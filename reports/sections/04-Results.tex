\section{Results}

Pyro\cite{bingham2019pyro} was used to perform stochastic variational inference automatically and ArviZ\cite{kumar2019arviz} was used to generate simple statistics, summaries and visualizations for the inference results.

\begin{figure}
    \centering
    \includegraphics[scale=0.6]{figures/Mean_Var_really_accurate.PNG}
\caption{Experimentation on fitting planar flows on Gaussian Mixture Model with increasing variance and comparing mean (red) and variance (blue) of the two. Here the x-axis denotes the difference in variance $\sigma^2_{\mathrm{NF}}-\sigma^2_{\mathrm{gmm}}$ and the y-axis denotes the difference in mean $|\mu_{\mathrm{gmm}}-\mu_{\mathrm{NF}}|$ }
    \label{fig:my_label}
\end{figure}

\subsection{Multivatiate normal}

\lipsum[3]

With covariance matrix

\begin{figure}
    \centering
    % % This file was created with tikzplotlib v0.9.13.
\begin{tikzpicture}

\begin{axis}[
legend cell align={left},
legend style={
  fill opacity=0.8,
  draw opacity=1,
  text opacity=1,
  at={(0.97,0.03)},
  anchor=south east,
  draw=white!80!black
},
tick align=outside,
tick pos=left,
x grid style={white!69.0196078431373!black},
xlabel={Number of flows},
xmin=2.6, xmax=33.4,
xtick style={color=black},
y grid style={white!69.0196078431373!black},
ylabel={ELBO},
ymin=-18107.9666384118, ymax=-17870.5896779011,
ytick style={color=black}
]
\addplot [draw=blue, fill=blue, mark=*, only marks]
table{%
x  y
4 -18097.1767765704
8 -17904.5776323087
16 -17888.9819208821
32 -17881.3795397425
};
\addlegendentry{Planar}
\addplot [draw=red, fill=red, mark=*, only marks]
table{%
x  y
4 -18038.6877921755
8 -17931.7873051845
16 -17915.0520597312
32 -17885.7323459633
};
\addlegendentry{Radial}
\addplot [semithick, blue, dashed, forget plot]
table {%
4 -18097.1767765704
8 -17904.5776323087
16 -17888.9819208821
32 -17881.3795397425
};
\addplot [semithick, red, dashed, forget plot]
table {%
4 -18038.6877921755
8 -17931.7873051845
16 -17915.0520597312
32 -17885.7323459633
};
\end{axis}

\end{tikzpicture}

    \caption{Final ELBO of normalizing flow as a function of number of flows for correlated multivariate Normal posterior. The final ELBO was calculated as the mean of the ELBOs of the last 500 iterations from training.}
    \label{fig:planarvsradial}
\end{figure}

\subsection{Energy functions}

\lipsum[1]

\begin{table}
    \caption{Final ELBO for different potential energy posteriors from \textcite{rezende2015variational} (\cref{tab:test_energy_functions}).}
    \label{tab:energyresults}
    \centering
    \begin{tabular}{lrrrrrr}
\toprule
{} & \multicolumn{3}{l}{Planar} & \multicolumn{3}{l}{Radial} \\
{} &     2  &     8  &     32 &     2  &     8  &     32 \\
\midrule
\$U\_1(z)\$ &  74.72 & 269.66 & 278.05 & -59.80 & 229.70 & 262.65 \\
\$U\_2(z)\$ & 272.11 & 526.72 & 600.33 & -98.17 & 177.55 & 529.82 \\
\$U\_3(z)\$ & 530.81 & 692.82 & 876.17 & 221.41 & 537.42 & 761.59 \\
\$U\_4(z)\$ & 473.85 & 638.39 & 750.76 & 229.20 & 462.36 & 648.31 \\
\bottomrule
\end{tabular}

\end{table}

The normalizing flows were trained for 10000 epochs with an ADAM learning rate of $0.005$, 16 Markov Chain gradient estimation samples and 256 base distribution samples ($\Vec z_0 \sim q_0\left(\bullet |\Vec x\right)$).

\begin{figure}
    \centering
    % \resizebox{\textwidth}{!}{
    %     %% Creator: Matplotlib, PGF backend
%%
%% To include the figure in your LaTeX document, write
%%   \input{<filename>.pgf}
%%
%% Make sure the required packages are loaded in your preamble
%%   \usepackage{pgf}
%%
%% and, on pdftex
%%   \usepackage[utf8]{inputenc}\DeclareUnicodeCharacter{2212}{-}
%%
%% or, on luatex and xetex
%%   \usepackage{unicode-math}
%%
%% Figures using additional raster images can only be included by \input if
%% they are in the same directory as the main LaTeX file. For loading figures
%% from other directories you can use the `import` package
%%   \usepackage{import}
%%
%% and then include the figures with
%%   \import{<path to file>}{<filename>.pgf}
%%
%% Matplotlib used the following preamble
%%
\begingroup%
\makeatletter%
\begin{pgfpicture}%
\pgfpathrectangle{\pgfpointorigin}{\pgfqpoint{6.815597in}{5.000000in}}%
\pgfusepath{use as bounding box, clip}%
\begin{pgfscope}%
\pgfsetbuttcap%
\pgfsetmiterjoin%
\definecolor{currentfill}{rgb}{1.000000,1.000000,1.000000}%
\pgfsetfillcolor{currentfill}%
\pgfsetlinewidth{0.000000pt}%
\definecolor{currentstroke}{rgb}{1.000000,1.000000,1.000000}%
\pgfsetstrokecolor{currentstroke}%
\pgfsetdash{}{0pt}%
\pgfpathmoveto{\pgfqpoint{0.000000in}{0.000000in}}%
\pgfpathlineto{\pgfqpoint{6.815597in}{0.000000in}}%
\pgfpathlineto{\pgfqpoint{6.815597in}{5.000000in}}%
\pgfpathlineto{\pgfqpoint{0.000000in}{5.000000in}}%
\pgfpathclose%
\pgfusepath{fill}%
\end{pgfscope}%
\begin{pgfscope}%
\pgfsetbuttcap%
\pgfsetmiterjoin%
\definecolor{currentfill}{rgb}{1.000000,1.000000,1.000000}%
\pgfsetfillcolor{currentfill}%
\pgfsetlinewidth{0.000000pt}%
\definecolor{currentstroke}{rgb}{0.000000,0.000000,0.000000}%
\pgfsetstrokecolor{currentstroke}%
\pgfsetstrokeopacity{0.000000}%
\pgfsetdash{}{0pt}%
\pgfpathmoveto{\pgfqpoint{0.712202in}{2.803906in}}%
\pgfpathlineto{\pgfqpoint{3.095758in}{2.803906in}}%
\pgfpathlineto{\pgfqpoint{3.095758in}{4.673040in}}%
\pgfpathlineto{\pgfqpoint{0.712202in}{4.673040in}}%
\pgfpathclose%
\pgfusepath{fill}%
\end{pgfscope}%
\begin{pgfscope}%
\pgfsetbuttcap%
\pgfsetroundjoin%
\definecolor{currentfill}{rgb}{0.150000,0.150000,0.150000}%
\pgfsetfillcolor{currentfill}%
\pgfsetlinewidth{1.003750pt}%
\definecolor{currentstroke}{rgb}{0.150000,0.150000,0.150000}%
\pgfsetstrokecolor{currentstroke}%
\pgfsetdash{}{0pt}%
\pgfsys@defobject{currentmarker}{\pgfqpoint{0.000000in}{-0.066667in}}{\pgfqpoint{0.000000in}{0.000000in}}{%
\pgfpathmoveto{\pgfqpoint{0.000000in}{0.000000in}}%
\pgfpathlineto{\pgfqpoint{0.000000in}{-0.066667in}}%
\pgfusepath{stroke,fill}%
}%
\begin{pgfscope}%
\pgfsys@transformshift{0.820546in}{2.803906in}%
\pgfsys@useobject{currentmarker}{}%
\end{pgfscope}%
\end{pgfscope}%
\begin{pgfscope}%
\pgfsetbuttcap%
\pgfsetroundjoin%
\definecolor{currentfill}{rgb}{0.150000,0.150000,0.150000}%
\pgfsetfillcolor{currentfill}%
\pgfsetlinewidth{1.003750pt}%
\definecolor{currentstroke}{rgb}{0.150000,0.150000,0.150000}%
\pgfsetstrokecolor{currentstroke}%
\pgfsetdash{}{0pt}%
\pgfsys@defobject{currentmarker}{\pgfqpoint{0.000000in}{-0.066667in}}{\pgfqpoint{0.000000in}{0.000000in}}{%
\pgfpathmoveto{\pgfqpoint{0.000000in}{0.000000in}}%
\pgfpathlineto{\pgfqpoint{0.000000in}{-0.066667in}}%
\pgfusepath{stroke,fill}%
}%
\begin{pgfscope}%
\pgfsys@transformshift{1.903980in}{2.803906in}%
\pgfsys@useobject{currentmarker}{}%
\end{pgfscope}%
\end{pgfscope}%
\begin{pgfscope}%
\pgfsetbuttcap%
\pgfsetroundjoin%
\definecolor{currentfill}{rgb}{0.150000,0.150000,0.150000}%
\pgfsetfillcolor{currentfill}%
\pgfsetlinewidth{1.003750pt}%
\definecolor{currentstroke}{rgb}{0.150000,0.150000,0.150000}%
\pgfsetstrokecolor{currentstroke}%
\pgfsetdash{}{0pt}%
\pgfsys@defobject{currentmarker}{\pgfqpoint{0.000000in}{-0.066667in}}{\pgfqpoint{0.000000in}{0.000000in}}{%
\pgfpathmoveto{\pgfqpoint{0.000000in}{0.000000in}}%
\pgfpathlineto{\pgfqpoint{0.000000in}{-0.066667in}}%
\pgfusepath{stroke,fill}%
}%
\begin{pgfscope}%
\pgfsys@transformshift{2.987414in}{2.803906in}%
\pgfsys@useobject{currentmarker}{}%
\end{pgfscope}%
\end{pgfscope}%
\begin{pgfscope}%
\pgfsetbuttcap%
\pgfsetroundjoin%
\definecolor{currentfill}{rgb}{0.150000,0.150000,0.150000}%
\pgfsetfillcolor{currentfill}%
\pgfsetlinewidth{1.003750pt}%
\definecolor{currentstroke}{rgb}{0.150000,0.150000,0.150000}%
\pgfsetstrokecolor{currentstroke}%
\pgfsetdash{}{0pt}%
\pgfsys@defobject{currentmarker}{\pgfqpoint{-0.066667in}{0.000000in}}{\pgfqpoint{-0.000000in}{0.000000in}}{%
\pgfpathmoveto{\pgfqpoint{-0.000000in}{0.000000in}}%
\pgfpathlineto{\pgfqpoint{-0.066667in}{0.000000in}}%
\pgfusepath{stroke,fill}%
}%
\begin{pgfscope}%
\pgfsys@transformshift{0.712202in}{3.189610in}%
\pgfsys@useobject{currentmarker}{}%
\end{pgfscope}%
\end{pgfscope}%
\begin{pgfscope}%
\definecolor{textcolor}{rgb}{0.150000,0.150000,0.150000}%
\pgfsetstrokecolor{textcolor}%
\pgfsetfillcolor{textcolor}%
\pgftext[x=0.532689in, y=3.146207in, left, base]{\color{textcolor}\rmfamily\fontsize{8.800000}{10.560000}\selectfont \(\displaystyle {0}\)}%
\end{pgfscope}%
\begin{pgfscope}%
\pgfsetbuttcap%
\pgfsetroundjoin%
\definecolor{currentfill}{rgb}{0.150000,0.150000,0.150000}%
\pgfsetfillcolor{currentfill}%
\pgfsetlinewidth{1.003750pt}%
\definecolor{currentstroke}{rgb}{0.150000,0.150000,0.150000}%
\pgfsetstrokecolor{currentstroke}%
\pgfsetdash{}{0pt}%
\pgfsys@defobject{currentmarker}{\pgfqpoint{-0.066667in}{0.000000in}}{\pgfqpoint{-0.000000in}{0.000000in}}{%
\pgfpathmoveto{\pgfqpoint{-0.000000in}{0.000000in}}%
\pgfpathlineto{\pgfqpoint{-0.066667in}{0.000000in}}%
\pgfusepath{stroke,fill}%
}%
\begin{pgfscope}%
\pgfsys@transformshift{0.712202in}{3.692566in}%
\pgfsys@useobject{currentmarker}{}%
\end{pgfscope}%
\end{pgfscope}%
\begin{pgfscope}%
\definecolor{textcolor}{rgb}{0.150000,0.150000,0.150000}%
\pgfsetstrokecolor{textcolor}%
\pgfsetfillcolor{textcolor}%
\pgftext[x=0.404218in, y=3.649163in, left, base]{\color{textcolor}\rmfamily\fontsize{8.800000}{10.560000}\selectfont \(\displaystyle {100}\)}%
\end{pgfscope}%
\begin{pgfscope}%
\pgfsetbuttcap%
\pgfsetroundjoin%
\definecolor{currentfill}{rgb}{0.150000,0.150000,0.150000}%
\pgfsetfillcolor{currentfill}%
\pgfsetlinewidth{1.003750pt}%
\definecolor{currentstroke}{rgb}{0.150000,0.150000,0.150000}%
\pgfsetstrokecolor{currentstroke}%
\pgfsetdash{}{0pt}%
\pgfsys@defobject{currentmarker}{\pgfqpoint{-0.066667in}{0.000000in}}{\pgfqpoint{-0.000000in}{0.000000in}}{%
\pgfpathmoveto{\pgfqpoint{-0.000000in}{0.000000in}}%
\pgfpathlineto{\pgfqpoint{-0.066667in}{0.000000in}}%
\pgfusepath{stroke,fill}%
}%
\begin{pgfscope}%
\pgfsys@transformshift{0.712202in}{4.195522in}%
\pgfsys@useobject{currentmarker}{}%
\end{pgfscope}%
\end{pgfscope}%
\begin{pgfscope}%
\definecolor{textcolor}{rgb}{0.150000,0.150000,0.150000}%
\pgfsetstrokecolor{textcolor}%
\pgfsetfillcolor{textcolor}%
\pgftext[x=0.404218in, y=4.152119in, left, base]{\color{textcolor}\rmfamily\fontsize{8.800000}{10.560000}\selectfont \(\displaystyle {200}\)}%
\end{pgfscope}%
\begin{pgfscope}%
\definecolor{textcolor}{rgb}{0.150000,0.150000,0.150000}%
\pgfsetstrokecolor{textcolor}%
\pgfsetfillcolor{textcolor}%
\pgftext[x=0.348662in,y=3.738473in,,bottom,rotate=90.000000]{\color{textcolor}\rmfamily\fontsize{9.600000}{11.520000}\selectfont ELBO}%
\end{pgfscope}%
\begin{pgfscope}%
\pgfpathrectangle{\pgfqpoint{0.712202in}{2.803906in}}{\pgfqpoint{2.383555in}{1.869134in}}%
\pgfusepath{clip}%
\pgfsetbuttcap%
\pgfsetroundjoin%
\definecolor{currentfill}{rgb}{0.768627,0.305882,0.321569}%
\pgfsetfillcolor{currentfill}%
\pgfsetlinewidth{0.385440pt}%
\definecolor{currentstroke}{rgb}{1.000000,1.000000,1.000000}%
\pgfsetstrokecolor{currentstroke}%
\pgfsetdash{}{0pt}%
\pgfpathmoveto{\pgfqpoint{0.820546in}{3.532091in}}%
\pgfpathcurveto{\pgfqpoint{0.829386in}{3.532091in}}{\pgfqpoint{0.837865in}{3.535604in}}{\pgfqpoint{0.844116in}{3.541855in}}%
\pgfpathcurveto{\pgfqpoint{0.850367in}{3.548105in}}{\pgfqpoint{0.853879in}{3.556585in}}{\pgfqpoint{0.853879in}{3.565425in}}%
\pgfpathcurveto{\pgfqpoint{0.853879in}{3.574265in}}{\pgfqpoint{0.850367in}{3.582744in}}{\pgfqpoint{0.844116in}{3.588995in}}%
\pgfpathcurveto{\pgfqpoint{0.837865in}{3.595246in}}{\pgfqpoint{0.829386in}{3.598758in}}{\pgfqpoint{0.820546in}{3.598758in}}%
\pgfpathcurveto{\pgfqpoint{0.811706in}{3.598758in}}{\pgfqpoint{0.803226in}{3.595246in}}{\pgfqpoint{0.796976in}{3.588995in}}%
\pgfpathcurveto{\pgfqpoint{0.790725in}{3.582744in}}{\pgfqpoint{0.787212in}{3.574265in}}{\pgfqpoint{0.787212in}{3.565425in}}%
\pgfpathcurveto{\pgfqpoint{0.787212in}{3.556585in}}{\pgfqpoint{0.790725in}{3.548105in}}{\pgfqpoint{0.796976in}{3.541855in}}%
\pgfpathcurveto{\pgfqpoint{0.803226in}{3.535604in}}{\pgfqpoint{0.811706in}{3.532091in}}{\pgfqpoint{0.820546in}{3.532091in}}%
\pgfpathclose%
\pgfusepath{stroke,fill}%
\end{pgfscope}%
\begin{pgfscope}%
\pgfpathrectangle{\pgfqpoint{0.712202in}{2.803906in}}{\pgfqpoint{2.383555in}{1.869134in}}%
\pgfusepath{clip}%
\pgfsetbuttcap%
\pgfsetroundjoin%
\definecolor{currentfill}{rgb}{0.768627,0.305882,0.321569}%
\pgfsetfillcolor{currentfill}%
\pgfsetlinewidth{0.385440pt}%
\definecolor{currentstroke}{rgb}{1.000000,1.000000,1.000000}%
\pgfsetstrokecolor{currentstroke}%
\pgfsetdash{}{0pt}%
\pgfpathmoveto{\pgfqpoint{1.903980in}{4.512537in}}%
\pgfpathcurveto{\pgfqpoint{1.912820in}{4.512537in}}{\pgfqpoint{1.921299in}{4.516049in}}{\pgfqpoint{1.927550in}{4.522300in}}%
\pgfpathcurveto{\pgfqpoint{1.933801in}{4.528551in}}{\pgfqpoint{1.937313in}{4.537030in}}{\pgfqpoint{1.937313in}{4.545870in}}%
\pgfpathcurveto{\pgfqpoint{1.937313in}{4.554710in}}{\pgfqpoint{1.933801in}{4.563189in}}{\pgfqpoint{1.927550in}{4.569440in}}%
\pgfpathcurveto{\pgfqpoint{1.921299in}{4.575691in}}{\pgfqpoint{1.912820in}{4.579203in}}{\pgfqpoint{1.903980in}{4.579203in}}%
\pgfpathcurveto{\pgfqpoint{1.895140in}{4.579203in}}{\pgfqpoint{1.886661in}{4.575691in}}{\pgfqpoint{1.880410in}{4.569440in}}%
\pgfpathcurveto{\pgfqpoint{1.874159in}{4.563189in}}{\pgfqpoint{1.870647in}{4.554710in}}{\pgfqpoint{1.870647in}{4.545870in}}%
\pgfpathcurveto{\pgfqpoint{1.870647in}{4.537030in}}{\pgfqpoint{1.874159in}{4.528551in}}{\pgfqpoint{1.880410in}{4.522300in}}%
\pgfpathcurveto{\pgfqpoint{1.886661in}{4.516049in}}{\pgfqpoint{1.895140in}{4.512537in}}{\pgfqpoint{1.903980in}{4.512537in}}%
\pgfpathclose%
\pgfusepath{stroke,fill}%
\end{pgfscope}%
\begin{pgfscope}%
\pgfpathrectangle{\pgfqpoint{0.712202in}{2.803906in}}{\pgfqpoint{2.383555in}{1.869134in}}%
\pgfusepath{clip}%
\pgfsetbuttcap%
\pgfsetroundjoin%
\definecolor{currentfill}{rgb}{0.768627,0.305882,0.321569}%
\pgfsetfillcolor{currentfill}%
\pgfsetlinewidth{0.385440pt}%
\definecolor{currentstroke}{rgb}{1.000000,1.000000,1.000000}%
\pgfsetstrokecolor{currentstroke}%
\pgfsetdash{}{0pt}%
\pgfpathmoveto{\pgfqpoint{2.987414in}{4.554746in}}%
\pgfpathcurveto{\pgfqpoint{2.996254in}{4.554746in}}{\pgfqpoint{3.004734in}{4.558259in}}{\pgfqpoint{3.010985in}{4.564509in}}%
\pgfpathcurveto{\pgfqpoint{3.017235in}{4.570760in}}{\pgfqpoint{3.020748in}{4.579240in}}{\pgfqpoint{3.020748in}{4.588080in}}%
\pgfpathcurveto{\pgfqpoint{3.020748in}{4.596920in}}{\pgfqpoint{3.017235in}{4.605399in}}{\pgfqpoint{3.010985in}{4.611650in}}%
\pgfpathcurveto{\pgfqpoint{3.004734in}{4.617901in}}{\pgfqpoint{2.996254in}{4.621413in}}{\pgfqpoint{2.987414in}{4.621413in}}%
\pgfpathcurveto{\pgfqpoint{2.978574in}{4.621413in}}{\pgfqpoint{2.970095in}{4.617901in}}{\pgfqpoint{2.963844in}{4.611650in}}%
\pgfpathcurveto{\pgfqpoint{2.957593in}{4.605399in}}{\pgfqpoint{2.954081in}{4.596920in}}{\pgfqpoint{2.954081in}{4.588080in}}%
\pgfpathcurveto{\pgfqpoint{2.954081in}{4.579240in}}{\pgfqpoint{2.957593in}{4.570760in}}{\pgfqpoint{2.963844in}{4.564509in}}%
\pgfpathcurveto{\pgfqpoint{2.970095in}{4.558259in}}{\pgfqpoint{2.978574in}{4.554746in}}{\pgfqpoint{2.987414in}{4.554746in}}%
\pgfpathclose%
\pgfusepath{stroke,fill}%
\end{pgfscope}%
\begin{pgfscope}%
\pgfpathrectangle{\pgfqpoint{0.712202in}{2.803906in}}{\pgfqpoint{2.383555in}{1.869134in}}%
\pgfusepath{clip}%
\pgfsetbuttcap%
\pgfsetroundjoin%
\definecolor{currentfill}{rgb}{0.298039,0.447059,0.690196}%
\pgfsetfillcolor{currentfill}%
\pgfsetlinewidth{0.385440pt}%
\definecolor{currentstroke}{rgb}{1.000000,1.000000,1.000000}%
\pgfsetstrokecolor{currentstroke}%
\pgfsetdash{}{0pt}%
\pgfpathmoveto{\pgfqpoint{0.820546in}{2.855533in}}%
\pgfpathcurveto{\pgfqpoint{0.829386in}{2.855533in}}{\pgfqpoint{0.837865in}{2.859045in}}{\pgfqpoint{0.844116in}{2.865296in}}%
\pgfpathcurveto{\pgfqpoint{0.850367in}{2.871547in}}{\pgfqpoint{0.853879in}{2.880026in}}{\pgfqpoint{0.853879in}{2.888867in}}%
\pgfpathcurveto{\pgfqpoint{0.853879in}{2.897707in}}{\pgfqpoint{0.850367in}{2.906186in}}{\pgfqpoint{0.844116in}{2.912437in}}%
\pgfpathcurveto{\pgfqpoint{0.837865in}{2.918688in}}{\pgfqpoint{0.829386in}{2.922200in}}{\pgfqpoint{0.820546in}{2.922200in}}%
\pgfpathcurveto{\pgfqpoint{0.811706in}{2.922200in}}{\pgfqpoint{0.803226in}{2.918688in}}{\pgfqpoint{0.796976in}{2.912437in}}%
\pgfpathcurveto{\pgfqpoint{0.790725in}{2.906186in}}{\pgfqpoint{0.787212in}{2.897707in}}{\pgfqpoint{0.787212in}{2.888867in}}%
\pgfpathcurveto{\pgfqpoint{0.787212in}{2.880026in}}{\pgfqpoint{0.790725in}{2.871547in}}{\pgfqpoint{0.796976in}{2.865296in}}%
\pgfpathcurveto{\pgfqpoint{0.803226in}{2.859045in}}{\pgfqpoint{0.811706in}{2.855533in}}{\pgfqpoint{0.820546in}{2.855533in}}%
\pgfpathclose%
\pgfusepath{stroke,fill}%
\end{pgfscope}%
\begin{pgfscope}%
\pgfpathrectangle{\pgfqpoint{0.712202in}{2.803906in}}{\pgfqpoint{2.383555in}{1.869134in}}%
\pgfusepath{clip}%
\pgfsetbuttcap%
\pgfsetroundjoin%
\definecolor{currentfill}{rgb}{0.298039,0.447059,0.690196}%
\pgfsetfillcolor{currentfill}%
\pgfsetlinewidth{0.385440pt}%
\definecolor{currentstroke}{rgb}{1.000000,1.000000,1.000000}%
\pgfsetstrokecolor{currentstroke}%
\pgfsetdash{}{0pt}%
\pgfpathmoveto{\pgfqpoint{1.903980in}{4.311583in}}%
\pgfpathcurveto{\pgfqpoint{1.912820in}{4.311583in}}{\pgfqpoint{1.921299in}{4.315095in}}{\pgfqpoint{1.927550in}{4.321346in}}%
\pgfpathcurveto{\pgfqpoint{1.933801in}{4.327597in}}{\pgfqpoint{1.937313in}{4.336076in}}{\pgfqpoint{1.937313in}{4.344916in}}%
\pgfpathcurveto{\pgfqpoint{1.937313in}{4.353757in}}{\pgfqpoint{1.933801in}{4.362236in}}{\pgfqpoint{1.927550in}{4.368487in}}%
\pgfpathcurveto{\pgfqpoint{1.921299in}{4.374738in}}{\pgfqpoint{1.912820in}{4.378250in}}{\pgfqpoint{1.903980in}{4.378250in}}%
\pgfpathcurveto{\pgfqpoint{1.895140in}{4.378250in}}{\pgfqpoint{1.886661in}{4.374738in}}{\pgfqpoint{1.880410in}{4.368487in}}%
\pgfpathcurveto{\pgfqpoint{1.874159in}{4.362236in}}{\pgfqpoint{1.870647in}{4.353757in}}{\pgfqpoint{1.870647in}{4.344916in}}%
\pgfpathcurveto{\pgfqpoint{1.870647in}{4.336076in}}{\pgfqpoint{1.874159in}{4.327597in}}{\pgfqpoint{1.880410in}{4.321346in}}%
\pgfpathcurveto{\pgfqpoint{1.886661in}{4.315095in}}{\pgfqpoint{1.895140in}{4.311583in}}{\pgfqpoint{1.903980in}{4.311583in}}%
\pgfpathclose%
\pgfusepath{stroke,fill}%
\end{pgfscope}%
\begin{pgfscope}%
\pgfpathrectangle{\pgfqpoint{0.712202in}{2.803906in}}{\pgfqpoint{2.383555in}{1.869134in}}%
\pgfusepath{clip}%
\pgfsetbuttcap%
\pgfsetroundjoin%
\definecolor{currentfill}{rgb}{0.298039,0.447059,0.690196}%
\pgfsetfillcolor{currentfill}%
\pgfsetlinewidth{0.385440pt}%
\definecolor{currentstroke}{rgb}{1.000000,1.000000,1.000000}%
\pgfsetstrokecolor{currentstroke}%
\pgfsetdash{}{0pt}%
\pgfpathmoveto{\pgfqpoint{2.987414in}{4.477273in}}%
\pgfpathcurveto{\pgfqpoint{2.996254in}{4.477273in}}{\pgfqpoint{3.004734in}{4.480785in}}{\pgfqpoint{3.010985in}{4.487036in}}%
\pgfpathcurveto{\pgfqpoint{3.017235in}{4.493287in}}{\pgfqpoint{3.020748in}{4.501766in}}{\pgfqpoint{3.020748in}{4.510606in}}%
\pgfpathcurveto{\pgfqpoint{3.020748in}{4.519446in}}{\pgfqpoint{3.017235in}{4.527925in}}{\pgfqpoint{3.010985in}{4.534176in}}%
\pgfpathcurveto{\pgfqpoint{3.004734in}{4.540427in}}{\pgfqpoint{2.996254in}{4.543939in}}{\pgfqpoint{2.987414in}{4.543939in}}%
\pgfpathcurveto{\pgfqpoint{2.978574in}{4.543939in}}{\pgfqpoint{2.970095in}{4.540427in}}{\pgfqpoint{2.963844in}{4.534176in}}%
\pgfpathcurveto{\pgfqpoint{2.957593in}{4.527925in}}{\pgfqpoint{2.954081in}{4.519446in}}{\pgfqpoint{2.954081in}{4.510606in}}%
\pgfpathcurveto{\pgfqpoint{2.954081in}{4.501766in}}{\pgfqpoint{2.957593in}{4.493287in}}{\pgfqpoint{2.963844in}{4.487036in}}%
\pgfpathcurveto{\pgfqpoint{2.970095in}{4.480785in}}{\pgfqpoint{2.978574in}{4.477273in}}{\pgfqpoint{2.987414in}{4.477273in}}%
\pgfpathclose%
\pgfusepath{stroke,fill}%
\end{pgfscope}%
\begin{pgfscope}%
\pgfpathrectangle{\pgfqpoint{0.712202in}{2.803906in}}{\pgfqpoint{2.383555in}{1.869134in}}%
\pgfusepath{clip}%
\pgfsetbuttcap%
\pgfsetroundjoin%
\definecolor{currentfill}{rgb}{0.768627,0.305882,0.321569}%
\pgfsetfillcolor{currentfill}%
\pgfsetlinewidth{0.803000pt}%
\definecolor{currentstroke}{rgb}{0.768627,0.305882,0.321569}%
\pgfsetstrokecolor{currentstroke}%
\pgfsetdash{}{0pt}%
\pgfsys@defobject{currentmarker}{\pgfqpoint{-0.033333in}{-0.033333in}}{\pgfqpoint{0.033333in}{0.033333in}}{%
\pgfpathmoveto{\pgfqpoint{0.000000in}{-0.033333in}}%
\pgfpathcurveto{\pgfqpoint{0.008840in}{-0.033333in}}{\pgfqpoint{0.017319in}{-0.029821in}}{\pgfqpoint{0.023570in}{-0.023570in}}%
\pgfpathcurveto{\pgfqpoint{0.029821in}{-0.017319in}}{\pgfqpoint{0.033333in}{-0.008840in}}{\pgfqpoint{0.033333in}{0.000000in}}%
\pgfpathcurveto{\pgfqpoint{0.033333in}{0.008840in}}{\pgfqpoint{0.029821in}{0.017319in}}{\pgfqpoint{0.023570in}{0.023570in}}%
\pgfpathcurveto{\pgfqpoint{0.017319in}{0.029821in}}{\pgfqpoint{0.008840in}{0.033333in}}{\pgfqpoint{0.000000in}{0.033333in}}%
\pgfpathcurveto{\pgfqpoint{-0.008840in}{0.033333in}}{\pgfqpoint{-0.017319in}{0.029821in}}{\pgfqpoint{-0.023570in}{0.023570in}}%
\pgfpathcurveto{\pgfqpoint{-0.029821in}{0.017319in}}{\pgfqpoint{-0.033333in}{0.008840in}}{\pgfqpoint{-0.033333in}{0.000000in}}%
\pgfpathcurveto{\pgfqpoint{-0.033333in}{-0.008840in}}{\pgfqpoint{-0.029821in}{-0.017319in}}{\pgfqpoint{-0.023570in}{-0.023570in}}%
\pgfpathcurveto{\pgfqpoint{-0.017319in}{-0.029821in}}{\pgfqpoint{-0.008840in}{-0.033333in}}{\pgfqpoint{0.000000in}{-0.033333in}}%
\pgfpathclose%
\pgfusepath{stroke,fill}%
}%
\end{pgfscope}%
\begin{pgfscope}%
\pgfpathrectangle{\pgfqpoint{0.712202in}{2.803906in}}{\pgfqpoint{2.383555in}{1.869134in}}%
\pgfusepath{clip}%
\pgfsetbuttcap%
\pgfsetroundjoin%
\definecolor{currentfill}{rgb}{0.298039,0.447059,0.690196}%
\pgfsetfillcolor{currentfill}%
\pgfsetlinewidth{0.803000pt}%
\definecolor{currentstroke}{rgb}{0.298039,0.447059,0.690196}%
\pgfsetstrokecolor{currentstroke}%
\pgfsetdash{}{0pt}%
\pgfsys@defobject{currentmarker}{\pgfqpoint{-0.033333in}{-0.033333in}}{\pgfqpoint{0.033333in}{0.033333in}}{%
\pgfpathmoveto{\pgfqpoint{0.000000in}{-0.033333in}}%
\pgfpathcurveto{\pgfqpoint{0.008840in}{-0.033333in}}{\pgfqpoint{0.017319in}{-0.029821in}}{\pgfqpoint{0.023570in}{-0.023570in}}%
\pgfpathcurveto{\pgfqpoint{0.029821in}{-0.017319in}}{\pgfqpoint{0.033333in}{-0.008840in}}{\pgfqpoint{0.033333in}{0.000000in}}%
\pgfpathcurveto{\pgfqpoint{0.033333in}{0.008840in}}{\pgfqpoint{0.029821in}{0.017319in}}{\pgfqpoint{0.023570in}{0.023570in}}%
\pgfpathcurveto{\pgfqpoint{0.017319in}{0.029821in}}{\pgfqpoint{0.008840in}{0.033333in}}{\pgfqpoint{0.000000in}{0.033333in}}%
\pgfpathcurveto{\pgfqpoint{-0.008840in}{0.033333in}}{\pgfqpoint{-0.017319in}{0.029821in}}{\pgfqpoint{-0.023570in}{0.023570in}}%
\pgfpathcurveto{\pgfqpoint{-0.029821in}{0.017319in}}{\pgfqpoint{-0.033333in}{0.008840in}}{\pgfqpoint{-0.033333in}{0.000000in}}%
\pgfpathcurveto{\pgfqpoint{-0.033333in}{-0.008840in}}{\pgfqpoint{-0.029821in}{-0.017319in}}{\pgfqpoint{-0.023570in}{-0.023570in}}%
\pgfpathcurveto{\pgfqpoint{-0.017319in}{-0.029821in}}{\pgfqpoint{-0.008840in}{-0.033333in}}{\pgfqpoint{0.000000in}{-0.033333in}}%
\pgfpathclose%
\pgfusepath{stroke,fill}%
}%
\end{pgfscope}%
\begin{pgfscope}%
\pgfpathrectangle{\pgfqpoint{0.712202in}{2.803906in}}{\pgfqpoint{2.383555in}{1.869134in}}%
\pgfusepath{clip}%
\pgfsetbuttcap%
\pgfsetroundjoin%
\pgfsetlinewidth{2.007500pt}%
\definecolor{currentstroke}{rgb}{0.768627,0.305882,0.321569}%
\pgfsetstrokecolor{currentstroke}%
\pgfsetdash{{7.400000pt}{3.200000pt}}{0.000000pt}%
\pgfpathmoveto{\pgfqpoint{0.820546in}{3.565425in}}%
\pgfpathlineto{\pgfqpoint{1.903980in}{4.545870in}}%
\pgfpathlineto{\pgfqpoint{2.987414in}{4.588080in}}%
\pgfusepath{stroke}%
\end{pgfscope}%
\begin{pgfscope}%
\pgfpathrectangle{\pgfqpoint{0.712202in}{2.803906in}}{\pgfqpoint{2.383555in}{1.869134in}}%
\pgfusepath{clip}%
\pgfsetbuttcap%
\pgfsetroundjoin%
\pgfsetlinewidth{2.007500pt}%
\definecolor{currentstroke}{rgb}{0.298039,0.447059,0.690196}%
\pgfsetstrokecolor{currentstroke}%
\pgfsetdash{{7.400000pt}{3.200000pt}}{0.000000pt}%
\pgfpathmoveto{\pgfqpoint{0.820546in}{2.888867in}}%
\pgfpathlineto{\pgfqpoint{1.903980in}{4.344916in}}%
\pgfpathlineto{\pgfqpoint{2.987414in}{4.510606in}}%
\pgfusepath{stroke}%
\end{pgfscope}%
\begin{pgfscope}%
\pgfsetrectcap%
\pgfsetmiterjoin%
\pgfsetlinewidth{1.003750pt}%
\definecolor{currentstroke}{rgb}{0.150000,0.150000,0.150000}%
\pgfsetstrokecolor{currentstroke}%
\pgfsetdash{}{0pt}%
\pgfpathmoveto{\pgfqpoint{0.712202in}{2.803906in}}%
\pgfpathlineto{\pgfqpoint{0.712202in}{4.673040in}}%
\pgfusepath{stroke}%
\end{pgfscope}%
\begin{pgfscope}%
\pgfsetrectcap%
\pgfsetmiterjoin%
\pgfsetlinewidth{1.003750pt}%
\definecolor{currentstroke}{rgb}{0.150000,0.150000,0.150000}%
\pgfsetstrokecolor{currentstroke}%
\pgfsetdash{}{0pt}%
\pgfpathmoveto{\pgfqpoint{0.712202in}{2.803906in}}%
\pgfpathlineto{\pgfqpoint{3.095758in}{2.803906in}}%
\pgfusepath{stroke}%
\end{pgfscope}%
\begin{pgfscope}%
\definecolor{textcolor}{rgb}{0.150000,0.150000,0.150000}%
\pgfsetstrokecolor{textcolor}%
\pgfsetfillcolor{textcolor}%
\pgftext[x=1.903980in,y=4.756374in,,base]{\color{textcolor}\rmfamily\fontsize{9.600000}{11.520000}\selectfont \(\displaystyle U_1(z)\)}%
\end{pgfscope}%
\begin{pgfscope}%
\pgfsetbuttcap%
\pgfsetmiterjoin%
\definecolor{currentfill}{rgb}{1.000000,1.000000,1.000000}%
\pgfsetfillcolor{currentfill}%
\pgfsetlinewidth{0.000000pt}%
\definecolor{currentstroke}{rgb}{0.000000,0.000000,0.000000}%
\pgfsetstrokecolor{currentstroke}%
\pgfsetstrokeopacity{0.000000}%
\pgfsetdash{}{0pt}%
\pgfpathmoveto{\pgfqpoint{3.548289in}{2.803906in}}%
\pgfpathlineto{\pgfqpoint{5.931844in}{2.803906in}}%
\pgfpathlineto{\pgfqpoint{5.931844in}{4.673040in}}%
\pgfpathlineto{\pgfqpoint{3.548289in}{4.673040in}}%
\pgfpathclose%
\pgfusepath{fill}%
\end{pgfscope}%
\begin{pgfscope}%
\pgfsetbuttcap%
\pgfsetroundjoin%
\definecolor{currentfill}{rgb}{0.150000,0.150000,0.150000}%
\pgfsetfillcolor{currentfill}%
\pgfsetlinewidth{1.003750pt}%
\definecolor{currentstroke}{rgb}{0.150000,0.150000,0.150000}%
\pgfsetstrokecolor{currentstroke}%
\pgfsetdash{}{0pt}%
\pgfsys@defobject{currentmarker}{\pgfqpoint{0.000000in}{-0.066667in}}{\pgfqpoint{0.000000in}{0.000000in}}{%
\pgfpathmoveto{\pgfqpoint{0.000000in}{0.000000in}}%
\pgfpathlineto{\pgfqpoint{0.000000in}{-0.066667in}}%
\pgfusepath{stroke,fill}%
}%
\begin{pgfscope}%
\pgfsys@transformshift{3.656632in}{2.803906in}%
\pgfsys@useobject{currentmarker}{}%
\end{pgfscope}%
\end{pgfscope}%
\begin{pgfscope}%
\pgfsetbuttcap%
\pgfsetroundjoin%
\definecolor{currentfill}{rgb}{0.150000,0.150000,0.150000}%
\pgfsetfillcolor{currentfill}%
\pgfsetlinewidth{1.003750pt}%
\definecolor{currentstroke}{rgb}{0.150000,0.150000,0.150000}%
\pgfsetstrokecolor{currentstroke}%
\pgfsetdash{}{0pt}%
\pgfsys@defobject{currentmarker}{\pgfqpoint{0.000000in}{-0.066667in}}{\pgfqpoint{0.000000in}{0.000000in}}{%
\pgfpathmoveto{\pgfqpoint{0.000000in}{0.000000in}}%
\pgfpathlineto{\pgfqpoint{0.000000in}{-0.066667in}}%
\pgfusepath{stroke,fill}%
}%
\begin{pgfscope}%
\pgfsys@transformshift{4.740066in}{2.803906in}%
\pgfsys@useobject{currentmarker}{}%
\end{pgfscope}%
\end{pgfscope}%
\begin{pgfscope}%
\pgfsetbuttcap%
\pgfsetroundjoin%
\definecolor{currentfill}{rgb}{0.150000,0.150000,0.150000}%
\pgfsetfillcolor{currentfill}%
\pgfsetlinewidth{1.003750pt}%
\definecolor{currentstroke}{rgb}{0.150000,0.150000,0.150000}%
\pgfsetstrokecolor{currentstroke}%
\pgfsetdash{}{0pt}%
\pgfsys@defobject{currentmarker}{\pgfqpoint{0.000000in}{-0.066667in}}{\pgfqpoint{0.000000in}{0.000000in}}{%
\pgfpathmoveto{\pgfqpoint{0.000000in}{0.000000in}}%
\pgfpathlineto{\pgfqpoint{0.000000in}{-0.066667in}}%
\pgfusepath{stroke,fill}%
}%
\begin{pgfscope}%
\pgfsys@transformshift{5.823501in}{2.803906in}%
\pgfsys@useobject{currentmarker}{}%
\end{pgfscope}%
\end{pgfscope}%
\begin{pgfscope}%
\pgfsetbuttcap%
\pgfsetroundjoin%
\definecolor{currentfill}{rgb}{0.150000,0.150000,0.150000}%
\pgfsetfillcolor{currentfill}%
\pgfsetlinewidth{1.003750pt}%
\definecolor{currentstroke}{rgb}{0.150000,0.150000,0.150000}%
\pgfsetstrokecolor{currentstroke}%
\pgfsetdash{}{0pt}%
\pgfsys@defobject{currentmarker}{\pgfqpoint{-0.066667in}{0.000000in}}{\pgfqpoint{-0.000000in}{0.000000in}}{%
\pgfpathmoveto{\pgfqpoint{-0.000000in}{0.000000in}}%
\pgfpathlineto{\pgfqpoint{-0.066667in}{0.000000in}}%
\pgfusepath{stroke,fill}%
}%
\begin{pgfscope}%
\pgfsys@transformshift{3.548289in}{3.127689in}%
\pgfsys@useobject{currentmarker}{}%
\end{pgfscope}%
\end{pgfscope}%
\begin{pgfscope}%
\definecolor{textcolor}{rgb}{0.150000,0.150000,0.150000}%
\pgfsetstrokecolor{textcolor}%
\pgfsetfillcolor{textcolor}%
\pgftext[x=3.368775in, y=3.084286in, left, base]{\color{textcolor}\rmfamily\fontsize{8.800000}{10.560000}\selectfont \(\displaystyle {0}\)}%
\end{pgfscope}%
\begin{pgfscope}%
\pgfsetbuttcap%
\pgfsetroundjoin%
\definecolor{currentfill}{rgb}{0.150000,0.150000,0.150000}%
\pgfsetfillcolor{currentfill}%
\pgfsetlinewidth{1.003750pt}%
\definecolor{currentstroke}{rgb}{0.150000,0.150000,0.150000}%
\pgfsetstrokecolor{currentstroke}%
\pgfsetdash{}{0pt}%
\pgfsys@defobject{currentmarker}{\pgfqpoint{-0.066667in}{0.000000in}}{\pgfqpoint{-0.000000in}{0.000000in}}{%
\pgfpathmoveto{\pgfqpoint{-0.000000in}{0.000000in}}%
\pgfpathlineto{\pgfqpoint{-0.066667in}{0.000000in}}%
\pgfusepath{stroke,fill}%
}%
\begin{pgfscope}%
\pgfsys@transformshift{3.548289in}{3.614215in}%
\pgfsys@useobject{currentmarker}{}%
\end{pgfscope}%
\end{pgfscope}%
\begin{pgfscope}%
\definecolor{textcolor}{rgb}{0.150000,0.150000,0.150000}%
\pgfsetstrokecolor{textcolor}%
\pgfsetfillcolor{textcolor}%
\pgftext[x=3.240304in, y=3.570812in, left, base]{\color{textcolor}\rmfamily\fontsize{8.800000}{10.560000}\selectfont \(\displaystyle {200}\)}%
\end{pgfscope}%
\begin{pgfscope}%
\pgfsetbuttcap%
\pgfsetroundjoin%
\definecolor{currentfill}{rgb}{0.150000,0.150000,0.150000}%
\pgfsetfillcolor{currentfill}%
\pgfsetlinewidth{1.003750pt}%
\definecolor{currentstroke}{rgb}{0.150000,0.150000,0.150000}%
\pgfsetstrokecolor{currentstroke}%
\pgfsetdash{}{0pt}%
\pgfsys@defobject{currentmarker}{\pgfqpoint{-0.066667in}{0.000000in}}{\pgfqpoint{-0.000000in}{0.000000in}}{%
\pgfpathmoveto{\pgfqpoint{-0.000000in}{0.000000in}}%
\pgfpathlineto{\pgfqpoint{-0.066667in}{0.000000in}}%
\pgfusepath{stroke,fill}%
}%
\begin{pgfscope}%
\pgfsys@transformshift{3.548289in}{4.100740in}%
\pgfsys@useobject{currentmarker}{}%
\end{pgfscope}%
\end{pgfscope}%
\begin{pgfscope}%
\definecolor{textcolor}{rgb}{0.150000,0.150000,0.150000}%
\pgfsetstrokecolor{textcolor}%
\pgfsetfillcolor{textcolor}%
\pgftext[x=3.240304in, y=4.057337in, left, base]{\color{textcolor}\rmfamily\fontsize{8.800000}{10.560000}\selectfont \(\displaystyle {400}\)}%
\end{pgfscope}%
\begin{pgfscope}%
\pgfsetbuttcap%
\pgfsetroundjoin%
\definecolor{currentfill}{rgb}{0.150000,0.150000,0.150000}%
\pgfsetfillcolor{currentfill}%
\pgfsetlinewidth{1.003750pt}%
\definecolor{currentstroke}{rgb}{0.150000,0.150000,0.150000}%
\pgfsetstrokecolor{currentstroke}%
\pgfsetdash{}{0pt}%
\pgfsys@defobject{currentmarker}{\pgfqpoint{-0.066667in}{0.000000in}}{\pgfqpoint{-0.000000in}{0.000000in}}{%
\pgfpathmoveto{\pgfqpoint{-0.000000in}{0.000000in}}%
\pgfpathlineto{\pgfqpoint{-0.066667in}{0.000000in}}%
\pgfusepath{stroke,fill}%
}%
\begin{pgfscope}%
\pgfsys@transformshift{3.548289in}{4.587266in}%
\pgfsys@useobject{currentmarker}{}%
\end{pgfscope}%
\end{pgfscope}%
\begin{pgfscope}%
\definecolor{textcolor}{rgb}{0.150000,0.150000,0.150000}%
\pgfsetstrokecolor{textcolor}%
\pgfsetfillcolor{textcolor}%
\pgftext[x=3.240304in, y=4.543863in, left, base]{\color{textcolor}\rmfamily\fontsize{8.800000}{10.560000}\selectfont \(\displaystyle {600}\)}%
\end{pgfscope}%
\begin{pgfscope}%
\pgfpathrectangle{\pgfqpoint{3.548289in}{2.803906in}}{\pgfqpoint{2.383555in}{1.869134in}}%
\pgfusepath{clip}%
\pgfsetbuttcap%
\pgfsetroundjoin%
\definecolor{currentfill}{rgb}{0.768627,0.305882,0.321569}%
\pgfsetfillcolor{currentfill}%
\pgfsetlinewidth{0.385440pt}%
\definecolor{currentstroke}{rgb}{1.000000,1.000000,1.000000}%
\pgfsetstrokecolor{currentstroke}%
\pgfsetdash{}{0pt}%
\pgfpathmoveto{\pgfqpoint{3.656632in}{3.756299in}}%
\pgfpathcurveto{\pgfqpoint{3.665472in}{3.756299in}}{\pgfqpoint{3.673951in}{3.759811in}}{\pgfqpoint{3.680202in}{3.766062in}}%
\pgfpathcurveto{\pgfqpoint{3.686453in}{3.772313in}}{\pgfqpoint{3.689965in}{3.780792in}}{\pgfqpoint{3.689965in}{3.789632in}}%
\pgfpathcurveto{\pgfqpoint{3.689965in}{3.798472in}}{\pgfqpoint{3.686453in}{3.806952in}}{\pgfqpoint{3.680202in}{3.813202in}}%
\pgfpathcurveto{\pgfqpoint{3.673951in}{3.819453in}}{\pgfqpoint{3.665472in}{3.822966in}}{\pgfqpoint{3.656632in}{3.822966in}}%
\pgfpathcurveto{\pgfqpoint{3.647792in}{3.822966in}}{\pgfqpoint{3.639313in}{3.819453in}}{\pgfqpoint{3.633062in}{3.813202in}}%
\pgfpathcurveto{\pgfqpoint{3.626811in}{3.806952in}}{\pgfqpoint{3.623299in}{3.798472in}}{\pgfqpoint{3.623299in}{3.789632in}}%
\pgfpathcurveto{\pgfqpoint{3.623299in}{3.780792in}}{\pgfqpoint{3.626811in}{3.772313in}}{\pgfqpoint{3.633062in}{3.766062in}}%
\pgfpathcurveto{\pgfqpoint{3.639313in}{3.759811in}}{\pgfqpoint{3.647792in}{3.756299in}}{\pgfqpoint{3.656632in}{3.756299in}}%
\pgfpathclose%
\pgfusepath{stroke,fill}%
\end{pgfscope}%
\begin{pgfscope}%
\pgfpathrectangle{\pgfqpoint{3.548289in}{2.803906in}}{\pgfqpoint{2.383555in}{1.869134in}}%
\pgfusepath{clip}%
\pgfsetbuttcap%
\pgfsetroundjoin%
\definecolor{currentfill}{rgb}{0.768627,0.305882,0.321569}%
\pgfsetfillcolor{currentfill}%
\pgfsetlinewidth{0.385440pt}%
\definecolor{currentstroke}{rgb}{1.000000,1.000000,1.000000}%
\pgfsetstrokecolor{currentstroke}%
\pgfsetdash{}{0pt}%
\pgfpathmoveto{\pgfqpoint{4.740066in}{4.375680in}}%
\pgfpathcurveto{\pgfqpoint{4.748906in}{4.375680in}}{\pgfqpoint{4.757386in}{4.379192in}}{\pgfqpoint{4.763637in}{4.385443in}}%
\pgfpathcurveto{\pgfqpoint{4.769887in}{4.391694in}}{\pgfqpoint{4.773400in}{4.400173in}}{\pgfqpoint{4.773400in}{4.409013in}}%
\pgfpathcurveto{\pgfqpoint{4.773400in}{4.417853in}}{\pgfqpoint{4.769887in}{4.426332in}}{\pgfqpoint{4.763637in}{4.432583in}}%
\pgfpathcurveto{\pgfqpoint{4.757386in}{4.438834in}}{\pgfqpoint{4.748906in}{4.442346in}}{\pgfqpoint{4.740066in}{4.442346in}}%
\pgfpathcurveto{\pgfqpoint{4.731226in}{4.442346in}}{\pgfqpoint{4.722747in}{4.438834in}}{\pgfqpoint{4.716496in}{4.432583in}}%
\pgfpathcurveto{\pgfqpoint{4.710245in}{4.426332in}}{\pgfqpoint{4.706733in}{4.417853in}}{\pgfqpoint{4.706733in}{4.409013in}}%
\pgfpathcurveto{\pgfqpoint{4.706733in}{4.400173in}}{\pgfqpoint{4.710245in}{4.391694in}}{\pgfqpoint{4.716496in}{4.385443in}}%
\pgfpathcurveto{\pgfqpoint{4.722747in}{4.379192in}}{\pgfqpoint{4.731226in}{4.375680in}}{\pgfqpoint{4.740066in}{4.375680in}}%
\pgfpathclose%
\pgfusepath{stroke,fill}%
\end{pgfscope}%
\begin{pgfscope}%
\pgfpathrectangle{\pgfqpoint{3.548289in}{2.803906in}}{\pgfqpoint{2.383555in}{1.869134in}}%
\pgfusepath{clip}%
\pgfsetbuttcap%
\pgfsetroundjoin%
\definecolor{currentfill}{rgb}{0.768627,0.305882,0.321569}%
\pgfsetfillcolor{currentfill}%
\pgfsetlinewidth{0.385440pt}%
\definecolor{currentstroke}{rgb}{1.000000,1.000000,1.000000}%
\pgfsetstrokecolor{currentstroke}%
\pgfsetdash{}{0pt}%
\pgfpathmoveto{\pgfqpoint{5.823501in}{4.554746in}}%
\pgfpathcurveto{\pgfqpoint{5.832341in}{4.554746in}}{\pgfqpoint{5.840820in}{4.558259in}}{\pgfqpoint{5.847071in}{4.564509in}}%
\pgfpathcurveto{\pgfqpoint{5.853322in}{4.570760in}}{\pgfqpoint{5.856834in}{4.579240in}}{\pgfqpoint{5.856834in}{4.588080in}}%
\pgfpathcurveto{\pgfqpoint{5.856834in}{4.596920in}}{\pgfqpoint{5.853322in}{4.605399in}}{\pgfqpoint{5.847071in}{4.611650in}}%
\pgfpathcurveto{\pgfqpoint{5.840820in}{4.617901in}}{\pgfqpoint{5.832341in}{4.621413in}}{\pgfqpoint{5.823501in}{4.621413in}}%
\pgfpathcurveto{\pgfqpoint{5.814660in}{4.621413in}}{\pgfqpoint{5.806181in}{4.617901in}}{\pgfqpoint{5.799930in}{4.611650in}}%
\pgfpathcurveto{\pgfqpoint{5.793679in}{4.605399in}}{\pgfqpoint{5.790167in}{4.596920in}}{\pgfqpoint{5.790167in}{4.588080in}}%
\pgfpathcurveto{\pgfqpoint{5.790167in}{4.579240in}}{\pgfqpoint{5.793679in}{4.570760in}}{\pgfqpoint{5.799930in}{4.564509in}}%
\pgfpathcurveto{\pgfqpoint{5.806181in}{4.558259in}}{\pgfqpoint{5.814660in}{4.554746in}}{\pgfqpoint{5.823501in}{4.554746in}}%
\pgfpathclose%
\pgfusepath{stroke,fill}%
\end{pgfscope}%
\begin{pgfscope}%
\pgfpathrectangle{\pgfqpoint{3.548289in}{2.803906in}}{\pgfqpoint{2.383555in}{1.869134in}}%
\pgfusepath{clip}%
\pgfsetbuttcap%
\pgfsetroundjoin%
\definecolor{currentfill}{rgb}{0.298039,0.447059,0.690196}%
\pgfsetfillcolor{currentfill}%
\pgfsetlinewidth{0.385440pt}%
\definecolor{currentstroke}{rgb}{1.000000,1.000000,1.000000}%
\pgfsetstrokecolor{currentstroke}%
\pgfsetdash{}{0pt}%
\pgfpathmoveto{\pgfqpoint{3.656632in}{2.855533in}}%
\pgfpathcurveto{\pgfqpoint{3.665472in}{2.855533in}}{\pgfqpoint{3.673951in}{2.859045in}}{\pgfqpoint{3.680202in}{2.865296in}}%
\pgfpathcurveto{\pgfqpoint{3.686453in}{2.871547in}}{\pgfqpoint{3.689965in}{2.880026in}}{\pgfqpoint{3.689965in}{2.888867in}}%
\pgfpathcurveto{\pgfqpoint{3.689965in}{2.897707in}}{\pgfqpoint{3.686453in}{2.906186in}}{\pgfqpoint{3.680202in}{2.912437in}}%
\pgfpathcurveto{\pgfqpoint{3.673951in}{2.918688in}}{\pgfqpoint{3.665472in}{2.922200in}}{\pgfqpoint{3.656632in}{2.922200in}}%
\pgfpathcurveto{\pgfqpoint{3.647792in}{2.922200in}}{\pgfqpoint{3.639313in}{2.918688in}}{\pgfqpoint{3.633062in}{2.912437in}}%
\pgfpathcurveto{\pgfqpoint{3.626811in}{2.906186in}}{\pgfqpoint{3.623299in}{2.897707in}}{\pgfqpoint{3.623299in}{2.888867in}}%
\pgfpathcurveto{\pgfqpoint{3.623299in}{2.880026in}}{\pgfqpoint{3.626811in}{2.871547in}}{\pgfqpoint{3.633062in}{2.865296in}}%
\pgfpathcurveto{\pgfqpoint{3.639313in}{2.859045in}}{\pgfqpoint{3.647792in}{2.855533in}}{\pgfqpoint{3.656632in}{2.855533in}}%
\pgfpathclose%
\pgfusepath{stroke,fill}%
\end{pgfscope}%
\begin{pgfscope}%
\pgfpathrectangle{\pgfqpoint{3.548289in}{2.803906in}}{\pgfqpoint{2.383555in}{1.869134in}}%
\pgfusepath{clip}%
\pgfsetbuttcap%
\pgfsetroundjoin%
\definecolor{currentfill}{rgb}{0.298039,0.447059,0.690196}%
\pgfsetfillcolor{currentfill}%
\pgfsetlinewidth{0.385440pt}%
\definecolor{currentstroke}{rgb}{1.000000,1.000000,1.000000}%
\pgfsetstrokecolor{currentstroke}%
\pgfsetdash{}{0pt}%
\pgfpathmoveto{\pgfqpoint{4.740066in}{3.526279in}}%
\pgfpathcurveto{\pgfqpoint{4.748906in}{3.526279in}}{\pgfqpoint{4.757386in}{3.529791in}}{\pgfqpoint{4.763637in}{3.536042in}}%
\pgfpathcurveto{\pgfqpoint{4.769887in}{3.542293in}}{\pgfqpoint{4.773400in}{3.550773in}}{\pgfqpoint{4.773400in}{3.559613in}}%
\pgfpathcurveto{\pgfqpoint{4.773400in}{3.568453in}}{\pgfqpoint{4.769887in}{3.576932in}}{\pgfqpoint{4.763637in}{3.583183in}}%
\pgfpathcurveto{\pgfqpoint{4.757386in}{3.589434in}}{\pgfqpoint{4.748906in}{3.592946in}}{\pgfqpoint{4.740066in}{3.592946in}}%
\pgfpathcurveto{\pgfqpoint{4.731226in}{3.592946in}}{\pgfqpoint{4.722747in}{3.589434in}}{\pgfqpoint{4.716496in}{3.583183in}}%
\pgfpathcurveto{\pgfqpoint{4.710245in}{3.576932in}}{\pgfqpoint{4.706733in}{3.568453in}}{\pgfqpoint{4.706733in}{3.559613in}}%
\pgfpathcurveto{\pgfqpoint{4.706733in}{3.550773in}}{\pgfqpoint{4.710245in}{3.542293in}}{\pgfqpoint{4.716496in}{3.536042in}}%
\pgfpathcurveto{\pgfqpoint{4.722747in}{3.529791in}}{\pgfqpoint{4.731226in}{3.526279in}}{\pgfqpoint{4.740066in}{3.526279in}}%
\pgfpathclose%
\pgfusepath{stroke,fill}%
\end{pgfscope}%
\begin{pgfscope}%
\pgfpathrectangle{\pgfqpoint{3.548289in}{2.803906in}}{\pgfqpoint{2.383555in}{1.869134in}}%
\pgfusepath{clip}%
\pgfsetbuttcap%
\pgfsetroundjoin%
\definecolor{currentfill}{rgb}{0.298039,0.447059,0.690196}%
\pgfsetfillcolor{currentfill}%
\pgfsetlinewidth{0.385440pt}%
\definecolor{currentstroke}{rgb}{1.000000,1.000000,1.000000}%
\pgfsetstrokecolor{currentstroke}%
\pgfsetdash{}{0pt}%
\pgfpathmoveto{\pgfqpoint{5.823501in}{4.383214in}}%
\pgfpathcurveto{\pgfqpoint{5.832341in}{4.383214in}}{\pgfqpoint{5.840820in}{4.386726in}}{\pgfqpoint{5.847071in}{4.392977in}}%
\pgfpathcurveto{\pgfqpoint{5.853322in}{4.399228in}}{\pgfqpoint{5.856834in}{4.407707in}}{\pgfqpoint{5.856834in}{4.416548in}}%
\pgfpathcurveto{\pgfqpoint{5.856834in}{4.425388in}}{\pgfqpoint{5.853322in}{4.433867in}}{\pgfqpoint{5.847071in}{4.440118in}}%
\pgfpathcurveto{\pgfqpoint{5.840820in}{4.446369in}}{\pgfqpoint{5.832341in}{4.449881in}}{\pgfqpoint{5.823501in}{4.449881in}}%
\pgfpathcurveto{\pgfqpoint{5.814660in}{4.449881in}}{\pgfqpoint{5.806181in}{4.446369in}}{\pgfqpoint{5.799930in}{4.440118in}}%
\pgfpathcurveto{\pgfqpoint{5.793679in}{4.433867in}}{\pgfqpoint{5.790167in}{4.425388in}}{\pgfqpoint{5.790167in}{4.416548in}}%
\pgfpathcurveto{\pgfqpoint{5.790167in}{4.407707in}}{\pgfqpoint{5.793679in}{4.399228in}}{\pgfqpoint{5.799930in}{4.392977in}}%
\pgfpathcurveto{\pgfqpoint{5.806181in}{4.386726in}}{\pgfqpoint{5.814660in}{4.383214in}}{\pgfqpoint{5.823501in}{4.383214in}}%
\pgfpathclose%
\pgfusepath{stroke,fill}%
\end{pgfscope}%
\begin{pgfscope}%
\pgfpathrectangle{\pgfqpoint{3.548289in}{2.803906in}}{\pgfqpoint{2.383555in}{1.869134in}}%
\pgfusepath{clip}%
\pgfsetbuttcap%
\pgfsetroundjoin%
\definecolor{currentfill}{rgb}{0.768627,0.305882,0.321569}%
\pgfsetfillcolor{currentfill}%
\pgfsetlinewidth{0.803000pt}%
\definecolor{currentstroke}{rgb}{0.768627,0.305882,0.321569}%
\pgfsetstrokecolor{currentstroke}%
\pgfsetdash{}{0pt}%
\pgfsys@defobject{currentmarker}{\pgfqpoint{-0.033333in}{-0.033333in}}{\pgfqpoint{0.033333in}{0.033333in}}{%
\pgfpathmoveto{\pgfqpoint{0.000000in}{-0.033333in}}%
\pgfpathcurveto{\pgfqpoint{0.008840in}{-0.033333in}}{\pgfqpoint{0.017319in}{-0.029821in}}{\pgfqpoint{0.023570in}{-0.023570in}}%
\pgfpathcurveto{\pgfqpoint{0.029821in}{-0.017319in}}{\pgfqpoint{0.033333in}{-0.008840in}}{\pgfqpoint{0.033333in}{0.000000in}}%
\pgfpathcurveto{\pgfqpoint{0.033333in}{0.008840in}}{\pgfqpoint{0.029821in}{0.017319in}}{\pgfqpoint{0.023570in}{0.023570in}}%
\pgfpathcurveto{\pgfqpoint{0.017319in}{0.029821in}}{\pgfqpoint{0.008840in}{0.033333in}}{\pgfqpoint{0.000000in}{0.033333in}}%
\pgfpathcurveto{\pgfqpoint{-0.008840in}{0.033333in}}{\pgfqpoint{-0.017319in}{0.029821in}}{\pgfqpoint{-0.023570in}{0.023570in}}%
\pgfpathcurveto{\pgfqpoint{-0.029821in}{0.017319in}}{\pgfqpoint{-0.033333in}{0.008840in}}{\pgfqpoint{-0.033333in}{0.000000in}}%
\pgfpathcurveto{\pgfqpoint{-0.033333in}{-0.008840in}}{\pgfqpoint{-0.029821in}{-0.017319in}}{\pgfqpoint{-0.023570in}{-0.023570in}}%
\pgfpathcurveto{\pgfqpoint{-0.017319in}{-0.029821in}}{\pgfqpoint{-0.008840in}{-0.033333in}}{\pgfqpoint{0.000000in}{-0.033333in}}%
\pgfpathclose%
\pgfusepath{stroke,fill}%
}%
\end{pgfscope}%
\begin{pgfscope}%
\pgfpathrectangle{\pgfqpoint{3.548289in}{2.803906in}}{\pgfqpoint{2.383555in}{1.869134in}}%
\pgfusepath{clip}%
\pgfsetbuttcap%
\pgfsetroundjoin%
\definecolor{currentfill}{rgb}{0.298039,0.447059,0.690196}%
\pgfsetfillcolor{currentfill}%
\pgfsetlinewidth{0.803000pt}%
\definecolor{currentstroke}{rgb}{0.298039,0.447059,0.690196}%
\pgfsetstrokecolor{currentstroke}%
\pgfsetdash{}{0pt}%
\pgfsys@defobject{currentmarker}{\pgfqpoint{-0.033333in}{-0.033333in}}{\pgfqpoint{0.033333in}{0.033333in}}{%
\pgfpathmoveto{\pgfqpoint{0.000000in}{-0.033333in}}%
\pgfpathcurveto{\pgfqpoint{0.008840in}{-0.033333in}}{\pgfqpoint{0.017319in}{-0.029821in}}{\pgfqpoint{0.023570in}{-0.023570in}}%
\pgfpathcurveto{\pgfqpoint{0.029821in}{-0.017319in}}{\pgfqpoint{0.033333in}{-0.008840in}}{\pgfqpoint{0.033333in}{0.000000in}}%
\pgfpathcurveto{\pgfqpoint{0.033333in}{0.008840in}}{\pgfqpoint{0.029821in}{0.017319in}}{\pgfqpoint{0.023570in}{0.023570in}}%
\pgfpathcurveto{\pgfqpoint{0.017319in}{0.029821in}}{\pgfqpoint{0.008840in}{0.033333in}}{\pgfqpoint{0.000000in}{0.033333in}}%
\pgfpathcurveto{\pgfqpoint{-0.008840in}{0.033333in}}{\pgfqpoint{-0.017319in}{0.029821in}}{\pgfqpoint{-0.023570in}{0.023570in}}%
\pgfpathcurveto{\pgfqpoint{-0.029821in}{0.017319in}}{\pgfqpoint{-0.033333in}{0.008840in}}{\pgfqpoint{-0.033333in}{0.000000in}}%
\pgfpathcurveto{\pgfqpoint{-0.033333in}{-0.008840in}}{\pgfqpoint{-0.029821in}{-0.017319in}}{\pgfqpoint{-0.023570in}{-0.023570in}}%
\pgfpathcurveto{\pgfqpoint{-0.017319in}{-0.029821in}}{\pgfqpoint{-0.008840in}{-0.033333in}}{\pgfqpoint{0.000000in}{-0.033333in}}%
\pgfpathclose%
\pgfusepath{stroke,fill}%
}%
\end{pgfscope}%
\begin{pgfscope}%
\pgfpathrectangle{\pgfqpoint{3.548289in}{2.803906in}}{\pgfqpoint{2.383555in}{1.869134in}}%
\pgfusepath{clip}%
\pgfsetbuttcap%
\pgfsetroundjoin%
\pgfsetlinewidth{2.007500pt}%
\definecolor{currentstroke}{rgb}{0.768627,0.305882,0.321569}%
\pgfsetstrokecolor{currentstroke}%
\pgfsetdash{{7.400000pt}{3.200000pt}}{0.000000pt}%
\pgfpathmoveto{\pgfqpoint{3.656632in}{3.789632in}}%
\pgfpathlineto{\pgfqpoint{4.740066in}{4.409013in}}%
\pgfpathlineto{\pgfqpoint{5.823501in}{4.588080in}}%
\pgfusepath{stroke}%
\end{pgfscope}%
\begin{pgfscope}%
\pgfpathrectangle{\pgfqpoint{3.548289in}{2.803906in}}{\pgfqpoint{2.383555in}{1.869134in}}%
\pgfusepath{clip}%
\pgfsetbuttcap%
\pgfsetroundjoin%
\pgfsetlinewidth{2.007500pt}%
\definecolor{currentstroke}{rgb}{0.298039,0.447059,0.690196}%
\pgfsetstrokecolor{currentstroke}%
\pgfsetdash{{7.400000pt}{3.200000pt}}{0.000000pt}%
\pgfpathmoveto{\pgfqpoint{3.656632in}{2.888867in}}%
\pgfpathlineto{\pgfqpoint{4.740066in}{3.559613in}}%
\pgfpathlineto{\pgfqpoint{5.823501in}{4.416548in}}%
\pgfusepath{stroke}%
\end{pgfscope}%
\begin{pgfscope}%
\pgfsetrectcap%
\pgfsetmiterjoin%
\pgfsetlinewidth{1.003750pt}%
\definecolor{currentstroke}{rgb}{0.150000,0.150000,0.150000}%
\pgfsetstrokecolor{currentstroke}%
\pgfsetdash{}{0pt}%
\pgfpathmoveto{\pgfqpoint{3.548289in}{2.803906in}}%
\pgfpathlineto{\pgfqpoint{3.548289in}{4.673040in}}%
\pgfusepath{stroke}%
\end{pgfscope}%
\begin{pgfscope}%
\pgfsetrectcap%
\pgfsetmiterjoin%
\pgfsetlinewidth{1.003750pt}%
\definecolor{currentstroke}{rgb}{0.150000,0.150000,0.150000}%
\pgfsetstrokecolor{currentstroke}%
\pgfsetdash{}{0pt}%
\pgfpathmoveto{\pgfqpoint{3.548289in}{2.803906in}}%
\pgfpathlineto{\pgfqpoint{5.931844in}{2.803906in}}%
\pgfusepath{stroke}%
\end{pgfscope}%
\begin{pgfscope}%
\definecolor{textcolor}{rgb}{0.150000,0.150000,0.150000}%
\pgfsetstrokecolor{textcolor}%
\pgfsetfillcolor{textcolor}%
\pgftext[x=4.740066in,y=4.756374in,,base]{\color{textcolor}\rmfamily\fontsize{9.600000}{11.520000}\selectfont \(\displaystyle U_2(z)\)}%
\end{pgfscope}%
\begin{pgfscope}%
\pgfsetbuttcap%
\pgfsetmiterjoin%
\definecolor{currentfill}{rgb}{1.000000,1.000000,1.000000}%
\pgfsetfillcolor{currentfill}%
\pgfsetlinewidth{0.000000pt}%
\definecolor{currentstroke}{rgb}{0.000000,0.000000,0.000000}%
\pgfsetstrokecolor{currentstroke}%
\pgfsetstrokeopacity{0.000000}%
\pgfsetdash{}{0pt}%
\pgfpathmoveto{\pgfqpoint{0.712202in}{0.541145in}}%
\pgfpathlineto{\pgfqpoint{3.095758in}{0.541145in}}%
\pgfpathlineto{\pgfqpoint{3.095758in}{2.410280in}}%
\pgfpathlineto{\pgfqpoint{0.712202in}{2.410280in}}%
\pgfpathclose%
\pgfusepath{fill}%
\end{pgfscope}%
\begin{pgfscope}%
\pgfsetbuttcap%
\pgfsetroundjoin%
\definecolor{currentfill}{rgb}{0.150000,0.150000,0.150000}%
\pgfsetfillcolor{currentfill}%
\pgfsetlinewidth{1.003750pt}%
\definecolor{currentstroke}{rgb}{0.150000,0.150000,0.150000}%
\pgfsetstrokecolor{currentstroke}%
\pgfsetdash{}{0pt}%
\pgfsys@defobject{currentmarker}{\pgfqpoint{0.000000in}{-0.066667in}}{\pgfqpoint{0.000000in}{0.000000in}}{%
\pgfpathmoveto{\pgfqpoint{0.000000in}{0.000000in}}%
\pgfpathlineto{\pgfqpoint{0.000000in}{-0.066667in}}%
\pgfusepath{stroke,fill}%
}%
\begin{pgfscope}%
\pgfsys@transformshift{0.820546in}{0.541145in}%
\pgfsys@useobject{currentmarker}{}%
\end{pgfscope}%
\end{pgfscope}%
\begin{pgfscope}%
\definecolor{textcolor}{rgb}{0.150000,0.150000,0.150000}%
\pgfsetstrokecolor{textcolor}%
\pgfsetfillcolor{textcolor}%
\pgftext[x=0.820546in,y=0.425867in,,top]{\color{textcolor}\rmfamily\fontsize{8.800000}{10.560000}\selectfont 2}%
\end{pgfscope}%
\begin{pgfscope}%
\pgfsetbuttcap%
\pgfsetroundjoin%
\definecolor{currentfill}{rgb}{0.150000,0.150000,0.150000}%
\pgfsetfillcolor{currentfill}%
\pgfsetlinewidth{1.003750pt}%
\definecolor{currentstroke}{rgb}{0.150000,0.150000,0.150000}%
\pgfsetstrokecolor{currentstroke}%
\pgfsetdash{}{0pt}%
\pgfsys@defobject{currentmarker}{\pgfqpoint{0.000000in}{-0.066667in}}{\pgfqpoint{0.000000in}{0.000000in}}{%
\pgfpathmoveto{\pgfqpoint{0.000000in}{0.000000in}}%
\pgfpathlineto{\pgfqpoint{0.000000in}{-0.066667in}}%
\pgfusepath{stroke,fill}%
}%
\begin{pgfscope}%
\pgfsys@transformshift{1.903980in}{0.541145in}%
\pgfsys@useobject{currentmarker}{}%
\end{pgfscope}%
\end{pgfscope}%
\begin{pgfscope}%
\definecolor{textcolor}{rgb}{0.150000,0.150000,0.150000}%
\pgfsetstrokecolor{textcolor}%
\pgfsetfillcolor{textcolor}%
\pgftext[x=1.903980in,y=0.425867in,,top]{\color{textcolor}\rmfamily\fontsize{8.800000}{10.560000}\selectfont 8}%
\end{pgfscope}%
\begin{pgfscope}%
\pgfsetbuttcap%
\pgfsetroundjoin%
\definecolor{currentfill}{rgb}{0.150000,0.150000,0.150000}%
\pgfsetfillcolor{currentfill}%
\pgfsetlinewidth{1.003750pt}%
\definecolor{currentstroke}{rgb}{0.150000,0.150000,0.150000}%
\pgfsetstrokecolor{currentstroke}%
\pgfsetdash{}{0pt}%
\pgfsys@defobject{currentmarker}{\pgfqpoint{0.000000in}{-0.066667in}}{\pgfqpoint{0.000000in}{0.000000in}}{%
\pgfpathmoveto{\pgfqpoint{0.000000in}{0.000000in}}%
\pgfpathlineto{\pgfqpoint{0.000000in}{-0.066667in}}%
\pgfusepath{stroke,fill}%
}%
\begin{pgfscope}%
\pgfsys@transformshift{2.987414in}{0.541145in}%
\pgfsys@useobject{currentmarker}{}%
\end{pgfscope}%
\end{pgfscope}%
\begin{pgfscope}%
\definecolor{textcolor}{rgb}{0.150000,0.150000,0.150000}%
\pgfsetstrokecolor{textcolor}%
\pgfsetfillcolor{textcolor}%
\pgftext[x=2.987414in,y=0.425867in,,top]{\color{textcolor}\rmfamily\fontsize{8.800000}{10.560000}\selectfont 32}%
\end{pgfscope}%
\begin{pgfscope}%
\definecolor{textcolor}{rgb}{0.150000,0.150000,0.150000}%
\pgfsetstrokecolor{textcolor}%
\pgfsetfillcolor{textcolor}%
\pgftext[x=1.903980in,y=0.259201in,,top]{\color{textcolor}\rmfamily\fontsize{9.600000}{11.520000}\selectfont Number of flows}%
\end{pgfscope}%
\begin{pgfscope}%
\pgfsetbuttcap%
\pgfsetroundjoin%
\definecolor{currentfill}{rgb}{0.150000,0.150000,0.150000}%
\pgfsetfillcolor{currentfill}%
\pgfsetlinewidth{1.003750pt}%
\definecolor{currentstroke}{rgb}{0.150000,0.150000,0.150000}%
\pgfsetstrokecolor{currentstroke}%
\pgfsetdash{}{0pt}%
\pgfsys@defobject{currentmarker}{\pgfqpoint{-0.066667in}{0.000000in}}{\pgfqpoint{-0.000000in}{0.000000in}}{%
\pgfpathmoveto{\pgfqpoint{-0.000000in}{0.000000in}}%
\pgfpathlineto{\pgfqpoint{-0.066667in}{0.000000in}}%
\pgfusepath{stroke,fill}%
}%
\begin{pgfscope}%
\pgfsys@transformshift{0.712202in}{0.570533in}%
\pgfsys@useobject{currentmarker}{}%
\end{pgfscope}%
\end{pgfscope}%
\begin{pgfscope}%
\definecolor{textcolor}{rgb}{0.150000,0.150000,0.150000}%
\pgfsetstrokecolor{textcolor}%
\pgfsetfillcolor{textcolor}%
\pgftext[x=0.404218in, y=0.527130in, left, base]{\color{textcolor}\rmfamily\fontsize{8.800000}{10.560000}\selectfont \(\displaystyle {200}\)}%
\end{pgfscope}%
\begin{pgfscope}%
\pgfsetbuttcap%
\pgfsetroundjoin%
\definecolor{currentfill}{rgb}{0.150000,0.150000,0.150000}%
\pgfsetfillcolor{currentfill}%
\pgfsetlinewidth{1.003750pt}%
\definecolor{currentstroke}{rgb}{0.150000,0.150000,0.150000}%
\pgfsetstrokecolor{currentstroke}%
\pgfsetdash{}{0pt}%
\pgfsys@defobject{currentmarker}{\pgfqpoint{-0.066667in}{0.000000in}}{\pgfqpoint{-0.000000in}{0.000000in}}{%
\pgfpathmoveto{\pgfqpoint{-0.000000in}{0.000000in}}%
\pgfpathlineto{\pgfqpoint{-0.066667in}{0.000000in}}%
\pgfusepath{stroke,fill}%
}%
\begin{pgfscope}%
\pgfsys@transformshift{0.712202in}{1.089567in}%
\pgfsys@useobject{currentmarker}{}%
\end{pgfscope}%
\end{pgfscope}%
\begin{pgfscope}%
\definecolor{textcolor}{rgb}{0.150000,0.150000,0.150000}%
\pgfsetstrokecolor{textcolor}%
\pgfsetfillcolor{textcolor}%
\pgftext[x=0.404218in, y=1.046164in, left, base]{\color{textcolor}\rmfamily\fontsize{8.800000}{10.560000}\selectfont \(\displaystyle {400}\)}%
\end{pgfscope}%
\begin{pgfscope}%
\pgfsetbuttcap%
\pgfsetroundjoin%
\definecolor{currentfill}{rgb}{0.150000,0.150000,0.150000}%
\pgfsetfillcolor{currentfill}%
\pgfsetlinewidth{1.003750pt}%
\definecolor{currentstroke}{rgb}{0.150000,0.150000,0.150000}%
\pgfsetstrokecolor{currentstroke}%
\pgfsetdash{}{0pt}%
\pgfsys@defobject{currentmarker}{\pgfqpoint{-0.066667in}{0.000000in}}{\pgfqpoint{-0.000000in}{0.000000in}}{%
\pgfpathmoveto{\pgfqpoint{-0.000000in}{0.000000in}}%
\pgfpathlineto{\pgfqpoint{-0.066667in}{0.000000in}}%
\pgfusepath{stroke,fill}%
}%
\begin{pgfscope}%
\pgfsys@transformshift{0.712202in}{1.608601in}%
\pgfsys@useobject{currentmarker}{}%
\end{pgfscope}%
\end{pgfscope}%
\begin{pgfscope}%
\definecolor{textcolor}{rgb}{0.150000,0.150000,0.150000}%
\pgfsetstrokecolor{textcolor}%
\pgfsetfillcolor{textcolor}%
\pgftext[x=0.404218in, y=1.565198in, left, base]{\color{textcolor}\rmfamily\fontsize{8.800000}{10.560000}\selectfont \(\displaystyle {600}\)}%
\end{pgfscope}%
\begin{pgfscope}%
\pgfsetbuttcap%
\pgfsetroundjoin%
\definecolor{currentfill}{rgb}{0.150000,0.150000,0.150000}%
\pgfsetfillcolor{currentfill}%
\pgfsetlinewidth{1.003750pt}%
\definecolor{currentstroke}{rgb}{0.150000,0.150000,0.150000}%
\pgfsetstrokecolor{currentstroke}%
\pgfsetdash{}{0pt}%
\pgfsys@defobject{currentmarker}{\pgfqpoint{-0.066667in}{0.000000in}}{\pgfqpoint{-0.000000in}{0.000000in}}{%
\pgfpathmoveto{\pgfqpoint{-0.000000in}{0.000000in}}%
\pgfpathlineto{\pgfqpoint{-0.066667in}{0.000000in}}%
\pgfusepath{stroke,fill}%
}%
\begin{pgfscope}%
\pgfsys@transformshift{0.712202in}{2.127635in}%
\pgfsys@useobject{currentmarker}{}%
\end{pgfscope}%
\end{pgfscope}%
\begin{pgfscope}%
\definecolor{textcolor}{rgb}{0.150000,0.150000,0.150000}%
\pgfsetstrokecolor{textcolor}%
\pgfsetfillcolor{textcolor}%
\pgftext[x=0.404218in, y=2.084232in, left, base]{\color{textcolor}\rmfamily\fontsize{8.800000}{10.560000}\selectfont \(\displaystyle {800}\)}%
\end{pgfscope}%
\begin{pgfscope}%
\definecolor{textcolor}{rgb}{0.150000,0.150000,0.150000}%
\pgfsetstrokecolor{textcolor}%
\pgfsetfillcolor{textcolor}%
\pgftext[x=0.348662in,y=1.475712in,,bottom,rotate=90.000000]{\color{textcolor}\rmfamily\fontsize{9.600000}{11.520000}\selectfont ELBO}%
\end{pgfscope}%
\begin{pgfscope}%
\pgfpathrectangle{\pgfqpoint{0.712202in}{0.541145in}}{\pgfqpoint{2.383555in}{1.869134in}}%
\pgfusepath{clip}%
\pgfsetbuttcap%
\pgfsetroundjoin%
\definecolor{currentfill}{rgb}{0.768627,0.305882,0.321569}%
\pgfsetfillcolor{currentfill}%
\pgfsetlinewidth{0.385440pt}%
\definecolor{currentstroke}{rgb}{1.000000,1.000000,1.000000}%
\pgfsetstrokecolor{currentstroke}%
\pgfsetdash{}{0pt}%
\pgfpathmoveto{\pgfqpoint{0.820546in}{1.395704in}}%
\pgfpathcurveto{\pgfqpoint{0.829386in}{1.395704in}}{\pgfqpoint{0.837865in}{1.399216in}}{\pgfqpoint{0.844116in}{1.405467in}}%
\pgfpathcurveto{\pgfqpoint{0.850367in}{1.411718in}}{\pgfqpoint{0.853879in}{1.420197in}}{\pgfqpoint{0.853879in}{1.429037in}}%
\pgfpathcurveto{\pgfqpoint{0.853879in}{1.437877in}}{\pgfqpoint{0.850367in}{1.446357in}}{\pgfqpoint{0.844116in}{1.452608in}}%
\pgfpathcurveto{\pgfqpoint{0.837865in}{1.458858in}}{\pgfqpoint{0.829386in}{1.462371in}}{\pgfqpoint{0.820546in}{1.462371in}}%
\pgfpathcurveto{\pgfqpoint{0.811706in}{1.462371in}}{\pgfqpoint{0.803226in}{1.458858in}}{\pgfqpoint{0.796976in}{1.452608in}}%
\pgfpathcurveto{\pgfqpoint{0.790725in}{1.446357in}}{\pgfqpoint{0.787212in}{1.437877in}}{\pgfqpoint{0.787212in}{1.429037in}}%
\pgfpathcurveto{\pgfqpoint{0.787212in}{1.420197in}}{\pgfqpoint{0.790725in}{1.411718in}}{\pgfqpoint{0.796976in}{1.405467in}}%
\pgfpathcurveto{\pgfqpoint{0.803226in}{1.399216in}}{\pgfqpoint{0.811706in}{1.395704in}}{\pgfqpoint{0.820546in}{1.395704in}}%
\pgfpathclose%
\pgfusepath{stroke,fill}%
\end{pgfscope}%
\begin{pgfscope}%
\pgfpathrectangle{\pgfqpoint{0.712202in}{0.541145in}}{\pgfqpoint{2.383555in}{1.869134in}}%
\pgfusepath{clip}%
\pgfsetbuttcap%
\pgfsetroundjoin%
\definecolor{currentfill}{rgb}{0.768627,0.305882,0.321569}%
\pgfsetfillcolor{currentfill}%
\pgfsetlinewidth{0.385440pt}%
\definecolor{currentstroke}{rgb}{1.000000,1.000000,1.000000}%
\pgfsetstrokecolor{currentstroke}%
\pgfsetdash{}{0pt}%
\pgfpathmoveto{\pgfqpoint{1.903980in}{1.816147in}}%
\pgfpathcurveto{\pgfqpoint{1.912820in}{1.816147in}}{\pgfqpoint{1.921299in}{1.819660in}}{\pgfqpoint{1.927550in}{1.825911in}}%
\pgfpathcurveto{\pgfqpoint{1.933801in}{1.832161in}}{\pgfqpoint{1.937313in}{1.840641in}}{\pgfqpoint{1.937313in}{1.849481in}}%
\pgfpathcurveto{\pgfqpoint{1.937313in}{1.858321in}}{\pgfqpoint{1.933801in}{1.866800in}}{\pgfqpoint{1.927550in}{1.873051in}}%
\pgfpathcurveto{\pgfqpoint{1.921299in}{1.879302in}}{\pgfqpoint{1.912820in}{1.882814in}}{\pgfqpoint{1.903980in}{1.882814in}}%
\pgfpathcurveto{\pgfqpoint{1.895140in}{1.882814in}}{\pgfqpoint{1.886661in}{1.879302in}}{\pgfqpoint{1.880410in}{1.873051in}}%
\pgfpathcurveto{\pgfqpoint{1.874159in}{1.866800in}}{\pgfqpoint{1.870647in}{1.858321in}}{\pgfqpoint{1.870647in}{1.849481in}}%
\pgfpathcurveto{\pgfqpoint{1.870647in}{1.840641in}}{\pgfqpoint{1.874159in}{1.832161in}}{\pgfqpoint{1.880410in}{1.825911in}}%
\pgfpathcurveto{\pgfqpoint{1.886661in}{1.819660in}}{\pgfqpoint{1.895140in}{1.816147in}}{\pgfqpoint{1.903980in}{1.816147in}}%
\pgfpathclose%
\pgfusepath{stroke,fill}%
\end{pgfscope}%
\begin{pgfscope}%
\pgfpathrectangle{\pgfqpoint{0.712202in}{0.541145in}}{\pgfqpoint{2.383555in}{1.869134in}}%
\pgfusepath{clip}%
\pgfsetbuttcap%
\pgfsetroundjoin%
\definecolor{currentfill}{rgb}{0.768627,0.305882,0.321569}%
\pgfsetfillcolor{currentfill}%
\pgfsetlinewidth{0.385440pt}%
\definecolor{currentstroke}{rgb}{1.000000,1.000000,1.000000}%
\pgfsetstrokecolor{currentstroke}%
\pgfsetdash{}{0pt}%
\pgfpathmoveto{\pgfqpoint{2.987414in}{2.291986in}}%
\pgfpathcurveto{\pgfqpoint{2.996254in}{2.291986in}}{\pgfqpoint{3.004734in}{2.295498in}}{\pgfqpoint{3.010985in}{2.301749in}}%
\pgfpathcurveto{\pgfqpoint{3.017235in}{2.308000in}}{\pgfqpoint{3.020748in}{2.316479in}}{\pgfqpoint{3.020748in}{2.325319in}}%
\pgfpathcurveto{\pgfqpoint{3.020748in}{2.334159in}}{\pgfqpoint{3.017235in}{2.342638in}}{\pgfqpoint{3.010985in}{2.348889in}}%
\pgfpathcurveto{\pgfqpoint{3.004734in}{2.355140in}}{\pgfqpoint{2.996254in}{2.358652in}}{\pgfqpoint{2.987414in}{2.358652in}}%
\pgfpathcurveto{\pgfqpoint{2.978574in}{2.358652in}}{\pgfqpoint{2.970095in}{2.355140in}}{\pgfqpoint{2.963844in}{2.348889in}}%
\pgfpathcurveto{\pgfqpoint{2.957593in}{2.342638in}}{\pgfqpoint{2.954081in}{2.334159in}}{\pgfqpoint{2.954081in}{2.325319in}}%
\pgfpathcurveto{\pgfqpoint{2.954081in}{2.316479in}}{\pgfqpoint{2.957593in}{2.308000in}}{\pgfqpoint{2.963844in}{2.301749in}}%
\pgfpathcurveto{\pgfqpoint{2.970095in}{2.295498in}}{\pgfqpoint{2.978574in}{2.291986in}}{\pgfqpoint{2.987414in}{2.291986in}}%
\pgfpathclose%
\pgfusepath{stroke,fill}%
\end{pgfscope}%
\begin{pgfscope}%
\pgfpathrectangle{\pgfqpoint{0.712202in}{0.541145in}}{\pgfqpoint{2.383555in}{1.869134in}}%
\pgfusepath{clip}%
\pgfsetbuttcap%
\pgfsetroundjoin%
\definecolor{currentfill}{rgb}{0.298039,0.447059,0.690196}%
\pgfsetfillcolor{currentfill}%
\pgfsetlinewidth{0.385440pt}%
\definecolor{currentstroke}{rgb}{1.000000,1.000000,1.000000}%
\pgfsetstrokecolor{currentstroke}%
\pgfsetdash{}{0pt}%
\pgfpathmoveto{\pgfqpoint{0.820546in}{0.592773in}}%
\pgfpathcurveto{\pgfqpoint{0.829386in}{0.592773in}}{\pgfqpoint{0.837865in}{0.596285in}}{\pgfqpoint{0.844116in}{0.602536in}}%
\pgfpathcurveto{\pgfqpoint{0.850367in}{0.608787in}}{\pgfqpoint{0.853879in}{0.617266in}}{\pgfqpoint{0.853879in}{0.626106in}}%
\pgfpathcurveto{\pgfqpoint{0.853879in}{0.634946in}}{\pgfqpoint{0.850367in}{0.643425in}}{\pgfqpoint{0.844116in}{0.649676in}}%
\pgfpathcurveto{\pgfqpoint{0.837865in}{0.655927in}}{\pgfqpoint{0.829386in}{0.659439in}}{\pgfqpoint{0.820546in}{0.659439in}}%
\pgfpathcurveto{\pgfqpoint{0.811706in}{0.659439in}}{\pgfqpoint{0.803226in}{0.655927in}}{\pgfqpoint{0.796976in}{0.649676in}}%
\pgfpathcurveto{\pgfqpoint{0.790725in}{0.643425in}}{\pgfqpoint{0.787212in}{0.634946in}}{\pgfqpoint{0.787212in}{0.626106in}}%
\pgfpathcurveto{\pgfqpoint{0.787212in}{0.617266in}}{\pgfqpoint{0.790725in}{0.608787in}}{\pgfqpoint{0.796976in}{0.602536in}}%
\pgfpathcurveto{\pgfqpoint{0.803226in}{0.596285in}}{\pgfqpoint{0.811706in}{0.592773in}}{\pgfqpoint{0.820546in}{0.592773in}}%
\pgfpathclose%
\pgfusepath{stroke,fill}%
\end{pgfscope}%
\begin{pgfscope}%
\pgfpathrectangle{\pgfqpoint{0.712202in}{0.541145in}}{\pgfqpoint{2.383555in}{1.869134in}}%
\pgfusepath{clip}%
\pgfsetbuttcap%
\pgfsetroundjoin%
\definecolor{currentfill}{rgb}{0.298039,0.447059,0.690196}%
\pgfsetfillcolor{currentfill}%
\pgfsetlinewidth{0.385440pt}%
\definecolor{currentstroke}{rgb}{1.000000,1.000000,1.000000}%
\pgfsetstrokecolor{currentstroke}%
\pgfsetdash{}{0pt}%
\pgfpathmoveto{\pgfqpoint{1.903980in}{1.412874in}}%
\pgfpathcurveto{\pgfqpoint{1.912820in}{1.412874in}}{\pgfqpoint{1.921299in}{1.416387in}}{\pgfqpoint{1.927550in}{1.422638in}}%
\pgfpathcurveto{\pgfqpoint{1.933801in}{1.428888in}}{\pgfqpoint{1.937313in}{1.437368in}}{\pgfqpoint{1.937313in}{1.446208in}}%
\pgfpathcurveto{\pgfqpoint{1.937313in}{1.455048in}}{\pgfqpoint{1.933801in}{1.463527in}}{\pgfqpoint{1.927550in}{1.469778in}}%
\pgfpathcurveto{\pgfqpoint{1.921299in}{1.476029in}}{\pgfqpoint{1.912820in}{1.479541in}}{\pgfqpoint{1.903980in}{1.479541in}}%
\pgfpathcurveto{\pgfqpoint{1.895140in}{1.479541in}}{\pgfqpoint{1.886661in}{1.476029in}}{\pgfqpoint{1.880410in}{1.469778in}}%
\pgfpathcurveto{\pgfqpoint{1.874159in}{1.463527in}}{\pgfqpoint{1.870647in}{1.455048in}}{\pgfqpoint{1.870647in}{1.446208in}}%
\pgfpathcurveto{\pgfqpoint{1.870647in}{1.437368in}}{\pgfqpoint{1.874159in}{1.428888in}}{\pgfqpoint{1.880410in}{1.422638in}}%
\pgfpathcurveto{\pgfqpoint{1.886661in}{1.416387in}}{\pgfqpoint{1.895140in}{1.412874in}}{\pgfqpoint{1.903980in}{1.412874in}}%
\pgfpathclose%
\pgfusepath{stroke,fill}%
\end{pgfscope}%
\begin{pgfscope}%
\pgfpathrectangle{\pgfqpoint{0.712202in}{0.541145in}}{\pgfqpoint{2.383555in}{1.869134in}}%
\pgfusepath{clip}%
\pgfsetbuttcap%
\pgfsetroundjoin%
\definecolor{currentfill}{rgb}{0.298039,0.447059,0.690196}%
\pgfsetfillcolor{currentfill}%
\pgfsetlinewidth{0.385440pt}%
\definecolor{currentstroke}{rgb}{1.000000,1.000000,1.000000}%
\pgfsetstrokecolor{currentstroke}%
\pgfsetdash{}{0pt}%
\pgfpathmoveto{\pgfqpoint{2.987414in}{1.994613in}}%
\pgfpathcurveto{\pgfqpoint{2.996254in}{1.994613in}}{\pgfqpoint{3.004734in}{1.998125in}}{\pgfqpoint{3.010985in}{2.004376in}}%
\pgfpathcurveto{\pgfqpoint{3.017235in}{2.010627in}}{\pgfqpoint{3.020748in}{2.019106in}}{\pgfqpoint{3.020748in}{2.027947in}}%
\pgfpathcurveto{\pgfqpoint{3.020748in}{2.036787in}}{\pgfqpoint{3.017235in}{2.045266in}}{\pgfqpoint{3.010985in}{2.051517in}}%
\pgfpathcurveto{\pgfqpoint{3.004734in}{2.057768in}}{\pgfqpoint{2.996254in}{2.061280in}}{\pgfqpoint{2.987414in}{2.061280in}}%
\pgfpathcurveto{\pgfqpoint{2.978574in}{2.061280in}}{\pgfqpoint{2.970095in}{2.057768in}}{\pgfqpoint{2.963844in}{2.051517in}}%
\pgfpathcurveto{\pgfqpoint{2.957593in}{2.045266in}}{\pgfqpoint{2.954081in}{2.036787in}}{\pgfqpoint{2.954081in}{2.027947in}}%
\pgfpathcurveto{\pgfqpoint{2.954081in}{2.019106in}}{\pgfqpoint{2.957593in}{2.010627in}}{\pgfqpoint{2.963844in}{2.004376in}}%
\pgfpathcurveto{\pgfqpoint{2.970095in}{1.998125in}}{\pgfqpoint{2.978574in}{1.994613in}}{\pgfqpoint{2.987414in}{1.994613in}}%
\pgfpathclose%
\pgfusepath{stroke,fill}%
\end{pgfscope}%
\begin{pgfscope}%
\pgfpathrectangle{\pgfqpoint{0.712202in}{0.541145in}}{\pgfqpoint{2.383555in}{1.869134in}}%
\pgfusepath{clip}%
\pgfsetbuttcap%
\pgfsetroundjoin%
\definecolor{currentfill}{rgb}{0.768627,0.305882,0.321569}%
\pgfsetfillcolor{currentfill}%
\pgfsetlinewidth{0.803000pt}%
\definecolor{currentstroke}{rgb}{0.768627,0.305882,0.321569}%
\pgfsetstrokecolor{currentstroke}%
\pgfsetdash{}{0pt}%
\pgfsys@defobject{currentmarker}{\pgfqpoint{-0.033333in}{-0.033333in}}{\pgfqpoint{0.033333in}{0.033333in}}{%
\pgfpathmoveto{\pgfqpoint{0.000000in}{-0.033333in}}%
\pgfpathcurveto{\pgfqpoint{0.008840in}{-0.033333in}}{\pgfqpoint{0.017319in}{-0.029821in}}{\pgfqpoint{0.023570in}{-0.023570in}}%
\pgfpathcurveto{\pgfqpoint{0.029821in}{-0.017319in}}{\pgfqpoint{0.033333in}{-0.008840in}}{\pgfqpoint{0.033333in}{0.000000in}}%
\pgfpathcurveto{\pgfqpoint{0.033333in}{0.008840in}}{\pgfqpoint{0.029821in}{0.017319in}}{\pgfqpoint{0.023570in}{0.023570in}}%
\pgfpathcurveto{\pgfqpoint{0.017319in}{0.029821in}}{\pgfqpoint{0.008840in}{0.033333in}}{\pgfqpoint{0.000000in}{0.033333in}}%
\pgfpathcurveto{\pgfqpoint{-0.008840in}{0.033333in}}{\pgfqpoint{-0.017319in}{0.029821in}}{\pgfqpoint{-0.023570in}{0.023570in}}%
\pgfpathcurveto{\pgfqpoint{-0.029821in}{0.017319in}}{\pgfqpoint{-0.033333in}{0.008840in}}{\pgfqpoint{-0.033333in}{0.000000in}}%
\pgfpathcurveto{\pgfqpoint{-0.033333in}{-0.008840in}}{\pgfqpoint{-0.029821in}{-0.017319in}}{\pgfqpoint{-0.023570in}{-0.023570in}}%
\pgfpathcurveto{\pgfqpoint{-0.017319in}{-0.029821in}}{\pgfqpoint{-0.008840in}{-0.033333in}}{\pgfqpoint{0.000000in}{-0.033333in}}%
\pgfpathclose%
\pgfusepath{stroke,fill}%
}%
\end{pgfscope}%
\begin{pgfscope}%
\pgfpathrectangle{\pgfqpoint{0.712202in}{0.541145in}}{\pgfqpoint{2.383555in}{1.869134in}}%
\pgfusepath{clip}%
\pgfsetbuttcap%
\pgfsetroundjoin%
\definecolor{currentfill}{rgb}{0.298039,0.447059,0.690196}%
\pgfsetfillcolor{currentfill}%
\pgfsetlinewidth{0.803000pt}%
\definecolor{currentstroke}{rgb}{0.298039,0.447059,0.690196}%
\pgfsetstrokecolor{currentstroke}%
\pgfsetdash{}{0pt}%
\pgfsys@defobject{currentmarker}{\pgfqpoint{-0.033333in}{-0.033333in}}{\pgfqpoint{0.033333in}{0.033333in}}{%
\pgfpathmoveto{\pgfqpoint{0.000000in}{-0.033333in}}%
\pgfpathcurveto{\pgfqpoint{0.008840in}{-0.033333in}}{\pgfqpoint{0.017319in}{-0.029821in}}{\pgfqpoint{0.023570in}{-0.023570in}}%
\pgfpathcurveto{\pgfqpoint{0.029821in}{-0.017319in}}{\pgfqpoint{0.033333in}{-0.008840in}}{\pgfqpoint{0.033333in}{0.000000in}}%
\pgfpathcurveto{\pgfqpoint{0.033333in}{0.008840in}}{\pgfqpoint{0.029821in}{0.017319in}}{\pgfqpoint{0.023570in}{0.023570in}}%
\pgfpathcurveto{\pgfqpoint{0.017319in}{0.029821in}}{\pgfqpoint{0.008840in}{0.033333in}}{\pgfqpoint{0.000000in}{0.033333in}}%
\pgfpathcurveto{\pgfqpoint{-0.008840in}{0.033333in}}{\pgfqpoint{-0.017319in}{0.029821in}}{\pgfqpoint{-0.023570in}{0.023570in}}%
\pgfpathcurveto{\pgfqpoint{-0.029821in}{0.017319in}}{\pgfqpoint{-0.033333in}{0.008840in}}{\pgfqpoint{-0.033333in}{0.000000in}}%
\pgfpathcurveto{\pgfqpoint{-0.033333in}{-0.008840in}}{\pgfqpoint{-0.029821in}{-0.017319in}}{\pgfqpoint{-0.023570in}{-0.023570in}}%
\pgfpathcurveto{\pgfqpoint{-0.017319in}{-0.029821in}}{\pgfqpoint{-0.008840in}{-0.033333in}}{\pgfqpoint{0.000000in}{-0.033333in}}%
\pgfpathclose%
\pgfusepath{stroke,fill}%
}%
\end{pgfscope}%
\begin{pgfscope}%
\pgfpathrectangle{\pgfqpoint{0.712202in}{0.541145in}}{\pgfqpoint{2.383555in}{1.869134in}}%
\pgfusepath{clip}%
\pgfsetbuttcap%
\pgfsetroundjoin%
\pgfsetlinewidth{2.007500pt}%
\definecolor{currentstroke}{rgb}{0.768627,0.305882,0.321569}%
\pgfsetstrokecolor{currentstroke}%
\pgfsetdash{{7.400000pt}{3.200000pt}}{0.000000pt}%
\pgfpathmoveto{\pgfqpoint{0.820546in}{1.429037in}}%
\pgfpathlineto{\pgfqpoint{1.903980in}{1.849481in}}%
\pgfpathlineto{\pgfqpoint{2.987414in}{2.325319in}}%
\pgfusepath{stroke}%
\end{pgfscope}%
\begin{pgfscope}%
\pgfpathrectangle{\pgfqpoint{0.712202in}{0.541145in}}{\pgfqpoint{2.383555in}{1.869134in}}%
\pgfusepath{clip}%
\pgfsetbuttcap%
\pgfsetroundjoin%
\pgfsetlinewidth{2.007500pt}%
\definecolor{currentstroke}{rgb}{0.298039,0.447059,0.690196}%
\pgfsetstrokecolor{currentstroke}%
\pgfsetdash{{7.400000pt}{3.200000pt}}{0.000000pt}%
\pgfpathmoveto{\pgfqpoint{0.820546in}{0.626106in}}%
\pgfpathlineto{\pgfqpoint{1.903980in}{1.446208in}}%
\pgfpathlineto{\pgfqpoint{2.987414in}{2.027947in}}%
\pgfusepath{stroke}%
\end{pgfscope}%
\begin{pgfscope}%
\pgfsetrectcap%
\pgfsetmiterjoin%
\pgfsetlinewidth{1.003750pt}%
\definecolor{currentstroke}{rgb}{0.150000,0.150000,0.150000}%
\pgfsetstrokecolor{currentstroke}%
\pgfsetdash{}{0pt}%
\pgfpathmoveto{\pgfqpoint{0.712202in}{0.541145in}}%
\pgfpathlineto{\pgfqpoint{0.712202in}{2.410280in}}%
\pgfusepath{stroke}%
\end{pgfscope}%
\begin{pgfscope}%
\pgfsetrectcap%
\pgfsetmiterjoin%
\pgfsetlinewidth{1.003750pt}%
\definecolor{currentstroke}{rgb}{0.150000,0.150000,0.150000}%
\pgfsetstrokecolor{currentstroke}%
\pgfsetdash{}{0pt}%
\pgfpathmoveto{\pgfqpoint{0.712202in}{0.541145in}}%
\pgfpathlineto{\pgfqpoint{3.095758in}{0.541145in}}%
\pgfusepath{stroke}%
\end{pgfscope}%
\begin{pgfscope}%
\definecolor{textcolor}{rgb}{0.150000,0.150000,0.150000}%
\pgfsetstrokecolor{textcolor}%
\pgfsetfillcolor{textcolor}%
\pgftext[x=1.903980in,y=2.493613in,,base]{\color{textcolor}\rmfamily\fontsize{9.600000}{11.520000}\selectfont \(\displaystyle U_3(z)\)}%
\end{pgfscope}%
\begin{pgfscope}%
\pgfsetbuttcap%
\pgfsetmiterjoin%
\definecolor{currentfill}{rgb}{1.000000,1.000000,1.000000}%
\pgfsetfillcolor{currentfill}%
\pgfsetlinewidth{0.000000pt}%
\definecolor{currentstroke}{rgb}{0.000000,0.000000,0.000000}%
\pgfsetstrokecolor{currentstroke}%
\pgfsetstrokeopacity{0.000000}%
\pgfsetdash{}{0pt}%
\pgfpathmoveto{\pgfqpoint{3.548289in}{0.541145in}}%
\pgfpathlineto{\pgfqpoint{5.931844in}{0.541145in}}%
\pgfpathlineto{\pgfqpoint{5.931844in}{2.410280in}}%
\pgfpathlineto{\pgfqpoint{3.548289in}{2.410280in}}%
\pgfpathclose%
\pgfusepath{fill}%
\end{pgfscope}%
\begin{pgfscope}%
\pgfsetbuttcap%
\pgfsetroundjoin%
\definecolor{currentfill}{rgb}{0.150000,0.150000,0.150000}%
\pgfsetfillcolor{currentfill}%
\pgfsetlinewidth{1.003750pt}%
\definecolor{currentstroke}{rgb}{0.150000,0.150000,0.150000}%
\pgfsetstrokecolor{currentstroke}%
\pgfsetdash{}{0pt}%
\pgfsys@defobject{currentmarker}{\pgfqpoint{0.000000in}{-0.066667in}}{\pgfqpoint{0.000000in}{0.000000in}}{%
\pgfpathmoveto{\pgfqpoint{0.000000in}{0.000000in}}%
\pgfpathlineto{\pgfqpoint{0.000000in}{-0.066667in}}%
\pgfusepath{stroke,fill}%
}%
\begin{pgfscope}%
\pgfsys@transformshift{3.656632in}{0.541145in}%
\pgfsys@useobject{currentmarker}{}%
\end{pgfscope}%
\end{pgfscope}%
\begin{pgfscope}%
\definecolor{textcolor}{rgb}{0.150000,0.150000,0.150000}%
\pgfsetstrokecolor{textcolor}%
\pgfsetfillcolor{textcolor}%
\pgftext[x=3.656632in,y=0.425867in,,top]{\color{textcolor}\rmfamily\fontsize{8.800000}{10.560000}\selectfont 2}%
\end{pgfscope}%
\begin{pgfscope}%
\pgfsetbuttcap%
\pgfsetroundjoin%
\definecolor{currentfill}{rgb}{0.150000,0.150000,0.150000}%
\pgfsetfillcolor{currentfill}%
\pgfsetlinewidth{1.003750pt}%
\definecolor{currentstroke}{rgb}{0.150000,0.150000,0.150000}%
\pgfsetstrokecolor{currentstroke}%
\pgfsetdash{}{0pt}%
\pgfsys@defobject{currentmarker}{\pgfqpoint{0.000000in}{-0.066667in}}{\pgfqpoint{0.000000in}{0.000000in}}{%
\pgfpathmoveto{\pgfqpoint{0.000000in}{0.000000in}}%
\pgfpathlineto{\pgfqpoint{0.000000in}{-0.066667in}}%
\pgfusepath{stroke,fill}%
}%
\begin{pgfscope}%
\pgfsys@transformshift{4.740066in}{0.541145in}%
\pgfsys@useobject{currentmarker}{}%
\end{pgfscope}%
\end{pgfscope}%
\begin{pgfscope}%
\definecolor{textcolor}{rgb}{0.150000,0.150000,0.150000}%
\pgfsetstrokecolor{textcolor}%
\pgfsetfillcolor{textcolor}%
\pgftext[x=4.740066in,y=0.425867in,,top]{\color{textcolor}\rmfamily\fontsize{8.800000}{10.560000}\selectfont 8}%
\end{pgfscope}%
\begin{pgfscope}%
\pgfsetbuttcap%
\pgfsetroundjoin%
\definecolor{currentfill}{rgb}{0.150000,0.150000,0.150000}%
\pgfsetfillcolor{currentfill}%
\pgfsetlinewidth{1.003750pt}%
\definecolor{currentstroke}{rgb}{0.150000,0.150000,0.150000}%
\pgfsetstrokecolor{currentstroke}%
\pgfsetdash{}{0pt}%
\pgfsys@defobject{currentmarker}{\pgfqpoint{0.000000in}{-0.066667in}}{\pgfqpoint{0.000000in}{0.000000in}}{%
\pgfpathmoveto{\pgfqpoint{0.000000in}{0.000000in}}%
\pgfpathlineto{\pgfqpoint{0.000000in}{-0.066667in}}%
\pgfusepath{stroke,fill}%
}%
\begin{pgfscope}%
\pgfsys@transformshift{5.823501in}{0.541145in}%
\pgfsys@useobject{currentmarker}{}%
\end{pgfscope}%
\end{pgfscope}%
\begin{pgfscope}%
\definecolor{textcolor}{rgb}{0.150000,0.150000,0.150000}%
\pgfsetstrokecolor{textcolor}%
\pgfsetfillcolor{textcolor}%
\pgftext[x=5.823501in,y=0.425867in,,top]{\color{textcolor}\rmfamily\fontsize{8.800000}{10.560000}\selectfont 32}%
\end{pgfscope}%
\begin{pgfscope}%
\definecolor{textcolor}{rgb}{0.150000,0.150000,0.150000}%
\pgfsetstrokecolor{textcolor}%
\pgfsetfillcolor{textcolor}%
\pgftext[x=4.740066in,y=0.259201in,,top]{\color{textcolor}\rmfamily\fontsize{9.600000}{11.520000}\selectfont Number of flows}%
\end{pgfscope}%
\begin{pgfscope}%
\pgfsetbuttcap%
\pgfsetroundjoin%
\definecolor{currentfill}{rgb}{0.150000,0.150000,0.150000}%
\pgfsetfillcolor{currentfill}%
\pgfsetlinewidth{1.003750pt}%
\definecolor{currentstroke}{rgb}{0.150000,0.150000,0.150000}%
\pgfsetstrokecolor{currentstroke}%
\pgfsetdash{}{0pt}%
\pgfsys@defobject{currentmarker}{\pgfqpoint{-0.066667in}{0.000000in}}{\pgfqpoint{-0.000000in}{0.000000in}}{%
\pgfpathmoveto{\pgfqpoint{-0.000000in}{0.000000in}}%
\pgfpathlineto{\pgfqpoint{-0.066667in}{0.000000in}}%
\pgfusepath{stroke,fill}%
}%
\begin{pgfscope}%
\pgfsys@transformshift{3.548289in}{0.856782in}%
\pgfsys@useobject{currentmarker}{}%
\end{pgfscope}%
\end{pgfscope}%
\begin{pgfscope}%
\definecolor{textcolor}{rgb}{0.150000,0.150000,0.150000}%
\pgfsetstrokecolor{textcolor}%
\pgfsetfillcolor{textcolor}%
\pgftext[x=3.240304in, y=0.813380in, left, base]{\color{textcolor}\rmfamily\fontsize{8.800000}{10.560000}\selectfont \(\displaystyle {300}\)}%
\end{pgfscope}%
\begin{pgfscope}%
\pgfsetbuttcap%
\pgfsetroundjoin%
\definecolor{currentfill}{rgb}{0.150000,0.150000,0.150000}%
\pgfsetfillcolor{currentfill}%
\pgfsetlinewidth{1.003750pt}%
\definecolor{currentstroke}{rgb}{0.150000,0.150000,0.150000}%
\pgfsetstrokecolor{currentstroke}%
\pgfsetdash{}{0pt}%
\pgfsys@defobject{currentmarker}{\pgfqpoint{-0.066667in}{0.000000in}}{\pgfqpoint{-0.000000in}{0.000000in}}{%
\pgfpathmoveto{\pgfqpoint{-0.000000in}{0.000000in}}%
\pgfpathlineto{\pgfqpoint{-0.066667in}{0.000000in}}%
\pgfusepath{stroke,fill}%
}%
\begin{pgfscope}%
\pgfsys@transformshift{3.548289in}{1.182576in}%
\pgfsys@useobject{currentmarker}{}%
\end{pgfscope}%
\end{pgfscope}%
\begin{pgfscope}%
\definecolor{textcolor}{rgb}{0.150000,0.150000,0.150000}%
\pgfsetstrokecolor{textcolor}%
\pgfsetfillcolor{textcolor}%
\pgftext[x=3.240304in, y=1.139173in, left, base]{\color{textcolor}\rmfamily\fontsize{8.800000}{10.560000}\selectfont \(\displaystyle {400}\)}%
\end{pgfscope}%
\begin{pgfscope}%
\pgfsetbuttcap%
\pgfsetroundjoin%
\definecolor{currentfill}{rgb}{0.150000,0.150000,0.150000}%
\pgfsetfillcolor{currentfill}%
\pgfsetlinewidth{1.003750pt}%
\definecolor{currentstroke}{rgb}{0.150000,0.150000,0.150000}%
\pgfsetstrokecolor{currentstroke}%
\pgfsetdash{}{0pt}%
\pgfsys@defobject{currentmarker}{\pgfqpoint{-0.066667in}{0.000000in}}{\pgfqpoint{-0.000000in}{0.000000in}}{%
\pgfpathmoveto{\pgfqpoint{-0.000000in}{0.000000in}}%
\pgfpathlineto{\pgfqpoint{-0.066667in}{0.000000in}}%
\pgfusepath{stroke,fill}%
}%
\begin{pgfscope}%
\pgfsys@transformshift{3.548289in}{1.508369in}%
\pgfsys@useobject{currentmarker}{}%
\end{pgfscope}%
\end{pgfscope}%
\begin{pgfscope}%
\definecolor{textcolor}{rgb}{0.150000,0.150000,0.150000}%
\pgfsetstrokecolor{textcolor}%
\pgfsetfillcolor{textcolor}%
\pgftext[x=3.240304in, y=1.464966in, left, base]{\color{textcolor}\rmfamily\fontsize{8.800000}{10.560000}\selectfont \(\displaystyle {500}\)}%
\end{pgfscope}%
\begin{pgfscope}%
\pgfsetbuttcap%
\pgfsetroundjoin%
\definecolor{currentfill}{rgb}{0.150000,0.150000,0.150000}%
\pgfsetfillcolor{currentfill}%
\pgfsetlinewidth{1.003750pt}%
\definecolor{currentstroke}{rgb}{0.150000,0.150000,0.150000}%
\pgfsetstrokecolor{currentstroke}%
\pgfsetdash{}{0pt}%
\pgfsys@defobject{currentmarker}{\pgfqpoint{-0.066667in}{0.000000in}}{\pgfqpoint{-0.000000in}{0.000000in}}{%
\pgfpathmoveto{\pgfqpoint{-0.000000in}{0.000000in}}%
\pgfpathlineto{\pgfqpoint{-0.066667in}{0.000000in}}%
\pgfusepath{stroke,fill}%
}%
\begin{pgfscope}%
\pgfsys@transformshift{3.548289in}{1.834163in}%
\pgfsys@useobject{currentmarker}{}%
\end{pgfscope}%
\end{pgfscope}%
\begin{pgfscope}%
\definecolor{textcolor}{rgb}{0.150000,0.150000,0.150000}%
\pgfsetstrokecolor{textcolor}%
\pgfsetfillcolor{textcolor}%
\pgftext[x=3.240304in, y=1.790760in, left, base]{\color{textcolor}\rmfamily\fontsize{8.800000}{10.560000}\selectfont \(\displaystyle {600}\)}%
\end{pgfscope}%
\begin{pgfscope}%
\pgfsetbuttcap%
\pgfsetroundjoin%
\definecolor{currentfill}{rgb}{0.150000,0.150000,0.150000}%
\pgfsetfillcolor{currentfill}%
\pgfsetlinewidth{1.003750pt}%
\definecolor{currentstroke}{rgb}{0.150000,0.150000,0.150000}%
\pgfsetstrokecolor{currentstroke}%
\pgfsetdash{}{0pt}%
\pgfsys@defobject{currentmarker}{\pgfqpoint{-0.066667in}{0.000000in}}{\pgfqpoint{-0.000000in}{0.000000in}}{%
\pgfpathmoveto{\pgfqpoint{-0.000000in}{0.000000in}}%
\pgfpathlineto{\pgfqpoint{-0.066667in}{0.000000in}}%
\pgfusepath{stroke,fill}%
}%
\begin{pgfscope}%
\pgfsys@transformshift{3.548289in}{2.159956in}%
\pgfsys@useobject{currentmarker}{}%
\end{pgfscope}%
\end{pgfscope}%
\begin{pgfscope}%
\definecolor{textcolor}{rgb}{0.150000,0.150000,0.150000}%
\pgfsetstrokecolor{textcolor}%
\pgfsetfillcolor{textcolor}%
\pgftext[x=3.240304in, y=2.116553in, left, base]{\color{textcolor}\rmfamily\fontsize{8.800000}{10.560000}\selectfont \(\displaystyle {700}\)}%
\end{pgfscope}%
\begin{pgfscope}%
\pgfpathrectangle{\pgfqpoint{3.548289in}{0.541145in}}{\pgfqpoint{2.383555in}{1.869134in}}%
\pgfusepath{clip}%
\pgfsetbuttcap%
\pgfsetroundjoin%
\definecolor{currentfill}{rgb}{0.768627,0.305882,0.321569}%
\pgfsetfillcolor{currentfill}%
\pgfsetlinewidth{0.385440pt}%
\definecolor{currentstroke}{rgb}{1.000000,1.000000,1.000000}%
\pgfsetstrokecolor{currentstroke}%
\pgfsetdash{}{0pt}%
\pgfpathmoveto{\pgfqpoint{3.656632in}{1.389855in}}%
\pgfpathcurveto{\pgfqpoint{3.665472in}{1.389855in}}{\pgfqpoint{3.673951in}{1.393367in}}{\pgfqpoint{3.680202in}{1.399618in}}%
\pgfpathcurveto{\pgfqpoint{3.686453in}{1.405869in}}{\pgfqpoint{3.689965in}{1.414348in}}{\pgfqpoint{3.689965in}{1.423188in}}%
\pgfpathcurveto{\pgfqpoint{3.689965in}{1.432028in}}{\pgfqpoint{3.686453in}{1.440508in}}{\pgfqpoint{3.680202in}{1.446759in}}%
\pgfpathcurveto{\pgfqpoint{3.673951in}{1.453009in}}{\pgfqpoint{3.665472in}{1.456522in}}{\pgfqpoint{3.656632in}{1.456522in}}%
\pgfpathcurveto{\pgfqpoint{3.647792in}{1.456522in}}{\pgfqpoint{3.639313in}{1.453009in}}{\pgfqpoint{3.633062in}{1.446759in}}%
\pgfpathcurveto{\pgfqpoint{3.626811in}{1.440508in}}{\pgfqpoint{3.623299in}{1.432028in}}{\pgfqpoint{3.623299in}{1.423188in}}%
\pgfpathcurveto{\pgfqpoint{3.623299in}{1.414348in}}{\pgfqpoint{3.626811in}{1.405869in}}{\pgfqpoint{3.633062in}{1.399618in}}%
\pgfpathcurveto{\pgfqpoint{3.639313in}{1.393367in}}{\pgfqpoint{3.647792in}{1.389855in}}{\pgfqpoint{3.656632in}{1.389855in}}%
\pgfpathclose%
\pgfusepath{stroke,fill}%
\end{pgfscope}%
\begin{pgfscope}%
\pgfpathrectangle{\pgfqpoint{3.548289in}{0.541145in}}{\pgfqpoint{2.383555in}{1.869134in}}%
\pgfusepath{clip}%
\pgfsetbuttcap%
\pgfsetroundjoin%
\definecolor{currentfill}{rgb}{0.768627,0.305882,0.321569}%
\pgfsetfillcolor{currentfill}%
\pgfsetlinewidth{0.385440pt}%
\definecolor{currentstroke}{rgb}{1.000000,1.000000,1.000000}%
\pgfsetstrokecolor{currentstroke}%
\pgfsetdash{}{0pt}%
\pgfpathmoveto{\pgfqpoint{4.740066in}{1.925894in}}%
\pgfpathcurveto{\pgfqpoint{4.748906in}{1.925894in}}{\pgfqpoint{4.757386in}{1.929407in}}{\pgfqpoint{4.763637in}{1.935658in}}%
\pgfpathcurveto{\pgfqpoint{4.769887in}{1.941908in}}{\pgfqpoint{4.773400in}{1.950388in}}{\pgfqpoint{4.773400in}{1.959228in}}%
\pgfpathcurveto{\pgfqpoint{4.773400in}{1.968068in}}{\pgfqpoint{4.769887in}{1.976547in}}{\pgfqpoint{4.763637in}{1.982798in}}%
\pgfpathcurveto{\pgfqpoint{4.757386in}{1.989049in}}{\pgfqpoint{4.748906in}{1.992561in}}{\pgfqpoint{4.740066in}{1.992561in}}%
\pgfpathcurveto{\pgfqpoint{4.731226in}{1.992561in}}{\pgfqpoint{4.722747in}{1.989049in}}{\pgfqpoint{4.716496in}{1.982798in}}%
\pgfpathcurveto{\pgfqpoint{4.710245in}{1.976547in}}{\pgfqpoint{4.706733in}{1.968068in}}{\pgfqpoint{4.706733in}{1.959228in}}%
\pgfpathcurveto{\pgfqpoint{4.706733in}{1.950388in}}{\pgfqpoint{4.710245in}{1.941908in}}{\pgfqpoint{4.716496in}{1.935658in}}%
\pgfpathcurveto{\pgfqpoint{4.722747in}{1.929407in}}{\pgfqpoint{4.731226in}{1.925894in}}{\pgfqpoint{4.740066in}{1.925894in}}%
\pgfpathclose%
\pgfusepath{stroke,fill}%
\end{pgfscope}%
\begin{pgfscope}%
\pgfpathrectangle{\pgfqpoint{3.548289in}{0.541145in}}{\pgfqpoint{2.383555in}{1.869134in}}%
\pgfusepath{clip}%
\pgfsetbuttcap%
\pgfsetroundjoin%
\definecolor{currentfill}{rgb}{0.768627,0.305882,0.321569}%
\pgfsetfillcolor{currentfill}%
\pgfsetlinewidth{0.385440pt}%
\definecolor{currentstroke}{rgb}{1.000000,1.000000,1.000000}%
\pgfsetstrokecolor{currentstroke}%
\pgfsetdash{}{0pt}%
\pgfpathmoveto{\pgfqpoint{5.823501in}{2.291986in}}%
\pgfpathcurveto{\pgfqpoint{5.832341in}{2.291986in}}{\pgfqpoint{5.840820in}{2.295498in}}{\pgfqpoint{5.847071in}{2.301749in}}%
\pgfpathcurveto{\pgfqpoint{5.853322in}{2.308000in}}{\pgfqpoint{5.856834in}{2.316479in}}{\pgfqpoint{5.856834in}{2.325319in}}%
\pgfpathcurveto{\pgfqpoint{5.856834in}{2.334159in}}{\pgfqpoint{5.853322in}{2.342638in}}{\pgfqpoint{5.847071in}{2.348889in}}%
\pgfpathcurveto{\pgfqpoint{5.840820in}{2.355140in}}{\pgfqpoint{5.832341in}{2.358652in}}{\pgfqpoint{5.823501in}{2.358652in}}%
\pgfpathcurveto{\pgfqpoint{5.814660in}{2.358652in}}{\pgfqpoint{5.806181in}{2.355140in}}{\pgfqpoint{5.799930in}{2.348889in}}%
\pgfpathcurveto{\pgfqpoint{5.793679in}{2.342638in}}{\pgfqpoint{5.790167in}{2.334159in}}{\pgfqpoint{5.790167in}{2.325319in}}%
\pgfpathcurveto{\pgfqpoint{5.790167in}{2.316479in}}{\pgfqpoint{5.793679in}{2.308000in}}{\pgfqpoint{5.799930in}{2.301749in}}%
\pgfpathcurveto{\pgfqpoint{5.806181in}{2.295498in}}{\pgfqpoint{5.814660in}{2.291986in}}{\pgfqpoint{5.823501in}{2.291986in}}%
\pgfpathclose%
\pgfusepath{stroke,fill}%
\end{pgfscope}%
\begin{pgfscope}%
\pgfpathrectangle{\pgfqpoint{3.548289in}{0.541145in}}{\pgfqpoint{2.383555in}{1.869134in}}%
\pgfusepath{clip}%
\pgfsetbuttcap%
\pgfsetroundjoin%
\definecolor{currentfill}{rgb}{0.298039,0.447059,0.690196}%
\pgfsetfillcolor{currentfill}%
\pgfsetlinewidth{0.385440pt}%
\definecolor{currentstroke}{rgb}{1.000000,1.000000,1.000000}%
\pgfsetstrokecolor{currentstroke}%
\pgfsetdash{}{0pt}%
\pgfpathmoveto{\pgfqpoint{3.656632in}{0.592773in}}%
\pgfpathcurveto{\pgfqpoint{3.665472in}{0.592773in}}{\pgfqpoint{3.673951in}{0.596285in}}{\pgfqpoint{3.680202in}{0.602536in}}%
\pgfpathcurveto{\pgfqpoint{3.686453in}{0.608787in}}{\pgfqpoint{3.689965in}{0.617266in}}{\pgfqpoint{3.689965in}{0.626106in}}%
\pgfpathcurveto{\pgfqpoint{3.689965in}{0.634946in}}{\pgfqpoint{3.686453in}{0.643425in}}{\pgfqpoint{3.680202in}{0.649676in}}%
\pgfpathcurveto{\pgfqpoint{3.673951in}{0.655927in}}{\pgfqpoint{3.665472in}{0.659439in}}{\pgfqpoint{3.656632in}{0.659439in}}%
\pgfpathcurveto{\pgfqpoint{3.647792in}{0.659439in}}{\pgfqpoint{3.639313in}{0.655927in}}{\pgfqpoint{3.633062in}{0.649676in}}%
\pgfpathcurveto{\pgfqpoint{3.626811in}{0.643425in}}{\pgfqpoint{3.623299in}{0.634946in}}{\pgfqpoint{3.623299in}{0.626106in}}%
\pgfpathcurveto{\pgfqpoint{3.623299in}{0.617266in}}{\pgfqpoint{3.626811in}{0.608787in}}{\pgfqpoint{3.633062in}{0.602536in}}%
\pgfpathcurveto{\pgfqpoint{3.639313in}{0.596285in}}{\pgfqpoint{3.647792in}{0.592773in}}{\pgfqpoint{3.656632in}{0.592773in}}%
\pgfpathclose%
\pgfusepath{stroke,fill}%
\end{pgfscope}%
\begin{pgfscope}%
\pgfpathrectangle{\pgfqpoint{3.548289in}{0.541145in}}{\pgfqpoint{2.383555in}{1.869134in}}%
\pgfusepath{clip}%
\pgfsetbuttcap%
\pgfsetroundjoin%
\definecolor{currentfill}{rgb}{0.298039,0.447059,0.690196}%
\pgfsetfillcolor{currentfill}%
\pgfsetlinewidth{0.385440pt}%
\definecolor{currentstroke}{rgb}{1.000000,1.000000,1.000000}%
\pgfsetstrokecolor{currentstroke}%
\pgfsetdash{}{0pt}%
\pgfpathmoveto{\pgfqpoint{4.740066in}{1.352397in}}%
\pgfpathcurveto{\pgfqpoint{4.748906in}{1.352397in}}{\pgfqpoint{4.757386in}{1.355910in}}{\pgfqpoint{4.763637in}{1.362161in}}%
\pgfpathcurveto{\pgfqpoint{4.769887in}{1.368411in}}{\pgfqpoint{4.773400in}{1.376891in}}{\pgfqpoint{4.773400in}{1.385731in}}%
\pgfpathcurveto{\pgfqpoint{4.773400in}{1.394571in}}{\pgfqpoint{4.769887in}{1.403050in}}{\pgfqpoint{4.763637in}{1.409301in}}%
\pgfpathcurveto{\pgfqpoint{4.757386in}{1.415552in}}{\pgfqpoint{4.748906in}{1.419064in}}{\pgfqpoint{4.740066in}{1.419064in}}%
\pgfpathcurveto{\pgfqpoint{4.731226in}{1.419064in}}{\pgfqpoint{4.722747in}{1.415552in}}{\pgfqpoint{4.716496in}{1.409301in}}%
\pgfpathcurveto{\pgfqpoint{4.710245in}{1.403050in}}{\pgfqpoint{4.706733in}{1.394571in}}{\pgfqpoint{4.706733in}{1.385731in}}%
\pgfpathcurveto{\pgfqpoint{4.706733in}{1.376891in}}{\pgfqpoint{4.710245in}{1.368411in}}{\pgfqpoint{4.716496in}{1.362161in}}%
\pgfpathcurveto{\pgfqpoint{4.722747in}{1.355910in}}{\pgfqpoint{4.731226in}{1.352397in}}{\pgfqpoint{4.740066in}{1.352397in}}%
\pgfpathclose%
\pgfusepath{stroke,fill}%
\end{pgfscope}%
\begin{pgfscope}%
\pgfpathrectangle{\pgfqpoint{3.548289in}{0.541145in}}{\pgfqpoint{2.383555in}{1.869134in}}%
\pgfusepath{clip}%
\pgfsetbuttcap%
\pgfsetroundjoin%
\definecolor{currentfill}{rgb}{0.298039,0.447059,0.690196}%
\pgfsetfillcolor{currentfill}%
\pgfsetlinewidth{0.385440pt}%
\definecolor{currentstroke}{rgb}{1.000000,1.000000,1.000000}%
\pgfsetstrokecolor{currentstroke}%
\pgfsetdash{}{0pt}%
\pgfpathmoveto{\pgfqpoint{5.823501in}{1.958225in}}%
\pgfpathcurveto{\pgfqpoint{5.832341in}{1.958225in}}{\pgfqpoint{5.840820in}{1.961737in}}{\pgfqpoint{5.847071in}{1.967988in}}%
\pgfpathcurveto{\pgfqpoint{5.853322in}{1.974239in}}{\pgfqpoint{5.856834in}{1.982718in}}{\pgfqpoint{5.856834in}{1.991558in}}%
\pgfpathcurveto{\pgfqpoint{5.856834in}{2.000398in}}{\pgfqpoint{5.853322in}{2.008877in}}{\pgfqpoint{5.847071in}{2.015128in}}%
\pgfpathcurveto{\pgfqpoint{5.840820in}{2.021379in}}{\pgfqpoint{5.832341in}{2.024891in}}{\pgfqpoint{5.823501in}{2.024891in}}%
\pgfpathcurveto{\pgfqpoint{5.814660in}{2.024891in}}{\pgfqpoint{5.806181in}{2.021379in}}{\pgfqpoint{5.799930in}{2.015128in}}%
\pgfpathcurveto{\pgfqpoint{5.793679in}{2.008877in}}{\pgfqpoint{5.790167in}{2.000398in}}{\pgfqpoint{5.790167in}{1.991558in}}%
\pgfpathcurveto{\pgfqpoint{5.790167in}{1.982718in}}{\pgfqpoint{5.793679in}{1.974239in}}{\pgfqpoint{5.799930in}{1.967988in}}%
\pgfpathcurveto{\pgfqpoint{5.806181in}{1.961737in}}{\pgfqpoint{5.814660in}{1.958225in}}{\pgfqpoint{5.823501in}{1.958225in}}%
\pgfpathclose%
\pgfusepath{stroke,fill}%
\end{pgfscope}%
\begin{pgfscope}%
\pgfpathrectangle{\pgfqpoint{3.548289in}{0.541145in}}{\pgfqpoint{2.383555in}{1.869134in}}%
\pgfusepath{clip}%
\pgfsetbuttcap%
\pgfsetroundjoin%
\definecolor{currentfill}{rgb}{0.768627,0.305882,0.321569}%
\pgfsetfillcolor{currentfill}%
\pgfsetlinewidth{0.803000pt}%
\definecolor{currentstroke}{rgb}{0.768627,0.305882,0.321569}%
\pgfsetstrokecolor{currentstroke}%
\pgfsetdash{}{0pt}%
\pgfsys@defobject{currentmarker}{\pgfqpoint{-0.033333in}{-0.033333in}}{\pgfqpoint{0.033333in}{0.033333in}}{%
\pgfpathmoveto{\pgfqpoint{0.000000in}{-0.033333in}}%
\pgfpathcurveto{\pgfqpoint{0.008840in}{-0.033333in}}{\pgfqpoint{0.017319in}{-0.029821in}}{\pgfqpoint{0.023570in}{-0.023570in}}%
\pgfpathcurveto{\pgfqpoint{0.029821in}{-0.017319in}}{\pgfqpoint{0.033333in}{-0.008840in}}{\pgfqpoint{0.033333in}{0.000000in}}%
\pgfpathcurveto{\pgfqpoint{0.033333in}{0.008840in}}{\pgfqpoint{0.029821in}{0.017319in}}{\pgfqpoint{0.023570in}{0.023570in}}%
\pgfpathcurveto{\pgfqpoint{0.017319in}{0.029821in}}{\pgfqpoint{0.008840in}{0.033333in}}{\pgfqpoint{0.000000in}{0.033333in}}%
\pgfpathcurveto{\pgfqpoint{-0.008840in}{0.033333in}}{\pgfqpoint{-0.017319in}{0.029821in}}{\pgfqpoint{-0.023570in}{0.023570in}}%
\pgfpathcurveto{\pgfqpoint{-0.029821in}{0.017319in}}{\pgfqpoint{-0.033333in}{0.008840in}}{\pgfqpoint{-0.033333in}{0.000000in}}%
\pgfpathcurveto{\pgfqpoint{-0.033333in}{-0.008840in}}{\pgfqpoint{-0.029821in}{-0.017319in}}{\pgfqpoint{-0.023570in}{-0.023570in}}%
\pgfpathcurveto{\pgfqpoint{-0.017319in}{-0.029821in}}{\pgfqpoint{-0.008840in}{-0.033333in}}{\pgfqpoint{0.000000in}{-0.033333in}}%
\pgfpathclose%
\pgfusepath{stroke,fill}%
}%
\end{pgfscope}%
\begin{pgfscope}%
\pgfpathrectangle{\pgfqpoint{3.548289in}{0.541145in}}{\pgfqpoint{2.383555in}{1.869134in}}%
\pgfusepath{clip}%
\pgfsetbuttcap%
\pgfsetroundjoin%
\definecolor{currentfill}{rgb}{0.298039,0.447059,0.690196}%
\pgfsetfillcolor{currentfill}%
\pgfsetlinewidth{0.803000pt}%
\definecolor{currentstroke}{rgb}{0.298039,0.447059,0.690196}%
\pgfsetstrokecolor{currentstroke}%
\pgfsetdash{}{0pt}%
\pgfsys@defobject{currentmarker}{\pgfqpoint{-0.033333in}{-0.033333in}}{\pgfqpoint{0.033333in}{0.033333in}}{%
\pgfpathmoveto{\pgfqpoint{0.000000in}{-0.033333in}}%
\pgfpathcurveto{\pgfqpoint{0.008840in}{-0.033333in}}{\pgfqpoint{0.017319in}{-0.029821in}}{\pgfqpoint{0.023570in}{-0.023570in}}%
\pgfpathcurveto{\pgfqpoint{0.029821in}{-0.017319in}}{\pgfqpoint{0.033333in}{-0.008840in}}{\pgfqpoint{0.033333in}{0.000000in}}%
\pgfpathcurveto{\pgfqpoint{0.033333in}{0.008840in}}{\pgfqpoint{0.029821in}{0.017319in}}{\pgfqpoint{0.023570in}{0.023570in}}%
\pgfpathcurveto{\pgfqpoint{0.017319in}{0.029821in}}{\pgfqpoint{0.008840in}{0.033333in}}{\pgfqpoint{0.000000in}{0.033333in}}%
\pgfpathcurveto{\pgfqpoint{-0.008840in}{0.033333in}}{\pgfqpoint{-0.017319in}{0.029821in}}{\pgfqpoint{-0.023570in}{0.023570in}}%
\pgfpathcurveto{\pgfqpoint{-0.029821in}{0.017319in}}{\pgfqpoint{-0.033333in}{0.008840in}}{\pgfqpoint{-0.033333in}{0.000000in}}%
\pgfpathcurveto{\pgfqpoint{-0.033333in}{-0.008840in}}{\pgfqpoint{-0.029821in}{-0.017319in}}{\pgfqpoint{-0.023570in}{-0.023570in}}%
\pgfpathcurveto{\pgfqpoint{-0.017319in}{-0.029821in}}{\pgfqpoint{-0.008840in}{-0.033333in}}{\pgfqpoint{0.000000in}{-0.033333in}}%
\pgfpathclose%
\pgfusepath{stroke,fill}%
}%
\end{pgfscope}%
\begin{pgfscope}%
\pgfpathrectangle{\pgfqpoint{3.548289in}{0.541145in}}{\pgfqpoint{2.383555in}{1.869134in}}%
\pgfusepath{clip}%
\pgfsetbuttcap%
\pgfsetroundjoin%
\pgfsetlinewidth{2.007500pt}%
\definecolor{currentstroke}{rgb}{0.768627,0.305882,0.321569}%
\pgfsetstrokecolor{currentstroke}%
\pgfsetdash{{7.400000pt}{3.200000pt}}{0.000000pt}%
\pgfpathmoveto{\pgfqpoint{3.656632in}{1.423188in}}%
\pgfpathlineto{\pgfqpoint{4.740066in}{1.959228in}}%
\pgfpathlineto{\pgfqpoint{5.823501in}{2.325319in}}%
\pgfusepath{stroke}%
\end{pgfscope}%
\begin{pgfscope}%
\pgfpathrectangle{\pgfqpoint{3.548289in}{0.541145in}}{\pgfqpoint{2.383555in}{1.869134in}}%
\pgfusepath{clip}%
\pgfsetbuttcap%
\pgfsetroundjoin%
\pgfsetlinewidth{2.007500pt}%
\definecolor{currentstroke}{rgb}{0.298039,0.447059,0.690196}%
\pgfsetstrokecolor{currentstroke}%
\pgfsetdash{{7.400000pt}{3.200000pt}}{0.000000pt}%
\pgfpathmoveto{\pgfqpoint{3.656632in}{0.626106in}}%
\pgfpathlineto{\pgfqpoint{4.740066in}{1.385731in}}%
\pgfpathlineto{\pgfqpoint{5.823501in}{1.991558in}}%
\pgfusepath{stroke}%
\end{pgfscope}%
\begin{pgfscope}%
\pgfsetrectcap%
\pgfsetmiterjoin%
\pgfsetlinewidth{1.003750pt}%
\definecolor{currentstroke}{rgb}{0.150000,0.150000,0.150000}%
\pgfsetstrokecolor{currentstroke}%
\pgfsetdash{}{0pt}%
\pgfpathmoveto{\pgfqpoint{3.548289in}{0.541145in}}%
\pgfpathlineto{\pgfqpoint{3.548289in}{2.410280in}}%
\pgfusepath{stroke}%
\end{pgfscope}%
\begin{pgfscope}%
\pgfsetrectcap%
\pgfsetmiterjoin%
\pgfsetlinewidth{1.003750pt}%
\definecolor{currentstroke}{rgb}{0.150000,0.150000,0.150000}%
\pgfsetstrokecolor{currentstroke}%
\pgfsetdash{}{0pt}%
\pgfpathmoveto{\pgfqpoint{3.548289in}{0.541145in}}%
\pgfpathlineto{\pgfqpoint{5.931844in}{0.541145in}}%
\pgfusepath{stroke}%
\end{pgfscope}%
\begin{pgfscope}%
\definecolor{textcolor}{rgb}{0.150000,0.150000,0.150000}%
\pgfsetstrokecolor{textcolor}%
\pgfsetfillcolor{textcolor}%
\pgftext[x=4.740066in,y=2.493613in,,base]{\color{textcolor}\rmfamily\fontsize{9.600000}{11.520000}\selectfont \(\displaystyle U_4(z)\)}%
\end{pgfscope}%
\begin{pgfscope}%
\pgfsetbuttcap%
\pgfsetroundjoin%
\definecolor{currentfill}{rgb}{0.768627,0.305882,0.321569}%
\pgfsetfillcolor{currentfill}%
\pgfsetlinewidth{0.803000pt}%
\definecolor{currentstroke}{rgb}{0.768627,0.305882,0.321569}%
\pgfsetstrokecolor{currentstroke}%
\pgfsetdash{}{0pt}%
\pgfsys@defobject{currentmarker}{\pgfqpoint{-0.033333in}{-0.033333in}}{\pgfqpoint{0.033333in}{0.033333in}}{%
\pgfpathmoveto{\pgfqpoint{0.000000in}{-0.033333in}}%
\pgfpathcurveto{\pgfqpoint{0.008840in}{-0.033333in}}{\pgfqpoint{0.017319in}{-0.029821in}}{\pgfqpoint{0.023570in}{-0.023570in}}%
\pgfpathcurveto{\pgfqpoint{0.029821in}{-0.017319in}}{\pgfqpoint{0.033333in}{-0.008840in}}{\pgfqpoint{0.033333in}{0.000000in}}%
\pgfpathcurveto{\pgfqpoint{0.033333in}{0.008840in}}{\pgfqpoint{0.029821in}{0.017319in}}{\pgfqpoint{0.023570in}{0.023570in}}%
\pgfpathcurveto{\pgfqpoint{0.017319in}{0.029821in}}{\pgfqpoint{0.008840in}{0.033333in}}{\pgfqpoint{0.000000in}{0.033333in}}%
\pgfpathcurveto{\pgfqpoint{-0.008840in}{0.033333in}}{\pgfqpoint{-0.017319in}{0.029821in}}{\pgfqpoint{-0.023570in}{0.023570in}}%
\pgfpathcurveto{\pgfqpoint{-0.029821in}{0.017319in}}{\pgfqpoint{-0.033333in}{0.008840in}}{\pgfqpoint{-0.033333in}{0.000000in}}%
\pgfpathcurveto{\pgfqpoint{-0.033333in}{-0.008840in}}{\pgfqpoint{-0.029821in}{-0.017319in}}{\pgfqpoint{-0.023570in}{-0.023570in}}%
\pgfpathcurveto{\pgfqpoint{-0.017319in}{-0.029821in}}{\pgfqpoint{-0.008840in}{-0.033333in}}{\pgfqpoint{0.000000in}{-0.033333in}}%
\pgfpathclose%
\pgfusepath{stroke,fill}%
}%
\begin{pgfscope}%
\pgfsys@transformshift{6.112394in}{2.586944in}%
\pgfsys@useobject{currentmarker}{}%
\end{pgfscope}%
\end{pgfscope}%
\begin{pgfscope}%
\definecolor{textcolor}{rgb}{0.150000,0.150000,0.150000}%
\pgfsetstrokecolor{textcolor}%
\pgfsetfillcolor{textcolor}%
\pgftext[x=6.332394in,y=2.554861in,left,base]{\color{textcolor}\rmfamily\fontsize{8.800000}{10.560000}\selectfont Planar}%
\end{pgfscope}%
\begin{pgfscope}%
\pgfsetbuttcap%
\pgfsetroundjoin%
\definecolor{currentfill}{rgb}{0.298039,0.447059,0.690196}%
\pgfsetfillcolor{currentfill}%
\pgfsetlinewidth{0.803000pt}%
\definecolor{currentstroke}{rgb}{0.298039,0.447059,0.690196}%
\pgfsetstrokecolor{currentstroke}%
\pgfsetdash{}{0pt}%
\pgfsys@defobject{currentmarker}{\pgfqpoint{-0.033333in}{-0.033333in}}{\pgfqpoint{0.033333in}{0.033333in}}{%
\pgfpathmoveto{\pgfqpoint{0.000000in}{-0.033333in}}%
\pgfpathcurveto{\pgfqpoint{0.008840in}{-0.033333in}}{\pgfqpoint{0.017319in}{-0.029821in}}{\pgfqpoint{0.023570in}{-0.023570in}}%
\pgfpathcurveto{\pgfqpoint{0.029821in}{-0.017319in}}{\pgfqpoint{0.033333in}{-0.008840in}}{\pgfqpoint{0.033333in}{0.000000in}}%
\pgfpathcurveto{\pgfqpoint{0.033333in}{0.008840in}}{\pgfqpoint{0.029821in}{0.017319in}}{\pgfqpoint{0.023570in}{0.023570in}}%
\pgfpathcurveto{\pgfqpoint{0.017319in}{0.029821in}}{\pgfqpoint{0.008840in}{0.033333in}}{\pgfqpoint{0.000000in}{0.033333in}}%
\pgfpathcurveto{\pgfqpoint{-0.008840in}{0.033333in}}{\pgfqpoint{-0.017319in}{0.029821in}}{\pgfqpoint{-0.023570in}{0.023570in}}%
\pgfpathcurveto{\pgfqpoint{-0.029821in}{0.017319in}}{\pgfqpoint{-0.033333in}{0.008840in}}{\pgfqpoint{-0.033333in}{0.000000in}}%
\pgfpathcurveto{\pgfqpoint{-0.033333in}{-0.008840in}}{\pgfqpoint{-0.029821in}{-0.017319in}}{\pgfqpoint{-0.023570in}{-0.023570in}}%
\pgfpathcurveto{\pgfqpoint{-0.017319in}{-0.029821in}}{\pgfqpoint{-0.008840in}{-0.033333in}}{\pgfqpoint{0.000000in}{-0.033333in}}%
\pgfpathclose%
\pgfusepath{stroke,fill}%
}%
\begin{pgfscope}%
\pgfsys@transformshift{6.112394in}{2.414722in}%
\pgfsys@useobject{currentmarker}{}%
\end{pgfscope}%
\end{pgfscope}%
\begin{pgfscope}%
\definecolor{textcolor}{rgb}{0.150000,0.150000,0.150000}%
\pgfsetstrokecolor{textcolor}%
\pgfsetfillcolor{textcolor}%
\pgftext[x=6.332394in,y=2.382639in,left,base]{\color{textcolor}\rmfamily\fontsize{8.800000}{10.560000}\selectfont Radial}%
\end{pgfscope}%
\end{pgfpicture}%
\makeatother%
\endgroup%

    % }
    \caption{Average ELBO of the last 1000 iterations for ...}
\end{figure}

\textcite{rezende2015variational} used normalizing flows for variational inference on four target posterior distributions, defined as in \cref{tab:test_energy_functions}.

\begin{table}
\caption{Test energy functions.}
\label{tab:test_energy_functions}
\begin{tabular}{@{}l@{}}
\toprule
\textbf{Potential} $U(\mathbf{z})$ \\ \midrule\midrule
\textbf{1:} $\frac{1}{2}\left(\frac{\|\mathbf{z}\|-2}{0.4}\right)^{2}-\ln \left(e^{-\frac{1}{2}\left[\frac{\mathbf{z}_{1}-2}{0.6}\right]^{2}}+e^{-\frac{1}{2}\left[\frac{\mathbf{z}_{1}+2}{0.6}\right]^{2}}\right)$ \\
\textbf{2:} $\frac{1}{2}\left[\frac{\mathbf{z}_{2}-w_{1}(\mathbf{z})}{0.4}\right]^{2}$ \\
\textbf{3:} $-\ln \left(e^{-\frac{1}{2}\left[\frac{\mathbf{z}_{2}-w_{1}(\mathbf{z})}{0.35}\right]^{2}}+e^{-\frac{1}{2}\left[\frac{\mathbf{z}_{2}-w_{1}(\mathbf{z})+w_{2}(\mathbf{z})}{0.35}\right]^{2}}\right)$ \\
\textbf{4:} $-\ln \left(e^{-\frac{1}{2}\left[\frac{\mathbf{z}_{2}-w_{1}(\mathbf{z})}{0.4}\right]^{2}}+e^{-\frac{1}{2}\left[\frac{\mathbf{z}_{2}-w_{1}(\mathbf{z})+w_{3}(\mathbf{z})}{0.35}\right]^{2}}\right)$ \\ \midrule
 with $w_1(\mathbf{z}) = \sin\left(\frac{2 \pi \mathbf{z}_1}{4}\right)$,
      $w_2(\mathbf{z}) = 3 e^{-\frac{1}{2} \left[ \frac{(\mathbf{z}_1 - 1)}{0.6} \right]^2}$,\\
     $w_3(\mathbf{z}) = 3 \sigma \left( \frac{\mathbf{z}_1-1}{0.3} \right)$ and $\sigma(x) = 1 / (1 + e^{-x})$.  \\ \midrule           
\end{tabular}
\end{table}

\begin{figure}
    \centering
    % \resizebox{\textwidth}{!}{
    %     %% Creator: Matplotlib, PGF backend
%%
%% To include the figure in your LaTeX document, write
%%   \input{<filename>.pgf}
%%
%% Make sure the required packages are loaded in your preamble
%%   \usepackage{pgf}
%%
%% Figures using additional raster images can only be included by \input if
%% they are in the same directory as the main LaTeX file. For loading figures
%% from other directories you can use the `import` package
%%   \usepackage{import}
%% and then include the figures with
%%   \import{<path to file>}{<filename>.pgf}
%%
%% Matplotlib used the following preamble
%%   \usepackage{fontspec}
%%   \setmainfont{DejaVuSerif.ttf}[Path=C:/ProgramData/Anaconda3/lib/site-packages/matplotlib/mpl-data/fonts/ttf/]
%%   \setsansfont{arial.ttf}[Path=C:/Windows/Fonts/]
%%   \setmonofont{DejaVuSansMono.ttf}[Path=C:/ProgramData/Anaconda3/lib/site-packages/matplotlib/mpl-data/fonts/ttf/]
%%
\begingroup%
\makeatletter%
\begin{pgfpicture}%
\pgfpathrectangle{\pgfpointorigin}{\pgfqpoint{20.000000in}{10.000000in}}%
\pgfusepath{use as bounding box, clip}%

    % }
    \caption{\the\textwidth}
\end{figure}

\lipsum[2]

\lipsum[3]

\begin{table}[htb]
    \centering
    \caption{Metrics}
    \label{tab:efawf}
    \begin{tabular}{lccccccccc} 
    \toprule
    Number of flows             &    & 2   &     &    & 8   &     &    & 32  &      \\ 
    \cmidrule(lr){2-4}\cmidrule(lr){5-7}\cmidrule(l){8-10}
    ELBO samples                & 32 & 128 & 512 & 32 & 128 & 512 & 32 & 128 & 512  \\ 
    \midrule
    ELBO                        &    &     &     &    &     &     &    &     &      \\
    $\mu - \hat{\mu}$           &    &     &     &    &     &     &    &     &      \\
    $\sigma^2 - \hat{\sigma}^2$ &    &     &     &    &     &     &    &     &      \\
    $\hat{k}$                   &    &     &     &    &     &     &    &     &      \\
    \bottomrule
    \end{tabular}
\end{table}

\lipsum[4]

% (move stuff here from caption of \ref{fig:training_curve_elbow_example})
% This can be seen in \cref{fig:elbow_example_9000,fig:elbow_example_11000}

\setlength\figureheight{5in}
\setlength\figurewidth{5in}

% \begin{figure}
%     \begin{adjustwidth}{-2cm}{-2cm}
%     \centering
%      %% Creator: Matplotlib, PGF backend
%%
%% To include the figure in your LaTeX document, write
%%   \input{<filename>.pgf}
%%
%% Make sure the required packages are loaded in your preamble
%%   \usepackage{pgf}
%%
%% Figures using additional raster images can only be included by \input if
%% they are in the same directory as the main LaTeX file. For loading figures
%% from other directories you can use the `import` package
%%   \usepackage{import}
%%
%% and then include the figures with
%%   \import{<path to file>}{<filename>.pgf}
%%
%% Matplotlib used the following preamble
%%
\begingroup%
\makeatletter%
\begin{pgfpicture}%
\pgfpathrectangle{\pgfpointorigin}{\pgfqpoint{7.500000in}{2.500000in}}%
\pgfusepath{use as bounding box, clip}%
\begin{pgfscope}%
\pgfsetbuttcap%
\pgfsetmiterjoin%
\definecolor{currentfill}{rgb}{1.000000,1.000000,1.000000}%
\pgfsetfillcolor{currentfill}%
\pgfsetlinewidth{0.000000pt}%
\definecolor{currentstroke}{rgb}{1.000000,1.000000,1.000000}%
\pgfsetstrokecolor{currentstroke}%
\pgfsetdash{}{0pt}%
\pgfpathmoveto{\pgfqpoint{0.000000in}{0.000000in}}%
\pgfpathlineto{\pgfqpoint{7.500000in}{0.000000in}}%
\pgfpathlineto{\pgfqpoint{7.500000in}{2.500000in}}%
\pgfpathlineto{\pgfqpoint{0.000000in}{2.500000in}}%
\pgfpathclose%
\pgfusepath{fill}%
\end{pgfscope}%
\begin{pgfscope}%
\pgfsetbuttcap%
\pgfsetmiterjoin%
\definecolor{currentfill}{rgb}{1.000000,1.000000,1.000000}%
\pgfsetfillcolor{currentfill}%
\pgfsetlinewidth{0.000000pt}%
\definecolor{currentstroke}{rgb}{0.000000,0.000000,0.000000}%
\pgfsetstrokecolor{currentstroke}%
\pgfsetstrokeopacity{0.000000}%
\pgfsetdash{}{0pt}%
\pgfpathmoveto{\pgfqpoint{0.750000in}{0.500000in}}%
\pgfpathlineto{\pgfqpoint{2.514706in}{0.500000in}}%
\pgfpathlineto{\pgfqpoint{2.514706in}{2.200000in}}%
\pgfpathlineto{\pgfqpoint{0.750000in}{2.200000in}}%
\pgfpathclose%
\pgfusepath{fill}%
\end{pgfscope}%
\begin{pgfscope}%
\pgfsetbuttcap%
\pgfsetroundjoin%
\definecolor{currentfill}{rgb}{0.150000,0.150000,0.150000}%
\pgfsetfillcolor{currentfill}%
\pgfsetlinewidth{1.003750pt}%
\definecolor{currentstroke}{rgb}{0.150000,0.150000,0.150000}%
\pgfsetstrokecolor{currentstroke}%
\pgfsetdash{}{0pt}%
\pgfsys@defobject{currentmarker}{\pgfqpoint{0.000000in}{-0.066667in}}{\pgfqpoint{0.000000in}{0.000000in}}{%
\pgfpathmoveto{\pgfqpoint{0.000000in}{0.000000in}}%
\pgfpathlineto{\pgfqpoint{0.000000in}{-0.066667in}}%
\pgfusepath{stroke,fill}%
}%
\begin{pgfscope}%
\pgfsys@transformshift{0.750000in}{0.500000in}%
\pgfsys@useobject{currentmarker}{}%
\end{pgfscope}%
\end{pgfscope}%
\begin{pgfscope}%
\definecolor{textcolor}{rgb}{0.150000,0.150000,0.150000}%
\pgfsetstrokecolor{textcolor}%
\pgfsetfillcolor{textcolor}%
\pgftext[x=0.750000in,y=0.384722in,,top]{\color{textcolor}\rmfamily\fontsize{8.800000}{10.560000}\selectfont \(\displaystyle {4500}\)}%
\end{pgfscope}%
\begin{pgfscope}%
\pgfsetbuttcap%
\pgfsetroundjoin%
\definecolor{currentfill}{rgb}{0.150000,0.150000,0.150000}%
\pgfsetfillcolor{currentfill}%
\pgfsetlinewidth{1.003750pt}%
\definecolor{currentstroke}{rgb}{0.150000,0.150000,0.150000}%
\pgfsetstrokecolor{currentstroke}%
\pgfsetdash{}{0pt}%
\pgfsys@defobject{currentmarker}{\pgfqpoint{0.000000in}{-0.066667in}}{\pgfqpoint{0.000000in}{0.000000in}}{%
\pgfpathmoveto{\pgfqpoint{0.000000in}{0.000000in}}%
\pgfpathlineto{\pgfqpoint{0.000000in}{-0.066667in}}%
\pgfusepath{stroke,fill}%
}%
\begin{pgfscope}%
\pgfsys@transformshift{1.338235in}{0.500000in}%
\pgfsys@useobject{currentmarker}{}%
\end{pgfscope}%
\end{pgfscope}%
\begin{pgfscope}%
\definecolor{textcolor}{rgb}{0.150000,0.150000,0.150000}%
\pgfsetstrokecolor{textcolor}%
\pgfsetfillcolor{textcolor}%
\pgftext[x=1.338235in,y=0.384722in,,top]{\color{textcolor}\rmfamily\fontsize{8.800000}{10.560000}\selectfont \(\displaystyle {5000}\)}%
\end{pgfscope}%
\begin{pgfscope}%
\pgfsetbuttcap%
\pgfsetroundjoin%
\definecolor{currentfill}{rgb}{0.150000,0.150000,0.150000}%
\pgfsetfillcolor{currentfill}%
\pgfsetlinewidth{1.003750pt}%
\definecolor{currentstroke}{rgb}{0.150000,0.150000,0.150000}%
\pgfsetstrokecolor{currentstroke}%
\pgfsetdash{}{0pt}%
\pgfsys@defobject{currentmarker}{\pgfqpoint{0.000000in}{-0.066667in}}{\pgfqpoint{0.000000in}{0.000000in}}{%
\pgfpathmoveto{\pgfqpoint{0.000000in}{0.000000in}}%
\pgfpathlineto{\pgfqpoint{0.000000in}{-0.066667in}}%
\pgfusepath{stroke,fill}%
}%
\begin{pgfscope}%
\pgfsys@transformshift{1.926471in}{0.500000in}%
\pgfsys@useobject{currentmarker}{}%
\end{pgfscope}%
\end{pgfscope}%
\begin{pgfscope}%
\definecolor{textcolor}{rgb}{0.150000,0.150000,0.150000}%
\pgfsetstrokecolor{textcolor}%
\pgfsetfillcolor{textcolor}%
\pgftext[x=1.926471in,y=0.384722in,,top]{\color{textcolor}\rmfamily\fontsize{8.800000}{10.560000}\selectfont \(\displaystyle {5500}\)}%
\end{pgfscope}%
\begin{pgfscope}%
\pgfsetbuttcap%
\pgfsetroundjoin%
\definecolor{currentfill}{rgb}{0.150000,0.150000,0.150000}%
\pgfsetfillcolor{currentfill}%
\pgfsetlinewidth{1.003750pt}%
\definecolor{currentstroke}{rgb}{0.150000,0.150000,0.150000}%
\pgfsetstrokecolor{currentstroke}%
\pgfsetdash{}{0pt}%
\pgfsys@defobject{currentmarker}{\pgfqpoint{0.000000in}{-0.066667in}}{\pgfqpoint{0.000000in}{0.000000in}}{%
\pgfpathmoveto{\pgfqpoint{0.000000in}{0.000000in}}%
\pgfpathlineto{\pgfqpoint{0.000000in}{-0.066667in}}%
\pgfusepath{stroke,fill}%
}%
\begin{pgfscope}%
\pgfsys@transformshift{2.514706in}{0.500000in}%
\pgfsys@useobject{currentmarker}{}%
\end{pgfscope}%
\end{pgfscope}%
\begin{pgfscope}%
\definecolor{textcolor}{rgb}{0.150000,0.150000,0.150000}%
\pgfsetstrokecolor{textcolor}%
\pgfsetfillcolor{textcolor}%
\pgftext[x=2.514706in,y=0.384722in,,top]{\color{textcolor}\rmfamily\fontsize{8.800000}{10.560000}\selectfont \(\displaystyle {6000}\)}%
\end{pgfscope}%
\begin{pgfscope}%
\definecolor{textcolor}{rgb}{0.150000,0.150000,0.150000}%
\pgfsetstrokecolor{textcolor}%
\pgfsetfillcolor{textcolor}%
\pgftext[x=1.632353in,y=0.218056in,,top]{\color{textcolor}\rmfamily\fontsize{9.600000}{11.520000}\selectfont Iterations}%
\end{pgfscope}%
\begin{pgfscope}%
\pgfsetbuttcap%
\pgfsetroundjoin%
\definecolor{currentfill}{rgb}{0.150000,0.150000,0.150000}%
\pgfsetfillcolor{currentfill}%
\pgfsetlinewidth{1.003750pt}%
\definecolor{currentstroke}{rgb}{0.150000,0.150000,0.150000}%
\pgfsetstrokecolor{currentstroke}%
\pgfsetdash{}{0pt}%
\pgfsys@defobject{currentmarker}{\pgfqpoint{-0.066667in}{0.000000in}}{\pgfqpoint{-0.000000in}{0.000000in}}{%
\pgfpathmoveto{\pgfqpoint{-0.000000in}{0.000000in}}%
\pgfpathlineto{\pgfqpoint{-0.066667in}{0.000000in}}%
\pgfusepath{stroke,fill}%
}%
\begin{pgfscope}%
\pgfsys@transformshift{0.750000in}{0.712500in}%
\pgfsys@useobject{currentmarker}{}%
\end{pgfscope}%
\end{pgfscope}%
\begin{pgfscope}%
\definecolor{textcolor}{rgb}{0.150000,0.150000,0.150000}%
\pgfsetstrokecolor{textcolor}%
\pgfsetfillcolor{textcolor}%
\pgftext[x=0.570487in, y=0.669097in, left, base]{\color{textcolor}\rmfamily\fontsize{8.800000}{10.560000}\selectfont \(\displaystyle {0}\)}%
\end{pgfscope}%
\begin{pgfscope}%
\pgfsetbuttcap%
\pgfsetroundjoin%
\definecolor{currentfill}{rgb}{0.150000,0.150000,0.150000}%
\pgfsetfillcolor{currentfill}%
\pgfsetlinewidth{1.003750pt}%
\definecolor{currentstroke}{rgb}{0.150000,0.150000,0.150000}%
\pgfsetstrokecolor{currentstroke}%
\pgfsetdash{}{0pt}%
\pgfsys@defobject{currentmarker}{\pgfqpoint{-0.066667in}{0.000000in}}{\pgfqpoint{-0.000000in}{0.000000in}}{%
\pgfpathmoveto{\pgfqpoint{-0.000000in}{0.000000in}}%
\pgfpathlineto{\pgfqpoint{-0.066667in}{0.000000in}}%
\pgfusepath{stroke,fill}%
}%
\begin{pgfscope}%
\pgfsys@transformshift{0.750000in}{1.137500in}%
\pgfsys@useobject{currentmarker}{}%
\end{pgfscope}%
\end{pgfscope}%
\begin{pgfscope}%
\definecolor{textcolor}{rgb}{0.150000,0.150000,0.150000}%
\pgfsetstrokecolor{textcolor}%
\pgfsetfillcolor{textcolor}%
\pgftext[x=0.442015in, y=1.094097in, left, base]{\color{textcolor}\rmfamily\fontsize{8.800000}{10.560000}\selectfont \(\displaystyle {100}\)}%
\end{pgfscope}%
\begin{pgfscope}%
\pgfsetbuttcap%
\pgfsetroundjoin%
\definecolor{currentfill}{rgb}{0.150000,0.150000,0.150000}%
\pgfsetfillcolor{currentfill}%
\pgfsetlinewidth{1.003750pt}%
\definecolor{currentstroke}{rgb}{0.150000,0.150000,0.150000}%
\pgfsetstrokecolor{currentstroke}%
\pgfsetdash{}{0pt}%
\pgfsys@defobject{currentmarker}{\pgfqpoint{-0.066667in}{0.000000in}}{\pgfqpoint{-0.000000in}{0.000000in}}{%
\pgfpathmoveto{\pgfqpoint{-0.000000in}{0.000000in}}%
\pgfpathlineto{\pgfqpoint{-0.066667in}{0.000000in}}%
\pgfusepath{stroke,fill}%
}%
\begin{pgfscope}%
\pgfsys@transformshift{0.750000in}{1.562500in}%
\pgfsys@useobject{currentmarker}{}%
\end{pgfscope}%
\end{pgfscope}%
\begin{pgfscope}%
\definecolor{textcolor}{rgb}{0.150000,0.150000,0.150000}%
\pgfsetstrokecolor{textcolor}%
\pgfsetfillcolor{textcolor}%
\pgftext[x=0.442015in, y=1.519097in, left, base]{\color{textcolor}\rmfamily\fontsize{8.800000}{10.560000}\selectfont \(\displaystyle {200}\)}%
\end{pgfscope}%
\begin{pgfscope}%
\pgfsetbuttcap%
\pgfsetroundjoin%
\definecolor{currentfill}{rgb}{0.150000,0.150000,0.150000}%
\pgfsetfillcolor{currentfill}%
\pgfsetlinewidth{1.003750pt}%
\definecolor{currentstroke}{rgb}{0.150000,0.150000,0.150000}%
\pgfsetstrokecolor{currentstroke}%
\pgfsetdash{}{0pt}%
\pgfsys@defobject{currentmarker}{\pgfqpoint{-0.066667in}{0.000000in}}{\pgfqpoint{-0.000000in}{0.000000in}}{%
\pgfpathmoveto{\pgfqpoint{-0.000000in}{0.000000in}}%
\pgfpathlineto{\pgfqpoint{-0.066667in}{0.000000in}}%
\pgfusepath{stroke,fill}%
}%
\begin{pgfscope}%
\pgfsys@transformshift{0.750000in}{1.987500in}%
\pgfsys@useobject{currentmarker}{}%
\end{pgfscope}%
\end{pgfscope}%
\begin{pgfscope}%
\definecolor{textcolor}{rgb}{0.150000,0.150000,0.150000}%
\pgfsetstrokecolor{textcolor}%
\pgfsetfillcolor{textcolor}%
\pgftext[x=0.442015in, y=1.944097in, left, base]{\color{textcolor}\rmfamily\fontsize{8.800000}{10.560000}\selectfont \(\displaystyle {300}\)}%
\end{pgfscope}%
\begin{pgfscope}%
\definecolor{textcolor}{rgb}{0.150000,0.150000,0.150000}%
\pgfsetstrokecolor{textcolor}%
\pgfsetfillcolor{textcolor}%
\pgftext[x=0.386460in,y=1.350000in,,bottom,rotate=90.000000]{\color{textcolor}\rmfamily\fontsize{9.600000}{11.520000}\selectfont ELBO}%
\end{pgfscope}%
\begin{pgfscope}%
\pgfpathrectangle{\pgfqpoint{0.750000in}{0.500000in}}{\pgfqpoint{1.764706in}{1.700000in}}%
\pgfusepath{clip}%
\pgfsetroundcap%
\pgfsetroundjoin%
\pgfsetlinewidth{1.204500pt}%
\definecolor{currentstroke}{rgb}{0.100000,0.100000,0.100000}%
\pgfsetstrokecolor{currentstroke}%
\pgfsetdash{}{0pt}%
\pgfpathmoveto{\pgfqpoint{0.748824in}{1.097109in}}%
\pgfpathlineto{\pgfqpoint{0.750000in}{1.144939in}}%
\pgfpathlineto{\pgfqpoint{0.751176in}{1.001161in}}%
\pgfpathlineto{\pgfqpoint{0.752353in}{1.103711in}}%
\pgfpathlineto{\pgfqpoint{0.753529in}{1.026955in}}%
\pgfpathlineto{\pgfqpoint{0.754706in}{1.207400in}}%
\pgfpathlineto{\pgfqpoint{0.757059in}{1.003409in}}%
\pgfpathlineto{\pgfqpoint{0.759412in}{1.215592in}}%
\pgfpathlineto{\pgfqpoint{0.760588in}{1.265564in}}%
\pgfpathlineto{\pgfqpoint{0.762941in}{1.082606in}}%
\pgfpathlineto{\pgfqpoint{0.765294in}{1.228878in}}%
\pgfpathlineto{\pgfqpoint{0.766471in}{1.130279in}}%
\pgfpathlineto{\pgfqpoint{0.767647in}{1.145044in}}%
\pgfpathlineto{\pgfqpoint{0.770000in}{1.067868in}}%
\pgfpathlineto{\pgfqpoint{0.771176in}{1.078670in}}%
\pgfpathlineto{\pgfqpoint{0.772353in}{1.091795in}}%
\pgfpathlineto{\pgfqpoint{0.773529in}{1.142839in}}%
\pgfpathlineto{\pgfqpoint{0.775882in}{0.955390in}}%
\pgfpathlineto{\pgfqpoint{0.777059in}{1.241173in}}%
\pgfpathlineto{\pgfqpoint{0.778235in}{1.123307in}}%
\pgfpathlineto{\pgfqpoint{0.779412in}{1.166810in}}%
\pgfpathlineto{\pgfqpoint{0.780588in}{1.104688in}}%
\pgfpathlineto{\pgfqpoint{0.781765in}{1.209898in}}%
\pgfpathlineto{\pgfqpoint{0.784118in}{1.081102in}}%
\pgfpathlineto{\pgfqpoint{0.785294in}{1.188419in}}%
\pgfpathlineto{\pgfqpoint{0.786471in}{1.172935in}}%
\pgfpathlineto{\pgfqpoint{0.787647in}{1.098329in}}%
\pgfpathlineto{\pgfqpoint{0.788824in}{1.146069in}}%
\pgfpathlineto{\pgfqpoint{0.790000in}{1.109911in}}%
\pgfpathlineto{\pgfqpoint{0.791176in}{1.032051in}}%
\pgfpathlineto{\pgfqpoint{0.792353in}{1.168567in}}%
\pgfpathlineto{\pgfqpoint{0.794706in}{1.065382in}}%
\pgfpathlineto{\pgfqpoint{0.795882in}{1.132517in}}%
\pgfpathlineto{\pgfqpoint{0.797059in}{0.953867in}}%
\pgfpathlineto{\pgfqpoint{0.798235in}{1.171518in}}%
\pgfpathlineto{\pgfqpoint{0.800588in}{1.097528in}}%
\pgfpathlineto{\pgfqpoint{0.801765in}{1.126203in}}%
\pgfpathlineto{\pgfqpoint{0.802941in}{1.046479in}}%
\pgfpathlineto{\pgfqpoint{0.804118in}{1.130805in}}%
\pgfpathlineto{\pgfqpoint{0.805294in}{1.126923in}}%
\pgfpathlineto{\pgfqpoint{0.806471in}{1.039668in}}%
\pgfpathlineto{\pgfqpoint{0.807647in}{1.073035in}}%
\pgfpathlineto{\pgfqpoint{0.808824in}{1.184421in}}%
\pgfpathlineto{\pgfqpoint{0.810000in}{1.171722in}}%
\pgfpathlineto{\pgfqpoint{0.812353in}{1.206637in}}%
\pgfpathlineto{\pgfqpoint{0.813529in}{1.102343in}}%
\pgfpathlineto{\pgfqpoint{0.814706in}{1.106694in}}%
\pgfpathlineto{\pgfqpoint{0.815882in}{1.082682in}}%
\pgfpathlineto{\pgfqpoint{0.817059in}{1.083058in}}%
\pgfpathlineto{\pgfqpoint{0.818235in}{1.091493in}}%
\pgfpathlineto{\pgfqpoint{0.820588in}{1.043982in}}%
\pgfpathlineto{\pgfqpoint{0.822941in}{1.244660in}}%
\pgfpathlineto{\pgfqpoint{0.825294in}{1.181471in}}%
\pgfpathlineto{\pgfqpoint{0.826471in}{1.136287in}}%
\pgfpathlineto{\pgfqpoint{0.827647in}{1.199171in}}%
\pgfpathlineto{\pgfqpoint{0.828824in}{1.192016in}}%
\pgfpathlineto{\pgfqpoint{0.830000in}{1.065678in}}%
\pgfpathlineto{\pgfqpoint{0.832353in}{1.141169in}}%
\pgfpathlineto{\pgfqpoint{0.834706in}{0.990447in}}%
\pgfpathlineto{\pgfqpoint{0.835882in}{1.190000in}}%
\pgfpathlineto{\pgfqpoint{0.837059in}{1.184863in}}%
\pgfpathlineto{\pgfqpoint{0.839412in}{1.131231in}}%
\pgfpathlineto{\pgfqpoint{0.840588in}{1.177828in}}%
\pgfpathlineto{\pgfqpoint{0.842941in}{1.145205in}}%
\pgfpathlineto{\pgfqpoint{0.844118in}{1.214353in}}%
\pgfpathlineto{\pgfqpoint{0.846471in}{1.035607in}}%
\pgfpathlineto{\pgfqpoint{0.848824in}{1.104091in}}%
\pgfpathlineto{\pgfqpoint{0.850000in}{1.085391in}}%
\pgfpathlineto{\pgfqpoint{0.851176in}{1.112420in}}%
\pgfpathlineto{\pgfqpoint{0.852353in}{0.984521in}}%
\pgfpathlineto{\pgfqpoint{0.855882in}{1.325632in}}%
\pgfpathlineto{\pgfqpoint{0.857059in}{1.263144in}}%
\pgfpathlineto{\pgfqpoint{0.858235in}{0.993668in}}%
\pgfpathlineto{\pgfqpoint{0.859412in}{1.103360in}}%
\pgfpathlineto{\pgfqpoint{0.860588in}{1.069199in}}%
\pgfpathlineto{\pgfqpoint{0.862941in}{1.193320in}}%
\pgfpathlineto{\pgfqpoint{0.864118in}{1.179245in}}%
\pgfpathlineto{\pgfqpoint{0.865294in}{1.080364in}}%
\pgfpathlineto{\pgfqpoint{0.866471in}{1.109747in}}%
\pgfpathlineto{\pgfqpoint{0.868824in}{1.199271in}}%
\pgfpathlineto{\pgfqpoint{0.870000in}{1.000035in}}%
\pgfpathlineto{\pgfqpoint{0.872353in}{1.236414in}}%
\pgfpathlineto{\pgfqpoint{0.873529in}{1.121497in}}%
\pgfpathlineto{\pgfqpoint{0.875882in}{1.211054in}}%
\pgfpathlineto{\pgfqpoint{0.877059in}{1.168267in}}%
\pgfpathlineto{\pgfqpoint{0.879412in}{0.928387in}}%
\pgfpathlineto{\pgfqpoint{0.880588in}{1.166432in}}%
\pgfpathlineto{\pgfqpoint{0.881765in}{1.073251in}}%
\pgfpathlineto{\pgfqpoint{0.882941in}{1.171532in}}%
\pgfpathlineto{\pgfqpoint{0.884118in}{1.129875in}}%
\pgfpathlineto{\pgfqpoint{0.885294in}{1.198430in}}%
\pgfpathlineto{\pgfqpoint{0.887647in}{1.116061in}}%
\pgfpathlineto{\pgfqpoint{0.888824in}{1.147147in}}%
\pgfpathlineto{\pgfqpoint{0.890000in}{1.073085in}}%
\pgfpathlineto{\pgfqpoint{0.892353in}{1.164568in}}%
\pgfpathlineto{\pgfqpoint{0.893529in}{1.033753in}}%
\pgfpathlineto{\pgfqpoint{0.895882in}{1.212969in}}%
\pgfpathlineto{\pgfqpoint{0.898235in}{1.079192in}}%
\pgfpathlineto{\pgfqpoint{0.899412in}{1.107632in}}%
\pgfpathlineto{\pgfqpoint{0.900588in}{1.169132in}}%
\pgfpathlineto{\pgfqpoint{0.904118in}{1.124774in}}%
\pgfpathlineto{\pgfqpoint{0.905294in}{1.092580in}}%
\pgfpathlineto{\pgfqpoint{0.907647in}{1.164223in}}%
\pgfpathlineto{\pgfqpoint{0.908824in}{1.095243in}}%
\pgfpathlineto{\pgfqpoint{0.911176in}{1.153574in}}%
\pgfpathlineto{\pgfqpoint{0.912353in}{1.133798in}}%
\pgfpathlineto{\pgfqpoint{0.913529in}{1.202640in}}%
\pgfpathlineto{\pgfqpoint{0.914706in}{1.129332in}}%
\pgfpathlineto{\pgfqpoint{0.915882in}{1.250604in}}%
\pgfpathlineto{\pgfqpoint{0.917059in}{1.072165in}}%
\pgfpathlineto{\pgfqpoint{0.918235in}{1.085762in}}%
\pgfpathlineto{\pgfqpoint{0.919412in}{1.113550in}}%
\pgfpathlineto{\pgfqpoint{0.921765in}{1.189784in}}%
\pgfpathlineto{\pgfqpoint{0.922941in}{1.064205in}}%
\pgfpathlineto{\pgfqpoint{0.924118in}{1.139093in}}%
\pgfpathlineto{\pgfqpoint{0.925294in}{1.107168in}}%
\pgfpathlineto{\pgfqpoint{0.926471in}{1.211910in}}%
\pgfpathlineto{\pgfqpoint{0.927647in}{1.137395in}}%
\pgfpathlineto{\pgfqpoint{0.928824in}{1.179470in}}%
\pgfpathlineto{\pgfqpoint{0.930000in}{1.038135in}}%
\pgfpathlineto{\pgfqpoint{0.931176in}{1.141557in}}%
\pgfpathlineto{\pgfqpoint{0.932353in}{1.146508in}}%
\pgfpathlineto{\pgfqpoint{0.934706in}{1.008445in}}%
\pgfpathlineto{\pgfqpoint{0.935882in}{1.100029in}}%
\pgfpathlineto{\pgfqpoint{0.937059in}{1.104479in}}%
\pgfpathlineto{\pgfqpoint{0.938235in}{1.144967in}}%
\pgfpathlineto{\pgfqpoint{0.939412in}{0.994819in}}%
\pgfpathlineto{\pgfqpoint{0.941765in}{1.232222in}}%
\pgfpathlineto{\pgfqpoint{0.942941in}{1.161942in}}%
\pgfpathlineto{\pgfqpoint{0.944118in}{1.225868in}}%
\pgfpathlineto{\pgfqpoint{0.945294in}{1.119649in}}%
\pgfpathlineto{\pgfqpoint{0.946471in}{1.161614in}}%
\pgfpathlineto{\pgfqpoint{0.947647in}{1.070872in}}%
\pgfpathlineto{\pgfqpoint{0.948824in}{1.182052in}}%
\pgfpathlineto{\pgfqpoint{0.950000in}{1.088392in}}%
\pgfpathlineto{\pgfqpoint{0.951176in}{1.165902in}}%
\pgfpathlineto{\pgfqpoint{0.952353in}{1.141107in}}%
\pgfpathlineto{\pgfqpoint{0.953529in}{1.204940in}}%
\pgfpathlineto{\pgfqpoint{0.954706in}{1.192235in}}%
\pgfpathlineto{\pgfqpoint{0.955882in}{1.081745in}}%
\pgfpathlineto{\pgfqpoint{0.957059in}{1.079940in}}%
\pgfpathlineto{\pgfqpoint{0.958235in}{1.035036in}}%
\pgfpathlineto{\pgfqpoint{0.959412in}{1.189732in}}%
\pgfpathlineto{\pgfqpoint{0.960588in}{1.127613in}}%
\pgfpathlineto{\pgfqpoint{0.961765in}{1.140430in}}%
\pgfpathlineto{\pgfqpoint{0.962941in}{1.071589in}}%
\pgfpathlineto{\pgfqpoint{0.964118in}{1.166370in}}%
\pgfpathlineto{\pgfqpoint{0.965294in}{1.159927in}}%
\pgfpathlineto{\pgfqpoint{0.966471in}{1.140565in}}%
\pgfpathlineto{\pgfqpoint{0.967647in}{1.140210in}}%
\pgfpathlineto{\pgfqpoint{0.968824in}{1.164019in}}%
\pgfpathlineto{\pgfqpoint{0.970000in}{1.026587in}}%
\pgfpathlineto{\pgfqpoint{0.971176in}{1.233687in}}%
\pgfpathlineto{\pgfqpoint{0.973529in}{1.068819in}}%
\pgfpathlineto{\pgfqpoint{0.975882in}{1.178416in}}%
\pgfpathlineto{\pgfqpoint{0.977059in}{1.130182in}}%
\pgfpathlineto{\pgfqpoint{0.978235in}{1.146808in}}%
\pgfpathlineto{\pgfqpoint{0.979412in}{1.196846in}}%
\pgfpathlineto{\pgfqpoint{0.981765in}{1.091543in}}%
\pgfpathlineto{\pgfqpoint{0.982941in}{1.096233in}}%
\pgfpathlineto{\pgfqpoint{0.984118in}{1.089943in}}%
\pgfpathlineto{\pgfqpoint{0.985294in}{1.302695in}}%
\pgfpathlineto{\pgfqpoint{0.986471in}{1.048393in}}%
\pgfpathlineto{\pgfqpoint{0.987647in}{1.216352in}}%
\pgfpathlineto{\pgfqpoint{0.990000in}{1.116257in}}%
\pgfpathlineto{\pgfqpoint{0.991176in}{1.060774in}}%
\pgfpathlineto{\pgfqpoint{0.992353in}{1.059650in}}%
\pgfpathlineto{\pgfqpoint{0.993529in}{1.074969in}}%
\pgfpathlineto{\pgfqpoint{0.994706in}{1.020941in}}%
\pgfpathlineto{\pgfqpoint{0.997059in}{1.164204in}}%
\pgfpathlineto{\pgfqpoint{0.999412in}{1.028988in}}%
\pgfpathlineto{\pgfqpoint{1.001765in}{1.172464in}}%
\pgfpathlineto{\pgfqpoint{1.002941in}{1.194157in}}%
\pgfpathlineto{\pgfqpoint{1.004118in}{1.136581in}}%
\pgfpathlineto{\pgfqpoint{1.005294in}{1.137469in}}%
\pgfpathlineto{\pgfqpoint{1.006471in}{1.226018in}}%
\pgfpathlineto{\pgfqpoint{1.007647in}{1.041440in}}%
\pgfpathlineto{\pgfqpoint{1.008824in}{1.157784in}}%
\pgfpathlineto{\pgfqpoint{1.011176in}{1.027967in}}%
\pgfpathlineto{\pgfqpoint{1.012353in}{1.093410in}}%
\pgfpathlineto{\pgfqpoint{1.013529in}{1.015093in}}%
\pgfpathlineto{\pgfqpoint{1.015882in}{1.082638in}}%
\pgfpathlineto{\pgfqpoint{1.017059in}{1.033693in}}%
\pgfpathlineto{\pgfqpoint{1.019412in}{1.254941in}}%
\pgfpathlineto{\pgfqpoint{1.020588in}{1.078221in}}%
\pgfpathlineto{\pgfqpoint{1.022941in}{1.147413in}}%
\pgfpathlineto{\pgfqpoint{1.024118in}{1.056977in}}%
\pgfpathlineto{\pgfqpoint{1.025294in}{1.075034in}}%
\pgfpathlineto{\pgfqpoint{1.026471in}{1.220921in}}%
\pgfpathlineto{\pgfqpoint{1.027647in}{1.065335in}}%
\pgfpathlineto{\pgfqpoint{1.028824in}{1.187421in}}%
\pgfpathlineto{\pgfqpoint{1.031176in}{1.091232in}}%
\pgfpathlineto{\pgfqpoint{1.032353in}{1.115003in}}%
\pgfpathlineto{\pgfqpoint{1.033529in}{0.933701in}}%
\pgfpathlineto{\pgfqpoint{1.035882in}{1.153190in}}%
\pgfpathlineto{\pgfqpoint{1.037059in}{1.196959in}}%
\pgfpathlineto{\pgfqpoint{1.038235in}{1.033039in}}%
\pgfpathlineto{\pgfqpoint{1.040588in}{1.176975in}}%
\pgfpathlineto{\pgfqpoint{1.042941in}{1.138578in}}%
\pgfpathlineto{\pgfqpoint{1.044118in}{1.129275in}}%
\pgfpathlineto{\pgfqpoint{1.046471in}{1.094759in}}%
\pgfpathlineto{\pgfqpoint{1.047647in}{1.098547in}}%
\pgfpathlineto{\pgfqpoint{1.048824in}{1.024920in}}%
\pgfpathlineto{\pgfqpoint{1.050000in}{1.073350in}}%
\pgfpathlineto{\pgfqpoint{1.051176in}{0.952580in}}%
\pgfpathlineto{\pgfqpoint{1.053529in}{1.229714in}}%
\pgfpathlineto{\pgfqpoint{1.055882in}{1.063117in}}%
\pgfpathlineto{\pgfqpoint{1.057059in}{1.200735in}}%
\pgfpathlineto{\pgfqpoint{1.058235in}{1.192829in}}%
\pgfpathlineto{\pgfqpoint{1.059412in}{1.061738in}}%
\pgfpathlineto{\pgfqpoint{1.060588in}{1.140967in}}%
\pgfpathlineto{\pgfqpoint{1.061765in}{1.097577in}}%
\pgfpathlineto{\pgfqpoint{1.062941in}{1.191452in}}%
\pgfpathlineto{\pgfqpoint{1.066471in}{1.018448in}}%
\pgfpathlineto{\pgfqpoint{1.067647in}{1.048552in}}%
\pgfpathlineto{\pgfqpoint{1.068824in}{1.231252in}}%
\pgfpathlineto{\pgfqpoint{1.070000in}{1.210422in}}%
\pgfpathlineto{\pgfqpoint{1.073529in}{1.081666in}}%
\pgfpathlineto{\pgfqpoint{1.075882in}{1.143698in}}%
\pgfpathlineto{\pgfqpoint{1.077059in}{1.079469in}}%
\pgfpathlineto{\pgfqpoint{1.078235in}{1.307881in}}%
\pgfpathlineto{\pgfqpoint{1.080588in}{1.196969in}}%
\pgfpathlineto{\pgfqpoint{1.081765in}{1.154728in}}%
\pgfpathlineto{\pgfqpoint{1.082941in}{1.057350in}}%
\pgfpathlineto{\pgfqpoint{1.084118in}{1.105052in}}%
\pgfpathlineto{\pgfqpoint{1.085294in}{1.102479in}}%
\pgfpathlineto{\pgfqpoint{1.086471in}{1.083236in}}%
\pgfpathlineto{\pgfqpoint{1.088824in}{1.106698in}}%
\pgfpathlineto{\pgfqpoint{1.090000in}{1.178788in}}%
\pgfpathlineto{\pgfqpoint{1.091176in}{1.101633in}}%
\pgfpathlineto{\pgfqpoint{1.092353in}{1.157004in}}%
\pgfpathlineto{\pgfqpoint{1.093529in}{1.026355in}}%
\pgfpathlineto{\pgfqpoint{1.094706in}{1.049228in}}%
\pgfpathlineto{\pgfqpoint{1.095882in}{1.037952in}}%
\pgfpathlineto{\pgfqpoint{1.097059in}{1.189889in}}%
\pgfpathlineto{\pgfqpoint{1.098235in}{1.069352in}}%
\pgfpathlineto{\pgfqpoint{1.099412in}{1.081360in}}%
\pgfpathlineto{\pgfqpoint{1.100588in}{1.120484in}}%
\pgfpathlineto{\pgfqpoint{1.101765in}{1.072788in}}%
\pgfpathlineto{\pgfqpoint{1.102941in}{1.227205in}}%
\pgfpathlineto{\pgfqpoint{1.104118in}{1.087830in}}%
\pgfpathlineto{\pgfqpoint{1.105294in}{1.232107in}}%
\pgfpathlineto{\pgfqpoint{1.106471in}{1.051522in}}%
\pgfpathlineto{\pgfqpoint{1.107647in}{1.087101in}}%
\pgfpathlineto{\pgfqpoint{1.108824in}{1.061758in}}%
\pgfpathlineto{\pgfqpoint{1.110000in}{1.177704in}}%
\pgfpathlineto{\pgfqpoint{1.111176in}{1.180598in}}%
\pgfpathlineto{\pgfqpoint{1.112353in}{1.046701in}}%
\pgfpathlineto{\pgfqpoint{1.115882in}{1.155691in}}%
\pgfpathlineto{\pgfqpoint{1.118235in}{1.045938in}}%
\pgfpathlineto{\pgfqpoint{1.119412in}{1.084955in}}%
\pgfpathlineto{\pgfqpoint{1.120588in}{0.997223in}}%
\pgfpathlineto{\pgfqpoint{1.121765in}{1.221178in}}%
\pgfpathlineto{\pgfqpoint{1.122941in}{1.117074in}}%
\pgfpathlineto{\pgfqpoint{1.125294in}{1.231201in}}%
\pgfpathlineto{\pgfqpoint{1.126471in}{1.042051in}}%
\pgfpathlineto{\pgfqpoint{1.128824in}{1.122804in}}%
\pgfpathlineto{\pgfqpoint{1.130000in}{1.041990in}}%
\pgfpathlineto{\pgfqpoint{1.131176in}{1.150041in}}%
\pgfpathlineto{\pgfqpoint{1.132353in}{1.068202in}}%
\pgfpathlineto{\pgfqpoint{1.133529in}{1.076374in}}%
\pgfpathlineto{\pgfqpoint{1.135882in}{1.257440in}}%
\pgfpathlineto{\pgfqpoint{1.137059in}{1.065524in}}%
\pgfpathlineto{\pgfqpoint{1.139412in}{1.211355in}}%
\pgfpathlineto{\pgfqpoint{1.141765in}{1.014694in}}%
\pgfpathlineto{\pgfqpoint{1.142941in}{1.189945in}}%
\pgfpathlineto{\pgfqpoint{1.144118in}{1.163667in}}%
\pgfpathlineto{\pgfqpoint{1.145294in}{1.219779in}}%
\pgfpathlineto{\pgfqpoint{1.146471in}{1.102605in}}%
\pgfpathlineto{\pgfqpoint{1.147647in}{1.103775in}}%
\pgfpathlineto{\pgfqpoint{1.148824in}{1.068967in}}%
\pgfpathlineto{\pgfqpoint{1.150000in}{0.981873in}}%
\pgfpathlineto{\pgfqpoint{1.151176in}{1.178880in}}%
\pgfpathlineto{\pgfqpoint{1.153529in}{0.963653in}}%
\pgfpathlineto{\pgfqpoint{1.155882in}{1.226523in}}%
\pgfpathlineto{\pgfqpoint{1.157059in}{1.089640in}}%
\pgfpathlineto{\pgfqpoint{1.158235in}{1.103041in}}%
\pgfpathlineto{\pgfqpoint{1.159412in}{1.089994in}}%
\pgfpathlineto{\pgfqpoint{1.160588in}{1.162318in}}%
\pgfpathlineto{\pgfqpoint{1.162941in}{1.141119in}}%
\pgfpathlineto{\pgfqpoint{1.164118in}{1.013383in}}%
\pgfpathlineto{\pgfqpoint{1.165294in}{1.042158in}}%
\pgfpathlineto{\pgfqpoint{1.166471in}{1.145771in}}%
\pgfpathlineto{\pgfqpoint{1.167647in}{1.144459in}}%
\pgfpathlineto{\pgfqpoint{1.168824in}{1.107431in}}%
\pgfpathlineto{\pgfqpoint{1.170000in}{1.109230in}}%
\pgfpathlineto{\pgfqpoint{1.171176in}{1.102947in}}%
\pgfpathlineto{\pgfqpoint{1.173529in}{1.173914in}}%
\pgfpathlineto{\pgfqpoint{1.174706in}{1.087218in}}%
\pgfpathlineto{\pgfqpoint{1.175882in}{1.102887in}}%
\pgfpathlineto{\pgfqpoint{1.177059in}{1.216903in}}%
\pgfpathlineto{\pgfqpoint{1.178235in}{1.105343in}}%
\pgfpathlineto{\pgfqpoint{1.179412in}{1.174135in}}%
\pgfpathlineto{\pgfqpoint{1.180588in}{1.086178in}}%
\pgfpathlineto{\pgfqpoint{1.181765in}{1.096289in}}%
\pgfpathlineto{\pgfqpoint{1.182941in}{1.157381in}}%
\pgfpathlineto{\pgfqpoint{1.185294in}{1.108656in}}%
\pgfpathlineto{\pgfqpoint{1.186471in}{1.102364in}}%
\pgfpathlineto{\pgfqpoint{1.188824in}{1.043142in}}%
\pgfpathlineto{\pgfqpoint{1.191176in}{1.230926in}}%
\pgfpathlineto{\pgfqpoint{1.192353in}{1.196872in}}%
\pgfpathlineto{\pgfqpoint{1.193529in}{1.093333in}}%
\pgfpathlineto{\pgfqpoint{1.194706in}{1.105993in}}%
\pgfpathlineto{\pgfqpoint{1.195882in}{1.137718in}}%
\pgfpathlineto{\pgfqpoint{1.197059in}{1.077631in}}%
\pgfpathlineto{\pgfqpoint{1.198235in}{1.128276in}}%
\pgfpathlineto{\pgfqpoint{1.199412in}{1.048147in}}%
\pgfpathlineto{\pgfqpoint{1.201765in}{1.134341in}}%
\pgfpathlineto{\pgfqpoint{1.202941in}{1.116454in}}%
\pgfpathlineto{\pgfqpoint{1.205294in}{1.129958in}}%
\pgfpathlineto{\pgfqpoint{1.206471in}{1.114841in}}%
\pgfpathlineto{\pgfqpoint{1.207647in}{1.080797in}}%
\pgfpathlineto{\pgfqpoint{1.210000in}{1.140933in}}%
\pgfpathlineto{\pgfqpoint{1.211176in}{1.142066in}}%
\pgfpathlineto{\pgfqpoint{1.212353in}{1.154928in}}%
\pgfpathlineto{\pgfqpoint{1.213529in}{1.205655in}}%
\pgfpathlineto{\pgfqpoint{1.214706in}{1.166152in}}%
\pgfpathlineto{\pgfqpoint{1.215882in}{1.256790in}}%
\pgfpathlineto{\pgfqpoint{1.218235in}{1.160862in}}%
\pgfpathlineto{\pgfqpoint{1.219412in}{1.172227in}}%
\pgfpathlineto{\pgfqpoint{1.220588in}{1.163439in}}%
\pgfpathlineto{\pgfqpoint{1.221765in}{1.188094in}}%
\pgfpathlineto{\pgfqpoint{1.224118in}{1.096773in}}%
\pgfpathlineto{\pgfqpoint{1.225294in}{1.046765in}}%
\pgfpathlineto{\pgfqpoint{1.226471in}{1.153825in}}%
\pgfpathlineto{\pgfqpoint{1.227647in}{1.135076in}}%
\pgfpathlineto{\pgfqpoint{1.228824in}{1.071615in}}%
\pgfpathlineto{\pgfqpoint{1.230000in}{1.076146in}}%
\pgfpathlineto{\pgfqpoint{1.231176in}{1.069549in}}%
\pgfpathlineto{\pgfqpoint{1.232353in}{1.154124in}}%
\pgfpathlineto{\pgfqpoint{1.233529in}{1.101401in}}%
\pgfpathlineto{\pgfqpoint{1.234706in}{1.189306in}}%
\pgfpathlineto{\pgfqpoint{1.237059in}{1.116683in}}%
\pgfpathlineto{\pgfqpoint{1.239412in}{1.221176in}}%
\pgfpathlineto{\pgfqpoint{1.241765in}{1.068546in}}%
\pgfpathlineto{\pgfqpoint{1.242941in}{1.153083in}}%
\pgfpathlineto{\pgfqpoint{1.245294in}{1.041822in}}%
\pgfpathlineto{\pgfqpoint{1.247647in}{1.233498in}}%
\pgfpathlineto{\pgfqpoint{1.248824in}{1.088418in}}%
\pgfpathlineto{\pgfqpoint{1.251176in}{1.149444in}}%
\pgfpathlineto{\pgfqpoint{1.252353in}{1.131302in}}%
\pgfpathlineto{\pgfqpoint{1.253529in}{1.170982in}}%
\pgfpathlineto{\pgfqpoint{1.254706in}{1.016582in}}%
\pgfpathlineto{\pgfqpoint{1.255882in}{1.244581in}}%
\pgfpathlineto{\pgfqpoint{1.258235in}{1.172205in}}%
\pgfpathlineto{\pgfqpoint{1.259412in}{1.172482in}}%
\pgfpathlineto{\pgfqpoint{1.261765in}{1.001610in}}%
\pgfpathlineto{\pgfqpoint{1.264118in}{1.166609in}}%
\pgfpathlineto{\pgfqpoint{1.265294in}{1.064856in}}%
\pgfpathlineto{\pgfqpoint{1.266471in}{1.155165in}}%
\pgfpathlineto{\pgfqpoint{1.267647in}{1.102616in}}%
\pgfpathlineto{\pgfqpoint{1.268824in}{1.175831in}}%
\pgfpathlineto{\pgfqpoint{1.270000in}{1.071973in}}%
\pgfpathlineto{\pgfqpoint{1.271176in}{1.201800in}}%
\pgfpathlineto{\pgfqpoint{1.272353in}{1.147688in}}%
\pgfpathlineto{\pgfqpoint{1.274706in}{1.210990in}}%
\pgfpathlineto{\pgfqpoint{1.278235in}{1.108639in}}%
\pgfpathlineto{\pgfqpoint{1.279412in}{1.030469in}}%
\pgfpathlineto{\pgfqpoint{1.280588in}{1.154058in}}%
\pgfpathlineto{\pgfqpoint{1.281765in}{1.093552in}}%
\pgfpathlineto{\pgfqpoint{1.282941in}{1.164624in}}%
\pgfpathlineto{\pgfqpoint{1.284118in}{1.087973in}}%
\pgfpathlineto{\pgfqpoint{1.285294in}{1.097413in}}%
\pgfpathlineto{\pgfqpoint{1.286471in}{1.229717in}}%
\pgfpathlineto{\pgfqpoint{1.287647in}{1.032802in}}%
\pgfpathlineto{\pgfqpoint{1.288824in}{1.113292in}}%
\pgfpathlineto{\pgfqpoint{1.290000in}{1.116140in}}%
\pgfpathlineto{\pgfqpoint{1.291176in}{1.089173in}}%
\pgfpathlineto{\pgfqpoint{1.292353in}{1.103225in}}%
\pgfpathlineto{\pgfqpoint{1.293529in}{1.150022in}}%
\pgfpathlineto{\pgfqpoint{1.294706in}{1.001932in}}%
\pgfpathlineto{\pgfqpoint{1.295882in}{1.107339in}}%
\pgfpathlineto{\pgfqpoint{1.297059in}{1.046035in}}%
\pgfpathlineto{\pgfqpoint{1.299412in}{1.123569in}}%
\pgfpathlineto{\pgfqpoint{1.300588in}{1.201558in}}%
\pgfpathlineto{\pgfqpoint{1.301765in}{1.201240in}}%
\pgfpathlineto{\pgfqpoint{1.304118in}{0.976151in}}%
\pgfpathlineto{\pgfqpoint{1.305294in}{1.156018in}}%
\pgfpathlineto{\pgfqpoint{1.306471in}{1.129284in}}%
\pgfpathlineto{\pgfqpoint{1.307647in}{1.049870in}}%
\pgfpathlineto{\pgfqpoint{1.310000in}{1.161601in}}%
\pgfpathlineto{\pgfqpoint{1.311176in}{1.044582in}}%
\pgfpathlineto{\pgfqpoint{1.312353in}{1.109459in}}%
\pgfpathlineto{\pgfqpoint{1.313529in}{1.090281in}}%
\pgfpathlineto{\pgfqpoint{1.314706in}{1.112573in}}%
\pgfpathlineto{\pgfqpoint{1.315882in}{1.104681in}}%
\pgfpathlineto{\pgfqpoint{1.317059in}{1.044868in}}%
\pgfpathlineto{\pgfqpoint{1.318235in}{1.239160in}}%
\pgfpathlineto{\pgfqpoint{1.321765in}{1.073409in}}%
\pgfpathlineto{\pgfqpoint{1.322941in}{1.104415in}}%
\pgfpathlineto{\pgfqpoint{1.324118in}{1.222833in}}%
\pgfpathlineto{\pgfqpoint{1.325294in}{1.165669in}}%
\pgfpathlineto{\pgfqpoint{1.327647in}{1.217539in}}%
\pgfpathlineto{\pgfqpoint{1.328824in}{1.215778in}}%
\pgfpathlineto{\pgfqpoint{1.330000in}{1.169749in}}%
\pgfpathlineto{\pgfqpoint{1.331176in}{1.172629in}}%
\pgfpathlineto{\pgfqpoint{1.332353in}{1.081217in}}%
\pgfpathlineto{\pgfqpoint{1.333529in}{1.124844in}}%
\pgfpathlineto{\pgfqpoint{1.334706in}{1.126018in}}%
\pgfpathlineto{\pgfqpoint{1.335882in}{1.008276in}}%
\pgfpathlineto{\pgfqpoint{1.337059in}{1.131020in}}%
\pgfpathlineto{\pgfqpoint{1.338235in}{0.860221in}}%
\pgfpathlineto{\pgfqpoint{1.340588in}{1.133996in}}%
\pgfpathlineto{\pgfqpoint{1.341765in}{1.186500in}}%
\pgfpathlineto{\pgfqpoint{1.344118in}{1.134675in}}%
\pgfpathlineto{\pgfqpoint{1.345294in}{1.141494in}}%
\pgfpathlineto{\pgfqpoint{1.346471in}{1.132085in}}%
\pgfpathlineto{\pgfqpoint{1.348824in}{1.224782in}}%
\pgfpathlineto{\pgfqpoint{1.351176in}{1.016302in}}%
\pgfpathlineto{\pgfqpoint{1.352353in}{1.169364in}}%
\pgfpathlineto{\pgfqpoint{1.353529in}{1.018226in}}%
\pgfpathlineto{\pgfqpoint{1.355882in}{1.148009in}}%
\pgfpathlineto{\pgfqpoint{1.357059in}{1.103517in}}%
\pgfpathlineto{\pgfqpoint{1.358235in}{1.128796in}}%
\pgfpathlineto{\pgfqpoint{1.359412in}{1.000308in}}%
\pgfpathlineto{\pgfqpoint{1.360588in}{1.177546in}}%
\pgfpathlineto{\pgfqpoint{1.361765in}{1.182452in}}%
\pgfpathlineto{\pgfqpoint{1.364118in}{1.053107in}}%
\pgfpathlineto{\pgfqpoint{1.365294in}{1.233093in}}%
\pgfpathlineto{\pgfqpoint{1.367647in}{1.055457in}}%
\pgfpathlineto{\pgfqpoint{1.368824in}{1.098670in}}%
\pgfpathlineto{\pgfqpoint{1.370000in}{1.096514in}}%
\pgfpathlineto{\pgfqpoint{1.371176in}{1.173218in}}%
\pgfpathlineto{\pgfqpoint{1.372353in}{1.170220in}}%
\pgfpathlineto{\pgfqpoint{1.374706in}{1.069025in}}%
\pgfpathlineto{\pgfqpoint{1.377059in}{1.189574in}}%
\pgfpathlineto{\pgfqpoint{1.378235in}{1.045489in}}%
\pgfpathlineto{\pgfqpoint{1.379412in}{1.083426in}}%
\pgfpathlineto{\pgfqpoint{1.380588in}{1.034601in}}%
\pgfpathlineto{\pgfqpoint{1.381765in}{1.192786in}}%
\pgfpathlineto{\pgfqpoint{1.382941in}{1.064908in}}%
\pgfpathlineto{\pgfqpoint{1.385294in}{1.086977in}}%
\pgfpathlineto{\pgfqpoint{1.386471in}{1.165723in}}%
\pgfpathlineto{\pgfqpoint{1.387647in}{1.101647in}}%
\pgfpathlineto{\pgfqpoint{1.388824in}{1.210299in}}%
\pgfpathlineto{\pgfqpoint{1.390000in}{1.068985in}}%
\pgfpathlineto{\pgfqpoint{1.391176in}{1.081777in}}%
\pgfpathlineto{\pgfqpoint{1.392353in}{1.040662in}}%
\pgfpathlineto{\pgfqpoint{1.393529in}{1.232567in}}%
\pgfpathlineto{\pgfqpoint{1.394706in}{1.113560in}}%
\pgfpathlineto{\pgfqpoint{1.395882in}{1.170121in}}%
\pgfpathlineto{\pgfqpoint{1.397059in}{1.115982in}}%
\pgfpathlineto{\pgfqpoint{1.398235in}{1.115826in}}%
\pgfpathlineto{\pgfqpoint{1.399412in}{1.103963in}}%
\pgfpathlineto{\pgfqpoint{1.400588in}{1.205219in}}%
\pgfpathlineto{\pgfqpoint{1.401765in}{1.046430in}}%
\pgfpathlineto{\pgfqpoint{1.402941in}{1.077245in}}%
\pgfpathlineto{\pgfqpoint{1.404118in}{1.206824in}}%
\pgfpathlineto{\pgfqpoint{1.405294in}{1.108100in}}%
\pgfpathlineto{\pgfqpoint{1.406471in}{1.130184in}}%
\pgfpathlineto{\pgfqpoint{1.407647in}{1.190612in}}%
\pgfpathlineto{\pgfqpoint{1.410000in}{1.072430in}}%
\pgfpathlineto{\pgfqpoint{1.411176in}{1.216754in}}%
\pgfpathlineto{\pgfqpoint{1.412353in}{1.039910in}}%
\pgfpathlineto{\pgfqpoint{1.413529in}{1.043291in}}%
\pgfpathlineto{\pgfqpoint{1.414706in}{1.144680in}}%
\pgfpathlineto{\pgfqpoint{1.415882in}{1.113323in}}%
\pgfpathlineto{\pgfqpoint{1.418235in}{1.151915in}}%
\pgfpathlineto{\pgfqpoint{1.419412in}{0.977656in}}%
\pgfpathlineto{\pgfqpoint{1.422941in}{1.160403in}}%
\pgfpathlineto{\pgfqpoint{1.424118in}{1.079599in}}%
\pgfpathlineto{\pgfqpoint{1.425294in}{1.141981in}}%
\pgfpathlineto{\pgfqpoint{1.427647in}{1.064398in}}%
\pgfpathlineto{\pgfqpoint{1.428824in}{1.146223in}}%
\pgfpathlineto{\pgfqpoint{1.430000in}{1.002947in}}%
\pgfpathlineto{\pgfqpoint{1.431176in}{1.147974in}}%
\pgfpathlineto{\pgfqpoint{1.432353in}{1.129761in}}%
\pgfpathlineto{\pgfqpoint{1.433529in}{1.054096in}}%
\pgfpathlineto{\pgfqpoint{1.435882in}{1.195245in}}%
\pgfpathlineto{\pgfqpoint{1.437059in}{1.101740in}}%
\pgfpathlineto{\pgfqpoint{1.438235in}{1.188113in}}%
\pgfpathlineto{\pgfqpoint{1.440588in}{1.079307in}}%
\pgfpathlineto{\pgfqpoint{1.441765in}{0.872918in}}%
\pgfpathlineto{\pgfqpoint{1.444118in}{1.159606in}}%
\pgfpathlineto{\pgfqpoint{1.446471in}{1.062883in}}%
\pgfpathlineto{\pgfqpoint{1.447647in}{1.000968in}}%
\pgfpathlineto{\pgfqpoint{1.451176in}{1.110555in}}%
\pgfpathlineto{\pgfqpoint{1.452353in}{1.100173in}}%
\pgfpathlineto{\pgfqpoint{1.453529in}{1.042793in}}%
\pgfpathlineto{\pgfqpoint{1.454706in}{1.190983in}}%
\pgfpathlineto{\pgfqpoint{1.455882in}{1.184369in}}%
\pgfpathlineto{\pgfqpoint{1.457059in}{1.162359in}}%
\pgfpathlineto{\pgfqpoint{1.458235in}{1.075393in}}%
\pgfpathlineto{\pgfqpoint{1.459412in}{1.164383in}}%
\pgfpathlineto{\pgfqpoint{1.460588in}{1.061367in}}%
\pgfpathlineto{\pgfqpoint{1.462941in}{1.184346in}}%
\pgfpathlineto{\pgfqpoint{1.464118in}{1.230448in}}%
\pgfpathlineto{\pgfqpoint{1.466471in}{0.993048in}}%
\pgfpathlineto{\pgfqpoint{1.468824in}{1.185993in}}%
\pgfpathlineto{\pgfqpoint{1.470000in}{1.069716in}}%
\pgfpathlineto{\pgfqpoint{1.471176in}{1.213585in}}%
\pgfpathlineto{\pgfqpoint{1.473529in}{1.043896in}}%
\pgfpathlineto{\pgfqpoint{1.474706in}{1.235781in}}%
\pgfpathlineto{\pgfqpoint{1.478235in}{1.067971in}}%
\pgfpathlineto{\pgfqpoint{1.479412in}{1.112361in}}%
\pgfpathlineto{\pgfqpoint{1.480588in}{1.026553in}}%
\pgfpathlineto{\pgfqpoint{1.481765in}{1.124389in}}%
\pgfpathlineto{\pgfqpoint{1.482941in}{1.089982in}}%
\pgfpathlineto{\pgfqpoint{1.484118in}{1.102535in}}%
\pgfpathlineto{\pgfqpoint{1.485294in}{0.981132in}}%
\pgfpathlineto{\pgfqpoint{1.486471in}{1.099966in}}%
\pgfpathlineto{\pgfqpoint{1.487647in}{1.016540in}}%
\pgfpathlineto{\pgfqpoint{1.488824in}{1.097065in}}%
\pgfpathlineto{\pgfqpoint{1.490000in}{1.021228in}}%
\pgfpathlineto{\pgfqpoint{1.491176in}{1.151276in}}%
\pgfpathlineto{\pgfqpoint{1.494706in}{1.025301in}}%
\pgfpathlineto{\pgfqpoint{1.497059in}{0.936814in}}%
\pgfpathlineto{\pgfqpoint{1.498235in}{0.992238in}}%
\pgfpathlineto{\pgfqpoint{1.499412in}{1.123983in}}%
\pgfpathlineto{\pgfqpoint{1.500588in}{1.053696in}}%
\pgfpathlineto{\pgfqpoint{1.502941in}{1.066061in}}%
\pgfpathlineto{\pgfqpoint{1.504118in}{1.010379in}}%
\pgfpathlineto{\pgfqpoint{1.506471in}{1.169215in}}%
\pgfpathlineto{\pgfqpoint{1.507647in}{0.989175in}}%
\pgfpathlineto{\pgfqpoint{1.508824in}{1.090223in}}%
\pgfpathlineto{\pgfqpoint{1.510000in}{1.006471in}}%
\pgfpathlineto{\pgfqpoint{1.512353in}{1.230519in}}%
\pgfpathlineto{\pgfqpoint{1.513529in}{1.060728in}}%
\pgfpathlineto{\pgfqpoint{1.514706in}{1.086432in}}%
\pgfpathlineto{\pgfqpoint{1.515882in}{1.052542in}}%
\pgfpathlineto{\pgfqpoint{1.517059in}{1.091559in}}%
\pgfpathlineto{\pgfqpoint{1.518235in}{0.999588in}}%
\pgfpathlineto{\pgfqpoint{1.519412in}{1.007204in}}%
\pgfpathlineto{\pgfqpoint{1.522941in}{1.096601in}}%
\pgfpathlineto{\pgfqpoint{1.525294in}{1.035618in}}%
\pgfpathlineto{\pgfqpoint{1.527647in}{1.131900in}}%
\pgfpathlineto{\pgfqpoint{1.530000in}{1.227869in}}%
\pgfpathlineto{\pgfqpoint{1.531176in}{1.051302in}}%
\pgfpathlineto{\pgfqpoint{1.533529in}{1.131841in}}%
\pgfpathlineto{\pgfqpoint{1.534706in}{1.128414in}}%
\pgfpathlineto{\pgfqpoint{1.535882in}{1.118908in}}%
\pgfpathlineto{\pgfqpoint{1.537059in}{1.024440in}}%
\pgfpathlineto{\pgfqpoint{1.539412in}{1.185305in}}%
\pgfpathlineto{\pgfqpoint{1.541765in}{0.899821in}}%
\pgfpathlineto{\pgfqpoint{1.542941in}{1.118841in}}%
\pgfpathlineto{\pgfqpoint{1.544118in}{0.982765in}}%
\pgfpathlineto{\pgfqpoint{1.545294in}{1.139386in}}%
\pgfpathlineto{\pgfqpoint{1.546471in}{1.082240in}}%
\pgfpathlineto{\pgfqpoint{1.547647in}{1.099601in}}%
\pgfpathlineto{\pgfqpoint{1.548824in}{1.140603in}}%
\pgfpathlineto{\pgfqpoint{1.550000in}{0.985498in}}%
\pgfpathlineto{\pgfqpoint{1.551176in}{1.004739in}}%
\pgfpathlineto{\pgfqpoint{1.552353in}{0.873805in}}%
\pgfpathlineto{\pgfqpoint{1.553529in}{1.073491in}}%
\pgfpathlineto{\pgfqpoint{1.554706in}{1.029813in}}%
\pgfpathlineto{\pgfqpoint{1.555882in}{1.066079in}}%
\pgfpathlineto{\pgfqpoint{1.557059in}{1.056109in}}%
\pgfpathlineto{\pgfqpoint{1.558235in}{1.010148in}}%
\pgfpathlineto{\pgfqpoint{1.559412in}{1.055540in}}%
\pgfpathlineto{\pgfqpoint{1.560588in}{1.001726in}}%
\pgfpathlineto{\pgfqpoint{1.562941in}{1.065341in}}%
\pgfpathlineto{\pgfqpoint{1.564118in}{0.866496in}}%
\pgfpathlineto{\pgfqpoint{1.566471in}{1.056195in}}%
\pgfpathlineto{\pgfqpoint{1.568824in}{1.014511in}}%
\pgfpathlineto{\pgfqpoint{1.570000in}{1.003538in}}%
\pgfpathlineto{\pgfqpoint{1.571176in}{0.973128in}}%
\pgfpathlineto{\pgfqpoint{1.572353in}{1.109087in}}%
\pgfpathlineto{\pgfqpoint{1.574706in}{0.972086in}}%
\pgfpathlineto{\pgfqpoint{1.575882in}{1.023515in}}%
\pgfpathlineto{\pgfqpoint{1.577059in}{0.969631in}}%
\pgfpathlineto{\pgfqpoint{1.578235in}{1.083165in}}%
\pgfpathlineto{\pgfqpoint{1.579412in}{0.962965in}}%
\pgfpathlineto{\pgfqpoint{1.580588in}{0.969377in}}%
\pgfpathlineto{\pgfqpoint{1.581765in}{1.149948in}}%
\pgfpathlineto{\pgfqpoint{1.582941in}{0.959704in}}%
\pgfpathlineto{\pgfqpoint{1.585294in}{1.072340in}}%
\pgfpathlineto{\pgfqpoint{1.586471in}{0.948619in}}%
\pgfpathlineto{\pgfqpoint{1.587647in}{1.061717in}}%
\pgfpathlineto{\pgfqpoint{1.588824in}{1.003954in}}%
\pgfpathlineto{\pgfqpoint{1.590000in}{1.066475in}}%
\pgfpathlineto{\pgfqpoint{1.591176in}{1.194766in}}%
\pgfpathlineto{\pgfqpoint{1.592353in}{0.981000in}}%
\pgfpathlineto{\pgfqpoint{1.593529in}{1.024473in}}%
\pgfpathlineto{\pgfqpoint{1.594706in}{1.134519in}}%
\pgfpathlineto{\pgfqpoint{1.595882in}{0.962776in}}%
\pgfpathlineto{\pgfqpoint{1.597059in}{1.028473in}}%
\pgfpathlineto{\pgfqpoint{1.598235in}{0.942749in}}%
\pgfpathlineto{\pgfqpoint{1.599412in}{1.119037in}}%
\pgfpathlineto{\pgfqpoint{1.600588in}{1.049743in}}%
\pgfpathlineto{\pgfqpoint{1.601765in}{1.154458in}}%
\pgfpathlineto{\pgfqpoint{1.602941in}{0.877793in}}%
\pgfpathlineto{\pgfqpoint{1.605294in}{1.079530in}}%
\pgfpathlineto{\pgfqpoint{1.607647in}{1.063107in}}%
\pgfpathlineto{\pgfqpoint{1.608824in}{1.097501in}}%
\pgfpathlineto{\pgfqpoint{1.610000in}{0.882657in}}%
\pgfpathlineto{\pgfqpoint{1.611176in}{1.015628in}}%
\pgfpathlineto{\pgfqpoint{1.612353in}{1.006304in}}%
\pgfpathlineto{\pgfqpoint{1.613529in}{0.821215in}}%
\pgfpathlineto{\pgfqpoint{1.615882in}{1.065308in}}%
\pgfpathlineto{\pgfqpoint{1.617059in}{0.986438in}}%
\pgfpathlineto{\pgfqpoint{1.618235in}{1.010830in}}%
\pgfpathlineto{\pgfqpoint{1.619412in}{0.919877in}}%
\pgfpathlineto{\pgfqpoint{1.620588in}{1.011304in}}%
\pgfpathlineto{\pgfqpoint{1.621765in}{0.941599in}}%
\pgfpathlineto{\pgfqpoint{1.622941in}{1.179519in}}%
\pgfpathlineto{\pgfqpoint{1.625294in}{1.056492in}}%
\pgfpathlineto{\pgfqpoint{1.626471in}{0.984856in}}%
\pgfpathlineto{\pgfqpoint{1.628824in}{1.219252in}}%
\pgfpathlineto{\pgfqpoint{1.630000in}{1.001429in}}%
\pgfpathlineto{\pgfqpoint{1.631176in}{1.050385in}}%
\pgfpathlineto{\pgfqpoint{1.633529in}{0.919262in}}%
\pgfpathlineto{\pgfqpoint{1.635882in}{1.108544in}}%
\pgfpathlineto{\pgfqpoint{1.637059in}{1.006696in}}%
\pgfpathlineto{\pgfqpoint{1.638235in}{1.048461in}}%
\pgfpathlineto{\pgfqpoint{1.639412in}{0.941201in}}%
\pgfpathlineto{\pgfqpoint{1.640588in}{1.193538in}}%
\pgfpathlineto{\pgfqpoint{1.641765in}{1.203455in}}%
\pgfpathlineto{\pgfqpoint{1.642941in}{1.107936in}}%
\pgfpathlineto{\pgfqpoint{1.644118in}{1.221277in}}%
\pgfpathlineto{\pgfqpoint{1.645294in}{1.094041in}}%
\pgfpathlineto{\pgfqpoint{1.647647in}{1.259012in}}%
\pgfpathlineto{\pgfqpoint{1.648824in}{1.356628in}}%
\pgfpathlineto{\pgfqpoint{1.650000in}{1.193264in}}%
\pgfpathlineto{\pgfqpoint{1.652353in}{1.461575in}}%
\pgfpathlineto{\pgfqpoint{1.654706in}{1.246626in}}%
\pgfpathlineto{\pgfqpoint{1.655882in}{1.483351in}}%
\pgfpathlineto{\pgfqpoint{1.657059in}{1.433021in}}%
\pgfpathlineto{\pgfqpoint{1.658235in}{1.660475in}}%
\pgfpathlineto{\pgfqpoint{1.659412in}{1.604418in}}%
\pgfpathlineto{\pgfqpoint{1.661765in}{1.686047in}}%
\pgfpathlineto{\pgfqpoint{1.662941in}{1.668062in}}%
\pgfpathlineto{\pgfqpoint{1.664118in}{1.561859in}}%
\pgfpathlineto{\pgfqpoint{1.665294in}{1.582548in}}%
\pgfpathlineto{\pgfqpoint{1.666471in}{1.676059in}}%
\pgfpathlineto{\pgfqpoint{1.667647in}{1.597329in}}%
\pgfpathlineto{\pgfqpoint{1.668824in}{1.411838in}}%
\pgfpathlineto{\pgfqpoint{1.670000in}{1.662856in}}%
\pgfpathlineto{\pgfqpoint{1.671176in}{1.500219in}}%
\pgfpathlineto{\pgfqpoint{1.672353in}{1.504200in}}%
\pgfpathlineto{\pgfqpoint{1.673529in}{1.550371in}}%
\pgfpathlineto{\pgfqpoint{1.674706in}{1.682421in}}%
\pgfpathlineto{\pgfqpoint{1.675882in}{1.626994in}}%
\pgfpathlineto{\pgfqpoint{1.677059in}{1.807197in}}%
\pgfpathlineto{\pgfqpoint{1.678235in}{1.600900in}}%
\pgfpathlineto{\pgfqpoint{1.679412in}{1.663383in}}%
\pgfpathlineto{\pgfqpoint{1.680588in}{1.846406in}}%
\pgfpathlineto{\pgfqpoint{1.681765in}{1.840865in}}%
\pgfpathlineto{\pgfqpoint{1.682941in}{1.749431in}}%
\pgfpathlineto{\pgfqpoint{1.684118in}{1.807628in}}%
\pgfpathlineto{\pgfqpoint{1.686471in}{1.666243in}}%
\pgfpathlineto{\pgfqpoint{1.687647in}{1.663581in}}%
\pgfpathlineto{\pgfqpoint{1.688824in}{1.813861in}}%
\pgfpathlineto{\pgfqpoint{1.690000in}{1.751649in}}%
\pgfpathlineto{\pgfqpoint{1.691176in}{1.874336in}}%
\pgfpathlineto{\pgfqpoint{1.692353in}{1.670352in}}%
\pgfpathlineto{\pgfqpoint{1.693529in}{1.681553in}}%
\pgfpathlineto{\pgfqpoint{1.694706in}{1.856492in}}%
\pgfpathlineto{\pgfqpoint{1.695882in}{1.796854in}}%
\pgfpathlineto{\pgfqpoint{1.697059in}{1.806097in}}%
\pgfpathlineto{\pgfqpoint{1.698235in}{1.661178in}}%
\pgfpathlineto{\pgfqpoint{1.700588in}{1.809961in}}%
\pgfpathlineto{\pgfqpoint{1.701765in}{1.678987in}}%
\pgfpathlineto{\pgfqpoint{1.702941in}{1.852344in}}%
\pgfpathlineto{\pgfqpoint{1.705294in}{1.764299in}}%
\pgfpathlineto{\pgfqpoint{1.706471in}{1.787071in}}%
\pgfpathlineto{\pgfqpoint{1.708824in}{1.712405in}}%
\pgfpathlineto{\pgfqpoint{1.711176in}{1.797193in}}%
\pgfpathlineto{\pgfqpoint{1.712353in}{1.822714in}}%
\pgfpathlineto{\pgfqpoint{1.713529in}{1.877445in}}%
\pgfpathlineto{\pgfqpoint{1.715882in}{1.558704in}}%
\pgfpathlineto{\pgfqpoint{1.718235in}{1.752316in}}%
\pgfpathlineto{\pgfqpoint{1.719412in}{1.842069in}}%
\pgfpathlineto{\pgfqpoint{1.721765in}{1.692009in}}%
\pgfpathlineto{\pgfqpoint{1.722941in}{1.824507in}}%
\pgfpathlineto{\pgfqpoint{1.724118in}{1.673283in}}%
\pgfpathlineto{\pgfqpoint{1.725294in}{1.853945in}}%
\pgfpathlineto{\pgfqpoint{1.727647in}{1.739807in}}%
\pgfpathlineto{\pgfqpoint{1.730000in}{1.860579in}}%
\pgfpathlineto{\pgfqpoint{1.731176in}{1.888149in}}%
\pgfpathlineto{\pgfqpoint{1.733529in}{1.694052in}}%
\pgfpathlineto{\pgfqpoint{1.734706in}{1.746342in}}%
\pgfpathlineto{\pgfqpoint{1.735882in}{1.873701in}}%
\pgfpathlineto{\pgfqpoint{1.737059in}{1.825399in}}%
\pgfpathlineto{\pgfqpoint{1.738235in}{1.695783in}}%
\pgfpathlineto{\pgfqpoint{1.740588in}{1.884143in}}%
\pgfpathlineto{\pgfqpoint{1.741765in}{1.684911in}}%
\pgfpathlineto{\pgfqpoint{1.742941in}{1.830776in}}%
\pgfpathlineto{\pgfqpoint{1.744118in}{1.794534in}}%
\pgfpathlineto{\pgfqpoint{1.745294in}{1.871455in}}%
\pgfpathlineto{\pgfqpoint{1.746471in}{1.652917in}}%
\pgfpathlineto{\pgfqpoint{1.747647in}{1.862148in}}%
\pgfpathlineto{\pgfqpoint{1.748824in}{1.630997in}}%
\pgfpathlineto{\pgfqpoint{1.751176in}{1.808715in}}%
\pgfpathlineto{\pgfqpoint{1.752353in}{1.947108in}}%
\pgfpathlineto{\pgfqpoint{1.754706in}{1.759384in}}%
\pgfpathlineto{\pgfqpoint{1.755882in}{1.877575in}}%
\pgfpathlineto{\pgfqpoint{1.757059in}{1.810334in}}%
\pgfpathlineto{\pgfqpoint{1.758235in}{1.832372in}}%
\pgfpathlineto{\pgfqpoint{1.759412in}{1.897398in}}%
\pgfpathlineto{\pgfqpoint{1.760588in}{1.887855in}}%
\pgfpathlineto{\pgfqpoint{1.761765in}{1.720150in}}%
\pgfpathlineto{\pgfqpoint{1.762941in}{1.809919in}}%
\pgfpathlineto{\pgfqpoint{1.765294in}{1.736476in}}%
\pgfpathlineto{\pgfqpoint{1.766471in}{1.844995in}}%
\pgfpathlineto{\pgfqpoint{1.767647in}{1.804982in}}%
\pgfpathlineto{\pgfqpoint{1.768824in}{1.897959in}}%
\pgfpathlineto{\pgfqpoint{1.771176in}{1.731549in}}%
\pgfpathlineto{\pgfqpoint{1.772353in}{1.862666in}}%
\pgfpathlineto{\pgfqpoint{1.773529in}{1.854702in}}%
\pgfpathlineto{\pgfqpoint{1.774706in}{1.748819in}}%
\pgfpathlineto{\pgfqpoint{1.777059in}{1.842311in}}%
\pgfpathlineto{\pgfqpoint{1.778235in}{1.891349in}}%
\pgfpathlineto{\pgfqpoint{1.779412in}{1.874795in}}%
\pgfpathlineto{\pgfqpoint{1.782941in}{1.755791in}}%
\pgfpathlineto{\pgfqpoint{1.784118in}{1.877321in}}%
\pgfpathlineto{\pgfqpoint{1.786471in}{1.770836in}}%
\pgfpathlineto{\pgfqpoint{1.787647in}{1.805943in}}%
\pgfpathlineto{\pgfqpoint{1.788824in}{1.899557in}}%
\pgfpathlineto{\pgfqpoint{1.790000in}{1.834870in}}%
\pgfpathlineto{\pgfqpoint{1.792353in}{1.896979in}}%
\pgfpathlineto{\pgfqpoint{1.793529in}{1.845765in}}%
\pgfpathlineto{\pgfqpoint{1.795882in}{1.694642in}}%
\pgfpathlineto{\pgfqpoint{1.798235in}{1.799705in}}%
\pgfpathlineto{\pgfqpoint{1.799412in}{1.767746in}}%
\pgfpathlineto{\pgfqpoint{1.800588in}{1.855979in}}%
\pgfpathlineto{\pgfqpoint{1.801765in}{1.821129in}}%
\pgfpathlineto{\pgfqpoint{1.802941in}{1.747094in}}%
\pgfpathlineto{\pgfqpoint{1.804118in}{1.772232in}}%
\pgfpathlineto{\pgfqpoint{1.806471in}{1.969845in}}%
\pgfpathlineto{\pgfqpoint{1.808824in}{1.725361in}}%
\pgfpathlineto{\pgfqpoint{1.810000in}{1.690347in}}%
\pgfpathlineto{\pgfqpoint{1.811176in}{1.929265in}}%
\pgfpathlineto{\pgfqpoint{1.814706in}{1.845771in}}%
\pgfpathlineto{\pgfqpoint{1.815882in}{1.870790in}}%
\pgfpathlineto{\pgfqpoint{1.817059in}{1.780397in}}%
\pgfpathlineto{\pgfqpoint{1.818235in}{1.934383in}}%
\pgfpathlineto{\pgfqpoint{1.819412in}{1.892745in}}%
\pgfpathlineto{\pgfqpoint{1.820588in}{1.912715in}}%
\pgfpathlineto{\pgfqpoint{1.821765in}{1.781830in}}%
\pgfpathlineto{\pgfqpoint{1.822941in}{1.861072in}}%
\pgfpathlineto{\pgfqpoint{1.824118in}{1.851053in}}%
\pgfpathlineto{\pgfqpoint{1.825294in}{1.763166in}}%
\pgfpathlineto{\pgfqpoint{1.826471in}{1.968990in}}%
\pgfpathlineto{\pgfqpoint{1.828824in}{1.738599in}}%
\pgfpathlineto{\pgfqpoint{1.830000in}{1.884541in}}%
\pgfpathlineto{\pgfqpoint{1.831176in}{1.756638in}}%
\pgfpathlineto{\pgfqpoint{1.832353in}{1.856104in}}%
\pgfpathlineto{\pgfqpoint{1.834706in}{1.696105in}}%
\pgfpathlineto{\pgfqpoint{1.837059in}{1.906238in}}%
\pgfpathlineto{\pgfqpoint{1.838235in}{1.878282in}}%
\pgfpathlineto{\pgfqpoint{1.839412in}{1.904552in}}%
\pgfpathlineto{\pgfqpoint{1.841765in}{1.778057in}}%
\pgfpathlineto{\pgfqpoint{1.842941in}{1.639644in}}%
\pgfpathlineto{\pgfqpoint{1.844118in}{1.995259in}}%
\pgfpathlineto{\pgfqpoint{1.845294in}{1.946436in}}%
\pgfpathlineto{\pgfqpoint{1.846471in}{1.966362in}}%
\pgfpathlineto{\pgfqpoint{1.847647in}{1.802247in}}%
\pgfpathlineto{\pgfqpoint{1.848824in}{1.808822in}}%
\pgfpathlineto{\pgfqpoint{1.850000in}{1.780589in}}%
\pgfpathlineto{\pgfqpoint{1.851176in}{1.864215in}}%
\pgfpathlineto{\pgfqpoint{1.852353in}{1.850150in}}%
\pgfpathlineto{\pgfqpoint{1.853529in}{1.789499in}}%
\pgfpathlineto{\pgfqpoint{1.854706in}{1.840679in}}%
\pgfpathlineto{\pgfqpoint{1.857059in}{1.715529in}}%
\pgfpathlineto{\pgfqpoint{1.858235in}{1.937213in}}%
\pgfpathlineto{\pgfqpoint{1.859412in}{1.739427in}}%
\pgfpathlineto{\pgfqpoint{1.860588in}{1.825125in}}%
\pgfpathlineto{\pgfqpoint{1.861765in}{1.810402in}}%
\pgfpathlineto{\pgfqpoint{1.862941in}{1.879318in}}%
\pgfpathlineto{\pgfqpoint{1.865294in}{1.717895in}}%
\pgfpathlineto{\pgfqpoint{1.866471in}{1.913373in}}%
\pgfpathlineto{\pgfqpoint{1.867647in}{1.739538in}}%
\pgfpathlineto{\pgfqpoint{1.868824in}{1.997551in}}%
\pgfpathlineto{\pgfqpoint{1.871176in}{1.832813in}}%
\pgfpathlineto{\pgfqpoint{1.872353in}{1.868593in}}%
\pgfpathlineto{\pgfqpoint{1.873529in}{1.851208in}}%
\pgfpathlineto{\pgfqpoint{1.874706in}{1.798963in}}%
\pgfpathlineto{\pgfqpoint{1.878235in}{1.999018in}}%
\pgfpathlineto{\pgfqpoint{1.879412in}{1.818943in}}%
\pgfpathlineto{\pgfqpoint{1.880588in}{1.843325in}}%
\pgfpathlineto{\pgfqpoint{1.881765in}{1.803912in}}%
\pgfpathlineto{\pgfqpoint{1.882941in}{1.805796in}}%
\pgfpathlineto{\pgfqpoint{1.884118in}{1.790610in}}%
\pgfpathlineto{\pgfqpoint{1.887647in}{1.949248in}}%
\pgfpathlineto{\pgfqpoint{1.890000in}{1.813522in}}%
\pgfpathlineto{\pgfqpoint{1.891176in}{1.921406in}}%
\pgfpathlineto{\pgfqpoint{1.892353in}{1.914297in}}%
\pgfpathlineto{\pgfqpoint{1.895882in}{1.810619in}}%
\pgfpathlineto{\pgfqpoint{1.897059in}{1.889149in}}%
\pgfpathlineto{\pgfqpoint{1.898235in}{1.848845in}}%
\pgfpathlineto{\pgfqpoint{1.899412in}{1.852172in}}%
\pgfpathlineto{\pgfqpoint{1.900588in}{1.820435in}}%
\pgfpathlineto{\pgfqpoint{1.902941in}{1.835920in}}%
\pgfpathlineto{\pgfqpoint{1.905294in}{1.947201in}}%
\pgfpathlineto{\pgfqpoint{1.906471in}{1.870309in}}%
\pgfpathlineto{\pgfqpoint{1.907647in}{1.927356in}}%
\pgfpathlineto{\pgfqpoint{1.910000in}{1.862581in}}%
\pgfpathlineto{\pgfqpoint{1.911176in}{1.740829in}}%
\pgfpathlineto{\pgfqpoint{1.912353in}{1.740266in}}%
\pgfpathlineto{\pgfqpoint{1.914706in}{1.797685in}}%
\pgfpathlineto{\pgfqpoint{1.915882in}{1.969175in}}%
\pgfpathlineto{\pgfqpoint{1.917059in}{1.846069in}}%
\pgfpathlineto{\pgfqpoint{1.918235in}{1.917190in}}%
\pgfpathlineto{\pgfqpoint{1.919412in}{1.808509in}}%
\pgfpathlineto{\pgfqpoint{1.920588in}{1.955612in}}%
\pgfpathlineto{\pgfqpoint{1.922941in}{1.802440in}}%
\pgfpathlineto{\pgfqpoint{1.924118in}{1.903765in}}%
\pgfpathlineto{\pgfqpoint{1.925294in}{1.880720in}}%
\pgfpathlineto{\pgfqpoint{1.926471in}{1.880438in}}%
\pgfpathlineto{\pgfqpoint{1.927647in}{1.810821in}}%
\pgfpathlineto{\pgfqpoint{1.930000in}{1.950265in}}%
\pgfpathlineto{\pgfqpoint{1.932353in}{1.918442in}}%
\pgfpathlineto{\pgfqpoint{1.933529in}{1.991624in}}%
\pgfpathlineto{\pgfqpoint{1.934706in}{1.830832in}}%
\pgfpathlineto{\pgfqpoint{1.937059in}{1.913055in}}%
\pgfpathlineto{\pgfqpoint{1.938235in}{1.928193in}}%
\pgfpathlineto{\pgfqpoint{1.939412in}{1.799174in}}%
\pgfpathlineto{\pgfqpoint{1.940588in}{1.855563in}}%
\pgfpathlineto{\pgfqpoint{1.941765in}{1.830161in}}%
\pgfpathlineto{\pgfqpoint{1.942941in}{1.912921in}}%
\pgfpathlineto{\pgfqpoint{1.944118in}{1.916064in}}%
\pgfpathlineto{\pgfqpoint{1.945294in}{1.807888in}}%
\pgfpathlineto{\pgfqpoint{1.946471in}{1.823626in}}%
\pgfpathlineto{\pgfqpoint{1.947647in}{1.896896in}}%
\pgfpathlineto{\pgfqpoint{1.948824in}{1.800610in}}%
\pgfpathlineto{\pgfqpoint{1.950000in}{1.838743in}}%
\pgfpathlineto{\pgfqpoint{1.951176in}{1.829151in}}%
\pgfpathlineto{\pgfqpoint{1.952353in}{1.878383in}}%
\pgfpathlineto{\pgfqpoint{1.954706in}{1.772656in}}%
\pgfpathlineto{\pgfqpoint{1.957059in}{1.972193in}}%
\pgfpathlineto{\pgfqpoint{1.958235in}{1.842715in}}%
\pgfpathlineto{\pgfqpoint{1.960588in}{1.959839in}}%
\pgfpathlineto{\pgfqpoint{1.961765in}{1.894147in}}%
\pgfpathlineto{\pgfqpoint{1.962941in}{1.909610in}}%
\pgfpathlineto{\pgfqpoint{1.964118in}{1.901122in}}%
\pgfpathlineto{\pgfqpoint{1.966471in}{1.759813in}}%
\pgfpathlineto{\pgfqpoint{1.967647in}{1.829676in}}%
\pgfpathlineto{\pgfqpoint{1.968824in}{1.822796in}}%
\pgfpathlineto{\pgfqpoint{1.970000in}{1.962138in}}%
\pgfpathlineto{\pgfqpoint{1.971176in}{1.902340in}}%
\pgfpathlineto{\pgfqpoint{1.972353in}{1.953328in}}%
\pgfpathlineto{\pgfqpoint{1.973529in}{1.791670in}}%
\pgfpathlineto{\pgfqpoint{1.975882in}{1.900661in}}%
\pgfpathlineto{\pgfqpoint{1.977059in}{1.820825in}}%
\pgfpathlineto{\pgfqpoint{1.978235in}{1.889691in}}%
\pgfpathlineto{\pgfqpoint{1.979412in}{1.845047in}}%
\pgfpathlineto{\pgfqpoint{1.980588in}{1.910178in}}%
\pgfpathlineto{\pgfqpoint{1.982941in}{1.723609in}}%
\pgfpathlineto{\pgfqpoint{1.984118in}{1.891306in}}%
\pgfpathlineto{\pgfqpoint{1.985294in}{1.804865in}}%
\pgfpathlineto{\pgfqpoint{1.986471in}{1.840922in}}%
\pgfpathlineto{\pgfqpoint{1.988824in}{1.944804in}}%
\pgfpathlineto{\pgfqpoint{1.990000in}{1.960271in}}%
\pgfpathlineto{\pgfqpoint{1.992353in}{1.846817in}}%
\pgfpathlineto{\pgfqpoint{1.994706in}{1.948806in}}%
\pgfpathlineto{\pgfqpoint{1.995882in}{1.795202in}}%
\pgfpathlineto{\pgfqpoint{1.998235in}{1.883270in}}%
\pgfpathlineto{\pgfqpoint{1.999412in}{1.849685in}}%
\pgfpathlineto{\pgfqpoint{2.000588in}{1.914950in}}%
\pgfpathlineto{\pgfqpoint{2.001765in}{1.874106in}}%
\pgfpathlineto{\pgfqpoint{2.002941in}{1.874834in}}%
\pgfpathlineto{\pgfqpoint{2.004118in}{1.719458in}}%
\pgfpathlineto{\pgfqpoint{2.006471in}{1.870938in}}%
\pgfpathlineto{\pgfqpoint{2.007647in}{1.794998in}}%
\pgfpathlineto{\pgfqpoint{2.010000in}{1.867600in}}%
\pgfpathlineto{\pgfqpoint{2.011176in}{1.789744in}}%
\pgfpathlineto{\pgfqpoint{2.012353in}{1.906621in}}%
\pgfpathlineto{\pgfqpoint{2.013529in}{1.840161in}}%
\pgfpathlineto{\pgfqpoint{2.014706in}{1.839711in}}%
\pgfpathlineto{\pgfqpoint{2.015882in}{1.876559in}}%
\pgfpathlineto{\pgfqpoint{2.017059in}{1.844266in}}%
\pgfpathlineto{\pgfqpoint{2.018235in}{1.928652in}}%
\pgfpathlineto{\pgfqpoint{2.019412in}{1.688771in}}%
\pgfpathlineto{\pgfqpoint{2.022941in}{1.871905in}}%
\pgfpathlineto{\pgfqpoint{2.024118in}{1.753063in}}%
\pgfpathlineto{\pgfqpoint{2.025294in}{1.920522in}}%
\pgfpathlineto{\pgfqpoint{2.026471in}{1.925921in}}%
\pgfpathlineto{\pgfqpoint{2.027647in}{1.936932in}}%
\pgfpathlineto{\pgfqpoint{2.028824in}{1.898261in}}%
\pgfpathlineto{\pgfqpoint{2.030000in}{1.818786in}}%
\pgfpathlineto{\pgfqpoint{2.031176in}{1.819403in}}%
\pgfpathlineto{\pgfqpoint{2.032353in}{1.833063in}}%
\pgfpathlineto{\pgfqpoint{2.033529in}{1.755795in}}%
\pgfpathlineto{\pgfqpoint{2.034706in}{1.865931in}}%
\pgfpathlineto{\pgfqpoint{2.035882in}{1.864350in}}%
\pgfpathlineto{\pgfqpoint{2.037059in}{1.859415in}}%
\pgfpathlineto{\pgfqpoint{2.038235in}{1.938430in}}%
\pgfpathlineto{\pgfqpoint{2.040588in}{1.880844in}}%
\pgfpathlineto{\pgfqpoint{2.041765in}{1.990543in}}%
\pgfpathlineto{\pgfqpoint{2.042941in}{1.988741in}}%
\pgfpathlineto{\pgfqpoint{2.046471in}{1.911148in}}%
\pgfpathlineto{\pgfqpoint{2.047647in}{1.852775in}}%
\pgfpathlineto{\pgfqpoint{2.048824in}{1.857421in}}%
\pgfpathlineto{\pgfqpoint{2.050000in}{1.841858in}}%
\pgfpathlineto{\pgfqpoint{2.052353in}{1.949163in}}%
\pgfpathlineto{\pgfqpoint{2.053529in}{1.865530in}}%
\pgfpathlineto{\pgfqpoint{2.054706in}{1.968088in}}%
\pgfpathlineto{\pgfqpoint{2.057059in}{1.813453in}}%
\pgfpathlineto{\pgfqpoint{2.059412in}{1.937607in}}%
\pgfpathlineto{\pgfqpoint{2.060588in}{1.817841in}}%
\pgfpathlineto{\pgfqpoint{2.062941in}{1.947844in}}%
\pgfpathlineto{\pgfqpoint{2.064118in}{1.959444in}}%
\pgfpathlineto{\pgfqpoint{2.065294in}{1.880375in}}%
\pgfpathlineto{\pgfqpoint{2.066471in}{1.948456in}}%
\pgfpathlineto{\pgfqpoint{2.067647in}{1.886645in}}%
\pgfpathlineto{\pgfqpoint{2.068824in}{1.896157in}}%
\pgfpathlineto{\pgfqpoint{2.070000in}{1.808124in}}%
\pgfpathlineto{\pgfqpoint{2.071176in}{1.845306in}}%
\pgfpathlineto{\pgfqpoint{2.072353in}{1.826828in}}%
\pgfpathlineto{\pgfqpoint{2.073529in}{1.921585in}}%
\pgfpathlineto{\pgfqpoint{2.075882in}{1.843712in}}%
\pgfpathlineto{\pgfqpoint{2.077059in}{1.965502in}}%
\pgfpathlineto{\pgfqpoint{2.078235in}{1.855323in}}%
\pgfpathlineto{\pgfqpoint{2.079412in}{1.914810in}}%
\pgfpathlineto{\pgfqpoint{2.081765in}{1.873419in}}%
\pgfpathlineto{\pgfqpoint{2.082941in}{1.936412in}}%
\pgfpathlineto{\pgfqpoint{2.085294in}{1.874018in}}%
\pgfpathlineto{\pgfqpoint{2.086471in}{1.896534in}}%
\pgfpathlineto{\pgfqpoint{2.087647in}{1.880511in}}%
\pgfpathlineto{\pgfqpoint{2.090000in}{1.977310in}}%
\pgfpathlineto{\pgfqpoint{2.091176in}{1.944093in}}%
\pgfpathlineto{\pgfqpoint{2.092353in}{1.791205in}}%
\pgfpathlineto{\pgfqpoint{2.093529in}{1.897302in}}%
\pgfpathlineto{\pgfqpoint{2.094706in}{1.735290in}}%
\pgfpathlineto{\pgfqpoint{2.095882in}{1.797394in}}%
\pgfpathlineto{\pgfqpoint{2.097059in}{1.958102in}}%
\pgfpathlineto{\pgfqpoint{2.098235in}{1.847111in}}%
\pgfpathlineto{\pgfqpoint{2.099412in}{1.842733in}}%
\pgfpathlineto{\pgfqpoint{2.100588in}{1.723686in}}%
\pgfpathlineto{\pgfqpoint{2.102941in}{1.966600in}}%
\pgfpathlineto{\pgfqpoint{2.105294in}{1.785604in}}%
\pgfpathlineto{\pgfqpoint{2.107647in}{1.965507in}}%
\pgfpathlineto{\pgfqpoint{2.110000in}{1.907360in}}%
\pgfpathlineto{\pgfqpoint{2.112353in}{1.929861in}}%
\pgfpathlineto{\pgfqpoint{2.113529in}{1.827742in}}%
\pgfpathlineto{\pgfqpoint{2.114706in}{1.833636in}}%
\pgfpathlineto{\pgfqpoint{2.115882in}{1.949868in}}%
\pgfpathlineto{\pgfqpoint{2.117059in}{1.917317in}}%
\pgfpathlineto{\pgfqpoint{2.118235in}{1.951733in}}%
\pgfpathlineto{\pgfqpoint{2.119412in}{1.846881in}}%
\pgfpathlineto{\pgfqpoint{2.120588in}{1.921552in}}%
\pgfpathlineto{\pgfqpoint{2.121765in}{1.867790in}}%
\pgfpathlineto{\pgfqpoint{2.122941in}{1.873126in}}%
\pgfpathlineto{\pgfqpoint{2.125294in}{1.897987in}}%
\pgfpathlineto{\pgfqpoint{2.127647in}{1.857759in}}%
\pgfpathlineto{\pgfqpoint{2.128824in}{1.708168in}}%
\pgfpathlineto{\pgfqpoint{2.131176in}{1.886575in}}%
\pgfpathlineto{\pgfqpoint{2.132353in}{1.911308in}}%
\pgfpathlineto{\pgfqpoint{2.134706in}{1.746665in}}%
\pgfpathlineto{\pgfqpoint{2.135882in}{1.780670in}}%
\pgfpathlineto{\pgfqpoint{2.137059in}{1.943669in}}%
\pgfpathlineto{\pgfqpoint{2.138235in}{1.817510in}}%
\pgfpathlineto{\pgfqpoint{2.139412in}{1.827805in}}%
\pgfpathlineto{\pgfqpoint{2.141765in}{1.937996in}}%
\pgfpathlineto{\pgfqpoint{2.142941in}{1.826600in}}%
\pgfpathlineto{\pgfqpoint{2.144118in}{1.858806in}}%
\pgfpathlineto{\pgfqpoint{2.145294in}{1.834194in}}%
\pgfpathlineto{\pgfqpoint{2.146471in}{1.773094in}}%
\pgfpathlineto{\pgfqpoint{2.147647in}{1.851979in}}%
\pgfpathlineto{\pgfqpoint{2.148824in}{1.798519in}}%
\pgfpathlineto{\pgfqpoint{2.150000in}{1.833461in}}%
\pgfpathlineto{\pgfqpoint{2.152353in}{1.914812in}}%
\pgfpathlineto{\pgfqpoint{2.153529in}{1.933521in}}%
\pgfpathlineto{\pgfqpoint{2.154706in}{1.783862in}}%
\pgfpathlineto{\pgfqpoint{2.155882in}{1.939853in}}%
\pgfpathlineto{\pgfqpoint{2.157059in}{1.817271in}}%
\pgfpathlineto{\pgfqpoint{2.159412in}{1.927362in}}%
\pgfpathlineto{\pgfqpoint{2.160588in}{1.807048in}}%
\pgfpathlineto{\pgfqpoint{2.161765in}{1.869074in}}%
\pgfpathlineto{\pgfqpoint{2.162941in}{1.677780in}}%
\pgfpathlineto{\pgfqpoint{2.165294in}{1.825867in}}%
\pgfpathlineto{\pgfqpoint{2.167647in}{1.911725in}}%
\pgfpathlineto{\pgfqpoint{2.168824in}{1.980043in}}%
\pgfpathlineto{\pgfqpoint{2.170000in}{1.867709in}}%
\pgfpathlineto{\pgfqpoint{2.171176in}{1.960971in}}%
\pgfpathlineto{\pgfqpoint{2.173529in}{1.823989in}}%
\pgfpathlineto{\pgfqpoint{2.174706in}{1.949063in}}%
\pgfpathlineto{\pgfqpoint{2.175882in}{1.925404in}}%
\pgfpathlineto{\pgfqpoint{2.178235in}{1.843480in}}%
\pgfpathlineto{\pgfqpoint{2.180588in}{1.767701in}}%
\pgfpathlineto{\pgfqpoint{2.182941in}{1.814042in}}%
\pgfpathlineto{\pgfqpoint{2.185294in}{1.834357in}}%
\pgfpathlineto{\pgfqpoint{2.186471in}{1.695515in}}%
\pgfpathlineto{\pgfqpoint{2.187647in}{1.957896in}}%
\pgfpathlineto{\pgfqpoint{2.190000in}{1.873438in}}%
\pgfpathlineto{\pgfqpoint{2.191176in}{1.910564in}}%
\pgfpathlineto{\pgfqpoint{2.193529in}{1.751114in}}%
\pgfpathlineto{\pgfqpoint{2.195882in}{1.891089in}}%
\pgfpathlineto{\pgfqpoint{2.197059in}{1.853766in}}%
\pgfpathlineto{\pgfqpoint{2.198235in}{1.948267in}}%
\pgfpathlineto{\pgfqpoint{2.199412in}{1.810759in}}%
\pgfpathlineto{\pgfqpoint{2.200588in}{1.883666in}}%
\pgfpathlineto{\pgfqpoint{2.204118in}{1.809111in}}%
\pgfpathlineto{\pgfqpoint{2.205294in}{1.868596in}}%
\pgfpathlineto{\pgfqpoint{2.208824in}{1.729380in}}%
\pgfpathlineto{\pgfqpoint{2.210000in}{1.927941in}}%
\pgfpathlineto{\pgfqpoint{2.212353in}{1.888739in}}%
\pgfpathlineto{\pgfqpoint{2.213529in}{1.878540in}}%
\pgfpathlineto{\pgfqpoint{2.214706in}{1.989417in}}%
\pgfpathlineto{\pgfqpoint{2.217059in}{1.778485in}}%
\pgfpathlineto{\pgfqpoint{2.218235in}{1.882248in}}%
\pgfpathlineto{\pgfqpoint{2.219412in}{1.813421in}}%
\pgfpathlineto{\pgfqpoint{2.220588in}{1.927047in}}%
\pgfpathlineto{\pgfqpoint{2.221765in}{1.750036in}}%
\pgfpathlineto{\pgfqpoint{2.224118in}{1.971198in}}%
\pgfpathlineto{\pgfqpoint{2.225294in}{1.824569in}}%
\pgfpathlineto{\pgfqpoint{2.226471in}{1.961349in}}%
\pgfpathlineto{\pgfqpoint{2.227647in}{1.798625in}}%
\pgfpathlineto{\pgfqpoint{2.228824in}{1.989311in}}%
\pgfpathlineto{\pgfqpoint{2.232353in}{1.722684in}}%
\pgfpathlineto{\pgfqpoint{2.235882in}{1.916580in}}%
\pgfpathlineto{\pgfqpoint{2.237059in}{1.884255in}}%
\pgfpathlineto{\pgfqpoint{2.238235in}{1.916463in}}%
\pgfpathlineto{\pgfqpoint{2.239412in}{1.849611in}}%
\pgfpathlineto{\pgfqpoint{2.240588in}{1.846297in}}%
\pgfpathlineto{\pgfqpoint{2.241765in}{1.847661in}}%
\pgfpathlineto{\pgfqpoint{2.242941in}{1.971009in}}%
\pgfpathlineto{\pgfqpoint{2.244118in}{1.919815in}}%
\pgfpathlineto{\pgfqpoint{2.245294in}{1.770764in}}%
\pgfpathlineto{\pgfqpoint{2.246471in}{1.863016in}}%
\pgfpathlineto{\pgfqpoint{2.247647in}{1.809139in}}%
\pgfpathlineto{\pgfqpoint{2.248824in}{1.935225in}}%
\pgfpathlineto{\pgfqpoint{2.251176in}{1.749907in}}%
\pgfpathlineto{\pgfqpoint{2.253529in}{1.872225in}}%
\pgfpathlineto{\pgfqpoint{2.254706in}{1.836869in}}%
\pgfpathlineto{\pgfqpoint{2.255882in}{1.884138in}}%
\pgfpathlineto{\pgfqpoint{2.257059in}{1.854480in}}%
\pgfpathlineto{\pgfqpoint{2.260588in}{1.973866in}}%
\pgfpathlineto{\pgfqpoint{2.261765in}{1.806153in}}%
\pgfpathlineto{\pgfqpoint{2.262941in}{1.841828in}}%
\pgfpathlineto{\pgfqpoint{2.264118in}{2.007218in}}%
\pgfpathlineto{\pgfqpoint{2.265294in}{1.978614in}}%
\pgfpathlineto{\pgfqpoint{2.266471in}{1.884868in}}%
\pgfpathlineto{\pgfqpoint{2.267647in}{1.885950in}}%
\pgfpathlineto{\pgfqpoint{2.270000in}{1.939975in}}%
\pgfpathlineto{\pgfqpoint{2.271176in}{1.800672in}}%
\pgfpathlineto{\pgfqpoint{2.272353in}{1.860025in}}%
\pgfpathlineto{\pgfqpoint{2.273529in}{1.987174in}}%
\pgfpathlineto{\pgfqpoint{2.274706in}{1.829940in}}%
\pgfpathlineto{\pgfqpoint{2.275882in}{1.993238in}}%
\pgfpathlineto{\pgfqpoint{2.277059in}{1.891705in}}%
\pgfpathlineto{\pgfqpoint{2.278235in}{1.896115in}}%
\pgfpathlineto{\pgfqpoint{2.281765in}{1.788279in}}%
\pgfpathlineto{\pgfqpoint{2.282941in}{1.912486in}}%
\pgfpathlineto{\pgfqpoint{2.285294in}{1.862728in}}%
\pgfpathlineto{\pgfqpoint{2.286471in}{1.960900in}}%
\pgfpathlineto{\pgfqpoint{2.288824in}{1.796219in}}%
\pgfpathlineto{\pgfqpoint{2.290000in}{1.952035in}}%
\pgfpathlineto{\pgfqpoint{2.291176in}{1.919726in}}%
\pgfpathlineto{\pgfqpoint{2.292353in}{1.762305in}}%
\pgfpathlineto{\pgfqpoint{2.293529in}{2.021019in}}%
\pgfpathlineto{\pgfqpoint{2.294706in}{1.998030in}}%
\pgfpathlineto{\pgfqpoint{2.297059in}{1.913245in}}%
\pgfpathlineto{\pgfqpoint{2.298235in}{1.880718in}}%
\pgfpathlineto{\pgfqpoint{2.299412in}{2.012750in}}%
\pgfpathlineto{\pgfqpoint{2.300588in}{1.785189in}}%
\pgfpathlineto{\pgfqpoint{2.302941in}{2.015136in}}%
\pgfpathlineto{\pgfqpoint{2.306471in}{1.803836in}}%
\pgfpathlineto{\pgfqpoint{2.307647in}{1.905361in}}%
\pgfpathlineto{\pgfqpoint{2.308824in}{1.851400in}}%
\pgfpathlineto{\pgfqpoint{2.310000in}{1.971576in}}%
\pgfpathlineto{\pgfqpoint{2.311176in}{1.917109in}}%
\pgfpathlineto{\pgfqpoint{2.312353in}{1.702753in}}%
\pgfpathlineto{\pgfqpoint{2.314706in}{1.805569in}}%
\pgfpathlineto{\pgfqpoint{2.318235in}{1.993188in}}%
\pgfpathlineto{\pgfqpoint{2.319412in}{1.851748in}}%
\pgfpathlineto{\pgfqpoint{2.320588in}{1.865446in}}%
\pgfpathlineto{\pgfqpoint{2.322941in}{1.936468in}}%
\pgfpathlineto{\pgfqpoint{2.325294in}{1.816665in}}%
\pgfpathlineto{\pgfqpoint{2.327647in}{1.877610in}}%
\pgfpathlineto{\pgfqpoint{2.328824in}{1.702078in}}%
\pgfpathlineto{\pgfqpoint{2.330000in}{1.870811in}}%
\pgfpathlineto{\pgfqpoint{2.331176in}{1.840256in}}%
\pgfpathlineto{\pgfqpoint{2.332353in}{1.759595in}}%
\pgfpathlineto{\pgfqpoint{2.334706in}{1.886159in}}%
\pgfpathlineto{\pgfqpoint{2.337059in}{1.928441in}}%
\pgfpathlineto{\pgfqpoint{2.338235in}{1.913654in}}%
\pgfpathlineto{\pgfqpoint{2.339412in}{1.780852in}}%
\pgfpathlineto{\pgfqpoint{2.340588in}{1.867712in}}%
\pgfpathlineto{\pgfqpoint{2.341765in}{1.851161in}}%
\pgfpathlineto{\pgfqpoint{2.342941in}{1.907690in}}%
\pgfpathlineto{\pgfqpoint{2.344118in}{1.870843in}}%
\pgfpathlineto{\pgfqpoint{2.345294in}{1.797265in}}%
\pgfpathlineto{\pgfqpoint{2.346471in}{1.911949in}}%
\pgfpathlineto{\pgfqpoint{2.347647in}{1.849619in}}%
\pgfpathlineto{\pgfqpoint{2.348824in}{2.012075in}}%
\pgfpathlineto{\pgfqpoint{2.351176in}{1.749339in}}%
\pgfpathlineto{\pgfqpoint{2.353529in}{2.002223in}}%
\pgfpathlineto{\pgfqpoint{2.354706in}{1.858153in}}%
\pgfpathlineto{\pgfqpoint{2.355882in}{1.906546in}}%
\pgfpathlineto{\pgfqpoint{2.357059in}{1.860526in}}%
\pgfpathlineto{\pgfqpoint{2.358235in}{1.918493in}}%
\pgfpathlineto{\pgfqpoint{2.359412in}{1.884937in}}%
\pgfpathlineto{\pgfqpoint{2.360588in}{1.771295in}}%
\pgfpathlineto{\pgfqpoint{2.361765in}{1.853813in}}%
\pgfpathlineto{\pgfqpoint{2.362941in}{1.751259in}}%
\pgfpathlineto{\pgfqpoint{2.364118in}{1.826758in}}%
\pgfpathlineto{\pgfqpoint{2.365294in}{1.819603in}}%
\pgfpathlineto{\pgfqpoint{2.366471in}{1.968428in}}%
\pgfpathlineto{\pgfqpoint{2.368824in}{1.724852in}}%
\pgfpathlineto{\pgfqpoint{2.371176in}{1.926707in}}%
\pgfpathlineto{\pgfqpoint{2.372353in}{1.864774in}}%
\pgfpathlineto{\pgfqpoint{2.373529in}{1.886247in}}%
\pgfpathlineto{\pgfqpoint{2.374706in}{1.884125in}}%
\pgfpathlineto{\pgfqpoint{2.377059in}{1.827730in}}%
\pgfpathlineto{\pgfqpoint{2.379412in}{1.901470in}}%
\pgfpathlineto{\pgfqpoint{2.380588in}{1.694968in}}%
\pgfpathlineto{\pgfqpoint{2.381765in}{1.917018in}}%
\pgfpathlineto{\pgfqpoint{2.382941in}{1.840566in}}%
\pgfpathlineto{\pgfqpoint{2.384118in}{1.954687in}}%
\pgfpathlineto{\pgfqpoint{2.385294in}{1.919809in}}%
\pgfpathlineto{\pgfqpoint{2.388824in}{1.964139in}}%
\pgfpathlineto{\pgfqpoint{2.391176in}{1.840985in}}%
\pgfpathlineto{\pgfqpoint{2.392353in}{1.940190in}}%
\pgfpathlineto{\pgfqpoint{2.393529in}{1.762648in}}%
\pgfpathlineto{\pgfqpoint{2.397059in}{1.931688in}}%
\pgfpathlineto{\pgfqpoint{2.398235in}{1.784409in}}%
\pgfpathlineto{\pgfqpoint{2.400588in}{1.884447in}}%
\pgfpathlineto{\pgfqpoint{2.401765in}{1.803833in}}%
\pgfpathlineto{\pgfqpoint{2.404118in}{1.877920in}}%
\pgfpathlineto{\pgfqpoint{2.405294in}{1.879532in}}%
\pgfpathlineto{\pgfqpoint{2.406471in}{1.935332in}}%
\pgfpathlineto{\pgfqpoint{2.407647in}{1.886282in}}%
\pgfpathlineto{\pgfqpoint{2.408824in}{1.924076in}}%
\pgfpathlineto{\pgfqpoint{2.410000in}{1.897122in}}%
\pgfpathlineto{\pgfqpoint{2.411176in}{1.947100in}}%
\pgfpathlineto{\pgfqpoint{2.412353in}{1.917505in}}%
\pgfpathlineto{\pgfqpoint{2.413529in}{1.988653in}}%
\pgfpathlineto{\pgfqpoint{2.415882in}{1.973041in}}%
\pgfpathlineto{\pgfqpoint{2.417059in}{1.905502in}}%
\pgfpathlineto{\pgfqpoint{2.418235in}{1.920307in}}%
\pgfpathlineto{\pgfqpoint{2.419412in}{1.710945in}}%
\pgfpathlineto{\pgfqpoint{2.420588in}{1.829443in}}%
\pgfpathlineto{\pgfqpoint{2.421765in}{1.764256in}}%
\pgfpathlineto{\pgfqpoint{2.422941in}{1.910341in}}%
\pgfpathlineto{\pgfqpoint{2.424118in}{1.909523in}}%
\pgfpathlineto{\pgfqpoint{2.426471in}{1.785909in}}%
\pgfpathlineto{\pgfqpoint{2.428824in}{2.021093in}}%
\pgfpathlineto{\pgfqpoint{2.430000in}{1.877069in}}%
\pgfpathlineto{\pgfqpoint{2.431176in}{1.974462in}}%
\pgfpathlineto{\pgfqpoint{2.432353in}{1.769230in}}%
\pgfpathlineto{\pgfqpoint{2.434706in}{1.951821in}}%
\pgfpathlineto{\pgfqpoint{2.435882in}{1.865911in}}%
\pgfpathlineto{\pgfqpoint{2.437059in}{1.903378in}}%
\pgfpathlineto{\pgfqpoint{2.438235in}{1.842027in}}%
\pgfpathlineto{\pgfqpoint{2.439412in}{1.906794in}}%
\pgfpathlineto{\pgfqpoint{2.440588in}{1.791438in}}%
\pgfpathlineto{\pgfqpoint{2.441765in}{1.894258in}}%
\pgfpathlineto{\pgfqpoint{2.442941in}{1.869689in}}%
\pgfpathlineto{\pgfqpoint{2.444118in}{1.969162in}}%
\pgfpathlineto{\pgfqpoint{2.445294in}{1.921800in}}%
\pgfpathlineto{\pgfqpoint{2.446471in}{1.960258in}}%
\pgfpathlineto{\pgfqpoint{2.451176in}{1.757983in}}%
\pgfpathlineto{\pgfqpoint{2.453529in}{1.931248in}}%
\pgfpathlineto{\pgfqpoint{2.455882in}{1.840121in}}%
\pgfpathlineto{\pgfqpoint{2.457059in}{1.880115in}}%
\pgfpathlineto{\pgfqpoint{2.458235in}{1.847355in}}%
\pgfpathlineto{\pgfqpoint{2.459412in}{1.962137in}}%
\pgfpathlineto{\pgfqpoint{2.460588in}{1.813127in}}%
\pgfpathlineto{\pgfqpoint{2.462941in}{1.935559in}}%
\pgfpathlineto{\pgfqpoint{2.464118in}{1.750966in}}%
\pgfpathlineto{\pgfqpoint{2.466471in}{1.907096in}}%
\pgfpathlineto{\pgfqpoint{2.467647in}{1.871056in}}%
\pgfpathlineto{\pgfqpoint{2.468824in}{1.891416in}}%
\pgfpathlineto{\pgfqpoint{2.470000in}{1.944249in}}%
\pgfpathlineto{\pgfqpoint{2.472353in}{1.918194in}}%
\pgfpathlineto{\pgfqpoint{2.473529in}{1.765559in}}%
\pgfpathlineto{\pgfqpoint{2.474706in}{1.814945in}}%
\pgfpathlineto{\pgfqpoint{2.475882in}{1.970550in}}%
\pgfpathlineto{\pgfqpoint{2.477059in}{1.915392in}}%
\pgfpathlineto{\pgfqpoint{2.478235in}{1.919703in}}%
\pgfpathlineto{\pgfqpoint{2.480588in}{1.976126in}}%
\pgfpathlineto{\pgfqpoint{2.481765in}{1.807721in}}%
\pgfpathlineto{\pgfqpoint{2.484118in}{1.911902in}}%
\pgfpathlineto{\pgfqpoint{2.485294in}{1.946275in}}%
\pgfpathlineto{\pgfqpoint{2.486471in}{1.941356in}}%
\pgfpathlineto{\pgfqpoint{2.488824in}{1.819368in}}%
\pgfpathlineto{\pgfqpoint{2.490000in}{1.931100in}}%
\pgfpathlineto{\pgfqpoint{2.492353in}{1.770487in}}%
\pgfpathlineto{\pgfqpoint{2.494706in}{1.897490in}}%
\pgfpathlineto{\pgfqpoint{2.495882in}{1.961971in}}%
\pgfpathlineto{\pgfqpoint{2.498235in}{1.787708in}}%
\pgfpathlineto{\pgfqpoint{2.499412in}{1.864313in}}%
\pgfpathlineto{\pgfqpoint{2.500588in}{1.826289in}}%
\pgfpathlineto{\pgfqpoint{2.501765in}{1.934812in}}%
\pgfpathlineto{\pgfqpoint{2.502941in}{1.787340in}}%
\pgfpathlineto{\pgfqpoint{2.504118in}{1.907377in}}%
\pgfpathlineto{\pgfqpoint{2.505294in}{1.908987in}}%
\pgfpathlineto{\pgfqpoint{2.506471in}{1.790274in}}%
\pgfpathlineto{\pgfqpoint{2.507647in}{1.892946in}}%
\pgfpathlineto{\pgfqpoint{2.508824in}{1.842318in}}%
\pgfpathlineto{\pgfqpoint{2.510000in}{1.946747in}}%
\pgfpathlineto{\pgfqpoint{2.512353in}{1.845962in}}%
\pgfpathlineto{\pgfqpoint{2.513529in}{1.893727in}}%
\pgfpathlineto{\pgfqpoint{2.514706in}{1.820411in}}%
\pgfpathlineto{\pgfqpoint{2.515882in}{1.889943in}}%
\pgfpathlineto{\pgfqpoint{2.515882in}{1.889943in}}%
\pgfusepath{stroke}%
\end{pgfscope}%
\begin{pgfscope}%
\pgfsetrectcap%
\pgfsetmiterjoin%
\pgfsetlinewidth{1.003750pt}%
\definecolor{currentstroke}{rgb}{0.150000,0.150000,0.150000}%
\pgfsetstrokecolor{currentstroke}%
\pgfsetdash{}{0pt}%
\pgfpathmoveto{\pgfqpoint{0.750000in}{0.500000in}}%
\pgfpathlineto{\pgfqpoint{0.750000in}{2.200000in}}%
\pgfusepath{stroke}%
\end{pgfscope}%
\begin{pgfscope}%
\pgfsetrectcap%
\pgfsetmiterjoin%
\pgfsetlinewidth{1.003750pt}%
\definecolor{currentstroke}{rgb}{0.150000,0.150000,0.150000}%
\pgfsetstrokecolor{currentstroke}%
\pgfsetdash{}{0pt}%
\pgfpathmoveto{\pgfqpoint{2.514706in}{0.500000in}}%
\pgfpathlineto{\pgfqpoint{2.514706in}{2.200000in}}%
\pgfusepath{stroke}%
\end{pgfscope}%
\begin{pgfscope}%
\pgfsetrectcap%
\pgfsetmiterjoin%
\pgfsetlinewidth{1.003750pt}%
\definecolor{currentstroke}{rgb}{0.150000,0.150000,0.150000}%
\pgfsetstrokecolor{currentstroke}%
\pgfsetdash{}{0pt}%
\pgfpathmoveto{\pgfqpoint{0.750000in}{0.500000in}}%
\pgfpathlineto{\pgfqpoint{2.514706in}{0.500000in}}%
\pgfusepath{stroke}%
\end{pgfscope}%
\begin{pgfscope}%
\pgfsetrectcap%
\pgfsetmiterjoin%
\pgfsetlinewidth{1.003750pt}%
\definecolor{currentstroke}{rgb}{0.150000,0.150000,0.150000}%
\pgfsetstrokecolor{currentstroke}%
\pgfsetdash{}{0pt}%
\pgfpathmoveto{\pgfqpoint{0.750000in}{2.200000in}}%
\pgfpathlineto{\pgfqpoint{2.514706in}{2.200000in}}%
\pgfusepath{stroke}%
\end{pgfscope}%
\begin{pgfscope}%
\definecolor{textcolor}{rgb}{0.150000,0.150000,0.150000}%
\pgfsetstrokecolor{textcolor}%
\pgfsetfillcolor{textcolor}%
\pgftext[x=1.632353in,y=2.283333in,,base]{\color{textcolor}\rmfamily\fontsize{9.600000}{11.520000}\selectfont Training curve}%
\end{pgfscope}%
\begin{pgfscope}%
\pgfsetbuttcap%
\pgfsetmiterjoin%
\definecolor{currentfill}{rgb}{1.000000,1.000000,1.000000}%
\pgfsetfillcolor{currentfill}%
\pgfsetlinewidth{0.000000pt}%
\definecolor{currentstroke}{rgb}{0.000000,0.000000,0.000000}%
\pgfsetstrokecolor{currentstroke}%
\pgfsetstrokeopacity{0.000000}%
\pgfsetdash{}{0pt}%
\pgfpathmoveto{\pgfqpoint{2.867647in}{0.500000in}}%
\pgfpathlineto{\pgfqpoint{4.632353in}{0.500000in}}%
\pgfpathlineto{\pgfqpoint{4.632353in}{2.200000in}}%
\pgfpathlineto{\pgfqpoint{2.867647in}{2.200000in}}%
\pgfpathclose%
\pgfusepath{fill}%
\end{pgfscope}%
\begin{pgfscope}%
\pgfpathrectangle{\pgfqpoint{2.867647in}{0.500000in}}{\pgfqpoint{1.764706in}{1.700000in}}%
\pgfusepath{clip}%
\pgfsetbuttcap%
\pgfsetroundjoin%
\definecolor{currentfill}{rgb}{0.980678,0.914765,0.856766}%
\pgfsetfillcolor{currentfill}%
\pgfsetlinewidth{0.311001pt}%
\definecolor{currentstroke}{rgb}{1.000000,1.000000,1.000000}%
\pgfsetstrokecolor{currentstroke}%
\pgfsetdash{}{0pt}%
\pgfpathmoveto{\pgfqpoint{4.154766in}{1.477265in}}%
\pgfpathcurveto{\pgfqpoint{4.161899in}{1.477265in}}{\pgfqpoint{4.168740in}{1.480099in}}{\pgfqpoint{4.173784in}{1.485142in}}%
\pgfpathcurveto{\pgfqpoint{4.178828in}{1.490186in}}{\pgfqpoint{4.181661in}{1.497028in}}{\pgfqpoint{4.181661in}{1.504160in}}%
\pgfpathcurveto{\pgfqpoint{4.181661in}{1.511293in}}{\pgfqpoint{4.178828in}{1.518135in}}{\pgfqpoint{4.173784in}{1.523179in}}%
\pgfpathcurveto{\pgfqpoint{4.168740in}{1.528222in}}{\pgfqpoint{4.161899in}{1.531056in}}{\pgfqpoint{4.154766in}{1.531056in}}%
\pgfpathcurveto{\pgfqpoint{4.147633in}{1.531056in}}{\pgfqpoint{4.140791in}{1.528222in}}{\pgfqpoint{4.135748in}{1.523179in}}%
\pgfpathcurveto{\pgfqpoint{4.130704in}{1.518135in}}{\pgfqpoint{4.127870in}{1.511293in}}{\pgfqpoint{4.127870in}{1.504160in}}%
\pgfpathcurveto{\pgfqpoint{4.127870in}{1.497028in}}{\pgfqpoint{4.130704in}{1.490186in}}{\pgfqpoint{4.135748in}{1.485142in}}%
\pgfpathcurveto{\pgfqpoint{4.140791in}{1.480099in}}{\pgfqpoint{4.147633in}{1.477265in}}{\pgfqpoint{4.154766in}{1.477265in}}%
\pgfpathclose%
\pgfusepath{stroke,fill}%
\end{pgfscope}%
\begin{pgfscope}%
\pgfpathrectangle{\pgfqpoint{2.867647in}{0.500000in}}{\pgfqpoint{1.764706in}{1.700000in}}%
\pgfusepath{clip}%
\pgfsetbuttcap%
\pgfsetroundjoin%
\definecolor{currentfill}{rgb}{0.966328,0.750560,0.616961}%
\pgfsetfillcolor{currentfill}%
\pgfsetlinewidth{0.311001pt}%
\definecolor{currentstroke}{rgb}{1.000000,1.000000,1.000000}%
\pgfsetstrokecolor{currentstroke}%
\pgfsetdash{}{0pt}%
\pgfpathmoveto{\pgfqpoint{4.296817in}{1.326114in}}%
\pgfpathcurveto{\pgfqpoint{4.303950in}{1.326114in}}{\pgfqpoint{4.310791in}{1.328948in}}{\pgfqpoint{4.315835in}{1.333992in}}%
\pgfpathcurveto{\pgfqpoint{4.320879in}{1.339035in}}{\pgfqpoint{4.323713in}{1.345877in}}{\pgfqpoint{4.323713in}{1.353010in}}%
\pgfpathcurveto{\pgfqpoint{4.323713in}{1.360143in}}{\pgfqpoint{4.320879in}{1.366984in}}{\pgfqpoint{4.315835in}{1.372028in}}%
\pgfpathcurveto{\pgfqpoint{4.310791in}{1.377072in}}{\pgfqpoint{4.303950in}{1.379905in}}{\pgfqpoint{4.296817in}{1.379905in}}%
\pgfpathcurveto{\pgfqpoint{4.289684in}{1.379905in}}{\pgfqpoint{4.282842in}{1.377072in}}{\pgfqpoint{4.277799in}{1.372028in}}%
\pgfpathcurveto{\pgfqpoint{4.272755in}{1.366984in}}{\pgfqpoint{4.269921in}{1.360143in}}{\pgfqpoint{4.269921in}{1.353010in}}%
\pgfpathcurveto{\pgfqpoint{4.269921in}{1.345877in}}{\pgfqpoint{4.272755in}{1.339035in}}{\pgfqpoint{4.277799in}{1.333992in}}%
\pgfpathcurveto{\pgfqpoint{4.282842in}{1.328948in}}{\pgfqpoint{4.289684in}{1.326114in}}{\pgfqpoint{4.296817in}{1.326114in}}%
\pgfpathclose%
\pgfusepath{stroke,fill}%
\end{pgfscope}%
\begin{pgfscope}%
\pgfpathrectangle{\pgfqpoint{2.867647in}{0.500000in}}{\pgfqpoint{1.764706in}{1.700000in}}%
\pgfusepath{clip}%
\pgfsetbuttcap%
\pgfsetroundjoin%
\definecolor{currentfill}{rgb}{0.960778,0.559972,0.399412}%
\pgfsetfillcolor{currentfill}%
\pgfsetlinewidth{0.311001pt}%
\definecolor{currentstroke}{rgb}{1.000000,1.000000,1.000000}%
\pgfsetstrokecolor{currentstroke}%
\pgfsetdash{}{0pt}%
\pgfpathmoveto{\pgfqpoint{3.934982in}{1.777294in}}%
\pgfpathcurveto{\pgfqpoint{3.942115in}{1.777294in}}{\pgfqpoint{3.948957in}{1.780128in}}{\pgfqpoint{3.954000in}{1.785171in}}%
\pgfpathcurveto{\pgfqpoint{3.959044in}{1.790215in}}{\pgfqpoint{3.961878in}{1.797057in}}{\pgfqpoint{3.961878in}{1.804190in}}%
\pgfpathcurveto{\pgfqpoint{3.961878in}{1.811322in}}{\pgfqpoint{3.959044in}{1.818164in}}{\pgfqpoint{3.954000in}{1.823208in}}%
\pgfpathcurveto{\pgfqpoint{3.948957in}{1.828251in}}{\pgfqpoint{3.942115in}{1.831085in}}{\pgfqpoint{3.934982in}{1.831085in}}%
\pgfpathcurveto{\pgfqpoint{3.927849in}{1.831085in}}{\pgfqpoint{3.921008in}{1.828251in}}{\pgfqpoint{3.915964in}{1.823208in}}%
\pgfpathcurveto{\pgfqpoint{3.910920in}{1.818164in}}{\pgfqpoint{3.908087in}{1.811322in}}{\pgfqpoint{3.908087in}{1.804190in}}%
\pgfpathcurveto{\pgfqpoint{3.908087in}{1.797057in}}{\pgfqpoint{3.910920in}{1.790215in}}{\pgfqpoint{3.915964in}{1.785171in}}%
\pgfpathcurveto{\pgfqpoint{3.921008in}{1.780128in}}{\pgfqpoint{3.927849in}{1.777294in}}{\pgfqpoint{3.934982in}{1.777294in}}%
\pgfpathclose%
\pgfusepath{stroke,fill}%
\end{pgfscope}%
\begin{pgfscope}%
\pgfpathrectangle{\pgfqpoint{2.867647in}{0.500000in}}{\pgfqpoint{1.764706in}{1.700000in}}%
\pgfusepath{clip}%
\pgfsetbuttcap%
\pgfsetroundjoin%
\definecolor{currentfill}{rgb}{0.950017,0.427714,0.292447}%
\pgfsetfillcolor{currentfill}%
\pgfsetlinewidth{0.311001pt}%
\definecolor{currentstroke}{rgb}{1.000000,1.000000,1.000000}%
\pgfsetstrokecolor{currentstroke}%
\pgfsetdash{}{0pt}%
\pgfpathmoveto{\pgfqpoint{4.361975in}{1.348192in}}%
\pgfpathcurveto{\pgfqpoint{4.369108in}{1.348192in}}{\pgfqpoint{4.375949in}{1.351026in}}{\pgfqpoint{4.380993in}{1.356069in}}%
\pgfpathcurveto{\pgfqpoint{4.386037in}{1.361113in}}{\pgfqpoint{4.388871in}{1.367955in}}{\pgfqpoint{4.388871in}{1.375088in}}%
\pgfpathcurveto{\pgfqpoint{4.388871in}{1.382220in}}{\pgfqpoint{4.386037in}{1.389062in}}{\pgfqpoint{4.380993in}{1.394106in}}%
\pgfpathcurveto{\pgfqpoint{4.375949in}{1.399149in}}{\pgfqpoint{4.369108in}{1.401983in}}{\pgfqpoint{4.361975in}{1.401983in}}%
\pgfpathcurveto{\pgfqpoint{4.354842in}{1.401983in}}{\pgfqpoint{4.348000in}{1.399149in}}{\pgfqpoint{4.342957in}{1.394106in}}%
\pgfpathcurveto{\pgfqpoint{4.337913in}{1.389062in}}{\pgfqpoint{4.335079in}{1.382220in}}{\pgfqpoint{4.335079in}{1.375088in}}%
\pgfpathcurveto{\pgfqpoint{4.335079in}{1.367955in}}{\pgfqpoint{4.337913in}{1.361113in}}{\pgfqpoint{4.342957in}{1.356069in}}%
\pgfpathcurveto{\pgfqpoint{4.348000in}{1.351026in}}{\pgfqpoint{4.354842in}{1.348192in}}{\pgfqpoint{4.361975in}{1.348192in}}%
\pgfpathclose%
\pgfusepath{stroke,fill}%
\end{pgfscope}%
\begin{pgfscope}%
\pgfpathrectangle{\pgfqpoint{2.867647in}{0.500000in}}{\pgfqpoint{1.764706in}{1.700000in}}%
\pgfusepath{clip}%
\pgfsetbuttcap%
\pgfsetroundjoin%
\definecolor{currentfill}{rgb}{0.975018,0.868213,0.788710}%
\pgfsetfillcolor{currentfill}%
\pgfsetlinewidth{0.311001pt}%
\definecolor{currentstroke}{rgb}{1.000000,1.000000,1.000000}%
\pgfsetstrokecolor{currentstroke}%
\pgfsetdash{}{0pt}%
\pgfpathmoveto{\pgfqpoint{4.127732in}{1.267059in}}%
\pgfpathcurveto{\pgfqpoint{4.134865in}{1.267059in}}{\pgfqpoint{4.141707in}{1.269893in}}{\pgfqpoint{4.146750in}{1.274936in}}%
\pgfpathcurveto{\pgfqpoint{4.151794in}{1.279980in}}{\pgfqpoint{4.154628in}{1.286822in}}{\pgfqpoint{4.154628in}{1.293955in}}%
\pgfpathcurveto{\pgfqpoint{4.154628in}{1.301087in}}{\pgfqpoint{4.151794in}{1.307929in}}{\pgfqpoint{4.146750in}{1.312973in}}%
\pgfpathcurveto{\pgfqpoint{4.141707in}{1.318016in}}{\pgfqpoint{4.134865in}{1.320850in}}{\pgfqpoint{4.127732in}{1.320850in}}%
\pgfpathcurveto{\pgfqpoint{4.120599in}{1.320850in}}{\pgfqpoint{4.113758in}{1.318016in}}{\pgfqpoint{4.108714in}{1.312973in}}%
\pgfpathcurveto{\pgfqpoint{4.103670in}{1.307929in}}{\pgfqpoint{4.100836in}{1.301087in}}{\pgfqpoint{4.100836in}{1.293955in}}%
\pgfpathcurveto{\pgfqpoint{4.100836in}{1.286822in}}{\pgfqpoint{4.103670in}{1.279980in}}{\pgfqpoint{4.108714in}{1.274936in}}%
\pgfpathcurveto{\pgfqpoint{4.113758in}{1.269893in}}{\pgfqpoint{4.120599in}{1.267059in}}{\pgfqpoint{4.127732in}{1.267059in}}%
\pgfpathclose%
\pgfusepath{stroke,fill}%
\end{pgfscope}%
\begin{pgfscope}%
\pgfpathrectangle{\pgfqpoint{2.867647in}{0.500000in}}{\pgfqpoint{1.764706in}{1.700000in}}%
\pgfusepath{clip}%
\pgfsetbuttcap%
\pgfsetroundjoin%
\definecolor{currentfill}{rgb}{0.961115,0.566634,0.405693}%
\pgfsetfillcolor{currentfill}%
\pgfsetlinewidth{0.311001pt}%
\definecolor{currentstroke}{rgb}{1.000000,1.000000,1.000000}%
\pgfsetstrokecolor{currentstroke}%
\pgfsetdash{}{0pt}%
\pgfpathmoveto{\pgfqpoint{4.304428in}{1.538324in}}%
\pgfpathcurveto{\pgfqpoint{4.311560in}{1.538324in}}{\pgfqpoint{4.318402in}{1.541158in}}{\pgfqpoint{4.323446in}{1.546202in}}%
\pgfpathcurveto{\pgfqpoint{4.328489in}{1.551246in}}{\pgfqpoint{4.331323in}{1.558087in}}{\pgfqpoint{4.331323in}{1.565220in}}%
\pgfpathcurveto{\pgfqpoint{4.331323in}{1.572353in}}{\pgfqpoint{4.328489in}{1.579195in}}{\pgfqpoint{4.323446in}{1.584238in}}%
\pgfpathcurveto{\pgfqpoint{4.318402in}{1.589282in}}{\pgfqpoint{4.311560in}{1.592116in}}{\pgfqpoint{4.304428in}{1.592116in}}%
\pgfpathcurveto{\pgfqpoint{4.297295in}{1.592116in}}{\pgfqpoint{4.290453in}{1.589282in}}{\pgfqpoint{4.285409in}{1.584238in}}%
\pgfpathcurveto{\pgfqpoint{4.280366in}{1.579195in}}{\pgfqpoint{4.277532in}{1.572353in}}{\pgfqpoint{4.277532in}{1.565220in}}%
\pgfpathcurveto{\pgfqpoint{4.277532in}{1.558087in}}{\pgfqpoint{4.280366in}{1.551246in}}{\pgfqpoint{4.285409in}{1.546202in}}%
\pgfpathcurveto{\pgfqpoint{4.290453in}{1.541158in}}{\pgfqpoint{4.297295in}{1.538324in}}{\pgfqpoint{4.304428in}{1.538324in}}%
\pgfpathclose%
\pgfusepath{stroke,fill}%
\end{pgfscope}%
\begin{pgfscope}%
\pgfpathrectangle{\pgfqpoint{2.867647in}{0.500000in}}{\pgfqpoint{1.764706in}{1.700000in}}%
\pgfusepath{clip}%
\pgfsetbuttcap%
\pgfsetroundjoin%
\definecolor{currentfill}{rgb}{0.967735,0.780441,0.659127}%
\pgfsetfillcolor{currentfill}%
\pgfsetlinewidth{0.311001pt}%
\definecolor{currentstroke}{rgb}{1.000000,1.000000,1.000000}%
\pgfsetstrokecolor{currentstroke}%
\pgfsetdash{}{0pt}%
\pgfpathmoveto{\pgfqpoint{4.027077in}{1.011458in}}%
\pgfpathcurveto{\pgfqpoint{4.034210in}{1.011458in}}{\pgfqpoint{4.041052in}{1.014292in}}{\pgfqpoint{4.046095in}{1.019336in}}%
\pgfpathcurveto{\pgfqpoint{4.051139in}{1.024380in}}{\pgfqpoint{4.053973in}{1.031221in}}{\pgfqpoint{4.053973in}{1.038354in}}%
\pgfpathcurveto{\pgfqpoint{4.053973in}{1.045487in}}{\pgfqpoint{4.051139in}{1.052329in}}{\pgfqpoint{4.046095in}{1.057372in}}%
\pgfpathcurveto{\pgfqpoint{4.041052in}{1.062416in}}{\pgfqpoint{4.034210in}{1.065250in}}{\pgfqpoint{4.027077in}{1.065250in}}%
\pgfpathcurveto{\pgfqpoint{4.019944in}{1.065250in}}{\pgfqpoint{4.013103in}{1.062416in}}{\pgfqpoint{4.008059in}{1.057372in}}%
\pgfpathcurveto{\pgfqpoint{4.003015in}{1.052329in}}{\pgfqpoint{4.000181in}{1.045487in}}{\pgfqpoint{4.000181in}{1.038354in}}%
\pgfpathcurveto{\pgfqpoint{4.000181in}{1.031221in}}{\pgfqpoint{4.003015in}{1.024380in}}{\pgfqpoint{4.008059in}{1.019336in}}%
\pgfpathcurveto{\pgfqpoint{4.013103in}{1.014292in}}{\pgfqpoint{4.019944in}{1.011458in}}{\pgfqpoint{4.027077in}{1.011458in}}%
\pgfpathclose%
\pgfusepath{stroke,fill}%
\end{pgfscope}%
\begin{pgfscope}%
\pgfpathrectangle{\pgfqpoint{2.867647in}{0.500000in}}{\pgfqpoint{1.764706in}{1.700000in}}%
\pgfusepath{clip}%
\pgfsetbuttcap%
\pgfsetroundjoin%
\definecolor{currentfill}{rgb}{0.968931,0.798091,0.685123}%
\pgfsetfillcolor{currentfill}%
\pgfsetlinewidth{0.311001pt}%
\definecolor{currentstroke}{rgb}{1.000000,1.000000,1.000000}%
\pgfsetstrokecolor{currentstroke}%
\pgfsetdash{}{0pt}%
\pgfpathmoveto{\pgfqpoint{4.278398in}{1.378357in}}%
\pgfpathcurveto{\pgfqpoint{4.285530in}{1.378357in}}{\pgfqpoint{4.292372in}{1.381191in}}{\pgfqpoint{4.297416in}{1.386234in}}%
\pgfpathcurveto{\pgfqpoint{4.302459in}{1.391278in}}{\pgfqpoint{4.305293in}{1.398120in}}{\pgfqpoint{4.305293in}{1.405252in}}%
\pgfpathcurveto{\pgfqpoint{4.305293in}{1.412385in}}{\pgfqpoint{4.302459in}{1.419227in}}{\pgfqpoint{4.297416in}{1.424271in}}%
\pgfpathcurveto{\pgfqpoint{4.292372in}{1.429314in}}{\pgfqpoint{4.285530in}{1.432148in}}{\pgfqpoint{4.278398in}{1.432148in}}%
\pgfpathcurveto{\pgfqpoint{4.271265in}{1.432148in}}{\pgfqpoint{4.264423in}{1.429314in}}{\pgfqpoint{4.259379in}{1.424271in}}%
\pgfpathcurveto{\pgfqpoint{4.254336in}{1.419227in}}{\pgfqpoint{4.251502in}{1.412385in}}{\pgfqpoint{4.251502in}{1.405252in}}%
\pgfpathcurveto{\pgfqpoint{4.251502in}{1.398120in}}{\pgfqpoint{4.254336in}{1.391278in}}{\pgfqpoint{4.259379in}{1.386234in}}%
\pgfpathcurveto{\pgfqpoint{4.264423in}{1.381191in}}{\pgfqpoint{4.271265in}{1.378357in}}{\pgfqpoint{4.278398in}{1.378357in}}%
\pgfpathclose%
\pgfusepath{stroke,fill}%
\end{pgfscope}%
\begin{pgfscope}%
\pgfpathrectangle{\pgfqpoint{2.867647in}{0.500000in}}{\pgfqpoint{1.764706in}{1.700000in}}%
\pgfusepath{clip}%
\pgfsetbuttcap%
\pgfsetroundjoin%
\definecolor{currentfill}{rgb}{0.980678,0.914765,0.856766}%
\pgfsetfillcolor{currentfill}%
\pgfsetlinewidth{0.311001pt}%
\definecolor{currentstroke}{rgb}{1.000000,1.000000,1.000000}%
\pgfsetstrokecolor{currentstroke}%
\pgfsetdash{}{0pt}%
\pgfpathmoveto{\pgfqpoint{4.145980in}{1.499547in}}%
\pgfpathcurveto{\pgfqpoint{4.153113in}{1.499547in}}{\pgfqpoint{4.159955in}{1.502381in}}{\pgfqpoint{4.164998in}{1.507425in}}%
\pgfpathcurveto{\pgfqpoint{4.170042in}{1.512468in}}{\pgfqpoint{4.172876in}{1.519310in}}{\pgfqpoint{4.172876in}{1.526443in}}%
\pgfpathcurveto{\pgfqpoint{4.172876in}{1.533576in}}{\pgfqpoint{4.170042in}{1.540417in}}{\pgfqpoint{4.164998in}{1.545461in}}%
\pgfpathcurveto{\pgfqpoint{4.159955in}{1.550505in}}{\pgfqpoint{4.153113in}{1.553338in}}{\pgfqpoint{4.145980in}{1.553338in}}%
\pgfpathcurveto{\pgfqpoint{4.138847in}{1.553338in}}{\pgfqpoint{4.132006in}{1.550505in}}{\pgfqpoint{4.126962in}{1.545461in}}%
\pgfpathcurveto{\pgfqpoint{4.121918in}{1.540417in}}{\pgfqpoint{4.119084in}{1.533576in}}{\pgfqpoint{4.119084in}{1.526443in}}%
\pgfpathcurveto{\pgfqpoint{4.119084in}{1.519310in}}{\pgfqpoint{4.121918in}{1.512468in}}{\pgfqpoint{4.126962in}{1.507425in}}%
\pgfpathcurveto{\pgfqpoint{4.132006in}{1.502381in}}{\pgfqpoint{4.138847in}{1.499547in}}{\pgfqpoint{4.145980in}{1.499547in}}%
\pgfpathclose%
\pgfusepath{stroke,fill}%
\end{pgfscope}%
\begin{pgfscope}%
\pgfpathrectangle{\pgfqpoint{2.867647in}{0.500000in}}{\pgfqpoint{1.764706in}{1.700000in}}%
\pgfusepath{clip}%
\pgfsetbuttcap%
\pgfsetroundjoin%
\definecolor{currentfill}{rgb}{0.968105,0.786346,0.667739}%
\pgfsetfillcolor{currentfill}%
\pgfsetlinewidth{0.311001pt}%
\definecolor{currentstroke}{rgb}{1.000000,1.000000,1.000000}%
\pgfsetstrokecolor{currentstroke}%
\pgfsetdash{}{0pt}%
\pgfpathmoveto{\pgfqpoint{4.071988in}{1.720687in}}%
\pgfpathcurveto{\pgfqpoint{4.079121in}{1.720687in}}{\pgfqpoint{4.085963in}{1.723521in}}{\pgfqpoint{4.091007in}{1.728564in}}%
\pgfpathcurveto{\pgfqpoint{4.096050in}{1.733608in}}{\pgfqpoint{4.098884in}{1.740450in}}{\pgfqpoint{4.098884in}{1.747582in}}%
\pgfpathcurveto{\pgfqpoint{4.098884in}{1.754715in}}{\pgfqpoint{4.096050in}{1.761557in}}{\pgfqpoint{4.091007in}{1.766601in}}%
\pgfpathcurveto{\pgfqpoint{4.085963in}{1.771644in}}{\pgfqpoint{4.079121in}{1.774478in}}{\pgfqpoint{4.071988in}{1.774478in}}%
\pgfpathcurveto{\pgfqpoint{4.064856in}{1.774478in}}{\pgfqpoint{4.058014in}{1.771644in}}{\pgfqpoint{4.052970in}{1.766601in}}%
\pgfpathcurveto{\pgfqpoint{4.047927in}{1.761557in}}{\pgfqpoint{4.045093in}{1.754715in}}{\pgfqpoint{4.045093in}{1.747582in}}%
\pgfpathcurveto{\pgfqpoint{4.045093in}{1.740450in}}{\pgfqpoint{4.047927in}{1.733608in}}{\pgfqpoint{4.052970in}{1.728564in}}%
\pgfpathcurveto{\pgfqpoint{4.058014in}{1.723521in}}{\pgfqpoint{4.064856in}{1.720687in}}{\pgfqpoint{4.071988in}{1.720687in}}%
\pgfpathclose%
\pgfusepath{stroke,fill}%
\end{pgfscope}%
\begin{pgfscope}%
\pgfpathrectangle{\pgfqpoint{2.867647in}{0.500000in}}{\pgfqpoint{1.764706in}{1.700000in}}%
\pgfusepath{clip}%
\pgfsetbuttcap%
\pgfsetroundjoin%
\definecolor{currentfill}{rgb}{0.964173,0.657587,0.500469}%
\pgfsetfillcolor{currentfill}%
\pgfsetlinewidth{0.311001pt}%
\definecolor{currentstroke}{rgb}{1.000000,1.000000,1.000000}%
\pgfsetstrokecolor{currentstroke}%
\pgfsetdash{}{0pt}%
\pgfpathmoveto{\pgfqpoint{4.217098in}{1.671904in}}%
\pgfpathcurveto{\pgfqpoint{4.224231in}{1.671904in}}{\pgfqpoint{4.231072in}{1.674738in}}{\pgfqpoint{4.236116in}{1.679781in}}%
\pgfpathcurveto{\pgfqpoint{4.241160in}{1.684825in}}{\pgfqpoint{4.243994in}{1.691667in}}{\pgfqpoint{4.243994in}{1.698799in}}%
\pgfpathcurveto{\pgfqpoint{4.243994in}{1.705932in}}{\pgfqpoint{4.241160in}{1.712774in}}{\pgfqpoint{4.236116in}{1.717818in}}%
\pgfpathcurveto{\pgfqpoint{4.231072in}{1.722861in}}{\pgfqpoint{4.224231in}{1.725695in}}{\pgfqpoint{4.217098in}{1.725695in}}%
\pgfpathcurveto{\pgfqpoint{4.209965in}{1.725695in}}{\pgfqpoint{4.203123in}{1.722861in}}{\pgfqpoint{4.198080in}{1.717818in}}%
\pgfpathcurveto{\pgfqpoint{4.193036in}{1.712774in}}{\pgfqpoint{4.190202in}{1.705932in}}{\pgfqpoint{4.190202in}{1.698799in}}%
\pgfpathcurveto{\pgfqpoint{4.190202in}{1.691667in}}{\pgfqpoint{4.193036in}{1.684825in}}{\pgfqpoint{4.198080in}{1.679781in}}%
\pgfpathcurveto{\pgfqpoint{4.203123in}{1.674738in}}{\pgfqpoint{4.209965in}{1.671904in}}{\pgfqpoint{4.217098in}{1.671904in}}%
\pgfpathclose%
\pgfusepath{stroke,fill}%
\end{pgfscope}%
\begin{pgfscope}%
\pgfpathrectangle{\pgfqpoint{2.867647in}{0.500000in}}{\pgfqpoint{1.764706in}{1.700000in}}%
\pgfusepath{clip}%
\pgfsetbuttcap%
\pgfsetroundjoin%
\definecolor{currentfill}{rgb}{0.971202,0.827364,0.728520}%
\pgfsetfillcolor{currentfill}%
\pgfsetlinewidth{0.311001pt}%
\definecolor{currentstroke}{rgb}{1.000000,1.000000,1.000000}%
\pgfsetstrokecolor{currentstroke}%
\pgfsetdash{}{0pt}%
\pgfpathmoveto{\pgfqpoint{4.129165in}{1.675845in}}%
\pgfpathcurveto{\pgfqpoint{4.136298in}{1.675845in}}{\pgfqpoint{4.143140in}{1.678679in}}{\pgfqpoint{4.148184in}{1.683722in}}%
\pgfpathcurveto{\pgfqpoint{4.153227in}{1.688766in}}{\pgfqpoint{4.156061in}{1.695608in}}{\pgfqpoint{4.156061in}{1.702741in}}%
\pgfpathcurveto{\pgfqpoint{4.156061in}{1.709873in}}{\pgfqpoint{4.153227in}{1.716715in}}{\pgfqpoint{4.148184in}{1.721759in}}%
\pgfpathcurveto{\pgfqpoint{4.143140in}{1.726802in}}{\pgfqpoint{4.136298in}{1.729636in}}{\pgfqpoint{4.129165in}{1.729636in}}%
\pgfpathcurveto{\pgfqpoint{4.122033in}{1.729636in}}{\pgfqpoint{4.115191in}{1.726802in}}{\pgfqpoint{4.110147in}{1.721759in}}%
\pgfpathcurveto{\pgfqpoint{4.105104in}{1.716715in}}{\pgfqpoint{4.102270in}{1.709873in}}{\pgfqpoint{4.102270in}{1.702741in}}%
\pgfpathcurveto{\pgfqpoint{4.102270in}{1.695608in}}{\pgfqpoint{4.105104in}{1.688766in}}{\pgfqpoint{4.110147in}{1.683722in}}%
\pgfpathcurveto{\pgfqpoint{4.115191in}{1.678679in}}{\pgfqpoint{4.122033in}{1.675845in}}{\pgfqpoint{4.129165in}{1.675845in}}%
\pgfpathclose%
\pgfusepath{stroke,fill}%
\end{pgfscope}%
\begin{pgfscope}%
\pgfpathrectangle{\pgfqpoint{2.867647in}{0.500000in}}{\pgfqpoint{1.764706in}{1.700000in}}%
\pgfusepath{clip}%
\pgfsetbuttcap%
\pgfsetroundjoin%
\definecolor{currentfill}{rgb}{0.972726,0.844889,0.754401}%
\pgfsetfillcolor{currentfill}%
\pgfsetlinewidth{0.311001pt}%
\definecolor{currentstroke}{rgb}{1.000000,1.000000,1.000000}%
\pgfsetstrokecolor{currentstroke}%
\pgfsetdash{}{0pt}%
\pgfpathmoveto{\pgfqpoint{4.082342in}{1.115379in}}%
\pgfpathcurveto{\pgfqpoint{4.089475in}{1.115379in}}{\pgfqpoint{4.096317in}{1.118213in}}{\pgfqpoint{4.101361in}{1.123257in}}%
\pgfpathcurveto{\pgfqpoint{4.106404in}{1.128300in}}{\pgfqpoint{4.109238in}{1.135142in}}{\pgfqpoint{4.109238in}{1.142275in}}%
\pgfpathcurveto{\pgfqpoint{4.109238in}{1.149408in}}{\pgfqpoint{4.106404in}{1.156249in}}{\pgfqpoint{4.101361in}{1.161293in}}%
\pgfpathcurveto{\pgfqpoint{4.096317in}{1.166337in}}{\pgfqpoint{4.089475in}{1.169171in}}{\pgfqpoint{4.082342in}{1.169171in}}%
\pgfpathcurveto{\pgfqpoint{4.075210in}{1.169171in}}{\pgfqpoint{4.068368in}{1.166337in}}{\pgfqpoint{4.063324in}{1.161293in}}%
\pgfpathcurveto{\pgfqpoint{4.058281in}{1.156249in}}{\pgfqpoint{4.055447in}{1.149408in}}{\pgfqpoint{4.055447in}{1.142275in}}%
\pgfpathcurveto{\pgfqpoint{4.055447in}{1.135142in}}{\pgfqpoint{4.058281in}{1.128300in}}{\pgfqpoint{4.063324in}{1.123257in}}%
\pgfpathcurveto{\pgfqpoint{4.068368in}{1.118213in}}{\pgfqpoint{4.075210in}{1.115379in}}{\pgfqpoint{4.082342in}{1.115379in}}%
\pgfpathclose%
\pgfusepath{stroke,fill}%
\end{pgfscope}%
\begin{pgfscope}%
\pgfpathrectangle{\pgfqpoint{2.867647in}{0.500000in}}{\pgfqpoint{1.764706in}{1.700000in}}%
\pgfusepath{clip}%
\pgfsetbuttcap%
\pgfsetroundjoin%
\definecolor{currentfill}{rgb}{0.965302,0.713942,0.568499}%
\pgfsetfillcolor{currentfill}%
\pgfsetlinewidth{0.311001pt}%
\definecolor{currentstroke}{rgb}{1.000000,1.000000,1.000000}%
\pgfsetstrokecolor{currentstroke}%
\pgfsetdash{}{0pt}%
\pgfpathmoveto{\pgfqpoint{4.009014in}{1.047947in}}%
\pgfpathcurveto{\pgfqpoint{4.016147in}{1.047947in}}{\pgfqpoint{4.022988in}{1.050781in}}{\pgfqpoint{4.028032in}{1.055824in}}%
\pgfpathcurveto{\pgfqpoint{4.033076in}{1.060868in}}{\pgfqpoint{4.035910in}{1.067710in}}{\pgfqpoint{4.035910in}{1.074843in}}%
\pgfpathcurveto{\pgfqpoint{4.035910in}{1.081975in}}{\pgfqpoint{4.033076in}{1.088817in}}{\pgfqpoint{4.028032in}{1.093861in}}%
\pgfpathcurveto{\pgfqpoint{4.022988in}{1.098904in}}{\pgfqpoint{4.016147in}{1.101738in}}{\pgfqpoint{4.009014in}{1.101738in}}%
\pgfpathcurveto{\pgfqpoint{4.001881in}{1.101738in}}{\pgfqpoint{3.995039in}{1.098904in}}{\pgfqpoint{3.989996in}{1.093861in}}%
\pgfpathcurveto{\pgfqpoint{3.984952in}{1.088817in}}{\pgfqpoint{3.982118in}{1.081975in}}{\pgfqpoint{3.982118in}{1.074843in}}%
\pgfpathcurveto{\pgfqpoint{3.982118in}{1.067710in}}{\pgfqpoint{3.984952in}{1.060868in}}{\pgfqpoint{3.989996in}{1.055824in}}%
\pgfpathcurveto{\pgfqpoint{3.995039in}{1.050781in}}{\pgfqpoint{4.001881in}{1.047947in}}{\pgfqpoint{4.009014in}{1.047947in}}%
\pgfpathclose%
\pgfusepath{stroke,fill}%
\end{pgfscope}%
\begin{pgfscope}%
\pgfpathrectangle{\pgfqpoint{2.867647in}{0.500000in}}{\pgfqpoint{1.764706in}{1.700000in}}%
\pgfusepath{clip}%
\pgfsetbuttcap%
\pgfsetroundjoin%
\definecolor{currentfill}{rgb}{0.965169,0.707764,0.560659}%
\pgfsetfillcolor{currentfill}%
\pgfsetlinewidth{0.311001pt}%
\definecolor{currentstroke}{rgb}{1.000000,1.000000,1.000000}%
\pgfsetstrokecolor{currentstroke}%
\pgfsetdash{}{0pt}%
\pgfpathmoveto{\pgfqpoint{4.220870in}{0.998676in}}%
\pgfpathcurveto{\pgfqpoint{4.228003in}{0.998676in}}{\pgfqpoint{4.234844in}{1.001510in}}{\pgfqpoint{4.239888in}{1.006554in}}%
\pgfpathcurveto{\pgfqpoint{4.244932in}{1.011598in}}{\pgfqpoint{4.247766in}{1.018439in}}{\pgfqpoint{4.247766in}{1.025572in}}%
\pgfpathcurveto{\pgfqpoint{4.247766in}{1.032705in}}{\pgfqpoint{4.244932in}{1.039546in}}{\pgfqpoint{4.239888in}{1.044590in}}%
\pgfpathcurveto{\pgfqpoint{4.234844in}{1.049634in}}{\pgfqpoint{4.228003in}{1.052468in}}{\pgfqpoint{4.220870in}{1.052468in}}%
\pgfpathcurveto{\pgfqpoint{4.213737in}{1.052468in}}{\pgfqpoint{4.206895in}{1.049634in}}{\pgfqpoint{4.201852in}{1.044590in}}%
\pgfpathcurveto{\pgfqpoint{4.196808in}{1.039546in}}{\pgfqpoint{4.193974in}{1.032705in}}{\pgfqpoint{4.193974in}{1.025572in}}%
\pgfpathcurveto{\pgfqpoint{4.193974in}{1.018439in}}{\pgfqpoint{4.196808in}{1.011598in}}{\pgfqpoint{4.201852in}{1.006554in}}%
\pgfpathcurveto{\pgfqpoint{4.206895in}{1.001510in}}{\pgfqpoint{4.213737in}{0.998676in}}{\pgfqpoint{4.220870in}{0.998676in}}%
\pgfpathclose%
\pgfusepath{stroke,fill}%
\end{pgfscope}%
\begin{pgfscope}%
\pgfpathrectangle{\pgfqpoint{2.867647in}{0.500000in}}{\pgfqpoint{1.764706in}{1.700000in}}%
\pgfusepath{clip}%
\pgfsetbuttcap%
\pgfsetroundjoin%
\definecolor{currentfill}{rgb}{0.975644,0.874038,0.797253}%
\pgfsetfillcolor{currentfill}%
\pgfsetlinewidth{0.311001pt}%
\definecolor{currentstroke}{rgb}{1.000000,1.000000,1.000000}%
\pgfsetstrokecolor{currentstroke}%
\pgfsetdash{}{0pt}%
\pgfpathmoveto{\pgfqpoint{4.242036in}{1.370944in}}%
\pgfpathcurveto{\pgfqpoint{4.249169in}{1.370944in}}{\pgfqpoint{4.256010in}{1.373778in}}{\pgfqpoint{4.261054in}{1.378822in}}%
\pgfpathcurveto{\pgfqpoint{4.266098in}{1.383866in}}{\pgfqpoint{4.268932in}{1.390707in}}{\pgfqpoint{4.268932in}{1.397840in}}%
\pgfpathcurveto{\pgfqpoint{4.268932in}{1.404973in}}{\pgfqpoint{4.266098in}{1.411815in}}{\pgfqpoint{4.261054in}{1.416858in}}%
\pgfpathcurveto{\pgfqpoint{4.256010in}{1.421902in}}{\pgfqpoint{4.249169in}{1.424736in}}{\pgfqpoint{4.242036in}{1.424736in}}%
\pgfpathcurveto{\pgfqpoint{4.234903in}{1.424736in}}{\pgfqpoint{4.228061in}{1.421902in}}{\pgfqpoint{4.223018in}{1.416858in}}%
\pgfpathcurveto{\pgfqpoint{4.217974in}{1.411815in}}{\pgfqpoint{4.215140in}{1.404973in}}{\pgfqpoint{4.215140in}{1.397840in}}%
\pgfpathcurveto{\pgfqpoint{4.215140in}{1.390707in}}{\pgfqpoint{4.217974in}{1.383866in}}{\pgfqpoint{4.223018in}{1.378822in}}%
\pgfpathcurveto{\pgfqpoint{4.228061in}{1.373778in}}{\pgfqpoint{4.234903in}{1.370944in}}{\pgfqpoint{4.242036in}{1.370944in}}%
\pgfpathclose%
\pgfusepath{stroke,fill}%
\end{pgfscope}%
\begin{pgfscope}%
\pgfpathrectangle{\pgfqpoint{2.867647in}{0.500000in}}{\pgfqpoint{1.764706in}{1.700000in}}%
\pgfusepath{clip}%
\pgfsetbuttcap%
\pgfsetroundjoin%
\definecolor{currentfill}{rgb}{0.968509,0.792226,0.676405}%
\pgfsetfillcolor{currentfill}%
\pgfsetlinewidth{0.311001pt}%
\definecolor{currentstroke}{rgb}{1.000000,1.000000,1.000000}%
\pgfsetstrokecolor{currentstroke}%
\pgfsetdash{}{0pt}%
\pgfpathmoveto{\pgfqpoint{4.032605in}{1.667485in}}%
\pgfpathcurveto{\pgfqpoint{4.039738in}{1.667485in}}{\pgfqpoint{4.046579in}{1.670319in}}{\pgfqpoint{4.051623in}{1.675362in}}%
\pgfpathcurveto{\pgfqpoint{4.056667in}{1.680406in}}{\pgfqpoint{4.059500in}{1.687248in}}{\pgfqpoint{4.059500in}{1.694381in}}%
\pgfpathcurveto{\pgfqpoint{4.059500in}{1.701513in}}{\pgfqpoint{4.056667in}{1.708355in}}{\pgfqpoint{4.051623in}{1.713399in}}%
\pgfpathcurveto{\pgfqpoint{4.046579in}{1.718442in}}{\pgfqpoint{4.039738in}{1.721276in}}{\pgfqpoint{4.032605in}{1.721276in}}%
\pgfpathcurveto{\pgfqpoint{4.025472in}{1.721276in}}{\pgfqpoint{4.018630in}{1.718442in}}{\pgfqpoint{4.013587in}{1.713399in}}%
\pgfpathcurveto{\pgfqpoint{4.008543in}{1.708355in}}{\pgfqpoint{4.005709in}{1.701513in}}{\pgfqpoint{4.005709in}{1.694381in}}%
\pgfpathcurveto{\pgfqpoint{4.005709in}{1.687248in}}{\pgfqpoint{4.008543in}{1.680406in}}{\pgfqpoint{4.013587in}{1.675362in}}%
\pgfpathcurveto{\pgfqpoint{4.018630in}{1.670319in}}{\pgfqpoint{4.025472in}{1.667485in}}{\pgfqpoint{4.032605in}{1.667485in}}%
\pgfpathclose%
\pgfusepath{stroke,fill}%
\end{pgfscope}%
\begin{pgfscope}%
\pgfpathrectangle{\pgfqpoint{2.867647in}{0.500000in}}{\pgfqpoint{1.764706in}{1.700000in}}%
\pgfusepath{clip}%
\pgfsetbuttcap%
\pgfsetroundjoin%
\definecolor{currentfill}{rgb}{0.975018,0.868213,0.788710}%
\pgfsetfillcolor{currentfill}%
\pgfsetlinewidth{0.311001pt}%
\definecolor{currentstroke}{rgb}{1.000000,1.000000,1.000000}%
\pgfsetstrokecolor{currentstroke}%
\pgfsetdash{}{0pt}%
\pgfpathmoveto{\pgfqpoint{4.221110in}{1.501040in}}%
\pgfpathcurveto{\pgfqpoint{4.228243in}{1.501040in}}{\pgfqpoint{4.235084in}{1.503874in}}{\pgfqpoint{4.240128in}{1.508918in}}%
\pgfpathcurveto{\pgfqpoint{4.245172in}{1.513962in}}{\pgfqpoint{4.248006in}{1.520803in}}{\pgfqpoint{4.248006in}{1.527936in}}%
\pgfpathcurveto{\pgfqpoint{4.248006in}{1.535069in}}{\pgfqpoint{4.245172in}{1.541911in}}{\pgfqpoint{4.240128in}{1.546954in}}%
\pgfpathcurveto{\pgfqpoint{4.235084in}{1.551998in}}{\pgfqpoint{4.228243in}{1.554832in}}{\pgfqpoint{4.221110in}{1.554832in}}%
\pgfpathcurveto{\pgfqpoint{4.213977in}{1.554832in}}{\pgfqpoint{4.207135in}{1.551998in}}{\pgfqpoint{4.202092in}{1.546954in}}%
\pgfpathcurveto{\pgfqpoint{4.197048in}{1.541911in}}{\pgfqpoint{4.194214in}{1.535069in}}{\pgfqpoint{4.194214in}{1.527936in}}%
\pgfpathcurveto{\pgfqpoint{4.194214in}{1.520803in}}{\pgfqpoint{4.197048in}{1.513962in}}{\pgfqpoint{4.202092in}{1.508918in}}%
\pgfpathcurveto{\pgfqpoint{4.207135in}{1.503874in}}{\pgfqpoint{4.213977in}{1.501040in}}{\pgfqpoint{4.221110in}{1.501040in}}%
\pgfpathclose%
\pgfusepath{stroke,fill}%
\end{pgfscope}%
\begin{pgfscope}%
\pgfpathrectangle{\pgfqpoint{2.867647in}{0.500000in}}{\pgfqpoint{1.764706in}{1.700000in}}%
\pgfusepath{clip}%
\pgfsetbuttcap%
\pgfsetroundjoin%
\definecolor{currentfill}{rgb}{0.965440,0.720101,0.576404}%
\pgfsetfillcolor{currentfill}%
\pgfsetlinewidth{0.311001pt}%
\definecolor{currentstroke}{rgb}{1.000000,1.000000,1.000000}%
\pgfsetstrokecolor{currentstroke}%
\pgfsetdash{}{0pt}%
\pgfpathmoveto{\pgfqpoint{4.076000in}{1.276494in}}%
\pgfpathcurveto{\pgfqpoint{4.083133in}{1.276494in}}{\pgfqpoint{4.089974in}{1.279328in}}{\pgfqpoint{4.095018in}{1.284372in}}%
\pgfpathcurveto{\pgfqpoint{4.100062in}{1.289415in}}{\pgfqpoint{4.102896in}{1.296257in}}{\pgfqpoint{4.102896in}{1.303390in}}%
\pgfpathcurveto{\pgfqpoint{4.102896in}{1.310523in}}{\pgfqpoint{4.100062in}{1.317364in}}{\pgfqpoint{4.095018in}{1.322408in}}%
\pgfpathcurveto{\pgfqpoint{4.089974in}{1.327452in}}{\pgfqpoint{4.083133in}{1.330286in}}{\pgfqpoint{4.076000in}{1.330286in}}%
\pgfpathcurveto{\pgfqpoint{4.068867in}{1.330286in}}{\pgfqpoint{4.062025in}{1.327452in}}{\pgfqpoint{4.056982in}{1.322408in}}%
\pgfpathcurveto{\pgfqpoint{4.051938in}{1.317364in}}{\pgfqpoint{4.049104in}{1.310523in}}{\pgfqpoint{4.049104in}{1.303390in}}%
\pgfpathcurveto{\pgfqpoint{4.049104in}{1.296257in}}{\pgfqpoint{4.051938in}{1.289415in}}{\pgfqpoint{4.056982in}{1.284372in}}%
\pgfpathcurveto{\pgfqpoint{4.062025in}{1.279328in}}{\pgfqpoint{4.068867in}{1.276494in}}{\pgfqpoint{4.076000in}{1.276494in}}%
\pgfpathclose%
\pgfusepath{stroke,fill}%
\end{pgfscope}%
\begin{pgfscope}%
\pgfpathrectangle{\pgfqpoint{2.867647in}{0.500000in}}{\pgfqpoint{1.764706in}{1.700000in}}%
\pgfusepath{clip}%
\pgfsetbuttcap%
\pgfsetroundjoin%
\definecolor{currentfill}{rgb}{0.980678,0.914765,0.856766}%
\pgfsetfillcolor{currentfill}%
\pgfsetlinewidth{0.311001pt}%
\definecolor{currentstroke}{rgb}{1.000000,1.000000,1.000000}%
\pgfsetstrokecolor{currentstroke}%
\pgfsetdash{}{0pt}%
\pgfpathmoveto{\pgfqpoint{4.145821in}{1.151600in}}%
\pgfpathcurveto{\pgfqpoint{4.152954in}{1.151600in}}{\pgfqpoint{4.159796in}{1.154434in}}{\pgfqpoint{4.164839in}{1.159477in}}%
\pgfpathcurveto{\pgfqpoint{4.169883in}{1.164521in}}{\pgfqpoint{4.172717in}{1.171363in}}{\pgfqpoint{4.172717in}{1.178496in}}%
\pgfpathcurveto{\pgfqpoint{4.172717in}{1.185628in}}{\pgfqpoint{4.169883in}{1.192470in}}{\pgfqpoint{4.164839in}{1.197514in}}%
\pgfpathcurveto{\pgfqpoint{4.159796in}{1.202557in}}{\pgfqpoint{4.152954in}{1.205391in}}{\pgfqpoint{4.145821in}{1.205391in}}%
\pgfpathcurveto{\pgfqpoint{4.138688in}{1.205391in}}{\pgfqpoint{4.131847in}{1.202557in}}{\pgfqpoint{4.126803in}{1.197514in}}%
\pgfpathcurveto{\pgfqpoint{4.121759in}{1.192470in}}{\pgfqpoint{4.118926in}{1.185628in}}{\pgfqpoint{4.118926in}{1.178496in}}%
\pgfpathcurveto{\pgfqpoint{4.118926in}{1.171363in}}{\pgfqpoint{4.121759in}{1.164521in}}{\pgfqpoint{4.126803in}{1.159477in}}%
\pgfpathcurveto{\pgfqpoint{4.131847in}{1.154434in}}{\pgfqpoint{4.138688in}{1.151600in}}{\pgfqpoint{4.145821in}{1.151600in}}%
\pgfpathclose%
\pgfusepath{stroke,fill}%
\end{pgfscope}%
\begin{pgfscope}%
\pgfpathrectangle{\pgfqpoint{2.867647in}{0.500000in}}{\pgfqpoint{1.764706in}{1.700000in}}%
\pgfusepath{clip}%
\pgfsetbuttcap%
\pgfsetroundjoin%
\definecolor{currentfill}{rgb}{0.970718,0.821518,0.719872}%
\pgfsetfillcolor{currentfill}%
\pgfsetlinewidth{0.311001pt}%
\definecolor{currentstroke}{rgb}{1.000000,1.000000,1.000000}%
\pgfsetstrokecolor{currentstroke}%
\pgfsetdash{}{0pt}%
\pgfpathmoveto{\pgfqpoint{4.119910in}{0.963370in}}%
\pgfpathcurveto{\pgfqpoint{4.127043in}{0.963370in}}{\pgfqpoint{4.133884in}{0.966204in}}{\pgfqpoint{4.138928in}{0.971248in}}%
\pgfpathcurveto{\pgfqpoint{4.143972in}{0.976292in}}{\pgfqpoint{4.146806in}{0.983133in}}{\pgfqpoint{4.146806in}{0.990266in}}%
\pgfpathcurveto{\pgfqpoint{4.146806in}{0.997399in}}{\pgfqpoint{4.143972in}{1.004241in}}{\pgfqpoint{4.138928in}{1.009284in}}%
\pgfpathcurveto{\pgfqpoint{4.133884in}{1.014328in}}{\pgfqpoint{4.127043in}{1.017162in}}{\pgfqpoint{4.119910in}{1.017162in}}%
\pgfpathcurveto{\pgfqpoint{4.112777in}{1.017162in}}{\pgfqpoint{4.105935in}{1.014328in}}{\pgfqpoint{4.100892in}{1.009284in}}%
\pgfpathcurveto{\pgfqpoint{4.095848in}{1.004241in}}{\pgfqpoint{4.093014in}{0.997399in}}{\pgfqpoint{4.093014in}{0.990266in}}%
\pgfpathcurveto{\pgfqpoint{4.093014in}{0.983133in}}{\pgfqpoint{4.095848in}{0.976292in}}{\pgfqpoint{4.100892in}{0.971248in}}%
\pgfpathcurveto{\pgfqpoint{4.105935in}{0.966204in}}{\pgfqpoint{4.112777in}{0.963370in}}{\pgfqpoint{4.119910in}{0.963370in}}%
\pgfpathclose%
\pgfusepath{stroke,fill}%
\end{pgfscope}%
\begin{pgfscope}%
\pgfpathrectangle{\pgfqpoint{2.867647in}{0.500000in}}{\pgfqpoint{1.764706in}{1.700000in}}%
\pgfusepath{clip}%
\pgfsetbuttcap%
\pgfsetroundjoin%
\definecolor{currentfill}{rgb}{0.146334,0.079734,0.198615}%
\pgfsetfillcolor{currentfill}%
\pgfsetlinewidth{0.311001pt}%
\definecolor{currentstroke}{rgb}{1.000000,1.000000,1.000000}%
\pgfsetstrokecolor{currentstroke}%
\pgfsetdash{}{0pt}%
\pgfpathmoveto{\pgfqpoint{3.600039in}{0.724237in}}%
\pgfpathcurveto{\pgfqpoint{3.607172in}{0.724237in}}{\pgfqpoint{3.614013in}{0.727071in}}{\pgfqpoint{3.619057in}{0.732114in}}%
\pgfpathcurveto{\pgfqpoint{3.624101in}{0.737158in}}{\pgfqpoint{3.626935in}{0.744000in}}{\pgfqpoint{3.626935in}{0.751132in}}%
\pgfpathcurveto{\pgfqpoint{3.626935in}{0.758265in}}{\pgfqpoint{3.624101in}{0.765107in}}{\pgfqpoint{3.619057in}{0.770151in}}%
\pgfpathcurveto{\pgfqpoint{3.614013in}{0.775194in}}{\pgfqpoint{3.607172in}{0.778028in}}{\pgfqpoint{3.600039in}{0.778028in}}%
\pgfpathcurveto{\pgfqpoint{3.592906in}{0.778028in}}{\pgfqpoint{3.586064in}{0.775194in}}{\pgfqpoint{3.581021in}{0.770151in}}%
\pgfpathcurveto{\pgfqpoint{3.575977in}{0.765107in}}{\pgfqpoint{3.573143in}{0.758265in}}{\pgfqpoint{3.573143in}{0.751132in}}%
\pgfpathcurveto{\pgfqpoint{3.573143in}{0.744000in}}{\pgfqpoint{3.575977in}{0.737158in}}{\pgfqpoint{3.581021in}{0.732114in}}%
\pgfpathcurveto{\pgfqpoint{3.586064in}{0.727071in}}{\pgfqpoint{3.592906in}{0.724237in}}{\pgfqpoint{3.600039in}{0.724237in}}%
\pgfpathclose%
\pgfusepath{stroke,fill}%
\end{pgfscope}%
\begin{pgfscope}%
\pgfpathrectangle{\pgfqpoint{2.867647in}{0.500000in}}{\pgfqpoint{1.764706in}{1.700000in}}%
\pgfusepath{clip}%
\pgfsetbuttcap%
\pgfsetroundjoin%
\definecolor{currentfill}{rgb}{0.968509,0.792226,0.676405}%
\pgfsetfillcolor{currentfill}%
\pgfsetlinewidth{0.311001pt}%
\definecolor{currentstroke}{rgb}{1.000000,1.000000,1.000000}%
\pgfsetstrokecolor{currentstroke}%
\pgfsetdash{}{0pt}%
\pgfpathmoveto{\pgfqpoint{4.132333in}{0.951529in}}%
\pgfpathcurveto{\pgfqpoint{4.139466in}{0.951529in}}{\pgfqpoint{4.146307in}{0.954363in}}{\pgfqpoint{4.151351in}{0.959407in}}%
\pgfpathcurveto{\pgfqpoint{4.156395in}{0.964450in}}{\pgfqpoint{4.159229in}{0.971292in}}{\pgfqpoint{4.159229in}{0.978425in}}%
\pgfpathcurveto{\pgfqpoint{4.159229in}{0.985558in}}{\pgfqpoint{4.156395in}{0.992399in}}{\pgfqpoint{4.151351in}{0.997443in}}%
\pgfpathcurveto{\pgfqpoint{4.146307in}{1.002487in}}{\pgfqpoint{4.139466in}{1.005320in}}{\pgfqpoint{4.132333in}{1.005320in}}%
\pgfpathcurveto{\pgfqpoint{4.125200in}{1.005320in}}{\pgfqpoint{4.118359in}{1.002487in}}{\pgfqpoint{4.113315in}{0.997443in}}%
\pgfpathcurveto{\pgfqpoint{4.108271in}{0.992399in}}{\pgfqpoint{4.105437in}{0.985558in}}{\pgfqpoint{4.105437in}{0.978425in}}%
\pgfpathcurveto{\pgfqpoint{4.105437in}{0.971292in}}{\pgfqpoint{4.108271in}{0.964450in}}{\pgfqpoint{4.113315in}{0.959407in}}%
\pgfpathcurveto{\pgfqpoint{4.118359in}{0.954363in}}{\pgfqpoint{4.125200in}{0.951529in}}{\pgfqpoint{4.132333in}{0.951529in}}%
\pgfpathclose%
\pgfusepath{stroke,fill}%
\end{pgfscope}%
\begin{pgfscope}%
\pgfpathrectangle{\pgfqpoint{2.867647in}{0.500000in}}{\pgfqpoint{1.764706in}{1.700000in}}%
\pgfusepath{clip}%
\pgfsetbuttcap%
\pgfsetroundjoin%
\definecolor{currentfill}{rgb}{0.963559,0.632016,0.472047}%
\pgfsetfillcolor{currentfill}%
\pgfsetlinewidth{0.311001pt}%
\definecolor{currentstroke}{rgb}{1.000000,1.000000,1.000000}%
\pgfsetstrokecolor{currentstroke}%
\pgfsetdash{}{0pt}%
\pgfpathmoveto{\pgfqpoint{4.031516in}{1.458767in}}%
\pgfpathcurveto{\pgfqpoint{4.038649in}{1.458767in}}{\pgfqpoint{4.045491in}{1.461601in}}{\pgfqpoint{4.050534in}{1.466644in}}%
\pgfpathcurveto{\pgfqpoint{4.055578in}{1.471688in}}{\pgfqpoint{4.058412in}{1.478530in}}{\pgfqpoint{4.058412in}{1.485662in}}%
\pgfpathcurveto{\pgfqpoint{4.058412in}{1.492795in}}{\pgfqpoint{4.055578in}{1.499637in}}{\pgfqpoint{4.050534in}{1.504681in}}%
\pgfpathcurveto{\pgfqpoint{4.045491in}{1.509724in}}{\pgfqpoint{4.038649in}{1.512558in}}{\pgfqpoint{4.031516in}{1.512558in}}%
\pgfpathcurveto{\pgfqpoint{4.024383in}{1.512558in}}{\pgfqpoint{4.017542in}{1.509724in}}{\pgfqpoint{4.012498in}{1.504681in}}%
\pgfpathcurveto{\pgfqpoint{4.007454in}{1.499637in}}{\pgfqpoint{4.004620in}{1.492795in}}{\pgfqpoint{4.004620in}{1.485662in}}%
\pgfpathcurveto{\pgfqpoint{4.004620in}{1.478530in}}{\pgfqpoint{4.007454in}{1.471688in}}{\pgfqpoint{4.012498in}{1.466644in}}%
\pgfpathcurveto{\pgfqpoint{4.017542in}{1.461601in}}{\pgfqpoint{4.024383in}{1.458767in}}{\pgfqpoint{4.031516in}{1.458767in}}%
\pgfpathclose%
\pgfusepath{stroke,fill}%
\end{pgfscope}%
\begin{pgfscope}%
\pgfpathrectangle{\pgfqpoint{2.867647in}{0.500000in}}{\pgfqpoint{1.764706in}{1.700000in}}%
\pgfusepath{clip}%
\pgfsetbuttcap%
\pgfsetroundjoin%
\definecolor{currentfill}{rgb}{0.979124,0.903132,0.839793}%
\pgfsetfillcolor{currentfill}%
\pgfsetlinewidth{0.311001pt}%
\definecolor{currentstroke}{rgb}{1.000000,1.000000,1.000000}%
\pgfsetstrokecolor{currentstroke}%
\pgfsetdash{}{0pt}%
\pgfpathmoveto{\pgfqpoint{4.148825in}{1.233089in}}%
\pgfpathcurveto{\pgfqpoint{4.155958in}{1.233089in}}{\pgfqpoint{4.162800in}{1.235923in}}{\pgfqpoint{4.167843in}{1.240967in}}%
\pgfpathcurveto{\pgfqpoint{4.172887in}{1.246010in}}{\pgfqpoint{4.175721in}{1.252852in}}{\pgfqpoint{4.175721in}{1.259985in}}%
\pgfpathcurveto{\pgfqpoint{4.175721in}{1.267118in}}{\pgfqpoint{4.172887in}{1.273959in}}{\pgfqpoint{4.167843in}{1.279003in}}%
\pgfpathcurveto{\pgfqpoint{4.162800in}{1.284047in}}{\pgfqpoint{4.155958in}{1.286881in}}{\pgfqpoint{4.148825in}{1.286881in}}%
\pgfpathcurveto{\pgfqpoint{4.141692in}{1.286881in}}{\pgfqpoint{4.134851in}{1.284047in}}{\pgfqpoint{4.129807in}{1.279003in}}%
\pgfpathcurveto{\pgfqpoint{4.124763in}{1.273959in}}{\pgfqpoint{4.121929in}{1.267118in}}{\pgfqpoint{4.121929in}{1.259985in}}%
\pgfpathcurveto{\pgfqpoint{4.121929in}{1.252852in}}{\pgfqpoint{4.124763in}{1.246010in}}{\pgfqpoint{4.129807in}{1.240967in}}%
\pgfpathcurveto{\pgfqpoint{4.134851in}{1.235923in}}{\pgfqpoint{4.141692in}{1.233089in}}{\pgfqpoint{4.148825in}{1.233089in}}%
\pgfpathclose%
\pgfusepath{stroke,fill}%
\end{pgfscope}%
\begin{pgfscope}%
\pgfpathrectangle{\pgfqpoint{2.867647in}{0.500000in}}{\pgfqpoint{1.764706in}{1.700000in}}%
\pgfusepath{clip}%
\pgfsetbuttcap%
\pgfsetroundjoin%
\definecolor{currentfill}{rgb}{0.979124,0.903132,0.839793}%
\pgfsetfillcolor{currentfill}%
\pgfsetlinewidth{0.311001pt}%
\definecolor{currentstroke}{rgb}{1.000000,1.000000,1.000000}%
\pgfsetstrokecolor{currentstroke}%
\pgfsetdash{}{0pt}%
\pgfpathmoveto{\pgfqpoint{4.224040in}{1.252498in}}%
\pgfpathcurveto{\pgfqpoint{4.231173in}{1.252498in}}{\pgfqpoint{4.238015in}{1.255332in}}{\pgfqpoint{4.243059in}{1.260376in}}%
\pgfpathcurveto{\pgfqpoint{4.248102in}{1.265420in}}{\pgfqpoint{4.250936in}{1.272261in}}{\pgfqpoint{4.250936in}{1.279394in}}%
\pgfpathcurveto{\pgfqpoint{4.250936in}{1.286527in}}{\pgfqpoint{4.248102in}{1.293369in}}{\pgfqpoint{4.243059in}{1.298412in}}%
\pgfpathcurveto{\pgfqpoint{4.238015in}{1.303456in}}{\pgfqpoint{4.231173in}{1.306290in}}{\pgfqpoint{4.224040in}{1.306290in}}%
\pgfpathcurveto{\pgfqpoint{4.216908in}{1.306290in}}{\pgfqpoint{4.210066in}{1.303456in}}{\pgfqpoint{4.205022in}{1.298412in}}%
\pgfpathcurveto{\pgfqpoint{4.199979in}{1.293369in}}{\pgfqpoint{4.197145in}{1.286527in}}{\pgfqpoint{4.197145in}{1.279394in}}%
\pgfpathcurveto{\pgfqpoint{4.197145in}{1.272261in}}{\pgfqpoint{4.199979in}{1.265420in}}{\pgfqpoint{4.205022in}{1.260376in}}%
\pgfpathcurveto{\pgfqpoint{4.210066in}{1.255332in}}{\pgfqpoint{4.216908in}{1.252498in}}{\pgfqpoint{4.224040in}{1.252498in}}%
\pgfpathclose%
\pgfusepath{stroke,fill}%
\end{pgfscope}%
\begin{pgfscope}%
\pgfpathrectangle{\pgfqpoint{2.867647in}{0.500000in}}{\pgfqpoint{1.764706in}{1.700000in}}%
\pgfusepath{clip}%
\pgfsetbuttcap%
\pgfsetroundjoin%
\definecolor{currentfill}{rgb}{0.979124,0.903132,0.839793}%
\pgfsetfillcolor{currentfill}%
\pgfsetlinewidth{0.311001pt}%
\definecolor{currentstroke}{rgb}{1.000000,1.000000,1.000000}%
\pgfsetstrokecolor{currentstroke}%
\pgfsetdash{}{0pt}%
\pgfpathmoveto{\pgfqpoint{4.223625in}{1.319719in}}%
\pgfpathcurveto{\pgfqpoint{4.230757in}{1.319719in}}{\pgfqpoint{4.237599in}{1.322553in}}{\pgfqpoint{4.242643in}{1.327597in}}%
\pgfpathcurveto{\pgfqpoint{4.247686in}{1.332641in}}{\pgfqpoint{4.250520in}{1.339482in}}{\pgfqpoint{4.250520in}{1.346615in}}%
\pgfpathcurveto{\pgfqpoint{4.250520in}{1.353748in}}{\pgfqpoint{4.247686in}{1.360590in}}{\pgfqpoint{4.242643in}{1.365633in}}%
\pgfpathcurveto{\pgfqpoint{4.237599in}{1.370677in}}{\pgfqpoint{4.230757in}{1.373511in}}{\pgfqpoint{4.223625in}{1.373511in}}%
\pgfpathcurveto{\pgfqpoint{4.216492in}{1.373511in}}{\pgfqpoint{4.209650in}{1.370677in}}{\pgfqpoint{4.204606in}{1.365633in}}%
\pgfpathcurveto{\pgfqpoint{4.199563in}{1.360590in}}{\pgfqpoint{4.196729in}{1.353748in}}{\pgfqpoint{4.196729in}{1.346615in}}%
\pgfpathcurveto{\pgfqpoint{4.196729in}{1.339482in}}{\pgfqpoint{4.199563in}{1.332641in}}{\pgfqpoint{4.204606in}{1.327597in}}%
\pgfpathcurveto{\pgfqpoint{4.209650in}{1.322553in}}{\pgfqpoint{4.216492in}{1.319719in}}{\pgfqpoint{4.223625in}{1.319719in}}%
\pgfpathclose%
\pgfusepath{stroke,fill}%
\end{pgfscope}%
\begin{pgfscope}%
\pgfpathrectangle{\pgfqpoint{2.867647in}{0.500000in}}{\pgfqpoint{1.764706in}{1.700000in}}%
\pgfusepath{clip}%
\pgfsetbuttcap%
\pgfsetroundjoin%
\definecolor{currentfill}{rgb}{0.971202,0.827364,0.728520}%
\pgfsetfillcolor{currentfill}%
\pgfsetlinewidth{0.311001pt}%
\definecolor{currentstroke}{rgb}{1.000000,1.000000,1.000000}%
\pgfsetstrokecolor{currentstroke}%
\pgfsetdash{}{0pt}%
\pgfpathmoveto{\pgfqpoint{4.080837in}{1.490931in}}%
\pgfpathcurveto{\pgfqpoint{4.087970in}{1.490931in}}{\pgfqpoint{4.094812in}{1.493765in}}{\pgfqpoint{4.099856in}{1.498808in}}%
\pgfpathcurveto{\pgfqpoint{4.104899in}{1.503852in}}{\pgfqpoint{4.107733in}{1.510694in}}{\pgfqpoint{4.107733in}{1.517826in}}%
\pgfpathcurveto{\pgfqpoint{4.107733in}{1.524959in}}{\pgfqpoint{4.104899in}{1.531801in}}{\pgfqpoint{4.099856in}{1.536845in}}%
\pgfpathcurveto{\pgfqpoint{4.094812in}{1.541888in}}{\pgfqpoint{4.087970in}{1.544722in}}{\pgfqpoint{4.080837in}{1.544722in}}%
\pgfpathcurveto{\pgfqpoint{4.073705in}{1.544722in}}{\pgfqpoint{4.066863in}{1.541888in}}{\pgfqpoint{4.061819in}{1.536845in}}%
\pgfpathcurveto{\pgfqpoint{4.056776in}{1.531801in}}{\pgfqpoint{4.053942in}{1.524959in}}{\pgfqpoint{4.053942in}{1.517826in}}%
\pgfpathcurveto{\pgfqpoint{4.053942in}{1.510694in}}{\pgfqpoint{4.056776in}{1.503852in}}{\pgfqpoint{4.061819in}{1.498808in}}%
\pgfpathcurveto{\pgfqpoint{4.066863in}{1.493765in}}{\pgfqpoint{4.073705in}{1.490931in}}{\pgfqpoint{4.080837in}{1.490931in}}%
\pgfpathclose%
\pgfusepath{stroke,fill}%
\end{pgfscope}%
\begin{pgfscope}%
\pgfpathrectangle{\pgfqpoint{2.867647in}{0.500000in}}{\pgfqpoint{1.764706in}{1.700000in}}%
\pgfusepath{clip}%
\pgfsetbuttcap%
\pgfsetroundjoin%
\definecolor{currentfill}{rgb}{0.978376,0.897317,0.831308}%
\pgfsetfillcolor{currentfill}%
\pgfsetlinewidth{0.311001pt}%
\definecolor{currentstroke}{rgb}{1.000000,1.000000,1.000000}%
\pgfsetstrokecolor{currentstroke}%
\pgfsetdash{}{0pt}%
\pgfpathmoveto{\pgfqpoint{4.121935in}{1.088768in}}%
\pgfpathcurveto{\pgfqpoint{4.129068in}{1.088768in}}{\pgfqpoint{4.135910in}{1.091602in}}{\pgfqpoint{4.140953in}{1.096646in}}%
\pgfpathcurveto{\pgfqpoint{4.145997in}{1.101689in}}{\pgfqpoint{4.148831in}{1.108531in}}{\pgfqpoint{4.148831in}{1.115664in}}%
\pgfpathcurveto{\pgfqpoint{4.148831in}{1.122797in}}{\pgfqpoint{4.145997in}{1.129638in}}{\pgfqpoint{4.140953in}{1.134682in}}%
\pgfpathcurveto{\pgfqpoint{4.135910in}{1.139726in}}{\pgfqpoint{4.129068in}{1.142559in}}{\pgfqpoint{4.121935in}{1.142559in}}%
\pgfpathcurveto{\pgfqpoint{4.114803in}{1.142559in}}{\pgfqpoint{4.107961in}{1.139726in}}{\pgfqpoint{4.102917in}{1.134682in}}%
\pgfpathcurveto{\pgfqpoint{4.097874in}{1.129638in}}{\pgfqpoint{4.095040in}{1.122797in}}{\pgfqpoint{4.095040in}{1.115664in}}%
\pgfpathcurveto{\pgfqpoint{4.095040in}{1.108531in}}{\pgfqpoint{4.097874in}{1.101689in}}{\pgfqpoint{4.102917in}{1.096646in}}%
\pgfpathcurveto{\pgfqpoint{4.107961in}{1.091602in}}{\pgfqpoint{4.114803in}{1.088768in}}{\pgfqpoint{4.121935in}{1.088768in}}%
\pgfpathclose%
\pgfusepath{stroke,fill}%
\end{pgfscope}%
\begin{pgfscope}%
\pgfpathrectangle{\pgfqpoint{2.867647in}{0.500000in}}{\pgfqpoint{1.764706in}{1.700000in}}%
\pgfusepath{clip}%
\pgfsetbuttcap%
\pgfsetroundjoin%
\definecolor{currentfill}{rgb}{0.980678,0.914765,0.856766}%
\pgfsetfillcolor{currentfill}%
\pgfsetlinewidth{0.311001pt}%
\definecolor{currentstroke}{rgb}{1.000000,1.000000,1.000000}%
\pgfsetstrokecolor{currentstroke}%
\pgfsetdash{}{0pt}%
\pgfpathmoveto{\pgfqpoint{4.174788in}{1.479710in}}%
\pgfpathcurveto{\pgfqpoint{4.181921in}{1.479710in}}{\pgfqpoint{4.188763in}{1.482543in}}{\pgfqpoint{4.193806in}{1.487587in}}%
\pgfpathcurveto{\pgfqpoint{4.198850in}{1.492631in}}{\pgfqpoint{4.201684in}{1.499472in}}{\pgfqpoint{4.201684in}{1.506605in}}%
\pgfpathcurveto{\pgfqpoint{4.201684in}{1.513738in}}{\pgfqpoint{4.198850in}{1.520580in}}{\pgfqpoint{4.193806in}{1.525623in}}%
\pgfpathcurveto{\pgfqpoint{4.188763in}{1.530667in}}{\pgfqpoint{4.181921in}{1.533501in}}{\pgfqpoint{4.174788in}{1.533501in}}%
\pgfpathcurveto{\pgfqpoint{4.167655in}{1.533501in}}{\pgfqpoint{4.160814in}{1.530667in}}{\pgfqpoint{4.155770in}{1.525623in}}%
\pgfpathcurveto{\pgfqpoint{4.150726in}{1.520580in}}{\pgfqpoint{4.147892in}{1.513738in}}{\pgfqpoint{4.147892in}{1.506605in}}%
\pgfpathcurveto{\pgfqpoint{4.147892in}{1.499472in}}{\pgfqpoint{4.150726in}{1.492631in}}{\pgfqpoint{4.155770in}{1.487587in}}%
\pgfpathcurveto{\pgfqpoint{4.160814in}{1.482543in}}{\pgfqpoint{4.167655in}{1.479710in}}{\pgfqpoint{4.174788in}{1.479710in}}%
\pgfpathclose%
\pgfusepath{stroke,fill}%
\end{pgfscope}%
\begin{pgfscope}%
\pgfpathrectangle{\pgfqpoint{2.867647in}{0.500000in}}{\pgfqpoint{1.764706in}{1.700000in}}%
\pgfusepath{clip}%
\pgfsetbuttcap%
\pgfsetroundjoin%
\definecolor{currentfill}{rgb}{0.980678,0.914765,0.856766}%
\pgfsetfillcolor{currentfill}%
\pgfsetlinewidth{0.311001pt}%
\definecolor{currentstroke}{rgb}{1.000000,1.000000,1.000000}%
\pgfsetstrokecolor{currentstroke}%
\pgfsetdash{}{0pt}%
\pgfpathmoveto{\pgfqpoint{4.183368in}{1.453553in}}%
\pgfpathcurveto{\pgfqpoint{4.190501in}{1.453553in}}{\pgfqpoint{4.197342in}{1.456387in}}{\pgfqpoint{4.202386in}{1.461430in}}%
\pgfpathcurveto{\pgfqpoint{4.207430in}{1.466474in}}{\pgfqpoint{4.210264in}{1.473316in}}{\pgfqpoint{4.210264in}{1.480449in}}%
\pgfpathcurveto{\pgfqpoint{4.210264in}{1.487581in}}{\pgfqpoint{4.207430in}{1.494423in}}{\pgfqpoint{4.202386in}{1.499467in}}%
\pgfpathcurveto{\pgfqpoint{4.197342in}{1.504510in}}{\pgfqpoint{4.190501in}{1.507344in}}{\pgfqpoint{4.183368in}{1.507344in}}%
\pgfpathcurveto{\pgfqpoint{4.176235in}{1.507344in}}{\pgfqpoint{4.169393in}{1.504510in}}{\pgfqpoint{4.164350in}{1.499467in}}%
\pgfpathcurveto{\pgfqpoint{4.159306in}{1.494423in}}{\pgfqpoint{4.156472in}{1.487581in}}{\pgfqpoint{4.156472in}{1.480449in}}%
\pgfpathcurveto{\pgfqpoint{4.156472in}{1.473316in}}{\pgfqpoint{4.159306in}{1.466474in}}{\pgfqpoint{4.164350in}{1.461430in}}%
\pgfpathcurveto{\pgfqpoint{4.169393in}{1.456387in}}{\pgfqpoint{4.176235in}{1.453553in}}{\pgfqpoint{4.183368in}{1.453553in}}%
\pgfpathclose%
\pgfusepath{stroke,fill}%
\end{pgfscope}%
\begin{pgfscope}%
\pgfpathrectangle{\pgfqpoint{2.867647in}{0.500000in}}{\pgfqpoint{1.764706in}{1.700000in}}%
\pgfusepath{clip}%
\pgfsetbuttcap%
\pgfsetroundjoin%
\definecolor{currentfill}{rgb}{0.979124,0.903132,0.839793}%
\pgfsetfillcolor{currentfill}%
\pgfsetlinewidth{0.311001pt}%
\definecolor{currentstroke}{rgb}{1.000000,1.000000,1.000000}%
\pgfsetstrokecolor{currentstroke}%
\pgfsetdash{}{0pt}%
\pgfpathmoveto{\pgfqpoint{4.148554in}{1.246671in}}%
\pgfpathcurveto{\pgfqpoint{4.155686in}{1.246671in}}{\pgfqpoint{4.162528in}{1.249505in}}{\pgfqpoint{4.167572in}{1.254549in}}%
\pgfpathcurveto{\pgfqpoint{4.172615in}{1.259592in}}{\pgfqpoint{4.175449in}{1.266434in}}{\pgfqpoint{4.175449in}{1.273567in}}%
\pgfpathcurveto{\pgfqpoint{4.175449in}{1.280700in}}{\pgfqpoint{4.172615in}{1.287541in}}{\pgfqpoint{4.167572in}{1.292585in}}%
\pgfpathcurveto{\pgfqpoint{4.162528in}{1.297629in}}{\pgfqpoint{4.155686in}{1.300463in}}{\pgfqpoint{4.148554in}{1.300463in}}%
\pgfpathcurveto{\pgfqpoint{4.141421in}{1.300463in}}{\pgfqpoint{4.134579in}{1.297629in}}{\pgfqpoint{4.129535in}{1.292585in}}%
\pgfpathcurveto{\pgfqpoint{4.124492in}{1.287541in}}{\pgfqpoint{4.121658in}{1.280700in}}{\pgfqpoint{4.121658in}{1.273567in}}%
\pgfpathcurveto{\pgfqpoint{4.121658in}{1.266434in}}{\pgfqpoint{4.124492in}{1.259592in}}{\pgfqpoint{4.129535in}{1.254549in}}%
\pgfpathcurveto{\pgfqpoint{4.134579in}{1.249505in}}{\pgfqpoint{4.141421in}{1.246671in}}{\pgfqpoint{4.148554in}{1.246671in}}%
\pgfpathclose%
\pgfusepath{stroke,fill}%
\end{pgfscope}%
\begin{pgfscope}%
\pgfpathrectangle{\pgfqpoint{2.867647in}{0.500000in}}{\pgfqpoint{1.764706in}{1.700000in}}%
\pgfusepath{clip}%
\pgfsetbuttcap%
\pgfsetroundjoin%
\definecolor{currentfill}{rgb}{0.968931,0.798091,0.685123}%
\pgfsetfillcolor{currentfill}%
\pgfsetlinewidth{0.311001pt}%
\definecolor{currentstroke}{rgb}{1.000000,1.000000,1.000000}%
\pgfsetstrokecolor{currentstroke}%
\pgfsetdash{}{0pt}%
\pgfpathmoveto{\pgfqpoint{4.044647in}{1.690768in}}%
\pgfpathcurveto{\pgfqpoint{4.051780in}{1.690768in}}{\pgfqpoint{4.058621in}{1.693602in}}{\pgfqpoint{4.063665in}{1.698645in}}%
\pgfpathcurveto{\pgfqpoint{4.068709in}{1.703689in}}{\pgfqpoint{4.071543in}{1.710531in}}{\pgfqpoint{4.071543in}{1.717663in}}%
\pgfpathcurveto{\pgfqpoint{4.071543in}{1.724796in}}{\pgfqpoint{4.068709in}{1.731638in}}{\pgfqpoint{4.063665in}{1.736682in}}%
\pgfpathcurveto{\pgfqpoint{4.058621in}{1.741725in}}{\pgfqpoint{4.051780in}{1.744559in}}{\pgfqpoint{4.044647in}{1.744559in}}%
\pgfpathcurveto{\pgfqpoint{4.037514in}{1.744559in}}{\pgfqpoint{4.030672in}{1.741725in}}{\pgfqpoint{4.025629in}{1.736682in}}%
\pgfpathcurveto{\pgfqpoint{4.020585in}{1.731638in}}{\pgfqpoint{4.017751in}{1.724796in}}{\pgfqpoint{4.017751in}{1.717663in}}%
\pgfpathcurveto{\pgfqpoint{4.017751in}{1.710531in}}{\pgfqpoint{4.020585in}{1.703689in}}{\pgfqpoint{4.025629in}{1.698645in}}%
\pgfpathcurveto{\pgfqpoint{4.030672in}{1.693602in}}{\pgfqpoint{4.037514in}{1.690768in}}{\pgfqpoint{4.044647in}{1.690768in}}%
\pgfpathclose%
\pgfusepath{stroke,fill}%
\end{pgfscope}%
\begin{pgfscope}%
\pgfpathrectangle{\pgfqpoint{2.867647in}{0.500000in}}{\pgfqpoint{1.764706in}{1.700000in}}%
\pgfusepath{clip}%
\pgfsetbuttcap%
\pgfsetroundjoin%
\definecolor{currentfill}{rgb}{0.949145,0.420383,0.287810}%
\pgfsetfillcolor{currentfill}%
\pgfsetlinewidth{0.311001pt}%
\definecolor{currentstroke}{rgb}{1.000000,1.000000,1.000000}%
\pgfsetstrokecolor{currentstroke}%
\pgfsetdash{}{0pt}%
\pgfpathmoveto{\pgfqpoint{3.904109in}{1.633278in}}%
\pgfpathcurveto{\pgfqpoint{3.911242in}{1.633278in}}{\pgfqpoint{3.918084in}{1.636112in}}{\pgfqpoint{3.923128in}{1.641156in}}%
\pgfpathcurveto{\pgfqpoint{3.928171in}{1.646199in}}{\pgfqpoint{3.931005in}{1.653041in}}{\pgfqpoint{3.931005in}{1.660174in}}%
\pgfpathcurveto{\pgfqpoint{3.931005in}{1.667307in}}{\pgfqpoint{3.928171in}{1.674148in}}{\pgfqpoint{3.923128in}{1.679192in}}%
\pgfpathcurveto{\pgfqpoint{3.918084in}{1.684236in}}{\pgfqpoint{3.911242in}{1.687070in}}{\pgfqpoint{3.904109in}{1.687070in}}%
\pgfpathcurveto{\pgfqpoint{3.896977in}{1.687070in}}{\pgfqpoint{3.890135in}{1.684236in}}{\pgfqpoint{3.885091in}{1.679192in}}%
\pgfpathcurveto{\pgfqpoint{3.880048in}{1.674148in}}{\pgfqpoint{3.877214in}{1.667307in}}{\pgfqpoint{3.877214in}{1.660174in}}%
\pgfpathcurveto{\pgfqpoint{3.877214in}{1.653041in}}{\pgfqpoint{3.880048in}{1.646199in}}{\pgfqpoint{3.885091in}{1.641156in}}%
\pgfpathcurveto{\pgfqpoint{3.890135in}{1.636112in}}{\pgfqpoint{3.896977in}{1.633278in}}{\pgfqpoint{3.904109in}{1.633278in}}%
\pgfpathclose%
\pgfusepath{stroke,fill}%
\end{pgfscope}%
\begin{pgfscope}%
\pgfpathrectangle{\pgfqpoint{2.867647in}{0.500000in}}{\pgfqpoint{1.764706in}{1.700000in}}%
\pgfusepath{clip}%
\pgfsetbuttcap%
\pgfsetroundjoin%
\definecolor{currentfill}{rgb}{0.978376,0.897317,0.831308}%
\pgfsetfillcolor{currentfill}%
\pgfsetlinewidth{0.311001pt}%
\definecolor{currentstroke}{rgb}{1.000000,1.000000,1.000000}%
\pgfsetstrokecolor{currentstroke}%
\pgfsetdash{}{0pt}%
\pgfpathmoveto{\pgfqpoint{4.127369in}{1.088090in}}%
\pgfpathcurveto{\pgfqpoint{4.134501in}{1.088090in}}{\pgfqpoint{4.141343in}{1.090924in}}{\pgfqpoint{4.146387in}{1.095968in}}%
\pgfpathcurveto{\pgfqpoint{4.151430in}{1.101011in}}{\pgfqpoint{4.154264in}{1.107853in}}{\pgfqpoint{4.154264in}{1.114986in}}%
\pgfpathcurveto{\pgfqpoint{4.154264in}{1.122119in}}{\pgfqpoint{4.151430in}{1.128960in}}{\pgfqpoint{4.146387in}{1.134004in}}%
\pgfpathcurveto{\pgfqpoint{4.141343in}{1.139048in}}{\pgfqpoint{4.134501in}{1.141881in}}{\pgfqpoint{4.127369in}{1.141881in}}%
\pgfpathcurveto{\pgfqpoint{4.120236in}{1.141881in}}{\pgfqpoint{4.113394in}{1.139048in}}{\pgfqpoint{4.108350in}{1.134004in}}%
\pgfpathcurveto{\pgfqpoint{4.103307in}{1.128960in}}{\pgfqpoint{4.100473in}{1.122119in}}{\pgfqpoint{4.100473in}{1.114986in}}%
\pgfpathcurveto{\pgfqpoint{4.100473in}{1.107853in}}{\pgfqpoint{4.103307in}{1.101011in}}{\pgfqpoint{4.108350in}{1.095968in}}%
\pgfpathcurveto{\pgfqpoint{4.113394in}{1.090924in}}{\pgfqpoint{4.120236in}{1.088090in}}{\pgfqpoint{4.127369in}{1.088090in}}%
\pgfpathclose%
\pgfusepath{stroke,fill}%
\end{pgfscope}%
\begin{pgfscope}%
\pgfpathrectangle{\pgfqpoint{2.867647in}{0.500000in}}{\pgfqpoint{1.764706in}{1.700000in}}%
\pgfusepath{clip}%
\pgfsetbuttcap%
\pgfsetroundjoin%
\definecolor{currentfill}{rgb}{0.978376,0.897317,0.831308}%
\pgfsetfillcolor{currentfill}%
\pgfsetlinewidth{0.311001pt}%
\definecolor{currentstroke}{rgb}{1.000000,1.000000,1.000000}%
\pgfsetstrokecolor{currentstroke}%
\pgfsetdash{}{0pt}%
\pgfpathmoveto{\pgfqpoint{4.162214in}{1.072624in}}%
\pgfpathcurveto{\pgfqpoint{4.169347in}{1.072624in}}{\pgfqpoint{4.176188in}{1.075458in}}{\pgfqpoint{4.181232in}{1.080502in}}%
\pgfpathcurveto{\pgfqpoint{4.186276in}{1.085546in}}{\pgfqpoint{4.189110in}{1.092387in}}{\pgfqpoint{4.189110in}{1.099520in}}%
\pgfpathcurveto{\pgfqpoint{4.189110in}{1.106653in}}{\pgfqpoint{4.186276in}{1.113494in}}{\pgfqpoint{4.181232in}{1.118538in}}%
\pgfpathcurveto{\pgfqpoint{4.176188in}{1.123582in}}{\pgfqpoint{4.169347in}{1.126416in}}{\pgfqpoint{4.162214in}{1.126416in}}%
\pgfpathcurveto{\pgfqpoint{4.155081in}{1.126416in}}{\pgfqpoint{4.148239in}{1.123582in}}{\pgfqpoint{4.143196in}{1.118538in}}%
\pgfpathcurveto{\pgfqpoint{4.138152in}{1.113494in}}{\pgfqpoint{4.135318in}{1.106653in}}{\pgfqpoint{4.135318in}{1.099520in}}%
\pgfpathcurveto{\pgfqpoint{4.135318in}{1.092387in}}{\pgfqpoint{4.138152in}{1.085546in}}{\pgfqpoint{4.143196in}{1.080502in}}%
\pgfpathcurveto{\pgfqpoint{4.148239in}{1.075458in}}{\pgfqpoint{4.155081in}{1.072624in}}{\pgfqpoint{4.162214in}{1.072624in}}%
\pgfpathclose%
\pgfusepath{stroke,fill}%
\end{pgfscope}%
\begin{pgfscope}%
\pgfpathrectangle{\pgfqpoint{2.867647in}{0.500000in}}{\pgfqpoint{1.764706in}{1.700000in}}%
\pgfusepath{clip}%
\pgfsetbuttcap%
\pgfsetroundjoin%
\definecolor{currentfill}{rgb}{0.966120,0.744512,0.608720}%
\pgfsetfillcolor{currentfill}%
\pgfsetlinewidth{0.311001pt}%
\definecolor{currentstroke}{rgb}{1.000000,1.000000,1.000000}%
\pgfsetstrokecolor{currentstroke}%
\pgfsetdash{}{0pt}%
\pgfpathmoveto{\pgfqpoint{4.203616in}{0.992282in}}%
\pgfpathcurveto{\pgfqpoint{4.210749in}{0.992282in}}{\pgfqpoint{4.217590in}{0.995116in}}{\pgfqpoint{4.222634in}{1.000160in}}%
\pgfpathcurveto{\pgfqpoint{4.227678in}{1.005203in}}{\pgfqpoint{4.230512in}{1.012045in}}{\pgfqpoint{4.230512in}{1.019178in}}%
\pgfpathcurveto{\pgfqpoint{4.230512in}{1.026311in}}{\pgfqpoint{4.227678in}{1.033152in}}{\pgfqpoint{4.222634in}{1.038196in}}%
\pgfpathcurveto{\pgfqpoint{4.217590in}{1.043240in}}{\pgfqpoint{4.210749in}{1.046074in}}{\pgfqpoint{4.203616in}{1.046074in}}%
\pgfpathcurveto{\pgfqpoint{4.196483in}{1.046074in}}{\pgfqpoint{4.189641in}{1.043240in}}{\pgfqpoint{4.184598in}{1.038196in}}%
\pgfpathcurveto{\pgfqpoint{4.179554in}{1.033152in}}{\pgfqpoint{4.176720in}{1.026311in}}{\pgfqpoint{4.176720in}{1.019178in}}%
\pgfpathcurveto{\pgfqpoint{4.176720in}{1.012045in}}{\pgfqpoint{4.179554in}{1.005203in}}{\pgfqpoint{4.184598in}{1.000160in}}%
\pgfpathcurveto{\pgfqpoint{4.189641in}{0.995116in}}{\pgfqpoint{4.196483in}{0.992282in}}{\pgfqpoint{4.203616in}{0.992282in}}%
\pgfpathclose%
\pgfusepath{stroke,fill}%
\end{pgfscope}%
\begin{pgfscope}%
\pgfpathrectangle{\pgfqpoint{2.867647in}{0.500000in}}{\pgfqpoint{1.764706in}{1.700000in}}%
\pgfusepath{clip}%
\pgfsetbuttcap%
\pgfsetroundjoin%
\definecolor{currentfill}{rgb}{0.970255,0.815666,0.711203}%
\pgfsetfillcolor{currentfill}%
\pgfsetlinewidth{0.311001pt}%
\definecolor{currentstroke}{rgb}{1.000000,1.000000,1.000000}%
\pgfsetstrokecolor{currentstroke}%
\pgfsetdash{}{0pt}%
\pgfpathmoveto{\pgfqpoint{4.233179in}{1.542534in}}%
\pgfpathcurveto{\pgfqpoint{4.240312in}{1.542534in}}{\pgfqpoint{4.247154in}{1.545367in}}{\pgfqpoint{4.252197in}{1.550411in}}%
\pgfpathcurveto{\pgfqpoint{4.257241in}{1.555455in}}{\pgfqpoint{4.260075in}{1.562296in}}{\pgfqpoint{4.260075in}{1.569429in}}%
\pgfpathcurveto{\pgfqpoint{4.260075in}{1.576562in}}{\pgfqpoint{4.257241in}{1.583404in}}{\pgfqpoint{4.252197in}{1.588447in}}%
\pgfpathcurveto{\pgfqpoint{4.247154in}{1.593491in}}{\pgfqpoint{4.240312in}{1.596325in}}{\pgfqpoint{4.233179in}{1.596325in}}%
\pgfpathcurveto{\pgfqpoint{4.226047in}{1.596325in}}{\pgfqpoint{4.219205in}{1.593491in}}{\pgfqpoint{4.214161in}{1.588447in}}%
\pgfpathcurveto{\pgfqpoint{4.209118in}{1.583404in}}{\pgfqpoint{4.206284in}{1.576562in}}{\pgfqpoint{4.206284in}{1.569429in}}%
\pgfpathcurveto{\pgfqpoint{4.206284in}{1.562296in}}{\pgfqpoint{4.209118in}{1.555455in}}{\pgfqpoint{4.214161in}{1.550411in}}%
\pgfpathcurveto{\pgfqpoint{4.219205in}{1.545367in}}{\pgfqpoint{4.226047in}{1.542534in}}{\pgfqpoint{4.233179in}{1.542534in}}%
\pgfpathclose%
\pgfusepath{stroke,fill}%
\end{pgfscope}%
\begin{pgfscope}%
\pgfpathrectangle{\pgfqpoint{2.867647in}{0.500000in}}{\pgfqpoint{1.764706in}{1.700000in}}%
\pgfusepath{clip}%
\pgfsetbuttcap%
\pgfsetroundjoin%
\definecolor{currentfill}{rgb}{0.966812,0.762584,0.633643}%
\pgfsetfillcolor{currentfill}%
\pgfsetlinewidth{0.311001pt}%
\definecolor{currentstroke}{rgb}{1.000000,1.000000,1.000000}%
\pgfsetstrokecolor{currentstroke}%
\pgfsetdash{}{0pt}%
\pgfpathmoveto{\pgfqpoint{4.016387in}{1.630057in}}%
\pgfpathcurveto{\pgfqpoint{4.023520in}{1.630057in}}{\pgfqpoint{4.030361in}{1.632891in}}{\pgfqpoint{4.035405in}{1.637935in}}%
\pgfpathcurveto{\pgfqpoint{4.040449in}{1.642978in}}{\pgfqpoint{4.043283in}{1.649820in}}{\pgfqpoint{4.043283in}{1.656953in}}%
\pgfpathcurveto{\pgfqpoint{4.043283in}{1.664086in}}{\pgfqpoint{4.040449in}{1.670927in}}{\pgfqpoint{4.035405in}{1.675971in}}%
\pgfpathcurveto{\pgfqpoint{4.030361in}{1.681015in}}{\pgfqpoint{4.023520in}{1.683849in}}{\pgfqpoint{4.016387in}{1.683849in}}%
\pgfpathcurveto{\pgfqpoint{4.009254in}{1.683849in}}{\pgfqpoint{4.002413in}{1.681015in}}{\pgfqpoint{3.997369in}{1.675971in}}%
\pgfpathcurveto{\pgfqpoint{3.992325in}{1.670927in}}{\pgfqpoint{3.989491in}{1.664086in}}{\pgfqpoint{3.989491in}{1.656953in}}%
\pgfpathcurveto{\pgfqpoint{3.989491in}{1.649820in}}{\pgfqpoint{3.992325in}{1.642978in}}{\pgfqpoint{3.997369in}{1.637935in}}%
\pgfpathcurveto{\pgfqpoint{4.002413in}{1.632891in}}{\pgfqpoint{4.009254in}{1.630057in}}{\pgfqpoint{4.016387in}{1.630057in}}%
\pgfpathclose%
\pgfusepath{stroke,fill}%
\end{pgfscope}%
\begin{pgfscope}%
\pgfpathrectangle{\pgfqpoint{2.867647in}{0.500000in}}{\pgfqpoint{1.764706in}{1.700000in}}%
\pgfusepath{clip}%
\pgfsetbuttcap%
\pgfsetroundjoin%
\definecolor{currentfill}{rgb}{0.967398,0.774513,0.650573}%
\pgfsetfillcolor{currentfill}%
\pgfsetlinewidth{0.311001pt}%
\definecolor{currentstroke}{rgb}{1.000000,1.000000,1.000000}%
\pgfsetstrokecolor{currentstroke}%
\pgfsetdash{}{0pt}%
\pgfpathmoveto{\pgfqpoint{4.023259in}{0.989934in}}%
\pgfpathcurveto{\pgfqpoint{4.030392in}{0.989934in}}{\pgfqpoint{4.037233in}{0.992767in}}{\pgfqpoint{4.042277in}{0.997811in}}%
\pgfpathcurveto{\pgfqpoint{4.047321in}{1.002855in}}{\pgfqpoint{4.050155in}{1.009696in}}{\pgfqpoint{4.050155in}{1.016829in}}%
\pgfpathcurveto{\pgfqpoint{4.050155in}{1.023962in}}{\pgfqpoint{4.047321in}{1.030804in}}{\pgfqpoint{4.042277in}{1.035847in}}%
\pgfpathcurveto{\pgfqpoint{4.037233in}{1.040891in}}{\pgfqpoint{4.030392in}{1.043725in}}{\pgfqpoint{4.023259in}{1.043725in}}%
\pgfpathcurveto{\pgfqpoint{4.016126in}{1.043725in}}{\pgfqpoint{4.009284in}{1.040891in}}{\pgfqpoint{4.004241in}{1.035847in}}%
\pgfpathcurveto{\pgfqpoint{3.999197in}{1.030804in}}{\pgfqpoint{3.996363in}{1.023962in}}{\pgfqpoint{3.996363in}{1.016829in}}%
\pgfpathcurveto{\pgfqpoint{3.996363in}{1.009696in}}{\pgfqpoint{3.999197in}{1.002855in}}{\pgfqpoint{4.004241in}{0.997811in}}%
\pgfpathcurveto{\pgfqpoint{4.009284in}{0.992767in}}{\pgfqpoint{4.016126in}{0.989934in}}{\pgfqpoint{4.023259in}{0.989934in}}%
\pgfpathclose%
\pgfusepath{stroke,fill}%
\end{pgfscope}%
\begin{pgfscope}%
\pgfpathrectangle{\pgfqpoint{2.867647in}{0.500000in}}{\pgfqpoint{1.764706in}{1.700000in}}%
\pgfusepath{clip}%
\pgfsetbuttcap%
\pgfsetroundjoin%
\definecolor{currentfill}{rgb}{0.972726,0.844889,0.754401}%
\pgfsetfillcolor{currentfill}%
\pgfsetlinewidth{0.311001pt}%
\definecolor{currentstroke}{rgb}{1.000000,1.000000,1.000000}%
\pgfsetstrokecolor{currentstroke}%
\pgfsetdash{}{0pt}%
\pgfpathmoveto{\pgfqpoint{4.133322in}{0.990433in}}%
\pgfpathcurveto{\pgfqpoint{4.140455in}{0.990433in}}{\pgfqpoint{4.147297in}{0.993267in}}{\pgfqpoint{4.152341in}{0.998311in}}%
\pgfpathcurveto{\pgfqpoint{4.157384in}{1.003354in}}{\pgfqpoint{4.160218in}{1.010196in}}{\pgfqpoint{4.160218in}{1.017329in}}%
\pgfpathcurveto{\pgfqpoint{4.160218in}{1.024461in}}{\pgfqpoint{4.157384in}{1.031303in}}{\pgfqpoint{4.152341in}{1.036347in}}%
\pgfpathcurveto{\pgfqpoint{4.147297in}{1.041390in}}{\pgfqpoint{4.140455in}{1.044224in}}{\pgfqpoint{4.133322in}{1.044224in}}%
\pgfpathcurveto{\pgfqpoint{4.126190in}{1.044224in}}{\pgfqpoint{4.119348in}{1.041390in}}{\pgfqpoint{4.114304in}{1.036347in}}%
\pgfpathcurveto{\pgfqpoint{4.109261in}{1.031303in}}{\pgfqpoint{4.106427in}{1.024461in}}{\pgfqpoint{4.106427in}{1.017329in}}%
\pgfpathcurveto{\pgfqpoint{4.106427in}{1.010196in}}{\pgfqpoint{4.109261in}{1.003354in}}{\pgfqpoint{4.114304in}{0.998311in}}%
\pgfpathcurveto{\pgfqpoint{4.119348in}{0.993267in}}{\pgfqpoint{4.126190in}{0.990433in}}{\pgfqpoint{4.133322in}{0.990433in}}%
\pgfpathclose%
\pgfusepath{stroke,fill}%
\end{pgfscope}%
\begin{pgfscope}%
\pgfpathrectangle{\pgfqpoint{2.867647in}{0.500000in}}{\pgfqpoint{1.764706in}{1.700000in}}%
\pgfusepath{clip}%
\pgfsetbuttcap%
\pgfsetroundjoin%
\definecolor{currentfill}{rgb}{0.967735,0.780441,0.659127}%
\pgfsetfillcolor{currentfill}%
\pgfsetlinewidth{0.311001pt}%
\definecolor{currentstroke}{rgb}{1.000000,1.000000,1.000000}%
\pgfsetstrokecolor{currentstroke}%
\pgfsetdash{}{0pt}%
\pgfpathmoveto{\pgfqpoint{4.052426in}{1.717865in}}%
\pgfpathcurveto{\pgfqpoint{4.059559in}{1.717865in}}{\pgfqpoint{4.066401in}{1.720698in}}{\pgfqpoint{4.071444in}{1.725742in}}%
\pgfpathcurveto{\pgfqpoint{4.076488in}{1.730786in}}{\pgfqpoint{4.079322in}{1.737627in}}{\pgfqpoint{4.079322in}{1.744760in}}%
\pgfpathcurveto{\pgfqpoint{4.079322in}{1.751893in}}{\pgfqpoint{4.076488in}{1.758735in}}{\pgfqpoint{4.071444in}{1.763778in}}%
\pgfpathcurveto{\pgfqpoint{4.066401in}{1.768822in}}{\pgfqpoint{4.059559in}{1.771656in}}{\pgfqpoint{4.052426in}{1.771656in}}%
\pgfpathcurveto{\pgfqpoint{4.045293in}{1.771656in}}{\pgfqpoint{4.038452in}{1.768822in}}{\pgfqpoint{4.033408in}{1.763778in}}%
\pgfpathcurveto{\pgfqpoint{4.028364in}{1.758735in}}{\pgfqpoint{4.025530in}{1.751893in}}{\pgfqpoint{4.025530in}{1.744760in}}%
\pgfpathcurveto{\pgfqpoint{4.025530in}{1.737627in}}{\pgfqpoint{4.028364in}{1.730786in}}{\pgfqpoint{4.033408in}{1.725742in}}%
\pgfpathcurveto{\pgfqpoint{4.038452in}{1.720698in}}{\pgfqpoint{4.045293in}{1.717865in}}{\pgfqpoint{4.052426in}{1.717865in}}%
\pgfpathclose%
\pgfusepath{stroke,fill}%
\end{pgfscope}%
\begin{pgfscope}%
\pgfpathrectangle{\pgfqpoint{2.867647in}{0.500000in}}{\pgfqpoint{1.764706in}{1.700000in}}%
\pgfusepath{clip}%
\pgfsetbuttcap%
\pgfsetroundjoin%
\definecolor{currentfill}{rgb}{0.976287,0.879862,0.805788}%
\pgfsetfillcolor{currentfill}%
\pgfsetlinewidth{0.311001pt}%
\definecolor{currentstroke}{rgb}{1.000000,1.000000,1.000000}%
\pgfsetstrokecolor{currentstroke}%
\pgfsetdash{}{0pt}%
\pgfpathmoveto{\pgfqpoint{4.131577in}{1.417050in}}%
\pgfpathcurveto{\pgfqpoint{4.138710in}{1.417050in}}{\pgfqpoint{4.145552in}{1.419884in}}{\pgfqpoint{4.150596in}{1.424928in}}%
\pgfpathcurveto{\pgfqpoint{4.155639in}{1.429971in}}{\pgfqpoint{4.158473in}{1.436813in}}{\pgfqpoint{4.158473in}{1.443946in}}%
\pgfpathcurveto{\pgfqpoint{4.158473in}{1.451079in}}{\pgfqpoint{4.155639in}{1.457920in}}{\pgfqpoint{4.150596in}{1.462964in}}%
\pgfpathcurveto{\pgfqpoint{4.145552in}{1.468008in}}{\pgfqpoint{4.138710in}{1.470842in}}{\pgfqpoint{4.131577in}{1.470842in}}%
\pgfpathcurveto{\pgfqpoint{4.124445in}{1.470842in}}{\pgfqpoint{4.117603in}{1.468008in}}{\pgfqpoint{4.112559in}{1.462964in}}%
\pgfpathcurveto{\pgfqpoint{4.107516in}{1.457920in}}{\pgfqpoint{4.104682in}{1.451079in}}{\pgfqpoint{4.104682in}{1.443946in}}%
\pgfpathcurveto{\pgfqpoint{4.104682in}{1.436813in}}{\pgfqpoint{4.107516in}{1.429971in}}{\pgfqpoint{4.112559in}{1.424928in}}%
\pgfpathcurveto{\pgfqpoint{4.117603in}{1.419884in}}{\pgfqpoint{4.124445in}{1.417050in}}{\pgfqpoint{4.131577in}{1.417050in}}%
\pgfpathclose%
\pgfusepath{stroke,fill}%
\end{pgfscope}%
\begin{pgfscope}%
\pgfpathrectangle{\pgfqpoint{2.867647in}{0.500000in}}{\pgfqpoint{1.764706in}{1.700000in}}%
\pgfusepath{clip}%
\pgfsetbuttcap%
\pgfsetroundjoin%
\definecolor{currentfill}{rgb}{0.979124,0.903132,0.839793}%
\pgfsetfillcolor{currentfill}%
\pgfsetlinewidth{0.311001pt}%
\definecolor{currentstroke}{rgb}{1.000000,1.000000,1.000000}%
\pgfsetstrokecolor{currentstroke}%
\pgfsetdash{}{0pt}%
\pgfpathmoveto{\pgfqpoint{4.210953in}{1.429948in}}%
\pgfpathcurveto{\pgfqpoint{4.218086in}{1.429948in}}{\pgfqpoint{4.224927in}{1.432782in}}{\pgfqpoint{4.229971in}{1.437826in}}%
\pgfpathcurveto{\pgfqpoint{4.235015in}{1.442869in}}{\pgfqpoint{4.237849in}{1.449711in}}{\pgfqpoint{4.237849in}{1.456844in}}%
\pgfpathcurveto{\pgfqpoint{4.237849in}{1.463977in}}{\pgfqpoint{4.235015in}{1.470818in}}{\pgfqpoint{4.229971in}{1.475862in}}%
\pgfpathcurveto{\pgfqpoint{4.224927in}{1.480906in}}{\pgfqpoint{4.218086in}{1.483740in}}{\pgfqpoint{4.210953in}{1.483740in}}%
\pgfpathcurveto{\pgfqpoint{4.203820in}{1.483740in}}{\pgfqpoint{4.196979in}{1.480906in}}{\pgfqpoint{4.191935in}{1.475862in}}%
\pgfpathcurveto{\pgfqpoint{4.186891in}{1.470818in}}{\pgfqpoint{4.184057in}{1.463977in}}{\pgfqpoint{4.184057in}{1.456844in}}%
\pgfpathcurveto{\pgfqpoint{4.184057in}{1.449711in}}{\pgfqpoint{4.186891in}{1.442869in}}{\pgfqpoint{4.191935in}{1.437826in}}%
\pgfpathcurveto{\pgfqpoint{4.196979in}{1.432782in}}{\pgfqpoint{4.203820in}{1.429948in}}{\pgfqpoint{4.210953in}{1.429948in}}%
\pgfpathclose%
\pgfusepath{stroke,fill}%
\end{pgfscope}%
\begin{pgfscope}%
\pgfpathrectangle{\pgfqpoint{2.867647in}{0.500000in}}{\pgfqpoint{1.764706in}{1.700000in}}%
\pgfusepath{clip}%
\pgfsetbuttcap%
\pgfsetroundjoin%
\definecolor{currentfill}{rgb}{0.967398,0.774513,0.650573}%
\pgfsetfillcolor{currentfill}%
\pgfsetlinewidth{0.311001pt}%
\definecolor{currentstroke}{rgb}{1.000000,1.000000,1.000000}%
\pgfsetstrokecolor{currentstroke}%
\pgfsetdash{}{0pt}%
\pgfpathmoveto{\pgfqpoint{4.020874in}{1.660409in}}%
\pgfpathcurveto{\pgfqpoint{4.028007in}{1.660409in}}{\pgfqpoint{4.034849in}{1.663243in}}{\pgfqpoint{4.039892in}{1.668286in}}%
\pgfpathcurveto{\pgfqpoint{4.044936in}{1.673330in}}{\pgfqpoint{4.047770in}{1.680172in}}{\pgfqpoint{4.047770in}{1.687305in}}%
\pgfpathcurveto{\pgfqpoint{4.047770in}{1.694437in}}{\pgfqpoint{4.044936in}{1.701279in}}{\pgfqpoint{4.039892in}{1.706323in}}%
\pgfpathcurveto{\pgfqpoint{4.034849in}{1.711366in}}{\pgfqpoint{4.028007in}{1.714200in}}{\pgfqpoint{4.020874in}{1.714200in}}%
\pgfpathcurveto{\pgfqpoint{4.013741in}{1.714200in}}{\pgfqpoint{4.006900in}{1.711366in}}{\pgfqpoint{4.001856in}{1.706323in}}%
\pgfpathcurveto{\pgfqpoint{3.996812in}{1.701279in}}{\pgfqpoint{3.993978in}{1.694437in}}{\pgfqpoint{3.993978in}{1.687305in}}%
\pgfpathcurveto{\pgfqpoint{3.993978in}{1.680172in}}{\pgfqpoint{3.996812in}{1.673330in}}{\pgfqpoint{4.001856in}{1.668286in}}%
\pgfpathcurveto{\pgfqpoint{4.006900in}{1.663243in}}{\pgfqpoint{4.013741in}{1.660409in}}{\pgfqpoint{4.020874in}{1.660409in}}%
\pgfpathclose%
\pgfusepath{stroke,fill}%
\end{pgfscope}%
\begin{pgfscope}%
\pgfpathrectangle{\pgfqpoint{2.867647in}{0.500000in}}{\pgfqpoint{1.764706in}{1.700000in}}%
\pgfusepath{clip}%
\pgfsetbuttcap%
\pgfsetroundjoin%
\definecolor{currentfill}{rgb}{0.970255,0.815666,0.711203}%
\pgfsetfillcolor{currentfill}%
\pgfsetlinewidth{0.311001pt}%
\definecolor{currentstroke}{rgb}{1.000000,1.000000,1.000000}%
\pgfsetstrokecolor{currentstroke}%
\pgfsetdash{}{0pt}%
\pgfpathmoveto{\pgfqpoint{4.165273in}{0.991705in}}%
\pgfpathcurveto{\pgfqpoint{4.172406in}{0.991705in}}{\pgfqpoint{4.179248in}{0.994539in}}{\pgfqpoint{4.184291in}{0.999583in}}%
\pgfpathcurveto{\pgfqpoint{4.189335in}{1.004627in}}{\pgfqpoint{4.192169in}{1.011468in}}{\pgfqpoint{4.192169in}{1.018601in}}%
\pgfpathcurveto{\pgfqpoint{4.192169in}{1.025734in}}{\pgfqpoint{4.189335in}{1.032576in}}{\pgfqpoint{4.184291in}{1.037619in}}%
\pgfpathcurveto{\pgfqpoint{4.179248in}{1.042663in}}{\pgfqpoint{4.172406in}{1.045497in}}{\pgfqpoint{4.165273in}{1.045497in}}%
\pgfpathcurveto{\pgfqpoint{4.158140in}{1.045497in}}{\pgfqpoint{4.151299in}{1.042663in}}{\pgfqpoint{4.146255in}{1.037619in}}%
\pgfpathcurveto{\pgfqpoint{4.141212in}{1.032576in}}{\pgfqpoint{4.138378in}{1.025734in}}{\pgfqpoint{4.138378in}{1.018601in}}%
\pgfpathcurveto{\pgfqpoint{4.138378in}{1.011468in}}{\pgfqpoint{4.141212in}{1.004627in}}{\pgfqpoint{4.146255in}{0.999583in}}%
\pgfpathcurveto{\pgfqpoint{4.151299in}{0.994539in}}{\pgfqpoint{4.158140in}{0.991705in}}{\pgfqpoint{4.165273in}{0.991705in}}%
\pgfpathclose%
\pgfusepath{stroke,fill}%
\end{pgfscope}%
\begin{pgfscope}%
\pgfpathrectangle{\pgfqpoint{2.867647in}{0.500000in}}{\pgfqpoint{1.764706in}{1.700000in}}%
\pgfusepath{clip}%
\pgfsetbuttcap%
\pgfsetroundjoin%
\definecolor{currentfill}{rgb}{0.968931,0.798091,0.685123}%
\pgfsetfillcolor{currentfill}%
\pgfsetlinewidth{0.311001pt}%
\definecolor{currentstroke}{rgb}{1.000000,1.000000,1.000000}%
\pgfsetstrokecolor{currentstroke}%
\pgfsetdash{}{0pt}%
\pgfpathmoveto{\pgfqpoint{4.239015in}{1.550307in}}%
\pgfpathcurveto{\pgfqpoint{4.246147in}{1.550307in}}{\pgfqpoint{4.252989in}{1.553141in}}{\pgfqpoint{4.258033in}{1.558184in}}%
\pgfpathcurveto{\pgfqpoint{4.263076in}{1.563228in}}{\pgfqpoint{4.265910in}{1.570069in}}{\pgfqpoint{4.265910in}{1.577202in}}%
\pgfpathcurveto{\pgfqpoint{4.265910in}{1.584335in}}{\pgfqpoint{4.263076in}{1.591177in}}{\pgfqpoint{4.258033in}{1.596220in}}%
\pgfpathcurveto{\pgfqpoint{4.252989in}{1.601264in}}{\pgfqpoint{4.246147in}{1.604098in}}{\pgfqpoint{4.239015in}{1.604098in}}%
\pgfpathcurveto{\pgfqpoint{4.231882in}{1.604098in}}{\pgfqpoint{4.225040in}{1.601264in}}{\pgfqpoint{4.219996in}{1.596220in}}%
\pgfpathcurveto{\pgfqpoint{4.214953in}{1.591177in}}{\pgfqpoint{4.212119in}{1.584335in}}{\pgfqpoint{4.212119in}{1.577202in}}%
\pgfpathcurveto{\pgfqpoint{4.212119in}{1.570069in}}{\pgfqpoint{4.214953in}{1.563228in}}{\pgfqpoint{4.219996in}{1.558184in}}%
\pgfpathcurveto{\pgfqpoint{4.225040in}{1.553141in}}{\pgfqpoint{4.231882in}{1.550307in}}{\pgfqpoint{4.239015in}{1.550307in}}%
\pgfpathclose%
\pgfusepath{stroke,fill}%
\end{pgfscope}%
\begin{pgfscope}%
\pgfpathrectangle{\pgfqpoint{2.867647in}{0.500000in}}{\pgfqpoint{1.764706in}{1.700000in}}%
\pgfusepath{clip}%
\pgfsetbuttcap%
\pgfsetroundjoin%
\definecolor{currentfill}{rgb}{0.966120,0.744512,0.608720}%
\pgfsetfillcolor{currentfill}%
\pgfsetlinewidth{0.311001pt}%
\definecolor{currentstroke}{rgb}{1.000000,1.000000,1.000000}%
\pgfsetstrokecolor{currentstroke}%
\pgfsetdash{}{0pt}%
\pgfpathmoveto{\pgfqpoint{4.028121in}{1.726294in}}%
\pgfpathcurveto{\pgfqpoint{4.035253in}{1.726294in}}{\pgfqpoint{4.042095in}{1.729128in}}{\pgfqpoint{4.047139in}{1.734172in}}%
\pgfpathcurveto{\pgfqpoint{4.052182in}{1.739216in}}{\pgfqpoint{4.055016in}{1.746057in}}{\pgfqpoint{4.055016in}{1.753190in}}%
\pgfpathcurveto{\pgfqpoint{4.055016in}{1.760323in}}{\pgfqpoint{4.052182in}{1.767165in}}{\pgfqpoint{4.047139in}{1.772208in}}%
\pgfpathcurveto{\pgfqpoint{4.042095in}{1.777252in}}{\pgfqpoint{4.035253in}{1.780086in}}{\pgfqpoint{4.028121in}{1.780086in}}%
\pgfpathcurveto{\pgfqpoint{4.020988in}{1.780086in}}{\pgfqpoint{4.014146in}{1.777252in}}{\pgfqpoint{4.009102in}{1.772208in}}%
\pgfpathcurveto{\pgfqpoint{4.004059in}{1.767165in}}{\pgfqpoint{4.001225in}{1.760323in}}{\pgfqpoint{4.001225in}{1.753190in}}%
\pgfpathcurveto{\pgfqpoint{4.001225in}{1.746057in}}{\pgfqpoint{4.004059in}{1.739216in}}{\pgfqpoint{4.009102in}{1.734172in}}%
\pgfpathcurveto{\pgfqpoint{4.014146in}{1.729128in}}{\pgfqpoint{4.020988in}{1.726294in}}{\pgfqpoint{4.028121in}{1.726294in}}%
\pgfpathclose%
\pgfusepath{stroke,fill}%
\end{pgfscope}%
\begin{pgfscope}%
\pgfpathrectangle{\pgfqpoint{2.867647in}{0.500000in}}{\pgfqpoint{1.764706in}{1.700000in}}%
\pgfusepath{clip}%
\pgfsetbuttcap%
\pgfsetroundjoin%
\definecolor{currentfill}{rgb}{0.980678,0.914765,0.856766}%
\pgfsetfillcolor{currentfill}%
\pgfsetlinewidth{0.311001pt}%
\definecolor{currentstroke}{rgb}{1.000000,1.000000,1.000000}%
\pgfsetstrokecolor{currentstroke}%
\pgfsetdash{}{0pt}%
\pgfpathmoveto{\pgfqpoint{4.209343in}{1.328997in}}%
\pgfpathcurveto{\pgfqpoint{4.216476in}{1.328997in}}{\pgfqpoint{4.223318in}{1.331831in}}{\pgfqpoint{4.228361in}{1.336875in}}%
\pgfpathcurveto{\pgfqpoint{4.233405in}{1.341919in}}{\pgfqpoint{4.236239in}{1.348760in}}{\pgfqpoint{4.236239in}{1.355893in}}%
\pgfpathcurveto{\pgfqpoint{4.236239in}{1.363026in}}{\pgfqpoint{4.233405in}{1.369868in}}{\pgfqpoint{4.228361in}{1.374911in}}%
\pgfpathcurveto{\pgfqpoint{4.223318in}{1.379955in}}{\pgfqpoint{4.216476in}{1.382789in}}{\pgfqpoint{4.209343in}{1.382789in}}%
\pgfpathcurveto{\pgfqpoint{4.202210in}{1.382789in}}{\pgfqpoint{4.195369in}{1.379955in}}{\pgfqpoint{4.190325in}{1.374911in}}%
\pgfpathcurveto{\pgfqpoint{4.185281in}{1.369868in}}{\pgfqpoint{4.182447in}{1.363026in}}{\pgfqpoint{4.182447in}{1.355893in}}%
\pgfpathcurveto{\pgfqpoint{4.182447in}{1.348760in}}{\pgfqpoint{4.185281in}{1.341919in}}{\pgfqpoint{4.190325in}{1.336875in}}%
\pgfpathcurveto{\pgfqpoint{4.195369in}{1.331831in}}{\pgfqpoint{4.202210in}{1.328997in}}{\pgfqpoint{4.209343in}{1.328997in}}%
\pgfpathclose%
\pgfusepath{stroke,fill}%
\end{pgfscope}%
\begin{pgfscope}%
\pgfpathrectangle{\pgfqpoint{2.867647in}{0.500000in}}{\pgfqpoint{1.764706in}{1.700000in}}%
\pgfusepath{clip}%
\pgfsetbuttcap%
\pgfsetroundjoin%
\definecolor{currentfill}{rgb}{0.976961,0.885681,0.814303}%
\pgfsetfillcolor{currentfill}%
\pgfsetlinewidth{0.311001pt}%
\definecolor{currentstroke}{rgb}{1.000000,1.000000,1.000000}%
\pgfsetstrokecolor{currentstroke}%
\pgfsetdash{}{0pt}%
\pgfpathmoveto{\pgfqpoint{4.108641in}{1.044716in}}%
\pgfpathcurveto{\pgfqpoint{4.115774in}{1.044716in}}{\pgfqpoint{4.122616in}{1.047550in}}{\pgfqpoint{4.127659in}{1.052594in}}%
\pgfpathcurveto{\pgfqpoint{4.132703in}{1.057637in}}{\pgfqpoint{4.135537in}{1.064479in}}{\pgfqpoint{4.135537in}{1.071612in}}%
\pgfpathcurveto{\pgfqpoint{4.135537in}{1.078745in}}{\pgfqpoint{4.132703in}{1.085586in}}{\pgfqpoint{4.127659in}{1.090630in}}%
\pgfpathcurveto{\pgfqpoint{4.122616in}{1.095674in}}{\pgfqpoint{4.115774in}{1.098508in}}{\pgfqpoint{4.108641in}{1.098508in}}%
\pgfpathcurveto{\pgfqpoint{4.101508in}{1.098508in}}{\pgfqpoint{4.094667in}{1.095674in}}{\pgfqpoint{4.089623in}{1.090630in}}%
\pgfpathcurveto{\pgfqpoint{4.084579in}{1.085586in}}{\pgfqpoint{4.081745in}{1.078745in}}{\pgfqpoint{4.081745in}{1.071612in}}%
\pgfpathcurveto{\pgfqpoint{4.081745in}{1.064479in}}{\pgfqpoint{4.084579in}{1.057637in}}{\pgfqpoint{4.089623in}{1.052594in}}%
\pgfpathcurveto{\pgfqpoint{4.094667in}{1.047550in}}{\pgfqpoint{4.101508in}{1.044716in}}{\pgfqpoint{4.108641in}{1.044716in}}%
\pgfpathclose%
\pgfusepath{stroke,fill}%
\end{pgfscope}%
\begin{pgfscope}%
\pgfpathrectangle{\pgfqpoint{2.867647in}{0.500000in}}{\pgfqpoint{1.764706in}{1.700000in}}%
\pgfusepath{clip}%
\pgfsetbuttcap%
\pgfsetroundjoin%
\definecolor{currentfill}{rgb}{0.979124,0.903132,0.839793}%
\pgfsetfillcolor{currentfill}%
\pgfsetlinewidth{0.311001pt}%
\definecolor{currentstroke}{rgb}{1.000000,1.000000,1.000000}%
\pgfsetstrokecolor{currentstroke}%
\pgfsetdash{}{0pt}%
\pgfpathmoveto{\pgfqpoint{4.119537in}{1.537848in}}%
\pgfpathcurveto{\pgfqpoint{4.126670in}{1.537848in}}{\pgfqpoint{4.133512in}{1.540682in}}{\pgfqpoint{4.138555in}{1.545726in}}%
\pgfpathcurveto{\pgfqpoint{4.143599in}{1.550770in}}{\pgfqpoint{4.146433in}{1.557611in}}{\pgfqpoint{4.146433in}{1.564744in}}%
\pgfpathcurveto{\pgfqpoint{4.146433in}{1.571877in}}{\pgfqpoint{4.143599in}{1.578719in}}{\pgfqpoint{4.138555in}{1.583762in}}%
\pgfpathcurveto{\pgfqpoint{4.133512in}{1.588806in}}{\pgfqpoint{4.126670in}{1.591640in}}{\pgfqpoint{4.119537in}{1.591640in}}%
\pgfpathcurveto{\pgfqpoint{4.112404in}{1.591640in}}{\pgfqpoint{4.105563in}{1.588806in}}{\pgfqpoint{4.100519in}{1.583762in}}%
\pgfpathcurveto{\pgfqpoint{4.095475in}{1.578719in}}{\pgfqpoint{4.092642in}{1.571877in}}{\pgfqpoint{4.092642in}{1.564744in}}%
\pgfpathcurveto{\pgfqpoint{4.092642in}{1.557611in}}{\pgfqpoint{4.095475in}{1.550770in}}{\pgfqpoint{4.100519in}{1.545726in}}%
\pgfpathcurveto{\pgfqpoint{4.105563in}{1.540682in}}{\pgfqpoint{4.112404in}{1.537848in}}{\pgfqpoint{4.119537in}{1.537848in}}%
\pgfpathclose%
\pgfusepath{stroke,fill}%
\end{pgfscope}%
\begin{pgfscope}%
\pgfpathrectangle{\pgfqpoint{2.867647in}{0.500000in}}{\pgfqpoint{1.764706in}{1.700000in}}%
\pgfusepath{clip}%
\pgfsetbuttcap%
\pgfsetroundjoin%
\definecolor{currentfill}{rgb}{0.977657,0.891500,0.822809}%
\pgfsetfillcolor{currentfill}%
\pgfsetlinewidth{0.311001pt}%
\definecolor{currentstroke}{rgb}{1.000000,1.000000,1.000000}%
\pgfsetstrokecolor{currentstroke}%
\pgfsetdash{}{0pt}%
\pgfpathmoveto{\pgfqpoint{4.148402in}{1.291462in}}%
\pgfpathcurveto{\pgfqpoint{4.155535in}{1.291462in}}{\pgfqpoint{4.162377in}{1.294295in}}{\pgfqpoint{4.167420in}{1.299339in}}%
\pgfpathcurveto{\pgfqpoint{4.172464in}{1.304383in}}{\pgfqpoint{4.175298in}{1.311224in}}{\pgfqpoint{4.175298in}{1.318357in}}%
\pgfpathcurveto{\pgfqpoint{4.175298in}{1.325490in}}{\pgfqpoint{4.172464in}{1.332332in}}{\pgfqpoint{4.167420in}{1.337375in}}%
\pgfpathcurveto{\pgfqpoint{4.162377in}{1.342419in}}{\pgfqpoint{4.155535in}{1.345253in}}{\pgfqpoint{4.148402in}{1.345253in}}%
\pgfpathcurveto{\pgfqpoint{4.141269in}{1.345253in}}{\pgfqpoint{4.134428in}{1.342419in}}{\pgfqpoint{4.129384in}{1.337375in}}%
\pgfpathcurveto{\pgfqpoint{4.124340in}{1.332332in}}{\pgfqpoint{4.121506in}{1.325490in}}{\pgfqpoint{4.121506in}{1.318357in}}%
\pgfpathcurveto{\pgfqpoint{4.121506in}{1.311224in}}{\pgfqpoint{4.124340in}{1.304383in}}{\pgfqpoint{4.129384in}{1.299339in}}%
\pgfpathcurveto{\pgfqpoint{4.134428in}{1.294295in}}{\pgfqpoint{4.141269in}{1.291462in}}{\pgfqpoint{4.148402in}{1.291462in}}%
\pgfpathclose%
\pgfusepath{stroke,fill}%
\end{pgfscope}%
\begin{pgfscope}%
\pgfpathrectangle{\pgfqpoint{2.867647in}{0.500000in}}{\pgfqpoint{1.764706in}{1.700000in}}%
\pgfusepath{clip}%
\pgfsetbuttcap%
\pgfsetroundjoin%
\definecolor{currentfill}{rgb}{0.975644,0.874038,0.797253}%
\pgfsetfillcolor{currentfill}%
\pgfsetlinewidth{0.311001pt}%
\definecolor{currentstroke}{rgb}{1.000000,1.000000,1.000000}%
\pgfsetstrokecolor{currentstroke}%
\pgfsetdash{}{0pt}%
\pgfpathmoveto{\pgfqpoint{4.129674in}{1.402207in}}%
\pgfpathcurveto{\pgfqpoint{4.136807in}{1.402207in}}{\pgfqpoint{4.143649in}{1.405041in}}{\pgfqpoint{4.148692in}{1.410085in}}%
\pgfpathcurveto{\pgfqpoint{4.153736in}{1.415128in}}{\pgfqpoint{4.156570in}{1.421970in}}{\pgfqpoint{4.156570in}{1.429103in}}%
\pgfpathcurveto{\pgfqpoint{4.156570in}{1.436236in}}{\pgfqpoint{4.153736in}{1.443077in}}{\pgfqpoint{4.148692in}{1.448121in}}%
\pgfpathcurveto{\pgfqpoint{4.143649in}{1.453165in}}{\pgfqpoint{4.136807in}{1.455999in}}{\pgfqpoint{4.129674in}{1.455999in}}%
\pgfpathcurveto{\pgfqpoint{4.122541in}{1.455999in}}{\pgfqpoint{4.115700in}{1.453165in}}{\pgfqpoint{4.110656in}{1.448121in}}%
\pgfpathcurveto{\pgfqpoint{4.105612in}{1.443077in}}{\pgfqpoint{4.102779in}{1.436236in}}{\pgfqpoint{4.102779in}{1.429103in}}%
\pgfpathcurveto{\pgfqpoint{4.102779in}{1.421970in}}{\pgfqpoint{4.105612in}{1.415128in}}{\pgfqpoint{4.110656in}{1.410085in}}%
\pgfpathcurveto{\pgfqpoint{4.115700in}{1.405041in}}{\pgfqpoint{4.122541in}{1.402207in}}{\pgfqpoint{4.129674in}{1.402207in}}%
\pgfpathclose%
\pgfusepath{stroke,fill}%
\end{pgfscope}%
\begin{pgfscope}%
\pgfpathrectangle{\pgfqpoint{2.867647in}{0.500000in}}{\pgfqpoint{1.764706in}{1.700000in}}%
\pgfusepath{clip}%
\pgfsetbuttcap%
\pgfsetroundjoin%
\definecolor{currentfill}{rgb}{0.969803,0.809811,0.702523}%
\pgfsetfillcolor{currentfill}%
\pgfsetlinewidth{0.311001pt}%
\definecolor{currentstroke}{rgb}{1.000000,1.000000,1.000000}%
\pgfsetstrokecolor{currentstroke}%
\pgfsetdash{}{0pt}%
\pgfpathmoveto{\pgfqpoint{4.107307in}{0.950615in}}%
\pgfpathcurveto{\pgfqpoint{4.114440in}{0.950615in}}{\pgfqpoint{4.121282in}{0.953449in}}{\pgfqpoint{4.126325in}{0.958493in}}%
\pgfpathcurveto{\pgfqpoint{4.131369in}{0.963536in}}{\pgfqpoint{4.134203in}{0.970378in}}{\pgfqpoint{4.134203in}{0.977511in}}%
\pgfpathcurveto{\pgfqpoint{4.134203in}{0.984644in}}{\pgfqpoint{4.131369in}{0.991485in}}{\pgfqpoint{4.126325in}{0.996529in}}%
\pgfpathcurveto{\pgfqpoint{4.121282in}{1.001573in}}{\pgfqpoint{4.114440in}{1.004407in}}{\pgfqpoint{4.107307in}{1.004407in}}%
\pgfpathcurveto{\pgfqpoint{4.100174in}{1.004407in}}{\pgfqpoint{4.093333in}{1.001573in}}{\pgfqpoint{4.088289in}{0.996529in}}%
\pgfpathcurveto{\pgfqpoint{4.083245in}{0.991485in}}{\pgfqpoint{4.080411in}{0.984644in}}{\pgfqpoint{4.080411in}{0.977511in}}%
\pgfpathcurveto{\pgfqpoint{4.080411in}{0.970378in}}{\pgfqpoint{4.083245in}{0.963536in}}{\pgfqpoint{4.088289in}{0.958493in}}%
\pgfpathcurveto{\pgfqpoint{4.093333in}{0.953449in}}{\pgfqpoint{4.100174in}{0.950615in}}{\pgfqpoint{4.107307in}{0.950615in}}%
\pgfpathclose%
\pgfusepath{stroke,fill}%
\end{pgfscope}%
\begin{pgfscope}%
\pgfpathrectangle{\pgfqpoint{2.867647in}{0.500000in}}{\pgfqpoint{1.764706in}{1.700000in}}%
\pgfusepath{clip}%
\pgfsetbuttcap%
\pgfsetroundjoin%
\definecolor{currentfill}{rgb}{0.975018,0.868213,0.788710}%
\pgfsetfillcolor{currentfill}%
\pgfsetlinewidth{0.311001pt}%
\definecolor{currentstroke}{rgb}{1.000000,1.000000,1.000000}%
\pgfsetstrokecolor{currentstroke}%
\pgfsetdash{}{0pt}%
\pgfpathmoveto{\pgfqpoint{4.089462in}{1.025155in}}%
\pgfpathcurveto{\pgfqpoint{4.096595in}{1.025155in}}{\pgfqpoint{4.103436in}{1.027989in}}{\pgfqpoint{4.108480in}{1.033033in}}%
\pgfpathcurveto{\pgfqpoint{4.113524in}{1.038077in}}{\pgfqpoint{4.116358in}{1.044918in}}{\pgfqpoint{4.116358in}{1.052051in}}%
\pgfpathcurveto{\pgfqpoint{4.116358in}{1.059184in}}{\pgfqpoint{4.113524in}{1.066026in}}{\pgfqpoint{4.108480in}{1.071069in}}%
\pgfpathcurveto{\pgfqpoint{4.103436in}{1.076113in}}{\pgfqpoint{4.096595in}{1.078947in}}{\pgfqpoint{4.089462in}{1.078947in}}%
\pgfpathcurveto{\pgfqpoint{4.082329in}{1.078947in}}{\pgfqpoint{4.075487in}{1.076113in}}{\pgfqpoint{4.070444in}{1.071069in}}%
\pgfpathcurveto{\pgfqpoint{4.065400in}{1.066026in}}{\pgfqpoint{4.062566in}{1.059184in}}{\pgfqpoint{4.062566in}{1.052051in}}%
\pgfpathcurveto{\pgfqpoint{4.062566in}{1.044918in}}{\pgfqpoint{4.065400in}{1.038077in}}{\pgfqpoint{4.070444in}{1.033033in}}%
\pgfpathcurveto{\pgfqpoint{4.075487in}{1.027989in}}{\pgfqpoint{4.082329in}{1.025155in}}{\pgfqpoint{4.089462in}{1.025155in}}%
\pgfpathclose%
\pgfusepath{stroke,fill}%
\end{pgfscope}%
\begin{pgfscope}%
\pgfpathrectangle{\pgfqpoint{2.867647in}{0.500000in}}{\pgfqpoint{1.764706in}{1.700000in}}%
\pgfusepath{clip}%
\pgfsetbuttcap%
\pgfsetroundjoin%
\definecolor{currentfill}{rgb}{0.980678,0.914765,0.856766}%
\pgfsetfillcolor{currentfill}%
\pgfsetlinewidth{0.311001pt}%
\definecolor{currentstroke}{rgb}{1.000000,1.000000,1.000000}%
\pgfsetstrokecolor{currentstroke}%
\pgfsetdash{}{0pt}%
\pgfpathmoveto{\pgfqpoint{4.142892in}{1.540276in}}%
\pgfpathcurveto{\pgfqpoint{4.150025in}{1.540276in}}{\pgfqpoint{4.156867in}{1.543110in}}{\pgfqpoint{4.161911in}{1.548154in}}%
\pgfpathcurveto{\pgfqpoint{4.166954in}{1.553198in}}{\pgfqpoint{4.169788in}{1.560039in}}{\pgfqpoint{4.169788in}{1.567172in}}%
\pgfpathcurveto{\pgfqpoint{4.169788in}{1.574305in}}{\pgfqpoint{4.166954in}{1.581147in}}{\pgfqpoint{4.161911in}{1.586190in}}%
\pgfpathcurveto{\pgfqpoint{4.156867in}{1.591234in}}{\pgfqpoint{4.150025in}{1.594068in}}{\pgfqpoint{4.142892in}{1.594068in}}%
\pgfpathcurveto{\pgfqpoint{4.135760in}{1.594068in}}{\pgfqpoint{4.128918in}{1.591234in}}{\pgfqpoint{4.123874in}{1.586190in}}%
\pgfpathcurveto{\pgfqpoint{4.118831in}{1.581147in}}{\pgfqpoint{4.115997in}{1.574305in}}{\pgfqpoint{4.115997in}{1.567172in}}%
\pgfpathcurveto{\pgfqpoint{4.115997in}{1.560039in}}{\pgfqpoint{4.118831in}{1.553198in}}{\pgfqpoint{4.123874in}{1.548154in}}%
\pgfpathcurveto{\pgfqpoint{4.128918in}{1.543110in}}{\pgfqpoint{4.135760in}{1.540276in}}{\pgfqpoint{4.142892in}{1.540276in}}%
\pgfpathclose%
\pgfusepath{stroke,fill}%
\end{pgfscope}%
\begin{pgfscope}%
\pgfpathrectangle{\pgfqpoint{2.867647in}{0.500000in}}{\pgfqpoint{1.764706in}{1.700000in}}%
\pgfusepath{clip}%
\pgfsetbuttcap%
\pgfsetroundjoin%
\definecolor{currentfill}{rgb}{0.935991,0.337039,0.249722}%
\pgfsetfillcolor{currentfill}%
\pgfsetlinewidth{0.311001pt}%
\definecolor{currentstroke}{rgb}{1.000000,1.000000,1.000000}%
\pgfsetstrokecolor{currentstroke}%
\pgfsetdash{}{0pt}%
\pgfpathmoveto{\pgfqpoint{4.295400in}{1.662546in}}%
\pgfpathcurveto{\pgfqpoint{4.302533in}{1.662546in}}{\pgfqpoint{4.309374in}{1.665380in}}{\pgfqpoint{4.314418in}{1.670424in}}%
\pgfpathcurveto{\pgfqpoint{4.319462in}{1.675467in}}{\pgfqpoint{4.322296in}{1.682309in}}{\pgfqpoint{4.322296in}{1.689442in}}%
\pgfpathcurveto{\pgfqpoint{4.322296in}{1.696575in}}{\pgfqpoint{4.319462in}{1.703416in}}{\pgfqpoint{4.314418in}{1.708460in}}%
\pgfpathcurveto{\pgfqpoint{4.309374in}{1.713504in}}{\pgfqpoint{4.302533in}{1.716338in}}{\pgfqpoint{4.295400in}{1.716338in}}%
\pgfpathcurveto{\pgfqpoint{4.288267in}{1.716338in}}{\pgfqpoint{4.281425in}{1.713504in}}{\pgfqpoint{4.276382in}{1.708460in}}%
\pgfpathcurveto{\pgfqpoint{4.271338in}{1.703416in}}{\pgfqpoint{4.268504in}{1.696575in}}{\pgfqpoint{4.268504in}{1.689442in}}%
\pgfpathcurveto{\pgfqpoint{4.268504in}{1.682309in}}{\pgfqpoint{4.271338in}{1.675467in}}{\pgfqpoint{4.276382in}{1.670424in}}%
\pgfpathcurveto{\pgfqpoint{4.281425in}{1.665380in}}{\pgfqpoint{4.288267in}{1.662546in}}{\pgfqpoint{4.295400in}{1.662546in}}%
\pgfpathclose%
\pgfusepath{stroke,fill}%
\end{pgfscope}%
\begin{pgfscope}%
\pgfpathrectangle{\pgfqpoint{2.867647in}{0.500000in}}{\pgfqpoint{1.764706in}{1.700000in}}%
\pgfusepath{clip}%
\pgfsetbuttcap%
\pgfsetroundjoin%
\definecolor{currentfill}{rgb}{0.973271,0.850724,0.762998}%
\pgfsetfillcolor{currentfill}%
\pgfsetlinewidth{0.311001pt}%
\definecolor{currentstroke}{rgb}{1.000000,1.000000,1.000000}%
\pgfsetstrokecolor{currentstroke}%
\pgfsetdash{}{0pt}%
\pgfpathmoveto{\pgfqpoint{4.204260in}{1.072838in}}%
\pgfpathcurveto{\pgfqpoint{4.211393in}{1.072838in}}{\pgfqpoint{4.218235in}{1.075672in}}{\pgfqpoint{4.223278in}{1.080715in}}%
\pgfpathcurveto{\pgfqpoint{4.228322in}{1.085759in}}{\pgfqpoint{4.231156in}{1.092601in}}{\pgfqpoint{4.231156in}{1.099734in}}%
\pgfpathcurveto{\pgfqpoint{4.231156in}{1.106866in}}{\pgfqpoint{4.228322in}{1.113708in}}{\pgfqpoint{4.223278in}{1.118752in}}%
\pgfpathcurveto{\pgfqpoint{4.218235in}{1.123795in}}{\pgfqpoint{4.211393in}{1.126629in}}{\pgfqpoint{4.204260in}{1.126629in}}%
\pgfpathcurveto{\pgfqpoint{4.197127in}{1.126629in}}{\pgfqpoint{4.190286in}{1.123795in}}{\pgfqpoint{4.185242in}{1.118752in}}%
\pgfpathcurveto{\pgfqpoint{4.180198in}{1.113708in}}{\pgfqpoint{4.177364in}{1.106866in}}{\pgfqpoint{4.177364in}{1.099734in}}%
\pgfpathcurveto{\pgfqpoint{4.177364in}{1.092601in}}{\pgfqpoint{4.180198in}{1.085759in}}{\pgfqpoint{4.185242in}{1.080715in}}%
\pgfpathcurveto{\pgfqpoint{4.190286in}{1.075672in}}{\pgfqpoint{4.197127in}{1.072838in}}{\pgfqpoint{4.204260in}{1.072838in}}%
\pgfpathclose%
\pgfusepath{stroke,fill}%
\end{pgfscope}%
\begin{pgfscope}%
\pgfpathrectangle{\pgfqpoint{2.867647in}{0.500000in}}{\pgfqpoint{1.764706in}{1.700000in}}%
\pgfusepath{clip}%
\pgfsetbuttcap%
\pgfsetroundjoin%
\definecolor{currentfill}{rgb}{0.972201,0.839051,0.745789}%
\pgfsetfillcolor{currentfill}%
\pgfsetlinewidth{0.311001pt}%
\definecolor{currentstroke}{rgb}{1.000000,1.000000,1.000000}%
\pgfsetstrokecolor{currentstroke}%
\pgfsetdash{}{0pt}%
\pgfpathmoveto{\pgfqpoint{4.057558in}{1.597154in}}%
\pgfpathcurveto{\pgfqpoint{4.064691in}{1.597154in}}{\pgfqpoint{4.071533in}{1.599988in}}{\pgfqpoint{4.076577in}{1.605032in}}%
\pgfpathcurveto{\pgfqpoint{4.081620in}{1.610076in}}{\pgfqpoint{4.084454in}{1.616917in}}{\pgfqpoint{4.084454in}{1.624050in}}%
\pgfpathcurveto{\pgfqpoint{4.084454in}{1.631183in}}{\pgfqpoint{4.081620in}{1.638025in}}{\pgfqpoint{4.076577in}{1.643068in}}%
\pgfpathcurveto{\pgfqpoint{4.071533in}{1.648112in}}{\pgfqpoint{4.064691in}{1.650946in}}{\pgfqpoint{4.057558in}{1.650946in}}%
\pgfpathcurveto{\pgfqpoint{4.050426in}{1.650946in}}{\pgfqpoint{4.043584in}{1.648112in}}{\pgfqpoint{4.038540in}{1.643068in}}%
\pgfpathcurveto{\pgfqpoint{4.033497in}{1.638025in}}{\pgfqpoint{4.030663in}{1.631183in}}{\pgfqpoint{4.030663in}{1.624050in}}%
\pgfpathcurveto{\pgfqpoint{4.030663in}{1.616917in}}{\pgfqpoint{4.033497in}{1.610076in}}{\pgfqpoint{4.038540in}{1.605032in}}%
\pgfpathcurveto{\pgfqpoint{4.043584in}{1.599988in}}{\pgfqpoint{4.050426in}{1.597154in}}{\pgfqpoint{4.057558in}{1.597154in}}%
\pgfpathclose%
\pgfusepath{stroke,fill}%
\end{pgfscope}%
\begin{pgfscope}%
\pgfpathrectangle{\pgfqpoint{2.867647in}{0.500000in}}{\pgfqpoint{1.764706in}{1.700000in}}%
\pgfusepath{clip}%
\pgfsetbuttcap%
\pgfsetroundjoin%
\definecolor{currentfill}{rgb}{0.971694,0.833208,0.737161}%
\pgfsetfillcolor{currentfill}%
\pgfsetlinewidth{0.311001pt}%
\definecolor{currentstroke}{rgb}{1.000000,1.000000,1.000000}%
\pgfsetstrokecolor{currentstroke}%
\pgfsetdash{}{0pt}%
\pgfpathmoveto{\pgfqpoint{4.076318in}{0.975899in}}%
\pgfpathcurveto{\pgfqpoint{4.083451in}{0.975899in}}{\pgfqpoint{4.090293in}{0.978733in}}{\pgfqpoint{4.095336in}{0.983777in}}%
\pgfpathcurveto{\pgfqpoint{4.100380in}{0.988821in}}{\pgfqpoint{4.103214in}{0.995662in}}{\pgfqpoint{4.103214in}{1.002795in}}%
\pgfpathcurveto{\pgfqpoint{4.103214in}{1.009928in}}{\pgfqpoint{4.100380in}{1.016770in}}{\pgfqpoint{4.095336in}{1.021813in}}%
\pgfpathcurveto{\pgfqpoint{4.090293in}{1.026857in}}{\pgfqpoint{4.083451in}{1.029691in}}{\pgfqpoint{4.076318in}{1.029691in}}%
\pgfpathcurveto{\pgfqpoint{4.069185in}{1.029691in}}{\pgfqpoint{4.062344in}{1.026857in}}{\pgfqpoint{4.057300in}{1.021813in}}%
\pgfpathcurveto{\pgfqpoint{4.052256in}{1.016770in}}{\pgfqpoint{4.049422in}{1.009928in}}{\pgfqpoint{4.049422in}{1.002795in}}%
\pgfpathcurveto{\pgfqpoint{4.049422in}{0.995662in}}{\pgfqpoint{4.052256in}{0.988821in}}{\pgfqpoint{4.057300in}{0.983777in}}%
\pgfpathcurveto{\pgfqpoint{4.062344in}{0.978733in}}{\pgfqpoint{4.069185in}{0.975899in}}{\pgfqpoint{4.076318in}{0.975899in}}%
\pgfpathclose%
\pgfusepath{stroke,fill}%
\end{pgfscope}%
\begin{pgfscope}%
\pgfpathrectangle{\pgfqpoint{2.867647in}{0.500000in}}{\pgfqpoint{1.764706in}{1.700000in}}%
\pgfusepath{clip}%
\pgfsetbuttcap%
\pgfsetroundjoin%
\definecolor{currentfill}{rgb}{0.968105,0.786346,0.667739}%
\pgfsetfillcolor{currentfill}%
\pgfsetlinewidth{0.311001pt}%
\definecolor{currentstroke}{rgb}{1.000000,1.000000,1.000000}%
\pgfsetstrokecolor{currentstroke}%
\pgfsetdash{}{0pt}%
\pgfpathmoveto{\pgfqpoint{4.108667in}{1.712427in}}%
\pgfpathcurveto{\pgfqpoint{4.115800in}{1.712427in}}{\pgfqpoint{4.122642in}{1.715261in}}{\pgfqpoint{4.127685in}{1.720304in}}%
\pgfpathcurveto{\pgfqpoint{4.132729in}{1.725348in}}{\pgfqpoint{4.135563in}{1.732190in}}{\pgfqpoint{4.135563in}{1.739322in}}%
\pgfpathcurveto{\pgfqpoint{4.135563in}{1.746455in}}{\pgfqpoint{4.132729in}{1.753297in}}{\pgfqpoint{4.127685in}{1.758341in}}%
\pgfpathcurveto{\pgfqpoint{4.122642in}{1.763384in}}{\pgfqpoint{4.115800in}{1.766218in}}{\pgfqpoint{4.108667in}{1.766218in}}%
\pgfpathcurveto{\pgfqpoint{4.101534in}{1.766218in}}{\pgfqpoint{4.094693in}{1.763384in}}{\pgfqpoint{4.089649in}{1.758341in}}%
\pgfpathcurveto{\pgfqpoint{4.084605in}{1.753297in}}{\pgfqpoint{4.081772in}{1.746455in}}{\pgfqpoint{4.081772in}{1.739322in}}%
\pgfpathcurveto{\pgfqpoint{4.081772in}{1.732190in}}{\pgfqpoint{4.084605in}{1.725348in}}{\pgfqpoint{4.089649in}{1.720304in}}%
\pgfpathcurveto{\pgfqpoint{4.094693in}{1.715261in}}{\pgfqpoint{4.101534in}{1.712427in}}{\pgfqpoint{4.108667in}{1.712427in}}%
\pgfpathclose%
\pgfusepath{stroke,fill}%
\end{pgfscope}%
\begin{pgfscope}%
\pgfpathrectangle{\pgfqpoint{2.867647in}{0.500000in}}{\pgfqpoint{1.764706in}{1.700000in}}%
\pgfusepath{clip}%
\pgfsetbuttcap%
\pgfsetroundjoin%
\definecolor{currentfill}{rgb}{0.964306,0.663930,0.507747}%
\pgfsetfillcolor{currentfill}%
\pgfsetlinewidth{0.311001pt}%
\definecolor{currentstroke}{rgb}{1.000000,1.000000,1.000000}%
\pgfsetstrokecolor{currentstroke}%
\pgfsetdash{}{0pt}%
\pgfpathmoveto{\pgfqpoint{4.009309in}{1.770205in}}%
\pgfpathcurveto{\pgfqpoint{4.016442in}{1.770205in}}{\pgfqpoint{4.023284in}{1.773038in}}{\pgfqpoint{4.028328in}{1.778082in}}%
\pgfpathcurveto{\pgfqpoint{4.033371in}{1.783126in}}{\pgfqpoint{4.036205in}{1.789967in}}{\pgfqpoint{4.036205in}{1.797100in}}%
\pgfpathcurveto{\pgfqpoint{4.036205in}{1.804233in}}{\pgfqpoint{4.033371in}{1.811075in}}{\pgfqpoint{4.028328in}{1.816118in}}%
\pgfpathcurveto{\pgfqpoint{4.023284in}{1.821162in}}{\pgfqpoint{4.016442in}{1.823996in}}{\pgfqpoint{4.009309in}{1.823996in}}%
\pgfpathcurveto{\pgfqpoint{4.002177in}{1.823996in}}{\pgfqpoint{3.995335in}{1.821162in}}{\pgfqpoint{3.990291in}{1.816118in}}%
\pgfpathcurveto{\pgfqpoint{3.985248in}{1.811075in}}{\pgfqpoint{3.982414in}{1.804233in}}{\pgfqpoint{3.982414in}{1.797100in}}%
\pgfpathcurveto{\pgfqpoint{3.982414in}{1.789967in}}{\pgfqpoint{3.985248in}{1.783126in}}{\pgfqpoint{3.990291in}{1.778082in}}%
\pgfpathcurveto{\pgfqpoint{3.995335in}{1.773038in}}{\pgfqpoint{4.002177in}{1.770205in}}{\pgfqpoint{4.009309in}{1.770205in}}%
\pgfpathclose%
\pgfusepath{stroke,fill}%
\end{pgfscope}%
\begin{pgfscope}%
\pgfpathrectangle{\pgfqpoint{2.867647in}{0.500000in}}{\pgfqpoint{1.764706in}{1.700000in}}%
\pgfusepath{clip}%
\pgfsetbuttcap%
\pgfsetroundjoin%
\definecolor{currentfill}{rgb}{0.973832,0.856556,0.771584}%
\pgfsetfillcolor{currentfill}%
\pgfsetlinewidth{0.311001pt}%
\definecolor{currentstroke}{rgb}{1.000000,1.000000,1.000000}%
\pgfsetstrokecolor{currentstroke}%
\pgfsetdash{}{0pt}%
\pgfpathmoveto{\pgfqpoint{4.218206in}{1.528568in}}%
\pgfpathcurveto{\pgfqpoint{4.225339in}{1.528568in}}{\pgfqpoint{4.232181in}{1.531402in}}{\pgfqpoint{4.237224in}{1.536445in}}%
\pgfpathcurveto{\pgfqpoint{4.242268in}{1.541489in}}{\pgfqpoint{4.245102in}{1.548331in}}{\pgfqpoint{4.245102in}{1.555463in}}%
\pgfpathcurveto{\pgfqpoint{4.245102in}{1.562596in}}{\pgfqpoint{4.242268in}{1.569438in}}{\pgfqpoint{4.237224in}{1.574481in}}%
\pgfpathcurveto{\pgfqpoint{4.232181in}{1.579525in}}{\pgfqpoint{4.225339in}{1.582359in}}{\pgfqpoint{4.218206in}{1.582359in}}%
\pgfpathcurveto{\pgfqpoint{4.211074in}{1.582359in}}{\pgfqpoint{4.204232in}{1.579525in}}{\pgfqpoint{4.199188in}{1.574481in}}%
\pgfpathcurveto{\pgfqpoint{4.194145in}{1.569438in}}{\pgfqpoint{4.191311in}{1.562596in}}{\pgfqpoint{4.191311in}{1.555463in}}%
\pgfpathcurveto{\pgfqpoint{4.191311in}{1.548331in}}{\pgfqpoint{4.194145in}{1.541489in}}{\pgfqpoint{4.199188in}{1.536445in}}%
\pgfpathcurveto{\pgfqpoint{4.204232in}{1.531402in}}{\pgfqpoint{4.211074in}{1.528568in}}{\pgfqpoint{4.218206in}{1.528568in}}%
\pgfpathclose%
\pgfusepath{stroke,fill}%
\end{pgfscope}%
\begin{pgfscope}%
\pgfpathrectangle{\pgfqpoint{2.867647in}{0.500000in}}{\pgfqpoint{1.764706in}{1.700000in}}%
\pgfusepath{clip}%
\pgfsetbuttcap%
\pgfsetroundjoin%
\definecolor{currentfill}{rgb}{0.979124,0.903132,0.839793}%
\pgfsetfillcolor{currentfill}%
\pgfsetlinewidth{0.311001pt}%
\definecolor{currentstroke}{rgb}{1.000000,1.000000,1.000000}%
\pgfsetstrokecolor{currentstroke}%
\pgfsetdash{}{0pt}%
\pgfpathmoveto{\pgfqpoint{4.157462in}{1.400837in}}%
\pgfpathcurveto{\pgfqpoint{4.164595in}{1.400837in}}{\pgfqpoint{4.171437in}{1.403671in}}{\pgfqpoint{4.176480in}{1.408715in}}%
\pgfpathcurveto{\pgfqpoint{4.181524in}{1.413758in}}{\pgfqpoint{4.184358in}{1.420600in}}{\pgfqpoint{4.184358in}{1.427733in}}%
\pgfpathcurveto{\pgfqpoint{4.184358in}{1.434866in}}{\pgfqpoint{4.181524in}{1.441707in}}{\pgfqpoint{4.176480in}{1.446751in}}%
\pgfpathcurveto{\pgfqpoint{4.171437in}{1.451795in}}{\pgfqpoint{4.164595in}{1.454629in}}{\pgfqpoint{4.157462in}{1.454629in}}%
\pgfpathcurveto{\pgfqpoint{4.150329in}{1.454629in}}{\pgfqpoint{4.143488in}{1.451795in}}{\pgfqpoint{4.138444in}{1.446751in}}%
\pgfpathcurveto{\pgfqpoint{4.133400in}{1.441707in}}{\pgfqpoint{4.130567in}{1.434866in}}{\pgfqpoint{4.130567in}{1.427733in}}%
\pgfpathcurveto{\pgfqpoint{4.130567in}{1.420600in}}{\pgfqpoint{4.133400in}{1.413758in}}{\pgfqpoint{4.138444in}{1.408715in}}%
\pgfpathcurveto{\pgfqpoint{4.143488in}{1.403671in}}{\pgfqpoint{4.150329in}{1.400837in}}{\pgfqpoint{4.157462in}{1.400837in}}%
\pgfpathclose%
\pgfusepath{stroke,fill}%
\end{pgfscope}%
\begin{pgfscope}%
\pgfpathrectangle{\pgfqpoint{2.867647in}{0.500000in}}{\pgfqpoint{1.764706in}{1.700000in}}%
\pgfusepath{clip}%
\pgfsetbuttcap%
\pgfsetroundjoin%
\definecolor{currentfill}{rgb}{0.975018,0.868213,0.788710}%
\pgfsetfillcolor{currentfill}%
\pgfsetlinewidth{0.311001pt}%
\definecolor{currentstroke}{rgb}{1.000000,1.000000,1.000000}%
\pgfsetstrokecolor{currentstroke}%
\pgfsetdash{}{0pt}%
\pgfpathmoveto{\pgfqpoint{4.125742in}{1.406553in}}%
\pgfpathcurveto{\pgfqpoint{4.132875in}{1.406553in}}{\pgfqpoint{4.139716in}{1.409387in}}{\pgfqpoint{4.144760in}{1.414430in}}%
\pgfpathcurveto{\pgfqpoint{4.149804in}{1.419474in}}{\pgfqpoint{4.152638in}{1.426316in}}{\pgfqpoint{4.152638in}{1.433448in}}%
\pgfpathcurveto{\pgfqpoint{4.152638in}{1.440581in}}{\pgfqpoint{4.149804in}{1.447423in}}{\pgfqpoint{4.144760in}{1.452467in}}%
\pgfpathcurveto{\pgfqpoint{4.139716in}{1.457510in}}{\pgfqpoint{4.132875in}{1.460344in}}{\pgfqpoint{4.125742in}{1.460344in}}%
\pgfpathcurveto{\pgfqpoint{4.118609in}{1.460344in}}{\pgfqpoint{4.111767in}{1.457510in}}{\pgfqpoint{4.106724in}{1.452467in}}%
\pgfpathcurveto{\pgfqpoint{4.101680in}{1.447423in}}{\pgfqpoint{4.098846in}{1.440581in}}{\pgfqpoint{4.098846in}{1.433448in}}%
\pgfpathcurveto{\pgfqpoint{4.098846in}{1.426316in}}{\pgfqpoint{4.101680in}{1.419474in}}{\pgfqpoint{4.106724in}{1.414430in}}%
\pgfpathcurveto{\pgfqpoint{4.111767in}{1.409387in}}{\pgfqpoint{4.118609in}{1.406553in}}{\pgfqpoint{4.125742in}{1.406553in}}%
\pgfpathclose%
\pgfusepath{stroke,fill}%
\end{pgfscope}%
\begin{pgfscope}%
\pgfpathrectangle{\pgfqpoint{2.867647in}{0.500000in}}{\pgfqpoint{1.764706in}{1.700000in}}%
\pgfusepath{clip}%
\pgfsetbuttcap%
\pgfsetroundjoin%
\definecolor{currentfill}{rgb}{0.975018,0.868213,0.788710}%
\pgfsetfillcolor{currentfill}%
\pgfsetlinewidth{0.311001pt}%
\definecolor{currentstroke}{rgb}{1.000000,1.000000,1.000000}%
\pgfsetstrokecolor{currentstroke}%
\pgfsetdash{}{0pt}%
\pgfpathmoveto{\pgfqpoint{4.087737in}{1.065311in}}%
\pgfpathcurveto{\pgfqpoint{4.094869in}{1.065311in}}{\pgfqpoint{4.101711in}{1.068145in}}{\pgfqpoint{4.106755in}{1.073189in}}%
\pgfpathcurveto{\pgfqpoint{4.111798in}{1.078232in}}{\pgfqpoint{4.114632in}{1.085074in}}{\pgfqpoint{4.114632in}{1.092207in}}%
\pgfpathcurveto{\pgfqpoint{4.114632in}{1.099340in}}{\pgfqpoint{4.111798in}{1.106181in}}{\pgfqpoint{4.106755in}{1.111225in}}%
\pgfpathcurveto{\pgfqpoint{4.101711in}{1.116269in}}{\pgfqpoint{4.094869in}{1.119103in}}{\pgfqpoint{4.087737in}{1.119103in}}%
\pgfpathcurveto{\pgfqpoint{4.080604in}{1.119103in}}{\pgfqpoint{4.073762in}{1.116269in}}{\pgfqpoint{4.068718in}{1.111225in}}%
\pgfpathcurveto{\pgfqpoint{4.063675in}{1.106181in}}{\pgfqpoint{4.060841in}{1.099340in}}{\pgfqpoint{4.060841in}{1.092207in}}%
\pgfpathcurveto{\pgfqpoint{4.060841in}{1.085074in}}{\pgfqpoint{4.063675in}{1.078232in}}{\pgfqpoint{4.068718in}{1.073189in}}%
\pgfpathcurveto{\pgfqpoint{4.073762in}{1.068145in}}{\pgfqpoint{4.080604in}{1.065311in}}{\pgfqpoint{4.087737in}{1.065311in}}%
\pgfpathclose%
\pgfusepath{stroke,fill}%
\end{pgfscope}%
\begin{pgfscope}%
\pgfpathrectangle{\pgfqpoint{2.867647in}{0.500000in}}{\pgfqpoint{1.764706in}{1.700000in}}%
\pgfusepath{clip}%
\pgfsetbuttcap%
\pgfsetroundjoin%
\definecolor{currentfill}{rgb}{0.966328,0.750560,0.616961}%
\pgfsetfillcolor{currentfill}%
\pgfsetlinewidth{0.311001pt}%
\definecolor{currentstroke}{rgb}{1.000000,1.000000,1.000000}%
\pgfsetstrokecolor{currentstroke}%
\pgfsetdash{}{0pt}%
\pgfpathmoveto{\pgfqpoint{4.092985in}{1.739322in}}%
\pgfpathcurveto{\pgfqpoint{4.100118in}{1.739322in}}{\pgfqpoint{4.106960in}{1.742156in}}{\pgfqpoint{4.112004in}{1.747199in}}%
\pgfpathcurveto{\pgfqpoint{4.117047in}{1.752243in}}{\pgfqpoint{4.119881in}{1.759085in}}{\pgfqpoint{4.119881in}{1.766218in}}%
\pgfpathcurveto{\pgfqpoint{4.119881in}{1.773350in}}{\pgfqpoint{4.117047in}{1.780192in}}{\pgfqpoint{4.112004in}{1.785236in}}%
\pgfpathcurveto{\pgfqpoint{4.106960in}{1.790279in}}{\pgfqpoint{4.100118in}{1.793113in}}{\pgfqpoint{4.092985in}{1.793113in}}%
\pgfpathcurveto{\pgfqpoint{4.085853in}{1.793113in}}{\pgfqpoint{4.079011in}{1.790279in}}{\pgfqpoint{4.073967in}{1.785236in}}%
\pgfpathcurveto{\pgfqpoint{4.068924in}{1.780192in}}{\pgfqpoint{4.066090in}{1.773350in}}{\pgfqpoint{4.066090in}{1.766218in}}%
\pgfpathcurveto{\pgfqpoint{4.066090in}{1.759085in}}{\pgfqpoint{4.068924in}{1.752243in}}{\pgfqpoint{4.073967in}{1.747199in}}%
\pgfpathcurveto{\pgfqpoint{4.079011in}{1.742156in}}{\pgfqpoint{4.085853in}{1.739322in}}{\pgfqpoint{4.092985in}{1.739322in}}%
\pgfpathclose%
\pgfusepath{stroke,fill}%
\end{pgfscope}%
\begin{pgfscope}%
\pgfpathrectangle{\pgfqpoint{2.867647in}{0.500000in}}{\pgfqpoint{1.764706in}{1.700000in}}%
\pgfusepath{clip}%
\pgfsetbuttcap%
\pgfsetroundjoin%
\definecolor{currentfill}{rgb}{0.961734,0.579886,0.418445}%
\pgfsetfillcolor{currentfill}%
\pgfsetlinewidth{0.311001pt}%
\definecolor{currentstroke}{rgb}{1.000000,1.000000,1.000000}%
\pgfsetstrokecolor{currentstroke}%
\pgfsetdash{}{0pt}%
\pgfpathmoveto{\pgfqpoint{4.005571in}{1.138942in}}%
\pgfpathcurveto{\pgfqpoint{4.012704in}{1.138942in}}{\pgfqpoint{4.019546in}{1.141776in}}{\pgfqpoint{4.024589in}{1.146820in}}%
\pgfpathcurveto{\pgfqpoint{4.029633in}{1.151863in}}{\pgfqpoint{4.032467in}{1.158705in}}{\pgfqpoint{4.032467in}{1.165838in}}%
\pgfpathcurveto{\pgfqpoint{4.032467in}{1.172971in}}{\pgfqpoint{4.029633in}{1.179812in}}{\pgfqpoint{4.024589in}{1.184856in}}%
\pgfpathcurveto{\pgfqpoint{4.019546in}{1.189900in}}{\pgfqpoint{4.012704in}{1.192734in}}{\pgfqpoint{4.005571in}{1.192734in}}%
\pgfpathcurveto{\pgfqpoint{3.998438in}{1.192734in}}{\pgfqpoint{3.991597in}{1.189900in}}{\pgfqpoint{3.986553in}{1.184856in}}%
\pgfpathcurveto{\pgfqpoint{3.981509in}{1.179812in}}{\pgfqpoint{3.978675in}{1.172971in}}{\pgfqpoint{3.978675in}{1.165838in}}%
\pgfpathcurveto{\pgfqpoint{3.978675in}{1.158705in}}{\pgfqpoint{3.981509in}{1.151863in}}{\pgfqpoint{3.986553in}{1.146820in}}%
\pgfpathcurveto{\pgfqpoint{3.991597in}{1.141776in}}{\pgfqpoint{3.998438in}{1.138942in}}{\pgfqpoint{4.005571in}{1.138942in}}%
\pgfpathclose%
\pgfusepath{stroke,fill}%
\end{pgfscope}%
\begin{pgfscope}%
\pgfpathrectangle{\pgfqpoint{2.867647in}{0.500000in}}{\pgfqpoint{1.764706in}{1.700000in}}%
\pgfusepath{clip}%
\pgfsetbuttcap%
\pgfsetroundjoin%
\definecolor{currentfill}{rgb}{0.971694,0.833208,0.737161}%
\pgfsetfillcolor{currentfill}%
\pgfsetlinewidth{0.311001pt}%
\definecolor{currentstroke}{rgb}{1.000000,1.000000,1.000000}%
\pgfsetstrokecolor{currentstroke}%
\pgfsetdash{}{0pt}%
\pgfpathmoveto{\pgfqpoint{4.064808in}{1.064654in}}%
\pgfpathcurveto{\pgfqpoint{4.071940in}{1.064654in}}{\pgfqpoint{4.078782in}{1.067488in}}{\pgfqpoint{4.083826in}{1.072532in}}%
\pgfpathcurveto{\pgfqpoint{4.088869in}{1.077576in}}{\pgfqpoint{4.091703in}{1.084417in}}{\pgfqpoint{4.091703in}{1.091550in}}%
\pgfpathcurveto{\pgfqpoint{4.091703in}{1.098683in}}{\pgfqpoint{4.088869in}{1.105525in}}{\pgfqpoint{4.083826in}{1.110568in}}%
\pgfpathcurveto{\pgfqpoint{4.078782in}{1.115612in}}{\pgfqpoint{4.071940in}{1.118446in}}{\pgfqpoint{4.064808in}{1.118446in}}%
\pgfpathcurveto{\pgfqpoint{4.057675in}{1.118446in}}{\pgfqpoint{4.050833in}{1.115612in}}{\pgfqpoint{4.045789in}{1.110568in}}%
\pgfpathcurveto{\pgfqpoint{4.040746in}{1.105525in}}{\pgfqpoint{4.037912in}{1.098683in}}{\pgfqpoint{4.037912in}{1.091550in}}%
\pgfpathcurveto{\pgfqpoint{4.037912in}{1.084417in}}{\pgfqpoint{4.040746in}{1.077576in}}{\pgfqpoint{4.045789in}{1.072532in}}%
\pgfpathcurveto{\pgfqpoint{4.050833in}{1.067488in}}{\pgfqpoint{4.057675in}{1.064654in}}{\pgfqpoint{4.064808in}{1.064654in}}%
\pgfpathclose%
\pgfusepath{stroke,fill}%
\end{pgfscope}%
\begin{pgfscope}%
\pgfpathrectangle{\pgfqpoint{2.867647in}{0.500000in}}{\pgfqpoint{1.764706in}{1.700000in}}%
\pgfusepath{clip}%
\pgfsetbuttcap%
\pgfsetroundjoin%
\definecolor{currentfill}{rgb}{0.973832,0.856556,0.771584}%
\pgfsetfillcolor{currentfill}%
\pgfsetlinewidth{0.311001pt}%
\definecolor{currentstroke}{rgb}{1.000000,1.000000,1.000000}%
\pgfsetstrokecolor{currentstroke}%
\pgfsetdash{}{0pt}%
\pgfpathmoveto{\pgfqpoint{4.156093in}{1.629116in}}%
\pgfpathcurveto{\pgfqpoint{4.163226in}{1.629116in}}{\pgfqpoint{4.170068in}{1.631949in}}{\pgfqpoint{4.175111in}{1.636993in}}%
\pgfpathcurveto{\pgfqpoint{4.180155in}{1.642037in}}{\pgfqpoint{4.182989in}{1.648878in}}{\pgfqpoint{4.182989in}{1.656011in}}%
\pgfpathcurveto{\pgfqpoint{4.182989in}{1.663144in}}{\pgfqpoint{4.180155in}{1.669986in}}{\pgfqpoint{4.175111in}{1.675029in}}%
\pgfpathcurveto{\pgfqpoint{4.170068in}{1.680073in}}{\pgfqpoint{4.163226in}{1.682907in}}{\pgfqpoint{4.156093in}{1.682907in}}%
\pgfpathcurveto{\pgfqpoint{4.148960in}{1.682907in}}{\pgfqpoint{4.142119in}{1.680073in}}{\pgfqpoint{4.137075in}{1.675029in}}%
\pgfpathcurveto{\pgfqpoint{4.132031in}{1.669986in}}{\pgfqpoint{4.129198in}{1.663144in}}{\pgfqpoint{4.129198in}{1.656011in}}%
\pgfpathcurveto{\pgfqpoint{4.129198in}{1.648878in}}{\pgfqpoint{4.132031in}{1.642037in}}{\pgfqpoint{4.137075in}{1.636993in}}%
\pgfpathcurveto{\pgfqpoint{4.142119in}{1.631949in}}{\pgfqpoint{4.148960in}{1.629116in}}{\pgfqpoint{4.156093in}{1.629116in}}%
\pgfpathclose%
\pgfusepath{stroke,fill}%
\end{pgfscope}%
\begin{pgfscope}%
\pgfpathrectangle{\pgfqpoint{2.867647in}{0.500000in}}{\pgfqpoint{1.764706in}{1.700000in}}%
\pgfusepath{clip}%
\pgfsetbuttcap%
\pgfsetroundjoin%
\definecolor{currentfill}{rgb}{0.971694,0.833208,0.737161}%
\pgfsetfillcolor{currentfill}%
\pgfsetlinewidth{0.311001pt}%
\definecolor{currentstroke}{rgb}{1.000000,1.000000,1.000000}%
\pgfsetstrokecolor{currentstroke}%
\pgfsetdash{}{0pt}%
\pgfpathmoveto{\pgfqpoint{4.081393in}{1.125922in}}%
\pgfpathcurveto{\pgfqpoint{4.088526in}{1.125922in}}{\pgfqpoint{4.095367in}{1.128756in}}{\pgfqpoint{4.100411in}{1.133800in}}%
\pgfpathcurveto{\pgfqpoint{4.105455in}{1.138844in}}{\pgfqpoint{4.108288in}{1.145685in}}{\pgfqpoint{4.108288in}{1.152818in}}%
\pgfpathcurveto{\pgfqpoint{4.108288in}{1.159951in}}{\pgfqpoint{4.105455in}{1.166792in}}{\pgfqpoint{4.100411in}{1.171836in}}%
\pgfpathcurveto{\pgfqpoint{4.095367in}{1.176880in}}{\pgfqpoint{4.088526in}{1.179714in}}{\pgfqpoint{4.081393in}{1.179714in}}%
\pgfpathcurveto{\pgfqpoint{4.074260in}{1.179714in}}{\pgfqpoint{4.067418in}{1.176880in}}{\pgfqpoint{4.062375in}{1.171836in}}%
\pgfpathcurveto{\pgfqpoint{4.057331in}{1.166792in}}{\pgfqpoint{4.054497in}{1.159951in}}{\pgfqpoint{4.054497in}{1.152818in}}%
\pgfpathcurveto{\pgfqpoint{4.054497in}{1.145685in}}{\pgfqpoint{4.057331in}{1.138844in}}{\pgfqpoint{4.062375in}{1.133800in}}%
\pgfpathcurveto{\pgfqpoint{4.067418in}{1.128756in}}{\pgfqpoint{4.074260in}{1.125922in}}{\pgfqpoint{4.081393in}{1.125922in}}%
\pgfpathclose%
\pgfusepath{stroke,fill}%
\end{pgfscope}%
\begin{pgfscope}%
\pgfpathrectangle{\pgfqpoint{2.867647in}{0.500000in}}{\pgfqpoint{1.764706in}{1.700000in}}%
\pgfusepath{clip}%
\pgfsetbuttcap%
\pgfsetroundjoin%
\definecolor{currentfill}{rgb}{0.972201,0.839051,0.745789}%
\pgfsetfillcolor{currentfill}%
\pgfsetlinewidth{0.311001pt}%
\definecolor{currentstroke}{rgb}{1.000000,1.000000,1.000000}%
\pgfsetstrokecolor{currentstroke}%
\pgfsetdash{}{0pt}%
\pgfpathmoveto{\pgfqpoint{4.065324in}{0.998847in}}%
\pgfpathcurveto{\pgfqpoint{4.072457in}{0.998847in}}{\pgfqpoint{4.079298in}{1.001681in}}{\pgfqpoint{4.084342in}{1.006724in}}%
\pgfpathcurveto{\pgfqpoint{4.089386in}{1.011768in}}{\pgfqpoint{4.092219in}{1.018610in}}{\pgfqpoint{4.092219in}{1.025743in}}%
\pgfpathcurveto{\pgfqpoint{4.092219in}{1.032875in}}{\pgfqpoint{4.089386in}{1.039717in}}{\pgfqpoint{4.084342in}{1.044761in}}%
\pgfpathcurveto{\pgfqpoint{4.079298in}{1.049804in}}{\pgfqpoint{4.072457in}{1.052638in}}{\pgfqpoint{4.065324in}{1.052638in}}%
\pgfpathcurveto{\pgfqpoint{4.058191in}{1.052638in}}{\pgfqpoint{4.051349in}{1.049804in}}{\pgfqpoint{4.046306in}{1.044761in}}%
\pgfpathcurveto{\pgfqpoint{4.041262in}{1.039717in}}{\pgfqpoint{4.038428in}{1.032875in}}{\pgfqpoint{4.038428in}{1.025743in}}%
\pgfpathcurveto{\pgfqpoint{4.038428in}{1.018610in}}{\pgfqpoint{4.041262in}{1.011768in}}{\pgfqpoint{4.046306in}{1.006724in}}%
\pgfpathcurveto{\pgfqpoint{4.051349in}{1.001681in}}{\pgfqpoint{4.058191in}{0.998847in}}{\pgfqpoint{4.065324in}{0.998847in}}%
\pgfpathclose%
\pgfusepath{stroke,fill}%
\end{pgfscope}%
\begin{pgfscope}%
\pgfpathrectangle{\pgfqpoint{2.867647in}{0.500000in}}{\pgfqpoint{1.764706in}{1.700000in}}%
\pgfusepath{clip}%
\pgfsetbuttcap%
\pgfsetroundjoin%
\definecolor{currentfill}{rgb}{0.978376,0.897317,0.831308}%
\pgfsetfillcolor{currentfill}%
\pgfsetlinewidth{0.311001pt}%
\definecolor{currentstroke}{rgb}{1.000000,1.000000,1.000000}%
\pgfsetstrokecolor{currentstroke}%
\pgfsetdash{}{0pt}%
\pgfpathmoveto{\pgfqpoint{4.209900in}{1.450339in}}%
\pgfpathcurveto{\pgfqpoint{4.217033in}{1.450339in}}{\pgfqpoint{4.223874in}{1.453173in}}{\pgfqpoint{4.228918in}{1.458217in}}%
\pgfpathcurveto{\pgfqpoint{4.233962in}{1.463260in}}{\pgfqpoint{4.236795in}{1.470102in}}{\pgfqpoint{4.236795in}{1.477235in}}%
\pgfpathcurveto{\pgfqpoint{4.236795in}{1.484368in}}{\pgfqpoint{4.233962in}{1.491209in}}{\pgfqpoint{4.228918in}{1.496253in}}%
\pgfpathcurveto{\pgfqpoint{4.223874in}{1.501297in}}{\pgfqpoint{4.217033in}{1.504130in}}{\pgfqpoint{4.209900in}{1.504130in}}%
\pgfpathcurveto{\pgfqpoint{4.202767in}{1.504130in}}{\pgfqpoint{4.195925in}{1.501297in}}{\pgfqpoint{4.190882in}{1.496253in}}%
\pgfpathcurveto{\pgfqpoint{4.185838in}{1.491209in}}{\pgfqpoint{4.183004in}{1.484368in}}{\pgfqpoint{4.183004in}{1.477235in}}%
\pgfpathcurveto{\pgfqpoint{4.183004in}{1.470102in}}{\pgfqpoint{4.185838in}{1.463260in}}{\pgfqpoint{4.190882in}{1.458217in}}%
\pgfpathcurveto{\pgfqpoint{4.195925in}{1.453173in}}{\pgfqpoint{4.202767in}{1.450339in}}{\pgfqpoint{4.209900in}{1.450339in}}%
\pgfpathclose%
\pgfusepath{stroke,fill}%
\end{pgfscope}%
\begin{pgfscope}%
\pgfpathrectangle{\pgfqpoint{2.867647in}{0.500000in}}{\pgfqpoint{1.764706in}{1.700000in}}%
\pgfusepath{clip}%
\pgfsetbuttcap%
\pgfsetroundjoin%
\definecolor{currentfill}{rgb}{0.965928,0.738443,0.600540}%
\pgfsetfillcolor{currentfill}%
\pgfsetlinewidth{0.311001pt}%
\definecolor{currentstroke}{rgb}{1.000000,1.000000,1.000000}%
\pgfsetstrokecolor{currentstroke}%
\pgfsetdash{}{0pt}%
\pgfpathmoveto{\pgfqpoint{4.155241in}{0.941060in}}%
\pgfpathcurveto{\pgfqpoint{4.162374in}{0.941060in}}{\pgfqpoint{4.169216in}{0.943893in}}{\pgfqpoint{4.174259in}{0.948937in}}%
\pgfpathcurveto{\pgfqpoint{4.179303in}{0.953981in}}{\pgfqpoint{4.182137in}{0.960822in}}{\pgfqpoint{4.182137in}{0.967955in}}%
\pgfpathcurveto{\pgfqpoint{4.182137in}{0.975088in}}{\pgfqpoint{4.179303in}{0.981930in}}{\pgfqpoint{4.174259in}{0.986973in}}%
\pgfpathcurveto{\pgfqpoint{4.169216in}{0.992017in}}{\pgfqpoint{4.162374in}{0.994851in}}{\pgfqpoint{4.155241in}{0.994851in}}%
\pgfpathcurveto{\pgfqpoint{4.148108in}{0.994851in}}{\pgfqpoint{4.141267in}{0.992017in}}{\pgfqpoint{4.136223in}{0.986973in}}%
\pgfpathcurveto{\pgfqpoint{4.131179in}{0.981930in}}{\pgfqpoint{4.128345in}{0.975088in}}{\pgfqpoint{4.128345in}{0.967955in}}%
\pgfpathcurveto{\pgfqpoint{4.128345in}{0.960822in}}{\pgfqpoint{4.131179in}{0.953981in}}{\pgfqpoint{4.136223in}{0.948937in}}%
\pgfpathcurveto{\pgfqpoint{4.141267in}{0.943893in}}{\pgfqpoint{4.148108in}{0.941060in}}{\pgfqpoint{4.155241in}{0.941060in}}%
\pgfpathclose%
\pgfusepath{stroke,fill}%
\end{pgfscope}%
\begin{pgfscope}%
\pgfpathrectangle{\pgfqpoint{2.867647in}{0.500000in}}{\pgfqpoint{1.764706in}{1.700000in}}%
\pgfusepath{clip}%
\pgfsetbuttcap%
\pgfsetroundjoin%
\definecolor{currentfill}{rgb}{0.962283,0.593046,0.431453}%
\pgfsetfillcolor{currentfill}%
\pgfsetlinewidth{0.311001pt}%
\definecolor{currentstroke}{rgb}{1.000000,1.000000,1.000000}%
\pgfsetstrokecolor{currentstroke}%
\pgfsetdash{}{0pt}%
\pgfpathmoveto{\pgfqpoint{4.083399in}{0.853882in}}%
\pgfpathcurveto{\pgfqpoint{4.090532in}{0.853882in}}{\pgfqpoint{4.097373in}{0.856716in}}{\pgfqpoint{4.102417in}{0.861759in}}%
\pgfpathcurveto{\pgfqpoint{4.107461in}{0.866803in}}{\pgfqpoint{4.110294in}{0.873645in}}{\pgfqpoint{4.110294in}{0.880778in}}%
\pgfpathcurveto{\pgfqpoint{4.110294in}{0.887910in}}{\pgfqpoint{4.107461in}{0.894752in}}{\pgfqpoint{4.102417in}{0.899796in}}%
\pgfpathcurveto{\pgfqpoint{4.097373in}{0.904839in}}{\pgfqpoint{4.090532in}{0.907673in}}{\pgfqpoint{4.083399in}{0.907673in}}%
\pgfpathcurveto{\pgfqpoint{4.076266in}{0.907673in}}{\pgfqpoint{4.069424in}{0.904839in}}{\pgfqpoint{4.064381in}{0.899796in}}%
\pgfpathcurveto{\pgfqpoint{4.059337in}{0.894752in}}{\pgfqpoint{4.056503in}{0.887910in}}{\pgfqpoint{4.056503in}{0.880778in}}%
\pgfpathcurveto{\pgfqpoint{4.056503in}{0.873645in}}{\pgfqpoint{4.059337in}{0.866803in}}{\pgfqpoint{4.064381in}{0.861759in}}%
\pgfpathcurveto{\pgfqpoint{4.069424in}{0.856716in}}{\pgfqpoint{4.076266in}{0.853882in}}{\pgfqpoint{4.083399in}{0.853882in}}%
\pgfpathclose%
\pgfusepath{stroke,fill}%
\end{pgfscope}%
\begin{pgfscope}%
\pgfpathrectangle{\pgfqpoint{2.867647in}{0.500000in}}{\pgfqpoint{1.764706in}{1.700000in}}%
\pgfusepath{clip}%
\pgfsetbuttcap%
\pgfsetroundjoin%
\definecolor{currentfill}{rgb}{0.972726,0.844889,0.754401}%
\pgfsetfillcolor{currentfill}%
\pgfsetlinewidth{0.311001pt}%
\definecolor{currentstroke}{rgb}{1.000000,1.000000,1.000000}%
\pgfsetstrokecolor{currentstroke}%
\pgfsetdash{}{0pt}%
\pgfpathmoveto{\pgfqpoint{4.120657in}{1.285803in}}%
\pgfpathcurveto{\pgfqpoint{4.127790in}{1.285803in}}{\pgfqpoint{4.134631in}{1.288637in}}{\pgfqpoint{4.139675in}{1.293681in}}%
\pgfpathcurveto{\pgfqpoint{4.144719in}{1.298725in}}{\pgfqpoint{4.147552in}{1.305566in}}{\pgfqpoint{4.147552in}{1.312699in}}%
\pgfpathcurveto{\pgfqpoint{4.147552in}{1.319832in}}{\pgfqpoint{4.144719in}{1.326674in}}{\pgfqpoint{4.139675in}{1.331717in}}%
\pgfpathcurveto{\pgfqpoint{4.134631in}{1.336761in}}{\pgfqpoint{4.127790in}{1.339595in}}{\pgfqpoint{4.120657in}{1.339595in}}%
\pgfpathcurveto{\pgfqpoint{4.113524in}{1.339595in}}{\pgfqpoint{4.106682in}{1.336761in}}{\pgfqpoint{4.101639in}{1.331717in}}%
\pgfpathcurveto{\pgfqpoint{4.096595in}{1.326674in}}{\pgfqpoint{4.093761in}{1.319832in}}{\pgfqpoint{4.093761in}{1.312699in}}%
\pgfpathcurveto{\pgfqpoint{4.093761in}{1.305566in}}{\pgfqpoint{4.096595in}{1.298725in}}{\pgfqpoint{4.101639in}{1.293681in}}%
\pgfpathcurveto{\pgfqpoint{4.106682in}{1.288637in}}{\pgfqpoint{4.113524in}{1.285803in}}{\pgfqpoint{4.120657in}{1.285803in}}%
\pgfpathclose%
\pgfusepath{stroke,fill}%
\end{pgfscope}%
\begin{pgfscope}%
\pgfpathrectangle{\pgfqpoint{2.867647in}{0.500000in}}{\pgfqpoint{1.764706in}{1.700000in}}%
\pgfusepath{clip}%
\pgfsetbuttcap%
\pgfsetroundjoin%
\definecolor{currentfill}{rgb}{0.969803,0.809811,0.702523}%
\pgfsetfillcolor{currentfill}%
\pgfsetlinewidth{0.311001pt}%
\definecolor{currentstroke}{rgb}{1.000000,1.000000,1.000000}%
\pgfsetstrokecolor{currentstroke}%
\pgfsetdash{}{0pt}%
\pgfpathmoveto{\pgfqpoint{4.065737in}{1.109683in}}%
\pgfpathcurveto{\pgfqpoint{4.072870in}{1.109683in}}{\pgfqpoint{4.079711in}{1.112517in}}{\pgfqpoint{4.084755in}{1.117561in}}%
\pgfpathcurveto{\pgfqpoint{4.089799in}{1.122605in}}{\pgfqpoint{4.092633in}{1.129446in}}{\pgfqpoint{4.092633in}{1.136579in}}%
\pgfpathcurveto{\pgfqpoint{4.092633in}{1.143712in}}{\pgfqpoint{4.089799in}{1.150553in}}{\pgfqpoint{4.084755in}{1.155597in}}%
\pgfpathcurveto{\pgfqpoint{4.079711in}{1.160641in}}{\pgfqpoint{4.072870in}{1.163475in}}{\pgfqpoint{4.065737in}{1.163475in}}%
\pgfpathcurveto{\pgfqpoint{4.058604in}{1.163475in}}{\pgfqpoint{4.051762in}{1.160641in}}{\pgfqpoint{4.046719in}{1.155597in}}%
\pgfpathcurveto{\pgfqpoint{4.041675in}{1.150553in}}{\pgfqpoint{4.038841in}{1.143712in}}{\pgfqpoint{4.038841in}{1.136579in}}%
\pgfpathcurveto{\pgfqpoint{4.038841in}{1.129446in}}{\pgfqpoint{4.041675in}{1.122605in}}{\pgfqpoint{4.046719in}{1.117561in}}%
\pgfpathcurveto{\pgfqpoint{4.051762in}{1.112517in}}{\pgfqpoint{4.058604in}{1.109683in}}{\pgfqpoint{4.065737in}{1.109683in}}%
\pgfpathclose%
\pgfusepath{stroke,fill}%
\end{pgfscope}%
\begin{pgfscope}%
\pgfpathrectangle{\pgfqpoint{2.867647in}{0.500000in}}{\pgfqpoint{1.764706in}{1.700000in}}%
\pgfusepath{clip}%
\pgfsetbuttcap%
\pgfsetroundjoin%
\definecolor{currentfill}{rgb}{0.966812,0.762584,0.633643}%
\pgfsetfillcolor{currentfill}%
\pgfsetlinewidth{0.311001pt}%
\definecolor{currentstroke}{rgb}{1.000000,1.000000,1.000000}%
\pgfsetstrokecolor{currentstroke}%
\pgfsetdash{}{0pt}%
\pgfpathmoveto{\pgfqpoint{4.249941in}{1.554532in}}%
\pgfpathcurveto{\pgfqpoint{4.257073in}{1.554532in}}{\pgfqpoint{4.263915in}{1.557366in}}{\pgfqpoint{4.268959in}{1.562409in}}%
\pgfpathcurveto{\pgfqpoint{4.274002in}{1.567453in}}{\pgfqpoint{4.276836in}{1.574295in}}{\pgfqpoint{4.276836in}{1.581427in}}%
\pgfpathcurveto{\pgfqpoint{4.276836in}{1.588560in}}{\pgfqpoint{4.274002in}{1.595402in}}{\pgfqpoint{4.268959in}{1.600446in}}%
\pgfpathcurveto{\pgfqpoint{4.263915in}{1.605489in}}{\pgfqpoint{4.257073in}{1.608323in}}{\pgfqpoint{4.249941in}{1.608323in}}%
\pgfpathcurveto{\pgfqpoint{4.242808in}{1.608323in}}{\pgfqpoint{4.235966in}{1.605489in}}{\pgfqpoint{4.230922in}{1.600446in}}%
\pgfpathcurveto{\pgfqpoint{4.225879in}{1.595402in}}{\pgfqpoint{4.223045in}{1.588560in}}{\pgfqpoint{4.223045in}{1.581427in}}%
\pgfpathcurveto{\pgfqpoint{4.223045in}{1.574295in}}{\pgfqpoint{4.225879in}{1.567453in}}{\pgfqpoint{4.230922in}{1.562409in}}%
\pgfpathcurveto{\pgfqpoint{4.235966in}{1.557366in}}{\pgfqpoint{4.242808in}{1.554532in}}{\pgfqpoint{4.249941in}{1.554532in}}%
\pgfpathclose%
\pgfusepath{stroke,fill}%
\end{pgfscope}%
\begin{pgfscope}%
\pgfpathrectangle{\pgfqpoint{2.867647in}{0.500000in}}{\pgfqpoint{1.764706in}{1.700000in}}%
\pgfusepath{clip}%
\pgfsetbuttcap%
\pgfsetroundjoin%
\definecolor{currentfill}{rgb}{0.964920,0.695342,0.545192}%
\pgfsetfillcolor{currentfill}%
\pgfsetlinewidth{0.311001pt}%
\definecolor{currentstroke}{rgb}{1.000000,1.000000,1.000000}%
\pgfsetstrokecolor{currentstroke}%
\pgfsetdash{}{0pt}%
\pgfpathmoveto{\pgfqpoint{4.072000in}{1.293453in}}%
\pgfpathcurveto{\pgfqpoint{4.079133in}{1.293453in}}{\pgfqpoint{4.085975in}{1.296287in}}{\pgfqpoint{4.091018in}{1.301331in}}%
\pgfpathcurveto{\pgfqpoint{4.096062in}{1.306374in}}{\pgfqpoint{4.098896in}{1.313216in}}{\pgfqpoint{4.098896in}{1.320349in}}%
\pgfpathcurveto{\pgfqpoint{4.098896in}{1.327482in}}{\pgfqpoint{4.096062in}{1.334323in}}{\pgfqpoint{4.091018in}{1.339367in}}%
\pgfpathcurveto{\pgfqpoint{4.085975in}{1.344410in}}{\pgfqpoint{4.079133in}{1.347244in}}{\pgfqpoint{4.072000in}{1.347244in}}%
\pgfpathcurveto{\pgfqpoint{4.064867in}{1.347244in}}{\pgfqpoint{4.058026in}{1.344410in}}{\pgfqpoint{4.052982in}{1.339367in}}%
\pgfpathcurveto{\pgfqpoint{4.047938in}{1.334323in}}{\pgfqpoint{4.045104in}{1.327482in}}{\pgfqpoint{4.045104in}{1.320349in}}%
\pgfpathcurveto{\pgfqpoint{4.045104in}{1.313216in}}{\pgfqpoint{4.047938in}{1.306374in}}{\pgfqpoint{4.052982in}{1.301331in}}%
\pgfpathcurveto{\pgfqpoint{4.058026in}{1.296287in}}{\pgfqpoint{4.064867in}{1.293453in}}{\pgfqpoint{4.072000in}{1.293453in}}%
\pgfpathclose%
\pgfusepath{stroke,fill}%
\end{pgfscope}%
\begin{pgfscope}%
\pgfpathrectangle{\pgfqpoint{2.867647in}{0.500000in}}{\pgfqpoint{1.764706in}{1.700000in}}%
\pgfusepath{clip}%
\pgfsetbuttcap%
\pgfsetroundjoin%
\definecolor{currentfill}{rgb}{0.963884,0.644842,0.486120}%
\pgfsetfillcolor{currentfill}%
\pgfsetlinewidth{0.311001pt}%
\definecolor{currentstroke}{rgb}{1.000000,1.000000,1.000000}%
\pgfsetstrokecolor{currentstroke}%
\pgfsetdash{}{0pt}%
\pgfpathmoveto{\pgfqpoint{4.059288in}{1.342379in}}%
\pgfpathcurveto{\pgfqpoint{4.066421in}{1.342379in}}{\pgfqpoint{4.073263in}{1.345213in}}{\pgfqpoint{4.078306in}{1.350256in}}%
\pgfpathcurveto{\pgfqpoint{4.083350in}{1.355300in}}{\pgfqpoint{4.086184in}{1.362142in}}{\pgfqpoint{4.086184in}{1.369274in}}%
\pgfpathcurveto{\pgfqpoint{4.086184in}{1.376407in}}{\pgfqpoint{4.083350in}{1.383249in}}{\pgfqpoint{4.078306in}{1.388293in}}%
\pgfpathcurveto{\pgfqpoint{4.073263in}{1.393336in}}{\pgfqpoint{4.066421in}{1.396170in}}{\pgfqpoint{4.059288in}{1.396170in}}%
\pgfpathcurveto{\pgfqpoint{4.052155in}{1.396170in}}{\pgfqpoint{4.045314in}{1.393336in}}{\pgfqpoint{4.040270in}{1.388293in}}%
\pgfpathcurveto{\pgfqpoint{4.035226in}{1.383249in}}{\pgfqpoint{4.032392in}{1.376407in}}{\pgfqpoint{4.032392in}{1.369274in}}%
\pgfpathcurveto{\pgfqpoint{4.032392in}{1.362142in}}{\pgfqpoint{4.035226in}{1.355300in}}{\pgfqpoint{4.040270in}{1.350256in}}%
\pgfpathcurveto{\pgfqpoint{4.045314in}{1.345213in}}{\pgfqpoint{4.052155in}{1.342379in}}{\pgfqpoint{4.059288in}{1.342379in}}%
\pgfpathclose%
\pgfusepath{stroke,fill}%
\end{pgfscope}%
\begin{pgfscope}%
\pgfpathrectangle{\pgfqpoint{2.867647in}{0.500000in}}{\pgfqpoint{1.764706in}{1.700000in}}%
\pgfusepath{clip}%
\pgfsetbuttcap%
\pgfsetroundjoin%
\definecolor{currentfill}{rgb}{0.964920,0.695342,0.545192}%
\pgfsetfillcolor{currentfill}%
\pgfsetlinewidth{0.311001pt}%
\definecolor{currentstroke}{rgb}{1.000000,1.000000,1.000000}%
\pgfsetstrokecolor{currentstroke}%
\pgfsetdash{}{0pt}%
\pgfpathmoveto{\pgfqpoint{4.061309in}{1.769049in}}%
\pgfpathcurveto{\pgfqpoint{4.068442in}{1.769049in}}{\pgfqpoint{4.075284in}{1.771883in}}{\pgfqpoint{4.080327in}{1.776927in}}%
\pgfpathcurveto{\pgfqpoint{4.085371in}{1.781970in}}{\pgfqpoint{4.088205in}{1.788812in}}{\pgfqpoint{4.088205in}{1.795945in}}%
\pgfpathcurveto{\pgfqpoint{4.088205in}{1.803078in}}{\pgfqpoint{4.085371in}{1.809919in}}{\pgfqpoint{4.080327in}{1.814963in}}%
\pgfpathcurveto{\pgfqpoint{4.075284in}{1.820007in}}{\pgfqpoint{4.068442in}{1.822841in}}{\pgfqpoint{4.061309in}{1.822841in}}%
\pgfpathcurveto{\pgfqpoint{4.054176in}{1.822841in}}{\pgfqpoint{4.047335in}{1.820007in}}{\pgfqpoint{4.042291in}{1.814963in}}%
\pgfpathcurveto{\pgfqpoint{4.037247in}{1.809919in}}{\pgfqpoint{4.034414in}{1.803078in}}{\pgfqpoint{4.034414in}{1.795945in}}%
\pgfpathcurveto{\pgfqpoint{4.034414in}{1.788812in}}{\pgfqpoint{4.037247in}{1.781970in}}{\pgfqpoint{4.042291in}{1.776927in}}%
\pgfpathcurveto{\pgfqpoint{4.047335in}{1.771883in}}{\pgfqpoint{4.054176in}{1.769049in}}{\pgfqpoint{4.061309in}{1.769049in}}%
\pgfpathclose%
\pgfusepath{stroke,fill}%
\end{pgfscope}%
\begin{pgfscope}%
\pgfpathrectangle{\pgfqpoint{2.867647in}{0.500000in}}{\pgfqpoint{1.764706in}{1.700000in}}%
\pgfusepath{clip}%
\pgfsetbuttcap%
\pgfsetroundjoin%
\definecolor{currentfill}{rgb}{0.942910,0.375495,0.263698}%
\pgfsetfillcolor{currentfill}%
\pgfsetlinewidth{0.311001pt}%
\definecolor{currentstroke}{rgb}{1.000000,1.000000,1.000000}%
\pgfsetstrokecolor{currentstroke}%
\pgfsetdash{}{0pt}%
\pgfpathmoveto{\pgfqpoint{4.299491in}{1.642476in}}%
\pgfpathcurveto{\pgfqpoint{4.306624in}{1.642476in}}{\pgfqpoint{4.313466in}{1.645310in}}{\pgfqpoint{4.318510in}{1.650354in}}%
\pgfpathcurveto{\pgfqpoint{4.323553in}{1.655398in}}{\pgfqpoint{4.326387in}{1.662239in}}{\pgfqpoint{4.326387in}{1.669372in}}%
\pgfpathcurveto{\pgfqpoint{4.326387in}{1.676505in}}{\pgfqpoint{4.323553in}{1.683346in}}{\pgfqpoint{4.318510in}{1.688390in}}%
\pgfpathcurveto{\pgfqpoint{4.313466in}{1.693434in}}{\pgfqpoint{4.306624in}{1.696268in}}{\pgfqpoint{4.299491in}{1.696268in}}%
\pgfpathcurveto{\pgfqpoint{4.292359in}{1.696268in}}{\pgfqpoint{4.285517in}{1.693434in}}{\pgfqpoint{4.280473in}{1.688390in}}%
\pgfpathcurveto{\pgfqpoint{4.275430in}{1.683346in}}{\pgfqpoint{4.272596in}{1.676505in}}{\pgfqpoint{4.272596in}{1.669372in}}%
\pgfpathcurveto{\pgfqpoint{4.272596in}{1.662239in}}{\pgfqpoint{4.275430in}{1.655398in}}{\pgfqpoint{4.280473in}{1.650354in}}%
\pgfpathcurveto{\pgfqpoint{4.285517in}{1.645310in}}{\pgfqpoint{4.292359in}{1.642476in}}{\pgfqpoint{4.299491in}{1.642476in}}%
\pgfpathclose%
\pgfusepath{stroke,fill}%
\end{pgfscope}%
\begin{pgfscope}%
\pgfpathrectangle{\pgfqpoint{2.867647in}{0.500000in}}{\pgfqpoint{1.764706in}{1.700000in}}%
\pgfusepath{clip}%
\pgfsetbuttcap%
\pgfsetroundjoin%
\definecolor{currentfill}{rgb}{0.971694,0.833208,0.737161}%
\pgfsetfillcolor{currentfill}%
\pgfsetlinewidth{0.311001pt}%
\definecolor{currentstroke}{rgb}{1.000000,1.000000,1.000000}%
\pgfsetstrokecolor{currentstroke}%
\pgfsetdash{}{0pt}%
\pgfpathmoveto{\pgfqpoint{4.072856in}{1.097272in}}%
\pgfpathcurveto{\pgfqpoint{4.079989in}{1.097272in}}{\pgfqpoint{4.086831in}{1.100106in}}{\pgfqpoint{4.091874in}{1.105150in}}%
\pgfpathcurveto{\pgfqpoint{4.096918in}{1.110194in}}{\pgfqpoint{4.099752in}{1.117035in}}{\pgfqpoint{4.099752in}{1.124168in}}%
\pgfpathcurveto{\pgfqpoint{4.099752in}{1.131301in}}{\pgfqpoint{4.096918in}{1.138143in}}{\pgfqpoint{4.091874in}{1.143186in}}%
\pgfpathcurveto{\pgfqpoint{4.086831in}{1.148230in}}{\pgfqpoint{4.079989in}{1.151064in}}{\pgfqpoint{4.072856in}{1.151064in}}%
\pgfpathcurveto{\pgfqpoint{4.065723in}{1.151064in}}{\pgfqpoint{4.058882in}{1.148230in}}{\pgfqpoint{4.053838in}{1.143186in}}%
\pgfpathcurveto{\pgfqpoint{4.048794in}{1.138143in}}{\pgfqpoint{4.045961in}{1.131301in}}{\pgfqpoint{4.045961in}{1.124168in}}%
\pgfpathcurveto{\pgfqpoint{4.045961in}{1.117035in}}{\pgfqpoint{4.048794in}{1.110194in}}{\pgfqpoint{4.053838in}{1.105150in}}%
\pgfpathcurveto{\pgfqpoint{4.058882in}{1.100106in}}{\pgfqpoint{4.065723in}{1.097272in}}{\pgfqpoint{4.072856in}{1.097272in}}%
\pgfpathclose%
\pgfusepath{stroke,fill}%
\end{pgfscope}%
\begin{pgfscope}%
\pgfpathrectangle{\pgfqpoint{2.867647in}{0.500000in}}{\pgfqpoint{1.764706in}{1.700000in}}%
\pgfusepath{clip}%
\pgfsetbuttcap%
\pgfsetroundjoin%
\definecolor{currentfill}{rgb}{0.975018,0.868213,0.788710}%
\pgfsetfillcolor{currentfill}%
\pgfsetlinewidth{0.311001pt}%
\definecolor{currentstroke}{rgb}{1.000000,1.000000,1.000000}%
\pgfsetstrokecolor{currentstroke}%
\pgfsetdash{}{0pt}%
\pgfpathmoveto{\pgfqpoint{4.103594in}{1.487428in}}%
\pgfpathcurveto{\pgfqpoint{4.110727in}{1.487428in}}{\pgfqpoint{4.117568in}{1.490262in}}{\pgfqpoint{4.122612in}{1.495305in}}%
\pgfpathcurveto{\pgfqpoint{4.127656in}{1.500349in}}{\pgfqpoint{4.130489in}{1.507191in}}{\pgfqpoint{4.130489in}{1.514323in}}%
\pgfpathcurveto{\pgfqpoint{4.130489in}{1.521456in}}{\pgfqpoint{4.127656in}{1.528298in}}{\pgfqpoint{4.122612in}{1.533342in}}%
\pgfpathcurveto{\pgfqpoint{4.117568in}{1.538385in}}{\pgfqpoint{4.110727in}{1.541219in}}{\pgfqpoint{4.103594in}{1.541219in}}%
\pgfpathcurveto{\pgfqpoint{4.096461in}{1.541219in}}{\pgfqpoint{4.089619in}{1.538385in}}{\pgfqpoint{4.084576in}{1.533342in}}%
\pgfpathcurveto{\pgfqpoint{4.079532in}{1.528298in}}{\pgfqpoint{4.076698in}{1.521456in}}{\pgfqpoint{4.076698in}{1.514323in}}%
\pgfpathcurveto{\pgfqpoint{4.076698in}{1.507191in}}{\pgfqpoint{4.079532in}{1.500349in}}{\pgfqpoint{4.084576in}{1.495305in}}%
\pgfpathcurveto{\pgfqpoint{4.089619in}{1.490262in}}{\pgfqpoint{4.096461in}{1.487428in}}{\pgfqpoint{4.103594in}{1.487428in}}%
\pgfpathclose%
\pgfusepath{stroke,fill}%
\end{pgfscope}%
\begin{pgfscope}%
\pgfpathrectangle{\pgfqpoint{2.867647in}{0.500000in}}{\pgfqpoint{1.764706in}{1.700000in}}%
\pgfusepath{clip}%
\pgfsetbuttcap%
\pgfsetroundjoin%
\definecolor{currentfill}{rgb}{0.980678,0.914765,0.856766}%
\pgfsetfillcolor{currentfill}%
\pgfsetlinewidth{0.311001pt}%
\definecolor{currentstroke}{rgb}{1.000000,1.000000,1.000000}%
\pgfsetstrokecolor{currentstroke}%
\pgfsetdash{}{0pt}%
\pgfpathmoveto{\pgfqpoint{4.153048in}{1.202962in}}%
\pgfpathcurveto{\pgfqpoint{4.160181in}{1.202962in}}{\pgfqpoint{4.167023in}{1.205796in}}{\pgfqpoint{4.172066in}{1.210839in}}%
\pgfpathcurveto{\pgfqpoint{4.177110in}{1.215883in}}{\pgfqpoint{4.179944in}{1.222725in}}{\pgfqpoint{4.179944in}{1.229857in}}%
\pgfpathcurveto{\pgfqpoint{4.179944in}{1.236990in}}{\pgfqpoint{4.177110in}{1.243832in}}{\pgfqpoint{4.172066in}{1.248876in}}%
\pgfpathcurveto{\pgfqpoint{4.167023in}{1.253919in}}{\pgfqpoint{4.160181in}{1.256753in}}{\pgfqpoint{4.153048in}{1.256753in}}%
\pgfpathcurveto{\pgfqpoint{4.145916in}{1.256753in}}{\pgfqpoint{4.139074in}{1.253919in}}{\pgfqpoint{4.134030in}{1.248876in}}%
\pgfpathcurveto{\pgfqpoint{4.128987in}{1.243832in}}{\pgfqpoint{4.126153in}{1.236990in}}{\pgfqpoint{4.126153in}{1.229857in}}%
\pgfpathcurveto{\pgfqpoint{4.126153in}{1.222725in}}{\pgfqpoint{4.128987in}{1.215883in}}{\pgfqpoint{4.134030in}{1.210839in}}%
\pgfpathcurveto{\pgfqpoint{4.139074in}{1.205796in}}{\pgfqpoint{4.145916in}{1.202962in}}{\pgfqpoint{4.153048in}{1.202962in}}%
\pgfpathclose%
\pgfusepath{stroke,fill}%
\end{pgfscope}%
\begin{pgfscope}%
\pgfpathrectangle{\pgfqpoint{2.867647in}{0.500000in}}{\pgfqpoint{1.764706in}{1.700000in}}%
\pgfusepath{clip}%
\pgfsetbuttcap%
\pgfsetroundjoin%
\definecolor{currentfill}{rgb}{0.978376,0.897317,0.831308}%
\pgfsetfillcolor{currentfill}%
\pgfsetlinewidth{0.311001pt}%
\definecolor{currentstroke}{rgb}{1.000000,1.000000,1.000000}%
\pgfsetstrokecolor{currentstroke}%
\pgfsetdash{}{0pt}%
\pgfpathmoveto{\pgfqpoint{4.162846in}{1.080637in}}%
\pgfpathcurveto{\pgfqpoint{4.169978in}{1.080637in}}{\pgfqpoint{4.176820in}{1.083471in}}{\pgfqpoint{4.181864in}{1.088515in}}%
\pgfpathcurveto{\pgfqpoint{4.186907in}{1.093558in}}{\pgfqpoint{4.189741in}{1.100400in}}{\pgfqpoint{4.189741in}{1.107533in}}%
\pgfpathcurveto{\pgfqpoint{4.189741in}{1.114666in}}{\pgfqpoint{4.186907in}{1.121507in}}{\pgfqpoint{4.181864in}{1.126551in}}%
\pgfpathcurveto{\pgfqpoint{4.176820in}{1.131594in}}{\pgfqpoint{4.169978in}{1.134428in}}{\pgfqpoint{4.162846in}{1.134428in}}%
\pgfpathcurveto{\pgfqpoint{4.155713in}{1.134428in}}{\pgfqpoint{4.148871in}{1.131594in}}{\pgfqpoint{4.143827in}{1.126551in}}%
\pgfpathcurveto{\pgfqpoint{4.138784in}{1.121507in}}{\pgfqpoint{4.135950in}{1.114666in}}{\pgfqpoint{4.135950in}{1.107533in}}%
\pgfpathcurveto{\pgfqpoint{4.135950in}{1.100400in}}{\pgfqpoint{4.138784in}{1.093558in}}{\pgfqpoint{4.143827in}{1.088515in}}%
\pgfpathcurveto{\pgfqpoint{4.148871in}{1.083471in}}{\pgfqpoint{4.155713in}{1.080637in}}{\pgfqpoint{4.162846in}{1.080637in}}%
\pgfpathclose%
\pgfusepath{stroke,fill}%
\end{pgfscope}%
\begin{pgfscope}%
\pgfpathrectangle{\pgfqpoint{2.867647in}{0.500000in}}{\pgfqpoint{1.764706in}{1.700000in}}%
\pgfusepath{clip}%
\pgfsetbuttcap%
\pgfsetroundjoin%
\definecolor{currentfill}{rgb}{0.979124,0.903132,0.839793}%
\pgfsetfillcolor{currentfill}%
\pgfsetlinewidth{0.311001pt}%
\definecolor{currentstroke}{rgb}{1.000000,1.000000,1.000000}%
\pgfsetstrokecolor{currentstroke}%
\pgfsetdash{}{0pt}%
\pgfpathmoveto{\pgfqpoint{4.214053in}{1.190156in}}%
\pgfpathcurveto{\pgfqpoint{4.221185in}{1.190156in}}{\pgfqpoint{4.228027in}{1.192989in}}{\pgfqpoint{4.233071in}{1.198033in}}%
\pgfpathcurveto{\pgfqpoint{4.238114in}{1.203077in}}{\pgfqpoint{4.240948in}{1.209918in}}{\pgfqpoint{4.240948in}{1.217051in}}%
\pgfpathcurveto{\pgfqpoint{4.240948in}{1.224184in}}{\pgfqpoint{4.238114in}{1.231026in}}{\pgfqpoint{4.233071in}{1.236069in}}%
\pgfpathcurveto{\pgfqpoint{4.228027in}{1.241113in}}{\pgfqpoint{4.221185in}{1.243947in}}{\pgfqpoint{4.214053in}{1.243947in}}%
\pgfpathcurveto{\pgfqpoint{4.206920in}{1.243947in}}{\pgfqpoint{4.200078in}{1.241113in}}{\pgfqpoint{4.195034in}{1.236069in}}%
\pgfpathcurveto{\pgfqpoint{4.189991in}{1.231026in}}{\pgfqpoint{4.187157in}{1.224184in}}{\pgfqpoint{4.187157in}{1.217051in}}%
\pgfpathcurveto{\pgfqpoint{4.187157in}{1.209918in}}{\pgfqpoint{4.189991in}{1.203077in}}{\pgfqpoint{4.195034in}{1.198033in}}%
\pgfpathcurveto{\pgfqpoint{4.200078in}{1.192989in}}{\pgfqpoint{4.206920in}{1.190156in}}{\pgfqpoint{4.214053in}{1.190156in}}%
\pgfpathclose%
\pgfusepath{stroke,fill}%
\end{pgfscope}%
\begin{pgfscope}%
\pgfpathrectangle{\pgfqpoint{2.867647in}{0.500000in}}{\pgfqpoint{1.764706in}{1.700000in}}%
\pgfusepath{clip}%
\pgfsetbuttcap%
\pgfsetroundjoin%
\definecolor{currentfill}{rgb}{0.968931,0.798091,0.685123}%
\pgfsetfillcolor{currentfill}%
\pgfsetlinewidth{0.311001pt}%
\definecolor{currentstroke}{rgb}{1.000000,1.000000,1.000000}%
\pgfsetstrokecolor{currentstroke}%
\pgfsetdash{}{0pt}%
\pgfpathmoveto{\pgfqpoint{4.049875in}{1.548848in}}%
\pgfpathcurveto{\pgfqpoint{4.057008in}{1.548848in}}{\pgfqpoint{4.063849in}{1.551682in}}{\pgfqpoint{4.068893in}{1.556726in}}%
\pgfpathcurveto{\pgfqpoint{4.073937in}{1.561769in}}{\pgfqpoint{4.076770in}{1.568611in}}{\pgfqpoint{4.076770in}{1.575744in}}%
\pgfpathcurveto{\pgfqpoint{4.076770in}{1.582877in}}{\pgfqpoint{4.073937in}{1.589718in}}{\pgfqpoint{4.068893in}{1.594762in}}%
\pgfpathcurveto{\pgfqpoint{4.063849in}{1.599806in}}{\pgfqpoint{4.057008in}{1.602640in}}{\pgfqpoint{4.049875in}{1.602640in}}%
\pgfpathcurveto{\pgfqpoint{4.042742in}{1.602640in}}{\pgfqpoint{4.035900in}{1.599806in}}{\pgfqpoint{4.030857in}{1.594762in}}%
\pgfpathcurveto{\pgfqpoint{4.025813in}{1.589718in}}{\pgfqpoint{4.022979in}{1.582877in}}{\pgfqpoint{4.022979in}{1.575744in}}%
\pgfpathcurveto{\pgfqpoint{4.022979in}{1.568611in}}{\pgfqpoint{4.025813in}{1.561769in}}{\pgfqpoint{4.030857in}{1.556726in}}%
\pgfpathcurveto{\pgfqpoint{4.035900in}{1.551682in}}{\pgfqpoint{4.042742in}{1.548848in}}{\pgfqpoint{4.049875in}{1.548848in}}%
\pgfpathclose%
\pgfusepath{stroke,fill}%
\end{pgfscope}%
\begin{pgfscope}%
\pgfpathrectangle{\pgfqpoint{2.867647in}{0.500000in}}{\pgfqpoint{1.764706in}{1.700000in}}%
\pgfusepath{clip}%
\pgfsetbuttcap%
\pgfsetroundjoin%
\definecolor{currentfill}{rgb}{0.980678,0.914765,0.856766}%
\pgfsetfillcolor{currentfill}%
\pgfsetlinewidth{0.311001pt}%
\definecolor{currentstroke}{rgb}{1.000000,1.000000,1.000000}%
\pgfsetstrokecolor{currentstroke}%
\pgfsetdash{}{0pt}%
\pgfpathmoveto{\pgfqpoint{4.199489in}{1.394330in}}%
\pgfpathcurveto{\pgfqpoint{4.206621in}{1.394330in}}{\pgfqpoint{4.213463in}{1.397164in}}{\pgfqpoint{4.218507in}{1.402208in}}%
\pgfpathcurveto{\pgfqpoint{4.223550in}{1.407252in}}{\pgfqpoint{4.226384in}{1.414093in}}{\pgfqpoint{4.226384in}{1.421226in}}%
\pgfpathcurveto{\pgfqpoint{4.226384in}{1.428359in}}{\pgfqpoint{4.223550in}{1.435200in}}{\pgfqpoint{4.218507in}{1.440244in}}%
\pgfpathcurveto{\pgfqpoint{4.213463in}{1.445288in}}{\pgfqpoint{4.206621in}{1.448122in}}{\pgfqpoint{4.199489in}{1.448122in}}%
\pgfpathcurveto{\pgfqpoint{4.192356in}{1.448122in}}{\pgfqpoint{4.185514in}{1.445288in}}{\pgfqpoint{4.180470in}{1.440244in}}%
\pgfpathcurveto{\pgfqpoint{4.175427in}{1.435200in}}{\pgfqpoint{4.172593in}{1.428359in}}{\pgfqpoint{4.172593in}{1.421226in}}%
\pgfpathcurveto{\pgfqpoint{4.172593in}{1.414093in}}{\pgfqpoint{4.175427in}{1.407252in}}{\pgfqpoint{4.180470in}{1.402208in}}%
\pgfpathcurveto{\pgfqpoint{4.185514in}{1.397164in}}{\pgfqpoint{4.192356in}{1.394330in}}{\pgfqpoint{4.199489in}{1.394330in}}%
\pgfpathclose%
\pgfusepath{stroke,fill}%
\end{pgfscope}%
\begin{pgfscope}%
\pgfpathrectangle{\pgfqpoint{2.867647in}{0.500000in}}{\pgfqpoint{1.764706in}{1.700000in}}%
\pgfusepath{clip}%
\pgfsetbuttcap%
\pgfsetroundjoin%
\definecolor{currentfill}{rgb}{0.577499,0.110312,0.358417}%
\pgfsetfillcolor{currentfill}%
\pgfsetlinewidth{0.311001pt}%
\definecolor{currentstroke}{rgb}{1.000000,1.000000,1.000000}%
\pgfsetstrokecolor{currentstroke}%
\pgfsetdash{}{0pt}%
\pgfpathmoveto{\pgfqpoint{4.050628in}{0.698879in}}%
\pgfpathcurveto{\pgfqpoint{4.057761in}{0.698879in}}{\pgfqpoint{4.064603in}{0.701713in}}{\pgfqpoint{4.069646in}{0.706756in}}%
\pgfpathcurveto{\pgfqpoint{4.074690in}{0.711800in}}{\pgfqpoint{4.077524in}{0.718642in}}{\pgfqpoint{4.077524in}{0.725774in}}%
\pgfpathcurveto{\pgfqpoint{4.077524in}{0.732907in}}{\pgfqpoint{4.074690in}{0.739749in}}{\pgfqpoint{4.069646in}{0.744793in}}%
\pgfpathcurveto{\pgfqpoint{4.064603in}{0.749836in}}{\pgfqpoint{4.057761in}{0.752670in}}{\pgfqpoint{4.050628in}{0.752670in}}%
\pgfpathcurveto{\pgfqpoint{4.043495in}{0.752670in}}{\pgfqpoint{4.036654in}{0.749836in}}{\pgfqpoint{4.031610in}{0.744793in}}%
\pgfpathcurveto{\pgfqpoint{4.026566in}{0.739749in}}{\pgfqpoint{4.023732in}{0.732907in}}{\pgfqpoint{4.023732in}{0.725774in}}%
\pgfpathcurveto{\pgfqpoint{4.023732in}{0.718642in}}{\pgfqpoint{4.026566in}{0.711800in}}{\pgfqpoint{4.031610in}{0.706756in}}%
\pgfpathcurveto{\pgfqpoint{4.036654in}{0.701713in}}{\pgfqpoint{4.043495in}{0.698879in}}{\pgfqpoint{4.050628in}{0.698879in}}%
\pgfpathclose%
\pgfusepath{stroke,fill}%
\end{pgfscope}%
\begin{pgfscope}%
\pgfpathrectangle{\pgfqpoint{2.867647in}{0.500000in}}{\pgfqpoint{1.764706in}{1.700000in}}%
\pgfusepath{clip}%
\pgfsetbuttcap%
\pgfsetroundjoin%
\definecolor{currentfill}{rgb}{0.977657,0.891500,0.822809}%
\pgfsetfillcolor{currentfill}%
\pgfsetlinewidth{0.311001pt}%
\definecolor{currentstroke}{rgb}{1.000000,1.000000,1.000000}%
\pgfsetstrokecolor{currentstroke}%
\pgfsetdash{}{0pt}%
\pgfpathmoveto{\pgfqpoint{4.208338in}{1.481543in}}%
\pgfpathcurveto{\pgfqpoint{4.215470in}{1.481543in}}{\pgfqpoint{4.222312in}{1.484377in}}{\pgfqpoint{4.227356in}{1.489420in}}%
\pgfpathcurveto{\pgfqpoint{4.232399in}{1.494464in}}{\pgfqpoint{4.235233in}{1.501306in}}{\pgfqpoint{4.235233in}{1.508438in}}%
\pgfpathcurveto{\pgfqpoint{4.235233in}{1.515571in}}{\pgfqpoint{4.232399in}{1.522413in}}{\pgfqpoint{4.227356in}{1.527457in}}%
\pgfpathcurveto{\pgfqpoint{4.222312in}{1.532500in}}{\pgfqpoint{4.215470in}{1.535334in}}{\pgfqpoint{4.208338in}{1.535334in}}%
\pgfpathcurveto{\pgfqpoint{4.201205in}{1.535334in}}{\pgfqpoint{4.194363in}{1.532500in}}{\pgfqpoint{4.189319in}{1.527457in}}%
\pgfpathcurveto{\pgfqpoint{4.184276in}{1.522413in}}{\pgfqpoint{4.181442in}{1.515571in}}{\pgfqpoint{4.181442in}{1.508438in}}%
\pgfpathcurveto{\pgfqpoint{4.181442in}{1.501306in}}{\pgfqpoint{4.184276in}{1.494464in}}{\pgfqpoint{4.189319in}{1.489420in}}%
\pgfpathcurveto{\pgfqpoint{4.194363in}{1.484377in}}{\pgfqpoint{4.201205in}{1.481543in}}{\pgfqpoint{4.208338in}{1.481543in}}%
\pgfpathclose%
\pgfusepath{stroke,fill}%
\end{pgfscope}%
\begin{pgfscope}%
\pgfpathrectangle{\pgfqpoint{2.867647in}{0.500000in}}{\pgfqpoint{1.764706in}{1.700000in}}%
\pgfusepath{clip}%
\pgfsetbuttcap%
\pgfsetroundjoin%
\definecolor{currentfill}{rgb}{0.959229,0.533075,0.374889}%
\pgfsetfillcolor{currentfill}%
\pgfsetlinewidth{0.311001pt}%
\definecolor{currentstroke}{rgb}{1.000000,1.000000,1.000000}%
\pgfsetstrokecolor{currentstroke}%
\pgfsetdash{}{0pt}%
\pgfpathmoveto{\pgfqpoint{4.195325in}{0.904382in}}%
\pgfpathcurveto{\pgfqpoint{4.202458in}{0.904382in}}{\pgfqpoint{4.209299in}{0.907216in}}{\pgfqpoint{4.214343in}{0.912260in}}%
\pgfpathcurveto{\pgfqpoint{4.219387in}{0.917303in}}{\pgfqpoint{4.222221in}{0.924145in}}{\pgfqpoint{4.222221in}{0.931278in}}%
\pgfpathcurveto{\pgfqpoint{4.222221in}{0.938411in}}{\pgfqpoint{4.219387in}{0.945252in}}{\pgfqpoint{4.214343in}{0.950296in}}%
\pgfpathcurveto{\pgfqpoint{4.209299in}{0.955340in}}{\pgfqpoint{4.202458in}{0.958174in}}{\pgfqpoint{4.195325in}{0.958174in}}%
\pgfpathcurveto{\pgfqpoint{4.188192in}{0.958174in}}{\pgfqpoint{4.181350in}{0.955340in}}{\pgfqpoint{4.176307in}{0.950296in}}%
\pgfpathcurveto{\pgfqpoint{4.171263in}{0.945252in}}{\pgfqpoint{4.168429in}{0.938411in}}{\pgfqpoint{4.168429in}{0.931278in}}%
\pgfpathcurveto{\pgfqpoint{4.168429in}{0.924145in}}{\pgfqpoint{4.171263in}{0.917303in}}{\pgfqpoint{4.176307in}{0.912260in}}%
\pgfpathcurveto{\pgfqpoint{4.181350in}{0.907216in}}{\pgfqpoint{4.188192in}{0.904382in}}{\pgfqpoint{4.195325in}{0.904382in}}%
\pgfpathclose%
\pgfusepath{stroke,fill}%
\end{pgfscope}%
\begin{pgfscope}%
\pgfpathrectangle{\pgfqpoint{2.867647in}{0.500000in}}{\pgfqpoint{1.764706in}{1.700000in}}%
\pgfusepath{clip}%
\pgfsetbuttcap%
\pgfsetroundjoin%
\definecolor{currentfill}{rgb}{0.970718,0.821518,0.719872}%
\pgfsetfillcolor{currentfill}%
\pgfsetlinewidth{0.311001pt}%
\definecolor{currentstroke}{rgb}{1.000000,1.000000,1.000000}%
\pgfsetstrokecolor{currentstroke}%
\pgfsetdash{}{0pt}%
\pgfpathmoveto{\pgfqpoint{4.066643in}{1.098172in}}%
\pgfpathcurveto{\pgfqpoint{4.073776in}{1.098172in}}{\pgfqpoint{4.080617in}{1.101006in}}{\pgfqpoint{4.085661in}{1.106050in}}%
\pgfpathcurveto{\pgfqpoint{4.090705in}{1.111093in}}{\pgfqpoint{4.093539in}{1.117935in}}{\pgfqpoint{4.093539in}{1.125068in}}%
\pgfpathcurveto{\pgfqpoint{4.093539in}{1.132201in}}{\pgfqpoint{4.090705in}{1.139042in}}{\pgfqpoint{4.085661in}{1.144086in}}%
\pgfpathcurveto{\pgfqpoint{4.080617in}{1.149130in}}{\pgfqpoint{4.073776in}{1.151964in}}{\pgfqpoint{4.066643in}{1.151964in}}%
\pgfpathcurveto{\pgfqpoint{4.059510in}{1.151964in}}{\pgfqpoint{4.052668in}{1.149130in}}{\pgfqpoint{4.047625in}{1.144086in}}%
\pgfpathcurveto{\pgfqpoint{4.042581in}{1.139042in}}{\pgfqpoint{4.039747in}{1.132201in}}{\pgfqpoint{4.039747in}{1.125068in}}%
\pgfpathcurveto{\pgfqpoint{4.039747in}{1.117935in}}{\pgfqpoint{4.042581in}{1.111093in}}{\pgfqpoint{4.047625in}{1.106050in}}%
\pgfpathcurveto{\pgfqpoint{4.052668in}{1.101006in}}{\pgfqpoint{4.059510in}{1.098172in}}{\pgfqpoint{4.066643in}{1.098172in}}%
\pgfpathclose%
\pgfusepath{stroke,fill}%
\end{pgfscope}%
\begin{pgfscope}%
\pgfpathrectangle{\pgfqpoint{2.867647in}{0.500000in}}{\pgfqpoint{1.764706in}{1.700000in}}%
\pgfusepath{clip}%
\pgfsetbuttcap%
\pgfsetroundjoin%
\definecolor{currentfill}{rgb}{0.965302,0.713942,0.568499}%
\pgfsetfillcolor{currentfill}%
\pgfsetlinewidth{0.311001pt}%
\definecolor{currentstroke}{rgb}{1.000000,1.000000,1.000000}%
\pgfsetstrokecolor{currentstroke}%
\pgfsetdash{}{0pt}%
\pgfpathmoveto{\pgfqpoint{4.042516in}{1.488843in}}%
\pgfpathcurveto{\pgfqpoint{4.049649in}{1.488843in}}{\pgfqpoint{4.056491in}{1.491677in}}{\pgfqpoint{4.061535in}{1.496721in}}%
\pgfpathcurveto{\pgfqpoint{4.066578in}{1.501765in}}{\pgfqpoint{4.069412in}{1.508606in}}{\pgfqpoint{4.069412in}{1.515739in}}%
\pgfpathcurveto{\pgfqpoint{4.069412in}{1.522872in}}{\pgfqpoint{4.066578in}{1.529714in}}{\pgfqpoint{4.061535in}{1.534757in}}%
\pgfpathcurveto{\pgfqpoint{4.056491in}{1.539801in}}{\pgfqpoint{4.049649in}{1.542635in}}{\pgfqpoint{4.042516in}{1.542635in}}%
\pgfpathcurveto{\pgfqpoint{4.035384in}{1.542635in}}{\pgfqpoint{4.028542in}{1.539801in}}{\pgfqpoint{4.023498in}{1.534757in}}%
\pgfpathcurveto{\pgfqpoint{4.018455in}{1.529714in}}{\pgfqpoint{4.015621in}{1.522872in}}{\pgfqpoint{4.015621in}{1.515739in}}%
\pgfpathcurveto{\pgfqpoint{4.015621in}{1.508606in}}{\pgfqpoint{4.018455in}{1.501765in}}{\pgfqpoint{4.023498in}{1.496721in}}%
\pgfpathcurveto{\pgfqpoint{4.028542in}{1.491677in}}{\pgfqpoint{4.035384in}{1.488843in}}{\pgfqpoint{4.042516in}{1.488843in}}%
\pgfpathclose%
\pgfusepath{stroke,fill}%
\end{pgfscope}%
\begin{pgfscope}%
\pgfpathrectangle{\pgfqpoint{2.867647in}{0.500000in}}{\pgfqpoint{1.764706in}{1.700000in}}%
\pgfusepath{clip}%
\pgfsetbuttcap%
\pgfsetroundjoin%
\definecolor{currentfill}{rgb}{0.968105,0.786346,0.667739}%
\pgfsetfillcolor{currentfill}%
\pgfsetlinewidth{0.311001pt}%
\definecolor{currentstroke}{rgb}{1.000000,1.000000,1.000000}%
\pgfsetstrokecolor{currentstroke}%
\pgfsetdash{}{0pt}%
\pgfpathmoveto{\pgfqpoint{4.071307in}{1.164845in}}%
\pgfpathcurveto{\pgfqpoint{4.078440in}{1.164845in}}{\pgfqpoint{4.085282in}{1.167678in}}{\pgfqpoint{4.090325in}{1.172722in}}%
\pgfpathcurveto{\pgfqpoint{4.095369in}{1.177766in}}{\pgfqpoint{4.098203in}{1.184607in}}{\pgfqpoint{4.098203in}{1.191740in}}%
\pgfpathcurveto{\pgfqpoint{4.098203in}{1.198873in}}{\pgfqpoint{4.095369in}{1.205715in}}{\pgfqpoint{4.090325in}{1.210758in}}%
\pgfpathcurveto{\pgfqpoint{4.085282in}{1.215802in}}{\pgfqpoint{4.078440in}{1.218636in}}{\pgfqpoint{4.071307in}{1.218636in}}%
\pgfpathcurveto{\pgfqpoint{4.064174in}{1.218636in}}{\pgfqpoint{4.057333in}{1.215802in}}{\pgfqpoint{4.052289in}{1.210758in}}%
\pgfpathcurveto{\pgfqpoint{4.047245in}{1.205715in}}{\pgfqpoint{4.044412in}{1.198873in}}{\pgfqpoint{4.044412in}{1.191740in}}%
\pgfpathcurveto{\pgfqpoint{4.044412in}{1.184607in}}{\pgfqpoint{4.047245in}{1.177766in}}{\pgfqpoint{4.052289in}{1.172722in}}%
\pgfpathcurveto{\pgfqpoint{4.057333in}{1.167678in}}{\pgfqpoint{4.064174in}{1.164845in}}{\pgfqpoint{4.071307in}{1.164845in}}%
\pgfpathclose%
\pgfusepath{stroke,fill}%
\end{pgfscope}%
\begin{pgfscope}%
\pgfpathrectangle{\pgfqpoint{2.867647in}{0.500000in}}{\pgfqpoint{1.764706in}{1.700000in}}%
\pgfusepath{clip}%
\pgfsetbuttcap%
\pgfsetroundjoin%
\definecolor{currentfill}{rgb}{0.972726,0.844889,0.754401}%
\pgfsetfillcolor{currentfill}%
\pgfsetlinewidth{0.311001pt}%
\definecolor{currentstroke}{rgb}{1.000000,1.000000,1.000000}%
\pgfsetstrokecolor{currentstroke}%
\pgfsetdash{}{0pt}%
\pgfpathmoveto{\pgfqpoint{4.068506in}{1.034124in}}%
\pgfpathcurveto{\pgfqpoint{4.075638in}{1.034124in}}{\pgfqpoint{4.082480in}{1.036958in}}{\pgfqpoint{4.087524in}{1.042002in}}%
\pgfpathcurveto{\pgfqpoint{4.092567in}{1.047045in}}{\pgfqpoint{4.095401in}{1.053887in}}{\pgfqpoint{4.095401in}{1.061020in}}%
\pgfpathcurveto{\pgfqpoint{4.095401in}{1.068153in}}{\pgfqpoint{4.092567in}{1.074994in}}{\pgfqpoint{4.087524in}{1.080038in}}%
\pgfpathcurveto{\pgfqpoint{4.082480in}{1.085082in}}{\pgfqpoint{4.075638in}{1.087916in}}{\pgfqpoint{4.068506in}{1.087916in}}%
\pgfpathcurveto{\pgfqpoint{4.061373in}{1.087916in}}{\pgfqpoint{4.054531in}{1.085082in}}{\pgfqpoint{4.049487in}{1.080038in}}%
\pgfpathcurveto{\pgfqpoint{4.044444in}{1.074994in}}{\pgfqpoint{4.041610in}{1.068153in}}{\pgfqpoint{4.041610in}{1.061020in}}%
\pgfpathcurveto{\pgfqpoint{4.041610in}{1.053887in}}{\pgfqpoint{4.044444in}{1.047045in}}{\pgfqpoint{4.049487in}{1.042002in}}%
\pgfpathcurveto{\pgfqpoint{4.054531in}{1.036958in}}{\pgfqpoint{4.061373in}{1.034124in}}{\pgfqpoint{4.068506in}{1.034124in}}%
\pgfpathclose%
\pgfusepath{stroke,fill}%
\end{pgfscope}%
\begin{pgfscope}%
\pgfpathrectangle{\pgfqpoint{2.867647in}{0.500000in}}{\pgfqpoint{1.764706in}{1.700000in}}%
\pgfusepath{clip}%
\pgfsetbuttcap%
\pgfsetroundjoin%
\definecolor{currentfill}{rgb}{0.979124,0.903132,0.839793}%
\pgfsetfillcolor{currentfill}%
\pgfsetlinewidth{0.311001pt}%
\definecolor{currentstroke}{rgb}{1.000000,1.000000,1.000000}%
\pgfsetstrokecolor{currentstroke}%
\pgfsetdash{}{0pt}%
\pgfpathmoveto{\pgfqpoint{4.155053in}{1.271333in}}%
\pgfpathcurveto{\pgfqpoint{4.162185in}{1.271333in}}{\pgfqpoint{4.169027in}{1.274167in}}{\pgfqpoint{4.174071in}{1.279210in}}%
\pgfpathcurveto{\pgfqpoint{4.179114in}{1.284254in}}{\pgfqpoint{4.181948in}{1.291096in}}{\pgfqpoint{4.181948in}{1.298228in}}%
\pgfpathcurveto{\pgfqpoint{4.181948in}{1.305361in}}{\pgfqpoint{4.179114in}{1.312203in}}{\pgfqpoint{4.174071in}{1.317246in}}%
\pgfpathcurveto{\pgfqpoint{4.169027in}{1.322290in}}{\pgfqpoint{4.162185in}{1.325124in}}{\pgfqpoint{4.155053in}{1.325124in}}%
\pgfpathcurveto{\pgfqpoint{4.147920in}{1.325124in}}{\pgfqpoint{4.141078in}{1.322290in}}{\pgfqpoint{4.136034in}{1.317246in}}%
\pgfpathcurveto{\pgfqpoint{4.130991in}{1.312203in}}{\pgfqpoint{4.128157in}{1.305361in}}{\pgfqpoint{4.128157in}{1.298228in}}%
\pgfpathcurveto{\pgfqpoint{4.128157in}{1.291096in}}{\pgfqpoint{4.130991in}{1.284254in}}{\pgfqpoint{4.136034in}{1.279210in}}%
\pgfpathcurveto{\pgfqpoint{4.141078in}{1.274167in}}{\pgfqpoint{4.147920in}{1.271333in}}{\pgfqpoint{4.155053in}{1.271333in}}%
\pgfpathclose%
\pgfusepath{stroke,fill}%
\end{pgfscope}%
\begin{pgfscope}%
\pgfpathrectangle{\pgfqpoint{2.867647in}{0.500000in}}{\pgfqpoint{1.764706in}{1.700000in}}%
\pgfusepath{clip}%
\pgfsetbuttcap%
\pgfsetroundjoin%
\definecolor{currentfill}{rgb}{0.975644,0.874038,0.797253}%
\pgfsetfillcolor{currentfill}%
\pgfsetlinewidth{0.311001pt}%
\definecolor{currentstroke}{rgb}{1.000000,1.000000,1.000000}%
\pgfsetstrokecolor{currentstroke}%
\pgfsetdash{}{0pt}%
\pgfpathmoveto{\pgfqpoint{4.108539in}{1.486402in}}%
\pgfpathcurveto{\pgfqpoint{4.115672in}{1.486402in}}{\pgfqpoint{4.122514in}{1.489236in}}{\pgfqpoint{4.127557in}{1.494280in}}%
\pgfpathcurveto{\pgfqpoint{4.132601in}{1.499324in}}{\pgfqpoint{4.135435in}{1.506165in}}{\pgfqpoint{4.135435in}{1.513298in}}%
\pgfpathcurveto{\pgfqpoint{4.135435in}{1.520431in}}{\pgfqpoint{4.132601in}{1.527273in}}{\pgfqpoint{4.127557in}{1.532316in}}%
\pgfpathcurveto{\pgfqpoint{4.122514in}{1.537360in}}{\pgfqpoint{4.115672in}{1.540194in}}{\pgfqpoint{4.108539in}{1.540194in}}%
\pgfpathcurveto{\pgfqpoint{4.101406in}{1.540194in}}{\pgfqpoint{4.094565in}{1.537360in}}{\pgfqpoint{4.089521in}{1.532316in}}%
\pgfpathcurveto{\pgfqpoint{4.084477in}{1.527273in}}{\pgfqpoint{4.081643in}{1.520431in}}{\pgfqpoint{4.081643in}{1.513298in}}%
\pgfpathcurveto{\pgfqpoint{4.081643in}{1.506165in}}{\pgfqpoint{4.084477in}{1.499324in}}{\pgfqpoint{4.089521in}{1.494280in}}%
\pgfpathcurveto{\pgfqpoint{4.094565in}{1.489236in}}{\pgfqpoint{4.101406in}{1.486402in}}{\pgfqpoint{4.108539in}{1.486402in}}%
\pgfpathclose%
\pgfusepath{stroke,fill}%
\end{pgfscope}%
\begin{pgfscope}%
\pgfpathrectangle{\pgfqpoint{2.867647in}{0.500000in}}{\pgfqpoint{1.764706in}{1.700000in}}%
\pgfusepath{clip}%
\pgfsetbuttcap%
\pgfsetroundjoin%
\definecolor{currentfill}{rgb}{0.980678,0.914765,0.856766}%
\pgfsetfillcolor{currentfill}%
\pgfsetlinewidth{0.311001pt}%
\definecolor{currentstroke}{rgb}{1.000000,1.000000,1.000000}%
\pgfsetstrokecolor{currentstroke}%
\pgfsetdash{}{0pt}%
\pgfpathmoveto{\pgfqpoint{4.206713in}{1.362794in}}%
\pgfpathcurveto{\pgfqpoint{4.213846in}{1.362794in}}{\pgfqpoint{4.220687in}{1.365628in}}{\pgfqpoint{4.225731in}{1.370672in}}%
\pgfpathcurveto{\pgfqpoint{4.230775in}{1.375715in}}{\pgfqpoint{4.233609in}{1.382557in}}{\pgfqpoint{4.233609in}{1.389690in}}%
\pgfpathcurveto{\pgfqpoint{4.233609in}{1.396823in}}{\pgfqpoint{4.230775in}{1.403664in}}{\pgfqpoint{4.225731in}{1.408708in}}%
\pgfpathcurveto{\pgfqpoint{4.220687in}{1.413752in}}{\pgfqpoint{4.213846in}{1.416585in}}{\pgfqpoint{4.206713in}{1.416585in}}%
\pgfpathcurveto{\pgfqpoint{4.199580in}{1.416585in}}{\pgfqpoint{4.192738in}{1.413752in}}{\pgfqpoint{4.187695in}{1.408708in}}%
\pgfpathcurveto{\pgfqpoint{4.182651in}{1.403664in}}{\pgfqpoint{4.179817in}{1.396823in}}{\pgfqpoint{4.179817in}{1.389690in}}%
\pgfpathcurveto{\pgfqpoint{4.179817in}{1.382557in}}{\pgfqpoint{4.182651in}{1.375715in}}{\pgfqpoint{4.187695in}{1.370672in}}%
\pgfpathcurveto{\pgfqpoint{4.192738in}{1.365628in}}{\pgfqpoint{4.199580in}{1.362794in}}{\pgfqpoint{4.206713in}{1.362794in}}%
\pgfpathclose%
\pgfusepath{stroke,fill}%
\end{pgfscope}%
\begin{pgfscope}%
\pgfpathrectangle{\pgfqpoint{2.867647in}{0.500000in}}{\pgfqpoint{1.764706in}{1.700000in}}%
\pgfusepath{clip}%
\pgfsetbuttcap%
\pgfsetroundjoin%
\definecolor{currentfill}{rgb}{0.970255,0.815666,0.711203}%
\pgfsetfillcolor{currentfill}%
\pgfsetlinewidth{0.311001pt}%
\definecolor{currentstroke}{rgb}{1.000000,1.000000,1.000000}%
\pgfsetstrokecolor{currentstroke}%
\pgfsetdash{}{0pt}%
\pgfpathmoveto{\pgfqpoint{4.268263in}{1.405984in}}%
\pgfpathcurveto{\pgfqpoint{4.275395in}{1.405984in}}{\pgfqpoint{4.282237in}{1.408818in}}{\pgfqpoint{4.287281in}{1.413862in}}%
\pgfpathcurveto{\pgfqpoint{4.292324in}{1.418905in}}{\pgfqpoint{4.295158in}{1.425747in}}{\pgfqpoint{4.295158in}{1.432880in}}%
\pgfpathcurveto{\pgfqpoint{4.295158in}{1.440013in}}{\pgfqpoint{4.292324in}{1.446854in}}{\pgfqpoint{4.287281in}{1.451898in}}%
\pgfpathcurveto{\pgfqpoint{4.282237in}{1.456941in}}{\pgfqpoint{4.275395in}{1.459775in}}{\pgfqpoint{4.268263in}{1.459775in}}%
\pgfpathcurveto{\pgfqpoint{4.261130in}{1.459775in}}{\pgfqpoint{4.254288in}{1.456941in}}{\pgfqpoint{4.249244in}{1.451898in}}%
\pgfpathcurveto{\pgfqpoint{4.244201in}{1.446854in}}{\pgfqpoint{4.241367in}{1.440013in}}{\pgfqpoint{4.241367in}{1.432880in}}%
\pgfpathcurveto{\pgfqpoint{4.241367in}{1.425747in}}{\pgfqpoint{4.244201in}{1.418905in}}{\pgfqpoint{4.249244in}{1.413862in}}%
\pgfpathcurveto{\pgfqpoint{4.254288in}{1.408818in}}{\pgfqpoint{4.261130in}{1.405984in}}{\pgfqpoint{4.268263in}{1.405984in}}%
\pgfpathclose%
\pgfusepath{stroke,fill}%
\end{pgfscope}%
\begin{pgfscope}%
\pgfpathrectangle{\pgfqpoint{2.867647in}{0.500000in}}{\pgfqpoint{1.764706in}{1.700000in}}%
\pgfusepath{clip}%
\pgfsetbuttcap%
\pgfsetroundjoin%
\definecolor{currentfill}{rgb}{0.975644,0.874038,0.797253}%
\pgfsetfillcolor{currentfill}%
\pgfsetlinewidth{0.311001pt}%
\definecolor{currentstroke}{rgb}{1.000000,1.000000,1.000000}%
\pgfsetstrokecolor{currentstroke}%
\pgfsetdash{}{0pt}%
\pgfpathmoveto{\pgfqpoint{4.093890in}{1.047083in}}%
\pgfpathcurveto{\pgfqpoint{4.101023in}{1.047083in}}{\pgfqpoint{4.107865in}{1.049917in}}{\pgfqpoint{4.112908in}{1.054960in}}%
\pgfpathcurveto{\pgfqpoint{4.117952in}{1.060004in}}{\pgfqpoint{4.120786in}{1.066846in}}{\pgfqpoint{4.120786in}{1.073979in}}%
\pgfpathcurveto{\pgfqpoint{4.120786in}{1.081111in}}{\pgfqpoint{4.117952in}{1.087953in}}{\pgfqpoint{4.112908in}{1.092997in}}%
\pgfpathcurveto{\pgfqpoint{4.107865in}{1.098040in}}{\pgfqpoint{4.101023in}{1.100874in}}{\pgfqpoint{4.093890in}{1.100874in}}%
\pgfpathcurveto{\pgfqpoint{4.086757in}{1.100874in}}{\pgfqpoint{4.079916in}{1.098040in}}{\pgfqpoint{4.074872in}{1.092997in}}%
\pgfpathcurveto{\pgfqpoint{4.069828in}{1.087953in}}{\pgfqpoint{4.066994in}{1.081111in}}{\pgfqpoint{4.066994in}{1.073979in}}%
\pgfpathcurveto{\pgfqpoint{4.066994in}{1.066846in}}{\pgfqpoint{4.069828in}{1.060004in}}{\pgfqpoint{4.074872in}{1.054960in}}%
\pgfpathcurveto{\pgfqpoint{4.079916in}{1.049917in}}{\pgfqpoint{4.086757in}{1.047083in}}{\pgfqpoint{4.093890in}{1.047083in}}%
\pgfpathclose%
\pgfusepath{stroke,fill}%
\end{pgfscope}%
\begin{pgfscope}%
\pgfpathrectangle{\pgfqpoint{2.867647in}{0.500000in}}{\pgfqpoint{1.764706in}{1.700000in}}%
\pgfusepath{clip}%
\pgfsetbuttcap%
\pgfsetroundjoin%
\definecolor{currentfill}{rgb}{0.972726,0.844889,0.754401}%
\pgfsetfillcolor{currentfill}%
\pgfsetlinewidth{0.311001pt}%
\definecolor{currentstroke}{rgb}{1.000000,1.000000,1.000000}%
\pgfsetstrokecolor{currentstroke}%
\pgfsetdash{}{0pt}%
\pgfpathmoveto{\pgfqpoint{4.115256in}{0.981960in}}%
\pgfpathcurveto{\pgfqpoint{4.122389in}{0.981960in}}{\pgfqpoint{4.129230in}{0.984794in}}{\pgfqpoint{4.134274in}{0.989838in}}%
\pgfpathcurveto{\pgfqpoint{4.139318in}{0.994882in}}{\pgfqpoint{4.142151in}{1.001723in}}{\pgfqpoint{4.142151in}{1.008856in}}%
\pgfpathcurveto{\pgfqpoint{4.142151in}{1.015989in}}{\pgfqpoint{4.139318in}{1.022831in}}{\pgfqpoint{4.134274in}{1.027874in}}%
\pgfpathcurveto{\pgfqpoint{4.129230in}{1.032918in}}{\pgfqpoint{4.122389in}{1.035752in}}{\pgfqpoint{4.115256in}{1.035752in}}%
\pgfpathcurveto{\pgfqpoint{4.108123in}{1.035752in}}{\pgfqpoint{4.101281in}{1.032918in}}{\pgfqpoint{4.096238in}{1.027874in}}%
\pgfpathcurveto{\pgfqpoint{4.091194in}{1.022831in}}{\pgfqpoint{4.088360in}{1.015989in}}{\pgfqpoint{4.088360in}{1.008856in}}%
\pgfpathcurveto{\pgfqpoint{4.088360in}{1.001723in}}{\pgfqpoint{4.091194in}{0.994882in}}{\pgfqpoint{4.096238in}{0.989838in}}%
\pgfpathcurveto{\pgfqpoint{4.101281in}{0.984794in}}{\pgfqpoint{4.108123in}{0.981960in}}{\pgfqpoint{4.115256in}{0.981960in}}%
\pgfpathclose%
\pgfusepath{stroke,fill}%
\end{pgfscope}%
\begin{pgfscope}%
\pgfpathrectangle{\pgfqpoint{2.867647in}{0.500000in}}{\pgfqpoint{1.764706in}{1.700000in}}%
\pgfusepath{clip}%
\pgfsetbuttcap%
\pgfsetroundjoin%
\definecolor{currentfill}{rgb}{0.967092,0.768560,0.642079}%
\pgfsetfillcolor{currentfill}%
\pgfsetlinewidth{0.311001pt}%
\definecolor{currentstroke}{rgb}{1.000000,1.000000,1.000000}%
\pgfsetstrokecolor{currentstroke}%
\pgfsetdash{}{0pt}%
\pgfpathmoveto{\pgfqpoint{4.095031in}{0.921710in}}%
\pgfpathcurveto{\pgfqpoint{4.102163in}{0.921710in}}{\pgfqpoint{4.109005in}{0.924543in}}{\pgfqpoint{4.114049in}{0.929587in}}%
\pgfpathcurveto{\pgfqpoint{4.119092in}{0.934631in}}{\pgfqpoint{4.121926in}{0.941472in}}{\pgfqpoint{4.121926in}{0.948605in}}%
\pgfpathcurveto{\pgfqpoint{4.121926in}{0.955738in}}{\pgfqpoint{4.119092in}{0.962580in}}{\pgfqpoint{4.114049in}{0.967623in}}%
\pgfpathcurveto{\pgfqpoint{4.109005in}{0.972667in}}{\pgfqpoint{4.102163in}{0.975501in}}{\pgfqpoint{4.095031in}{0.975501in}}%
\pgfpathcurveto{\pgfqpoint{4.087898in}{0.975501in}}{\pgfqpoint{4.081056in}{0.972667in}}{\pgfqpoint{4.076012in}{0.967623in}}%
\pgfpathcurveto{\pgfqpoint{4.070969in}{0.962580in}}{\pgfqpoint{4.068135in}{0.955738in}}{\pgfqpoint{4.068135in}{0.948605in}}%
\pgfpathcurveto{\pgfqpoint{4.068135in}{0.941472in}}{\pgfqpoint{4.070969in}{0.934631in}}{\pgfqpoint{4.076012in}{0.929587in}}%
\pgfpathcurveto{\pgfqpoint{4.081056in}{0.924543in}}{\pgfqpoint{4.087898in}{0.921710in}}{\pgfqpoint{4.095031in}{0.921710in}}%
\pgfpathclose%
\pgfusepath{stroke,fill}%
\end{pgfscope}%
\begin{pgfscope}%
\pgfpathrectangle{\pgfqpoint{2.867647in}{0.500000in}}{\pgfqpoint{1.764706in}{1.700000in}}%
\pgfusepath{clip}%
\pgfsetbuttcap%
\pgfsetroundjoin%
\definecolor{currentfill}{rgb}{0.979124,0.903132,0.839793}%
\pgfsetfillcolor{currentfill}%
\pgfsetlinewidth{0.311001pt}%
\definecolor{currentstroke}{rgb}{1.000000,1.000000,1.000000}%
\pgfsetstrokecolor{currentstroke}%
\pgfsetdash{}{0pt}%
\pgfpathmoveto{\pgfqpoint{4.222953in}{1.251231in}}%
\pgfpathcurveto{\pgfqpoint{4.230086in}{1.251231in}}{\pgfqpoint{4.236927in}{1.254065in}}{\pgfqpoint{4.241971in}{1.259109in}}%
\pgfpathcurveto{\pgfqpoint{4.247015in}{1.264152in}}{\pgfqpoint{4.249849in}{1.270994in}}{\pgfqpoint{4.249849in}{1.278127in}}%
\pgfpathcurveto{\pgfqpoint{4.249849in}{1.285260in}}{\pgfqpoint{4.247015in}{1.292101in}}{\pgfqpoint{4.241971in}{1.297145in}}%
\pgfpathcurveto{\pgfqpoint{4.236927in}{1.302188in}}{\pgfqpoint{4.230086in}{1.305022in}}{\pgfqpoint{4.222953in}{1.305022in}}%
\pgfpathcurveto{\pgfqpoint{4.215820in}{1.305022in}}{\pgfqpoint{4.208979in}{1.302188in}}{\pgfqpoint{4.203935in}{1.297145in}}%
\pgfpathcurveto{\pgfqpoint{4.198891in}{1.292101in}}{\pgfqpoint{4.196057in}{1.285260in}}{\pgfqpoint{4.196057in}{1.278127in}}%
\pgfpathcurveto{\pgfqpoint{4.196057in}{1.270994in}}{\pgfqpoint{4.198891in}{1.264152in}}{\pgfqpoint{4.203935in}{1.259109in}}%
\pgfpathcurveto{\pgfqpoint{4.208979in}{1.254065in}}{\pgfqpoint{4.215820in}{1.251231in}}{\pgfqpoint{4.222953in}{1.251231in}}%
\pgfpathclose%
\pgfusepath{stroke,fill}%
\end{pgfscope}%
\begin{pgfscope}%
\pgfpathrectangle{\pgfqpoint{2.867647in}{0.500000in}}{\pgfqpoint{1.764706in}{1.700000in}}%
\pgfusepath{clip}%
\pgfsetbuttcap%
\pgfsetroundjoin%
\definecolor{currentfill}{rgb}{0.966812,0.762584,0.633643}%
\pgfsetfillcolor{currentfill}%
\pgfsetlinewidth{0.311001pt}%
\definecolor{currentstroke}{rgb}{1.000000,1.000000,1.000000}%
\pgfsetstrokecolor{currentstroke}%
\pgfsetdash{}{0pt}%
\pgfpathmoveto{\pgfqpoint{4.053158in}{1.730320in}}%
\pgfpathcurveto{\pgfqpoint{4.060291in}{1.730320in}}{\pgfqpoint{4.067132in}{1.733154in}}{\pgfqpoint{4.072176in}{1.738198in}}%
\pgfpathcurveto{\pgfqpoint{4.077220in}{1.743241in}}{\pgfqpoint{4.080053in}{1.750083in}}{\pgfqpoint{4.080053in}{1.757216in}}%
\pgfpathcurveto{\pgfqpoint{4.080053in}{1.764349in}}{\pgfqpoint{4.077220in}{1.771190in}}{\pgfqpoint{4.072176in}{1.776234in}}%
\pgfpathcurveto{\pgfqpoint{4.067132in}{1.781278in}}{\pgfqpoint{4.060291in}{1.784112in}}{\pgfqpoint{4.053158in}{1.784112in}}%
\pgfpathcurveto{\pgfqpoint{4.046025in}{1.784112in}}{\pgfqpoint{4.039183in}{1.781278in}}{\pgfqpoint{4.034140in}{1.776234in}}%
\pgfpathcurveto{\pgfqpoint{4.029096in}{1.771190in}}{\pgfqpoint{4.026262in}{1.764349in}}{\pgfqpoint{4.026262in}{1.757216in}}%
\pgfpathcurveto{\pgfqpoint{4.026262in}{1.750083in}}{\pgfqpoint{4.029096in}{1.743241in}}{\pgfqpoint{4.034140in}{1.738198in}}%
\pgfpathcurveto{\pgfqpoint{4.039183in}{1.733154in}}{\pgfqpoint{4.046025in}{1.730320in}}{\pgfqpoint{4.053158in}{1.730320in}}%
\pgfpathclose%
\pgfusepath{stroke,fill}%
\end{pgfscope}%
\begin{pgfscope}%
\pgfpathrectangle{\pgfqpoint{2.867647in}{0.500000in}}{\pgfqpoint{1.764706in}{1.700000in}}%
\pgfusepath{clip}%
\pgfsetbuttcap%
\pgfsetroundjoin%
\definecolor{currentfill}{rgb}{0.974412,0.862387,0.780156}%
\pgfsetfillcolor{currentfill}%
\pgfsetlinewidth{0.311001pt}%
\definecolor{currentstroke}{rgb}{1.000000,1.000000,1.000000}%
\pgfsetstrokecolor{currentstroke}%
\pgfsetdash{}{0pt}%
\pgfpathmoveto{\pgfqpoint{4.099930in}{1.493192in}}%
\pgfpathcurveto{\pgfqpoint{4.107063in}{1.493192in}}{\pgfqpoint{4.113904in}{1.496026in}}{\pgfqpoint{4.118948in}{1.501069in}}%
\pgfpathcurveto{\pgfqpoint{4.123992in}{1.506113in}}{\pgfqpoint{4.126825in}{1.512955in}}{\pgfqpoint{4.126825in}{1.520087in}}%
\pgfpathcurveto{\pgfqpoint{4.126825in}{1.527220in}}{\pgfqpoint{4.123992in}{1.534062in}}{\pgfqpoint{4.118948in}{1.539106in}}%
\pgfpathcurveto{\pgfqpoint{4.113904in}{1.544149in}}{\pgfqpoint{4.107063in}{1.546983in}}{\pgfqpoint{4.099930in}{1.546983in}}%
\pgfpathcurveto{\pgfqpoint{4.092797in}{1.546983in}}{\pgfqpoint{4.085955in}{1.544149in}}{\pgfqpoint{4.080912in}{1.539106in}}%
\pgfpathcurveto{\pgfqpoint{4.075868in}{1.534062in}}{\pgfqpoint{4.073034in}{1.527220in}}{\pgfqpoint{4.073034in}{1.520087in}}%
\pgfpathcurveto{\pgfqpoint{4.073034in}{1.512955in}}{\pgfqpoint{4.075868in}{1.506113in}}{\pgfqpoint{4.080912in}{1.501069in}}%
\pgfpathcurveto{\pgfqpoint{4.085955in}{1.496026in}}{\pgfqpoint{4.092797in}{1.493192in}}{\pgfqpoint{4.099930in}{1.493192in}}%
\pgfpathclose%
\pgfusepath{stroke,fill}%
\end{pgfscope}%
\begin{pgfscope}%
\pgfpathrectangle{\pgfqpoint{2.867647in}{0.500000in}}{\pgfqpoint{1.764706in}{1.700000in}}%
\pgfusepath{clip}%
\pgfsetbuttcap%
\pgfsetroundjoin%
\definecolor{currentfill}{rgb}{0.967735,0.780441,0.659127}%
\pgfsetfillcolor{currentfill}%
\pgfsetlinewidth{0.311001pt}%
\definecolor{currentstroke}{rgb}{1.000000,1.000000,1.000000}%
\pgfsetstrokecolor{currentstroke}%
\pgfsetdash{}{0pt}%
\pgfpathmoveto{\pgfqpoint{4.079964in}{1.199839in}}%
\pgfpathcurveto{\pgfqpoint{4.087097in}{1.199839in}}{\pgfqpoint{4.093938in}{1.202673in}}{\pgfqpoint{4.098982in}{1.207717in}}%
\pgfpathcurveto{\pgfqpoint{4.104026in}{1.212761in}}{\pgfqpoint{4.106860in}{1.219602in}}{\pgfqpoint{4.106860in}{1.226735in}}%
\pgfpathcurveto{\pgfqpoint{4.106860in}{1.233868in}}{\pgfqpoint{4.104026in}{1.240710in}}{\pgfqpoint{4.098982in}{1.245753in}}%
\pgfpathcurveto{\pgfqpoint{4.093938in}{1.250797in}}{\pgfqpoint{4.087097in}{1.253631in}}{\pgfqpoint{4.079964in}{1.253631in}}%
\pgfpathcurveto{\pgfqpoint{4.072831in}{1.253631in}}{\pgfqpoint{4.065990in}{1.250797in}}{\pgfqpoint{4.060946in}{1.245753in}}%
\pgfpathcurveto{\pgfqpoint{4.055902in}{1.240710in}}{\pgfqpoint{4.053068in}{1.233868in}}{\pgfqpoint{4.053068in}{1.226735in}}%
\pgfpathcurveto{\pgfqpoint{4.053068in}{1.219602in}}{\pgfqpoint{4.055902in}{1.212761in}}{\pgfqpoint{4.060946in}{1.207717in}}%
\pgfpathcurveto{\pgfqpoint{4.065990in}{1.202673in}}{\pgfqpoint{4.072831in}{1.199839in}}{\pgfqpoint{4.079964in}{1.199839in}}%
\pgfpathclose%
\pgfusepath{stroke,fill}%
\end{pgfscope}%
\begin{pgfscope}%
\pgfpathrectangle{\pgfqpoint{2.867647in}{0.500000in}}{\pgfqpoint{1.764706in}{1.700000in}}%
\pgfusepath{clip}%
\pgfsetbuttcap%
\pgfsetroundjoin%
\definecolor{currentfill}{rgb}{0.972201,0.839051,0.745789}%
\pgfsetfillcolor{currentfill}%
\pgfsetlinewidth{0.311001pt}%
\definecolor{currentstroke}{rgb}{1.000000,1.000000,1.000000}%
\pgfsetstrokecolor{currentstroke}%
\pgfsetdash{}{0pt}%
\pgfpathmoveto{\pgfqpoint{4.181134in}{1.024014in}}%
\pgfpathcurveto{\pgfqpoint{4.188267in}{1.024014in}}{\pgfqpoint{4.195109in}{1.026848in}}{\pgfqpoint{4.200152in}{1.031892in}}%
\pgfpathcurveto{\pgfqpoint{4.205196in}{1.036936in}}{\pgfqpoint{4.208030in}{1.043777in}}{\pgfqpoint{4.208030in}{1.050910in}}%
\pgfpathcurveto{\pgfqpoint{4.208030in}{1.058043in}}{\pgfqpoint{4.205196in}{1.064885in}}{\pgfqpoint{4.200152in}{1.069928in}}%
\pgfpathcurveto{\pgfqpoint{4.195109in}{1.074972in}}{\pgfqpoint{4.188267in}{1.077806in}}{\pgfqpoint{4.181134in}{1.077806in}}%
\pgfpathcurveto{\pgfqpoint{4.174001in}{1.077806in}}{\pgfqpoint{4.167160in}{1.074972in}}{\pgfqpoint{4.162116in}{1.069928in}}%
\pgfpathcurveto{\pgfqpoint{4.157072in}{1.064885in}}{\pgfqpoint{4.154238in}{1.058043in}}{\pgfqpoint{4.154238in}{1.050910in}}%
\pgfpathcurveto{\pgfqpoint{4.154238in}{1.043777in}}{\pgfqpoint{4.157072in}{1.036936in}}{\pgfqpoint{4.162116in}{1.031892in}}%
\pgfpathcurveto{\pgfqpoint{4.167160in}{1.026848in}}{\pgfqpoint{4.174001in}{1.024014in}}{\pgfqpoint{4.181134in}{1.024014in}}%
\pgfpathclose%
\pgfusepath{stroke,fill}%
\end{pgfscope}%
\begin{pgfscope}%
\pgfpathrectangle{\pgfqpoint{2.867647in}{0.500000in}}{\pgfqpoint{1.764706in}{1.700000in}}%
\pgfusepath{clip}%
\pgfsetbuttcap%
\pgfsetroundjoin%
\definecolor{currentfill}{rgb}{0.952404,0.449449,0.307210}%
\pgfsetfillcolor{currentfill}%
\pgfsetlinewidth{0.311001pt}%
\definecolor{currentstroke}{rgb}{1.000000,1.000000,1.000000}%
\pgfsetstrokecolor{currentstroke}%
\pgfsetdash{}{0pt}%
\pgfpathmoveto{\pgfqpoint{3.962845in}{1.547612in}}%
\pgfpathcurveto{\pgfqpoint{3.969978in}{1.547612in}}{\pgfqpoint{3.976820in}{1.550446in}}{\pgfqpoint{3.981863in}{1.555490in}}%
\pgfpathcurveto{\pgfqpoint{3.986907in}{1.560533in}}{\pgfqpoint{3.989741in}{1.567375in}}{\pgfqpoint{3.989741in}{1.574508in}}%
\pgfpathcurveto{\pgfqpoint{3.989741in}{1.581641in}}{\pgfqpoint{3.986907in}{1.588482in}}{\pgfqpoint{3.981863in}{1.593526in}}%
\pgfpathcurveto{\pgfqpoint{3.976820in}{1.598570in}}{\pgfqpoint{3.969978in}{1.601403in}}{\pgfqpoint{3.962845in}{1.601403in}}%
\pgfpathcurveto{\pgfqpoint{3.955712in}{1.601403in}}{\pgfqpoint{3.948871in}{1.598570in}}{\pgfqpoint{3.943827in}{1.593526in}}%
\pgfpathcurveto{\pgfqpoint{3.938783in}{1.588482in}}{\pgfqpoint{3.935949in}{1.581641in}}{\pgfqpoint{3.935949in}{1.574508in}}%
\pgfpathcurveto{\pgfqpoint{3.935949in}{1.567375in}}{\pgfqpoint{3.938783in}{1.560533in}}{\pgfqpoint{3.943827in}{1.555490in}}%
\pgfpathcurveto{\pgfqpoint{3.948871in}{1.550446in}}{\pgfqpoint{3.955712in}{1.547612in}}{\pgfqpoint{3.962845in}{1.547612in}}%
\pgfpathclose%
\pgfusepath{stroke,fill}%
\end{pgfscope}%
\begin{pgfscope}%
\pgfpathrectangle{\pgfqpoint{2.867647in}{0.500000in}}{\pgfqpoint{1.764706in}{1.700000in}}%
\pgfusepath{clip}%
\pgfsetbuttcap%
\pgfsetroundjoin%
\definecolor{currentfill}{rgb}{0.980678,0.914765,0.856766}%
\pgfsetfillcolor{currentfill}%
\pgfsetlinewidth{0.311001pt}%
\definecolor{currentstroke}{rgb}{1.000000,1.000000,1.000000}%
\pgfsetstrokecolor{currentstroke}%
\pgfsetdash{}{0pt}%
\pgfpathmoveto{\pgfqpoint{4.159748in}{1.476632in}}%
\pgfpathcurveto{\pgfqpoint{4.166881in}{1.476632in}}{\pgfqpoint{4.173723in}{1.479466in}}{\pgfqpoint{4.178767in}{1.484510in}}%
\pgfpathcurveto{\pgfqpoint{4.183810in}{1.489553in}}{\pgfqpoint{4.186644in}{1.496395in}}{\pgfqpoint{4.186644in}{1.503528in}}%
\pgfpathcurveto{\pgfqpoint{4.186644in}{1.510660in}}{\pgfqpoint{4.183810in}{1.517502in}}{\pgfqpoint{4.178767in}{1.522546in}}%
\pgfpathcurveto{\pgfqpoint{4.173723in}{1.527589in}}{\pgfqpoint{4.166881in}{1.530423in}}{\pgfqpoint{4.159748in}{1.530423in}}%
\pgfpathcurveto{\pgfqpoint{4.152616in}{1.530423in}}{\pgfqpoint{4.145774in}{1.527589in}}{\pgfqpoint{4.140730in}{1.522546in}}%
\pgfpathcurveto{\pgfqpoint{4.135687in}{1.517502in}}{\pgfqpoint{4.132853in}{1.510660in}}{\pgfqpoint{4.132853in}{1.503528in}}%
\pgfpathcurveto{\pgfqpoint{4.132853in}{1.496395in}}{\pgfqpoint{4.135687in}{1.489553in}}{\pgfqpoint{4.140730in}{1.484510in}}%
\pgfpathcurveto{\pgfqpoint{4.145774in}{1.479466in}}{\pgfqpoint{4.152616in}{1.476632in}}{\pgfqpoint{4.159748in}{1.476632in}}%
\pgfpathclose%
\pgfusepath{stroke,fill}%
\end{pgfscope}%
\begin{pgfscope}%
\pgfpathrectangle{\pgfqpoint{2.867647in}{0.500000in}}{\pgfqpoint{1.764706in}{1.700000in}}%
\pgfusepath{clip}%
\pgfsetbuttcap%
\pgfsetroundjoin%
\definecolor{currentfill}{rgb}{0.973271,0.850724,0.762998}%
\pgfsetfillcolor{currentfill}%
\pgfsetlinewidth{0.311001pt}%
\definecolor{currentstroke}{rgb}{1.000000,1.000000,1.000000}%
\pgfsetstrokecolor{currentstroke}%
\pgfsetdash{}{0pt}%
\pgfpathmoveto{\pgfqpoint{4.076998in}{1.001594in}}%
\pgfpathcurveto{\pgfqpoint{4.084131in}{1.001594in}}{\pgfqpoint{4.090972in}{1.004428in}}{\pgfqpoint{4.096016in}{1.009472in}}%
\pgfpathcurveto{\pgfqpoint{4.101060in}{1.014516in}}{\pgfqpoint{4.103893in}{1.021357in}}{\pgfqpoint{4.103893in}{1.028490in}}%
\pgfpathcurveto{\pgfqpoint{4.103893in}{1.035623in}}{\pgfqpoint{4.101060in}{1.042465in}}{\pgfqpoint{4.096016in}{1.047508in}}%
\pgfpathcurveto{\pgfqpoint{4.090972in}{1.052552in}}{\pgfqpoint{4.084131in}{1.055386in}}{\pgfqpoint{4.076998in}{1.055386in}}%
\pgfpathcurveto{\pgfqpoint{4.069865in}{1.055386in}}{\pgfqpoint{4.063023in}{1.052552in}}{\pgfqpoint{4.057980in}{1.047508in}}%
\pgfpathcurveto{\pgfqpoint{4.052936in}{1.042465in}}{\pgfqpoint{4.050102in}{1.035623in}}{\pgfqpoint{4.050102in}{1.028490in}}%
\pgfpathcurveto{\pgfqpoint{4.050102in}{1.021357in}}{\pgfqpoint{4.052936in}{1.014516in}}{\pgfqpoint{4.057980in}{1.009472in}}%
\pgfpathcurveto{\pgfqpoint{4.063023in}{1.004428in}}{\pgfqpoint{4.069865in}{1.001594in}}{\pgfqpoint{4.076998in}{1.001594in}}%
\pgfpathclose%
\pgfusepath{stroke,fill}%
\end{pgfscope}%
\begin{pgfscope}%
\pgfpathrectangle{\pgfqpoint{2.867647in}{0.500000in}}{\pgfqpoint{1.764706in}{1.700000in}}%
\pgfusepath{clip}%
\pgfsetbuttcap%
\pgfsetroundjoin%
\definecolor{currentfill}{rgb}{0.967092,0.768560,0.642079}%
\pgfsetfillcolor{currentfill}%
\pgfsetlinewidth{0.311001pt}%
\definecolor{currentstroke}{rgb}{1.000000,1.000000,1.000000}%
\pgfsetstrokecolor{currentstroke}%
\pgfsetdash{}{0pt}%
\pgfpathmoveto{\pgfqpoint{4.086946in}{1.246769in}}%
\pgfpathcurveto{\pgfqpoint{4.094079in}{1.246769in}}{\pgfqpoint{4.100920in}{1.249602in}}{\pgfqpoint{4.105964in}{1.254646in}}%
\pgfpathcurveto{\pgfqpoint{4.111008in}{1.259690in}}{\pgfqpoint{4.113842in}{1.266531in}}{\pgfqpoint{4.113842in}{1.273664in}}%
\pgfpathcurveto{\pgfqpoint{4.113842in}{1.280797in}}{\pgfqpoint{4.111008in}{1.287639in}}{\pgfqpoint{4.105964in}{1.292682in}}%
\pgfpathcurveto{\pgfqpoint{4.100920in}{1.297726in}}{\pgfqpoint{4.094079in}{1.300560in}}{\pgfqpoint{4.086946in}{1.300560in}}%
\pgfpathcurveto{\pgfqpoint{4.079813in}{1.300560in}}{\pgfqpoint{4.072971in}{1.297726in}}{\pgfqpoint{4.067928in}{1.292682in}}%
\pgfpathcurveto{\pgfqpoint{4.062884in}{1.287639in}}{\pgfqpoint{4.060050in}{1.280797in}}{\pgfqpoint{4.060050in}{1.273664in}}%
\pgfpathcurveto{\pgfqpoint{4.060050in}{1.266531in}}{\pgfqpoint{4.062884in}{1.259690in}}{\pgfqpoint{4.067928in}{1.254646in}}%
\pgfpathcurveto{\pgfqpoint{4.072971in}{1.249602in}}{\pgfqpoint{4.079813in}{1.246769in}}{\pgfqpoint{4.086946in}{1.246769in}}%
\pgfpathclose%
\pgfusepath{stroke,fill}%
\end{pgfscope}%
\begin{pgfscope}%
\pgfpathrectangle{\pgfqpoint{2.867647in}{0.500000in}}{\pgfqpoint{1.764706in}{1.700000in}}%
\pgfusepath{clip}%
\pgfsetbuttcap%
\pgfsetroundjoin%
\definecolor{currentfill}{rgb}{0.977657,0.891500,0.822809}%
\pgfsetfillcolor{currentfill}%
\pgfsetlinewidth{0.311001pt}%
\definecolor{currentstroke}{rgb}{1.000000,1.000000,1.000000}%
\pgfsetstrokecolor{currentstroke}%
\pgfsetdash{}{0pt}%
\pgfpathmoveto{\pgfqpoint{4.104668in}{1.564919in}}%
\pgfpathcurveto{\pgfqpoint{4.111801in}{1.564919in}}{\pgfqpoint{4.118643in}{1.567753in}}{\pgfqpoint{4.123686in}{1.572797in}}%
\pgfpathcurveto{\pgfqpoint{4.128730in}{1.577841in}}{\pgfqpoint{4.131564in}{1.584682in}}{\pgfqpoint{4.131564in}{1.591815in}}%
\pgfpathcurveto{\pgfqpoint{4.131564in}{1.598948in}}{\pgfqpoint{4.128730in}{1.605789in}}{\pgfqpoint{4.123686in}{1.610833in}}%
\pgfpathcurveto{\pgfqpoint{4.118643in}{1.615877in}}{\pgfqpoint{4.111801in}{1.618711in}}{\pgfqpoint{4.104668in}{1.618711in}}%
\pgfpathcurveto{\pgfqpoint{4.097535in}{1.618711in}}{\pgfqpoint{4.090694in}{1.615877in}}{\pgfqpoint{4.085650in}{1.610833in}}%
\pgfpathcurveto{\pgfqpoint{4.080606in}{1.605789in}}{\pgfqpoint{4.077772in}{1.598948in}}{\pgfqpoint{4.077772in}{1.591815in}}%
\pgfpathcurveto{\pgfqpoint{4.077772in}{1.584682in}}{\pgfqpoint{4.080606in}{1.577841in}}{\pgfqpoint{4.085650in}{1.572797in}}%
\pgfpathcurveto{\pgfqpoint{4.090694in}{1.567753in}}{\pgfqpoint{4.097535in}{1.564919in}}{\pgfqpoint{4.104668in}{1.564919in}}%
\pgfpathclose%
\pgfusepath{stroke,fill}%
\end{pgfscope}%
\begin{pgfscope}%
\pgfpathrectangle{\pgfqpoint{2.867647in}{0.500000in}}{\pgfqpoint{1.764706in}{1.700000in}}%
\pgfusepath{clip}%
\pgfsetbuttcap%
\pgfsetroundjoin%
\definecolor{currentfill}{rgb}{0.870791,0.179821,0.267974}%
\pgfsetfillcolor{currentfill}%
\pgfsetlinewidth{0.311001pt}%
\definecolor{currentstroke}{rgb}{1.000000,1.000000,1.000000}%
\pgfsetstrokecolor{currentstroke}%
\pgfsetdash{}{0pt}%
\pgfpathmoveto{\pgfqpoint{3.972228in}{1.393495in}}%
\pgfpathcurveto{\pgfqpoint{3.979360in}{1.393495in}}{\pgfqpoint{3.986202in}{1.396329in}}{\pgfqpoint{3.991246in}{1.401373in}}%
\pgfpathcurveto{\pgfqpoint{3.996289in}{1.406417in}}{\pgfqpoint{3.999123in}{1.413258in}}{\pgfqpoint{3.999123in}{1.420391in}}%
\pgfpathcurveto{\pgfqpoint{3.999123in}{1.427524in}}{\pgfqpoint{3.996289in}{1.434366in}}{\pgfqpoint{3.991246in}{1.439409in}}%
\pgfpathcurveto{\pgfqpoint{3.986202in}{1.444453in}}{\pgfqpoint{3.979360in}{1.447287in}}{\pgfqpoint{3.972228in}{1.447287in}}%
\pgfpathcurveto{\pgfqpoint{3.965095in}{1.447287in}}{\pgfqpoint{3.958253in}{1.444453in}}{\pgfqpoint{3.953209in}{1.439409in}}%
\pgfpathcurveto{\pgfqpoint{3.948166in}{1.434366in}}{\pgfqpoint{3.945332in}{1.427524in}}{\pgfqpoint{3.945332in}{1.420391in}}%
\pgfpathcurveto{\pgfqpoint{3.945332in}{1.413258in}}{\pgfqpoint{3.948166in}{1.406417in}}{\pgfqpoint{3.953209in}{1.401373in}}%
\pgfpathcurveto{\pgfqpoint{3.958253in}{1.396329in}}{\pgfqpoint{3.965095in}{1.393495in}}{\pgfqpoint{3.972228in}{1.393495in}}%
\pgfpathclose%
\pgfusepath{stroke,fill}%
\end{pgfscope}%
\begin{pgfscope}%
\pgfpathrectangle{\pgfqpoint{2.867647in}{0.500000in}}{\pgfqpoint{1.764706in}{1.700000in}}%
\pgfusepath{clip}%
\pgfsetbuttcap%
\pgfsetroundjoin%
\definecolor{currentfill}{rgb}{0.972726,0.844889,0.754401}%
\pgfsetfillcolor{currentfill}%
\pgfsetlinewidth{0.311001pt}%
\definecolor{currentstroke}{rgb}{1.000000,1.000000,1.000000}%
\pgfsetstrokecolor{currentstroke}%
\pgfsetdash{}{0pt}%
\pgfpathmoveto{\pgfqpoint{4.117156in}{1.376457in}}%
\pgfpathcurveto{\pgfqpoint{4.124289in}{1.376457in}}{\pgfqpoint{4.131131in}{1.379290in}}{\pgfqpoint{4.136175in}{1.384334in}}%
\pgfpathcurveto{\pgfqpoint{4.141218in}{1.389378in}}{\pgfqpoint{4.144052in}{1.396219in}}{\pgfqpoint{4.144052in}{1.403352in}}%
\pgfpathcurveto{\pgfqpoint{4.144052in}{1.410485in}}{\pgfqpoint{4.141218in}{1.417327in}}{\pgfqpoint{4.136175in}{1.422370in}}%
\pgfpathcurveto{\pgfqpoint{4.131131in}{1.427414in}}{\pgfqpoint{4.124289in}{1.430248in}}{\pgfqpoint{4.117156in}{1.430248in}}%
\pgfpathcurveto{\pgfqpoint{4.110024in}{1.430248in}}{\pgfqpoint{4.103182in}{1.427414in}}{\pgfqpoint{4.098138in}{1.422370in}}%
\pgfpathcurveto{\pgfqpoint{4.093095in}{1.417327in}}{\pgfqpoint{4.090261in}{1.410485in}}{\pgfqpoint{4.090261in}{1.403352in}}%
\pgfpathcurveto{\pgfqpoint{4.090261in}{1.396219in}}{\pgfqpoint{4.093095in}{1.389378in}}{\pgfqpoint{4.098138in}{1.384334in}}%
\pgfpathcurveto{\pgfqpoint{4.103182in}{1.379290in}}{\pgfqpoint{4.110024in}{1.376457in}}{\pgfqpoint{4.117156in}{1.376457in}}%
\pgfpathclose%
\pgfusepath{stroke,fill}%
\end{pgfscope}%
\begin{pgfscope}%
\pgfpathrectangle{\pgfqpoint{2.867647in}{0.500000in}}{\pgfqpoint{1.764706in}{1.700000in}}%
\pgfusepath{clip}%
\pgfsetbuttcap%
\pgfsetroundjoin%
\definecolor{currentfill}{rgb}{0.975644,0.874038,0.797253}%
\pgfsetfillcolor{currentfill}%
\pgfsetlinewidth{0.311001pt}%
\definecolor{currentstroke}{rgb}{1.000000,1.000000,1.000000}%
\pgfsetstrokecolor{currentstroke}%
\pgfsetdash{}{0pt}%
\pgfpathmoveto{\pgfqpoint{4.093102in}{1.067992in}}%
\pgfpathcurveto{\pgfqpoint{4.100235in}{1.067992in}}{\pgfqpoint{4.107076in}{1.070826in}}{\pgfqpoint{4.112120in}{1.075870in}}%
\pgfpathcurveto{\pgfqpoint{4.117164in}{1.080914in}}{\pgfqpoint{4.119998in}{1.087755in}}{\pgfqpoint{4.119998in}{1.094888in}}%
\pgfpathcurveto{\pgfqpoint{4.119998in}{1.102021in}}{\pgfqpoint{4.117164in}{1.108863in}}{\pgfqpoint{4.112120in}{1.113906in}}%
\pgfpathcurveto{\pgfqpoint{4.107076in}{1.118950in}}{\pgfqpoint{4.100235in}{1.121784in}}{\pgfqpoint{4.093102in}{1.121784in}}%
\pgfpathcurveto{\pgfqpoint{4.085969in}{1.121784in}}{\pgfqpoint{4.079127in}{1.118950in}}{\pgfqpoint{4.074084in}{1.113906in}}%
\pgfpathcurveto{\pgfqpoint{4.069040in}{1.108863in}}{\pgfqpoint{4.066206in}{1.102021in}}{\pgfqpoint{4.066206in}{1.094888in}}%
\pgfpathcurveto{\pgfqpoint{4.066206in}{1.087755in}}{\pgfqpoint{4.069040in}{1.080914in}}{\pgfqpoint{4.074084in}{1.075870in}}%
\pgfpathcurveto{\pgfqpoint{4.079127in}{1.070826in}}{\pgfqpoint{4.085969in}{1.067992in}}{\pgfqpoint{4.093102in}{1.067992in}}%
\pgfpathclose%
\pgfusepath{stroke,fill}%
\end{pgfscope}%
\begin{pgfscope}%
\pgfpathrectangle{\pgfqpoint{2.867647in}{0.500000in}}{\pgfqpoint{1.764706in}{1.700000in}}%
\pgfusepath{clip}%
\pgfsetbuttcap%
\pgfsetroundjoin%
\definecolor{currentfill}{rgb}{0.966560,0.756582,0.625273}%
\pgfsetfillcolor{currentfill}%
\pgfsetlinewidth{0.311001pt}%
\definecolor{currentstroke}{rgb}{1.000000,1.000000,1.000000}%
\pgfsetstrokecolor{currentstroke}%
\pgfsetdash{}{0pt}%
\pgfpathmoveto{\pgfqpoint{4.013906in}{0.972035in}}%
\pgfpathcurveto{\pgfqpoint{4.021039in}{0.972035in}}{\pgfqpoint{4.027881in}{0.974869in}}{\pgfqpoint{4.032924in}{0.979913in}}%
\pgfpathcurveto{\pgfqpoint{4.037968in}{0.984956in}}{\pgfqpoint{4.040802in}{0.991798in}}{\pgfqpoint{4.040802in}{0.998931in}}%
\pgfpathcurveto{\pgfqpoint{4.040802in}{1.006064in}}{\pgfqpoint{4.037968in}{1.012905in}}{\pgfqpoint{4.032924in}{1.017949in}}%
\pgfpathcurveto{\pgfqpoint{4.027881in}{1.022993in}}{\pgfqpoint{4.021039in}{1.025826in}}{\pgfqpoint{4.013906in}{1.025826in}}%
\pgfpathcurveto{\pgfqpoint{4.006773in}{1.025826in}}{\pgfqpoint{3.999932in}{1.022993in}}{\pgfqpoint{3.994888in}{1.017949in}}%
\pgfpathcurveto{\pgfqpoint{3.989844in}{1.012905in}}{\pgfqpoint{3.987011in}{1.006064in}}{\pgfqpoint{3.987011in}{0.998931in}}%
\pgfpathcurveto{\pgfqpoint{3.987011in}{0.991798in}}{\pgfqpoint{3.989844in}{0.984956in}}{\pgfqpoint{3.994888in}{0.979913in}}%
\pgfpathcurveto{\pgfqpoint{3.999932in}{0.974869in}}{\pgfqpoint{4.006773in}{0.972035in}}{\pgfqpoint{4.013906in}{0.972035in}}%
\pgfpathclose%
\pgfusepath{stroke,fill}%
\end{pgfscope}%
\begin{pgfscope}%
\pgfpathrectangle{\pgfqpoint{2.867647in}{0.500000in}}{\pgfqpoint{1.764706in}{1.700000in}}%
\pgfusepath{clip}%
\pgfsetbuttcap%
\pgfsetroundjoin%
\definecolor{currentfill}{rgb}{0.972726,0.844889,0.754401}%
\pgfsetfillcolor{currentfill}%
\pgfsetlinewidth{0.311001pt}%
\definecolor{currentstroke}{rgb}{1.000000,1.000000,1.000000}%
\pgfsetstrokecolor{currentstroke}%
\pgfsetdash{}{0pt}%
\pgfpathmoveto{\pgfqpoint{4.261216in}{1.277942in}}%
\pgfpathcurveto{\pgfqpoint{4.268349in}{1.277942in}}{\pgfqpoint{4.275191in}{1.280776in}}{\pgfqpoint{4.280234in}{1.285820in}}%
\pgfpathcurveto{\pgfqpoint{4.285278in}{1.290863in}}{\pgfqpoint{4.288112in}{1.297705in}}{\pgfqpoint{4.288112in}{1.304838in}}%
\pgfpathcurveto{\pgfqpoint{4.288112in}{1.311971in}}{\pgfqpoint{4.285278in}{1.318812in}}{\pgfqpoint{4.280234in}{1.323856in}}%
\pgfpathcurveto{\pgfqpoint{4.275191in}{1.328900in}}{\pgfqpoint{4.268349in}{1.331734in}}{\pgfqpoint{4.261216in}{1.331734in}}%
\pgfpathcurveto{\pgfqpoint{4.254083in}{1.331734in}}{\pgfqpoint{4.247242in}{1.328900in}}{\pgfqpoint{4.242198in}{1.323856in}}%
\pgfpathcurveto{\pgfqpoint{4.237154in}{1.318812in}}{\pgfqpoint{4.234320in}{1.311971in}}{\pgfqpoint{4.234320in}{1.304838in}}%
\pgfpathcurveto{\pgfqpoint{4.234320in}{1.297705in}}{\pgfqpoint{4.237154in}{1.290863in}}{\pgfqpoint{4.242198in}{1.285820in}}%
\pgfpathcurveto{\pgfqpoint{4.247242in}{1.280776in}}{\pgfqpoint{4.254083in}{1.277942in}}{\pgfqpoint{4.261216in}{1.277942in}}%
\pgfpathclose%
\pgfusepath{stroke,fill}%
\end{pgfscope}%
\begin{pgfscope}%
\pgfpathrectangle{\pgfqpoint{2.867647in}{0.500000in}}{\pgfqpoint{1.764706in}{1.700000in}}%
\pgfusepath{clip}%
\pgfsetbuttcap%
\pgfsetroundjoin%
\definecolor{currentfill}{rgb}{0.966328,0.750560,0.616961}%
\pgfsetfillcolor{currentfill}%
\pgfsetlinewidth{0.311001pt}%
\definecolor{currentstroke}{rgb}{1.000000,1.000000,1.000000}%
\pgfsetstrokecolor{currentstroke}%
\pgfsetdash{}{0pt}%
\pgfpathmoveto{\pgfqpoint{4.116410in}{0.923104in}}%
\pgfpathcurveto{\pgfqpoint{4.123543in}{0.923104in}}{\pgfqpoint{4.130385in}{0.925938in}}{\pgfqpoint{4.135429in}{0.930982in}}%
\pgfpathcurveto{\pgfqpoint{4.140472in}{0.936025in}}{\pgfqpoint{4.143306in}{0.942867in}}{\pgfqpoint{4.143306in}{0.950000in}}%
\pgfpathcurveto{\pgfqpoint{4.143306in}{0.957133in}}{\pgfqpoint{4.140472in}{0.963974in}}{\pgfqpoint{4.135429in}{0.969018in}}%
\pgfpathcurveto{\pgfqpoint{4.130385in}{0.974062in}}{\pgfqpoint{4.123543in}{0.976896in}}{\pgfqpoint{4.116410in}{0.976896in}}%
\pgfpathcurveto{\pgfqpoint{4.109278in}{0.976896in}}{\pgfqpoint{4.102436in}{0.974062in}}{\pgfqpoint{4.097392in}{0.969018in}}%
\pgfpathcurveto{\pgfqpoint{4.092349in}{0.963974in}}{\pgfqpoint{4.089515in}{0.957133in}}{\pgfqpoint{4.089515in}{0.950000in}}%
\pgfpathcurveto{\pgfqpoint{4.089515in}{0.942867in}}{\pgfqpoint{4.092349in}{0.936025in}}{\pgfqpoint{4.097392in}{0.930982in}}%
\pgfpathcurveto{\pgfqpoint{4.102436in}{0.925938in}}{\pgfqpoint{4.109278in}{0.923104in}}{\pgfqpoint{4.116410in}{0.923104in}}%
\pgfpathclose%
\pgfusepath{stroke,fill}%
\end{pgfscope}%
\begin{pgfscope}%
\pgfpathrectangle{\pgfqpoint{2.867647in}{0.500000in}}{\pgfqpoint{1.764706in}{1.700000in}}%
\pgfusepath{clip}%
\pgfsetbuttcap%
\pgfsetroundjoin%
\definecolor{currentfill}{rgb}{0.966120,0.744512,0.608720}%
\pgfsetfillcolor{currentfill}%
\pgfsetlinewidth{0.311001pt}%
\definecolor{currentstroke}{rgb}{1.000000,1.000000,1.000000}%
\pgfsetstrokecolor{currentstroke}%
\pgfsetdash{}{0pt}%
\pgfpathmoveto{\pgfqpoint{4.091208in}{0.909692in}}%
\pgfpathcurveto{\pgfqpoint{4.098341in}{0.909692in}}{\pgfqpoint{4.105183in}{0.912526in}}{\pgfqpoint{4.110226in}{0.917570in}}%
\pgfpathcurveto{\pgfqpoint{4.115270in}{0.922613in}}{\pgfqpoint{4.118104in}{0.929455in}}{\pgfqpoint{4.118104in}{0.936588in}}%
\pgfpathcurveto{\pgfqpoint{4.118104in}{0.943721in}}{\pgfqpoint{4.115270in}{0.950562in}}{\pgfqpoint{4.110226in}{0.955606in}}%
\pgfpathcurveto{\pgfqpoint{4.105183in}{0.960650in}}{\pgfqpoint{4.098341in}{0.963484in}}{\pgfqpoint{4.091208in}{0.963484in}}%
\pgfpathcurveto{\pgfqpoint{4.084075in}{0.963484in}}{\pgfqpoint{4.077234in}{0.960650in}}{\pgfqpoint{4.072190in}{0.955606in}}%
\pgfpathcurveto{\pgfqpoint{4.067146in}{0.950562in}}{\pgfqpoint{4.064312in}{0.943721in}}{\pgfqpoint{4.064312in}{0.936588in}}%
\pgfpathcurveto{\pgfqpoint{4.064312in}{0.929455in}}{\pgfqpoint{4.067146in}{0.922613in}}{\pgfqpoint{4.072190in}{0.917570in}}%
\pgfpathcurveto{\pgfqpoint{4.077234in}{0.912526in}}{\pgfqpoint{4.084075in}{0.909692in}}{\pgfqpoint{4.091208in}{0.909692in}}%
\pgfpathclose%
\pgfusepath{stroke,fill}%
\end{pgfscope}%
\begin{pgfscope}%
\pgfpathrectangle{\pgfqpoint{2.867647in}{0.500000in}}{\pgfqpoint{1.764706in}{1.700000in}}%
\pgfusepath{clip}%
\pgfsetbuttcap%
\pgfsetroundjoin%
\definecolor{currentfill}{rgb}{0.970718,0.821518,0.719872}%
\pgfsetfillcolor{currentfill}%
\pgfsetlinewidth{0.311001pt}%
\definecolor{currentstroke}{rgb}{1.000000,1.000000,1.000000}%
\pgfsetstrokecolor{currentstroke}%
\pgfsetdash{}{0pt}%
\pgfpathmoveto{\pgfqpoint{4.051126in}{1.651183in}}%
\pgfpathcurveto{\pgfqpoint{4.058259in}{1.651183in}}{\pgfqpoint{4.065101in}{1.654017in}}{\pgfqpoint{4.070144in}{1.659061in}}%
\pgfpathcurveto{\pgfqpoint{4.075188in}{1.664105in}}{\pgfqpoint{4.078022in}{1.670946in}}{\pgfqpoint{4.078022in}{1.678079in}}%
\pgfpathcurveto{\pgfqpoint{4.078022in}{1.685212in}}{\pgfqpoint{4.075188in}{1.692054in}}{\pgfqpoint{4.070144in}{1.697097in}}%
\pgfpathcurveto{\pgfqpoint{4.065101in}{1.702141in}}{\pgfqpoint{4.058259in}{1.704975in}}{\pgfqpoint{4.051126in}{1.704975in}}%
\pgfpathcurveto{\pgfqpoint{4.043993in}{1.704975in}}{\pgfqpoint{4.037152in}{1.702141in}}{\pgfqpoint{4.032108in}{1.697097in}}%
\pgfpathcurveto{\pgfqpoint{4.027064in}{1.692054in}}{\pgfqpoint{4.024230in}{1.685212in}}{\pgfqpoint{4.024230in}{1.678079in}}%
\pgfpathcurveto{\pgfqpoint{4.024230in}{1.670946in}}{\pgfqpoint{4.027064in}{1.664105in}}{\pgfqpoint{4.032108in}{1.659061in}}%
\pgfpathcurveto{\pgfqpoint{4.037152in}{1.654017in}}{\pgfqpoint{4.043993in}{1.651183in}}{\pgfqpoint{4.051126in}{1.651183in}}%
\pgfpathclose%
\pgfusepath{stroke,fill}%
\end{pgfscope}%
\begin{pgfscope}%
\pgfpathrectangle{\pgfqpoint{2.867647in}{0.500000in}}{\pgfqpoint{1.764706in}{1.700000in}}%
\pgfusepath{clip}%
\pgfsetbuttcap%
\pgfsetroundjoin%
\definecolor{currentfill}{rgb}{0.970255,0.815666,0.711203}%
\pgfsetfillcolor{currentfill}%
\pgfsetlinewidth{0.311001pt}%
\definecolor{currentstroke}{rgb}{1.000000,1.000000,1.000000}%
\pgfsetstrokecolor{currentstroke}%
\pgfsetdash{}{0pt}%
\pgfpathmoveto{\pgfqpoint{4.268731in}{1.392351in}}%
\pgfpathcurveto{\pgfqpoint{4.275864in}{1.392351in}}{\pgfqpoint{4.282705in}{1.395185in}}{\pgfqpoint{4.287749in}{1.400229in}}%
\pgfpathcurveto{\pgfqpoint{4.292793in}{1.405272in}}{\pgfqpoint{4.295627in}{1.412114in}}{\pgfqpoint{4.295627in}{1.419247in}}%
\pgfpathcurveto{\pgfqpoint{4.295627in}{1.426380in}}{\pgfqpoint{4.292793in}{1.433221in}}{\pgfqpoint{4.287749in}{1.438265in}}%
\pgfpathcurveto{\pgfqpoint{4.282705in}{1.443309in}}{\pgfqpoint{4.275864in}{1.446143in}}{\pgfqpoint{4.268731in}{1.446143in}}%
\pgfpathcurveto{\pgfqpoint{4.261598in}{1.446143in}}{\pgfqpoint{4.254757in}{1.443309in}}{\pgfqpoint{4.249713in}{1.438265in}}%
\pgfpathcurveto{\pgfqpoint{4.244669in}{1.433221in}}{\pgfqpoint{4.241835in}{1.426380in}}{\pgfqpoint{4.241835in}{1.419247in}}%
\pgfpathcurveto{\pgfqpoint{4.241835in}{1.412114in}}{\pgfqpoint{4.244669in}{1.405272in}}{\pgfqpoint{4.249713in}{1.400229in}}%
\pgfpathcurveto{\pgfqpoint{4.254757in}{1.395185in}}{\pgfqpoint{4.261598in}{1.392351in}}{\pgfqpoint{4.268731in}{1.392351in}}%
\pgfpathclose%
\pgfusepath{stroke,fill}%
\end{pgfscope}%
\begin{pgfscope}%
\pgfpathrectangle{\pgfqpoint{2.867647in}{0.500000in}}{\pgfqpoint{1.764706in}{1.700000in}}%
\pgfusepath{clip}%
\pgfsetbuttcap%
\pgfsetroundjoin%
\definecolor{currentfill}{rgb}{0.981377,0.920617,0.865369}%
\pgfsetfillcolor{currentfill}%
\pgfsetlinewidth{0.311001pt}%
\definecolor{currentstroke}{rgb}{1.000000,1.000000,1.000000}%
\pgfsetstrokecolor{currentstroke}%
\pgfsetdash{}{0pt}%
\pgfpathmoveto{\pgfqpoint{4.189085in}{1.264972in}}%
\pgfpathcurveto{\pgfqpoint{4.196217in}{1.264972in}}{\pgfqpoint{4.203059in}{1.267806in}}{\pgfqpoint{4.208103in}{1.272850in}}%
\pgfpathcurveto{\pgfqpoint{4.213146in}{1.277893in}}{\pgfqpoint{4.215980in}{1.284735in}}{\pgfqpoint{4.215980in}{1.291868in}}%
\pgfpathcurveto{\pgfqpoint{4.215980in}{1.299001in}}{\pgfqpoint{4.213146in}{1.305842in}}{\pgfqpoint{4.208103in}{1.310886in}}%
\pgfpathcurveto{\pgfqpoint{4.203059in}{1.315930in}}{\pgfqpoint{4.196217in}{1.318764in}}{\pgfqpoint{4.189085in}{1.318764in}}%
\pgfpathcurveto{\pgfqpoint{4.181952in}{1.318764in}}{\pgfqpoint{4.175110in}{1.315930in}}{\pgfqpoint{4.170067in}{1.310886in}}%
\pgfpathcurveto{\pgfqpoint{4.165023in}{1.305842in}}{\pgfqpoint{4.162189in}{1.299001in}}{\pgfqpoint{4.162189in}{1.291868in}}%
\pgfpathcurveto{\pgfqpoint{4.162189in}{1.284735in}}{\pgfqpoint{4.165023in}{1.277893in}}{\pgfqpoint{4.170067in}{1.272850in}}%
\pgfpathcurveto{\pgfqpoint{4.175110in}{1.267806in}}{\pgfqpoint{4.181952in}{1.264972in}}{\pgfqpoint{4.189085in}{1.264972in}}%
\pgfpathclose%
\pgfusepath{stroke,fill}%
\end{pgfscope}%
\begin{pgfscope}%
\pgfpathrectangle{\pgfqpoint{2.867647in}{0.500000in}}{\pgfqpoint{1.764706in}{1.700000in}}%
\pgfusepath{clip}%
\pgfsetbuttcap%
\pgfsetroundjoin%
\definecolor{currentfill}{rgb}{0.823415,0.125353,0.296370}%
\pgfsetfillcolor{currentfill}%
\pgfsetlinewidth{0.311001pt}%
\definecolor{currentstroke}{rgb}{1.000000,1.000000,1.000000}%
\pgfsetstrokecolor{currentstroke}%
\pgfsetdash{}{0pt}%
\pgfpathmoveto{\pgfqpoint{4.342473in}{0.989155in}}%
\pgfpathcurveto{\pgfqpoint{4.349606in}{0.989155in}}{\pgfqpoint{4.356448in}{0.991989in}}{\pgfqpoint{4.361492in}{0.997032in}}%
\pgfpathcurveto{\pgfqpoint{4.366535in}{1.002076in}}{\pgfqpoint{4.369369in}{1.008918in}}{\pgfqpoint{4.369369in}{1.016051in}}%
\pgfpathcurveto{\pgfqpoint{4.369369in}{1.023183in}}{\pgfqpoint{4.366535in}{1.030025in}}{\pgfqpoint{4.361492in}{1.035069in}}%
\pgfpathcurveto{\pgfqpoint{4.356448in}{1.040112in}}{\pgfqpoint{4.349606in}{1.042946in}}{\pgfqpoint{4.342473in}{1.042946in}}%
\pgfpathcurveto{\pgfqpoint{4.335341in}{1.042946in}}{\pgfqpoint{4.328499in}{1.040112in}}{\pgfqpoint{4.323455in}{1.035069in}}%
\pgfpathcurveto{\pgfqpoint{4.318412in}{1.030025in}}{\pgfqpoint{4.315578in}{1.023183in}}{\pgfqpoint{4.315578in}{1.016051in}}%
\pgfpathcurveto{\pgfqpoint{4.315578in}{1.008918in}}{\pgfqpoint{4.318412in}{1.002076in}}{\pgfqpoint{4.323455in}{0.997032in}}%
\pgfpathcurveto{\pgfqpoint{4.328499in}{0.991989in}}{\pgfqpoint{4.335341in}{0.989155in}}{\pgfqpoint{4.342473in}{0.989155in}}%
\pgfpathclose%
\pgfusepath{stroke,fill}%
\end{pgfscope}%
\begin{pgfscope}%
\pgfpathrectangle{\pgfqpoint{2.867647in}{0.500000in}}{\pgfqpoint{1.764706in}{1.700000in}}%
\pgfusepath{clip}%
\pgfsetbuttcap%
\pgfsetroundjoin%
\definecolor{currentfill}{rgb}{0.978376,0.897317,0.831308}%
\pgfsetfillcolor{currentfill}%
\pgfsetlinewidth{0.311001pt}%
\definecolor{currentstroke}{rgb}{1.000000,1.000000,1.000000}%
\pgfsetstrokecolor{currentstroke}%
\pgfsetdash{}{0pt}%
\pgfpathmoveto{\pgfqpoint{4.145590in}{1.254437in}}%
\pgfpathcurveto{\pgfqpoint{4.152722in}{1.254437in}}{\pgfqpoint{4.159564in}{1.257271in}}{\pgfqpoint{4.164608in}{1.262314in}}%
\pgfpathcurveto{\pgfqpoint{4.169651in}{1.267358in}}{\pgfqpoint{4.172485in}{1.274200in}}{\pgfqpoint{4.172485in}{1.281332in}}%
\pgfpathcurveto{\pgfqpoint{4.172485in}{1.288465in}}{\pgfqpoint{4.169651in}{1.295307in}}{\pgfqpoint{4.164608in}{1.300351in}}%
\pgfpathcurveto{\pgfqpoint{4.159564in}{1.305394in}}{\pgfqpoint{4.152722in}{1.308228in}}{\pgfqpoint{4.145590in}{1.308228in}}%
\pgfpathcurveto{\pgfqpoint{4.138457in}{1.308228in}}{\pgfqpoint{4.131615in}{1.305394in}}{\pgfqpoint{4.126571in}{1.300351in}}%
\pgfpathcurveto{\pgfqpoint{4.121528in}{1.295307in}}{\pgfqpoint{4.118694in}{1.288465in}}{\pgfqpoint{4.118694in}{1.281332in}}%
\pgfpathcurveto{\pgfqpoint{4.118694in}{1.274200in}}{\pgfqpoint{4.121528in}{1.267358in}}{\pgfqpoint{4.126571in}{1.262314in}}%
\pgfpathcurveto{\pgfqpoint{4.131615in}{1.257271in}}{\pgfqpoint{4.138457in}{1.254437in}}{\pgfqpoint{4.145590in}{1.254437in}}%
\pgfpathclose%
\pgfusepath{stroke,fill}%
\end{pgfscope}%
\begin{pgfscope}%
\pgfpathrectangle{\pgfqpoint{2.867647in}{0.500000in}}{\pgfqpoint{1.764706in}{1.700000in}}%
\pgfusepath{clip}%
\pgfsetbuttcap%
\pgfsetroundjoin%
\definecolor{currentfill}{rgb}{0.977657,0.891500,0.822809}%
\pgfsetfillcolor{currentfill}%
\pgfsetlinewidth{0.311001pt}%
\definecolor{currentstroke}{rgb}{1.000000,1.000000,1.000000}%
\pgfsetstrokecolor{currentstroke}%
\pgfsetdash{}{0pt}%
\pgfpathmoveto{\pgfqpoint{4.191219in}{1.536971in}}%
\pgfpathcurveto{\pgfqpoint{4.198352in}{1.536971in}}{\pgfqpoint{4.205193in}{1.539804in}}{\pgfqpoint{4.210237in}{1.544848in}}%
\pgfpathcurveto{\pgfqpoint{4.215281in}{1.549892in}}{\pgfqpoint{4.218114in}{1.556733in}}{\pgfqpoint{4.218114in}{1.563866in}}%
\pgfpathcurveto{\pgfqpoint{4.218114in}{1.570999in}}{\pgfqpoint{4.215281in}{1.577841in}}{\pgfqpoint{4.210237in}{1.582884in}}%
\pgfpathcurveto{\pgfqpoint{4.205193in}{1.587928in}}{\pgfqpoint{4.198352in}{1.590762in}}{\pgfqpoint{4.191219in}{1.590762in}}%
\pgfpathcurveto{\pgfqpoint{4.184086in}{1.590762in}}{\pgfqpoint{4.177244in}{1.587928in}}{\pgfqpoint{4.172201in}{1.582884in}}%
\pgfpathcurveto{\pgfqpoint{4.167157in}{1.577841in}}{\pgfqpoint{4.164323in}{1.570999in}}{\pgfqpoint{4.164323in}{1.563866in}}%
\pgfpathcurveto{\pgfqpoint{4.164323in}{1.556733in}}{\pgfqpoint{4.167157in}{1.549892in}}{\pgfqpoint{4.172201in}{1.544848in}}%
\pgfpathcurveto{\pgfqpoint{4.177244in}{1.539804in}}{\pgfqpoint{4.184086in}{1.536971in}}{\pgfqpoint{4.191219in}{1.536971in}}%
\pgfpathclose%
\pgfusepath{stroke,fill}%
\end{pgfscope}%
\begin{pgfscope}%
\pgfpathrectangle{\pgfqpoint{2.867647in}{0.500000in}}{\pgfqpoint{1.764706in}{1.700000in}}%
\pgfusepath{clip}%
\pgfsetbuttcap%
\pgfsetroundjoin%
\definecolor{currentfill}{rgb}{0.978376,0.897317,0.831308}%
\pgfsetfillcolor{currentfill}%
\pgfsetlinewidth{0.311001pt}%
\definecolor{currentstroke}{rgb}{1.000000,1.000000,1.000000}%
\pgfsetstrokecolor{currentstroke}%
\pgfsetdash{}{0pt}%
\pgfpathmoveto{\pgfqpoint{4.215157in}{1.178824in}}%
\pgfpathcurveto{\pgfqpoint{4.222290in}{1.178824in}}{\pgfqpoint{4.229131in}{1.181658in}}{\pgfqpoint{4.234175in}{1.186701in}}%
\pgfpathcurveto{\pgfqpoint{4.239219in}{1.191745in}}{\pgfqpoint{4.242053in}{1.198587in}}{\pgfqpoint{4.242053in}{1.205719in}}%
\pgfpathcurveto{\pgfqpoint{4.242053in}{1.212852in}}{\pgfqpoint{4.239219in}{1.219694in}}{\pgfqpoint{4.234175in}{1.224738in}}%
\pgfpathcurveto{\pgfqpoint{4.229131in}{1.229781in}}{\pgfqpoint{4.222290in}{1.232615in}}{\pgfqpoint{4.215157in}{1.232615in}}%
\pgfpathcurveto{\pgfqpoint{4.208024in}{1.232615in}}{\pgfqpoint{4.201183in}{1.229781in}}{\pgfqpoint{4.196139in}{1.224738in}}%
\pgfpathcurveto{\pgfqpoint{4.191095in}{1.219694in}}{\pgfqpoint{4.188261in}{1.212852in}}{\pgfqpoint{4.188261in}{1.205719in}}%
\pgfpathcurveto{\pgfqpoint{4.188261in}{1.198587in}}{\pgfqpoint{4.191095in}{1.191745in}}{\pgfqpoint{4.196139in}{1.186701in}}%
\pgfpathcurveto{\pgfqpoint{4.201183in}{1.181658in}}{\pgfqpoint{4.208024in}{1.178824in}}{\pgfqpoint{4.215157in}{1.178824in}}%
\pgfpathclose%
\pgfusepath{stroke,fill}%
\end{pgfscope}%
\begin{pgfscope}%
\pgfpathrectangle{\pgfqpoint{2.867647in}{0.500000in}}{\pgfqpoint{1.764706in}{1.700000in}}%
\pgfusepath{clip}%
\pgfsetbuttcap%
\pgfsetroundjoin%
\definecolor{currentfill}{rgb}{0.979891,0.908948,0.848279}%
\pgfsetfillcolor{currentfill}%
\pgfsetlinewidth{0.311001pt}%
\definecolor{currentstroke}{rgb}{1.000000,1.000000,1.000000}%
\pgfsetstrokecolor{currentstroke}%
\pgfsetdash{}{0pt}%
\pgfpathmoveto{\pgfqpoint{4.161409in}{1.400615in}}%
\pgfpathcurveto{\pgfqpoint{4.168542in}{1.400615in}}{\pgfqpoint{4.175384in}{1.403448in}}{\pgfqpoint{4.180427in}{1.408492in}}%
\pgfpathcurveto{\pgfqpoint{4.185471in}{1.413536in}}{\pgfqpoint{4.188305in}{1.420377in}}{\pgfqpoint{4.188305in}{1.427510in}}%
\pgfpathcurveto{\pgfqpoint{4.188305in}{1.434643in}}{\pgfqpoint{4.185471in}{1.441485in}}{\pgfqpoint{4.180427in}{1.446528in}}%
\pgfpathcurveto{\pgfqpoint{4.175384in}{1.451572in}}{\pgfqpoint{4.168542in}{1.454406in}}{\pgfqpoint{4.161409in}{1.454406in}}%
\pgfpathcurveto{\pgfqpoint{4.154276in}{1.454406in}}{\pgfqpoint{4.147435in}{1.451572in}}{\pgfqpoint{4.142391in}{1.446528in}}%
\pgfpathcurveto{\pgfqpoint{4.137347in}{1.441485in}}{\pgfqpoint{4.134513in}{1.434643in}}{\pgfqpoint{4.134513in}{1.427510in}}%
\pgfpathcurveto{\pgfqpoint{4.134513in}{1.420377in}}{\pgfqpoint{4.137347in}{1.413536in}}{\pgfqpoint{4.142391in}{1.408492in}}%
\pgfpathcurveto{\pgfqpoint{4.147435in}{1.403448in}}{\pgfqpoint{4.154276in}{1.400615in}}{\pgfqpoint{4.161409in}{1.400615in}}%
\pgfpathclose%
\pgfusepath{stroke,fill}%
\end{pgfscope}%
\begin{pgfscope}%
\pgfpathrectangle{\pgfqpoint{2.867647in}{0.500000in}}{\pgfqpoint{1.764706in}{1.700000in}}%
\pgfusepath{clip}%
\pgfsetbuttcap%
\pgfsetroundjoin%
\definecolor{currentfill}{rgb}{0.979891,0.908948,0.848279}%
\pgfsetfillcolor{currentfill}%
\pgfsetlinewidth{0.311001pt}%
\definecolor{currentstroke}{rgb}{1.000000,1.000000,1.000000}%
\pgfsetstrokecolor{currentstroke}%
\pgfsetdash{}{0pt}%
\pgfpathmoveto{\pgfqpoint{4.204099in}{1.428364in}}%
\pgfpathcurveto{\pgfqpoint{4.211232in}{1.428364in}}{\pgfqpoint{4.218074in}{1.431198in}}{\pgfqpoint{4.223117in}{1.436242in}}%
\pgfpathcurveto{\pgfqpoint{4.228161in}{1.441286in}}{\pgfqpoint{4.230995in}{1.448127in}}{\pgfqpoint{4.230995in}{1.455260in}}%
\pgfpathcurveto{\pgfqpoint{4.230995in}{1.462393in}}{\pgfqpoint{4.228161in}{1.469235in}}{\pgfqpoint{4.223117in}{1.474278in}}%
\pgfpathcurveto{\pgfqpoint{4.218074in}{1.479322in}}{\pgfqpoint{4.211232in}{1.482156in}}{\pgfqpoint{4.204099in}{1.482156in}}%
\pgfpathcurveto{\pgfqpoint{4.196966in}{1.482156in}}{\pgfqpoint{4.190125in}{1.479322in}}{\pgfqpoint{4.185081in}{1.474278in}}%
\pgfpathcurveto{\pgfqpoint{4.180037in}{1.469235in}}{\pgfqpoint{4.177203in}{1.462393in}}{\pgfqpoint{4.177203in}{1.455260in}}%
\pgfpathcurveto{\pgfqpoint{4.177203in}{1.448127in}}{\pgfqpoint{4.180037in}{1.441286in}}{\pgfqpoint{4.185081in}{1.436242in}}%
\pgfpathcurveto{\pgfqpoint{4.190125in}{1.431198in}}{\pgfqpoint{4.196966in}{1.428364in}}{\pgfqpoint{4.204099in}{1.428364in}}%
\pgfpathclose%
\pgfusepath{stroke,fill}%
\end{pgfscope}%
\begin{pgfscope}%
\pgfpathrectangle{\pgfqpoint{2.867647in}{0.500000in}}{\pgfqpoint{1.764706in}{1.700000in}}%
\pgfusepath{clip}%
\pgfsetbuttcap%
\pgfsetroundjoin%
\definecolor{currentfill}{rgb}{0.964558,0.676556,0.522514}%
\pgfsetfillcolor{currentfill}%
\pgfsetlinewidth{0.311001pt}%
\definecolor{currentstroke}{rgb}{1.000000,1.000000,1.000000}%
\pgfsetstrokecolor{currentstroke}%
\pgfsetdash{}{0pt}%
\pgfpathmoveto{\pgfqpoint{3.970968in}{1.703814in}}%
\pgfpathcurveto{\pgfqpoint{3.978101in}{1.703814in}}{\pgfqpoint{3.984942in}{1.706648in}}{\pgfqpoint{3.989986in}{1.711691in}}%
\pgfpathcurveto{\pgfqpoint{3.995030in}{1.716735in}}{\pgfqpoint{3.997864in}{1.723577in}}{\pgfqpoint{3.997864in}{1.730709in}}%
\pgfpathcurveto{\pgfqpoint{3.997864in}{1.737842in}}{\pgfqpoint{3.995030in}{1.744684in}}{\pgfqpoint{3.989986in}{1.749728in}}%
\pgfpathcurveto{\pgfqpoint{3.984942in}{1.754771in}}{\pgfqpoint{3.978101in}{1.757605in}}{\pgfqpoint{3.970968in}{1.757605in}}%
\pgfpathcurveto{\pgfqpoint{3.963835in}{1.757605in}}{\pgfqpoint{3.956993in}{1.754771in}}{\pgfqpoint{3.951950in}{1.749728in}}%
\pgfpathcurveto{\pgfqpoint{3.946906in}{1.744684in}}{\pgfqpoint{3.944072in}{1.737842in}}{\pgfqpoint{3.944072in}{1.730709in}}%
\pgfpathcurveto{\pgfqpoint{3.944072in}{1.723577in}}{\pgfqpoint{3.946906in}{1.716735in}}{\pgfqpoint{3.951950in}{1.711691in}}%
\pgfpathcurveto{\pgfqpoint{3.956993in}{1.706648in}}{\pgfqpoint{3.963835in}{1.703814in}}{\pgfqpoint{3.970968in}{1.703814in}}%
\pgfpathclose%
\pgfusepath{stroke,fill}%
\end{pgfscope}%
\begin{pgfscope}%
\pgfpathrectangle{\pgfqpoint{2.867647in}{0.500000in}}{\pgfqpoint{1.764706in}{1.700000in}}%
\pgfusepath{clip}%
\pgfsetbuttcap%
\pgfsetroundjoin%
\definecolor{currentfill}{rgb}{0.969803,0.809811,0.702523}%
\pgfsetfillcolor{currentfill}%
\pgfsetlinewidth{0.311001pt}%
\definecolor{currentstroke}{rgb}{1.000000,1.000000,1.000000}%
\pgfsetstrokecolor{currentstroke}%
\pgfsetdash{}{0pt}%
\pgfpathmoveto{\pgfqpoint{4.080279in}{0.949751in}}%
\pgfpathcurveto{\pgfqpoint{4.087411in}{0.949751in}}{\pgfqpoint{4.094253in}{0.952585in}}{\pgfqpoint{4.099297in}{0.957629in}}%
\pgfpathcurveto{\pgfqpoint{4.104340in}{0.962673in}}{\pgfqpoint{4.107174in}{0.969514in}}{\pgfqpoint{4.107174in}{0.976647in}}%
\pgfpathcurveto{\pgfqpoint{4.107174in}{0.983780in}}{\pgfqpoint{4.104340in}{0.990622in}}{\pgfqpoint{4.099297in}{0.995665in}}%
\pgfpathcurveto{\pgfqpoint{4.094253in}{1.000709in}}{\pgfqpoint{4.087411in}{1.003543in}}{\pgfqpoint{4.080279in}{1.003543in}}%
\pgfpathcurveto{\pgfqpoint{4.073146in}{1.003543in}}{\pgfqpoint{4.066304in}{1.000709in}}{\pgfqpoint{4.061261in}{0.995665in}}%
\pgfpathcurveto{\pgfqpoint{4.056217in}{0.990622in}}{\pgfqpoint{4.053383in}{0.983780in}}{\pgfqpoint{4.053383in}{0.976647in}}%
\pgfpathcurveto{\pgfqpoint{4.053383in}{0.969514in}}{\pgfqpoint{4.056217in}{0.962673in}}{\pgfqpoint{4.061261in}{0.957629in}}%
\pgfpathcurveto{\pgfqpoint{4.066304in}{0.952585in}}{\pgfqpoint{4.073146in}{0.949751in}}{\pgfqpoint{4.080279in}{0.949751in}}%
\pgfpathclose%
\pgfusepath{stroke,fill}%
\end{pgfscope}%
\begin{pgfscope}%
\pgfpathrectangle{\pgfqpoint{2.867647in}{0.500000in}}{\pgfqpoint{1.764706in}{1.700000in}}%
\pgfusepath{clip}%
\pgfsetbuttcap%
\pgfsetroundjoin%
\definecolor{currentfill}{rgb}{0.958791,0.526283,0.368909}%
\pgfsetfillcolor{currentfill}%
\pgfsetlinewidth{0.311001pt}%
\definecolor{currentstroke}{rgb}{1.000000,1.000000,1.000000}%
\pgfsetstrokecolor{currentstroke}%
\pgfsetdash{}{0pt}%
\pgfpathmoveto{\pgfqpoint{4.018233in}{1.203469in}}%
\pgfpathcurveto{\pgfqpoint{4.025366in}{1.203469in}}{\pgfqpoint{4.032208in}{1.206303in}}{\pgfqpoint{4.037251in}{1.211347in}}%
\pgfpathcurveto{\pgfqpoint{4.042295in}{1.216390in}}{\pgfqpoint{4.045129in}{1.223232in}}{\pgfqpoint{4.045129in}{1.230365in}}%
\pgfpathcurveto{\pgfqpoint{4.045129in}{1.237498in}}{\pgfqpoint{4.042295in}{1.244339in}}{\pgfqpoint{4.037251in}{1.249383in}}%
\pgfpathcurveto{\pgfqpoint{4.032208in}{1.254427in}}{\pgfqpoint{4.025366in}{1.257260in}}{\pgfqpoint{4.018233in}{1.257260in}}%
\pgfpathcurveto{\pgfqpoint{4.011100in}{1.257260in}}{\pgfqpoint{4.004259in}{1.254427in}}{\pgfqpoint{3.999215in}{1.249383in}}%
\pgfpathcurveto{\pgfqpoint{3.994171in}{1.244339in}}{\pgfqpoint{3.991337in}{1.237498in}}{\pgfqpoint{3.991337in}{1.230365in}}%
\pgfpathcurveto{\pgfqpoint{3.991337in}{1.223232in}}{\pgfqpoint{3.994171in}{1.216390in}}{\pgfqpoint{3.999215in}{1.211347in}}%
\pgfpathcurveto{\pgfqpoint{4.004259in}{1.206303in}}{\pgfqpoint{4.011100in}{1.203469in}}{\pgfqpoint{4.018233in}{1.203469in}}%
\pgfpathclose%
\pgfusepath{stroke,fill}%
\end{pgfscope}%
\begin{pgfscope}%
\pgfpathrectangle{\pgfqpoint{2.867647in}{0.500000in}}{\pgfqpoint{1.764706in}{1.700000in}}%
\pgfusepath{clip}%
\pgfsetbuttcap%
\pgfsetroundjoin%
\definecolor{currentfill}{rgb}{0.972201,0.839051,0.745789}%
\pgfsetfillcolor{currentfill}%
\pgfsetlinewidth{0.311001pt}%
\definecolor{currentstroke}{rgb}{1.000000,1.000000,1.000000}%
\pgfsetstrokecolor{currentstroke}%
\pgfsetdash{}{0pt}%
\pgfpathmoveto{\pgfqpoint{4.073355in}{0.990541in}}%
\pgfpathcurveto{\pgfqpoint{4.080488in}{0.990541in}}{\pgfqpoint{4.087329in}{0.993374in}}{\pgfqpoint{4.092373in}{0.998418in}}%
\pgfpathcurveto{\pgfqpoint{4.097417in}{1.003462in}}{\pgfqpoint{4.100251in}{1.010303in}}{\pgfqpoint{4.100251in}{1.017436in}}%
\pgfpathcurveto{\pgfqpoint{4.100251in}{1.024569in}}{\pgfqpoint{4.097417in}{1.031411in}}{\pgfqpoint{4.092373in}{1.036454in}}%
\pgfpathcurveto{\pgfqpoint{4.087329in}{1.041498in}}{\pgfqpoint{4.080488in}{1.044332in}}{\pgfqpoint{4.073355in}{1.044332in}}%
\pgfpathcurveto{\pgfqpoint{4.066222in}{1.044332in}}{\pgfqpoint{4.059380in}{1.041498in}}{\pgfqpoint{4.054337in}{1.036454in}}%
\pgfpathcurveto{\pgfqpoint{4.049293in}{1.031411in}}{\pgfqpoint{4.046459in}{1.024569in}}{\pgfqpoint{4.046459in}{1.017436in}}%
\pgfpathcurveto{\pgfqpoint{4.046459in}{1.010303in}}{\pgfqpoint{4.049293in}{1.003462in}}{\pgfqpoint{4.054337in}{0.998418in}}%
\pgfpathcurveto{\pgfqpoint{4.059380in}{0.993374in}}{\pgfqpoint{4.066222in}{0.990541in}}{\pgfqpoint{4.073355in}{0.990541in}}%
\pgfpathclose%
\pgfusepath{stroke,fill}%
\end{pgfscope}%
\begin{pgfscope}%
\pgfpathrectangle{\pgfqpoint{2.867647in}{0.500000in}}{\pgfqpoint{1.764706in}{1.700000in}}%
\pgfusepath{clip}%
\pgfsetbuttcap%
\pgfsetroundjoin%
\definecolor{currentfill}{rgb}{0.975644,0.874038,0.797253}%
\pgfsetfillcolor{currentfill}%
\pgfsetlinewidth{0.311001pt}%
\definecolor{currentstroke}{rgb}{1.000000,1.000000,1.000000}%
\pgfsetstrokecolor{currentstroke}%
\pgfsetdash{}{0pt}%
\pgfpathmoveto{\pgfqpoint{4.235976in}{1.193756in}}%
\pgfpathcurveto{\pgfqpoint{4.243108in}{1.193756in}}{\pgfqpoint{4.249950in}{1.196590in}}{\pgfqpoint{4.254994in}{1.201634in}}%
\pgfpathcurveto{\pgfqpoint{4.260037in}{1.206678in}}{\pgfqpoint{4.262871in}{1.213519in}}{\pgfqpoint{4.262871in}{1.220652in}}%
\pgfpathcurveto{\pgfqpoint{4.262871in}{1.227785in}}{\pgfqpoint{4.260037in}{1.234627in}}{\pgfqpoint{4.254994in}{1.239670in}}%
\pgfpathcurveto{\pgfqpoint{4.249950in}{1.244714in}}{\pgfqpoint{4.243108in}{1.247548in}}{\pgfqpoint{4.235976in}{1.247548in}}%
\pgfpathcurveto{\pgfqpoint{4.228843in}{1.247548in}}{\pgfqpoint{4.222001in}{1.244714in}}{\pgfqpoint{4.216958in}{1.239670in}}%
\pgfpathcurveto{\pgfqpoint{4.211914in}{1.234627in}}{\pgfqpoint{4.209080in}{1.227785in}}{\pgfqpoint{4.209080in}{1.220652in}}%
\pgfpathcurveto{\pgfqpoint{4.209080in}{1.213519in}}{\pgfqpoint{4.211914in}{1.206678in}}{\pgfqpoint{4.216958in}{1.201634in}}%
\pgfpathcurveto{\pgfqpoint{4.222001in}{1.196590in}}{\pgfqpoint{4.228843in}{1.193756in}}{\pgfqpoint{4.235976in}{1.193756in}}%
\pgfpathclose%
\pgfusepath{stroke,fill}%
\end{pgfscope}%
\begin{pgfscope}%
\pgfpathrectangle{\pgfqpoint{2.867647in}{0.500000in}}{\pgfqpoint{1.764706in}{1.700000in}}%
\pgfusepath{clip}%
\pgfsetbuttcap%
\pgfsetroundjoin%
\definecolor{currentfill}{rgb}{0.978376,0.897317,0.831308}%
\pgfsetfillcolor{currentfill}%
\pgfsetlinewidth{0.311001pt}%
\definecolor{currentstroke}{rgb}{1.000000,1.000000,1.000000}%
\pgfsetstrokecolor{currentstroke}%
\pgfsetdash{}{0pt}%
\pgfpathmoveto{\pgfqpoint{4.129239in}{1.060411in}}%
\pgfpathcurveto{\pgfqpoint{4.136371in}{1.060411in}}{\pgfqpoint{4.143213in}{1.063245in}}{\pgfqpoint{4.148257in}{1.068288in}}%
\pgfpathcurveto{\pgfqpoint{4.153300in}{1.073332in}}{\pgfqpoint{4.156134in}{1.080174in}}{\pgfqpoint{4.156134in}{1.087307in}}%
\pgfpathcurveto{\pgfqpoint{4.156134in}{1.094439in}}{\pgfqpoint{4.153300in}{1.101281in}}{\pgfqpoint{4.148257in}{1.106325in}}%
\pgfpathcurveto{\pgfqpoint{4.143213in}{1.111368in}}{\pgfqpoint{4.136371in}{1.114202in}}{\pgfqpoint{4.129239in}{1.114202in}}%
\pgfpathcurveto{\pgfqpoint{4.122106in}{1.114202in}}{\pgfqpoint{4.115264in}{1.111368in}}{\pgfqpoint{4.110220in}{1.106325in}}%
\pgfpathcurveto{\pgfqpoint{4.105177in}{1.101281in}}{\pgfqpoint{4.102343in}{1.094439in}}{\pgfqpoint{4.102343in}{1.087307in}}%
\pgfpathcurveto{\pgfqpoint{4.102343in}{1.080174in}}{\pgfqpoint{4.105177in}{1.073332in}}{\pgfqpoint{4.110220in}{1.068288in}}%
\pgfpathcurveto{\pgfqpoint{4.115264in}{1.063245in}}{\pgfqpoint{4.122106in}{1.060411in}}{\pgfqpoint{4.129239in}{1.060411in}}%
\pgfpathclose%
\pgfusepath{stroke,fill}%
\end{pgfscope}%
\begin{pgfscope}%
\pgfpathrectangle{\pgfqpoint{2.867647in}{0.500000in}}{\pgfqpoint{1.764706in}{1.700000in}}%
\pgfusepath{clip}%
\pgfsetbuttcap%
\pgfsetroundjoin%
\definecolor{currentfill}{rgb}{0.975644,0.874038,0.797253}%
\pgfsetfillcolor{currentfill}%
\pgfsetlinewidth{0.311001pt}%
\definecolor{currentstroke}{rgb}{1.000000,1.000000,1.000000}%
\pgfsetstrokecolor{currentstroke}%
\pgfsetdash{}{0pt}%
\pgfpathmoveto{\pgfqpoint{4.243504in}{1.256459in}}%
\pgfpathcurveto{\pgfqpoint{4.250637in}{1.256459in}}{\pgfqpoint{4.257479in}{1.259293in}}{\pgfqpoint{4.262523in}{1.264337in}}%
\pgfpathcurveto{\pgfqpoint{4.267566in}{1.269380in}}{\pgfqpoint{4.270400in}{1.276222in}}{\pgfqpoint{4.270400in}{1.283355in}}%
\pgfpathcurveto{\pgfqpoint{4.270400in}{1.290488in}}{\pgfqpoint{4.267566in}{1.297329in}}{\pgfqpoint{4.262523in}{1.302373in}}%
\pgfpathcurveto{\pgfqpoint{4.257479in}{1.307417in}}{\pgfqpoint{4.250637in}{1.310250in}}{\pgfqpoint{4.243504in}{1.310250in}}%
\pgfpathcurveto{\pgfqpoint{4.236372in}{1.310250in}}{\pgfqpoint{4.229530in}{1.307417in}}{\pgfqpoint{4.224486in}{1.302373in}}%
\pgfpathcurveto{\pgfqpoint{4.219443in}{1.297329in}}{\pgfqpoint{4.216609in}{1.290488in}}{\pgfqpoint{4.216609in}{1.283355in}}%
\pgfpathcurveto{\pgfqpoint{4.216609in}{1.276222in}}{\pgfqpoint{4.219443in}{1.269380in}}{\pgfqpoint{4.224486in}{1.264337in}}%
\pgfpathcurveto{\pgfqpoint{4.229530in}{1.259293in}}{\pgfqpoint{4.236372in}{1.256459in}}{\pgfqpoint{4.243504in}{1.256459in}}%
\pgfpathclose%
\pgfusepath{stroke,fill}%
\end{pgfscope}%
\begin{pgfscope}%
\pgfpathrectangle{\pgfqpoint{2.867647in}{0.500000in}}{\pgfqpoint{1.764706in}{1.700000in}}%
\pgfusepath{clip}%
\pgfsetbuttcap%
\pgfsetroundjoin%
\definecolor{currentfill}{rgb}{0.935991,0.337039,0.249722}%
\pgfsetfillcolor{currentfill}%
\pgfsetlinewidth{0.311001pt}%
\definecolor{currentstroke}{rgb}{1.000000,1.000000,1.000000}%
\pgfsetstrokecolor{currentstroke}%
\pgfsetdash{}{0pt}%
\pgfpathmoveto{\pgfqpoint{4.124137in}{0.809568in}}%
\pgfpathcurveto{\pgfqpoint{4.131270in}{0.809568in}}{\pgfqpoint{4.138112in}{0.812401in}}{\pgfqpoint{4.143156in}{0.817445in}}%
\pgfpathcurveto{\pgfqpoint{4.148199in}{0.822489in}}{\pgfqpoint{4.151033in}{0.829330in}}{\pgfqpoint{4.151033in}{0.836463in}}%
\pgfpathcurveto{\pgfqpoint{4.151033in}{0.843596in}}{\pgfqpoint{4.148199in}{0.850438in}}{\pgfqpoint{4.143156in}{0.855481in}}%
\pgfpathcurveto{\pgfqpoint{4.138112in}{0.860525in}}{\pgfqpoint{4.131270in}{0.863359in}}{\pgfqpoint{4.124137in}{0.863359in}}%
\pgfpathcurveto{\pgfqpoint{4.117005in}{0.863359in}}{\pgfqpoint{4.110163in}{0.860525in}}{\pgfqpoint{4.105119in}{0.855481in}}%
\pgfpathcurveto{\pgfqpoint{4.100076in}{0.850438in}}{\pgfqpoint{4.097242in}{0.843596in}}{\pgfqpoint{4.097242in}{0.836463in}}%
\pgfpathcurveto{\pgfqpoint{4.097242in}{0.829330in}}{\pgfqpoint{4.100076in}{0.822489in}}{\pgfqpoint{4.105119in}{0.817445in}}%
\pgfpathcurveto{\pgfqpoint{4.110163in}{0.812401in}}{\pgfqpoint{4.117005in}{0.809568in}}{\pgfqpoint{4.124137in}{0.809568in}}%
\pgfpathclose%
\pgfusepath{stroke,fill}%
\end{pgfscope}%
\begin{pgfscope}%
\pgfpathrectangle{\pgfqpoint{2.867647in}{0.500000in}}{\pgfqpoint{1.764706in}{1.700000in}}%
\pgfusepath{clip}%
\pgfsetbuttcap%
\pgfsetroundjoin%
\definecolor{currentfill}{rgb}{0.975644,0.874038,0.797253}%
\pgfsetfillcolor{currentfill}%
\pgfsetlinewidth{0.311001pt}%
\definecolor{currentstroke}{rgb}{1.000000,1.000000,1.000000}%
\pgfsetstrokecolor{currentstroke}%
\pgfsetdash{}{0pt}%
\pgfpathmoveto{\pgfqpoint{4.127589in}{1.635360in}}%
\pgfpathcurveto{\pgfqpoint{4.134722in}{1.635360in}}{\pgfqpoint{4.141563in}{1.638194in}}{\pgfqpoint{4.146607in}{1.643238in}}%
\pgfpathcurveto{\pgfqpoint{4.151651in}{1.648281in}}{\pgfqpoint{4.154485in}{1.655123in}}{\pgfqpoint{4.154485in}{1.662256in}}%
\pgfpathcurveto{\pgfqpoint{4.154485in}{1.669389in}}{\pgfqpoint{4.151651in}{1.676230in}}{\pgfqpoint{4.146607in}{1.681274in}}%
\pgfpathcurveto{\pgfqpoint{4.141563in}{1.686318in}}{\pgfqpoint{4.134722in}{1.689151in}}{\pgfqpoint{4.127589in}{1.689151in}}%
\pgfpathcurveto{\pgfqpoint{4.120456in}{1.689151in}}{\pgfqpoint{4.113614in}{1.686318in}}{\pgfqpoint{4.108571in}{1.681274in}}%
\pgfpathcurveto{\pgfqpoint{4.103527in}{1.676230in}}{\pgfqpoint{4.100693in}{1.669389in}}{\pgfqpoint{4.100693in}{1.662256in}}%
\pgfpathcurveto{\pgfqpoint{4.100693in}{1.655123in}}{\pgfqpoint{4.103527in}{1.648281in}}{\pgfqpoint{4.108571in}{1.643238in}}%
\pgfpathcurveto{\pgfqpoint{4.113614in}{1.638194in}}{\pgfqpoint{4.120456in}{1.635360in}}{\pgfqpoint{4.127589in}{1.635360in}}%
\pgfpathclose%
\pgfusepath{stroke,fill}%
\end{pgfscope}%
\begin{pgfscope}%
\pgfpathrectangle{\pgfqpoint{2.867647in}{0.500000in}}{\pgfqpoint{1.764706in}{1.700000in}}%
\pgfusepath{clip}%
\pgfsetbuttcap%
\pgfsetroundjoin%
\definecolor{currentfill}{rgb}{0.979124,0.903132,0.839793}%
\pgfsetfillcolor{currentfill}%
\pgfsetlinewidth{0.311001pt}%
\definecolor{currentstroke}{rgb}{1.000000,1.000000,1.000000}%
\pgfsetstrokecolor{currentstroke}%
\pgfsetdash{}{0pt}%
\pgfpathmoveto{\pgfqpoint{4.213148in}{1.397678in}}%
\pgfpathcurveto{\pgfqpoint{4.220281in}{1.397678in}}{\pgfqpoint{4.227123in}{1.400512in}}{\pgfqpoint{4.232166in}{1.405555in}}%
\pgfpathcurveto{\pgfqpoint{4.237210in}{1.410599in}}{\pgfqpoint{4.240044in}{1.417441in}}{\pgfqpoint{4.240044in}{1.424573in}}%
\pgfpathcurveto{\pgfqpoint{4.240044in}{1.431706in}}{\pgfqpoint{4.237210in}{1.438548in}}{\pgfqpoint{4.232166in}{1.443592in}}%
\pgfpathcurveto{\pgfqpoint{4.227123in}{1.448635in}}{\pgfqpoint{4.220281in}{1.451469in}}{\pgfqpoint{4.213148in}{1.451469in}}%
\pgfpathcurveto{\pgfqpoint{4.206015in}{1.451469in}}{\pgfqpoint{4.199174in}{1.448635in}}{\pgfqpoint{4.194130in}{1.443592in}}%
\pgfpathcurveto{\pgfqpoint{4.189086in}{1.438548in}}{\pgfqpoint{4.186252in}{1.431706in}}{\pgfqpoint{4.186252in}{1.424573in}}%
\pgfpathcurveto{\pgfqpoint{4.186252in}{1.417441in}}{\pgfqpoint{4.189086in}{1.410599in}}{\pgfqpoint{4.194130in}{1.405555in}}%
\pgfpathcurveto{\pgfqpoint{4.199174in}{1.400512in}}{\pgfqpoint{4.206015in}{1.397678in}}{\pgfqpoint{4.213148in}{1.397678in}}%
\pgfpathclose%
\pgfusepath{stroke,fill}%
\end{pgfscope}%
\begin{pgfscope}%
\pgfpathrectangle{\pgfqpoint{2.867647in}{0.500000in}}{\pgfqpoint{1.764706in}{1.700000in}}%
\pgfusepath{clip}%
\pgfsetbuttcap%
\pgfsetroundjoin%
\definecolor{currentfill}{rgb}{0.966560,0.756582,0.625273}%
\pgfsetfillcolor{currentfill}%
\pgfsetlinewidth{0.311001pt}%
\definecolor{currentstroke}{rgb}{1.000000,1.000000,1.000000}%
\pgfsetstrokecolor{currentstroke}%
\pgfsetdash{}{0pt}%
\pgfpathmoveto{\pgfqpoint{4.011541in}{1.676952in}}%
\pgfpathcurveto{\pgfqpoint{4.018674in}{1.676952in}}{\pgfqpoint{4.025516in}{1.679785in}}{\pgfqpoint{4.030559in}{1.684829in}}%
\pgfpathcurveto{\pgfqpoint{4.035603in}{1.689873in}}{\pgfqpoint{4.038437in}{1.696714in}}{\pgfqpoint{4.038437in}{1.703847in}}%
\pgfpathcurveto{\pgfqpoint{4.038437in}{1.710980in}}{\pgfqpoint{4.035603in}{1.717822in}}{\pgfqpoint{4.030559in}{1.722865in}}%
\pgfpathcurveto{\pgfqpoint{4.025516in}{1.727909in}}{\pgfqpoint{4.018674in}{1.730743in}}{\pgfqpoint{4.011541in}{1.730743in}}%
\pgfpathcurveto{\pgfqpoint{4.004408in}{1.730743in}}{\pgfqpoint{3.997567in}{1.727909in}}{\pgfqpoint{3.992523in}{1.722865in}}%
\pgfpathcurveto{\pgfqpoint{3.987479in}{1.717822in}}{\pgfqpoint{3.984645in}{1.710980in}}{\pgfqpoint{3.984645in}{1.703847in}}%
\pgfpathcurveto{\pgfqpoint{3.984645in}{1.696714in}}{\pgfqpoint{3.987479in}{1.689873in}}{\pgfqpoint{3.992523in}{1.684829in}}%
\pgfpathcurveto{\pgfqpoint{3.997567in}{1.679785in}}{\pgfqpoint{4.004408in}{1.676952in}}{\pgfqpoint{4.011541in}{1.676952in}}%
\pgfpathclose%
\pgfusepath{stroke,fill}%
\end{pgfscope}%
\begin{pgfscope}%
\pgfpathrectangle{\pgfqpoint{2.867647in}{0.500000in}}{\pgfqpoint{1.764706in}{1.700000in}}%
\pgfusepath{clip}%
\pgfsetbuttcap%
\pgfsetroundjoin%
\definecolor{currentfill}{rgb}{0.976961,0.885681,0.814303}%
\pgfsetfillcolor{currentfill}%
\pgfsetlinewidth{0.311001pt}%
\definecolor{currentstroke}{rgb}{1.000000,1.000000,1.000000}%
\pgfsetstrokecolor{currentstroke}%
\pgfsetdash{}{0pt}%
\pgfpathmoveto{\pgfqpoint{4.142497in}{1.394908in}}%
\pgfpathcurveto{\pgfqpoint{4.149630in}{1.394908in}}{\pgfqpoint{4.156472in}{1.397742in}}{\pgfqpoint{4.161515in}{1.402786in}}%
\pgfpathcurveto{\pgfqpoint{4.166559in}{1.407829in}}{\pgfqpoint{4.169393in}{1.414671in}}{\pgfqpoint{4.169393in}{1.421804in}}%
\pgfpathcurveto{\pgfqpoint{4.169393in}{1.428937in}}{\pgfqpoint{4.166559in}{1.435778in}}{\pgfqpoint{4.161515in}{1.440822in}}%
\pgfpathcurveto{\pgfqpoint{4.156472in}{1.445866in}}{\pgfqpoint{4.149630in}{1.448699in}}{\pgfqpoint{4.142497in}{1.448699in}}%
\pgfpathcurveto{\pgfqpoint{4.135364in}{1.448699in}}{\pgfqpoint{4.128523in}{1.445866in}}{\pgfqpoint{4.123479in}{1.440822in}}%
\pgfpathcurveto{\pgfqpoint{4.118435in}{1.435778in}}{\pgfqpoint{4.115602in}{1.428937in}}{\pgfqpoint{4.115602in}{1.421804in}}%
\pgfpathcurveto{\pgfqpoint{4.115602in}{1.414671in}}{\pgfqpoint{4.118435in}{1.407829in}}{\pgfqpoint{4.123479in}{1.402786in}}%
\pgfpathcurveto{\pgfqpoint{4.128523in}{1.397742in}}{\pgfqpoint{4.135364in}{1.394908in}}{\pgfqpoint{4.142497in}{1.394908in}}%
\pgfpathclose%
\pgfusepath{stroke,fill}%
\end{pgfscope}%
\begin{pgfscope}%
\pgfpathrectangle{\pgfqpoint{2.867647in}{0.500000in}}{\pgfqpoint{1.764706in}{1.700000in}}%
\pgfusepath{clip}%
\pgfsetbuttcap%
\pgfsetroundjoin%
\definecolor{currentfill}{rgb}{0.964679,0.682838,0.530002}%
\pgfsetfillcolor{currentfill}%
\pgfsetlinewidth{0.311001pt}%
\definecolor{currentstroke}{rgb}{1.000000,1.000000,1.000000}%
\pgfsetstrokecolor{currentstroke}%
\pgfsetdash{}{0pt}%
\pgfpathmoveto{\pgfqpoint{4.101604in}{1.763718in}}%
\pgfpathcurveto{\pgfqpoint{4.108736in}{1.763718in}}{\pgfqpoint{4.115578in}{1.766552in}}{\pgfqpoint{4.120622in}{1.771595in}}%
\pgfpathcurveto{\pgfqpoint{4.125665in}{1.776639in}}{\pgfqpoint{4.128499in}{1.783481in}}{\pgfqpoint{4.128499in}{1.790613in}}%
\pgfpathcurveto{\pgfqpoint{4.128499in}{1.797746in}}{\pgfqpoint{4.125665in}{1.804588in}}{\pgfqpoint{4.120622in}{1.809632in}}%
\pgfpathcurveto{\pgfqpoint{4.115578in}{1.814675in}}{\pgfqpoint{4.108736in}{1.817509in}}{\pgfqpoint{4.101604in}{1.817509in}}%
\pgfpathcurveto{\pgfqpoint{4.094471in}{1.817509in}}{\pgfqpoint{4.087629in}{1.814675in}}{\pgfqpoint{4.082585in}{1.809632in}}%
\pgfpathcurveto{\pgfqpoint{4.077542in}{1.804588in}}{\pgfqpoint{4.074708in}{1.797746in}}{\pgfqpoint{4.074708in}{1.790613in}}%
\pgfpathcurveto{\pgfqpoint{4.074708in}{1.783481in}}{\pgfqpoint{4.077542in}{1.776639in}}{\pgfqpoint{4.082585in}{1.771595in}}%
\pgfpathcurveto{\pgfqpoint{4.087629in}{1.766552in}}{\pgfqpoint{4.094471in}{1.763718in}}{\pgfqpoint{4.101604in}{1.763718in}}%
\pgfpathclose%
\pgfusepath{stroke,fill}%
\end{pgfscope}%
\begin{pgfscope}%
\pgfpathrectangle{\pgfqpoint{2.867647in}{0.500000in}}{\pgfqpoint{1.764706in}{1.700000in}}%
\pgfusepath{clip}%
\pgfsetbuttcap%
\pgfsetroundjoin%
\definecolor{currentfill}{rgb}{0.964679,0.682838,0.530002}%
\pgfsetfillcolor{currentfill}%
\pgfsetlinewidth{0.311001pt}%
\definecolor{currentstroke}{rgb}{1.000000,1.000000,1.000000}%
\pgfsetstrokecolor{currentstroke}%
\pgfsetdash{}{0pt}%
\pgfpathmoveto{\pgfqpoint{3.981742in}{1.645654in}}%
\pgfpathcurveto{\pgfqpoint{3.988875in}{1.645654in}}{\pgfqpoint{3.995717in}{1.648488in}}{\pgfqpoint{4.000761in}{1.653531in}}%
\pgfpathcurveto{\pgfqpoint{4.005804in}{1.658575in}}{\pgfqpoint{4.008638in}{1.665417in}}{\pgfqpoint{4.008638in}{1.672549in}}%
\pgfpathcurveto{\pgfqpoint{4.008638in}{1.679682in}}{\pgfqpoint{4.005804in}{1.686524in}}{\pgfqpoint{4.000761in}{1.691568in}}%
\pgfpathcurveto{\pgfqpoint{3.995717in}{1.696611in}}{\pgfqpoint{3.988875in}{1.699445in}}{\pgfqpoint{3.981742in}{1.699445in}}%
\pgfpathcurveto{\pgfqpoint{3.974610in}{1.699445in}}{\pgfqpoint{3.967768in}{1.696611in}}{\pgfqpoint{3.962724in}{1.691568in}}%
\pgfpathcurveto{\pgfqpoint{3.957681in}{1.686524in}}{\pgfqpoint{3.954847in}{1.679682in}}{\pgfqpoint{3.954847in}{1.672549in}}%
\pgfpathcurveto{\pgfqpoint{3.954847in}{1.665417in}}{\pgfqpoint{3.957681in}{1.658575in}}{\pgfqpoint{3.962724in}{1.653531in}}%
\pgfpathcurveto{\pgfqpoint{3.967768in}{1.648488in}}{\pgfqpoint{3.974610in}{1.645654in}}{\pgfqpoint{3.981742in}{1.645654in}}%
\pgfpathclose%
\pgfusepath{stroke,fill}%
\end{pgfscope}%
\begin{pgfscope}%
\pgfpathrectangle{\pgfqpoint{2.867647in}{0.500000in}}{\pgfqpoint{1.764706in}{1.700000in}}%
\pgfusepath{clip}%
\pgfsetbuttcap%
\pgfsetroundjoin%
\definecolor{currentfill}{rgb}{0.779326,0.096348,0.318766}%
\pgfsetfillcolor{currentfill}%
\pgfsetlinewidth{0.311001pt}%
\definecolor{currentstroke}{rgb}{1.000000,1.000000,1.000000}%
\pgfsetstrokecolor{currentstroke}%
\pgfsetdash{}{0pt}%
\pgfpathmoveto{\pgfqpoint{3.742199in}{1.833808in}}%
\pgfpathcurveto{\pgfqpoint{3.749332in}{1.833808in}}{\pgfqpoint{3.756174in}{1.836642in}}{\pgfqpoint{3.761217in}{1.841685in}}%
\pgfpathcurveto{\pgfqpoint{3.766261in}{1.846729in}}{\pgfqpoint{3.769095in}{1.853571in}}{\pgfqpoint{3.769095in}{1.860703in}}%
\pgfpathcurveto{\pgfqpoint{3.769095in}{1.867836in}}{\pgfqpoint{3.766261in}{1.874678in}}{\pgfqpoint{3.761217in}{1.879721in}}%
\pgfpathcurveto{\pgfqpoint{3.756174in}{1.884765in}}{\pgfqpoint{3.749332in}{1.887599in}}{\pgfqpoint{3.742199in}{1.887599in}}%
\pgfpathcurveto{\pgfqpoint{3.735066in}{1.887599in}}{\pgfqpoint{3.728225in}{1.884765in}}{\pgfqpoint{3.723181in}{1.879721in}}%
\pgfpathcurveto{\pgfqpoint{3.718137in}{1.874678in}}{\pgfqpoint{3.715303in}{1.867836in}}{\pgfqpoint{3.715303in}{1.860703in}}%
\pgfpathcurveto{\pgfqpoint{3.715303in}{1.853571in}}{\pgfqpoint{3.718137in}{1.846729in}}{\pgfqpoint{3.723181in}{1.841685in}}%
\pgfpathcurveto{\pgfqpoint{3.728225in}{1.836642in}}{\pgfqpoint{3.735066in}{1.833808in}}{\pgfqpoint{3.742199in}{1.833808in}}%
\pgfpathclose%
\pgfusepath{stroke,fill}%
\end{pgfscope}%
\begin{pgfscope}%
\pgfpathrectangle{\pgfqpoint{2.867647in}{0.500000in}}{\pgfqpoint{1.764706in}{1.700000in}}%
\pgfusepath{clip}%
\pgfsetbuttcap%
\pgfsetroundjoin%
\definecolor{currentfill}{rgb}{0.971694,0.833208,0.737161}%
\pgfsetfillcolor{currentfill}%
\pgfsetlinewidth{0.311001pt}%
\definecolor{currentstroke}{rgb}{1.000000,1.000000,1.000000}%
\pgfsetstrokecolor{currentstroke}%
\pgfsetdash{}{0pt}%
\pgfpathmoveto{\pgfqpoint{4.087564in}{0.968894in}}%
\pgfpathcurveto{\pgfqpoint{4.094697in}{0.968894in}}{\pgfqpoint{4.101539in}{0.971728in}}{\pgfqpoint{4.106583in}{0.976772in}}%
\pgfpathcurveto{\pgfqpoint{4.111626in}{0.981816in}}{\pgfqpoint{4.114460in}{0.988657in}}{\pgfqpoint{4.114460in}{0.995790in}}%
\pgfpathcurveto{\pgfqpoint{4.114460in}{1.002923in}}{\pgfqpoint{4.111626in}{1.009765in}}{\pgfqpoint{4.106583in}{1.014808in}}%
\pgfpathcurveto{\pgfqpoint{4.101539in}{1.019852in}}{\pgfqpoint{4.094697in}{1.022686in}}{\pgfqpoint{4.087564in}{1.022686in}}%
\pgfpathcurveto{\pgfqpoint{4.080432in}{1.022686in}}{\pgfqpoint{4.073590in}{1.019852in}}{\pgfqpoint{4.068546in}{1.014808in}}%
\pgfpathcurveto{\pgfqpoint{4.063503in}{1.009765in}}{\pgfqpoint{4.060669in}{1.002923in}}{\pgfqpoint{4.060669in}{0.995790in}}%
\pgfpathcurveto{\pgfqpoint{4.060669in}{0.988657in}}{\pgfqpoint{4.063503in}{0.981816in}}{\pgfqpoint{4.068546in}{0.976772in}}%
\pgfpathcurveto{\pgfqpoint{4.073590in}{0.971728in}}{\pgfqpoint{4.080432in}{0.968894in}}{\pgfqpoint{4.087564in}{0.968894in}}%
\pgfpathclose%
\pgfusepath{stroke,fill}%
\end{pgfscope}%
\begin{pgfscope}%
\pgfpathrectangle{\pgfqpoint{2.867647in}{0.500000in}}{\pgfqpoint{1.764706in}{1.700000in}}%
\pgfusepath{clip}%
\pgfsetbuttcap%
\pgfsetroundjoin%
\definecolor{currentfill}{rgb}{0.961734,0.579886,0.418445}%
\pgfsetfillcolor{currentfill}%
\pgfsetlinewidth{0.311001pt}%
\definecolor{currentstroke}{rgb}{1.000000,1.000000,1.000000}%
\pgfsetstrokecolor{currentstroke}%
\pgfsetdash{}{0pt}%
\pgfpathmoveto{\pgfqpoint{4.240546in}{0.973709in}}%
\pgfpathcurveto{\pgfqpoint{4.247679in}{0.973709in}}{\pgfqpoint{4.254520in}{0.976543in}}{\pgfqpoint{4.259564in}{0.981586in}}%
\pgfpathcurveto{\pgfqpoint{4.264608in}{0.986630in}}{\pgfqpoint{4.267442in}{0.993472in}}{\pgfqpoint{4.267442in}{1.000605in}}%
\pgfpathcurveto{\pgfqpoint{4.267442in}{1.007737in}}{\pgfqpoint{4.264608in}{1.014579in}}{\pgfqpoint{4.259564in}{1.019623in}}%
\pgfpathcurveto{\pgfqpoint{4.254520in}{1.024666in}}{\pgfqpoint{4.247679in}{1.027500in}}{\pgfqpoint{4.240546in}{1.027500in}}%
\pgfpathcurveto{\pgfqpoint{4.233413in}{1.027500in}}{\pgfqpoint{4.226571in}{1.024666in}}{\pgfqpoint{4.221528in}{1.019623in}}%
\pgfpathcurveto{\pgfqpoint{4.216484in}{1.014579in}}{\pgfqpoint{4.213650in}{1.007737in}}{\pgfqpoint{4.213650in}{1.000605in}}%
\pgfpathcurveto{\pgfqpoint{4.213650in}{0.993472in}}{\pgfqpoint{4.216484in}{0.986630in}}{\pgfqpoint{4.221528in}{0.981586in}}%
\pgfpathcurveto{\pgfqpoint{4.226571in}{0.976543in}}{\pgfqpoint{4.233413in}{0.973709in}}{\pgfqpoint{4.240546in}{0.973709in}}%
\pgfpathclose%
\pgfusepath{stroke,fill}%
\end{pgfscope}%
\begin{pgfscope}%
\pgfpathrectangle{\pgfqpoint{2.867647in}{0.500000in}}{\pgfqpoint{1.764706in}{1.700000in}}%
\pgfusepath{clip}%
\pgfsetbuttcap%
\pgfsetroundjoin%
\definecolor{currentfill}{rgb}{0.972726,0.844889,0.754401}%
\pgfsetfillcolor{currentfill}%
\pgfsetlinewidth{0.311001pt}%
\definecolor{currentstroke}{rgb}{1.000000,1.000000,1.000000}%
\pgfsetstrokecolor{currentstroke}%
\pgfsetdash{}{0pt}%
\pgfpathmoveto{\pgfqpoint{4.066199in}{1.033560in}}%
\pgfpathcurveto{\pgfqpoint{4.073332in}{1.033560in}}{\pgfqpoint{4.080173in}{1.036394in}}{\pgfqpoint{4.085217in}{1.041437in}}%
\pgfpathcurveto{\pgfqpoint{4.090261in}{1.046481in}}{\pgfqpoint{4.093094in}{1.053323in}}{\pgfqpoint{4.093094in}{1.060456in}}%
\pgfpathcurveto{\pgfqpoint{4.093094in}{1.067588in}}{\pgfqpoint{4.090261in}{1.074430in}}{\pgfqpoint{4.085217in}{1.079474in}}%
\pgfpathcurveto{\pgfqpoint{4.080173in}{1.084517in}}{\pgfqpoint{4.073332in}{1.087351in}}{\pgfqpoint{4.066199in}{1.087351in}}%
\pgfpathcurveto{\pgfqpoint{4.059066in}{1.087351in}}{\pgfqpoint{4.052224in}{1.084517in}}{\pgfqpoint{4.047181in}{1.079474in}}%
\pgfpathcurveto{\pgfqpoint{4.042137in}{1.074430in}}{\pgfqpoint{4.039303in}{1.067588in}}{\pgfqpoint{4.039303in}{1.060456in}}%
\pgfpathcurveto{\pgfqpoint{4.039303in}{1.053323in}}{\pgfqpoint{4.042137in}{1.046481in}}{\pgfqpoint{4.047181in}{1.041437in}}%
\pgfpathcurveto{\pgfqpoint{4.052224in}{1.036394in}}{\pgfqpoint{4.059066in}{1.033560in}}{\pgfqpoint{4.066199in}{1.033560in}}%
\pgfpathclose%
\pgfusepath{stroke,fill}%
\end{pgfscope}%
\begin{pgfscope}%
\pgfpathrectangle{\pgfqpoint{2.867647in}{0.500000in}}{\pgfqpoint{1.764706in}{1.700000in}}%
\pgfusepath{clip}%
\pgfsetbuttcap%
\pgfsetroundjoin%
\definecolor{currentfill}{rgb}{0.978376,0.897317,0.831308}%
\pgfsetfillcolor{currentfill}%
\pgfsetlinewidth{0.311001pt}%
\definecolor{currentstroke}{rgb}{1.000000,1.000000,1.000000}%
\pgfsetstrokecolor{currentstroke}%
\pgfsetdash{}{0pt}%
\pgfpathmoveto{\pgfqpoint{4.221589in}{1.211142in}}%
\pgfpathcurveto{\pgfqpoint{4.228722in}{1.211142in}}{\pgfqpoint{4.235564in}{1.213976in}}{\pgfqpoint{4.240607in}{1.219019in}}%
\pgfpathcurveto{\pgfqpoint{4.245651in}{1.224063in}}{\pgfqpoint{4.248485in}{1.230905in}}{\pgfqpoint{4.248485in}{1.238038in}}%
\pgfpathcurveto{\pgfqpoint{4.248485in}{1.245170in}}{\pgfqpoint{4.245651in}{1.252012in}}{\pgfqpoint{4.240607in}{1.257056in}}%
\pgfpathcurveto{\pgfqpoint{4.235564in}{1.262099in}}{\pgfqpoint{4.228722in}{1.264933in}}{\pgfqpoint{4.221589in}{1.264933in}}%
\pgfpathcurveto{\pgfqpoint{4.214456in}{1.264933in}}{\pgfqpoint{4.207615in}{1.262099in}}{\pgfqpoint{4.202571in}{1.257056in}}%
\pgfpathcurveto{\pgfqpoint{4.197527in}{1.252012in}}{\pgfqpoint{4.194693in}{1.245170in}}{\pgfqpoint{4.194693in}{1.238038in}}%
\pgfpathcurveto{\pgfqpoint{4.194693in}{1.230905in}}{\pgfqpoint{4.197527in}{1.224063in}}{\pgfqpoint{4.202571in}{1.219019in}}%
\pgfpathcurveto{\pgfqpoint{4.207615in}{1.213976in}}{\pgfqpoint{4.214456in}{1.211142in}}{\pgfqpoint{4.221589in}{1.211142in}}%
\pgfpathclose%
\pgfusepath{stroke,fill}%
\end{pgfscope}%
\begin{pgfscope}%
\pgfpathrectangle{\pgfqpoint{2.867647in}{0.500000in}}{\pgfqpoint{1.764706in}{1.700000in}}%
\pgfusepath{clip}%
\pgfsetbuttcap%
\pgfsetroundjoin%
\definecolor{currentfill}{rgb}{0.964433,0.670254,0.515093}%
\pgfsetfillcolor{currentfill}%
\pgfsetlinewidth{0.311001pt}%
\definecolor{currentstroke}{rgb}{1.000000,1.000000,1.000000}%
\pgfsetstrokecolor{currentstroke}%
\pgfsetdash{}{0pt}%
\pgfpathmoveto{\pgfqpoint{4.312910in}{1.394270in}}%
\pgfpathcurveto{\pgfqpoint{4.320043in}{1.394270in}}{\pgfqpoint{4.326885in}{1.397104in}}{\pgfqpoint{4.331929in}{1.402148in}}%
\pgfpathcurveto{\pgfqpoint{4.336972in}{1.407192in}}{\pgfqpoint{4.339806in}{1.414033in}}{\pgfqpoint{4.339806in}{1.421166in}}%
\pgfpathcurveto{\pgfqpoint{4.339806in}{1.428299in}}{\pgfqpoint{4.336972in}{1.435141in}}{\pgfqpoint{4.331929in}{1.440184in}}%
\pgfpathcurveto{\pgfqpoint{4.326885in}{1.445228in}}{\pgfqpoint{4.320043in}{1.448062in}}{\pgfqpoint{4.312910in}{1.448062in}}%
\pgfpathcurveto{\pgfqpoint{4.305778in}{1.448062in}}{\pgfqpoint{4.298936in}{1.445228in}}{\pgfqpoint{4.293892in}{1.440184in}}%
\pgfpathcurveto{\pgfqpoint{4.288849in}{1.435141in}}{\pgfqpoint{4.286015in}{1.428299in}}{\pgfqpoint{4.286015in}{1.421166in}}%
\pgfpathcurveto{\pgfqpoint{4.286015in}{1.414033in}}{\pgfqpoint{4.288849in}{1.407192in}}{\pgfqpoint{4.293892in}{1.402148in}}%
\pgfpathcurveto{\pgfqpoint{4.298936in}{1.397104in}}{\pgfqpoint{4.305778in}{1.394270in}}{\pgfqpoint{4.312910in}{1.394270in}}%
\pgfpathclose%
\pgfusepath{stroke,fill}%
\end{pgfscope}%
\begin{pgfscope}%
\pgfpathrectangle{\pgfqpoint{2.867647in}{0.500000in}}{\pgfqpoint{1.764706in}{1.700000in}}%
\pgfusepath{clip}%
\pgfsetbuttcap%
\pgfsetroundjoin%
\definecolor{currentfill}{rgb}{0.967735,0.780441,0.659127}%
\pgfsetfillcolor{currentfill}%
\pgfsetlinewidth{0.311001pt}%
\definecolor{currentstroke}{rgb}{1.000000,1.000000,1.000000}%
\pgfsetstrokecolor{currentstroke}%
\pgfsetdash{}{0pt}%
\pgfpathmoveto{\pgfqpoint{4.274905in}{1.438525in}}%
\pgfpathcurveto{\pgfqpoint{4.282038in}{1.438525in}}{\pgfqpoint{4.288879in}{1.441359in}}{\pgfqpoint{4.293923in}{1.446403in}}%
\pgfpathcurveto{\pgfqpoint{4.298967in}{1.451446in}}{\pgfqpoint{4.301801in}{1.458288in}}{\pgfqpoint{4.301801in}{1.465421in}}%
\pgfpathcurveto{\pgfqpoint{4.301801in}{1.472554in}}{\pgfqpoint{4.298967in}{1.479395in}}{\pgfqpoint{4.293923in}{1.484439in}}%
\pgfpathcurveto{\pgfqpoint{4.288879in}{1.489482in}}{\pgfqpoint{4.282038in}{1.492316in}}{\pgfqpoint{4.274905in}{1.492316in}}%
\pgfpathcurveto{\pgfqpoint{4.267772in}{1.492316in}}{\pgfqpoint{4.260931in}{1.489482in}}{\pgfqpoint{4.255887in}{1.484439in}}%
\pgfpathcurveto{\pgfqpoint{4.250843in}{1.479395in}}{\pgfqpoint{4.248009in}{1.472554in}}{\pgfqpoint{4.248009in}{1.465421in}}%
\pgfpathcurveto{\pgfqpoint{4.248009in}{1.458288in}}{\pgfqpoint{4.250843in}{1.451446in}}{\pgfqpoint{4.255887in}{1.446403in}}%
\pgfpathcurveto{\pgfqpoint{4.260931in}{1.441359in}}{\pgfqpoint{4.267772in}{1.438525in}}{\pgfqpoint{4.274905in}{1.438525in}}%
\pgfpathclose%
\pgfusepath{stroke,fill}%
\end{pgfscope}%
\begin{pgfscope}%
\pgfpathrectangle{\pgfqpoint{2.867647in}{0.500000in}}{\pgfqpoint{1.764706in}{1.700000in}}%
\pgfusepath{clip}%
\pgfsetbuttcap%
\pgfsetroundjoin%
\definecolor{currentfill}{rgb}{0.980678,0.914765,0.856766}%
\pgfsetfillcolor{currentfill}%
\pgfsetlinewidth{0.311001pt}%
\definecolor{currentstroke}{rgb}{1.000000,1.000000,1.000000}%
\pgfsetstrokecolor{currentstroke}%
\pgfsetdash{}{0pt}%
\pgfpathmoveto{\pgfqpoint{4.149229in}{1.168841in}}%
\pgfpathcurveto{\pgfqpoint{4.156362in}{1.168841in}}{\pgfqpoint{4.163203in}{1.171675in}}{\pgfqpoint{4.168247in}{1.176718in}}%
\pgfpathcurveto{\pgfqpoint{4.173291in}{1.181762in}}{\pgfqpoint{4.176125in}{1.188604in}}{\pgfqpoint{4.176125in}{1.195737in}}%
\pgfpathcurveto{\pgfqpoint{4.176125in}{1.202869in}}{\pgfqpoint{4.173291in}{1.209711in}}{\pgfqpoint{4.168247in}{1.214755in}}%
\pgfpathcurveto{\pgfqpoint{4.163203in}{1.219798in}}{\pgfqpoint{4.156362in}{1.222632in}}{\pgfqpoint{4.149229in}{1.222632in}}%
\pgfpathcurveto{\pgfqpoint{4.142096in}{1.222632in}}{\pgfqpoint{4.135254in}{1.219798in}}{\pgfqpoint{4.130211in}{1.214755in}}%
\pgfpathcurveto{\pgfqpoint{4.125167in}{1.209711in}}{\pgfqpoint{4.122333in}{1.202869in}}{\pgfqpoint{4.122333in}{1.195737in}}%
\pgfpathcurveto{\pgfqpoint{4.122333in}{1.188604in}}{\pgfqpoint{4.125167in}{1.181762in}}{\pgfqpoint{4.130211in}{1.176718in}}%
\pgfpathcurveto{\pgfqpoint{4.135254in}{1.171675in}}{\pgfqpoint{4.142096in}{1.168841in}}{\pgfqpoint{4.149229in}{1.168841in}}%
\pgfpathclose%
\pgfusepath{stroke,fill}%
\end{pgfscope}%
\begin{pgfscope}%
\pgfpathrectangle{\pgfqpoint{2.867647in}{0.500000in}}{\pgfqpoint{1.764706in}{1.700000in}}%
\pgfusepath{clip}%
\pgfsetbuttcap%
\pgfsetroundjoin%
\definecolor{currentfill}{rgb}{0.973832,0.856556,0.771584}%
\pgfsetfillcolor{currentfill}%
\pgfsetlinewidth{0.311001pt}%
\definecolor{currentstroke}{rgb}{1.000000,1.000000,1.000000}%
\pgfsetstrokecolor{currentstroke}%
\pgfsetdash{}{0pt}%
\pgfpathmoveto{\pgfqpoint{4.179508in}{1.603553in}}%
\pgfpathcurveto{\pgfqpoint{4.186641in}{1.603553in}}{\pgfqpoint{4.193483in}{1.606387in}}{\pgfqpoint{4.198526in}{1.611431in}}%
\pgfpathcurveto{\pgfqpoint{4.203570in}{1.616475in}}{\pgfqpoint{4.206404in}{1.623316in}}{\pgfqpoint{4.206404in}{1.630449in}}%
\pgfpathcurveto{\pgfqpoint{4.206404in}{1.637582in}}{\pgfqpoint{4.203570in}{1.644424in}}{\pgfqpoint{4.198526in}{1.649467in}}%
\pgfpathcurveto{\pgfqpoint{4.193483in}{1.654511in}}{\pgfqpoint{4.186641in}{1.657345in}}{\pgfqpoint{4.179508in}{1.657345in}}%
\pgfpathcurveto{\pgfqpoint{4.172375in}{1.657345in}}{\pgfqpoint{4.165534in}{1.654511in}}{\pgfqpoint{4.160490in}{1.649467in}}%
\pgfpathcurveto{\pgfqpoint{4.155446in}{1.644424in}}{\pgfqpoint{4.152612in}{1.637582in}}{\pgfqpoint{4.152612in}{1.630449in}}%
\pgfpathcurveto{\pgfqpoint{4.152612in}{1.623316in}}{\pgfqpoint{4.155446in}{1.616475in}}{\pgfqpoint{4.160490in}{1.611431in}}%
\pgfpathcurveto{\pgfqpoint{4.165534in}{1.606387in}}{\pgfqpoint{4.172375in}{1.603553in}}{\pgfqpoint{4.179508in}{1.603553in}}%
\pgfpathclose%
\pgfusepath{stroke,fill}%
\end{pgfscope}%
\begin{pgfscope}%
\pgfpathrectangle{\pgfqpoint{2.867647in}{0.500000in}}{\pgfqpoint{1.764706in}{1.700000in}}%
\pgfusepath{clip}%
\pgfsetbuttcap%
\pgfsetroundjoin%
\definecolor{currentfill}{rgb}{0.964679,0.682838,0.530002}%
\pgfsetfillcolor{currentfill}%
\pgfsetlinewidth{0.311001pt}%
\definecolor{currentstroke}{rgb}{1.000000,1.000000,1.000000}%
\pgfsetstrokecolor{currentstroke}%
\pgfsetdash{}{0pt}%
\pgfpathmoveto{\pgfqpoint{4.126404in}{0.899171in}}%
\pgfpathcurveto{\pgfqpoint{4.133537in}{0.899171in}}{\pgfqpoint{4.140378in}{0.902005in}}{\pgfqpoint{4.145422in}{0.907049in}}%
\pgfpathcurveto{\pgfqpoint{4.150466in}{0.912092in}}{\pgfqpoint{4.153299in}{0.918934in}}{\pgfqpoint{4.153299in}{0.926067in}}%
\pgfpathcurveto{\pgfqpoint{4.153299in}{0.933200in}}{\pgfqpoint{4.150466in}{0.940041in}}{\pgfqpoint{4.145422in}{0.945085in}}%
\pgfpathcurveto{\pgfqpoint{4.140378in}{0.950129in}}{\pgfqpoint{4.133537in}{0.952962in}}{\pgfqpoint{4.126404in}{0.952962in}}%
\pgfpathcurveto{\pgfqpoint{4.119271in}{0.952962in}}{\pgfqpoint{4.112429in}{0.950129in}}{\pgfqpoint{4.107386in}{0.945085in}}%
\pgfpathcurveto{\pgfqpoint{4.102342in}{0.940041in}}{\pgfqpoint{4.099508in}{0.933200in}}{\pgfqpoint{4.099508in}{0.926067in}}%
\pgfpathcurveto{\pgfqpoint{4.099508in}{0.918934in}}{\pgfqpoint{4.102342in}{0.912092in}}{\pgfqpoint{4.107386in}{0.907049in}}%
\pgfpathcurveto{\pgfqpoint{4.112429in}{0.902005in}}{\pgfqpoint{4.119271in}{0.899171in}}{\pgfqpoint{4.126404in}{0.899171in}}%
\pgfpathclose%
\pgfusepath{stroke,fill}%
\end{pgfscope}%
\begin{pgfscope}%
\pgfpathrectangle{\pgfqpoint{2.867647in}{0.500000in}}{\pgfqpoint{1.764706in}{1.700000in}}%
\pgfusepath{clip}%
\pgfsetbuttcap%
\pgfsetroundjoin%
\definecolor{currentfill}{rgb}{0.979124,0.903132,0.839793}%
\pgfsetfillcolor{currentfill}%
\pgfsetlinewidth{0.311001pt}%
\definecolor{currentstroke}{rgb}{1.000000,1.000000,1.000000}%
\pgfsetstrokecolor{currentstroke}%
\pgfsetdash{}{0pt}%
\pgfpathmoveto{\pgfqpoint{4.209891in}{1.185966in}}%
\pgfpathcurveto{\pgfqpoint{4.217024in}{1.185966in}}{\pgfqpoint{4.223866in}{1.188800in}}{\pgfqpoint{4.228909in}{1.193844in}}%
\pgfpathcurveto{\pgfqpoint{4.233953in}{1.198887in}}{\pgfqpoint{4.236787in}{1.205729in}}{\pgfqpoint{4.236787in}{1.212862in}}%
\pgfpathcurveto{\pgfqpoint{4.236787in}{1.219995in}}{\pgfqpoint{4.233953in}{1.226836in}}{\pgfqpoint{4.228909in}{1.231880in}}%
\pgfpathcurveto{\pgfqpoint{4.223866in}{1.236924in}}{\pgfqpoint{4.217024in}{1.239758in}}{\pgfqpoint{4.209891in}{1.239758in}}%
\pgfpathcurveto{\pgfqpoint{4.202759in}{1.239758in}}{\pgfqpoint{4.195917in}{1.236924in}}{\pgfqpoint{4.190873in}{1.231880in}}%
\pgfpathcurveto{\pgfqpoint{4.185830in}{1.226836in}}{\pgfqpoint{4.182996in}{1.219995in}}{\pgfqpoint{4.182996in}{1.212862in}}%
\pgfpathcurveto{\pgfqpoint{4.182996in}{1.205729in}}{\pgfqpoint{4.185830in}{1.198887in}}{\pgfqpoint{4.190873in}{1.193844in}}%
\pgfpathcurveto{\pgfqpoint{4.195917in}{1.188800in}}{\pgfqpoint{4.202759in}{1.185966in}}{\pgfqpoint{4.209891in}{1.185966in}}%
\pgfpathclose%
\pgfusepath{stroke,fill}%
\end{pgfscope}%
\begin{pgfscope}%
\pgfpathrectangle{\pgfqpoint{2.867647in}{0.500000in}}{\pgfqpoint{1.764706in}{1.700000in}}%
\pgfusepath{clip}%
\pgfsetbuttcap%
\pgfsetroundjoin%
\definecolor{currentfill}{rgb}{0.973832,0.856556,0.771584}%
\pgfsetfillcolor{currentfill}%
\pgfsetlinewidth{0.311001pt}%
\definecolor{currentstroke}{rgb}{1.000000,1.000000,1.000000}%
\pgfsetstrokecolor{currentstroke}%
\pgfsetdash{}{0pt}%
\pgfpathmoveto{\pgfqpoint{4.225697in}{1.507508in}}%
\pgfpathcurveto{\pgfqpoint{4.232830in}{1.507508in}}{\pgfqpoint{4.239671in}{1.510342in}}{\pgfqpoint{4.244715in}{1.515386in}}%
\pgfpathcurveto{\pgfqpoint{4.249759in}{1.520430in}}{\pgfqpoint{4.252593in}{1.527271in}}{\pgfqpoint{4.252593in}{1.534404in}}%
\pgfpathcurveto{\pgfqpoint{4.252593in}{1.541537in}}{\pgfqpoint{4.249759in}{1.548379in}}{\pgfqpoint{4.244715in}{1.553422in}}%
\pgfpathcurveto{\pgfqpoint{4.239671in}{1.558466in}}{\pgfqpoint{4.232830in}{1.561300in}}{\pgfqpoint{4.225697in}{1.561300in}}%
\pgfpathcurveto{\pgfqpoint{4.218564in}{1.561300in}}{\pgfqpoint{4.211722in}{1.558466in}}{\pgfqpoint{4.206679in}{1.553422in}}%
\pgfpathcurveto{\pgfqpoint{4.201635in}{1.548379in}}{\pgfqpoint{4.198801in}{1.541537in}}{\pgfqpoint{4.198801in}{1.534404in}}%
\pgfpathcurveto{\pgfqpoint{4.198801in}{1.527271in}}{\pgfqpoint{4.201635in}{1.520430in}}{\pgfqpoint{4.206679in}{1.515386in}}%
\pgfpathcurveto{\pgfqpoint{4.211722in}{1.510342in}}{\pgfqpoint{4.218564in}{1.507508in}}{\pgfqpoint{4.225697in}{1.507508in}}%
\pgfpathclose%
\pgfusepath{stroke,fill}%
\end{pgfscope}%
\begin{pgfscope}%
\pgfpathrectangle{\pgfqpoint{2.867647in}{0.500000in}}{\pgfqpoint{1.764706in}{1.700000in}}%
\pgfusepath{clip}%
\pgfsetbuttcap%
\pgfsetroundjoin%
\definecolor{currentfill}{rgb}{0.977657,0.891500,0.822809}%
\pgfsetfillcolor{currentfill}%
\pgfsetlinewidth{0.311001pt}%
\definecolor{currentstroke}{rgb}{1.000000,1.000000,1.000000}%
\pgfsetstrokecolor{currentstroke}%
\pgfsetdash{}{0pt}%
\pgfpathmoveto{\pgfqpoint{4.222120in}{1.428624in}}%
\pgfpathcurveto{\pgfqpoint{4.229253in}{1.428624in}}{\pgfqpoint{4.236094in}{1.431458in}}{\pgfqpoint{4.241138in}{1.436501in}}%
\pgfpathcurveto{\pgfqpoint{4.246182in}{1.441545in}}{\pgfqpoint{4.249016in}{1.448387in}}{\pgfqpoint{4.249016in}{1.455520in}}%
\pgfpathcurveto{\pgfqpoint{4.249016in}{1.462652in}}{\pgfqpoint{4.246182in}{1.469494in}}{\pgfqpoint{4.241138in}{1.474538in}}%
\pgfpathcurveto{\pgfqpoint{4.236094in}{1.479581in}}{\pgfqpoint{4.229253in}{1.482415in}}{\pgfqpoint{4.222120in}{1.482415in}}%
\pgfpathcurveto{\pgfqpoint{4.214987in}{1.482415in}}{\pgfqpoint{4.208145in}{1.479581in}}{\pgfqpoint{4.203102in}{1.474538in}}%
\pgfpathcurveto{\pgfqpoint{4.198058in}{1.469494in}}{\pgfqpoint{4.195224in}{1.462652in}}{\pgfqpoint{4.195224in}{1.455520in}}%
\pgfpathcurveto{\pgfqpoint{4.195224in}{1.448387in}}{\pgfqpoint{4.198058in}{1.441545in}}{\pgfqpoint{4.203102in}{1.436501in}}%
\pgfpathcurveto{\pgfqpoint{4.208145in}{1.431458in}}{\pgfqpoint{4.214987in}{1.428624in}}{\pgfqpoint{4.222120in}{1.428624in}}%
\pgfpathclose%
\pgfusepath{stroke,fill}%
\end{pgfscope}%
\begin{pgfscope}%
\pgfpathrectangle{\pgfqpoint{2.867647in}{0.500000in}}{\pgfqpoint{1.764706in}{1.700000in}}%
\pgfusepath{clip}%
\pgfsetbuttcap%
\pgfsetroundjoin%
\definecolor{currentfill}{rgb}{0.981377,0.920617,0.865369}%
\pgfsetfillcolor{currentfill}%
\pgfsetlinewidth{0.311001pt}%
\definecolor{currentstroke}{rgb}{1.000000,1.000000,1.000000}%
\pgfsetstrokecolor{currentstroke}%
\pgfsetdash{}{0pt}%
\pgfpathmoveto{\pgfqpoint{4.206263in}{1.296539in}}%
\pgfpathcurveto{\pgfqpoint{4.213396in}{1.296539in}}{\pgfqpoint{4.220237in}{1.299373in}}{\pgfqpoint{4.225281in}{1.304417in}}%
\pgfpathcurveto{\pgfqpoint{4.230325in}{1.309460in}}{\pgfqpoint{4.233159in}{1.316302in}}{\pgfqpoint{4.233159in}{1.323435in}}%
\pgfpathcurveto{\pgfqpoint{4.233159in}{1.330568in}}{\pgfqpoint{4.230325in}{1.337409in}}{\pgfqpoint{4.225281in}{1.342453in}}%
\pgfpathcurveto{\pgfqpoint{4.220237in}{1.347497in}}{\pgfqpoint{4.213396in}{1.350331in}}{\pgfqpoint{4.206263in}{1.350331in}}%
\pgfpathcurveto{\pgfqpoint{4.199130in}{1.350331in}}{\pgfqpoint{4.192288in}{1.347497in}}{\pgfqpoint{4.187245in}{1.342453in}}%
\pgfpathcurveto{\pgfqpoint{4.182201in}{1.337409in}}{\pgfqpoint{4.179367in}{1.330568in}}{\pgfqpoint{4.179367in}{1.323435in}}%
\pgfpathcurveto{\pgfqpoint{4.179367in}{1.316302in}}{\pgfqpoint{4.182201in}{1.309460in}}{\pgfqpoint{4.187245in}{1.304417in}}%
\pgfpathcurveto{\pgfqpoint{4.192288in}{1.299373in}}{\pgfqpoint{4.199130in}{1.296539in}}{\pgfqpoint{4.206263in}{1.296539in}}%
\pgfpathclose%
\pgfusepath{stroke,fill}%
\end{pgfscope}%
\begin{pgfscope}%
\pgfpathrectangle{\pgfqpoint{2.867647in}{0.500000in}}{\pgfqpoint{1.764706in}{1.700000in}}%
\pgfusepath{clip}%
\pgfsetbuttcap%
\pgfsetroundjoin%
\definecolor{currentfill}{rgb}{0.961115,0.566634,0.405693}%
\pgfsetfillcolor{currentfill}%
\pgfsetlinewidth{0.311001pt}%
\definecolor{currentstroke}{rgb}{1.000000,1.000000,1.000000}%
\pgfsetstrokecolor{currentstroke}%
\pgfsetdash{}{0pt}%
\pgfpathmoveto{\pgfqpoint{3.974698in}{1.801224in}}%
\pgfpathcurveto{\pgfqpoint{3.981831in}{1.801224in}}{\pgfqpoint{3.988673in}{1.804058in}}{\pgfqpoint{3.993716in}{1.809102in}}%
\pgfpathcurveto{\pgfqpoint{3.998760in}{1.814145in}}{\pgfqpoint{4.001594in}{1.820987in}}{\pgfqpoint{4.001594in}{1.828120in}}%
\pgfpathcurveto{\pgfqpoint{4.001594in}{1.835253in}}{\pgfqpoint{3.998760in}{1.842094in}}{\pgfqpoint{3.993716in}{1.847138in}}%
\pgfpathcurveto{\pgfqpoint{3.988673in}{1.852182in}}{\pgfqpoint{3.981831in}{1.855015in}}{\pgfqpoint{3.974698in}{1.855015in}}%
\pgfpathcurveto{\pgfqpoint{3.967565in}{1.855015in}}{\pgfqpoint{3.960724in}{1.852182in}}{\pgfqpoint{3.955680in}{1.847138in}}%
\pgfpathcurveto{\pgfqpoint{3.950636in}{1.842094in}}{\pgfqpoint{3.947803in}{1.835253in}}{\pgfqpoint{3.947803in}{1.828120in}}%
\pgfpathcurveto{\pgfqpoint{3.947803in}{1.820987in}}{\pgfqpoint{3.950636in}{1.814145in}}{\pgfqpoint{3.955680in}{1.809102in}}%
\pgfpathcurveto{\pgfqpoint{3.960724in}{1.804058in}}{\pgfqpoint{3.967565in}{1.801224in}}{\pgfqpoint{3.974698in}{1.801224in}}%
\pgfpathclose%
\pgfusepath{stroke,fill}%
\end{pgfscope}%
\begin{pgfscope}%
\pgfpathrectangle{\pgfqpoint{2.867647in}{0.500000in}}{\pgfqpoint{1.764706in}{1.700000in}}%
\pgfusepath{clip}%
\pgfsetbuttcap%
\pgfsetroundjoin%
\definecolor{currentfill}{rgb}{0.973832,0.856556,0.771584}%
\pgfsetfillcolor{currentfill}%
\pgfsetlinewidth{0.311001pt}%
\definecolor{currentstroke}{rgb}{1.000000,1.000000,1.000000}%
\pgfsetstrokecolor{currentstroke}%
\pgfsetdash{}{0pt}%
\pgfpathmoveto{\pgfqpoint{4.244999in}{1.435196in}}%
\pgfpathcurveto{\pgfqpoint{4.252132in}{1.435196in}}{\pgfqpoint{4.258973in}{1.438030in}}{\pgfqpoint{4.264017in}{1.443074in}}%
\pgfpathcurveto{\pgfqpoint{4.269060in}{1.448117in}}{\pgfqpoint{4.271894in}{1.454959in}}{\pgfqpoint{4.271894in}{1.462092in}}%
\pgfpathcurveto{\pgfqpoint{4.271894in}{1.469225in}}{\pgfqpoint{4.269060in}{1.476066in}}{\pgfqpoint{4.264017in}{1.481110in}}%
\pgfpathcurveto{\pgfqpoint{4.258973in}{1.486154in}}{\pgfqpoint{4.252132in}{1.488988in}}{\pgfqpoint{4.244999in}{1.488988in}}%
\pgfpathcurveto{\pgfqpoint{4.237866in}{1.488988in}}{\pgfqpoint{4.231024in}{1.486154in}}{\pgfqpoint{4.225981in}{1.481110in}}%
\pgfpathcurveto{\pgfqpoint{4.220937in}{1.476066in}}{\pgfqpoint{4.218103in}{1.469225in}}{\pgfqpoint{4.218103in}{1.462092in}}%
\pgfpathcurveto{\pgfqpoint{4.218103in}{1.454959in}}{\pgfqpoint{4.220937in}{1.448117in}}{\pgfqpoint{4.225981in}{1.443074in}}%
\pgfpathcurveto{\pgfqpoint{4.231024in}{1.438030in}}{\pgfqpoint{4.237866in}{1.435196in}}{\pgfqpoint{4.244999in}{1.435196in}}%
\pgfpathclose%
\pgfusepath{stroke,fill}%
\end{pgfscope}%
\begin{pgfscope}%
\pgfpathrectangle{\pgfqpoint{2.867647in}{0.500000in}}{\pgfqpoint{1.764706in}{1.700000in}}%
\pgfusepath{clip}%
\pgfsetbuttcap%
\pgfsetroundjoin%
\definecolor{currentfill}{rgb}{0.964920,0.695342,0.545192}%
\pgfsetfillcolor{currentfill}%
\pgfsetlinewidth{0.311001pt}%
\definecolor{currentstroke}{rgb}{1.000000,1.000000,1.000000}%
\pgfsetstrokecolor{currentstroke}%
\pgfsetdash{}{0pt}%
\pgfpathmoveto{\pgfqpoint{4.044540in}{1.467590in}}%
\pgfpathcurveto{\pgfqpoint{4.051673in}{1.467590in}}{\pgfqpoint{4.058515in}{1.470424in}}{\pgfqpoint{4.063558in}{1.475468in}}%
\pgfpathcurveto{\pgfqpoint{4.068602in}{1.480511in}}{\pgfqpoint{4.071436in}{1.487353in}}{\pgfqpoint{4.071436in}{1.494486in}}%
\pgfpathcurveto{\pgfqpoint{4.071436in}{1.501619in}}{\pgfqpoint{4.068602in}{1.508460in}}{\pgfqpoint{4.063558in}{1.513504in}}%
\pgfpathcurveto{\pgfqpoint{4.058515in}{1.518548in}}{\pgfqpoint{4.051673in}{1.521382in}}{\pgfqpoint{4.044540in}{1.521382in}}%
\pgfpathcurveto{\pgfqpoint{4.037407in}{1.521382in}}{\pgfqpoint{4.030566in}{1.518548in}}{\pgfqpoint{4.025522in}{1.513504in}}%
\pgfpathcurveto{\pgfqpoint{4.020478in}{1.508460in}}{\pgfqpoint{4.017644in}{1.501619in}}{\pgfqpoint{4.017644in}{1.494486in}}%
\pgfpathcurveto{\pgfqpoint{4.017644in}{1.487353in}}{\pgfqpoint{4.020478in}{1.480511in}}{\pgfqpoint{4.025522in}{1.475468in}}%
\pgfpathcurveto{\pgfqpoint{4.030566in}{1.470424in}}{\pgfqpoint{4.037407in}{1.467590in}}{\pgfqpoint{4.044540in}{1.467590in}}%
\pgfpathclose%
\pgfusepath{stroke,fill}%
\end{pgfscope}%
\begin{pgfscope}%
\pgfpathrectangle{\pgfqpoint{2.867647in}{0.500000in}}{\pgfqpoint{1.764706in}{1.700000in}}%
\pgfusepath{clip}%
\pgfsetbuttcap%
\pgfsetroundjoin%
\definecolor{currentfill}{rgb}{0.980678,0.914765,0.856766}%
\pgfsetfillcolor{currentfill}%
\pgfsetlinewidth{0.311001pt}%
\definecolor{currentstroke}{rgb}{1.000000,1.000000,1.000000}%
\pgfsetstrokecolor{currentstroke}%
\pgfsetdash{}{0pt}%
\pgfpathmoveto{\pgfqpoint{4.157293in}{1.202995in}}%
\pgfpathcurveto{\pgfqpoint{4.164426in}{1.202995in}}{\pgfqpoint{4.171267in}{1.205829in}}{\pgfqpoint{4.176311in}{1.210873in}}%
\pgfpathcurveto{\pgfqpoint{4.181355in}{1.215916in}}{\pgfqpoint{4.184189in}{1.222758in}}{\pgfqpoint{4.184189in}{1.229891in}}%
\pgfpathcurveto{\pgfqpoint{4.184189in}{1.237024in}}{\pgfqpoint{4.181355in}{1.243865in}}{\pgfqpoint{4.176311in}{1.248909in}}%
\pgfpathcurveto{\pgfqpoint{4.171267in}{1.253953in}}{\pgfqpoint{4.164426in}{1.256787in}}{\pgfqpoint{4.157293in}{1.256787in}}%
\pgfpathcurveto{\pgfqpoint{4.150160in}{1.256787in}}{\pgfqpoint{4.143318in}{1.253953in}}{\pgfqpoint{4.138275in}{1.248909in}}%
\pgfpathcurveto{\pgfqpoint{4.133231in}{1.243865in}}{\pgfqpoint{4.130397in}{1.237024in}}{\pgfqpoint{4.130397in}{1.229891in}}%
\pgfpathcurveto{\pgfqpoint{4.130397in}{1.222758in}}{\pgfqpoint{4.133231in}{1.215916in}}{\pgfqpoint{4.138275in}{1.210873in}}%
\pgfpathcurveto{\pgfqpoint{4.143318in}{1.205829in}}{\pgfqpoint{4.150160in}{1.202995in}}{\pgfqpoint{4.157293in}{1.202995in}}%
\pgfpathclose%
\pgfusepath{stroke,fill}%
\end{pgfscope}%
\begin{pgfscope}%
\pgfpathrectangle{\pgfqpoint{2.867647in}{0.500000in}}{\pgfqpoint{1.764706in}{1.700000in}}%
\pgfusepath{clip}%
\pgfsetbuttcap%
\pgfsetroundjoin%
\definecolor{currentfill}{rgb}{0.973832,0.856556,0.771584}%
\pgfsetfillcolor{currentfill}%
\pgfsetlinewidth{0.311001pt}%
\definecolor{currentstroke}{rgb}{1.000000,1.000000,1.000000}%
\pgfsetstrokecolor{currentstroke}%
\pgfsetdash{}{0pt}%
\pgfpathmoveto{\pgfqpoint{4.136924in}{1.001976in}}%
\pgfpathcurveto{\pgfqpoint{4.144057in}{1.001976in}}{\pgfqpoint{4.150898in}{1.004810in}}{\pgfqpoint{4.155942in}{1.009854in}}%
\pgfpathcurveto{\pgfqpoint{4.160986in}{1.014897in}}{\pgfqpoint{4.163819in}{1.021739in}}{\pgfqpoint{4.163819in}{1.028872in}}%
\pgfpathcurveto{\pgfqpoint{4.163819in}{1.036005in}}{\pgfqpoint{4.160986in}{1.042846in}}{\pgfqpoint{4.155942in}{1.047890in}}%
\pgfpathcurveto{\pgfqpoint{4.150898in}{1.052934in}}{\pgfqpoint{4.144057in}{1.055768in}}{\pgfqpoint{4.136924in}{1.055768in}}%
\pgfpathcurveto{\pgfqpoint{4.129791in}{1.055768in}}{\pgfqpoint{4.122949in}{1.052934in}}{\pgfqpoint{4.117906in}{1.047890in}}%
\pgfpathcurveto{\pgfqpoint{4.112862in}{1.042846in}}{\pgfqpoint{4.110028in}{1.036005in}}{\pgfqpoint{4.110028in}{1.028872in}}%
\pgfpathcurveto{\pgfqpoint{4.110028in}{1.021739in}}{\pgfqpoint{4.112862in}{1.014897in}}{\pgfqpoint{4.117906in}{1.009854in}}%
\pgfpathcurveto{\pgfqpoint{4.122949in}{1.004810in}}{\pgfqpoint{4.129791in}{1.001976in}}{\pgfqpoint{4.136924in}{1.001976in}}%
\pgfpathclose%
\pgfusepath{stroke,fill}%
\end{pgfscope}%
\begin{pgfscope}%
\pgfpathrectangle{\pgfqpoint{2.867647in}{0.500000in}}{\pgfqpoint{1.764706in}{1.700000in}}%
\pgfusepath{clip}%
\pgfsetbuttcap%
\pgfsetroundjoin%
\definecolor{currentfill}{rgb}{0.981377,0.920617,0.865369}%
\pgfsetfillcolor{currentfill}%
\pgfsetlinewidth{0.311001pt}%
\definecolor{currentstroke}{rgb}{1.000000,1.000000,1.000000}%
\pgfsetstrokecolor{currentstroke}%
\pgfsetdash{}{0pt}%
\pgfpathmoveto{\pgfqpoint{4.193659in}{1.321933in}}%
\pgfpathcurveto{\pgfqpoint{4.200791in}{1.321933in}}{\pgfqpoint{4.207633in}{1.324767in}}{\pgfqpoint{4.212677in}{1.329811in}}%
\pgfpathcurveto{\pgfqpoint{4.217720in}{1.334855in}}{\pgfqpoint{4.220554in}{1.341696in}}{\pgfqpoint{4.220554in}{1.348829in}}%
\pgfpathcurveto{\pgfqpoint{4.220554in}{1.355962in}}{\pgfqpoint{4.217720in}{1.362803in}}{\pgfqpoint{4.212677in}{1.367847in}}%
\pgfpathcurveto{\pgfqpoint{4.207633in}{1.372891in}}{\pgfqpoint{4.200791in}{1.375725in}}{\pgfqpoint{4.193659in}{1.375725in}}%
\pgfpathcurveto{\pgfqpoint{4.186526in}{1.375725in}}{\pgfqpoint{4.179684in}{1.372891in}}{\pgfqpoint{4.174641in}{1.367847in}}%
\pgfpathcurveto{\pgfqpoint{4.169597in}{1.362803in}}{\pgfqpoint{4.166763in}{1.355962in}}{\pgfqpoint{4.166763in}{1.348829in}}%
\pgfpathcurveto{\pgfqpoint{4.166763in}{1.341696in}}{\pgfqpoint{4.169597in}{1.334855in}}{\pgfqpoint{4.174641in}{1.329811in}}%
\pgfpathcurveto{\pgfqpoint{4.179684in}{1.324767in}}{\pgfqpoint{4.186526in}{1.321933in}}{\pgfqpoint{4.193659in}{1.321933in}}%
\pgfpathclose%
\pgfusepath{stroke,fill}%
\end{pgfscope}%
\begin{pgfscope}%
\pgfpathrectangle{\pgfqpoint{2.867647in}{0.500000in}}{\pgfqpoint{1.764706in}{1.700000in}}%
\pgfusepath{clip}%
\pgfsetbuttcap%
\pgfsetroundjoin%
\definecolor{currentfill}{rgb}{0.965169,0.707764,0.560659}%
\pgfsetfillcolor{currentfill}%
\pgfsetlinewidth{0.311001pt}%
\definecolor{currentstroke}{rgb}{1.000000,1.000000,1.000000}%
\pgfsetstrokecolor{currentstroke}%
\pgfsetdash{}{0pt}%
\pgfpathmoveto{\pgfqpoint{4.003814in}{0.916716in}}%
\pgfpathcurveto{\pgfqpoint{4.010947in}{0.916716in}}{\pgfqpoint{4.017788in}{0.919550in}}{\pgfqpoint{4.022832in}{0.924594in}}%
\pgfpathcurveto{\pgfqpoint{4.027876in}{0.929637in}}{\pgfqpoint{4.030709in}{0.936479in}}{\pgfqpoint{4.030709in}{0.943612in}}%
\pgfpathcurveto{\pgfqpoint{4.030709in}{0.950745in}}{\pgfqpoint{4.027876in}{0.957586in}}{\pgfqpoint{4.022832in}{0.962630in}}%
\pgfpathcurveto{\pgfqpoint{4.017788in}{0.967674in}}{\pgfqpoint{4.010947in}{0.970508in}}{\pgfqpoint{4.003814in}{0.970508in}}%
\pgfpathcurveto{\pgfqpoint{3.996681in}{0.970508in}}{\pgfqpoint{3.989839in}{0.967674in}}{\pgfqpoint{3.984796in}{0.962630in}}%
\pgfpathcurveto{\pgfqpoint{3.979752in}{0.957586in}}{\pgfqpoint{3.976918in}{0.950745in}}{\pgfqpoint{3.976918in}{0.943612in}}%
\pgfpathcurveto{\pgfqpoint{3.976918in}{0.936479in}}{\pgfqpoint{3.979752in}{0.929637in}}{\pgfqpoint{3.984796in}{0.924594in}}%
\pgfpathcurveto{\pgfqpoint{3.989839in}{0.919550in}}{\pgfqpoint{3.996681in}{0.916716in}}{\pgfqpoint{4.003814in}{0.916716in}}%
\pgfpathclose%
\pgfusepath{stroke,fill}%
\end{pgfscope}%
\begin{pgfscope}%
\pgfpathrectangle{\pgfqpoint{2.867647in}{0.500000in}}{\pgfqpoint{1.764706in}{1.700000in}}%
\pgfusepath{clip}%
\pgfsetbuttcap%
\pgfsetroundjoin%
\definecolor{currentfill}{rgb}{0.960421,0.553286,0.393191}%
\pgfsetfillcolor{currentfill}%
\pgfsetlinewidth{0.311001pt}%
\definecolor{currentstroke}{rgb}{1.000000,1.000000,1.000000}%
\pgfsetstrokecolor{currentstroke}%
\pgfsetdash{}{0pt}%
\pgfpathmoveto{\pgfqpoint{4.172141in}{1.752993in}}%
\pgfpathcurveto{\pgfqpoint{4.179274in}{1.752993in}}{\pgfqpoint{4.186116in}{1.755826in}}{\pgfqpoint{4.191160in}{1.760870in}}%
\pgfpathcurveto{\pgfqpoint{4.196203in}{1.765914in}}{\pgfqpoint{4.199037in}{1.772755in}}{\pgfqpoint{4.199037in}{1.779888in}}%
\pgfpathcurveto{\pgfqpoint{4.199037in}{1.787021in}}{\pgfqpoint{4.196203in}{1.793863in}}{\pgfqpoint{4.191160in}{1.798906in}}%
\pgfpathcurveto{\pgfqpoint{4.186116in}{1.803950in}}{\pgfqpoint{4.179274in}{1.806784in}}{\pgfqpoint{4.172141in}{1.806784in}}%
\pgfpathcurveto{\pgfqpoint{4.165009in}{1.806784in}}{\pgfqpoint{4.158167in}{1.803950in}}{\pgfqpoint{4.153123in}{1.798906in}}%
\pgfpathcurveto{\pgfqpoint{4.148080in}{1.793863in}}{\pgfqpoint{4.145246in}{1.787021in}}{\pgfqpoint{4.145246in}{1.779888in}}%
\pgfpathcurveto{\pgfqpoint{4.145246in}{1.772755in}}{\pgfqpoint{4.148080in}{1.765914in}}{\pgfqpoint{4.153123in}{1.760870in}}%
\pgfpathcurveto{\pgfqpoint{4.158167in}{1.755826in}}{\pgfqpoint{4.165009in}{1.752993in}}{\pgfqpoint{4.172141in}{1.752993in}}%
\pgfpathclose%
\pgfusepath{stroke,fill}%
\end{pgfscope}%
\begin{pgfscope}%
\pgfpathrectangle{\pgfqpoint{2.867647in}{0.500000in}}{\pgfqpoint{1.764706in}{1.700000in}}%
\pgfusepath{clip}%
\pgfsetbuttcap%
\pgfsetroundjoin%
\definecolor{currentfill}{rgb}{0.976287,0.879862,0.805788}%
\pgfsetfillcolor{currentfill}%
\pgfsetlinewidth{0.311001pt}%
\definecolor{currentstroke}{rgb}{1.000000,1.000000,1.000000}%
\pgfsetstrokecolor{currentstroke}%
\pgfsetdash{}{0pt}%
\pgfpathmoveto{\pgfqpoint{4.239693in}{1.252594in}}%
\pgfpathcurveto{\pgfqpoint{4.246826in}{1.252594in}}{\pgfqpoint{4.253667in}{1.255428in}}{\pgfqpoint{4.258711in}{1.260472in}}%
\pgfpathcurveto{\pgfqpoint{4.263755in}{1.265515in}}{\pgfqpoint{4.266589in}{1.272357in}}{\pgfqpoint{4.266589in}{1.279490in}}%
\pgfpathcurveto{\pgfqpoint{4.266589in}{1.286623in}}{\pgfqpoint{4.263755in}{1.293464in}}{\pgfqpoint{4.258711in}{1.298508in}}%
\pgfpathcurveto{\pgfqpoint{4.253667in}{1.303552in}}{\pgfqpoint{4.246826in}{1.306386in}}{\pgfqpoint{4.239693in}{1.306386in}}%
\pgfpathcurveto{\pgfqpoint{4.232560in}{1.306386in}}{\pgfqpoint{4.225719in}{1.303552in}}{\pgfqpoint{4.220675in}{1.298508in}}%
\pgfpathcurveto{\pgfqpoint{4.215631in}{1.293464in}}{\pgfqpoint{4.212797in}{1.286623in}}{\pgfqpoint{4.212797in}{1.279490in}}%
\pgfpathcurveto{\pgfqpoint{4.212797in}{1.272357in}}{\pgfqpoint{4.215631in}{1.265515in}}{\pgfqpoint{4.220675in}{1.260472in}}%
\pgfpathcurveto{\pgfqpoint{4.225719in}{1.255428in}}{\pgfqpoint{4.232560in}{1.252594in}}{\pgfqpoint{4.239693in}{1.252594in}}%
\pgfpathclose%
\pgfusepath{stroke,fill}%
\end{pgfscope}%
\begin{pgfscope}%
\pgfpathrectangle{\pgfqpoint{2.867647in}{0.500000in}}{\pgfqpoint{1.764706in}{1.700000in}}%
\pgfusepath{clip}%
\pgfsetbuttcap%
\pgfsetroundjoin%
\definecolor{currentfill}{rgb}{0.978376,0.897317,0.831308}%
\pgfsetfillcolor{currentfill}%
\pgfsetlinewidth{0.311001pt}%
\definecolor{currentstroke}{rgb}{1.000000,1.000000,1.000000}%
\pgfsetstrokecolor{currentstroke}%
\pgfsetdash{}{0pt}%
\pgfpathmoveto{\pgfqpoint{4.150054in}{1.293925in}}%
\pgfpathcurveto{\pgfqpoint{4.157187in}{1.293925in}}{\pgfqpoint{4.164029in}{1.296759in}}{\pgfqpoint{4.169072in}{1.301802in}}%
\pgfpathcurveto{\pgfqpoint{4.174116in}{1.306846in}}{\pgfqpoint{4.176950in}{1.313688in}}{\pgfqpoint{4.176950in}{1.320820in}}%
\pgfpathcurveto{\pgfqpoint{4.176950in}{1.327953in}}{\pgfqpoint{4.174116in}{1.334795in}}{\pgfqpoint{4.169072in}{1.339838in}}%
\pgfpathcurveto{\pgfqpoint{4.164029in}{1.344882in}}{\pgfqpoint{4.157187in}{1.347716in}}{\pgfqpoint{4.150054in}{1.347716in}}%
\pgfpathcurveto{\pgfqpoint{4.142921in}{1.347716in}}{\pgfqpoint{4.136080in}{1.344882in}}{\pgfqpoint{4.131036in}{1.339838in}}%
\pgfpathcurveto{\pgfqpoint{4.125992in}{1.334795in}}{\pgfqpoint{4.123158in}{1.327953in}}{\pgfqpoint{4.123158in}{1.320820in}}%
\pgfpathcurveto{\pgfqpoint{4.123158in}{1.313688in}}{\pgfqpoint{4.125992in}{1.306846in}}{\pgfqpoint{4.131036in}{1.301802in}}%
\pgfpathcurveto{\pgfqpoint{4.136080in}{1.296759in}}{\pgfqpoint{4.142921in}{1.293925in}}{\pgfqpoint{4.150054in}{1.293925in}}%
\pgfpathclose%
\pgfusepath{stroke,fill}%
\end{pgfscope}%
\begin{pgfscope}%
\pgfpathrectangle{\pgfqpoint{2.867647in}{0.500000in}}{\pgfqpoint{1.764706in}{1.700000in}}%
\pgfusepath{clip}%
\pgfsetbuttcap%
\pgfsetroundjoin%
\definecolor{currentfill}{rgb}{0.947270,0.405591,0.279023}%
\pgfsetfillcolor{currentfill}%
\pgfsetlinewidth{0.311001pt}%
\definecolor{currentstroke}{rgb}{1.000000,1.000000,1.000000}%
\pgfsetstrokecolor{currentstroke}%
\pgfsetdash{}{0pt}%
\pgfpathmoveto{\pgfqpoint{3.884279in}{0.948368in}}%
\pgfpathcurveto{\pgfqpoint{3.891411in}{0.948368in}}{\pgfqpoint{3.898253in}{0.951202in}}{\pgfqpoint{3.903297in}{0.956246in}}%
\pgfpathcurveto{\pgfqpoint{3.908340in}{0.961290in}}{\pgfqpoint{3.911174in}{0.968131in}}{\pgfqpoint{3.911174in}{0.975264in}}%
\pgfpathcurveto{\pgfqpoint{3.911174in}{0.982397in}}{\pgfqpoint{3.908340in}{0.989239in}}{\pgfqpoint{3.903297in}{0.994282in}}%
\pgfpathcurveto{\pgfqpoint{3.898253in}{0.999326in}}{\pgfqpoint{3.891411in}{1.002160in}}{\pgfqpoint{3.884279in}{1.002160in}}%
\pgfpathcurveto{\pgfqpoint{3.877146in}{1.002160in}}{\pgfqpoint{3.870304in}{0.999326in}}{\pgfqpoint{3.865260in}{0.994282in}}%
\pgfpathcurveto{\pgfqpoint{3.860217in}{0.989239in}}{\pgfqpoint{3.857383in}{0.982397in}}{\pgfqpoint{3.857383in}{0.975264in}}%
\pgfpathcurveto{\pgfqpoint{3.857383in}{0.968131in}}{\pgfqpoint{3.860217in}{0.961290in}}{\pgfqpoint{3.865260in}{0.956246in}}%
\pgfpathcurveto{\pgfqpoint{3.870304in}{0.951202in}}{\pgfqpoint{3.877146in}{0.948368in}}{\pgfqpoint{3.884279in}{0.948368in}}%
\pgfpathclose%
\pgfusepath{stroke,fill}%
\end{pgfscope}%
\begin{pgfscope}%
\pgfpathrectangle{\pgfqpoint{2.867647in}{0.500000in}}{\pgfqpoint{1.764706in}{1.700000in}}%
\pgfusepath{clip}%
\pgfsetbuttcap%
\pgfsetroundjoin%
\definecolor{currentfill}{rgb}{0.971202,0.827364,0.728520}%
\pgfsetfillcolor{currentfill}%
\pgfsetlinewidth{0.311001pt}%
\definecolor{currentstroke}{rgb}{1.000000,1.000000,1.000000}%
\pgfsetstrokecolor{currentstroke}%
\pgfsetdash{}{0pt}%
\pgfpathmoveto{\pgfqpoint{4.073923in}{1.683102in}}%
\pgfpathcurveto{\pgfqpoint{4.081056in}{1.683102in}}{\pgfqpoint{4.087898in}{1.685935in}}{\pgfqpoint{4.092941in}{1.690979in}}%
\pgfpathcurveto{\pgfqpoint{4.097985in}{1.696023in}}{\pgfqpoint{4.100819in}{1.702864in}}{\pgfqpoint{4.100819in}{1.709997in}}%
\pgfpathcurveto{\pgfqpoint{4.100819in}{1.717130in}}{\pgfqpoint{4.097985in}{1.723972in}}{\pgfqpoint{4.092941in}{1.729015in}}%
\pgfpathcurveto{\pgfqpoint{4.087898in}{1.734059in}}{\pgfqpoint{4.081056in}{1.736893in}}{\pgfqpoint{4.073923in}{1.736893in}}%
\pgfpathcurveto{\pgfqpoint{4.066790in}{1.736893in}}{\pgfqpoint{4.059949in}{1.734059in}}{\pgfqpoint{4.054905in}{1.729015in}}%
\pgfpathcurveto{\pgfqpoint{4.049861in}{1.723972in}}{\pgfqpoint{4.047028in}{1.717130in}}{\pgfqpoint{4.047028in}{1.709997in}}%
\pgfpathcurveto{\pgfqpoint{4.047028in}{1.702864in}}{\pgfqpoint{4.049861in}{1.696023in}}{\pgfqpoint{4.054905in}{1.690979in}}%
\pgfpathcurveto{\pgfqpoint{4.059949in}{1.685935in}}{\pgfqpoint{4.066790in}{1.683102in}}{\pgfqpoint{4.073923in}{1.683102in}}%
\pgfpathclose%
\pgfusepath{stroke,fill}%
\end{pgfscope}%
\begin{pgfscope}%
\pgfpathrectangle{\pgfqpoint{2.867647in}{0.500000in}}{\pgfqpoint{1.764706in}{1.700000in}}%
\pgfusepath{clip}%
\pgfsetbuttcap%
\pgfsetroundjoin%
\definecolor{currentfill}{rgb}{0.980678,0.914765,0.856766}%
\pgfsetfillcolor{currentfill}%
\pgfsetlinewidth{0.311001pt}%
\definecolor{currentstroke}{rgb}{1.000000,1.000000,1.000000}%
\pgfsetstrokecolor{currentstroke}%
\pgfsetdash{}{0pt}%
\pgfpathmoveto{\pgfqpoint{4.174426in}{1.361781in}}%
\pgfpathcurveto{\pgfqpoint{4.181559in}{1.361781in}}{\pgfqpoint{4.188401in}{1.364615in}}{\pgfqpoint{4.193444in}{1.369659in}}%
\pgfpathcurveto{\pgfqpoint{4.198488in}{1.374702in}}{\pgfqpoint{4.201322in}{1.381544in}}{\pgfqpoint{4.201322in}{1.388677in}}%
\pgfpathcurveto{\pgfqpoint{4.201322in}{1.395810in}}{\pgfqpoint{4.198488in}{1.402651in}}{\pgfqpoint{4.193444in}{1.407695in}}%
\pgfpathcurveto{\pgfqpoint{4.188401in}{1.412739in}}{\pgfqpoint{4.181559in}{1.415572in}}{\pgfqpoint{4.174426in}{1.415572in}}%
\pgfpathcurveto{\pgfqpoint{4.167293in}{1.415572in}}{\pgfqpoint{4.160452in}{1.412739in}}{\pgfqpoint{4.155408in}{1.407695in}}%
\pgfpathcurveto{\pgfqpoint{4.150364in}{1.402651in}}{\pgfqpoint{4.147531in}{1.395810in}}{\pgfqpoint{4.147531in}{1.388677in}}%
\pgfpathcurveto{\pgfqpoint{4.147531in}{1.381544in}}{\pgfqpoint{4.150364in}{1.374702in}}{\pgfqpoint{4.155408in}{1.369659in}}%
\pgfpathcurveto{\pgfqpoint{4.160452in}{1.364615in}}{\pgfqpoint{4.167293in}{1.361781in}}{\pgfqpoint{4.174426in}{1.361781in}}%
\pgfpathclose%
\pgfusepath{stroke,fill}%
\end{pgfscope}%
\begin{pgfscope}%
\pgfpathrectangle{\pgfqpoint{2.867647in}{0.500000in}}{\pgfqpoint{1.764706in}{1.700000in}}%
\pgfusepath{clip}%
\pgfsetbuttcap%
\pgfsetroundjoin%
\definecolor{currentfill}{rgb}{0.979124,0.903132,0.839793}%
\pgfsetfillcolor{currentfill}%
\pgfsetlinewidth{0.311001pt}%
\definecolor{currentstroke}{rgb}{1.000000,1.000000,1.000000}%
\pgfsetstrokecolor{currentstroke}%
\pgfsetdash{}{0pt}%
\pgfpathmoveto{\pgfqpoint{4.155907in}{1.285453in}}%
\pgfpathcurveto{\pgfqpoint{4.163040in}{1.285453in}}{\pgfqpoint{4.169882in}{1.288286in}}{\pgfqpoint{4.174925in}{1.293330in}}%
\pgfpathcurveto{\pgfqpoint{4.179969in}{1.298374in}}{\pgfqpoint{4.182803in}{1.305215in}}{\pgfqpoint{4.182803in}{1.312348in}}%
\pgfpathcurveto{\pgfqpoint{4.182803in}{1.319481in}}{\pgfqpoint{4.179969in}{1.326323in}}{\pgfqpoint{4.174925in}{1.331366in}}%
\pgfpathcurveto{\pgfqpoint{4.169882in}{1.336410in}}{\pgfqpoint{4.163040in}{1.339244in}}{\pgfqpoint{4.155907in}{1.339244in}}%
\pgfpathcurveto{\pgfqpoint{4.148774in}{1.339244in}}{\pgfqpoint{4.141933in}{1.336410in}}{\pgfqpoint{4.136889in}{1.331366in}}%
\pgfpathcurveto{\pgfqpoint{4.131845in}{1.326323in}}{\pgfqpoint{4.129011in}{1.319481in}}{\pgfqpoint{4.129011in}{1.312348in}}%
\pgfpathcurveto{\pgfqpoint{4.129011in}{1.305215in}}{\pgfqpoint{4.131845in}{1.298374in}}{\pgfqpoint{4.136889in}{1.293330in}}%
\pgfpathcurveto{\pgfqpoint{4.141933in}{1.288286in}}{\pgfqpoint{4.148774in}{1.285453in}}{\pgfqpoint{4.155907in}{1.285453in}}%
\pgfpathclose%
\pgfusepath{stroke,fill}%
\end{pgfscope}%
\begin{pgfscope}%
\pgfpathrectangle{\pgfqpoint{2.867647in}{0.500000in}}{\pgfqpoint{1.764706in}{1.700000in}}%
\pgfusepath{clip}%
\pgfsetbuttcap%
\pgfsetroundjoin%
\definecolor{currentfill}{rgb}{0.971694,0.833208,0.737161}%
\pgfsetfillcolor{currentfill}%
\pgfsetlinewidth{0.311001pt}%
\definecolor{currentstroke}{rgb}{1.000000,1.000000,1.000000}%
\pgfsetstrokecolor{currentstroke}%
\pgfsetdash{}{0pt}%
\pgfpathmoveto{\pgfqpoint{4.059605in}{1.577803in}}%
\pgfpathcurveto{\pgfqpoint{4.066738in}{1.577803in}}{\pgfqpoint{4.073580in}{1.580637in}}{\pgfqpoint{4.078623in}{1.585680in}}%
\pgfpathcurveto{\pgfqpoint{4.083667in}{1.590724in}}{\pgfqpoint{4.086501in}{1.597566in}}{\pgfqpoint{4.086501in}{1.604699in}}%
\pgfpathcurveto{\pgfqpoint{4.086501in}{1.611831in}}{\pgfqpoint{4.083667in}{1.618673in}}{\pgfqpoint{4.078623in}{1.623717in}}%
\pgfpathcurveto{\pgfqpoint{4.073580in}{1.628760in}}{\pgfqpoint{4.066738in}{1.631594in}}{\pgfqpoint{4.059605in}{1.631594in}}%
\pgfpathcurveto{\pgfqpoint{4.052472in}{1.631594in}}{\pgfqpoint{4.045631in}{1.628760in}}{\pgfqpoint{4.040587in}{1.623717in}}%
\pgfpathcurveto{\pgfqpoint{4.035543in}{1.618673in}}{\pgfqpoint{4.032709in}{1.611831in}}{\pgfqpoint{4.032709in}{1.604699in}}%
\pgfpathcurveto{\pgfqpoint{4.032709in}{1.597566in}}{\pgfqpoint{4.035543in}{1.590724in}}{\pgfqpoint{4.040587in}{1.585680in}}%
\pgfpathcurveto{\pgfqpoint{4.045631in}{1.580637in}}{\pgfqpoint{4.052472in}{1.577803in}}{\pgfqpoint{4.059605in}{1.577803in}}%
\pgfpathclose%
\pgfusepath{stroke,fill}%
\end{pgfscope}%
\begin{pgfscope}%
\pgfpathrectangle{\pgfqpoint{2.867647in}{0.500000in}}{\pgfqpoint{1.764706in}{1.700000in}}%
\pgfusepath{clip}%
\pgfsetbuttcap%
\pgfsetroundjoin%
\definecolor{currentfill}{rgb}{0.964433,0.670254,0.515093}%
\pgfsetfillcolor{currentfill}%
\pgfsetlinewidth{0.311001pt}%
\definecolor{currentstroke}{rgb}{1.000000,1.000000,1.000000}%
\pgfsetstrokecolor{currentstroke}%
\pgfsetdash{}{0pt}%
\pgfpathmoveto{\pgfqpoint{3.994234in}{1.758466in}}%
\pgfpathcurveto{\pgfqpoint{4.001367in}{1.758466in}}{\pgfqpoint{4.008209in}{1.761300in}}{\pgfqpoint{4.013252in}{1.766343in}}%
\pgfpathcurveto{\pgfqpoint{4.018296in}{1.771387in}}{\pgfqpoint{4.021130in}{1.778229in}}{\pgfqpoint{4.021130in}{1.785361in}}%
\pgfpathcurveto{\pgfqpoint{4.021130in}{1.792494in}}{\pgfqpoint{4.018296in}{1.799336in}}{\pgfqpoint{4.013252in}{1.804380in}}%
\pgfpathcurveto{\pgfqpoint{4.008209in}{1.809423in}}{\pgfqpoint{4.001367in}{1.812257in}}{\pgfqpoint{3.994234in}{1.812257in}}%
\pgfpathcurveto{\pgfqpoint{3.987101in}{1.812257in}}{\pgfqpoint{3.980260in}{1.809423in}}{\pgfqpoint{3.975216in}{1.804380in}}%
\pgfpathcurveto{\pgfqpoint{3.970172in}{1.799336in}}{\pgfqpoint{3.967339in}{1.792494in}}{\pgfqpoint{3.967339in}{1.785361in}}%
\pgfpathcurveto{\pgfqpoint{3.967339in}{1.778229in}}{\pgfqpoint{3.970172in}{1.771387in}}{\pgfqpoint{3.975216in}{1.766343in}}%
\pgfpathcurveto{\pgfqpoint{3.980260in}{1.761300in}}{\pgfqpoint{3.987101in}{1.758466in}}{\pgfqpoint{3.994234in}{1.758466in}}%
\pgfpathclose%
\pgfusepath{stroke,fill}%
\end{pgfscope}%
\begin{pgfscope}%
\pgfpathrectangle{\pgfqpoint{2.867647in}{0.500000in}}{\pgfqpoint{1.764706in}{1.700000in}}%
\pgfusepath{clip}%
\pgfsetbuttcap%
\pgfsetroundjoin%
\definecolor{currentfill}{rgb}{0.967398,0.774513,0.650573}%
\pgfsetfillcolor{currentfill}%
\pgfsetlinewidth{0.311001pt}%
\definecolor{currentstroke}{rgb}{1.000000,1.000000,1.000000}%
\pgfsetstrokecolor{currentstroke}%
\pgfsetdash{}{0pt}%
\pgfpathmoveto{\pgfqpoint{4.023402in}{0.982605in}}%
\pgfpathcurveto{\pgfqpoint{4.030535in}{0.982605in}}{\pgfqpoint{4.037377in}{0.985439in}}{\pgfqpoint{4.042420in}{0.990482in}}%
\pgfpathcurveto{\pgfqpoint{4.047464in}{0.995526in}}{\pgfqpoint{4.050298in}{1.002368in}}{\pgfqpoint{4.050298in}{1.009500in}}%
\pgfpathcurveto{\pgfqpoint{4.050298in}{1.016633in}}{\pgfqpoint{4.047464in}{1.023475in}}{\pgfqpoint{4.042420in}{1.028519in}}%
\pgfpathcurveto{\pgfqpoint{4.037377in}{1.033562in}}{\pgfqpoint{4.030535in}{1.036396in}}{\pgfqpoint{4.023402in}{1.036396in}}%
\pgfpathcurveto{\pgfqpoint{4.016269in}{1.036396in}}{\pgfqpoint{4.009428in}{1.033562in}}{\pgfqpoint{4.004384in}{1.028519in}}%
\pgfpathcurveto{\pgfqpoint{3.999340in}{1.023475in}}{\pgfqpoint{3.996507in}{1.016633in}}{\pgfqpoint{3.996507in}{1.009500in}}%
\pgfpathcurveto{\pgfqpoint{3.996507in}{1.002368in}}{\pgfqpoint{3.999340in}{0.995526in}}{\pgfqpoint{4.004384in}{0.990482in}}%
\pgfpathcurveto{\pgfqpoint{4.009428in}{0.985439in}}{\pgfqpoint{4.016269in}{0.982605in}}{\pgfqpoint{4.023402in}{0.982605in}}%
\pgfpathclose%
\pgfusepath{stroke,fill}%
\end{pgfscope}%
\begin{pgfscope}%
\pgfpathrectangle{\pgfqpoint{2.867647in}{0.500000in}}{\pgfqpoint{1.764706in}{1.700000in}}%
\pgfusepath{clip}%
\pgfsetbuttcap%
\pgfsetroundjoin%
\definecolor{currentfill}{rgb}{0.970255,0.815666,0.711203}%
\pgfsetfillcolor{currentfill}%
\pgfsetlinewidth{0.311001pt}%
\definecolor{currentstroke}{rgb}{1.000000,1.000000,1.000000}%
\pgfsetstrokecolor{currentstroke}%
\pgfsetdash{}{0pt}%
\pgfpathmoveto{\pgfqpoint{4.093152in}{1.201536in}}%
\pgfpathcurveto{\pgfqpoint{4.100285in}{1.201536in}}{\pgfqpoint{4.107126in}{1.204370in}}{\pgfqpoint{4.112170in}{1.209414in}}%
\pgfpathcurveto{\pgfqpoint{4.117214in}{1.214457in}}{\pgfqpoint{4.120048in}{1.221299in}}{\pgfqpoint{4.120048in}{1.228432in}}%
\pgfpathcurveto{\pgfqpoint{4.120048in}{1.235565in}}{\pgfqpoint{4.117214in}{1.242406in}}{\pgfqpoint{4.112170in}{1.247450in}}%
\pgfpathcurveto{\pgfqpoint{4.107126in}{1.252493in}}{\pgfqpoint{4.100285in}{1.255327in}}{\pgfqpoint{4.093152in}{1.255327in}}%
\pgfpathcurveto{\pgfqpoint{4.086019in}{1.255327in}}{\pgfqpoint{4.079177in}{1.252493in}}{\pgfqpoint{4.074134in}{1.247450in}}%
\pgfpathcurveto{\pgfqpoint{4.069090in}{1.242406in}}{\pgfqpoint{4.066256in}{1.235565in}}{\pgfqpoint{4.066256in}{1.228432in}}%
\pgfpathcurveto{\pgfqpoint{4.066256in}{1.221299in}}{\pgfqpoint{4.069090in}{1.214457in}}{\pgfqpoint{4.074134in}{1.209414in}}%
\pgfpathcurveto{\pgfqpoint{4.079177in}{1.204370in}}{\pgfqpoint{4.086019in}{1.201536in}}{\pgfqpoint{4.093152in}{1.201536in}}%
\pgfpathclose%
\pgfusepath{stroke,fill}%
\end{pgfscope}%
\begin{pgfscope}%
\pgfpathrectangle{\pgfqpoint{2.867647in}{0.500000in}}{\pgfqpoint{1.764706in}{1.700000in}}%
\pgfusepath{clip}%
\pgfsetbuttcap%
\pgfsetroundjoin%
\definecolor{currentfill}{rgb}{0.969359,0.803954,0.693832}%
\pgfsetfillcolor{currentfill}%
\pgfsetlinewidth{0.311001pt}%
\definecolor{currentstroke}{rgb}{1.000000,1.000000,1.000000}%
\pgfsetstrokecolor{currentstroke}%
\pgfsetdash{}{0pt}%
\pgfpathmoveto{\pgfqpoint{4.229223in}{1.564556in}}%
\pgfpathcurveto{\pgfqpoint{4.236356in}{1.564556in}}{\pgfqpoint{4.243197in}{1.567390in}}{\pgfqpoint{4.248241in}{1.572434in}}%
\pgfpathcurveto{\pgfqpoint{4.253285in}{1.577478in}}{\pgfqpoint{4.256118in}{1.584319in}}{\pgfqpoint{4.256118in}{1.591452in}}%
\pgfpathcurveto{\pgfqpoint{4.256118in}{1.598585in}}{\pgfqpoint{4.253285in}{1.605427in}}{\pgfqpoint{4.248241in}{1.610470in}}%
\pgfpathcurveto{\pgfqpoint{4.243197in}{1.615514in}}{\pgfqpoint{4.236356in}{1.618348in}}{\pgfqpoint{4.229223in}{1.618348in}}%
\pgfpathcurveto{\pgfqpoint{4.222090in}{1.618348in}}{\pgfqpoint{4.215248in}{1.615514in}}{\pgfqpoint{4.210205in}{1.610470in}}%
\pgfpathcurveto{\pgfqpoint{4.205161in}{1.605427in}}{\pgfqpoint{4.202327in}{1.598585in}}{\pgfqpoint{4.202327in}{1.591452in}}%
\pgfpathcurveto{\pgfqpoint{4.202327in}{1.584319in}}{\pgfqpoint{4.205161in}{1.577478in}}{\pgfqpoint{4.210205in}{1.572434in}}%
\pgfpathcurveto{\pgfqpoint{4.215248in}{1.567390in}}{\pgfqpoint{4.222090in}{1.564556in}}{\pgfqpoint{4.229223in}{1.564556in}}%
\pgfpathclose%
\pgfusepath{stroke,fill}%
\end{pgfscope}%
\begin{pgfscope}%
\pgfpathrectangle{\pgfqpoint{2.867647in}{0.500000in}}{\pgfqpoint{1.764706in}{1.700000in}}%
\pgfusepath{clip}%
\pgfsetbuttcap%
\pgfsetroundjoin%
\definecolor{currentfill}{rgb}{0.974412,0.862387,0.780156}%
\pgfsetfillcolor{currentfill}%
\pgfsetlinewidth{0.311001pt}%
\definecolor{currentstroke}{rgb}{1.000000,1.000000,1.000000}%
\pgfsetstrokecolor{currentstroke}%
\pgfsetdash{}{0pt}%
\pgfpathmoveto{\pgfqpoint{4.122006in}{1.414443in}}%
\pgfpathcurveto{\pgfqpoint{4.129139in}{1.414443in}}{\pgfqpoint{4.135981in}{1.417277in}}{\pgfqpoint{4.141024in}{1.422320in}}%
\pgfpathcurveto{\pgfqpoint{4.146068in}{1.427364in}}{\pgfqpoint{4.148902in}{1.434206in}}{\pgfqpoint{4.148902in}{1.441338in}}%
\pgfpathcurveto{\pgfqpoint{4.148902in}{1.448471in}}{\pgfqpoint{4.146068in}{1.455313in}}{\pgfqpoint{4.141024in}{1.460357in}}%
\pgfpathcurveto{\pgfqpoint{4.135981in}{1.465400in}}{\pgfqpoint{4.129139in}{1.468234in}}{\pgfqpoint{4.122006in}{1.468234in}}%
\pgfpathcurveto{\pgfqpoint{4.114874in}{1.468234in}}{\pgfqpoint{4.108032in}{1.465400in}}{\pgfqpoint{4.102988in}{1.460357in}}%
\pgfpathcurveto{\pgfqpoint{4.097945in}{1.455313in}}{\pgfqpoint{4.095111in}{1.448471in}}{\pgfqpoint{4.095111in}{1.441338in}}%
\pgfpathcurveto{\pgfqpoint{4.095111in}{1.434206in}}{\pgfqpoint{4.097945in}{1.427364in}}{\pgfqpoint{4.102988in}{1.422320in}}%
\pgfpathcurveto{\pgfqpoint{4.108032in}{1.417277in}}{\pgfqpoint{4.114874in}{1.414443in}}{\pgfqpoint{4.122006in}{1.414443in}}%
\pgfpathclose%
\pgfusepath{stroke,fill}%
\end{pgfscope}%
\begin{pgfscope}%
\pgfpathrectangle{\pgfqpoint{2.867647in}{0.500000in}}{\pgfqpoint{1.764706in}{1.700000in}}%
\pgfusepath{clip}%
\pgfsetbuttcap%
\pgfsetroundjoin%
\definecolor{currentfill}{rgb}{0.969359,0.803954,0.693832}%
\pgfsetfillcolor{currentfill}%
\pgfsetlinewidth{0.311001pt}%
\definecolor{currentstroke}{rgb}{1.000000,1.000000,1.000000}%
\pgfsetstrokecolor{currentstroke}%
\pgfsetdash{}{0pt}%
\pgfpathmoveto{\pgfqpoint{4.143054in}{0.966770in}}%
\pgfpathcurveto{\pgfqpoint{4.150186in}{0.966770in}}{\pgfqpoint{4.157028in}{0.969604in}}{\pgfqpoint{4.162072in}{0.974648in}}%
\pgfpathcurveto{\pgfqpoint{4.167115in}{0.979692in}}{\pgfqpoint{4.169949in}{0.986533in}}{\pgfqpoint{4.169949in}{0.993666in}}%
\pgfpathcurveto{\pgfqpoint{4.169949in}{1.000799in}}{\pgfqpoint{4.167115in}{1.007641in}}{\pgfqpoint{4.162072in}{1.012684in}}%
\pgfpathcurveto{\pgfqpoint{4.157028in}{1.017728in}}{\pgfqpoint{4.150186in}{1.020562in}}{\pgfqpoint{4.143054in}{1.020562in}}%
\pgfpathcurveto{\pgfqpoint{4.135921in}{1.020562in}}{\pgfqpoint{4.129079in}{1.017728in}}{\pgfqpoint{4.124035in}{1.012684in}}%
\pgfpathcurveto{\pgfqpoint{4.118992in}{1.007641in}}{\pgfqpoint{4.116158in}{1.000799in}}{\pgfqpoint{4.116158in}{0.993666in}}%
\pgfpathcurveto{\pgfqpoint{4.116158in}{0.986533in}}{\pgfqpoint{4.118992in}{0.979692in}}{\pgfqpoint{4.124035in}{0.974648in}}%
\pgfpathcurveto{\pgfqpoint{4.129079in}{0.969604in}}{\pgfqpoint{4.135921in}{0.966770in}}{\pgfqpoint{4.143054in}{0.966770in}}%
\pgfpathclose%
\pgfusepath{stroke,fill}%
\end{pgfscope}%
\begin{pgfscope}%
\pgfpathrectangle{\pgfqpoint{2.867647in}{0.500000in}}{\pgfqpoint{1.764706in}{1.700000in}}%
\pgfusepath{clip}%
\pgfsetbuttcap%
\pgfsetroundjoin%
\definecolor{currentfill}{rgb}{0.964558,0.676556,0.522514}%
\pgfsetfillcolor{currentfill}%
\pgfsetlinewidth{0.311001pt}%
\definecolor{currentstroke}{rgb}{1.000000,1.000000,1.000000}%
\pgfsetstrokecolor{currentstroke}%
\pgfsetdash{}{0pt}%
\pgfpathmoveto{\pgfqpoint{4.307372in}{1.416677in}}%
\pgfpathcurveto{\pgfqpoint{4.314505in}{1.416677in}}{\pgfqpoint{4.321347in}{1.419510in}}{\pgfqpoint{4.326390in}{1.424554in}}%
\pgfpathcurveto{\pgfqpoint{4.331434in}{1.429598in}}{\pgfqpoint{4.334268in}{1.436439in}}{\pgfqpoint{4.334268in}{1.443572in}}%
\pgfpathcurveto{\pgfqpoint{4.334268in}{1.450705in}}{\pgfqpoint{4.331434in}{1.457547in}}{\pgfqpoint{4.326390in}{1.462590in}}%
\pgfpathcurveto{\pgfqpoint{4.321347in}{1.467634in}}{\pgfqpoint{4.314505in}{1.470468in}}{\pgfqpoint{4.307372in}{1.470468in}}%
\pgfpathcurveto{\pgfqpoint{4.300240in}{1.470468in}}{\pgfqpoint{4.293398in}{1.467634in}}{\pgfqpoint{4.288354in}{1.462590in}}%
\pgfpathcurveto{\pgfqpoint{4.283311in}{1.457547in}}{\pgfqpoint{4.280477in}{1.450705in}}{\pgfqpoint{4.280477in}{1.443572in}}%
\pgfpathcurveto{\pgfqpoint{4.280477in}{1.436439in}}{\pgfqpoint{4.283311in}{1.429598in}}{\pgfqpoint{4.288354in}{1.424554in}}%
\pgfpathcurveto{\pgfqpoint{4.293398in}{1.419510in}}{\pgfqpoint{4.300240in}{1.416677in}}{\pgfqpoint{4.307372in}{1.416677in}}%
\pgfpathclose%
\pgfusepath{stroke,fill}%
\end{pgfscope}%
\begin{pgfscope}%
\pgfpathrectangle{\pgfqpoint{2.867647in}{0.500000in}}{\pgfqpoint{1.764706in}{1.700000in}}%
\pgfusepath{clip}%
\pgfsetbuttcap%
\pgfsetroundjoin%
\definecolor{currentfill}{rgb}{0.956817,0.498820,0.345554}%
\pgfsetfillcolor{currentfill}%
\pgfsetlinewidth{0.311001pt}%
\definecolor{currentstroke}{rgb}{1.000000,1.000000,1.000000}%
\pgfsetstrokecolor{currentstroke}%
\pgfsetdash{}{0pt}%
\pgfpathmoveto{\pgfqpoint{4.351019in}{1.279593in}}%
\pgfpathcurveto{\pgfqpoint{4.358152in}{1.279593in}}{\pgfqpoint{4.364994in}{1.282427in}}{\pgfqpoint{4.370037in}{1.287471in}}%
\pgfpathcurveto{\pgfqpoint{4.375081in}{1.292515in}}{\pgfqpoint{4.377915in}{1.299356in}}{\pgfqpoint{4.377915in}{1.306489in}}%
\pgfpathcurveto{\pgfqpoint{4.377915in}{1.313622in}}{\pgfqpoint{4.375081in}{1.320463in}}{\pgfqpoint{4.370037in}{1.325507in}}%
\pgfpathcurveto{\pgfqpoint{4.364994in}{1.330551in}}{\pgfqpoint{4.358152in}{1.333385in}}{\pgfqpoint{4.351019in}{1.333385in}}%
\pgfpathcurveto{\pgfqpoint{4.343886in}{1.333385in}}{\pgfqpoint{4.337045in}{1.330551in}}{\pgfqpoint{4.332001in}{1.325507in}}%
\pgfpathcurveto{\pgfqpoint{4.326957in}{1.320463in}}{\pgfqpoint{4.324124in}{1.313622in}}{\pgfqpoint{4.324124in}{1.306489in}}%
\pgfpathcurveto{\pgfqpoint{4.324124in}{1.299356in}}{\pgfqpoint{4.326957in}{1.292515in}}{\pgfqpoint{4.332001in}{1.287471in}}%
\pgfpathcurveto{\pgfqpoint{4.337045in}{1.282427in}}{\pgfqpoint{4.343886in}{1.279593in}}{\pgfqpoint{4.351019in}{1.279593in}}%
\pgfpathclose%
\pgfusepath{stroke,fill}%
\end{pgfscope}%
\begin{pgfscope}%
\pgfpathrectangle{\pgfqpoint{2.867647in}{0.500000in}}{\pgfqpoint{1.764706in}{1.700000in}}%
\pgfusepath{clip}%
\pgfsetbuttcap%
\pgfsetroundjoin%
\definecolor{currentfill}{rgb}{0.975644,0.874038,0.797253}%
\pgfsetfillcolor{currentfill}%
\pgfsetlinewidth{0.311001pt}%
\definecolor{currentstroke}{rgb}{1.000000,1.000000,1.000000}%
\pgfsetstrokecolor{currentstroke}%
\pgfsetdash{}{0pt}%
\pgfpathmoveto{\pgfqpoint{4.086388in}{1.619905in}}%
\pgfpathcurveto{\pgfqpoint{4.093520in}{1.619905in}}{\pgfqpoint{4.100362in}{1.622739in}}{\pgfqpoint{4.105406in}{1.627782in}}%
\pgfpathcurveto{\pgfqpoint{4.110449in}{1.632826in}}{\pgfqpoint{4.113283in}{1.639668in}}{\pgfqpoint{4.113283in}{1.646801in}}%
\pgfpathcurveto{\pgfqpoint{4.113283in}{1.653933in}}{\pgfqpoint{4.110449in}{1.660775in}}{\pgfqpoint{4.105406in}{1.665819in}}%
\pgfpathcurveto{\pgfqpoint{4.100362in}{1.670862in}}{\pgfqpoint{4.093520in}{1.673696in}}{\pgfqpoint{4.086388in}{1.673696in}}%
\pgfpathcurveto{\pgfqpoint{4.079255in}{1.673696in}}{\pgfqpoint{4.072413in}{1.670862in}}{\pgfqpoint{4.067369in}{1.665819in}}%
\pgfpathcurveto{\pgfqpoint{4.062326in}{1.660775in}}{\pgfqpoint{4.059492in}{1.653933in}}{\pgfqpoint{4.059492in}{1.646801in}}%
\pgfpathcurveto{\pgfqpoint{4.059492in}{1.639668in}}{\pgfqpoint{4.062326in}{1.632826in}}{\pgfqpoint{4.067369in}{1.627782in}}%
\pgfpathcurveto{\pgfqpoint{4.072413in}{1.622739in}}{\pgfqpoint{4.079255in}{1.619905in}}{\pgfqpoint{4.086388in}{1.619905in}}%
\pgfpathclose%
\pgfusepath{stroke,fill}%
\end{pgfscope}%
\begin{pgfscope}%
\pgfpathrectangle{\pgfqpoint{2.867647in}{0.500000in}}{\pgfqpoint{1.764706in}{1.700000in}}%
\pgfusepath{clip}%
\pgfsetbuttcap%
\pgfsetroundjoin%
\definecolor{currentfill}{rgb}{0.964173,0.657587,0.500469}%
\pgfsetfillcolor{currentfill}%
\pgfsetlinewidth{0.311001pt}%
\definecolor{currentstroke}{rgb}{1.000000,1.000000,1.000000}%
\pgfsetstrokecolor{currentstroke}%
\pgfsetdash{}{0pt}%
\pgfpathmoveto{\pgfqpoint{4.130589in}{0.893094in}}%
\pgfpathcurveto{\pgfqpoint{4.137722in}{0.893094in}}{\pgfqpoint{4.144564in}{0.895928in}}{\pgfqpoint{4.149607in}{0.900972in}}%
\pgfpathcurveto{\pgfqpoint{4.154651in}{0.906016in}}{\pgfqpoint{4.157485in}{0.912857in}}{\pgfqpoint{4.157485in}{0.919990in}}%
\pgfpathcurveto{\pgfqpoint{4.157485in}{0.927123in}}{\pgfqpoint{4.154651in}{0.933965in}}{\pgfqpoint{4.149607in}{0.939008in}}%
\pgfpathcurveto{\pgfqpoint{4.144564in}{0.944052in}}{\pgfqpoint{4.137722in}{0.946886in}}{\pgfqpoint{4.130589in}{0.946886in}}%
\pgfpathcurveto{\pgfqpoint{4.123456in}{0.946886in}}{\pgfqpoint{4.116615in}{0.944052in}}{\pgfqpoint{4.111571in}{0.939008in}}%
\pgfpathcurveto{\pgfqpoint{4.106527in}{0.933965in}}{\pgfqpoint{4.103693in}{0.927123in}}{\pgfqpoint{4.103693in}{0.919990in}}%
\pgfpathcurveto{\pgfqpoint{4.103693in}{0.912857in}}{\pgfqpoint{4.106527in}{0.906016in}}{\pgfqpoint{4.111571in}{0.900972in}}%
\pgfpathcurveto{\pgfqpoint{4.116615in}{0.895928in}}{\pgfqpoint{4.123456in}{0.893094in}}{\pgfqpoint{4.130589in}{0.893094in}}%
\pgfpathclose%
\pgfusepath{stroke,fill}%
\end{pgfscope}%
\begin{pgfscope}%
\pgfpathrectangle{\pgfqpoint{2.867647in}{0.500000in}}{\pgfqpoint{1.764706in}{1.700000in}}%
\pgfusepath{clip}%
\pgfsetbuttcap%
\pgfsetroundjoin%
\definecolor{currentfill}{rgb}{0.955103,0.477872,0.328626}%
\pgfsetfillcolor{currentfill}%
\pgfsetlinewidth{0.311001pt}%
\definecolor{currentstroke}{rgb}{1.000000,1.000000,1.000000}%
\pgfsetstrokecolor{currentstroke}%
\pgfsetdash{}{0pt}%
\pgfpathmoveto{\pgfqpoint{3.904781in}{0.915653in}}%
\pgfpathcurveto{\pgfqpoint{3.911913in}{0.915653in}}{\pgfqpoint{3.918755in}{0.918487in}}{\pgfqpoint{3.923799in}{0.923530in}}%
\pgfpathcurveto{\pgfqpoint{3.928842in}{0.928574in}}{\pgfqpoint{3.931676in}{0.935416in}}{\pgfqpoint{3.931676in}{0.942548in}}%
\pgfpathcurveto{\pgfqpoint{3.931676in}{0.949681in}}{\pgfqpoint{3.928842in}{0.956523in}}{\pgfqpoint{3.923799in}{0.961567in}}%
\pgfpathcurveto{\pgfqpoint{3.918755in}{0.966610in}}{\pgfqpoint{3.911913in}{0.969444in}}{\pgfqpoint{3.904781in}{0.969444in}}%
\pgfpathcurveto{\pgfqpoint{3.897648in}{0.969444in}}{\pgfqpoint{3.890806in}{0.966610in}}{\pgfqpoint{3.885762in}{0.961567in}}%
\pgfpathcurveto{\pgfqpoint{3.880719in}{0.956523in}}{\pgfqpoint{3.877885in}{0.949681in}}{\pgfqpoint{3.877885in}{0.942548in}}%
\pgfpathcurveto{\pgfqpoint{3.877885in}{0.935416in}}{\pgfqpoint{3.880719in}{0.928574in}}{\pgfqpoint{3.885762in}{0.923530in}}%
\pgfpathcurveto{\pgfqpoint{3.890806in}{0.918487in}}{\pgfqpoint{3.897648in}{0.915653in}}{\pgfqpoint{3.904781in}{0.915653in}}%
\pgfpathclose%
\pgfusepath{stroke,fill}%
\end{pgfscope}%
\begin{pgfscope}%
\pgfpathrectangle{\pgfqpoint{2.867647in}{0.500000in}}{\pgfqpoint{1.764706in}{1.700000in}}%
\pgfusepath{clip}%
\pgfsetbuttcap%
\pgfsetroundjoin%
\definecolor{currentfill}{rgb}{0.973832,0.856556,0.771584}%
\pgfsetfillcolor{currentfill}%
\pgfsetlinewidth{0.311001pt}%
\definecolor{currentstroke}{rgb}{1.000000,1.000000,1.000000}%
\pgfsetstrokecolor{currentstroke}%
\pgfsetdash{}{0pt}%
\pgfpathmoveto{\pgfqpoint{4.220390in}{1.109562in}}%
\pgfpathcurveto{\pgfqpoint{4.227523in}{1.109562in}}{\pgfqpoint{4.234365in}{1.112396in}}{\pgfqpoint{4.239408in}{1.117440in}}%
\pgfpathcurveto{\pgfqpoint{4.244452in}{1.122483in}}{\pgfqpoint{4.247286in}{1.129325in}}{\pgfqpoint{4.247286in}{1.136458in}}%
\pgfpathcurveto{\pgfqpoint{4.247286in}{1.143591in}}{\pgfqpoint{4.244452in}{1.150432in}}{\pgfqpoint{4.239408in}{1.155476in}}%
\pgfpathcurveto{\pgfqpoint{4.234365in}{1.160520in}}{\pgfqpoint{4.227523in}{1.163353in}}{\pgfqpoint{4.220390in}{1.163353in}}%
\pgfpathcurveto{\pgfqpoint{4.213257in}{1.163353in}}{\pgfqpoint{4.206416in}{1.160520in}}{\pgfqpoint{4.201372in}{1.155476in}}%
\pgfpathcurveto{\pgfqpoint{4.196328in}{1.150432in}}{\pgfqpoint{4.193494in}{1.143591in}}{\pgfqpoint{4.193494in}{1.136458in}}%
\pgfpathcurveto{\pgfqpoint{4.193494in}{1.129325in}}{\pgfqpoint{4.196328in}{1.122483in}}{\pgfqpoint{4.201372in}{1.117440in}}%
\pgfpathcurveto{\pgfqpoint{4.206416in}{1.112396in}}{\pgfqpoint{4.213257in}{1.109562in}}{\pgfqpoint{4.220390in}{1.109562in}}%
\pgfpathclose%
\pgfusepath{stroke,fill}%
\end{pgfscope}%
\begin{pgfscope}%
\pgfpathrectangle{\pgfqpoint{2.867647in}{0.500000in}}{\pgfqpoint{1.764706in}{1.700000in}}%
\pgfusepath{clip}%
\pgfsetbuttcap%
\pgfsetroundjoin%
\definecolor{currentfill}{rgb}{0.917171,0.267738,0.242941}%
\pgfsetfillcolor{currentfill}%
\pgfsetlinewidth{0.311001pt}%
\definecolor{currentstroke}{rgb}{1.000000,1.000000,1.000000}%
\pgfsetstrokecolor{currentstroke}%
\pgfsetdash{}{0pt}%
\pgfpathmoveto{\pgfqpoint{3.917424in}{1.089366in}}%
\pgfpathcurveto{\pgfqpoint{3.924557in}{1.089366in}}{\pgfqpoint{3.931398in}{1.092200in}}{\pgfqpoint{3.936442in}{1.097243in}}%
\pgfpathcurveto{\pgfqpoint{3.941486in}{1.102287in}}{\pgfqpoint{3.944320in}{1.109129in}}{\pgfqpoint{3.944320in}{1.116261in}}%
\pgfpathcurveto{\pgfqpoint{3.944320in}{1.123394in}}{\pgfqpoint{3.941486in}{1.130236in}}{\pgfqpoint{3.936442in}{1.135280in}}%
\pgfpathcurveto{\pgfqpoint{3.931398in}{1.140323in}}{\pgfqpoint{3.924557in}{1.143157in}}{\pgfqpoint{3.917424in}{1.143157in}}%
\pgfpathcurveto{\pgfqpoint{3.910291in}{1.143157in}}{\pgfqpoint{3.903449in}{1.140323in}}{\pgfqpoint{3.898406in}{1.135280in}}%
\pgfpathcurveto{\pgfqpoint{3.893362in}{1.130236in}}{\pgfqpoint{3.890528in}{1.123394in}}{\pgfqpoint{3.890528in}{1.116261in}}%
\pgfpathcurveto{\pgfqpoint{3.890528in}{1.109129in}}{\pgfqpoint{3.893362in}{1.102287in}}{\pgfqpoint{3.898406in}{1.097243in}}%
\pgfpathcurveto{\pgfqpoint{3.903449in}{1.092200in}}{\pgfqpoint{3.910291in}{1.089366in}}{\pgfqpoint{3.917424in}{1.089366in}}%
\pgfpathclose%
\pgfusepath{stroke,fill}%
\end{pgfscope}%
\begin{pgfscope}%
\pgfpathrectangle{\pgfqpoint{2.867647in}{0.500000in}}{\pgfqpoint{1.764706in}{1.700000in}}%
\pgfusepath{clip}%
\pgfsetbuttcap%
\pgfsetroundjoin%
\definecolor{currentfill}{rgb}{0.976961,0.885681,0.814303}%
\pgfsetfillcolor{currentfill}%
\pgfsetlinewidth{0.311001pt}%
\definecolor{currentstroke}{rgb}{1.000000,1.000000,1.000000}%
\pgfsetstrokecolor{currentstroke}%
\pgfsetdash{}{0pt}%
\pgfpathmoveto{\pgfqpoint{4.135229in}{1.039912in}}%
\pgfpathcurveto{\pgfqpoint{4.142362in}{1.039912in}}{\pgfqpoint{4.149204in}{1.042746in}}{\pgfqpoint{4.154247in}{1.047790in}}%
\pgfpathcurveto{\pgfqpoint{4.159291in}{1.052834in}}{\pgfqpoint{4.162125in}{1.059675in}}{\pgfqpoint{4.162125in}{1.066808in}}%
\pgfpathcurveto{\pgfqpoint{4.162125in}{1.073941in}}{\pgfqpoint{4.159291in}{1.080782in}}{\pgfqpoint{4.154247in}{1.085826in}}%
\pgfpathcurveto{\pgfqpoint{4.149204in}{1.090870in}}{\pgfqpoint{4.142362in}{1.093704in}}{\pgfqpoint{4.135229in}{1.093704in}}%
\pgfpathcurveto{\pgfqpoint{4.128097in}{1.093704in}}{\pgfqpoint{4.121255in}{1.090870in}}{\pgfqpoint{4.116211in}{1.085826in}}%
\pgfpathcurveto{\pgfqpoint{4.111168in}{1.080782in}}{\pgfqpoint{4.108334in}{1.073941in}}{\pgfqpoint{4.108334in}{1.066808in}}%
\pgfpathcurveto{\pgfqpoint{4.108334in}{1.059675in}}{\pgfqpoint{4.111168in}{1.052834in}}{\pgfqpoint{4.116211in}{1.047790in}}%
\pgfpathcurveto{\pgfqpoint{4.121255in}{1.042746in}}{\pgfqpoint{4.128097in}{1.039912in}}{\pgfqpoint{4.135229in}{1.039912in}}%
\pgfpathclose%
\pgfusepath{stroke,fill}%
\end{pgfscope}%
\begin{pgfscope}%
\pgfpathrectangle{\pgfqpoint{2.867647in}{0.500000in}}{\pgfqpoint{1.764706in}{1.700000in}}%
\pgfusepath{clip}%
\pgfsetbuttcap%
\pgfsetroundjoin%
\definecolor{currentfill}{rgb}{0.979891,0.908948,0.848279}%
\pgfsetfillcolor{currentfill}%
\pgfsetlinewidth{0.311001pt}%
\definecolor{currentstroke}{rgb}{1.000000,1.000000,1.000000}%
\pgfsetstrokecolor{currentstroke}%
\pgfsetdash{}{0pt}%
\pgfpathmoveto{\pgfqpoint{4.160039in}{1.306083in}}%
\pgfpathcurveto{\pgfqpoint{4.167172in}{1.306083in}}{\pgfqpoint{4.174014in}{1.308917in}}{\pgfqpoint{4.179057in}{1.313960in}}%
\pgfpathcurveto{\pgfqpoint{4.184101in}{1.319004in}}{\pgfqpoint{4.186935in}{1.325846in}}{\pgfqpoint{4.186935in}{1.332978in}}%
\pgfpathcurveto{\pgfqpoint{4.186935in}{1.340111in}}{\pgfqpoint{4.184101in}{1.346953in}}{\pgfqpoint{4.179057in}{1.351997in}}%
\pgfpathcurveto{\pgfqpoint{4.174014in}{1.357040in}}{\pgfqpoint{4.167172in}{1.359874in}}{\pgfqpoint{4.160039in}{1.359874in}}%
\pgfpathcurveto{\pgfqpoint{4.152906in}{1.359874in}}{\pgfqpoint{4.146065in}{1.357040in}}{\pgfqpoint{4.141021in}{1.351997in}}%
\pgfpathcurveto{\pgfqpoint{4.135977in}{1.346953in}}{\pgfqpoint{4.133144in}{1.340111in}}{\pgfqpoint{4.133144in}{1.332978in}}%
\pgfpathcurveto{\pgfqpoint{4.133144in}{1.325846in}}{\pgfqpoint{4.135977in}{1.319004in}}{\pgfqpoint{4.141021in}{1.313960in}}%
\pgfpathcurveto{\pgfqpoint{4.146065in}{1.308917in}}{\pgfqpoint{4.152906in}{1.306083in}}{\pgfqpoint{4.160039in}{1.306083in}}%
\pgfpathclose%
\pgfusepath{stroke,fill}%
\end{pgfscope}%
\begin{pgfscope}%
\pgfpathrectangle{\pgfqpoint{2.867647in}{0.500000in}}{\pgfqpoint{1.764706in}{1.700000in}}%
\pgfusepath{clip}%
\pgfsetbuttcap%
\pgfsetroundjoin%
\definecolor{currentfill}{rgb}{0.969803,0.809811,0.702523}%
\pgfsetfillcolor{currentfill}%
\pgfsetlinewidth{0.311001pt}%
\definecolor{currentstroke}{rgb}{1.000000,1.000000,1.000000}%
\pgfsetstrokecolor{currentstroke}%
\pgfsetdash{}{0pt}%
\pgfpathmoveto{\pgfqpoint{4.070003in}{1.127392in}}%
\pgfpathcurveto{\pgfqpoint{4.077136in}{1.127392in}}{\pgfqpoint{4.083978in}{1.130226in}}{\pgfqpoint{4.089021in}{1.135270in}}%
\pgfpathcurveto{\pgfqpoint{4.094065in}{1.140314in}}{\pgfqpoint{4.096899in}{1.147155in}}{\pgfqpoint{4.096899in}{1.154288in}}%
\pgfpathcurveto{\pgfqpoint{4.096899in}{1.161421in}}{\pgfqpoint{4.094065in}{1.168263in}}{\pgfqpoint{4.089021in}{1.173306in}}%
\pgfpathcurveto{\pgfqpoint{4.083978in}{1.178350in}}{\pgfqpoint{4.077136in}{1.181184in}}{\pgfqpoint{4.070003in}{1.181184in}}%
\pgfpathcurveto{\pgfqpoint{4.062870in}{1.181184in}}{\pgfqpoint{4.056029in}{1.178350in}}{\pgfqpoint{4.050985in}{1.173306in}}%
\pgfpathcurveto{\pgfqpoint{4.045941in}{1.168263in}}{\pgfqpoint{4.043107in}{1.161421in}}{\pgfqpoint{4.043107in}{1.154288in}}%
\pgfpathcurveto{\pgfqpoint{4.043107in}{1.147155in}}{\pgfqpoint{4.045941in}{1.140314in}}{\pgfqpoint{4.050985in}{1.135270in}}%
\pgfpathcurveto{\pgfqpoint{4.056029in}{1.130226in}}{\pgfqpoint{4.062870in}{1.127392in}}{\pgfqpoint{4.070003in}{1.127392in}}%
\pgfpathclose%
\pgfusepath{stroke,fill}%
\end{pgfscope}%
\begin{pgfscope}%
\pgfpathrectangle{\pgfqpoint{2.867647in}{0.500000in}}{\pgfqpoint{1.764706in}{1.700000in}}%
\pgfusepath{clip}%
\pgfsetbuttcap%
\pgfsetroundjoin%
\definecolor{currentfill}{rgb}{0.981377,0.920617,0.865369}%
\pgfsetfillcolor{currentfill}%
\pgfsetlinewidth{0.311001pt}%
\definecolor{currentstroke}{rgb}{1.000000,1.000000,1.000000}%
\pgfsetstrokecolor{currentstroke}%
\pgfsetdash{}{0pt}%
\pgfpathmoveto{\pgfqpoint{4.184847in}{1.317950in}}%
\pgfpathcurveto{\pgfqpoint{4.191980in}{1.317950in}}{\pgfqpoint{4.198822in}{1.320784in}}{\pgfqpoint{4.203865in}{1.325828in}}%
\pgfpathcurveto{\pgfqpoint{4.208909in}{1.330871in}}{\pgfqpoint{4.211743in}{1.337713in}}{\pgfqpoint{4.211743in}{1.344846in}}%
\pgfpathcurveto{\pgfqpoint{4.211743in}{1.351979in}}{\pgfqpoint{4.208909in}{1.358820in}}{\pgfqpoint{4.203865in}{1.363864in}}%
\pgfpathcurveto{\pgfqpoint{4.198822in}{1.368908in}}{\pgfqpoint{4.191980in}{1.371741in}}{\pgfqpoint{4.184847in}{1.371741in}}%
\pgfpathcurveto{\pgfqpoint{4.177715in}{1.371741in}}{\pgfqpoint{4.170873in}{1.368908in}}{\pgfqpoint{4.165829in}{1.363864in}}%
\pgfpathcurveto{\pgfqpoint{4.160786in}{1.358820in}}{\pgfqpoint{4.157952in}{1.351979in}}{\pgfqpoint{4.157952in}{1.344846in}}%
\pgfpathcurveto{\pgfqpoint{4.157952in}{1.337713in}}{\pgfqpoint{4.160786in}{1.330871in}}{\pgfqpoint{4.165829in}{1.325828in}}%
\pgfpathcurveto{\pgfqpoint{4.170873in}{1.320784in}}{\pgfqpoint{4.177715in}{1.317950in}}{\pgfqpoint{4.184847in}{1.317950in}}%
\pgfpathclose%
\pgfusepath{stroke,fill}%
\end{pgfscope}%
\begin{pgfscope}%
\pgfpathrectangle{\pgfqpoint{2.867647in}{0.500000in}}{\pgfqpoint{1.764706in}{1.700000in}}%
\pgfusepath{clip}%
\pgfsetbuttcap%
\pgfsetroundjoin%
\definecolor{currentfill}{rgb}{0.978376,0.897317,0.831308}%
\pgfsetfillcolor{currentfill}%
\pgfsetlinewidth{0.311001pt}%
\definecolor{currentstroke}{rgb}{1.000000,1.000000,1.000000}%
\pgfsetstrokecolor{currentstroke}%
\pgfsetdash{}{0pt}%
\pgfpathmoveto{\pgfqpoint{4.198638in}{1.492357in}}%
\pgfpathcurveto{\pgfqpoint{4.205771in}{1.492357in}}{\pgfqpoint{4.212613in}{1.495191in}}{\pgfqpoint{4.217656in}{1.500235in}}%
\pgfpathcurveto{\pgfqpoint{4.222700in}{1.505278in}}{\pgfqpoint{4.225534in}{1.512120in}}{\pgfqpoint{4.225534in}{1.519253in}}%
\pgfpathcurveto{\pgfqpoint{4.225534in}{1.526386in}}{\pgfqpoint{4.222700in}{1.533227in}}{\pgfqpoint{4.217656in}{1.538271in}}%
\pgfpathcurveto{\pgfqpoint{4.212613in}{1.543315in}}{\pgfqpoint{4.205771in}{1.546148in}}{\pgfqpoint{4.198638in}{1.546148in}}%
\pgfpathcurveto{\pgfqpoint{4.191505in}{1.546148in}}{\pgfqpoint{4.184664in}{1.543315in}}{\pgfqpoint{4.179620in}{1.538271in}}%
\pgfpathcurveto{\pgfqpoint{4.174577in}{1.533227in}}{\pgfqpoint{4.171743in}{1.526386in}}{\pgfqpoint{4.171743in}{1.519253in}}%
\pgfpathcurveto{\pgfqpoint{4.171743in}{1.512120in}}{\pgfqpoint{4.174577in}{1.505278in}}{\pgfqpoint{4.179620in}{1.500235in}}%
\pgfpathcurveto{\pgfqpoint{4.184664in}{1.495191in}}{\pgfqpoint{4.191505in}{1.492357in}}{\pgfqpoint{4.198638in}{1.492357in}}%
\pgfpathclose%
\pgfusepath{stroke,fill}%
\end{pgfscope}%
\begin{pgfscope}%
\pgfpathrectangle{\pgfqpoint{2.867647in}{0.500000in}}{\pgfqpoint{1.764706in}{1.700000in}}%
\pgfusepath{clip}%
\pgfsetbuttcap%
\pgfsetroundjoin%
\definecolor{currentfill}{rgb}{0.979891,0.908948,0.848279}%
\pgfsetfillcolor{currentfill}%
\pgfsetlinewidth{0.311001pt}%
\definecolor{currentstroke}{rgb}{1.000000,1.000000,1.000000}%
\pgfsetstrokecolor{currentstroke}%
\pgfsetdash{}{0pt}%
\pgfpathmoveto{\pgfqpoint{4.155349in}{1.453939in}}%
\pgfpathcurveto{\pgfqpoint{4.162482in}{1.453939in}}{\pgfqpoint{4.169324in}{1.456773in}}{\pgfqpoint{4.174368in}{1.461816in}}%
\pgfpathcurveto{\pgfqpoint{4.179411in}{1.466860in}}{\pgfqpoint{4.182245in}{1.473702in}}{\pgfqpoint{4.182245in}{1.480835in}}%
\pgfpathcurveto{\pgfqpoint{4.182245in}{1.487967in}}{\pgfqpoint{4.179411in}{1.494809in}}{\pgfqpoint{4.174368in}{1.499853in}}%
\pgfpathcurveto{\pgfqpoint{4.169324in}{1.504896in}}{\pgfqpoint{4.162482in}{1.507730in}}{\pgfqpoint{4.155349in}{1.507730in}}%
\pgfpathcurveto{\pgfqpoint{4.148217in}{1.507730in}}{\pgfqpoint{4.141375in}{1.504896in}}{\pgfqpoint{4.136331in}{1.499853in}}%
\pgfpathcurveto{\pgfqpoint{4.131288in}{1.494809in}}{\pgfqpoint{4.128454in}{1.487967in}}{\pgfqpoint{4.128454in}{1.480835in}}%
\pgfpathcurveto{\pgfqpoint{4.128454in}{1.473702in}}{\pgfqpoint{4.131288in}{1.466860in}}{\pgfqpoint{4.136331in}{1.461816in}}%
\pgfpathcurveto{\pgfqpoint{4.141375in}{1.456773in}}{\pgfqpoint{4.148217in}{1.453939in}}{\pgfqpoint{4.155349in}{1.453939in}}%
\pgfpathclose%
\pgfusepath{stroke,fill}%
\end{pgfscope}%
\begin{pgfscope}%
\pgfpathrectangle{\pgfqpoint{2.867647in}{0.500000in}}{\pgfqpoint{1.764706in}{1.700000in}}%
\pgfusepath{clip}%
\pgfsetbuttcap%
\pgfsetroundjoin%
\definecolor{currentfill}{rgb}{0.973271,0.850724,0.762998}%
\pgfsetfillcolor{currentfill}%
\pgfsetlinewidth{0.311001pt}%
\definecolor{currentstroke}{rgb}{1.000000,1.000000,1.000000}%
\pgfsetstrokecolor{currentstroke}%
\pgfsetdash{}{0pt}%
\pgfpathmoveto{\pgfqpoint{4.182857in}{1.040671in}}%
\pgfpathcurveto{\pgfqpoint{4.189989in}{1.040671in}}{\pgfqpoint{4.196831in}{1.043505in}}{\pgfqpoint{4.201875in}{1.048549in}}%
\pgfpathcurveto{\pgfqpoint{4.206918in}{1.053593in}}{\pgfqpoint{4.209752in}{1.060434in}}{\pgfqpoint{4.209752in}{1.067567in}}%
\pgfpathcurveto{\pgfqpoint{4.209752in}{1.074700in}}{\pgfqpoint{4.206918in}{1.081542in}}{\pgfqpoint{4.201875in}{1.086585in}}%
\pgfpathcurveto{\pgfqpoint{4.196831in}{1.091629in}}{\pgfqpoint{4.189989in}{1.094463in}}{\pgfqpoint{4.182857in}{1.094463in}}%
\pgfpathcurveto{\pgfqpoint{4.175724in}{1.094463in}}{\pgfqpoint{4.168882in}{1.091629in}}{\pgfqpoint{4.163838in}{1.086585in}}%
\pgfpathcurveto{\pgfqpoint{4.158795in}{1.081542in}}{\pgfqpoint{4.155961in}{1.074700in}}{\pgfqpoint{4.155961in}{1.067567in}}%
\pgfpathcurveto{\pgfqpoint{4.155961in}{1.060434in}}{\pgfqpoint{4.158795in}{1.053593in}}{\pgfqpoint{4.163838in}{1.048549in}}%
\pgfpathcurveto{\pgfqpoint{4.168882in}{1.043505in}}{\pgfqpoint{4.175724in}{1.040671in}}{\pgfqpoint{4.182857in}{1.040671in}}%
\pgfpathclose%
\pgfusepath{stroke,fill}%
\end{pgfscope}%
\begin{pgfscope}%
\pgfpathrectangle{\pgfqpoint{2.867647in}{0.500000in}}{\pgfqpoint{1.764706in}{1.700000in}}%
\pgfusepath{clip}%
\pgfsetbuttcap%
\pgfsetroundjoin%
\definecolor{currentfill}{rgb}{0.981377,0.920617,0.865369}%
\pgfsetfillcolor{currentfill}%
\pgfsetlinewidth{0.311001pt}%
\definecolor{currentstroke}{rgb}{1.000000,1.000000,1.000000}%
\pgfsetstrokecolor{currentstroke}%
\pgfsetdash{}{0pt}%
\pgfpathmoveto{\pgfqpoint{4.168852in}{1.199299in}}%
\pgfpathcurveto{\pgfqpoint{4.175985in}{1.199299in}}{\pgfqpoint{4.182826in}{1.202133in}}{\pgfqpoint{4.187870in}{1.207177in}}%
\pgfpathcurveto{\pgfqpoint{4.192914in}{1.212221in}}{\pgfqpoint{4.195748in}{1.219062in}}{\pgfqpoint{4.195748in}{1.226195in}}%
\pgfpathcurveto{\pgfqpoint{4.195748in}{1.233328in}}{\pgfqpoint{4.192914in}{1.240170in}}{\pgfqpoint{4.187870in}{1.245213in}}%
\pgfpathcurveto{\pgfqpoint{4.182826in}{1.250257in}}{\pgfqpoint{4.175985in}{1.253091in}}{\pgfqpoint{4.168852in}{1.253091in}}%
\pgfpathcurveto{\pgfqpoint{4.161719in}{1.253091in}}{\pgfqpoint{4.154877in}{1.250257in}}{\pgfqpoint{4.149834in}{1.245213in}}%
\pgfpathcurveto{\pgfqpoint{4.144790in}{1.240170in}}{\pgfqpoint{4.141956in}{1.233328in}}{\pgfqpoint{4.141956in}{1.226195in}}%
\pgfpathcurveto{\pgfqpoint{4.141956in}{1.219062in}}{\pgfqpoint{4.144790in}{1.212221in}}{\pgfqpoint{4.149834in}{1.207177in}}%
\pgfpathcurveto{\pgfqpoint{4.154877in}{1.202133in}}{\pgfqpoint{4.161719in}{1.199299in}}{\pgfqpoint{4.168852in}{1.199299in}}%
\pgfpathclose%
\pgfusepath{stroke,fill}%
\end{pgfscope}%
\begin{pgfscope}%
\pgfpathrectangle{\pgfqpoint{2.867647in}{0.500000in}}{\pgfqpoint{1.764706in}{1.700000in}}%
\pgfusepath{clip}%
\pgfsetbuttcap%
\pgfsetroundjoin%
\definecolor{currentfill}{rgb}{0.971202,0.827364,0.728520}%
\pgfsetfillcolor{currentfill}%
\pgfsetlinewidth{0.311001pt}%
\definecolor{currentstroke}{rgb}{1.000000,1.000000,1.000000}%
\pgfsetstrokecolor{currentstroke}%
\pgfsetdash{}{0pt}%
\pgfpathmoveto{\pgfqpoint{4.084292in}{0.961172in}}%
\pgfpathcurveto{\pgfqpoint{4.091425in}{0.961172in}}{\pgfqpoint{4.098267in}{0.964006in}}{\pgfqpoint{4.103310in}{0.969050in}}%
\pgfpathcurveto{\pgfqpoint{4.108354in}{0.974093in}}{\pgfqpoint{4.111188in}{0.980935in}}{\pgfqpoint{4.111188in}{0.988068in}}%
\pgfpathcurveto{\pgfqpoint{4.111188in}{0.995201in}}{\pgfqpoint{4.108354in}{1.002042in}}{\pgfqpoint{4.103310in}{1.007086in}}%
\pgfpathcurveto{\pgfqpoint{4.098267in}{1.012130in}}{\pgfqpoint{4.091425in}{1.014964in}}{\pgfqpoint{4.084292in}{1.014964in}}%
\pgfpathcurveto{\pgfqpoint{4.077159in}{1.014964in}}{\pgfqpoint{4.070318in}{1.012130in}}{\pgfqpoint{4.065274in}{1.007086in}}%
\pgfpathcurveto{\pgfqpoint{4.060230in}{1.002042in}}{\pgfqpoint{4.057397in}{0.995201in}}{\pgfqpoint{4.057397in}{0.988068in}}%
\pgfpathcurveto{\pgfqpoint{4.057397in}{0.980935in}}{\pgfqpoint{4.060230in}{0.974093in}}{\pgfqpoint{4.065274in}{0.969050in}}%
\pgfpathcurveto{\pgfqpoint{4.070318in}{0.964006in}}{\pgfqpoint{4.077159in}{0.961172in}}{\pgfqpoint{4.084292in}{0.961172in}}%
\pgfpathclose%
\pgfusepath{stroke,fill}%
\end{pgfscope}%
\begin{pgfscope}%
\pgfpathrectangle{\pgfqpoint{2.867647in}{0.500000in}}{\pgfqpoint{1.764706in}{1.700000in}}%
\pgfusepath{clip}%
\pgfsetbuttcap%
\pgfsetroundjoin%
\definecolor{currentfill}{rgb}{0.802060,0.108583,0.307830}%
\pgfsetfillcolor{currentfill}%
\pgfsetlinewidth{0.311001pt}%
\definecolor{currentstroke}{rgb}{1.000000,1.000000,1.000000}%
\pgfsetstrokecolor{currentstroke}%
\pgfsetdash{}{0pt}%
\pgfpathmoveto{\pgfqpoint{3.784646in}{0.829579in}}%
\pgfpathcurveto{\pgfqpoint{3.791779in}{0.829579in}}{\pgfqpoint{3.798620in}{0.832413in}}{\pgfqpoint{3.803664in}{0.837456in}}%
\pgfpathcurveto{\pgfqpoint{3.808708in}{0.842500in}}{\pgfqpoint{3.811542in}{0.849342in}}{\pgfqpoint{3.811542in}{0.856474in}}%
\pgfpathcurveto{\pgfqpoint{3.811542in}{0.863607in}}{\pgfqpoint{3.808708in}{0.870449in}}{\pgfqpoint{3.803664in}{0.875493in}}%
\pgfpathcurveto{\pgfqpoint{3.798620in}{0.880536in}}{\pgfqpoint{3.791779in}{0.883370in}}{\pgfqpoint{3.784646in}{0.883370in}}%
\pgfpathcurveto{\pgfqpoint{3.777513in}{0.883370in}}{\pgfqpoint{3.770672in}{0.880536in}}{\pgfqpoint{3.765628in}{0.875493in}}%
\pgfpathcurveto{\pgfqpoint{3.760584in}{0.870449in}}{\pgfqpoint{3.757750in}{0.863607in}}{\pgfqpoint{3.757750in}{0.856474in}}%
\pgfpathcurveto{\pgfqpoint{3.757750in}{0.849342in}}{\pgfqpoint{3.760584in}{0.842500in}}{\pgfqpoint{3.765628in}{0.837456in}}%
\pgfpathcurveto{\pgfqpoint{3.770672in}{0.832413in}}{\pgfqpoint{3.777513in}{0.829579in}}{\pgfqpoint{3.784646in}{0.829579in}}%
\pgfpathclose%
\pgfusepath{stroke,fill}%
\end{pgfscope}%
\begin{pgfscope}%
\pgfpathrectangle{\pgfqpoint{2.867647in}{0.500000in}}{\pgfqpoint{1.764706in}{1.700000in}}%
\pgfusepath{clip}%
\pgfsetbuttcap%
\pgfsetroundjoin%
\definecolor{currentfill}{rgb}{0.964558,0.676556,0.522514}%
\pgfsetfillcolor{currentfill}%
\pgfsetlinewidth{0.311001pt}%
\definecolor{currentstroke}{rgb}{1.000000,1.000000,1.000000}%
\pgfsetstrokecolor{currentstroke}%
\pgfsetdash{}{0pt}%
\pgfpathmoveto{\pgfqpoint{4.315269in}{1.283141in}}%
\pgfpathcurveto{\pgfqpoint{4.322402in}{1.283141in}}{\pgfqpoint{4.329244in}{1.285975in}}{\pgfqpoint{4.334287in}{1.291018in}}%
\pgfpathcurveto{\pgfqpoint{4.339331in}{1.296062in}}{\pgfqpoint{4.342165in}{1.302904in}}{\pgfqpoint{4.342165in}{1.310036in}}%
\pgfpathcurveto{\pgfqpoint{4.342165in}{1.317169in}}{\pgfqpoint{4.339331in}{1.324011in}}{\pgfqpoint{4.334287in}{1.329055in}}%
\pgfpathcurveto{\pgfqpoint{4.329244in}{1.334098in}}{\pgfqpoint{4.322402in}{1.336932in}}{\pgfqpoint{4.315269in}{1.336932in}}%
\pgfpathcurveto{\pgfqpoint{4.308136in}{1.336932in}}{\pgfqpoint{4.301295in}{1.334098in}}{\pgfqpoint{4.296251in}{1.329055in}}%
\pgfpathcurveto{\pgfqpoint{4.291207in}{1.324011in}}{\pgfqpoint{4.288373in}{1.317169in}}{\pgfqpoint{4.288373in}{1.310036in}}%
\pgfpathcurveto{\pgfqpoint{4.288373in}{1.302904in}}{\pgfqpoint{4.291207in}{1.296062in}}{\pgfqpoint{4.296251in}{1.291018in}}%
\pgfpathcurveto{\pgfqpoint{4.301295in}{1.285975in}}{\pgfqpoint{4.308136in}{1.283141in}}{\pgfqpoint{4.315269in}{1.283141in}}%
\pgfpathclose%
\pgfusepath{stroke,fill}%
\end{pgfscope}%
\begin{pgfscope}%
\pgfpathrectangle{\pgfqpoint{2.867647in}{0.500000in}}{\pgfqpoint{1.764706in}{1.700000in}}%
\pgfusepath{clip}%
\pgfsetbuttcap%
\pgfsetroundjoin%
\definecolor{currentfill}{rgb}{0.981377,0.920617,0.865369}%
\pgfsetfillcolor{currentfill}%
\pgfsetlinewidth{0.311001pt}%
\definecolor{currentstroke}{rgb}{1.000000,1.000000,1.000000}%
\pgfsetstrokecolor{currentstroke}%
\pgfsetdash{}{0pt}%
\pgfpathmoveto{\pgfqpoint{4.191071in}{1.262022in}}%
\pgfpathcurveto{\pgfqpoint{4.198204in}{1.262022in}}{\pgfqpoint{4.205046in}{1.264856in}}{\pgfqpoint{4.210089in}{1.269899in}}%
\pgfpathcurveto{\pgfqpoint{4.215133in}{1.274943in}}{\pgfqpoint{4.217967in}{1.281785in}}{\pgfqpoint{4.217967in}{1.288918in}}%
\pgfpathcurveto{\pgfqpoint{4.217967in}{1.296050in}}{\pgfqpoint{4.215133in}{1.302892in}}{\pgfqpoint{4.210089in}{1.307936in}}%
\pgfpathcurveto{\pgfqpoint{4.205046in}{1.312979in}}{\pgfqpoint{4.198204in}{1.315813in}}{\pgfqpoint{4.191071in}{1.315813in}}%
\pgfpathcurveto{\pgfqpoint{4.183939in}{1.315813in}}{\pgfqpoint{4.177097in}{1.312979in}}{\pgfqpoint{4.172053in}{1.307936in}}%
\pgfpathcurveto{\pgfqpoint{4.167010in}{1.302892in}}{\pgfqpoint{4.164176in}{1.296050in}}{\pgfqpoint{4.164176in}{1.288918in}}%
\pgfpathcurveto{\pgfqpoint{4.164176in}{1.281785in}}{\pgfqpoint{4.167010in}{1.274943in}}{\pgfqpoint{4.172053in}{1.269899in}}%
\pgfpathcurveto{\pgfqpoint{4.177097in}{1.264856in}}{\pgfqpoint{4.183939in}{1.262022in}}{\pgfqpoint{4.191071in}{1.262022in}}%
\pgfpathclose%
\pgfusepath{stroke,fill}%
\end{pgfscope}%
\begin{pgfscope}%
\pgfpathrectangle{\pgfqpoint{2.867647in}{0.500000in}}{\pgfqpoint{1.764706in}{1.700000in}}%
\pgfusepath{clip}%
\pgfsetbuttcap%
\pgfsetroundjoin%
\definecolor{currentfill}{rgb}{0.971694,0.833208,0.737161}%
\pgfsetfillcolor{currentfill}%
\pgfsetlinewidth{0.311001pt}%
\definecolor{currentstroke}{rgb}{1.000000,1.000000,1.000000}%
\pgfsetstrokecolor{currentstroke}%
\pgfsetdash{}{0pt}%
\pgfpathmoveto{\pgfqpoint{4.149484in}{1.657039in}}%
\pgfpathcurveto{\pgfqpoint{4.156617in}{1.657039in}}{\pgfqpoint{4.163458in}{1.659873in}}{\pgfqpoint{4.168502in}{1.664916in}}%
\pgfpathcurveto{\pgfqpoint{4.173546in}{1.669960in}}{\pgfqpoint{4.176379in}{1.676802in}}{\pgfqpoint{4.176379in}{1.683935in}}%
\pgfpathcurveto{\pgfqpoint{4.176379in}{1.691067in}}{\pgfqpoint{4.173546in}{1.697909in}}{\pgfqpoint{4.168502in}{1.702953in}}%
\pgfpathcurveto{\pgfqpoint{4.163458in}{1.707996in}}{\pgfqpoint{4.156617in}{1.710830in}}{\pgfqpoint{4.149484in}{1.710830in}}%
\pgfpathcurveto{\pgfqpoint{4.142351in}{1.710830in}}{\pgfqpoint{4.135509in}{1.707996in}}{\pgfqpoint{4.130466in}{1.702953in}}%
\pgfpathcurveto{\pgfqpoint{4.125422in}{1.697909in}}{\pgfqpoint{4.122588in}{1.691067in}}{\pgfqpoint{4.122588in}{1.683935in}}%
\pgfpathcurveto{\pgfqpoint{4.122588in}{1.676802in}}{\pgfqpoint{4.125422in}{1.669960in}}{\pgfqpoint{4.130466in}{1.664916in}}%
\pgfpathcurveto{\pgfqpoint{4.135509in}{1.659873in}}{\pgfqpoint{4.142351in}{1.657039in}}{\pgfqpoint{4.149484in}{1.657039in}}%
\pgfpathclose%
\pgfusepath{stroke,fill}%
\end{pgfscope}%
\begin{pgfscope}%
\pgfpathrectangle{\pgfqpoint{2.867647in}{0.500000in}}{\pgfqpoint{1.764706in}{1.700000in}}%
\pgfusepath{clip}%
\pgfsetbuttcap%
\pgfsetroundjoin%
\definecolor{currentfill}{rgb}{0.917171,0.267738,0.242941}%
\pgfsetfillcolor{currentfill}%
\pgfsetlinewidth{0.311001pt}%
\definecolor{currentstroke}{rgb}{1.000000,1.000000,1.000000}%
\pgfsetstrokecolor{currentstroke}%
\pgfsetdash{}{0pt}%
\pgfpathmoveto{\pgfqpoint{4.376327in}{1.206988in}}%
\pgfpathcurveto{\pgfqpoint{4.383460in}{1.206988in}}{\pgfqpoint{4.390302in}{1.209822in}}{\pgfqpoint{4.395345in}{1.214865in}}%
\pgfpathcurveto{\pgfqpoint{4.400389in}{1.219909in}}{\pgfqpoint{4.403223in}{1.226751in}}{\pgfqpoint{4.403223in}{1.233883in}}%
\pgfpathcurveto{\pgfqpoint{4.403223in}{1.241016in}}{\pgfqpoint{4.400389in}{1.247858in}}{\pgfqpoint{4.395345in}{1.252902in}}%
\pgfpathcurveto{\pgfqpoint{4.390302in}{1.257945in}}{\pgfqpoint{4.383460in}{1.260779in}}{\pgfqpoint{4.376327in}{1.260779in}}%
\pgfpathcurveto{\pgfqpoint{4.369194in}{1.260779in}}{\pgfqpoint{4.362353in}{1.257945in}}{\pgfqpoint{4.357309in}{1.252902in}}%
\pgfpathcurveto{\pgfqpoint{4.352265in}{1.247858in}}{\pgfqpoint{4.349431in}{1.241016in}}{\pgfqpoint{4.349431in}{1.233883in}}%
\pgfpathcurveto{\pgfqpoint{4.349431in}{1.226751in}}{\pgfqpoint{4.352265in}{1.219909in}}{\pgfqpoint{4.357309in}{1.214865in}}%
\pgfpathcurveto{\pgfqpoint{4.362353in}{1.209822in}}{\pgfqpoint{4.369194in}{1.206988in}}{\pgfqpoint{4.376327in}{1.206988in}}%
\pgfpathclose%
\pgfusepath{stroke,fill}%
\end{pgfscope}%
\begin{pgfscope}%
\pgfpathrectangle{\pgfqpoint{2.867647in}{0.500000in}}{\pgfqpoint{1.764706in}{1.700000in}}%
\pgfusepath{clip}%
\pgfsetbuttcap%
\pgfsetroundjoin%
\definecolor{currentfill}{rgb}{0.963559,0.632016,0.472047}%
\pgfsetfillcolor{currentfill}%
\pgfsetlinewidth{0.311001pt}%
\definecolor{currentstroke}{rgb}{1.000000,1.000000,1.000000}%
\pgfsetstrokecolor{currentstroke}%
\pgfsetdash{}{0pt}%
\pgfpathmoveto{\pgfqpoint{4.260877in}{1.615112in}}%
\pgfpathcurveto{\pgfqpoint{4.268010in}{1.615112in}}{\pgfqpoint{4.274851in}{1.617946in}}{\pgfqpoint{4.279895in}{1.622989in}}%
\pgfpathcurveto{\pgfqpoint{4.284939in}{1.628033in}}{\pgfqpoint{4.287773in}{1.634875in}}{\pgfqpoint{4.287773in}{1.642007in}}%
\pgfpathcurveto{\pgfqpoint{4.287773in}{1.649140in}}{\pgfqpoint{4.284939in}{1.655982in}}{\pgfqpoint{4.279895in}{1.661026in}}%
\pgfpathcurveto{\pgfqpoint{4.274851in}{1.666069in}}{\pgfqpoint{4.268010in}{1.668903in}}{\pgfqpoint{4.260877in}{1.668903in}}%
\pgfpathcurveto{\pgfqpoint{4.253744in}{1.668903in}}{\pgfqpoint{4.246903in}{1.666069in}}{\pgfqpoint{4.241859in}{1.661026in}}%
\pgfpathcurveto{\pgfqpoint{4.236815in}{1.655982in}}{\pgfqpoint{4.233981in}{1.649140in}}{\pgfqpoint{4.233981in}{1.642007in}}%
\pgfpathcurveto{\pgfqpoint{4.233981in}{1.634875in}}{\pgfqpoint{4.236815in}{1.628033in}}{\pgfqpoint{4.241859in}{1.622989in}}%
\pgfpathcurveto{\pgfqpoint{4.246903in}{1.617946in}}{\pgfqpoint{4.253744in}{1.615112in}}{\pgfqpoint{4.260877in}{1.615112in}}%
\pgfpathclose%
\pgfusepath{stroke,fill}%
\end{pgfscope}%
\begin{pgfscope}%
\pgfpathrectangle{\pgfqpoint{2.867647in}{0.500000in}}{\pgfqpoint{1.764706in}{1.700000in}}%
\pgfusepath{clip}%
\pgfsetbuttcap%
\pgfsetroundjoin%
\definecolor{currentfill}{rgb}{0.980678,0.914765,0.856766}%
\pgfsetfillcolor{currentfill}%
\pgfsetlinewidth{0.311001pt}%
\definecolor{currentstroke}{rgb}{1.000000,1.000000,1.000000}%
\pgfsetstrokecolor{currentstroke}%
\pgfsetdash{}{0pt}%
\pgfpathmoveto{\pgfqpoint{4.179745in}{1.407527in}}%
\pgfpathcurveto{\pgfqpoint{4.186878in}{1.407527in}}{\pgfqpoint{4.193719in}{1.410361in}}{\pgfqpoint{4.198763in}{1.415405in}}%
\pgfpathcurveto{\pgfqpoint{4.203807in}{1.420449in}}{\pgfqpoint{4.206641in}{1.427290in}}{\pgfqpoint{4.206641in}{1.434423in}}%
\pgfpathcurveto{\pgfqpoint{4.206641in}{1.441556in}}{\pgfqpoint{4.203807in}{1.448398in}}{\pgfqpoint{4.198763in}{1.453441in}}%
\pgfpathcurveto{\pgfqpoint{4.193719in}{1.458485in}}{\pgfqpoint{4.186878in}{1.461319in}}{\pgfqpoint{4.179745in}{1.461319in}}%
\pgfpathcurveto{\pgfqpoint{4.172612in}{1.461319in}}{\pgfqpoint{4.165771in}{1.458485in}}{\pgfqpoint{4.160727in}{1.453441in}}%
\pgfpathcurveto{\pgfqpoint{4.155683in}{1.448398in}}{\pgfqpoint{4.152849in}{1.441556in}}{\pgfqpoint{4.152849in}{1.434423in}}%
\pgfpathcurveto{\pgfqpoint{4.152849in}{1.427290in}}{\pgfqpoint{4.155683in}{1.420449in}}{\pgfqpoint{4.160727in}{1.415405in}}%
\pgfpathcurveto{\pgfqpoint{4.165771in}{1.410361in}}{\pgfqpoint{4.172612in}{1.407527in}}{\pgfqpoint{4.179745in}{1.407527in}}%
\pgfpathclose%
\pgfusepath{stroke,fill}%
\end{pgfscope}%
\begin{pgfscope}%
\pgfpathrectangle{\pgfqpoint{2.867647in}{0.500000in}}{\pgfqpoint{1.764706in}{1.700000in}}%
\pgfusepath{clip}%
\pgfsetbuttcap%
\pgfsetroundjoin%
\definecolor{currentfill}{rgb}{0.971694,0.833208,0.737161}%
\pgfsetfillcolor{currentfill}%
\pgfsetlinewidth{0.311001pt}%
\definecolor{currentstroke}{rgb}{1.000000,1.000000,1.000000}%
\pgfsetstrokecolor{currentstroke}%
\pgfsetdash{}{0pt}%
\pgfpathmoveto{\pgfqpoint{4.264800in}{1.363856in}}%
\pgfpathcurveto{\pgfqpoint{4.271933in}{1.363856in}}{\pgfqpoint{4.278774in}{1.366690in}}{\pgfqpoint{4.283818in}{1.371734in}}%
\pgfpathcurveto{\pgfqpoint{4.288862in}{1.376777in}}{\pgfqpoint{4.291695in}{1.383619in}}{\pgfqpoint{4.291695in}{1.390752in}}%
\pgfpathcurveto{\pgfqpoint{4.291695in}{1.397885in}}{\pgfqpoint{4.288862in}{1.404726in}}{\pgfqpoint{4.283818in}{1.409770in}}%
\pgfpathcurveto{\pgfqpoint{4.278774in}{1.414814in}}{\pgfqpoint{4.271933in}{1.417648in}}{\pgfqpoint{4.264800in}{1.417648in}}%
\pgfpathcurveto{\pgfqpoint{4.257667in}{1.417648in}}{\pgfqpoint{4.250825in}{1.414814in}}{\pgfqpoint{4.245782in}{1.409770in}}%
\pgfpathcurveto{\pgfqpoint{4.240738in}{1.404726in}}{\pgfqpoint{4.237904in}{1.397885in}}{\pgfqpoint{4.237904in}{1.390752in}}%
\pgfpathcurveto{\pgfqpoint{4.237904in}{1.383619in}}{\pgfqpoint{4.240738in}{1.376777in}}{\pgfqpoint{4.245782in}{1.371734in}}%
\pgfpathcurveto{\pgfqpoint{4.250825in}{1.366690in}}{\pgfqpoint{4.257667in}{1.363856in}}{\pgfqpoint{4.264800in}{1.363856in}}%
\pgfpathclose%
\pgfusepath{stroke,fill}%
\end{pgfscope}%
\begin{pgfscope}%
\pgfpathrectangle{\pgfqpoint{2.867647in}{0.500000in}}{\pgfqpoint{1.764706in}{1.700000in}}%
\pgfusepath{clip}%
\pgfsetbuttcap%
\pgfsetroundjoin%
\definecolor{currentfill}{rgb}{0.972201,0.839051,0.745789}%
\pgfsetfillcolor{currentfill}%
\pgfsetlinewidth{0.311001pt}%
\definecolor{currentstroke}{rgb}{1.000000,1.000000,1.000000}%
\pgfsetstrokecolor{currentstroke}%
\pgfsetdash{}{0pt}%
\pgfpathmoveto{\pgfqpoint{4.257238in}{1.404854in}}%
\pgfpathcurveto{\pgfqpoint{4.264370in}{1.404854in}}{\pgfqpoint{4.271212in}{1.407688in}}{\pgfqpoint{4.276256in}{1.412732in}}%
\pgfpathcurveto{\pgfqpoint{4.281299in}{1.417775in}}{\pgfqpoint{4.284133in}{1.424617in}}{\pgfqpoint{4.284133in}{1.431750in}}%
\pgfpathcurveto{\pgfqpoint{4.284133in}{1.438883in}}{\pgfqpoint{4.281299in}{1.445724in}}{\pgfqpoint{4.276256in}{1.450768in}}%
\pgfpathcurveto{\pgfqpoint{4.271212in}{1.455812in}}{\pgfqpoint{4.264370in}{1.458646in}}{\pgfqpoint{4.257238in}{1.458646in}}%
\pgfpathcurveto{\pgfqpoint{4.250105in}{1.458646in}}{\pgfqpoint{4.243263in}{1.455812in}}{\pgfqpoint{4.238220in}{1.450768in}}%
\pgfpathcurveto{\pgfqpoint{4.233176in}{1.445724in}}{\pgfqpoint{4.230342in}{1.438883in}}{\pgfqpoint{4.230342in}{1.431750in}}%
\pgfpathcurveto{\pgfqpoint{4.230342in}{1.424617in}}{\pgfqpoint{4.233176in}{1.417775in}}{\pgfqpoint{4.238220in}{1.412732in}}%
\pgfpathcurveto{\pgfqpoint{4.243263in}{1.407688in}}{\pgfqpoint{4.250105in}{1.404854in}}{\pgfqpoint{4.257238in}{1.404854in}}%
\pgfpathclose%
\pgfusepath{stroke,fill}%
\end{pgfscope}%
\begin{pgfscope}%
\pgfpathrectangle{\pgfqpoint{2.867647in}{0.500000in}}{\pgfqpoint{1.764706in}{1.700000in}}%
\pgfusepath{clip}%
\pgfsetbuttcap%
\pgfsetroundjoin%
\definecolor{currentfill}{rgb}{0.958791,0.526283,0.368909}%
\pgfsetfillcolor{currentfill}%
\pgfsetlinewidth{0.311001pt}%
\definecolor{currentstroke}{rgb}{1.000000,1.000000,1.000000}%
\pgfsetstrokecolor{currentstroke}%
\pgfsetdash{}{0pt}%
\pgfpathmoveto{\pgfqpoint{3.952601in}{1.046319in}}%
\pgfpathcurveto{\pgfqpoint{3.959734in}{1.046319in}}{\pgfqpoint{3.966575in}{1.049153in}}{\pgfqpoint{3.971619in}{1.054196in}}%
\pgfpathcurveto{\pgfqpoint{3.976663in}{1.059240in}}{\pgfqpoint{3.979496in}{1.066082in}}{\pgfqpoint{3.979496in}{1.073215in}}%
\pgfpathcurveto{\pgfqpoint{3.979496in}{1.080347in}}{\pgfqpoint{3.976663in}{1.087189in}}{\pgfqpoint{3.971619in}{1.092233in}}%
\pgfpathcurveto{\pgfqpoint{3.966575in}{1.097276in}}{\pgfqpoint{3.959734in}{1.100110in}}{\pgfqpoint{3.952601in}{1.100110in}}%
\pgfpathcurveto{\pgfqpoint{3.945468in}{1.100110in}}{\pgfqpoint{3.938626in}{1.097276in}}{\pgfqpoint{3.933583in}{1.092233in}}%
\pgfpathcurveto{\pgfqpoint{3.928539in}{1.087189in}}{\pgfqpoint{3.925705in}{1.080347in}}{\pgfqpoint{3.925705in}{1.073215in}}%
\pgfpathcurveto{\pgfqpoint{3.925705in}{1.066082in}}{\pgfqpoint{3.928539in}{1.059240in}}{\pgfqpoint{3.933583in}{1.054196in}}%
\pgfpathcurveto{\pgfqpoint{3.938626in}{1.049153in}}{\pgfqpoint{3.945468in}{1.046319in}}{\pgfqpoint{3.952601in}{1.046319in}}%
\pgfpathclose%
\pgfusepath{stroke,fill}%
\end{pgfscope}%
\begin{pgfscope}%
\pgfpathrectangle{\pgfqpoint{2.867647in}{0.500000in}}{\pgfqpoint{1.764706in}{1.700000in}}%
\pgfusepath{clip}%
\pgfsetbuttcap%
\pgfsetroundjoin%
\definecolor{currentfill}{rgb}{0.964558,0.676556,0.522514}%
\pgfsetfillcolor{currentfill}%
\pgfsetlinewidth{0.311001pt}%
\definecolor{currentstroke}{rgb}{1.000000,1.000000,1.000000}%
\pgfsetstrokecolor{currentstroke}%
\pgfsetdash{}{0pt}%
\pgfpathmoveto{\pgfqpoint{3.969597in}{1.694527in}}%
\pgfpathcurveto{\pgfqpoint{3.976730in}{1.694527in}}{\pgfqpoint{3.983572in}{1.697361in}}{\pgfqpoint{3.988615in}{1.702404in}}%
\pgfpathcurveto{\pgfqpoint{3.993659in}{1.707448in}}{\pgfqpoint{3.996493in}{1.714290in}}{\pgfqpoint{3.996493in}{1.721422in}}%
\pgfpathcurveto{\pgfqpoint{3.996493in}{1.728555in}}{\pgfqpoint{3.993659in}{1.735397in}}{\pgfqpoint{3.988615in}{1.740441in}}%
\pgfpathcurveto{\pgfqpoint{3.983572in}{1.745484in}}{\pgfqpoint{3.976730in}{1.748318in}}{\pgfqpoint{3.969597in}{1.748318in}}%
\pgfpathcurveto{\pgfqpoint{3.962464in}{1.748318in}}{\pgfqpoint{3.955623in}{1.745484in}}{\pgfqpoint{3.950579in}{1.740441in}}%
\pgfpathcurveto{\pgfqpoint{3.945535in}{1.735397in}}{\pgfqpoint{3.942702in}{1.728555in}}{\pgfqpoint{3.942702in}{1.721422in}}%
\pgfpathcurveto{\pgfqpoint{3.942702in}{1.714290in}}{\pgfqpoint{3.945535in}{1.707448in}}{\pgfqpoint{3.950579in}{1.702404in}}%
\pgfpathcurveto{\pgfqpoint{3.955623in}{1.697361in}}{\pgfqpoint{3.962464in}{1.694527in}}{\pgfqpoint{3.969597in}{1.694527in}}%
\pgfpathclose%
\pgfusepath{stroke,fill}%
\end{pgfscope}%
\begin{pgfscope}%
\pgfpathrectangle{\pgfqpoint{2.867647in}{0.500000in}}{\pgfqpoint{1.764706in}{1.700000in}}%
\pgfusepath{clip}%
\pgfsetbuttcap%
\pgfsetroundjoin%
\definecolor{currentfill}{rgb}{0.964920,0.695342,0.545192}%
\pgfsetfillcolor{currentfill}%
\pgfsetlinewidth{0.311001pt}%
\definecolor{currentstroke}{rgb}{1.000000,1.000000,1.000000}%
\pgfsetstrokecolor{currentstroke}%
\pgfsetdash{}{0pt}%
\pgfpathmoveto{\pgfqpoint{4.194721in}{1.680863in}}%
\pgfpathcurveto{\pgfqpoint{4.201854in}{1.680863in}}{\pgfqpoint{4.208696in}{1.683697in}}{\pgfqpoint{4.213740in}{1.688740in}}%
\pgfpathcurveto{\pgfqpoint{4.218783in}{1.693784in}}{\pgfqpoint{4.221617in}{1.700626in}}{\pgfqpoint{4.221617in}{1.707758in}}%
\pgfpathcurveto{\pgfqpoint{4.221617in}{1.714891in}}{\pgfqpoint{4.218783in}{1.721733in}}{\pgfqpoint{4.213740in}{1.726776in}}%
\pgfpathcurveto{\pgfqpoint{4.208696in}{1.731820in}}{\pgfqpoint{4.201854in}{1.734654in}}{\pgfqpoint{4.194721in}{1.734654in}}%
\pgfpathcurveto{\pgfqpoint{4.187589in}{1.734654in}}{\pgfqpoint{4.180747in}{1.731820in}}{\pgfqpoint{4.175703in}{1.726776in}}%
\pgfpathcurveto{\pgfqpoint{4.170660in}{1.721733in}}{\pgfqpoint{4.167826in}{1.714891in}}{\pgfqpoint{4.167826in}{1.707758in}}%
\pgfpathcurveto{\pgfqpoint{4.167826in}{1.700626in}}{\pgfqpoint{4.170660in}{1.693784in}}{\pgfqpoint{4.175703in}{1.688740in}}%
\pgfpathcurveto{\pgfqpoint{4.180747in}{1.683697in}}{\pgfqpoint{4.187589in}{1.680863in}}{\pgfqpoint{4.194721in}{1.680863in}}%
\pgfpathclose%
\pgfusepath{stroke,fill}%
\end{pgfscope}%
\begin{pgfscope}%
\pgfpathrectangle{\pgfqpoint{2.867647in}{0.500000in}}{\pgfqpoint{1.764706in}{1.700000in}}%
\pgfusepath{clip}%
\pgfsetbuttcap%
\pgfsetroundjoin%
\definecolor{currentfill}{rgb}{0.971202,0.827364,0.728520}%
\pgfsetfillcolor{currentfill}%
\pgfsetlinewidth{0.311001pt}%
\definecolor{currentstroke}{rgb}{1.000000,1.000000,1.000000}%
\pgfsetstrokecolor{currentstroke}%
\pgfsetdash{}{0pt}%
\pgfpathmoveto{\pgfqpoint{4.139822in}{1.665403in}}%
\pgfpathcurveto{\pgfqpoint{4.146955in}{1.665403in}}{\pgfqpoint{4.153797in}{1.668237in}}{\pgfqpoint{4.158841in}{1.673280in}}%
\pgfpathcurveto{\pgfqpoint{4.163884in}{1.678324in}}{\pgfqpoint{4.166718in}{1.685166in}}{\pgfqpoint{4.166718in}{1.692299in}}%
\pgfpathcurveto{\pgfqpoint{4.166718in}{1.699431in}}{\pgfqpoint{4.163884in}{1.706273in}}{\pgfqpoint{4.158841in}{1.711317in}}%
\pgfpathcurveto{\pgfqpoint{4.153797in}{1.716360in}}{\pgfqpoint{4.146955in}{1.719194in}}{\pgfqpoint{4.139822in}{1.719194in}}%
\pgfpathcurveto{\pgfqpoint{4.132690in}{1.719194in}}{\pgfqpoint{4.125848in}{1.716360in}}{\pgfqpoint{4.120804in}{1.711317in}}%
\pgfpathcurveto{\pgfqpoint{4.115761in}{1.706273in}}{\pgfqpoint{4.112927in}{1.699431in}}{\pgfqpoint{4.112927in}{1.692299in}}%
\pgfpathcurveto{\pgfqpoint{4.112927in}{1.685166in}}{\pgfqpoint{4.115761in}{1.678324in}}{\pgfqpoint{4.120804in}{1.673280in}}%
\pgfpathcurveto{\pgfqpoint{4.125848in}{1.668237in}}{\pgfqpoint{4.132690in}{1.665403in}}{\pgfqpoint{4.139822in}{1.665403in}}%
\pgfpathclose%
\pgfusepath{stroke,fill}%
\end{pgfscope}%
\begin{pgfscope}%
\pgfpathrectangle{\pgfqpoint{2.867647in}{0.500000in}}{\pgfqpoint{1.764706in}{1.700000in}}%
\pgfusepath{clip}%
\pgfsetbuttcap%
\pgfsetroundjoin%
\definecolor{currentfill}{rgb}{0.955697,0.484891,0.334214}%
\pgfsetfillcolor{currentfill}%
\pgfsetlinewidth{0.311001pt}%
\definecolor{currentstroke}{rgb}{1.000000,1.000000,1.000000}%
\pgfsetstrokecolor{currentstroke}%
\pgfsetdash{}{0pt}%
\pgfpathmoveto{\pgfqpoint{4.337394in}{1.459539in}}%
\pgfpathcurveto{\pgfqpoint{4.344527in}{1.459539in}}{\pgfqpoint{4.351368in}{1.462372in}}{\pgfqpoint{4.356412in}{1.467416in}}%
\pgfpathcurveto{\pgfqpoint{4.361456in}{1.472460in}}{\pgfqpoint{4.364289in}{1.479301in}}{\pgfqpoint{4.364289in}{1.486434in}}%
\pgfpathcurveto{\pgfqpoint{4.364289in}{1.493567in}}{\pgfqpoint{4.361456in}{1.500409in}}{\pgfqpoint{4.356412in}{1.505452in}}%
\pgfpathcurveto{\pgfqpoint{4.351368in}{1.510496in}}{\pgfqpoint{4.344527in}{1.513330in}}{\pgfqpoint{4.337394in}{1.513330in}}%
\pgfpathcurveto{\pgfqpoint{4.330261in}{1.513330in}}{\pgfqpoint{4.323419in}{1.510496in}}{\pgfqpoint{4.318376in}{1.505452in}}%
\pgfpathcurveto{\pgfqpoint{4.313332in}{1.500409in}}{\pgfqpoint{4.310498in}{1.493567in}}{\pgfqpoint{4.310498in}{1.486434in}}%
\pgfpathcurveto{\pgfqpoint{4.310498in}{1.479301in}}{\pgfqpoint{4.313332in}{1.472460in}}{\pgfqpoint{4.318376in}{1.467416in}}%
\pgfpathcurveto{\pgfqpoint{4.323419in}{1.462372in}}{\pgfqpoint{4.330261in}{1.459539in}}{\pgfqpoint{4.337394in}{1.459539in}}%
\pgfpathclose%
\pgfusepath{stroke,fill}%
\end{pgfscope}%
\begin{pgfscope}%
\pgfpathrectangle{\pgfqpoint{2.867647in}{0.500000in}}{\pgfqpoint{1.764706in}{1.700000in}}%
\pgfusepath{clip}%
\pgfsetbuttcap%
\pgfsetroundjoin%
\definecolor{currentfill}{rgb}{0.980678,0.914765,0.856766}%
\pgfsetfillcolor{currentfill}%
\pgfsetlinewidth{0.311001pt}%
\definecolor{currentstroke}{rgb}{1.000000,1.000000,1.000000}%
\pgfsetstrokecolor{currentstroke}%
\pgfsetdash{}{0pt}%
\pgfpathmoveto{\pgfqpoint{4.172245in}{1.419615in}}%
\pgfpathcurveto{\pgfqpoint{4.179378in}{1.419615in}}{\pgfqpoint{4.186220in}{1.422449in}}{\pgfqpoint{4.191263in}{1.427492in}}%
\pgfpathcurveto{\pgfqpoint{4.196307in}{1.432536in}}{\pgfqpoint{4.199141in}{1.439378in}}{\pgfqpoint{4.199141in}{1.446510in}}%
\pgfpathcurveto{\pgfqpoint{4.199141in}{1.453643in}}{\pgfqpoint{4.196307in}{1.460485in}}{\pgfqpoint{4.191263in}{1.465529in}}%
\pgfpathcurveto{\pgfqpoint{4.186220in}{1.470572in}}{\pgfqpoint{4.179378in}{1.473406in}}{\pgfqpoint{4.172245in}{1.473406in}}%
\pgfpathcurveto{\pgfqpoint{4.165112in}{1.473406in}}{\pgfqpoint{4.158271in}{1.470572in}}{\pgfqpoint{4.153227in}{1.465529in}}%
\pgfpathcurveto{\pgfqpoint{4.148183in}{1.460485in}}{\pgfqpoint{4.145350in}{1.453643in}}{\pgfqpoint{4.145350in}{1.446510in}}%
\pgfpathcurveto{\pgfqpoint{4.145350in}{1.439378in}}{\pgfqpoint{4.148183in}{1.432536in}}{\pgfqpoint{4.153227in}{1.427492in}}%
\pgfpathcurveto{\pgfqpoint{4.158271in}{1.422449in}}{\pgfqpoint{4.165112in}{1.419615in}}{\pgfqpoint{4.172245in}{1.419615in}}%
\pgfpathclose%
\pgfusepath{stroke,fill}%
\end{pgfscope}%
\begin{pgfscope}%
\pgfpathrectangle{\pgfqpoint{2.867647in}{0.500000in}}{\pgfqpoint{1.764706in}{1.700000in}}%
\pgfusepath{clip}%
\pgfsetbuttcap%
\pgfsetroundjoin%
\definecolor{currentfill}{rgb}{0.974412,0.862387,0.780156}%
\pgfsetfillcolor{currentfill}%
\pgfsetlinewidth{0.311001pt}%
\definecolor{currentstroke}{rgb}{1.000000,1.000000,1.000000}%
\pgfsetstrokecolor{currentstroke}%
\pgfsetdash{}{0pt}%
\pgfpathmoveto{\pgfqpoint{4.176746in}{1.043929in}}%
\pgfpathcurveto{\pgfqpoint{4.183878in}{1.043929in}}{\pgfqpoint{4.190720in}{1.046762in}}{\pgfqpoint{4.195764in}{1.051806in}}%
\pgfpathcurveto{\pgfqpoint{4.200807in}{1.056850in}}{\pgfqpoint{4.203641in}{1.063691in}}{\pgfqpoint{4.203641in}{1.070824in}}%
\pgfpathcurveto{\pgfqpoint{4.203641in}{1.077957in}}{\pgfqpoint{4.200807in}{1.084799in}}{\pgfqpoint{4.195764in}{1.089842in}}%
\pgfpathcurveto{\pgfqpoint{4.190720in}{1.094886in}}{\pgfqpoint{4.183878in}{1.097720in}}{\pgfqpoint{4.176746in}{1.097720in}}%
\pgfpathcurveto{\pgfqpoint{4.169613in}{1.097720in}}{\pgfqpoint{4.162771in}{1.094886in}}{\pgfqpoint{4.157727in}{1.089842in}}%
\pgfpathcurveto{\pgfqpoint{4.152684in}{1.084799in}}{\pgfqpoint{4.149850in}{1.077957in}}{\pgfqpoint{4.149850in}{1.070824in}}%
\pgfpathcurveto{\pgfqpoint{4.149850in}{1.063691in}}{\pgfqpoint{4.152684in}{1.056850in}}{\pgfqpoint{4.157727in}{1.051806in}}%
\pgfpathcurveto{\pgfqpoint{4.162771in}{1.046762in}}{\pgfqpoint{4.169613in}{1.043929in}}{\pgfqpoint{4.176746in}{1.043929in}}%
\pgfpathclose%
\pgfusepath{stroke,fill}%
\end{pgfscope}%
\begin{pgfscope}%
\pgfpathrectangle{\pgfqpoint{2.867647in}{0.500000in}}{\pgfqpoint{1.764706in}{1.700000in}}%
\pgfusepath{clip}%
\pgfsetbuttcap%
\pgfsetroundjoin%
\definecolor{currentfill}{rgb}{0.964173,0.657587,0.500469}%
\pgfsetfillcolor{currentfill}%
\pgfsetlinewidth{0.311001pt}%
\definecolor{currentstroke}{rgb}{1.000000,1.000000,1.000000}%
\pgfsetstrokecolor{currentstroke}%
\pgfsetdash{}{0pt}%
\pgfpathmoveto{\pgfqpoint{4.233445in}{1.648552in}}%
\pgfpathcurveto{\pgfqpoint{4.240578in}{1.648552in}}{\pgfqpoint{4.247420in}{1.651386in}}{\pgfqpoint{4.252464in}{1.656430in}}%
\pgfpathcurveto{\pgfqpoint{4.257507in}{1.661473in}}{\pgfqpoint{4.260341in}{1.668315in}}{\pgfqpoint{4.260341in}{1.675448in}}%
\pgfpathcurveto{\pgfqpoint{4.260341in}{1.682581in}}{\pgfqpoint{4.257507in}{1.689422in}}{\pgfqpoint{4.252464in}{1.694466in}}%
\pgfpathcurveto{\pgfqpoint{4.247420in}{1.699510in}}{\pgfqpoint{4.240578in}{1.702344in}}{\pgfqpoint{4.233445in}{1.702344in}}%
\pgfpathcurveto{\pgfqpoint{4.226313in}{1.702344in}}{\pgfqpoint{4.219471in}{1.699510in}}{\pgfqpoint{4.214427in}{1.694466in}}%
\pgfpathcurveto{\pgfqpoint{4.209384in}{1.689422in}}{\pgfqpoint{4.206550in}{1.682581in}}{\pgfqpoint{4.206550in}{1.675448in}}%
\pgfpathcurveto{\pgfqpoint{4.206550in}{1.668315in}}{\pgfqpoint{4.209384in}{1.661473in}}{\pgfqpoint{4.214427in}{1.656430in}}%
\pgfpathcurveto{\pgfqpoint{4.219471in}{1.651386in}}{\pgfqpoint{4.226313in}{1.648552in}}{\pgfqpoint{4.233445in}{1.648552in}}%
\pgfpathclose%
\pgfusepath{stroke,fill}%
\end{pgfscope}%
\begin{pgfscope}%
\pgfpathrectangle{\pgfqpoint{2.867647in}{0.500000in}}{\pgfqpoint{1.764706in}{1.700000in}}%
\pgfusepath{clip}%
\pgfsetbuttcap%
\pgfsetroundjoin%
\definecolor{currentfill}{rgb}{0.755358,0.089198,0.328762}%
\pgfsetfillcolor{currentfill}%
\pgfsetlinewidth{0.311001pt}%
\definecolor{currentstroke}{rgb}{1.000000,1.000000,1.000000}%
\pgfsetstrokecolor{currentstroke}%
\pgfsetdash{}{0pt}%
\pgfpathmoveto{\pgfqpoint{4.071627in}{0.733998in}}%
\pgfpathcurveto{\pgfqpoint{4.078759in}{0.733998in}}{\pgfqpoint{4.085601in}{0.736832in}}{\pgfqpoint{4.090645in}{0.741875in}}%
\pgfpathcurveto{\pgfqpoint{4.095688in}{0.746919in}}{\pgfqpoint{4.098522in}{0.753761in}}{\pgfqpoint{4.098522in}{0.760893in}}%
\pgfpathcurveto{\pgfqpoint{4.098522in}{0.768026in}}{\pgfqpoint{4.095688in}{0.774868in}}{\pgfqpoint{4.090645in}{0.779912in}}%
\pgfpathcurveto{\pgfqpoint{4.085601in}{0.784955in}}{\pgfqpoint{4.078759in}{0.787789in}}{\pgfqpoint{4.071627in}{0.787789in}}%
\pgfpathcurveto{\pgfqpoint{4.064494in}{0.787789in}}{\pgfqpoint{4.057652in}{0.784955in}}{\pgfqpoint{4.052608in}{0.779912in}}%
\pgfpathcurveto{\pgfqpoint{4.047565in}{0.774868in}}{\pgfqpoint{4.044731in}{0.768026in}}{\pgfqpoint{4.044731in}{0.760893in}}%
\pgfpathcurveto{\pgfqpoint{4.044731in}{0.753761in}}{\pgfqpoint{4.047565in}{0.746919in}}{\pgfqpoint{4.052608in}{0.741875in}}%
\pgfpathcurveto{\pgfqpoint{4.057652in}{0.736832in}}{\pgfqpoint{4.064494in}{0.733998in}}{\pgfqpoint{4.071627in}{0.733998in}}%
\pgfpathclose%
\pgfusepath{stroke,fill}%
\end{pgfscope}%
\begin{pgfscope}%
\pgfpathrectangle{\pgfqpoint{2.867647in}{0.500000in}}{\pgfqpoint{1.764706in}{1.700000in}}%
\pgfusepath{clip}%
\pgfsetbuttcap%
\pgfsetroundjoin%
\definecolor{currentfill}{rgb}{0.962765,0.606121,0.444717}%
\pgfsetfillcolor{currentfill}%
\pgfsetlinewidth{0.311001pt}%
\definecolor{currentstroke}{rgb}{1.000000,1.000000,1.000000}%
\pgfsetstrokecolor{currentstroke}%
\pgfsetdash{}{0pt}%
\pgfpathmoveto{\pgfqpoint{3.951761in}{1.761382in}}%
\pgfpathcurveto{\pgfqpoint{3.958894in}{1.761382in}}{\pgfqpoint{3.965736in}{1.764216in}}{\pgfqpoint{3.970780in}{1.769260in}}%
\pgfpathcurveto{\pgfqpoint{3.975823in}{1.774304in}}{\pgfqpoint{3.978657in}{1.781145in}}{\pgfqpoint{3.978657in}{1.788278in}}%
\pgfpathcurveto{\pgfqpoint{3.978657in}{1.795411in}}{\pgfqpoint{3.975823in}{1.802253in}}{\pgfqpoint{3.970780in}{1.807296in}}%
\pgfpathcurveto{\pgfqpoint{3.965736in}{1.812340in}}{\pgfqpoint{3.958894in}{1.815174in}}{\pgfqpoint{3.951761in}{1.815174in}}%
\pgfpathcurveto{\pgfqpoint{3.944629in}{1.815174in}}{\pgfqpoint{3.937787in}{1.812340in}}{\pgfqpoint{3.932743in}{1.807296in}}%
\pgfpathcurveto{\pgfqpoint{3.927700in}{1.802253in}}{\pgfqpoint{3.924866in}{1.795411in}}{\pgfqpoint{3.924866in}{1.788278in}}%
\pgfpathcurveto{\pgfqpoint{3.924866in}{1.781145in}}{\pgfqpoint{3.927700in}{1.774304in}}{\pgfqpoint{3.932743in}{1.769260in}}%
\pgfpathcurveto{\pgfqpoint{3.937787in}{1.764216in}}{\pgfqpoint{3.944629in}{1.761382in}}{\pgfqpoint{3.951761in}{1.761382in}}%
\pgfpathclose%
\pgfusepath{stroke,fill}%
\end{pgfscope}%
\begin{pgfscope}%
\pgfpathrectangle{\pgfqpoint{2.867647in}{0.500000in}}{\pgfqpoint{1.764706in}{1.700000in}}%
\pgfusepath{clip}%
\pgfsetbuttcap%
\pgfsetroundjoin%
\definecolor{currentfill}{rgb}{0.966812,0.762584,0.633643}%
\pgfsetfillcolor{currentfill}%
\pgfsetlinewidth{0.311001pt}%
\definecolor{currentstroke}{rgb}{1.000000,1.000000,1.000000}%
\pgfsetstrokecolor{currentstroke}%
\pgfsetdash{}{0pt}%
\pgfpathmoveto{\pgfqpoint{4.018809in}{1.689991in}}%
\pgfpathcurveto{\pgfqpoint{4.025942in}{1.689991in}}{\pgfqpoint{4.032783in}{1.692825in}}{\pgfqpoint{4.037827in}{1.697868in}}%
\pgfpathcurveto{\pgfqpoint{4.042871in}{1.702912in}}{\pgfqpoint{4.045705in}{1.709754in}}{\pgfqpoint{4.045705in}{1.716887in}}%
\pgfpathcurveto{\pgfqpoint{4.045705in}{1.724019in}}{\pgfqpoint{4.042871in}{1.730861in}}{\pgfqpoint{4.037827in}{1.735905in}}%
\pgfpathcurveto{\pgfqpoint{4.032783in}{1.740948in}}{\pgfqpoint{4.025942in}{1.743782in}}{\pgfqpoint{4.018809in}{1.743782in}}%
\pgfpathcurveto{\pgfqpoint{4.011676in}{1.743782in}}{\pgfqpoint{4.004834in}{1.740948in}}{\pgfqpoint{3.999791in}{1.735905in}}%
\pgfpathcurveto{\pgfqpoint{3.994747in}{1.730861in}}{\pgfqpoint{3.991913in}{1.724019in}}{\pgfqpoint{3.991913in}{1.716887in}}%
\pgfpathcurveto{\pgfqpoint{3.991913in}{1.709754in}}{\pgfqpoint{3.994747in}{1.702912in}}{\pgfqpoint{3.999791in}{1.697868in}}%
\pgfpathcurveto{\pgfqpoint{4.004834in}{1.692825in}}{\pgfqpoint{4.011676in}{1.689991in}}{\pgfqpoint{4.018809in}{1.689991in}}%
\pgfpathclose%
\pgfusepath{stroke,fill}%
\end{pgfscope}%
\begin{pgfscope}%
\pgfpathrectangle{\pgfqpoint{2.867647in}{0.500000in}}{\pgfqpoint{1.764706in}{1.700000in}}%
\pgfusepath{clip}%
\pgfsetbuttcap%
\pgfsetroundjoin%
\definecolor{currentfill}{rgb}{0.965440,0.720101,0.576404}%
\pgfsetfillcolor{currentfill}%
\pgfsetlinewidth{0.311001pt}%
\definecolor{currentstroke}{rgb}{1.000000,1.000000,1.000000}%
\pgfsetstrokecolor{currentstroke}%
\pgfsetdash{}{0pt}%
\pgfpathmoveto{\pgfqpoint{3.998025in}{0.946395in}}%
\pgfpathcurveto{\pgfqpoint{4.005158in}{0.946395in}}{\pgfqpoint{4.011999in}{0.949229in}}{\pgfqpoint{4.017043in}{0.954273in}}%
\pgfpathcurveto{\pgfqpoint{4.022087in}{0.959316in}}{\pgfqpoint{4.024921in}{0.966158in}}{\pgfqpoint{4.024921in}{0.973291in}}%
\pgfpathcurveto{\pgfqpoint{4.024921in}{0.980424in}}{\pgfqpoint{4.022087in}{0.987265in}}{\pgfqpoint{4.017043in}{0.992309in}}%
\pgfpathcurveto{\pgfqpoint{4.011999in}{0.997353in}}{\pgfqpoint{4.005158in}{1.000187in}}{\pgfqpoint{3.998025in}{1.000187in}}%
\pgfpathcurveto{\pgfqpoint{3.990892in}{1.000187in}}{\pgfqpoint{3.984051in}{0.997353in}}{\pgfqpoint{3.979007in}{0.992309in}}%
\pgfpathcurveto{\pgfqpoint{3.973963in}{0.987265in}}{\pgfqpoint{3.971129in}{0.980424in}}{\pgfqpoint{3.971129in}{0.973291in}}%
\pgfpathcurveto{\pgfqpoint{3.971129in}{0.966158in}}{\pgfqpoint{3.973963in}{0.959316in}}{\pgfqpoint{3.979007in}{0.954273in}}%
\pgfpathcurveto{\pgfqpoint{3.984051in}{0.949229in}}{\pgfqpoint{3.990892in}{0.946395in}}{\pgfqpoint{3.998025in}{0.946395in}}%
\pgfpathclose%
\pgfusepath{stroke,fill}%
\end{pgfscope}%
\begin{pgfscope}%
\pgfpathrectangle{\pgfqpoint{2.867647in}{0.500000in}}{\pgfqpoint{1.764706in}{1.700000in}}%
\pgfusepath{clip}%
\pgfsetbuttcap%
\pgfsetroundjoin%
\definecolor{currentfill}{rgb}{0.973832,0.856556,0.771584}%
\pgfsetfillcolor{currentfill}%
\pgfsetlinewidth{0.311001pt}%
\definecolor{currentstroke}{rgb}{1.000000,1.000000,1.000000}%
\pgfsetstrokecolor{currentstroke}%
\pgfsetdash{}{0pt}%
\pgfpathmoveto{\pgfqpoint{4.161809in}{1.018363in}}%
\pgfpathcurveto{\pgfqpoint{4.168942in}{1.018363in}}{\pgfqpoint{4.175784in}{1.021196in}}{\pgfqpoint{4.180828in}{1.026240in}}%
\pgfpathcurveto{\pgfqpoint{4.185871in}{1.031284in}}{\pgfqpoint{4.188705in}{1.038125in}}{\pgfqpoint{4.188705in}{1.045258in}}%
\pgfpathcurveto{\pgfqpoint{4.188705in}{1.052391in}}{\pgfqpoint{4.185871in}{1.059233in}}{\pgfqpoint{4.180828in}{1.064276in}}%
\pgfpathcurveto{\pgfqpoint{4.175784in}{1.069320in}}{\pgfqpoint{4.168942in}{1.072154in}}{\pgfqpoint{4.161809in}{1.072154in}}%
\pgfpathcurveto{\pgfqpoint{4.154677in}{1.072154in}}{\pgfqpoint{4.147835in}{1.069320in}}{\pgfqpoint{4.142791in}{1.064276in}}%
\pgfpathcurveto{\pgfqpoint{4.137748in}{1.059233in}}{\pgfqpoint{4.134914in}{1.052391in}}{\pgfqpoint{4.134914in}{1.045258in}}%
\pgfpathcurveto{\pgfqpoint{4.134914in}{1.038125in}}{\pgfqpoint{4.137748in}{1.031284in}}{\pgfqpoint{4.142791in}{1.026240in}}%
\pgfpathcurveto{\pgfqpoint{4.147835in}{1.021196in}}{\pgfqpoint{4.154677in}{1.018363in}}{\pgfqpoint{4.161809in}{1.018363in}}%
\pgfpathclose%
\pgfusepath{stroke,fill}%
\end{pgfscope}%
\begin{pgfscope}%
\pgfpathrectangle{\pgfqpoint{2.867647in}{0.500000in}}{\pgfqpoint{1.764706in}{1.700000in}}%
\pgfusepath{clip}%
\pgfsetbuttcap%
\pgfsetroundjoin%
\definecolor{currentfill}{rgb}{0.973832,0.856556,0.771584}%
\pgfsetfillcolor{currentfill}%
\pgfsetlinewidth{0.311001pt}%
\definecolor{currentstroke}{rgb}{1.000000,1.000000,1.000000}%
\pgfsetstrokecolor{currentstroke}%
\pgfsetdash{}{0pt}%
\pgfpathmoveto{\pgfqpoint{4.126545in}{1.328710in}}%
\pgfpathcurveto{\pgfqpoint{4.133678in}{1.328710in}}{\pgfqpoint{4.140520in}{1.331544in}}{\pgfqpoint{4.145564in}{1.336588in}}%
\pgfpathcurveto{\pgfqpoint{4.150607in}{1.341631in}}{\pgfqpoint{4.153441in}{1.348473in}}{\pgfqpoint{4.153441in}{1.355606in}}%
\pgfpathcurveto{\pgfqpoint{4.153441in}{1.362739in}}{\pgfqpoint{4.150607in}{1.369580in}}{\pgfqpoint{4.145564in}{1.374624in}}%
\pgfpathcurveto{\pgfqpoint{4.140520in}{1.379668in}}{\pgfqpoint{4.133678in}{1.382501in}}{\pgfqpoint{4.126545in}{1.382501in}}%
\pgfpathcurveto{\pgfqpoint{4.119413in}{1.382501in}}{\pgfqpoint{4.112571in}{1.379668in}}{\pgfqpoint{4.107527in}{1.374624in}}%
\pgfpathcurveto{\pgfqpoint{4.102484in}{1.369580in}}{\pgfqpoint{4.099650in}{1.362739in}}{\pgfqpoint{4.099650in}{1.355606in}}%
\pgfpathcurveto{\pgfqpoint{4.099650in}{1.348473in}}{\pgfqpoint{4.102484in}{1.341631in}}{\pgfqpoint{4.107527in}{1.336588in}}%
\pgfpathcurveto{\pgfqpoint{4.112571in}{1.331544in}}{\pgfqpoint{4.119413in}{1.328710in}}{\pgfqpoint{4.126545in}{1.328710in}}%
\pgfpathclose%
\pgfusepath{stroke,fill}%
\end{pgfscope}%
\begin{pgfscope}%
\pgfpathrectangle{\pgfqpoint{2.867647in}{0.500000in}}{\pgfqpoint{1.764706in}{1.700000in}}%
\pgfusepath{clip}%
\pgfsetbuttcap%
\pgfsetroundjoin%
\definecolor{currentfill}{rgb}{0.922239,0.282873,0.242296}%
\pgfsetfillcolor{currentfill}%
\pgfsetlinewidth{0.311001pt}%
\definecolor{currentstroke}{rgb}{1.000000,1.000000,1.000000}%
\pgfsetstrokecolor{currentstroke}%
\pgfsetdash{}{0pt}%
\pgfpathmoveto{\pgfqpoint{4.271514in}{1.715759in}}%
\pgfpathcurveto{\pgfqpoint{4.278646in}{1.715759in}}{\pgfqpoint{4.285488in}{1.718593in}}{\pgfqpoint{4.290532in}{1.723637in}}%
\pgfpathcurveto{\pgfqpoint{4.295575in}{1.728680in}}{\pgfqpoint{4.298409in}{1.735522in}}{\pgfqpoint{4.298409in}{1.742655in}}%
\pgfpathcurveto{\pgfqpoint{4.298409in}{1.749788in}}{\pgfqpoint{4.295575in}{1.756629in}}{\pgfqpoint{4.290532in}{1.761673in}}%
\pgfpathcurveto{\pgfqpoint{4.285488in}{1.766717in}}{\pgfqpoint{4.278646in}{1.769550in}}{\pgfqpoint{4.271514in}{1.769550in}}%
\pgfpathcurveto{\pgfqpoint{4.264381in}{1.769550in}}{\pgfqpoint{4.257539in}{1.766717in}}{\pgfqpoint{4.252495in}{1.761673in}}%
\pgfpathcurveto{\pgfqpoint{4.247452in}{1.756629in}}{\pgfqpoint{4.244618in}{1.749788in}}{\pgfqpoint{4.244618in}{1.742655in}}%
\pgfpathcurveto{\pgfqpoint{4.244618in}{1.735522in}}{\pgfqpoint{4.247452in}{1.728680in}}{\pgfqpoint{4.252495in}{1.723637in}}%
\pgfpathcurveto{\pgfqpoint{4.257539in}{1.718593in}}{\pgfqpoint{4.264381in}{1.715759in}}{\pgfqpoint{4.271514in}{1.715759in}}%
\pgfpathclose%
\pgfusepath{stroke,fill}%
\end{pgfscope}%
\begin{pgfscope}%
\pgfpathrectangle{\pgfqpoint{2.867647in}{0.500000in}}{\pgfqpoint{1.764706in}{1.700000in}}%
\pgfusepath{clip}%
\pgfsetbuttcap%
\pgfsetroundjoin%
\definecolor{currentfill}{rgb}{0.972201,0.839051,0.745789}%
\pgfsetfillcolor{currentfill}%
\pgfsetlinewidth{0.311001pt}%
\definecolor{currentstroke}{rgb}{1.000000,1.000000,1.000000}%
\pgfsetstrokecolor{currentstroke}%
\pgfsetdash{}{0pt}%
\pgfpathmoveto{\pgfqpoint{4.235260in}{1.506340in}}%
\pgfpathcurveto{\pgfqpoint{4.242393in}{1.506340in}}{\pgfqpoint{4.249234in}{1.509174in}}{\pgfqpoint{4.254278in}{1.514217in}}%
\pgfpathcurveto{\pgfqpoint{4.259322in}{1.519261in}}{\pgfqpoint{4.262156in}{1.526103in}}{\pgfqpoint{4.262156in}{1.533236in}}%
\pgfpathcurveto{\pgfqpoint{4.262156in}{1.540368in}}{\pgfqpoint{4.259322in}{1.547210in}}{\pgfqpoint{4.254278in}{1.552254in}}%
\pgfpathcurveto{\pgfqpoint{4.249234in}{1.557297in}}{\pgfqpoint{4.242393in}{1.560131in}}{\pgfqpoint{4.235260in}{1.560131in}}%
\pgfpathcurveto{\pgfqpoint{4.228127in}{1.560131in}}{\pgfqpoint{4.221285in}{1.557297in}}{\pgfqpoint{4.216242in}{1.552254in}}%
\pgfpathcurveto{\pgfqpoint{4.211198in}{1.547210in}}{\pgfqpoint{4.208364in}{1.540368in}}{\pgfqpoint{4.208364in}{1.533236in}}%
\pgfpathcurveto{\pgfqpoint{4.208364in}{1.526103in}}{\pgfqpoint{4.211198in}{1.519261in}}{\pgfqpoint{4.216242in}{1.514217in}}%
\pgfpathcurveto{\pgfqpoint{4.221285in}{1.509174in}}{\pgfqpoint{4.228127in}{1.506340in}}{\pgfqpoint{4.235260in}{1.506340in}}%
\pgfpathclose%
\pgfusepath{stroke,fill}%
\end{pgfscope}%
\begin{pgfscope}%
\pgfpathrectangle{\pgfqpoint{2.867647in}{0.500000in}}{\pgfqpoint{1.764706in}{1.700000in}}%
\pgfusepath{clip}%
\pgfsetbuttcap%
\pgfsetroundjoin%
\definecolor{currentfill}{rgb}{0.977657,0.891500,0.822809}%
\pgfsetfillcolor{currentfill}%
\pgfsetlinewidth{0.311001pt}%
\definecolor{currentstroke}{rgb}{1.000000,1.000000,1.000000}%
\pgfsetstrokecolor{currentstroke}%
\pgfsetdash{}{0pt}%
\pgfpathmoveto{\pgfqpoint{4.145852in}{1.282356in}}%
\pgfpathcurveto{\pgfqpoint{4.152984in}{1.282356in}}{\pgfqpoint{4.159826in}{1.285190in}}{\pgfqpoint{4.164870in}{1.290234in}}%
\pgfpathcurveto{\pgfqpoint{4.169913in}{1.295277in}}{\pgfqpoint{4.172747in}{1.302119in}}{\pgfqpoint{4.172747in}{1.309252in}}%
\pgfpathcurveto{\pgfqpoint{4.172747in}{1.316385in}}{\pgfqpoint{4.169913in}{1.323226in}}{\pgfqpoint{4.164870in}{1.328270in}}%
\pgfpathcurveto{\pgfqpoint{4.159826in}{1.333314in}}{\pgfqpoint{4.152984in}{1.336147in}}{\pgfqpoint{4.145852in}{1.336147in}}%
\pgfpathcurveto{\pgfqpoint{4.138719in}{1.336147in}}{\pgfqpoint{4.131877in}{1.333314in}}{\pgfqpoint{4.126833in}{1.328270in}}%
\pgfpathcurveto{\pgfqpoint{4.121790in}{1.323226in}}{\pgfqpoint{4.118956in}{1.316385in}}{\pgfqpoint{4.118956in}{1.309252in}}%
\pgfpathcurveto{\pgfqpoint{4.118956in}{1.302119in}}{\pgfqpoint{4.121790in}{1.295277in}}{\pgfqpoint{4.126833in}{1.290234in}}%
\pgfpathcurveto{\pgfqpoint{4.131877in}{1.285190in}}{\pgfqpoint{4.138719in}{1.282356in}}{\pgfqpoint{4.145852in}{1.282356in}}%
\pgfpathclose%
\pgfusepath{stroke,fill}%
\end{pgfscope}%
\begin{pgfscope}%
\pgfpathrectangle{\pgfqpoint{2.867647in}{0.500000in}}{\pgfqpoint{1.764706in}{1.700000in}}%
\pgfusepath{clip}%
\pgfsetbuttcap%
\pgfsetroundjoin%
\definecolor{currentfill}{rgb}{0.979891,0.908948,0.848279}%
\pgfsetfillcolor{currentfill}%
\pgfsetlinewidth{0.311001pt}%
\definecolor{currentstroke}{rgb}{1.000000,1.000000,1.000000}%
\pgfsetstrokecolor{currentstroke}%
\pgfsetdash{}{0pt}%
\pgfpathmoveto{\pgfqpoint{4.133670in}{1.528455in}}%
\pgfpathcurveto{\pgfqpoint{4.140803in}{1.528455in}}{\pgfqpoint{4.147644in}{1.531289in}}{\pgfqpoint{4.152688in}{1.536332in}}%
\pgfpathcurveto{\pgfqpoint{4.157732in}{1.541376in}}{\pgfqpoint{4.160565in}{1.548218in}}{\pgfqpoint{4.160565in}{1.555350in}}%
\pgfpathcurveto{\pgfqpoint{4.160565in}{1.562483in}}{\pgfqpoint{4.157732in}{1.569325in}}{\pgfqpoint{4.152688in}{1.574369in}}%
\pgfpathcurveto{\pgfqpoint{4.147644in}{1.579412in}}{\pgfqpoint{4.140803in}{1.582246in}}{\pgfqpoint{4.133670in}{1.582246in}}%
\pgfpathcurveto{\pgfqpoint{4.126537in}{1.582246in}}{\pgfqpoint{4.119695in}{1.579412in}}{\pgfqpoint{4.114652in}{1.574369in}}%
\pgfpathcurveto{\pgfqpoint{4.109608in}{1.569325in}}{\pgfqpoint{4.106774in}{1.562483in}}{\pgfqpoint{4.106774in}{1.555350in}}%
\pgfpathcurveto{\pgfqpoint{4.106774in}{1.548218in}}{\pgfqpoint{4.109608in}{1.541376in}}{\pgfqpoint{4.114652in}{1.536332in}}%
\pgfpathcurveto{\pgfqpoint{4.119695in}{1.531289in}}{\pgfqpoint{4.126537in}{1.528455in}}{\pgfqpoint{4.133670in}{1.528455in}}%
\pgfpathclose%
\pgfusepath{stroke,fill}%
\end{pgfscope}%
\begin{pgfscope}%
\pgfpathrectangle{\pgfqpoint{2.867647in}{0.500000in}}{\pgfqpoint{1.764706in}{1.700000in}}%
\pgfusepath{clip}%
\pgfsetbuttcap%
\pgfsetroundjoin%
\definecolor{currentfill}{rgb}{0.961115,0.566634,0.405693}%
\pgfsetfillcolor{currentfill}%
\pgfsetlinewidth{0.311001pt}%
\definecolor{currentstroke}{rgb}{1.000000,1.000000,1.000000}%
\pgfsetstrokecolor{currentstroke}%
\pgfsetdash{}{0pt}%
\pgfpathmoveto{\pgfqpoint{4.107492in}{0.854700in}}%
\pgfpathcurveto{\pgfqpoint{4.114625in}{0.854700in}}{\pgfqpoint{4.121466in}{0.857534in}}{\pgfqpoint{4.126510in}{0.862577in}}%
\pgfpathcurveto{\pgfqpoint{4.131553in}{0.867621in}}{\pgfqpoint{4.134387in}{0.874463in}}{\pgfqpoint{4.134387in}{0.881595in}}%
\pgfpathcurveto{\pgfqpoint{4.134387in}{0.888728in}}{\pgfqpoint{4.131553in}{0.895570in}}{\pgfqpoint{4.126510in}{0.900614in}}%
\pgfpathcurveto{\pgfqpoint{4.121466in}{0.905657in}}{\pgfqpoint{4.114625in}{0.908491in}}{\pgfqpoint{4.107492in}{0.908491in}}%
\pgfpathcurveto{\pgfqpoint{4.100359in}{0.908491in}}{\pgfqpoint{4.093517in}{0.905657in}}{\pgfqpoint{4.088474in}{0.900614in}}%
\pgfpathcurveto{\pgfqpoint{4.083430in}{0.895570in}}{\pgfqpoint{4.080596in}{0.888728in}}{\pgfqpoint{4.080596in}{0.881595in}}%
\pgfpathcurveto{\pgfqpoint{4.080596in}{0.874463in}}{\pgfqpoint{4.083430in}{0.867621in}}{\pgfqpoint{4.088474in}{0.862577in}}%
\pgfpathcurveto{\pgfqpoint{4.093517in}{0.857534in}}{\pgfqpoint{4.100359in}{0.854700in}}{\pgfqpoint{4.107492in}{0.854700in}}%
\pgfpathclose%
\pgfusepath{stroke,fill}%
\end{pgfscope}%
\begin{pgfscope}%
\pgfpathrectangle{\pgfqpoint{2.867647in}{0.500000in}}{\pgfqpoint{1.764706in}{1.700000in}}%
\pgfusepath{clip}%
\pgfsetbuttcap%
\pgfsetroundjoin%
\definecolor{currentfill}{rgb}{0.972201,0.839051,0.745789}%
\pgfsetfillcolor{currentfill}%
\pgfsetlinewidth{0.311001pt}%
\definecolor{currentstroke}{rgb}{1.000000,1.000000,1.000000}%
\pgfsetstrokecolor{currentstroke}%
\pgfsetdash{}{0pt}%
\pgfpathmoveto{\pgfqpoint{4.123756in}{0.976560in}}%
\pgfpathcurveto{\pgfqpoint{4.130889in}{0.976560in}}{\pgfqpoint{4.137730in}{0.979394in}}{\pgfqpoint{4.142774in}{0.984438in}}%
\pgfpathcurveto{\pgfqpoint{4.147818in}{0.989482in}}{\pgfqpoint{4.150652in}{0.996323in}}{\pgfqpoint{4.150652in}{1.003456in}}%
\pgfpathcurveto{\pgfqpoint{4.150652in}{1.010589in}}{\pgfqpoint{4.147818in}{1.017430in}}{\pgfqpoint{4.142774in}{1.022474in}}%
\pgfpathcurveto{\pgfqpoint{4.137730in}{1.027518in}}{\pgfqpoint{4.130889in}{1.030352in}}{\pgfqpoint{4.123756in}{1.030352in}}%
\pgfpathcurveto{\pgfqpoint{4.116623in}{1.030352in}}{\pgfqpoint{4.109781in}{1.027518in}}{\pgfqpoint{4.104738in}{1.022474in}}%
\pgfpathcurveto{\pgfqpoint{4.099694in}{1.017430in}}{\pgfqpoint{4.096860in}{1.010589in}}{\pgfqpoint{4.096860in}{1.003456in}}%
\pgfpathcurveto{\pgfqpoint{4.096860in}{0.996323in}}{\pgfqpoint{4.099694in}{0.989482in}}{\pgfqpoint{4.104738in}{0.984438in}}%
\pgfpathcurveto{\pgfqpoint{4.109781in}{0.979394in}}{\pgfqpoint{4.116623in}{0.976560in}}{\pgfqpoint{4.123756in}{0.976560in}}%
\pgfpathclose%
\pgfusepath{stroke,fill}%
\end{pgfscope}%
\begin{pgfscope}%
\pgfpathrectangle{\pgfqpoint{2.867647in}{0.500000in}}{\pgfqpoint{1.764706in}{1.700000in}}%
\pgfusepath{clip}%
\pgfsetbuttcap%
\pgfsetroundjoin%
\definecolor{currentfill}{rgb}{0.964306,0.663930,0.507747}%
\pgfsetfillcolor{currentfill}%
\pgfsetlinewidth{0.311001pt}%
\definecolor{currentstroke}{rgb}{1.000000,1.000000,1.000000}%
\pgfsetstrokecolor{currentstroke}%
\pgfsetdash{}{0pt}%
\pgfpathmoveto{\pgfqpoint{4.036761in}{1.465484in}}%
\pgfpathcurveto{\pgfqpoint{4.043894in}{1.465484in}}{\pgfqpoint{4.050736in}{1.468318in}}{\pgfqpoint{4.055779in}{1.473361in}}%
\pgfpathcurveto{\pgfqpoint{4.060823in}{1.478405in}}{\pgfqpoint{4.063657in}{1.485247in}}{\pgfqpoint{4.063657in}{1.492379in}}%
\pgfpathcurveto{\pgfqpoint{4.063657in}{1.499512in}}{\pgfqpoint{4.060823in}{1.506354in}}{\pgfqpoint{4.055779in}{1.511397in}}%
\pgfpathcurveto{\pgfqpoint{4.050736in}{1.516441in}}{\pgfqpoint{4.043894in}{1.519275in}}{\pgfqpoint{4.036761in}{1.519275in}}%
\pgfpathcurveto{\pgfqpoint{4.029628in}{1.519275in}}{\pgfqpoint{4.022787in}{1.516441in}}{\pgfqpoint{4.017743in}{1.511397in}}%
\pgfpathcurveto{\pgfqpoint{4.012699in}{1.506354in}}{\pgfqpoint{4.009865in}{1.499512in}}{\pgfqpoint{4.009865in}{1.492379in}}%
\pgfpathcurveto{\pgfqpoint{4.009865in}{1.485247in}}{\pgfqpoint{4.012699in}{1.478405in}}{\pgfqpoint{4.017743in}{1.473361in}}%
\pgfpathcurveto{\pgfqpoint{4.022787in}{1.468318in}}{\pgfqpoint{4.029628in}{1.465484in}}{\pgfqpoint{4.036761in}{1.465484in}}%
\pgfpathclose%
\pgfusepath{stroke,fill}%
\end{pgfscope}%
\begin{pgfscope}%
\pgfpathrectangle{\pgfqpoint{2.867647in}{0.500000in}}{\pgfqpoint{1.764706in}{1.700000in}}%
\pgfusepath{clip}%
\pgfsetbuttcap%
\pgfsetroundjoin%
\definecolor{currentfill}{rgb}{0.965928,0.738443,0.600540}%
\pgfsetfillcolor{currentfill}%
\pgfsetlinewidth{0.311001pt}%
\definecolor{currentstroke}{rgb}{1.000000,1.000000,1.000000}%
\pgfsetstrokecolor{currentstroke}%
\pgfsetdash{}{0pt}%
\pgfpathmoveto{\pgfqpoint{4.066560in}{1.749331in}}%
\pgfpathcurveto{\pgfqpoint{4.073693in}{1.749331in}}{\pgfqpoint{4.080535in}{1.752165in}}{\pgfqpoint{4.085578in}{1.757208in}}%
\pgfpathcurveto{\pgfqpoint{4.090622in}{1.762252in}}{\pgfqpoint{4.093456in}{1.769094in}}{\pgfqpoint{4.093456in}{1.776227in}}%
\pgfpathcurveto{\pgfqpoint{4.093456in}{1.783359in}}{\pgfqpoint{4.090622in}{1.790201in}}{\pgfqpoint{4.085578in}{1.795245in}}%
\pgfpathcurveto{\pgfqpoint{4.080535in}{1.800288in}}{\pgfqpoint{4.073693in}{1.803122in}}{\pgfqpoint{4.066560in}{1.803122in}}%
\pgfpathcurveto{\pgfqpoint{4.059427in}{1.803122in}}{\pgfqpoint{4.052586in}{1.800288in}}{\pgfqpoint{4.047542in}{1.795245in}}%
\pgfpathcurveto{\pgfqpoint{4.042498in}{1.790201in}}{\pgfqpoint{4.039664in}{1.783359in}}{\pgfqpoint{4.039664in}{1.776227in}}%
\pgfpathcurveto{\pgfqpoint{4.039664in}{1.769094in}}{\pgfqpoint{4.042498in}{1.762252in}}{\pgfqpoint{4.047542in}{1.757208in}}%
\pgfpathcurveto{\pgfqpoint{4.052586in}{1.752165in}}{\pgfqpoint{4.059427in}{1.749331in}}{\pgfqpoint{4.066560in}{1.749331in}}%
\pgfpathclose%
\pgfusepath{stroke,fill}%
\end{pgfscope}%
\begin{pgfscope}%
\pgfpathrectangle{\pgfqpoint{2.867647in}{0.500000in}}{\pgfqpoint{1.764706in}{1.700000in}}%
\pgfusepath{clip}%
\pgfsetbuttcap%
\pgfsetroundjoin%
\definecolor{currentfill}{rgb}{0.981377,0.920617,0.865369}%
\pgfsetfillcolor{currentfill}%
\pgfsetlinewidth{0.311001pt}%
\definecolor{currentstroke}{rgb}{1.000000,1.000000,1.000000}%
\pgfsetstrokecolor{currentstroke}%
\pgfsetdash{}{0pt}%
\pgfpathmoveto{\pgfqpoint{4.185975in}{1.309917in}}%
\pgfpathcurveto{\pgfqpoint{4.193107in}{1.309917in}}{\pgfqpoint{4.199949in}{1.312751in}}{\pgfqpoint{4.204993in}{1.317795in}}%
\pgfpathcurveto{\pgfqpoint{4.210036in}{1.322839in}}{\pgfqpoint{4.212870in}{1.329680in}}{\pgfqpoint{4.212870in}{1.336813in}}%
\pgfpathcurveto{\pgfqpoint{4.212870in}{1.343946in}}{\pgfqpoint{4.210036in}{1.350788in}}{\pgfqpoint{4.204993in}{1.355831in}}%
\pgfpathcurveto{\pgfqpoint{4.199949in}{1.360875in}}{\pgfqpoint{4.193107in}{1.363709in}}{\pgfqpoint{4.185975in}{1.363709in}}%
\pgfpathcurveto{\pgfqpoint{4.178842in}{1.363709in}}{\pgfqpoint{4.172000in}{1.360875in}}{\pgfqpoint{4.166957in}{1.355831in}}%
\pgfpathcurveto{\pgfqpoint{4.161913in}{1.350788in}}{\pgfqpoint{4.159079in}{1.343946in}}{\pgfqpoint{4.159079in}{1.336813in}}%
\pgfpathcurveto{\pgfqpoint{4.159079in}{1.329680in}}{\pgfqpoint{4.161913in}{1.322839in}}{\pgfqpoint{4.166957in}{1.317795in}}%
\pgfpathcurveto{\pgfqpoint{4.172000in}{1.312751in}}{\pgfqpoint{4.178842in}{1.309917in}}{\pgfqpoint{4.185975in}{1.309917in}}%
\pgfpathclose%
\pgfusepath{stroke,fill}%
\end{pgfscope}%
\begin{pgfscope}%
\pgfpathrectangle{\pgfqpoint{2.867647in}{0.500000in}}{\pgfqpoint{1.764706in}{1.700000in}}%
\pgfusepath{clip}%
\pgfsetbuttcap%
\pgfsetroundjoin%
\definecolor{currentfill}{rgb}{0.964679,0.682838,0.530002}%
\pgfsetfillcolor{currentfill}%
\pgfsetlinewidth{0.311001pt}%
\definecolor{currentstroke}{rgb}{1.000000,1.000000,1.000000}%
\pgfsetstrokecolor{currentstroke}%
\pgfsetdash{}{0pt}%
\pgfpathmoveto{\pgfqpoint{4.044865in}{1.455836in}}%
\pgfpathcurveto{\pgfqpoint{4.051998in}{1.455836in}}{\pgfqpoint{4.058840in}{1.458670in}}{\pgfqpoint{4.063883in}{1.463713in}}%
\pgfpathcurveto{\pgfqpoint{4.068927in}{1.468757in}}{\pgfqpoint{4.071761in}{1.475599in}}{\pgfqpoint{4.071761in}{1.482731in}}%
\pgfpathcurveto{\pgfqpoint{4.071761in}{1.489864in}}{\pgfqpoint{4.068927in}{1.496706in}}{\pgfqpoint{4.063883in}{1.501750in}}%
\pgfpathcurveto{\pgfqpoint{4.058840in}{1.506793in}}{\pgfqpoint{4.051998in}{1.509627in}}{\pgfqpoint{4.044865in}{1.509627in}}%
\pgfpathcurveto{\pgfqpoint{4.037732in}{1.509627in}}{\pgfqpoint{4.030891in}{1.506793in}}{\pgfqpoint{4.025847in}{1.501750in}}%
\pgfpathcurveto{\pgfqpoint{4.020803in}{1.496706in}}{\pgfqpoint{4.017969in}{1.489864in}}{\pgfqpoint{4.017969in}{1.482731in}}%
\pgfpathcurveto{\pgfqpoint{4.017969in}{1.475599in}}{\pgfqpoint{4.020803in}{1.468757in}}{\pgfqpoint{4.025847in}{1.463713in}}%
\pgfpathcurveto{\pgfqpoint{4.030891in}{1.458670in}}{\pgfqpoint{4.037732in}{1.455836in}}{\pgfqpoint{4.044865in}{1.455836in}}%
\pgfpathclose%
\pgfusepath{stroke,fill}%
\end{pgfscope}%
\begin{pgfscope}%
\pgfpathrectangle{\pgfqpoint{2.867647in}{0.500000in}}{\pgfqpoint{1.764706in}{1.700000in}}%
\pgfusepath{clip}%
\pgfsetbuttcap%
\pgfsetroundjoin%
\definecolor{currentfill}{rgb}{0.979124,0.903132,0.839793}%
\pgfsetfillcolor{currentfill}%
\pgfsetlinewidth{0.311001pt}%
\definecolor{currentstroke}{rgb}{1.000000,1.000000,1.000000}%
\pgfsetstrokecolor{currentstroke}%
\pgfsetdash{}{0pt}%
\pgfpathmoveto{\pgfqpoint{4.159342in}{1.390539in}}%
\pgfpathcurveto{\pgfqpoint{4.166475in}{1.390539in}}{\pgfqpoint{4.173316in}{1.393373in}}{\pgfqpoint{4.178360in}{1.398417in}}%
\pgfpathcurveto{\pgfqpoint{4.183404in}{1.403461in}}{\pgfqpoint{4.186238in}{1.410302in}}{\pgfqpoint{4.186238in}{1.417435in}}%
\pgfpathcurveto{\pgfqpoint{4.186238in}{1.424568in}}{\pgfqpoint{4.183404in}{1.431410in}}{\pgfqpoint{4.178360in}{1.436453in}}%
\pgfpathcurveto{\pgfqpoint{4.173316in}{1.441497in}}{\pgfqpoint{4.166475in}{1.444331in}}{\pgfqpoint{4.159342in}{1.444331in}}%
\pgfpathcurveto{\pgfqpoint{4.152209in}{1.444331in}}{\pgfqpoint{4.145368in}{1.441497in}}{\pgfqpoint{4.140324in}{1.436453in}}%
\pgfpathcurveto{\pgfqpoint{4.135280in}{1.431410in}}{\pgfqpoint{4.132446in}{1.424568in}}{\pgfqpoint{4.132446in}{1.417435in}}%
\pgfpathcurveto{\pgfqpoint{4.132446in}{1.410302in}}{\pgfqpoint{4.135280in}{1.403461in}}{\pgfqpoint{4.140324in}{1.398417in}}%
\pgfpathcurveto{\pgfqpoint{4.145368in}{1.393373in}}{\pgfqpoint{4.152209in}{1.390539in}}{\pgfqpoint{4.159342in}{1.390539in}}%
\pgfpathclose%
\pgfusepath{stroke,fill}%
\end{pgfscope}%
\begin{pgfscope}%
\pgfpathrectangle{\pgfqpoint{2.867647in}{0.500000in}}{\pgfqpoint{1.764706in}{1.700000in}}%
\pgfusepath{clip}%
\pgfsetbuttcap%
\pgfsetroundjoin%
\definecolor{currentfill}{rgb}{0.978376,0.897317,0.831308}%
\pgfsetfillcolor{currentfill}%
\pgfsetlinewidth{0.311001pt}%
\definecolor{currentstroke}{rgb}{1.000000,1.000000,1.000000}%
\pgfsetstrokecolor{currentstroke}%
\pgfsetdash{}{0pt}%
\pgfpathmoveto{\pgfqpoint{4.227380in}{1.251188in}}%
\pgfpathcurveto{\pgfqpoint{4.234513in}{1.251188in}}{\pgfqpoint{4.241354in}{1.254022in}}{\pgfqpoint{4.246398in}{1.259066in}}%
\pgfpathcurveto{\pgfqpoint{4.251442in}{1.264109in}}{\pgfqpoint{4.254275in}{1.270951in}}{\pgfqpoint{4.254275in}{1.278084in}}%
\pgfpathcurveto{\pgfqpoint{4.254275in}{1.285217in}}{\pgfqpoint{4.251442in}{1.292058in}}{\pgfqpoint{4.246398in}{1.297102in}}%
\pgfpathcurveto{\pgfqpoint{4.241354in}{1.302146in}}{\pgfqpoint{4.234513in}{1.304979in}}{\pgfqpoint{4.227380in}{1.304979in}}%
\pgfpathcurveto{\pgfqpoint{4.220247in}{1.304979in}}{\pgfqpoint{4.213405in}{1.302146in}}{\pgfqpoint{4.208362in}{1.297102in}}%
\pgfpathcurveto{\pgfqpoint{4.203318in}{1.292058in}}{\pgfqpoint{4.200484in}{1.285217in}}{\pgfqpoint{4.200484in}{1.278084in}}%
\pgfpathcurveto{\pgfqpoint{4.200484in}{1.270951in}}{\pgfqpoint{4.203318in}{1.264109in}}{\pgfqpoint{4.208362in}{1.259066in}}%
\pgfpathcurveto{\pgfqpoint{4.213405in}{1.254022in}}{\pgfqpoint{4.220247in}{1.251188in}}{\pgfqpoint{4.227380in}{1.251188in}}%
\pgfpathclose%
\pgfusepath{stroke,fill}%
\end{pgfscope}%
\begin{pgfscope}%
\pgfpathrectangle{\pgfqpoint{2.867647in}{0.500000in}}{\pgfqpoint{1.764706in}{1.700000in}}%
\pgfusepath{clip}%
\pgfsetbuttcap%
\pgfsetroundjoin%
\definecolor{currentfill}{rgb}{0.979124,0.903132,0.839793}%
\pgfsetfillcolor{currentfill}%
\pgfsetlinewidth{0.311001pt}%
\definecolor{currentstroke}{rgb}{1.000000,1.000000,1.000000}%
\pgfsetstrokecolor{currentstroke}%
\pgfsetdash{}{0pt}%
\pgfpathmoveto{\pgfqpoint{4.156037in}{1.308824in}}%
\pgfpathcurveto{\pgfqpoint{4.163170in}{1.308824in}}{\pgfqpoint{4.170011in}{1.311658in}}{\pgfqpoint{4.175055in}{1.316701in}}%
\pgfpathcurveto{\pgfqpoint{4.180099in}{1.321745in}}{\pgfqpoint{4.182933in}{1.328587in}}{\pgfqpoint{4.182933in}{1.335719in}}%
\pgfpathcurveto{\pgfqpoint{4.182933in}{1.342852in}}{\pgfqpoint{4.180099in}{1.349694in}}{\pgfqpoint{4.175055in}{1.354738in}}%
\pgfpathcurveto{\pgfqpoint{4.170011in}{1.359781in}}{\pgfqpoint{4.163170in}{1.362615in}}{\pgfqpoint{4.156037in}{1.362615in}}%
\pgfpathcurveto{\pgfqpoint{4.148904in}{1.362615in}}{\pgfqpoint{4.142062in}{1.359781in}}{\pgfqpoint{4.137019in}{1.354738in}}%
\pgfpathcurveto{\pgfqpoint{4.131975in}{1.349694in}}{\pgfqpoint{4.129141in}{1.342852in}}{\pgfqpoint{4.129141in}{1.335719in}}%
\pgfpathcurveto{\pgfqpoint{4.129141in}{1.328587in}}{\pgfqpoint{4.131975in}{1.321745in}}{\pgfqpoint{4.137019in}{1.316701in}}%
\pgfpathcurveto{\pgfqpoint{4.142062in}{1.311658in}}{\pgfqpoint{4.148904in}{1.308824in}}{\pgfqpoint{4.156037in}{1.308824in}}%
\pgfpathclose%
\pgfusepath{stroke,fill}%
\end{pgfscope}%
\begin{pgfscope}%
\pgfpathrectangle{\pgfqpoint{2.867647in}{0.500000in}}{\pgfqpoint{1.764706in}{1.700000in}}%
\pgfusepath{clip}%
\pgfsetbuttcap%
\pgfsetroundjoin%
\definecolor{currentfill}{rgb}{0.966328,0.750560,0.616961}%
\pgfsetfillcolor{currentfill}%
\pgfsetlinewidth{0.311001pt}%
\definecolor{currentstroke}{rgb}{1.000000,1.000000,1.000000}%
\pgfsetstrokecolor{currentstroke}%
\pgfsetdash{}{0pt}%
\pgfpathmoveto{\pgfqpoint{4.099300in}{0.915167in}}%
\pgfpathcurveto{\pgfqpoint{4.106433in}{0.915167in}}{\pgfqpoint{4.113274in}{0.918001in}}{\pgfqpoint{4.118318in}{0.923044in}}%
\pgfpathcurveto{\pgfqpoint{4.123362in}{0.928088in}}{\pgfqpoint{4.126196in}{0.934930in}}{\pgfqpoint{4.126196in}{0.942063in}}%
\pgfpathcurveto{\pgfqpoint{4.126196in}{0.949195in}}{\pgfqpoint{4.123362in}{0.956037in}}{\pgfqpoint{4.118318in}{0.961081in}}%
\pgfpathcurveto{\pgfqpoint{4.113274in}{0.966124in}}{\pgfqpoint{4.106433in}{0.968958in}}{\pgfqpoint{4.099300in}{0.968958in}}%
\pgfpathcurveto{\pgfqpoint{4.092167in}{0.968958in}}{\pgfqpoint{4.085325in}{0.966124in}}{\pgfqpoint{4.080282in}{0.961081in}}%
\pgfpathcurveto{\pgfqpoint{4.075238in}{0.956037in}}{\pgfqpoint{4.072404in}{0.949195in}}{\pgfqpoint{4.072404in}{0.942063in}}%
\pgfpathcurveto{\pgfqpoint{4.072404in}{0.934930in}}{\pgfqpoint{4.075238in}{0.928088in}}{\pgfqpoint{4.080282in}{0.923044in}}%
\pgfpathcurveto{\pgfqpoint{4.085325in}{0.918001in}}{\pgfqpoint{4.092167in}{0.915167in}}{\pgfqpoint{4.099300in}{0.915167in}}%
\pgfpathclose%
\pgfusepath{stroke,fill}%
\end{pgfscope}%
\begin{pgfscope}%
\pgfpathrectangle{\pgfqpoint{2.867647in}{0.500000in}}{\pgfqpoint{1.764706in}{1.700000in}}%
\pgfusepath{clip}%
\pgfsetbuttcap%
\pgfsetroundjoin%
\definecolor{currentfill}{rgb}{0.970255,0.815666,0.711203}%
\pgfsetfillcolor{currentfill}%
\pgfsetlinewidth{0.311001pt}%
\definecolor{currentstroke}{rgb}{1.000000,1.000000,1.000000}%
\pgfsetstrokecolor{currentstroke}%
\pgfsetdash{}{0pt}%
\pgfpathmoveto{\pgfqpoint{4.214057in}{1.585399in}}%
\pgfpathcurveto{\pgfqpoint{4.221190in}{1.585399in}}{\pgfqpoint{4.228031in}{1.588233in}}{\pgfqpoint{4.233075in}{1.593277in}}%
\pgfpathcurveto{\pgfqpoint{4.238119in}{1.598320in}}{\pgfqpoint{4.240952in}{1.605162in}}{\pgfqpoint{4.240952in}{1.612295in}}%
\pgfpathcurveto{\pgfqpoint{4.240952in}{1.619428in}}{\pgfqpoint{4.238119in}{1.626269in}}{\pgfqpoint{4.233075in}{1.631313in}}%
\pgfpathcurveto{\pgfqpoint{4.228031in}{1.636356in}}{\pgfqpoint{4.221190in}{1.639190in}}{\pgfqpoint{4.214057in}{1.639190in}}%
\pgfpathcurveto{\pgfqpoint{4.206924in}{1.639190in}}{\pgfqpoint{4.200082in}{1.636356in}}{\pgfqpoint{4.195039in}{1.631313in}}%
\pgfpathcurveto{\pgfqpoint{4.189995in}{1.626269in}}{\pgfqpoint{4.187161in}{1.619428in}}{\pgfqpoint{4.187161in}{1.612295in}}%
\pgfpathcurveto{\pgfqpoint{4.187161in}{1.605162in}}{\pgfqpoint{4.189995in}{1.598320in}}{\pgfqpoint{4.195039in}{1.593277in}}%
\pgfpathcurveto{\pgfqpoint{4.200082in}{1.588233in}}{\pgfqpoint{4.206924in}{1.585399in}}{\pgfqpoint{4.214057in}{1.585399in}}%
\pgfpathclose%
\pgfusepath{stroke,fill}%
\end{pgfscope}%
\begin{pgfscope}%
\pgfpathrectangle{\pgfqpoint{2.867647in}{0.500000in}}{\pgfqpoint{1.764706in}{1.700000in}}%
\pgfusepath{clip}%
\pgfsetbuttcap%
\pgfsetroundjoin%
\definecolor{currentfill}{rgb}{0.971694,0.833208,0.737161}%
\pgfsetfillcolor{currentfill}%
\pgfsetlinewidth{0.311001pt}%
\definecolor{currentstroke}{rgb}{1.000000,1.000000,1.000000}%
\pgfsetstrokecolor{currentstroke}%
\pgfsetdash{}{0pt}%
\pgfpathmoveto{\pgfqpoint{4.210876in}{1.062133in}}%
\pgfpathcurveto{\pgfqpoint{4.218009in}{1.062133in}}{\pgfqpoint{4.224851in}{1.064967in}}{\pgfqpoint{4.229894in}{1.070011in}}%
\pgfpathcurveto{\pgfqpoint{4.234938in}{1.075055in}}{\pgfqpoint{4.237772in}{1.081896in}}{\pgfqpoint{4.237772in}{1.089029in}}%
\pgfpathcurveto{\pgfqpoint{4.237772in}{1.096162in}}{\pgfqpoint{4.234938in}{1.103004in}}{\pgfqpoint{4.229894in}{1.108047in}}%
\pgfpathcurveto{\pgfqpoint{4.224851in}{1.113091in}}{\pgfqpoint{4.218009in}{1.115925in}}{\pgfqpoint{4.210876in}{1.115925in}}%
\pgfpathcurveto{\pgfqpoint{4.203743in}{1.115925in}}{\pgfqpoint{4.196902in}{1.113091in}}{\pgfqpoint{4.191858in}{1.108047in}}%
\pgfpathcurveto{\pgfqpoint{4.186814in}{1.103004in}}{\pgfqpoint{4.183980in}{1.096162in}}{\pgfqpoint{4.183980in}{1.089029in}}%
\pgfpathcurveto{\pgfqpoint{4.183980in}{1.081896in}}{\pgfqpoint{4.186814in}{1.075055in}}{\pgfqpoint{4.191858in}{1.070011in}}%
\pgfpathcurveto{\pgfqpoint{4.196902in}{1.064967in}}{\pgfqpoint{4.203743in}{1.062133in}}{\pgfqpoint{4.210876in}{1.062133in}}%
\pgfpathclose%
\pgfusepath{stroke,fill}%
\end{pgfscope}%
\begin{pgfscope}%
\pgfpathrectangle{\pgfqpoint{2.867647in}{0.500000in}}{\pgfqpoint{1.764706in}{1.700000in}}%
\pgfusepath{clip}%
\pgfsetbuttcap%
\pgfsetroundjoin%
\definecolor{currentfill}{rgb}{0.976961,0.885681,0.814303}%
\pgfsetfillcolor{currentfill}%
\pgfsetlinewidth{0.311001pt}%
\definecolor{currentstroke}{rgb}{1.000000,1.000000,1.000000}%
\pgfsetstrokecolor{currentstroke}%
\pgfsetdash{}{0pt}%
\pgfpathmoveto{\pgfqpoint{4.125639in}{1.452057in}}%
\pgfpathcurveto{\pgfqpoint{4.132771in}{1.452057in}}{\pgfqpoint{4.139613in}{1.454890in}}{\pgfqpoint{4.144657in}{1.459934in}}%
\pgfpathcurveto{\pgfqpoint{4.149700in}{1.464978in}}{\pgfqpoint{4.152534in}{1.471819in}}{\pgfqpoint{4.152534in}{1.478952in}}%
\pgfpathcurveto{\pgfqpoint{4.152534in}{1.486085in}}{\pgfqpoint{4.149700in}{1.492927in}}{\pgfqpoint{4.144657in}{1.497970in}}%
\pgfpathcurveto{\pgfqpoint{4.139613in}{1.503014in}}{\pgfqpoint{4.132771in}{1.505848in}}{\pgfqpoint{4.125639in}{1.505848in}}%
\pgfpathcurveto{\pgfqpoint{4.118506in}{1.505848in}}{\pgfqpoint{4.111664in}{1.503014in}}{\pgfqpoint{4.106620in}{1.497970in}}%
\pgfpathcurveto{\pgfqpoint{4.101577in}{1.492927in}}{\pgfqpoint{4.098743in}{1.486085in}}{\pgfqpoint{4.098743in}{1.478952in}}%
\pgfpathcurveto{\pgfqpoint{4.098743in}{1.471819in}}{\pgfqpoint{4.101577in}{1.464978in}}{\pgfqpoint{4.106620in}{1.459934in}}%
\pgfpathcurveto{\pgfqpoint{4.111664in}{1.454890in}}{\pgfqpoint{4.118506in}{1.452057in}}{\pgfqpoint{4.125639in}{1.452057in}}%
\pgfpathclose%
\pgfusepath{stroke,fill}%
\end{pgfscope}%
\begin{pgfscope}%
\pgfpathrectangle{\pgfqpoint{2.867647in}{0.500000in}}{\pgfqpoint{1.764706in}{1.700000in}}%
\pgfusepath{clip}%
\pgfsetbuttcap%
\pgfsetroundjoin%
\definecolor{currentfill}{rgb}{0.981377,0.920617,0.865369}%
\pgfsetfillcolor{currentfill}%
\pgfsetlinewidth{0.311001pt}%
\definecolor{currentstroke}{rgb}{1.000000,1.000000,1.000000}%
\pgfsetstrokecolor{currentstroke}%
\pgfsetdash{}{0pt}%
\pgfpathmoveto{\pgfqpoint{4.197247in}{1.307224in}}%
\pgfpathcurveto{\pgfqpoint{4.204380in}{1.307224in}}{\pgfqpoint{4.211221in}{1.310058in}}{\pgfqpoint{4.216265in}{1.315101in}}%
\pgfpathcurveto{\pgfqpoint{4.221308in}{1.320145in}}{\pgfqpoint{4.224142in}{1.326987in}}{\pgfqpoint{4.224142in}{1.334119in}}%
\pgfpathcurveto{\pgfqpoint{4.224142in}{1.341252in}}{\pgfqpoint{4.221308in}{1.348094in}}{\pgfqpoint{4.216265in}{1.353137in}}%
\pgfpathcurveto{\pgfqpoint{4.211221in}{1.358181in}}{\pgfqpoint{4.204380in}{1.361015in}}{\pgfqpoint{4.197247in}{1.361015in}}%
\pgfpathcurveto{\pgfqpoint{4.190114in}{1.361015in}}{\pgfqpoint{4.183272in}{1.358181in}}{\pgfqpoint{4.178229in}{1.353137in}}%
\pgfpathcurveto{\pgfqpoint{4.173185in}{1.348094in}}{\pgfqpoint{4.170351in}{1.341252in}}{\pgfqpoint{4.170351in}{1.334119in}}%
\pgfpathcurveto{\pgfqpoint{4.170351in}{1.326987in}}{\pgfqpoint{4.173185in}{1.320145in}}{\pgfqpoint{4.178229in}{1.315101in}}%
\pgfpathcurveto{\pgfqpoint{4.183272in}{1.310058in}}{\pgfqpoint{4.190114in}{1.307224in}}{\pgfqpoint{4.197247in}{1.307224in}}%
\pgfpathclose%
\pgfusepath{stroke,fill}%
\end{pgfscope}%
\begin{pgfscope}%
\pgfpathrectangle{\pgfqpoint{2.867647in}{0.500000in}}{\pgfqpoint{1.764706in}{1.700000in}}%
\pgfusepath{clip}%
\pgfsetbuttcap%
\pgfsetroundjoin%
\definecolor{currentfill}{rgb}{0.975018,0.868213,0.788710}%
\pgfsetfillcolor{currentfill}%
\pgfsetlinewidth{0.311001pt}%
\definecolor{currentstroke}{rgb}{1.000000,1.000000,1.000000}%
\pgfsetstrokecolor{currentstroke}%
\pgfsetdash{}{0pt}%
\pgfpathmoveto{\pgfqpoint{4.130329in}{1.394176in}}%
\pgfpathcurveto{\pgfqpoint{4.137462in}{1.394176in}}{\pgfqpoint{4.144304in}{1.397010in}}{\pgfqpoint{4.149347in}{1.402054in}}%
\pgfpathcurveto{\pgfqpoint{4.154391in}{1.407097in}}{\pgfqpoint{4.157225in}{1.413939in}}{\pgfqpoint{4.157225in}{1.421072in}}%
\pgfpathcurveto{\pgfqpoint{4.157225in}{1.428204in}}{\pgfqpoint{4.154391in}{1.435046in}}{\pgfqpoint{4.149347in}{1.440090in}}%
\pgfpathcurveto{\pgfqpoint{4.144304in}{1.445133in}}{\pgfqpoint{4.137462in}{1.447967in}}{\pgfqpoint{4.130329in}{1.447967in}}%
\pgfpathcurveto{\pgfqpoint{4.123196in}{1.447967in}}{\pgfqpoint{4.116355in}{1.445133in}}{\pgfqpoint{4.111311in}{1.440090in}}%
\pgfpathcurveto{\pgfqpoint{4.106267in}{1.435046in}}{\pgfqpoint{4.103434in}{1.428204in}}{\pgfqpoint{4.103434in}{1.421072in}}%
\pgfpathcurveto{\pgfqpoint{4.103434in}{1.413939in}}{\pgfqpoint{4.106267in}{1.407097in}}{\pgfqpoint{4.111311in}{1.402054in}}%
\pgfpathcurveto{\pgfqpoint{4.116355in}{1.397010in}}{\pgfqpoint{4.123196in}{1.394176in}}{\pgfqpoint{4.130329in}{1.394176in}}%
\pgfpathclose%
\pgfusepath{stroke,fill}%
\end{pgfscope}%
\begin{pgfscope}%
\pgfpathrectangle{\pgfqpoint{2.867647in}{0.500000in}}{\pgfqpoint{1.764706in}{1.700000in}}%
\pgfusepath{clip}%
\pgfsetbuttcap%
\pgfsetroundjoin%
\definecolor{currentfill}{rgb}{0.964920,0.695342,0.545192}%
\pgfsetfillcolor{currentfill}%
\pgfsetlinewidth{0.311001pt}%
\definecolor{currentstroke}{rgb}{1.000000,1.000000,1.000000}%
\pgfsetstrokecolor{currentstroke}%
\pgfsetdash{}{0pt}%
\pgfpathmoveto{\pgfqpoint{4.300015in}{1.439557in}}%
\pgfpathcurveto{\pgfqpoint{4.307148in}{1.439557in}}{\pgfqpoint{4.313989in}{1.442391in}}{\pgfqpoint{4.319033in}{1.447434in}}%
\pgfpathcurveto{\pgfqpoint{4.324077in}{1.452478in}}{\pgfqpoint{4.326910in}{1.459320in}}{\pgfqpoint{4.326910in}{1.466452in}}%
\pgfpathcurveto{\pgfqpoint{4.326910in}{1.473585in}}{\pgfqpoint{4.324077in}{1.480427in}}{\pgfqpoint{4.319033in}{1.485471in}}%
\pgfpathcurveto{\pgfqpoint{4.313989in}{1.490514in}}{\pgfqpoint{4.307148in}{1.493348in}}{\pgfqpoint{4.300015in}{1.493348in}}%
\pgfpathcurveto{\pgfqpoint{4.292882in}{1.493348in}}{\pgfqpoint{4.286040in}{1.490514in}}{\pgfqpoint{4.280997in}{1.485471in}}%
\pgfpathcurveto{\pgfqpoint{4.275953in}{1.480427in}}{\pgfqpoint{4.273119in}{1.473585in}}{\pgfqpoint{4.273119in}{1.466452in}}%
\pgfpathcurveto{\pgfqpoint{4.273119in}{1.459320in}}{\pgfqpoint{4.275953in}{1.452478in}}{\pgfqpoint{4.280997in}{1.447434in}}%
\pgfpathcurveto{\pgfqpoint{4.286040in}{1.442391in}}{\pgfqpoint{4.292882in}{1.439557in}}{\pgfqpoint{4.300015in}{1.439557in}}%
\pgfpathclose%
\pgfusepath{stroke,fill}%
\end{pgfscope}%
\begin{pgfscope}%
\pgfpathrectangle{\pgfqpoint{2.867647in}{0.500000in}}{\pgfqpoint{1.764706in}{1.700000in}}%
\pgfusepath{clip}%
\pgfsetbuttcap%
\pgfsetroundjoin%
\definecolor{currentfill}{rgb}{0.978376,0.897317,0.831308}%
\pgfsetfillcolor{currentfill}%
\pgfsetlinewidth{0.311001pt}%
\definecolor{currentstroke}{rgb}{1.000000,1.000000,1.000000}%
\pgfsetstrokecolor{currentstroke}%
\pgfsetdash{}{0pt}%
\pgfpathmoveto{\pgfqpoint{4.231145in}{1.281824in}}%
\pgfpathcurveto{\pgfqpoint{4.238278in}{1.281824in}}{\pgfqpoint{4.245120in}{1.284658in}}{\pgfqpoint{4.250163in}{1.289702in}}%
\pgfpathcurveto{\pgfqpoint{4.255207in}{1.294745in}}{\pgfqpoint{4.258041in}{1.301587in}}{\pgfqpoint{4.258041in}{1.308720in}}%
\pgfpathcurveto{\pgfqpoint{4.258041in}{1.315853in}}{\pgfqpoint{4.255207in}{1.322694in}}{\pgfqpoint{4.250163in}{1.327738in}}%
\pgfpathcurveto{\pgfqpoint{4.245120in}{1.332782in}}{\pgfqpoint{4.238278in}{1.335616in}}{\pgfqpoint{4.231145in}{1.335616in}}%
\pgfpathcurveto{\pgfqpoint{4.224013in}{1.335616in}}{\pgfqpoint{4.217171in}{1.332782in}}{\pgfqpoint{4.212127in}{1.327738in}}%
\pgfpathcurveto{\pgfqpoint{4.207084in}{1.322694in}}{\pgfqpoint{4.204250in}{1.315853in}}{\pgfqpoint{4.204250in}{1.308720in}}%
\pgfpathcurveto{\pgfqpoint{4.204250in}{1.301587in}}{\pgfqpoint{4.207084in}{1.294745in}}{\pgfqpoint{4.212127in}{1.289702in}}%
\pgfpathcurveto{\pgfqpoint{4.217171in}{1.284658in}}{\pgfqpoint{4.224013in}{1.281824in}}{\pgfqpoint{4.231145in}{1.281824in}}%
\pgfpathclose%
\pgfusepath{stroke,fill}%
\end{pgfscope}%
\begin{pgfscope}%
\pgfpathrectangle{\pgfqpoint{2.867647in}{0.500000in}}{\pgfqpoint{1.764706in}{1.700000in}}%
\pgfusepath{clip}%
\pgfsetbuttcap%
\pgfsetroundjoin%
\definecolor{currentfill}{rgb}{0.967398,0.774513,0.650573}%
\pgfsetfillcolor{currentfill}%
\pgfsetlinewidth{0.311001pt}%
\definecolor{currentstroke}{rgb}{1.000000,1.000000,1.000000}%
\pgfsetstrokecolor{currentstroke}%
\pgfsetdash{}{0pt}%
\pgfpathmoveto{\pgfqpoint{4.212152in}{1.020480in}}%
\pgfpathcurveto{\pgfqpoint{4.219285in}{1.020480in}}{\pgfqpoint{4.226126in}{1.023314in}}{\pgfqpoint{4.231170in}{1.028357in}}%
\pgfpathcurveto{\pgfqpoint{4.236214in}{1.033401in}}{\pgfqpoint{4.239047in}{1.040243in}}{\pgfqpoint{4.239047in}{1.047376in}}%
\pgfpathcurveto{\pgfqpoint{4.239047in}{1.054508in}}{\pgfqpoint{4.236214in}{1.061350in}}{\pgfqpoint{4.231170in}{1.066394in}}%
\pgfpathcurveto{\pgfqpoint{4.226126in}{1.071437in}}{\pgfqpoint{4.219285in}{1.074271in}}{\pgfqpoint{4.212152in}{1.074271in}}%
\pgfpathcurveto{\pgfqpoint{4.205019in}{1.074271in}}{\pgfqpoint{4.198177in}{1.071437in}}{\pgfqpoint{4.193134in}{1.066394in}}%
\pgfpathcurveto{\pgfqpoint{4.188090in}{1.061350in}}{\pgfqpoint{4.185256in}{1.054508in}}{\pgfqpoint{4.185256in}{1.047376in}}%
\pgfpathcurveto{\pgfqpoint{4.185256in}{1.040243in}}{\pgfqpoint{4.188090in}{1.033401in}}{\pgfqpoint{4.193134in}{1.028357in}}%
\pgfpathcurveto{\pgfqpoint{4.198177in}{1.023314in}}{\pgfqpoint{4.205019in}{1.020480in}}{\pgfqpoint{4.212152in}{1.020480in}}%
\pgfpathclose%
\pgfusepath{stroke,fill}%
\end{pgfscope}%
\begin{pgfscope}%
\pgfpathrectangle{\pgfqpoint{2.867647in}{0.500000in}}{\pgfqpoint{1.764706in}{1.700000in}}%
\pgfusepath{clip}%
\pgfsetbuttcap%
\pgfsetroundjoin%
\definecolor{currentfill}{rgb}{0.970718,0.821518,0.719872}%
\pgfsetfillcolor{currentfill}%
\pgfsetlinewidth{0.311001pt}%
\definecolor{currentstroke}{rgb}{1.000000,1.000000,1.000000}%
\pgfsetstrokecolor{currentstroke}%
\pgfsetdash{}{0pt}%
\pgfpathmoveto{\pgfqpoint{4.152688in}{1.660313in}}%
\pgfpathcurveto{\pgfqpoint{4.159821in}{1.660313in}}{\pgfqpoint{4.166663in}{1.663147in}}{\pgfqpoint{4.171706in}{1.668191in}}%
\pgfpathcurveto{\pgfqpoint{4.176750in}{1.673234in}}{\pgfqpoint{4.179584in}{1.680076in}}{\pgfqpoint{4.179584in}{1.687209in}}%
\pgfpathcurveto{\pgfqpoint{4.179584in}{1.694342in}}{\pgfqpoint{4.176750in}{1.701183in}}{\pgfqpoint{4.171706in}{1.706227in}}%
\pgfpathcurveto{\pgfqpoint{4.166663in}{1.711271in}}{\pgfqpoint{4.159821in}{1.714105in}}{\pgfqpoint{4.152688in}{1.714105in}}%
\pgfpathcurveto{\pgfqpoint{4.145555in}{1.714105in}}{\pgfqpoint{4.138714in}{1.711271in}}{\pgfqpoint{4.133670in}{1.706227in}}%
\pgfpathcurveto{\pgfqpoint{4.128626in}{1.701183in}}{\pgfqpoint{4.125792in}{1.694342in}}{\pgfqpoint{4.125792in}{1.687209in}}%
\pgfpathcurveto{\pgfqpoint{4.125792in}{1.680076in}}{\pgfqpoint{4.128626in}{1.673234in}}{\pgfqpoint{4.133670in}{1.668191in}}%
\pgfpathcurveto{\pgfqpoint{4.138714in}{1.663147in}}{\pgfqpoint{4.145555in}{1.660313in}}{\pgfqpoint{4.152688in}{1.660313in}}%
\pgfpathclose%
\pgfusepath{stroke,fill}%
\end{pgfscope}%
\begin{pgfscope}%
\pgfpathrectangle{\pgfqpoint{2.867647in}{0.500000in}}{\pgfqpoint{1.764706in}{1.700000in}}%
\pgfusepath{clip}%
\pgfsetbuttcap%
\pgfsetroundjoin%
\definecolor{currentfill}{rgb}{0.967735,0.780441,0.659127}%
\pgfsetfillcolor{currentfill}%
\pgfsetlinewidth{0.311001pt}%
\definecolor{currentstroke}{rgb}{1.000000,1.000000,1.000000}%
\pgfsetstrokecolor{currentstroke}%
\pgfsetdash{}{0pt}%
\pgfpathmoveto{\pgfqpoint{4.091695in}{1.382268in}}%
\pgfpathcurveto{\pgfqpoint{4.098828in}{1.382268in}}{\pgfqpoint{4.105670in}{1.385102in}}{\pgfqpoint{4.110713in}{1.390146in}}%
\pgfpathcurveto{\pgfqpoint{4.115757in}{1.395189in}}{\pgfqpoint{4.118591in}{1.402031in}}{\pgfqpoint{4.118591in}{1.409164in}}%
\pgfpathcurveto{\pgfqpoint{4.118591in}{1.416297in}}{\pgfqpoint{4.115757in}{1.423138in}}{\pgfqpoint{4.110713in}{1.428182in}}%
\pgfpathcurveto{\pgfqpoint{4.105670in}{1.433226in}}{\pgfqpoint{4.098828in}{1.436060in}}{\pgfqpoint{4.091695in}{1.436060in}}%
\pgfpathcurveto{\pgfqpoint{4.084563in}{1.436060in}}{\pgfqpoint{4.077721in}{1.433226in}}{\pgfqpoint{4.072677in}{1.428182in}}%
\pgfpathcurveto{\pgfqpoint{4.067634in}{1.423138in}}{\pgfqpoint{4.064800in}{1.416297in}}{\pgfqpoint{4.064800in}{1.409164in}}%
\pgfpathcurveto{\pgfqpoint{4.064800in}{1.402031in}}{\pgfqpoint{4.067634in}{1.395189in}}{\pgfqpoint{4.072677in}{1.390146in}}%
\pgfpathcurveto{\pgfqpoint{4.077721in}{1.385102in}}{\pgfqpoint{4.084563in}{1.382268in}}{\pgfqpoint{4.091695in}{1.382268in}}%
\pgfpathclose%
\pgfusepath{stroke,fill}%
\end{pgfscope}%
\begin{pgfscope}%
\pgfpathrectangle{\pgfqpoint{2.867647in}{0.500000in}}{\pgfqpoint{1.764706in}{1.700000in}}%
\pgfusepath{clip}%
\pgfsetbuttcap%
\pgfsetroundjoin%
\definecolor{currentfill}{rgb}{0.962532,0.599594,0.438051}%
\pgfsetfillcolor{currentfill}%
\pgfsetlinewidth{0.311001pt}%
\definecolor{currentstroke}{rgb}{1.000000,1.000000,1.000000}%
\pgfsetstrokecolor{currentstroke}%
\pgfsetdash{}{0pt}%
\pgfpathmoveto{\pgfqpoint{4.028379in}{0.849951in}}%
\pgfpathcurveto{\pgfqpoint{4.035511in}{0.849951in}}{\pgfqpoint{4.042353in}{0.852785in}}{\pgfqpoint{4.047397in}{0.857829in}}%
\pgfpathcurveto{\pgfqpoint{4.052440in}{0.862873in}}{\pgfqpoint{4.055274in}{0.869714in}}{\pgfqpoint{4.055274in}{0.876847in}}%
\pgfpathcurveto{\pgfqpoint{4.055274in}{0.883980in}}{\pgfqpoint{4.052440in}{0.890822in}}{\pgfqpoint{4.047397in}{0.895865in}}%
\pgfpathcurveto{\pgfqpoint{4.042353in}{0.900909in}}{\pgfqpoint{4.035511in}{0.903743in}}{\pgfqpoint{4.028379in}{0.903743in}}%
\pgfpathcurveto{\pgfqpoint{4.021246in}{0.903743in}}{\pgfqpoint{4.014404in}{0.900909in}}{\pgfqpoint{4.009360in}{0.895865in}}%
\pgfpathcurveto{\pgfqpoint{4.004317in}{0.890822in}}{\pgfqpoint{4.001483in}{0.883980in}}{\pgfqpoint{4.001483in}{0.876847in}}%
\pgfpathcurveto{\pgfqpoint{4.001483in}{0.869714in}}{\pgfqpoint{4.004317in}{0.862873in}}{\pgfqpoint{4.009360in}{0.857829in}}%
\pgfpathcurveto{\pgfqpoint{4.014404in}{0.852785in}}{\pgfqpoint{4.021246in}{0.849951in}}{\pgfqpoint{4.028379in}{0.849951in}}%
\pgfpathclose%
\pgfusepath{stroke,fill}%
\end{pgfscope}%
\begin{pgfscope}%
\pgfpathrectangle{\pgfqpoint{2.867647in}{0.500000in}}{\pgfqpoint{1.764706in}{1.700000in}}%
\pgfusepath{clip}%
\pgfsetbuttcap%
\pgfsetroundjoin%
\definecolor{currentfill}{rgb}{0.979124,0.903132,0.839793}%
\pgfsetfillcolor{currentfill}%
\pgfsetlinewidth{0.311001pt}%
\definecolor{currentstroke}{rgb}{1.000000,1.000000,1.000000}%
\pgfsetstrokecolor{currentstroke}%
\pgfsetdash{}{0pt}%
\pgfpathmoveto{\pgfqpoint{4.156541in}{1.380666in}}%
\pgfpathcurveto{\pgfqpoint{4.163674in}{1.380666in}}{\pgfqpoint{4.170516in}{1.383500in}}{\pgfqpoint{4.175559in}{1.388544in}}%
\pgfpathcurveto{\pgfqpoint{4.180603in}{1.393587in}}{\pgfqpoint{4.183437in}{1.400429in}}{\pgfqpoint{4.183437in}{1.407562in}}%
\pgfpathcurveto{\pgfqpoint{4.183437in}{1.414695in}}{\pgfqpoint{4.180603in}{1.421536in}}{\pgfqpoint{4.175559in}{1.426580in}}%
\pgfpathcurveto{\pgfqpoint{4.170516in}{1.431623in}}{\pgfqpoint{4.163674in}{1.434457in}}{\pgfqpoint{4.156541in}{1.434457in}}%
\pgfpathcurveto{\pgfqpoint{4.149408in}{1.434457in}}{\pgfqpoint{4.142567in}{1.431623in}}{\pgfqpoint{4.137523in}{1.426580in}}%
\pgfpathcurveto{\pgfqpoint{4.132479in}{1.421536in}}{\pgfqpoint{4.129646in}{1.414695in}}{\pgfqpoint{4.129646in}{1.407562in}}%
\pgfpathcurveto{\pgfqpoint{4.129646in}{1.400429in}}{\pgfqpoint{4.132479in}{1.393587in}}{\pgfqpoint{4.137523in}{1.388544in}}%
\pgfpathcurveto{\pgfqpoint{4.142567in}{1.383500in}}{\pgfqpoint{4.149408in}{1.380666in}}{\pgfqpoint{4.156541in}{1.380666in}}%
\pgfpathclose%
\pgfusepath{stroke,fill}%
\end{pgfscope}%
\begin{pgfscope}%
\pgfpathrectangle{\pgfqpoint{2.867647in}{0.500000in}}{\pgfqpoint{1.764706in}{1.700000in}}%
\pgfusepath{clip}%
\pgfsetbuttcap%
\pgfsetroundjoin%
\definecolor{currentfill}{rgb}{0.978376,0.897317,0.831308}%
\pgfsetfillcolor{currentfill}%
\pgfsetlinewidth{0.311001pt}%
\definecolor{currentstroke}{rgb}{1.000000,1.000000,1.000000}%
\pgfsetstrokecolor{currentstroke}%
\pgfsetdash{}{0pt}%
\pgfpathmoveto{\pgfqpoint{4.232167in}{1.326399in}}%
\pgfpathcurveto{\pgfqpoint{4.239300in}{1.326399in}}{\pgfqpoint{4.246141in}{1.329233in}}{\pgfqpoint{4.251185in}{1.334277in}}%
\pgfpathcurveto{\pgfqpoint{4.256228in}{1.339321in}}{\pgfqpoint{4.259062in}{1.346162in}}{\pgfqpoint{4.259062in}{1.353295in}}%
\pgfpathcurveto{\pgfqpoint{4.259062in}{1.360428in}}{\pgfqpoint{4.256228in}{1.367270in}}{\pgfqpoint{4.251185in}{1.372313in}}%
\pgfpathcurveto{\pgfqpoint{4.246141in}{1.377357in}}{\pgfqpoint{4.239300in}{1.380191in}}{\pgfqpoint{4.232167in}{1.380191in}}%
\pgfpathcurveto{\pgfqpoint{4.225034in}{1.380191in}}{\pgfqpoint{4.218192in}{1.377357in}}{\pgfqpoint{4.213149in}{1.372313in}}%
\pgfpathcurveto{\pgfqpoint{4.208105in}{1.367270in}}{\pgfqpoint{4.205271in}{1.360428in}}{\pgfqpoint{4.205271in}{1.353295in}}%
\pgfpathcurveto{\pgfqpoint{4.205271in}{1.346162in}}{\pgfqpoint{4.208105in}{1.339321in}}{\pgfqpoint{4.213149in}{1.334277in}}%
\pgfpathcurveto{\pgfqpoint{4.218192in}{1.329233in}}{\pgfqpoint{4.225034in}{1.326399in}}{\pgfqpoint{4.232167in}{1.326399in}}%
\pgfpathclose%
\pgfusepath{stroke,fill}%
\end{pgfscope}%
\begin{pgfscope}%
\pgfpathrectangle{\pgfqpoint{2.867647in}{0.500000in}}{\pgfqpoint{1.764706in}{1.700000in}}%
\pgfusepath{clip}%
\pgfsetbuttcap%
\pgfsetroundjoin%
\definecolor{currentfill}{rgb}{0.977657,0.891500,0.822809}%
\pgfsetfillcolor{currentfill}%
\pgfsetlinewidth{0.311001pt}%
\definecolor{currentstroke}{rgb}{1.000000,1.000000,1.000000}%
\pgfsetstrokecolor{currentstroke}%
\pgfsetdash{}{0pt}%
\pgfpathmoveto{\pgfqpoint{4.213210in}{1.464894in}}%
\pgfpathcurveto{\pgfqpoint{4.220343in}{1.464894in}}{\pgfqpoint{4.227185in}{1.467728in}}{\pgfqpoint{4.232228in}{1.472772in}}%
\pgfpathcurveto{\pgfqpoint{4.237272in}{1.477816in}}{\pgfqpoint{4.240106in}{1.484657in}}{\pgfqpoint{4.240106in}{1.491790in}}%
\pgfpathcurveto{\pgfqpoint{4.240106in}{1.498923in}}{\pgfqpoint{4.237272in}{1.505765in}}{\pgfqpoint{4.232228in}{1.510808in}}%
\pgfpathcurveto{\pgfqpoint{4.227185in}{1.515852in}}{\pgfqpoint{4.220343in}{1.518686in}}{\pgfqpoint{4.213210in}{1.518686in}}%
\pgfpathcurveto{\pgfqpoint{4.206077in}{1.518686in}}{\pgfqpoint{4.199236in}{1.515852in}}{\pgfqpoint{4.194192in}{1.510808in}}%
\pgfpathcurveto{\pgfqpoint{4.189148in}{1.505765in}}{\pgfqpoint{4.186315in}{1.498923in}}{\pgfqpoint{4.186315in}{1.491790in}}%
\pgfpathcurveto{\pgfqpoint{4.186315in}{1.484657in}}{\pgfqpoint{4.189148in}{1.477816in}}{\pgfqpoint{4.194192in}{1.472772in}}%
\pgfpathcurveto{\pgfqpoint{4.199236in}{1.467728in}}{\pgfqpoint{4.206077in}{1.464894in}}{\pgfqpoint{4.213210in}{1.464894in}}%
\pgfpathclose%
\pgfusepath{stroke,fill}%
\end{pgfscope}%
\begin{pgfscope}%
\pgfpathrectangle{\pgfqpoint{2.867647in}{0.500000in}}{\pgfqpoint{1.764706in}{1.700000in}}%
\pgfusepath{clip}%
\pgfsetbuttcap%
\pgfsetroundjoin%
\definecolor{currentfill}{rgb}{0.965753,0.732351,0.592427}%
\pgfsetfillcolor{currentfill}%
\pgfsetlinewidth{0.311001pt}%
\definecolor{currentstroke}{rgb}{1.000000,1.000000,1.000000}%
\pgfsetstrokecolor{currentstroke}%
\pgfsetdash{}{0pt}%
\pgfpathmoveto{\pgfqpoint{4.091370in}{1.747215in}}%
\pgfpathcurveto{\pgfqpoint{4.098503in}{1.747215in}}{\pgfqpoint{4.105345in}{1.750049in}}{\pgfqpoint{4.110388in}{1.755092in}}%
\pgfpathcurveto{\pgfqpoint{4.115432in}{1.760136in}}{\pgfqpoint{4.118266in}{1.766978in}}{\pgfqpoint{4.118266in}{1.774111in}}%
\pgfpathcurveto{\pgfqpoint{4.118266in}{1.781243in}}{\pgfqpoint{4.115432in}{1.788085in}}{\pgfqpoint{4.110388in}{1.793129in}}%
\pgfpathcurveto{\pgfqpoint{4.105345in}{1.798172in}}{\pgfqpoint{4.098503in}{1.801006in}}{\pgfqpoint{4.091370in}{1.801006in}}%
\pgfpathcurveto{\pgfqpoint{4.084237in}{1.801006in}}{\pgfqpoint{4.077396in}{1.798172in}}{\pgfqpoint{4.072352in}{1.793129in}}%
\pgfpathcurveto{\pgfqpoint{4.067308in}{1.788085in}}{\pgfqpoint{4.064475in}{1.781243in}}{\pgfqpoint{4.064475in}{1.774111in}}%
\pgfpathcurveto{\pgfqpoint{4.064475in}{1.766978in}}{\pgfqpoint{4.067308in}{1.760136in}}{\pgfqpoint{4.072352in}{1.755092in}}%
\pgfpathcurveto{\pgfqpoint{4.077396in}{1.750049in}}{\pgfqpoint{4.084237in}{1.747215in}}{\pgfqpoint{4.091370in}{1.747215in}}%
\pgfpathclose%
\pgfusepath{stroke,fill}%
\end{pgfscope}%
\begin{pgfscope}%
\pgfpathrectangle{\pgfqpoint{2.867647in}{0.500000in}}{\pgfqpoint{1.764706in}{1.700000in}}%
\pgfusepath{clip}%
\pgfsetbuttcap%
\pgfsetroundjoin%
\definecolor{currentfill}{rgb}{0.973271,0.850724,0.762998}%
\pgfsetfillcolor{currentfill}%
\pgfsetlinewidth{0.311001pt}%
\definecolor{currentstroke}{rgb}{1.000000,1.000000,1.000000}%
\pgfsetstrokecolor{currentstroke}%
\pgfsetdash{}{0pt}%
\pgfpathmoveto{\pgfqpoint{4.107294in}{1.195631in}}%
\pgfpathcurveto{\pgfqpoint{4.114427in}{1.195631in}}{\pgfqpoint{4.121269in}{1.198465in}}{\pgfqpoint{4.126313in}{1.203509in}}%
\pgfpathcurveto{\pgfqpoint{4.131356in}{1.208553in}}{\pgfqpoint{4.134190in}{1.215394in}}{\pgfqpoint{4.134190in}{1.222527in}}%
\pgfpathcurveto{\pgfqpoint{4.134190in}{1.229660in}}{\pgfqpoint{4.131356in}{1.236501in}}{\pgfqpoint{4.126313in}{1.241545in}}%
\pgfpathcurveto{\pgfqpoint{4.121269in}{1.246589in}}{\pgfqpoint{4.114427in}{1.249423in}}{\pgfqpoint{4.107294in}{1.249423in}}%
\pgfpathcurveto{\pgfqpoint{4.100162in}{1.249423in}}{\pgfqpoint{4.093320in}{1.246589in}}{\pgfqpoint{4.088276in}{1.241545in}}%
\pgfpathcurveto{\pgfqpoint{4.083233in}{1.236501in}}{\pgfqpoint{4.080399in}{1.229660in}}{\pgfqpoint{4.080399in}{1.222527in}}%
\pgfpathcurveto{\pgfqpoint{4.080399in}{1.215394in}}{\pgfqpoint{4.083233in}{1.208553in}}{\pgfqpoint{4.088276in}{1.203509in}}%
\pgfpathcurveto{\pgfqpoint{4.093320in}{1.198465in}}{\pgfqpoint{4.100162in}{1.195631in}}{\pgfqpoint{4.107294in}{1.195631in}}%
\pgfpathclose%
\pgfusepath{stroke,fill}%
\end{pgfscope}%
\begin{pgfscope}%
\pgfpathrectangle{\pgfqpoint{2.867647in}{0.500000in}}{\pgfqpoint{1.764706in}{1.700000in}}%
\pgfusepath{clip}%
\pgfsetbuttcap%
\pgfsetroundjoin%
\definecolor{currentfill}{rgb}{0.957848,0.512613,0.357119}%
\pgfsetfillcolor{currentfill}%
\pgfsetlinewidth{0.311001pt}%
\definecolor{currentstroke}{rgb}{1.000000,1.000000,1.000000}%
\pgfsetstrokecolor{currentstroke}%
\pgfsetdash{}{0pt}%
\pgfpathmoveto{\pgfqpoint{3.928293in}{0.880092in}}%
\pgfpathcurveto{\pgfqpoint{3.935426in}{0.880092in}}{\pgfqpoint{3.942267in}{0.882926in}}{\pgfqpoint{3.947311in}{0.887969in}}%
\pgfpathcurveto{\pgfqpoint{3.952355in}{0.893013in}}{\pgfqpoint{3.955188in}{0.899855in}}{\pgfqpoint{3.955188in}{0.906987in}}%
\pgfpathcurveto{\pgfqpoint{3.955188in}{0.914120in}}{\pgfqpoint{3.952355in}{0.920962in}}{\pgfqpoint{3.947311in}{0.926006in}}%
\pgfpathcurveto{\pgfqpoint{3.942267in}{0.931049in}}{\pgfqpoint{3.935426in}{0.933883in}}{\pgfqpoint{3.928293in}{0.933883in}}%
\pgfpathcurveto{\pgfqpoint{3.921160in}{0.933883in}}{\pgfqpoint{3.914318in}{0.931049in}}{\pgfqpoint{3.909275in}{0.926006in}}%
\pgfpathcurveto{\pgfqpoint{3.904231in}{0.920962in}}{\pgfqpoint{3.901397in}{0.914120in}}{\pgfqpoint{3.901397in}{0.906987in}}%
\pgfpathcurveto{\pgfqpoint{3.901397in}{0.899855in}}{\pgfqpoint{3.904231in}{0.893013in}}{\pgfqpoint{3.909275in}{0.887969in}}%
\pgfpathcurveto{\pgfqpoint{3.914318in}{0.882926in}}{\pgfqpoint{3.921160in}{0.880092in}}{\pgfqpoint{3.928293in}{0.880092in}}%
\pgfpathclose%
\pgfusepath{stroke,fill}%
\end{pgfscope}%
\begin{pgfscope}%
\pgfpathrectangle{\pgfqpoint{2.867647in}{0.500000in}}{\pgfqpoint{1.764706in}{1.700000in}}%
\pgfusepath{clip}%
\pgfsetbuttcap%
\pgfsetroundjoin%
\definecolor{currentfill}{rgb}{0.972201,0.839051,0.745789}%
\pgfsetfillcolor{currentfill}%
\pgfsetlinewidth{0.311001pt}%
\definecolor{currentstroke}{rgb}{1.000000,1.000000,1.000000}%
\pgfsetstrokecolor{currentstroke}%
\pgfsetdash{}{0pt}%
\pgfpathmoveto{\pgfqpoint{4.261661in}{1.368470in}}%
\pgfpathcurveto{\pgfqpoint{4.268794in}{1.368470in}}{\pgfqpoint{4.275636in}{1.371304in}}{\pgfqpoint{4.280679in}{1.376348in}}%
\pgfpathcurveto{\pgfqpoint{4.285723in}{1.381391in}}{\pgfqpoint{4.288557in}{1.388233in}}{\pgfqpoint{4.288557in}{1.395366in}}%
\pgfpathcurveto{\pgfqpoint{4.288557in}{1.402499in}}{\pgfqpoint{4.285723in}{1.409340in}}{\pgfqpoint{4.280679in}{1.414384in}}%
\pgfpathcurveto{\pgfqpoint{4.275636in}{1.419428in}}{\pgfqpoint{4.268794in}{1.422262in}}{\pgfqpoint{4.261661in}{1.422262in}}%
\pgfpathcurveto{\pgfqpoint{4.254528in}{1.422262in}}{\pgfqpoint{4.247687in}{1.419428in}}{\pgfqpoint{4.242643in}{1.414384in}}%
\pgfpathcurveto{\pgfqpoint{4.237599in}{1.409340in}}{\pgfqpoint{4.234766in}{1.402499in}}{\pgfqpoint{4.234766in}{1.395366in}}%
\pgfpathcurveto{\pgfqpoint{4.234766in}{1.388233in}}{\pgfqpoint{4.237599in}{1.381391in}}{\pgfqpoint{4.242643in}{1.376348in}}%
\pgfpathcurveto{\pgfqpoint{4.247687in}{1.371304in}}{\pgfqpoint{4.254528in}{1.368470in}}{\pgfqpoint{4.261661in}{1.368470in}}%
\pgfpathclose%
\pgfusepath{stroke,fill}%
\end{pgfscope}%
\begin{pgfscope}%
\pgfpathrectangle{\pgfqpoint{2.867647in}{0.500000in}}{\pgfqpoint{1.764706in}{1.700000in}}%
\pgfusepath{clip}%
\pgfsetbuttcap%
\pgfsetroundjoin%
\definecolor{currentfill}{rgb}{0.963884,0.644842,0.486120}%
\pgfsetfillcolor{currentfill}%
\pgfsetlinewidth{0.311001pt}%
\definecolor{currentstroke}{rgb}{1.000000,1.000000,1.000000}%
\pgfsetstrokecolor{currentstroke}%
\pgfsetdash{}{0pt}%
\pgfpathmoveto{\pgfqpoint{4.173121in}{1.722381in}}%
\pgfpathcurveto{\pgfqpoint{4.180254in}{1.722381in}}{\pgfqpoint{4.187095in}{1.725215in}}{\pgfqpoint{4.192139in}{1.730259in}}%
\pgfpathcurveto{\pgfqpoint{4.197183in}{1.735303in}}{\pgfqpoint{4.200017in}{1.742144in}}{\pgfqpoint{4.200017in}{1.749277in}}%
\pgfpathcurveto{\pgfqpoint{4.200017in}{1.756410in}}{\pgfqpoint{4.197183in}{1.763252in}}{\pgfqpoint{4.192139in}{1.768295in}}%
\pgfpathcurveto{\pgfqpoint{4.187095in}{1.773339in}}{\pgfqpoint{4.180254in}{1.776173in}}{\pgfqpoint{4.173121in}{1.776173in}}%
\pgfpathcurveto{\pgfqpoint{4.165988in}{1.776173in}}{\pgfqpoint{4.159147in}{1.773339in}}{\pgfqpoint{4.154103in}{1.768295in}}%
\pgfpathcurveto{\pgfqpoint{4.149059in}{1.763252in}}{\pgfqpoint{4.146225in}{1.756410in}}{\pgfqpoint{4.146225in}{1.749277in}}%
\pgfpathcurveto{\pgfqpoint{4.146225in}{1.742144in}}{\pgfqpoint{4.149059in}{1.735303in}}{\pgfqpoint{4.154103in}{1.730259in}}%
\pgfpathcurveto{\pgfqpoint{4.159147in}{1.725215in}}{\pgfqpoint{4.165988in}{1.722381in}}{\pgfqpoint{4.173121in}{1.722381in}}%
\pgfpathclose%
\pgfusepath{stroke,fill}%
\end{pgfscope}%
\begin{pgfscope}%
\pgfpathrectangle{\pgfqpoint{2.867647in}{0.500000in}}{\pgfqpoint{1.764706in}{1.700000in}}%
\pgfusepath{clip}%
\pgfsetbuttcap%
\pgfsetroundjoin%
\definecolor{currentfill}{rgb}{0.973271,0.850724,0.762998}%
\pgfsetfillcolor{currentfill}%
\pgfsetlinewidth{0.311001pt}%
\definecolor{currentstroke}{rgb}{1.000000,1.000000,1.000000}%
\pgfsetstrokecolor{currentstroke}%
\pgfsetdash{}{0pt}%
\pgfpathmoveto{\pgfqpoint{4.247291in}{1.188303in}}%
\pgfpathcurveto{\pgfqpoint{4.254424in}{1.188303in}}{\pgfqpoint{4.261265in}{1.191137in}}{\pgfqpoint{4.266309in}{1.196180in}}%
\pgfpathcurveto{\pgfqpoint{4.271353in}{1.201224in}}{\pgfqpoint{4.274187in}{1.208066in}}{\pgfqpoint{4.274187in}{1.215198in}}%
\pgfpathcurveto{\pgfqpoint{4.274187in}{1.222331in}}{\pgfqpoint{4.271353in}{1.229173in}}{\pgfqpoint{4.266309in}{1.234216in}}%
\pgfpathcurveto{\pgfqpoint{4.261265in}{1.239260in}}{\pgfqpoint{4.254424in}{1.242094in}}{\pgfqpoint{4.247291in}{1.242094in}}%
\pgfpathcurveto{\pgfqpoint{4.240158in}{1.242094in}}{\pgfqpoint{4.233316in}{1.239260in}}{\pgfqpoint{4.228273in}{1.234216in}}%
\pgfpathcurveto{\pgfqpoint{4.223229in}{1.229173in}}{\pgfqpoint{4.220395in}{1.222331in}}{\pgfqpoint{4.220395in}{1.215198in}}%
\pgfpathcurveto{\pgfqpoint{4.220395in}{1.208066in}}{\pgfqpoint{4.223229in}{1.201224in}}{\pgfqpoint{4.228273in}{1.196180in}}%
\pgfpathcurveto{\pgfqpoint{4.233316in}{1.191137in}}{\pgfqpoint{4.240158in}{1.188303in}}{\pgfqpoint{4.247291in}{1.188303in}}%
\pgfpathclose%
\pgfusepath{stroke,fill}%
\end{pgfscope}%
\begin{pgfscope}%
\pgfpathrectangle{\pgfqpoint{2.867647in}{0.500000in}}{\pgfqpoint{1.764706in}{1.700000in}}%
\pgfusepath{clip}%
\pgfsetbuttcap%
\pgfsetroundjoin%
\definecolor{currentfill}{rgb}{0.980678,0.914765,0.856766}%
\pgfsetfillcolor{currentfill}%
\pgfsetlinewidth{0.311001pt}%
\definecolor{currentstroke}{rgb}{1.000000,1.000000,1.000000}%
\pgfsetstrokecolor{currentstroke}%
\pgfsetdash{}{0pt}%
\pgfpathmoveto{\pgfqpoint{4.149562in}{1.502985in}}%
\pgfpathcurveto{\pgfqpoint{4.156695in}{1.502985in}}{\pgfqpoint{4.163537in}{1.505819in}}{\pgfqpoint{4.168581in}{1.510863in}}%
\pgfpathcurveto{\pgfqpoint{4.173624in}{1.515906in}}{\pgfqpoint{4.176458in}{1.522748in}}{\pgfqpoint{4.176458in}{1.529881in}}%
\pgfpathcurveto{\pgfqpoint{4.176458in}{1.537014in}}{\pgfqpoint{4.173624in}{1.543855in}}{\pgfqpoint{4.168581in}{1.548899in}}%
\pgfpathcurveto{\pgfqpoint{4.163537in}{1.553943in}}{\pgfqpoint{4.156695in}{1.556776in}}{\pgfqpoint{4.149562in}{1.556776in}}%
\pgfpathcurveto{\pgfqpoint{4.142430in}{1.556776in}}{\pgfqpoint{4.135588in}{1.553943in}}{\pgfqpoint{4.130544in}{1.548899in}}%
\pgfpathcurveto{\pgfqpoint{4.125501in}{1.543855in}}{\pgfqpoint{4.122667in}{1.537014in}}{\pgfqpoint{4.122667in}{1.529881in}}%
\pgfpathcurveto{\pgfqpoint{4.122667in}{1.522748in}}{\pgfqpoint{4.125501in}{1.515906in}}{\pgfqpoint{4.130544in}{1.510863in}}%
\pgfpathcurveto{\pgfqpoint{4.135588in}{1.505819in}}{\pgfqpoint{4.142430in}{1.502985in}}{\pgfqpoint{4.149562in}{1.502985in}}%
\pgfpathclose%
\pgfusepath{stroke,fill}%
\end{pgfscope}%
\begin{pgfscope}%
\pgfpathrectangle{\pgfqpoint{2.867647in}{0.500000in}}{\pgfqpoint{1.764706in}{1.700000in}}%
\pgfusepath{clip}%
\pgfsetbuttcap%
\pgfsetroundjoin%
\definecolor{currentfill}{rgb}{0.965169,0.707764,0.560659}%
\pgfsetfillcolor{currentfill}%
\pgfsetlinewidth{0.311001pt}%
\definecolor{currentstroke}{rgb}{1.000000,1.000000,1.000000}%
\pgfsetstrokecolor{currentstroke}%
\pgfsetdash{}{0pt}%
\pgfpathmoveto{\pgfqpoint{4.046238in}{1.158721in}}%
\pgfpathcurveto{\pgfqpoint{4.053371in}{1.158721in}}{\pgfqpoint{4.060213in}{1.161555in}}{\pgfqpoint{4.065256in}{1.166599in}}%
\pgfpathcurveto{\pgfqpoint{4.070300in}{1.171642in}}{\pgfqpoint{4.073134in}{1.178484in}}{\pgfqpoint{4.073134in}{1.185617in}}%
\pgfpathcurveto{\pgfqpoint{4.073134in}{1.192750in}}{\pgfqpoint{4.070300in}{1.199591in}}{\pgfqpoint{4.065256in}{1.204635in}}%
\pgfpathcurveto{\pgfqpoint{4.060213in}{1.209679in}}{\pgfqpoint{4.053371in}{1.212513in}}{\pgfqpoint{4.046238in}{1.212513in}}%
\pgfpathcurveto{\pgfqpoint{4.039105in}{1.212513in}}{\pgfqpoint{4.032264in}{1.209679in}}{\pgfqpoint{4.027220in}{1.204635in}}%
\pgfpathcurveto{\pgfqpoint{4.022176in}{1.199591in}}{\pgfqpoint{4.019342in}{1.192750in}}{\pgfqpoint{4.019342in}{1.185617in}}%
\pgfpathcurveto{\pgfqpoint{4.019342in}{1.178484in}}{\pgfqpoint{4.022176in}{1.171642in}}{\pgfqpoint{4.027220in}{1.166599in}}%
\pgfpathcurveto{\pgfqpoint{4.032264in}{1.161555in}}{\pgfqpoint{4.039105in}{1.158721in}}{\pgfqpoint{4.046238in}{1.158721in}}%
\pgfpathclose%
\pgfusepath{stroke,fill}%
\end{pgfscope}%
\begin{pgfscope}%
\pgfpathrectangle{\pgfqpoint{2.867647in}{0.500000in}}{\pgfqpoint{1.764706in}{1.700000in}}%
\pgfusepath{clip}%
\pgfsetbuttcap%
\pgfsetroundjoin%
\definecolor{currentfill}{rgb}{0.971202,0.827364,0.728520}%
\pgfsetfillcolor{currentfill}%
\pgfsetlinewidth{0.311001pt}%
\definecolor{currentstroke}{rgb}{1.000000,1.000000,1.000000}%
\pgfsetstrokecolor{currentstroke}%
\pgfsetdash{}{0pt}%
\pgfpathmoveto{\pgfqpoint{4.266482in}{1.380059in}}%
\pgfpathcurveto{\pgfqpoint{4.273615in}{1.380059in}}{\pgfqpoint{4.280457in}{1.382893in}}{\pgfqpoint{4.285500in}{1.387936in}}%
\pgfpathcurveto{\pgfqpoint{4.290544in}{1.392980in}}{\pgfqpoint{4.293378in}{1.399822in}}{\pgfqpoint{4.293378in}{1.406954in}}%
\pgfpathcurveto{\pgfqpoint{4.293378in}{1.414087in}}{\pgfqpoint{4.290544in}{1.420929in}}{\pgfqpoint{4.285500in}{1.425972in}}%
\pgfpathcurveto{\pgfqpoint{4.280457in}{1.431016in}}{\pgfqpoint{4.273615in}{1.433850in}}{\pgfqpoint{4.266482in}{1.433850in}}%
\pgfpathcurveto{\pgfqpoint{4.259350in}{1.433850in}}{\pgfqpoint{4.252508in}{1.431016in}}{\pgfqpoint{4.247464in}{1.425972in}}%
\pgfpathcurveto{\pgfqpoint{4.242421in}{1.420929in}}{\pgfqpoint{4.239587in}{1.414087in}}{\pgfqpoint{4.239587in}{1.406954in}}%
\pgfpathcurveto{\pgfqpoint{4.239587in}{1.399822in}}{\pgfqpoint{4.242421in}{1.392980in}}{\pgfqpoint{4.247464in}{1.387936in}}%
\pgfpathcurveto{\pgfqpoint{4.252508in}{1.382893in}}{\pgfqpoint{4.259350in}{1.380059in}}{\pgfqpoint{4.266482in}{1.380059in}}%
\pgfpathclose%
\pgfusepath{stroke,fill}%
\end{pgfscope}%
\begin{pgfscope}%
\pgfpathrectangle{\pgfqpoint{2.867647in}{0.500000in}}{\pgfqpoint{1.764706in}{1.700000in}}%
\pgfusepath{clip}%
\pgfsetbuttcap%
\pgfsetroundjoin%
\definecolor{currentfill}{rgb}{0.964173,0.657587,0.500469}%
\pgfsetfillcolor{currentfill}%
\pgfsetlinewidth{0.311001pt}%
\definecolor{currentstroke}{rgb}{1.000000,1.000000,1.000000}%
\pgfsetstrokecolor{currentstroke}%
\pgfsetdash{}{0pt}%
\pgfpathmoveto{\pgfqpoint{3.969466in}{0.938538in}}%
\pgfpathcurveto{\pgfqpoint{3.976598in}{0.938538in}}{\pgfqpoint{3.983440in}{0.941371in}}{\pgfqpoint{3.988484in}{0.946415in}}%
\pgfpathcurveto{\pgfqpoint{3.993527in}{0.951459in}}{\pgfqpoint{3.996361in}{0.958300in}}{\pgfqpoint{3.996361in}{0.965433in}}%
\pgfpathcurveto{\pgfqpoint{3.996361in}{0.972566in}}{\pgfqpoint{3.993527in}{0.979408in}}{\pgfqpoint{3.988484in}{0.984451in}}%
\pgfpathcurveto{\pgfqpoint{3.983440in}{0.989495in}}{\pgfqpoint{3.976598in}{0.992329in}}{\pgfqpoint{3.969466in}{0.992329in}}%
\pgfpathcurveto{\pgfqpoint{3.962333in}{0.992329in}}{\pgfqpoint{3.955491in}{0.989495in}}{\pgfqpoint{3.950448in}{0.984451in}}%
\pgfpathcurveto{\pgfqpoint{3.945404in}{0.979408in}}{\pgfqpoint{3.942570in}{0.972566in}}{\pgfqpoint{3.942570in}{0.965433in}}%
\pgfpathcurveto{\pgfqpoint{3.942570in}{0.958300in}}{\pgfqpoint{3.945404in}{0.951459in}}{\pgfqpoint{3.950448in}{0.946415in}}%
\pgfpathcurveto{\pgfqpoint{3.955491in}{0.941371in}}{\pgfqpoint{3.962333in}{0.938538in}}{\pgfqpoint{3.969466in}{0.938538in}}%
\pgfpathclose%
\pgfusepath{stroke,fill}%
\end{pgfscope}%
\begin{pgfscope}%
\pgfpathrectangle{\pgfqpoint{2.867647in}{0.500000in}}{\pgfqpoint{1.764706in}{1.700000in}}%
\pgfusepath{clip}%
\pgfsetbuttcap%
\pgfsetroundjoin%
\definecolor{currentfill}{rgb}{0.975018,0.868213,0.788710}%
\pgfsetfillcolor{currentfill}%
\pgfsetlinewidth{0.311001pt}%
\definecolor{currentstroke}{rgb}{1.000000,1.000000,1.000000}%
\pgfsetstrokecolor{currentstroke}%
\pgfsetdash{}{0pt}%
\pgfpathmoveto{\pgfqpoint{4.139870in}{1.014696in}}%
\pgfpathcurveto{\pgfqpoint{4.147002in}{1.014696in}}{\pgfqpoint{4.153844in}{1.017530in}}{\pgfqpoint{4.158888in}{1.022574in}}%
\pgfpathcurveto{\pgfqpoint{4.163931in}{1.027617in}}{\pgfqpoint{4.166765in}{1.034459in}}{\pgfqpoint{4.166765in}{1.041592in}}%
\pgfpathcurveto{\pgfqpoint{4.166765in}{1.048725in}}{\pgfqpoint{4.163931in}{1.055566in}}{\pgfqpoint{4.158888in}{1.060610in}}%
\pgfpathcurveto{\pgfqpoint{4.153844in}{1.065654in}}{\pgfqpoint{4.147002in}{1.068488in}}{\pgfqpoint{4.139870in}{1.068488in}}%
\pgfpathcurveto{\pgfqpoint{4.132737in}{1.068488in}}{\pgfqpoint{4.125895in}{1.065654in}}{\pgfqpoint{4.120851in}{1.060610in}}%
\pgfpathcurveto{\pgfqpoint{4.115808in}{1.055566in}}{\pgfqpoint{4.112974in}{1.048725in}}{\pgfqpoint{4.112974in}{1.041592in}}%
\pgfpathcurveto{\pgfqpoint{4.112974in}{1.034459in}}{\pgfqpoint{4.115808in}{1.027617in}}{\pgfqpoint{4.120851in}{1.022574in}}%
\pgfpathcurveto{\pgfqpoint{4.125895in}{1.017530in}}{\pgfqpoint{4.132737in}{1.014696in}}{\pgfqpoint{4.139870in}{1.014696in}}%
\pgfpathclose%
\pgfusepath{stroke,fill}%
\end{pgfscope}%
\begin{pgfscope}%
\pgfpathrectangle{\pgfqpoint{2.867647in}{0.500000in}}{\pgfqpoint{1.764706in}{1.700000in}}%
\pgfusepath{clip}%
\pgfsetbuttcap%
\pgfsetroundjoin%
\definecolor{currentfill}{rgb}{0.981377,0.920617,0.865369}%
\pgfsetfillcolor{currentfill}%
\pgfsetlinewidth{0.311001pt}%
\definecolor{currentstroke}{rgb}{1.000000,1.000000,1.000000}%
\pgfsetstrokecolor{currentstroke}%
\pgfsetdash{}{0pt}%
\pgfpathmoveto{\pgfqpoint{4.196479in}{1.222755in}}%
\pgfpathcurveto{\pgfqpoint{4.203611in}{1.222755in}}{\pgfqpoint{4.210453in}{1.225589in}}{\pgfqpoint{4.215497in}{1.230632in}}%
\pgfpathcurveto{\pgfqpoint{4.220540in}{1.235676in}}{\pgfqpoint{4.223374in}{1.242518in}}{\pgfqpoint{4.223374in}{1.249651in}}%
\pgfpathcurveto{\pgfqpoint{4.223374in}{1.256783in}}{\pgfqpoint{4.220540in}{1.263625in}}{\pgfqpoint{4.215497in}{1.268669in}}%
\pgfpathcurveto{\pgfqpoint{4.210453in}{1.273712in}}{\pgfqpoint{4.203611in}{1.276546in}}{\pgfqpoint{4.196479in}{1.276546in}}%
\pgfpathcurveto{\pgfqpoint{4.189346in}{1.276546in}}{\pgfqpoint{4.182504in}{1.273712in}}{\pgfqpoint{4.177460in}{1.268669in}}%
\pgfpathcurveto{\pgfqpoint{4.172417in}{1.263625in}}{\pgfqpoint{4.169583in}{1.256783in}}{\pgfqpoint{4.169583in}{1.249651in}}%
\pgfpathcurveto{\pgfqpoint{4.169583in}{1.242518in}}{\pgfqpoint{4.172417in}{1.235676in}}{\pgfqpoint{4.177460in}{1.230632in}}%
\pgfpathcurveto{\pgfqpoint{4.182504in}{1.225589in}}{\pgfqpoint{4.189346in}{1.222755in}}{\pgfqpoint{4.196479in}{1.222755in}}%
\pgfpathclose%
\pgfusepath{stroke,fill}%
\end{pgfscope}%
\begin{pgfscope}%
\pgfpathrectangle{\pgfqpoint{2.867647in}{0.500000in}}{\pgfqpoint{1.764706in}{1.700000in}}%
\pgfusepath{clip}%
\pgfsetbuttcap%
\pgfsetroundjoin%
\definecolor{currentfill}{rgb}{0.966560,0.756582,0.625273}%
\pgfsetfillcolor{currentfill}%
\pgfsetlinewidth{0.311001pt}%
\definecolor{currentstroke}{rgb}{1.000000,1.000000,1.000000}%
\pgfsetstrokecolor{currentstroke}%
\pgfsetdash{}{0pt}%
\pgfpathmoveto{\pgfqpoint{4.203345in}{0.999219in}}%
\pgfpathcurveto{\pgfqpoint{4.210477in}{0.999219in}}{\pgfqpoint{4.217319in}{1.002053in}}{\pgfqpoint{4.222363in}{1.007097in}}%
\pgfpathcurveto{\pgfqpoint{4.227406in}{1.012140in}}{\pgfqpoint{4.230240in}{1.018982in}}{\pgfqpoint{4.230240in}{1.026115in}}%
\pgfpathcurveto{\pgfqpoint{4.230240in}{1.033248in}}{\pgfqpoint{4.227406in}{1.040089in}}{\pgfqpoint{4.222363in}{1.045133in}}%
\pgfpathcurveto{\pgfqpoint{4.217319in}{1.050177in}}{\pgfqpoint{4.210477in}{1.053010in}}{\pgfqpoint{4.203345in}{1.053010in}}%
\pgfpathcurveto{\pgfqpoint{4.196212in}{1.053010in}}{\pgfqpoint{4.189370in}{1.050177in}}{\pgfqpoint{4.184327in}{1.045133in}}%
\pgfpathcurveto{\pgfqpoint{4.179283in}{1.040089in}}{\pgfqpoint{4.176449in}{1.033248in}}{\pgfqpoint{4.176449in}{1.026115in}}%
\pgfpathcurveto{\pgfqpoint{4.176449in}{1.018982in}}{\pgfqpoint{4.179283in}{1.012140in}}{\pgfqpoint{4.184327in}{1.007097in}}%
\pgfpathcurveto{\pgfqpoint{4.189370in}{1.002053in}}{\pgfqpoint{4.196212in}{0.999219in}}{\pgfqpoint{4.203345in}{0.999219in}}%
\pgfpathclose%
\pgfusepath{stroke,fill}%
\end{pgfscope}%
\begin{pgfscope}%
\pgfpathrectangle{\pgfqpoint{2.867647in}{0.500000in}}{\pgfqpoint{1.764706in}{1.700000in}}%
\pgfusepath{clip}%
\pgfsetbuttcap%
\pgfsetroundjoin%
\definecolor{currentfill}{rgb}{0.976287,0.879862,0.805788}%
\pgfsetfillcolor{currentfill}%
\pgfsetlinewidth{0.311001pt}%
\definecolor{currentstroke}{rgb}{1.000000,1.000000,1.000000}%
\pgfsetstrokecolor{currentstroke}%
\pgfsetdash{}{0pt}%
\pgfpathmoveto{\pgfqpoint{4.122437in}{1.187050in}}%
\pgfpathcurveto{\pgfqpoint{4.129569in}{1.187050in}}{\pgfqpoint{4.136411in}{1.189884in}}{\pgfqpoint{4.141455in}{1.194927in}}%
\pgfpathcurveto{\pgfqpoint{4.146498in}{1.199971in}}{\pgfqpoint{4.149332in}{1.206813in}}{\pgfqpoint{4.149332in}{1.213946in}}%
\pgfpathcurveto{\pgfqpoint{4.149332in}{1.221078in}}{\pgfqpoint{4.146498in}{1.227920in}}{\pgfqpoint{4.141455in}{1.232964in}}%
\pgfpathcurveto{\pgfqpoint{4.136411in}{1.238007in}}{\pgfqpoint{4.129569in}{1.240841in}}{\pgfqpoint{4.122437in}{1.240841in}}%
\pgfpathcurveto{\pgfqpoint{4.115304in}{1.240841in}}{\pgfqpoint{4.108462in}{1.238007in}}{\pgfqpoint{4.103418in}{1.232964in}}%
\pgfpathcurveto{\pgfqpoint{4.098375in}{1.227920in}}{\pgfqpoint{4.095541in}{1.221078in}}{\pgfqpoint{4.095541in}{1.213946in}}%
\pgfpathcurveto{\pgfqpoint{4.095541in}{1.206813in}}{\pgfqpoint{4.098375in}{1.199971in}}{\pgfqpoint{4.103418in}{1.194927in}}%
\pgfpathcurveto{\pgfqpoint{4.108462in}{1.189884in}}{\pgfqpoint{4.115304in}{1.187050in}}{\pgfqpoint{4.122437in}{1.187050in}}%
\pgfpathclose%
\pgfusepath{stroke,fill}%
\end{pgfscope}%
\begin{pgfscope}%
\pgfpathrectangle{\pgfqpoint{2.867647in}{0.500000in}}{\pgfqpoint{1.764706in}{1.700000in}}%
\pgfusepath{clip}%
\pgfsetbuttcap%
\pgfsetroundjoin%
\definecolor{currentfill}{rgb}{0.972201,0.839051,0.745789}%
\pgfsetfillcolor{currentfill}%
\pgfsetlinewidth{0.311001pt}%
\definecolor{currentstroke}{rgb}{1.000000,1.000000,1.000000}%
\pgfsetstrokecolor{currentstroke}%
\pgfsetdash{}{0pt}%
\pgfpathmoveto{\pgfqpoint{4.247542in}{1.172497in}}%
\pgfpathcurveto{\pgfqpoint{4.254675in}{1.172497in}}{\pgfqpoint{4.261517in}{1.175331in}}{\pgfqpoint{4.266560in}{1.180375in}}%
\pgfpathcurveto{\pgfqpoint{4.271604in}{1.185418in}}{\pgfqpoint{4.274438in}{1.192260in}}{\pgfqpoint{4.274438in}{1.199393in}}%
\pgfpathcurveto{\pgfqpoint{4.274438in}{1.206526in}}{\pgfqpoint{4.271604in}{1.213367in}}{\pgfqpoint{4.266560in}{1.218411in}}%
\pgfpathcurveto{\pgfqpoint{4.261517in}{1.223455in}}{\pgfqpoint{4.254675in}{1.226288in}}{\pgfqpoint{4.247542in}{1.226288in}}%
\pgfpathcurveto{\pgfqpoint{4.240409in}{1.226288in}}{\pgfqpoint{4.233568in}{1.223455in}}{\pgfqpoint{4.228524in}{1.218411in}}%
\pgfpathcurveto{\pgfqpoint{4.223480in}{1.213367in}}{\pgfqpoint{4.220647in}{1.206526in}}{\pgfqpoint{4.220647in}{1.199393in}}%
\pgfpathcurveto{\pgfqpoint{4.220647in}{1.192260in}}{\pgfqpoint{4.223480in}{1.185418in}}{\pgfqpoint{4.228524in}{1.180375in}}%
\pgfpathcurveto{\pgfqpoint{4.233568in}{1.175331in}}{\pgfqpoint{4.240409in}{1.172497in}}{\pgfqpoint{4.247542in}{1.172497in}}%
\pgfpathclose%
\pgfusepath{stroke,fill}%
\end{pgfscope}%
\begin{pgfscope}%
\pgfpathrectangle{\pgfqpoint{2.867647in}{0.500000in}}{\pgfqpoint{1.764706in}{1.700000in}}%
\pgfusepath{clip}%
\pgfsetbuttcap%
\pgfsetroundjoin%
\definecolor{currentfill}{rgb}{0.963884,0.644842,0.486120}%
\pgfsetfillcolor{currentfill}%
\pgfsetlinewidth{0.311001pt}%
\definecolor{currentstroke}{rgb}{1.000000,1.000000,1.000000}%
\pgfsetstrokecolor{currentstroke}%
\pgfsetdash{}{0pt}%
\pgfpathmoveto{\pgfqpoint{4.276537in}{1.569855in}}%
\pgfpathcurveto{\pgfqpoint{4.283669in}{1.569855in}}{\pgfqpoint{4.290511in}{1.572689in}}{\pgfqpoint{4.295555in}{1.577733in}}%
\pgfpathcurveto{\pgfqpoint{4.300598in}{1.582777in}}{\pgfqpoint{4.303432in}{1.589618in}}{\pgfqpoint{4.303432in}{1.596751in}}%
\pgfpathcurveto{\pgfqpoint{4.303432in}{1.603884in}}{\pgfqpoint{4.300598in}{1.610725in}}{\pgfqpoint{4.295555in}{1.615769in}}%
\pgfpathcurveto{\pgfqpoint{4.290511in}{1.620813in}}{\pgfqpoint{4.283669in}{1.623647in}}{\pgfqpoint{4.276537in}{1.623647in}}%
\pgfpathcurveto{\pgfqpoint{4.269404in}{1.623647in}}{\pgfqpoint{4.262562in}{1.620813in}}{\pgfqpoint{4.257518in}{1.615769in}}%
\pgfpathcurveto{\pgfqpoint{4.252475in}{1.610725in}}{\pgfqpoint{4.249641in}{1.603884in}}{\pgfqpoint{4.249641in}{1.596751in}}%
\pgfpathcurveto{\pgfqpoint{4.249641in}{1.589618in}}{\pgfqpoint{4.252475in}{1.582777in}}{\pgfqpoint{4.257518in}{1.577733in}}%
\pgfpathcurveto{\pgfqpoint{4.262562in}{1.572689in}}{\pgfqpoint{4.269404in}{1.569855in}}{\pgfqpoint{4.276537in}{1.569855in}}%
\pgfpathclose%
\pgfusepath{stroke,fill}%
\end{pgfscope}%
\begin{pgfscope}%
\pgfpathrectangle{\pgfqpoint{2.867647in}{0.500000in}}{\pgfqpoint{1.764706in}{1.700000in}}%
\pgfusepath{clip}%
\pgfsetbuttcap%
\pgfsetroundjoin%
\definecolor{currentfill}{rgb}{0.971694,0.833208,0.737161}%
\pgfsetfillcolor{currentfill}%
\pgfsetlinewidth{0.311001pt}%
\definecolor{currentstroke}{rgb}{1.000000,1.000000,1.000000}%
\pgfsetstrokecolor{currentstroke}%
\pgfsetdash{}{0pt}%
\pgfpathmoveto{\pgfqpoint{4.166965in}{1.006178in}}%
\pgfpathcurveto{\pgfqpoint{4.174097in}{1.006178in}}{\pgfqpoint{4.180939in}{1.009012in}}{\pgfqpoint{4.185983in}{1.014055in}}%
\pgfpathcurveto{\pgfqpoint{4.191026in}{1.019099in}}{\pgfqpoint{4.193860in}{1.025941in}}{\pgfqpoint{4.193860in}{1.033074in}}%
\pgfpathcurveto{\pgfqpoint{4.193860in}{1.040206in}}{\pgfqpoint{4.191026in}{1.047048in}}{\pgfqpoint{4.185983in}{1.052092in}}%
\pgfpathcurveto{\pgfqpoint{4.180939in}{1.057135in}}{\pgfqpoint{4.174097in}{1.059969in}}{\pgfqpoint{4.166965in}{1.059969in}}%
\pgfpathcurveto{\pgfqpoint{4.159832in}{1.059969in}}{\pgfqpoint{4.152990in}{1.057135in}}{\pgfqpoint{4.147946in}{1.052092in}}%
\pgfpathcurveto{\pgfqpoint{4.142903in}{1.047048in}}{\pgfqpoint{4.140069in}{1.040206in}}{\pgfqpoint{4.140069in}{1.033074in}}%
\pgfpathcurveto{\pgfqpoint{4.140069in}{1.025941in}}{\pgfqpoint{4.142903in}{1.019099in}}{\pgfqpoint{4.147946in}{1.014055in}}%
\pgfpathcurveto{\pgfqpoint{4.152990in}{1.009012in}}{\pgfqpoint{4.159832in}{1.006178in}}{\pgfqpoint{4.166965in}{1.006178in}}%
\pgfpathclose%
\pgfusepath{stroke,fill}%
\end{pgfscope}%
\begin{pgfscope}%
\pgfpathrectangle{\pgfqpoint{2.867647in}{0.500000in}}{\pgfqpoint{1.764706in}{1.700000in}}%
\pgfusepath{clip}%
\pgfsetbuttcap%
\pgfsetroundjoin%
\definecolor{currentfill}{rgb}{0.976961,0.885681,0.814303}%
\pgfsetfillcolor{currentfill}%
\pgfsetlinewidth{0.311001pt}%
\definecolor{currentstroke}{rgb}{1.000000,1.000000,1.000000}%
\pgfsetstrokecolor{currentstroke}%
\pgfsetdash{}{0pt}%
\pgfpathmoveto{\pgfqpoint{4.110857in}{1.106580in}}%
\pgfpathcurveto{\pgfqpoint{4.117990in}{1.106580in}}{\pgfqpoint{4.124831in}{1.109414in}}{\pgfqpoint{4.129875in}{1.114457in}}%
\pgfpathcurveto{\pgfqpoint{4.134919in}{1.119501in}}{\pgfqpoint{4.137753in}{1.126343in}}{\pgfqpoint{4.137753in}{1.133475in}}%
\pgfpathcurveto{\pgfqpoint{4.137753in}{1.140608in}}{\pgfqpoint{4.134919in}{1.147450in}}{\pgfqpoint{4.129875in}{1.152494in}}%
\pgfpathcurveto{\pgfqpoint{4.124831in}{1.157537in}}{\pgfqpoint{4.117990in}{1.160371in}}{\pgfqpoint{4.110857in}{1.160371in}}%
\pgfpathcurveto{\pgfqpoint{4.103724in}{1.160371in}}{\pgfqpoint{4.096882in}{1.157537in}}{\pgfqpoint{4.091839in}{1.152494in}}%
\pgfpathcurveto{\pgfqpoint{4.086795in}{1.147450in}}{\pgfqpoint{4.083961in}{1.140608in}}{\pgfqpoint{4.083961in}{1.133475in}}%
\pgfpathcurveto{\pgfqpoint{4.083961in}{1.126343in}}{\pgfqpoint{4.086795in}{1.119501in}}{\pgfqpoint{4.091839in}{1.114457in}}%
\pgfpathcurveto{\pgfqpoint{4.096882in}{1.109414in}}{\pgfqpoint{4.103724in}{1.106580in}}{\pgfqpoint{4.110857in}{1.106580in}}%
\pgfpathclose%
\pgfusepath{stroke,fill}%
\end{pgfscope}%
\begin{pgfscope}%
\pgfpathrectangle{\pgfqpoint{2.867647in}{0.500000in}}{\pgfqpoint{1.764706in}{1.700000in}}%
\pgfusepath{clip}%
\pgfsetbuttcap%
\pgfsetroundjoin%
\definecolor{currentfill}{rgb}{0.965302,0.713942,0.568499}%
\pgfsetfillcolor{currentfill}%
\pgfsetlinewidth{0.311001pt}%
\definecolor{currentstroke}{rgb}{1.000000,1.000000,1.000000}%
\pgfsetstrokecolor{currentstroke}%
\pgfsetdash{}{0pt}%
\pgfpathmoveto{\pgfqpoint{4.302269in}{1.390142in}}%
\pgfpathcurveto{\pgfqpoint{4.309402in}{1.390142in}}{\pgfqpoint{4.316243in}{1.392976in}}{\pgfqpoint{4.321287in}{1.398020in}}%
\pgfpathcurveto{\pgfqpoint{4.326331in}{1.403063in}}{\pgfqpoint{4.329164in}{1.409905in}}{\pgfqpoint{4.329164in}{1.417038in}}%
\pgfpathcurveto{\pgfqpoint{4.329164in}{1.424171in}}{\pgfqpoint{4.326331in}{1.431012in}}{\pgfqpoint{4.321287in}{1.436056in}}%
\pgfpathcurveto{\pgfqpoint{4.316243in}{1.441100in}}{\pgfqpoint{4.309402in}{1.443934in}}{\pgfqpoint{4.302269in}{1.443934in}}%
\pgfpathcurveto{\pgfqpoint{4.295136in}{1.443934in}}{\pgfqpoint{4.288294in}{1.441100in}}{\pgfqpoint{4.283251in}{1.436056in}}%
\pgfpathcurveto{\pgfqpoint{4.278207in}{1.431012in}}{\pgfqpoint{4.275373in}{1.424171in}}{\pgfqpoint{4.275373in}{1.417038in}}%
\pgfpathcurveto{\pgfqpoint{4.275373in}{1.409905in}}{\pgfqpoint{4.278207in}{1.403063in}}{\pgfqpoint{4.283251in}{1.398020in}}%
\pgfpathcurveto{\pgfqpoint{4.288294in}{1.392976in}}{\pgfqpoint{4.295136in}{1.390142in}}{\pgfqpoint{4.302269in}{1.390142in}}%
\pgfpathclose%
\pgfusepath{stroke,fill}%
\end{pgfscope}%
\begin{pgfscope}%
\pgfpathrectangle{\pgfqpoint{2.867647in}{0.500000in}}{\pgfqpoint{1.764706in}{1.700000in}}%
\pgfusepath{clip}%
\pgfsetbuttcap%
\pgfsetroundjoin%
\definecolor{currentfill}{rgb}{0.971202,0.827364,0.728520}%
\pgfsetfillcolor{currentfill}%
\pgfsetlinewidth{0.311001pt}%
\definecolor{currentstroke}{rgb}{1.000000,1.000000,1.000000}%
\pgfsetstrokecolor{currentstroke}%
\pgfsetdash{}{0pt}%
\pgfpathmoveto{\pgfqpoint{4.262454in}{1.407588in}}%
\pgfpathcurveto{\pgfqpoint{4.269587in}{1.407588in}}{\pgfqpoint{4.276429in}{1.410422in}}{\pgfqpoint{4.281472in}{1.415465in}}%
\pgfpathcurveto{\pgfqpoint{4.286516in}{1.420509in}}{\pgfqpoint{4.289350in}{1.427351in}}{\pgfqpoint{4.289350in}{1.434484in}}%
\pgfpathcurveto{\pgfqpoint{4.289350in}{1.441616in}}{\pgfqpoint{4.286516in}{1.448458in}}{\pgfqpoint{4.281472in}{1.453502in}}%
\pgfpathcurveto{\pgfqpoint{4.276429in}{1.458545in}}{\pgfqpoint{4.269587in}{1.461379in}}{\pgfqpoint{4.262454in}{1.461379in}}%
\pgfpathcurveto{\pgfqpoint{4.255321in}{1.461379in}}{\pgfqpoint{4.248480in}{1.458545in}}{\pgfqpoint{4.243436in}{1.453502in}}%
\pgfpathcurveto{\pgfqpoint{4.238392in}{1.448458in}}{\pgfqpoint{4.235559in}{1.441616in}}{\pgfqpoint{4.235559in}{1.434484in}}%
\pgfpathcurveto{\pgfqpoint{4.235559in}{1.427351in}}{\pgfqpoint{4.238392in}{1.420509in}}{\pgfqpoint{4.243436in}{1.415465in}}%
\pgfpathcurveto{\pgfqpoint{4.248480in}{1.410422in}}{\pgfqpoint{4.255321in}{1.407588in}}{\pgfqpoint{4.262454in}{1.407588in}}%
\pgfpathclose%
\pgfusepath{stroke,fill}%
\end{pgfscope}%
\begin{pgfscope}%
\pgfpathrectangle{\pgfqpoint{2.867647in}{0.500000in}}{\pgfqpoint{1.764706in}{1.700000in}}%
\pgfusepath{clip}%
\pgfsetbuttcap%
\pgfsetroundjoin%
\definecolor{currentfill}{rgb}{0.964799,0.689101,0.537560}%
\pgfsetfillcolor{currentfill}%
\pgfsetlinewidth{0.311001pt}%
\definecolor{currentstroke}{rgb}{1.000000,1.000000,1.000000}%
\pgfsetstrokecolor{currentstroke}%
\pgfsetdash{}{0pt}%
\pgfpathmoveto{\pgfqpoint{4.010716in}{1.561718in}}%
\pgfpathcurveto{\pgfqpoint{4.017849in}{1.561718in}}{\pgfqpoint{4.024691in}{1.564552in}}{\pgfqpoint{4.029734in}{1.569596in}}%
\pgfpathcurveto{\pgfqpoint{4.034778in}{1.574639in}}{\pgfqpoint{4.037612in}{1.581481in}}{\pgfqpoint{4.037612in}{1.588614in}}%
\pgfpathcurveto{\pgfqpoint{4.037612in}{1.595747in}}{\pgfqpoint{4.034778in}{1.602588in}}{\pgfqpoint{4.029734in}{1.607632in}}%
\pgfpathcurveto{\pgfqpoint{4.024691in}{1.612676in}}{\pgfqpoint{4.017849in}{1.615509in}}{\pgfqpoint{4.010716in}{1.615509in}}%
\pgfpathcurveto{\pgfqpoint{4.003583in}{1.615509in}}{\pgfqpoint{3.996742in}{1.612676in}}{\pgfqpoint{3.991698in}{1.607632in}}%
\pgfpathcurveto{\pgfqpoint{3.986655in}{1.602588in}}{\pgfqpoint{3.983821in}{1.595747in}}{\pgfqpoint{3.983821in}{1.588614in}}%
\pgfpathcurveto{\pgfqpoint{3.983821in}{1.581481in}}{\pgfqpoint{3.986655in}{1.574639in}}{\pgfqpoint{3.991698in}{1.569596in}}%
\pgfpathcurveto{\pgfqpoint{3.996742in}{1.564552in}}{\pgfqpoint{4.003583in}{1.561718in}}{\pgfqpoint{4.010716in}{1.561718in}}%
\pgfpathclose%
\pgfusepath{stroke,fill}%
\end{pgfscope}%
\begin{pgfscope}%
\pgfpathrectangle{\pgfqpoint{2.867647in}{0.500000in}}{\pgfqpoint{1.764706in}{1.700000in}}%
\pgfusepath{clip}%
\pgfsetbuttcap%
\pgfsetroundjoin%
\definecolor{currentfill}{rgb}{0.980678,0.914765,0.856766}%
\pgfsetfillcolor{currentfill}%
\pgfsetlinewidth{0.311001pt}%
\definecolor{currentstroke}{rgb}{1.000000,1.000000,1.000000}%
\pgfsetstrokecolor{currentstroke}%
\pgfsetdash{}{0pt}%
\pgfpathmoveto{\pgfqpoint{4.152716in}{1.143263in}}%
\pgfpathcurveto{\pgfqpoint{4.159848in}{1.143263in}}{\pgfqpoint{4.166690in}{1.146096in}}{\pgfqpoint{4.171734in}{1.151140in}}%
\pgfpathcurveto{\pgfqpoint{4.176777in}{1.156184in}}{\pgfqpoint{4.179611in}{1.163025in}}{\pgfqpoint{4.179611in}{1.170158in}}%
\pgfpathcurveto{\pgfqpoint{4.179611in}{1.177291in}}{\pgfqpoint{4.176777in}{1.184133in}}{\pgfqpoint{4.171734in}{1.189176in}}%
\pgfpathcurveto{\pgfqpoint{4.166690in}{1.194220in}}{\pgfqpoint{4.159848in}{1.197054in}}{\pgfqpoint{4.152716in}{1.197054in}}%
\pgfpathcurveto{\pgfqpoint{4.145583in}{1.197054in}}{\pgfqpoint{4.138741in}{1.194220in}}{\pgfqpoint{4.133697in}{1.189176in}}%
\pgfpathcurveto{\pgfqpoint{4.128654in}{1.184133in}}{\pgfqpoint{4.125820in}{1.177291in}}{\pgfqpoint{4.125820in}{1.170158in}}%
\pgfpathcurveto{\pgfqpoint{4.125820in}{1.163025in}}{\pgfqpoint{4.128654in}{1.156184in}}{\pgfqpoint{4.133697in}{1.151140in}}%
\pgfpathcurveto{\pgfqpoint{4.138741in}{1.146096in}}{\pgfqpoint{4.145583in}{1.143263in}}{\pgfqpoint{4.152716in}{1.143263in}}%
\pgfpathclose%
\pgfusepath{stroke,fill}%
\end{pgfscope}%
\begin{pgfscope}%
\pgfpathrectangle{\pgfqpoint{2.867647in}{0.500000in}}{\pgfqpoint{1.764706in}{1.700000in}}%
\pgfusepath{clip}%
\pgfsetbuttcap%
\pgfsetroundjoin%
\definecolor{currentfill}{rgb}{0.977657,0.891500,0.822809}%
\pgfsetfillcolor{currentfill}%
\pgfsetlinewidth{0.311001pt}%
\definecolor{currentstroke}{rgb}{1.000000,1.000000,1.000000}%
\pgfsetstrokecolor{currentstroke}%
\pgfsetdash{}{0pt}%
\pgfpathmoveto{\pgfqpoint{4.172047in}{1.073339in}}%
\pgfpathcurveto{\pgfqpoint{4.179180in}{1.073339in}}{\pgfqpoint{4.186022in}{1.076173in}}{\pgfqpoint{4.191065in}{1.081216in}}%
\pgfpathcurveto{\pgfqpoint{4.196109in}{1.086260in}}{\pgfqpoint{4.198943in}{1.093102in}}{\pgfqpoint{4.198943in}{1.100235in}}%
\pgfpathcurveto{\pgfqpoint{4.198943in}{1.107367in}}{\pgfqpoint{4.196109in}{1.114209in}}{\pgfqpoint{4.191065in}{1.119253in}}%
\pgfpathcurveto{\pgfqpoint{4.186022in}{1.124296in}}{\pgfqpoint{4.179180in}{1.127130in}}{\pgfqpoint{4.172047in}{1.127130in}}%
\pgfpathcurveto{\pgfqpoint{4.164914in}{1.127130in}}{\pgfqpoint{4.158073in}{1.124296in}}{\pgfqpoint{4.153029in}{1.119253in}}%
\pgfpathcurveto{\pgfqpoint{4.147985in}{1.114209in}}{\pgfqpoint{4.145151in}{1.107367in}}{\pgfqpoint{4.145151in}{1.100235in}}%
\pgfpathcurveto{\pgfqpoint{4.145151in}{1.093102in}}{\pgfqpoint{4.147985in}{1.086260in}}{\pgfqpoint{4.153029in}{1.081216in}}%
\pgfpathcurveto{\pgfqpoint{4.158073in}{1.076173in}}{\pgfqpoint{4.164914in}{1.073339in}}{\pgfqpoint{4.172047in}{1.073339in}}%
\pgfpathclose%
\pgfusepath{stroke,fill}%
\end{pgfscope}%
\begin{pgfscope}%
\pgfpathrectangle{\pgfqpoint{2.867647in}{0.500000in}}{\pgfqpoint{1.764706in}{1.700000in}}%
\pgfusepath{clip}%
\pgfsetbuttcap%
\pgfsetroundjoin%
\definecolor{currentfill}{rgb}{0.966328,0.750560,0.616961}%
\pgfsetfillcolor{currentfill}%
\pgfsetlinewidth{0.311001pt}%
\definecolor{currentstroke}{rgb}{1.000000,1.000000,1.000000}%
\pgfsetstrokecolor{currentstroke}%
\pgfsetdash{}{0pt}%
\pgfpathmoveto{\pgfqpoint{4.263398in}{1.523593in}}%
\pgfpathcurveto{\pgfqpoint{4.270531in}{1.523593in}}{\pgfqpoint{4.277372in}{1.526427in}}{\pgfqpoint{4.282416in}{1.531470in}}%
\pgfpathcurveto{\pgfqpoint{4.287460in}{1.536514in}}{\pgfqpoint{4.290294in}{1.543356in}}{\pgfqpoint{4.290294in}{1.550488in}}%
\pgfpathcurveto{\pgfqpoint{4.290294in}{1.557621in}}{\pgfqpoint{4.287460in}{1.564463in}}{\pgfqpoint{4.282416in}{1.569507in}}%
\pgfpathcurveto{\pgfqpoint{4.277372in}{1.574550in}}{\pgfqpoint{4.270531in}{1.577384in}}{\pgfqpoint{4.263398in}{1.577384in}}%
\pgfpathcurveto{\pgfqpoint{4.256265in}{1.577384in}}{\pgfqpoint{4.249423in}{1.574550in}}{\pgfqpoint{4.244380in}{1.569507in}}%
\pgfpathcurveto{\pgfqpoint{4.239336in}{1.564463in}}{\pgfqpoint{4.236502in}{1.557621in}}{\pgfqpoint{4.236502in}{1.550488in}}%
\pgfpathcurveto{\pgfqpoint{4.236502in}{1.543356in}}{\pgfqpoint{4.239336in}{1.536514in}}{\pgfqpoint{4.244380in}{1.531470in}}%
\pgfpathcurveto{\pgfqpoint{4.249423in}{1.526427in}}{\pgfqpoint{4.256265in}{1.523593in}}{\pgfqpoint{4.263398in}{1.523593in}}%
\pgfpathclose%
\pgfusepath{stroke,fill}%
\end{pgfscope}%
\begin{pgfscope}%
\pgfpathrectangle{\pgfqpoint{2.867647in}{0.500000in}}{\pgfqpoint{1.764706in}{1.700000in}}%
\pgfusepath{clip}%
\pgfsetbuttcap%
\pgfsetroundjoin%
\definecolor{currentfill}{rgb}{0.973271,0.850724,0.762998}%
\pgfsetfillcolor{currentfill}%
\pgfsetlinewidth{0.311001pt}%
\definecolor{currentstroke}{rgb}{1.000000,1.000000,1.000000}%
\pgfsetstrokecolor{currentstroke}%
\pgfsetdash{}{0pt}%
\pgfpathmoveto{\pgfqpoint{4.101472in}{1.669493in}}%
\pgfpathcurveto{\pgfqpoint{4.108604in}{1.669493in}}{\pgfqpoint{4.115446in}{1.672327in}}{\pgfqpoint{4.120490in}{1.677371in}}%
\pgfpathcurveto{\pgfqpoint{4.125533in}{1.682414in}}{\pgfqpoint{4.128367in}{1.689256in}}{\pgfqpoint{4.128367in}{1.696389in}}%
\pgfpathcurveto{\pgfqpoint{4.128367in}{1.703522in}}{\pgfqpoint{4.125533in}{1.710363in}}{\pgfqpoint{4.120490in}{1.715407in}}%
\pgfpathcurveto{\pgfqpoint{4.115446in}{1.720451in}}{\pgfqpoint{4.108604in}{1.723285in}}{\pgfqpoint{4.101472in}{1.723285in}}%
\pgfpathcurveto{\pgfqpoint{4.094339in}{1.723285in}}{\pgfqpoint{4.087497in}{1.720451in}}{\pgfqpoint{4.082453in}{1.715407in}}%
\pgfpathcurveto{\pgfqpoint{4.077410in}{1.710363in}}{\pgfqpoint{4.074576in}{1.703522in}}{\pgfqpoint{4.074576in}{1.696389in}}%
\pgfpathcurveto{\pgfqpoint{4.074576in}{1.689256in}}{\pgfqpoint{4.077410in}{1.682414in}}{\pgfqpoint{4.082453in}{1.677371in}}%
\pgfpathcurveto{\pgfqpoint{4.087497in}{1.672327in}}{\pgfqpoint{4.094339in}{1.669493in}}{\pgfqpoint{4.101472in}{1.669493in}}%
\pgfpathclose%
\pgfusepath{stroke,fill}%
\end{pgfscope}%
\begin{pgfscope}%
\pgfpathrectangle{\pgfqpoint{2.867647in}{0.500000in}}{\pgfqpoint{1.764706in}{1.700000in}}%
\pgfusepath{clip}%
\pgfsetbuttcap%
\pgfsetroundjoin%
\definecolor{currentfill}{rgb}{0.979891,0.908948,0.848279}%
\pgfsetfillcolor{currentfill}%
\pgfsetlinewidth{0.311001pt}%
\definecolor{currentstroke}{rgb}{1.000000,1.000000,1.000000}%
\pgfsetstrokecolor{currentstroke}%
\pgfsetdash{}{0pt}%
\pgfpathmoveto{\pgfqpoint{4.192150in}{1.147171in}}%
\pgfpathcurveto{\pgfqpoint{4.199283in}{1.147171in}}{\pgfqpoint{4.206125in}{1.150004in}}{\pgfqpoint{4.211169in}{1.155048in}}%
\pgfpathcurveto{\pgfqpoint{4.216212in}{1.160092in}}{\pgfqpoint{4.219046in}{1.166933in}}{\pgfqpoint{4.219046in}{1.174066in}}%
\pgfpathcurveto{\pgfqpoint{4.219046in}{1.181199in}}{\pgfqpoint{4.216212in}{1.188041in}}{\pgfqpoint{4.211169in}{1.193084in}}%
\pgfpathcurveto{\pgfqpoint{4.206125in}{1.198128in}}{\pgfqpoint{4.199283in}{1.200962in}}{\pgfqpoint{4.192150in}{1.200962in}}%
\pgfpathcurveto{\pgfqpoint{4.185018in}{1.200962in}}{\pgfqpoint{4.178176in}{1.198128in}}{\pgfqpoint{4.173132in}{1.193084in}}%
\pgfpathcurveto{\pgfqpoint{4.168089in}{1.188041in}}{\pgfqpoint{4.165255in}{1.181199in}}{\pgfqpoint{4.165255in}{1.174066in}}%
\pgfpathcurveto{\pgfqpoint{4.165255in}{1.166933in}}{\pgfqpoint{4.168089in}{1.160092in}}{\pgfqpoint{4.173132in}{1.155048in}}%
\pgfpathcurveto{\pgfqpoint{4.178176in}{1.150004in}}{\pgfqpoint{4.185018in}{1.147171in}}{\pgfqpoint{4.192150in}{1.147171in}}%
\pgfpathclose%
\pgfusepath{stroke,fill}%
\end{pgfscope}%
\begin{pgfscope}%
\pgfpathrectangle{\pgfqpoint{2.867647in}{0.500000in}}{\pgfqpoint{1.764706in}{1.700000in}}%
\pgfusepath{clip}%
\pgfsetbuttcap%
\pgfsetroundjoin%
\definecolor{currentfill}{rgb}{0.965302,0.713942,0.568499}%
\pgfsetfillcolor{currentfill}%
\pgfsetlinewidth{0.311001pt}%
\definecolor{currentstroke}{rgb}{1.000000,1.000000,1.000000}%
\pgfsetstrokecolor{currentstroke}%
\pgfsetdash{}{0pt}%
\pgfpathmoveto{\pgfqpoint{4.039843in}{0.893418in}}%
\pgfpathcurveto{\pgfqpoint{4.046975in}{0.893418in}}{\pgfqpoint{4.053817in}{0.896252in}}{\pgfqpoint{4.058861in}{0.901295in}}%
\pgfpathcurveto{\pgfqpoint{4.063904in}{0.906339in}}{\pgfqpoint{4.066738in}{0.913181in}}{\pgfqpoint{4.066738in}{0.920313in}}%
\pgfpathcurveto{\pgfqpoint{4.066738in}{0.927446in}}{\pgfqpoint{4.063904in}{0.934288in}}{\pgfqpoint{4.058861in}{0.939332in}}%
\pgfpathcurveto{\pgfqpoint{4.053817in}{0.944375in}}{\pgfqpoint{4.046975in}{0.947209in}}{\pgfqpoint{4.039843in}{0.947209in}}%
\pgfpathcurveto{\pgfqpoint{4.032710in}{0.947209in}}{\pgfqpoint{4.025868in}{0.944375in}}{\pgfqpoint{4.020824in}{0.939332in}}%
\pgfpathcurveto{\pgfqpoint{4.015781in}{0.934288in}}{\pgfqpoint{4.012947in}{0.927446in}}{\pgfqpoint{4.012947in}{0.920313in}}%
\pgfpathcurveto{\pgfqpoint{4.012947in}{0.913181in}}{\pgfqpoint{4.015781in}{0.906339in}}{\pgfqpoint{4.020824in}{0.901295in}}%
\pgfpathcurveto{\pgfqpoint{4.025868in}{0.896252in}}{\pgfqpoint{4.032710in}{0.893418in}}{\pgfqpoint{4.039843in}{0.893418in}}%
\pgfpathclose%
\pgfusepath{stroke,fill}%
\end{pgfscope}%
\begin{pgfscope}%
\pgfpathrectangle{\pgfqpoint{2.867647in}{0.500000in}}{\pgfqpoint{1.764706in}{1.700000in}}%
\pgfusepath{clip}%
\pgfsetbuttcap%
\pgfsetroundjoin%
\definecolor{currentfill}{rgb}{0.964799,0.689101,0.537560}%
\pgfsetfillcolor{currentfill}%
\pgfsetlinewidth{0.311001pt}%
\definecolor{currentstroke}{rgb}{1.000000,1.000000,1.000000}%
\pgfsetstrokecolor{currentstroke}%
\pgfsetdash{}{0pt}%
\pgfpathmoveto{\pgfqpoint{4.050362in}{1.190917in}}%
\pgfpathcurveto{\pgfqpoint{4.057495in}{1.190917in}}{\pgfqpoint{4.064337in}{1.193751in}}{\pgfqpoint{4.069381in}{1.198794in}}%
\pgfpathcurveto{\pgfqpoint{4.074424in}{1.203838in}}{\pgfqpoint{4.077258in}{1.210680in}}{\pgfqpoint{4.077258in}{1.217813in}}%
\pgfpathcurveto{\pgfqpoint{4.077258in}{1.224945in}}{\pgfqpoint{4.074424in}{1.231787in}}{\pgfqpoint{4.069381in}{1.236831in}}%
\pgfpathcurveto{\pgfqpoint{4.064337in}{1.241874in}}{\pgfqpoint{4.057495in}{1.244708in}}{\pgfqpoint{4.050362in}{1.244708in}}%
\pgfpathcurveto{\pgfqpoint{4.043230in}{1.244708in}}{\pgfqpoint{4.036388in}{1.241874in}}{\pgfqpoint{4.031344in}{1.236831in}}%
\pgfpathcurveto{\pgfqpoint{4.026301in}{1.231787in}}{\pgfqpoint{4.023467in}{1.224945in}}{\pgfqpoint{4.023467in}{1.217813in}}%
\pgfpathcurveto{\pgfqpoint{4.023467in}{1.210680in}}{\pgfqpoint{4.026301in}{1.203838in}}{\pgfqpoint{4.031344in}{1.198794in}}%
\pgfpathcurveto{\pgfqpoint{4.036388in}{1.193751in}}{\pgfqpoint{4.043230in}{1.190917in}}{\pgfqpoint{4.050362in}{1.190917in}}%
\pgfpathclose%
\pgfusepath{stroke,fill}%
\end{pgfscope}%
\begin{pgfscope}%
\pgfpathrectangle{\pgfqpoint{2.867647in}{0.500000in}}{\pgfqpoint{1.764706in}{1.700000in}}%
\pgfusepath{clip}%
\pgfsetbuttcap%
\pgfsetroundjoin%
\definecolor{currentfill}{rgb}{0.971694,0.833208,0.737161}%
\pgfsetfillcolor{currentfill}%
\pgfsetlinewidth{0.311001pt}%
\definecolor{currentstroke}{rgb}{1.000000,1.000000,1.000000}%
\pgfsetstrokecolor{currentstroke}%
\pgfsetdash{}{0pt}%
\pgfpathmoveto{\pgfqpoint{4.092600in}{1.176473in}}%
\pgfpathcurveto{\pgfqpoint{4.099732in}{1.176473in}}{\pgfqpoint{4.106574in}{1.179307in}}{\pgfqpoint{4.111618in}{1.184350in}}%
\pgfpathcurveto{\pgfqpoint{4.116661in}{1.189394in}}{\pgfqpoint{4.119495in}{1.196236in}}{\pgfqpoint{4.119495in}{1.203369in}}%
\pgfpathcurveto{\pgfqpoint{4.119495in}{1.210501in}}{\pgfqpoint{4.116661in}{1.217343in}}{\pgfqpoint{4.111618in}{1.222387in}}%
\pgfpathcurveto{\pgfqpoint{4.106574in}{1.227430in}}{\pgfqpoint{4.099732in}{1.230264in}}{\pgfqpoint{4.092600in}{1.230264in}}%
\pgfpathcurveto{\pgfqpoint{4.085467in}{1.230264in}}{\pgfqpoint{4.078625in}{1.227430in}}{\pgfqpoint{4.073581in}{1.222387in}}%
\pgfpathcurveto{\pgfqpoint{4.068538in}{1.217343in}}{\pgfqpoint{4.065704in}{1.210501in}}{\pgfqpoint{4.065704in}{1.203369in}}%
\pgfpathcurveto{\pgfqpoint{4.065704in}{1.196236in}}{\pgfqpoint{4.068538in}{1.189394in}}{\pgfqpoint{4.073581in}{1.184350in}}%
\pgfpathcurveto{\pgfqpoint{4.078625in}{1.179307in}}{\pgfqpoint{4.085467in}{1.176473in}}{\pgfqpoint{4.092600in}{1.176473in}}%
\pgfpathclose%
\pgfusepath{stroke,fill}%
\end{pgfscope}%
\begin{pgfscope}%
\pgfpathrectangle{\pgfqpoint{2.867647in}{0.500000in}}{\pgfqpoint{1.764706in}{1.700000in}}%
\pgfusepath{clip}%
\pgfsetbuttcap%
\pgfsetroundjoin%
\definecolor{currentfill}{rgb}{0.964558,0.676556,0.522514}%
\pgfsetfillcolor{currentfill}%
\pgfsetlinewidth{0.311001pt}%
\definecolor{currentstroke}{rgb}{1.000000,1.000000,1.000000}%
\pgfsetstrokecolor{currentstroke}%
\pgfsetdash{}{0pt}%
\pgfpathmoveto{\pgfqpoint{4.246098in}{1.022961in}}%
\pgfpathcurveto{\pgfqpoint{4.253231in}{1.022961in}}{\pgfqpoint{4.260072in}{1.025795in}}{\pgfqpoint{4.265116in}{1.030838in}}%
\pgfpathcurveto{\pgfqpoint{4.270160in}{1.035882in}}{\pgfqpoint{4.272994in}{1.042724in}}{\pgfqpoint{4.272994in}{1.049857in}}%
\pgfpathcurveto{\pgfqpoint{4.272994in}{1.056989in}}{\pgfqpoint{4.270160in}{1.063831in}}{\pgfqpoint{4.265116in}{1.068875in}}%
\pgfpathcurveto{\pgfqpoint{4.260072in}{1.073918in}}{\pgfqpoint{4.253231in}{1.076752in}}{\pgfqpoint{4.246098in}{1.076752in}}%
\pgfpathcurveto{\pgfqpoint{4.238965in}{1.076752in}}{\pgfqpoint{4.232123in}{1.073918in}}{\pgfqpoint{4.227080in}{1.068875in}}%
\pgfpathcurveto{\pgfqpoint{4.222036in}{1.063831in}}{\pgfqpoint{4.219202in}{1.056989in}}{\pgfqpoint{4.219202in}{1.049857in}}%
\pgfpathcurveto{\pgfqpoint{4.219202in}{1.042724in}}{\pgfqpoint{4.222036in}{1.035882in}}{\pgfqpoint{4.227080in}{1.030838in}}%
\pgfpathcurveto{\pgfqpoint{4.232123in}{1.025795in}}{\pgfqpoint{4.238965in}{1.022961in}}{\pgfqpoint{4.246098in}{1.022961in}}%
\pgfpathclose%
\pgfusepath{stroke,fill}%
\end{pgfscope}%
\begin{pgfscope}%
\pgfpathrectangle{\pgfqpoint{2.867647in}{0.500000in}}{\pgfqpoint{1.764706in}{1.700000in}}%
\pgfusepath{clip}%
\pgfsetbuttcap%
\pgfsetroundjoin%
\definecolor{currentfill}{rgb}{0.981377,0.920617,0.865369}%
\pgfsetfillcolor{currentfill}%
\pgfsetlinewidth{0.311001pt}%
\definecolor{currentstroke}{rgb}{1.000000,1.000000,1.000000}%
\pgfsetstrokecolor{currentstroke}%
\pgfsetdash{}{0pt}%
\pgfpathmoveto{\pgfqpoint{4.196002in}{1.296957in}}%
\pgfpathcurveto{\pgfqpoint{4.203135in}{1.296957in}}{\pgfqpoint{4.209977in}{1.299791in}}{\pgfqpoint{4.215020in}{1.304835in}}%
\pgfpathcurveto{\pgfqpoint{4.220064in}{1.309878in}}{\pgfqpoint{4.222898in}{1.316720in}}{\pgfqpoint{4.222898in}{1.323853in}}%
\pgfpathcurveto{\pgfqpoint{4.222898in}{1.330986in}}{\pgfqpoint{4.220064in}{1.337827in}}{\pgfqpoint{4.215020in}{1.342871in}}%
\pgfpathcurveto{\pgfqpoint{4.209977in}{1.347915in}}{\pgfqpoint{4.203135in}{1.350749in}}{\pgfqpoint{4.196002in}{1.350749in}}%
\pgfpathcurveto{\pgfqpoint{4.188869in}{1.350749in}}{\pgfqpoint{4.182028in}{1.347915in}}{\pgfqpoint{4.176984in}{1.342871in}}%
\pgfpathcurveto{\pgfqpoint{4.171940in}{1.337827in}}{\pgfqpoint{4.169106in}{1.330986in}}{\pgfqpoint{4.169106in}{1.323853in}}%
\pgfpathcurveto{\pgfqpoint{4.169106in}{1.316720in}}{\pgfqpoint{4.171940in}{1.309878in}}{\pgfqpoint{4.176984in}{1.304835in}}%
\pgfpathcurveto{\pgfqpoint{4.182028in}{1.299791in}}{\pgfqpoint{4.188869in}{1.296957in}}{\pgfqpoint{4.196002in}{1.296957in}}%
\pgfpathclose%
\pgfusepath{stroke,fill}%
\end{pgfscope}%
\begin{pgfscope}%
\pgfpathrectangle{\pgfqpoint{2.867647in}{0.500000in}}{\pgfqpoint{1.764706in}{1.700000in}}%
\pgfusepath{clip}%
\pgfsetbuttcap%
\pgfsetroundjoin%
\definecolor{currentfill}{rgb}{0.967735,0.780441,0.659127}%
\pgfsetfillcolor{currentfill}%
\pgfsetlinewidth{0.311001pt}%
\definecolor{currentstroke}{rgb}{1.000000,1.000000,1.000000}%
\pgfsetstrokecolor{currentstroke}%
\pgfsetdash{}{0pt}%
\pgfpathmoveto{\pgfqpoint{4.255853in}{1.118592in}}%
\pgfpathcurveto{\pgfqpoint{4.262986in}{1.118592in}}{\pgfqpoint{4.269827in}{1.121426in}}{\pgfqpoint{4.274871in}{1.126469in}}%
\pgfpathcurveto{\pgfqpoint{4.279915in}{1.131513in}}{\pgfqpoint{4.282749in}{1.138355in}}{\pgfqpoint{4.282749in}{1.145488in}}%
\pgfpathcurveto{\pgfqpoint{4.282749in}{1.152620in}}{\pgfqpoint{4.279915in}{1.159462in}}{\pgfqpoint{4.274871in}{1.164506in}}%
\pgfpathcurveto{\pgfqpoint{4.269827in}{1.169549in}}{\pgfqpoint{4.262986in}{1.172383in}}{\pgfqpoint{4.255853in}{1.172383in}}%
\pgfpathcurveto{\pgfqpoint{4.248720in}{1.172383in}}{\pgfqpoint{4.241879in}{1.169549in}}{\pgfqpoint{4.236835in}{1.164506in}}%
\pgfpathcurveto{\pgfqpoint{4.231791in}{1.159462in}}{\pgfqpoint{4.228957in}{1.152620in}}{\pgfqpoint{4.228957in}{1.145488in}}%
\pgfpathcurveto{\pgfqpoint{4.228957in}{1.138355in}}{\pgfqpoint{4.231791in}{1.131513in}}{\pgfqpoint{4.236835in}{1.126469in}}%
\pgfpathcurveto{\pgfqpoint{4.241879in}{1.121426in}}{\pgfqpoint{4.248720in}{1.118592in}}{\pgfqpoint{4.255853in}{1.118592in}}%
\pgfpathclose%
\pgfusepath{stroke,fill}%
\end{pgfscope}%
\begin{pgfscope}%
\pgfpathrectangle{\pgfqpoint{2.867647in}{0.500000in}}{\pgfqpoint{1.764706in}{1.700000in}}%
\pgfusepath{clip}%
\pgfsetbuttcap%
\pgfsetroundjoin%
\definecolor{currentfill}{rgb}{0.977657,0.891500,0.822809}%
\pgfsetfillcolor{currentfill}%
\pgfsetlinewidth{0.311001pt}%
\definecolor{currentstroke}{rgb}{1.000000,1.000000,1.000000}%
\pgfsetstrokecolor{currentstroke}%
\pgfsetdash{}{0pt}%
\pgfpathmoveto{\pgfqpoint{4.129462in}{1.052563in}}%
\pgfpathcurveto{\pgfqpoint{4.136594in}{1.052563in}}{\pgfqpoint{4.143436in}{1.055397in}}{\pgfqpoint{4.148480in}{1.060441in}}%
\pgfpathcurveto{\pgfqpoint{4.153523in}{1.065484in}}{\pgfqpoint{4.156357in}{1.072326in}}{\pgfqpoint{4.156357in}{1.079459in}}%
\pgfpathcurveto{\pgfqpoint{4.156357in}{1.086591in}}{\pgfqpoint{4.153523in}{1.093433in}}{\pgfqpoint{4.148480in}{1.098477in}}%
\pgfpathcurveto{\pgfqpoint{4.143436in}{1.103520in}}{\pgfqpoint{4.136594in}{1.106354in}}{\pgfqpoint{4.129462in}{1.106354in}}%
\pgfpathcurveto{\pgfqpoint{4.122329in}{1.106354in}}{\pgfqpoint{4.115487in}{1.103520in}}{\pgfqpoint{4.110443in}{1.098477in}}%
\pgfpathcurveto{\pgfqpoint{4.105400in}{1.093433in}}{\pgfqpoint{4.102566in}{1.086591in}}{\pgfqpoint{4.102566in}{1.079459in}}%
\pgfpathcurveto{\pgfqpoint{4.102566in}{1.072326in}}{\pgfqpoint{4.105400in}{1.065484in}}{\pgfqpoint{4.110443in}{1.060441in}}%
\pgfpathcurveto{\pgfqpoint{4.115487in}{1.055397in}}{\pgfqpoint{4.122329in}{1.052563in}}{\pgfqpoint{4.129462in}{1.052563in}}%
\pgfpathclose%
\pgfusepath{stroke,fill}%
\end{pgfscope}%
\begin{pgfscope}%
\pgfpathrectangle{\pgfqpoint{2.867647in}{0.500000in}}{\pgfqpoint{1.764706in}{1.700000in}}%
\pgfusepath{clip}%
\pgfsetbuttcap%
\pgfsetroundjoin%
\definecolor{currentfill}{rgb}{0.950851,0.435000,0.297228}%
\pgfsetfillcolor{currentfill}%
\pgfsetlinewidth{0.311001pt}%
\definecolor{currentstroke}{rgb}{1.000000,1.000000,1.000000}%
\pgfsetstrokecolor{currentstroke}%
\pgfsetdash{}{0pt}%
\pgfpathmoveto{\pgfqpoint{4.320728in}{1.559062in}}%
\pgfpathcurveto{\pgfqpoint{4.327861in}{1.559062in}}{\pgfqpoint{4.334703in}{1.561896in}}{\pgfqpoint{4.339746in}{1.566940in}}%
\pgfpathcurveto{\pgfqpoint{4.344790in}{1.571984in}}{\pgfqpoint{4.347624in}{1.578825in}}{\pgfqpoint{4.347624in}{1.585958in}}%
\pgfpathcurveto{\pgfqpoint{4.347624in}{1.593091in}}{\pgfqpoint{4.344790in}{1.599932in}}{\pgfqpoint{4.339746in}{1.604976in}}%
\pgfpathcurveto{\pgfqpoint{4.334703in}{1.610020in}}{\pgfqpoint{4.327861in}{1.612854in}}{\pgfqpoint{4.320728in}{1.612854in}}%
\pgfpathcurveto{\pgfqpoint{4.313595in}{1.612854in}}{\pgfqpoint{4.306754in}{1.610020in}}{\pgfqpoint{4.301710in}{1.604976in}}%
\pgfpathcurveto{\pgfqpoint{4.296666in}{1.599932in}}{\pgfqpoint{4.293832in}{1.593091in}}{\pgfqpoint{4.293832in}{1.585958in}}%
\pgfpathcurveto{\pgfqpoint{4.293832in}{1.578825in}}{\pgfqpoint{4.296666in}{1.571984in}}{\pgfqpoint{4.301710in}{1.566940in}}%
\pgfpathcurveto{\pgfqpoint{4.306754in}{1.561896in}}{\pgfqpoint{4.313595in}{1.559062in}}{\pgfqpoint{4.320728in}{1.559062in}}%
\pgfpathclose%
\pgfusepath{stroke,fill}%
\end{pgfscope}%
\begin{pgfscope}%
\pgfpathrectangle{\pgfqpoint{2.867647in}{0.500000in}}{\pgfqpoint{1.764706in}{1.700000in}}%
\pgfusepath{clip}%
\pgfsetbuttcap%
\pgfsetroundjoin%
\definecolor{currentfill}{rgb}{0.970718,0.821518,0.719872}%
\pgfsetfillcolor{currentfill}%
\pgfsetlinewidth{0.311001pt}%
\definecolor{currentstroke}{rgb}{1.000000,1.000000,1.000000}%
\pgfsetstrokecolor{currentstroke}%
\pgfsetdash{}{0pt}%
\pgfpathmoveto{\pgfqpoint{4.085809in}{1.693097in}}%
\pgfpathcurveto{\pgfqpoint{4.092942in}{1.693097in}}{\pgfqpoint{4.099783in}{1.695931in}}{\pgfqpoint{4.104827in}{1.700975in}}%
\pgfpathcurveto{\pgfqpoint{4.109871in}{1.706019in}}{\pgfqpoint{4.112705in}{1.712860in}}{\pgfqpoint{4.112705in}{1.719993in}}%
\pgfpathcurveto{\pgfqpoint{4.112705in}{1.727126in}}{\pgfqpoint{4.109871in}{1.733968in}}{\pgfqpoint{4.104827in}{1.739011in}}%
\pgfpathcurveto{\pgfqpoint{4.099783in}{1.744055in}}{\pgfqpoint{4.092942in}{1.746889in}}{\pgfqpoint{4.085809in}{1.746889in}}%
\pgfpathcurveto{\pgfqpoint{4.078676in}{1.746889in}}{\pgfqpoint{4.071834in}{1.744055in}}{\pgfqpoint{4.066791in}{1.739011in}}%
\pgfpathcurveto{\pgfqpoint{4.061747in}{1.733968in}}{\pgfqpoint{4.058913in}{1.727126in}}{\pgfqpoint{4.058913in}{1.719993in}}%
\pgfpathcurveto{\pgfqpoint{4.058913in}{1.712860in}}{\pgfqpoint{4.061747in}{1.706019in}}{\pgfqpoint{4.066791in}{1.700975in}}%
\pgfpathcurveto{\pgfqpoint{4.071834in}{1.695931in}}{\pgfqpoint{4.078676in}{1.693097in}}{\pgfqpoint{4.085809in}{1.693097in}}%
\pgfpathclose%
\pgfusepath{stroke,fill}%
\end{pgfscope}%
\begin{pgfscope}%
\pgfpathrectangle{\pgfqpoint{2.867647in}{0.500000in}}{\pgfqpoint{1.764706in}{1.700000in}}%
\pgfusepath{clip}%
\pgfsetbuttcap%
\pgfsetroundjoin%
\definecolor{currentfill}{rgb}{0.977657,0.891500,0.822809}%
\pgfsetfillcolor{currentfill}%
\pgfsetlinewidth{0.311001pt}%
\definecolor{currentstroke}{rgb}{1.000000,1.000000,1.000000}%
\pgfsetstrokecolor{currentstroke}%
\pgfsetdash{}{0pt}%
\pgfpathmoveto{\pgfqpoint{4.148922in}{1.351852in}}%
\pgfpathcurveto{\pgfqpoint{4.156055in}{1.351852in}}{\pgfqpoint{4.162897in}{1.354685in}}{\pgfqpoint{4.167940in}{1.359729in}}%
\pgfpathcurveto{\pgfqpoint{4.172984in}{1.364773in}}{\pgfqpoint{4.175818in}{1.371614in}}{\pgfqpoint{4.175818in}{1.378747in}}%
\pgfpathcurveto{\pgfqpoint{4.175818in}{1.385880in}}{\pgfqpoint{4.172984in}{1.392722in}}{\pgfqpoint{4.167940in}{1.397765in}}%
\pgfpathcurveto{\pgfqpoint{4.162897in}{1.402809in}}{\pgfqpoint{4.156055in}{1.405643in}}{\pgfqpoint{4.148922in}{1.405643in}}%
\pgfpathcurveto{\pgfqpoint{4.141789in}{1.405643in}}{\pgfqpoint{4.134948in}{1.402809in}}{\pgfqpoint{4.129904in}{1.397765in}}%
\pgfpathcurveto{\pgfqpoint{4.124860in}{1.392722in}}{\pgfqpoint{4.122027in}{1.385880in}}{\pgfqpoint{4.122027in}{1.378747in}}%
\pgfpathcurveto{\pgfqpoint{4.122027in}{1.371614in}}{\pgfqpoint{4.124860in}{1.364773in}}{\pgfqpoint{4.129904in}{1.359729in}}%
\pgfpathcurveto{\pgfqpoint{4.134948in}{1.354685in}}{\pgfqpoint{4.141789in}{1.351852in}}{\pgfqpoint{4.148922in}{1.351852in}}%
\pgfpathclose%
\pgfusepath{stroke,fill}%
\end{pgfscope}%
\begin{pgfscope}%
\pgfpathrectangle{\pgfqpoint{2.867647in}{0.500000in}}{\pgfqpoint{1.764706in}{1.700000in}}%
\pgfusepath{clip}%
\pgfsetbuttcap%
\pgfsetroundjoin%
\definecolor{currentfill}{rgb}{0.961734,0.579886,0.418445}%
\pgfsetfillcolor{currentfill}%
\pgfsetlinewidth{0.311001pt}%
\definecolor{currentstroke}{rgb}{1.000000,1.000000,1.000000}%
\pgfsetstrokecolor{currentstroke}%
\pgfsetdash{}{0pt}%
\pgfpathmoveto{\pgfqpoint{3.946470in}{0.982317in}}%
\pgfpathcurveto{\pgfqpoint{3.953602in}{0.982317in}}{\pgfqpoint{3.960444in}{0.985151in}}{\pgfqpoint{3.965488in}{0.990194in}}%
\pgfpathcurveto{\pgfqpoint{3.970531in}{0.995238in}}{\pgfqpoint{3.973365in}{1.002080in}}{\pgfqpoint{3.973365in}{1.009212in}}%
\pgfpathcurveto{\pgfqpoint{3.973365in}{1.016345in}}{\pgfqpoint{3.970531in}{1.023187in}}{\pgfqpoint{3.965488in}{1.028230in}}%
\pgfpathcurveto{\pgfqpoint{3.960444in}{1.033274in}}{\pgfqpoint{3.953602in}{1.036108in}}{\pgfqpoint{3.946470in}{1.036108in}}%
\pgfpathcurveto{\pgfqpoint{3.939337in}{1.036108in}}{\pgfqpoint{3.932495in}{1.033274in}}{\pgfqpoint{3.927451in}{1.028230in}}%
\pgfpathcurveto{\pgfqpoint{3.922408in}{1.023187in}}{\pgfqpoint{3.919574in}{1.016345in}}{\pgfqpoint{3.919574in}{1.009212in}}%
\pgfpathcurveto{\pgfqpoint{3.919574in}{1.002080in}}{\pgfqpoint{3.922408in}{0.995238in}}{\pgfqpoint{3.927451in}{0.990194in}}%
\pgfpathcurveto{\pgfqpoint{3.932495in}{0.985151in}}{\pgfqpoint{3.939337in}{0.982317in}}{\pgfqpoint{3.946470in}{0.982317in}}%
\pgfpathclose%
\pgfusepath{stroke,fill}%
\end{pgfscope}%
\begin{pgfscope}%
\pgfpathrectangle{\pgfqpoint{2.867647in}{0.500000in}}{\pgfqpoint{1.764706in}{1.700000in}}%
\pgfusepath{clip}%
\pgfsetbuttcap%
\pgfsetroundjoin%
\definecolor{currentfill}{rgb}{0.964032,0.651225,0.493258}%
\pgfsetfillcolor{currentfill}%
\pgfsetlinewidth{0.311001pt}%
\definecolor{currentstroke}{rgb}{1.000000,1.000000,1.000000}%
\pgfsetstrokecolor{currentstroke}%
\pgfsetdash{}{0pt}%
\pgfpathmoveto{\pgfqpoint{3.984839in}{1.034979in}}%
\pgfpathcurveto{\pgfqpoint{3.991972in}{1.034979in}}{\pgfqpoint{3.998813in}{1.037813in}}{\pgfqpoint{4.003857in}{1.042856in}}%
\pgfpathcurveto{\pgfqpoint{4.008901in}{1.047900in}}{\pgfqpoint{4.011735in}{1.054742in}}{\pgfqpoint{4.011735in}{1.061874in}}%
\pgfpathcurveto{\pgfqpoint{4.011735in}{1.069007in}}{\pgfqpoint{4.008901in}{1.075849in}}{\pgfqpoint{4.003857in}{1.080892in}}%
\pgfpathcurveto{\pgfqpoint{3.998813in}{1.085936in}}{\pgfqpoint{3.991972in}{1.088770in}}{\pgfqpoint{3.984839in}{1.088770in}}%
\pgfpathcurveto{\pgfqpoint{3.977706in}{1.088770in}}{\pgfqpoint{3.970864in}{1.085936in}}{\pgfqpoint{3.965821in}{1.080892in}}%
\pgfpathcurveto{\pgfqpoint{3.960777in}{1.075849in}}{\pgfqpoint{3.957943in}{1.069007in}}{\pgfqpoint{3.957943in}{1.061874in}}%
\pgfpathcurveto{\pgfqpoint{3.957943in}{1.054742in}}{\pgfqpoint{3.960777in}{1.047900in}}{\pgfqpoint{3.965821in}{1.042856in}}%
\pgfpathcurveto{\pgfqpoint{3.970864in}{1.037813in}}{\pgfqpoint{3.977706in}{1.034979in}}{\pgfqpoint{3.984839in}{1.034979in}}%
\pgfpathclose%
\pgfusepath{stroke,fill}%
\end{pgfscope}%
\begin{pgfscope}%
\pgfpathrectangle{\pgfqpoint{2.867647in}{0.500000in}}{\pgfqpoint{1.764706in}{1.700000in}}%
\pgfusepath{clip}%
\pgfsetbuttcap%
\pgfsetroundjoin%
\definecolor{currentfill}{rgb}{0.964306,0.663930,0.507747}%
\pgfsetfillcolor{currentfill}%
\pgfsetlinewidth{0.311001pt}%
\definecolor{currentstroke}{rgb}{1.000000,1.000000,1.000000}%
\pgfsetstrokecolor{currentstroke}%
\pgfsetdash{}{0pt}%
\pgfpathmoveto{\pgfqpoint{4.053651in}{1.224396in}}%
\pgfpathcurveto{\pgfqpoint{4.060784in}{1.224396in}}{\pgfqpoint{4.067626in}{1.227230in}}{\pgfqpoint{4.072670in}{1.232274in}}%
\pgfpathcurveto{\pgfqpoint{4.077713in}{1.237318in}}{\pgfqpoint{4.080547in}{1.244159in}}{\pgfqpoint{4.080547in}{1.251292in}}%
\pgfpathcurveto{\pgfqpoint{4.080547in}{1.258425in}}{\pgfqpoint{4.077713in}{1.265267in}}{\pgfqpoint{4.072670in}{1.270310in}}%
\pgfpathcurveto{\pgfqpoint{4.067626in}{1.275354in}}{\pgfqpoint{4.060784in}{1.278188in}}{\pgfqpoint{4.053651in}{1.278188in}}%
\pgfpathcurveto{\pgfqpoint{4.046519in}{1.278188in}}{\pgfqpoint{4.039677in}{1.275354in}}{\pgfqpoint{4.034633in}{1.270310in}}%
\pgfpathcurveto{\pgfqpoint{4.029590in}{1.265267in}}{\pgfqpoint{4.026756in}{1.258425in}}{\pgfqpoint{4.026756in}{1.251292in}}%
\pgfpathcurveto{\pgfqpoint{4.026756in}{1.244159in}}{\pgfqpoint{4.029590in}{1.237318in}}{\pgfqpoint{4.034633in}{1.232274in}}%
\pgfpathcurveto{\pgfqpoint{4.039677in}{1.227230in}}{\pgfqpoint{4.046519in}{1.224396in}}{\pgfqpoint{4.053651in}{1.224396in}}%
\pgfpathclose%
\pgfusepath{stroke,fill}%
\end{pgfscope}%
\begin{pgfscope}%
\pgfpathrectangle{\pgfqpoint{2.867647in}{0.500000in}}{\pgfqpoint{1.764706in}{1.700000in}}%
\pgfusepath{clip}%
\pgfsetbuttcap%
\pgfsetroundjoin%
\definecolor{currentfill}{rgb}{0.968509,0.792226,0.676405}%
\pgfsetfillcolor{currentfill}%
\pgfsetlinewidth{0.311001pt}%
\definecolor{currentstroke}{rgb}{1.000000,1.000000,1.000000}%
\pgfsetstrokecolor{currentstroke}%
\pgfsetdash{}{0pt}%
\pgfpathmoveto{\pgfqpoint{4.050791in}{1.089248in}}%
\pgfpathcurveto{\pgfqpoint{4.057924in}{1.089248in}}{\pgfqpoint{4.064765in}{1.092081in}}{\pgfqpoint{4.069809in}{1.097125in}}%
\pgfpathcurveto{\pgfqpoint{4.074853in}{1.102169in}}{\pgfqpoint{4.077686in}{1.109010in}}{\pgfqpoint{4.077686in}{1.116143in}}%
\pgfpathcurveto{\pgfqpoint{4.077686in}{1.123276in}}{\pgfqpoint{4.074853in}{1.130118in}}{\pgfqpoint{4.069809in}{1.135161in}}%
\pgfpathcurveto{\pgfqpoint{4.064765in}{1.140205in}}{\pgfqpoint{4.057924in}{1.143039in}}{\pgfqpoint{4.050791in}{1.143039in}}%
\pgfpathcurveto{\pgfqpoint{4.043658in}{1.143039in}}{\pgfqpoint{4.036816in}{1.140205in}}{\pgfqpoint{4.031773in}{1.135161in}}%
\pgfpathcurveto{\pgfqpoint{4.026729in}{1.130118in}}{\pgfqpoint{4.023895in}{1.123276in}}{\pgfqpoint{4.023895in}{1.116143in}}%
\pgfpathcurveto{\pgfqpoint{4.023895in}{1.109010in}}{\pgfqpoint{4.026729in}{1.102169in}}{\pgfqpoint{4.031773in}{1.097125in}}%
\pgfpathcurveto{\pgfqpoint{4.036816in}{1.092081in}}{\pgfqpoint{4.043658in}{1.089248in}}{\pgfqpoint{4.050791in}{1.089248in}}%
\pgfpathclose%
\pgfusepath{stroke,fill}%
\end{pgfscope}%
\begin{pgfscope}%
\pgfpathrectangle{\pgfqpoint{2.867647in}{0.500000in}}{\pgfqpoint{1.764706in}{1.700000in}}%
\pgfusepath{clip}%
\pgfsetbuttcap%
\pgfsetroundjoin%
\definecolor{currentfill}{rgb}{0.978376,0.897317,0.831308}%
\pgfsetfillcolor{currentfill}%
\pgfsetlinewidth{0.311001pt}%
\definecolor{currentstroke}{rgb}{1.000000,1.000000,1.000000}%
\pgfsetstrokecolor{currentstroke}%
\pgfsetdash{}{0pt}%
\pgfpathmoveto{\pgfqpoint{4.199832in}{1.498406in}}%
\pgfpathcurveto{\pgfqpoint{4.206965in}{1.498406in}}{\pgfqpoint{4.213806in}{1.501240in}}{\pgfqpoint{4.218850in}{1.506283in}}%
\pgfpathcurveto{\pgfqpoint{4.223894in}{1.511327in}}{\pgfqpoint{4.226727in}{1.518169in}}{\pgfqpoint{4.226727in}{1.525301in}}%
\pgfpathcurveto{\pgfqpoint{4.226727in}{1.532434in}}{\pgfqpoint{4.223894in}{1.539276in}}{\pgfqpoint{4.218850in}{1.544320in}}%
\pgfpathcurveto{\pgfqpoint{4.213806in}{1.549363in}}{\pgfqpoint{4.206965in}{1.552197in}}{\pgfqpoint{4.199832in}{1.552197in}}%
\pgfpathcurveto{\pgfqpoint{4.192699in}{1.552197in}}{\pgfqpoint{4.185857in}{1.549363in}}{\pgfqpoint{4.180814in}{1.544320in}}%
\pgfpathcurveto{\pgfqpoint{4.175770in}{1.539276in}}{\pgfqpoint{4.172936in}{1.532434in}}{\pgfqpoint{4.172936in}{1.525301in}}%
\pgfpathcurveto{\pgfqpoint{4.172936in}{1.518169in}}{\pgfqpoint{4.175770in}{1.511327in}}{\pgfqpoint{4.180814in}{1.506283in}}%
\pgfpathcurveto{\pgfqpoint{4.185857in}{1.501240in}}{\pgfqpoint{4.192699in}{1.498406in}}{\pgfqpoint{4.199832in}{1.498406in}}%
\pgfpathclose%
\pgfusepath{stroke,fill}%
\end{pgfscope}%
\begin{pgfscope}%
\pgfpathrectangle{\pgfqpoint{2.867647in}{0.500000in}}{\pgfqpoint{1.764706in}{1.700000in}}%
\pgfusepath{clip}%
\pgfsetbuttcap%
\pgfsetroundjoin%
\definecolor{currentfill}{rgb}{0.976287,0.879862,0.805788}%
\pgfsetfillcolor{currentfill}%
\pgfsetlinewidth{0.311001pt}%
\definecolor{currentstroke}{rgb}{1.000000,1.000000,1.000000}%
\pgfsetstrokecolor{currentstroke}%
\pgfsetdash{}{0pt}%
\pgfpathmoveto{\pgfqpoint{4.244560in}{1.312865in}}%
\pgfpathcurveto{\pgfqpoint{4.251692in}{1.312865in}}{\pgfqpoint{4.258534in}{1.315699in}}{\pgfqpoint{4.263578in}{1.320742in}}%
\pgfpathcurveto{\pgfqpoint{4.268621in}{1.325786in}}{\pgfqpoint{4.271455in}{1.332628in}}{\pgfqpoint{4.271455in}{1.339760in}}%
\pgfpathcurveto{\pgfqpoint{4.271455in}{1.346893in}}{\pgfqpoint{4.268621in}{1.353735in}}{\pgfqpoint{4.263578in}{1.358778in}}%
\pgfpathcurveto{\pgfqpoint{4.258534in}{1.363822in}}{\pgfqpoint{4.251692in}{1.366656in}}{\pgfqpoint{4.244560in}{1.366656in}}%
\pgfpathcurveto{\pgfqpoint{4.237427in}{1.366656in}}{\pgfqpoint{4.230585in}{1.363822in}}{\pgfqpoint{4.225541in}{1.358778in}}%
\pgfpathcurveto{\pgfqpoint{4.220498in}{1.353735in}}{\pgfqpoint{4.217664in}{1.346893in}}{\pgfqpoint{4.217664in}{1.339760in}}%
\pgfpathcurveto{\pgfqpoint{4.217664in}{1.332628in}}{\pgfqpoint{4.220498in}{1.325786in}}{\pgfqpoint{4.225541in}{1.320742in}}%
\pgfpathcurveto{\pgfqpoint{4.230585in}{1.315699in}}{\pgfqpoint{4.237427in}{1.312865in}}{\pgfqpoint{4.244560in}{1.312865in}}%
\pgfpathclose%
\pgfusepath{stroke,fill}%
\end{pgfscope}%
\begin{pgfscope}%
\pgfpathrectangle{\pgfqpoint{2.867647in}{0.500000in}}{\pgfqpoint{1.764706in}{1.700000in}}%
\pgfusepath{clip}%
\pgfsetbuttcap%
\pgfsetroundjoin%
\definecolor{currentfill}{rgb}{0.968105,0.786346,0.667739}%
\pgfsetfillcolor{currentfill}%
\pgfsetlinewidth{0.311001pt}%
\definecolor{currentstroke}{rgb}{1.000000,1.000000,1.000000}%
\pgfsetstrokecolor{currentstroke}%
\pgfsetdash{}{0pt}%
\pgfpathmoveto{\pgfqpoint{4.284889in}{1.332265in}}%
\pgfpathcurveto{\pgfqpoint{4.292022in}{1.332265in}}{\pgfqpoint{4.298864in}{1.335099in}}{\pgfqpoint{4.303907in}{1.340143in}}%
\pgfpathcurveto{\pgfqpoint{4.308951in}{1.345186in}}{\pgfqpoint{4.311785in}{1.352028in}}{\pgfqpoint{4.311785in}{1.359161in}}%
\pgfpathcurveto{\pgfqpoint{4.311785in}{1.366294in}}{\pgfqpoint{4.308951in}{1.373135in}}{\pgfqpoint{4.303907in}{1.378179in}}%
\pgfpathcurveto{\pgfqpoint{4.298864in}{1.383223in}}{\pgfqpoint{4.292022in}{1.386057in}}{\pgfqpoint{4.284889in}{1.386057in}}%
\pgfpathcurveto{\pgfqpoint{4.277756in}{1.386057in}}{\pgfqpoint{4.270915in}{1.383223in}}{\pgfqpoint{4.265871in}{1.378179in}}%
\pgfpathcurveto{\pgfqpoint{4.260827in}{1.373135in}}{\pgfqpoint{4.257994in}{1.366294in}}{\pgfqpoint{4.257994in}{1.359161in}}%
\pgfpathcurveto{\pgfqpoint{4.257994in}{1.352028in}}{\pgfqpoint{4.260827in}{1.345186in}}{\pgfqpoint{4.265871in}{1.340143in}}%
\pgfpathcurveto{\pgfqpoint{4.270915in}{1.335099in}}{\pgfqpoint{4.277756in}{1.332265in}}{\pgfqpoint{4.284889in}{1.332265in}}%
\pgfpathclose%
\pgfusepath{stroke,fill}%
\end{pgfscope}%
\begin{pgfscope}%
\pgfpathrectangle{\pgfqpoint{2.867647in}{0.500000in}}{\pgfqpoint{1.764706in}{1.700000in}}%
\pgfusepath{clip}%
\pgfsetbuttcap%
\pgfsetroundjoin%
\definecolor{currentfill}{rgb}{0.977657,0.891500,0.822809}%
\pgfsetfillcolor{currentfill}%
\pgfsetlinewidth{0.311001pt}%
\definecolor{currentstroke}{rgb}{1.000000,1.000000,1.000000}%
\pgfsetstrokecolor{currentstroke}%
\pgfsetdash{}{0pt}%
\pgfpathmoveto{\pgfqpoint{4.112889in}{1.608743in}}%
\pgfpathcurveto{\pgfqpoint{4.120021in}{1.608743in}}{\pgfqpoint{4.126863in}{1.611577in}}{\pgfqpoint{4.131907in}{1.616621in}}%
\pgfpathcurveto{\pgfqpoint{4.136950in}{1.621665in}}{\pgfqpoint{4.139784in}{1.628506in}}{\pgfqpoint{4.139784in}{1.635639in}}%
\pgfpathcurveto{\pgfqpoint{4.139784in}{1.642772in}}{\pgfqpoint{4.136950in}{1.649614in}}{\pgfqpoint{4.131907in}{1.654657in}}%
\pgfpathcurveto{\pgfqpoint{4.126863in}{1.659701in}}{\pgfqpoint{4.120021in}{1.662535in}}{\pgfqpoint{4.112889in}{1.662535in}}%
\pgfpathcurveto{\pgfqpoint{4.105756in}{1.662535in}}{\pgfqpoint{4.098914in}{1.659701in}}{\pgfqpoint{4.093870in}{1.654657in}}%
\pgfpathcurveto{\pgfqpoint{4.088827in}{1.649614in}}{\pgfqpoint{4.085993in}{1.642772in}}{\pgfqpoint{4.085993in}{1.635639in}}%
\pgfpathcurveto{\pgfqpoint{4.085993in}{1.628506in}}{\pgfqpoint{4.088827in}{1.621665in}}{\pgfqpoint{4.093870in}{1.616621in}}%
\pgfpathcurveto{\pgfqpoint{4.098914in}{1.611577in}}{\pgfqpoint{4.105756in}{1.608743in}}{\pgfqpoint{4.112889in}{1.608743in}}%
\pgfpathclose%
\pgfusepath{stroke,fill}%
\end{pgfscope}%
\begin{pgfscope}%
\pgfpathrectangle{\pgfqpoint{2.867647in}{0.500000in}}{\pgfqpoint{1.764706in}{1.700000in}}%
\pgfusepath{clip}%
\pgfsetbuttcap%
\pgfsetroundjoin%
\definecolor{currentfill}{rgb}{0.971202,0.827364,0.728520}%
\pgfsetfillcolor{currentfill}%
\pgfsetlinewidth{0.311001pt}%
\definecolor{currentstroke}{rgb}{1.000000,1.000000,1.000000}%
\pgfsetstrokecolor{currentstroke}%
\pgfsetdash{}{0pt}%
\pgfpathmoveto{\pgfqpoint{4.257422in}{1.186263in}}%
\pgfpathcurveto{\pgfqpoint{4.264554in}{1.186263in}}{\pgfqpoint{4.271396in}{1.189097in}}{\pgfqpoint{4.276440in}{1.194141in}}%
\pgfpathcurveto{\pgfqpoint{4.281483in}{1.199185in}}{\pgfqpoint{4.284317in}{1.206026in}}{\pgfqpoint{4.284317in}{1.213159in}}%
\pgfpathcurveto{\pgfqpoint{4.284317in}{1.220292in}}{\pgfqpoint{4.281483in}{1.227133in}}{\pgfqpoint{4.276440in}{1.232177in}}%
\pgfpathcurveto{\pgfqpoint{4.271396in}{1.237221in}}{\pgfqpoint{4.264554in}{1.240055in}}{\pgfqpoint{4.257422in}{1.240055in}}%
\pgfpathcurveto{\pgfqpoint{4.250289in}{1.240055in}}{\pgfqpoint{4.243447in}{1.237221in}}{\pgfqpoint{4.238404in}{1.232177in}}%
\pgfpathcurveto{\pgfqpoint{4.233360in}{1.227133in}}{\pgfqpoint{4.230526in}{1.220292in}}{\pgfqpoint{4.230526in}{1.213159in}}%
\pgfpathcurveto{\pgfqpoint{4.230526in}{1.206026in}}{\pgfqpoint{4.233360in}{1.199185in}}{\pgfqpoint{4.238404in}{1.194141in}}%
\pgfpathcurveto{\pgfqpoint{4.243447in}{1.189097in}}{\pgfqpoint{4.250289in}{1.186263in}}{\pgfqpoint{4.257422in}{1.186263in}}%
\pgfpathclose%
\pgfusepath{stroke,fill}%
\end{pgfscope}%
\begin{pgfscope}%
\pgfpathrectangle{\pgfqpoint{2.867647in}{0.500000in}}{\pgfqpoint{1.764706in}{1.700000in}}%
\pgfusepath{clip}%
\pgfsetbuttcap%
\pgfsetroundjoin%
\definecolor{currentfill}{rgb}{0.976287,0.879862,0.805788}%
\pgfsetfillcolor{currentfill}%
\pgfsetlinewidth{0.311001pt}%
\definecolor{currentstroke}{rgb}{1.000000,1.000000,1.000000}%
\pgfsetstrokecolor{currentstroke}%
\pgfsetdash{}{0pt}%
\pgfpathmoveto{\pgfqpoint{4.113635in}{1.036328in}}%
\pgfpathcurveto{\pgfqpoint{4.120768in}{1.036328in}}{\pgfqpoint{4.127609in}{1.039162in}}{\pgfqpoint{4.132653in}{1.044206in}}%
\pgfpathcurveto{\pgfqpoint{4.137697in}{1.049249in}}{\pgfqpoint{4.140531in}{1.056091in}}{\pgfqpoint{4.140531in}{1.063224in}}%
\pgfpathcurveto{\pgfqpoint{4.140531in}{1.070357in}}{\pgfqpoint{4.137697in}{1.077198in}}{\pgfqpoint{4.132653in}{1.082242in}}%
\pgfpathcurveto{\pgfqpoint{4.127609in}{1.087286in}}{\pgfqpoint{4.120768in}{1.090119in}}{\pgfqpoint{4.113635in}{1.090119in}}%
\pgfpathcurveto{\pgfqpoint{4.106502in}{1.090119in}}{\pgfqpoint{4.099660in}{1.087286in}}{\pgfqpoint{4.094617in}{1.082242in}}%
\pgfpathcurveto{\pgfqpoint{4.089573in}{1.077198in}}{\pgfqpoint{4.086739in}{1.070357in}}{\pgfqpoint{4.086739in}{1.063224in}}%
\pgfpathcurveto{\pgfqpoint{4.086739in}{1.056091in}}{\pgfqpoint{4.089573in}{1.049249in}}{\pgfqpoint{4.094617in}{1.044206in}}%
\pgfpathcurveto{\pgfqpoint{4.099660in}{1.039162in}}{\pgfqpoint{4.106502in}{1.036328in}}{\pgfqpoint{4.113635in}{1.036328in}}%
\pgfpathclose%
\pgfusepath{stroke,fill}%
\end{pgfscope}%
\begin{pgfscope}%
\pgfpathrectangle{\pgfqpoint{2.867647in}{0.500000in}}{\pgfqpoint{1.764706in}{1.700000in}}%
\pgfusepath{clip}%
\pgfsetbuttcap%
\pgfsetroundjoin%
\definecolor{currentfill}{rgb}{0.977657,0.891500,0.822809}%
\pgfsetfillcolor{currentfill}%
\pgfsetlinewidth{0.311001pt}%
\definecolor{currentstroke}{rgb}{1.000000,1.000000,1.000000}%
\pgfsetstrokecolor{currentstroke}%
\pgfsetdash{}{0pt}%
\pgfpathmoveto{\pgfqpoint{4.170515in}{1.077781in}}%
\pgfpathcurveto{\pgfqpoint{4.177648in}{1.077781in}}{\pgfqpoint{4.184489in}{1.080615in}}{\pgfqpoint{4.189533in}{1.085659in}}%
\pgfpathcurveto{\pgfqpoint{4.194577in}{1.090702in}}{\pgfqpoint{4.197411in}{1.097544in}}{\pgfqpoint{4.197411in}{1.104677in}}%
\pgfpathcurveto{\pgfqpoint{4.197411in}{1.111810in}}{\pgfqpoint{4.194577in}{1.118651in}}{\pgfqpoint{4.189533in}{1.123695in}}%
\pgfpathcurveto{\pgfqpoint{4.184489in}{1.128739in}}{\pgfqpoint{4.177648in}{1.131573in}}{\pgfqpoint{4.170515in}{1.131573in}}%
\pgfpathcurveto{\pgfqpoint{4.163382in}{1.131573in}}{\pgfqpoint{4.156540in}{1.128739in}}{\pgfqpoint{4.151497in}{1.123695in}}%
\pgfpathcurveto{\pgfqpoint{4.146453in}{1.118651in}}{\pgfqpoint{4.143619in}{1.111810in}}{\pgfqpoint{4.143619in}{1.104677in}}%
\pgfpathcurveto{\pgfqpoint{4.143619in}{1.097544in}}{\pgfqpoint{4.146453in}{1.090702in}}{\pgfqpoint{4.151497in}{1.085659in}}%
\pgfpathcurveto{\pgfqpoint{4.156540in}{1.080615in}}{\pgfqpoint{4.163382in}{1.077781in}}{\pgfqpoint{4.170515in}{1.077781in}}%
\pgfpathclose%
\pgfusepath{stroke,fill}%
\end{pgfscope}%
\begin{pgfscope}%
\pgfpathrectangle{\pgfqpoint{2.867647in}{0.500000in}}{\pgfqpoint{1.764706in}{1.700000in}}%
\pgfusepath{clip}%
\pgfsetbuttcap%
\pgfsetroundjoin%
\definecolor{currentfill}{rgb}{0.980678,0.914765,0.856766}%
\pgfsetfillcolor{currentfill}%
\pgfsetlinewidth{0.311001pt}%
\definecolor{currentstroke}{rgb}{1.000000,1.000000,1.000000}%
\pgfsetstrokecolor{currentstroke}%
\pgfsetdash{}{0pt}%
\pgfpathmoveto{\pgfqpoint{4.188431in}{1.405042in}}%
\pgfpathcurveto{\pgfqpoint{4.195564in}{1.405042in}}{\pgfqpoint{4.202406in}{1.407876in}}{\pgfqpoint{4.207449in}{1.412920in}}%
\pgfpathcurveto{\pgfqpoint{4.212493in}{1.417963in}}{\pgfqpoint{4.215327in}{1.424805in}}{\pgfqpoint{4.215327in}{1.431938in}}%
\pgfpathcurveto{\pgfqpoint{4.215327in}{1.439071in}}{\pgfqpoint{4.212493in}{1.445912in}}{\pgfqpoint{4.207449in}{1.450956in}}%
\pgfpathcurveto{\pgfqpoint{4.202406in}{1.456000in}}{\pgfqpoint{4.195564in}{1.458833in}}{\pgfqpoint{4.188431in}{1.458833in}}%
\pgfpathcurveto{\pgfqpoint{4.181298in}{1.458833in}}{\pgfqpoint{4.174457in}{1.456000in}}{\pgfqpoint{4.169413in}{1.450956in}}%
\pgfpathcurveto{\pgfqpoint{4.164369in}{1.445912in}}{\pgfqpoint{4.161535in}{1.439071in}}{\pgfqpoint{4.161535in}{1.431938in}}%
\pgfpathcurveto{\pgfqpoint{4.161535in}{1.424805in}}{\pgfqpoint{4.164369in}{1.417963in}}{\pgfqpoint{4.169413in}{1.412920in}}%
\pgfpathcurveto{\pgfqpoint{4.174457in}{1.407876in}}{\pgfqpoint{4.181298in}{1.405042in}}{\pgfqpoint{4.188431in}{1.405042in}}%
\pgfpathclose%
\pgfusepath{stroke,fill}%
\end{pgfscope}%
\begin{pgfscope}%
\pgfpathrectangle{\pgfqpoint{2.867647in}{0.500000in}}{\pgfqpoint{1.764706in}{1.700000in}}%
\pgfusepath{clip}%
\pgfsetbuttcap%
\pgfsetroundjoin%
\definecolor{currentfill}{rgb}{0.964173,0.657587,0.500469}%
\pgfsetfillcolor{currentfill}%
\pgfsetlinewidth{0.311001pt}%
\definecolor{currentstroke}{rgb}{1.000000,1.000000,1.000000}%
\pgfsetstrokecolor{currentstroke}%
\pgfsetdash{}{0pt}%
\pgfpathmoveto{\pgfqpoint{4.319803in}{1.332957in}}%
\pgfpathcurveto{\pgfqpoint{4.326936in}{1.332957in}}{\pgfqpoint{4.333778in}{1.335791in}}{\pgfqpoint{4.338822in}{1.340835in}}%
\pgfpathcurveto{\pgfqpoint{4.343865in}{1.345879in}}{\pgfqpoint{4.346699in}{1.352720in}}{\pgfqpoint{4.346699in}{1.359853in}}%
\pgfpathcurveto{\pgfqpoint{4.346699in}{1.366986in}}{\pgfqpoint{4.343865in}{1.373828in}}{\pgfqpoint{4.338822in}{1.378871in}}%
\pgfpathcurveto{\pgfqpoint{4.333778in}{1.383915in}}{\pgfqpoint{4.326936in}{1.386749in}}{\pgfqpoint{4.319803in}{1.386749in}}%
\pgfpathcurveto{\pgfqpoint{4.312671in}{1.386749in}}{\pgfqpoint{4.305829in}{1.383915in}}{\pgfqpoint{4.300785in}{1.378871in}}%
\pgfpathcurveto{\pgfqpoint{4.295742in}{1.373828in}}{\pgfqpoint{4.292908in}{1.366986in}}{\pgfqpoint{4.292908in}{1.359853in}}%
\pgfpathcurveto{\pgfqpoint{4.292908in}{1.352720in}}{\pgfqpoint{4.295742in}{1.345879in}}{\pgfqpoint{4.300785in}{1.340835in}}%
\pgfpathcurveto{\pgfqpoint{4.305829in}{1.335791in}}{\pgfqpoint{4.312671in}{1.332957in}}{\pgfqpoint{4.319803in}{1.332957in}}%
\pgfpathclose%
\pgfusepath{stroke,fill}%
\end{pgfscope}%
\begin{pgfscope}%
\pgfpathrectangle{\pgfqpoint{2.867647in}{0.500000in}}{\pgfqpoint{1.764706in}{1.700000in}}%
\pgfusepath{clip}%
\pgfsetbuttcap%
\pgfsetroundjoin%
\definecolor{currentfill}{rgb}{0.969803,0.809811,0.702523}%
\pgfsetfillcolor{currentfill}%
\pgfsetlinewidth{0.311001pt}%
\definecolor{currentstroke}{rgb}{1.000000,1.000000,1.000000}%
\pgfsetstrokecolor{currentstroke}%
\pgfsetdash{}{0pt}%
\pgfpathmoveto{\pgfqpoint{4.162861in}{0.985784in}}%
\pgfpathcurveto{\pgfqpoint{4.169994in}{0.985784in}}{\pgfqpoint{4.176836in}{0.988618in}}{\pgfqpoint{4.181879in}{0.993662in}}%
\pgfpathcurveto{\pgfqpoint{4.186923in}{0.998705in}}{\pgfqpoint{4.189757in}{1.005547in}}{\pgfqpoint{4.189757in}{1.012680in}}%
\pgfpathcurveto{\pgfqpoint{4.189757in}{1.019813in}}{\pgfqpoint{4.186923in}{1.026654in}}{\pgfqpoint{4.181879in}{1.031698in}}%
\pgfpathcurveto{\pgfqpoint{4.176836in}{1.036742in}}{\pgfqpoint{4.169994in}{1.039576in}}{\pgfqpoint{4.162861in}{1.039576in}}%
\pgfpathcurveto{\pgfqpoint{4.155728in}{1.039576in}}{\pgfqpoint{4.148887in}{1.036742in}}{\pgfqpoint{4.143843in}{1.031698in}}%
\pgfpathcurveto{\pgfqpoint{4.138799in}{1.026654in}}{\pgfqpoint{4.135965in}{1.019813in}}{\pgfqpoint{4.135965in}{1.012680in}}%
\pgfpathcurveto{\pgfqpoint{4.135965in}{1.005547in}}{\pgfqpoint{4.138799in}{0.998705in}}{\pgfqpoint{4.143843in}{0.993662in}}%
\pgfpathcurveto{\pgfqpoint{4.148887in}{0.988618in}}{\pgfqpoint{4.155728in}{0.985784in}}{\pgfqpoint{4.162861in}{0.985784in}}%
\pgfpathclose%
\pgfusepath{stroke,fill}%
\end{pgfscope}%
\begin{pgfscope}%
\pgfpathrectangle{\pgfqpoint{2.867647in}{0.500000in}}{\pgfqpoint{1.764706in}{1.700000in}}%
\pgfusepath{clip}%
\pgfsetbuttcap%
\pgfsetroundjoin%
\definecolor{currentfill}{rgb}{0.980678,0.914765,0.856766}%
\pgfsetfillcolor{currentfill}%
\pgfsetlinewidth{0.311001pt}%
\definecolor{currentstroke}{rgb}{1.000000,1.000000,1.000000}%
\pgfsetstrokecolor{currentstroke}%
\pgfsetdash{}{0pt}%
\pgfpathmoveto{\pgfqpoint{4.205082in}{1.359838in}}%
\pgfpathcurveto{\pgfqpoint{4.212214in}{1.359838in}}{\pgfqpoint{4.219056in}{1.362672in}}{\pgfqpoint{4.224100in}{1.367715in}}%
\pgfpathcurveto{\pgfqpoint{4.229143in}{1.372759in}}{\pgfqpoint{4.231977in}{1.379601in}}{\pgfqpoint{4.231977in}{1.386733in}}%
\pgfpathcurveto{\pgfqpoint{4.231977in}{1.393866in}}{\pgfqpoint{4.229143in}{1.400708in}}{\pgfqpoint{4.224100in}{1.405752in}}%
\pgfpathcurveto{\pgfqpoint{4.219056in}{1.410795in}}{\pgfqpoint{4.212214in}{1.413629in}}{\pgfqpoint{4.205082in}{1.413629in}}%
\pgfpathcurveto{\pgfqpoint{4.197949in}{1.413629in}}{\pgfqpoint{4.191107in}{1.410795in}}{\pgfqpoint{4.186063in}{1.405752in}}%
\pgfpathcurveto{\pgfqpoint{4.181020in}{1.400708in}}{\pgfqpoint{4.178186in}{1.393866in}}{\pgfqpoint{4.178186in}{1.386733in}}%
\pgfpathcurveto{\pgfqpoint{4.178186in}{1.379601in}}{\pgfqpoint{4.181020in}{1.372759in}}{\pgfqpoint{4.186063in}{1.367715in}}%
\pgfpathcurveto{\pgfqpoint{4.191107in}{1.362672in}}{\pgfqpoint{4.197949in}{1.359838in}}{\pgfqpoint{4.205082in}{1.359838in}}%
\pgfpathclose%
\pgfusepath{stroke,fill}%
\end{pgfscope}%
\begin{pgfscope}%
\pgfpathrectangle{\pgfqpoint{2.867647in}{0.500000in}}{\pgfqpoint{1.764706in}{1.700000in}}%
\pgfusepath{clip}%
\pgfsetbuttcap%
\pgfsetroundjoin%
\definecolor{currentfill}{rgb}{0.969359,0.803954,0.693832}%
\pgfsetfillcolor{currentfill}%
\pgfsetlinewidth{0.311001pt}%
\definecolor{currentstroke}{rgb}{1.000000,1.000000,1.000000}%
\pgfsetstrokecolor{currentstroke}%
\pgfsetdash{}{0pt}%
\pgfpathmoveto{\pgfqpoint{4.263014in}{1.171866in}}%
\pgfpathcurveto{\pgfqpoint{4.270147in}{1.171866in}}{\pgfqpoint{4.276988in}{1.174700in}}{\pgfqpoint{4.282032in}{1.179744in}}%
\pgfpathcurveto{\pgfqpoint{4.287076in}{1.184787in}}{\pgfqpoint{4.289909in}{1.191629in}}{\pgfqpoint{4.289909in}{1.198762in}}%
\pgfpathcurveto{\pgfqpoint{4.289909in}{1.205895in}}{\pgfqpoint{4.287076in}{1.212736in}}{\pgfqpoint{4.282032in}{1.217780in}}%
\pgfpathcurveto{\pgfqpoint{4.276988in}{1.222824in}}{\pgfqpoint{4.270147in}{1.225657in}}{\pgfqpoint{4.263014in}{1.225657in}}%
\pgfpathcurveto{\pgfqpoint{4.255881in}{1.225657in}}{\pgfqpoint{4.249039in}{1.222824in}}{\pgfqpoint{4.243996in}{1.217780in}}%
\pgfpathcurveto{\pgfqpoint{4.238952in}{1.212736in}}{\pgfqpoint{4.236118in}{1.205895in}}{\pgfqpoint{4.236118in}{1.198762in}}%
\pgfpathcurveto{\pgfqpoint{4.236118in}{1.191629in}}{\pgfqpoint{4.238952in}{1.184787in}}{\pgfqpoint{4.243996in}{1.179744in}}%
\pgfpathcurveto{\pgfqpoint{4.249039in}{1.174700in}}{\pgfqpoint{4.255881in}{1.171866in}}{\pgfqpoint{4.263014in}{1.171866in}}%
\pgfpathclose%
\pgfusepath{stroke,fill}%
\end{pgfscope}%
\begin{pgfscope}%
\pgfpathrectangle{\pgfqpoint{2.867647in}{0.500000in}}{\pgfqpoint{1.764706in}{1.700000in}}%
\pgfusepath{clip}%
\pgfsetbuttcap%
\pgfsetroundjoin%
\definecolor{currentfill}{rgb}{0.981377,0.920617,0.865369}%
\pgfsetfillcolor{currentfill}%
\pgfsetlinewidth{0.311001pt}%
\definecolor{currentstroke}{rgb}{1.000000,1.000000,1.000000}%
\pgfsetstrokecolor{currentstroke}%
\pgfsetdash{}{0pt}%
\pgfpathmoveto{\pgfqpoint{4.178293in}{1.204845in}}%
\pgfpathcurveto{\pgfqpoint{4.185426in}{1.204845in}}{\pgfqpoint{4.192267in}{1.207679in}}{\pgfqpoint{4.197311in}{1.212723in}}%
\pgfpathcurveto{\pgfqpoint{4.202355in}{1.217766in}}{\pgfqpoint{4.205188in}{1.224608in}}{\pgfqpoint{4.205188in}{1.231741in}}%
\pgfpathcurveto{\pgfqpoint{4.205188in}{1.238874in}}{\pgfqpoint{4.202355in}{1.245715in}}{\pgfqpoint{4.197311in}{1.250759in}}%
\pgfpathcurveto{\pgfqpoint{4.192267in}{1.255803in}}{\pgfqpoint{4.185426in}{1.258636in}}{\pgfqpoint{4.178293in}{1.258636in}}%
\pgfpathcurveto{\pgfqpoint{4.171160in}{1.258636in}}{\pgfqpoint{4.164318in}{1.255803in}}{\pgfqpoint{4.159275in}{1.250759in}}%
\pgfpathcurveto{\pgfqpoint{4.154231in}{1.245715in}}{\pgfqpoint{4.151397in}{1.238874in}}{\pgfqpoint{4.151397in}{1.231741in}}%
\pgfpathcurveto{\pgfqpoint{4.151397in}{1.224608in}}{\pgfqpoint{4.154231in}{1.217766in}}{\pgfqpoint{4.159275in}{1.212723in}}%
\pgfpathcurveto{\pgfqpoint{4.164318in}{1.207679in}}{\pgfqpoint{4.171160in}{1.204845in}}{\pgfqpoint{4.178293in}{1.204845in}}%
\pgfpathclose%
\pgfusepath{stroke,fill}%
\end{pgfscope}%
\begin{pgfscope}%
\pgfpathrectangle{\pgfqpoint{2.867647in}{0.500000in}}{\pgfqpoint{1.764706in}{1.700000in}}%
\pgfusepath{clip}%
\pgfsetbuttcap%
\pgfsetroundjoin%
\definecolor{currentfill}{rgb}{0.975644,0.874038,0.797253}%
\pgfsetfillcolor{currentfill}%
\pgfsetlinewidth{0.311001pt}%
\definecolor{currentstroke}{rgb}{1.000000,1.000000,1.000000}%
\pgfsetstrokecolor{currentstroke}%
\pgfsetdash{}{0pt}%
\pgfpathmoveto{\pgfqpoint{4.238897in}{1.208159in}}%
\pgfpathcurveto{\pgfqpoint{4.246030in}{1.208159in}}{\pgfqpoint{4.252871in}{1.210993in}}{\pgfqpoint{4.257915in}{1.216037in}}%
\pgfpathcurveto{\pgfqpoint{4.262959in}{1.221081in}}{\pgfqpoint{4.265792in}{1.227922in}}{\pgfqpoint{4.265792in}{1.235055in}}%
\pgfpathcurveto{\pgfqpoint{4.265792in}{1.242188in}}{\pgfqpoint{4.262959in}{1.249030in}}{\pgfqpoint{4.257915in}{1.254073in}}%
\pgfpathcurveto{\pgfqpoint{4.252871in}{1.259117in}}{\pgfqpoint{4.246030in}{1.261951in}}{\pgfqpoint{4.238897in}{1.261951in}}%
\pgfpathcurveto{\pgfqpoint{4.231764in}{1.261951in}}{\pgfqpoint{4.224922in}{1.259117in}}{\pgfqpoint{4.219879in}{1.254073in}}%
\pgfpathcurveto{\pgfqpoint{4.214835in}{1.249030in}}{\pgfqpoint{4.212001in}{1.242188in}}{\pgfqpoint{4.212001in}{1.235055in}}%
\pgfpathcurveto{\pgfqpoint{4.212001in}{1.227922in}}{\pgfqpoint{4.214835in}{1.221081in}}{\pgfqpoint{4.219879in}{1.216037in}}%
\pgfpathcurveto{\pgfqpoint{4.224922in}{1.210993in}}{\pgfqpoint{4.231764in}{1.208159in}}{\pgfqpoint{4.238897in}{1.208159in}}%
\pgfpathclose%
\pgfusepath{stroke,fill}%
\end{pgfscope}%
\begin{pgfscope}%
\pgfpathrectangle{\pgfqpoint{2.867647in}{0.500000in}}{\pgfqpoint{1.764706in}{1.700000in}}%
\pgfusepath{clip}%
\pgfsetbuttcap%
\pgfsetroundjoin%
\definecolor{currentfill}{rgb}{0.975644,0.874038,0.797253}%
\pgfsetfillcolor{currentfill}%
\pgfsetlinewidth{0.311001pt}%
\definecolor{currentstroke}{rgb}{1.000000,1.000000,1.000000}%
\pgfsetstrokecolor{currentstroke}%
\pgfsetdash{}{0pt}%
\pgfpathmoveto{\pgfqpoint{4.098849in}{1.029236in}}%
\pgfpathcurveto{\pgfqpoint{4.105982in}{1.029236in}}{\pgfqpoint{4.112824in}{1.032070in}}{\pgfqpoint{4.117867in}{1.037114in}}%
\pgfpathcurveto{\pgfqpoint{4.122911in}{1.042157in}}{\pgfqpoint{4.125745in}{1.048999in}}{\pgfqpoint{4.125745in}{1.056132in}}%
\pgfpathcurveto{\pgfqpoint{4.125745in}{1.063265in}}{\pgfqpoint{4.122911in}{1.070106in}}{\pgfqpoint{4.117867in}{1.075150in}}%
\pgfpathcurveto{\pgfqpoint{4.112824in}{1.080194in}}{\pgfqpoint{4.105982in}{1.083028in}}{\pgfqpoint{4.098849in}{1.083028in}}%
\pgfpathcurveto{\pgfqpoint{4.091716in}{1.083028in}}{\pgfqpoint{4.084875in}{1.080194in}}{\pgfqpoint{4.079831in}{1.075150in}}%
\pgfpathcurveto{\pgfqpoint{4.074787in}{1.070106in}}{\pgfqpoint{4.071953in}{1.063265in}}{\pgfqpoint{4.071953in}{1.056132in}}%
\pgfpathcurveto{\pgfqpoint{4.071953in}{1.048999in}}{\pgfqpoint{4.074787in}{1.042157in}}{\pgfqpoint{4.079831in}{1.037114in}}%
\pgfpathcurveto{\pgfqpoint{4.084875in}{1.032070in}}{\pgfqpoint{4.091716in}{1.029236in}}{\pgfqpoint{4.098849in}{1.029236in}}%
\pgfpathclose%
\pgfusepath{stroke,fill}%
\end{pgfscope}%
\begin{pgfscope}%
\pgfpathrectangle{\pgfqpoint{2.867647in}{0.500000in}}{\pgfqpoint{1.764706in}{1.700000in}}%
\pgfusepath{clip}%
\pgfsetbuttcap%
\pgfsetroundjoin%
\definecolor{currentfill}{rgb}{0.968105,0.786346,0.667739}%
\pgfsetfillcolor{currentfill}%
\pgfsetlinewidth{0.311001pt}%
\definecolor{currentstroke}{rgb}{1.000000,1.000000,1.000000}%
\pgfsetstrokecolor{currentstroke}%
\pgfsetdash{}{0pt}%
\pgfpathmoveto{\pgfqpoint{4.096421in}{1.352980in}}%
\pgfpathcurveto{\pgfqpoint{4.103554in}{1.352980in}}{\pgfqpoint{4.110396in}{1.355814in}}{\pgfqpoint{4.115440in}{1.360858in}}%
\pgfpathcurveto{\pgfqpoint{4.120483in}{1.365902in}}{\pgfqpoint{4.123317in}{1.372743in}}{\pgfqpoint{4.123317in}{1.379876in}}%
\pgfpathcurveto{\pgfqpoint{4.123317in}{1.387009in}}{\pgfqpoint{4.120483in}{1.393851in}}{\pgfqpoint{4.115440in}{1.398894in}}%
\pgfpathcurveto{\pgfqpoint{4.110396in}{1.403938in}}{\pgfqpoint{4.103554in}{1.406772in}}{\pgfqpoint{4.096421in}{1.406772in}}%
\pgfpathcurveto{\pgfqpoint{4.089289in}{1.406772in}}{\pgfqpoint{4.082447in}{1.403938in}}{\pgfqpoint{4.077403in}{1.398894in}}%
\pgfpathcurveto{\pgfqpoint{4.072360in}{1.393851in}}{\pgfqpoint{4.069526in}{1.387009in}}{\pgfqpoint{4.069526in}{1.379876in}}%
\pgfpathcurveto{\pgfqpoint{4.069526in}{1.372743in}}{\pgfqpoint{4.072360in}{1.365902in}}{\pgfqpoint{4.077403in}{1.360858in}}%
\pgfpathcurveto{\pgfqpoint{4.082447in}{1.355814in}}{\pgfqpoint{4.089289in}{1.352980in}}{\pgfqpoint{4.096421in}{1.352980in}}%
\pgfpathclose%
\pgfusepath{stroke,fill}%
\end{pgfscope}%
\begin{pgfscope}%
\pgfpathrectangle{\pgfqpoint{2.867647in}{0.500000in}}{\pgfqpoint{1.764706in}{1.700000in}}%
\pgfusepath{clip}%
\pgfsetbuttcap%
\pgfsetroundjoin%
\definecolor{currentfill}{rgb}{0.974412,0.862387,0.780156}%
\pgfsetfillcolor{currentfill}%
\pgfsetlinewidth{0.311001pt}%
\definecolor{currentstroke}{rgb}{1.000000,1.000000,1.000000}%
\pgfsetstrokecolor{currentstroke}%
\pgfsetdash{}{0pt}%
\pgfpathmoveto{\pgfqpoint{4.165293in}{1.613791in}}%
\pgfpathcurveto{\pgfqpoint{4.172426in}{1.613791in}}{\pgfqpoint{4.179267in}{1.616625in}}{\pgfqpoint{4.184311in}{1.621668in}}%
\pgfpathcurveto{\pgfqpoint{4.189355in}{1.626712in}}{\pgfqpoint{4.192189in}{1.633554in}}{\pgfqpoint{4.192189in}{1.640686in}}%
\pgfpathcurveto{\pgfqpoint{4.192189in}{1.647819in}}{\pgfqpoint{4.189355in}{1.654661in}}{\pgfqpoint{4.184311in}{1.659705in}}%
\pgfpathcurveto{\pgfqpoint{4.179267in}{1.664748in}}{\pgfqpoint{4.172426in}{1.667582in}}{\pgfqpoint{4.165293in}{1.667582in}}%
\pgfpathcurveto{\pgfqpoint{4.158160in}{1.667582in}}{\pgfqpoint{4.151318in}{1.664748in}}{\pgfqpoint{4.146275in}{1.659705in}}%
\pgfpathcurveto{\pgfqpoint{4.141231in}{1.654661in}}{\pgfqpoint{4.138397in}{1.647819in}}{\pgfqpoint{4.138397in}{1.640686in}}%
\pgfpathcurveto{\pgfqpoint{4.138397in}{1.633554in}}{\pgfqpoint{4.141231in}{1.626712in}}{\pgfqpoint{4.146275in}{1.621668in}}%
\pgfpathcurveto{\pgfqpoint{4.151318in}{1.616625in}}{\pgfqpoint{4.158160in}{1.613791in}}{\pgfqpoint{4.165293in}{1.613791in}}%
\pgfpathclose%
\pgfusepath{stroke,fill}%
\end{pgfscope}%
\begin{pgfscope}%
\pgfpathrectangle{\pgfqpoint{2.867647in}{0.500000in}}{\pgfqpoint{1.764706in}{1.700000in}}%
\pgfusepath{clip}%
\pgfsetbuttcap%
\pgfsetroundjoin%
\definecolor{currentfill}{rgb}{0.975644,0.874038,0.797253}%
\pgfsetfillcolor{currentfill}%
\pgfsetlinewidth{0.311001pt}%
\definecolor{currentstroke}{rgb}{1.000000,1.000000,1.000000}%
\pgfsetstrokecolor{currentstroke}%
\pgfsetdash{}{0pt}%
\pgfpathmoveto{\pgfqpoint{4.089696in}{1.626164in}}%
\pgfpathcurveto{\pgfqpoint{4.096829in}{1.626164in}}{\pgfqpoint{4.103670in}{1.628998in}}{\pgfqpoint{4.108714in}{1.634042in}}%
\pgfpathcurveto{\pgfqpoint{4.113758in}{1.639085in}}{\pgfqpoint{4.116592in}{1.645927in}}{\pgfqpoint{4.116592in}{1.653060in}}%
\pgfpathcurveto{\pgfqpoint{4.116592in}{1.660193in}}{\pgfqpoint{4.113758in}{1.667034in}}{\pgfqpoint{4.108714in}{1.672078in}}%
\pgfpathcurveto{\pgfqpoint{4.103670in}{1.677122in}}{\pgfqpoint{4.096829in}{1.679956in}}{\pgfqpoint{4.089696in}{1.679956in}}%
\pgfpathcurveto{\pgfqpoint{4.082563in}{1.679956in}}{\pgfqpoint{4.075721in}{1.677122in}}{\pgfqpoint{4.070678in}{1.672078in}}%
\pgfpathcurveto{\pgfqpoint{4.065634in}{1.667034in}}{\pgfqpoint{4.062800in}{1.660193in}}{\pgfqpoint{4.062800in}{1.653060in}}%
\pgfpathcurveto{\pgfqpoint{4.062800in}{1.645927in}}{\pgfqpoint{4.065634in}{1.639085in}}{\pgfqpoint{4.070678in}{1.634042in}}%
\pgfpathcurveto{\pgfqpoint{4.075721in}{1.628998in}}{\pgfqpoint{4.082563in}{1.626164in}}{\pgfqpoint{4.089696in}{1.626164in}}%
\pgfpathclose%
\pgfusepath{stroke,fill}%
\end{pgfscope}%
\begin{pgfscope}%
\pgfpathrectangle{\pgfqpoint{2.867647in}{0.500000in}}{\pgfqpoint{1.764706in}{1.700000in}}%
\pgfusepath{clip}%
\pgfsetbuttcap%
\pgfsetroundjoin%
\definecolor{currentfill}{rgb}{0.966812,0.762584,0.633643}%
\pgfsetfillcolor{currentfill}%
\pgfsetlinewidth{0.311001pt}%
\definecolor{currentstroke}{rgb}{1.000000,1.000000,1.000000}%
\pgfsetstrokecolor{currentstroke}%
\pgfsetdash{}{0pt}%
\pgfpathmoveto{\pgfqpoint{4.190141in}{1.655936in}}%
\pgfpathcurveto{\pgfqpoint{4.197274in}{1.655936in}}{\pgfqpoint{4.204116in}{1.658770in}}{\pgfqpoint{4.209159in}{1.663813in}}%
\pgfpathcurveto{\pgfqpoint{4.214203in}{1.668857in}}{\pgfqpoint{4.217037in}{1.675699in}}{\pgfqpoint{4.217037in}{1.682832in}}%
\pgfpathcurveto{\pgfqpoint{4.217037in}{1.689964in}}{\pgfqpoint{4.214203in}{1.696806in}}{\pgfqpoint{4.209159in}{1.701850in}}%
\pgfpathcurveto{\pgfqpoint{4.204116in}{1.706893in}}{\pgfqpoint{4.197274in}{1.709727in}}{\pgfqpoint{4.190141in}{1.709727in}}%
\pgfpathcurveto{\pgfqpoint{4.183008in}{1.709727in}}{\pgfqpoint{4.176167in}{1.706893in}}{\pgfqpoint{4.171123in}{1.701850in}}%
\pgfpathcurveto{\pgfqpoint{4.166079in}{1.696806in}}{\pgfqpoint{4.163245in}{1.689964in}}{\pgfqpoint{4.163245in}{1.682832in}}%
\pgfpathcurveto{\pgfqpoint{4.163245in}{1.675699in}}{\pgfqpoint{4.166079in}{1.668857in}}{\pgfqpoint{4.171123in}{1.663813in}}%
\pgfpathcurveto{\pgfqpoint{4.176167in}{1.658770in}}{\pgfqpoint{4.183008in}{1.655936in}}{\pgfqpoint{4.190141in}{1.655936in}}%
\pgfpathclose%
\pgfusepath{stroke,fill}%
\end{pgfscope}%
\begin{pgfscope}%
\pgfpathrectangle{\pgfqpoint{2.867647in}{0.500000in}}{\pgfqpoint{1.764706in}{1.700000in}}%
\pgfusepath{clip}%
\pgfsetbuttcap%
\pgfsetroundjoin%
\definecolor{currentfill}{rgb}{0.967735,0.780441,0.659127}%
\pgfsetfillcolor{currentfill}%
\pgfsetlinewidth{0.311001pt}%
\definecolor{currentstroke}{rgb}{1.000000,1.000000,1.000000}%
\pgfsetstrokecolor{currentstroke}%
\pgfsetdash{}{0pt}%
\pgfpathmoveto{\pgfqpoint{4.271006in}{1.462509in}}%
\pgfpathcurveto{\pgfqpoint{4.278139in}{1.462509in}}{\pgfqpoint{4.284981in}{1.465343in}}{\pgfqpoint{4.290024in}{1.470387in}}%
\pgfpathcurveto{\pgfqpoint{4.295068in}{1.475430in}}{\pgfqpoint{4.297902in}{1.482272in}}{\pgfqpoint{4.297902in}{1.489405in}}%
\pgfpathcurveto{\pgfqpoint{4.297902in}{1.496538in}}{\pgfqpoint{4.295068in}{1.503379in}}{\pgfqpoint{4.290024in}{1.508423in}}%
\pgfpathcurveto{\pgfqpoint{4.284981in}{1.513467in}}{\pgfqpoint{4.278139in}{1.516300in}}{\pgfqpoint{4.271006in}{1.516300in}}%
\pgfpathcurveto{\pgfqpoint{4.263873in}{1.516300in}}{\pgfqpoint{4.257032in}{1.513467in}}{\pgfqpoint{4.251988in}{1.508423in}}%
\pgfpathcurveto{\pgfqpoint{4.246944in}{1.503379in}}{\pgfqpoint{4.244111in}{1.496538in}}{\pgfqpoint{4.244111in}{1.489405in}}%
\pgfpathcurveto{\pgfqpoint{4.244111in}{1.482272in}}{\pgfqpoint{4.246944in}{1.475430in}}{\pgfqpoint{4.251988in}{1.470387in}}%
\pgfpathcurveto{\pgfqpoint{4.257032in}{1.465343in}}{\pgfqpoint{4.263873in}{1.462509in}}{\pgfqpoint{4.271006in}{1.462509in}}%
\pgfpathclose%
\pgfusepath{stroke,fill}%
\end{pgfscope}%
\begin{pgfscope}%
\pgfpathrectangle{\pgfqpoint{2.867647in}{0.500000in}}{\pgfqpoint{1.764706in}{1.700000in}}%
\pgfusepath{clip}%
\pgfsetbuttcap%
\pgfsetroundjoin%
\definecolor{currentfill}{rgb}{0.922239,0.282873,0.242296}%
\pgfsetfillcolor{currentfill}%
\pgfsetlinewidth{0.311001pt}%
\definecolor{currentstroke}{rgb}{1.000000,1.000000,1.000000}%
\pgfsetstrokecolor{currentstroke}%
\pgfsetdash{}{0pt}%
\pgfpathmoveto{\pgfqpoint{3.993646in}{1.369410in}}%
\pgfpathcurveto{\pgfqpoint{4.000778in}{1.369410in}}{\pgfqpoint{4.007620in}{1.372244in}}{\pgfqpoint{4.012664in}{1.377288in}}%
\pgfpathcurveto{\pgfqpoint{4.017707in}{1.382332in}}{\pgfqpoint{4.020541in}{1.389173in}}{\pgfqpoint{4.020541in}{1.396306in}}%
\pgfpathcurveto{\pgfqpoint{4.020541in}{1.403439in}}{\pgfqpoint{4.017707in}{1.410281in}}{\pgfqpoint{4.012664in}{1.415324in}}%
\pgfpathcurveto{\pgfqpoint{4.007620in}{1.420368in}}{\pgfqpoint{4.000778in}{1.423202in}}{\pgfqpoint{3.993646in}{1.423202in}}%
\pgfpathcurveto{\pgfqpoint{3.986513in}{1.423202in}}{\pgfqpoint{3.979671in}{1.420368in}}{\pgfqpoint{3.974627in}{1.415324in}}%
\pgfpathcurveto{\pgfqpoint{3.969584in}{1.410281in}}{\pgfqpoint{3.966750in}{1.403439in}}{\pgfqpoint{3.966750in}{1.396306in}}%
\pgfpathcurveto{\pgfqpoint{3.966750in}{1.389173in}}{\pgfqpoint{3.969584in}{1.382332in}}{\pgfqpoint{3.974627in}{1.377288in}}%
\pgfpathcurveto{\pgfqpoint{3.979671in}{1.372244in}}{\pgfqpoint{3.986513in}{1.369410in}}{\pgfqpoint{3.993646in}{1.369410in}}%
\pgfpathclose%
\pgfusepath{stroke,fill}%
\end{pgfscope}%
\begin{pgfscope}%
\pgfpathrectangle{\pgfqpoint{2.867647in}{0.500000in}}{\pgfqpoint{1.764706in}{1.700000in}}%
\pgfusepath{clip}%
\pgfsetbuttcap%
\pgfsetroundjoin%
\definecolor{currentfill}{rgb}{0.972726,0.844889,0.754401}%
\pgfsetfillcolor{currentfill}%
\pgfsetlinewidth{0.311001pt}%
\definecolor{currentstroke}{rgb}{1.000000,1.000000,1.000000}%
\pgfsetstrokecolor{currentstroke}%
\pgfsetdash{}{0pt}%
\pgfpathmoveto{\pgfqpoint{4.073799in}{1.072522in}}%
\pgfpathcurveto{\pgfqpoint{4.080932in}{1.072522in}}{\pgfqpoint{4.087774in}{1.075356in}}{\pgfqpoint{4.092817in}{1.080399in}}%
\pgfpathcurveto{\pgfqpoint{4.097861in}{1.085443in}}{\pgfqpoint{4.100695in}{1.092285in}}{\pgfqpoint{4.100695in}{1.099417in}}%
\pgfpathcurveto{\pgfqpoint{4.100695in}{1.106550in}}{\pgfqpoint{4.097861in}{1.113392in}}{\pgfqpoint{4.092817in}{1.118436in}}%
\pgfpathcurveto{\pgfqpoint{4.087774in}{1.123479in}}{\pgfqpoint{4.080932in}{1.126313in}}{\pgfqpoint{4.073799in}{1.126313in}}%
\pgfpathcurveto{\pgfqpoint{4.066666in}{1.126313in}}{\pgfqpoint{4.059825in}{1.123479in}}{\pgfqpoint{4.054781in}{1.118436in}}%
\pgfpathcurveto{\pgfqpoint{4.049738in}{1.113392in}}{\pgfqpoint{4.046904in}{1.106550in}}{\pgfqpoint{4.046904in}{1.099417in}}%
\pgfpathcurveto{\pgfqpoint{4.046904in}{1.092285in}}{\pgfqpoint{4.049738in}{1.085443in}}{\pgfqpoint{4.054781in}{1.080399in}}%
\pgfpathcurveto{\pgfqpoint{4.059825in}{1.075356in}}{\pgfqpoint{4.066666in}{1.072522in}}{\pgfqpoint{4.073799in}{1.072522in}}%
\pgfpathclose%
\pgfusepath{stroke,fill}%
\end{pgfscope}%
\begin{pgfscope}%
\pgfpathrectangle{\pgfqpoint{2.867647in}{0.500000in}}{\pgfqpoint{1.764706in}{1.700000in}}%
\pgfusepath{clip}%
\pgfsetbuttcap%
\pgfsetroundjoin%
\definecolor{currentfill}{rgb}{0.975644,0.874038,0.797253}%
\pgfsetfillcolor{currentfill}%
\pgfsetlinewidth{0.311001pt}%
\definecolor{currentstroke}{rgb}{1.000000,1.000000,1.000000}%
\pgfsetstrokecolor{currentstroke}%
\pgfsetdash{}{0pt}%
\pgfpathmoveto{\pgfqpoint{4.147266in}{1.027787in}}%
\pgfpathcurveto{\pgfqpoint{4.154399in}{1.027787in}}{\pgfqpoint{4.161240in}{1.030621in}}{\pgfqpoint{4.166284in}{1.035665in}}%
\pgfpathcurveto{\pgfqpoint{4.171328in}{1.040709in}}{\pgfqpoint{4.174162in}{1.047550in}}{\pgfqpoint{4.174162in}{1.054683in}}%
\pgfpathcurveto{\pgfqpoint{4.174162in}{1.061816in}}{\pgfqpoint{4.171328in}{1.068658in}}{\pgfqpoint{4.166284in}{1.073701in}}%
\pgfpathcurveto{\pgfqpoint{4.161240in}{1.078745in}}{\pgfqpoint{4.154399in}{1.081579in}}{\pgfqpoint{4.147266in}{1.081579in}}%
\pgfpathcurveto{\pgfqpoint{4.140133in}{1.081579in}}{\pgfqpoint{4.133291in}{1.078745in}}{\pgfqpoint{4.128248in}{1.073701in}}%
\pgfpathcurveto{\pgfqpoint{4.123204in}{1.068658in}}{\pgfqpoint{4.120370in}{1.061816in}}{\pgfqpoint{4.120370in}{1.054683in}}%
\pgfpathcurveto{\pgfqpoint{4.120370in}{1.047550in}}{\pgfqpoint{4.123204in}{1.040709in}}{\pgfqpoint{4.128248in}{1.035665in}}%
\pgfpathcurveto{\pgfqpoint{4.133291in}{1.030621in}}{\pgfqpoint{4.140133in}{1.027787in}}{\pgfqpoint{4.147266in}{1.027787in}}%
\pgfpathclose%
\pgfusepath{stroke,fill}%
\end{pgfscope}%
\begin{pgfscope}%
\pgfpathrectangle{\pgfqpoint{2.867647in}{0.500000in}}{\pgfqpoint{1.764706in}{1.700000in}}%
\pgfusepath{clip}%
\pgfsetbuttcap%
\pgfsetroundjoin%
\definecolor{currentfill}{rgb}{0.980678,0.914765,0.856766}%
\pgfsetfillcolor{currentfill}%
\pgfsetlinewidth{0.311001pt}%
\definecolor{currentstroke}{rgb}{1.000000,1.000000,1.000000}%
\pgfsetstrokecolor{currentstroke}%
\pgfsetdash{}{0pt}%
\pgfpathmoveto{\pgfqpoint{4.197312in}{1.373542in}}%
\pgfpathcurveto{\pgfqpoint{4.204445in}{1.373542in}}{\pgfqpoint{4.211286in}{1.376376in}}{\pgfqpoint{4.216330in}{1.381420in}}%
\pgfpathcurveto{\pgfqpoint{4.221374in}{1.386463in}}{\pgfqpoint{4.224208in}{1.393305in}}{\pgfqpoint{4.224208in}{1.400438in}}%
\pgfpathcurveto{\pgfqpoint{4.224208in}{1.407571in}}{\pgfqpoint{4.221374in}{1.414412in}}{\pgfqpoint{4.216330in}{1.419456in}}%
\pgfpathcurveto{\pgfqpoint{4.211286in}{1.424500in}}{\pgfqpoint{4.204445in}{1.427333in}}{\pgfqpoint{4.197312in}{1.427333in}}%
\pgfpathcurveto{\pgfqpoint{4.190179in}{1.427333in}}{\pgfqpoint{4.183337in}{1.424500in}}{\pgfqpoint{4.178294in}{1.419456in}}%
\pgfpathcurveto{\pgfqpoint{4.173250in}{1.414412in}}{\pgfqpoint{4.170416in}{1.407571in}}{\pgfqpoint{4.170416in}{1.400438in}}%
\pgfpathcurveto{\pgfqpoint{4.170416in}{1.393305in}}{\pgfqpoint{4.173250in}{1.386463in}}{\pgfqpoint{4.178294in}{1.381420in}}%
\pgfpathcurveto{\pgfqpoint{4.183337in}{1.376376in}}{\pgfqpoint{4.190179in}{1.373542in}}{\pgfqpoint{4.197312in}{1.373542in}}%
\pgfpathclose%
\pgfusepath{stroke,fill}%
\end{pgfscope}%
\begin{pgfscope}%
\pgfpathrectangle{\pgfqpoint{2.867647in}{0.500000in}}{\pgfqpoint{1.764706in}{1.700000in}}%
\pgfusepath{clip}%
\pgfsetbuttcap%
\pgfsetroundjoin%
\definecolor{currentfill}{rgb}{0.962532,0.599594,0.438051}%
\pgfsetfillcolor{currentfill}%
\pgfsetlinewidth{0.311001pt}%
\definecolor{currentstroke}{rgb}{1.000000,1.000000,1.000000}%
\pgfsetstrokecolor{currentstroke}%
\pgfsetdash{}{0pt}%
\pgfpathmoveto{\pgfqpoint{3.970933in}{0.876920in}}%
\pgfpathcurveto{\pgfqpoint{3.978066in}{0.876920in}}{\pgfqpoint{3.984907in}{0.879754in}}{\pgfqpoint{3.989951in}{0.884797in}}%
\pgfpathcurveto{\pgfqpoint{3.994995in}{0.889841in}}{\pgfqpoint{3.997829in}{0.896683in}}{\pgfqpoint{3.997829in}{0.903816in}}%
\pgfpathcurveto{\pgfqpoint{3.997829in}{0.910948in}}{\pgfqpoint{3.994995in}{0.917790in}}{\pgfqpoint{3.989951in}{0.922834in}}%
\pgfpathcurveto{\pgfqpoint{3.984907in}{0.927877in}}{\pgfqpoint{3.978066in}{0.930711in}}{\pgfqpoint{3.970933in}{0.930711in}}%
\pgfpathcurveto{\pgfqpoint{3.963800in}{0.930711in}}{\pgfqpoint{3.956958in}{0.927877in}}{\pgfqpoint{3.951915in}{0.922834in}}%
\pgfpathcurveto{\pgfqpoint{3.946871in}{0.917790in}}{\pgfqpoint{3.944037in}{0.910948in}}{\pgfqpoint{3.944037in}{0.903816in}}%
\pgfpathcurveto{\pgfqpoint{3.944037in}{0.896683in}}{\pgfqpoint{3.946871in}{0.889841in}}{\pgfqpoint{3.951915in}{0.884797in}}%
\pgfpathcurveto{\pgfqpoint{3.956958in}{0.879754in}}{\pgfqpoint{3.963800in}{0.876920in}}{\pgfqpoint{3.970933in}{0.876920in}}%
\pgfpathclose%
\pgfusepath{stroke,fill}%
\end{pgfscope}%
\begin{pgfscope}%
\pgfpathrectangle{\pgfqpoint{2.867647in}{0.500000in}}{\pgfqpoint{1.764706in}{1.700000in}}%
\pgfusepath{clip}%
\pgfsetbuttcap%
\pgfsetroundjoin%
\definecolor{currentfill}{rgb}{0.980678,0.914765,0.856766}%
\pgfsetfillcolor{currentfill}%
\pgfsetlinewidth{0.311001pt}%
\definecolor{currentstroke}{rgb}{1.000000,1.000000,1.000000}%
\pgfsetstrokecolor{currentstroke}%
\pgfsetdash{}{0pt}%
\pgfpathmoveto{\pgfqpoint{4.211116in}{1.311828in}}%
\pgfpathcurveto{\pgfqpoint{4.218249in}{1.311828in}}{\pgfqpoint{4.225090in}{1.314662in}}{\pgfqpoint{4.230134in}{1.319706in}}%
\pgfpathcurveto{\pgfqpoint{4.235178in}{1.324750in}}{\pgfqpoint{4.238012in}{1.331591in}}{\pgfqpoint{4.238012in}{1.338724in}}%
\pgfpathcurveto{\pgfqpoint{4.238012in}{1.345857in}}{\pgfqpoint{4.235178in}{1.352699in}}{\pgfqpoint{4.230134in}{1.357742in}}%
\pgfpathcurveto{\pgfqpoint{4.225090in}{1.362786in}}{\pgfqpoint{4.218249in}{1.365620in}}{\pgfqpoint{4.211116in}{1.365620in}}%
\pgfpathcurveto{\pgfqpoint{4.203983in}{1.365620in}}{\pgfqpoint{4.197141in}{1.362786in}}{\pgfqpoint{4.192098in}{1.357742in}}%
\pgfpathcurveto{\pgfqpoint{4.187054in}{1.352699in}}{\pgfqpoint{4.184220in}{1.345857in}}{\pgfqpoint{4.184220in}{1.338724in}}%
\pgfpathcurveto{\pgfqpoint{4.184220in}{1.331591in}}{\pgfqpoint{4.187054in}{1.324750in}}{\pgfqpoint{4.192098in}{1.319706in}}%
\pgfpathcurveto{\pgfqpoint{4.197141in}{1.314662in}}{\pgfqpoint{4.203983in}{1.311828in}}{\pgfqpoint{4.211116in}{1.311828in}}%
\pgfpathclose%
\pgfusepath{stroke,fill}%
\end{pgfscope}%
\begin{pgfscope}%
\pgfpathrectangle{\pgfqpoint{2.867647in}{0.500000in}}{\pgfqpoint{1.764706in}{1.700000in}}%
\pgfusepath{clip}%
\pgfsetbuttcap%
\pgfsetroundjoin%
\definecolor{currentfill}{rgb}{0.976961,0.885681,0.814303}%
\pgfsetfillcolor{currentfill}%
\pgfsetlinewidth{0.311001pt}%
\definecolor{currentstroke}{rgb}{1.000000,1.000000,1.000000}%
\pgfsetstrokecolor{currentstroke}%
\pgfsetdash{}{0pt}%
\pgfpathmoveto{\pgfqpoint{4.112596in}{1.114717in}}%
\pgfpathcurveto{\pgfqpoint{4.119729in}{1.114717in}}{\pgfqpoint{4.126570in}{1.117551in}}{\pgfqpoint{4.131614in}{1.122594in}}%
\pgfpathcurveto{\pgfqpoint{4.136658in}{1.127638in}}{\pgfqpoint{4.139492in}{1.134480in}}{\pgfqpoint{4.139492in}{1.141612in}}%
\pgfpathcurveto{\pgfqpoint{4.139492in}{1.148745in}}{\pgfqpoint{4.136658in}{1.155587in}}{\pgfqpoint{4.131614in}{1.160631in}}%
\pgfpathcurveto{\pgfqpoint{4.126570in}{1.165674in}}{\pgfqpoint{4.119729in}{1.168508in}}{\pgfqpoint{4.112596in}{1.168508in}}%
\pgfpathcurveto{\pgfqpoint{4.105463in}{1.168508in}}{\pgfqpoint{4.098622in}{1.165674in}}{\pgfqpoint{4.093578in}{1.160631in}}%
\pgfpathcurveto{\pgfqpoint{4.088534in}{1.155587in}}{\pgfqpoint{4.085700in}{1.148745in}}{\pgfqpoint{4.085700in}{1.141612in}}%
\pgfpathcurveto{\pgfqpoint{4.085700in}{1.134480in}}{\pgfqpoint{4.088534in}{1.127638in}}{\pgfqpoint{4.093578in}{1.122594in}}%
\pgfpathcurveto{\pgfqpoint{4.098622in}{1.117551in}}{\pgfqpoint{4.105463in}{1.114717in}}{\pgfqpoint{4.112596in}{1.114717in}}%
\pgfpathclose%
\pgfusepath{stroke,fill}%
\end{pgfscope}%
\begin{pgfscope}%
\pgfpathrectangle{\pgfqpoint{2.867647in}{0.500000in}}{\pgfqpoint{1.764706in}{1.700000in}}%
\pgfusepath{clip}%
\pgfsetbuttcap%
\pgfsetroundjoin%
\definecolor{currentfill}{rgb}{0.684863,0.090856,0.349141}%
\pgfsetfillcolor{currentfill}%
\pgfsetlinewidth{0.311001pt}%
\definecolor{currentstroke}{rgb}{1.000000,1.000000,1.000000}%
\pgfsetstrokecolor{currentstroke}%
\pgfsetdash{}{0pt}%
\pgfpathmoveto{\pgfqpoint{3.694404in}{1.833568in}}%
\pgfpathcurveto{\pgfqpoint{3.701537in}{1.833568in}}{\pgfqpoint{3.708378in}{1.836402in}}{\pgfqpoint{3.713422in}{1.841446in}}%
\pgfpathcurveto{\pgfqpoint{3.718466in}{1.846489in}}{\pgfqpoint{3.721299in}{1.853331in}}{\pgfqpoint{3.721299in}{1.860464in}}%
\pgfpathcurveto{\pgfqpoint{3.721299in}{1.867597in}}{\pgfqpoint{3.718466in}{1.874438in}}{\pgfqpoint{3.713422in}{1.879482in}}%
\pgfpathcurveto{\pgfqpoint{3.708378in}{1.884526in}}{\pgfqpoint{3.701537in}{1.887360in}}{\pgfqpoint{3.694404in}{1.887360in}}%
\pgfpathcurveto{\pgfqpoint{3.687271in}{1.887360in}}{\pgfqpoint{3.680429in}{1.884526in}}{\pgfqpoint{3.675386in}{1.879482in}}%
\pgfpathcurveto{\pgfqpoint{3.670342in}{1.874438in}}{\pgfqpoint{3.667508in}{1.867597in}}{\pgfqpoint{3.667508in}{1.860464in}}%
\pgfpathcurveto{\pgfqpoint{3.667508in}{1.853331in}}{\pgfqpoint{3.670342in}{1.846489in}}{\pgfqpoint{3.675386in}{1.841446in}}%
\pgfpathcurveto{\pgfqpoint{3.680429in}{1.836402in}}{\pgfqpoint{3.687271in}{1.833568in}}{\pgfqpoint{3.694404in}{1.833568in}}%
\pgfpathclose%
\pgfusepath{stroke,fill}%
\end{pgfscope}%
\begin{pgfscope}%
\pgfpathrectangle{\pgfqpoint{2.867647in}{0.500000in}}{\pgfqpoint{1.764706in}{1.700000in}}%
\pgfusepath{clip}%
\pgfsetbuttcap%
\pgfsetroundjoin%
\definecolor{currentfill}{rgb}{0.981377,0.920617,0.865369}%
\pgfsetfillcolor{currentfill}%
\pgfsetlinewidth{0.311001pt}%
\definecolor{currentstroke}{rgb}{1.000000,1.000000,1.000000}%
\pgfsetstrokecolor{currentstroke}%
\pgfsetdash{}{0pt}%
\pgfpathmoveto{\pgfqpoint{4.165721in}{1.218152in}}%
\pgfpathcurveto{\pgfqpoint{4.172853in}{1.218152in}}{\pgfqpoint{4.179695in}{1.220986in}}{\pgfqpoint{4.184739in}{1.226029in}}%
\pgfpathcurveto{\pgfqpoint{4.189782in}{1.231073in}}{\pgfqpoint{4.192616in}{1.237915in}}{\pgfqpoint{4.192616in}{1.245048in}}%
\pgfpathcurveto{\pgfqpoint{4.192616in}{1.252180in}}{\pgfqpoint{4.189782in}{1.259022in}}{\pgfqpoint{4.184739in}{1.264066in}}%
\pgfpathcurveto{\pgfqpoint{4.179695in}{1.269109in}}{\pgfqpoint{4.172853in}{1.271943in}}{\pgfqpoint{4.165721in}{1.271943in}}%
\pgfpathcurveto{\pgfqpoint{4.158588in}{1.271943in}}{\pgfqpoint{4.151746in}{1.269109in}}{\pgfqpoint{4.146702in}{1.264066in}}%
\pgfpathcurveto{\pgfqpoint{4.141659in}{1.259022in}}{\pgfqpoint{4.138825in}{1.252180in}}{\pgfqpoint{4.138825in}{1.245048in}}%
\pgfpathcurveto{\pgfqpoint{4.138825in}{1.237915in}}{\pgfqpoint{4.141659in}{1.231073in}}{\pgfqpoint{4.146702in}{1.226029in}}%
\pgfpathcurveto{\pgfqpoint{4.151746in}{1.220986in}}{\pgfqpoint{4.158588in}{1.218152in}}{\pgfqpoint{4.165721in}{1.218152in}}%
\pgfpathclose%
\pgfusepath{stroke,fill}%
\end{pgfscope}%
\begin{pgfscope}%
\pgfpathrectangle{\pgfqpoint{2.867647in}{0.500000in}}{\pgfqpoint{1.764706in}{1.700000in}}%
\pgfusepath{clip}%
\pgfsetbuttcap%
\pgfsetroundjoin%
\definecolor{currentfill}{rgb}{0.978376,0.897317,0.831308}%
\pgfsetfillcolor{currentfill}%
\pgfsetlinewidth{0.311001pt}%
\definecolor{currentstroke}{rgb}{1.000000,1.000000,1.000000}%
\pgfsetstrokecolor{currentstroke}%
\pgfsetdash{}{0pt}%
\pgfpathmoveto{\pgfqpoint{4.150115in}{1.388768in}}%
\pgfpathcurveto{\pgfqpoint{4.157248in}{1.388768in}}{\pgfqpoint{4.164089in}{1.391602in}}{\pgfqpoint{4.169133in}{1.396646in}}%
\pgfpathcurveto{\pgfqpoint{4.174177in}{1.401690in}}{\pgfqpoint{4.177011in}{1.408531in}}{\pgfqpoint{4.177011in}{1.415664in}}%
\pgfpathcurveto{\pgfqpoint{4.177011in}{1.422797in}}{\pgfqpoint{4.174177in}{1.429639in}}{\pgfqpoint{4.169133in}{1.434682in}}%
\pgfpathcurveto{\pgfqpoint{4.164089in}{1.439726in}}{\pgfqpoint{4.157248in}{1.442560in}}{\pgfqpoint{4.150115in}{1.442560in}}%
\pgfpathcurveto{\pgfqpoint{4.142982in}{1.442560in}}{\pgfqpoint{4.136140in}{1.439726in}}{\pgfqpoint{4.131097in}{1.434682in}}%
\pgfpathcurveto{\pgfqpoint{4.126053in}{1.429639in}}{\pgfqpoint{4.123219in}{1.422797in}}{\pgfqpoint{4.123219in}{1.415664in}}%
\pgfpathcurveto{\pgfqpoint{4.123219in}{1.408531in}}{\pgfqpoint{4.126053in}{1.401690in}}{\pgfqpoint{4.131097in}{1.396646in}}%
\pgfpathcurveto{\pgfqpoint{4.136140in}{1.391602in}}{\pgfqpoint{4.142982in}{1.388768in}}{\pgfqpoint{4.150115in}{1.388768in}}%
\pgfpathclose%
\pgfusepath{stroke,fill}%
\end{pgfscope}%
\begin{pgfscope}%
\pgfpathrectangle{\pgfqpoint{2.867647in}{0.500000in}}{\pgfqpoint{1.764706in}{1.700000in}}%
\pgfusepath{clip}%
\pgfsetbuttcap%
\pgfsetroundjoin%
\definecolor{currentfill}{rgb}{0.973271,0.850724,0.762998}%
\pgfsetfillcolor{currentfill}%
\pgfsetlinewidth{0.311001pt}%
\definecolor{currentstroke}{rgb}{1.000000,1.000000,1.000000}%
\pgfsetstrokecolor{currentstroke}%
\pgfsetdash{}{0pt}%
\pgfpathmoveto{\pgfqpoint{4.066113in}{1.584790in}}%
\pgfpathcurveto{\pgfqpoint{4.073246in}{1.584790in}}{\pgfqpoint{4.080088in}{1.587624in}}{\pgfqpoint{4.085131in}{1.592668in}}%
\pgfpathcurveto{\pgfqpoint{4.090175in}{1.597711in}}{\pgfqpoint{4.093009in}{1.604553in}}{\pgfqpoint{4.093009in}{1.611686in}}%
\pgfpathcurveto{\pgfqpoint{4.093009in}{1.618819in}}{\pgfqpoint{4.090175in}{1.625660in}}{\pgfqpoint{4.085131in}{1.630704in}}%
\pgfpathcurveto{\pgfqpoint{4.080088in}{1.635748in}}{\pgfqpoint{4.073246in}{1.638582in}}{\pgfqpoint{4.066113in}{1.638582in}}%
\pgfpathcurveto{\pgfqpoint{4.058980in}{1.638582in}}{\pgfqpoint{4.052139in}{1.635748in}}{\pgfqpoint{4.047095in}{1.630704in}}%
\pgfpathcurveto{\pgfqpoint{4.042051in}{1.625660in}}{\pgfqpoint{4.039217in}{1.618819in}}{\pgfqpoint{4.039217in}{1.611686in}}%
\pgfpathcurveto{\pgfqpoint{4.039217in}{1.604553in}}{\pgfqpoint{4.042051in}{1.597711in}}{\pgfqpoint{4.047095in}{1.592668in}}%
\pgfpathcurveto{\pgfqpoint{4.052139in}{1.587624in}}{\pgfqpoint{4.058980in}{1.584790in}}{\pgfqpoint{4.066113in}{1.584790in}}%
\pgfpathclose%
\pgfusepath{stroke,fill}%
\end{pgfscope}%
\begin{pgfscope}%
\pgfpathrectangle{\pgfqpoint{2.867647in}{0.500000in}}{\pgfqpoint{1.764706in}{1.700000in}}%
\pgfusepath{clip}%
\pgfsetbuttcap%
\pgfsetroundjoin%
\definecolor{currentfill}{rgb}{0.977657,0.891500,0.822809}%
\pgfsetfillcolor{currentfill}%
\pgfsetlinewidth{0.311001pt}%
\definecolor{currentstroke}{rgb}{1.000000,1.000000,1.000000}%
\pgfsetstrokecolor{currentstroke}%
\pgfsetdash{}{0pt}%
\pgfpathmoveto{\pgfqpoint{4.114743in}{1.604254in}}%
\pgfpathcurveto{\pgfqpoint{4.121876in}{1.604254in}}{\pgfqpoint{4.128718in}{1.607088in}}{\pgfqpoint{4.133761in}{1.612132in}}%
\pgfpathcurveto{\pgfqpoint{4.138805in}{1.617176in}}{\pgfqpoint{4.141639in}{1.624017in}}{\pgfqpoint{4.141639in}{1.631150in}}%
\pgfpathcurveto{\pgfqpoint{4.141639in}{1.638283in}}{\pgfqpoint{4.138805in}{1.645125in}}{\pgfqpoint{4.133761in}{1.650168in}}%
\pgfpathcurveto{\pgfqpoint{4.128718in}{1.655212in}}{\pgfqpoint{4.121876in}{1.658046in}}{\pgfqpoint{4.114743in}{1.658046in}}%
\pgfpathcurveto{\pgfqpoint{4.107610in}{1.658046in}}{\pgfqpoint{4.100769in}{1.655212in}}{\pgfqpoint{4.095725in}{1.650168in}}%
\pgfpathcurveto{\pgfqpoint{4.090681in}{1.645125in}}{\pgfqpoint{4.087848in}{1.638283in}}{\pgfqpoint{4.087848in}{1.631150in}}%
\pgfpathcurveto{\pgfqpoint{4.087848in}{1.624017in}}{\pgfqpoint{4.090681in}{1.617176in}}{\pgfqpoint{4.095725in}{1.612132in}}%
\pgfpathcurveto{\pgfqpoint{4.100769in}{1.607088in}}{\pgfqpoint{4.107610in}{1.604254in}}{\pgfqpoint{4.114743in}{1.604254in}}%
\pgfpathclose%
\pgfusepath{stroke,fill}%
\end{pgfscope}%
\begin{pgfscope}%
\pgfpathrectangle{\pgfqpoint{2.867647in}{0.500000in}}{\pgfqpoint{1.764706in}{1.700000in}}%
\pgfusepath{clip}%
\pgfsetbuttcap%
\pgfsetroundjoin%
\definecolor{currentfill}{rgb}{0.979891,0.908948,0.848279}%
\pgfsetfillcolor{currentfill}%
\pgfsetlinewidth{0.311001pt}%
\definecolor{currentstroke}{rgb}{1.000000,1.000000,1.000000}%
\pgfsetstrokecolor{currentstroke}%
\pgfsetdash{}{0pt}%
\pgfpathmoveto{\pgfqpoint{4.150254in}{1.198870in}}%
\pgfpathcurveto{\pgfqpoint{4.157387in}{1.198870in}}{\pgfqpoint{4.164229in}{1.201704in}}{\pgfqpoint{4.169272in}{1.206748in}}%
\pgfpathcurveto{\pgfqpoint{4.174316in}{1.211791in}}{\pgfqpoint{4.177150in}{1.218633in}}{\pgfqpoint{4.177150in}{1.225766in}}%
\pgfpathcurveto{\pgfqpoint{4.177150in}{1.232899in}}{\pgfqpoint{4.174316in}{1.239740in}}{\pgfqpoint{4.169272in}{1.244784in}}%
\pgfpathcurveto{\pgfqpoint{4.164229in}{1.249828in}}{\pgfqpoint{4.157387in}{1.252662in}}{\pgfqpoint{4.150254in}{1.252662in}}%
\pgfpathcurveto{\pgfqpoint{4.143121in}{1.252662in}}{\pgfqpoint{4.136280in}{1.249828in}}{\pgfqpoint{4.131236in}{1.244784in}}%
\pgfpathcurveto{\pgfqpoint{4.126192in}{1.239740in}}{\pgfqpoint{4.123358in}{1.232899in}}{\pgfqpoint{4.123358in}{1.225766in}}%
\pgfpathcurveto{\pgfqpoint{4.123358in}{1.218633in}}{\pgfqpoint{4.126192in}{1.211791in}}{\pgfqpoint{4.131236in}{1.206748in}}%
\pgfpathcurveto{\pgfqpoint{4.136280in}{1.201704in}}{\pgfqpoint{4.143121in}{1.198870in}}{\pgfqpoint{4.150254in}{1.198870in}}%
\pgfpathclose%
\pgfusepath{stroke,fill}%
\end{pgfscope}%
\begin{pgfscope}%
\pgfpathrectangle{\pgfqpoint{2.867647in}{0.500000in}}{\pgfqpoint{1.764706in}{1.700000in}}%
\pgfusepath{clip}%
\pgfsetbuttcap%
\pgfsetroundjoin%
\definecolor{currentfill}{rgb}{0.960421,0.553286,0.393191}%
\pgfsetfillcolor{currentfill}%
\pgfsetlinewidth{0.311001pt}%
\definecolor{currentstroke}{rgb}{1.000000,1.000000,1.000000}%
\pgfsetstrokecolor{currentstroke}%
\pgfsetdash{}{0pt}%
\pgfpathmoveto{\pgfqpoint{3.964198in}{0.857922in}}%
\pgfpathcurveto{\pgfqpoint{3.971331in}{0.857922in}}{\pgfqpoint{3.978173in}{0.860756in}}{\pgfqpoint{3.983216in}{0.865800in}}%
\pgfpathcurveto{\pgfqpoint{3.988260in}{0.870843in}}{\pgfqpoint{3.991094in}{0.877685in}}{\pgfqpoint{3.991094in}{0.884818in}}%
\pgfpathcurveto{\pgfqpoint{3.991094in}{0.891951in}}{\pgfqpoint{3.988260in}{0.898792in}}{\pgfqpoint{3.983216in}{0.903836in}}%
\pgfpathcurveto{\pgfqpoint{3.978173in}{0.908880in}}{\pgfqpoint{3.971331in}{0.911713in}}{\pgfqpoint{3.964198in}{0.911713in}}%
\pgfpathcurveto{\pgfqpoint{3.957065in}{0.911713in}}{\pgfqpoint{3.950224in}{0.908880in}}{\pgfqpoint{3.945180in}{0.903836in}}%
\pgfpathcurveto{\pgfqpoint{3.940136in}{0.898792in}}{\pgfqpoint{3.937302in}{0.891951in}}{\pgfqpoint{3.937302in}{0.884818in}}%
\pgfpathcurveto{\pgfqpoint{3.937302in}{0.877685in}}{\pgfqpoint{3.940136in}{0.870843in}}{\pgfqpoint{3.945180in}{0.865800in}}%
\pgfpathcurveto{\pgfqpoint{3.950224in}{0.860756in}}{\pgfqpoint{3.957065in}{0.857922in}}{\pgfqpoint{3.964198in}{0.857922in}}%
\pgfpathclose%
\pgfusepath{stroke,fill}%
\end{pgfscope}%
\begin{pgfscope}%
\pgfpathrectangle{\pgfqpoint{2.867647in}{0.500000in}}{\pgfqpoint{1.764706in}{1.700000in}}%
\pgfusepath{clip}%
\pgfsetbuttcap%
\pgfsetroundjoin%
\definecolor{currentfill}{rgb}{0.937528,0.344792,0.251999}%
\pgfsetfillcolor{currentfill}%
\pgfsetlinewidth{0.311001pt}%
\definecolor{currentstroke}{rgb}{1.000000,1.000000,1.000000}%
\pgfsetstrokecolor{currentstroke}%
\pgfsetdash{}{0pt}%
\pgfpathmoveto{\pgfqpoint{4.005398in}{1.305683in}}%
\pgfpathcurveto{\pgfqpoint{4.012530in}{1.305683in}}{\pgfqpoint{4.019372in}{1.308517in}}{\pgfqpoint{4.024416in}{1.313561in}}%
\pgfpathcurveto{\pgfqpoint{4.029459in}{1.318604in}}{\pgfqpoint{4.032293in}{1.325446in}}{\pgfqpoint{4.032293in}{1.332579in}}%
\pgfpathcurveto{\pgfqpoint{4.032293in}{1.339712in}}{\pgfqpoint{4.029459in}{1.346553in}}{\pgfqpoint{4.024416in}{1.351597in}}%
\pgfpathcurveto{\pgfqpoint{4.019372in}{1.356641in}}{\pgfqpoint{4.012530in}{1.359474in}}{\pgfqpoint{4.005398in}{1.359474in}}%
\pgfpathcurveto{\pgfqpoint{3.998265in}{1.359474in}}{\pgfqpoint{3.991423in}{1.356641in}}{\pgfqpoint{3.986380in}{1.351597in}}%
\pgfpathcurveto{\pgfqpoint{3.981336in}{1.346553in}}{\pgfqpoint{3.978502in}{1.339712in}}{\pgfqpoint{3.978502in}{1.332579in}}%
\pgfpathcurveto{\pgfqpoint{3.978502in}{1.325446in}}{\pgfqpoint{3.981336in}{1.318604in}}{\pgfqpoint{3.986380in}{1.313561in}}%
\pgfpathcurveto{\pgfqpoint{3.991423in}{1.308517in}}{\pgfqpoint{3.998265in}{1.305683in}}{\pgfqpoint{4.005398in}{1.305683in}}%
\pgfpathclose%
\pgfusepath{stroke,fill}%
\end{pgfscope}%
\begin{pgfscope}%
\pgfpathrectangle{\pgfqpoint{2.867647in}{0.500000in}}{\pgfqpoint{1.764706in}{1.700000in}}%
\pgfusepath{clip}%
\pgfsetbuttcap%
\pgfsetroundjoin%
\definecolor{currentfill}{rgb}{0.965440,0.720101,0.576404}%
\pgfsetfillcolor{currentfill}%
\pgfsetlinewidth{0.311001pt}%
\definecolor{currentstroke}{rgb}{1.000000,1.000000,1.000000}%
\pgfsetstrokecolor{currentstroke}%
\pgfsetdash{}{0pt}%
\pgfpathmoveto{\pgfqpoint{4.002013in}{1.018067in}}%
\pgfpathcurveto{\pgfqpoint{4.009146in}{1.018067in}}{\pgfqpoint{4.015987in}{1.020900in}}{\pgfqpoint{4.021031in}{1.025944in}}%
\pgfpathcurveto{\pgfqpoint{4.026075in}{1.030988in}}{\pgfqpoint{4.028909in}{1.037829in}}{\pgfqpoint{4.028909in}{1.044962in}}%
\pgfpathcurveto{\pgfqpoint{4.028909in}{1.052095in}}{\pgfqpoint{4.026075in}{1.058937in}}{\pgfqpoint{4.021031in}{1.063980in}}%
\pgfpathcurveto{\pgfqpoint{4.015987in}{1.069024in}}{\pgfqpoint{4.009146in}{1.071858in}}{\pgfqpoint{4.002013in}{1.071858in}}%
\pgfpathcurveto{\pgfqpoint{3.994880in}{1.071858in}}{\pgfqpoint{3.988038in}{1.069024in}}{\pgfqpoint{3.982995in}{1.063980in}}%
\pgfpathcurveto{\pgfqpoint{3.977951in}{1.058937in}}{\pgfqpoint{3.975117in}{1.052095in}}{\pgfqpoint{3.975117in}{1.044962in}}%
\pgfpathcurveto{\pgfqpoint{3.975117in}{1.037829in}}{\pgfqpoint{3.977951in}{1.030988in}}{\pgfqpoint{3.982995in}{1.025944in}}%
\pgfpathcurveto{\pgfqpoint{3.988038in}{1.020900in}}{\pgfqpoint{3.994880in}{1.018067in}}{\pgfqpoint{4.002013in}{1.018067in}}%
\pgfpathclose%
\pgfusepath{stroke,fill}%
\end{pgfscope}%
\begin{pgfscope}%
\pgfpathrectangle{\pgfqpoint{2.867647in}{0.500000in}}{\pgfqpoint{1.764706in}{1.700000in}}%
\pgfusepath{clip}%
\pgfsetbuttcap%
\pgfsetroundjoin%
\definecolor{currentfill}{rgb}{0.978376,0.897317,0.831308}%
\pgfsetfillcolor{currentfill}%
\pgfsetlinewidth{0.311001pt}%
\definecolor{currentstroke}{rgb}{1.000000,1.000000,1.000000}%
\pgfsetstrokecolor{currentstroke}%
\pgfsetdash{}{0pt}%
\pgfpathmoveto{\pgfqpoint{4.144825in}{1.236857in}}%
\pgfpathcurveto{\pgfqpoint{4.151958in}{1.236857in}}{\pgfqpoint{4.158800in}{1.239691in}}{\pgfqpoint{4.163843in}{1.244735in}}%
\pgfpathcurveto{\pgfqpoint{4.168887in}{1.249778in}}{\pgfqpoint{4.171721in}{1.256620in}}{\pgfqpoint{4.171721in}{1.263753in}}%
\pgfpathcurveto{\pgfqpoint{4.171721in}{1.270886in}}{\pgfqpoint{4.168887in}{1.277727in}}{\pgfqpoint{4.163843in}{1.282771in}}%
\pgfpathcurveto{\pgfqpoint{4.158800in}{1.287815in}}{\pgfqpoint{4.151958in}{1.290648in}}{\pgfqpoint{4.144825in}{1.290648in}}%
\pgfpathcurveto{\pgfqpoint{4.137692in}{1.290648in}}{\pgfqpoint{4.130851in}{1.287815in}}{\pgfqpoint{4.125807in}{1.282771in}}%
\pgfpathcurveto{\pgfqpoint{4.120763in}{1.277727in}}{\pgfqpoint{4.117929in}{1.270886in}}{\pgfqpoint{4.117929in}{1.263753in}}%
\pgfpathcurveto{\pgfqpoint{4.117929in}{1.256620in}}{\pgfqpoint{4.120763in}{1.249778in}}{\pgfqpoint{4.125807in}{1.244735in}}%
\pgfpathcurveto{\pgfqpoint{4.130851in}{1.239691in}}{\pgfqpoint{4.137692in}{1.236857in}}{\pgfqpoint{4.144825in}{1.236857in}}%
\pgfpathclose%
\pgfusepath{stroke,fill}%
\end{pgfscope}%
\begin{pgfscope}%
\pgfpathrectangle{\pgfqpoint{2.867647in}{0.500000in}}{\pgfqpoint{1.764706in}{1.700000in}}%
\pgfusepath{clip}%
\pgfsetbuttcap%
\pgfsetroundjoin%
\definecolor{currentfill}{rgb}{0.967092,0.768560,0.642079}%
\pgfsetfillcolor{currentfill}%
\pgfsetlinewidth{0.311001pt}%
\definecolor{currentstroke}{rgb}{1.000000,1.000000,1.000000}%
\pgfsetstrokecolor{currentstroke}%
\pgfsetdash{}{0pt}%
\pgfpathmoveto{\pgfqpoint{4.053041in}{1.122003in}}%
\pgfpathcurveto{\pgfqpoint{4.060173in}{1.122003in}}{\pgfqpoint{4.067015in}{1.124837in}}{\pgfqpoint{4.072059in}{1.129881in}}%
\pgfpathcurveto{\pgfqpoint{4.077102in}{1.134925in}}{\pgfqpoint{4.079936in}{1.141766in}}{\pgfqpoint{4.079936in}{1.148899in}}%
\pgfpathcurveto{\pgfqpoint{4.079936in}{1.156032in}}{\pgfqpoint{4.077102in}{1.162873in}}{\pgfqpoint{4.072059in}{1.167917in}}%
\pgfpathcurveto{\pgfqpoint{4.067015in}{1.172961in}}{\pgfqpoint{4.060173in}{1.175795in}}{\pgfqpoint{4.053041in}{1.175795in}}%
\pgfpathcurveto{\pgfqpoint{4.045908in}{1.175795in}}{\pgfqpoint{4.039066in}{1.172961in}}{\pgfqpoint{4.034022in}{1.167917in}}%
\pgfpathcurveto{\pgfqpoint{4.028979in}{1.162873in}}{\pgfqpoint{4.026145in}{1.156032in}}{\pgfqpoint{4.026145in}{1.148899in}}%
\pgfpathcurveto{\pgfqpoint{4.026145in}{1.141766in}}{\pgfqpoint{4.028979in}{1.134925in}}{\pgfqpoint{4.034022in}{1.129881in}}%
\pgfpathcurveto{\pgfqpoint{4.039066in}{1.124837in}}{\pgfqpoint{4.045908in}{1.122003in}}{\pgfqpoint{4.053041in}{1.122003in}}%
\pgfpathclose%
\pgfusepath{stroke,fill}%
\end{pgfscope}%
\begin{pgfscope}%
\pgfpathrectangle{\pgfqpoint{2.867647in}{0.500000in}}{\pgfqpoint{1.764706in}{1.700000in}}%
\pgfusepath{clip}%
\pgfsetbuttcap%
\pgfsetroundjoin%
\definecolor{currentfill}{rgb}{0.960778,0.559972,0.399412}%
\pgfsetfillcolor{currentfill}%
\pgfsetlinewidth{0.311001pt}%
\definecolor{currentstroke}{rgb}{1.000000,1.000000,1.000000}%
\pgfsetstrokecolor{currentstroke}%
\pgfsetdash{}{0pt}%
\pgfpathmoveto{\pgfqpoint{3.918027in}{1.756127in}}%
\pgfpathcurveto{\pgfqpoint{3.925160in}{1.756127in}}{\pgfqpoint{3.932001in}{1.758960in}}{\pgfqpoint{3.937045in}{1.764004in}}%
\pgfpathcurveto{\pgfqpoint{3.942089in}{1.769048in}}{\pgfqpoint{3.944923in}{1.775889in}}{\pgfqpoint{3.944923in}{1.783022in}}%
\pgfpathcurveto{\pgfqpoint{3.944923in}{1.790155in}}{\pgfqpoint{3.942089in}{1.796997in}}{\pgfqpoint{3.937045in}{1.802040in}}%
\pgfpathcurveto{\pgfqpoint{3.932001in}{1.807084in}}{\pgfqpoint{3.925160in}{1.809918in}}{\pgfqpoint{3.918027in}{1.809918in}}%
\pgfpathcurveto{\pgfqpoint{3.910894in}{1.809918in}}{\pgfqpoint{3.904052in}{1.807084in}}{\pgfqpoint{3.899009in}{1.802040in}}%
\pgfpathcurveto{\pgfqpoint{3.893965in}{1.796997in}}{\pgfqpoint{3.891131in}{1.790155in}}{\pgfqpoint{3.891131in}{1.783022in}}%
\pgfpathcurveto{\pgfqpoint{3.891131in}{1.775889in}}{\pgfqpoint{3.893965in}{1.769048in}}{\pgfqpoint{3.899009in}{1.764004in}}%
\pgfpathcurveto{\pgfqpoint{3.904052in}{1.758960in}}{\pgfqpoint{3.910894in}{1.756127in}}{\pgfqpoint{3.918027in}{1.756127in}}%
\pgfpathclose%
\pgfusepath{stroke,fill}%
\end{pgfscope}%
\begin{pgfscope}%
\pgfpathrectangle{\pgfqpoint{2.867647in}{0.500000in}}{\pgfqpoint{1.764706in}{1.700000in}}%
\pgfusepath{clip}%
\pgfsetbuttcap%
\pgfsetroundjoin%
\definecolor{currentfill}{rgb}{0.966812,0.762584,0.633643}%
\pgfsetfillcolor{currentfill}%
\pgfsetlinewidth{0.311001pt}%
\definecolor{currentstroke}{rgb}{1.000000,1.000000,1.000000}%
\pgfsetstrokecolor{currentstroke}%
\pgfsetdash{}{0pt}%
\pgfpathmoveto{\pgfqpoint{4.015617in}{0.975703in}}%
\pgfpathcurveto{\pgfqpoint{4.022749in}{0.975703in}}{\pgfqpoint{4.029591in}{0.978537in}}{\pgfqpoint{4.034635in}{0.983581in}}%
\pgfpathcurveto{\pgfqpoint{4.039678in}{0.988624in}}{\pgfqpoint{4.042512in}{0.995466in}}{\pgfqpoint{4.042512in}{1.002599in}}%
\pgfpathcurveto{\pgfqpoint{4.042512in}{1.009732in}}{\pgfqpoint{4.039678in}{1.016573in}}{\pgfqpoint{4.034635in}{1.021617in}}%
\pgfpathcurveto{\pgfqpoint{4.029591in}{1.026661in}}{\pgfqpoint{4.022749in}{1.029494in}}{\pgfqpoint{4.015617in}{1.029494in}}%
\pgfpathcurveto{\pgfqpoint{4.008484in}{1.029494in}}{\pgfqpoint{4.001642in}{1.026661in}}{\pgfqpoint{3.996598in}{1.021617in}}%
\pgfpathcurveto{\pgfqpoint{3.991555in}{1.016573in}}{\pgfqpoint{3.988721in}{1.009732in}}{\pgfqpoint{3.988721in}{1.002599in}}%
\pgfpathcurveto{\pgfqpoint{3.988721in}{0.995466in}}{\pgfqpoint{3.991555in}{0.988624in}}{\pgfqpoint{3.996598in}{0.983581in}}%
\pgfpathcurveto{\pgfqpoint{4.001642in}{0.978537in}}{\pgfqpoint{4.008484in}{0.975703in}}{\pgfqpoint{4.015617in}{0.975703in}}%
\pgfpathclose%
\pgfusepath{stroke,fill}%
\end{pgfscope}%
\begin{pgfscope}%
\pgfpathrectangle{\pgfqpoint{2.867647in}{0.500000in}}{\pgfqpoint{1.764706in}{1.700000in}}%
\pgfusepath{clip}%
\pgfsetbuttcap%
\pgfsetroundjoin%
\definecolor{currentfill}{rgb}{0.975018,0.868213,0.788710}%
\pgfsetfillcolor{currentfill}%
\pgfsetlinewidth{0.311001pt}%
\definecolor{currentstroke}{rgb}{1.000000,1.000000,1.000000}%
\pgfsetstrokecolor{currentstroke}%
\pgfsetdash{}{0pt}%
\pgfpathmoveto{\pgfqpoint{4.110397in}{1.175912in}}%
\pgfpathcurveto{\pgfqpoint{4.117530in}{1.175912in}}{\pgfqpoint{4.124371in}{1.178746in}}{\pgfqpoint{4.129415in}{1.183790in}}%
\pgfpathcurveto{\pgfqpoint{4.134459in}{1.188834in}}{\pgfqpoint{4.137293in}{1.195675in}}{\pgfqpoint{4.137293in}{1.202808in}}%
\pgfpathcurveto{\pgfqpoint{4.137293in}{1.209941in}}{\pgfqpoint{4.134459in}{1.216782in}}{\pgfqpoint{4.129415in}{1.221826in}}%
\pgfpathcurveto{\pgfqpoint{4.124371in}{1.226870in}}{\pgfqpoint{4.117530in}{1.229704in}}{\pgfqpoint{4.110397in}{1.229704in}}%
\pgfpathcurveto{\pgfqpoint{4.103264in}{1.229704in}}{\pgfqpoint{4.096422in}{1.226870in}}{\pgfqpoint{4.091379in}{1.221826in}}%
\pgfpathcurveto{\pgfqpoint{4.086335in}{1.216782in}}{\pgfqpoint{4.083501in}{1.209941in}}{\pgfqpoint{4.083501in}{1.202808in}}%
\pgfpathcurveto{\pgfqpoint{4.083501in}{1.195675in}}{\pgfqpoint{4.086335in}{1.188834in}}{\pgfqpoint{4.091379in}{1.183790in}}%
\pgfpathcurveto{\pgfqpoint{4.096422in}{1.178746in}}{\pgfqpoint{4.103264in}{1.175912in}}{\pgfqpoint{4.110397in}{1.175912in}}%
\pgfpathclose%
\pgfusepath{stroke,fill}%
\end{pgfscope}%
\begin{pgfscope}%
\pgfpathrectangle{\pgfqpoint{2.867647in}{0.500000in}}{\pgfqpoint{1.764706in}{1.700000in}}%
\pgfusepath{clip}%
\pgfsetbuttcap%
\pgfsetroundjoin%
\definecolor{currentfill}{rgb}{0.973832,0.856556,0.771584}%
\pgfsetfillcolor{currentfill}%
\pgfsetlinewidth{0.311001pt}%
\definecolor{currentstroke}{rgb}{1.000000,1.000000,1.000000}%
\pgfsetstrokecolor{currentstroke}%
\pgfsetdash{}{0pt}%
\pgfpathmoveto{\pgfqpoint{4.126003in}{1.341436in}}%
\pgfpathcurveto{\pgfqpoint{4.133136in}{1.341436in}}{\pgfqpoint{4.139978in}{1.344270in}}{\pgfqpoint{4.145021in}{1.349314in}}%
\pgfpathcurveto{\pgfqpoint{4.150065in}{1.354358in}}{\pgfqpoint{4.152899in}{1.361199in}}{\pgfqpoint{4.152899in}{1.368332in}}%
\pgfpathcurveto{\pgfqpoint{4.152899in}{1.375465in}}{\pgfqpoint{4.150065in}{1.382307in}}{\pgfqpoint{4.145021in}{1.387350in}}%
\pgfpathcurveto{\pgfqpoint{4.139978in}{1.392394in}}{\pgfqpoint{4.133136in}{1.395228in}}{\pgfqpoint{4.126003in}{1.395228in}}%
\pgfpathcurveto{\pgfqpoint{4.118870in}{1.395228in}}{\pgfqpoint{4.112029in}{1.392394in}}{\pgfqpoint{4.106985in}{1.387350in}}%
\pgfpathcurveto{\pgfqpoint{4.101941in}{1.382307in}}{\pgfqpoint{4.099108in}{1.375465in}}{\pgfqpoint{4.099108in}{1.368332in}}%
\pgfpathcurveto{\pgfqpoint{4.099108in}{1.361199in}}{\pgfqpoint{4.101941in}{1.354358in}}{\pgfqpoint{4.106985in}{1.349314in}}%
\pgfpathcurveto{\pgfqpoint{4.112029in}{1.344270in}}{\pgfqpoint{4.118870in}{1.341436in}}{\pgfqpoint{4.126003in}{1.341436in}}%
\pgfpathclose%
\pgfusepath{stroke,fill}%
\end{pgfscope}%
\begin{pgfscope}%
\pgfpathrectangle{\pgfqpoint{2.867647in}{0.500000in}}{\pgfqpoint{1.764706in}{1.700000in}}%
\pgfusepath{clip}%
\pgfsetbuttcap%
\pgfsetroundjoin%
\definecolor{currentfill}{rgb}{0.978376,0.897317,0.831308}%
\pgfsetfillcolor{currentfill}%
\pgfsetlinewidth{0.311001pt}%
\definecolor{currentstroke}{rgb}{1.000000,1.000000,1.000000}%
\pgfsetstrokecolor{currentstroke}%
\pgfsetdash{}{0pt}%
\pgfpathmoveto{\pgfqpoint{4.138775in}{1.459274in}}%
\pgfpathcurveto{\pgfqpoint{4.145908in}{1.459274in}}{\pgfqpoint{4.152750in}{1.462108in}}{\pgfqpoint{4.157793in}{1.467151in}}%
\pgfpathcurveto{\pgfqpoint{4.162837in}{1.472195in}}{\pgfqpoint{4.165671in}{1.479037in}}{\pgfqpoint{4.165671in}{1.486170in}}%
\pgfpathcurveto{\pgfqpoint{4.165671in}{1.493302in}}{\pgfqpoint{4.162837in}{1.500144in}}{\pgfqpoint{4.157793in}{1.505188in}}%
\pgfpathcurveto{\pgfqpoint{4.152750in}{1.510231in}}{\pgfqpoint{4.145908in}{1.513065in}}{\pgfqpoint{4.138775in}{1.513065in}}%
\pgfpathcurveto{\pgfqpoint{4.131642in}{1.513065in}}{\pgfqpoint{4.124801in}{1.510231in}}{\pgfqpoint{4.119757in}{1.505188in}}%
\pgfpathcurveto{\pgfqpoint{4.114713in}{1.500144in}}{\pgfqpoint{4.111879in}{1.493302in}}{\pgfqpoint{4.111879in}{1.486170in}}%
\pgfpathcurveto{\pgfqpoint{4.111879in}{1.479037in}}{\pgfqpoint{4.114713in}{1.472195in}}{\pgfqpoint{4.119757in}{1.467151in}}%
\pgfpathcurveto{\pgfqpoint{4.124801in}{1.462108in}}{\pgfqpoint{4.131642in}{1.459274in}}{\pgfqpoint{4.138775in}{1.459274in}}%
\pgfpathclose%
\pgfusepath{stroke,fill}%
\end{pgfscope}%
\begin{pgfscope}%
\pgfpathrectangle{\pgfqpoint{2.867647in}{0.500000in}}{\pgfqpoint{1.764706in}{1.700000in}}%
\pgfusepath{clip}%
\pgfsetbuttcap%
\pgfsetroundjoin%
\definecolor{currentfill}{rgb}{0.965042,0.701564,0.552889}%
\pgfsetfillcolor{currentfill}%
\pgfsetlinewidth{0.311001pt}%
\definecolor{currentstroke}{rgb}{1.000000,1.000000,1.000000}%
\pgfsetstrokecolor{currentstroke}%
\pgfsetdash{}{0pt}%
\pgfpathmoveto{\pgfqpoint{4.260570in}{1.576734in}}%
\pgfpathcurveto{\pgfqpoint{4.267702in}{1.576734in}}{\pgfqpoint{4.274544in}{1.579568in}}{\pgfqpoint{4.279588in}{1.584612in}}%
\pgfpathcurveto{\pgfqpoint{4.284631in}{1.589656in}}{\pgfqpoint{4.287465in}{1.596497in}}{\pgfqpoint{4.287465in}{1.603630in}}%
\pgfpathcurveto{\pgfqpoint{4.287465in}{1.610763in}}{\pgfqpoint{4.284631in}{1.617605in}}{\pgfqpoint{4.279588in}{1.622648in}}%
\pgfpathcurveto{\pgfqpoint{4.274544in}{1.627692in}}{\pgfqpoint{4.267702in}{1.630526in}}{\pgfqpoint{4.260570in}{1.630526in}}%
\pgfpathcurveto{\pgfqpoint{4.253437in}{1.630526in}}{\pgfqpoint{4.246595in}{1.627692in}}{\pgfqpoint{4.241551in}{1.622648in}}%
\pgfpathcurveto{\pgfqpoint{4.236508in}{1.617605in}}{\pgfqpoint{4.233674in}{1.610763in}}{\pgfqpoint{4.233674in}{1.603630in}}%
\pgfpathcurveto{\pgfqpoint{4.233674in}{1.596497in}}{\pgfqpoint{4.236508in}{1.589656in}}{\pgfqpoint{4.241551in}{1.584612in}}%
\pgfpathcurveto{\pgfqpoint{4.246595in}{1.579568in}}{\pgfqpoint{4.253437in}{1.576734in}}{\pgfqpoint{4.260570in}{1.576734in}}%
\pgfpathclose%
\pgfusepath{stroke,fill}%
\end{pgfscope}%
\begin{pgfscope}%
\pgfpathrectangle{\pgfqpoint{2.867647in}{0.500000in}}{\pgfqpoint{1.764706in}{1.700000in}}%
\pgfusepath{clip}%
\pgfsetbuttcap%
\pgfsetroundjoin%
\definecolor{currentfill}{rgb}{0.974412,0.862387,0.780156}%
\pgfsetfillcolor{currentfill}%
\pgfsetlinewidth{0.311001pt}%
\definecolor{currentstroke}{rgb}{1.000000,1.000000,1.000000}%
\pgfsetstrokecolor{currentstroke}%
\pgfsetdash{}{0pt}%
\pgfpathmoveto{\pgfqpoint{4.247405in}{1.381985in}}%
\pgfpathcurveto{\pgfqpoint{4.254538in}{1.381985in}}{\pgfqpoint{4.261379in}{1.384819in}}{\pgfqpoint{4.266423in}{1.389863in}}%
\pgfpathcurveto{\pgfqpoint{4.271467in}{1.394906in}}{\pgfqpoint{4.274301in}{1.401748in}}{\pgfqpoint{4.274301in}{1.408881in}}%
\pgfpathcurveto{\pgfqpoint{4.274301in}{1.416014in}}{\pgfqpoint{4.271467in}{1.422855in}}{\pgfqpoint{4.266423in}{1.427899in}}%
\pgfpathcurveto{\pgfqpoint{4.261379in}{1.432943in}}{\pgfqpoint{4.254538in}{1.435777in}}{\pgfqpoint{4.247405in}{1.435777in}}%
\pgfpathcurveto{\pgfqpoint{4.240272in}{1.435777in}}{\pgfqpoint{4.233430in}{1.432943in}}{\pgfqpoint{4.228387in}{1.427899in}}%
\pgfpathcurveto{\pgfqpoint{4.223343in}{1.422855in}}{\pgfqpoint{4.220509in}{1.416014in}}{\pgfqpoint{4.220509in}{1.408881in}}%
\pgfpathcurveto{\pgfqpoint{4.220509in}{1.401748in}}{\pgfqpoint{4.223343in}{1.394906in}}{\pgfqpoint{4.228387in}{1.389863in}}%
\pgfpathcurveto{\pgfqpoint{4.233430in}{1.384819in}}{\pgfqpoint{4.240272in}{1.381985in}}{\pgfqpoint{4.247405in}{1.381985in}}%
\pgfpathclose%
\pgfusepath{stroke,fill}%
\end{pgfscope}%
\begin{pgfscope}%
\pgfpathrectangle{\pgfqpoint{2.867647in}{0.500000in}}{\pgfqpoint{1.764706in}{1.700000in}}%
\pgfusepath{clip}%
\pgfsetbuttcap%
\pgfsetroundjoin%
\definecolor{currentfill}{rgb}{0.972201,0.839051,0.745789}%
\pgfsetfillcolor{currentfill}%
\pgfsetlinewidth{0.311001pt}%
\definecolor{currentstroke}{rgb}{1.000000,1.000000,1.000000}%
\pgfsetstrokecolor{currentstroke}%
\pgfsetdash{}{0pt}%
\pgfpathmoveto{\pgfqpoint{4.066163in}{0.992489in}}%
\pgfpathcurveto{\pgfqpoint{4.073296in}{0.992489in}}{\pgfqpoint{4.080137in}{0.995322in}}{\pgfqpoint{4.085181in}{1.000366in}}%
\pgfpathcurveto{\pgfqpoint{4.090225in}{1.005410in}}{\pgfqpoint{4.093059in}{1.012251in}}{\pgfqpoint{4.093059in}{1.019384in}}%
\pgfpathcurveto{\pgfqpoint{4.093059in}{1.026517in}}{\pgfqpoint{4.090225in}{1.033359in}}{\pgfqpoint{4.085181in}{1.038402in}}%
\pgfpathcurveto{\pgfqpoint{4.080137in}{1.043446in}}{\pgfqpoint{4.073296in}{1.046280in}}{\pgfqpoint{4.066163in}{1.046280in}}%
\pgfpathcurveto{\pgfqpoint{4.059030in}{1.046280in}}{\pgfqpoint{4.052188in}{1.043446in}}{\pgfqpoint{4.047145in}{1.038402in}}%
\pgfpathcurveto{\pgfqpoint{4.042101in}{1.033359in}}{\pgfqpoint{4.039267in}{1.026517in}}{\pgfqpoint{4.039267in}{1.019384in}}%
\pgfpathcurveto{\pgfqpoint{4.039267in}{1.012251in}}{\pgfqpoint{4.042101in}{1.005410in}}{\pgfqpoint{4.047145in}{1.000366in}}%
\pgfpathcurveto{\pgfqpoint{4.052188in}{0.995322in}}{\pgfqpoint{4.059030in}{0.992489in}}{\pgfqpoint{4.066163in}{0.992489in}}%
\pgfpathclose%
\pgfusepath{stroke,fill}%
\end{pgfscope}%
\begin{pgfscope}%
\pgfpathrectangle{\pgfqpoint{2.867647in}{0.500000in}}{\pgfqpoint{1.764706in}{1.700000in}}%
\pgfusepath{clip}%
\pgfsetbuttcap%
\pgfsetroundjoin%
\definecolor{currentfill}{rgb}{0.973832,0.856556,0.771584}%
\pgfsetfillcolor{currentfill}%
\pgfsetlinewidth{0.311001pt}%
\definecolor{currentstroke}{rgb}{1.000000,1.000000,1.000000}%
\pgfsetstrokecolor{currentstroke}%
\pgfsetdash{}{0pt}%
\pgfpathmoveto{\pgfqpoint{4.078019in}{1.552959in}}%
\pgfpathcurveto{\pgfqpoint{4.085152in}{1.552959in}}{\pgfqpoint{4.091994in}{1.555793in}}{\pgfqpoint{4.097037in}{1.560837in}}%
\pgfpathcurveto{\pgfqpoint{4.102081in}{1.565880in}}{\pgfqpoint{4.104915in}{1.572722in}}{\pgfqpoint{4.104915in}{1.579855in}}%
\pgfpathcurveto{\pgfqpoint{4.104915in}{1.586988in}}{\pgfqpoint{4.102081in}{1.593829in}}{\pgfqpoint{4.097037in}{1.598873in}}%
\pgfpathcurveto{\pgfqpoint{4.091994in}{1.603916in}}{\pgfqpoint{4.085152in}{1.606750in}}{\pgfqpoint{4.078019in}{1.606750in}}%
\pgfpathcurveto{\pgfqpoint{4.070886in}{1.606750in}}{\pgfqpoint{4.064045in}{1.603916in}}{\pgfqpoint{4.059001in}{1.598873in}}%
\pgfpathcurveto{\pgfqpoint{4.053957in}{1.593829in}}{\pgfqpoint{4.051124in}{1.586988in}}{\pgfqpoint{4.051124in}{1.579855in}}%
\pgfpathcurveto{\pgfqpoint{4.051124in}{1.572722in}}{\pgfqpoint{4.053957in}{1.565880in}}{\pgfqpoint{4.059001in}{1.560837in}}%
\pgfpathcurveto{\pgfqpoint{4.064045in}{1.555793in}}{\pgfqpoint{4.070886in}{1.552959in}}{\pgfqpoint{4.078019in}{1.552959in}}%
\pgfpathclose%
\pgfusepath{stroke,fill}%
\end{pgfscope}%
\begin{pgfscope}%
\pgfpathrectangle{\pgfqpoint{2.867647in}{0.500000in}}{\pgfqpoint{1.764706in}{1.700000in}}%
\pgfusepath{clip}%
\pgfsetbuttcap%
\pgfsetroundjoin%
\definecolor{currentfill}{rgb}{0.972726,0.844889,0.754401}%
\pgfsetfillcolor{currentfill}%
\pgfsetlinewidth{0.311001pt}%
\definecolor{currentstroke}{rgb}{1.000000,1.000000,1.000000}%
\pgfsetstrokecolor{currentstroke}%
\pgfsetdash{}{0pt}%
\pgfpathmoveto{\pgfqpoint{4.104777in}{0.982096in}}%
\pgfpathcurveto{\pgfqpoint{4.111910in}{0.982096in}}{\pgfqpoint{4.118751in}{0.984930in}}{\pgfqpoint{4.123795in}{0.989974in}}%
\pgfpathcurveto{\pgfqpoint{4.128839in}{0.995018in}}{\pgfqpoint{4.131672in}{1.001859in}}{\pgfqpoint{4.131672in}{1.008992in}}%
\pgfpathcurveto{\pgfqpoint{4.131672in}{1.016125in}}{\pgfqpoint{4.128839in}{1.022967in}}{\pgfqpoint{4.123795in}{1.028010in}}%
\pgfpathcurveto{\pgfqpoint{4.118751in}{1.033054in}}{\pgfqpoint{4.111910in}{1.035888in}}{\pgfqpoint{4.104777in}{1.035888in}}%
\pgfpathcurveto{\pgfqpoint{4.097644in}{1.035888in}}{\pgfqpoint{4.090802in}{1.033054in}}{\pgfqpoint{4.085759in}{1.028010in}}%
\pgfpathcurveto{\pgfqpoint{4.080715in}{1.022967in}}{\pgfqpoint{4.077881in}{1.016125in}}{\pgfqpoint{4.077881in}{1.008992in}}%
\pgfpathcurveto{\pgfqpoint{4.077881in}{1.001859in}}{\pgfqpoint{4.080715in}{0.995018in}}{\pgfqpoint{4.085759in}{0.989974in}}%
\pgfpathcurveto{\pgfqpoint{4.090802in}{0.984930in}}{\pgfqpoint{4.097644in}{0.982096in}}{\pgfqpoint{4.104777in}{0.982096in}}%
\pgfpathclose%
\pgfusepath{stroke,fill}%
\end{pgfscope}%
\begin{pgfscope}%
\pgfpathrectangle{\pgfqpoint{2.867647in}{0.500000in}}{\pgfqpoint{1.764706in}{1.700000in}}%
\pgfusepath{clip}%
\pgfsetbuttcap%
\pgfsetroundjoin%
\definecolor{currentfill}{rgb}{0.980678,0.914765,0.856766}%
\pgfsetfillcolor{currentfill}%
\pgfsetlinewidth{0.311001pt}%
\definecolor{currentstroke}{rgb}{1.000000,1.000000,1.000000}%
\pgfsetstrokecolor{currentstroke}%
\pgfsetdash{}{0pt}%
\pgfpathmoveto{\pgfqpoint{4.173155in}{1.368611in}}%
\pgfpathcurveto{\pgfqpoint{4.180288in}{1.368611in}}{\pgfqpoint{4.187129in}{1.371445in}}{\pgfqpoint{4.192173in}{1.376489in}}%
\pgfpathcurveto{\pgfqpoint{4.197217in}{1.381533in}}{\pgfqpoint{4.200051in}{1.388374in}}{\pgfqpoint{4.200051in}{1.395507in}}%
\pgfpathcurveto{\pgfqpoint{4.200051in}{1.402640in}}{\pgfqpoint{4.197217in}{1.409482in}}{\pgfqpoint{4.192173in}{1.414525in}}%
\pgfpathcurveto{\pgfqpoint{4.187129in}{1.419569in}}{\pgfqpoint{4.180288in}{1.422403in}}{\pgfqpoint{4.173155in}{1.422403in}}%
\pgfpathcurveto{\pgfqpoint{4.166022in}{1.422403in}}{\pgfqpoint{4.159180in}{1.419569in}}{\pgfqpoint{4.154137in}{1.414525in}}%
\pgfpathcurveto{\pgfqpoint{4.149093in}{1.409482in}}{\pgfqpoint{4.146259in}{1.402640in}}{\pgfqpoint{4.146259in}{1.395507in}}%
\pgfpathcurveto{\pgfqpoint{4.146259in}{1.388374in}}{\pgfqpoint{4.149093in}{1.381533in}}{\pgfqpoint{4.154137in}{1.376489in}}%
\pgfpathcurveto{\pgfqpoint{4.159180in}{1.371445in}}{\pgfqpoint{4.166022in}{1.368611in}}{\pgfqpoint{4.173155in}{1.368611in}}%
\pgfpathclose%
\pgfusepath{stroke,fill}%
\end{pgfscope}%
\begin{pgfscope}%
\pgfpathrectangle{\pgfqpoint{2.867647in}{0.500000in}}{\pgfqpoint{1.764706in}{1.700000in}}%
\pgfusepath{clip}%
\pgfsetbuttcap%
\pgfsetroundjoin%
\definecolor{currentfill}{rgb}{0.977657,0.891500,0.822809}%
\pgfsetfillcolor{currentfill}%
\pgfsetlinewidth{0.311001pt}%
\definecolor{currentstroke}{rgb}{1.000000,1.000000,1.000000}%
\pgfsetstrokecolor{currentstroke}%
\pgfsetdash{}{0pt}%
\pgfpathmoveto{\pgfqpoint{4.144211in}{1.267345in}}%
\pgfpathcurveto{\pgfqpoint{4.151344in}{1.267345in}}{\pgfqpoint{4.158186in}{1.270179in}}{\pgfqpoint{4.163229in}{1.275222in}}%
\pgfpathcurveto{\pgfqpoint{4.168273in}{1.280266in}}{\pgfqpoint{4.171107in}{1.287108in}}{\pgfqpoint{4.171107in}{1.294241in}}%
\pgfpathcurveto{\pgfqpoint{4.171107in}{1.301373in}}{\pgfqpoint{4.168273in}{1.308215in}}{\pgfqpoint{4.163229in}{1.313259in}}%
\pgfpathcurveto{\pgfqpoint{4.158186in}{1.318302in}}{\pgfqpoint{4.151344in}{1.321136in}}{\pgfqpoint{4.144211in}{1.321136in}}%
\pgfpathcurveto{\pgfqpoint{4.137078in}{1.321136in}}{\pgfqpoint{4.130237in}{1.318302in}}{\pgfqpoint{4.125193in}{1.313259in}}%
\pgfpathcurveto{\pgfqpoint{4.120149in}{1.308215in}}{\pgfqpoint{4.117315in}{1.301373in}}{\pgfqpoint{4.117315in}{1.294241in}}%
\pgfpathcurveto{\pgfqpoint{4.117315in}{1.287108in}}{\pgfqpoint{4.120149in}{1.280266in}}{\pgfqpoint{4.125193in}{1.275222in}}%
\pgfpathcurveto{\pgfqpoint{4.130237in}{1.270179in}}{\pgfqpoint{4.137078in}{1.267345in}}{\pgfqpoint{4.144211in}{1.267345in}}%
\pgfpathclose%
\pgfusepath{stroke,fill}%
\end{pgfscope}%
\begin{pgfscope}%
\pgfpathrectangle{\pgfqpoint{2.867647in}{0.500000in}}{\pgfqpoint{1.764706in}{1.700000in}}%
\pgfusepath{clip}%
\pgfsetbuttcap%
\pgfsetroundjoin%
\definecolor{currentfill}{rgb}{0.978376,0.897317,0.831308}%
\pgfsetfillcolor{currentfill}%
\pgfsetlinewidth{0.311001pt}%
\definecolor{currentstroke}{rgb}{1.000000,1.000000,1.000000}%
\pgfsetstrokecolor{currentstroke}%
\pgfsetdash{}{0pt}%
\pgfpathmoveto{\pgfqpoint{4.155521in}{1.357300in}}%
\pgfpathcurveto{\pgfqpoint{4.162654in}{1.357300in}}{\pgfqpoint{4.169496in}{1.360134in}}{\pgfqpoint{4.174539in}{1.365177in}}%
\pgfpathcurveto{\pgfqpoint{4.179583in}{1.370221in}}{\pgfqpoint{4.182417in}{1.377063in}}{\pgfqpoint{4.182417in}{1.384195in}}%
\pgfpathcurveto{\pgfqpoint{4.182417in}{1.391328in}}{\pgfqpoint{4.179583in}{1.398170in}}{\pgfqpoint{4.174539in}{1.403214in}}%
\pgfpathcurveto{\pgfqpoint{4.169496in}{1.408257in}}{\pgfqpoint{4.162654in}{1.411091in}}{\pgfqpoint{4.155521in}{1.411091in}}%
\pgfpathcurveto{\pgfqpoint{4.148388in}{1.411091in}}{\pgfqpoint{4.141547in}{1.408257in}}{\pgfqpoint{4.136503in}{1.403214in}}%
\pgfpathcurveto{\pgfqpoint{4.131459in}{1.398170in}}{\pgfqpoint{4.128625in}{1.391328in}}{\pgfqpoint{4.128625in}{1.384195in}}%
\pgfpathcurveto{\pgfqpoint{4.128625in}{1.377063in}}{\pgfqpoint{4.131459in}{1.370221in}}{\pgfqpoint{4.136503in}{1.365177in}}%
\pgfpathcurveto{\pgfqpoint{4.141547in}{1.360134in}}{\pgfqpoint{4.148388in}{1.357300in}}{\pgfqpoint{4.155521in}{1.357300in}}%
\pgfpathclose%
\pgfusepath{stroke,fill}%
\end{pgfscope}%
\begin{pgfscope}%
\pgfpathrectangle{\pgfqpoint{2.867647in}{0.500000in}}{\pgfqpoint{1.764706in}{1.700000in}}%
\pgfusepath{clip}%
\pgfsetbuttcap%
\pgfsetroundjoin%
\definecolor{currentfill}{rgb}{0.971202,0.827364,0.728520}%
\pgfsetfillcolor{currentfill}%
\pgfsetlinewidth{0.311001pt}%
\definecolor{currentstroke}{rgb}{1.000000,1.000000,1.000000}%
\pgfsetstrokecolor{currentstroke}%
\pgfsetdash{}{0pt}%
\pgfpathmoveto{\pgfqpoint{4.185304in}{1.623241in}}%
\pgfpathcurveto{\pgfqpoint{4.192437in}{1.623241in}}{\pgfqpoint{4.199279in}{1.626075in}}{\pgfqpoint{4.204322in}{1.631119in}}%
\pgfpathcurveto{\pgfqpoint{4.209366in}{1.636163in}}{\pgfqpoint{4.212200in}{1.643004in}}{\pgfqpoint{4.212200in}{1.650137in}}%
\pgfpathcurveto{\pgfqpoint{4.212200in}{1.657270in}}{\pgfqpoint{4.209366in}{1.664112in}}{\pgfqpoint{4.204322in}{1.669155in}}%
\pgfpathcurveto{\pgfqpoint{4.199279in}{1.674199in}}{\pgfqpoint{4.192437in}{1.677033in}}{\pgfqpoint{4.185304in}{1.677033in}}%
\pgfpathcurveto{\pgfqpoint{4.178171in}{1.677033in}}{\pgfqpoint{4.171330in}{1.674199in}}{\pgfqpoint{4.166286in}{1.669155in}}%
\pgfpathcurveto{\pgfqpoint{4.161242in}{1.664112in}}{\pgfqpoint{4.158409in}{1.657270in}}{\pgfqpoint{4.158409in}{1.650137in}}%
\pgfpathcurveto{\pgfqpoint{4.158409in}{1.643004in}}{\pgfqpoint{4.161242in}{1.636163in}}{\pgfqpoint{4.166286in}{1.631119in}}%
\pgfpathcurveto{\pgfqpoint{4.171330in}{1.626075in}}{\pgfqpoint{4.178171in}{1.623241in}}{\pgfqpoint{4.185304in}{1.623241in}}%
\pgfpathclose%
\pgfusepath{stroke,fill}%
\end{pgfscope}%
\begin{pgfscope}%
\pgfpathrectangle{\pgfqpoint{2.867647in}{0.500000in}}{\pgfqpoint{1.764706in}{1.700000in}}%
\pgfusepath{clip}%
\pgfsetbuttcap%
\pgfsetroundjoin%
\definecolor{currentfill}{rgb}{0.975018,0.868213,0.788710}%
\pgfsetfillcolor{currentfill}%
\pgfsetlinewidth{0.311001pt}%
\definecolor{currentstroke}{rgb}{1.000000,1.000000,1.000000}%
\pgfsetstrokecolor{currentstroke}%
\pgfsetdash{}{0pt}%
\pgfpathmoveto{\pgfqpoint{4.239345in}{1.430638in}}%
\pgfpathcurveto{\pgfqpoint{4.246478in}{1.430638in}}{\pgfqpoint{4.253320in}{1.433472in}}{\pgfqpoint{4.258363in}{1.438516in}}%
\pgfpathcurveto{\pgfqpoint{4.263407in}{1.443559in}}{\pgfqpoint{4.266241in}{1.450401in}}{\pgfqpoint{4.266241in}{1.457534in}}%
\pgfpathcurveto{\pgfqpoint{4.266241in}{1.464667in}}{\pgfqpoint{4.263407in}{1.471508in}}{\pgfqpoint{4.258363in}{1.476552in}}%
\pgfpathcurveto{\pgfqpoint{4.253320in}{1.481596in}}{\pgfqpoint{4.246478in}{1.484429in}}{\pgfqpoint{4.239345in}{1.484429in}}%
\pgfpathcurveto{\pgfqpoint{4.232212in}{1.484429in}}{\pgfqpoint{4.225371in}{1.481596in}}{\pgfqpoint{4.220327in}{1.476552in}}%
\pgfpathcurveto{\pgfqpoint{4.215283in}{1.471508in}}{\pgfqpoint{4.212449in}{1.464667in}}{\pgfqpoint{4.212449in}{1.457534in}}%
\pgfpathcurveto{\pgfqpoint{4.212449in}{1.450401in}}{\pgfqpoint{4.215283in}{1.443559in}}{\pgfqpoint{4.220327in}{1.438516in}}%
\pgfpathcurveto{\pgfqpoint{4.225371in}{1.433472in}}{\pgfqpoint{4.232212in}{1.430638in}}{\pgfqpoint{4.239345in}{1.430638in}}%
\pgfpathclose%
\pgfusepath{stroke,fill}%
\end{pgfscope}%
\begin{pgfscope}%
\pgfpathrectangle{\pgfqpoint{2.867647in}{0.500000in}}{\pgfqpoint{1.764706in}{1.700000in}}%
\pgfusepath{clip}%
\pgfsetbuttcap%
\pgfsetroundjoin%
\definecolor{currentfill}{rgb}{0.979891,0.908948,0.848279}%
\pgfsetfillcolor{currentfill}%
\pgfsetlinewidth{0.311001pt}%
\definecolor{currentstroke}{rgb}{1.000000,1.000000,1.000000}%
\pgfsetstrokecolor{currentstroke}%
\pgfsetdash{}{0pt}%
\pgfpathmoveto{\pgfqpoint{4.204142in}{1.398927in}}%
\pgfpathcurveto{\pgfqpoint{4.211274in}{1.398927in}}{\pgfqpoint{4.218116in}{1.401761in}}{\pgfqpoint{4.223160in}{1.406804in}}%
\pgfpathcurveto{\pgfqpoint{4.228203in}{1.411848in}}{\pgfqpoint{4.231037in}{1.418690in}}{\pgfqpoint{4.231037in}{1.425822in}}%
\pgfpathcurveto{\pgfqpoint{4.231037in}{1.432955in}}{\pgfqpoint{4.228203in}{1.439797in}}{\pgfqpoint{4.223160in}{1.444841in}}%
\pgfpathcurveto{\pgfqpoint{4.218116in}{1.449884in}}{\pgfqpoint{4.211274in}{1.452718in}}{\pgfqpoint{4.204142in}{1.452718in}}%
\pgfpathcurveto{\pgfqpoint{4.197009in}{1.452718in}}{\pgfqpoint{4.190167in}{1.449884in}}{\pgfqpoint{4.185123in}{1.444841in}}%
\pgfpathcurveto{\pgfqpoint{4.180080in}{1.439797in}}{\pgfqpoint{4.177246in}{1.432955in}}{\pgfqpoint{4.177246in}{1.425822in}}%
\pgfpathcurveto{\pgfqpoint{4.177246in}{1.418690in}}{\pgfqpoint{4.180080in}{1.411848in}}{\pgfqpoint{4.185123in}{1.406804in}}%
\pgfpathcurveto{\pgfqpoint{4.190167in}{1.401761in}}{\pgfqpoint{4.197009in}{1.398927in}}{\pgfqpoint{4.204142in}{1.398927in}}%
\pgfpathclose%
\pgfusepath{stroke,fill}%
\end{pgfscope}%
\begin{pgfscope}%
\pgfpathrectangle{\pgfqpoint{2.867647in}{0.500000in}}{\pgfqpoint{1.764706in}{1.700000in}}%
\pgfusepath{clip}%
\pgfsetbuttcap%
\pgfsetroundjoin%
\definecolor{currentfill}{rgb}{0.979891,0.908948,0.848279}%
\pgfsetfillcolor{currentfill}%
\pgfsetlinewidth{0.311001pt}%
\definecolor{currentstroke}{rgb}{1.000000,1.000000,1.000000}%
\pgfsetstrokecolor{currentstroke}%
\pgfsetdash{}{0pt}%
\pgfpathmoveto{\pgfqpoint{4.142222in}{1.179839in}}%
\pgfpathcurveto{\pgfqpoint{4.149355in}{1.179839in}}{\pgfqpoint{4.156197in}{1.182673in}}{\pgfqpoint{4.161241in}{1.187716in}}%
\pgfpathcurveto{\pgfqpoint{4.166284in}{1.192760in}}{\pgfqpoint{4.169118in}{1.199602in}}{\pgfqpoint{4.169118in}{1.206735in}}%
\pgfpathcurveto{\pgfqpoint{4.169118in}{1.213867in}}{\pgfqpoint{4.166284in}{1.220709in}}{\pgfqpoint{4.161241in}{1.225753in}}%
\pgfpathcurveto{\pgfqpoint{4.156197in}{1.230796in}}{\pgfqpoint{4.149355in}{1.233630in}}{\pgfqpoint{4.142222in}{1.233630in}}%
\pgfpathcurveto{\pgfqpoint{4.135090in}{1.233630in}}{\pgfqpoint{4.128248in}{1.230796in}}{\pgfqpoint{4.123204in}{1.225753in}}%
\pgfpathcurveto{\pgfqpoint{4.118161in}{1.220709in}}{\pgfqpoint{4.115327in}{1.213867in}}{\pgfqpoint{4.115327in}{1.206735in}}%
\pgfpathcurveto{\pgfqpoint{4.115327in}{1.199602in}}{\pgfqpoint{4.118161in}{1.192760in}}{\pgfqpoint{4.123204in}{1.187716in}}%
\pgfpathcurveto{\pgfqpoint{4.128248in}{1.182673in}}{\pgfqpoint{4.135090in}{1.179839in}}{\pgfqpoint{4.142222in}{1.179839in}}%
\pgfpathclose%
\pgfusepath{stroke,fill}%
\end{pgfscope}%
\begin{pgfscope}%
\pgfpathrectangle{\pgfqpoint{2.867647in}{0.500000in}}{\pgfqpoint{1.764706in}{1.700000in}}%
\pgfusepath{clip}%
\pgfsetbuttcap%
\pgfsetroundjoin%
\definecolor{currentfill}{rgb}{0.970255,0.815666,0.711203}%
\pgfsetfillcolor{currentfill}%
\pgfsetlinewidth{0.311001pt}%
\definecolor{currentstroke}{rgb}{1.000000,1.000000,1.000000}%
\pgfsetstrokecolor{currentstroke}%
\pgfsetdash{}{0pt}%
\pgfpathmoveto{\pgfqpoint{4.044825in}{1.649246in}}%
\pgfpathcurveto{\pgfqpoint{4.051958in}{1.649246in}}{\pgfqpoint{4.058799in}{1.652080in}}{\pgfqpoint{4.063843in}{1.657124in}}%
\pgfpathcurveto{\pgfqpoint{4.068887in}{1.662167in}}{\pgfqpoint{4.071721in}{1.669009in}}{\pgfqpoint{4.071721in}{1.676142in}}%
\pgfpathcurveto{\pgfqpoint{4.071721in}{1.683275in}}{\pgfqpoint{4.068887in}{1.690116in}}{\pgfqpoint{4.063843in}{1.695160in}}%
\pgfpathcurveto{\pgfqpoint{4.058799in}{1.700204in}}{\pgfqpoint{4.051958in}{1.703037in}}{\pgfqpoint{4.044825in}{1.703037in}}%
\pgfpathcurveto{\pgfqpoint{4.037692in}{1.703037in}}{\pgfqpoint{4.030850in}{1.700204in}}{\pgfqpoint{4.025807in}{1.695160in}}%
\pgfpathcurveto{\pgfqpoint{4.020763in}{1.690116in}}{\pgfqpoint{4.017929in}{1.683275in}}{\pgfqpoint{4.017929in}{1.676142in}}%
\pgfpathcurveto{\pgfqpoint{4.017929in}{1.669009in}}{\pgfqpoint{4.020763in}{1.662167in}}{\pgfqpoint{4.025807in}{1.657124in}}%
\pgfpathcurveto{\pgfqpoint{4.030850in}{1.652080in}}{\pgfqpoint{4.037692in}{1.649246in}}{\pgfqpoint{4.044825in}{1.649246in}}%
\pgfpathclose%
\pgfusepath{stroke,fill}%
\end{pgfscope}%
\begin{pgfscope}%
\pgfpathrectangle{\pgfqpoint{2.867647in}{0.500000in}}{\pgfqpoint{1.764706in}{1.700000in}}%
\pgfusepath{clip}%
\pgfsetbuttcap%
\pgfsetroundjoin%
\definecolor{currentfill}{rgb}{0.964032,0.651225,0.493258}%
\pgfsetfillcolor{currentfill}%
\pgfsetlinewidth{0.311001pt}%
\definecolor{currentstroke}{rgb}{1.000000,1.000000,1.000000}%
\pgfsetstrokecolor{currentstroke}%
\pgfsetdash{}{0pt}%
\pgfpathmoveto{\pgfqpoint{4.022065in}{1.134598in}}%
\pgfpathcurveto{\pgfqpoint{4.029198in}{1.134598in}}{\pgfqpoint{4.036040in}{1.137432in}}{\pgfqpoint{4.041083in}{1.142475in}}%
\pgfpathcurveto{\pgfqpoint{4.046127in}{1.147519in}}{\pgfqpoint{4.048961in}{1.154361in}}{\pgfqpoint{4.048961in}{1.161493in}}%
\pgfpathcurveto{\pgfqpoint{4.048961in}{1.168626in}}{\pgfqpoint{4.046127in}{1.175468in}}{\pgfqpoint{4.041083in}{1.180512in}}%
\pgfpathcurveto{\pgfqpoint{4.036040in}{1.185555in}}{\pgfqpoint{4.029198in}{1.188389in}}{\pgfqpoint{4.022065in}{1.188389in}}%
\pgfpathcurveto{\pgfqpoint{4.014932in}{1.188389in}}{\pgfqpoint{4.008091in}{1.185555in}}{\pgfqpoint{4.003047in}{1.180512in}}%
\pgfpathcurveto{\pgfqpoint{3.998003in}{1.175468in}}{\pgfqpoint{3.995169in}{1.168626in}}{\pgfqpoint{3.995169in}{1.161493in}}%
\pgfpathcurveto{\pgfqpoint{3.995169in}{1.154361in}}{\pgfqpoint{3.998003in}{1.147519in}}{\pgfqpoint{4.003047in}{1.142475in}}%
\pgfpathcurveto{\pgfqpoint{4.008091in}{1.137432in}}{\pgfqpoint{4.014932in}{1.134598in}}{\pgfqpoint{4.022065in}{1.134598in}}%
\pgfpathclose%
\pgfusepath{stroke,fill}%
\end{pgfscope}%
\begin{pgfscope}%
\pgfpathrectangle{\pgfqpoint{2.867647in}{0.500000in}}{\pgfqpoint{1.764706in}{1.700000in}}%
\pgfusepath{clip}%
\pgfsetbuttcap%
\pgfsetroundjoin%
\definecolor{currentfill}{rgb}{0.976287,0.879862,0.805788}%
\pgfsetfillcolor{currentfill}%
\pgfsetlinewidth{0.311001pt}%
\definecolor{currentstroke}{rgb}{1.000000,1.000000,1.000000}%
\pgfsetstrokecolor{currentstroke}%
\pgfsetdash{}{0pt}%
\pgfpathmoveto{\pgfqpoint{4.194087in}{1.546638in}}%
\pgfpathcurveto{\pgfqpoint{4.201219in}{1.546638in}}{\pgfqpoint{4.208061in}{1.549472in}}{\pgfqpoint{4.213105in}{1.554515in}}%
\pgfpathcurveto{\pgfqpoint{4.218148in}{1.559559in}}{\pgfqpoint{4.220982in}{1.566401in}}{\pgfqpoint{4.220982in}{1.573534in}}%
\pgfpathcurveto{\pgfqpoint{4.220982in}{1.580666in}}{\pgfqpoint{4.218148in}{1.587508in}}{\pgfqpoint{4.213105in}{1.592552in}}%
\pgfpathcurveto{\pgfqpoint{4.208061in}{1.597595in}}{\pgfqpoint{4.201219in}{1.600429in}}{\pgfqpoint{4.194087in}{1.600429in}}%
\pgfpathcurveto{\pgfqpoint{4.186954in}{1.600429in}}{\pgfqpoint{4.180112in}{1.597595in}}{\pgfqpoint{4.175069in}{1.592552in}}%
\pgfpathcurveto{\pgfqpoint{4.170025in}{1.587508in}}{\pgfqpoint{4.167191in}{1.580666in}}{\pgfqpoint{4.167191in}{1.573534in}}%
\pgfpathcurveto{\pgfqpoint{4.167191in}{1.566401in}}{\pgfqpoint{4.170025in}{1.559559in}}{\pgfqpoint{4.175069in}{1.554515in}}%
\pgfpathcurveto{\pgfqpoint{4.180112in}{1.549472in}}{\pgfqpoint{4.186954in}{1.546638in}}{\pgfqpoint{4.194087in}{1.546638in}}%
\pgfpathclose%
\pgfusepath{stroke,fill}%
\end{pgfscope}%
\begin{pgfscope}%
\pgfpathrectangle{\pgfqpoint{2.867647in}{0.500000in}}{\pgfqpoint{1.764706in}{1.700000in}}%
\pgfusepath{clip}%
\pgfsetbuttcap%
\pgfsetroundjoin%
\definecolor{currentfill}{rgb}{0.966328,0.750560,0.616961}%
\pgfsetfillcolor{currentfill}%
\pgfsetlinewidth{0.311001pt}%
\definecolor{currentstroke}{rgb}{1.000000,1.000000,1.000000}%
\pgfsetstrokecolor{currentstroke}%
\pgfsetdash{}{0pt}%
\pgfpathmoveto{\pgfqpoint{4.074169in}{1.419261in}}%
\pgfpathcurveto{\pgfqpoint{4.081302in}{1.419261in}}{\pgfqpoint{4.088144in}{1.422095in}}{\pgfqpoint{4.093187in}{1.427138in}}%
\pgfpathcurveto{\pgfqpoint{4.098231in}{1.432182in}}{\pgfqpoint{4.101065in}{1.439024in}}{\pgfqpoint{4.101065in}{1.446157in}}%
\pgfpathcurveto{\pgfqpoint{4.101065in}{1.453289in}}{\pgfqpoint{4.098231in}{1.460131in}}{\pgfqpoint{4.093187in}{1.465175in}}%
\pgfpathcurveto{\pgfqpoint{4.088144in}{1.470218in}}{\pgfqpoint{4.081302in}{1.473052in}}{\pgfqpoint{4.074169in}{1.473052in}}%
\pgfpathcurveto{\pgfqpoint{4.067036in}{1.473052in}}{\pgfqpoint{4.060195in}{1.470218in}}{\pgfqpoint{4.055151in}{1.465175in}}%
\pgfpathcurveto{\pgfqpoint{4.050107in}{1.460131in}}{\pgfqpoint{4.047273in}{1.453289in}}{\pgfqpoint{4.047273in}{1.446157in}}%
\pgfpathcurveto{\pgfqpoint{4.047273in}{1.439024in}}{\pgfqpoint{4.050107in}{1.432182in}}{\pgfqpoint{4.055151in}{1.427138in}}%
\pgfpathcurveto{\pgfqpoint{4.060195in}{1.422095in}}{\pgfqpoint{4.067036in}{1.419261in}}{\pgfqpoint{4.074169in}{1.419261in}}%
\pgfpathclose%
\pgfusepath{stroke,fill}%
\end{pgfscope}%
\begin{pgfscope}%
\pgfpathrectangle{\pgfqpoint{2.867647in}{0.500000in}}{\pgfqpoint{1.764706in}{1.700000in}}%
\pgfusepath{clip}%
\pgfsetbuttcap%
\pgfsetroundjoin%
\definecolor{currentfill}{rgb}{0.969803,0.809811,0.702523}%
\pgfsetfillcolor{currentfill}%
\pgfsetlinewidth{0.311001pt}%
\definecolor{currentstroke}{rgb}{1.000000,1.000000,1.000000}%
\pgfsetstrokecolor{currentstroke}%
\pgfsetdash{}{0pt}%
\pgfpathmoveto{\pgfqpoint{4.047562in}{1.046258in}}%
\pgfpathcurveto{\pgfqpoint{4.054695in}{1.046258in}}{\pgfqpoint{4.061537in}{1.049092in}}{\pgfqpoint{4.066580in}{1.054136in}}%
\pgfpathcurveto{\pgfqpoint{4.071624in}{1.059179in}}{\pgfqpoint{4.074458in}{1.066021in}}{\pgfqpoint{4.074458in}{1.073154in}}%
\pgfpathcurveto{\pgfqpoint{4.074458in}{1.080287in}}{\pgfqpoint{4.071624in}{1.087128in}}{\pgfqpoint{4.066580in}{1.092172in}}%
\pgfpathcurveto{\pgfqpoint{4.061537in}{1.097216in}}{\pgfqpoint{4.054695in}{1.100050in}}{\pgfqpoint{4.047562in}{1.100050in}}%
\pgfpathcurveto{\pgfqpoint{4.040429in}{1.100050in}}{\pgfqpoint{4.033588in}{1.097216in}}{\pgfqpoint{4.028544in}{1.092172in}}%
\pgfpathcurveto{\pgfqpoint{4.023500in}{1.087128in}}{\pgfqpoint{4.020666in}{1.080287in}}{\pgfqpoint{4.020666in}{1.073154in}}%
\pgfpathcurveto{\pgfqpoint{4.020666in}{1.066021in}}{\pgfqpoint{4.023500in}{1.059179in}}{\pgfqpoint{4.028544in}{1.054136in}}%
\pgfpathcurveto{\pgfqpoint{4.033588in}{1.049092in}}{\pgfqpoint{4.040429in}{1.046258in}}{\pgfqpoint{4.047562in}{1.046258in}}%
\pgfpathclose%
\pgfusepath{stroke,fill}%
\end{pgfscope}%
\begin{pgfscope}%
\pgfpathrectangle{\pgfqpoint{2.867647in}{0.500000in}}{\pgfqpoint{1.764706in}{1.700000in}}%
\pgfusepath{clip}%
\pgfsetbuttcap%
\pgfsetroundjoin%
\definecolor{currentfill}{rgb}{0.976961,0.885681,0.814303}%
\pgfsetfillcolor{currentfill}%
\pgfsetlinewidth{0.311001pt}%
\definecolor{currentstroke}{rgb}{1.000000,1.000000,1.000000}%
\pgfsetstrokecolor{currentstroke}%
\pgfsetdash{}{0pt}%
\pgfpathmoveto{\pgfqpoint{4.138852in}{1.612513in}}%
\pgfpathcurveto{\pgfqpoint{4.145985in}{1.612513in}}{\pgfqpoint{4.152826in}{1.615347in}}{\pgfqpoint{4.157870in}{1.620390in}}%
\pgfpathcurveto{\pgfqpoint{4.162914in}{1.625434in}}{\pgfqpoint{4.165748in}{1.632276in}}{\pgfqpoint{4.165748in}{1.639408in}}%
\pgfpathcurveto{\pgfqpoint{4.165748in}{1.646541in}}{\pgfqpoint{4.162914in}{1.653383in}}{\pgfqpoint{4.157870in}{1.658426in}}%
\pgfpathcurveto{\pgfqpoint{4.152826in}{1.663470in}}{\pgfqpoint{4.145985in}{1.666304in}}{\pgfqpoint{4.138852in}{1.666304in}}%
\pgfpathcurveto{\pgfqpoint{4.131719in}{1.666304in}}{\pgfqpoint{4.124877in}{1.663470in}}{\pgfqpoint{4.119834in}{1.658426in}}%
\pgfpathcurveto{\pgfqpoint{4.114790in}{1.653383in}}{\pgfqpoint{4.111956in}{1.646541in}}{\pgfqpoint{4.111956in}{1.639408in}}%
\pgfpathcurveto{\pgfqpoint{4.111956in}{1.632276in}}{\pgfqpoint{4.114790in}{1.625434in}}{\pgfqpoint{4.119834in}{1.620390in}}%
\pgfpathcurveto{\pgfqpoint{4.124877in}{1.615347in}}{\pgfqpoint{4.131719in}{1.612513in}}{\pgfqpoint{4.138852in}{1.612513in}}%
\pgfpathclose%
\pgfusepath{stroke,fill}%
\end{pgfscope}%
\begin{pgfscope}%
\pgfpathrectangle{\pgfqpoint{2.867647in}{0.500000in}}{\pgfqpoint{1.764706in}{1.700000in}}%
\pgfusepath{clip}%
\pgfsetbuttcap%
\pgfsetroundjoin%
\definecolor{currentfill}{rgb}{0.976287,0.879862,0.805788}%
\pgfsetfillcolor{currentfill}%
\pgfsetlinewidth{0.311001pt}%
\definecolor{currentstroke}{rgb}{1.000000,1.000000,1.000000}%
\pgfsetstrokecolor{currentstroke}%
\pgfsetdash{}{0pt}%
\pgfpathmoveto{\pgfqpoint{4.095651in}{1.622429in}}%
\pgfpathcurveto{\pgfqpoint{4.102784in}{1.622429in}}{\pgfqpoint{4.109625in}{1.625263in}}{\pgfqpoint{4.114669in}{1.630307in}}%
\pgfpathcurveto{\pgfqpoint{4.119712in}{1.635351in}}{\pgfqpoint{4.122546in}{1.642192in}}{\pgfqpoint{4.122546in}{1.649325in}}%
\pgfpathcurveto{\pgfqpoint{4.122546in}{1.656458in}}{\pgfqpoint{4.119712in}{1.663300in}}{\pgfqpoint{4.114669in}{1.668343in}}%
\pgfpathcurveto{\pgfqpoint{4.109625in}{1.673387in}}{\pgfqpoint{4.102784in}{1.676221in}}{\pgfqpoint{4.095651in}{1.676221in}}%
\pgfpathcurveto{\pgfqpoint{4.088518in}{1.676221in}}{\pgfqpoint{4.081676in}{1.673387in}}{\pgfqpoint{4.076633in}{1.668343in}}%
\pgfpathcurveto{\pgfqpoint{4.071589in}{1.663300in}}{\pgfqpoint{4.068755in}{1.656458in}}{\pgfqpoint{4.068755in}{1.649325in}}%
\pgfpathcurveto{\pgfqpoint{4.068755in}{1.642192in}}{\pgfqpoint{4.071589in}{1.635351in}}{\pgfqpoint{4.076633in}{1.630307in}}%
\pgfpathcurveto{\pgfqpoint{4.081676in}{1.625263in}}{\pgfqpoint{4.088518in}{1.622429in}}{\pgfqpoint{4.095651in}{1.622429in}}%
\pgfpathclose%
\pgfusepath{stroke,fill}%
\end{pgfscope}%
\begin{pgfscope}%
\pgfpathrectangle{\pgfqpoint{2.867647in}{0.500000in}}{\pgfqpoint{1.764706in}{1.700000in}}%
\pgfusepath{clip}%
\pgfsetbuttcap%
\pgfsetroundjoin%
\definecolor{currentfill}{rgb}{0.965753,0.732351,0.592427}%
\pgfsetfillcolor{currentfill}%
\pgfsetlinewidth{0.311001pt}%
\definecolor{currentstroke}{rgb}{1.000000,1.000000,1.000000}%
\pgfsetstrokecolor{currentstroke}%
\pgfsetdash{}{0pt}%
\pgfpathmoveto{\pgfqpoint{4.246590in}{1.057566in}}%
\pgfpathcurveto{\pgfqpoint{4.253723in}{1.057566in}}{\pgfqpoint{4.260565in}{1.060399in}}{\pgfqpoint{4.265609in}{1.065443in}}%
\pgfpathcurveto{\pgfqpoint{4.270652in}{1.070487in}}{\pgfqpoint{4.273486in}{1.077328in}}{\pgfqpoint{4.273486in}{1.084461in}}%
\pgfpathcurveto{\pgfqpoint{4.273486in}{1.091594in}}{\pgfqpoint{4.270652in}{1.098436in}}{\pgfqpoint{4.265609in}{1.103479in}}%
\pgfpathcurveto{\pgfqpoint{4.260565in}{1.108523in}}{\pgfqpoint{4.253723in}{1.111357in}}{\pgfqpoint{4.246590in}{1.111357in}}%
\pgfpathcurveto{\pgfqpoint{4.239458in}{1.111357in}}{\pgfqpoint{4.232616in}{1.108523in}}{\pgfqpoint{4.227572in}{1.103479in}}%
\pgfpathcurveto{\pgfqpoint{4.222529in}{1.098436in}}{\pgfqpoint{4.219695in}{1.091594in}}{\pgfqpoint{4.219695in}{1.084461in}}%
\pgfpathcurveto{\pgfqpoint{4.219695in}{1.077328in}}{\pgfqpoint{4.222529in}{1.070487in}}{\pgfqpoint{4.227572in}{1.065443in}}%
\pgfpathcurveto{\pgfqpoint{4.232616in}{1.060399in}}{\pgfqpoint{4.239458in}{1.057566in}}{\pgfqpoint{4.246590in}{1.057566in}}%
\pgfpathclose%
\pgfusepath{stroke,fill}%
\end{pgfscope}%
\begin{pgfscope}%
\pgfpathrectangle{\pgfqpoint{2.867647in}{0.500000in}}{\pgfqpoint{1.764706in}{1.700000in}}%
\pgfusepath{clip}%
\pgfsetbuttcap%
\pgfsetroundjoin%
\definecolor{currentfill}{rgb}{0.971694,0.833208,0.737161}%
\pgfsetfillcolor{currentfill}%
\pgfsetlinewidth{0.311001pt}%
\definecolor{currentstroke}{rgb}{1.000000,1.000000,1.000000}%
\pgfsetstrokecolor{currentstroke}%
\pgfsetdash{}{0pt}%
\pgfpathmoveto{\pgfqpoint{4.085686in}{1.484124in}}%
\pgfpathcurveto{\pgfqpoint{4.092818in}{1.484124in}}{\pgfqpoint{4.099660in}{1.486958in}}{\pgfqpoint{4.104704in}{1.492002in}}%
\pgfpathcurveto{\pgfqpoint{4.109747in}{1.497045in}}{\pgfqpoint{4.112581in}{1.503887in}}{\pgfqpoint{4.112581in}{1.511020in}}%
\pgfpathcurveto{\pgfqpoint{4.112581in}{1.518152in}}{\pgfqpoint{4.109747in}{1.524994in}}{\pgfqpoint{4.104704in}{1.530038in}}%
\pgfpathcurveto{\pgfqpoint{4.099660in}{1.535081in}}{\pgfqpoint{4.092818in}{1.537915in}}{\pgfqpoint{4.085686in}{1.537915in}}%
\pgfpathcurveto{\pgfqpoint{4.078553in}{1.537915in}}{\pgfqpoint{4.071711in}{1.535081in}}{\pgfqpoint{4.066667in}{1.530038in}}%
\pgfpathcurveto{\pgfqpoint{4.061624in}{1.524994in}}{\pgfqpoint{4.058790in}{1.518152in}}{\pgfqpoint{4.058790in}{1.511020in}}%
\pgfpathcurveto{\pgfqpoint{4.058790in}{1.503887in}}{\pgfqpoint{4.061624in}{1.497045in}}{\pgfqpoint{4.066667in}{1.492002in}}%
\pgfpathcurveto{\pgfqpoint{4.071711in}{1.486958in}}{\pgfqpoint{4.078553in}{1.484124in}}{\pgfqpoint{4.085686in}{1.484124in}}%
\pgfpathclose%
\pgfusepath{stroke,fill}%
\end{pgfscope}%
\begin{pgfscope}%
\pgfpathrectangle{\pgfqpoint{2.867647in}{0.500000in}}{\pgfqpoint{1.764706in}{1.700000in}}%
\pgfusepath{clip}%
\pgfsetbuttcap%
\pgfsetroundjoin%
\definecolor{currentfill}{rgb}{0.972726,0.844889,0.754401}%
\pgfsetfillcolor{currentfill}%
\pgfsetlinewidth{0.311001pt}%
\definecolor{currentstroke}{rgb}{1.000000,1.000000,1.000000}%
\pgfsetstrokecolor{currentstroke}%
\pgfsetdash{}{0pt}%
\pgfpathmoveto{\pgfqpoint{4.200934in}{1.060767in}}%
\pgfpathcurveto{\pgfqpoint{4.208067in}{1.060767in}}{\pgfqpoint{4.214909in}{1.063601in}}{\pgfqpoint{4.219952in}{1.068645in}}%
\pgfpathcurveto{\pgfqpoint{4.224996in}{1.073688in}}{\pgfqpoint{4.227830in}{1.080530in}}{\pgfqpoint{4.227830in}{1.087663in}}%
\pgfpathcurveto{\pgfqpoint{4.227830in}{1.094796in}}{\pgfqpoint{4.224996in}{1.101637in}}{\pgfqpoint{4.219952in}{1.106681in}}%
\pgfpathcurveto{\pgfqpoint{4.214909in}{1.111725in}}{\pgfqpoint{4.208067in}{1.114559in}}{\pgfqpoint{4.200934in}{1.114559in}}%
\pgfpathcurveto{\pgfqpoint{4.193802in}{1.114559in}}{\pgfqpoint{4.186960in}{1.111725in}}{\pgfqpoint{4.181916in}{1.106681in}}%
\pgfpathcurveto{\pgfqpoint{4.176873in}{1.101637in}}{\pgfqpoint{4.174039in}{1.094796in}}{\pgfqpoint{4.174039in}{1.087663in}}%
\pgfpathcurveto{\pgfqpoint{4.174039in}{1.080530in}}{\pgfqpoint{4.176873in}{1.073688in}}{\pgfqpoint{4.181916in}{1.068645in}}%
\pgfpathcurveto{\pgfqpoint{4.186960in}{1.063601in}}{\pgfqpoint{4.193802in}{1.060767in}}{\pgfqpoint{4.200934in}{1.060767in}}%
\pgfpathclose%
\pgfusepath{stroke,fill}%
\end{pgfscope}%
\begin{pgfscope}%
\pgfpathrectangle{\pgfqpoint{2.867647in}{0.500000in}}{\pgfqpoint{1.764706in}{1.700000in}}%
\pgfusepath{clip}%
\pgfsetbuttcap%
\pgfsetroundjoin%
\definecolor{currentfill}{rgb}{0.979891,0.908948,0.848279}%
\pgfsetfillcolor{currentfill}%
\pgfsetlinewidth{0.311001pt}%
\definecolor{currentstroke}{rgb}{1.000000,1.000000,1.000000}%
\pgfsetstrokecolor{currentstroke}%
\pgfsetdash{}{0pt}%
\pgfpathmoveto{\pgfqpoint{4.212763in}{1.210845in}}%
\pgfpathcurveto{\pgfqpoint{4.219896in}{1.210845in}}{\pgfqpoint{4.226737in}{1.213679in}}{\pgfqpoint{4.231781in}{1.218723in}}%
\pgfpathcurveto{\pgfqpoint{4.236825in}{1.223766in}}{\pgfqpoint{4.239659in}{1.230608in}}{\pgfqpoint{4.239659in}{1.237741in}}%
\pgfpathcurveto{\pgfqpoint{4.239659in}{1.244874in}}{\pgfqpoint{4.236825in}{1.251715in}}{\pgfqpoint{4.231781in}{1.256759in}}%
\pgfpathcurveto{\pgfqpoint{4.226737in}{1.261803in}}{\pgfqpoint{4.219896in}{1.264637in}}{\pgfqpoint{4.212763in}{1.264637in}}%
\pgfpathcurveto{\pgfqpoint{4.205630in}{1.264637in}}{\pgfqpoint{4.198788in}{1.261803in}}{\pgfqpoint{4.193745in}{1.256759in}}%
\pgfpathcurveto{\pgfqpoint{4.188701in}{1.251715in}}{\pgfqpoint{4.185867in}{1.244874in}}{\pgfqpoint{4.185867in}{1.237741in}}%
\pgfpathcurveto{\pgfqpoint{4.185867in}{1.230608in}}{\pgfqpoint{4.188701in}{1.223766in}}{\pgfqpoint{4.193745in}{1.218723in}}%
\pgfpathcurveto{\pgfqpoint{4.198788in}{1.213679in}}{\pgfqpoint{4.205630in}{1.210845in}}{\pgfqpoint{4.212763in}{1.210845in}}%
\pgfpathclose%
\pgfusepath{stroke,fill}%
\end{pgfscope}%
\begin{pgfscope}%
\pgfpathrectangle{\pgfqpoint{2.867647in}{0.500000in}}{\pgfqpoint{1.764706in}{1.700000in}}%
\pgfusepath{clip}%
\pgfsetbuttcap%
\pgfsetroundjoin%
\definecolor{currentfill}{rgb}{0.979891,0.908948,0.848279}%
\pgfsetfillcolor{currentfill}%
\pgfsetlinewidth{0.311001pt}%
\definecolor{currentstroke}{rgb}{1.000000,1.000000,1.000000}%
\pgfsetstrokecolor{currentstroke}%
\pgfsetdash{}{0pt}%
\pgfpathmoveto{\pgfqpoint{4.153855in}{1.453668in}}%
\pgfpathcurveto{\pgfqpoint{4.160987in}{1.453668in}}{\pgfqpoint{4.167829in}{1.456502in}}{\pgfqpoint{4.172873in}{1.461546in}}%
\pgfpathcurveto{\pgfqpoint{4.177916in}{1.466589in}}{\pgfqpoint{4.180750in}{1.473431in}}{\pgfqpoint{4.180750in}{1.480564in}}%
\pgfpathcurveto{\pgfqpoint{4.180750in}{1.487697in}}{\pgfqpoint{4.177916in}{1.494538in}}{\pgfqpoint{4.172873in}{1.499582in}}%
\pgfpathcurveto{\pgfqpoint{4.167829in}{1.504626in}}{\pgfqpoint{4.160987in}{1.507460in}}{\pgfqpoint{4.153855in}{1.507460in}}%
\pgfpathcurveto{\pgfqpoint{4.146722in}{1.507460in}}{\pgfqpoint{4.139880in}{1.504626in}}{\pgfqpoint{4.134836in}{1.499582in}}%
\pgfpathcurveto{\pgfqpoint{4.129793in}{1.494538in}}{\pgfqpoint{4.126959in}{1.487697in}}{\pgfqpoint{4.126959in}{1.480564in}}%
\pgfpathcurveto{\pgfqpoint{4.126959in}{1.473431in}}{\pgfqpoint{4.129793in}{1.466589in}}{\pgfqpoint{4.134836in}{1.461546in}}%
\pgfpathcurveto{\pgfqpoint{4.139880in}{1.456502in}}{\pgfqpoint{4.146722in}{1.453668in}}{\pgfqpoint{4.153855in}{1.453668in}}%
\pgfpathclose%
\pgfusepath{stroke,fill}%
\end{pgfscope}%
\begin{pgfscope}%
\pgfpathrectangle{\pgfqpoint{2.867647in}{0.500000in}}{\pgfqpoint{1.764706in}{1.700000in}}%
\pgfusepath{clip}%
\pgfsetbuttcap%
\pgfsetroundjoin%
\definecolor{currentfill}{rgb}{0.969359,0.803954,0.693832}%
\pgfsetfillcolor{currentfill}%
\pgfsetlinewidth{0.311001pt}%
\definecolor{currentstroke}{rgb}{1.000000,1.000000,1.000000}%
\pgfsetstrokecolor{currentstroke}%
\pgfsetdash{}{0pt}%
\pgfpathmoveto{\pgfqpoint{4.104240in}{1.349633in}}%
\pgfpathcurveto{\pgfqpoint{4.111373in}{1.349633in}}{\pgfqpoint{4.118215in}{1.352467in}}{\pgfqpoint{4.123259in}{1.357511in}}%
\pgfpathcurveto{\pgfqpoint{4.128302in}{1.362555in}}{\pgfqpoint{4.131136in}{1.369396in}}{\pgfqpoint{4.131136in}{1.376529in}}%
\pgfpathcurveto{\pgfqpoint{4.131136in}{1.383662in}}{\pgfqpoint{4.128302in}{1.390503in}}{\pgfqpoint{4.123259in}{1.395547in}}%
\pgfpathcurveto{\pgfqpoint{4.118215in}{1.400591in}}{\pgfqpoint{4.111373in}{1.403425in}}{\pgfqpoint{4.104240in}{1.403425in}}%
\pgfpathcurveto{\pgfqpoint{4.097108in}{1.403425in}}{\pgfqpoint{4.090266in}{1.400591in}}{\pgfqpoint{4.085222in}{1.395547in}}%
\pgfpathcurveto{\pgfqpoint{4.080179in}{1.390503in}}{\pgfqpoint{4.077345in}{1.383662in}}{\pgfqpoint{4.077345in}{1.376529in}}%
\pgfpathcurveto{\pgfqpoint{4.077345in}{1.369396in}}{\pgfqpoint{4.080179in}{1.362555in}}{\pgfqpoint{4.085222in}{1.357511in}}%
\pgfpathcurveto{\pgfqpoint{4.090266in}{1.352467in}}{\pgfqpoint{4.097108in}{1.349633in}}{\pgfqpoint{4.104240in}{1.349633in}}%
\pgfpathclose%
\pgfusepath{stroke,fill}%
\end{pgfscope}%
\begin{pgfscope}%
\pgfpathrectangle{\pgfqpoint{2.867647in}{0.500000in}}{\pgfqpoint{1.764706in}{1.700000in}}%
\pgfusepath{clip}%
\pgfsetbuttcap%
\pgfsetroundjoin%
\definecolor{currentfill}{rgb}{0.965753,0.732351,0.592427}%
\pgfsetfillcolor{currentfill}%
\pgfsetlinewidth{0.311001pt}%
\definecolor{currentstroke}{rgb}{1.000000,1.000000,1.000000}%
\pgfsetstrokecolor{currentstroke}%
\pgfsetdash{}{0pt}%
\pgfpathmoveto{\pgfqpoint{4.242491in}{1.047965in}}%
\pgfpathcurveto{\pgfqpoint{4.249624in}{1.047965in}}{\pgfqpoint{4.256465in}{1.050798in}}{\pgfqpoint{4.261509in}{1.055842in}}%
\pgfpathcurveto{\pgfqpoint{4.266553in}{1.060886in}}{\pgfqpoint{4.269387in}{1.067727in}}{\pgfqpoint{4.269387in}{1.074860in}}%
\pgfpathcurveto{\pgfqpoint{4.269387in}{1.081993in}}{\pgfqpoint{4.266553in}{1.088835in}}{\pgfqpoint{4.261509in}{1.093878in}}%
\pgfpathcurveto{\pgfqpoint{4.256465in}{1.098922in}}{\pgfqpoint{4.249624in}{1.101756in}}{\pgfqpoint{4.242491in}{1.101756in}}%
\pgfpathcurveto{\pgfqpoint{4.235358in}{1.101756in}}{\pgfqpoint{4.228517in}{1.098922in}}{\pgfqpoint{4.223473in}{1.093878in}}%
\pgfpathcurveto{\pgfqpoint{4.218429in}{1.088835in}}{\pgfqpoint{4.215595in}{1.081993in}}{\pgfqpoint{4.215595in}{1.074860in}}%
\pgfpathcurveto{\pgfqpoint{4.215595in}{1.067727in}}{\pgfqpoint{4.218429in}{1.060886in}}{\pgfqpoint{4.223473in}{1.055842in}}%
\pgfpathcurveto{\pgfqpoint{4.228517in}{1.050798in}}{\pgfqpoint{4.235358in}{1.047965in}}{\pgfqpoint{4.242491in}{1.047965in}}%
\pgfpathclose%
\pgfusepath{stroke,fill}%
\end{pgfscope}%
\begin{pgfscope}%
\pgfpathrectangle{\pgfqpoint{2.867647in}{0.500000in}}{\pgfqpoint{1.764706in}{1.700000in}}%
\pgfusepath{clip}%
\pgfsetbuttcap%
\pgfsetroundjoin%
\definecolor{currentfill}{rgb}{0.976961,0.885681,0.814303}%
\pgfsetfillcolor{currentfill}%
\pgfsetlinewidth{0.311001pt}%
\definecolor{currentstroke}{rgb}{1.000000,1.000000,1.000000}%
\pgfsetstrokecolor{currentstroke}%
\pgfsetdash{}{0pt}%
\pgfpathmoveto{\pgfqpoint{4.143092in}{1.318728in}}%
\pgfpathcurveto{\pgfqpoint{4.150225in}{1.318728in}}{\pgfqpoint{4.157066in}{1.321562in}}{\pgfqpoint{4.162110in}{1.326606in}}%
\pgfpathcurveto{\pgfqpoint{4.167154in}{1.331649in}}{\pgfqpoint{4.169987in}{1.338491in}}{\pgfqpoint{4.169987in}{1.345624in}}%
\pgfpathcurveto{\pgfqpoint{4.169987in}{1.352757in}}{\pgfqpoint{4.167154in}{1.359598in}}{\pgfqpoint{4.162110in}{1.364642in}}%
\pgfpathcurveto{\pgfqpoint{4.157066in}{1.369686in}}{\pgfqpoint{4.150225in}{1.372520in}}{\pgfqpoint{4.143092in}{1.372520in}}%
\pgfpathcurveto{\pgfqpoint{4.135959in}{1.372520in}}{\pgfqpoint{4.129117in}{1.369686in}}{\pgfqpoint{4.124074in}{1.364642in}}%
\pgfpathcurveto{\pgfqpoint{4.119030in}{1.359598in}}{\pgfqpoint{4.116196in}{1.352757in}}{\pgfqpoint{4.116196in}{1.345624in}}%
\pgfpathcurveto{\pgfqpoint{4.116196in}{1.338491in}}{\pgfqpoint{4.119030in}{1.331649in}}{\pgfqpoint{4.124074in}{1.326606in}}%
\pgfpathcurveto{\pgfqpoint{4.129117in}{1.321562in}}{\pgfqpoint{4.135959in}{1.318728in}}{\pgfqpoint{4.143092in}{1.318728in}}%
\pgfpathclose%
\pgfusepath{stroke,fill}%
\end{pgfscope}%
\begin{pgfscope}%
\pgfpathrectangle{\pgfqpoint{2.867647in}{0.500000in}}{\pgfqpoint{1.764706in}{1.700000in}}%
\pgfusepath{clip}%
\pgfsetbuttcap%
\pgfsetroundjoin%
\definecolor{currentfill}{rgb}{0.966120,0.744512,0.608720}%
\pgfsetfillcolor{currentfill}%
\pgfsetlinewidth{0.311001pt}%
\definecolor{currentstroke}{rgb}{1.000000,1.000000,1.000000}%
\pgfsetstrokecolor{currentstroke}%
\pgfsetdash{}{0pt}%
\pgfpathmoveto{\pgfqpoint{4.056381in}{0.909177in}}%
\pgfpathcurveto{\pgfqpoint{4.063514in}{0.909177in}}{\pgfqpoint{4.070355in}{0.912011in}}{\pgfqpoint{4.075399in}{0.917055in}}%
\pgfpathcurveto{\pgfqpoint{4.080443in}{0.922098in}}{\pgfqpoint{4.083276in}{0.928940in}}{\pgfqpoint{4.083276in}{0.936073in}}%
\pgfpathcurveto{\pgfqpoint{4.083276in}{0.943206in}}{\pgfqpoint{4.080443in}{0.950047in}}{\pgfqpoint{4.075399in}{0.955091in}}%
\pgfpathcurveto{\pgfqpoint{4.070355in}{0.960135in}}{\pgfqpoint{4.063514in}{0.962969in}}{\pgfqpoint{4.056381in}{0.962969in}}%
\pgfpathcurveto{\pgfqpoint{4.049248in}{0.962969in}}{\pgfqpoint{4.042406in}{0.960135in}}{\pgfqpoint{4.037363in}{0.955091in}}%
\pgfpathcurveto{\pgfqpoint{4.032319in}{0.950047in}}{\pgfqpoint{4.029485in}{0.943206in}}{\pgfqpoint{4.029485in}{0.936073in}}%
\pgfpathcurveto{\pgfqpoint{4.029485in}{0.928940in}}{\pgfqpoint{4.032319in}{0.922098in}}{\pgfqpoint{4.037363in}{0.917055in}}%
\pgfpathcurveto{\pgfqpoint{4.042406in}{0.912011in}}{\pgfqpoint{4.049248in}{0.909177in}}{\pgfqpoint{4.056381in}{0.909177in}}%
\pgfpathclose%
\pgfusepath{stroke,fill}%
\end{pgfscope}%
\begin{pgfscope}%
\pgfpathrectangle{\pgfqpoint{2.867647in}{0.500000in}}{\pgfqpoint{1.764706in}{1.700000in}}%
\pgfusepath{clip}%
\pgfsetbuttcap%
\pgfsetroundjoin%
\definecolor{currentfill}{rgb}{0.894903,0.217856,0.253144}%
\pgfsetfillcolor{currentfill}%
\pgfsetlinewidth{0.311001pt}%
\definecolor{currentstroke}{rgb}{1.000000,1.000000,1.000000}%
\pgfsetstrokecolor{currentstroke}%
\pgfsetdash{}{0pt}%
\pgfpathmoveto{\pgfqpoint{3.762871in}{1.778546in}}%
\pgfpathcurveto{\pgfqpoint{3.770004in}{1.778546in}}{\pgfqpoint{3.776846in}{1.781380in}}{\pgfqpoint{3.781890in}{1.786423in}}%
\pgfpathcurveto{\pgfqpoint{3.786933in}{1.791467in}}{\pgfqpoint{3.789767in}{1.798309in}}{\pgfqpoint{3.789767in}{1.805441in}}%
\pgfpathcurveto{\pgfqpoint{3.789767in}{1.812574in}}{\pgfqpoint{3.786933in}{1.819416in}}{\pgfqpoint{3.781890in}{1.824460in}}%
\pgfpathcurveto{\pgfqpoint{3.776846in}{1.829503in}}{\pgfqpoint{3.770004in}{1.832337in}}{\pgfqpoint{3.762871in}{1.832337in}}%
\pgfpathcurveto{\pgfqpoint{3.755739in}{1.832337in}}{\pgfqpoint{3.748897in}{1.829503in}}{\pgfqpoint{3.743853in}{1.824460in}}%
\pgfpathcurveto{\pgfqpoint{3.738810in}{1.819416in}}{\pgfqpoint{3.735976in}{1.812574in}}{\pgfqpoint{3.735976in}{1.805441in}}%
\pgfpathcurveto{\pgfqpoint{3.735976in}{1.798309in}}{\pgfqpoint{3.738810in}{1.791467in}}{\pgfqpoint{3.743853in}{1.786423in}}%
\pgfpathcurveto{\pgfqpoint{3.748897in}{1.781380in}}{\pgfqpoint{3.755739in}{1.778546in}}{\pgfqpoint{3.762871in}{1.778546in}}%
\pgfpathclose%
\pgfusepath{stroke,fill}%
\end{pgfscope}%
\begin{pgfscope}%
\pgfpathrectangle{\pgfqpoint{2.867647in}{0.500000in}}{\pgfqpoint{1.764706in}{1.700000in}}%
\pgfusepath{clip}%
\pgfsetbuttcap%
\pgfsetroundjoin%
\definecolor{currentfill}{rgb}{0.966120,0.744512,0.608720}%
\pgfsetfillcolor{currentfill}%
\pgfsetlinewidth{0.311001pt}%
\definecolor{currentstroke}{rgb}{1.000000,1.000000,1.000000}%
\pgfsetstrokecolor{currentstroke}%
\pgfsetdash{}{0pt}%
\pgfpathmoveto{\pgfqpoint{4.021282in}{1.720324in}}%
\pgfpathcurveto{\pgfqpoint{4.028415in}{1.720324in}}{\pgfqpoint{4.035257in}{1.723157in}}{\pgfqpoint{4.040300in}{1.728201in}}%
\pgfpathcurveto{\pgfqpoint{4.045344in}{1.733245in}}{\pgfqpoint{4.048178in}{1.740086in}}{\pgfqpoint{4.048178in}{1.747219in}}%
\pgfpathcurveto{\pgfqpoint{4.048178in}{1.754352in}}{\pgfqpoint{4.045344in}{1.761194in}}{\pgfqpoint{4.040300in}{1.766237in}}%
\pgfpathcurveto{\pgfqpoint{4.035257in}{1.771281in}}{\pgfqpoint{4.028415in}{1.774115in}}{\pgfqpoint{4.021282in}{1.774115in}}%
\pgfpathcurveto{\pgfqpoint{4.014149in}{1.774115in}}{\pgfqpoint{4.007308in}{1.771281in}}{\pgfqpoint{4.002264in}{1.766237in}}%
\pgfpathcurveto{\pgfqpoint{3.997220in}{1.761194in}}{\pgfqpoint{3.994386in}{1.754352in}}{\pgfqpoint{3.994386in}{1.747219in}}%
\pgfpathcurveto{\pgfqpoint{3.994386in}{1.740086in}}{\pgfqpoint{3.997220in}{1.733245in}}{\pgfqpoint{4.002264in}{1.728201in}}%
\pgfpathcurveto{\pgfqpoint{4.007308in}{1.723157in}}{\pgfqpoint{4.014149in}{1.720324in}}{\pgfqpoint{4.021282in}{1.720324in}}%
\pgfpathclose%
\pgfusepath{stroke,fill}%
\end{pgfscope}%
\begin{pgfscope}%
\pgfpathrectangle{\pgfqpoint{2.867647in}{0.500000in}}{\pgfqpoint{1.764706in}{1.700000in}}%
\pgfusepath{clip}%
\pgfsetbuttcap%
\pgfsetroundjoin%
\definecolor{currentfill}{rgb}{0.965042,0.701564,0.552889}%
\pgfsetfillcolor{currentfill}%
\pgfsetlinewidth{0.311001pt}%
\definecolor{currentstroke}{rgb}{1.000000,1.000000,1.000000}%
\pgfsetstrokecolor{currentstroke}%
\pgfsetdash{}{0pt}%
\pgfpathmoveto{\pgfqpoint{4.230850in}{1.010230in}}%
\pgfpathcurveto{\pgfqpoint{4.237983in}{1.010230in}}{\pgfqpoint{4.244824in}{1.013064in}}{\pgfqpoint{4.249868in}{1.018108in}}%
\pgfpathcurveto{\pgfqpoint{4.254912in}{1.023151in}}{\pgfqpoint{4.257745in}{1.029993in}}{\pgfqpoint{4.257745in}{1.037126in}}%
\pgfpathcurveto{\pgfqpoint{4.257745in}{1.044259in}}{\pgfqpoint{4.254912in}{1.051100in}}{\pgfqpoint{4.249868in}{1.056144in}}%
\pgfpathcurveto{\pgfqpoint{4.244824in}{1.061187in}}{\pgfqpoint{4.237983in}{1.064021in}}{\pgfqpoint{4.230850in}{1.064021in}}%
\pgfpathcurveto{\pgfqpoint{4.223717in}{1.064021in}}{\pgfqpoint{4.216875in}{1.061187in}}{\pgfqpoint{4.211832in}{1.056144in}}%
\pgfpathcurveto{\pgfqpoint{4.206788in}{1.051100in}}{\pgfqpoint{4.203954in}{1.044259in}}{\pgfqpoint{4.203954in}{1.037126in}}%
\pgfpathcurveto{\pgfqpoint{4.203954in}{1.029993in}}{\pgfqpoint{4.206788in}{1.023151in}}{\pgfqpoint{4.211832in}{1.018108in}}%
\pgfpathcurveto{\pgfqpoint{4.216875in}{1.013064in}}{\pgfqpoint{4.223717in}{1.010230in}}{\pgfqpoint{4.230850in}{1.010230in}}%
\pgfpathclose%
\pgfusepath{stroke,fill}%
\end{pgfscope}%
\begin{pgfscope}%
\pgfpathrectangle{\pgfqpoint{2.867647in}{0.500000in}}{\pgfqpoint{1.764706in}{1.700000in}}%
\pgfusepath{clip}%
\pgfsetbuttcap%
\pgfsetroundjoin%
\definecolor{currentfill}{rgb}{0.976961,0.885681,0.814303}%
\pgfsetfillcolor{currentfill}%
\pgfsetlinewidth{0.311001pt}%
\definecolor{currentstroke}{rgb}{1.000000,1.000000,1.000000}%
\pgfsetstrokecolor{currentstroke}%
\pgfsetdash{}{0pt}%
\pgfpathmoveto{\pgfqpoint{4.144826in}{1.333771in}}%
\pgfpathcurveto{\pgfqpoint{4.151958in}{1.333771in}}{\pgfqpoint{4.158800in}{1.336605in}}{\pgfqpoint{4.163844in}{1.341649in}}%
\pgfpathcurveto{\pgfqpoint{4.168887in}{1.346692in}}{\pgfqpoint{4.171721in}{1.353534in}}{\pgfqpoint{4.171721in}{1.360667in}}%
\pgfpathcurveto{\pgfqpoint{4.171721in}{1.367800in}}{\pgfqpoint{4.168887in}{1.374641in}}{\pgfqpoint{4.163844in}{1.379685in}}%
\pgfpathcurveto{\pgfqpoint{4.158800in}{1.384729in}}{\pgfqpoint{4.151958in}{1.387562in}}{\pgfqpoint{4.144826in}{1.387562in}}%
\pgfpathcurveto{\pgfqpoint{4.137693in}{1.387562in}}{\pgfqpoint{4.130851in}{1.384729in}}{\pgfqpoint{4.125807in}{1.379685in}}%
\pgfpathcurveto{\pgfqpoint{4.120764in}{1.374641in}}{\pgfqpoint{4.117930in}{1.367800in}}{\pgfqpoint{4.117930in}{1.360667in}}%
\pgfpathcurveto{\pgfqpoint{4.117930in}{1.353534in}}{\pgfqpoint{4.120764in}{1.346692in}}{\pgfqpoint{4.125807in}{1.341649in}}%
\pgfpathcurveto{\pgfqpoint{4.130851in}{1.336605in}}{\pgfqpoint{4.137693in}{1.333771in}}{\pgfqpoint{4.144826in}{1.333771in}}%
\pgfpathclose%
\pgfusepath{stroke,fill}%
\end{pgfscope}%
\begin{pgfscope}%
\pgfpathrectangle{\pgfqpoint{2.867647in}{0.500000in}}{\pgfqpoint{1.764706in}{1.700000in}}%
\pgfusepath{clip}%
\pgfsetbuttcap%
\pgfsetroundjoin%
\definecolor{currentfill}{rgb}{0.962283,0.593046,0.431453}%
\pgfsetfillcolor{currentfill}%
\pgfsetlinewidth{0.311001pt}%
\definecolor{currentstroke}{rgb}{1.000000,1.000000,1.000000}%
\pgfsetstrokecolor{currentstroke}%
\pgfsetdash{}{0pt}%
\pgfpathmoveto{\pgfqpoint{4.166609in}{1.747669in}}%
\pgfpathcurveto{\pgfqpoint{4.173742in}{1.747669in}}{\pgfqpoint{4.180584in}{1.750503in}}{\pgfqpoint{4.185627in}{1.755547in}}%
\pgfpathcurveto{\pgfqpoint{4.190671in}{1.760591in}}{\pgfqpoint{4.193505in}{1.767432in}}{\pgfqpoint{4.193505in}{1.774565in}}%
\pgfpathcurveto{\pgfqpoint{4.193505in}{1.781698in}}{\pgfqpoint{4.190671in}{1.788540in}}{\pgfqpoint{4.185627in}{1.793583in}}%
\pgfpathcurveto{\pgfqpoint{4.180584in}{1.798627in}}{\pgfqpoint{4.173742in}{1.801461in}}{\pgfqpoint{4.166609in}{1.801461in}}%
\pgfpathcurveto{\pgfqpoint{4.159476in}{1.801461in}}{\pgfqpoint{4.152635in}{1.798627in}}{\pgfqpoint{4.147591in}{1.793583in}}%
\pgfpathcurveto{\pgfqpoint{4.142547in}{1.788540in}}{\pgfqpoint{4.139713in}{1.781698in}}{\pgfqpoint{4.139713in}{1.774565in}}%
\pgfpathcurveto{\pgfqpoint{4.139713in}{1.767432in}}{\pgfqpoint{4.142547in}{1.760591in}}{\pgfqpoint{4.147591in}{1.755547in}}%
\pgfpathcurveto{\pgfqpoint{4.152635in}{1.750503in}}{\pgfqpoint{4.159476in}{1.747669in}}{\pgfqpoint{4.166609in}{1.747669in}}%
\pgfpathclose%
\pgfusepath{stroke,fill}%
\end{pgfscope}%
\begin{pgfscope}%
\pgfpathrectangle{\pgfqpoint{2.867647in}{0.500000in}}{\pgfqpoint{1.764706in}{1.700000in}}%
\pgfusepath{clip}%
\pgfsetbuttcap%
\pgfsetroundjoin%
\definecolor{currentfill}{rgb}{0.905301,0.238545,0.247481}%
\pgfsetfillcolor{currentfill}%
\pgfsetlinewidth{0.311001pt}%
\definecolor{currentstroke}{rgb}{1.000000,1.000000,1.000000}%
\pgfsetstrokecolor{currentstroke}%
\pgfsetdash{}{0pt}%
\pgfpathmoveto{\pgfqpoint{3.882550in}{1.039157in}}%
\pgfpathcurveto{\pgfqpoint{3.889682in}{1.039157in}}{\pgfqpoint{3.896524in}{1.041991in}}{\pgfqpoint{3.901568in}{1.047035in}}%
\pgfpathcurveto{\pgfqpoint{3.906611in}{1.052078in}}{\pgfqpoint{3.909445in}{1.058920in}}{\pgfqpoint{3.909445in}{1.066053in}}%
\pgfpathcurveto{\pgfqpoint{3.909445in}{1.073186in}}{\pgfqpoint{3.906611in}{1.080027in}}{\pgfqpoint{3.901568in}{1.085071in}}%
\pgfpathcurveto{\pgfqpoint{3.896524in}{1.090115in}}{\pgfqpoint{3.889682in}{1.092949in}}{\pgfqpoint{3.882550in}{1.092949in}}%
\pgfpathcurveto{\pgfqpoint{3.875417in}{1.092949in}}{\pgfqpoint{3.868575in}{1.090115in}}{\pgfqpoint{3.863531in}{1.085071in}}%
\pgfpathcurveto{\pgfqpoint{3.858488in}{1.080027in}}{\pgfqpoint{3.855654in}{1.073186in}}{\pgfqpoint{3.855654in}{1.066053in}}%
\pgfpathcurveto{\pgfqpoint{3.855654in}{1.058920in}}{\pgfqpoint{3.858488in}{1.052078in}}{\pgfqpoint{3.863531in}{1.047035in}}%
\pgfpathcurveto{\pgfqpoint{3.868575in}{1.041991in}}{\pgfqpoint{3.875417in}{1.039157in}}{\pgfqpoint{3.882550in}{1.039157in}}%
\pgfpathclose%
\pgfusepath{stroke,fill}%
\end{pgfscope}%
\begin{pgfscope}%
\pgfpathrectangle{\pgfqpoint{2.867647in}{0.500000in}}{\pgfqpoint{1.764706in}{1.700000in}}%
\pgfusepath{clip}%
\pgfsetbuttcap%
\pgfsetroundjoin%
\definecolor{currentfill}{rgb}{0.980678,0.914765,0.856766}%
\pgfsetfillcolor{currentfill}%
\pgfsetlinewidth{0.311001pt}%
\definecolor{currentstroke}{rgb}{1.000000,1.000000,1.000000}%
\pgfsetstrokecolor{currentstroke}%
\pgfsetdash{}{0pt}%
\pgfpathmoveto{\pgfqpoint{4.173083in}{1.437238in}}%
\pgfpathcurveto{\pgfqpoint{4.180215in}{1.437238in}}{\pgfqpoint{4.187057in}{1.440072in}}{\pgfqpoint{4.192101in}{1.445115in}}%
\pgfpathcurveto{\pgfqpoint{4.197144in}{1.450159in}}{\pgfqpoint{4.199978in}{1.457001in}}{\pgfqpoint{4.199978in}{1.464133in}}%
\pgfpathcurveto{\pgfqpoint{4.199978in}{1.471266in}}{\pgfqpoint{4.197144in}{1.478108in}}{\pgfqpoint{4.192101in}{1.483151in}}%
\pgfpathcurveto{\pgfqpoint{4.187057in}{1.488195in}}{\pgfqpoint{4.180215in}{1.491029in}}{\pgfqpoint{4.173083in}{1.491029in}}%
\pgfpathcurveto{\pgfqpoint{4.165950in}{1.491029in}}{\pgfqpoint{4.159108in}{1.488195in}}{\pgfqpoint{4.154064in}{1.483151in}}%
\pgfpathcurveto{\pgfqpoint{4.149021in}{1.478108in}}{\pgfqpoint{4.146187in}{1.471266in}}{\pgfqpoint{4.146187in}{1.464133in}}%
\pgfpathcurveto{\pgfqpoint{4.146187in}{1.457001in}}{\pgfqpoint{4.149021in}{1.450159in}}{\pgfqpoint{4.154064in}{1.445115in}}%
\pgfpathcurveto{\pgfqpoint{4.159108in}{1.440072in}}{\pgfqpoint{4.165950in}{1.437238in}}{\pgfqpoint{4.173083in}{1.437238in}}%
\pgfpathclose%
\pgfusepath{stroke,fill}%
\end{pgfscope}%
\begin{pgfscope}%
\pgfpathrectangle{\pgfqpoint{2.867647in}{0.500000in}}{\pgfqpoint{1.764706in}{1.700000in}}%
\pgfusepath{clip}%
\pgfsetbuttcap%
\pgfsetroundjoin%
\definecolor{currentfill}{rgb}{0.979124,0.903132,0.839793}%
\pgfsetfillcolor{currentfill}%
\pgfsetlinewidth{0.311001pt}%
\definecolor{currentstroke}{rgb}{1.000000,1.000000,1.000000}%
\pgfsetstrokecolor{currentstroke}%
\pgfsetdash{}{0pt}%
\pgfpathmoveto{\pgfqpoint{4.142209in}{1.205379in}}%
\pgfpathcurveto{\pgfqpoint{4.149342in}{1.205379in}}{\pgfqpoint{4.156183in}{1.208213in}}{\pgfqpoint{4.161227in}{1.213256in}}%
\pgfpathcurveto{\pgfqpoint{4.166271in}{1.218300in}}{\pgfqpoint{4.169105in}{1.225141in}}{\pgfqpoint{4.169105in}{1.232274in}}%
\pgfpathcurveto{\pgfqpoint{4.169105in}{1.239407in}}{\pgfqpoint{4.166271in}{1.246249in}}{\pgfqpoint{4.161227in}{1.251292in}}%
\pgfpathcurveto{\pgfqpoint{4.156183in}{1.256336in}}{\pgfqpoint{4.149342in}{1.259170in}}{\pgfqpoint{4.142209in}{1.259170in}}%
\pgfpathcurveto{\pgfqpoint{4.135076in}{1.259170in}}{\pgfqpoint{4.128234in}{1.256336in}}{\pgfqpoint{4.123191in}{1.251292in}}%
\pgfpathcurveto{\pgfqpoint{4.118147in}{1.246249in}}{\pgfqpoint{4.115313in}{1.239407in}}{\pgfqpoint{4.115313in}{1.232274in}}%
\pgfpathcurveto{\pgfqpoint{4.115313in}{1.225141in}}{\pgfqpoint{4.118147in}{1.218300in}}{\pgfqpoint{4.123191in}{1.213256in}}%
\pgfpathcurveto{\pgfqpoint{4.128234in}{1.208213in}}{\pgfqpoint{4.135076in}{1.205379in}}{\pgfqpoint{4.142209in}{1.205379in}}%
\pgfpathclose%
\pgfusepath{stroke,fill}%
\end{pgfscope}%
\begin{pgfscope}%
\pgfpathrectangle{\pgfqpoint{2.867647in}{0.500000in}}{\pgfqpoint{1.764706in}{1.700000in}}%
\pgfusepath{clip}%
\pgfsetbuttcap%
\pgfsetroundjoin%
\definecolor{currentfill}{rgb}{0.967092,0.768560,0.642079}%
\pgfsetfillcolor{currentfill}%
\pgfsetlinewidth{0.311001pt}%
\definecolor{currentstroke}{rgb}{1.000000,1.000000,1.000000}%
\pgfsetstrokecolor{currentstroke}%
\pgfsetdash{}{0pt}%
\pgfpathmoveto{\pgfqpoint{4.039232in}{1.080661in}}%
\pgfpathcurveto{\pgfqpoint{4.046364in}{1.080661in}}{\pgfqpoint{4.053206in}{1.083495in}}{\pgfqpoint{4.058250in}{1.088538in}}%
\pgfpathcurveto{\pgfqpoint{4.063293in}{1.093582in}}{\pgfqpoint{4.066127in}{1.100424in}}{\pgfqpoint{4.066127in}{1.107557in}}%
\pgfpathcurveto{\pgfqpoint{4.066127in}{1.114689in}}{\pgfqpoint{4.063293in}{1.121531in}}{\pgfqpoint{4.058250in}{1.126575in}}%
\pgfpathcurveto{\pgfqpoint{4.053206in}{1.131618in}}{\pgfqpoint{4.046364in}{1.134452in}}{\pgfqpoint{4.039232in}{1.134452in}}%
\pgfpathcurveto{\pgfqpoint{4.032099in}{1.134452in}}{\pgfqpoint{4.025257in}{1.131618in}}{\pgfqpoint{4.020213in}{1.126575in}}%
\pgfpathcurveto{\pgfqpoint{4.015170in}{1.121531in}}{\pgfqpoint{4.012336in}{1.114689in}}{\pgfqpoint{4.012336in}{1.107557in}}%
\pgfpathcurveto{\pgfqpoint{4.012336in}{1.100424in}}{\pgfqpoint{4.015170in}{1.093582in}}{\pgfqpoint{4.020213in}{1.088538in}}%
\pgfpathcurveto{\pgfqpoint{4.025257in}{1.083495in}}{\pgfqpoint{4.032099in}{1.080661in}}{\pgfqpoint{4.039232in}{1.080661in}}%
\pgfpathclose%
\pgfusepath{stroke,fill}%
\end{pgfscope}%
\begin{pgfscope}%
\pgfpathrectangle{\pgfqpoint{2.867647in}{0.500000in}}{\pgfqpoint{1.764706in}{1.700000in}}%
\pgfusepath{clip}%
\pgfsetbuttcap%
\pgfsetroundjoin%
\definecolor{currentfill}{rgb}{0.970718,0.821518,0.719872}%
\pgfsetfillcolor{currentfill}%
\pgfsetlinewidth{0.311001pt}%
\definecolor{currentstroke}{rgb}{1.000000,1.000000,1.000000}%
\pgfsetstrokecolor{currentstroke}%
\pgfsetdash{}{0pt}%
\pgfpathmoveto{\pgfqpoint{4.068978in}{1.517106in}}%
\pgfpathcurveto{\pgfqpoint{4.076111in}{1.517106in}}{\pgfqpoint{4.082953in}{1.519940in}}{\pgfqpoint{4.087996in}{1.524984in}}%
\pgfpathcurveto{\pgfqpoint{4.093040in}{1.530027in}}{\pgfqpoint{4.095874in}{1.536869in}}{\pgfqpoint{4.095874in}{1.544002in}}%
\pgfpathcurveto{\pgfqpoint{4.095874in}{1.551135in}}{\pgfqpoint{4.093040in}{1.557976in}}{\pgfqpoint{4.087996in}{1.563020in}}%
\pgfpathcurveto{\pgfqpoint{4.082953in}{1.568064in}}{\pgfqpoint{4.076111in}{1.570897in}}{\pgfqpoint{4.068978in}{1.570897in}}%
\pgfpathcurveto{\pgfqpoint{4.061845in}{1.570897in}}{\pgfqpoint{4.055004in}{1.568064in}}{\pgfqpoint{4.049960in}{1.563020in}}%
\pgfpathcurveto{\pgfqpoint{4.044916in}{1.557976in}}{\pgfqpoint{4.042082in}{1.551135in}}{\pgfqpoint{4.042082in}{1.544002in}}%
\pgfpathcurveto{\pgfqpoint{4.042082in}{1.536869in}}{\pgfqpoint{4.044916in}{1.530027in}}{\pgfqpoint{4.049960in}{1.524984in}}%
\pgfpathcurveto{\pgfqpoint{4.055004in}{1.519940in}}{\pgfqpoint{4.061845in}{1.517106in}}{\pgfqpoint{4.068978in}{1.517106in}}%
\pgfpathclose%
\pgfusepath{stroke,fill}%
\end{pgfscope}%
\begin{pgfscope}%
\pgfpathrectangle{\pgfqpoint{2.867647in}{0.500000in}}{\pgfqpoint{1.764706in}{1.700000in}}%
\pgfusepath{clip}%
\pgfsetbuttcap%
\pgfsetroundjoin%
\definecolor{currentfill}{rgb}{0.979891,0.908948,0.848279}%
\pgfsetfillcolor{currentfill}%
\pgfsetlinewidth{0.311001pt}%
\definecolor{currentstroke}{rgb}{1.000000,1.000000,1.000000}%
\pgfsetstrokecolor{currentstroke}%
\pgfsetdash{}{0pt}%
\pgfpathmoveto{\pgfqpoint{4.162157in}{1.113177in}}%
\pgfpathcurveto{\pgfqpoint{4.169290in}{1.113177in}}{\pgfqpoint{4.176132in}{1.116011in}}{\pgfqpoint{4.181175in}{1.121054in}}%
\pgfpathcurveto{\pgfqpoint{4.186219in}{1.126098in}}{\pgfqpoint{4.189053in}{1.132940in}}{\pgfqpoint{4.189053in}{1.140073in}}%
\pgfpathcurveto{\pgfqpoint{4.189053in}{1.147205in}}{\pgfqpoint{4.186219in}{1.154047in}}{\pgfqpoint{4.181175in}{1.159091in}}%
\pgfpathcurveto{\pgfqpoint{4.176132in}{1.164134in}}{\pgfqpoint{4.169290in}{1.166968in}}{\pgfqpoint{4.162157in}{1.166968in}}%
\pgfpathcurveto{\pgfqpoint{4.155025in}{1.166968in}}{\pgfqpoint{4.148183in}{1.164134in}}{\pgfqpoint{4.143139in}{1.159091in}}%
\pgfpathcurveto{\pgfqpoint{4.138096in}{1.154047in}}{\pgfqpoint{4.135262in}{1.147205in}}{\pgfqpoint{4.135262in}{1.140073in}}%
\pgfpathcurveto{\pgfqpoint{4.135262in}{1.132940in}}{\pgfqpoint{4.138096in}{1.126098in}}{\pgfqpoint{4.143139in}{1.121054in}}%
\pgfpathcurveto{\pgfqpoint{4.148183in}{1.116011in}}{\pgfqpoint{4.155025in}{1.113177in}}{\pgfqpoint{4.162157in}{1.113177in}}%
\pgfpathclose%
\pgfusepath{stroke,fill}%
\end{pgfscope}%
\begin{pgfscope}%
\pgfpathrectangle{\pgfqpoint{2.867647in}{0.500000in}}{\pgfqpoint{1.764706in}{1.700000in}}%
\pgfusepath{clip}%
\pgfsetbuttcap%
\pgfsetroundjoin%
\definecolor{currentfill}{rgb}{0.960778,0.559972,0.399412}%
\pgfsetfillcolor{currentfill}%
\pgfsetlinewidth{0.311001pt}%
\definecolor{currentstroke}{rgb}{1.000000,1.000000,1.000000}%
\pgfsetstrokecolor{currentstroke}%
\pgfsetdash{}{0pt}%
\pgfpathmoveto{\pgfqpoint{3.928198in}{1.666901in}}%
\pgfpathcurveto{\pgfqpoint{3.935331in}{1.666901in}}{\pgfqpoint{3.942173in}{1.669735in}}{\pgfqpoint{3.947216in}{1.674779in}}%
\pgfpathcurveto{\pgfqpoint{3.952260in}{1.679823in}}{\pgfqpoint{3.955094in}{1.686664in}}{\pgfqpoint{3.955094in}{1.693797in}}%
\pgfpathcurveto{\pgfqpoint{3.955094in}{1.700930in}}{\pgfqpoint{3.952260in}{1.707772in}}{\pgfqpoint{3.947216in}{1.712815in}}%
\pgfpathcurveto{\pgfqpoint{3.942173in}{1.717859in}}{\pgfqpoint{3.935331in}{1.720693in}}{\pgfqpoint{3.928198in}{1.720693in}}%
\pgfpathcurveto{\pgfqpoint{3.921065in}{1.720693in}}{\pgfqpoint{3.914224in}{1.717859in}}{\pgfqpoint{3.909180in}{1.712815in}}%
\pgfpathcurveto{\pgfqpoint{3.904136in}{1.707772in}}{\pgfqpoint{3.901302in}{1.700930in}}{\pgfqpoint{3.901302in}{1.693797in}}%
\pgfpathcurveto{\pgfqpoint{3.901302in}{1.686664in}}{\pgfqpoint{3.904136in}{1.679823in}}{\pgfqpoint{3.909180in}{1.674779in}}%
\pgfpathcurveto{\pgfqpoint{3.914224in}{1.669735in}}{\pgfqpoint{3.921065in}{1.666901in}}{\pgfqpoint{3.928198in}{1.666901in}}%
\pgfpathclose%
\pgfusepath{stroke,fill}%
\end{pgfscope}%
\begin{pgfscope}%
\pgfpathrectangle{\pgfqpoint{2.867647in}{0.500000in}}{\pgfqpoint{1.764706in}{1.700000in}}%
\pgfusepath{clip}%
\pgfsetbuttcap%
\pgfsetroundjoin%
\definecolor{currentfill}{rgb}{0.965592,0.726236,0.584384}%
\pgfsetfillcolor{currentfill}%
\pgfsetlinewidth{0.311001pt}%
\definecolor{currentstroke}{rgb}{1.000000,1.000000,1.000000}%
\pgfsetstrokecolor{currentstroke}%
\pgfsetdash{}{0pt}%
\pgfpathmoveto{\pgfqpoint{4.232371in}{1.026656in}}%
\pgfpathcurveto{\pgfqpoint{4.239504in}{1.026656in}}{\pgfqpoint{4.246346in}{1.029490in}}{\pgfqpoint{4.251389in}{1.034534in}}%
\pgfpathcurveto{\pgfqpoint{4.256433in}{1.039577in}}{\pgfqpoint{4.259267in}{1.046419in}}{\pgfqpoint{4.259267in}{1.053552in}}%
\pgfpathcurveto{\pgfqpoint{4.259267in}{1.060684in}}{\pgfqpoint{4.256433in}{1.067526in}}{\pgfqpoint{4.251389in}{1.072570in}}%
\pgfpathcurveto{\pgfqpoint{4.246346in}{1.077613in}}{\pgfqpoint{4.239504in}{1.080447in}}{\pgfqpoint{4.232371in}{1.080447in}}%
\pgfpathcurveto{\pgfqpoint{4.225238in}{1.080447in}}{\pgfqpoint{4.218397in}{1.077613in}}{\pgfqpoint{4.213353in}{1.072570in}}%
\pgfpathcurveto{\pgfqpoint{4.208309in}{1.067526in}}{\pgfqpoint{4.205475in}{1.060684in}}{\pgfqpoint{4.205475in}{1.053552in}}%
\pgfpathcurveto{\pgfqpoint{4.205475in}{1.046419in}}{\pgfqpoint{4.208309in}{1.039577in}}{\pgfqpoint{4.213353in}{1.034534in}}%
\pgfpathcurveto{\pgfqpoint{4.218397in}{1.029490in}}{\pgfqpoint{4.225238in}{1.026656in}}{\pgfqpoint{4.232371in}{1.026656in}}%
\pgfpathclose%
\pgfusepath{stroke,fill}%
\end{pgfscope}%
\begin{pgfscope}%
\pgfpathrectangle{\pgfqpoint{2.867647in}{0.500000in}}{\pgfqpoint{1.764706in}{1.700000in}}%
\pgfusepath{clip}%
\pgfsetbuttcap%
\pgfsetroundjoin%
\definecolor{currentfill}{rgb}{0.973271,0.850724,0.762998}%
\pgfsetfillcolor{currentfill}%
\pgfsetlinewidth{0.311001pt}%
\definecolor{currentstroke}{rgb}{1.000000,1.000000,1.000000}%
\pgfsetstrokecolor{currentstroke}%
\pgfsetdash{}{0pt}%
\pgfpathmoveto{\pgfqpoint{4.208002in}{1.074680in}}%
\pgfpathcurveto{\pgfqpoint{4.215135in}{1.074680in}}{\pgfqpoint{4.221977in}{1.077514in}}{\pgfqpoint{4.227021in}{1.082557in}}%
\pgfpathcurveto{\pgfqpoint{4.232064in}{1.087601in}}{\pgfqpoint{4.234898in}{1.094443in}}{\pgfqpoint{4.234898in}{1.101576in}}%
\pgfpathcurveto{\pgfqpoint{4.234898in}{1.108708in}}{\pgfqpoint{4.232064in}{1.115550in}}{\pgfqpoint{4.227021in}{1.120594in}}%
\pgfpathcurveto{\pgfqpoint{4.221977in}{1.125637in}}{\pgfqpoint{4.215135in}{1.128471in}}{\pgfqpoint{4.208002in}{1.128471in}}%
\pgfpathcurveto{\pgfqpoint{4.200870in}{1.128471in}}{\pgfqpoint{4.194028in}{1.125637in}}{\pgfqpoint{4.188984in}{1.120594in}}%
\pgfpathcurveto{\pgfqpoint{4.183941in}{1.115550in}}{\pgfqpoint{4.181107in}{1.108708in}}{\pgfqpoint{4.181107in}{1.101576in}}%
\pgfpathcurveto{\pgfqpoint{4.181107in}{1.094443in}}{\pgfqpoint{4.183941in}{1.087601in}}{\pgfqpoint{4.188984in}{1.082557in}}%
\pgfpathcurveto{\pgfqpoint{4.194028in}{1.077514in}}{\pgfqpoint{4.200870in}{1.074680in}}{\pgfqpoint{4.208002in}{1.074680in}}%
\pgfpathclose%
\pgfusepath{stroke,fill}%
\end{pgfscope}%
\begin{pgfscope}%
\pgfpathrectangle{\pgfqpoint{2.867647in}{0.500000in}}{\pgfqpoint{1.764706in}{1.700000in}}%
\pgfusepath{clip}%
\pgfsetbuttcap%
\pgfsetroundjoin%
\definecolor{currentfill}{rgb}{0.976961,0.885681,0.814303}%
\pgfsetfillcolor{currentfill}%
\pgfsetlinewidth{0.311001pt}%
\definecolor{currentstroke}{rgb}{1.000000,1.000000,1.000000}%
\pgfsetstrokecolor{currentstroke}%
\pgfsetdash{}{0pt}%
\pgfpathmoveto{\pgfqpoint{4.121043in}{1.042158in}}%
\pgfpathcurveto{\pgfqpoint{4.128176in}{1.042158in}}{\pgfqpoint{4.135018in}{1.044992in}}{\pgfqpoint{4.140061in}{1.050036in}}%
\pgfpathcurveto{\pgfqpoint{4.145105in}{1.055079in}}{\pgfqpoint{4.147939in}{1.061921in}}{\pgfqpoint{4.147939in}{1.069054in}}%
\pgfpathcurveto{\pgfqpoint{4.147939in}{1.076187in}}{\pgfqpoint{4.145105in}{1.083028in}}{\pgfqpoint{4.140061in}{1.088072in}}%
\pgfpathcurveto{\pgfqpoint{4.135018in}{1.093116in}}{\pgfqpoint{4.128176in}{1.095950in}}{\pgfqpoint{4.121043in}{1.095950in}}%
\pgfpathcurveto{\pgfqpoint{4.113910in}{1.095950in}}{\pgfqpoint{4.107069in}{1.093116in}}{\pgfqpoint{4.102025in}{1.088072in}}%
\pgfpathcurveto{\pgfqpoint{4.096981in}{1.083028in}}{\pgfqpoint{4.094148in}{1.076187in}}{\pgfqpoint{4.094148in}{1.069054in}}%
\pgfpathcurveto{\pgfqpoint{4.094148in}{1.061921in}}{\pgfqpoint{4.096981in}{1.055079in}}{\pgfqpoint{4.102025in}{1.050036in}}%
\pgfpathcurveto{\pgfqpoint{4.107069in}{1.044992in}}{\pgfqpoint{4.113910in}{1.042158in}}{\pgfqpoint{4.121043in}{1.042158in}}%
\pgfpathclose%
\pgfusepath{stroke,fill}%
\end{pgfscope}%
\begin{pgfscope}%
\pgfpathrectangle{\pgfqpoint{2.867647in}{0.500000in}}{\pgfqpoint{1.764706in}{1.700000in}}%
\pgfusepath{clip}%
\pgfsetbuttcap%
\pgfsetroundjoin%
\definecolor{currentfill}{rgb}{0.597702,0.106938,0.358380}%
\pgfsetfillcolor{currentfill}%
\pgfsetlinewidth{0.311001pt}%
\definecolor{currentstroke}{rgb}{1.000000,1.000000,1.000000}%
\pgfsetstrokecolor{currentstroke}%
\pgfsetdash{}{0pt}%
\pgfpathmoveto{\pgfqpoint{3.925434in}{1.260134in}}%
\pgfpathcurveto{\pgfqpoint{3.932566in}{1.260134in}}{\pgfqpoint{3.939408in}{1.262968in}}{\pgfqpoint{3.944452in}{1.268011in}}%
\pgfpathcurveto{\pgfqpoint{3.949495in}{1.273055in}}{\pgfqpoint{3.952329in}{1.279897in}}{\pgfqpoint{3.952329in}{1.287029in}}%
\pgfpathcurveto{\pgfqpoint{3.952329in}{1.294162in}}{\pgfqpoint{3.949495in}{1.301004in}}{\pgfqpoint{3.944452in}{1.306048in}}%
\pgfpathcurveto{\pgfqpoint{3.939408in}{1.311091in}}{\pgfqpoint{3.932566in}{1.313925in}}{\pgfqpoint{3.925434in}{1.313925in}}%
\pgfpathcurveto{\pgfqpoint{3.918301in}{1.313925in}}{\pgfqpoint{3.911459in}{1.311091in}}{\pgfqpoint{3.906415in}{1.306048in}}%
\pgfpathcurveto{\pgfqpoint{3.901372in}{1.301004in}}{\pgfqpoint{3.898538in}{1.294162in}}{\pgfqpoint{3.898538in}{1.287029in}}%
\pgfpathcurveto{\pgfqpoint{3.898538in}{1.279897in}}{\pgfqpoint{3.901372in}{1.273055in}}{\pgfqpoint{3.906415in}{1.268011in}}%
\pgfpathcurveto{\pgfqpoint{3.911459in}{1.262968in}}{\pgfqpoint{3.918301in}{1.260134in}}{\pgfqpoint{3.925434in}{1.260134in}}%
\pgfpathclose%
\pgfusepath{stroke,fill}%
\end{pgfscope}%
\begin{pgfscope}%
\pgfpathrectangle{\pgfqpoint{2.867647in}{0.500000in}}{\pgfqpoint{1.764706in}{1.700000in}}%
\pgfusepath{clip}%
\pgfsetbuttcap%
\pgfsetroundjoin%
\definecolor{currentfill}{rgb}{0.968931,0.798091,0.685123}%
\pgfsetfillcolor{currentfill}%
\pgfsetlinewidth{0.311001pt}%
\definecolor{currentstroke}{rgb}{1.000000,1.000000,1.000000}%
\pgfsetstrokecolor{currentstroke}%
\pgfsetdash{}{0pt}%
\pgfpathmoveto{\pgfqpoint{4.044886in}{1.564810in}}%
\pgfpathcurveto{\pgfqpoint{4.052019in}{1.564810in}}{\pgfqpoint{4.058860in}{1.567644in}}{\pgfqpoint{4.063904in}{1.572688in}}%
\pgfpathcurveto{\pgfqpoint{4.068948in}{1.577732in}}{\pgfqpoint{4.071782in}{1.584573in}}{\pgfqpoint{4.071782in}{1.591706in}}%
\pgfpathcurveto{\pgfqpoint{4.071782in}{1.598839in}}{\pgfqpoint{4.068948in}{1.605680in}}{\pgfqpoint{4.063904in}{1.610724in}}%
\pgfpathcurveto{\pgfqpoint{4.058860in}{1.615768in}}{\pgfqpoint{4.052019in}{1.618602in}}{\pgfqpoint{4.044886in}{1.618602in}}%
\pgfpathcurveto{\pgfqpoint{4.037753in}{1.618602in}}{\pgfqpoint{4.030911in}{1.615768in}}{\pgfqpoint{4.025868in}{1.610724in}}%
\pgfpathcurveto{\pgfqpoint{4.020824in}{1.605680in}}{\pgfqpoint{4.017990in}{1.598839in}}{\pgfqpoint{4.017990in}{1.591706in}}%
\pgfpathcurveto{\pgfqpoint{4.017990in}{1.584573in}}{\pgfqpoint{4.020824in}{1.577732in}}{\pgfqpoint{4.025868in}{1.572688in}}%
\pgfpathcurveto{\pgfqpoint{4.030911in}{1.567644in}}{\pgfqpoint{4.037753in}{1.564810in}}{\pgfqpoint{4.044886in}{1.564810in}}%
\pgfpathclose%
\pgfusepath{stroke,fill}%
\end{pgfscope}%
\begin{pgfscope}%
\pgfpathrectangle{\pgfqpoint{2.867647in}{0.500000in}}{\pgfqpoint{1.764706in}{1.700000in}}%
\pgfusepath{clip}%
\pgfsetbuttcap%
\pgfsetroundjoin%
\definecolor{currentfill}{rgb}{0.980678,0.914765,0.856766}%
\pgfsetfillcolor{currentfill}%
\pgfsetlinewidth{0.311001pt}%
\definecolor{currentstroke}{rgb}{1.000000,1.000000,1.000000}%
\pgfsetstrokecolor{currentstroke}%
\pgfsetdash{}{0pt}%
\pgfpathmoveto{\pgfqpoint{4.173889in}{1.332639in}}%
\pgfpathcurveto{\pgfqpoint{4.181022in}{1.332639in}}{\pgfqpoint{4.187864in}{1.335473in}}{\pgfqpoint{4.192908in}{1.340517in}}%
\pgfpathcurveto{\pgfqpoint{4.197951in}{1.345561in}}{\pgfqpoint{4.200785in}{1.352402in}}{\pgfqpoint{4.200785in}{1.359535in}}%
\pgfpathcurveto{\pgfqpoint{4.200785in}{1.366668in}}{\pgfqpoint{4.197951in}{1.373510in}}{\pgfqpoint{4.192908in}{1.378553in}}%
\pgfpathcurveto{\pgfqpoint{4.187864in}{1.383597in}}{\pgfqpoint{4.181022in}{1.386431in}}{\pgfqpoint{4.173889in}{1.386431in}}%
\pgfpathcurveto{\pgfqpoint{4.166757in}{1.386431in}}{\pgfqpoint{4.159915in}{1.383597in}}{\pgfqpoint{4.154871in}{1.378553in}}%
\pgfpathcurveto{\pgfqpoint{4.149828in}{1.373510in}}{\pgfqpoint{4.146994in}{1.366668in}}{\pgfqpoint{4.146994in}{1.359535in}}%
\pgfpathcurveto{\pgfqpoint{4.146994in}{1.352402in}}{\pgfqpoint{4.149828in}{1.345561in}}{\pgfqpoint{4.154871in}{1.340517in}}%
\pgfpathcurveto{\pgfqpoint{4.159915in}{1.335473in}}{\pgfqpoint{4.166757in}{1.332639in}}{\pgfqpoint{4.173889in}{1.332639in}}%
\pgfpathclose%
\pgfusepath{stroke,fill}%
\end{pgfscope}%
\begin{pgfscope}%
\pgfpathrectangle{\pgfqpoint{2.867647in}{0.500000in}}{\pgfqpoint{1.764706in}{1.700000in}}%
\pgfusepath{clip}%
\pgfsetbuttcap%
\pgfsetroundjoin%
\definecolor{currentfill}{rgb}{0.975018,0.868213,0.788710}%
\pgfsetfillcolor{currentfill}%
\pgfsetlinewidth{0.311001pt}%
\definecolor{currentstroke}{rgb}{1.000000,1.000000,1.000000}%
\pgfsetstrokecolor{currentstroke}%
\pgfsetdash{}{0pt}%
\pgfpathmoveto{\pgfqpoint{4.232987in}{1.449079in}}%
\pgfpathcurveto{\pgfqpoint{4.240120in}{1.449079in}}{\pgfqpoint{4.246962in}{1.451913in}}{\pgfqpoint{4.252005in}{1.456957in}}%
\pgfpathcurveto{\pgfqpoint{4.257049in}{1.462000in}}{\pgfqpoint{4.259883in}{1.468842in}}{\pgfqpoint{4.259883in}{1.475975in}}%
\pgfpathcurveto{\pgfqpoint{4.259883in}{1.483108in}}{\pgfqpoint{4.257049in}{1.489949in}}{\pgfqpoint{4.252005in}{1.494993in}}%
\pgfpathcurveto{\pgfqpoint{4.246962in}{1.500037in}}{\pgfqpoint{4.240120in}{1.502871in}}{\pgfqpoint{4.232987in}{1.502871in}}%
\pgfpathcurveto{\pgfqpoint{4.225854in}{1.502871in}}{\pgfqpoint{4.219013in}{1.500037in}}{\pgfqpoint{4.213969in}{1.494993in}}%
\pgfpathcurveto{\pgfqpoint{4.208925in}{1.489949in}}{\pgfqpoint{4.206091in}{1.483108in}}{\pgfqpoint{4.206091in}{1.475975in}}%
\pgfpathcurveto{\pgfqpoint{4.206091in}{1.468842in}}{\pgfqpoint{4.208925in}{1.462000in}}{\pgfqpoint{4.213969in}{1.456957in}}%
\pgfpathcurveto{\pgfqpoint{4.219013in}{1.451913in}}{\pgfqpoint{4.225854in}{1.449079in}}{\pgfqpoint{4.232987in}{1.449079in}}%
\pgfpathclose%
\pgfusepath{stroke,fill}%
\end{pgfscope}%
\begin{pgfscope}%
\pgfpathrectangle{\pgfqpoint{2.867647in}{0.500000in}}{\pgfqpoint{1.764706in}{1.700000in}}%
\pgfusepath{clip}%
\pgfsetbuttcap%
\pgfsetroundjoin%
\definecolor{currentfill}{rgb}{0.981377,0.920617,0.865369}%
\pgfsetfillcolor{currentfill}%
\pgfsetlinewidth{0.311001pt}%
\definecolor{currentstroke}{rgb}{1.000000,1.000000,1.000000}%
\pgfsetstrokecolor{currentstroke}%
\pgfsetdash{}{0pt}%
\pgfpathmoveto{\pgfqpoint{4.174691in}{1.174563in}}%
\pgfpathcurveto{\pgfqpoint{4.181824in}{1.174563in}}{\pgfqpoint{4.188666in}{1.177397in}}{\pgfqpoint{4.193709in}{1.182440in}}%
\pgfpathcurveto{\pgfqpoint{4.198753in}{1.187484in}}{\pgfqpoint{4.201587in}{1.194326in}}{\pgfqpoint{4.201587in}{1.201459in}}%
\pgfpathcurveto{\pgfqpoint{4.201587in}{1.208591in}}{\pgfqpoint{4.198753in}{1.215433in}}{\pgfqpoint{4.193709in}{1.220477in}}%
\pgfpathcurveto{\pgfqpoint{4.188666in}{1.225520in}}{\pgfqpoint{4.181824in}{1.228354in}}{\pgfqpoint{4.174691in}{1.228354in}}%
\pgfpathcurveto{\pgfqpoint{4.167559in}{1.228354in}}{\pgfqpoint{4.160717in}{1.225520in}}{\pgfqpoint{4.155673in}{1.220477in}}%
\pgfpathcurveto{\pgfqpoint{4.150630in}{1.215433in}}{\pgfqpoint{4.147796in}{1.208591in}}{\pgfqpoint{4.147796in}{1.201459in}}%
\pgfpathcurveto{\pgfqpoint{4.147796in}{1.194326in}}{\pgfqpoint{4.150630in}{1.187484in}}{\pgfqpoint{4.155673in}{1.182440in}}%
\pgfpathcurveto{\pgfqpoint{4.160717in}{1.177397in}}{\pgfqpoint{4.167559in}{1.174563in}}{\pgfqpoint{4.174691in}{1.174563in}}%
\pgfpathclose%
\pgfusepath{stroke,fill}%
\end{pgfscope}%
\begin{pgfscope}%
\pgfpathrectangle{\pgfqpoint{2.867647in}{0.500000in}}{\pgfqpoint{1.764706in}{1.700000in}}%
\pgfusepath{clip}%
\pgfsetbuttcap%
\pgfsetroundjoin%
\definecolor{currentfill}{rgb}{0.965440,0.720101,0.576404}%
\pgfsetfillcolor{currentfill}%
\pgfsetlinewidth{0.311001pt}%
\definecolor{currentstroke}{rgb}{1.000000,1.000000,1.000000}%
\pgfsetstrokecolor{currentstroke}%
\pgfsetdash{}{0pt}%
\pgfpathmoveto{\pgfqpoint{4.210412in}{1.652817in}}%
\pgfpathcurveto{\pgfqpoint{4.217545in}{1.652817in}}{\pgfqpoint{4.224386in}{1.655650in}}{\pgfqpoint{4.229430in}{1.660694in}}%
\pgfpathcurveto{\pgfqpoint{4.234474in}{1.665738in}}{\pgfqpoint{4.237307in}{1.672579in}}{\pgfqpoint{4.237307in}{1.679712in}}%
\pgfpathcurveto{\pgfqpoint{4.237307in}{1.686845in}}{\pgfqpoint{4.234474in}{1.693687in}}{\pgfqpoint{4.229430in}{1.698730in}}%
\pgfpathcurveto{\pgfqpoint{4.224386in}{1.703774in}}{\pgfqpoint{4.217545in}{1.706608in}}{\pgfqpoint{4.210412in}{1.706608in}}%
\pgfpathcurveto{\pgfqpoint{4.203279in}{1.706608in}}{\pgfqpoint{4.196437in}{1.703774in}}{\pgfqpoint{4.191394in}{1.698730in}}%
\pgfpathcurveto{\pgfqpoint{4.186350in}{1.693687in}}{\pgfqpoint{4.183516in}{1.686845in}}{\pgfqpoint{4.183516in}{1.679712in}}%
\pgfpathcurveto{\pgfqpoint{4.183516in}{1.672579in}}{\pgfqpoint{4.186350in}{1.665738in}}{\pgfqpoint{4.191394in}{1.660694in}}%
\pgfpathcurveto{\pgfqpoint{4.196437in}{1.655650in}}{\pgfqpoint{4.203279in}{1.652817in}}{\pgfqpoint{4.210412in}{1.652817in}}%
\pgfpathclose%
\pgfusepath{stroke,fill}%
\end{pgfscope}%
\begin{pgfscope}%
\pgfpathrectangle{\pgfqpoint{2.867647in}{0.500000in}}{\pgfqpoint{1.764706in}{1.700000in}}%
\pgfusepath{clip}%
\pgfsetbuttcap%
\pgfsetroundjoin%
\definecolor{currentfill}{rgb}{0.976287,0.879862,0.805788}%
\pgfsetfillcolor{currentfill}%
\pgfsetlinewidth{0.311001pt}%
\definecolor{currentstroke}{rgb}{1.000000,1.000000,1.000000}%
\pgfsetstrokecolor{currentstroke}%
\pgfsetdash{}{0pt}%
\pgfpathmoveto{\pgfqpoint{4.222784in}{1.470232in}}%
\pgfpathcurveto{\pgfqpoint{4.229916in}{1.470232in}}{\pgfqpoint{4.236758in}{1.473066in}}{\pgfqpoint{4.241802in}{1.478110in}}%
\pgfpathcurveto{\pgfqpoint{4.246845in}{1.483154in}}{\pgfqpoint{4.249679in}{1.489995in}}{\pgfqpoint{4.249679in}{1.497128in}}%
\pgfpathcurveto{\pgfqpoint{4.249679in}{1.504261in}}{\pgfqpoint{4.246845in}{1.511103in}}{\pgfqpoint{4.241802in}{1.516146in}}%
\pgfpathcurveto{\pgfqpoint{4.236758in}{1.521190in}}{\pgfqpoint{4.229916in}{1.524024in}}{\pgfqpoint{4.222784in}{1.524024in}}%
\pgfpathcurveto{\pgfqpoint{4.215651in}{1.524024in}}{\pgfqpoint{4.208809in}{1.521190in}}{\pgfqpoint{4.203765in}{1.516146in}}%
\pgfpathcurveto{\pgfqpoint{4.198722in}{1.511103in}}{\pgfqpoint{4.195888in}{1.504261in}}{\pgfqpoint{4.195888in}{1.497128in}}%
\pgfpathcurveto{\pgfqpoint{4.195888in}{1.489995in}}{\pgfqpoint{4.198722in}{1.483154in}}{\pgfqpoint{4.203765in}{1.478110in}}%
\pgfpathcurveto{\pgfqpoint{4.208809in}{1.473066in}}{\pgfqpoint{4.215651in}{1.470232in}}{\pgfqpoint{4.222784in}{1.470232in}}%
\pgfpathclose%
\pgfusepath{stroke,fill}%
\end{pgfscope}%
\begin{pgfscope}%
\pgfpathrectangle{\pgfqpoint{2.867647in}{0.500000in}}{\pgfqpoint{1.764706in}{1.700000in}}%
\pgfusepath{clip}%
\pgfsetbuttcap%
\pgfsetroundjoin%
\definecolor{currentfill}{rgb}{0.972726,0.844889,0.754401}%
\pgfsetfillcolor{currentfill}%
\pgfsetlinewidth{0.311001pt}%
\definecolor{currentstroke}{rgb}{1.000000,1.000000,1.000000}%
\pgfsetstrokecolor{currentstroke}%
\pgfsetdash{}{0pt}%
\pgfpathmoveto{\pgfqpoint{4.077959in}{0.994469in}}%
\pgfpathcurveto{\pgfqpoint{4.085092in}{0.994469in}}{\pgfqpoint{4.091933in}{0.997303in}}{\pgfqpoint{4.096977in}{1.002346in}}%
\pgfpathcurveto{\pgfqpoint{4.102021in}{1.007390in}}{\pgfqpoint{4.104854in}{1.014232in}}{\pgfqpoint{4.104854in}{1.021365in}}%
\pgfpathcurveto{\pgfqpoint{4.104854in}{1.028497in}}{\pgfqpoint{4.102021in}{1.035339in}}{\pgfqpoint{4.096977in}{1.040383in}}%
\pgfpathcurveto{\pgfqpoint{4.091933in}{1.045426in}}{\pgfqpoint{4.085092in}{1.048260in}}{\pgfqpoint{4.077959in}{1.048260in}}%
\pgfpathcurveto{\pgfqpoint{4.070826in}{1.048260in}}{\pgfqpoint{4.063984in}{1.045426in}}{\pgfqpoint{4.058941in}{1.040383in}}%
\pgfpathcurveto{\pgfqpoint{4.053897in}{1.035339in}}{\pgfqpoint{4.051063in}{1.028497in}}{\pgfqpoint{4.051063in}{1.021365in}}%
\pgfpathcurveto{\pgfqpoint{4.051063in}{1.014232in}}{\pgfqpoint{4.053897in}{1.007390in}}{\pgfqpoint{4.058941in}{1.002346in}}%
\pgfpathcurveto{\pgfqpoint{4.063984in}{0.997303in}}{\pgfqpoint{4.070826in}{0.994469in}}{\pgfqpoint{4.077959in}{0.994469in}}%
\pgfpathclose%
\pgfusepath{stroke,fill}%
\end{pgfscope}%
\begin{pgfscope}%
\pgfpathrectangle{\pgfqpoint{2.867647in}{0.500000in}}{\pgfqpoint{1.764706in}{1.700000in}}%
\pgfusepath{clip}%
\pgfsetbuttcap%
\pgfsetroundjoin%
\definecolor{currentfill}{rgb}{0.973271,0.850724,0.762998}%
\pgfsetfillcolor{currentfill}%
\pgfsetlinewidth{0.311001pt}%
\definecolor{currentstroke}{rgb}{1.000000,1.000000,1.000000}%
\pgfsetstrokecolor{currentstroke}%
\pgfsetdash{}{0pt}%
\pgfpathmoveto{\pgfqpoint{4.257933in}{1.322900in}}%
\pgfpathcurveto{\pgfqpoint{4.265066in}{1.322900in}}{\pgfqpoint{4.271908in}{1.325734in}}{\pgfqpoint{4.276952in}{1.330777in}}%
\pgfpathcurveto{\pgfqpoint{4.281995in}{1.335821in}}{\pgfqpoint{4.284829in}{1.342663in}}{\pgfqpoint{4.284829in}{1.349796in}}%
\pgfpathcurveto{\pgfqpoint{4.284829in}{1.356928in}}{\pgfqpoint{4.281995in}{1.363770in}}{\pgfqpoint{4.276952in}{1.368814in}}%
\pgfpathcurveto{\pgfqpoint{4.271908in}{1.373857in}}{\pgfqpoint{4.265066in}{1.376691in}}{\pgfqpoint{4.257933in}{1.376691in}}%
\pgfpathcurveto{\pgfqpoint{4.250801in}{1.376691in}}{\pgfqpoint{4.243959in}{1.373857in}}{\pgfqpoint{4.238915in}{1.368814in}}%
\pgfpathcurveto{\pgfqpoint{4.233872in}{1.363770in}}{\pgfqpoint{4.231038in}{1.356928in}}{\pgfqpoint{4.231038in}{1.349796in}}%
\pgfpathcurveto{\pgfqpoint{4.231038in}{1.342663in}}{\pgfqpoint{4.233872in}{1.335821in}}{\pgfqpoint{4.238915in}{1.330777in}}%
\pgfpathcurveto{\pgfqpoint{4.243959in}{1.325734in}}{\pgfqpoint{4.250801in}{1.322900in}}{\pgfqpoint{4.257933in}{1.322900in}}%
\pgfpathclose%
\pgfusepath{stroke,fill}%
\end{pgfscope}%
\begin{pgfscope}%
\pgfpathrectangle{\pgfqpoint{2.867647in}{0.500000in}}{\pgfqpoint{1.764706in}{1.700000in}}%
\pgfusepath{clip}%
\pgfsetbuttcap%
\pgfsetroundjoin%
\definecolor{currentfill}{rgb}{0.976961,0.885681,0.814303}%
\pgfsetfillcolor{currentfill}%
\pgfsetlinewidth{0.311001pt}%
\definecolor{currentstroke}{rgb}{1.000000,1.000000,1.000000}%
\pgfsetstrokecolor{currentstroke}%
\pgfsetdash{}{0pt}%
\pgfpathmoveto{\pgfqpoint{4.098225in}{1.588798in}}%
\pgfpathcurveto{\pgfqpoint{4.105358in}{1.588798in}}{\pgfqpoint{4.112199in}{1.591632in}}{\pgfqpoint{4.117243in}{1.596676in}}%
\pgfpathcurveto{\pgfqpoint{4.122287in}{1.601719in}}{\pgfqpoint{4.125121in}{1.608561in}}{\pgfqpoint{4.125121in}{1.615694in}}%
\pgfpathcurveto{\pgfqpoint{4.125121in}{1.622827in}}{\pgfqpoint{4.122287in}{1.629668in}}{\pgfqpoint{4.117243in}{1.634712in}}%
\pgfpathcurveto{\pgfqpoint{4.112199in}{1.639756in}}{\pgfqpoint{4.105358in}{1.642589in}}{\pgfqpoint{4.098225in}{1.642589in}}%
\pgfpathcurveto{\pgfqpoint{4.091092in}{1.642589in}}{\pgfqpoint{4.084250in}{1.639756in}}{\pgfqpoint{4.079207in}{1.634712in}}%
\pgfpathcurveto{\pgfqpoint{4.074163in}{1.629668in}}{\pgfqpoint{4.071329in}{1.622827in}}{\pgfqpoint{4.071329in}{1.615694in}}%
\pgfpathcurveto{\pgfqpoint{4.071329in}{1.608561in}}{\pgfqpoint{4.074163in}{1.601719in}}{\pgfqpoint{4.079207in}{1.596676in}}%
\pgfpathcurveto{\pgfqpoint{4.084250in}{1.591632in}}{\pgfqpoint{4.091092in}{1.588798in}}{\pgfqpoint{4.098225in}{1.588798in}}%
\pgfpathclose%
\pgfusepath{stroke,fill}%
\end{pgfscope}%
\begin{pgfscope}%
\pgfpathrectangle{\pgfqpoint{2.867647in}{0.500000in}}{\pgfqpoint{1.764706in}{1.700000in}}%
\pgfusepath{clip}%
\pgfsetbuttcap%
\pgfsetroundjoin%
\definecolor{currentfill}{rgb}{0.976961,0.885681,0.814303}%
\pgfsetfillcolor{currentfill}%
\pgfsetlinewidth{0.311001pt}%
\definecolor{currentstroke}{rgb}{1.000000,1.000000,1.000000}%
\pgfsetstrokecolor{currentstroke}%
\pgfsetdash{}{0pt}%
\pgfpathmoveto{\pgfqpoint{4.144203in}{1.344816in}}%
\pgfpathcurveto{\pgfqpoint{4.151336in}{1.344816in}}{\pgfqpoint{4.158177in}{1.347649in}}{\pgfqpoint{4.163221in}{1.352693in}}%
\pgfpathcurveto{\pgfqpoint{4.168265in}{1.357737in}}{\pgfqpoint{4.171099in}{1.364578in}}{\pgfqpoint{4.171099in}{1.371711in}}%
\pgfpathcurveto{\pgfqpoint{4.171099in}{1.378844in}}{\pgfqpoint{4.168265in}{1.385686in}}{\pgfqpoint{4.163221in}{1.390729in}}%
\pgfpathcurveto{\pgfqpoint{4.158177in}{1.395773in}}{\pgfqpoint{4.151336in}{1.398607in}}{\pgfqpoint{4.144203in}{1.398607in}}%
\pgfpathcurveto{\pgfqpoint{4.137070in}{1.398607in}}{\pgfqpoint{4.130228in}{1.395773in}}{\pgfqpoint{4.125185in}{1.390729in}}%
\pgfpathcurveto{\pgfqpoint{4.120141in}{1.385686in}}{\pgfqpoint{4.117307in}{1.378844in}}{\pgfqpoint{4.117307in}{1.371711in}}%
\pgfpathcurveto{\pgfqpoint{4.117307in}{1.364578in}}{\pgfqpoint{4.120141in}{1.357737in}}{\pgfqpoint{4.125185in}{1.352693in}}%
\pgfpathcurveto{\pgfqpoint{4.130228in}{1.347649in}}{\pgfqpoint{4.137070in}{1.344816in}}{\pgfqpoint{4.144203in}{1.344816in}}%
\pgfpathclose%
\pgfusepath{stroke,fill}%
\end{pgfscope}%
\begin{pgfscope}%
\pgfpathrectangle{\pgfqpoint{2.867647in}{0.500000in}}{\pgfqpoint{1.764706in}{1.700000in}}%
\pgfusepath{clip}%
\pgfsetbuttcap%
\pgfsetroundjoin%
\definecolor{currentfill}{rgb}{0.969803,0.809811,0.702523}%
\pgfsetfillcolor{currentfill}%
\pgfsetlinewidth{0.311001pt}%
\definecolor{currentstroke}{rgb}{1.000000,1.000000,1.000000}%
\pgfsetstrokecolor{currentstroke}%
\pgfsetdash{}{0pt}%
\pgfpathmoveto{\pgfqpoint{4.273889in}{1.377297in}}%
\pgfpathcurveto{\pgfqpoint{4.281022in}{1.377297in}}{\pgfqpoint{4.287863in}{1.380131in}}{\pgfqpoint{4.292907in}{1.385174in}}%
\pgfpathcurveto{\pgfqpoint{4.297951in}{1.390218in}}{\pgfqpoint{4.300785in}{1.397060in}}{\pgfqpoint{4.300785in}{1.404192in}}%
\pgfpathcurveto{\pgfqpoint{4.300785in}{1.411325in}}{\pgfqpoint{4.297951in}{1.418167in}}{\pgfqpoint{4.292907in}{1.423211in}}%
\pgfpathcurveto{\pgfqpoint{4.287863in}{1.428254in}}{\pgfqpoint{4.281022in}{1.431088in}}{\pgfqpoint{4.273889in}{1.431088in}}%
\pgfpathcurveto{\pgfqpoint{4.266756in}{1.431088in}}{\pgfqpoint{4.259914in}{1.428254in}}{\pgfqpoint{4.254871in}{1.423211in}}%
\pgfpathcurveto{\pgfqpoint{4.249827in}{1.418167in}}{\pgfqpoint{4.246993in}{1.411325in}}{\pgfqpoint{4.246993in}{1.404192in}}%
\pgfpathcurveto{\pgfqpoint{4.246993in}{1.397060in}}{\pgfqpoint{4.249827in}{1.390218in}}{\pgfqpoint{4.254871in}{1.385174in}}%
\pgfpathcurveto{\pgfqpoint{4.259914in}{1.380131in}}{\pgfqpoint{4.266756in}{1.377297in}}{\pgfqpoint{4.273889in}{1.377297in}}%
\pgfpathclose%
\pgfusepath{stroke,fill}%
\end{pgfscope}%
\begin{pgfscope}%
\pgfpathrectangle{\pgfqpoint{2.867647in}{0.500000in}}{\pgfqpoint{1.764706in}{1.700000in}}%
\pgfusepath{clip}%
\pgfsetbuttcap%
\pgfsetroundjoin%
\definecolor{currentfill}{rgb}{0.980678,0.914765,0.856766}%
\pgfsetfillcolor{currentfill}%
\pgfsetlinewidth{0.311001pt}%
\definecolor{currentstroke}{rgb}{1.000000,1.000000,1.000000}%
\pgfsetstrokecolor{currentstroke}%
\pgfsetdash{}{0pt}%
\pgfpathmoveto{\pgfqpoint{4.213006in}{1.280212in}}%
\pgfpathcurveto{\pgfqpoint{4.220139in}{1.280212in}}{\pgfqpoint{4.226980in}{1.283046in}}{\pgfqpoint{4.232024in}{1.288089in}}%
\pgfpathcurveto{\pgfqpoint{4.237068in}{1.293133in}}{\pgfqpoint{4.239901in}{1.299974in}}{\pgfqpoint{4.239901in}{1.307107in}}%
\pgfpathcurveto{\pgfqpoint{4.239901in}{1.314240in}}{\pgfqpoint{4.237068in}{1.321082in}}{\pgfqpoint{4.232024in}{1.326125in}}%
\pgfpathcurveto{\pgfqpoint{4.226980in}{1.331169in}}{\pgfqpoint{4.220139in}{1.334003in}}{\pgfqpoint{4.213006in}{1.334003in}}%
\pgfpathcurveto{\pgfqpoint{4.205873in}{1.334003in}}{\pgfqpoint{4.199031in}{1.331169in}}{\pgfqpoint{4.193988in}{1.326125in}}%
\pgfpathcurveto{\pgfqpoint{4.188944in}{1.321082in}}{\pgfqpoint{4.186110in}{1.314240in}}{\pgfqpoint{4.186110in}{1.307107in}}%
\pgfpathcurveto{\pgfqpoint{4.186110in}{1.299974in}}{\pgfqpoint{4.188944in}{1.293133in}}{\pgfqpoint{4.193988in}{1.288089in}}%
\pgfpathcurveto{\pgfqpoint{4.199031in}{1.283046in}}{\pgfqpoint{4.205873in}{1.280212in}}{\pgfqpoint{4.213006in}{1.280212in}}%
\pgfpathclose%
\pgfusepath{stroke,fill}%
\end{pgfscope}%
\begin{pgfscope}%
\pgfpathrectangle{\pgfqpoint{2.867647in}{0.500000in}}{\pgfqpoint{1.764706in}{1.700000in}}%
\pgfusepath{clip}%
\pgfsetbuttcap%
\pgfsetroundjoin%
\definecolor{currentfill}{rgb}{0.979891,0.908948,0.848279}%
\pgfsetfillcolor{currentfill}%
\pgfsetlinewidth{0.311001pt}%
\definecolor{currentstroke}{rgb}{1.000000,1.000000,1.000000}%
\pgfsetstrokecolor{currentstroke}%
\pgfsetdash{}{0pt}%
\pgfpathmoveto{\pgfqpoint{4.196201in}{1.158650in}}%
\pgfpathcurveto{\pgfqpoint{4.203333in}{1.158650in}}{\pgfqpoint{4.210175in}{1.161484in}}{\pgfqpoint{4.215219in}{1.166528in}}%
\pgfpathcurveto{\pgfqpoint{4.220262in}{1.171571in}}{\pgfqpoint{4.223096in}{1.178413in}}{\pgfqpoint{4.223096in}{1.185546in}}%
\pgfpathcurveto{\pgfqpoint{4.223096in}{1.192678in}}{\pgfqpoint{4.220262in}{1.199520in}}{\pgfqpoint{4.215219in}{1.204564in}}%
\pgfpathcurveto{\pgfqpoint{4.210175in}{1.209607in}}{\pgfqpoint{4.203333in}{1.212441in}}{\pgfqpoint{4.196201in}{1.212441in}}%
\pgfpathcurveto{\pgfqpoint{4.189068in}{1.212441in}}{\pgfqpoint{4.182226in}{1.209607in}}{\pgfqpoint{4.177182in}{1.204564in}}%
\pgfpathcurveto{\pgfqpoint{4.172139in}{1.199520in}}{\pgfqpoint{4.169305in}{1.192678in}}{\pgfqpoint{4.169305in}{1.185546in}}%
\pgfpathcurveto{\pgfqpoint{4.169305in}{1.178413in}}{\pgfqpoint{4.172139in}{1.171571in}}{\pgfqpoint{4.177182in}{1.166528in}}%
\pgfpathcurveto{\pgfqpoint{4.182226in}{1.161484in}}{\pgfqpoint{4.189068in}{1.158650in}}{\pgfqpoint{4.196201in}{1.158650in}}%
\pgfpathclose%
\pgfusepath{stroke,fill}%
\end{pgfscope}%
\begin{pgfscope}%
\pgfpathrectangle{\pgfqpoint{2.867647in}{0.500000in}}{\pgfqpoint{1.764706in}{1.700000in}}%
\pgfusepath{clip}%
\pgfsetbuttcap%
\pgfsetroundjoin%
\definecolor{currentfill}{rgb}{0.981377,0.920617,0.865369}%
\pgfsetfillcolor{currentfill}%
\pgfsetlinewidth{0.311001pt}%
\definecolor{currentstroke}{rgb}{1.000000,1.000000,1.000000}%
\pgfsetstrokecolor{currentstroke}%
\pgfsetdash{}{0pt}%
\pgfpathmoveto{\pgfqpoint{4.167283in}{1.254800in}}%
\pgfpathcurveto{\pgfqpoint{4.174416in}{1.254800in}}{\pgfqpoint{4.181258in}{1.257634in}}{\pgfqpoint{4.186301in}{1.262677in}}%
\pgfpathcurveto{\pgfqpoint{4.191345in}{1.267721in}}{\pgfqpoint{4.194179in}{1.274563in}}{\pgfqpoint{4.194179in}{1.281696in}}%
\pgfpathcurveto{\pgfqpoint{4.194179in}{1.288828in}}{\pgfqpoint{4.191345in}{1.295670in}}{\pgfqpoint{4.186301in}{1.300714in}}%
\pgfpathcurveto{\pgfqpoint{4.181258in}{1.305757in}}{\pgfqpoint{4.174416in}{1.308591in}}{\pgfqpoint{4.167283in}{1.308591in}}%
\pgfpathcurveto{\pgfqpoint{4.160150in}{1.308591in}}{\pgfqpoint{4.153309in}{1.305757in}}{\pgfqpoint{4.148265in}{1.300714in}}%
\pgfpathcurveto{\pgfqpoint{4.143221in}{1.295670in}}{\pgfqpoint{4.140388in}{1.288828in}}{\pgfqpoint{4.140388in}{1.281696in}}%
\pgfpathcurveto{\pgfqpoint{4.140388in}{1.274563in}}{\pgfqpoint{4.143221in}{1.267721in}}{\pgfqpoint{4.148265in}{1.262677in}}%
\pgfpathcurveto{\pgfqpoint{4.153309in}{1.257634in}}{\pgfqpoint{4.160150in}{1.254800in}}{\pgfqpoint{4.167283in}{1.254800in}}%
\pgfpathclose%
\pgfusepath{stroke,fill}%
\end{pgfscope}%
\begin{pgfscope}%
\pgfpathrectangle{\pgfqpoint{2.867647in}{0.500000in}}{\pgfqpoint{1.764706in}{1.700000in}}%
\pgfusepath{clip}%
\pgfsetbuttcap%
\pgfsetroundjoin%
\definecolor{currentfill}{rgb}{0.979891,0.908948,0.848279}%
\pgfsetfillcolor{currentfill}%
\pgfsetlinewidth{0.311001pt}%
\definecolor{currentstroke}{rgb}{1.000000,1.000000,1.000000}%
\pgfsetstrokecolor{currentstroke}%
\pgfsetdash{}{0pt}%
\pgfpathmoveto{\pgfqpoint{4.213375in}{1.382576in}}%
\pgfpathcurveto{\pgfqpoint{4.220507in}{1.382576in}}{\pgfqpoint{4.227349in}{1.385410in}}{\pgfqpoint{4.232393in}{1.390453in}}%
\pgfpathcurveto{\pgfqpoint{4.237436in}{1.395497in}}{\pgfqpoint{4.240270in}{1.402339in}}{\pgfqpoint{4.240270in}{1.409472in}}%
\pgfpathcurveto{\pgfqpoint{4.240270in}{1.416604in}}{\pgfqpoint{4.237436in}{1.423446in}}{\pgfqpoint{4.232393in}{1.428490in}}%
\pgfpathcurveto{\pgfqpoint{4.227349in}{1.433533in}}{\pgfqpoint{4.220507in}{1.436367in}}{\pgfqpoint{4.213375in}{1.436367in}}%
\pgfpathcurveto{\pgfqpoint{4.206242in}{1.436367in}}{\pgfqpoint{4.199400in}{1.433533in}}{\pgfqpoint{4.194356in}{1.428490in}}%
\pgfpathcurveto{\pgfqpoint{4.189313in}{1.423446in}}{\pgfqpoint{4.186479in}{1.416604in}}{\pgfqpoint{4.186479in}{1.409472in}}%
\pgfpathcurveto{\pgfqpoint{4.186479in}{1.402339in}}{\pgfqpoint{4.189313in}{1.395497in}}{\pgfqpoint{4.194356in}{1.390453in}}%
\pgfpathcurveto{\pgfqpoint{4.199400in}{1.385410in}}{\pgfqpoint{4.206242in}{1.382576in}}{\pgfqpoint{4.213375in}{1.382576in}}%
\pgfpathclose%
\pgfusepath{stroke,fill}%
\end{pgfscope}%
\begin{pgfscope}%
\pgfpathrectangle{\pgfqpoint{2.867647in}{0.500000in}}{\pgfqpoint{1.764706in}{1.700000in}}%
\pgfusepath{clip}%
\pgfsetbuttcap%
\pgfsetroundjoin%
\definecolor{currentfill}{rgb}{0.969359,0.803954,0.693832}%
\pgfsetfillcolor{currentfill}%
\pgfsetlinewidth{0.311001pt}%
\definecolor{currentstroke}{rgb}{1.000000,1.000000,1.000000}%
\pgfsetstrokecolor{currentstroke}%
\pgfsetdash{}{0pt}%
\pgfpathmoveto{\pgfqpoint{4.097175in}{0.942970in}}%
\pgfpathcurveto{\pgfqpoint{4.104308in}{0.942970in}}{\pgfqpoint{4.111150in}{0.945804in}}{\pgfqpoint{4.116193in}{0.950847in}}%
\pgfpathcurveto{\pgfqpoint{4.121237in}{0.955891in}}{\pgfqpoint{4.124071in}{0.962733in}}{\pgfqpoint{4.124071in}{0.969865in}}%
\pgfpathcurveto{\pgfqpoint{4.124071in}{0.976998in}}{\pgfqpoint{4.121237in}{0.983840in}}{\pgfqpoint{4.116193in}{0.988884in}}%
\pgfpathcurveto{\pgfqpoint{4.111150in}{0.993927in}}{\pgfqpoint{4.104308in}{0.996761in}}{\pgfqpoint{4.097175in}{0.996761in}}%
\pgfpathcurveto{\pgfqpoint{4.090042in}{0.996761in}}{\pgfqpoint{4.083201in}{0.993927in}}{\pgfqpoint{4.078157in}{0.988884in}}%
\pgfpathcurveto{\pgfqpoint{4.073113in}{0.983840in}}{\pgfqpoint{4.070279in}{0.976998in}}{\pgfqpoint{4.070279in}{0.969865in}}%
\pgfpathcurveto{\pgfqpoint{4.070279in}{0.962733in}}{\pgfqpoint{4.073113in}{0.955891in}}{\pgfqpoint{4.078157in}{0.950847in}}%
\pgfpathcurveto{\pgfqpoint{4.083201in}{0.945804in}}{\pgfqpoint{4.090042in}{0.942970in}}{\pgfqpoint{4.097175in}{0.942970in}}%
\pgfpathclose%
\pgfusepath{stroke,fill}%
\end{pgfscope}%
\begin{pgfscope}%
\pgfpathrectangle{\pgfqpoint{2.867647in}{0.500000in}}{\pgfqpoint{1.764706in}{1.700000in}}%
\pgfusepath{clip}%
\pgfsetbuttcap%
\pgfsetroundjoin%
\definecolor{currentfill}{rgb}{0.964558,0.676556,0.522514}%
\pgfsetfillcolor{currentfill}%
\pgfsetlinewidth{0.311001pt}%
\definecolor{currentstroke}{rgb}{1.000000,1.000000,1.000000}%
\pgfsetstrokecolor{currentstroke}%
\pgfsetdash{}{0pt}%
\pgfpathmoveto{\pgfqpoint{4.152246in}{1.731982in}}%
\pgfpathcurveto{\pgfqpoint{4.159378in}{1.731982in}}{\pgfqpoint{4.166220in}{1.734815in}}{\pgfqpoint{4.171264in}{1.739859in}}%
\pgfpathcurveto{\pgfqpoint{4.176307in}{1.744903in}}{\pgfqpoint{4.179141in}{1.751744in}}{\pgfqpoint{4.179141in}{1.758877in}}%
\pgfpathcurveto{\pgfqpoint{4.179141in}{1.766010in}}{\pgfqpoint{4.176307in}{1.772852in}}{\pgfqpoint{4.171264in}{1.777895in}}%
\pgfpathcurveto{\pgfqpoint{4.166220in}{1.782939in}}{\pgfqpoint{4.159378in}{1.785773in}}{\pgfqpoint{4.152246in}{1.785773in}}%
\pgfpathcurveto{\pgfqpoint{4.145113in}{1.785773in}}{\pgfqpoint{4.138271in}{1.782939in}}{\pgfqpoint{4.133227in}{1.777895in}}%
\pgfpathcurveto{\pgfqpoint{4.128184in}{1.772852in}}{\pgfqpoint{4.125350in}{1.766010in}}{\pgfqpoint{4.125350in}{1.758877in}}%
\pgfpathcurveto{\pgfqpoint{4.125350in}{1.751744in}}{\pgfqpoint{4.128184in}{1.744903in}}{\pgfqpoint{4.133227in}{1.739859in}}%
\pgfpathcurveto{\pgfqpoint{4.138271in}{1.734815in}}{\pgfqpoint{4.145113in}{1.731982in}}{\pgfqpoint{4.152246in}{1.731982in}}%
\pgfpathclose%
\pgfusepath{stroke,fill}%
\end{pgfscope}%
\begin{pgfscope}%
\pgfpathrectangle{\pgfqpoint{2.867647in}{0.500000in}}{\pgfqpoint{1.764706in}{1.700000in}}%
\pgfusepath{clip}%
\pgfsetbuttcap%
\pgfsetroundjoin%
\definecolor{currentfill}{rgb}{0.975018,0.868213,0.788710}%
\pgfsetfillcolor{currentfill}%
\pgfsetlinewidth{0.311001pt}%
\definecolor{currentstroke}{rgb}{1.000000,1.000000,1.000000}%
\pgfsetstrokecolor{currentstroke}%
\pgfsetdash{}{0pt}%
\pgfpathmoveto{\pgfqpoint{4.222439in}{1.498332in}}%
\pgfpathcurveto{\pgfqpoint{4.229571in}{1.498332in}}{\pgfqpoint{4.236413in}{1.501166in}}{\pgfqpoint{4.241457in}{1.506209in}}%
\pgfpathcurveto{\pgfqpoint{4.246500in}{1.511253in}}{\pgfqpoint{4.249334in}{1.518095in}}{\pgfqpoint{4.249334in}{1.525227in}}%
\pgfpathcurveto{\pgfqpoint{4.249334in}{1.532360in}}{\pgfqpoint{4.246500in}{1.539202in}}{\pgfqpoint{4.241457in}{1.544245in}}%
\pgfpathcurveto{\pgfqpoint{4.236413in}{1.549289in}}{\pgfqpoint{4.229571in}{1.552123in}}{\pgfqpoint{4.222439in}{1.552123in}}%
\pgfpathcurveto{\pgfqpoint{4.215306in}{1.552123in}}{\pgfqpoint{4.208464in}{1.549289in}}{\pgfqpoint{4.203421in}{1.544245in}}%
\pgfpathcurveto{\pgfqpoint{4.198377in}{1.539202in}}{\pgfqpoint{4.195543in}{1.532360in}}{\pgfqpoint{4.195543in}{1.525227in}}%
\pgfpathcurveto{\pgfqpoint{4.195543in}{1.518095in}}{\pgfqpoint{4.198377in}{1.511253in}}{\pgfqpoint{4.203421in}{1.506209in}}%
\pgfpathcurveto{\pgfqpoint{4.208464in}{1.501166in}}{\pgfqpoint{4.215306in}{1.498332in}}{\pgfqpoint{4.222439in}{1.498332in}}%
\pgfpathclose%
\pgfusepath{stroke,fill}%
\end{pgfscope}%
\begin{pgfscope}%
\pgfpathrectangle{\pgfqpoint{2.867647in}{0.500000in}}{\pgfqpoint{1.764706in}{1.700000in}}%
\pgfusepath{clip}%
\pgfsetbuttcap%
\pgfsetroundjoin%
\definecolor{currentfill}{rgb}{0.967092,0.768560,0.642079}%
\pgfsetfillcolor{currentfill}%
\pgfsetlinewidth{0.311001pt}%
\definecolor{currentstroke}{rgb}{1.000000,1.000000,1.000000}%
\pgfsetstrokecolor{currentstroke}%
\pgfsetdash{}{0pt}%
\pgfpathmoveto{\pgfqpoint{4.028756in}{1.706446in}}%
\pgfpathcurveto{\pgfqpoint{4.035889in}{1.706446in}}{\pgfqpoint{4.042731in}{1.709280in}}{\pgfqpoint{4.047774in}{1.714324in}}%
\pgfpathcurveto{\pgfqpoint{4.052818in}{1.719368in}}{\pgfqpoint{4.055652in}{1.726209in}}{\pgfqpoint{4.055652in}{1.733342in}}%
\pgfpathcurveto{\pgfqpoint{4.055652in}{1.740475in}}{\pgfqpoint{4.052818in}{1.747317in}}{\pgfqpoint{4.047774in}{1.752360in}}%
\pgfpathcurveto{\pgfqpoint{4.042731in}{1.757404in}}{\pgfqpoint{4.035889in}{1.760238in}}{\pgfqpoint{4.028756in}{1.760238in}}%
\pgfpathcurveto{\pgfqpoint{4.021624in}{1.760238in}}{\pgfqpoint{4.014782in}{1.757404in}}{\pgfqpoint{4.009738in}{1.752360in}}%
\pgfpathcurveto{\pgfqpoint{4.004695in}{1.747317in}}{\pgfqpoint{4.001861in}{1.740475in}}{\pgfqpoint{4.001861in}{1.733342in}}%
\pgfpathcurveto{\pgfqpoint{4.001861in}{1.726209in}}{\pgfqpoint{4.004695in}{1.719368in}}{\pgfqpoint{4.009738in}{1.714324in}}%
\pgfpathcurveto{\pgfqpoint{4.014782in}{1.709280in}}{\pgfqpoint{4.021624in}{1.706446in}}{\pgfqpoint{4.028756in}{1.706446in}}%
\pgfpathclose%
\pgfusepath{stroke,fill}%
\end{pgfscope}%
\begin{pgfscope}%
\pgfpathrectangle{\pgfqpoint{2.867647in}{0.500000in}}{\pgfqpoint{1.764706in}{1.700000in}}%
\pgfusepath{clip}%
\pgfsetbuttcap%
\pgfsetroundjoin%
\definecolor{currentfill}{rgb}{0.968931,0.798091,0.685123}%
\pgfsetfillcolor{currentfill}%
\pgfsetlinewidth{0.311001pt}%
\definecolor{currentstroke}{rgb}{1.000000,1.000000,1.000000}%
\pgfsetstrokecolor{currentstroke}%
\pgfsetdash{}{0pt}%
\pgfpathmoveto{\pgfqpoint{4.208858in}{1.033452in}}%
\pgfpathcurveto{\pgfqpoint{4.215991in}{1.033452in}}{\pgfqpoint{4.222832in}{1.036286in}}{\pgfqpoint{4.227876in}{1.041330in}}%
\pgfpathcurveto{\pgfqpoint{4.232920in}{1.046373in}}{\pgfqpoint{4.235754in}{1.053215in}}{\pgfqpoint{4.235754in}{1.060348in}}%
\pgfpathcurveto{\pgfqpoint{4.235754in}{1.067481in}}{\pgfqpoint{4.232920in}{1.074322in}}{\pgfqpoint{4.227876in}{1.079366in}}%
\pgfpathcurveto{\pgfqpoint{4.222832in}{1.084409in}}{\pgfqpoint{4.215991in}{1.087243in}}{\pgfqpoint{4.208858in}{1.087243in}}%
\pgfpathcurveto{\pgfqpoint{4.201725in}{1.087243in}}{\pgfqpoint{4.194883in}{1.084409in}}{\pgfqpoint{4.189840in}{1.079366in}}%
\pgfpathcurveto{\pgfqpoint{4.184796in}{1.074322in}}{\pgfqpoint{4.181962in}{1.067481in}}{\pgfqpoint{4.181962in}{1.060348in}}%
\pgfpathcurveto{\pgfqpoint{4.181962in}{1.053215in}}{\pgfqpoint{4.184796in}{1.046373in}}{\pgfqpoint{4.189840in}{1.041330in}}%
\pgfpathcurveto{\pgfqpoint{4.194883in}{1.036286in}}{\pgfqpoint{4.201725in}{1.033452in}}{\pgfqpoint{4.208858in}{1.033452in}}%
\pgfpathclose%
\pgfusepath{stroke,fill}%
\end{pgfscope}%
\begin{pgfscope}%
\pgfpathrectangle{\pgfqpoint{2.867647in}{0.500000in}}{\pgfqpoint{1.764706in}{1.700000in}}%
\pgfusepath{clip}%
\pgfsetbuttcap%
\pgfsetroundjoin%
\definecolor{currentfill}{rgb}{0.974412,0.862387,0.780156}%
\pgfsetfillcolor{currentfill}%
\pgfsetlinewidth{0.311001pt}%
\definecolor{currentstroke}{rgb}{1.000000,1.000000,1.000000}%
\pgfsetstrokecolor{currentstroke}%
\pgfsetdash{}{0pt}%
\pgfpathmoveto{\pgfqpoint{4.100934in}{1.000500in}}%
\pgfpathcurveto{\pgfqpoint{4.108067in}{1.000500in}}{\pgfqpoint{4.114909in}{1.003333in}}{\pgfqpoint{4.119953in}{1.008377in}}%
\pgfpathcurveto{\pgfqpoint{4.124996in}{1.013421in}}{\pgfqpoint{4.127830in}{1.020262in}}{\pgfqpoint{4.127830in}{1.027395in}}%
\pgfpathcurveto{\pgfqpoint{4.127830in}{1.034528in}}{\pgfqpoint{4.124996in}{1.041370in}}{\pgfqpoint{4.119953in}{1.046413in}}%
\pgfpathcurveto{\pgfqpoint{4.114909in}{1.051457in}}{\pgfqpoint{4.108067in}{1.054291in}}{\pgfqpoint{4.100934in}{1.054291in}}%
\pgfpathcurveto{\pgfqpoint{4.093802in}{1.054291in}}{\pgfqpoint{4.086960in}{1.051457in}}{\pgfqpoint{4.081916in}{1.046413in}}%
\pgfpathcurveto{\pgfqpoint{4.076873in}{1.041370in}}{\pgfqpoint{4.074039in}{1.034528in}}{\pgfqpoint{4.074039in}{1.027395in}}%
\pgfpathcurveto{\pgfqpoint{4.074039in}{1.020262in}}{\pgfqpoint{4.076873in}{1.013421in}}{\pgfqpoint{4.081916in}{1.008377in}}%
\pgfpathcurveto{\pgfqpoint{4.086960in}{1.003333in}}{\pgfqpoint{4.093802in}{1.000500in}}{\pgfqpoint{4.100934in}{1.000500in}}%
\pgfpathclose%
\pgfusepath{stroke,fill}%
\end{pgfscope}%
\begin{pgfscope}%
\pgfpathrectangle{\pgfqpoint{2.867647in}{0.500000in}}{\pgfqpoint{1.764706in}{1.700000in}}%
\pgfusepath{clip}%
\pgfsetbuttcap%
\pgfsetroundjoin%
\definecolor{currentfill}{rgb}{0.973832,0.856556,0.771584}%
\pgfsetfillcolor{currentfill}%
\pgfsetlinewidth{0.311001pt}%
\definecolor{currentstroke}{rgb}{1.000000,1.000000,1.000000}%
\pgfsetstrokecolor{currentstroke}%
\pgfsetdash{}{0pt}%
\pgfpathmoveto{\pgfqpoint{4.255023in}{1.332173in}}%
\pgfpathcurveto{\pgfqpoint{4.262156in}{1.332173in}}{\pgfqpoint{4.268998in}{1.335007in}}{\pgfqpoint{4.274041in}{1.340050in}}%
\pgfpathcurveto{\pgfqpoint{4.279085in}{1.345094in}}{\pgfqpoint{4.281919in}{1.351936in}}{\pgfqpoint{4.281919in}{1.359068in}}%
\pgfpathcurveto{\pgfqpoint{4.281919in}{1.366201in}}{\pgfqpoint{4.279085in}{1.373043in}}{\pgfqpoint{4.274041in}{1.378087in}}%
\pgfpathcurveto{\pgfqpoint{4.268998in}{1.383130in}}{\pgfqpoint{4.262156in}{1.385964in}}{\pgfqpoint{4.255023in}{1.385964in}}%
\pgfpathcurveto{\pgfqpoint{4.247890in}{1.385964in}}{\pgfqpoint{4.241049in}{1.383130in}}{\pgfqpoint{4.236005in}{1.378087in}}%
\pgfpathcurveto{\pgfqpoint{4.230961in}{1.373043in}}{\pgfqpoint{4.228127in}{1.366201in}}{\pgfqpoint{4.228127in}{1.359068in}}%
\pgfpathcurveto{\pgfqpoint{4.228127in}{1.351936in}}{\pgfqpoint{4.230961in}{1.345094in}}{\pgfqpoint{4.236005in}{1.340050in}}%
\pgfpathcurveto{\pgfqpoint{4.241049in}{1.335007in}}{\pgfqpoint{4.247890in}{1.332173in}}{\pgfqpoint{4.255023in}{1.332173in}}%
\pgfpathclose%
\pgfusepath{stroke,fill}%
\end{pgfscope}%
\begin{pgfscope}%
\pgfpathrectangle{\pgfqpoint{2.867647in}{0.500000in}}{\pgfqpoint{1.764706in}{1.700000in}}%
\pgfusepath{clip}%
\pgfsetbuttcap%
\pgfsetroundjoin%
\definecolor{currentfill}{rgb}{0.981377,0.920617,0.865369}%
\pgfsetfillcolor{currentfill}%
\pgfsetlinewidth{0.311001pt}%
\definecolor{currentstroke}{rgb}{1.000000,1.000000,1.000000}%
\pgfsetstrokecolor{currentstroke}%
\pgfsetdash{}{0pt}%
\pgfpathmoveto{\pgfqpoint{4.200604in}{1.306394in}}%
\pgfpathcurveto{\pgfqpoint{4.207737in}{1.306394in}}{\pgfqpoint{4.214579in}{1.309228in}}{\pgfqpoint{4.219622in}{1.314271in}}%
\pgfpathcurveto{\pgfqpoint{4.224666in}{1.319315in}}{\pgfqpoint{4.227500in}{1.326156in}}{\pgfqpoint{4.227500in}{1.333289in}}%
\pgfpathcurveto{\pgfqpoint{4.227500in}{1.340422in}}{\pgfqpoint{4.224666in}{1.347264in}}{\pgfqpoint{4.219622in}{1.352307in}}%
\pgfpathcurveto{\pgfqpoint{4.214579in}{1.357351in}}{\pgfqpoint{4.207737in}{1.360185in}}{\pgfqpoint{4.200604in}{1.360185in}}%
\pgfpathcurveto{\pgfqpoint{4.193471in}{1.360185in}}{\pgfqpoint{4.186630in}{1.357351in}}{\pgfqpoint{4.181586in}{1.352307in}}%
\pgfpathcurveto{\pgfqpoint{4.176542in}{1.347264in}}{\pgfqpoint{4.173709in}{1.340422in}}{\pgfqpoint{4.173709in}{1.333289in}}%
\pgfpathcurveto{\pgfqpoint{4.173709in}{1.326156in}}{\pgfqpoint{4.176542in}{1.319315in}}{\pgfqpoint{4.181586in}{1.314271in}}%
\pgfpathcurveto{\pgfqpoint{4.186630in}{1.309228in}}{\pgfqpoint{4.193471in}{1.306394in}}{\pgfqpoint{4.200604in}{1.306394in}}%
\pgfpathclose%
\pgfusepath{stroke,fill}%
\end{pgfscope}%
\begin{pgfscope}%
\pgfpathrectangle{\pgfqpoint{2.867647in}{0.500000in}}{\pgfqpoint{1.764706in}{1.700000in}}%
\pgfusepath{clip}%
\pgfsetbuttcap%
\pgfsetroundjoin%
\definecolor{currentfill}{rgb}{0.953816,0.463738,0.317699}%
\pgfsetfillcolor{currentfill}%
\pgfsetlinewidth{0.311001pt}%
\definecolor{currentstroke}{rgb}{1.000000,1.000000,1.000000}%
\pgfsetstrokecolor{currentstroke}%
\pgfsetdash{}{0pt}%
\pgfpathmoveto{\pgfqpoint{4.135877in}{1.809094in}}%
\pgfpathcurveto{\pgfqpoint{4.143010in}{1.809094in}}{\pgfqpoint{4.149852in}{1.811928in}}{\pgfqpoint{4.154895in}{1.816971in}}%
\pgfpathcurveto{\pgfqpoint{4.159939in}{1.822015in}}{\pgfqpoint{4.162773in}{1.828857in}}{\pgfqpoint{4.162773in}{1.835989in}}%
\pgfpathcurveto{\pgfqpoint{4.162773in}{1.843122in}}{\pgfqpoint{4.159939in}{1.849964in}}{\pgfqpoint{4.154895in}{1.855008in}}%
\pgfpathcurveto{\pgfqpoint{4.149852in}{1.860051in}}{\pgfqpoint{4.143010in}{1.862885in}}{\pgfqpoint{4.135877in}{1.862885in}}%
\pgfpathcurveto{\pgfqpoint{4.128744in}{1.862885in}}{\pgfqpoint{4.121903in}{1.860051in}}{\pgfqpoint{4.116859in}{1.855008in}}%
\pgfpathcurveto{\pgfqpoint{4.111815in}{1.849964in}}{\pgfqpoint{4.108981in}{1.843122in}}{\pgfqpoint{4.108981in}{1.835989in}}%
\pgfpathcurveto{\pgfqpoint{4.108981in}{1.828857in}}{\pgfqpoint{4.111815in}{1.822015in}}{\pgfqpoint{4.116859in}{1.816971in}}%
\pgfpathcurveto{\pgfqpoint{4.121903in}{1.811928in}}{\pgfqpoint{4.128744in}{1.809094in}}{\pgfqpoint{4.135877in}{1.809094in}}%
\pgfpathclose%
\pgfusepath{stroke,fill}%
\end{pgfscope}%
\begin{pgfscope}%
\pgfpathrectangle{\pgfqpoint{2.867647in}{0.500000in}}{\pgfqpoint{1.764706in}{1.700000in}}%
\pgfusepath{clip}%
\pgfsetbuttcap%
\pgfsetroundjoin%
\definecolor{currentfill}{rgb}{0.969359,0.803954,0.693832}%
\pgfsetfillcolor{currentfill}%
\pgfsetlinewidth{0.311001pt}%
\definecolor{currentstroke}{rgb}{1.000000,1.000000,1.000000}%
\pgfsetstrokecolor{currentstroke}%
\pgfsetdash{}{0pt}%
\pgfpathmoveto{\pgfqpoint{4.102360in}{1.366266in}}%
\pgfpathcurveto{\pgfqpoint{4.109493in}{1.366266in}}{\pgfqpoint{4.116334in}{1.369099in}}{\pgfqpoint{4.121378in}{1.374143in}}%
\pgfpathcurveto{\pgfqpoint{4.126422in}{1.379187in}}{\pgfqpoint{4.129256in}{1.386028in}}{\pgfqpoint{4.129256in}{1.393161in}}%
\pgfpathcurveto{\pgfqpoint{4.129256in}{1.400294in}}{\pgfqpoint{4.126422in}{1.407136in}}{\pgfqpoint{4.121378in}{1.412179in}}%
\pgfpathcurveto{\pgfqpoint{4.116334in}{1.417223in}}{\pgfqpoint{4.109493in}{1.420057in}}{\pgfqpoint{4.102360in}{1.420057in}}%
\pgfpathcurveto{\pgfqpoint{4.095227in}{1.420057in}}{\pgfqpoint{4.088385in}{1.417223in}}{\pgfqpoint{4.083342in}{1.412179in}}%
\pgfpathcurveto{\pgfqpoint{4.078298in}{1.407136in}}{\pgfqpoint{4.075464in}{1.400294in}}{\pgfqpoint{4.075464in}{1.393161in}}%
\pgfpathcurveto{\pgfqpoint{4.075464in}{1.386028in}}{\pgfqpoint{4.078298in}{1.379187in}}{\pgfqpoint{4.083342in}{1.374143in}}%
\pgfpathcurveto{\pgfqpoint{4.088385in}{1.369099in}}{\pgfqpoint{4.095227in}{1.366266in}}{\pgfqpoint{4.102360in}{1.366266in}}%
\pgfpathclose%
\pgfusepath{stroke,fill}%
\end{pgfscope}%
\begin{pgfscope}%
\pgfpathrectangle{\pgfqpoint{2.867647in}{0.500000in}}{\pgfqpoint{1.764706in}{1.700000in}}%
\pgfusepath{clip}%
\pgfsetbuttcap%
\pgfsetroundjoin%
\definecolor{currentfill}{rgb}{0.956817,0.498820,0.345554}%
\pgfsetfillcolor{currentfill}%
\pgfsetlinewidth{0.311001pt}%
\definecolor{currentstroke}{rgb}{1.000000,1.000000,1.000000}%
\pgfsetstrokecolor{currentstroke}%
\pgfsetdash{}{0pt}%
\pgfpathmoveto{\pgfqpoint{3.983429in}{0.827239in}}%
\pgfpathcurveto{\pgfqpoint{3.990562in}{0.827239in}}{\pgfqpoint{3.997404in}{0.830072in}}{\pgfqpoint{4.002448in}{0.835116in}}%
\pgfpathcurveto{\pgfqpoint{4.007491in}{0.840160in}}{\pgfqpoint{4.010325in}{0.847001in}}{\pgfqpoint{4.010325in}{0.854134in}}%
\pgfpathcurveto{\pgfqpoint{4.010325in}{0.861267in}}{\pgfqpoint{4.007491in}{0.868109in}}{\pgfqpoint{4.002448in}{0.873152in}}%
\pgfpathcurveto{\pgfqpoint{3.997404in}{0.878196in}}{\pgfqpoint{3.990562in}{0.881030in}}{\pgfqpoint{3.983429in}{0.881030in}}%
\pgfpathcurveto{\pgfqpoint{3.976297in}{0.881030in}}{\pgfqpoint{3.969455in}{0.878196in}}{\pgfqpoint{3.964411in}{0.873152in}}%
\pgfpathcurveto{\pgfqpoint{3.959368in}{0.868109in}}{\pgfqpoint{3.956534in}{0.861267in}}{\pgfqpoint{3.956534in}{0.854134in}}%
\pgfpathcurveto{\pgfqpoint{3.956534in}{0.847001in}}{\pgfqpoint{3.959368in}{0.840160in}}{\pgfqpoint{3.964411in}{0.835116in}}%
\pgfpathcurveto{\pgfqpoint{3.969455in}{0.830072in}}{\pgfqpoint{3.976297in}{0.827239in}}{\pgfqpoint{3.983429in}{0.827239in}}%
\pgfpathclose%
\pgfusepath{stroke,fill}%
\end{pgfscope}%
\begin{pgfscope}%
\pgfpathrectangle{\pgfqpoint{2.867647in}{0.500000in}}{\pgfqpoint{1.764706in}{1.700000in}}%
\pgfusepath{clip}%
\pgfsetbuttcap%
\pgfsetroundjoin%
\definecolor{currentfill}{rgb}{0.922239,0.282873,0.242296}%
\pgfsetfillcolor{currentfill}%
\pgfsetlinewidth{0.311001pt}%
\definecolor{currentstroke}{rgb}{1.000000,1.000000,1.000000}%
\pgfsetstrokecolor{currentstroke}%
\pgfsetdash{}{0pt}%
\pgfpathmoveto{\pgfqpoint{3.866072in}{1.840712in}}%
\pgfpathcurveto{\pgfqpoint{3.873204in}{1.840712in}}{\pgfqpoint{3.880046in}{1.843546in}}{\pgfqpoint{3.885090in}{1.848589in}}%
\pgfpathcurveto{\pgfqpoint{3.890133in}{1.853633in}}{\pgfqpoint{3.892967in}{1.860475in}}{\pgfqpoint{3.892967in}{1.867608in}}%
\pgfpathcurveto{\pgfqpoint{3.892967in}{1.874740in}}{\pgfqpoint{3.890133in}{1.881582in}}{\pgfqpoint{3.885090in}{1.886626in}}%
\pgfpathcurveto{\pgfqpoint{3.880046in}{1.891669in}}{\pgfqpoint{3.873204in}{1.894503in}}{\pgfqpoint{3.866072in}{1.894503in}}%
\pgfpathcurveto{\pgfqpoint{3.858939in}{1.894503in}}{\pgfqpoint{3.852097in}{1.891669in}}{\pgfqpoint{3.847054in}{1.886626in}}%
\pgfpathcurveto{\pgfqpoint{3.842010in}{1.881582in}}{\pgfqpoint{3.839176in}{1.874740in}}{\pgfqpoint{3.839176in}{1.867608in}}%
\pgfpathcurveto{\pgfqpoint{3.839176in}{1.860475in}}{\pgfqpoint{3.842010in}{1.853633in}}{\pgfqpoint{3.847054in}{1.848589in}}%
\pgfpathcurveto{\pgfqpoint{3.852097in}{1.843546in}}{\pgfqpoint{3.858939in}{1.840712in}}{\pgfqpoint{3.866072in}{1.840712in}}%
\pgfpathclose%
\pgfusepath{stroke,fill}%
\end{pgfscope}%
\begin{pgfscope}%
\pgfpathrectangle{\pgfqpoint{2.867647in}{0.500000in}}{\pgfqpoint{1.764706in}{1.700000in}}%
\pgfusepath{clip}%
\pgfsetbuttcap%
\pgfsetroundjoin%
\definecolor{currentfill}{rgb}{0.969803,0.809811,0.702523}%
\pgfsetfillcolor{currentfill}%
\pgfsetlinewidth{0.311001pt}%
\definecolor{currentstroke}{rgb}{1.000000,1.000000,1.000000}%
\pgfsetstrokecolor{currentstroke}%
\pgfsetdash{}{0pt}%
\pgfpathmoveto{\pgfqpoint{4.275138in}{1.282795in}}%
\pgfpathcurveto{\pgfqpoint{4.282271in}{1.282795in}}{\pgfqpoint{4.289113in}{1.285629in}}{\pgfqpoint{4.294156in}{1.290673in}}%
\pgfpathcurveto{\pgfqpoint{4.299200in}{1.295717in}}{\pgfqpoint{4.302034in}{1.302558in}}{\pgfqpoint{4.302034in}{1.309691in}}%
\pgfpathcurveto{\pgfqpoint{4.302034in}{1.316824in}}{\pgfqpoint{4.299200in}{1.323665in}}{\pgfqpoint{4.294156in}{1.328709in}}%
\pgfpathcurveto{\pgfqpoint{4.289113in}{1.333753in}}{\pgfqpoint{4.282271in}{1.336587in}}{\pgfqpoint{4.275138in}{1.336587in}}%
\pgfpathcurveto{\pgfqpoint{4.268005in}{1.336587in}}{\pgfqpoint{4.261164in}{1.333753in}}{\pgfqpoint{4.256120in}{1.328709in}}%
\pgfpathcurveto{\pgfqpoint{4.251076in}{1.323665in}}{\pgfqpoint{4.248242in}{1.316824in}}{\pgfqpoint{4.248242in}{1.309691in}}%
\pgfpathcurveto{\pgfqpoint{4.248242in}{1.302558in}}{\pgfqpoint{4.251076in}{1.295717in}}{\pgfqpoint{4.256120in}{1.290673in}}%
\pgfpathcurveto{\pgfqpoint{4.261164in}{1.285629in}}{\pgfqpoint{4.268005in}{1.282795in}}{\pgfqpoint{4.275138in}{1.282795in}}%
\pgfpathclose%
\pgfusepath{stroke,fill}%
\end{pgfscope}%
\begin{pgfscope}%
\pgfpathrectangle{\pgfqpoint{2.867647in}{0.500000in}}{\pgfqpoint{1.764706in}{1.700000in}}%
\pgfusepath{clip}%
\pgfsetbuttcap%
\pgfsetroundjoin%
\definecolor{currentfill}{rgb}{0.971202,0.827364,0.728520}%
\pgfsetfillcolor{currentfill}%
\pgfsetlinewidth{0.311001pt}%
\definecolor{currentstroke}{rgb}{1.000000,1.000000,1.000000}%
\pgfsetstrokecolor{currentstroke}%
\pgfsetdash{}{0pt}%
\pgfpathmoveto{\pgfqpoint{4.063720in}{1.545579in}}%
\pgfpathcurveto{\pgfqpoint{4.070853in}{1.545579in}}{\pgfqpoint{4.077694in}{1.548412in}}{\pgfqpoint{4.082738in}{1.553456in}}%
\pgfpathcurveto{\pgfqpoint{4.087782in}{1.558500in}}{\pgfqpoint{4.090615in}{1.565341in}}{\pgfqpoint{4.090615in}{1.572474in}}%
\pgfpathcurveto{\pgfqpoint{4.090615in}{1.579607in}}{\pgfqpoint{4.087782in}{1.586449in}}{\pgfqpoint{4.082738in}{1.591492in}}%
\pgfpathcurveto{\pgfqpoint{4.077694in}{1.596536in}}{\pgfqpoint{4.070853in}{1.599370in}}{\pgfqpoint{4.063720in}{1.599370in}}%
\pgfpathcurveto{\pgfqpoint{4.056587in}{1.599370in}}{\pgfqpoint{4.049745in}{1.596536in}}{\pgfqpoint{4.044702in}{1.591492in}}%
\pgfpathcurveto{\pgfqpoint{4.039658in}{1.586449in}}{\pgfqpoint{4.036824in}{1.579607in}}{\pgfqpoint{4.036824in}{1.572474in}}%
\pgfpathcurveto{\pgfqpoint{4.036824in}{1.565341in}}{\pgfqpoint{4.039658in}{1.558500in}}{\pgfqpoint{4.044702in}{1.553456in}}%
\pgfpathcurveto{\pgfqpoint{4.049745in}{1.548412in}}{\pgfqpoint{4.056587in}{1.545579in}}{\pgfqpoint{4.063720in}{1.545579in}}%
\pgfpathclose%
\pgfusepath{stroke,fill}%
\end{pgfscope}%
\begin{pgfscope}%
\pgfpathrectangle{\pgfqpoint{2.867647in}{0.500000in}}{\pgfqpoint{1.764706in}{1.700000in}}%
\pgfusepath{clip}%
\pgfsetbuttcap%
\pgfsetroundjoin%
\definecolor{currentfill}{rgb}{0.972726,0.844889,0.754401}%
\pgfsetfillcolor{currentfill}%
\pgfsetlinewidth{0.311001pt}%
\definecolor{currentstroke}{rgb}{1.000000,1.000000,1.000000}%
\pgfsetstrokecolor{currentstroke}%
\pgfsetdash{}{0pt}%
\pgfpathmoveto{\pgfqpoint{4.259578in}{1.364757in}}%
\pgfpathcurveto{\pgfqpoint{4.266711in}{1.364757in}}{\pgfqpoint{4.273552in}{1.367590in}}{\pgfqpoint{4.278596in}{1.372634in}}%
\pgfpathcurveto{\pgfqpoint{4.283640in}{1.377678in}}{\pgfqpoint{4.286474in}{1.384519in}}{\pgfqpoint{4.286474in}{1.391652in}}%
\pgfpathcurveto{\pgfqpoint{4.286474in}{1.398785in}}{\pgfqpoint{4.283640in}{1.405627in}}{\pgfqpoint{4.278596in}{1.410670in}}%
\pgfpathcurveto{\pgfqpoint{4.273552in}{1.415714in}}{\pgfqpoint{4.266711in}{1.418548in}}{\pgfqpoint{4.259578in}{1.418548in}}%
\pgfpathcurveto{\pgfqpoint{4.252445in}{1.418548in}}{\pgfqpoint{4.245603in}{1.415714in}}{\pgfqpoint{4.240560in}{1.410670in}}%
\pgfpathcurveto{\pgfqpoint{4.235516in}{1.405627in}}{\pgfqpoint{4.232682in}{1.398785in}}{\pgfqpoint{4.232682in}{1.391652in}}%
\pgfpathcurveto{\pgfqpoint{4.232682in}{1.384519in}}{\pgfqpoint{4.235516in}{1.377678in}}{\pgfqpoint{4.240560in}{1.372634in}}%
\pgfpathcurveto{\pgfqpoint{4.245603in}{1.367590in}}{\pgfqpoint{4.252445in}{1.364757in}}{\pgfqpoint{4.259578in}{1.364757in}}%
\pgfpathclose%
\pgfusepath{stroke,fill}%
\end{pgfscope}%
\begin{pgfscope}%
\pgfpathrectangle{\pgfqpoint{2.867647in}{0.500000in}}{\pgfqpoint{1.764706in}{1.700000in}}%
\pgfusepath{clip}%
\pgfsetbuttcap%
\pgfsetroundjoin%
\definecolor{currentfill}{rgb}{0.973832,0.856556,0.771584}%
\pgfsetfillcolor{currentfill}%
\pgfsetlinewidth{0.311001pt}%
\definecolor{currentstroke}{rgb}{1.000000,1.000000,1.000000}%
\pgfsetstrokecolor{currentstroke}%
\pgfsetdash{}{0pt}%
\pgfpathmoveto{\pgfqpoint{4.254449in}{1.279760in}}%
\pgfpathcurveto{\pgfqpoint{4.261582in}{1.279760in}}{\pgfqpoint{4.268424in}{1.282594in}}{\pgfqpoint{4.273467in}{1.287638in}}%
\pgfpathcurveto{\pgfqpoint{4.278511in}{1.292681in}}{\pgfqpoint{4.281345in}{1.299523in}}{\pgfqpoint{4.281345in}{1.306656in}}%
\pgfpathcurveto{\pgfqpoint{4.281345in}{1.313789in}}{\pgfqpoint{4.278511in}{1.320630in}}{\pgfqpoint{4.273467in}{1.325674in}}%
\pgfpathcurveto{\pgfqpoint{4.268424in}{1.330718in}}{\pgfqpoint{4.261582in}{1.333552in}}{\pgfqpoint{4.254449in}{1.333552in}}%
\pgfpathcurveto{\pgfqpoint{4.247316in}{1.333552in}}{\pgfqpoint{4.240475in}{1.330718in}}{\pgfqpoint{4.235431in}{1.325674in}}%
\pgfpathcurveto{\pgfqpoint{4.230387in}{1.320630in}}{\pgfqpoint{4.227554in}{1.313789in}}{\pgfqpoint{4.227554in}{1.306656in}}%
\pgfpathcurveto{\pgfqpoint{4.227554in}{1.299523in}}{\pgfqpoint{4.230387in}{1.292681in}}{\pgfqpoint{4.235431in}{1.287638in}}%
\pgfpathcurveto{\pgfqpoint{4.240475in}{1.282594in}}{\pgfqpoint{4.247316in}{1.279760in}}{\pgfqpoint{4.254449in}{1.279760in}}%
\pgfpathclose%
\pgfusepath{stroke,fill}%
\end{pgfscope}%
\begin{pgfscope}%
\pgfpathrectangle{\pgfqpoint{2.867647in}{0.500000in}}{\pgfqpoint{1.764706in}{1.700000in}}%
\pgfusepath{clip}%
\pgfsetbuttcap%
\pgfsetroundjoin%
\definecolor{currentfill}{rgb}{0.979124,0.903132,0.839793}%
\pgfsetfillcolor{currentfill}%
\pgfsetlinewidth{0.311001pt}%
\definecolor{currentstroke}{rgb}{1.000000,1.000000,1.000000}%
\pgfsetstrokecolor{currentstroke}%
\pgfsetdash{}{0pt}%
\pgfpathmoveto{\pgfqpoint{4.122020in}{1.588268in}}%
\pgfpathcurveto{\pgfqpoint{4.129153in}{1.588268in}}{\pgfqpoint{4.135995in}{1.591102in}}{\pgfqpoint{4.141038in}{1.596146in}}%
\pgfpathcurveto{\pgfqpoint{4.146082in}{1.601190in}}{\pgfqpoint{4.148916in}{1.608031in}}{\pgfqpoint{4.148916in}{1.615164in}}%
\pgfpathcurveto{\pgfqpoint{4.148916in}{1.622297in}}{\pgfqpoint{4.146082in}{1.629139in}}{\pgfqpoint{4.141038in}{1.634182in}}%
\pgfpathcurveto{\pgfqpoint{4.135995in}{1.639226in}}{\pgfqpoint{4.129153in}{1.642060in}}{\pgfqpoint{4.122020in}{1.642060in}}%
\pgfpathcurveto{\pgfqpoint{4.114887in}{1.642060in}}{\pgfqpoint{4.108046in}{1.639226in}}{\pgfqpoint{4.103002in}{1.634182in}}%
\pgfpathcurveto{\pgfqpoint{4.097958in}{1.629139in}}{\pgfqpoint{4.095124in}{1.622297in}}{\pgfqpoint{4.095124in}{1.615164in}}%
\pgfpathcurveto{\pgfqpoint{4.095124in}{1.608031in}}{\pgfqpoint{4.097958in}{1.601190in}}{\pgfqpoint{4.103002in}{1.596146in}}%
\pgfpathcurveto{\pgfqpoint{4.108046in}{1.591102in}}{\pgfqpoint{4.114887in}{1.588268in}}{\pgfqpoint{4.122020in}{1.588268in}}%
\pgfpathclose%
\pgfusepath{stroke,fill}%
\end{pgfscope}%
\begin{pgfscope}%
\pgfpathrectangle{\pgfqpoint{2.867647in}{0.500000in}}{\pgfqpoint{1.764706in}{1.700000in}}%
\pgfusepath{clip}%
\pgfsetbuttcap%
\pgfsetroundjoin%
\definecolor{currentfill}{rgb}{0.964679,0.682838,0.530002}%
\pgfsetfillcolor{currentfill}%
\pgfsetlinewidth{0.311001pt}%
\definecolor{currentstroke}{rgb}{1.000000,1.000000,1.000000}%
\pgfsetstrokecolor{currentstroke}%
\pgfsetdash{}{0pt}%
\pgfpathmoveto{\pgfqpoint{4.078503in}{0.881391in}}%
\pgfpathcurveto{\pgfqpoint{4.085636in}{0.881391in}}{\pgfqpoint{4.092478in}{0.884225in}}{\pgfqpoint{4.097521in}{0.889269in}}%
\pgfpathcurveto{\pgfqpoint{4.102565in}{0.894313in}}{\pgfqpoint{4.105399in}{0.901154in}}{\pgfqpoint{4.105399in}{0.908287in}}%
\pgfpathcurveto{\pgfqpoint{4.105399in}{0.915420in}}{\pgfqpoint{4.102565in}{0.922262in}}{\pgfqpoint{4.097521in}{0.927305in}}%
\pgfpathcurveto{\pgfqpoint{4.092478in}{0.932349in}}{\pgfqpoint{4.085636in}{0.935183in}}{\pgfqpoint{4.078503in}{0.935183in}}%
\pgfpathcurveto{\pgfqpoint{4.071370in}{0.935183in}}{\pgfqpoint{4.064529in}{0.932349in}}{\pgfqpoint{4.059485in}{0.927305in}}%
\pgfpathcurveto{\pgfqpoint{4.054441in}{0.922262in}}{\pgfqpoint{4.051607in}{0.915420in}}{\pgfqpoint{4.051607in}{0.908287in}}%
\pgfpathcurveto{\pgfqpoint{4.051607in}{0.901154in}}{\pgfqpoint{4.054441in}{0.894313in}}{\pgfqpoint{4.059485in}{0.889269in}}%
\pgfpathcurveto{\pgfqpoint{4.064529in}{0.884225in}}{\pgfqpoint{4.071370in}{0.881391in}}{\pgfqpoint{4.078503in}{0.881391in}}%
\pgfpathclose%
\pgfusepath{stroke,fill}%
\end{pgfscope}%
\begin{pgfscope}%
\pgfpathrectangle{\pgfqpoint{2.867647in}{0.500000in}}{\pgfqpoint{1.764706in}{1.700000in}}%
\pgfusepath{clip}%
\pgfsetbuttcap%
\pgfsetroundjoin%
\definecolor{currentfill}{rgb}{0.973832,0.856556,0.771584}%
\pgfsetfillcolor{currentfill}%
\pgfsetlinewidth{0.311001pt}%
\definecolor{currentstroke}{rgb}{1.000000,1.000000,1.000000}%
\pgfsetstrokecolor{currentstroke}%
\pgfsetdash{}{0pt}%
\pgfpathmoveto{\pgfqpoint{4.247910in}{1.412859in}}%
\pgfpathcurveto{\pgfqpoint{4.255043in}{1.412859in}}{\pgfqpoint{4.261884in}{1.415693in}}{\pgfqpoint{4.266928in}{1.420737in}}%
\pgfpathcurveto{\pgfqpoint{4.271972in}{1.425781in}}{\pgfqpoint{4.274805in}{1.432622in}}{\pgfqpoint{4.274805in}{1.439755in}}%
\pgfpathcurveto{\pgfqpoint{4.274805in}{1.446888in}}{\pgfqpoint{4.271972in}{1.453730in}}{\pgfqpoint{4.266928in}{1.458773in}}%
\pgfpathcurveto{\pgfqpoint{4.261884in}{1.463817in}}{\pgfqpoint{4.255043in}{1.466651in}}{\pgfqpoint{4.247910in}{1.466651in}}%
\pgfpathcurveto{\pgfqpoint{4.240777in}{1.466651in}}{\pgfqpoint{4.233935in}{1.463817in}}{\pgfqpoint{4.228892in}{1.458773in}}%
\pgfpathcurveto{\pgfqpoint{4.223848in}{1.453730in}}{\pgfqpoint{4.221014in}{1.446888in}}{\pgfqpoint{4.221014in}{1.439755in}}%
\pgfpathcurveto{\pgfqpoint{4.221014in}{1.432622in}}{\pgfqpoint{4.223848in}{1.425781in}}{\pgfqpoint{4.228892in}{1.420737in}}%
\pgfpathcurveto{\pgfqpoint{4.233935in}{1.415693in}}{\pgfqpoint{4.240777in}{1.412859in}}{\pgfqpoint{4.247910in}{1.412859in}}%
\pgfpathclose%
\pgfusepath{stroke,fill}%
\end{pgfscope}%
\begin{pgfscope}%
\pgfpathrectangle{\pgfqpoint{2.867647in}{0.500000in}}{\pgfqpoint{1.764706in}{1.700000in}}%
\pgfusepath{clip}%
\pgfsetbuttcap%
\pgfsetroundjoin%
\definecolor{currentfill}{rgb}{0.972726,0.844889,0.754401}%
\pgfsetfillcolor{currentfill}%
\pgfsetlinewidth{0.311001pt}%
\definecolor{currentstroke}{rgb}{1.000000,1.000000,1.000000}%
\pgfsetstrokecolor{currentstroke}%
\pgfsetdash{}{0pt}%
\pgfpathmoveto{\pgfqpoint{4.120681in}{1.306621in}}%
\pgfpathcurveto{\pgfqpoint{4.127814in}{1.306621in}}{\pgfqpoint{4.134655in}{1.309455in}}{\pgfqpoint{4.139699in}{1.314499in}}%
\pgfpathcurveto{\pgfqpoint{4.144743in}{1.319543in}}{\pgfqpoint{4.147576in}{1.326384in}}{\pgfqpoint{4.147576in}{1.333517in}}%
\pgfpathcurveto{\pgfqpoint{4.147576in}{1.340650in}}{\pgfqpoint{4.144743in}{1.347492in}}{\pgfqpoint{4.139699in}{1.352535in}}%
\pgfpathcurveto{\pgfqpoint{4.134655in}{1.357579in}}{\pgfqpoint{4.127814in}{1.360413in}}{\pgfqpoint{4.120681in}{1.360413in}}%
\pgfpathcurveto{\pgfqpoint{4.113548in}{1.360413in}}{\pgfqpoint{4.106706in}{1.357579in}}{\pgfqpoint{4.101663in}{1.352535in}}%
\pgfpathcurveto{\pgfqpoint{4.096619in}{1.347492in}}{\pgfqpoint{4.093785in}{1.340650in}}{\pgfqpoint{4.093785in}{1.333517in}}%
\pgfpathcurveto{\pgfqpoint{4.093785in}{1.326384in}}{\pgfqpoint{4.096619in}{1.319543in}}{\pgfqpoint{4.101663in}{1.314499in}}%
\pgfpathcurveto{\pgfqpoint{4.106706in}{1.309455in}}{\pgfqpoint{4.113548in}{1.306621in}}{\pgfqpoint{4.120681in}{1.306621in}}%
\pgfpathclose%
\pgfusepath{stroke,fill}%
\end{pgfscope}%
\begin{pgfscope}%
\pgfpathrectangle{\pgfqpoint{2.867647in}{0.500000in}}{\pgfqpoint{1.764706in}{1.700000in}}%
\pgfusepath{clip}%
\pgfsetbuttcap%
\pgfsetroundjoin%
\definecolor{currentfill}{rgb}{0.965042,0.701564,0.552889}%
\pgfsetfillcolor{currentfill}%
\pgfsetlinewidth{0.311001pt}%
\definecolor{currentstroke}{rgb}{1.000000,1.000000,1.000000}%
\pgfsetstrokecolor{currentstroke}%
\pgfsetdash{}{0pt}%
\pgfpathmoveto{\pgfqpoint{4.102006in}{1.757909in}}%
\pgfpathcurveto{\pgfqpoint{4.109139in}{1.757909in}}{\pgfqpoint{4.115981in}{1.760743in}}{\pgfqpoint{4.121024in}{1.765786in}}%
\pgfpathcurveto{\pgfqpoint{4.126068in}{1.770830in}}{\pgfqpoint{4.128902in}{1.777672in}}{\pgfqpoint{4.128902in}{1.784804in}}%
\pgfpathcurveto{\pgfqpoint{4.128902in}{1.791937in}}{\pgfqpoint{4.126068in}{1.798779in}}{\pgfqpoint{4.121024in}{1.803823in}}%
\pgfpathcurveto{\pgfqpoint{4.115981in}{1.808866in}}{\pgfqpoint{4.109139in}{1.811700in}}{\pgfqpoint{4.102006in}{1.811700in}}%
\pgfpathcurveto{\pgfqpoint{4.094873in}{1.811700in}}{\pgfqpoint{4.088032in}{1.808866in}}{\pgfqpoint{4.082988in}{1.803823in}}%
\pgfpathcurveto{\pgfqpoint{4.077944in}{1.798779in}}{\pgfqpoint{4.075110in}{1.791937in}}{\pgfqpoint{4.075110in}{1.784804in}}%
\pgfpathcurveto{\pgfqpoint{4.075110in}{1.777672in}}{\pgfqpoint{4.077944in}{1.770830in}}{\pgfqpoint{4.082988in}{1.765786in}}%
\pgfpathcurveto{\pgfqpoint{4.088032in}{1.760743in}}{\pgfqpoint{4.094873in}{1.757909in}}{\pgfqpoint{4.102006in}{1.757909in}}%
\pgfpathclose%
\pgfusepath{stroke,fill}%
\end{pgfscope}%
\begin{pgfscope}%
\pgfpathrectangle{\pgfqpoint{2.867647in}{0.500000in}}{\pgfqpoint{1.764706in}{1.700000in}}%
\pgfusepath{clip}%
\pgfsetbuttcap%
\pgfsetroundjoin%
\definecolor{currentfill}{rgb}{0.979124,0.903132,0.839793}%
\pgfsetfillcolor{currentfill}%
\pgfsetlinewidth{0.311001pt}%
\definecolor{currentstroke}{rgb}{1.000000,1.000000,1.000000}%
\pgfsetstrokecolor{currentstroke}%
\pgfsetdash{}{0pt}%
\pgfpathmoveto{\pgfqpoint{4.128045in}{1.100925in}}%
\pgfpathcurveto{\pgfqpoint{4.135178in}{1.100925in}}{\pgfqpoint{4.142019in}{1.103759in}}{\pgfqpoint{4.147063in}{1.108803in}}%
\pgfpathcurveto{\pgfqpoint{4.152107in}{1.113846in}}{\pgfqpoint{4.154941in}{1.120688in}}{\pgfqpoint{4.154941in}{1.127821in}}%
\pgfpathcurveto{\pgfqpoint{4.154941in}{1.134954in}}{\pgfqpoint{4.152107in}{1.141795in}}{\pgfqpoint{4.147063in}{1.146839in}}%
\pgfpathcurveto{\pgfqpoint{4.142019in}{1.151883in}}{\pgfqpoint{4.135178in}{1.154717in}}{\pgfqpoint{4.128045in}{1.154717in}}%
\pgfpathcurveto{\pgfqpoint{4.120912in}{1.154717in}}{\pgfqpoint{4.114070in}{1.151883in}}{\pgfqpoint{4.109027in}{1.146839in}}%
\pgfpathcurveto{\pgfqpoint{4.103983in}{1.141795in}}{\pgfqpoint{4.101149in}{1.134954in}}{\pgfqpoint{4.101149in}{1.127821in}}%
\pgfpathcurveto{\pgfqpoint{4.101149in}{1.120688in}}{\pgfqpoint{4.103983in}{1.113846in}}{\pgfqpoint{4.109027in}{1.108803in}}%
\pgfpathcurveto{\pgfqpoint{4.114070in}{1.103759in}}{\pgfqpoint{4.120912in}{1.100925in}}{\pgfqpoint{4.128045in}{1.100925in}}%
\pgfpathclose%
\pgfusepath{stroke,fill}%
\end{pgfscope}%
\begin{pgfscope}%
\pgfpathrectangle{\pgfqpoint{2.867647in}{0.500000in}}{\pgfqpoint{1.764706in}{1.700000in}}%
\pgfusepath{clip}%
\pgfsetbuttcap%
\pgfsetroundjoin%
\definecolor{currentfill}{rgb}{0.971202,0.827364,0.728520}%
\pgfsetfillcolor{currentfill}%
\pgfsetlinewidth{0.311001pt}%
\definecolor{currentstroke}{rgb}{1.000000,1.000000,1.000000}%
\pgfsetstrokecolor{currentstroke}%
\pgfsetdash{}{0pt}%
\pgfpathmoveto{\pgfqpoint{4.131518in}{0.972915in}}%
\pgfpathcurveto{\pgfqpoint{4.138651in}{0.972915in}}{\pgfqpoint{4.145492in}{0.975749in}}{\pgfqpoint{4.150536in}{0.980793in}}%
\pgfpathcurveto{\pgfqpoint{4.155580in}{0.985837in}}{\pgfqpoint{4.158414in}{0.992678in}}{\pgfqpoint{4.158414in}{0.999811in}}%
\pgfpathcurveto{\pgfqpoint{4.158414in}{1.006944in}}{\pgfqpoint{4.155580in}{1.013786in}}{\pgfqpoint{4.150536in}{1.018829in}}%
\pgfpathcurveto{\pgfqpoint{4.145492in}{1.023873in}}{\pgfqpoint{4.138651in}{1.026707in}}{\pgfqpoint{4.131518in}{1.026707in}}%
\pgfpathcurveto{\pgfqpoint{4.124385in}{1.026707in}}{\pgfqpoint{4.117543in}{1.023873in}}{\pgfqpoint{4.112500in}{1.018829in}}%
\pgfpathcurveto{\pgfqpoint{4.107456in}{1.013786in}}{\pgfqpoint{4.104622in}{1.006944in}}{\pgfqpoint{4.104622in}{0.999811in}}%
\pgfpathcurveto{\pgfqpoint{4.104622in}{0.992678in}}{\pgfqpoint{4.107456in}{0.985837in}}{\pgfqpoint{4.112500in}{0.980793in}}%
\pgfpathcurveto{\pgfqpoint{4.117543in}{0.975749in}}{\pgfqpoint{4.124385in}{0.972915in}}{\pgfqpoint{4.131518in}{0.972915in}}%
\pgfpathclose%
\pgfusepath{stroke,fill}%
\end{pgfscope}%
\begin{pgfscope}%
\pgfpathrectangle{\pgfqpoint{2.867647in}{0.500000in}}{\pgfqpoint{1.764706in}{1.700000in}}%
\pgfusepath{clip}%
\pgfsetbuttcap%
\pgfsetroundjoin%
\definecolor{currentfill}{rgb}{0.978376,0.897317,0.831308}%
\pgfsetfillcolor{currentfill}%
\pgfsetlinewidth{0.311001pt}%
\definecolor{currentstroke}{rgb}{1.000000,1.000000,1.000000}%
\pgfsetstrokecolor{currentstroke}%
\pgfsetdash{}{0pt}%
\pgfpathmoveto{\pgfqpoint{4.146518in}{1.066452in}}%
\pgfpathcurveto{\pgfqpoint{4.153651in}{1.066452in}}{\pgfqpoint{4.160492in}{1.069286in}}{\pgfqpoint{4.165536in}{1.074329in}}%
\pgfpathcurveto{\pgfqpoint{4.170580in}{1.079373in}}{\pgfqpoint{4.173414in}{1.086215in}}{\pgfqpoint{4.173414in}{1.093348in}}%
\pgfpathcurveto{\pgfqpoint{4.173414in}{1.100480in}}{\pgfqpoint{4.170580in}{1.107322in}}{\pgfqpoint{4.165536in}{1.112366in}}%
\pgfpathcurveto{\pgfqpoint{4.160492in}{1.117409in}}{\pgfqpoint{4.153651in}{1.120243in}}{\pgfqpoint{4.146518in}{1.120243in}}%
\pgfpathcurveto{\pgfqpoint{4.139385in}{1.120243in}}{\pgfqpoint{4.132544in}{1.117409in}}{\pgfqpoint{4.127500in}{1.112366in}}%
\pgfpathcurveto{\pgfqpoint{4.122456in}{1.107322in}}{\pgfqpoint{4.119622in}{1.100480in}}{\pgfqpoint{4.119622in}{1.093348in}}%
\pgfpathcurveto{\pgfqpoint{4.119622in}{1.086215in}}{\pgfqpoint{4.122456in}{1.079373in}}{\pgfqpoint{4.127500in}{1.074329in}}%
\pgfpathcurveto{\pgfqpoint{4.132544in}{1.069286in}}{\pgfqpoint{4.139385in}{1.066452in}}{\pgfqpoint{4.146518in}{1.066452in}}%
\pgfpathclose%
\pgfusepath{stroke,fill}%
\end{pgfscope}%
\begin{pgfscope}%
\pgfpathrectangle{\pgfqpoint{2.867647in}{0.500000in}}{\pgfqpoint{1.764706in}{1.700000in}}%
\pgfusepath{clip}%
\pgfsetbuttcap%
\pgfsetroundjoin%
\definecolor{currentfill}{rgb}{0.962018,0.586477,0.424918}%
\pgfsetfillcolor{currentfill}%
\pgfsetlinewidth{0.311001pt}%
\definecolor{currentstroke}{rgb}{1.000000,1.000000,1.000000}%
\pgfsetstrokecolor{currentstroke}%
\pgfsetdash{}{0pt}%
\pgfpathmoveto{\pgfqpoint{4.270020in}{1.619920in}}%
\pgfpathcurveto{\pgfqpoint{4.277152in}{1.619920in}}{\pgfqpoint{4.283994in}{1.622754in}}{\pgfqpoint{4.289038in}{1.627797in}}%
\pgfpathcurveto{\pgfqpoint{4.294081in}{1.632841in}}{\pgfqpoint{4.296915in}{1.639683in}}{\pgfqpoint{4.296915in}{1.646816in}}%
\pgfpathcurveto{\pgfqpoint{4.296915in}{1.653948in}}{\pgfqpoint{4.294081in}{1.660790in}}{\pgfqpoint{4.289038in}{1.665834in}}%
\pgfpathcurveto{\pgfqpoint{4.283994in}{1.670877in}}{\pgfqpoint{4.277152in}{1.673711in}}{\pgfqpoint{4.270020in}{1.673711in}}%
\pgfpathcurveto{\pgfqpoint{4.262887in}{1.673711in}}{\pgfqpoint{4.256045in}{1.670877in}}{\pgfqpoint{4.251001in}{1.665834in}}%
\pgfpathcurveto{\pgfqpoint{4.245958in}{1.660790in}}{\pgfqpoint{4.243124in}{1.653948in}}{\pgfqpoint{4.243124in}{1.646816in}}%
\pgfpathcurveto{\pgfqpoint{4.243124in}{1.639683in}}{\pgfqpoint{4.245958in}{1.632841in}}{\pgfqpoint{4.251001in}{1.627797in}}%
\pgfpathcurveto{\pgfqpoint{4.256045in}{1.622754in}}{\pgfqpoint{4.262887in}{1.619920in}}{\pgfqpoint{4.270020in}{1.619920in}}%
\pgfpathclose%
\pgfusepath{stroke,fill}%
\end{pgfscope}%
\begin{pgfscope}%
\pgfpathrectangle{\pgfqpoint{2.867647in}{0.500000in}}{\pgfqpoint{1.764706in}{1.700000in}}%
\pgfusepath{clip}%
\pgfsetbuttcap%
\pgfsetroundjoin%
\definecolor{currentfill}{rgb}{0.960043,0.546576,0.387029}%
\pgfsetfillcolor{currentfill}%
\pgfsetlinewidth{0.311001pt}%
\definecolor{currentstroke}{rgb}{1.000000,1.000000,1.000000}%
\pgfsetstrokecolor{currentstroke}%
\pgfsetdash{}{0pt}%
\pgfpathmoveto{\pgfqpoint{3.928087in}{0.947136in}}%
\pgfpathcurveto{\pgfqpoint{3.935220in}{0.947136in}}{\pgfqpoint{3.942062in}{0.949970in}}{\pgfqpoint{3.947106in}{0.955013in}}%
\pgfpathcurveto{\pgfqpoint{3.952149in}{0.960057in}}{\pgfqpoint{3.954983in}{0.966899in}}{\pgfqpoint{3.954983in}{0.974032in}}%
\pgfpathcurveto{\pgfqpoint{3.954983in}{0.981164in}}{\pgfqpoint{3.952149in}{0.988006in}}{\pgfqpoint{3.947106in}{0.993050in}}%
\pgfpathcurveto{\pgfqpoint{3.942062in}{0.998093in}}{\pgfqpoint{3.935220in}{1.000927in}}{\pgfqpoint{3.928087in}{1.000927in}}%
\pgfpathcurveto{\pgfqpoint{3.920955in}{1.000927in}}{\pgfqpoint{3.914113in}{0.998093in}}{\pgfqpoint{3.909069in}{0.993050in}}%
\pgfpathcurveto{\pgfqpoint{3.904026in}{0.988006in}}{\pgfqpoint{3.901192in}{0.981164in}}{\pgfqpoint{3.901192in}{0.974032in}}%
\pgfpathcurveto{\pgfqpoint{3.901192in}{0.966899in}}{\pgfqpoint{3.904026in}{0.960057in}}{\pgfqpoint{3.909069in}{0.955013in}}%
\pgfpathcurveto{\pgfqpoint{3.914113in}{0.949970in}}{\pgfqpoint{3.920955in}{0.947136in}}{\pgfqpoint{3.928087in}{0.947136in}}%
\pgfpathclose%
\pgfusepath{stroke,fill}%
\end{pgfscope}%
\begin{pgfscope}%
\pgfpathrectangle{\pgfqpoint{2.867647in}{0.500000in}}{\pgfqpoint{1.764706in}{1.700000in}}%
\pgfusepath{clip}%
\pgfsetbuttcap%
\pgfsetroundjoin%
\definecolor{currentfill}{rgb}{0.979891,0.908948,0.848279}%
\pgfsetfillcolor{currentfill}%
\pgfsetlinewidth{0.311001pt}%
\definecolor{currentstroke}{rgb}{1.000000,1.000000,1.000000}%
\pgfsetstrokecolor{currentstroke}%
\pgfsetdash{}{0pt}%
\pgfpathmoveto{\pgfqpoint{4.148279in}{1.561820in}}%
\pgfpathcurveto{\pgfqpoint{4.155412in}{1.561820in}}{\pgfqpoint{4.162254in}{1.564654in}}{\pgfqpoint{4.167297in}{1.569697in}}%
\pgfpathcurveto{\pgfqpoint{4.172341in}{1.574741in}}{\pgfqpoint{4.175175in}{1.581583in}}{\pgfqpoint{4.175175in}{1.588715in}}%
\pgfpathcurveto{\pgfqpoint{4.175175in}{1.595848in}}{\pgfqpoint{4.172341in}{1.602690in}}{\pgfqpoint{4.167297in}{1.607734in}}%
\pgfpathcurveto{\pgfqpoint{4.162254in}{1.612777in}}{\pgfqpoint{4.155412in}{1.615611in}}{\pgfqpoint{4.148279in}{1.615611in}}%
\pgfpathcurveto{\pgfqpoint{4.141146in}{1.615611in}}{\pgfqpoint{4.134305in}{1.612777in}}{\pgfqpoint{4.129261in}{1.607734in}}%
\pgfpathcurveto{\pgfqpoint{4.124217in}{1.602690in}}{\pgfqpoint{4.121383in}{1.595848in}}{\pgfqpoint{4.121383in}{1.588715in}}%
\pgfpathcurveto{\pgfqpoint{4.121383in}{1.581583in}}{\pgfqpoint{4.124217in}{1.574741in}}{\pgfqpoint{4.129261in}{1.569697in}}%
\pgfpathcurveto{\pgfqpoint{4.134305in}{1.564654in}}{\pgfqpoint{4.141146in}{1.561820in}}{\pgfqpoint{4.148279in}{1.561820in}}%
\pgfpathclose%
\pgfusepath{stroke,fill}%
\end{pgfscope}%
\begin{pgfscope}%
\pgfpathrectangle{\pgfqpoint{2.867647in}{0.500000in}}{\pgfqpoint{1.764706in}{1.700000in}}%
\pgfusepath{clip}%
\pgfsetbuttcap%
\pgfsetroundjoin%
\definecolor{currentfill}{rgb}{0.967735,0.780441,0.659127}%
\pgfsetfillcolor{currentfill}%
\pgfsetlinewidth{0.311001pt}%
\definecolor{currentstroke}{rgb}{1.000000,1.000000,1.000000}%
\pgfsetstrokecolor{currentstroke}%
\pgfsetdash{}{0pt}%
\pgfpathmoveto{\pgfqpoint{4.171356in}{0.975996in}}%
\pgfpathcurveto{\pgfqpoint{4.178489in}{0.975996in}}{\pgfqpoint{4.185330in}{0.978830in}}{\pgfqpoint{4.190374in}{0.983874in}}%
\pgfpathcurveto{\pgfqpoint{4.195418in}{0.988918in}}{\pgfqpoint{4.198251in}{0.995759in}}{\pgfqpoint{4.198251in}{1.002892in}}%
\pgfpathcurveto{\pgfqpoint{4.198251in}{1.010025in}}{\pgfqpoint{4.195418in}{1.016866in}}{\pgfqpoint{4.190374in}{1.021910in}}%
\pgfpathcurveto{\pgfqpoint{4.185330in}{1.026954in}}{\pgfqpoint{4.178489in}{1.029788in}}{\pgfqpoint{4.171356in}{1.029788in}}%
\pgfpathcurveto{\pgfqpoint{4.164223in}{1.029788in}}{\pgfqpoint{4.157381in}{1.026954in}}{\pgfqpoint{4.152338in}{1.021910in}}%
\pgfpathcurveto{\pgfqpoint{4.147294in}{1.016866in}}{\pgfqpoint{4.144460in}{1.010025in}}{\pgfqpoint{4.144460in}{1.002892in}}%
\pgfpathcurveto{\pgfqpoint{4.144460in}{0.995759in}}{\pgfqpoint{4.147294in}{0.988918in}}{\pgfqpoint{4.152338in}{0.983874in}}%
\pgfpathcurveto{\pgfqpoint{4.157381in}{0.978830in}}{\pgfqpoint{4.164223in}{0.975996in}}{\pgfqpoint{4.171356in}{0.975996in}}%
\pgfpathclose%
\pgfusepath{stroke,fill}%
\end{pgfscope}%
\begin{pgfscope}%
\pgfpathrectangle{\pgfqpoint{2.867647in}{0.500000in}}{\pgfqpoint{1.764706in}{1.700000in}}%
\pgfusepath{clip}%
\pgfsetbuttcap%
\pgfsetroundjoin%
\definecolor{currentfill}{rgb}{0.979124,0.903132,0.839793}%
\pgfsetfillcolor{currentfill}%
\pgfsetlinewidth{0.311001pt}%
\definecolor{currentstroke}{rgb}{1.000000,1.000000,1.000000}%
\pgfsetstrokecolor{currentstroke}%
\pgfsetdash{}{0pt}%
\pgfpathmoveto{\pgfqpoint{4.223413in}{1.333485in}}%
\pgfpathcurveto{\pgfqpoint{4.230546in}{1.333485in}}{\pgfqpoint{4.237388in}{1.336319in}}{\pgfqpoint{4.242432in}{1.341362in}}%
\pgfpathcurveto{\pgfqpoint{4.247475in}{1.346406in}}{\pgfqpoint{4.250309in}{1.353248in}}{\pgfqpoint{4.250309in}{1.360380in}}%
\pgfpathcurveto{\pgfqpoint{4.250309in}{1.367513in}}{\pgfqpoint{4.247475in}{1.374355in}}{\pgfqpoint{4.242432in}{1.379398in}}%
\pgfpathcurveto{\pgfqpoint{4.237388in}{1.384442in}}{\pgfqpoint{4.230546in}{1.387276in}}{\pgfqpoint{4.223413in}{1.387276in}}%
\pgfpathcurveto{\pgfqpoint{4.216281in}{1.387276in}}{\pgfqpoint{4.209439in}{1.384442in}}{\pgfqpoint{4.204395in}{1.379398in}}%
\pgfpathcurveto{\pgfqpoint{4.199352in}{1.374355in}}{\pgfqpoint{4.196518in}{1.367513in}}{\pgfqpoint{4.196518in}{1.360380in}}%
\pgfpathcurveto{\pgfqpoint{4.196518in}{1.353248in}}{\pgfqpoint{4.199352in}{1.346406in}}{\pgfqpoint{4.204395in}{1.341362in}}%
\pgfpathcurveto{\pgfqpoint{4.209439in}{1.336319in}}{\pgfqpoint{4.216281in}{1.333485in}}{\pgfqpoint{4.223413in}{1.333485in}}%
\pgfpathclose%
\pgfusepath{stroke,fill}%
\end{pgfscope}%
\begin{pgfscope}%
\pgfpathrectangle{\pgfqpoint{2.867647in}{0.500000in}}{\pgfqpoint{1.764706in}{1.700000in}}%
\pgfusepath{clip}%
\pgfsetbuttcap%
\pgfsetroundjoin%
\definecolor{currentfill}{rgb}{0.964799,0.689101,0.537560}%
\pgfsetfillcolor{currentfill}%
\pgfsetlinewidth{0.311001pt}%
\definecolor{currentstroke}{rgb}{1.000000,1.000000,1.000000}%
\pgfsetstrokecolor{currentstroke}%
\pgfsetdash{}{0pt}%
\pgfpathmoveto{\pgfqpoint{3.975067in}{1.682862in}}%
\pgfpathcurveto{\pgfqpoint{3.982200in}{1.682862in}}{\pgfqpoint{3.989042in}{1.685696in}}{\pgfqpoint{3.994086in}{1.690740in}}%
\pgfpathcurveto{\pgfqpoint{3.999129in}{1.695783in}}{\pgfqpoint{4.001963in}{1.702625in}}{\pgfqpoint{4.001963in}{1.709758in}}%
\pgfpathcurveto{\pgfqpoint{4.001963in}{1.716891in}}{\pgfqpoint{3.999129in}{1.723732in}}{\pgfqpoint{3.994086in}{1.728776in}}%
\pgfpathcurveto{\pgfqpoint{3.989042in}{1.733820in}}{\pgfqpoint{3.982200in}{1.736654in}}{\pgfqpoint{3.975067in}{1.736654in}}%
\pgfpathcurveto{\pgfqpoint{3.967935in}{1.736654in}}{\pgfqpoint{3.961093in}{1.733820in}}{\pgfqpoint{3.956049in}{1.728776in}}%
\pgfpathcurveto{\pgfqpoint{3.951006in}{1.723732in}}{\pgfqpoint{3.948172in}{1.716891in}}{\pgfqpoint{3.948172in}{1.709758in}}%
\pgfpathcurveto{\pgfqpoint{3.948172in}{1.702625in}}{\pgfqpoint{3.951006in}{1.695783in}}{\pgfqpoint{3.956049in}{1.690740in}}%
\pgfpathcurveto{\pgfqpoint{3.961093in}{1.685696in}}{\pgfqpoint{3.967935in}{1.682862in}}{\pgfqpoint{3.975067in}{1.682862in}}%
\pgfpathclose%
\pgfusepath{stroke,fill}%
\end{pgfscope}%
\begin{pgfscope}%
\pgfpathrectangle{\pgfqpoint{2.867647in}{0.500000in}}{\pgfqpoint{1.764706in}{1.700000in}}%
\pgfusepath{clip}%
\pgfsetbuttcap%
\pgfsetroundjoin%
\definecolor{currentfill}{rgb}{0.964032,0.651225,0.493258}%
\pgfsetfillcolor{currentfill}%
\pgfsetlinewidth{0.311001pt}%
\definecolor{currentstroke}{rgb}{1.000000,1.000000,1.000000}%
\pgfsetstrokecolor{currentstroke}%
\pgfsetdash{}{0pt}%
\pgfpathmoveto{\pgfqpoint{4.320748in}{1.353040in}}%
\pgfpathcurveto{\pgfqpoint{4.327881in}{1.353040in}}{\pgfqpoint{4.334722in}{1.355874in}}{\pgfqpoint{4.339766in}{1.360918in}}%
\pgfpathcurveto{\pgfqpoint{4.344810in}{1.365961in}}{\pgfqpoint{4.347643in}{1.372803in}}{\pgfqpoint{4.347643in}{1.379936in}}%
\pgfpathcurveto{\pgfqpoint{4.347643in}{1.387069in}}{\pgfqpoint{4.344810in}{1.393910in}}{\pgfqpoint{4.339766in}{1.398954in}}%
\pgfpathcurveto{\pgfqpoint{4.334722in}{1.403998in}}{\pgfqpoint{4.327881in}{1.406831in}}{\pgfqpoint{4.320748in}{1.406831in}}%
\pgfpathcurveto{\pgfqpoint{4.313615in}{1.406831in}}{\pgfqpoint{4.306773in}{1.403998in}}{\pgfqpoint{4.301730in}{1.398954in}}%
\pgfpathcurveto{\pgfqpoint{4.296686in}{1.393910in}}{\pgfqpoint{4.293852in}{1.387069in}}{\pgfqpoint{4.293852in}{1.379936in}}%
\pgfpathcurveto{\pgfqpoint{4.293852in}{1.372803in}}{\pgfqpoint{4.296686in}{1.365961in}}{\pgfqpoint{4.301730in}{1.360918in}}%
\pgfpathcurveto{\pgfqpoint{4.306773in}{1.355874in}}{\pgfqpoint{4.313615in}{1.353040in}}{\pgfqpoint{4.320748in}{1.353040in}}%
\pgfpathclose%
\pgfusepath{stroke,fill}%
\end{pgfscope}%
\begin{pgfscope}%
\pgfpathrectangle{\pgfqpoint{2.867647in}{0.500000in}}{\pgfqpoint{1.764706in}{1.700000in}}%
\pgfusepath{clip}%
\pgfsetbuttcap%
\pgfsetroundjoin%
\definecolor{currentfill}{rgb}{0.972201,0.839051,0.745789}%
\pgfsetfillcolor{currentfill}%
\pgfsetlinewidth{0.311001pt}%
\definecolor{currentstroke}{rgb}{1.000000,1.000000,1.000000}%
\pgfsetstrokecolor{currentstroke}%
\pgfsetdash{}{0pt}%
\pgfpathmoveto{\pgfqpoint{4.222084in}{1.094640in}}%
\pgfpathcurveto{\pgfqpoint{4.229217in}{1.094640in}}{\pgfqpoint{4.236059in}{1.097474in}}{\pgfqpoint{4.241102in}{1.102518in}}%
\pgfpathcurveto{\pgfqpoint{4.246146in}{1.107561in}}{\pgfqpoint{4.248980in}{1.114403in}}{\pgfqpoint{4.248980in}{1.121536in}}%
\pgfpathcurveto{\pgfqpoint{4.248980in}{1.128669in}}{\pgfqpoint{4.246146in}{1.135510in}}{\pgfqpoint{4.241102in}{1.140554in}}%
\pgfpathcurveto{\pgfqpoint{4.236059in}{1.145598in}}{\pgfqpoint{4.229217in}{1.148432in}}{\pgfqpoint{4.222084in}{1.148432in}}%
\pgfpathcurveto{\pgfqpoint{4.214952in}{1.148432in}}{\pgfqpoint{4.208110in}{1.145598in}}{\pgfqpoint{4.203066in}{1.140554in}}%
\pgfpathcurveto{\pgfqpoint{4.198023in}{1.135510in}}{\pgfqpoint{4.195189in}{1.128669in}}{\pgfqpoint{4.195189in}{1.121536in}}%
\pgfpathcurveto{\pgfqpoint{4.195189in}{1.114403in}}{\pgfqpoint{4.198023in}{1.107561in}}{\pgfqpoint{4.203066in}{1.102518in}}%
\pgfpathcurveto{\pgfqpoint{4.208110in}{1.097474in}}{\pgfqpoint{4.214952in}{1.094640in}}{\pgfqpoint{4.222084in}{1.094640in}}%
\pgfpathclose%
\pgfusepath{stroke,fill}%
\end{pgfscope}%
\begin{pgfscope}%
\pgfpathrectangle{\pgfqpoint{2.867647in}{0.500000in}}{\pgfqpoint{1.764706in}{1.700000in}}%
\pgfusepath{clip}%
\pgfsetbuttcap%
\pgfsetroundjoin%
\definecolor{currentfill}{rgb}{0.980678,0.914765,0.856766}%
\pgfsetfillcolor{currentfill}%
\pgfsetlinewidth{0.311001pt}%
\definecolor{currentstroke}{rgb}{1.000000,1.000000,1.000000}%
\pgfsetstrokecolor{currentstroke}%
\pgfsetdash{}{0pt}%
\pgfpathmoveto{\pgfqpoint{4.189036in}{1.449123in}}%
\pgfpathcurveto{\pgfqpoint{4.196169in}{1.449123in}}{\pgfqpoint{4.203010in}{1.451957in}}{\pgfqpoint{4.208054in}{1.457001in}}%
\pgfpathcurveto{\pgfqpoint{4.213098in}{1.462045in}}{\pgfqpoint{4.215932in}{1.468886in}}{\pgfqpoint{4.215932in}{1.476019in}}%
\pgfpathcurveto{\pgfqpoint{4.215932in}{1.483152in}}{\pgfqpoint{4.213098in}{1.489994in}}{\pgfqpoint{4.208054in}{1.495037in}}%
\pgfpathcurveto{\pgfqpoint{4.203010in}{1.500081in}}{\pgfqpoint{4.196169in}{1.502915in}}{\pgfqpoint{4.189036in}{1.502915in}}%
\pgfpathcurveto{\pgfqpoint{4.181903in}{1.502915in}}{\pgfqpoint{4.175061in}{1.500081in}}{\pgfqpoint{4.170018in}{1.495037in}}%
\pgfpathcurveto{\pgfqpoint{4.164974in}{1.489994in}}{\pgfqpoint{4.162140in}{1.483152in}}{\pgfqpoint{4.162140in}{1.476019in}}%
\pgfpathcurveto{\pgfqpoint{4.162140in}{1.468886in}}{\pgfqpoint{4.164974in}{1.462045in}}{\pgfqpoint{4.170018in}{1.457001in}}%
\pgfpathcurveto{\pgfqpoint{4.175061in}{1.451957in}}{\pgfqpoint{4.181903in}{1.449123in}}{\pgfqpoint{4.189036in}{1.449123in}}%
\pgfpathclose%
\pgfusepath{stroke,fill}%
\end{pgfscope}%
\begin{pgfscope}%
\pgfpathrectangle{\pgfqpoint{2.867647in}{0.500000in}}{\pgfqpoint{1.764706in}{1.700000in}}%
\pgfusepath{clip}%
\pgfsetbuttcap%
\pgfsetroundjoin%
\definecolor{currentfill}{rgb}{0.978376,0.897317,0.831308}%
\pgfsetfillcolor{currentfill}%
\pgfsetlinewidth{0.311001pt}%
\definecolor{currentstroke}{rgb}{1.000000,1.000000,1.000000}%
\pgfsetstrokecolor{currentstroke}%
\pgfsetdash{}{0pt}%
\pgfpathmoveto{\pgfqpoint{4.114129in}{1.533854in}}%
\pgfpathcurveto{\pgfqpoint{4.121262in}{1.533854in}}{\pgfqpoint{4.128103in}{1.536688in}}{\pgfqpoint{4.133147in}{1.541732in}}%
\pgfpathcurveto{\pgfqpoint{4.138191in}{1.546775in}}{\pgfqpoint{4.141025in}{1.553617in}}{\pgfqpoint{4.141025in}{1.560750in}}%
\pgfpathcurveto{\pgfqpoint{4.141025in}{1.567883in}}{\pgfqpoint{4.138191in}{1.574724in}}{\pgfqpoint{4.133147in}{1.579768in}}%
\pgfpathcurveto{\pgfqpoint{4.128103in}{1.584812in}}{\pgfqpoint{4.121262in}{1.587646in}}{\pgfqpoint{4.114129in}{1.587646in}}%
\pgfpathcurveto{\pgfqpoint{4.106996in}{1.587646in}}{\pgfqpoint{4.100154in}{1.584812in}}{\pgfqpoint{4.095111in}{1.579768in}}%
\pgfpathcurveto{\pgfqpoint{4.090067in}{1.574724in}}{\pgfqpoint{4.087233in}{1.567883in}}{\pgfqpoint{4.087233in}{1.560750in}}%
\pgfpathcurveto{\pgfqpoint{4.087233in}{1.553617in}}{\pgfqpoint{4.090067in}{1.546775in}}{\pgfqpoint{4.095111in}{1.541732in}}%
\pgfpathcurveto{\pgfqpoint{4.100154in}{1.536688in}}{\pgfqpoint{4.106996in}{1.533854in}}{\pgfqpoint{4.114129in}{1.533854in}}%
\pgfpathclose%
\pgfusepath{stroke,fill}%
\end{pgfscope}%
\begin{pgfscope}%
\pgfpathrectangle{\pgfqpoint{2.867647in}{0.500000in}}{\pgfqpoint{1.764706in}{1.700000in}}%
\pgfusepath{clip}%
\pgfsetbuttcap%
\pgfsetroundjoin%
\definecolor{currentfill}{rgb}{0.971202,0.827364,0.728520}%
\pgfsetfillcolor{currentfill}%
\pgfsetlinewidth{0.311001pt}%
\definecolor{currentstroke}{rgb}{1.000000,1.000000,1.000000}%
\pgfsetstrokecolor{currentstroke}%
\pgfsetdash{}{0pt}%
\pgfpathmoveto{\pgfqpoint{4.056903in}{1.002409in}}%
\pgfpathcurveto{\pgfqpoint{4.064036in}{1.002409in}}{\pgfqpoint{4.070877in}{1.005243in}}{\pgfqpoint{4.075921in}{1.010287in}}%
\pgfpathcurveto{\pgfqpoint{4.080965in}{1.015330in}}{\pgfqpoint{4.083798in}{1.022172in}}{\pgfqpoint{4.083798in}{1.029305in}}%
\pgfpathcurveto{\pgfqpoint{4.083798in}{1.036438in}}{\pgfqpoint{4.080965in}{1.043279in}}{\pgfqpoint{4.075921in}{1.048323in}}%
\pgfpathcurveto{\pgfqpoint{4.070877in}{1.053367in}}{\pgfqpoint{4.064036in}{1.056200in}}{\pgfqpoint{4.056903in}{1.056200in}}%
\pgfpathcurveto{\pgfqpoint{4.049770in}{1.056200in}}{\pgfqpoint{4.042928in}{1.053367in}}{\pgfqpoint{4.037885in}{1.048323in}}%
\pgfpathcurveto{\pgfqpoint{4.032841in}{1.043279in}}{\pgfqpoint{4.030007in}{1.036438in}}{\pgfqpoint{4.030007in}{1.029305in}}%
\pgfpathcurveto{\pgfqpoint{4.030007in}{1.022172in}}{\pgfqpoint{4.032841in}{1.015330in}}{\pgfqpoint{4.037885in}{1.010287in}}%
\pgfpathcurveto{\pgfqpoint{4.042928in}{1.005243in}}{\pgfqpoint{4.049770in}{1.002409in}}{\pgfqpoint{4.056903in}{1.002409in}}%
\pgfpathclose%
\pgfusepath{stroke,fill}%
\end{pgfscope}%
\begin{pgfscope}%
\pgfpathrectangle{\pgfqpoint{2.867647in}{0.500000in}}{\pgfqpoint{1.764706in}{1.700000in}}%
\pgfusepath{clip}%
\pgfsetbuttcap%
\pgfsetroundjoin%
\definecolor{currentfill}{rgb}{0.967735,0.780441,0.659127}%
\pgfsetfillcolor{currentfill}%
\pgfsetlinewidth{0.311001pt}%
\definecolor{currentstroke}{rgb}{1.000000,1.000000,1.000000}%
\pgfsetstrokecolor{currentstroke}%
\pgfsetdash{}{0pt}%
\pgfpathmoveto{\pgfqpoint{4.262298in}{1.493270in}}%
\pgfpathcurveto{\pgfqpoint{4.269431in}{1.493270in}}{\pgfqpoint{4.276273in}{1.496104in}}{\pgfqpoint{4.281316in}{1.501147in}}%
\pgfpathcurveto{\pgfqpoint{4.286360in}{1.506191in}}{\pgfqpoint{4.289194in}{1.513033in}}{\pgfqpoint{4.289194in}{1.520165in}}%
\pgfpathcurveto{\pgfqpoint{4.289194in}{1.527298in}}{\pgfqpoint{4.286360in}{1.534140in}}{\pgfqpoint{4.281316in}{1.539184in}}%
\pgfpathcurveto{\pgfqpoint{4.276273in}{1.544227in}}{\pgfqpoint{4.269431in}{1.547061in}}{\pgfqpoint{4.262298in}{1.547061in}}%
\pgfpathcurveto{\pgfqpoint{4.255165in}{1.547061in}}{\pgfqpoint{4.248324in}{1.544227in}}{\pgfqpoint{4.243280in}{1.539184in}}%
\pgfpathcurveto{\pgfqpoint{4.238236in}{1.534140in}}{\pgfqpoint{4.235402in}{1.527298in}}{\pgfqpoint{4.235402in}{1.520165in}}%
\pgfpathcurveto{\pgfqpoint{4.235402in}{1.513033in}}{\pgfqpoint{4.238236in}{1.506191in}}{\pgfqpoint{4.243280in}{1.501147in}}%
\pgfpathcurveto{\pgfqpoint{4.248324in}{1.496104in}}{\pgfqpoint{4.255165in}{1.493270in}}{\pgfqpoint{4.262298in}{1.493270in}}%
\pgfpathclose%
\pgfusepath{stroke,fill}%
\end{pgfscope}%
\begin{pgfscope}%
\pgfpathrectangle{\pgfqpoint{2.867647in}{0.500000in}}{\pgfqpoint{1.764706in}{1.700000in}}%
\pgfusepath{clip}%
\pgfsetbuttcap%
\pgfsetroundjoin%
\definecolor{currentfill}{rgb}{0.975018,0.868213,0.788710}%
\pgfsetfillcolor{currentfill}%
\pgfsetlinewidth{0.311001pt}%
\definecolor{currentstroke}{rgb}{1.000000,1.000000,1.000000}%
\pgfsetstrokecolor{currentstroke}%
\pgfsetdash{}{0pt}%
\pgfpathmoveto{\pgfqpoint{4.146768in}{1.017138in}}%
\pgfpathcurveto{\pgfqpoint{4.153900in}{1.017138in}}{\pgfqpoint{4.160742in}{1.019972in}}{\pgfqpoint{4.165786in}{1.025015in}}%
\pgfpathcurveto{\pgfqpoint{4.170829in}{1.030059in}}{\pgfqpoint{4.173663in}{1.036901in}}{\pgfqpoint{4.173663in}{1.044033in}}%
\pgfpathcurveto{\pgfqpoint{4.173663in}{1.051166in}}{\pgfqpoint{4.170829in}{1.058008in}}{\pgfqpoint{4.165786in}{1.063052in}}%
\pgfpathcurveto{\pgfqpoint{4.160742in}{1.068095in}}{\pgfqpoint{4.153900in}{1.070929in}}{\pgfqpoint{4.146768in}{1.070929in}}%
\pgfpathcurveto{\pgfqpoint{4.139635in}{1.070929in}}{\pgfqpoint{4.132793in}{1.068095in}}{\pgfqpoint{4.127749in}{1.063052in}}%
\pgfpathcurveto{\pgfqpoint{4.122706in}{1.058008in}}{\pgfqpoint{4.119872in}{1.051166in}}{\pgfqpoint{4.119872in}{1.044033in}}%
\pgfpathcurveto{\pgfqpoint{4.119872in}{1.036901in}}{\pgfqpoint{4.122706in}{1.030059in}}{\pgfqpoint{4.127749in}{1.025015in}}%
\pgfpathcurveto{\pgfqpoint{4.132793in}{1.019972in}}{\pgfqpoint{4.139635in}{1.017138in}}{\pgfqpoint{4.146768in}{1.017138in}}%
\pgfpathclose%
\pgfusepath{stroke,fill}%
\end{pgfscope}%
\begin{pgfscope}%
\pgfpathrectangle{\pgfqpoint{2.867647in}{0.500000in}}{\pgfqpoint{1.764706in}{1.700000in}}%
\pgfusepath{clip}%
\pgfsetbuttcap%
\pgfsetroundjoin%
\definecolor{currentfill}{rgb}{0.965753,0.732351,0.592427}%
\pgfsetfillcolor{currentfill}%
\pgfsetlinewidth{0.311001pt}%
\definecolor{currentstroke}{rgb}{1.000000,1.000000,1.000000}%
\pgfsetstrokecolor{currentstroke}%
\pgfsetdash{}{0pt}%
\pgfpathmoveto{\pgfqpoint{4.049119in}{1.749661in}}%
\pgfpathcurveto{\pgfqpoint{4.056252in}{1.749661in}}{\pgfqpoint{4.063093in}{1.752495in}}{\pgfqpoint{4.068137in}{1.757539in}}%
\pgfpathcurveto{\pgfqpoint{4.073181in}{1.762583in}}{\pgfqpoint{4.076015in}{1.769424in}}{\pgfqpoint{4.076015in}{1.776557in}}%
\pgfpathcurveto{\pgfqpoint{4.076015in}{1.783690in}}{\pgfqpoint{4.073181in}{1.790532in}}{\pgfqpoint{4.068137in}{1.795575in}}%
\pgfpathcurveto{\pgfqpoint{4.063093in}{1.800619in}}{\pgfqpoint{4.056252in}{1.803453in}}{\pgfqpoint{4.049119in}{1.803453in}}%
\pgfpathcurveto{\pgfqpoint{4.041986in}{1.803453in}}{\pgfqpoint{4.035144in}{1.800619in}}{\pgfqpoint{4.030101in}{1.795575in}}%
\pgfpathcurveto{\pgfqpoint{4.025057in}{1.790532in}}{\pgfqpoint{4.022223in}{1.783690in}}{\pgfqpoint{4.022223in}{1.776557in}}%
\pgfpathcurveto{\pgfqpoint{4.022223in}{1.769424in}}{\pgfqpoint{4.025057in}{1.762583in}}{\pgfqpoint{4.030101in}{1.757539in}}%
\pgfpathcurveto{\pgfqpoint{4.035144in}{1.752495in}}{\pgfqpoint{4.041986in}{1.749661in}}{\pgfqpoint{4.049119in}{1.749661in}}%
\pgfpathclose%
\pgfusepath{stroke,fill}%
\end{pgfscope}%
\begin{pgfscope}%
\pgfpathrectangle{\pgfqpoint{2.867647in}{0.500000in}}{\pgfqpoint{1.764706in}{1.700000in}}%
\pgfusepath{clip}%
\pgfsetbuttcap%
\pgfsetroundjoin%
\definecolor{currentfill}{rgb}{0.961433,0.573272,0.412036}%
\pgfsetfillcolor{currentfill}%
\pgfsetlinewidth{0.311001pt}%
\definecolor{currentstroke}{rgb}{1.000000,1.000000,1.000000}%
\pgfsetstrokecolor{currentstroke}%
\pgfsetdash{}{0pt}%
\pgfpathmoveto{\pgfqpoint{4.071212in}{1.811973in}}%
\pgfpathcurveto{\pgfqpoint{4.078345in}{1.811973in}}{\pgfqpoint{4.085187in}{1.814806in}}{\pgfqpoint{4.090230in}{1.819850in}}%
\pgfpathcurveto{\pgfqpoint{4.095274in}{1.824894in}}{\pgfqpoint{4.098108in}{1.831735in}}{\pgfqpoint{4.098108in}{1.838868in}}%
\pgfpathcurveto{\pgfqpoint{4.098108in}{1.846001in}}{\pgfqpoint{4.095274in}{1.852843in}}{\pgfqpoint{4.090230in}{1.857886in}}%
\pgfpathcurveto{\pgfqpoint{4.085187in}{1.862930in}}{\pgfqpoint{4.078345in}{1.865764in}}{\pgfqpoint{4.071212in}{1.865764in}}%
\pgfpathcurveto{\pgfqpoint{4.064079in}{1.865764in}}{\pgfqpoint{4.057238in}{1.862930in}}{\pgfqpoint{4.052194in}{1.857886in}}%
\pgfpathcurveto{\pgfqpoint{4.047150in}{1.852843in}}{\pgfqpoint{4.044316in}{1.846001in}}{\pgfqpoint{4.044316in}{1.838868in}}%
\pgfpathcurveto{\pgfqpoint{4.044316in}{1.831735in}}{\pgfqpoint{4.047150in}{1.824894in}}{\pgfqpoint{4.052194in}{1.819850in}}%
\pgfpathcurveto{\pgfqpoint{4.057238in}{1.814806in}}{\pgfqpoint{4.064079in}{1.811973in}}{\pgfqpoint{4.071212in}{1.811973in}}%
\pgfpathclose%
\pgfusepath{stroke,fill}%
\end{pgfscope}%
\begin{pgfscope}%
\pgfpathrectangle{\pgfqpoint{2.867647in}{0.500000in}}{\pgfqpoint{1.764706in}{1.700000in}}%
\pgfusepath{clip}%
\pgfsetbuttcap%
\pgfsetroundjoin%
\definecolor{currentfill}{rgb}{0.965042,0.701564,0.552889}%
\pgfsetfillcolor{currentfill}%
\pgfsetlinewidth{0.311001pt}%
\definecolor{currentstroke}{rgb}{1.000000,1.000000,1.000000}%
\pgfsetstrokecolor{currentstroke}%
\pgfsetdash{}{0pt}%
\pgfpathmoveto{\pgfqpoint{4.117009in}{0.901688in}}%
\pgfpathcurveto{\pgfqpoint{4.124142in}{0.901688in}}{\pgfqpoint{4.130983in}{0.904521in}}{\pgfqpoint{4.136027in}{0.909565in}}%
\pgfpathcurveto{\pgfqpoint{4.141071in}{0.914609in}}{\pgfqpoint{4.143905in}{0.921450in}}{\pgfqpoint{4.143905in}{0.928583in}}%
\pgfpathcurveto{\pgfqpoint{4.143905in}{0.935716in}}{\pgfqpoint{4.141071in}{0.942558in}}{\pgfqpoint{4.136027in}{0.947601in}}%
\pgfpathcurveto{\pgfqpoint{4.130983in}{0.952645in}}{\pgfqpoint{4.124142in}{0.955479in}}{\pgfqpoint{4.117009in}{0.955479in}}%
\pgfpathcurveto{\pgfqpoint{4.109876in}{0.955479in}}{\pgfqpoint{4.103034in}{0.952645in}}{\pgfqpoint{4.097991in}{0.947601in}}%
\pgfpathcurveto{\pgfqpoint{4.092947in}{0.942558in}}{\pgfqpoint{4.090113in}{0.935716in}}{\pgfqpoint{4.090113in}{0.928583in}}%
\pgfpathcurveto{\pgfqpoint{4.090113in}{0.921450in}}{\pgfqpoint{4.092947in}{0.914609in}}{\pgfqpoint{4.097991in}{0.909565in}}%
\pgfpathcurveto{\pgfqpoint{4.103034in}{0.904521in}}{\pgfqpoint{4.109876in}{0.901688in}}{\pgfqpoint{4.117009in}{0.901688in}}%
\pgfpathclose%
\pgfusepath{stroke,fill}%
\end{pgfscope}%
\begin{pgfscope}%
\pgfpathrectangle{\pgfqpoint{2.867647in}{0.500000in}}{\pgfqpoint{1.764706in}{1.700000in}}%
\pgfusepath{clip}%
\pgfsetbuttcap%
\pgfsetroundjoin%
\definecolor{currentfill}{rgb}{0.976287,0.879862,0.805788}%
\pgfsetfillcolor{currentfill}%
\pgfsetlinewidth{0.311001pt}%
\definecolor{currentstroke}{rgb}{1.000000,1.000000,1.000000}%
\pgfsetstrokecolor{currentstroke}%
\pgfsetdash{}{0pt}%
\pgfpathmoveto{\pgfqpoint{4.202790in}{1.110558in}}%
\pgfpathcurveto{\pgfqpoint{4.209922in}{1.110558in}}{\pgfqpoint{4.216764in}{1.113392in}}{\pgfqpoint{4.221808in}{1.118436in}}%
\pgfpathcurveto{\pgfqpoint{4.226851in}{1.123479in}}{\pgfqpoint{4.229685in}{1.130321in}}{\pgfqpoint{4.229685in}{1.137454in}}%
\pgfpathcurveto{\pgfqpoint{4.229685in}{1.144587in}}{\pgfqpoint{4.226851in}{1.151428in}}{\pgfqpoint{4.221808in}{1.156472in}}%
\pgfpathcurveto{\pgfqpoint{4.216764in}{1.161516in}}{\pgfqpoint{4.209922in}{1.164350in}}{\pgfqpoint{4.202790in}{1.164350in}}%
\pgfpathcurveto{\pgfqpoint{4.195657in}{1.164350in}}{\pgfqpoint{4.188815in}{1.161516in}}{\pgfqpoint{4.183771in}{1.156472in}}%
\pgfpathcurveto{\pgfqpoint{4.178728in}{1.151428in}}{\pgfqpoint{4.175894in}{1.144587in}}{\pgfqpoint{4.175894in}{1.137454in}}%
\pgfpathcurveto{\pgfqpoint{4.175894in}{1.130321in}}{\pgfqpoint{4.178728in}{1.123479in}}{\pgfqpoint{4.183771in}{1.118436in}}%
\pgfpathcurveto{\pgfqpoint{4.188815in}{1.113392in}}{\pgfqpoint{4.195657in}{1.110558in}}{\pgfqpoint{4.202790in}{1.110558in}}%
\pgfpathclose%
\pgfusepath{stroke,fill}%
\end{pgfscope}%
\begin{pgfscope}%
\pgfpathrectangle{\pgfqpoint{2.867647in}{0.500000in}}{\pgfqpoint{1.764706in}{1.700000in}}%
\pgfusepath{clip}%
\pgfsetbuttcap%
\pgfsetroundjoin%
\definecolor{currentfill}{rgb}{0.977657,0.891500,0.822809}%
\pgfsetfillcolor{currentfill}%
\pgfsetlinewidth{0.311001pt}%
\definecolor{currentstroke}{rgb}{1.000000,1.000000,1.000000}%
\pgfsetstrokecolor{currentstroke}%
\pgfsetdash{}{0pt}%
\pgfpathmoveto{\pgfqpoint{4.231431in}{1.252143in}}%
\pgfpathcurveto{\pgfqpoint{4.238564in}{1.252143in}}{\pgfqpoint{4.245405in}{1.254976in}}{\pgfqpoint{4.250449in}{1.260020in}}%
\pgfpathcurveto{\pgfqpoint{4.255493in}{1.265064in}}{\pgfqpoint{4.258327in}{1.271905in}}{\pgfqpoint{4.258327in}{1.279038in}}%
\pgfpathcurveto{\pgfqpoint{4.258327in}{1.286171in}}{\pgfqpoint{4.255493in}{1.293013in}}{\pgfqpoint{4.250449in}{1.298056in}}%
\pgfpathcurveto{\pgfqpoint{4.245405in}{1.303100in}}{\pgfqpoint{4.238564in}{1.305934in}}{\pgfqpoint{4.231431in}{1.305934in}}%
\pgfpathcurveto{\pgfqpoint{4.224298in}{1.305934in}}{\pgfqpoint{4.217456in}{1.303100in}}{\pgfqpoint{4.212413in}{1.298056in}}%
\pgfpathcurveto{\pgfqpoint{4.207369in}{1.293013in}}{\pgfqpoint{4.204535in}{1.286171in}}{\pgfqpoint{4.204535in}{1.279038in}}%
\pgfpathcurveto{\pgfqpoint{4.204535in}{1.271905in}}{\pgfqpoint{4.207369in}{1.265064in}}{\pgfqpoint{4.212413in}{1.260020in}}%
\pgfpathcurveto{\pgfqpoint{4.217456in}{1.254976in}}{\pgfqpoint{4.224298in}{1.252143in}}{\pgfqpoint{4.231431in}{1.252143in}}%
\pgfpathclose%
\pgfusepath{stroke,fill}%
\end{pgfscope}%
\begin{pgfscope}%
\pgfpathrectangle{\pgfqpoint{2.867647in}{0.500000in}}{\pgfqpoint{1.764706in}{1.700000in}}%
\pgfusepath{clip}%
\pgfsetbuttcap%
\pgfsetroundjoin%
\definecolor{currentfill}{rgb}{0.972201,0.839051,0.745789}%
\pgfsetfillcolor{currentfill}%
\pgfsetlinewidth{0.311001pt}%
\definecolor{currentstroke}{rgb}{1.000000,1.000000,1.000000}%
\pgfsetstrokecolor{currentstroke}%
\pgfsetdash{}{0pt}%
\pgfpathmoveto{\pgfqpoint{4.247375in}{1.456809in}}%
\pgfpathcurveto{\pgfqpoint{4.254508in}{1.456809in}}{\pgfqpoint{4.261349in}{1.459643in}}{\pgfqpoint{4.266393in}{1.464687in}}%
\pgfpathcurveto{\pgfqpoint{4.271437in}{1.469730in}}{\pgfqpoint{4.274270in}{1.476572in}}{\pgfqpoint{4.274270in}{1.483705in}}%
\pgfpathcurveto{\pgfqpoint{4.274270in}{1.490838in}}{\pgfqpoint{4.271437in}{1.497679in}}{\pgfqpoint{4.266393in}{1.502723in}}%
\pgfpathcurveto{\pgfqpoint{4.261349in}{1.507767in}}{\pgfqpoint{4.254508in}{1.510600in}}{\pgfqpoint{4.247375in}{1.510600in}}%
\pgfpathcurveto{\pgfqpoint{4.240242in}{1.510600in}}{\pgfqpoint{4.233400in}{1.507767in}}{\pgfqpoint{4.228357in}{1.502723in}}%
\pgfpathcurveto{\pgfqpoint{4.223313in}{1.497679in}}{\pgfqpoint{4.220479in}{1.490838in}}{\pgfqpoint{4.220479in}{1.483705in}}%
\pgfpathcurveto{\pgfqpoint{4.220479in}{1.476572in}}{\pgfqpoint{4.223313in}{1.469730in}}{\pgfqpoint{4.228357in}{1.464687in}}%
\pgfpathcurveto{\pgfqpoint{4.233400in}{1.459643in}}{\pgfqpoint{4.240242in}{1.456809in}}{\pgfqpoint{4.247375in}{1.456809in}}%
\pgfpathclose%
\pgfusepath{stroke,fill}%
\end{pgfscope}%
\begin{pgfscope}%
\pgfpathrectangle{\pgfqpoint{2.867647in}{0.500000in}}{\pgfqpoint{1.764706in}{1.700000in}}%
\pgfusepath{clip}%
\pgfsetbuttcap%
\pgfsetroundjoin%
\definecolor{currentfill}{rgb}{0.968509,0.792226,0.676405}%
\pgfsetfillcolor{currentfill}%
\pgfsetlinewidth{0.311001pt}%
\definecolor{currentstroke}{rgb}{1.000000,1.000000,1.000000}%
\pgfsetstrokecolor{currentstroke}%
\pgfsetdash{}{0pt}%
\pgfpathmoveto{\pgfqpoint{4.038339in}{0.961645in}}%
\pgfpathcurveto{\pgfqpoint{4.045472in}{0.961645in}}{\pgfqpoint{4.052313in}{0.964479in}}{\pgfqpoint{4.057357in}{0.969523in}}%
\pgfpathcurveto{\pgfqpoint{4.062401in}{0.974566in}}{\pgfqpoint{4.065235in}{0.981408in}}{\pgfqpoint{4.065235in}{0.988541in}}%
\pgfpathcurveto{\pgfqpoint{4.065235in}{0.995674in}}{\pgfqpoint{4.062401in}{1.002515in}}{\pgfqpoint{4.057357in}{1.007559in}}%
\pgfpathcurveto{\pgfqpoint{4.052313in}{1.012603in}}{\pgfqpoint{4.045472in}{1.015436in}}{\pgfqpoint{4.038339in}{1.015436in}}%
\pgfpathcurveto{\pgfqpoint{4.031206in}{1.015436in}}{\pgfqpoint{4.024364in}{1.012603in}}{\pgfqpoint{4.019321in}{1.007559in}}%
\pgfpathcurveto{\pgfqpoint{4.014277in}{1.002515in}}{\pgfqpoint{4.011443in}{0.995674in}}{\pgfqpoint{4.011443in}{0.988541in}}%
\pgfpathcurveto{\pgfqpoint{4.011443in}{0.981408in}}{\pgfqpoint{4.014277in}{0.974566in}}{\pgfqpoint{4.019321in}{0.969523in}}%
\pgfpathcurveto{\pgfqpoint{4.024364in}{0.964479in}}{\pgfqpoint{4.031206in}{0.961645in}}{\pgfqpoint{4.038339in}{0.961645in}}%
\pgfpathclose%
\pgfusepath{stroke,fill}%
\end{pgfscope}%
\begin{pgfscope}%
\pgfpathrectangle{\pgfqpoint{2.867647in}{0.500000in}}{\pgfqpoint{1.764706in}{1.700000in}}%
\pgfusepath{clip}%
\pgfsetbuttcap%
\pgfsetroundjoin%
\definecolor{currentfill}{rgb}{0.958791,0.526283,0.368909}%
\pgfsetfillcolor{currentfill}%
\pgfsetlinewidth{0.311001pt}%
\definecolor{currentstroke}{rgb}{1.000000,1.000000,1.000000}%
\pgfsetstrokecolor{currentstroke}%
\pgfsetdash{}{0pt}%
\pgfpathmoveto{\pgfqpoint{4.231454in}{0.942599in}}%
\pgfpathcurveto{\pgfqpoint{4.238587in}{0.942599in}}{\pgfqpoint{4.245429in}{0.945432in}}{\pgfqpoint{4.250473in}{0.950476in}}%
\pgfpathcurveto{\pgfqpoint{4.255516in}{0.955520in}}{\pgfqpoint{4.258350in}{0.962361in}}{\pgfqpoint{4.258350in}{0.969494in}}%
\pgfpathcurveto{\pgfqpoint{4.258350in}{0.976627in}}{\pgfqpoint{4.255516in}{0.983469in}}{\pgfqpoint{4.250473in}{0.988512in}}%
\pgfpathcurveto{\pgfqpoint{4.245429in}{0.993556in}}{\pgfqpoint{4.238587in}{0.996390in}}{\pgfqpoint{4.231454in}{0.996390in}}%
\pgfpathcurveto{\pgfqpoint{4.224322in}{0.996390in}}{\pgfqpoint{4.217480in}{0.993556in}}{\pgfqpoint{4.212436in}{0.988512in}}%
\pgfpathcurveto{\pgfqpoint{4.207393in}{0.983469in}}{\pgfqpoint{4.204559in}{0.976627in}}{\pgfqpoint{4.204559in}{0.969494in}}%
\pgfpathcurveto{\pgfqpoint{4.204559in}{0.962361in}}{\pgfqpoint{4.207393in}{0.955520in}}{\pgfqpoint{4.212436in}{0.950476in}}%
\pgfpathcurveto{\pgfqpoint{4.217480in}{0.945432in}}{\pgfqpoint{4.224322in}{0.942599in}}{\pgfqpoint{4.231454in}{0.942599in}}%
\pgfpathclose%
\pgfusepath{stroke,fill}%
\end{pgfscope}%
\begin{pgfscope}%
\pgfpathrectangle{\pgfqpoint{2.867647in}{0.500000in}}{\pgfqpoint{1.764706in}{1.700000in}}%
\pgfusepath{clip}%
\pgfsetbuttcap%
\pgfsetroundjoin%
\definecolor{currentfill}{rgb}{0.979124,0.903132,0.839793}%
\pgfsetfillcolor{currentfill}%
\pgfsetlinewidth{0.311001pt}%
\definecolor{currentstroke}{rgb}{1.000000,1.000000,1.000000}%
\pgfsetstrokecolor{currentstroke}%
\pgfsetdash{}{0pt}%
\pgfpathmoveto{\pgfqpoint{4.127252in}{1.109447in}}%
\pgfpathcurveto{\pgfqpoint{4.134385in}{1.109447in}}{\pgfqpoint{4.141227in}{1.112281in}}{\pgfqpoint{4.146270in}{1.117325in}}%
\pgfpathcurveto{\pgfqpoint{4.151314in}{1.122368in}}{\pgfqpoint{4.154148in}{1.129210in}}{\pgfqpoint{4.154148in}{1.136343in}}%
\pgfpathcurveto{\pgfqpoint{4.154148in}{1.143476in}}{\pgfqpoint{4.151314in}{1.150317in}}{\pgfqpoint{4.146270in}{1.155361in}}%
\pgfpathcurveto{\pgfqpoint{4.141227in}{1.160405in}}{\pgfqpoint{4.134385in}{1.163239in}}{\pgfqpoint{4.127252in}{1.163239in}}%
\pgfpathcurveto{\pgfqpoint{4.120119in}{1.163239in}}{\pgfqpoint{4.113278in}{1.160405in}}{\pgfqpoint{4.108234in}{1.155361in}}%
\pgfpathcurveto{\pgfqpoint{4.103190in}{1.150317in}}{\pgfqpoint{4.100356in}{1.143476in}}{\pgfqpoint{4.100356in}{1.136343in}}%
\pgfpathcurveto{\pgfqpoint{4.100356in}{1.129210in}}{\pgfqpoint{4.103190in}{1.122368in}}{\pgfqpoint{4.108234in}{1.117325in}}%
\pgfpathcurveto{\pgfqpoint{4.113278in}{1.112281in}}{\pgfqpoint{4.120119in}{1.109447in}}{\pgfqpoint{4.127252in}{1.109447in}}%
\pgfpathclose%
\pgfusepath{stroke,fill}%
\end{pgfscope}%
\begin{pgfscope}%
\pgfpathrectangle{\pgfqpoint{2.867647in}{0.500000in}}{\pgfqpoint{1.764706in}{1.700000in}}%
\pgfusepath{clip}%
\pgfsetbuttcap%
\pgfsetroundjoin%
\definecolor{currentfill}{rgb}{0.973832,0.856556,0.771584}%
\pgfsetfillcolor{currentfill}%
\pgfsetlinewidth{0.311001pt}%
\definecolor{currentstroke}{rgb}{1.000000,1.000000,1.000000}%
\pgfsetstrokecolor{currentstroke}%
\pgfsetdash{}{0pt}%
\pgfpathmoveto{\pgfqpoint{4.253253in}{1.259046in}}%
\pgfpathcurveto{\pgfqpoint{4.260386in}{1.259046in}}{\pgfqpoint{4.267228in}{1.261880in}}{\pgfqpoint{4.272272in}{1.266924in}}%
\pgfpathcurveto{\pgfqpoint{4.277315in}{1.271968in}}{\pgfqpoint{4.280149in}{1.278809in}}{\pgfqpoint{4.280149in}{1.285942in}}%
\pgfpathcurveto{\pgfqpoint{4.280149in}{1.293075in}}{\pgfqpoint{4.277315in}{1.299917in}}{\pgfqpoint{4.272272in}{1.304960in}}%
\pgfpathcurveto{\pgfqpoint{4.267228in}{1.310004in}}{\pgfqpoint{4.260386in}{1.312838in}}{\pgfqpoint{4.253253in}{1.312838in}}%
\pgfpathcurveto{\pgfqpoint{4.246121in}{1.312838in}}{\pgfqpoint{4.239279in}{1.310004in}}{\pgfqpoint{4.234235in}{1.304960in}}%
\pgfpathcurveto{\pgfqpoint{4.229192in}{1.299917in}}{\pgfqpoint{4.226358in}{1.293075in}}{\pgfqpoint{4.226358in}{1.285942in}}%
\pgfpathcurveto{\pgfqpoint{4.226358in}{1.278809in}}{\pgfqpoint{4.229192in}{1.271968in}}{\pgfqpoint{4.234235in}{1.266924in}}%
\pgfpathcurveto{\pgfqpoint{4.239279in}{1.261880in}}{\pgfqpoint{4.246121in}{1.259046in}}{\pgfqpoint{4.253253in}{1.259046in}}%
\pgfpathclose%
\pgfusepath{stroke,fill}%
\end{pgfscope}%
\begin{pgfscope}%
\pgfpathrectangle{\pgfqpoint{2.867647in}{0.500000in}}{\pgfqpoint{1.764706in}{1.700000in}}%
\pgfusepath{clip}%
\pgfsetbuttcap%
\pgfsetroundjoin%
\definecolor{currentfill}{rgb}{0.979891,0.908948,0.848279}%
\pgfsetfillcolor{currentfill}%
\pgfsetlinewidth{0.311001pt}%
\definecolor{currentstroke}{rgb}{1.000000,1.000000,1.000000}%
\pgfsetstrokecolor{currentstroke}%
\pgfsetdash{}{0pt}%
\pgfpathmoveto{\pgfqpoint{4.166860in}{1.359381in}}%
\pgfpathcurveto{\pgfqpoint{4.173992in}{1.359381in}}{\pgfqpoint{4.180834in}{1.362215in}}{\pgfqpoint{4.185878in}{1.367259in}}%
\pgfpathcurveto{\pgfqpoint{4.190921in}{1.372302in}}{\pgfqpoint{4.193755in}{1.379144in}}{\pgfqpoint{4.193755in}{1.386277in}}%
\pgfpathcurveto{\pgfqpoint{4.193755in}{1.393410in}}{\pgfqpoint{4.190921in}{1.400251in}}{\pgfqpoint{4.185878in}{1.405295in}}%
\pgfpathcurveto{\pgfqpoint{4.180834in}{1.410339in}}{\pgfqpoint{4.173992in}{1.413172in}}{\pgfqpoint{4.166860in}{1.413172in}}%
\pgfpathcurveto{\pgfqpoint{4.159727in}{1.413172in}}{\pgfqpoint{4.152885in}{1.410339in}}{\pgfqpoint{4.147841in}{1.405295in}}%
\pgfpathcurveto{\pgfqpoint{4.142798in}{1.400251in}}{\pgfqpoint{4.139964in}{1.393410in}}{\pgfqpoint{4.139964in}{1.386277in}}%
\pgfpathcurveto{\pgfqpoint{4.139964in}{1.379144in}}{\pgfqpoint{4.142798in}{1.372302in}}{\pgfqpoint{4.147841in}{1.367259in}}%
\pgfpathcurveto{\pgfqpoint{4.152885in}{1.362215in}}{\pgfqpoint{4.159727in}{1.359381in}}{\pgfqpoint{4.166860in}{1.359381in}}%
\pgfpathclose%
\pgfusepath{stroke,fill}%
\end{pgfscope}%
\begin{pgfscope}%
\pgfpathrectangle{\pgfqpoint{2.867647in}{0.500000in}}{\pgfqpoint{1.764706in}{1.700000in}}%
\pgfusepath{clip}%
\pgfsetbuttcap%
\pgfsetroundjoin%
\definecolor{currentfill}{rgb}{0.977657,0.891500,0.822809}%
\pgfsetfillcolor{currentfill}%
\pgfsetlinewidth{0.311001pt}%
\definecolor{currentstroke}{rgb}{1.000000,1.000000,1.000000}%
\pgfsetstrokecolor{currentstroke}%
\pgfsetdash{}{0pt}%
\pgfpathmoveto{\pgfqpoint{4.113788in}{1.616319in}}%
\pgfpathcurveto{\pgfqpoint{4.120921in}{1.616319in}}{\pgfqpoint{4.127762in}{1.619153in}}{\pgfqpoint{4.132806in}{1.624197in}}%
\pgfpathcurveto{\pgfqpoint{4.137850in}{1.629241in}}{\pgfqpoint{4.140684in}{1.636082in}}{\pgfqpoint{4.140684in}{1.643215in}}%
\pgfpathcurveto{\pgfqpoint{4.140684in}{1.650348in}}{\pgfqpoint{4.137850in}{1.657190in}}{\pgfqpoint{4.132806in}{1.662233in}}%
\pgfpathcurveto{\pgfqpoint{4.127762in}{1.667277in}}{\pgfqpoint{4.120921in}{1.670111in}}{\pgfqpoint{4.113788in}{1.670111in}}%
\pgfpathcurveto{\pgfqpoint{4.106655in}{1.670111in}}{\pgfqpoint{4.099813in}{1.667277in}}{\pgfqpoint{4.094770in}{1.662233in}}%
\pgfpathcurveto{\pgfqpoint{4.089726in}{1.657190in}}{\pgfqpoint{4.086892in}{1.650348in}}{\pgfqpoint{4.086892in}{1.643215in}}%
\pgfpathcurveto{\pgfqpoint{4.086892in}{1.636082in}}{\pgfqpoint{4.089726in}{1.629241in}}{\pgfqpoint{4.094770in}{1.624197in}}%
\pgfpathcurveto{\pgfqpoint{4.099813in}{1.619153in}}{\pgfqpoint{4.106655in}{1.616319in}}{\pgfqpoint{4.113788in}{1.616319in}}%
\pgfpathclose%
\pgfusepath{stroke,fill}%
\end{pgfscope}%
\begin{pgfscope}%
\pgfpathrectangle{\pgfqpoint{2.867647in}{0.500000in}}{\pgfqpoint{1.764706in}{1.700000in}}%
\pgfusepath{clip}%
\pgfsetbuttcap%
\pgfsetroundjoin%
\definecolor{currentfill}{rgb}{0.963884,0.644842,0.486120}%
\pgfsetfillcolor{currentfill}%
\pgfsetlinewidth{0.311001pt}%
\definecolor{currentstroke}{rgb}{1.000000,1.000000,1.000000}%
\pgfsetstrokecolor{currentstroke}%
\pgfsetdash{}{0pt}%
\pgfpathmoveto{\pgfqpoint{4.266455in}{1.597564in}}%
\pgfpathcurveto{\pgfqpoint{4.273588in}{1.597564in}}{\pgfqpoint{4.280430in}{1.600398in}}{\pgfqpoint{4.285473in}{1.605441in}}%
\pgfpathcurveto{\pgfqpoint{4.290517in}{1.610485in}}{\pgfqpoint{4.293351in}{1.617327in}}{\pgfqpoint{4.293351in}{1.624459in}}%
\pgfpathcurveto{\pgfqpoint{4.293351in}{1.631592in}}{\pgfqpoint{4.290517in}{1.638434in}}{\pgfqpoint{4.285473in}{1.643478in}}%
\pgfpathcurveto{\pgfqpoint{4.280430in}{1.648521in}}{\pgfqpoint{4.273588in}{1.651355in}}{\pgfqpoint{4.266455in}{1.651355in}}%
\pgfpathcurveto{\pgfqpoint{4.259322in}{1.651355in}}{\pgfqpoint{4.252481in}{1.648521in}}{\pgfqpoint{4.247437in}{1.643478in}}%
\pgfpathcurveto{\pgfqpoint{4.242393in}{1.638434in}}{\pgfqpoint{4.239559in}{1.631592in}}{\pgfqpoint{4.239559in}{1.624459in}}%
\pgfpathcurveto{\pgfqpoint{4.239559in}{1.617327in}}{\pgfqpoint{4.242393in}{1.610485in}}{\pgfqpoint{4.247437in}{1.605441in}}%
\pgfpathcurveto{\pgfqpoint{4.252481in}{1.600398in}}{\pgfqpoint{4.259322in}{1.597564in}}{\pgfqpoint{4.266455in}{1.597564in}}%
\pgfpathclose%
\pgfusepath{stroke,fill}%
\end{pgfscope}%
\begin{pgfscope}%
\pgfpathrectangle{\pgfqpoint{2.867647in}{0.500000in}}{\pgfqpoint{1.764706in}{1.700000in}}%
\pgfusepath{clip}%
\pgfsetbuttcap%
\pgfsetroundjoin%
\definecolor{currentfill}{rgb}{0.970255,0.815666,0.711203}%
\pgfsetfillcolor{currentfill}%
\pgfsetlinewidth{0.311001pt}%
\definecolor{currentstroke}{rgb}{1.000000,1.000000,1.000000}%
\pgfsetstrokecolor{currentstroke}%
\pgfsetdash{}{0pt}%
\pgfpathmoveto{\pgfqpoint{4.178280in}{1.002855in}}%
\pgfpathcurveto{\pgfqpoint{4.185413in}{1.002855in}}{\pgfqpoint{4.192255in}{1.005689in}}{\pgfqpoint{4.197298in}{1.010733in}}%
\pgfpathcurveto{\pgfqpoint{4.202342in}{1.015777in}}{\pgfqpoint{4.205176in}{1.022618in}}{\pgfqpoint{4.205176in}{1.029751in}}%
\pgfpathcurveto{\pgfqpoint{4.205176in}{1.036884in}}{\pgfqpoint{4.202342in}{1.043726in}}{\pgfqpoint{4.197298in}{1.048769in}}%
\pgfpathcurveto{\pgfqpoint{4.192255in}{1.053813in}}{\pgfqpoint{4.185413in}{1.056647in}}{\pgfqpoint{4.178280in}{1.056647in}}%
\pgfpathcurveto{\pgfqpoint{4.171147in}{1.056647in}}{\pgfqpoint{4.164306in}{1.053813in}}{\pgfqpoint{4.159262in}{1.048769in}}%
\pgfpathcurveto{\pgfqpoint{4.154218in}{1.043726in}}{\pgfqpoint{4.151384in}{1.036884in}}{\pgfqpoint{4.151384in}{1.029751in}}%
\pgfpathcurveto{\pgfqpoint{4.151384in}{1.022618in}}{\pgfqpoint{4.154218in}{1.015777in}}{\pgfqpoint{4.159262in}{1.010733in}}%
\pgfpathcurveto{\pgfqpoint{4.164306in}{1.005689in}}{\pgfqpoint{4.171147in}{1.002855in}}{\pgfqpoint{4.178280in}{1.002855in}}%
\pgfpathclose%
\pgfusepath{stroke,fill}%
\end{pgfscope}%
\begin{pgfscope}%
\pgfpathrectangle{\pgfqpoint{2.867647in}{0.500000in}}{\pgfqpoint{1.764706in}{1.700000in}}%
\pgfusepath{clip}%
\pgfsetbuttcap%
\pgfsetroundjoin%
\definecolor{currentfill}{rgb}{0.937528,0.344792,0.251999}%
\pgfsetfillcolor{currentfill}%
\pgfsetlinewidth{0.311001pt}%
\definecolor{currentstroke}{rgb}{1.000000,1.000000,1.000000}%
\pgfsetstrokecolor{currentstroke}%
\pgfsetdash{}{0pt}%
\pgfpathmoveto{\pgfqpoint{4.367117in}{1.212378in}}%
\pgfpathcurveto{\pgfqpoint{4.374250in}{1.212378in}}{\pgfqpoint{4.381092in}{1.215212in}}{\pgfqpoint{4.386136in}{1.220256in}}%
\pgfpathcurveto{\pgfqpoint{4.391179in}{1.225299in}}{\pgfqpoint{4.394013in}{1.232141in}}{\pgfqpoint{4.394013in}{1.239274in}}%
\pgfpathcurveto{\pgfqpoint{4.394013in}{1.246407in}}{\pgfqpoint{4.391179in}{1.253248in}}{\pgfqpoint{4.386136in}{1.258292in}}%
\pgfpathcurveto{\pgfqpoint{4.381092in}{1.263336in}}{\pgfqpoint{4.374250in}{1.266169in}}{\pgfqpoint{4.367117in}{1.266169in}}%
\pgfpathcurveto{\pgfqpoint{4.359985in}{1.266169in}}{\pgfqpoint{4.353143in}{1.263336in}}{\pgfqpoint{4.348099in}{1.258292in}}%
\pgfpathcurveto{\pgfqpoint{4.343056in}{1.253248in}}{\pgfqpoint{4.340222in}{1.246407in}}{\pgfqpoint{4.340222in}{1.239274in}}%
\pgfpathcurveto{\pgfqpoint{4.340222in}{1.232141in}}{\pgfqpoint{4.343056in}{1.225299in}}{\pgfqpoint{4.348099in}{1.220256in}}%
\pgfpathcurveto{\pgfqpoint{4.353143in}{1.215212in}}{\pgfqpoint{4.359985in}{1.212378in}}{\pgfqpoint{4.367117in}{1.212378in}}%
\pgfpathclose%
\pgfusepath{stroke,fill}%
\end{pgfscope}%
\begin{pgfscope}%
\pgfpathrectangle{\pgfqpoint{2.867647in}{0.500000in}}{\pgfqpoint{1.764706in}{1.700000in}}%
\pgfusepath{clip}%
\pgfsetbuttcap%
\pgfsetroundjoin%
\definecolor{currentfill}{rgb}{0.956817,0.498820,0.345554}%
\pgfsetfillcolor{currentfill}%
\pgfsetlinewidth{0.311001pt}%
\definecolor{currentstroke}{rgb}{1.000000,1.000000,1.000000}%
\pgfsetstrokecolor{currentstroke}%
\pgfsetdash{}{0pt}%
\pgfpathmoveto{\pgfqpoint{3.879895in}{1.722965in}}%
\pgfpathcurveto{\pgfqpoint{3.887028in}{1.722965in}}{\pgfqpoint{3.893869in}{1.725799in}}{\pgfqpoint{3.898913in}{1.730843in}}%
\pgfpathcurveto{\pgfqpoint{3.903957in}{1.735886in}}{\pgfqpoint{3.906790in}{1.742728in}}{\pgfqpoint{3.906790in}{1.749861in}}%
\pgfpathcurveto{\pgfqpoint{3.906790in}{1.756994in}}{\pgfqpoint{3.903957in}{1.763835in}}{\pgfqpoint{3.898913in}{1.768879in}}%
\pgfpathcurveto{\pgfqpoint{3.893869in}{1.773923in}}{\pgfqpoint{3.887028in}{1.776757in}}{\pgfqpoint{3.879895in}{1.776757in}}%
\pgfpathcurveto{\pgfqpoint{3.872762in}{1.776757in}}{\pgfqpoint{3.865920in}{1.773923in}}{\pgfqpoint{3.860877in}{1.768879in}}%
\pgfpathcurveto{\pgfqpoint{3.855833in}{1.763835in}}{\pgfqpoint{3.852999in}{1.756994in}}{\pgfqpoint{3.852999in}{1.749861in}}%
\pgfpathcurveto{\pgfqpoint{3.852999in}{1.742728in}}{\pgfqpoint{3.855833in}{1.735886in}}{\pgfqpoint{3.860877in}{1.730843in}}%
\pgfpathcurveto{\pgfqpoint{3.865920in}{1.725799in}}{\pgfqpoint{3.872762in}{1.722965in}}{\pgfqpoint{3.879895in}{1.722965in}}%
\pgfpathclose%
\pgfusepath{stroke,fill}%
\end{pgfscope}%
\begin{pgfscope}%
\pgfpathrectangle{\pgfqpoint{2.867647in}{0.500000in}}{\pgfqpoint{1.764706in}{1.700000in}}%
\pgfusepath{clip}%
\pgfsetbuttcap%
\pgfsetroundjoin%
\definecolor{currentfill}{rgb}{0.879259,0.192033,0.262681}%
\pgfsetfillcolor{currentfill}%
\pgfsetlinewidth{0.311001pt}%
\definecolor{currentstroke}{rgb}{1.000000,1.000000,1.000000}%
\pgfsetstrokecolor{currentstroke}%
\pgfsetdash{}{0pt}%
\pgfpathmoveto{\pgfqpoint{3.831575in}{0.837018in}}%
\pgfpathcurveto{\pgfqpoint{3.838708in}{0.837018in}}{\pgfqpoint{3.845550in}{0.839852in}}{\pgfqpoint{3.850594in}{0.844896in}}%
\pgfpathcurveto{\pgfqpoint{3.855637in}{0.849939in}}{\pgfqpoint{3.858471in}{0.856781in}}{\pgfqpoint{3.858471in}{0.863914in}}%
\pgfpathcurveto{\pgfqpoint{3.858471in}{0.871047in}}{\pgfqpoint{3.855637in}{0.877888in}}{\pgfqpoint{3.850594in}{0.882932in}}%
\pgfpathcurveto{\pgfqpoint{3.845550in}{0.887976in}}{\pgfqpoint{3.838708in}{0.890809in}}{\pgfqpoint{3.831575in}{0.890809in}}%
\pgfpathcurveto{\pgfqpoint{3.824443in}{0.890809in}}{\pgfqpoint{3.817601in}{0.887976in}}{\pgfqpoint{3.812557in}{0.882932in}}%
\pgfpathcurveto{\pgfqpoint{3.807514in}{0.877888in}}{\pgfqpoint{3.804680in}{0.871047in}}{\pgfqpoint{3.804680in}{0.863914in}}%
\pgfpathcurveto{\pgfqpoint{3.804680in}{0.856781in}}{\pgfqpoint{3.807514in}{0.849939in}}{\pgfqpoint{3.812557in}{0.844896in}}%
\pgfpathcurveto{\pgfqpoint{3.817601in}{0.839852in}}{\pgfqpoint{3.824443in}{0.837018in}}{\pgfqpoint{3.831575in}{0.837018in}}%
\pgfpathclose%
\pgfusepath{stroke,fill}%
\end{pgfscope}%
\begin{pgfscope}%
\pgfpathrectangle{\pgfqpoint{2.867647in}{0.500000in}}{\pgfqpoint{1.764706in}{1.700000in}}%
\pgfusepath{clip}%
\pgfsetbuttcap%
\pgfsetroundjoin%
\definecolor{currentfill}{rgb}{0.971694,0.833208,0.737161}%
\pgfsetfillcolor{currentfill}%
\pgfsetlinewidth{0.311001pt}%
\definecolor{currentstroke}{rgb}{1.000000,1.000000,1.000000}%
\pgfsetstrokecolor{currentstroke}%
\pgfsetdash{}{0pt}%
\pgfpathmoveto{\pgfqpoint{4.228489in}{1.536542in}}%
\pgfpathcurveto{\pgfqpoint{4.235621in}{1.536542in}}{\pgfqpoint{4.242463in}{1.539376in}}{\pgfqpoint{4.247507in}{1.544420in}}%
\pgfpathcurveto{\pgfqpoint{4.252550in}{1.549463in}}{\pgfqpoint{4.255384in}{1.556305in}}{\pgfqpoint{4.255384in}{1.563438in}}%
\pgfpathcurveto{\pgfqpoint{4.255384in}{1.570571in}}{\pgfqpoint{4.252550in}{1.577412in}}{\pgfqpoint{4.247507in}{1.582456in}}%
\pgfpathcurveto{\pgfqpoint{4.242463in}{1.587500in}}{\pgfqpoint{4.235621in}{1.590334in}}{\pgfqpoint{4.228489in}{1.590334in}}%
\pgfpathcurveto{\pgfqpoint{4.221356in}{1.590334in}}{\pgfqpoint{4.214514in}{1.587500in}}{\pgfqpoint{4.209470in}{1.582456in}}%
\pgfpathcurveto{\pgfqpoint{4.204427in}{1.577412in}}{\pgfqpoint{4.201593in}{1.570571in}}{\pgfqpoint{4.201593in}{1.563438in}}%
\pgfpathcurveto{\pgfqpoint{4.201593in}{1.556305in}}{\pgfqpoint{4.204427in}{1.549463in}}{\pgfqpoint{4.209470in}{1.544420in}}%
\pgfpathcurveto{\pgfqpoint{4.214514in}{1.539376in}}{\pgfqpoint{4.221356in}{1.536542in}}{\pgfqpoint{4.228489in}{1.536542in}}%
\pgfpathclose%
\pgfusepath{stroke,fill}%
\end{pgfscope}%
\begin{pgfscope}%
\pgfpathrectangle{\pgfqpoint{2.867647in}{0.500000in}}{\pgfqpoint{1.764706in}{1.700000in}}%
\pgfusepath{clip}%
\pgfsetbuttcap%
\pgfsetroundjoin%
\definecolor{currentfill}{rgb}{0.964920,0.695342,0.545192}%
\pgfsetfillcolor{currentfill}%
\pgfsetlinewidth{0.311001pt}%
\definecolor{currentstroke}{rgb}{1.000000,1.000000,1.000000}%
\pgfsetstrokecolor{currentstroke}%
\pgfsetdash{}{0pt}%
\pgfpathmoveto{\pgfqpoint{3.993399in}{1.017181in}}%
\pgfpathcurveto{\pgfqpoint{4.000531in}{1.017181in}}{\pgfqpoint{4.007373in}{1.020015in}}{\pgfqpoint{4.012417in}{1.025058in}}%
\pgfpathcurveto{\pgfqpoint{4.017460in}{1.030102in}}{\pgfqpoint{4.020294in}{1.036944in}}{\pgfqpoint{4.020294in}{1.044077in}}%
\pgfpathcurveto{\pgfqpoint{4.020294in}{1.051209in}}{\pgfqpoint{4.017460in}{1.058051in}}{\pgfqpoint{4.012417in}{1.063095in}}%
\pgfpathcurveto{\pgfqpoint{4.007373in}{1.068138in}}{\pgfqpoint{4.000531in}{1.070972in}}{\pgfqpoint{3.993399in}{1.070972in}}%
\pgfpathcurveto{\pgfqpoint{3.986266in}{1.070972in}}{\pgfqpoint{3.979424in}{1.068138in}}{\pgfqpoint{3.974380in}{1.063095in}}%
\pgfpathcurveto{\pgfqpoint{3.969337in}{1.058051in}}{\pgfqpoint{3.966503in}{1.051209in}}{\pgfqpoint{3.966503in}{1.044077in}}%
\pgfpathcurveto{\pgfqpoint{3.966503in}{1.036944in}}{\pgfqpoint{3.969337in}{1.030102in}}{\pgfqpoint{3.974380in}{1.025058in}}%
\pgfpathcurveto{\pgfqpoint{3.979424in}{1.020015in}}{\pgfqpoint{3.986266in}{1.017181in}}{\pgfqpoint{3.993399in}{1.017181in}}%
\pgfpathclose%
\pgfusepath{stroke,fill}%
\end{pgfscope}%
\begin{pgfscope}%
\pgfpathrectangle{\pgfqpoint{2.867647in}{0.500000in}}{\pgfqpoint{1.764706in}{1.700000in}}%
\pgfusepath{clip}%
\pgfsetbuttcap%
\pgfsetroundjoin%
\definecolor{currentfill}{rgb}{0.981377,0.920617,0.865369}%
\pgfsetfillcolor{currentfill}%
\pgfsetlinewidth{0.311001pt}%
\definecolor{currentstroke}{rgb}{1.000000,1.000000,1.000000}%
\pgfsetstrokecolor{currentstroke}%
\pgfsetdash{}{0pt}%
\pgfpathmoveto{\pgfqpoint{4.181101in}{1.271425in}}%
\pgfpathcurveto{\pgfqpoint{4.188234in}{1.271425in}}{\pgfqpoint{4.195076in}{1.274259in}}{\pgfqpoint{4.200119in}{1.279302in}}%
\pgfpathcurveto{\pgfqpoint{4.205163in}{1.284346in}}{\pgfqpoint{4.207997in}{1.291188in}}{\pgfqpoint{4.207997in}{1.298320in}}%
\pgfpathcurveto{\pgfqpoint{4.207997in}{1.305453in}}{\pgfqpoint{4.205163in}{1.312295in}}{\pgfqpoint{4.200119in}{1.317338in}}%
\pgfpathcurveto{\pgfqpoint{4.195076in}{1.322382in}}{\pgfqpoint{4.188234in}{1.325216in}}{\pgfqpoint{4.181101in}{1.325216in}}%
\pgfpathcurveto{\pgfqpoint{4.173968in}{1.325216in}}{\pgfqpoint{4.167127in}{1.322382in}}{\pgfqpoint{4.162083in}{1.317338in}}%
\pgfpathcurveto{\pgfqpoint{4.157039in}{1.312295in}}{\pgfqpoint{4.154206in}{1.305453in}}{\pgfqpoint{4.154206in}{1.298320in}}%
\pgfpathcurveto{\pgfqpoint{4.154206in}{1.291188in}}{\pgfqpoint{4.157039in}{1.284346in}}{\pgfqpoint{4.162083in}{1.279302in}}%
\pgfpathcurveto{\pgfqpoint{4.167127in}{1.274259in}}{\pgfqpoint{4.173968in}{1.271425in}}{\pgfqpoint{4.181101in}{1.271425in}}%
\pgfpathclose%
\pgfusepath{stroke,fill}%
\end{pgfscope}%
\begin{pgfscope}%
\pgfpathrectangle{\pgfqpoint{2.867647in}{0.500000in}}{\pgfqpoint{1.764706in}{1.700000in}}%
\pgfusepath{clip}%
\pgfsetbuttcap%
\pgfsetroundjoin%
\definecolor{currentfill}{rgb}{0.972726,0.844889,0.754401}%
\pgfsetfillcolor{currentfill}%
\pgfsetlinewidth{0.311001pt}%
\definecolor{currentstroke}{rgb}{1.000000,1.000000,1.000000}%
\pgfsetstrokecolor{currentstroke}%
\pgfsetdash{}{0pt}%
\pgfpathmoveto{\pgfqpoint{4.242470in}{1.158899in}}%
\pgfpathcurveto{\pgfqpoint{4.249603in}{1.158899in}}{\pgfqpoint{4.256445in}{1.161733in}}{\pgfqpoint{4.261488in}{1.166777in}}%
\pgfpathcurveto{\pgfqpoint{4.266532in}{1.171820in}}{\pgfqpoint{4.269366in}{1.178662in}}{\pgfqpoint{4.269366in}{1.185795in}}%
\pgfpathcurveto{\pgfqpoint{4.269366in}{1.192928in}}{\pgfqpoint{4.266532in}{1.199769in}}{\pgfqpoint{4.261488in}{1.204813in}}%
\pgfpathcurveto{\pgfqpoint{4.256445in}{1.209857in}}{\pgfqpoint{4.249603in}{1.212690in}}{\pgfqpoint{4.242470in}{1.212690in}}%
\pgfpathcurveto{\pgfqpoint{4.235337in}{1.212690in}}{\pgfqpoint{4.228496in}{1.209857in}}{\pgfqpoint{4.223452in}{1.204813in}}%
\pgfpathcurveto{\pgfqpoint{4.218408in}{1.199769in}}{\pgfqpoint{4.215575in}{1.192928in}}{\pgfqpoint{4.215575in}{1.185795in}}%
\pgfpathcurveto{\pgfqpoint{4.215575in}{1.178662in}}{\pgfqpoint{4.218408in}{1.171820in}}{\pgfqpoint{4.223452in}{1.166777in}}%
\pgfpathcurveto{\pgfqpoint{4.228496in}{1.161733in}}{\pgfqpoint{4.235337in}{1.158899in}}{\pgfqpoint{4.242470in}{1.158899in}}%
\pgfpathclose%
\pgfusepath{stroke,fill}%
\end{pgfscope}%
\begin{pgfscope}%
\pgfpathrectangle{\pgfqpoint{2.867647in}{0.500000in}}{\pgfqpoint{1.764706in}{1.700000in}}%
\pgfusepath{clip}%
\pgfsetbuttcap%
\pgfsetroundjoin%
\definecolor{currentfill}{rgb}{0.974412,0.862387,0.780156}%
\pgfsetfillcolor{currentfill}%
\pgfsetlinewidth{0.311001pt}%
\definecolor{currentstroke}{rgb}{1.000000,1.000000,1.000000}%
\pgfsetstrokecolor{currentstroke}%
\pgfsetdash{}{0pt}%
\pgfpathmoveto{\pgfqpoint{4.090335in}{1.645502in}}%
\pgfpathcurveto{\pgfqpoint{4.097467in}{1.645502in}}{\pgfqpoint{4.104309in}{1.648336in}}{\pgfqpoint{4.109353in}{1.653379in}}%
\pgfpathcurveto{\pgfqpoint{4.114396in}{1.658423in}}{\pgfqpoint{4.117230in}{1.665265in}}{\pgfqpoint{4.117230in}{1.672398in}}%
\pgfpathcurveto{\pgfqpoint{4.117230in}{1.679530in}}{\pgfqpoint{4.114396in}{1.686372in}}{\pgfqpoint{4.109353in}{1.691416in}}%
\pgfpathcurveto{\pgfqpoint{4.104309in}{1.696459in}}{\pgfqpoint{4.097467in}{1.699293in}}{\pgfqpoint{4.090335in}{1.699293in}}%
\pgfpathcurveto{\pgfqpoint{4.083202in}{1.699293in}}{\pgfqpoint{4.076360in}{1.696459in}}{\pgfqpoint{4.071316in}{1.691416in}}%
\pgfpathcurveto{\pgfqpoint{4.066273in}{1.686372in}}{\pgfqpoint{4.063439in}{1.679530in}}{\pgfqpoint{4.063439in}{1.672398in}}%
\pgfpathcurveto{\pgfqpoint{4.063439in}{1.665265in}}{\pgfqpoint{4.066273in}{1.658423in}}{\pgfqpoint{4.071316in}{1.653379in}}%
\pgfpathcurveto{\pgfqpoint{4.076360in}{1.648336in}}{\pgfqpoint{4.083202in}{1.645502in}}{\pgfqpoint{4.090335in}{1.645502in}}%
\pgfpathclose%
\pgfusepath{stroke,fill}%
\end{pgfscope}%
\begin{pgfscope}%
\pgfpathrectangle{\pgfqpoint{2.867647in}{0.500000in}}{\pgfqpoint{1.764706in}{1.700000in}}%
\pgfusepath{clip}%
\pgfsetbuttcap%
\pgfsetroundjoin%
\definecolor{currentfill}{rgb}{0.966812,0.762584,0.633643}%
\pgfsetfillcolor{currentfill}%
\pgfsetlinewidth{0.311001pt}%
\definecolor{currentstroke}{rgb}{1.000000,1.000000,1.000000}%
\pgfsetstrokecolor{currentstroke}%
\pgfsetdash{}{0pt}%
\pgfpathmoveto{\pgfqpoint{4.019932in}{1.695981in}}%
\pgfpathcurveto{\pgfqpoint{4.027064in}{1.695981in}}{\pgfqpoint{4.033906in}{1.698815in}}{\pgfqpoint{4.038950in}{1.703859in}}%
\pgfpathcurveto{\pgfqpoint{4.043993in}{1.708903in}}{\pgfqpoint{4.046827in}{1.715744in}}{\pgfqpoint{4.046827in}{1.722877in}}%
\pgfpathcurveto{\pgfqpoint{4.046827in}{1.730010in}}{\pgfqpoint{4.043993in}{1.736852in}}{\pgfqpoint{4.038950in}{1.741895in}}%
\pgfpathcurveto{\pgfqpoint{4.033906in}{1.746939in}}{\pgfqpoint{4.027064in}{1.749773in}}{\pgfqpoint{4.019932in}{1.749773in}}%
\pgfpathcurveto{\pgfqpoint{4.012799in}{1.749773in}}{\pgfqpoint{4.005957in}{1.746939in}}{\pgfqpoint{4.000913in}{1.741895in}}%
\pgfpathcurveto{\pgfqpoint{3.995870in}{1.736852in}}{\pgfqpoint{3.993036in}{1.730010in}}{\pgfqpoint{3.993036in}{1.722877in}}%
\pgfpathcurveto{\pgfqpoint{3.993036in}{1.715744in}}{\pgfqpoint{3.995870in}{1.708903in}}{\pgfqpoint{4.000913in}{1.703859in}}%
\pgfpathcurveto{\pgfqpoint{4.005957in}{1.698815in}}{\pgfqpoint{4.012799in}{1.695981in}}{\pgfqpoint{4.019932in}{1.695981in}}%
\pgfpathclose%
\pgfusepath{stroke,fill}%
\end{pgfscope}%
\begin{pgfscope}%
\pgfpathrectangle{\pgfqpoint{2.867647in}{0.500000in}}{\pgfqpoint{1.764706in}{1.700000in}}%
\pgfusepath{clip}%
\pgfsetbuttcap%
\pgfsetroundjoin%
\definecolor{currentfill}{rgb}{0.970718,0.821518,0.719872}%
\pgfsetfillcolor{currentfill}%
\pgfsetlinewidth{0.311001pt}%
\definecolor{currentstroke}{rgb}{1.000000,1.000000,1.000000}%
\pgfsetstrokecolor{currentstroke}%
\pgfsetdash{}{0pt}%
\pgfpathmoveto{\pgfqpoint{4.079348in}{1.144919in}}%
\pgfpathcurveto{\pgfqpoint{4.086481in}{1.144919in}}{\pgfqpoint{4.093322in}{1.147753in}}{\pgfqpoint{4.098366in}{1.152797in}}%
\pgfpathcurveto{\pgfqpoint{4.103410in}{1.157841in}}{\pgfqpoint{4.106244in}{1.164682in}}{\pgfqpoint{4.106244in}{1.171815in}}%
\pgfpathcurveto{\pgfqpoint{4.106244in}{1.178948in}}{\pgfqpoint{4.103410in}{1.185790in}}{\pgfqpoint{4.098366in}{1.190833in}}%
\pgfpathcurveto{\pgfqpoint{4.093322in}{1.195877in}}{\pgfqpoint{4.086481in}{1.198711in}}{\pgfqpoint{4.079348in}{1.198711in}}%
\pgfpathcurveto{\pgfqpoint{4.072215in}{1.198711in}}{\pgfqpoint{4.065373in}{1.195877in}}{\pgfqpoint{4.060330in}{1.190833in}}%
\pgfpathcurveto{\pgfqpoint{4.055286in}{1.185790in}}{\pgfqpoint{4.052452in}{1.178948in}}{\pgfqpoint{4.052452in}{1.171815in}}%
\pgfpathcurveto{\pgfqpoint{4.052452in}{1.164682in}}{\pgfqpoint{4.055286in}{1.157841in}}{\pgfqpoint{4.060330in}{1.152797in}}%
\pgfpathcurveto{\pgfqpoint{4.065373in}{1.147753in}}{\pgfqpoint{4.072215in}{1.144919in}}{\pgfqpoint{4.079348in}{1.144919in}}%
\pgfpathclose%
\pgfusepath{stroke,fill}%
\end{pgfscope}%
\begin{pgfscope}%
\pgfpathrectangle{\pgfqpoint{2.867647in}{0.500000in}}{\pgfqpoint{1.764706in}{1.700000in}}%
\pgfusepath{clip}%
\pgfsetbuttcap%
\pgfsetroundjoin%
\definecolor{currentfill}{rgb}{0.981377,0.920617,0.865369}%
\pgfsetfillcolor{currentfill}%
\pgfsetlinewidth{0.311001pt}%
\definecolor{currentstroke}{rgb}{1.000000,1.000000,1.000000}%
\pgfsetstrokecolor{currentstroke}%
\pgfsetdash{}{0pt}%
\pgfpathmoveto{\pgfqpoint{4.185525in}{1.178827in}}%
\pgfpathcurveto{\pgfqpoint{4.192658in}{1.178827in}}{\pgfqpoint{4.199499in}{1.181660in}}{\pgfqpoint{4.204543in}{1.186704in}}%
\pgfpathcurveto{\pgfqpoint{4.209587in}{1.191748in}}{\pgfqpoint{4.212421in}{1.198589in}}{\pgfqpoint{4.212421in}{1.205722in}}%
\pgfpathcurveto{\pgfqpoint{4.212421in}{1.212855in}}{\pgfqpoint{4.209587in}{1.219697in}}{\pgfqpoint{4.204543in}{1.224740in}}%
\pgfpathcurveto{\pgfqpoint{4.199499in}{1.229784in}}{\pgfqpoint{4.192658in}{1.232618in}}{\pgfqpoint{4.185525in}{1.232618in}}%
\pgfpathcurveto{\pgfqpoint{4.178392in}{1.232618in}}{\pgfqpoint{4.171550in}{1.229784in}}{\pgfqpoint{4.166507in}{1.224740in}}%
\pgfpathcurveto{\pgfqpoint{4.161463in}{1.219697in}}{\pgfqpoint{4.158629in}{1.212855in}}{\pgfqpoint{4.158629in}{1.205722in}}%
\pgfpathcurveto{\pgfqpoint{4.158629in}{1.198589in}}{\pgfqpoint{4.161463in}{1.191748in}}{\pgfqpoint{4.166507in}{1.186704in}}%
\pgfpathcurveto{\pgfqpoint{4.171550in}{1.181660in}}{\pgfqpoint{4.178392in}{1.178827in}}{\pgfqpoint{4.185525in}{1.178827in}}%
\pgfpathclose%
\pgfusepath{stroke,fill}%
\end{pgfscope}%
\begin{pgfscope}%
\pgfpathrectangle{\pgfqpoint{2.867647in}{0.500000in}}{\pgfqpoint{1.764706in}{1.700000in}}%
\pgfusepath{clip}%
\pgfsetbuttcap%
\pgfsetroundjoin%
\definecolor{currentfill}{rgb}{0.981377,0.920617,0.865369}%
\pgfsetfillcolor{currentfill}%
\pgfsetlinewidth{0.311001pt}%
\definecolor{currentstroke}{rgb}{1.000000,1.000000,1.000000}%
\pgfsetstrokecolor{currentstroke}%
\pgfsetdash{}{0pt}%
\pgfpathmoveto{\pgfqpoint{4.198107in}{1.322131in}}%
\pgfpathcurveto{\pgfqpoint{4.205240in}{1.322131in}}{\pgfqpoint{4.212082in}{1.324965in}}{\pgfqpoint{4.217125in}{1.330009in}}%
\pgfpathcurveto{\pgfqpoint{4.222169in}{1.335053in}}{\pgfqpoint{4.225003in}{1.341894in}}{\pgfqpoint{4.225003in}{1.349027in}}%
\pgfpathcurveto{\pgfqpoint{4.225003in}{1.356160in}}{\pgfqpoint{4.222169in}{1.363002in}}{\pgfqpoint{4.217125in}{1.368045in}}%
\pgfpathcurveto{\pgfqpoint{4.212082in}{1.373089in}}{\pgfqpoint{4.205240in}{1.375923in}}{\pgfqpoint{4.198107in}{1.375923in}}%
\pgfpathcurveto{\pgfqpoint{4.190974in}{1.375923in}}{\pgfqpoint{4.184133in}{1.373089in}}{\pgfqpoint{4.179089in}{1.368045in}}%
\pgfpathcurveto{\pgfqpoint{4.174045in}{1.363002in}}{\pgfqpoint{4.171212in}{1.356160in}}{\pgfqpoint{4.171212in}{1.349027in}}%
\pgfpathcurveto{\pgfqpoint{4.171212in}{1.341894in}}{\pgfqpoint{4.174045in}{1.335053in}}{\pgfqpoint{4.179089in}{1.330009in}}%
\pgfpathcurveto{\pgfqpoint{4.184133in}{1.324965in}}{\pgfqpoint{4.190974in}{1.322131in}}{\pgfqpoint{4.198107in}{1.322131in}}%
\pgfpathclose%
\pgfusepath{stroke,fill}%
\end{pgfscope}%
\begin{pgfscope}%
\pgfpathrectangle{\pgfqpoint{2.867647in}{0.500000in}}{\pgfqpoint{1.764706in}{1.700000in}}%
\pgfusepath{clip}%
\pgfsetbuttcap%
\pgfsetroundjoin%
\definecolor{currentfill}{rgb}{0.972726,0.844889,0.754401}%
\pgfsetfillcolor{currentfill}%
\pgfsetlinewidth{0.311001pt}%
\definecolor{currentstroke}{rgb}{1.000000,1.000000,1.000000}%
\pgfsetstrokecolor{currentstroke}%
\pgfsetdash{}{0pt}%
\pgfpathmoveto{\pgfqpoint{4.069274in}{1.014366in}}%
\pgfpathcurveto{\pgfqpoint{4.076407in}{1.014366in}}{\pgfqpoint{4.083248in}{1.017200in}}{\pgfqpoint{4.088292in}{1.022244in}}%
\pgfpathcurveto{\pgfqpoint{4.093335in}{1.027288in}}{\pgfqpoint{4.096169in}{1.034129in}}{\pgfqpoint{4.096169in}{1.041262in}}%
\pgfpathcurveto{\pgfqpoint{4.096169in}{1.048395in}}{\pgfqpoint{4.093335in}{1.055236in}}{\pgfqpoint{4.088292in}{1.060280in}}%
\pgfpathcurveto{\pgfqpoint{4.083248in}{1.065324in}}{\pgfqpoint{4.076407in}{1.068158in}}{\pgfqpoint{4.069274in}{1.068158in}}%
\pgfpathcurveto{\pgfqpoint{4.062141in}{1.068158in}}{\pgfqpoint{4.055299in}{1.065324in}}{\pgfqpoint{4.050256in}{1.060280in}}%
\pgfpathcurveto{\pgfqpoint{4.045212in}{1.055236in}}{\pgfqpoint{4.042378in}{1.048395in}}{\pgfqpoint{4.042378in}{1.041262in}}%
\pgfpathcurveto{\pgfqpoint{4.042378in}{1.034129in}}{\pgfqpoint{4.045212in}{1.027288in}}{\pgfqpoint{4.050256in}{1.022244in}}%
\pgfpathcurveto{\pgfqpoint{4.055299in}{1.017200in}}{\pgfqpoint{4.062141in}{1.014366in}}{\pgfqpoint{4.069274in}{1.014366in}}%
\pgfpathclose%
\pgfusepath{stroke,fill}%
\end{pgfscope}%
\begin{pgfscope}%
\pgfpathrectangle{\pgfqpoint{2.867647in}{0.500000in}}{\pgfqpoint{1.764706in}{1.700000in}}%
\pgfusepath{clip}%
\pgfsetbuttcap%
\pgfsetroundjoin%
\definecolor{currentfill}{rgb}{0.970255,0.815666,0.711203}%
\pgfsetfillcolor{currentfill}%
\pgfsetlinewidth{0.311001pt}%
\definecolor{currentstroke}{rgb}{1.000000,1.000000,1.000000}%
\pgfsetstrokecolor{currentstroke}%
\pgfsetdash{}{0pt}%
\pgfpathmoveto{\pgfqpoint{4.076951in}{1.486931in}}%
\pgfpathcurveto{\pgfqpoint{4.084084in}{1.486931in}}{\pgfqpoint{4.090925in}{1.489765in}}{\pgfqpoint{4.095969in}{1.494808in}}%
\pgfpathcurveto{\pgfqpoint{4.101013in}{1.499852in}}{\pgfqpoint{4.103847in}{1.506694in}}{\pgfqpoint{4.103847in}{1.513826in}}%
\pgfpathcurveto{\pgfqpoint{4.103847in}{1.520959in}}{\pgfqpoint{4.101013in}{1.527801in}}{\pgfqpoint{4.095969in}{1.532845in}}%
\pgfpathcurveto{\pgfqpoint{4.090925in}{1.537888in}}{\pgfqpoint{4.084084in}{1.540722in}}{\pgfqpoint{4.076951in}{1.540722in}}%
\pgfpathcurveto{\pgfqpoint{4.069818in}{1.540722in}}{\pgfqpoint{4.062976in}{1.537888in}}{\pgfqpoint{4.057933in}{1.532845in}}%
\pgfpathcurveto{\pgfqpoint{4.052889in}{1.527801in}}{\pgfqpoint{4.050055in}{1.520959in}}{\pgfqpoint{4.050055in}{1.513826in}}%
\pgfpathcurveto{\pgfqpoint{4.050055in}{1.506694in}}{\pgfqpoint{4.052889in}{1.499852in}}{\pgfqpoint{4.057933in}{1.494808in}}%
\pgfpathcurveto{\pgfqpoint{4.062976in}{1.489765in}}{\pgfqpoint{4.069818in}{1.486931in}}{\pgfqpoint{4.076951in}{1.486931in}}%
\pgfpathclose%
\pgfusepath{stroke,fill}%
\end{pgfscope}%
\begin{pgfscope}%
\pgfpathrectangle{\pgfqpoint{2.867647in}{0.500000in}}{\pgfqpoint{1.764706in}{1.700000in}}%
\pgfusepath{clip}%
\pgfsetbuttcap%
\pgfsetroundjoin%
\definecolor{currentfill}{rgb}{0.979124,0.903132,0.839793}%
\pgfsetfillcolor{currentfill}%
\pgfsetlinewidth{0.311001pt}%
\definecolor{currentstroke}{rgb}{1.000000,1.000000,1.000000}%
\pgfsetstrokecolor{currentstroke}%
\pgfsetdash{}{0pt}%
\pgfpathmoveto{\pgfqpoint{4.169820in}{1.552512in}}%
\pgfpathcurveto{\pgfqpoint{4.176953in}{1.552512in}}{\pgfqpoint{4.183795in}{1.555345in}}{\pgfqpoint{4.188838in}{1.560389in}}%
\pgfpathcurveto{\pgfqpoint{4.193882in}{1.565433in}}{\pgfqpoint{4.196716in}{1.572274in}}{\pgfqpoint{4.196716in}{1.579407in}}%
\pgfpathcurveto{\pgfqpoint{4.196716in}{1.586540in}}{\pgfqpoint{4.193882in}{1.593382in}}{\pgfqpoint{4.188838in}{1.598425in}}%
\pgfpathcurveto{\pgfqpoint{4.183795in}{1.603469in}}{\pgfqpoint{4.176953in}{1.606303in}}{\pgfqpoint{4.169820in}{1.606303in}}%
\pgfpathcurveto{\pgfqpoint{4.162687in}{1.606303in}}{\pgfqpoint{4.155846in}{1.603469in}}{\pgfqpoint{4.150802in}{1.598425in}}%
\pgfpathcurveto{\pgfqpoint{4.145758in}{1.593382in}}{\pgfqpoint{4.142925in}{1.586540in}}{\pgfqpoint{4.142925in}{1.579407in}}%
\pgfpathcurveto{\pgfqpoint{4.142925in}{1.572274in}}{\pgfqpoint{4.145758in}{1.565433in}}{\pgfqpoint{4.150802in}{1.560389in}}%
\pgfpathcurveto{\pgfqpoint{4.155846in}{1.555345in}}{\pgfqpoint{4.162687in}{1.552512in}}{\pgfqpoint{4.169820in}{1.552512in}}%
\pgfpathclose%
\pgfusepath{stroke,fill}%
\end{pgfscope}%
\begin{pgfscope}%
\pgfpathrectangle{\pgfqpoint{2.867647in}{0.500000in}}{\pgfqpoint{1.764706in}{1.700000in}}%
\pgfusepath{clip}%
\pgfsetbuttcap%
\pgfsetroundjoin%
\definecolor{currentfill}{rgb}{0.950017,0.427714,0.292447}%
\pgfsetfillcolor{currentfill}%
\pgfsetlinewidth{0.311001pt}%
\definecolor{currentstroke}{rgb}{1.000000,1.000000,1.000000}%
\pgfsetstrokecolor{currentstroke}%
\pgfsetdash{}{0pt}%
\pgfpathmoveto{\pgfqpoint{3.982507in}{1.849752in}}%
\pgfpathcurveto{\pgfqpoint{3.989640in}{1.849752in}}{\pgfqpoint{3.996482in}{1.852586in}}{\pgfqpoint{4.001525in}{1.857630in}}%
\pgfpathcurveto{\pgfqpoint{4.006569in}{1.862674in}}{\pgfqpoint{4.009403in}{1.869515in}}{\pgfqpoint{4.009403in}{1.876648in}}%
\pgfpathcurveto{\pgfqpoint{4.009403in}{1.883781in}}{\pgfqpoint{4.006569in}{1.890623in}}{\pgfqpoint{4.001525in}{1.895666in}}%
\pgfpathcurveto{\pgfqpoint{3.996482in}{1.900710in}}{\pgfqpoint{3.989640in}{1.903544in}}{\pgfqpoint{3.982507in}{1.903544in}}%
\pgfpathcurveto{\pgfqpoint{3.975375in}{1.903544in}}{\pgfqpoint{3.968533in}{1.900710in}}{\pgfqpoint{3.963489in}{1.895666in}}%
\pgfpathcurveto{\pgfqpoint{3.958446in}{1.890623in}}{\pgfqpoint{3.955612in}{1.883781in}}{\pgfqpoint{3.955612in}{1.876648in}}%
\pgfpathcurveto{\pgfqpoint{3.955612in}{1.869515in}}{\pgfqpoint{3.958446in}{1.862674in}}{\pgfqpoint{3.963489in}{1.857630in}}%
\pgfpathcurveto{\pgfqpoint{3.968533in}{1.852586in}}{\pgfqpoint{3.975375in}{1.849752in}}{\pgfqpoint{3.982507in}{1.849752in}}%
\pgfpathclose%
\pgfusepath{stroke,fill}%
\end{pgfscope}%
\begin{pgfscope}%
\pgfpathrectangle{\pgfqpoint{2.867647in}{0.500000in}}{\pgfqpoint{1.764706in}{1.700000in}}%
\pgfusepath{clip}%
\pgfsetbuttcap%
\pgfsetroundjoin%
\definecolor{currentfill}{rgb}{0.974412,0.862387,0.780156}%
\pgfsetfillcolor{currentfill}%
\pgfsetlinewidth{0.311001pt}%
\definecolor{currentstroke}{rgb}{1.000000,1.000000,1.000000}%
\pgfsetstrokecolor{currentstroke}%
\pgfsetdash{}{0pt}%
\pgfpathmoveto{\pgfqpoint{4.114370in}{1.206901in}}%
\pgfpathcurveto{\pgfqpoint{4.121502in}{1.206901in}}{\pgfqpoint{4.128344in}{1.209735in}}{\pgfqpoint{4.133388in}{1.214779in}}%
\pgfpathcurveto{\pgfqpoint{4.138431in}{1.219823in}}{\pgfqpoint{4.141265in}{1.226664in}}{\pgfqpoint{4.141265in}{1.233797in}}%
\pgfpathcurveto{\pgfqpoint{4.141265in}{1.240930in}}{\pgfqpoint{4.138431in}{1.247772in}}{\pgfqpoint{4.133388in}{1.252815in}}%
\pgfpathcurveto{\pgfqpoint{4.128344in}{1.257859in}}{\pgfqpoint{4.121502in}{1.260693in}}{\pgfqpoint{4.114370in}{1.260693in}}%
\pgfpathcurveto{\pgfqpoint{4.107237in}{1.260693in}}{\pgfqpoint{4.100395in}{1.257859in}}{\pgfqpoint{4.095351in}{1.252815in}}%
\pgfpathcurveto{\pgfqpoint{4.090308in}{1.247772in}}{\pgfqpoint{4.087474in}{1.240930in}}{\pgfqpoint{4.087474in}{1.233797in}}%
\pgfpathcurveto{\pgfqpoint{4.087474in}{1.226664in}}{\pgfqpoint{4.090308in}{1.219823in}}{\pgfqpoint{4.095351in}{1.214779in}}%
\pgfpathcurveto{\pgfqpoint{4.100395in}{1.209735in}}{\pgfqpoint{4.107237in}{1.206901in}}{\pgfqpoint{4.114370in}{1.206901in}}%
\pgfpathclose%
\pgfusepath{stroke,fill}%
\end{pgfscope}%
\begin{pgfscope}%
\pgfpathrectangle{\pgfqpoint{2.867647in}{0.500000in}}{\pgfqpoint{1.764706in}{1.700000in}}%
\pgfusepath{clip}%
\pgfsetbuttcap%
\pgfsetroundjoin%
\definecolor{currentfill}{rgb}{0.979891,0.908948,0.848279}%
\pgfsetfillcolor{currentfill}%
\pgfsetlinewidth{0.311001pt}%
\definecolor{currentstroke}{rgb}{1.000000,1.000000,1.000000}%
\pgfsetstrokecolor{currentstroke}%
\pgfsetdash{}{0pt}%
\pgfpathmoveto{\pgfqpoint{4.212510in}{1.357633in}}%
\pgfpathcurveto{\pgfqpoint{4.219643in}{1.357633in}}{\pgfqpoint{4.226484in}{1.360467in}}{\pgfqpoint{4.231528in}{1.365510in}}%
\pgfpathcurveto{\pgfqpoint{4.236572in}{1.370554in}}{\pgfqpoint{4.239406in}{1.377396in}}{\pgfqpoint{4.239406in}{1.384528in}}%
\pgfpathcurveto{\pgfqpoint{4.239406in}{1.391661in}}{\pgfqpoint{4.236572in}{1.398503in}}{\pgfqpoint{4.231528in}{1.403546in}}%
\pgfpathcurveto{\pgfqpoint{4.226484in}{1.408590in}}{\pgfqpoint{4.219643in}{1.411424in}}{\pgfqpoint{4.212510in}{1.411424in}}%
\pgfpathcurveto{\pgfqpoint{4.205377in}{1.411424in}}{\pgfqpoint{4.198535in}{1.408590in}}{\pgfqpoint{4.193492in}{1.403546in}}%
\pgfpathcurveto{\pgfqpoint{4.188448in}{1.398503in}}{\pgfqpoint{4.185614in}{1.391661in}}{\pgfqpoint{4.185614in}{1.384528in}}%
\pgfpathcurveto{\pgfqpoint{4.185614in}{1.377396in}}{\pgfqpoint{4.188448in}{1.370554in}}{\pgfqpoint{4.193492in}{1.365510in}}%
\pgfpathcurveto{\pgfqpoint{4.198535in}{1.360467in}}{\pgfqpoint{4.205377in}{1.357633in}}{\pgfqpoint{4.212510in}{1.357633in}}%
\pgfpathclose%
\pgfusepath{stroke,fill}%
\end{pgfscope}%
\begin{pgfscope}%
\pgfpathrectangle{\pgfqpoint{2.867647in}{0.500000in}}{\pgfqpoint{1.764706in}{1.700000in}}%
\pgfusepath{clip}%
\pgfsetbuttcap%
\pgfsetroundjoin%
\definecolor{currentfill}{rgb}{0.964032,0.651225,0.493258}%
\pgfsetfillcolor{currentfill}%
\pgfsetlinewidth{0.311001pt}%
\definecolor{currentstroke}{rgb}{1.000000,1.000000,1.000000}%
\pgfsetstrokecolor{currentstroke}%
\pgfsetdash{}{0pt}%
\pgfpathmoveto{\pgfqpoint{3.968255in}{0.961912in}}%
\pgfpathcurveto{\pgfqpoint{3.975388in}{0.961912in}}{\pgfqpoint{3.982230in}{0.964746in}}{\pgfqpoint{3.987273in}{0.969789in}}%
\pgfpathcurveto{\pgfqpoint{3.992317in}{0.974833in}}{\pgfqpoint{3.995151in}{0.981675in}}{\pgfqpoint{3.995151in}{0.988807in}}%
\pgfpathcurveto{\pgfqpoint{3.995151in}{0.995940in}}{\pgfqpoint{3.992317in}{1.002782in}}{\pgfqpoint{3.987273in}{1.007826in}}%
\pgfpathcurveto{\pgfqpoint{3.982230in}{1.012869in}}{\pgfqpoint{3.975388in}{1.015703in}}{\pgfqpoint{3.968255in}{1.015703in}}%
\pgfpathcurveto{\pgfqpoint{3.961122in}{1.015703in}}{\pgfqpoint{3.954281in}{1.012869in}}{\pgfqpoint{3.949237in}{1.007826in}}%
\pgfpathcurveto{\pgfqpoint{3.944194in}{1.002782in}}{\pgfqpoint{3.941360in}{0.995940in}}{\pgfqpoint{3.941360in}{0.988807in}}%
\pgfpathcurveto{\pgfqpoint{3.941360in}{0.981675in}}{\pgfqpoint{3.944194in}{0.974833in}}{\pgfqpoint{3.949237in}{0.969789in}}%
\pgfpathcurveto{\pgfqpoint{3.954281in}{0.964746in}}{\pgfqpoint{3.961122in}{0.961912in}}{\pgfqpoint{3.968255in}{0.961912in}}%
\pgfpathclose%
\pgfusepath{stroke,fill}%
\end{pgfscope}%
\begin{pgfscope}%
\pgfpathrectangle{\pgfqpoint{2.867647in}{0.500000in}}{\pgfqpoint{1.764706in}{1.700000in}}%
\pgfusepath{clip}%
\pgfsetbuttcap%
\pgfsetroundjoin%
\definecolor{currentfill}{rgb}{0.965928,0.738443,0.600540}%
\pgfsetfillcolor{currentfill}%
\pgfsetlinewidth{0.311001pt}%
\definecolor{currentstroke}{rgb}{1.000000,1.000000,1.000000}%
\pgfsetstrokecolor{currentstroke}%
\pgfsetdash{}{0pt}%
\pgfpathmoveto{\pgfqpoint{4.082954in}{0.905022in}}%
\pgfpathcurveto{\pgfqpoint{4.090087in}{0.905022in}}{\pgfqpoint{4.096929in}{0.907856in}}{\pgfqpoint{4.101973in}{0.912900in}}%
\pgfpathcurveto{\pgfqpoint{4.107016in}{0.917944in}}{\pgfqpoint{4.109850in}{0.924785in}}{\pgfqpoint{4.109850in}{0.931918in}}%
\pgfpathcurveto{\pgfqpoint{4.109850in}{0.939051in}}{\pgfqpoint{4.107016in}{0.945893in}}{\pgfqpoint{4.101973in}{0.950936in}}%
\pgfpathcurveto{\pgfqpoint{4.096929in}{0.955980in}}{\pgfqpoint{4.090087in}{0.958814in}}{\pgfqpoint{4.082954in}{0.958814in}}%
\pgfpathcurveto{\pgfqpoint{4.075822in}{0.958814in}}{\pgfqpoint{4.068980in}{0.955980in}}{\pgfqpoint{4.063936in}{0.950936in}}%
\pgfpathcurveto{\pgfqpoint{4.058893in}{0.945893in}}{\pgfqpoint{4.056059in}{0.939051in}}{\pgfqpoint{4.056059in}{0.931918in}}%
\pgfpathcurveto{\pgfqpoint{4.056059in}{0.924785in}}{\pgfqpoint{4.058893in}{0.917944in}}{\pgfqpoint{4.063936in}{0.912900in}}%
\pgfpathcurveto{\pgfqpoint{4.068980in}{0.907856in}}{\pgfqpoint{4.075822in}{0.905022in}}{\pgfqpoint{4.082954in}{0.905022in}}%
\pgfpathclose%
\pgfusepath{stroke,fill}%
\end{pgfscope}%
\begin{pgfscope}%
\pgfpathrectangle{\pgfqpoint{2.867647in}{0.500000in}}{\pgfqpoint{1.764706in}{1.700000in}}%
\pgfusepath{clip}%
\pgfsetbuttcap%
\pgfsetroundjoin%
\definecolor{currentfill}{rgb}{0.966120,0.744512,0.608720}%
\pgfsetfillcolor{currentfill}%
\pgfsetlinewidth{0.311001pt}%
\definecolor{currentstroke}{rgb}{1.000000,1.000000,1.000000}%
\pgfsetstrokecolor{currentstroke}%
\pgfsetdash{}{0pt}%
\pgfpathmoveto{\pgfqpoint{4.274847in}{1.491528in}}%
\pgfpathcurveto{\pgfqpoint{4.281980in}{1.491528in}}{\pgfqpoint{4.288822in}{1.494362in}}{\pgfqpoint{4.293865in}{1.499405in}}%
\pgfpathcurveto{\pgfqpoint{4.298909in}{1.504449in}}{\pgfqpoint{4.301743in}{1.511291in}}{\pgfqpoint{4.301743in}{1.518423in}}%
\pgfpathcurveto{\pgfqpoint{4.301743in}{1.525556in}}{\pgfqpoint{4.298909in}{1.532398in}}{\pgfqpoint{4.293865in}{1.537442in}}%
\pgfpathcurveto{\pgfqpoint{4.288822in}{1.542485in}}{\pgfqpoint{4.281980in}{1.545319in}}{\pgfqpoint{4.274847in}{1.545319in}}%
\pgfpathcurveto{\pgfqpoint{4.267715in}{1.545319in}}{\pgfqpoint{4.260873in}{1.542485in}}{\pgfqpoint{4.255829in}{1.537442in}}%
\pgfpathcurveto{\pgfqpoint{4.250786in}{1.532398in}}{\pgfqpoint{4.247952in}{1.525556in}}{\pgfqpoint{4.247952in}{1.518423in}}%
\pgfpathcurveto{\pgfqpoint{4.247952in}{1.511291in}}{\pgfqpoint{4.250786in}{1.504449in}}{\pgfqpoint{4.255829in}{1.499405in}}%
\pgfpathcurveto{\pgfqpoint{4.260873in}{1.494362in}}{\pgfqpoint{4.267715in}{1.491528in}}{\pgfqpoint{4.274847in}{1.491528in}}%
\pgfpathclose%
\pgfusepath{stroke,fill}%
\end{pgfscope}%
\begin{pgfscope}%
\pgfpathrectangle{\pgfqpoint{2.867647in}{0.500000in}}{\pgfqpoint{1.764706in}{1.700000in}}%
\pgfusepath{clip}%
\pgfsetbuttcap%
\pgfsetroundjoin%
\definecolor{currentfill}{rgb}{0.971694,0.833208,0.737161}%
\pgfsetfillcolor{currentfill}%
\pgfsetlinewidth{0.311001pt}%
\definecolor{currentstroke}{rgb}{1.000000,1.000000,1.000000}%
\pgfsetstrokecolor{currentstroke}%
\pgfsetdash{}{0pt}%
\pgfpathmoveto{\pgfqpoint{4.068398in}{1.670130in}}%
\pgfpathcurveto{\pgfqpoint{4.075531in}{1.670130in}}{\pgfqpoint{4.082372in}{1.672964in}}{\pgfqpoint{4.087416in}{1.678008in}}%
\pgfpathcurveto{\pgfqpoint{4.092460in}{1.683052in}}{\pgfqpoint{4.095294in}{1.689893in}}{\pgfqpoint{4.095294in}{1.697026in}}%
\pgfpathcurveto{\pgfqpoint{4.095294in}{1.704159in}}{\pgfqpoint{4.092460in}{1.711001in}}{\pgfqpoint{4.087416in}{1.716044in}}%
\pgfpathcurveto{\pgfqpoint{4.082372in}{1.721088in}}{\pgfqpoint{4.075531in}{1.723922in}}{\pgfqpoint{4.068398in}{1.723922in}}%
\pgfpathcurveto{\pgfqpoint{4.061265in}{1.723922in}}{\pgfqpoint{4.054424in}{1.721088in}}{\pgfqpoint{4.049380in}{1.716044in}}%
\pgfpathcurveto{\pgfqpoint{4.044336in}{1.711001in}}{\pgfqpoint{4.041502in}{1.704159in}}{\pgfqpoint{4.041502in}{1.697026in}}%
\pgfpathcurveto{\pgfqpoint{4.041502in}{1.689893in}}{\pgfqpoint{4.044336in}{1.683052in}}{\pgfqpoint{4.049380in}{1.678008in}}%
\pgfpathcurveto{\pgfqpoint{4.054424in}{1.672964in}}{\pgfqpoint{4.061265in}{1.670130in}}{\pgfqpoint{4.068398in}{1.670130in}}%
\pgfpathclose%
\pgfusepath{stroke,fill}%
\end{pgfscope}%
\begin{pgfscope}%
\pgfpathrectangle{\pgfqpoint{2.867647in}{0.500000in}}{\pgfqpoint{1.764706in}{1.700000in}}%
\pgfusepath{clip}%
\pgfsetbuttcap%
\pgfsetroundjoin%
\definecolor{currentfill}{rgb}{0.981377,0.920617,0.865369}%
\pgfsetfillcolor{currentfill}%
\pgfsetlinewidth{0.311001pt}%
\definecolor{currentstroke}{rgb}{1.000000,1.000000,1.000000}%
\pgfsetstrokecolor{currentstroke}%
\pgfsetdash{}{0pt}%
\pgfpathmoveto{\pgfqpoint{4.187321in}{1.186011in}}%
\pgfpathcurveto{\pgfqpoint{4.194453in}{1.186011in}}{\pgfqpoint{4.201295in}{1.188845in}}{\pgfqpoint{4.206339in}{1.193888in}}%
\pgfpathcurveto{\pgfqpoint{4.211382in}{1.198932in}}{\pgfqpoint{4.214216in}{1.205774in}}{\pgfqpoint{4.214216in}{1.212906in}}%
\pgfpathcurveto{\pgfqpoint{4.214216in}{1.220039in}}{\pgfqpoint{4.211382in}{1.226881in}}{\pgfqpoint{4.206339in}{1.231925in}}%
\pgfpathcurveto{\pgfqpoint{4.201295in}{1.236968in}}{\pgfqpoint{4.194453in}{1.239802in}}{\pgfqpoint{4.187321in}{1.239802in}}%
\pgfpathcurveto{\pgfqpoint{4.180188in}{1.239802in}}{\pgfqpoint{4.173346in}{1.236968in}}{\pgfqpoint{4.168302in}{1.231925in}}%
\pgfpathcurveto{\pgfqpoint{4.163259in}{1.226881in}}{\pgfqpoint{4.160425in}{1.220039in}}{\pgfqpoint{4.160425in}{1.212906in}}%
\pgfpathcurveto{\pgfqpoint{4.160425in}{1.205774in}}{\pgfqpoint{4.163259in}{1.198932in}}{\pgfqpoint{4.168302in}{1.193888in}}%
\pgfpathcurveto{\pgfqpoint{4.173346in}{1.188845in}}{\pgfqpoint{4.180188in}{1.186011in}}{\pgfqpoint{4.187321in}{1.186011in}}%
\pgfpathclose%
\pgfusepath{stroke,fill}%
\end{pgfscope}%
\begin{pgfscope}%
\pgfpathrectangle{\pgfqpoint{2.867647in}{0.500000in}}{\pgfqpoint{1.764706in}{1.700000in}}%
\pgfusepath{clip}%
\pgfsetbuttcap%
\pgfsetroundjoin%
\definecolor{currentfill}{rgb}{0.971202,0.827364,0.728520}%
\pgfsetfillcolor{currentfill}%
\pgfsetlinewidth{0.311001pt}%
\definecolor{currentstroke}{rgb}{1.000000,1.000000,1.000000}%
\pgfsetstrokecolor{currentstroke}%
\pgfsetdash{}{0pt}%
\pgfpathmoveto{\pgfqpoint{4.083866in}{1.149193in}}%
\pgfpathcurveto{\pgfqpoint{4.090999in}{1.149193in}}{\pgfqpoint{4.097841in}{1.152027in}}{\pgfqpoint{4.102884in}{1.157070in}}%
\pgfpathcurveto{\pgfqpoint{4.107928in}{1.162114in}}{\pgfqpoint{4.110762in}{1.168956in}}{\pgfqpoint{4.110762in}{1.176088in}}%
\pgfpathcurveto{\pgfqpoint{4.110762in}{1.183221in}}{\pgfqpoint{4.107928in}{1.190063in}}{\pgfqpoint{4.102884in}{1.195107in}}%
\pgfpathcurveto{\pgfqpoint{4.097841in}{1.200150in}}{\pgfqpoint{4.090999in}{1.202984in}}{\pgfqpoint{4.083866in}{1.202984in}}%
\pgfpathcurveto{\pgfqpoint{4.076733in}{1.202984in}}{\pgfqpoint{4.069892in}{1.200150in}}{\pgfqpoint{4.064848in}{1.195107in}}%
\pgfpathcurveto{\pgfqpoint{4.059804in}{1.190063in}}{\pgfqpoint{4.056971in}{1.183221in}}{\pgfqpoint{4.056971in}{1.176088in}}%
\pgfpathcurveto{\pgfqpoint{4.056971in}{1.168956in}}{\pgfqpoint{4.059804in}{1.162114in}}{\pgfqpoint{4.064848in}{1.157070in}}%
\pgfpathcurveto{\pgfqpoint{4.069892in}{1.152027in}}{\pgfqpoint{4.076733in}{1.149193in}}{\pgfqpoint{4.083866in}{1.149193in}}%
\pgfpathclose%
\pgfusepath{stroke,fill}%
\end{pgfscope}%
\begin{pgfscope}%
\pgfpathrectangle{\pgfqpoint{2.867647in}{0.500000in}}{\pgfqpoint{1.764706in}{1.700000in}}%
\pgfusepath{clip}%
\pgfsetbuttcap%
\pgfsetroundjoin%
\definecolor{currentfill}{rgb}{0.974412,0.862387,0.780156}%
\pgfsetfillcolor{currentfill}%
\pgfsetlinewidth{0.311001pt}%
\definecolor{currentstroke}{rgb}{1.000000,1.000000,1.000000}%
\pgfsetstrokecolor{currentstroke}%
\pgfsetdash{}{0pt}%
\pgfpathmoveto{\pgfqpoint{4.126988in}{1.266687in}}%
\pgfpathcurveto{\pgfqpoint{4.134120in}{1.266687in}}{\pgfqpoint{4.140962in}{1.269521in}}{\pgfqpoint{4.146006in}{1.274565in}}%
\pgfpathcurveto{\pgfqpoint{4.151049in}{1.279609in}}{\pgfqpoint{4.153883in}{1.286450in}}{\pgfqpoint{4.153883in}{1.293583in}}%
\pgfpathcurveto{\pgfqpoint{4.153883in}{1.300716in}}{\pgfqpoint{4.151049in}{1.307557in}}{\pgfqpoint{4.146006in}{1.312601in}}%
\pgfpathcurveto{\pgfqpoint{4.140962in}{1.317645in}}{\pgfqpoint{4.134120in}{1.320479in}}{\pgfqpoint{4.126988in}{1.320479in}}%
\pgfpathcurveto{\pgfqpoint{4.119855in}{1.320479in}}{\pgfqpoint{4.113013in}{1.317645in}}{\pgfqpoint{4.107969in}{1.312601in}}%
\pgfpathcurveto{\pgfqpoint{4.102926in}{1.307557in}}{\pgfqpoint{4.100092in}{1.300716in}}{\pgfqpoint{4.100092in}{1.293583in}}%
\pgfpathcurveto{\pgfqpoint{4.100092in}{1.286450in}}{\pgfqpoint{4.102926in}{1.279609in}}{\pgfqpoint{4.107969in}{1.274565in}}%
\pgfpathcurveto{\pgfqpoint{4.113013in}{1.269521in}}{\pgfqpoint{4.119855in}{1.266687in}}{\pgfqpoint{4.126988in}{1.266687in}}%
\pgfpathclose%
\pgfusepath{stroke,fill}%
\end{pgfscope}%
\begin{pgfscope}%
\pgfpathrectangle{\pgfqpoint{2.867647in}{0.500000in}}{\pgfqpoint{1.764706in}{1.700000in}}%
\pgfusepath{clip}%
\pgfsetbuttcap%
\pgfsetroundjoin%
\definecolor{currentfill}{rgb}{0.979124,0.903132,0.839793}%
\pgfsetfillcolor{currentfill}%
\pgfsetlinewidth{0.311001pt}%
\definecolor{currentstroke}{rgb}{1.000000,1.000000,1.000000}%
\pgfsetstrokecolor{currentstroke}%
\pgfsetdash{}{0pt}%
\pgfpathmoveto{\pgfqpoint{4.132302in}{1.080271in}}%
\pgfpathcurveto{\pgfqpoint{4.139435in}{1.080271in}}{\pgfqpoint{4.146277in}{1.083105in}}{\pgfqpoint{4.151320in}{1.088149in}}%
\pgfpathcurveto{\pgfqpoint{4.156364in}{1.093193in}}{\pgfqpoint{4.159198in}{1.100034in}}{\pgfqpoint{4.159198in}{1.107167in}}%
\pgfpathcurveto{\pgfqpoint{4.159198in}{1.114300in}}{\pgfqpoint{4.156364in}{1.121142in}}{\pgfqpoint{4.151320in}{1.126185in}}%
\pgfpathcurveto{\pgfqpoint{4.146277in}{1.131229in}}{\pgfqpoint{4.139435in}{1.134063in}}{\pgfqpoint{4.132302in}{1.134063in}}%
\pgfpathcurveto{\pgfqpoint{4.125169in}{1.134063in}}{\pgfqpoint{4.118328in}{1.131229in}}{\pgfqpoint{4.113284in}{1.126185in}}%
\pgfpathcurveto{\pgfqpoint{4.108240in}{1.121142in}}{\pgfqpoint{4.105406in}{1.114300in}}{\pgfqpoint{4.105406in}{1.107167in}}%
\pgfpathcurveto{\pgfqpoint{4.105406in}{1.100034in}}{\pgfqpoint{4.108240in}{1.093193in}}{\pgfqpoint{4.113284in}{1.088149in}}%
\pgfpathcurveto{\pgfqpoint{4.118328in}{1.083105in}}{\pgfqpoint{4.125169in}{1.080271in}}{\pgfqpoint{4.132302in}{1.080271in}}%
\pgfpathclose%
\pgfusepath{stroke,fill}%
\end{pgfscope}%
\begin{pgfscope}%
\pgfpathrectangle{\pgfqpoint{2.867647in}{0.500000in}}{\pgfqpoint{1.764706in}{1.700000in}}%
\pgfusepath{clip}%
\pgfsetbuttcap%
\pgfsetroundjoin%
\definecolor{currentfill}{rgb}{0.970255,0.815666,0.711203}%
\pgfsetfillcolor{currentfill}%
\pgfsetlinewidth{0.311001pt}%
\definecolor{currentstroke}{rgb}{1.000000,1.000000,1.000000}%
\pgfsetstrokecolor{currentstroke}%
\pgfsetdash{}{0pt}%
\pgfpathmoveto{\pgfqpoint{4.099048in}{1.410353in}}%
\pgfpathcurveto{\pgfqpoint{4.106181in}{1.410353in}}{\pgfqpoint{4.113023in}{1.413187in}}{\pgfqpoint{4.118066in}{1.418230in}}%
\pgfpathcurveto{\pgfqpoint{4.123110in}{1.423274in}}{\pgfqpoint{4.125944in}{1.430116in}}{\pgfqpoint{4.125944in}{1.437248in}}%
\pgfpathcurveto{\pgfqpoint{4.125944in}{1.444381in}}{\pgfqpoint{4.123110in}{1.451223in}}{\pgfqpoint{4.118066in}{1.456266in}}%
\pgfpathcurveto{\pgfqpoint{4.113023in}{1.461310in}}{\pgfqpoint{4.106181in}{1.464144in}}{\pgfqpoint{4.099048in}{1.464144in}}%
\pgfpathcurveto{\pgfqpoint{4.091915in}{1.464144in}}{\pgfqpoint{4.085074in}{1.461310in}}{\pgfqpoint{4.080030in}{1.456266in}}%
\pgfpathcurveto{\pgfqpoint{4.074986in}{1.451223in}}{\pgfqpoint{4.072152in}{1.444381in}}{\pgfqpoint{4.072152in}{1.437248in}}%
\pgfpathcurveto{\pgfqpoint{4.072152in}{1.430116in}}{\pgfqpoint{4.074986in}{1.423274in}}{\pgfqpoint{4.080030in}{1.418230in}}%
\pgfpathcurveto{\pgfqpoint{4.085074in}{1.413187in}}{\pgfqpoint{4.091915in}{1.410353in}}{\pgfqpoint{4.099048in}{1.410353in}}%
\pgfpathclose%
\pgfusepath{stroke,fill}%
\end{pgfscope}%
\begin{pgfscope}%
\pgfpathrectangle{\pgfqpoint{2.867647in}{0.500000in}}{\pgfqpoint{1.764706in}{1.700000in}}%
\pgfusepath{clip}%
\pgfsetbuttcap%
\pgfsetroundjoin%
\definecolor{currentfill}{rgb}{0.947270,0.405591,0.279023}%
\pgfsetfillcolor{currentfill}%
\pgfsetlinewidth{0.311001pt}%
\definecolor{currentstroke}{rgb}{1.000000,1.000000,1.000000}%
\pgfsetstrokecolor{currentstroke}%
\pgfsetdash{}{0pt}%
\pgfpathmoveto{\pgfqpoint{3.843153in}{1.695418in}}%
\pgfpathcurveto{\pgfqpoint{3.850286in}{1.695418in}}{\pgfqpoint{3.857128in}{1.698252in}}{\pgfqpoint{3.862171in}{1.703296in}}%
\pgfpathcurveto{\pgfqpoint{3.867215in}{1.708339in}}{\pgfqpoint{3.870049in}{1.715181in}}{\pgfqpoint{3.870049in}{1.722314in}}%
\pgfpathcurveto{\pgfqpoint{3.870049in}{1.729447in}}{\pgfqpoint{3.867215in}{1.736288in}}{\pgfqpoint{3.862171in}{1.741332in}}%
\pgfpathcurveto{\pgfqpoint{3.857128in}{1.746376in}}{\pgfqpoint{3.850286in}{1.749209in}}{\pgfqpoint{3.843153in}{1.749209in}}%
\pgfpathcurveto{\pgfqpoint{3.836020in}{1.749209in}}{\pgfqpoint{3.829179in}{1.746376in}}{\pgfqpoint{3.824135in}{1.741332in}}%
\pgfpathcurveto{\pgfqpoint{3.819091in}{1.736288in}}{\pgfqpoint{3.816257in}{1.729447in}}{\pgfqpoint{3.816257in}{1.722314in}}%
\pgfpathcurveto{\pgfqpoint{3.816257in}{1.715181in}}{\pgfqpoint{3.819091in}{1.708339in}}{\pgfqpoint{3.824135in}{1.703296in}}%
\pgfpathcurveto{\pgfqpoint{3.829179in}{1.698252in}}{\pgfqpoint{3.836020in}{1.695418in}}{\pgfqpoint{3.843153in}{1.695418in}}%
\pgfpathclose%
\pgfusepath{stroke,fill}%
\end{pgfscope}%
\begin{pgfscope}%
\pgfpathrectangle{\pgfqpoint{2.867647in}{0.500000in}}{\pgfqpoint{1.764706in}{1.700000in}}%
\pgfusepath{clip}%
\pgfsetbuttcap%
\pgfsetroundjoin%
\definecolor{currentfill}{rgb}{0.979891,0.908948,0.848279}%
\pgfsetfillcolor{currentfill}%
\pgfsetlinewidth{0.311001pt}%
\definecolor{currentstroke}{rgb}{1.000000,1.000000,1.000000}%
\pgfsetstrokecolor{currentstroke}%
\pgfsetdash{}{0pt}%
\pgfpathmoveto{\pgfqpoint{4.176193in}{1.124580in}}%
\pgfpathcurveto{\pgfqpoint{4.183326in}{1.124580in}}{\pgfqpoint{4.190167in}{1.127414in}}{\pgfqpoint{4.195211in}{1.132458in}}%
\pgfpathcurveto{\pgfqpoint{4.200255in}{1.137501in}}{\pgfqpoint{4.203088in}{1.144343in}}{\pgfqpoint{4.203088in}{1.151476in}}%
\pgfpathcurveto{\pgfqpoint{4.203088in}{1.158609in}}{\pgfqpoint{4.200255in}{1.165450in}}{\pgfqpoint{4.195211in}{1.170494in}}%
\pgfpathcurveto{\pgfqpoint{4.190167in}{1.175538in}}{\pgfqpoint{4.183326in}{1.178372in}}{\pgfqpoint{4.176193in}{1.178372in}}%
\pgfpathcurveto{\pgfqpoint{4.169060in}{1.178372in}}{\pgfqpoint{4.162218in}{1.175538in}}{\pgfqpoint{4.157175in}{1.170494in}}%
\pgfpathcurveto{\pgfqpoint{4.152131in}{1.165450in}}{\pgfqpoint{4.149297in}{1.158609in}}{\pgfqpoint{4.149297in}{1.151476in}}%
\pgfpathcurveto{\pgfqpoint{4.149297in}{1.144343in}}{\pgfqpoint{4.152131in}{1.137501in}}{\pgfqpoint{4.157175in}{1.132458in}}%
\pgfpathcurveto{\pgfqpoint{4.162218in}{1.127414in}}{\pgfqpoint{4.169060in}{1.124580in}}{\pgfqpoint{4.176193in}{1.124580in}}%
\pgfpathclose%
\pgfusepath{stroke,fill}%
\end{pgfscope}%
\begin{pgfscope}%
\pgfpathrectangle{\pgfqpoint{2.867647in}{0.500000in}}{\pgfqpoint{1.764706in}{1.700000in}}%
\pgfusepath{clip}%
\pgfsetbuttcap%
\pgfsetroundjoin%
\definecolor{currentfill}{rgb}{0.962985,0.612625,0.451451}%
\pgfsetfillcolor{currentfill}%
\pgfsetlinewidth{0.311001pt}%
\definecolor{currentstroke}{rgb}{1.000000,1.000000,1.000000}%
\pgfsetstrokecolor{currentstroke}%
\pgfsetdash{}{0pt}%
\pgfpathmoveto{\pgfqpoint{3.982214in}{0.873152in}}%
\pgfpathcurveto{\pgfqpoint{3.989347in}{0.873152in}}{\pgfqpoint{3.996189in}{0.875986in}}{\pgfqpoint{4.001232in}{0.881030in}}%
\pgfpathcurveto{\pgfqpoint{4.006276in}{0.886073in}}{\pgfqpoint{4.009110in}{0.892915in}}{\pgfqpoint{4.009110in}{0.900048in}}%
\pgfpathcurveto{\pgfqpoint{4.009110in}{0.907181in}}{\pgfqpoint{4.006276in}{0.914022in}}{\pgfqpoint{4.001232in}{0.919066in}}%
\pgfpathcurveto{\pgfqpoint{3.996189in}{0.924110in}}{\pgfqpoint{3.989347in}{0.926944in}}{\pgfqpoint{3.982214in}{0.926944in}}%
\pgfpathcurveto{\pgfqpoint{3.975081in}{0.926944in}}{\pgfqpoint{3.968240in}{0.924110in}}{\pgfqpoint{3.963196in}{0.919066in}}%
\pgfpathcurveto{\pgfqpoint{3.958152in}{0.914022in}}{\pgfqpoint{3.955318in}{0.907181in}}{\pgfqpoint{3.955318in}{0.900048in}}%
\pgfpathcurveto{\pgfqpoint{3.955318in}{0.892915in}}{\pgfqpoint{3.958152in}{0.886073in}}{\pgfqpoint{3.963196in}{0.881030in}}%
\pgfpathcurveto{\pgfqpoint{3.968240in}{0.875986in}}{\pgfqpoint{3.975081in}{0.873152in}}{\pgfqpoint{3.982214in}{0.873152in}}%
\pgfpathclose%
\pgfusepath{stroke,fill}%
\end{pgfscope}%
\begin{pgfscope}%
\pgfpathrectangle{\pgfqpoint{2.867647in}{0.500000in}}{\pgfqpoint{1.764706in}{1.700000in}}%
\pgfusepath{clip}%
\pgfsetbuttcap%
\pgfsetroundjoin%
\definecolor{currentfill}{rgb}{0.970718,0.821518,0.719872}%
\pgfsetfillcolor{currentfill}%
\pgfsetlinewidth{0.311001pt}%
\definecolor{currentstroke}{rgb}{1.000000,1.000000,1.000000}%
\pgfsetstrokecolor{currentstroke}%
\pgfsetdash{}{0pt}%
\pgfpathmoveto{\pgfqpoint{4.087355in}{1.461236in}}%
\pgfpathcurveto{\pgfqpoint{4.094488in}{1.461236in}}{\pgfqpoint{4.101329in}{1.464070in}}{\pgfqpoint{4.106373in}{1.469113in}}%
\pgfpathcurveto{\pgfqpoint{4.111417in}{1.474157in}}{\pgfqpoint{4.114251in}{1.480999in}}{\pgfqpoint{4.114251in}{1.488132in}}%
\pgfpathcurveto{\pgfqpoint{4.114251in}{1.495264in}}{\pgfqpoint{4.111417in}{1.502106in}}{\pgfqpoint{4.106373in}{1.507150in}}%
\pgfpathcurveto{\pgfqpoint{4.101329in}{1.512193in}}{\pgfqpoint{4.094488in}{1.515027in}}{\pgfqpoint{4.087355in}{1.515027in}}%
\pgfpathcurveto{\pgfqpoint{4.080222in}{1.515027in}}{\pgfqpoint{4.073380in}{1.512193in}}{\pgfqpoint{4.068337in}{1.507150in}}%
\pgfpathcurveto{\pgfqpoint{4.063293in}{1.502106in}}{\pgfqpoint{4.060459in}{1.495264in}}{\pgfqpoint{4.060459in}{1.488132in}}%
\pgfpathcurveto{\pgfqpoint{4.060459in}{1.480999in}}{\pgfqpoint{4.063293in}{1.474157in}}{\pgfqpoint{4.068337in}{1.469113in}}%
\pgfpathcurveto{\pgfqpoint{4.073380in}{1.464070in}}{\pgfqpoint{4.080222in}{1.461236in}}{\pgfqpoint{4.087355in}{1.461236in}}%
\pgfpathclose%
\pgfusepath{stroke,fill}%
\end{pgfscope}%
\begin{pgfscope}%
\pgfpathrectangle{\pgfqpoint{2.867647in}{0.500000in}}{\pgfqpoint{1.764706in}{1.700000in}}%
\pgfusepath{clip}%
\pgfsetbuttcap%
\pgfsetroundjoin%
\definecolor{currentfill}{rgb}{0.973271,0.850724,0.762998}%
\pgfsetfillcolor{currentfill}%
\pgfsetlinewidth{0.311001pt}%
\definecolor{currentstroke}{rgb}{1.000000,1.000000,1.000000}%
\pgfsetstrokecolor{currentstroke}%
\pgfsetdash{}{0pt}%
\pgfpathmoveto{\pgfqpoint{4.165437in}{1.020391in}}%
\pgfpathcurveto{\pgfqpoint{4.172570in}{1.020391in}}{\pgfqpoint{4.179411in}{1.023224in}}{\pgfqpoint{4.184455in}{1.028268in}}%
\pgfpathcurveto{\pgfqpoint{4.189499in}{1.033312in}}{\pgfqpoint{4.192333in}{1.040153in}}{\pgfqpoint{4.192333in}{1.047286in}}%
\pgfpathcurveto{\pgfqpoint{4.192333in}{1.054419in}}{\pgfqpoint{4.189499in}{1.061261in}}{\pgfqpoint{4.184455in}{1.066304in}}%
\pgfpathcurveto{\pgfqpoint{4.179411in}{1.071348in}}{\pgfqpoint{4.172570in}{1.074182in}}{\pgfqpoint{4.165437in}{1.074182in}}%
\pgfpathcurveto{\pgfqpoint{4.158304in}{1.074182in}}{\pgfqpoint{4.151462in}{1.071348in}}{\pgfqpoint{4.146419in}{1.066304in}}%
\pgfpathcurveto{\pgfqpoint{4.141375in}{1.061261in}}{\pgfqpoint{4.138541in}{1.054419in}}{\pgfqpoint{4.138541in}{1.047286in}}%
\pgfpathcurveto{\pgfqpoint{4.138541in}{1.040153in}}{\pgfqpoint{4.141375in}{1.033312in}}{\pgfqpoint{4.146419in}{1.028268in}}%
\pgfpathcurveto{\pgfqpoint{4.151462in}{1.023224in}}{\pgfqpoint{4.158304in}{1.020391in}}{\pgfqpoint{4.165437in}{1.020391in}}%
\pgfpathclose%
\pgfusepath{stroke,fill}%
\end{pgfscope}%
\begin{pgfscope}%
\pgfpathrectangle{\pgfqpoint{2.867647in}{0.500000in}}{\pgfqpoint{1.764706in}{1.700000in}}%
\pgfusepath{clip}%
\pgfsetbuttcap%
\pgfsetroundjoin%
\definecolor{currentfill}{rgb}{0.981377,0.920617,0.865369}%
\pgfsetfillcolor{currentfill}%
\pgfsetlinewidth{0.311001pt}%
\definecolor{currentstroke}{rgb}{1.000000,1.000000,1.000000}%
\pgfsetstrokecolor{currentstroke}%
\pgfsetdash{}{0pt}%
\pgfpathmoveto{\pgfqpoint{4.181803in}{1.257197in}}%
\pgfpathcurveto{\pgfqpoint{4.188936in}{1.257197in}}{\pgfqpoint{4.195778in}{1.260031in}}{\pgfqpoint{4.200821in}{1.265075in}}%
\pgfpathcurveto{\pgfqpoint{4.205865in}{1.270119in}}{\pgfqpoint{4.208699in}{1.276960in}}{\pgfqpoint{4.208699in}{1.284093in}}%
\pgfpathcurveto{\pgfqpoint{4.208699in}{1.291226in}}{\pgfqpoint{4.205865in}{1.298067in}}{\pgfqpoint{4.200821in}{1.303111in}}%
\pgfpathcurveto{\pgfqpoint{4.195778in}{1.308155in}}{\pgfqpoint{4.188936in}{1.310989in}}{\pgfqpoint{4.181803in}{1.310989in}}%
\pgfpathcurveto{\pgfqpoint{4.174670in}{1.310989in}}{\pgfqpoint{4.167829in}{1.308155in}}{\pgfqpoint{4.162785in}{1.303111in}}%
\pgfpathcurveto{\pgfqpoint{4.157741in}{1.298067in}}{\pgfqpoint{4.154908in}{1.291226in}}{\pgfqpoint{4.154908in}{1.284093in}}%
\pgfpathcurveto{\pgfqpoint{4.154908in}{1.276960in}}{\pgfqpoint{4.157741in}{1.270119in}}{\pgfqpoint{4.162785in}{1.265075in}}%
\pgfpathcurveto{\pgfqpoint{4.167829in}{1.260031in}}{\pgfqpoint{4.174670in}{1.257197in}}{\pgfqpoint{4.181803in}{1.257197in}}%
\pgfpathclose%
\pgfusepath{stroke,fill}%
\end{pgfscope}%
\begin{pgfscope}%
\pgfpathrectangle{\pgfqpoint{2.867647in}{0.500000in}}{\pgfqpoint{1.764706in}{1.700000in}}%
\pgfusepath{clip}%
\pgfsetbuttcap%
\pgfsetroundjoin%
\definecolor{currentfill}{rgb}{0.977657,0.891500,0.822809}%
\pgfsetfillcolor{currentfill}%
\pgfsetlinewidth{0.311001pt}%
\definecolor{currentstroke}{rgb}{1.000000,1.000000,1.000000}%
\pgfsetstrokecolor{currentstroke}%
\pgfsetdash{}{0pt}%
\pgfpathmoveto{\pgfqpoint{4.226191in}{1.406555in}}%
\pgfpathcurveto{\pgfqpoint{4.233324in}{1.406555in}}{\pgfqpoint{4.240166in}{1.409389in}}{\pgfqpoint{4.245209in}{1.414433in}}%
\pgfpathcurveto{\pgfqpoint{4.250253in}{1.419477in}}{\pgfqpoint{4.253087in}{1.426318in}}{\pgfqpoint{4.253087in}{1.433451in}}%
\pgfpathcurveto{\pgfqpoint{4.253087in}{1.440584in}}{\pgfqpoint{4.250253in}{1.447425in}}{\pgfqpoint{4.245209in}{1.452469in}}%
\pgfpathcurveto{\pgfqpoint{4.240166in}{1.457513in}}{\pgfqpoint{4.233324in}{1.460347in}}{\pgfqpoint{4.226191in}{1.460347in}}%
\pgfpathcurveto{\pgfqpoint{4.219059in}{1.460347in}}{\pgfqpoint{4.212217in}{1.457513in}}{\pgfqpoint{4.207173in}{1.452469in}}%
\pgfpathcurveto{\pgfqpoint{4.202130in}{1.447425in}}{\pgfqpoint{4.199296in}{1.440584in}}{\pgfqpoint{4.199296in}{1.433451in}}%
\pgfpathcurveto{\pgfqpoint{4.199296in}{1.426318in}}{\pgfqpoint{4.202130in}{1.419477in}}{\pgfqpoint{4.207173in}{1.414433in}}%
\pgfpathcurveto{\pgfqpoint{4.212217in}{1.409389in}}{\pgfqpoint{4.219059in}{1.406555in}}{\pgfqpoint{4.226191in}{1.406555in}}%
\pgfpathclose%
\pgfusepath{stroke,fill}%
\end{pgfscope}%
\begin{pgfscope}%
\pgfpathrectangle{\pgfqpoint{2.867647in}{0.500000in}}{\pgfqpoint{1.764706in}{1.700000in}}%
\pgfusepath{clip}%
\pgfsetbuttcap%
\pgfsetroundjoin%
\definecolor{currentfill}{rgb}{0.951650,0.442241,0.302145}%
\pgfsetfillcolor{currentfill}%
\pgfsetlinewidth{0.311001pt}%
\definecolor{currentstroke}{rgb}{1.000000,1.000000,1.000000}%
\pgfsetstrokecolor{currentstroke}%
\pgfsetdash{}{0pt}%
\pgfpathmoveto{\pgfqpoint{3.850337in}{1.739075in}}%
\pgfpathcurveto{\pgfqpoint{3.857470in}{1.739075in}}{\pgfqpoint{3.864312in}{1.741908in}}{\pgfqpoint{3.869355in}{1.746952in}}%
\pgfpathcurveto{\pgfqpoint{3.874399in}{1.751996in}}{\pgfqpoint{3.877233in}{1.758837in}}{\pgfqpoint{3.877233in}{1.765970in}}%
\pgfpathcurveto{\pgfqpoint{3.877233in}{1.773103in}}{\pgfqpoint{3.874399in}{1.779945in}}{\pgfqpoint{3.869355in}{1.784988in}}%
\pgfpathcurveto{\pgfqpoint{3.864312in}{1.790032in}}{\pgfqpoint{3.857470in}{1.792866in}}{\pgfqpoint{3.850337in}{1.792866in}}%
\pgfpathcurveto{\pgfqpoint{3.843204in}{1.792866in}}{\pgfqpoint{3.836363in}{1.790032in}}{\pgfqpoint{3.831319in}{1.784988in}}%
\pgfpathcurveto{\pgfqpoint{3.826275in}{1.779945in}}{\pgfqpoint{3.823442in}{1.773103in}}{\pgfqpoint{3.823442in}{1.765970in}}%
\pgfpathcurveto{\pgfqpoint{3.823442in}{1.758837in}}{\pgfqpoint{3.826275in}{1.751996in}}{\pgfqpoint{3.831319in}{1.746952in}}%
\pgfpathcurveto{\pgfqpoint{3.836363in}{1.741908in}}{\pgfqpoint{3.843204in}{1.739075in}}{\pgfqpoint{3.850337in}{1.739075in}}%
\pgfpathclose%
\pgfusepath{stroke,fill}%
\end{pgfscope}%
\begin{pgfscope}%
\pgfpathrectangle{\pgfqpoint{2.867647in}{0.500000in}}{\pgfqpoint{1.764706in}{1.700000in}}%
\pgfusepath{clip}%
\pgfsetbuttcap%
\pgfsetroundjoin%
\definecolor{currentfill}{rgb}{0.966560,0.756582,0.625273}%
\pgfsetfillcolor{currentfill}%
\pgfsetlinewidth{0.311001pt}%
\definecolor{currentstroke}{rgb}{1.000000,1.000000,1.000000}%
\pgfsetstrokecolor{currentstroke}%
\pgfsetdash{}{0pt}%
\pgfpathmoveto{\pgfqpoint{4.020508in}{1.600919in}}%
\pgfpathcurveto{\pgfqpoint{4.027641in}{1.600919in}}{\pgfqpoint{4.034483in}{1.603752in}}{\pgfqpoint{4.039526in}{1.608796in}}%
\pgfpathcurveto{\pgfqpoint{4.044570in}{1.613840in}}{\pgfqpoint{4.047404in}{1.620681in}}{\pgfqpoint{4.047404in}{1.627814in}}%
\pgfpathcurveto{\pgfqpoint{4.047404in}{1.634947in}}{\pgfqpoint{4.044570in}{1.641789in}}{\pgfqpoint{4.039526in}{1.646832in}}%
\pgfpathcurveto{\pgfqpoint{4.034483in}{1.651876in}}{\pgfqpoint{4.027641in}{1.654710in}}{\pgfqpoint{4.020508in}{1.654710in}}%
\pgfpathcurveto{\pgfqpoint{4.013375in}{1.654710in}}{\pgfqpoint{4.006534in}{1.651876in}}{\pgfqpoint{4.001490in}{1.646832in}}%
\pgfpathcurveto{\pgfqpoint{3.996446in}{1.641789in}}{\pgfqpoint{3.993613in}{1.634947in}}{\pgfqpoint{3.993613in}{1.627814in}}%
\pgfpathcurveto{\pgfqpoint{3.993613in}{1.620681in}}{\pgfqpoint{3.996446in}{1.613840in}}{\pgfqpoint{4.001490in}{1.608796in}}%
\pgfpathcurveto{\pgfqpoint{4.006534in}{1.603752in}}{\pgfqpoint{4.013375in}{1.600919in}}{\pgfqpoint{4.020508in}{1.600919in}}%
\pgfpathclose%
\pgfusepath{stroke,fill}%
\end{pgfscope}%
\begin{pgfscope}%
\pgfpathrectangle{\pgfqpoint{2.867647in}{0.500000in}}{\pgfqpoint{1.764706in}{1.700000in}}%
\pgfusepath{clip}%
\pgfsetbuttcap%
\pgfsetroundjoin%
\definecolor{currentfill}{rgb}{0.970718,0.821518,0.719872}%
\pgfsetfillcolor{currentfill}%
\pgfsetlinewidth{0.311001pt}%
\definecolor{currentstroke}{rgb}{1.000000,1.000000,1.000000}%
\pgfsetstrokecolor{currentstroke}%
\pgfsetdash{}{0pt}%
\pgfpathmoveto{\pgfqpoint{4.109150in}{1.276517in}}%
\pgfpathcurveto{\pgfqpoint{4.116283in}{1.276517in}}{\pgfqpoint{4.123125in}{1.279350in}}{\pgfqpoint{4.128168in}{1.284394in}}%
\pgfpathcurveto{\pgfqpoint{4.133212in}{1.289438in}}{\pgfqpoint{4.136046in}{1.296279in}}{\pgfqpoint{4.136046in}{1.303412in}}%
\pgfpathcurveto{\pgfqpoint{4.136046in}{1.310545in}}{\pgfqpoint{4.133212in}{1.317387in}}{\pgfqpoint{4.128168in}{1.322430in}}%
\pgfpathcurveto{\pgfqpoint{4.123125in}{1.327474in}}{\pgfqpoint{4.116283in}{1.330308in}}{\pgfqpoint{4.109150in}{1.330308in}}%
\pgfpathcurveto{\pgfqpoint{4.102017in}{1.330308in}}{\pgfqpoint{4.095176in}{1.327474in}}{\pgfqpoint{4.090132in}{1.322430in}}%
\pgfpathcurveto{\pgfqpoint{4.085088in}{1.317387in}}{\pgfqpoint{4.082254in}{1.310545in}}{\pgfqpoint{4.082254in}{1.303412in}}%
\pgfpathcurveto{\pgfqpoint{4.082254in}{1.296279in}}{\pgfqpoint{4.085088in}{1.289438in}}{\pgfqpoint{4.090132in}{1.284394in}}%
\pgfpathcurveto{\pgfqpoint{4.095176in}{1.279350in}}{\pgfqpoint{4.102017in}{1.276517in}}{\pgfqpoint{4.109150in}{1.276517in}}%
\pgfpathclose%
\pgfusepath{stroke,fill}%
\end{pgfscope}%
\begin{pgfscope}%
\pgfpathrectangle{\pgfqpoint{2.867647in}{0.500000in}}{\pgfqpoint{1.764706in}{1.700000in}}%
\pgfusepath{clip}%
\pgfsetbuttcap%
\pgfsetroundjoin%
\definecolor{currentfill}{rgb}{0.965753,0.732351,0.592427}%
\pgfsetfillcolor{currentfill}%
\pgfsetlinewidth{0.311001pt}%
\definecolor{currentstroke}{rgb}{1.000000,1.000000,1.000000}%
\pgfsetstrokecolor{currentstroke}%
\pgfsetdash{}{0pt}%
\pgfpathmoveto{\pgfqpoint{4.291599in}{1.426035in}}%
\pgfpathcurveto{\pgfqpoint{4.298732in}{1.426035in}}{\pgfqpoint{4.305574in}{1.428869in}}{\pgfqpoint{4.310617in}{1.433913in}}%
\pgfpathcurveto{\pgfqpoint{4.315661in}{1.438957in}}{\pgfqpoint{4.318495in}{1.445798in}}{\pgfqpoint{4.318495in}{1.452931in}}%
\pgfpathcurveto{\pgfqpoint{4.318495in}{1.460064in}}{\pgfqpoint{4.315661in}{1.466905in}}{\pgfqpoint{4.310617in}{1.471949in}}%
\pgfpathcurveto{\pgfqpoint{4.305574in}{1.476993in}}{\pgfqpoint{4.298732in}{1.479827in}}{\pgfqpoint{4.291599in}{1.479827in}}%
\pgfpathcurveto{\pgfqpoint{4.284466in}{1.479827in}}{\pgfqpoint{4.277625in}{1.476993in}}{\pgfqpoint{4.272581in}{1.471949in}}%
\pgfpathcurveto{\pgfqpoint{4.267537in}{1.466905in}}{\pgfqpoint{4.264704in}{1.460064in}}{\pgfqpoint{4.264704in}{1.452931in}}%
\pgfpathcurveto{\pgfqpoint{4.264704in}{1.445798in}}{\pgfqpoint{4.267537in}{1.438957in}}{\pgfqpoint{4.272581in}{1.433913in}}%
\pgfpathcurveto{\pgfqpoint{4.277625in}{1.428869in}}{\pgfqpoint{4.284466in}{1.426035in}}{\pgfqpoint{4.291599in}{1.426035in}}%
\pgfpathclose%
\pgfusepath{stroke,fill}%
\end{pgfscope}%
\begin{pgfscope}%
\pgfpathrectangle{\pgfqpoint{2.867647in}{0.500000in}}{\pgfqpoint{1.764706in}{1.700000in}}%
\pgfusepath{clip}%
\pgfsetbuttcap%
\pgfsetroundjoin%
\definecolor{currentfill}{rgb}{0.965592,0.726236,0.584384}%
\pgfsetfillcolor{currentfill}%
\pgfsetlinewidth{0.311001pt}%
\definecolor{currentstroke}{rgb}{1.000000,1.000000,1.000000}%
\pgfsetstrokecolor{currentstroke}%
\pgfsetdash{}{0pt}%
\pgfpathmoveto{\pgfqpoint{4.179106in}{0.957048in}}%
\pgfpathcurveto{\pgfqpoint{4.186238in}{0.957048in}}{\pgfqpoint{4.193080in}{0.959882in}}{\pgfqpoint{4.198124in}{0.964926in}}%
\pgfpathcurveto{\pgfqpoint{4.203167in}{0.969969in}}{\pgfqpoint{4.206001in}{0.976811in}}{\pgfqpoint{4.206001in}{0.983944in}}%
\pgfpathcurveto{\pgfqpoint{4.206001in}{0.991077in}}{\pgfqpoint{4.203167in}{0.997918in}}{\pgfqpoint{4.198124in}{1.002962in}}%
\pgfpathcurveto{\pgfqpoint{4.193080in}{1.008006in}}{\pgfqpoint{4.186238in}{1.010840in}}{\pgfqpoint{4.179106in}{1.010840in}}%
\pgfpathcurveto{\pgfqpoint{4.171973in}{1.010840in}}{\pgfqpoint{4.165131in}{1.008006in}}{\pgfqpoint{4.160087in}{1.002962in}}%
\pgfpathcurveto{\pgfqpoint{4.155044in}{0.997918in}}{\pgfqpoint{4.152210in}{0.991077in}}{\pgfqpoint{4.152210in}{0.983944in}}%
\pgfpathcurveto{\pgfqpoint{4.152210in}{0.976811in}}{\pgfqpoint{4.155044in}{0.969969in}}{\pgfqpoint{4.160087in}{0.964926in}}%
\pgfpathcurveto{\pgfqpoint{4.165131in}{0.959882in}}{\pgfqpoint{4.171973in}{0.957048in}}{\pgfqpoint{4.179106in}{0.957048in}}%
\pgfpathclose%
\pgfusepath{stroke,fill}%
\end{pgfscope}%
\begin{pgfscope}%
\pgfpathrectangle{\pgfqpoint{2.867647in}{0.500000in}}{\pgfqpoint{1.764706in}{1.700000in}}%
\pgfusepath{clip}%
\pgfsetbuttcap%
\pgfsetroundjoin%
\definecolor{currentfill}{rgb}{0.975644,0.874038,0.797253}%
\pgfsetfillcolor{currentfill}%
\pgfsetlinewidth{0.311001pt}%
\definecolor{currentstroke}{rgb}{1.000000,1.000000,1.000000}%
\pgfsetstrokecolor{currentstroke}%
\pgfsetdash{}{0pt}%
\pgfpathmoveto{\pgfqpoint{4.237106in}{1.418487in}}%
\pgfpathcurveto{\pgfqpoint{4.244238in}{1.418487in}}{\pgfqpoint{4.251080in}{1.421321in}}{\pgfqpoint{4.256124in}{1.426365in}}%
\pgfpathcurveto{\pgfqpoint{4.261167in}{1.431408in}}{\pgfqpoint{4.264001in}{1.438250in}}{\pgfqpoint{4.264001in}{1.445383in}}%
\pgfpathcurveto{\pgfqpoint{4.264001in}{1.452516in}}{\pgfqpoint{4.261167in}{1.459357in}}{\pgfqpoint{4.256124in}{1.464401in}}%
\pgfpathcurveto{\pgfqpoint{4.251080in}{1.469445in}}{\pgfqpoint{4.244238in}{1.472279in}}{\pgfqpoint{4.237106in}{1.472279in}}%
\pgfpathcurveto{\pgfqpoint{4.229973in}{1.472279in}}{\pgfqpoint{4.223131in}{1.469445in}}{\pgfqpoint{4.218087in}{1.464401in}}%
\pgfpathcurveto{\pgfqpoint{4.213044in}{1.459357in}}{\pgfqpoint{4.210210in}{1.452516in}}{\pgfqpoint{4.210210in}{1.445383in}}%
\pgfpathcurveto{\pgfqpoint{4.210210in}{1.438250in}}{\pgfqpoint{4.213044in}{1.431408in}}{\pgfqpoint{4.218087in}{1.426365in}}%
\pgfpathcurveto{\pgfqpoint{4.223131in}{1.421321in}}{\pgfqpoint{4.229973in}{1.418487in}}{\pgfqpoint{4.237106in}{1.418487in}}%
\pgfpathclose%
\pgfusepath{stroke,fill}%
\end{pgfscope}%
\begin{pgfscope}%
\pgfpathrectangle{\pgfqpoint{2.867647in}{0.500000in}}{\pgfqpoint{1.764706in}{1.700000in}}%
\pgfusepath{clip}%
\pgfsetbuttcap%
\pgfsetroundjoin%
\definecolor{currentfill}{rgb}{0.964032,0.651225,0.493258}%
\pgfsetfillcolor{currentfill}%
\pgfsetlinewidth{0.311001pt}%
\definecolor{currentstroke}{rgb}{1.000000,1.000000,1.000000}%
\pgfsetstrokecolor{currentstroke}%
\pgfsetdash{}{0pt}%
\pgfpathmoveto{\pgfqpoint{4.321988in}{1.307489in}}%
\pgfpathcurveto{\pgfqpoint{4.329121in}{1.307489in}}{\pgfqpoint{4.335963in}{1.310323in}}{\pgfqpoint{4.341006in}{1.315367in}}%
\pgfpathcurveto{\pgfqpoint{4.346050in}{1.320410in}}{\pgfqpoint{4.348884in}{1.327252in}}{\pgfqpoint{4.348884in}{1.334385in}}%
\pgfpathcurveto{\pgfqpoint{4.348884in}{1.341518in}}{\pgfqpoint{4.346050in}{1.348359in}}{\pgfqpoint{4.341006in}{1.353403in}}%
\pgfpathcurveto{\pgfqpoint{4.335963in}{1.358447in}}{\pgfqpoint{4.329121in}{1.361281in}}{\pgfqpoint{4.321988in}{1.361281in}}%
\pgfpathcurveto{\pgfqpoint{4.314855in}{1.361281in}}{\pgfqpoint{4.308014in}{1.358447in}}{\pgfqpoint{4.302970in}{1.353403in}}%
\pgfpathcurveto{\pgfqpoint{4.297926in}{1.348359in}}{\pgfqpoint{4.295092in}{1.341518in}}{\pgfqpoint{4.295092in}{1.334385in}}%
\pgfpathcurveto{\pgfqpoint{4.295092in}{1.327252in}}{\pgfqpoint{4.297926in}{1.320410in}}{\pgfqpoint{4.302970in}{1.315367in}}%
\pgfpathcurveto{\pgfqpoint{4.308014in}{1.310323in}}{\pgfqpoint{4.314855in}{1.307489in}}{\pgfqpoint{4.321988in}{1.307489in}}%
\pgfpathclose%
\pgfusepath{stroke,fill}%
\end{pgfscope}%
\begin{pgfscope}%
\pgfpathrectangle{\pgfqpoint{2.867647in}{0.500000in}}{\pgfqpoint{1.764706in}{1.700000in}}%
\pgfusepath{clip}%
\pgfsetbuttcap%
\pgfsetroundjoin%
\definecolor{currentfill}{rgb}{0.979891,0.908948,0.848279}%
\pgfsetfillcolor{currentfill}%
\pgfsetlinewidth{0.311001pt}%
\definecolor{currentstroke}{rgb}{1.000000,1.000000,1.000000}%
\pgfsetstrokecolor{currentstroke}%
\pgfsetdash{}{0pt}%
\pgfpathmoveto{\pgfqpoint{4.146353in}{1.457233in}}%
\pgfpathcurveto{\pgfqpoint{4.153486in}{1.457233in}}{\pgfqpoint{4.160327in}{1.460067in}}{\pgfqpoint{4.165371in}{1.465110in}}%
\pgfpathcurveto{\pgfqpoint{4.170415in}{1.470154in}}{\pgfqpoint{4.173249in}{1.476996in}}{\pgfqpoint{4.173249in}{1.484129in}}%
\pgfpathcurveto{\pgfqpoint{4.173249in}{1.491261in}}{\pgfqpoint{4.170415in}{1.498103in}}{\pgfqpoint{4.165371in}{1.503147in}}%
\pgfpathcurveto{\pgfqpoint{4.160327in}{1.508190in}}{\pgfqpoint{4.153486in}{1.511024in}}{\pgfqpoint{4.146353in}{1.511024in}}%
\pgfpathcurveto{\pgfqpoint{4.139220in}{1.511024in}}{\pgfqpoint{4.132379in}{1.508190in}}{\pgfqpoint{4.127335in}{1.503147in}}%
\pgfpathcurveto{\pgfqpoint{4.122291in}{1.498103in}}{\pgfqpoint{4.119457in}{1.491261in}}{\pgfqpoint{4.119457in}{1.484129in}}%
\pgfpathcurveto{\pgfqpoint{4.119457in}{1.476996in}}{\pgfqpoint{4.122291in}{1.470154in}}{\pgfqpoint{4.127335in}{1.465110in}}%
\pgfpathcurveto{\pgfqpoint{4.132379in}{1.460067in}}{\pgfqpoint{4.139220in}{1.457233in}}{\pgfqpoint{4.146353in}{1.457233in}}%
\pgfpathclose%
\pgfusepath{stroke,fill}%
\end{pgfscope}%
\begin{pgfscope}%
\pgfpathrectangle{\pgfqpoint{2.867647in}{0.500000in}}{\pgfqpoint{1.764706in}{1.700000in}}%
\pgfusepath{clip}%
\pgfsetbuttcap%
\pgfsetroundjoin%
\definecolor{currentfill}{rgb}{0.975018,0.868213,0.788710}%
\pgfsetfillcolor{currentfill}%
\pgfsetlinewidth{0.311001pt}%
\definecolor{currentstroke}{rgb}{1.000000,1.000000,1.000000}%
\pgfsetstrokecolor{currentstroke}%
\pgfsetdash{}{0pt}%
\pgfpathmoveto{\pgfqpoint{4.117489in}{1.647461in}}%
\pgfpathcurveto{\pgfqpoint{4.124622in}{1.647461in}}{\pgfqpoint{4.131463in}{1.650295in}}{\pgfqpoint{4.136507in}{1.655339in}}%
\pgfpathcurveto{\pgfqpoint{4.141551in}{1.660382in}}{\pgfqpoint{4.144384in}{1.667224in}}{\pgfqpoint{4.144384in}{1.674357in}}%
\pgfpathcurveto{\pgfqpoint{4.144384in}{1.681490in}}{\pgfqpoint{4.141551in}{1.688331in}}{\pgfqpoint{4.136507in}{1.693375in}}%
\pgfpathcurveto{\pgfqpoint{4.131463in}{1.698419in}}{\pgfqpoint{4.124622in}{1.701253in}}{\pgfqpoint{4.117489in}{1.701253in}}%
\pgfpathcurveto{\pgfqpoint{4.110356in}{1.701253in}}{\pgfqpoint{4.103514in}{1.698419in}}{\pgfqpoint{4.098471in}{1.693375in}}%
\pgfpathcurveto{\pgfqpoint{4.093427in}{1.688331in}}{\pgfqpoint{4.090593in}{1.681490in}}{\pgfqpoint{4.090593in}{1.674357in}}%
\pgfpathcurveto{\pgfqpoint{4.090593in}{1.667224in}}{\pgfqpoint{4.093427in}{1.660382in}}{\pgfqpoint{4.098471in}{1.655339in}}%
\pgfpathcurveto{\pgfqpoint{4.103514in}{1.650295in}}{\pgfqpoint{4.110356in}{1.647461in}}{\pgfqpoint{4.117489in}{1.647461in}}%
\pgfpathclose%
\pgfusepath{stroke,fill}%
\end{pgfscope}%
\begin{pgfscope}%
\pgfpathrectangle{\pgfqpoint{2.867647in}{0.500000in}}{\pgfqpoint{1.764706in}{1.700000in}}%
\pgfusepath{clip}%
\pgfsetbuttcap%
\pgfsetroundjoin%
\definecolor{currentfill}{rgb}{0.970718,0.821518,0.719872}%
\pgfsetfillcolor{currentfill}%
\pgfsetlinewidth{0.311001pt}%
\definecolor{currentstroke}{rgb}{1.000000,1.000000,1.000000}%
\pgfsetstrokecolor{currentstroke}%
\pgfsetdash{}{0pt}%
\pgfpathmoveto{\pgfqpoint{4.257085in}{1.172210in}}%
\pgfpathcurveto{\pgfqpoint{4.264218in}{1.172210in}}{\pgfqpoint{4.271059in}{1.175044in}}{\pgfqpoint{4.276103in}{1.180088in}}%
\pgfpathcurveto{\pgfqpoint{4.281147in}{1.185131in}}{\pgfqpoint{4.283980in}{1.191973in}}{\pgfqpoint{4.283980in}{1.199106in}}%
\pgfpathcurveto{\pgfqpoint{4.283980in}{1.206239in}}{\pgfqpoint{4.281147in}{1.213080in}}{\pgfqpoint{4.276103in}{1.218124in}}%
\pgfpathcurveto{\pgfqpoint{4.271059in}{1.223168in}}{\pgfqpoint{4.264218in}{1.226002in}}{\pgfqpoint{4.257085in}{1.226002in}}%
\pgfpathcurveto{\pgfqpoint{4.249952in}{1.226002in}}{\pgfqpoint{4.243110in}{1.223168in}}{\pgfqpoint{4.238067in}{1.218124in}}%
\pgfpathcurveto{\pgfqpoint{4.233023in}{1.213080in}}{\pgfqpoint{4.230189in}{1.206239in}}{\pgfqpoint{4.230189in}{1.199106in}}%
\pgfpathcurveto{\pgfqpoint{4.230189in}{1.191973in}}{\pgfqpoint{4.233023in}{1.185131in}}{\pgfqpoint{4.238067in}{1.180088in}}%
\pgfpathcurveto{\pgfqpoint{4.243110in}{1.175044in}}{\pgfqpoint{4.249952in}{1.172210in}}{\pgfqpoint{4.257085in}{1.172210in}}%
\pgfpathclose%
\pgfusepath{stroke,fill}%
\end{pgfscope}%
\begin{pgfscope}%
\pgfpathrectangle{\pgfqpoint{2.867647in}{0.500000in}}{\pgfqpoint{1.764706in}{1.700000in}}%
\pgfusepath{clip}%
\pgfsetbuttcap%
\pgfsetroundjoin%
\definecolor{currentfill}{rgb}{0.968509,0.792226,0.676405}%
\pgfsetfillcolor{currentfill}%
\pgfsetlinewidth{0.311001pt}%
\definecolor{currentstroke}{rgb}{1.000000,1.000000,1.000000}%
\pgfsetstrokecolor{currentstroke}%
\pgfsetdash{}{0pt}%
\pgfpathmoveto{\pgfqpoint{4.260680in}{1.490406in}}%
\pgfpathcurveto{\pgfqpoint{4.267813in}{1.490406in}}{\pgfqpoint{4.274655in}{1.493240in}}{\pgfqpoint{4.279698in}{1.498284in}}%
\pgfpathcurveto{\pgfqpoint{4.284742in}{1.503327in}}{\pgfqpoint{4.287576in}{1.510169in}}{\pgfqpoint{4.287576in}{1.517302in}}%
\pgfpathcurveto{\pgfqpoint{4.287576in}{1.524435in}}{\pgfqpoint{4.284742in}{1.531276in}}{\pgfqpoint{4.279698in}{1.536320in}}%
\pgfpathcurveto{\pgfqpoint{4.274655in}{1.541364in}}{\pgfqpoint{4.267813in}{1.544197in}}{\pgfqpoint{4.260680in}{1.544197in}}%
\pgfpathcurveto{\pgfqpoint{4.253547in}{1.544197in}}{\pgfqpoint{4.246706in}{1.541364in}}{\pgfqpoint{4.241662in}{1.536320in}}%
\pgfpathcurveto{\pgfqpoint{4.236618in}{1.531276in}}{\pgfqpoint{4.233784in}{1.524435in}}{\pgfqpoint{4.233784in}{1.517302in}}%
\pgfpathcurveto{\pgfqpoint{4.233784in}{1.510169in}}{\pgfqpoint{4.236618in}{1.503327in}}{\pgfqpoint{4.241662in}{1.498284in}}%
\pgfpathcurveto{\pgfqpoint{4.246706in}{1.493240in}}{\pgfqpoint{4.253547in}{1.490406in}}{\pgfqpoint{4.260680in}{1.490406in}}%
\pgfpathclose%
\pgfusepath{stroke,fill}%
\end{pgfscope}%
\begin{pgfscope}%
\pgfpathrectangle{\pgfqpoint{2.867647in}{0.500000in}}{\pgfqpoint{1.764706in}{1.700000in}}%
\pgfusepath{clip}%
\pgfsetbuttcap%
\pgfsetroundjoin%
\definecolor{currentfill}{rgb}{0.968931,0.798091,0.685123}%
\pgfsetfillcolor{currentfill}%
\pgfsetlinewidth{0.311001pt}%
\definecolor{currentstroke}{rgb}{1.000000,1.000000,1.000000}%
\pgfsetstrokecolor{currentstroke}%
\pgfsetdash{}{0pt}%
\pgfpathmoveto{\pgfqpoint{4.117794in}{1.701274in}}%
\pgfpathcurveto{\pgfqpoint{4.124926in}{1.701274in}}{\pgfqpoint{4.131768in}{1.704108in}}{\pgfqpoint{4.136812in}{1.709151in}}%
\pgfpathcurveto{\pgfqpoint{4.141855in}{1.714195in}}{\pgfqpoint{4.144689in}{1.721037in}}{\pgfqpoint{4.144689in}{1.728170in}}%
\pgfpathcurveto{\pgfqpoint{4.144689in}{1.735302in}}{\pgfqpoint{4.141855in}{1.742144in}}{\pgfqpoint{4.136812in}{1.747188in}}%
\pgfpathcurveto{\pgfqpoint{4.131768in}{1.752231in}}{\pgfqpoint{4.124926in}{1.755065in}}{\pgfqpoint{4.117794in}{1.755065in}}%
\pgfpathcurveto{\pgfqpoint{4.110661in}{1.755065in}}{\pgfqpoint{4.103819in}{1.752231in}}{\pgfqpoint{4.098775in}{1.747188in}}%
\pgfpathcurveto{\pgfqpoint{4.093732in}{1.742144in}}{\pgfqpoint{4.090898in}{1.735302in}}{\pgfqpoint{4.090898in}{1.728170in}}%
\pgfpathcurveto{\pgfqpoint{4.090898in}{1.721037in}}{\pgfqpoint{4.093732in}{1.714195in}}{\pgfqpoint{4.098775in}{1.709151in}}%
\pgfpathcurveto{\pgfqpoint{4.103819in}{1.704108in}}{\pgfqpoint{4.110661in}{1.701274in}}{\pgfqpoint{4.117794in}{1.701274in}}%
\pgfpathclose%
\pgfusepath{stroke,fill}%
\end{pgfscope}%
\begin{pgfscope}%
\pgfpathrectangle{\pgfqpoint{2.867647in}{0.500000in}}{\pgfqpoint{1.764706in}{1.700000in}}%
\pgfusepath{clip}%
\pgfsetbuttcap%
\pgfsetroundjoin%
\definecolor{currentfill}{rgb}{0.976287,0.879862,0.805788}%
\pgfsetfillcolor{currentfill}%
\pgfsetlinewidth{0.311001pt}%
\definecolor{currentstroke}{rgb}{1.000000,1.000000,1.000000}%
\pgfsetstrokecolor{currentstroke}%
\pgfsetdash{}{0pt}%
\pgfpathmoveto{\pgfqpoint{4.140822in}{1.339397in}}%
\pgfpathcurveto{\pgfqpoint{4.147955in}{1.339397in}}{\pgfqpoint{4.154796in}{1.342231in}}{\pgfqpoint{4.159840in}{1.347274in}}%
\pgfpathcurveto{\pgfqpoint{4.164884in}{1.352318in}}{\pgfqpoint{4.167718in}{1.359160in}}{\pgfqpoint{4.167718in}{1.366293in}}%
\pgfpathcurveto{\pgfqpoint{4.167718in}{1.373425in}}{\pgfqpoint{4.164884in}{1.380267in}}{\pgfqpoint{4.159840in}{1.385311in}}%
\pgfpathcurveto{\pgfqpoint{4.154796in}{1.390354in}}{\pgfqpoint{4.147955in}{1.393188in}}{\pgfqpoint{4.140822in}{1.393188in}}%
\pgfpathcurveto{\pgfqpoint{4.133689in}{1.393188in}}{\pgfqpoint{4.126847in}{1.390354in}}{\pgfqpoint{4.121804in}{1.385311in}}%
\pgfpathcurveto{\pgfqpoint{4.116760in}{1.380267in}}{\pgfqpoint{4.113926in}{1.373425in}}{\pgfqpoint{4.113926in}{1.366293in}}%
\pgfpathcurveto{\pgfqpoint{4.113926in}{1.359160in}}{\pgfqpoint{4.116760in}{1.352318in}}{\pgfqpoint{4.121804in}{1.347274in}}%
\pgfpathcurveto{\pgfqpoint{4.126847in}{1.342231in}}{\pgfqpoint{4.133689in}{1.339397in}}{\pgfqpoint{4.140822in}{1.339397in}}%
\pgfpathclose%
\pgfusepath{stroke,fill}%
\end{pgfscope}%
\begin{pgfscope}%
\pgfpathrectangle{\pgfqpoint{2.867647in}{0.500000in}}{\pgfqpoint{1.764706in}{1.700000in}}%
\pgfusepath{clip}%
\pgfsetbuttcap%
\pgfsetroundjoin%
\definecolor{currentfill}{rgb}{0.974412,0.862387,0.780156}%
\pgfsetfillcolor{currentfill}%
\pgfsetlinewidth{0.311001pt}%
\definecolor{currentstroke}{rgb}{1.000000,1.000000,1.000000}%
\pgfsetstrokecolor{currentstroke}%
\pgfsetdash{}{0pt}%
\pgfpathmoveto{\pgfqpoint{4.115740in}{1.437731in}}%
\pgfpathcurveto{\pgfqpoint{4.122873in}{1.437731in}}{\pgfqpoint{4.129714in}{1.440565in}}{\pgfqpoint{4.134758in}{1.445608in}}%
\pgfpathcurveto{\pgfqpoint{4.139802in}{1.450652in}}{\pgfqpoint{4.142636in}{1.457494in}}{\pgfqpoint{4.142636in}{1.464626in}}%
\pgfpathcurveto{\pgfqpoint{4.142636in}{1.471759in}}{\pgfqpoint{4.139802in}{1.478601in}}{\pgfqpoint{4.134758in}{1.483645in}}%
\pgfpathcurveto{\pgfqpoint{4.129714in}{1.488688in}}{\pgfqpoint{4.122873in}{1.491522in}}{\pgfqpoint{4.115740in}{1.491522in}}%
\pgfpathcurveto{\pgfqpoint{4.108607in}{1.491522in}}{\pgfqpoint{4.101765in}{1.488688in}}{\pgfqpoint{4.096722in}{1.483645in}}%
\pgfpathcurveto{\pgfqpoint{4.091678in}{1.478601in}}{\pgfqpoint{4.088844in}{1.471759in}}{\pgfqpoint{4.088844in}{1.464626in}}%
\pgfpathcurveto{\pgfqpoint{4.088844in}{1.457494in}}{\pgfqpoint{4.091678in}{1.450652in}}{\pgfqpoint{4.096722in}{1.445608in}}%
\pgfpathcurveto{\pgfqpoint{4.101765in}{1.440565in}}{\pgfqpoint{4.108607in}{1.437731in}}{\pgfqpoint{4.115740in}{1.437731in}}%
\pgfpathclose%
\pgfusepath{stroke,fill}%
\end{pgfscope}%
\begin{pgfscope}%
\pgfpathrectangle{\pgfqpoint{2.867647in}{0.500000in}}{\pgfqpoint{1.764706in}{1.700000in}}%
\pgfusepath{clip}%
\pgfsetbuttcap%
\pgfsetroundjoin%
\definecolor{currentfill}{rgb}{0.981377,0.920617,0.865369}%
\pgfsetfillcolor{currentfill}%
\pgfsetlinewidth{0.311001pt}%
\definecolor{currentstroke}{rgb}{1.000000,1.000000,1.000000}%
\pgfsetstrokecolor{currentstroke}%
\pgfsetdash{}{0pt}%
\pgfpathmoveto{\pgfqpoint{4.180538in}{1.357857in}}%
\pgfpathcurveto{\pgfqpoint{4.187671in}{1.357857in}}{\pgfqpoint{4.194513in}{1.360691in}}{\pgfqpoint{4.199557in}{1.365735in}}%
\pgfpathcurveto{\pgfqpoint{4.204600in}{1.370778in}}{\pgfqpoint{4.207434in}{1.377620in}}{\pgfqpoint{4.207434in}{1.384753in}}%
\pgfpathcurveto{\pgfqpoint{4.207434in}{1.391886in}}{\pgfqpoint{4.204600in}{1.398727in}}{\pgfqpoint{4.199557in}{1.403771in}}%
\pgfpathcurveto{\pgfqpoint{4.194513in}{1.408815in}}{\pgfqpoint{4.187671in}{1.411649in}}{\pgfqpoint{4.180538in}{1.411649in}}%
\pgfpathcurveto{\pgfqpoint{4.173406in}{1.411649in}}{\pgfqpoint{4.166564in}{1.408815in}}{\pgfqpoint{4.161520in}{1.403771in}}%
\pgfpathcurveto{\pgfqpoint{4.156477in}{1.398727in}}{\pgfqpoint{4.153643in}{1.391886in}}{\pgfqpoint{4.153643in}{1.384753in}}%
\pgfpathcurveto{\pgfqpoint{4.153643in}{1.377620in}}{\pgfqpoint{4.156477in}{1.370778in}}{\pgfqpoint{4.161520in}{1.365735in}}%
\pgfpathcurveto{\pgfqpoint{4.166564in}{1.360691in}}{\pgfqpoint{4.173406in}{1.357857in}}{\pgfqpoint{4.180538in}{1.357857in}}%
\pgfpathclose%
\pgfusepath{stroke,fill}%
\end{pgfscope}%
\begin{pgfscope}%
\pgfpathrectangle{\pgfqpoint{2.867647in}{0.500000in}}{\pgfqpoint{1.764706in}{1.700000in}}%
\pgfusepath{clip}%
\pgfsetbuttcap%
\pgfsetroundjoin%
\definecolor{currentfill}{rgb}{0.962985,0.612625,0.451451}%
\pgfsetfillcolor{currentfill}%
\pgfsetlinewidth{0.311001pt}%
\definecolor{currentstroke}{rgb}{1.000000,1.000000,1.000000}%
\pgfsetstrokecolor{currentstroke}%
\pgfsetdash{}{0pt}%
\pgfpathmoveto{\pgfqpoint{3.983658in}{0.870628in}}%
\pgfpathcurveto{\pgfqpoint{3.990791in}{0.870628in}}{\pgfqpoint{3.997633in}{0.873462in}}{\pgfqpoint{4.002676in}{0.878506in}}%
\pgfpathcurveto{\pgfqpoint{4.007720in}{0.883549in}}{\pgfqpoint{4.010554in}{0.890391in}}{\pgfqpoint{4.010554in}{0.897524in}}%
\pgfpathcurveto{\pgfqpoint{4.010554in}{0.904657in}}{\pgfqpoint{4.007720in}{0.911498in}}{\pgfqpoint{4.002676in}{0.916542in}}%
\pgfpathcurveto{\pgfqpoint{3.997633in}{0.921585in}}{\pgfqpoint{3.990791in}{0.924419in}}{\pgfqpoint{3.983658in}{0.924419in}}%
\pgfpathcurveto{\pgfqpoint{3.976525in}{0.924419in}}{\pgfqpoint{3.969684in}{0.921585in}}{\pgfqpoint{3.964640in}{0.916542in}}%
\pgfpathcurveto{\pgfqpoint{3.959596in}{0.911498in}}{\pgfqpoint{3.956762in}{0.904657in}}{\pgfqpoint{3.956762in}{0.897524in}}%
\pgfpathcurveto{\pgfqpoint{3.956762in}{0.890391in}}{\pgfqpoint{3.959596in}{0.883549in}}{\pgfqpoint{3.964640in}{0.878506in}}%
\pgfpathcurveto{\pgfqpoint{3.969684in}{0.873462in}}{\pgfqpoint{3.976525in}{0.870628in}}{\pgfqpoint{3.983658in}{0.870628in}}%
\pgfpathclose%
\pgfusepath{stroke,fill}%
\end{pgfscope}%
\begin{pgfscope}%
\pgfpathrectangle{\pgfqpoint{2.867647in}{0.500000in}}{\pgfqpoint{1.764706in}{1.700000in}}%
\pgfusepath{clip}%
\pgfsetbuttcap%
\pgfsetroundjoin%
\definecolor{currentfill}{rgb}{0.975018,0.868213,0.788710}%
\pgfsetfillcolor{currentfill}%
\pgfsetlinewidth{0.311001pt}%
\definecolor{currentstroke}{rgb}{1.000000,1.000000,1.000000}%
\pgfsetstrokecolor{currentstroke}%
\pgfsetdash{}{0pt}%
\pgfpathmoveto{\pgfqpoint{4.228238in}{1.471087in}}%
\pgfpathcurveto{\pgfqpoint{4.235371in}{1.471087in}}{\pgfqpoint{4.242213in}{1.473921in}}{\pgfqpoint{4.247256in}{1.478965in}}%
\pgfpathcurveto{\pgfqpoint{4.252300in}{1.484008in}}{\pgfqpoint{4.255134in}{1.490850in}}{\pgfqpoint{4.255134in}{1.497983in}}%
\pgfpathcurveto{\pgfqpoint{4.255134in}{1.505116in}}{\pgfqpoint{4.252300in}{1.511957in}}{\pgfqpoint{4.247256in}{1.517001in}}%
\pgfpathcurveto{\pgfqpoint{4.242213in}{1.522045in}}{\pgfqpoint{4.235371in}{1.524879in}}{\pgfqpoint{4.228238in}{1.524879in}}%
\pgfpathcurveto{\pgfqpoint{4.221105in}{1.524879in}}{\pgfqpoint{4.214264in}{1.522045in}}{\pgfqpoint{4.209220in}{1.517001in}}%
\pgfpathcurveto{\pgfqpoint{4.204176in}{1.511957in}}{\pgfqpoint{4.201343in}{1.505116in}}{\pgfqpoint{4.201343in}{1.497983in}}%
\pgfpathcurveto{\pgfqpoint{4.201343in}{1.490850in}}{\pgfqpoint{4.204176in}{1.484008in}}{\pgfqpoint{4.209220in}{1.478965in}}%
\pgfpathcurveto{\pgfqpoint{4.214264in}{1.473921in}}{\pgfqpoint{4.221105in}{1.471087in}}{\pgfqpoint{4.228238in}{1.471087in}}%
\pgfpathclose%
\pgfusepath{stroke,fill}%
\end{pgfscope}%
\begin{pgfscope}%
\pgfpathrectangle{\pgfqpoint{2.867647in}{0.500000in}}{\pgfqpoint{1.764706in}{1.700000in}}%
\pgfusepath{clip}%
\pgfsetbuttcap%
\pgfsetroundjoin%
\definecolor{currentfill}{rgb}{0.970718,0.821518,0.719872}%
\pgfsetfillcolor{currentfill}%
\pgfsetlinewidth{0.311001pt}%
\definecolor{currentstroke}{rgb}{1.000000,1.000000,1.000000}%
\pgfsetstrokecolor{currentstroke}%
\pgfsetdash{}{0pt}%
\pgfpathmoveto{\pgfqpoint{4.122107in}{0.964969in}}%
\pgfpathcurveto{\pgfqpoint{4.129240in}{0.964969in}}{\pgfqpoint{4.136082in}{0.967803in}}{\pgfqpoint{4.141125in}{0.972846in}}%
\pgfpathcurveto{\pgfqpoint{4.146169in}{0.977890in}}{\pgfqpoint{4.149003in}{0.984732in}}{\pgfqpoint{4.149003in}{0.991864in}}%
\pgfpathcurveto{\pgfqpoint{4.149003in}{0.998997in}}{\pgfqpoint{4.146169in}{1.005839in}}{\pgfqpoint{4.141125in}{1.010883in}}%
\pgfpathcurveto{\pgfqpoint{4.136082in}{1.015926in}}{\pgfqpoint{4.129240in}{1.018760in}}{\pgfqpoint{4.122107in}{1.018760in}}%
\pgfpathcurveto{\pgfqpoint{4.114974in}{1.018760in}}{\pgfqpoint{4.108133in}{1.015926in}}{\pgfqpoint{4.103089in}{1.010883in}}%
\pgfpathcurveto{\pgfqpoint{4.098045in}{1.005839in}}{\pgfqpoint{4.095212in}{0.998997in}}{\pgfqpoint{4.095212in}{0.991864in}}%
\pgfpathcurveto{\pgfqpoint{4.095212in}{0.984732in}}{\pgfqpoint{4.098045in}{0.977890in}}{\pgfqpoint{4.103089in}{0.972846in}}%
\pgfpathcurveto{\pgfqpoint{4.108133in}{0.967803in}}{\pgfqpoint{4.114974in}{0.964969in}}{\pgfqpoint{4.122107in}{0.964969in}}%
\pgfpathclose%
\pgfusepath{stroke,fill}%
\end{pgfscope}%
\begin{pgfscope}%
\pgfpathrectangle{\pgfqpoint{2.867647in}{0.500000in}}{\pgfqpoint{1.764706in}{1.700000in}}%
\pgfusepath{clip}%
\pgfsetbuttcap%
\pgfsetroundjoin%
\definecolor{currentfill}{rgb}{0.977657,0.891500,0.822809}%
\pgfsetfillcolor{currentfill}%
\pgfsetlinewidth{0.311001pt}%
\definecolor{currentstroke}{rgb}{1.000000,1.000000,1.000000}%
\pgfsetstrokecolor{currentstroke}%
\pgfsetdash{}{0pt}%
\pgfpathmoveto{\pgfqpoint{4.125397in}{1.486177in}}%
\pgfpathcurveto{\pgfqpoint{4.132530in}{1.486177in}}{\pgfqpoint{4.139372in}{1.489010in}}{\pgfqpoint{4.144415in}{1.494054in}}%
\pgfpathcurveto{\pgfqpoint{4.149459in}{1.499098in}}{\pgfqpoint{4.152293in}{1.505939in}}{\pgfqpoint{4.152293in}{1.513072in}}%
\pgfpathcurveto{\pgfqpoint{4.152293in}{1.520205in}}{\pgfqpoint{4.149459in}{1.527047in}}{\pgfqpoint{4.144415in}{1.532090in}}%
\pgfpathcurveto{\pgfqpoint{4.139372in}{1.537134in}}{\pgfqpoint{4.132530in}{1.539968in}}{\pgfqpoint{4.125397in}{1.539968in}}%
\pgfpathcurveto{\pgfqpoint{4.118264in}{1.539968in}}{\pgfqpoint{4.111423in}{1.537134in}}{\pgfqpoint{4.106379in}{1.532090in}}%
\pgfpathcurveto{\pgfqpoint{4.101336in}{1.527047in}}{\pgfqpoint{4.098502in}{1.520205in}}{\pgfqpoint{4.098502in}{1.513072in}}%
\pgfpathcurveto{\pgfqpoint{4.098502in}{1.505939in}}{\pgfqpoint{4.101336in}{1.499098in}}{\pgfqpoint{4.106379in}{1.494054in}}%
\pgfpathcurveto{\pgfqpoint{4.111423in}{1.489010in}}{\pgfqpoint{4.118264in}{1.486177in}}{\pgfqpoint{4.125397in}{1.486177in}}%
\pgfpathclose%
\pgfusepath{stroke,fill}%
\end{pgfscope}%
\begin{pgfscope}%
\pgfpathrectangle{\pgfqpoint{2.867647in}{0.500000in}}{\pgfqpoint{1.764706in}{1.700000in}}%
\pgfusepath{clip}%
\pgfsetbuttcap%
\pgfsetroundjoin%
\definecolor{currentfill}{rgb}{0.977657,0.891500,0.822809}%
\pgfsetfillcolor{currentfill}%
\pgfsetlinewidth{0.311001pt}%
\definecolor{currentstroke}{rgb}{1.000000,1.000000,1.000000}%
\pgfsetstrokecolor{currentstroke}%
\pgfsetdash{}{0pt}%
\pgfpathmoveto{\pgfqpoint{4.105753in}{1.586282in}}%
\pgfpathcurveto{\pgfqpoint{4.112886in}{1.586282in}}{\pgfqpoint{4.119727in}{1.589116in}}{\pgfqpoint{4.124771in}{1.594160in}}%
\pgfpathcurveto{\pgfqpoint{4.129815in}{1.599203in}}{\pgfqpoint{4.132649in}{1.606045in}}{\pgfqpoint{4.132649in}{1.613178in}}%
\pgfpathcurveto{\pgfqpoint{4.132649in}{1.620311in}}{\pgfqpoint{4.129815in}{1.627152in}}{\pgfqpoint{4.124771in}{1.632196in}}%
\pgfpathcurveto{\pgfqpoint{4.119727in}{1.637240in}}{\pgfqpoint{4.112886in}{1.640073in}}{\pgfqpoint{4.105753in}{1.640073in}}%
\pgfpathcurveto{\pgfqpoint{4.098620in}{1.640073in}}{\pgfqpoint{4.091779in}{1.637240in}}{\pgfqpoint{4.086735in}{1.632196in}}%
\pgfpathcurveto{\pgfqpoint{4.081691in}{1.627152in}}{\pgfqpoint{4.078857in}{1.620311in}}{\pgfqpoint{4.078857in}{1.613178in}}%
\pgfpathcurveto{\pgfqpoint{4.078857in}{1.606045in}}{\pgfqpoint{4.081691in}{1.599203in}}{\pgfqpoint{4.086735in}{1.594160in}}%
\pgfpathcurveto{\pgfqpoint{4.091779in}{1.589116in}}{\pgfqpoint{4.098620in}{1.586282in}}{\pgfqpoint{4.105753in}{1.586282in}}%
\pgfpathclose%
\pgfusepath{stroke,fill}%
\end{pgfscope}%
\begin{pgfscope}%
\pgfpathrectangle{\pgfqpoint{2.867647in}{0.500000in}}{\pgfqpoint{1.764706in}{1.700000in}}%
\pgfusepath{clip}%
\pgfsetbuttcap%
\pgfsetroundjoin%
\definecolor{currentfill}{rgb}{0.976287,0.879862,0.805788}%
\pgfsetfillcolor{currentfill}%
\pgfsetlinewidth{0.311001pt}%
\definecolor{currentstroke}{rgb}{1.000000,1.000000,1.000000}%
\pgfsetstrokecolor{currentstroke}%
\pgfsetdash{}{0pt}%
\pgfpathmoveto{\pgfqpoint{4.136919in}{1.389781in}}%
\pgfpathcurveto{\pgfqpoint{4.144052in}{1.389781in}}{\pgfqpoint{4.150894in}{1.392614in}}{\pgfqpoint{4.155937in}{1.397658in}}%
\pgfpathcurveto{\pgfqpoint{4.160981in}{1.402702in}}{\pgfqpoint{4.163815in}{1.409543in}}{\pgfqpoint{4.163815in}{1.416676in}}%
\pgfpathcurveto{\pgfqpoint{4.163815in}{1.423809in}}{\pgfqpoint{4.160981in}{1.430651in}}{\pgfqpoint{4.155937in}{1.435694in}}%
\pgfpathcurveto{\pgfqpoint{4.150894in}{1.440738in}}{\pgfqpoint{4.144052in}{1.443572in}}{\pgfqpoint{4.136919in}{1.443572in}}%
\pgfpathcurveto{\pgfqpoint{4.129786in}{1.443572in}}{\pgfqpoint{4.122945in}{1.440738in}}{\pgfqpoint{4.117901in}{1.435694in}}%
\pgfpathcurveto{\pgfqpoint{4.112857in}{1.430651in}}{\pgfqpoint{4.110024in}{1.423809in}}{\pgfqpoint{4.110024in}{1.416676in}}%
\pgfpathcurveto{\pgfqpoint{4.110024in}{1.409543in}}{\pgfqpoint{4.112857in}{1.402702in}}{\pgfqpoint{4.117901in}{1.397658in}}%
\pgfpathcurveto{\pgfqpoint{4.122945in}{1.392614in}}{\pgfqpoint{4.129786in}{1.389781in}}{\pgfqpoint{4.136919in}{1.389781in}}%
\pgfpathclose%
\pgfusepath{stroke,fill}%
\end{pgfscope}%
\begin{pgfscope}%
\pgfpathrectangle{\pgfqpoint{2.867647in}{0.500000in}}{\pgfqpoint{1.764706in}{1.700000in}}%
\pgfusepath{clip}%
\pgfsetbuttcap%
\pgfsetroundjoin%
\definecolor{currentfill}{rgb}{0.976287,0.879862,0.805788}%
\pgfsetfillcolor{currentfill}%
\pgfsetlinewidth{0.311001pt}%
\definecolor{currentstroke}{rgb}{1.000000,1.000000,1.000000}%
\pgfsetstrokecolor{currentstroke}%
\pgfsetdash{}{0pt}%
\pgfpathmoveto{\pgfqpoint{4.201232in}{1.532183in}}%
\pgfpathcurveto{\pgfqpoint{4.208365in}{1.532183in}}{\pgfqpoint{4.215207in}{1.535017in}}{\pgfqpoint{4.220250in}{1.540060in}}%
\pgfpathcurveto{\pgfqpoint{4.225294in}{1.545104in}}{\pgfqpoint{4.228128in}{1.551946in}}{\pgfqpoint{4.228128in}{1.559078in}}%
\pgfpathcurveto{\pgfqpoint{4.228128in}{1.566211in}}{\pgfqpoint{4.225294in}{1.573053in}}{\pgfqpoint{4.220250in}{1.578097in}}%
\pgfpathcurveto{\pgfqpoint{4.215207in}{1.583140in}}{\pgfqpoint{4.208365in}{1.585974in}}{\pgfqpoint{4.201232in}{1.585974in}}%
\pgfpathcurveto{\pgfqpoint{4.194099in}{1.585974in}}{\pgfqpoint{4.187258in}{1.583140in}}{\pgfqpoint{4.182214in}{1.578097in}}%
\pgfpathcurveto{\pgfqpoint{4.177170in}{1.573053in}}{\pgfqpoint{4.174336in}{1.566211in}}{\pgfqpoint{4.174336in}{1.559078in}}%
\pgfpathcurveto{\pgfqpoint{4.174336in}{1.551946in}}{\pgfqpoint{4.177170in}{1.545104in}}{\pgfqpoint{4.182214in}{1.540060in}}%
\pgfpathcurveto{\pgfqpoint{4.187258in}{1.535017in}}{\pgfqpoint{4.194099in}{1.532183in}}{\pgfqpoint{4.201232in}{1.532183in}}%
\pgfpathclose%
\pgfusepath{stroke,fill}%
\end{pgfscope}%
\begin{pgfscope}%
\pgfpathrectangle{\pgfqpoint{2.867647in}{0.500000in}}{\pgfqpoint{1.764706in}{1.700000in}}%
\pgfusepath{clip}%
\pgfsetbuttcap%
\pgfsetroundjoin%
\definecolor{currentfill}{rgb}{0.961433,0.573272,0.412036}%
\pgfsetfillcolor{currentfill}%
\pgfsetlinewidth{0.311001pt}%
\definecolor{currentstroke}{rgb}{1.000000,1.000000,1.000000}%
\pgfsetstrokecolor{currentstroke}%
\pgfsetdash{}{0pt}%
\pgfpathmoveto{\pgfqpoint{4.155944in}{1.762476in}}%
\pgfpathcurveto{\pgfqpoint{4.163077in}{1.762476in}}{\pgfqpoint{4.169918in}{1.765310in}}{\pgfqpoint{4.174962in}{1.770354in}}%
\pgfpathcurveto{\pgfqpoint{4.180006in}{1.775398in}}{\pgfqpoint{4.182840in}{1.782239in}}{\pgfqpoint{4.182840in}{1.789372in}}%
\pgfpathcurveto{\pgfqpoint{4.182840in}{1.796505in}}{\pgfqpoint{4.180006in}{1.803347in}}{\pgfqpoint{4.174962in}{1.808390in}}%
\pgfpathcurveto{\pgfqpoint{4.169918in}{1.813434in}}{\pgfqpoint{4.163077in}{1.816268in}}{\pgfqpoint{4.155944in}{1.816268in}}%
\pgfpathcurveto{\pgfqpoint{4.148811in}{1.816268in}}{\pgfqpoint{4.141969in}{1.813434in}}{\pgfqpoint{4.136926in}{1.808390in}}%
\pgfpathcurveto{\pgfqpoint{4.131882in}{1.803347in}}{\pgfqpoint{4.129048in}{1.796505in}}{\pgfqpoint{4.129048in}{1.789372in}}%
\pgfpathcurveto{\pgfqpoint{4.129048in}{1.782239in}}{\pgfqpoint{4.131882in}{1.775398in}}{\pgfqpoint{4.136926in}{1.770354in}}%
\pgfpathcurveto{\pgfqpoint{4.141969in}{1.765310in}}{\pgfqpoint{4.148811in}{1.762476in}}{\pgfqpoint{4.155944in}{1.762476in}}%
\pgfpathclose%
\pgfusepath{stroke,fill}%
\end{pgfscope}%
\begin{pgfscope}%
\pgfpathrectangle{\pgfqpoint{2.867647in}{0.500000in}}{\pgfqpoint{1.764706in}{1.700000in}}%
\pgfusepath{clip}%
\pgfsetbuttcap%
\pgfsetroundjoin%
\definecolor{currentfill}{rgb}{0.957344,0.505732,0.351309}%
\pgfsetfillcolor{currentfill}%
\pgfsetlinewidth{0.311001pt}%
\definecolor{currentstroke}{rgb}{1.000000,1.000000,1.000000}%
\pgfsetstrokecolor{currentstroke}%
\pgfsetdash{}{0pt}%
\pgfpathmoveto{\pgfqpoint{4.336845in}{1.176773in}}%
\pgfpathcurveto{\pgfqpoint{4.343977in}{1.176773in}}{\pgfqpoint{4.350819in}{1.179607in}}{\pgfqpoint{4.355863in}{1.184651in}}%
\pgfpathcurveto{\pgfqpoint{4.360906in}{1.189695in}}{\pgfqpoint{4.363740in}{1.196536in}}{\pgfqpoint{4.363740in}{1.203669in}}%
\pgfpathcurveto{\pgfqpoint{4.363740in}{1.210802in}}{\pgfqpoint{4.360906in}{1.217643in}}{\pgfqpoint{4.355863in}{1.222687in}}%
\pgfpathcurveto{\pgfqpoint{4.350819in}{1.227731in}}{\pgfqpoint{4.343977in}{1.230565in}}{\pgfqpoint{4.336845in}{1.230565in}}%
\pgfpathcurveto{\pgfqpoint{4.329712in}{1.230565in}}{\pgfqpoint{4.322870in}{1.227731in}}{\pgfqpoint{4.317826in}{1.222687in}}%
\pgfpathcurveto{\pgfqpoint{4.312783in}{1.217643in}}{\pgfqpoint{4.309949in}{1.210802in}}{\pgfqpoint{4.309949in}{1.203669in}}%
\pgfpathcurveto{\pgfqpoint{4.309949in}{1.196536in}}{\pgfqpoint{4.312783in}{1.189695in}}{\pgfqpoint{4.317826in}{1.184651in}}%
\pgfpathcurveto{\pgfqpoint{4.322870in}{1.179607in}}{\pgfqpoint{4.329712in}{1.176773in}}{\pgfqpoint{4.336845in}{1.176773in}}%
\pgfpathclose%
\pgfusepath{stroke,fill}%
\end{pgfscope}%
\begin{pgfscope}%
\pgfpathrectangle{\pgfqpoint{2.867647in}{0.500000in}}{\pgfqpoint{1.764706in}{1.700000in}}%
\pgfusepath{clip}%
\pgfsetbuttcap%
\pgfsetroundjoin%
\definecolor{currentfill}{rgb}{0.966560,0.756582,0.625273}%
\pgfsetfillcolor{currentfill}%
\pgfsetlinewidth{0.311001pt}%
\definecolor{currentstroke}{rgb}{1.000000,1.000000,1.000000}%
\pgfsetstrokecolor{currentstroke}%
\pgfsetdash{}{0pt}%
\pgfpathmoveto{\pgfqpoint{4.118377in}{1.722733in}}%
\pgfpathcurveto{\pgfqpoint{4.125510in}{1.722733in}}{\pgfqpoint{4.132352in}{1.725567in}}{\pgfqpoint{4.137395in}{1.730610in}}%
\pgfpathcurveto{\pgfqpoint{4.142439in}{1.735654in}}{\pgfqpoint{4.145273in}{1.742496in}}{\pgfqpoint{4.145273in}{1.749628in}}%
\pgfpathcurveto{\pgfqpoint{4.145273in}{1.756761in}}{\pgfqpoint{4.142439in}{1.763603in}}{\pgfqpoint{4.137395in}{1.768646in}}%
\pgfpathcurveto{\pgfqpoint{4.132352in}{1.773690in}}{\pgfqpoint{4.125510in}{1.776524in}}{\pgfqpoint{4.118377in}{1.776524in}}%
\pgfpathcurveto{\pgfqpoint{4.111244in}{1.776524in}}{\pgfqpoint{4.104403in}{1.773690in}}{\pgfqpoint{4.099359in}{1.768646in}}%
\pgfpathcurveto{\pgfqpoint{4.094315in}{1.763603in}}{\pgfqpoint{4.091481in}{1.756761in}}{\pgfqpoint{4.091481in}{1.749628in}}%
\pgfpathcurveto{\pgfqpoint{4.091481in}{1.742496in}}{\pgfqpoint{4.094315in}{1.735654in}}{\pgfqpoint{4.099359in}{1.730610in}}%
\pgfpathcurveto{\pgfqpoint{4.104403in}{1.725567in}}{\pgfqpoint{4.111244in}{1.722733in}}{\pgfqpoint{4.118377in}{1.722733in}}%
\pgfpathclose%
\pgfusepath{stroke,fill}%
\end{pgfscope}%
\begin{pgfscope}%
\pgfpathrectangle{\pgfqpoint{2.867647in}{0.500000in}}{\pgfqpoint{1.764706in}{1.700000in}}%
\pgfusepath{clip}%
\pgfsetbuttcap%
\pgfsetroundjoin%
\definecolor{currentfill}{rgb}{0.976961,0.885681,0.814303}%
\pgfsetfillcolor{currentfill}%
\pgfsetlinewidth{0.311001pt}%
\definecolor{currentstroke}{rgb}{1.000000,1.000000,1.000000}%
\pgfsetstrokecolor{currentstroke}%
\pgfsetdash{}{0pt}%
\pgfpathmoveto{\pgfqpoint{4.216046in}{1.469806in}}%
\pgfpathcurveto{\pgfqpoint{4.223179in}{1.469806in}}{\pgfqpoint{4.230020in}{1.472640in}}{\pgfqpoint{4.235064in}{1.477683in}}%
\pgfpathcurveto{\pgfqpoint{4.240108in}{1.482727in}}{\pgfqpoint{4.242942in}{1.489569in}}{\pgfqpoint{4.242942in}{1.496702in}}%
\pgfpathcurveto{\pgfqpoint{4.242942in}{1.503834in}}{\pgfqpoint{4.240108in}{1.510676in}}{\pgfqpoint{4.235064in}{1.515720in}}%
\pgfpathcurveto{\pgfqpoint{4.230020in}{1.520763in}}{\pgfqpoint{4.223179in}{1.523597in}}{\pgfqpoint{4.216046in}{1.523597in}}%
\pgfpathcurveto{\pgfqpoint{4.208913in}{1.523597in}}{\pgfqpoint{4.202071in}{1.520763in}}{\pgfqpoint{4.197028in}{1.515720in}}%
\pgfpathcurveto{\pgfqpoint{4.191984in}{1.510676in}}{\pgfqpoint{4.189150in}{1.503834in}}{\pgfqpoint{4.189150in}{1.496702in}}%
\pgfpathcurveto{\pgfqpoint{4.189150in}{1.489569in}}{\pgfqpoint{4.191984in}{1.482727in}}{\pgfqpoint{4.197028in}{1.477683in}}%
\pgfpathcurveto{\pgfqpoint{4.202071in}{1.472640in}}{\pgfqpoint{4.208913in}{1.469806in}}{\pgfqpoint{4.216046in}{1.469806in}}%
\pgfpathclose%
\pgfusepath{stroke,fill}%
\end{pgfscope}%
\begin{pgfscope}%
\pgfpathrectangle{\pgfqpoint{2.867647in}{0.500000in}}{\pgfqpoint{1.764706in}{1.700000in}}%
\pgfusepath{clip}%
\pgfsetbuttcap%
\pgfsetroundjoin%
\definecolor{currentfill}{rgb}{0.965753,0.732351,0.592427}%
\pgfsetfillcolor{currentfill}%
\pgfsetlinewidth{0.311001pt}%
\definecolor{currentstroke}{rgb}{1.000000,1.000000,1.000000}%
\pgfsetstrokecolor{currentstroke}%
\pgfsetdash{}{0pt}%
\pgfpathmoveto{\pgfqpoint{4.047241in}{0.901372in}}%
\pgfpathcurveto{\pgfqpoint{4.054374in}{0.901372in}}{\pgfqpoint{4.061216in}{0.904206in}}{\pgfqpoint{4.066259in}{0.909250in}}%
\pgfpathcurveto{\pgfqpoint{4.071303in}{0.914293in}}{\pgfqpoint{4.074137in}{0.921135in}}{\pgfqpoint{4.074137in}{0.928268in}}%
\pgfpathcurveto{\pgfqpoint{4.074137in}{0.935400in}}{\pgfqpoint{4.071303in}{0.942242in}}{\pgfqpoint{4.066259in}{0.947286in}}%
\pgfpathcurveto{\pgfqpoint{4.061216in}{0.952329in}}{\pgfqpoint{4.054374in}{0.955163in}}{\pgfqpoint{4.047241in}{0.955163in}}%
\pgfpathcurveto{\pgfqpoint{4.040108in}{0.955163in}}{\pgfqpoint{4.033267in}{0.952329in}}{\pgfqpoint{4.028223in}{0.947286in}}%
\pgfpathcurveto{\pgfqpoint{4.023179in}{0.942242in}}{\pgfqpoint{4.020345in}{0.935400in}}{\pgfqpoint{4.020345in}{0.928268in}}%
\pgfpathcurveto{\pgfqpoint{4.020345in}{0.921135in}}{\pgfqpoint{4.023179in}{0.914293in}}{\pgfqpoint{4.028223in}{0.909250in}}%
\pgfpathcurveto{\pgfqpoint{4.033267in}{0.904206in}}{\pgfqpoint{4.040108in}{0.901372in}}{\pgfqpoint{4.047241in}{0.901372in}}%
\pgfpathclose%
\pgfusepath{stroke,fill}%
\end{pgfscope}%
\begin{pgfscope}%
\pgfpathrectangle{\pgfqpoint{2.867647in}{0.500000in}}{\pgfqpoint{1.764706in}{1.700000in}}%
\pgfusepath{clip}%
\pgfsetbuttcap%
\pgfsetroundjoin%
\definecolor{currentfill}{rgb}{0.976961,0.885681,0.814303}%
\pgfsetfillcolor{currentfill}%
\pgfsetlinewidth{0.311001pt}%
\definecolor{currentstroke}{rgb}{1.000000,1.000000,1.000000}%
\pgfsetstrokecolor{currentstroke}%
\pgfsetdash{}{0pt}%
\pgfpathmoveto{\pgfqpoint{4.234635in}{1.242154in}}%
\pgfpathcurveto{\pgfqpoint{4.241768in}{1.242154in}}{\pgfqpoint{4.248609in}{1.244988in}}{\pgfqpoint{4.253653in}{1.250031in}}%
\pgfpathcurveto{\pgfqpoint{4.258697in}{1.255075in}}{\pgfqpoint{4.261531in}{1.261917in}}{\pgfqpoint{4.261531in}{1.269050in}}%
\pgfpathcurveto{\pgfqpoint{4.261531in}{1.276182in}}{\pgfqpoint{4.258697in}{1.283024in}}{\pgfqpoint{4.253653in}{1.288068in}}%
\pgfpathcurveto{\pgfqpoint{4.248609in}{1.293111in}}{\pgfqpoint{4.241768in}{1.295945in}}{\pgfqpoint{4.234635in}{1.295945in}}%
\pgfpathcurveto{\pgfqpoint{4.227502in}{1.295945in}}{\pgfqpoint{4.220660in}{1.293111in}}{\pgfqpoint{4.215617in}{1.288068in}}%
\pgfpathcurveto{\pgfqpoint{4.210573in}{1.283024in}}{\pgfqpoint{4.207739in}{1.276182in}}{\pgfqpoint{4.207739in}{1.269050in}}%
\pgfpathcurveto{\pgfqpoint{4.207739in}{1.261917in}}{\pgfqpoint{4.210573in}{1.255075in}}{\pgfqpoint{4.215617in}{1.250031in}}%
\pgfpathcurveto{\pgfqpoint{4.220660in}{1.244988in}}{\pgfqpoint{4.227502in}{1.242154in}}{\pgfqpoint{4.234635in}{1.242154in}}%
\pgfpathclose%
\pgfusepath{stroke,fill}%
\end{pgfscope}%
\begin{pgfscope}%
\pgfpathrectangle{\pgfqpoint{2.867647in}{0.500000in}}{\pgfqpoint{1.764706in}{1.700000in}}%
\pgfusepath{clip}%
\pgfsetbuttcap%
\pgfsetroundjoin%
\definecolor{currentfill}{rgb}{0.967735,0.780441,0.659127}%
\pgfsetfillcolor{currentfill}%
\pgfsetlinewidth{0.311001pt}%
\definecolor{currentstroke}{rgb}{1.000000,1.000000,1.000000}%
\pgfsetstrokecolor{currentstroke}%
\pgfsetdash{}{0pt}%
\pgfpathmoveto{\pgfqpoint{4.062803in}{0.927541in}}%
\pgfpathcurveto{\pgfqpoint{4.069936in}{0.927541in}}{\pgfqpoint{4.076777in}{0.930375in}}{\pgfqpoint{4.081821in}{0.935418in}}%
\pgfpathcurveto{\pgfqpoint{4.086865in}{0.940462in}}{\pgfqpoint{4.089699in}{0.947304in}}{\pgfqpoint{4.089699in}{0.954437in}}%
\pgfpathcurveto{\pgfqpoint{4.089699in}{0.961569in}}{\pgfqpoint{4.086865in}{0.968411in}}{\pgfqpoint{4.081821in}{0.973455in}}%
\pgfpathcurveto{\pgfqpoint{4.076777in}{0.978498in}}{\pgfqpoint{4.069936in}{0.981332in}}{\pgfqpoint{4.062803in}{0.981332in}}%
\pgfpathcurveto{\pgfqpoint{4.055670in}{0.981332in}}{\pgfqpoint{4.048828in}{0.978498in}}{\pgfqpoint{4.043785in}{0.973455in}}%
\pgfpathcurveto{\pgfqpoint{4.038741in}{0.968411in}}{\pgfqpoint{4.035907in}{0.961569in}}{\pgfqpoint{4.035907in}{0.954437in}}%
\pgfpathcurveto{\pgfqpoint{4.035907in}{0.947304in}}{\pgfqpoint{4.038741in}{0.940462in}}{\pgfqpoint{4.043785in}{0.935418in}}%
\pgfpathcurveto{\pgfqpoint{4.048828in}{0.930375in}}{\pgfqpoint{4.055670in}{0.927541in}}{\pgfqpoint{4.062803in}{0.927541in}}%
\pgfpathclose%
\pgfusepath{stroke,fill}%
\end{pgfscope}%
\begin{pgfscope}%
\pgfpathrectangle{\pgfqpoint{2.867647in}{0.500000in}}{\pgfqpoint{1.764706in}{1.700000in}}%
\pgfusepath{clip}%
\pgfsetbuttcap%
\pgfsetroundjoin%
\definecolor{currentfill}{rgb}{0.973271,0.850724,0.762998}%
\pgfsetfillcolor{currentfill}%
\pgfsetlinewidth{0.311001pt}%
\definecolor{currentstroke}{rgb}{1.000000,1.000000,1.000000}%
\pgfsetstrokecolor{currentstroke}%
\pgfsetdash{}{0pt}%
\pgfpathmoveto{\pgfqpoint{4.120048in}{1.379030in}}%
\pgfpathcurveto{\pgfqpoint{4.127181in}{1.379030in}}{\pgfqpoint{4.134023in}{1.381864in}}{\pgfqpoint{4.139066in}{1.386908in}}%
\pgfpathcurveto{\pgfqpoint{4.144110in}{1.391952in}}{\pgfqpoint{4.146944in}{1.398793in}}{\pgfqpoint{4.146944in}{1.405926in}}%
\pgfpathcurveto{\pgfqpoint{4.146944in}{1.413059in}}{\pgfqpoint{4.144110in}{1.419901in}}{\pgfqpoint{4.139066in}{1.424944in}}%
\pgfpathcurveto{\pgfqpoint{4.134023in}{1.429988in}}{\pgfqpoint{4.127181in}{1.432822in}}{\pgfqpoint{4.120048in}{1.432822in}}%
\pgfpathcurveto{\pgfqpoint{4.112915in}{1.432822in}}{\pgfqpoint{4.106074in}{1.429988in}}{\pgfqpoint{4.101030in}{1.424944in}}%
\pgfpathcurveto{\pgfqpoint{4.095986in}{1.419901in}}{\pgfqpoint{4.093153in}{1.413059in}}{\pgfqpoint{4.093153in}{1.405926in}}%
\pgfpathcurveto{\pgfqpoint{4.093153in}{1.398793in}}{\pgfqpoint{4.095986in}{1.391952in}}{\pgfqpoint{4.101030in}{1.386908in}}%
\pgfpathcurveto{\pgfqpoint{4.106074in}{1.381864in}}{\pgfqpoint{4.112915in}{1.379030in}}{\pgfqpoint{4.120048in}{1.379030in}}%
\pgfpathclose%
\pgfusepath{stroke,fill}%
\end{pgfscope}%
\begin{pgfscope}%
\pgfpathrectangle{\pgfqpoint{2.867647in}{0.500000in}}{\pgfqpoint{1.764706in}{1.700000in}}%
\pgfusepath{clip}%
\pgfsetbuttcap%
\pgfsetroundjoin%
\definecolor{currentfill}{rgb}{0.969359,0.803954,0.693832}%
\pgfsetfillcolor{currentfill}%
\pgfsetlinewidth{0.311001pt}%
\definecolor{currentstroke}{rgb}{1.000000,1.000000,1.000000}%
\pgfsetstrokecolor{currentstroke}%
\pgfsetdash{}{0pt}%
\pgfpathmoveto{\pgfqpoint{4.088540in}{1.705783in}}%
\pgfpathcurveto{\pgfqpoint{4.095673in}{1.705783in}}{\pgfqpoint{4.102515in}{1.708617in}}{\pgfqpoint{4.107559in}{1.713660in}}%
\pgfpathcurveto{\pgfqpoint{4.112602in}{1.718704in}}{\pgfqpoint{4.115436in}{1.725546in}}{\pgfqpoint{4.115436in}{1.732679in}}%
\pgfpathcurveto{\pgfqpoint{4.115436in}{1.739811in}}{\pgfqpoint{4.112602in}{1.746653in}}{\pgfqpoint{4.107559in}{1.751697in}}%
\pgfpathcurveto{\pgfqpoint{4.102515in}{1.756740in}}{\pgfqpoint{4.095673in}{1.759574in}}{\pgfqpoint{4.088540in}{1.759574in}}%
\pgfpathcurveto{\pgfqpoint{4.081408in}{1.759574in}}{\pgfqpoint{4.074566in}{1.756740in}}{\pgfqpoint{4.069522in}{1.751697in}}%
\pgfpathcurveto{\pgfqpoint{4.064479in}{1.746653in}}{\pgfqpoint{4.061645in}{1.739811in}}{\pgfqpoint{4.061645in}{1.732679in}}%
\pgfpathcurveto{\pgfqpoint{4.061645in}{1.725546in}}{\pgfqpoint{4.064479in}{1.718704in}}{\pgfqpoint{4.069522in}{1.713660in}}%
\pgfpathcurveto{\pgfqpoint{4.074566in}{1.708617in}}{\pgfqpoint{4.081408in}{1.705783in}}{\pgfqpoint{4.088540in}{1.705783in}}%
\pgfpathclose%
\pgfusepath{stroke,fill}%
\end{pgfscope}%
\begin{pgfscope}%
\pgfpathrectangle{\pgfqpoint{2.867647in}{0.500000in}}{\pgfqpoint{1.764706in}{1.700000in}}%
\pgfusepath{clip}%
\pgfsetbuttcap%
\pgfsetroundjoin%
\definecolor{currentfill}{rgb}{0.962018,0.586477,0.424918}%
\pgfsetfillcolor{currentfill}%
\pgfsetlinewidth{0.311001pt}%
\definecolor{currentstroke}{rgb}{1.000000,1.000000,1.000000}%
\pgfsetstrokecolor{currentstroke}%
\pgfsetdash{}{0pt}%
\pgfpathmoveto{\pgfqpoint{4.046700in}{1.347925in}}%
\pgfpathcurveto{\pgfqpoint{4.053833in}{1.347925in}}{\pgfqpoint{4.060674in}{1.350759in}}{\pgfqpoint{4.065718in}{1.355803in}}%
\pgfpathcurveto{\pgfqpoint{4.070762in}{1.360847in}}{\pgfqpoint{4.073596in}{1.367688in}}{\pgfqpoint{4.073596in}{1.374821in}}%
\pgfpathcurveto{\pgfqpoint{4.073596in}{1.381954in}}{\pgfqpoint{4.070762in}{1.388796in}}{\pgfqpoint{4.065718in}{1.393839in}}%
\pgfpathcurveto{\pgfqpoint{4.060674in}{1.398883in}}{\pgfqpoint{4.053833in}{1.401717in}}{\pgfqpoint{4.046700in}{1.401717in}}%
\pgfpathcurveto{\pgfqpoint{4.039567in}{1.401717in}}{\pgfqpoint{4.032725in}{1.398883in}}{\pgfqpoint{4.027682in}{1.393839in}}%
\pgfpathcurveto{\pgfqpoint{4.022638in}{1.388796in}}{\pgfqpoint{4.019804in}{1.381954in}}{\pgfqpoint{4.019804in}{1.374821in}}%
\pgfpathcurveto{\pgfqpoint{4.019804in}{1.367688in}}{\pgfqpoint{4.022638in}{1.360847in}}{\pgfqpoint{4.027682in}{1.355803in}}%
\pgfpathcurveto{\pgfqpoint{4.032725in}{1.350759in}}{\pgfqpoint{4.039567in}{1.347925in}}{\pgfqpoint{4.046700in}{1.347925in}}%
\pgfpathclose%
\pgfusepath{stroke,fill}%
\end{pgfscope}%
\begin{pgfscope}%
\pgfpathrectangle{\pgfqpoint{2.867647in}{0.500000in}}{\pgfqpoint{1.764706in}{1.700000in}}%
\pgfusepath{clip}%
\pgfsetbuttcap%
\pgfsetroundjoin%
\definecolor{currentfill}{rgb}{0.963379,0.625574,0.465113}%
\pgfsetfillcolor{currentfill}%
\pgfsetlinewidth{0.311001pt}%
\definecolor{currentstroke}{rgb}{1.000000,1.000000,1.000000}%
\pgfsetstrokecolor{currentstroke}%
\pgfsetdash{}{0pt}%
\pgfpathmoveto{\pgfqpoint{4.042942in}{0.857272in}}%
\pgfpathcurveto{\pgfqpoint{4.050075in}{0.857272in}}{\pgfqpoint{4.056917in}{0.860106in}}{\pgfqpoint{4.061960in}{0.865149in}}%
\pgfpathcurveto{\pgfqpoint{4.067004in}{0.870193in}}{\pgfqpoint{4.069838in}{0.877035in}}{\pgfqpoint{4.069838in}{0.884168in}}%
\pgfpathcurveto{\pgfqpoint{4.069838in}{0.891300in}}{\pgfqpoint{4.067004in}{0.898142in}}{\pgfqpoint{4.061960in}{0.903186in}}%
\pgfpathcurveto{\pgfqpoint{4.056917in}{0.908229in}}{\pgfqpoint{4.050075in}{0.911063in}}{\pgfqpoint{4.042942in}{0.911063in}}%
\pgfpathcurveto{\pgfqpoint{4.035809in}{0.911063in}}{\pgfqpoint{4.028968in}{0.908229in}}{\pgfqpoint{4.023924in}{0.903186in}}%
\pgfpathcurveto{\pgfqpoint{4.018880in}{0.898142in}}{\pgfqpoint{4.016046in}{0.891300in}}{\pgfqpoint{4.016046in}{0.884168in}}%
\pgfpathcurveto{\pgfqpoint{4.016046in}{0.877035in}}{\pgfqpoint{4.018880in}{0.870193in}}{\pgfqpoint{4.023924in}{0.865149in}}%
\pgfpathcurveto{\pgfqpoint{4.028968in}{0.860106in}}{\pgfqpoint{4.035809in}{0.857272in}}{\pgfqpoint{4.042942in}{0.857272in}}%
\pgfpathclose%
\pgfusepath{stroke,fill}%
\end{pgfscope}%
\begin{pgfscope}%
\pgfpathrectangle{\pgfqpoint{2.867647in}{0.500000in}}{\pgfqpoint{1.764706in}{1.700000in}}%
\pgfusepath{clip}%
\pgfsetbuttcap%
\pgfsetroundjoin%
\definecolor{currentfill}{rgb}{0.981377,0.920617,0.865369}%
\pgfsetfillcolor{currentfill}%
\pgfsetlinewidth{0.311001pt}%
\definecolor{currentstroke}{rgb}{1.000000,1.000000,1.000000}%
\pgfsetstrokecolor{currentstroke}%
\pgfsetdash{}{0pt}%
\pgfpathmoveto{\pgfqpoint{4.158516in}{1.183116in}}%
\pgfpathcurveto{\pgfqpoint{4.165649in}{1.183116in}}{\pgfqpoint{4.172490in}{1.185949in}}{\pgfqpoint{4.177534in}{1.190993in}}%
\pgfpathcurveto{\pgfqpoint{4.182578in}{1.196037in}}{\pgfqpoint{4.185412in}{1.202878in}}{\pgfqpoint{4.185412in}{1.210011in}}%
\pgfpathcurveto{\pgfqpoint{4.185412in}{1.217144in}}{\pgfqpoint{4.182578in}{1.223986in}}{\pgfqpoint{4.177534in}{1.229029in}}%
\pgfpathcurveto{\pgfqpoint{4.172490in}{1.234073in}}{\pgfqpoint{4.165649in}{1.236907in}}{\pgfqpoint{4.158516in}{1.236907in}}%
\pgfpathcurveto{\pgfqpoint{4.151383in}{1.236907in}}{\pgfqpoint{4.144541in}{1.234073in}}{\pgfqpoint{4.139498in}{1.229029in}}%
\pgfpathcurveto{\pgfqpoint{4.134454in}{1.223986in}}{\pgfqpoint{4.131620in}{1.217144in}}{\pgfqpoint{4.131620in}{1.210011in}}%
\pgfpathcurveto{\pgfqpoint{4.131620in}{1.202878in}}{\pgfqpoint{4.134454in}{1.196037in}}{\pgfqpoint{4.139498in}{1.190993in}}%
\pgfpathcurveto{\pgfqpoint{4.144541in}{1.185949in}}{\pgfqpoint{4.151383in}{1.183116in}}{\pgfqpoint{4.158516in}{1.183116in}}%
\pgfpathclose%
\pgfusepath{stroke,fill}%
\end{pgfscope}%
\begin{pgfscope}%
\pgfpathrectangle{\pgfqpoint{2.867647in}{0.500000in}}{\pgfqpoint{1.764706in}{1.700000in}}%
\pgfusepath{clip}%
\pgfsetbuttcap%
\pgfsetroundjoin%
\definecolor{currentfill}{rgb}{0.966812,0.762584,0.633643}%
\pgfsetfillcolor{currentfill}%
\pgfsetlinewidth{0.311001pt}%
\definecolor{currentstroke}{rgb}{1.000000,1.000000,1.000000}%
\pgfsetstrokecolor{currentstroke}%
\pgfsetdash{}{0pt}%
\pgfpathmoveto{\pgfqpoint{4.218113in}{1.024227in}}%
\pgfpathcurveto{\pgfqpoint{4.225246in}{1.024227in}}{\pgfqpoint{4.232088in}{1.027060in}}{\pgfqpoint{4.237131in}{1.032104in}}%
\pgfpathcurveto{\pgfqpoint{4.242175in}{1.037148in}}{\pgfqpoint{4.245009in}{1.043989in}}{\pgfqpoint{4.245009in}{1.051122in}}%
\pgfpathcurveto{\pgfqpoint{4.245009in}{1.058255in}}{\pgfqpoint{4.242175in}{1.065097in}}{\pgfqpoint{4.237131in}{1.070140in}}%
\pgfpathcurveto{\pgfqpoint{4.232088in}{1.075184in}}{\pgfqpoint{4.225246in}{1.078018in}}{\pgfqpoint{4.218113in}{1.078018in}}%
\pgfpathcurveto{\pgfqpoint{4.210980in}{1.078018in}}{\pgfqpoint{4.204139in}{1.075184in}}{\pgfqpoint{4.199095in}{1.070140in}}%
\pgfpathcurveto{\pgfqpoint{4.194051in}{1.065097in}}{\pgfqpoint{4.191217in}{1.058255in}}{\pgfqpoint{4.191217in}{1.051122in}}%
\pgfpathcurveto{\pgfqpoint{4.191217in}{1.043989in}}{\pgfqpoint{4.194051in}{1.037148in}}{\pgfqpoint{4.199095in}{1.032104in}}%
\pgfpathcurveto{\pgfqpoint{4.204139in}{1.027060in}}{\pgfqpoint{4.210980in}{1.024227in}}{\pgfqpoint{4.218113in}{1.024227in}}%
\pgfpathclose%
\pgfusepath{stroke,fill}%
\end{pgfscope}%
\begin{pgfscope}%
\pgfpathrectangle{\pgfqpoint{2.867647in}{0.500000in}}{\pgfqpoint{1.764706in}{1.700000in}}%
\pgfusepath{clip}%
\pgfsetbuttcap%
\pgfsetroundjoin%
\definecolor{currentfill}{rgb}{0.698038,0.088972,0.346299}%
\pgfsetfillcolor{currentfill}%
\pgfsetlinewidth{0.311001pt}%
\definecolor{currentstroke}{rgb}{1.000000,1.000000,1.000000}%
\pgfsetstrokecolor{currentstroke}%
\pgfsetdash{}{0pt}%
\pgfpathmoveto{\pgfqpoint{3.917185in}{0.723352in}}%
\pgfpathcurveto{\pgfqpoint{3.924318in}{0.723352in}}{\pgfqpoint{3.931160in}{0.726186in}}{\pgfqpoint{3.936203in}{0.731229in}}%
\pgfpathcurveto{\pgfqpoint{3.941247in}{0.736273in}}{\pgfqpoint{3.944081in}{0.743115in}}{\pgfqpoint{3.944081in}{0.750247in}}%
\pgfpathcurveto{\pgfqpoint{3.944081in}{0.757380in}}{\pgfqpoint{3.941247in}{0.764222in}}{\pgfqpoint{3.936203in}{0.769266in}}%
\pgfpathcurveto{\pgfqpoint{3.931160in}{0.774309in}}{\pgfqpoint{3.924318in}{0.777143in}}{\pgfqpoint{3.917185in}{0.777143in}}%
\pgfpathcurveto{\pgfqpoint{3.910052in}{0.777143in}}{\pgfqpoint{3.903211in}{0.774309in}}{\pgfqpoint{3.898167in}{0.769266in}}%
\pgfpathcurveto{\pgfqpoint{3.893123in}{0.764222in}}{\pgfqpoint{3.890289in}{0.757380in}}{\pgfqpoint{3.890289in}{0.750247in}}%
\pgfpathcurveto{\pgfqpoint{3.890289in}{0.743115in}}{\pgfqpoint{3.893123in}{0.736273in}}{\pgfqpoint{3.898167in}{0.731229in}}%
\pgfpathcurveto{\pgfqpoint{3.903211in}{0.726186in}}{\pgfqpoint{3.910052in}{0.723352in}}{\pgfqpoint{3.917185in}{0.723352in}}%
\pgfpathclose%
\pgfusepath{stroke,fill}%
\end{pgfscope}%
\begin{pgfscope}%
\pgfpathrectangle{\pgfqpoint{2.867647in}{0.500000in}}{\pgfqpoint{1.764706in}{1.700000in}}%
\pgfusepath{clip}%
\pgfsetbuttcap%
\pgfsetroundjoin%
\definecolor{currentfill}{rgb}{0.975018,0.868213,0.788710}%
\pgfsetfillcolor{currentfill}%
\pgfsetlinewidth{0.311001pt}%
\definecolor{currentstroke}{rgb}{1.000000,1.000000,1.000000}%
\pgfsetstrokecolor{currentstroke}%
\pgfsetdash{}{0pt}%
\pgfpathmoveto{\pgfqpoint{4.160374in}{1.612242in}}%
\pgfpathcurveto{\pgfqpoint{4.167507in}{1.612242in}}{\pgfqpoint{4.174348in}{1.615076in}}{\pgfqpoint{4.179392in}{1.620119in}}%
\pgfpathcurveto{\pgfqpoint{4.184436in}{1.625163in}}{\pgfqpoint{4.187269in}{1.632005in}}{\pgfqpoint{4.187269in}{1.639137in}}%
\pgfpathcurveto{\pgfqpoint{4.187269in}{1.646270in}}{\pgfqpoint{4.184436in}{1.653112in}}{\pgfqpoint{4.179392in}{1.658156in}}%
\pgfpathcurveto{\pgfqpoint{4.174348in}{1.663199in}}{\pgfqpoint{4.167507in}{1.666033in}}{\pgfqpoint{4.160374in}{1.666033in}}%
\pgfpathcurveto{\pgfqpoint{4.153241in}{1.666033in}}{\pgfqpoint{4.146399in}{1.663199in}}{\pgfqpoint{4.141356in}{1.658156in}}%
\pgfpathcurveto{\pgfqpoint{4.136312in}{1.653112in}}{\pgfqpoint{4.133478in}{1.646270in}}{\pgfqpoint{4.133478in}{1.639137in}}%
\pgfpathcurveto{\pgfqpoint{4.133478in}{1.632005in}}{\pgfqpoint{4.136312in}{1.625163in}}{\pgfqpoint{4.141356in}{1.620119in}}%
\pgfpathcurveto{\pgfqpoint{4.146399in}{1.615076in}}{\pgfqpoint{4.153241in}{1.612242in}}{\pgfqpoint{4.160374in}{1.612242in}}%
\pgfpathclose%
\pgfusepath{stroke,fill}%
\end{pgfscope}%
\begin{pgfscope}%
\pgfpathrectangle{\pgfqpoint{2.867647in}{0.500000in}}{\pgfqpoint{1.764706in}{1.700000in}}%
\pgfusepath{clip}%
\pgfsetbuttcap%
\pgfsetroundjoin%
\definecolor{currentfill}{rgb}{0.951650,0.442241,0.302145}%
\pgfsetfillcolor{currentfill}%
\pgfsetlinewidth{0.311001pt}%
\definecolor{currentstroke}{rgb}{1.000000,1.000000,1.000000}%
\pgfsetstrokecolor{currentstroke}%
\pgfsetdash{}{0pt}%
\pgfpathmoveto{\pgfqpoint{3.937822in}{1.060974in}}%
\pgfpathcurveto{\pgfqpoint{3.944955in}{1.060974in}}{\pgfqpoint{3.951797in}{1.063807in}}{\pgfqpoint{3.956840in}{1.068851in}}%
\pgfpathcurveto{\pgfqpoint{3.961884in}{1.073895in}}{\pgfqpoint{3.964718in}{1.080736in}}{\pgfqpoint{3.964718in}{1.087869in}}%
\pgfpathcurveto{\pgfqpoint{3.964718in}{1.095002in}}{\pgfqpoint{3.961884in}{1.101844in}}{\pgfqpoint{3.956840in}{1.106887in}}%
\pgfpathcurveto{\pgfqpoint{3.951797in}{1.111931in}}{\pgfqpoint{3.944955in}{1.114765in}}{\pgfqpoint{3.937822in}{1.114765in}}%
\pgfpathcurveto{\pgfqpoint{3.930689in}{1.114765in}}{\pgfqpoint{3.923848in}{1.111931in}}{\pgfqpoint{3.918804in}{1.106887in}}%
\pgfpathcurveto{\pgfqpoint{3.913760in}{1.101844in}}{\pgfqpoint{3.910927in}{1.095002in}}{\pgfqpoint{3.910927in}{1.087869in}}%
\pgfpathcurveto{\pgfqpoint{3.910927in}{1.080736in}}{\pgfqpoint{3.913760in}{1.073895in}}{\pgfqpoint{3.918804in}{1.068851in}}%
\pgfpathcurveto{\pgfqpoint{3.923848in}{1.063807in}}{\pgfqpoint{3.930689in}{1.060974in}}{\pgfqpoint{3.937822in}{1.060974in}}%
\pgfpathclose%
\pgfusepath{stroke,fill}%
\end{pgfscope}%
\begin{pgfscope}%
\pgfpathrectangle{\pgfqpoint{2.867647in}{0.500000in}}{\pgfqpoint{1.764706in}{1.700000in}}%
\pgfusepath{clip}%
\pgfsetbuttcap%
\pgfsetroundjoin%
\definecolor{currentfill}{rgb}{0.964799,0.689101,0.537560}%
\pgfsetfillcolor{currentfill}%
\pgfsetlinewidth{0.311001pt}%
\definecolor{currentstroke}{rgb}{1.000000,1.000000,1.000000}%
\pgfsetstrokecolor{currentstroke}%
\pgfsetdash{}{0pt}%
\pgfpathmoveto{\pgfqpoint{4.019325in}{1.538266in}}%
\pgfpathcurveto{\pgfqpoint{4.026458in}{1.538266in}}{\pgfqpoint{4.033299in}{1.541100in}}{\pgfqpoint{4.038343in}{1.546144in}}%
\pgfpathcurveto{\pgfqpoint{4.043387in}{1.551187in}}{\pgfqpoint{4.046220in}{1.558029in}}{\pgfqpoint{4.046220in}{1.565162in}}%
\pgfpathcurveto{\pgfqpoint{4.046220in}{1.572295in}}{\pgfqpoint{4.043387in}{1.579136in}}{\pgfqpoint{4.038343in}{1.584180in}}%
\pgfpathcurveto{\pgfqpoint{4.033299in}{1.589224in}}{\pgfqpoint{4.026458in}{1.592057in}}{\pgfqpoint{4.019325in}{1.592057in}}%
\pgfpathcurveto{\pgfqpoint{4.012192in}{1.592057in}}{\pgfqpoint{4.005350in}{1.589224in}}{\pgfqpoint{4.000307in}{1.584180in}}%
\pgfpathcurveto{\pgfqpoint{3.995263in}{1.579136in}}{\pgfqpoint{3.992429in}{1.572295in}}{\pgfqpoint{3.992429in}{1.565162in}}%
\pgfpathcurveto{\pgfqpoint{3.992429in}{1.558029in}}{\pgfqpoint{3.995263in}{1.551187in}}{\pgfqpoint{4.000307in}{1.546144in}}%
\pgfpathcurveto{\pgfqpoint{4.005350in}{1.541100in}}{\pgfqpoint{4.012192in}{1.538266in}}{\pgfqpoint{4.019325in}{1.538266in}}%
\pgfpathclose%
\pgfusepath{stroke,fill}%
\end{pgfscope}%
\begin{pgfscope}%
\pgfpathrectangle{\pgfqpoint{2.867647in}{0.500000in}}{\pgfqpoint{1.764706in}{1.700000in}}%
\pgfusepath{clip}%
\pgfsetbuttcap%
\pgfsetroundjoin%
\definecolor{currentfill}{rgb}{0.974412,0.862387,0.780156}%
\pgfsetfillcolor{currentfill}%
\pgfsetlinewidth{0.311001pt}%
\definecolor{currentstroke}{rgb}{1.000000,1.000000,1.000000}%
\pgfsetstrokecolor{currentstroke}%
\pgfsetdash{}{0pt}%
\pgfpathmoveto{\pgfqpoint{4.105278in}{1.004093in}}%
\pgfpathcurveto{\pgfqpoint{4.112411in}{1.004093in}}{\pgfqpoint{4.119252in}{1.006927in}}{\pgfqpoint{4.124296in}{1.011971in}}%
\pgfpathcurveto{\pgfqpoint{4.129340in}{1.017014in}}{\pgfqpoint{4.132173in}{1.023856in}}{\pgfqpoint{4.132173in}{1.030989in}}%
\pgfpathcurveto{\pgfqpoint{4.132173in}{1.038122in}}{\pgfqpoint{4.129340in}{1.044963in}}{\pgfqpoint{4.124296in}{1.050007in}}%
\pgfpathcurveto{\pgfqpoint{4.119252in}{1.055051in}}{\pgfqpoint{4.112411in}{1.057885in}}{\pgfqpoint{4.105278in}{1.057885in}}%
\pgfpathcurveto{\pgfqpoint{4.098145in}{1.057885in}}{\pgfqpoint{4.091303in}{1.055051in}}{\pgfqpoint{4.086260in}{1.050007in}}%
\pgfpathcurveto{\pgfqpoint{4.081216in}{1.044963in}}{\pgfqpoint{4.078382in}{1.038122in}}{\pgfqpoint{4.078382in}{1.030989in}}%
\pgfpathcurveto{\pgfqpoint{4.078382in}{1.023856in}}{\pgfqpoint{4.081216in}{1.017014in}}{\pgfqpoint{4.086260in}{1.011971in}}%
\pgfpathcurveto{\pgfqpoint{4.091303in}{1.006927in}}{\pgfqpoint{4.098145in}{1.004093in}}{\pgfqpoint{4.105278in}{1.004093in}}%
\pgfpathclose%
\pgfusepath{stroke,fill}%
\end{pgfscope}%
\begin{pgfscope}%
\pgfpathrectangle{\pgfqpoint{2.867647in}{0.500000in}}{\pgfqpoint{1.764706in}{1.700000in}}%
\pgfusepath{clip}%
\pgfsetbuttcap%
\pgfsetroundjoin%
\definecolor{currentfill}{rgb}{0.971202,0.827364,0.728520}%
\pgfsetfillcolor{currentfill}%
\pgfsetlinewidth{0.311001pt}%
\definecolor{currentstroke}{rgb}{1.000000,1.000000,1.000000}%
\pgfsetstrokecolor{currentstroke}%
\pgfsetdash{}{0pt}%
\pgfpathmoveto{\pgfqpoint{4.209982in}{1.581774in}}%
\pgfpathcurveto{\pgfqpoint{4.217114in}{1.581774in}}{\pgfqpoint{4.223956in}{1.584608in}}{\pgfqpoint{4.229000in}{1.589652in}}%
\pgfpathcurveto{\pgfqpoint{4.234043in}{1.594695in}}{\pgfqpoint{4.236877in}{1.601537in}}{\pgfqpoint{4.236877in}{1.608670in}}%
\pgfpathcurveto{\pgfqpoint{4.236877in}{1.615803in}}{\pgfqpoint{4.234043in}{1.622644in}}{\pgfqpoint{4.229000in}{1.627688in}}%
\pgfpathcurveto{\pgfqpoint{4.223956in}{1.632732in}}{\pgfqpoint{4.217114in}{1.635565in}}{\pgfqpoint{4.209982in}{1.635565in}}%
\pgfpathcurveto{\pgfqpoint{4.202849in}{1.635565in}}{\pgfqpoint{4.196007in}{1.632732in}}{\pgfqpoint{4.190963in}{1.627688in}}%
\pgfpathcurveto{\pgfqpoint{4.185920in}{1.622644in}}{\pgfqpoint{4.183086in}{1.615803in}}{\pgfqpoint{4.183086in}{1.608670in}}%
\pgfpathcurveto{\pgfqpoint{4.183086in}{1.601537in}}{\pgfqpoint{4.185920in}{1.594695in}}{\pgfqpoint{4.190963in}{1.589652in}}%
\pgfpathcurveto{\pgfqpoint{4.196007in}{1.584608in}}{\pgfqpoint{4.202849in}{1.581774in}}{\pgfqpoint{4.209982in}{1.581774in}}%
\pgfpathclose%
\pgfusepath{stroke,fill}%
\end{pgfscope}%
\begin{pgfscope}%
\pgfpathrectangle{\pgfqpoint{2.867647in}{0.500000in}}{\pgfqpoint{1.764706in}{1.700000in}}%
\pgfusepath{clip}%
\pgfsetbuttcap%
\pgfsetroundjoin%
\definecolor{currentfill}{rgb}{0.977657,0.891500,0.822809}%
\pgfsetfillcolor{currentfill}%
\pgfsetlinewidth{0.311001pt}%
\definecolor{currentstroke}{rgb}{1.000000,1.000000,1.000000}%
\pgfsetstrokecolor{currentstroke}%
\pgfsetdash{}{0pt}%
\pgfpathmoveto{\pgfqpoint{4.235213in}{1.329042in}}%
\pgfpathcurveto{\pgfqpoint{4.242345in}{1.329042in}}{\pgfqpoint{4.249187in}{1.331876in}}{\pgfqpoint{4.254231in}{1.336920in}}%
\pgfpathcurveto{\pgfqpoint{4.259274in}{1.341964in}}{\pgfqpoint{4.262108in}{1.348805in}}{\pgfqpoint{4.262108in}{1.355938in}}%
\pgfpathcurveto{\pgfqpoint{4.262108in}{1.363071in}}{\pgfqpoint{4.259274in}{1.369912in}}{\pgfqpoint{4.254231in}{1.374956in}}%
\pgfpathcurveto{\pgfqpoint{4.249187in}{1.380000in}}{\pgfqpoint{4.242345in}{1.382834in}}{\pgfqpoint{4.235213in}{1.382834in}}%
\pgfpathcurveto{\pgfqpoint{4.228080in}{1.382834in}}{\pgfqpoint{4.221238in}{1.380000in}}{\pgfqpoint{4.216195in}{1.374956in}}%
\pgfpathcurveto{\pgfqpoint{4.211151in}{1.369912in}}{\pgfqpoint{4.208317in}{1.363071in}}{\pgfqpoint{4.208317in}{1.355938in}}%
\pgfpathcurveto{\pgfqpoint{4.208317in}{1.348805in}}{\pgfqpoint{4.211151in}{1.341964in}}{\pgfqpoint{4.216195in}{1.336920in}}%
\pgfpathcurveto{\pgfqpoint{4.221238in}{1.331876in}}{\pgfqpoint{4.228080in}{1.329042in}}{\pgfqpoint{4.235213in}{1.329042in}}%
\pgfpathclose%
\pgfusepath{stroke,fill}%
\end{pgfscope}%
\begin{pgfscope}%
\pgfpathrectangle{\pgfqpoint{2.867647in}{0.500000in}}{\pgfqpoint{1.764706in}{1.700000in}}%
\pgfusepath{clip}%
\pgfsetbuttcap%
\pgfsetroundjoin%
\definecolor{currentfill}{rgb}{0.981377,0.920617,0.865369}%
\pgfsetfillcolor{currentfill}%
\pgfsetlinewidth{0.311001pt}%
\definecolor{currentstroke}{rgb}{1.000000,1.000000,1.000000}%
\pgfsetstrokecolor{currentstroke}%
\pgfsetdash{}{0pt}%
\pgfpathmoveto{\pgfqpoint{4.187032in}{1.187194in}}%
\pgfpathcurveto{\pgfqpoint{4.194165in}{1.187194in}}{\pgfqpoint{4.201007in}{1.190028in}}{\pgfqpoint{4.206050in}{1.195072in}}%
\pgfpathcurveto{\pgfqpoint{4.211094in}{1.200115in}}{\pgfqpoint{4.213928in}{1.206957in}}{\pgfqpoint{4.213928in}{1.214090in}}%
\pgfpathcurveto{\pgfqpoint{4.213928in}{1.221222in}}{\pgfqpoint{4.211094in}{1.228064in}}{\pgfqpoint{4.206050in}{1.233108in}}%
\pgfpathcurveto{\pgfqpoint{4.201007in}{1.238151in}}{\pgfqpoint{4.194165in}{1.240985in}}{\pgfqpoint{4.187032in}{1.240985in}}%
\pgfpathcurveto{\pgfqpoint{4.179899in}{1.240985in}}{\pgfqpoint{4.173058in}{1.238151in}}{\pgfqpoint{4.168014in}{1.233108in}}%
\pgfpathcurveto{\pgfqpoint{4.162970in}{1.228064in}}{\pgfqpoint{4.160136in}{1.221222in}}{\pgfqpoint{4.160136in}{1.214090in}}%
\pgfpathcurveto{\pgfqpoint{4.160136in}{1.206957in}}{\pgfqpoint{4.162970in}{1.200115in}}{\pgfqpoint{4.168014in}{1.195072in}}%
\pgfpathcurveto{\pgfqpoint{4.173058in}{1.190028in}}{\pgfqpoint{4.179899in}{1.187194in}}{\pgfqpoint{4.187032in}{1.187194in}}%
\pgfpathclose%
\pgfusepath{stroke,fill}%
\end{pgfscope}%
\begin{pgfscope}%
\pgfpathrectangle{\pgfqpoint{2.867647in}{0.500000in}}{\pgfqpoint{1.764706in}{1.700000in}}%
\pgfusepath{clip}%
\pgfsetbuttcap%
\pgfsetroundjoin%
\definecolor{currentfill}{rgb}{0.971202,0.827364,0.728520}%
\pgfsetfillcolor{currentfill}%
\pgfsetlinewidth{0.311001pt}%
\definecolor{currentstroke}{rgb}{1.000000,1.000000,1.000000}%
\pgfsetstrokecolor{currentstroke}%
\pgfsetdash{}{0pt}%
\pgfpathmoveto{\pgfqpoint{4.053672in}{1.653352in}}%
\pgfpathcurveto{\pgfqpoint{4.060805in}{1.653352in}}{\pgfqpoint{4.067647in}{1.656185in}}{\pgfqpoint{4.072690in}{1.661229in}}%
\pgfpathcurveto{\pgfqpoint{4.077734in}{1.666273in}}{\pgfqpoint{4.080568in}{1.673114in}}{\pgfqpoint{4.080568in}{1.680247in}}%
\pgfpathcurveto{\pgfqpoint{4.080568in}{1.687380in}}{\pgfqpoint{4.077734in}{1.694222in}}{\pgfqpoint{4.072690in}{1.699265in}}%
\pgfpathcurveto{\pgfqpoint{4.067647in}{1.704309in}}{\pgfqpoint{4.060805in}{1.707143in}}{\pgfqpoint{4.053672in}{1.707143in}}%
\pgfpathcurveto{\pgfqpoint{4.046539in}{1.707143in}}{\pgfqpoint{4.039698in}{1.704309in}}{\pgfqpoint{4.034654in}{1.699265in}}%
\pgfpathcurveto{\pgfqpoint{4.029610in}{1.694222in}}{\pgfqpoint{4.026777in}{1.687380in}}{\pgfqpoint{4.026777in}{1.680247in}}%
\pgfpathcurveto{\pgfqpoint{4.026777in}{1.673114in}}{\pgfqpoint{4.029610in}{1.666273in}}{\pgfqpoint{4.034654in}{1.661229in}}%
\pgfpathcurveto{\pgfqpoint{4.039698in}{1.656185in}}{\pgfqpoint{4.046539in}{1.653352in}}{\pgfqpoint{4.053672in}{1.653352in}}%
\pgfpathclose%
\pgfusepath{stroke,fill}%
\end{pgfscope}%
\begin{pgfscope}%
\pgfpathrectangle{\pgfqpoint{2.867647in}{0.500000in}}{\pgfqpoint{1.764706in}{1.700000in}}%
\pgfusepath{clip}%
\pgfsetbuttcap%
\pgfsetroundjoin%
\definecolor{currentfill}{rgb}{0.974412,0.862387,0.780156}%
\pgfsetfillcolor{currentfill}%
\pgfsetlinewidth{0.311001pt}%
\definecolor{currentstroke}{rgb}{1.000000,1.000000,1.000000}%
\pgfsetstrokecolor{currentstroke}%
\pgfsetdash{}{0pt}%
\pgfpathmoveto{\pgfqpoint{4.124512in}{1.390234in}}%
\pgfpathcurveto{\pgfqpoint{4.131645in}{1.390234in}}{\pgfqpoint{4.138487in}{1.393067in}}{\pgfqpoint{4.143530in}{1.398111in}}%
\pgfpathcurveto{\pgfqpoint{4.148574in}{1.403155in}}{\pgfqpoint{4.151408in}{1.409996in}}{\pgfqpoint{4.151408in}{1.417129in}}%
\pgfpathcurveto{\pgfqpoint{4.151408in}{1.424262in}}{\pgfqpoint{4.148574in}{1.431104in}}{\pgfqpoint{4.143530in}{1.436147in}}%
\pgfpathcurveto{\pgfqpoint{4.138487in}{1.441191in}}{\pgfqpoint{4.131645in}{1.444025in}}{\pgfqpoint{4.124512in}{1.444025in}}%
\pgfpathcurveto{\pgfqpoint{4.117379in}{1.444025in}}{\pgfqpoint{4.110538in}{1.441191in}}{\pgfqpoint{4.105494in}{1.436147in}}%
\pgfpathcurveto{\pgfqpoint{4.100450in}{1.431104in}}{\pgfqpoint{4.097617in}{1.424262in}}{\pgfqpoint{4.097617in}{1.417129in}}%
\pgfpathcurveto{\pgfqpoint{4.097617in}{1.409996in}}{\pgfqpoint{4.100450in}{1.403155in}}{\pgfqpoint{4.105494in}{1.398111in}}%
\pgfpathcurveto{\pgfqpoint{4.110538in}{1.393067in}}{\pgfqpoint{4.117379in}{1.390234in}}{\pgfqpoint{4.124512in}{1.390234in}}%
\pgfpathclose%
\pgfusepath{stroke,fill}%
\end{pgfscope}%
\begin{pgfscope}%
\pgfpathrectangle{\pgfqpoint{2.867647in}{0.500000in}}{\pgfqpoint{1.764706in}{1.700000in}}%
\pgfusepath{clip}%
\pgfsetbuttcap%
\pgfsetroundjoin%
\definecolor{currentfill}{rgb}{0.964173,0.657587,0.500469}%
\pgfsetfillcolor{currentfill}%
\pgfsetlinewidth{0.311001pt}%
\definecolor{currentstroke}{rgb}{1.000000,1.000000,1.000000}%
\pgfsetstrokecolor{currentstroke}%
\pgfsetdash{}{0pt}%
\pgfpathmoveto{\pgfqpoint{4.307722in}{1.186381in}}%
\pgfpathcurveto{\pgfqpoint{4.314854in}{1.186381in}}{\pgfqpoint{4.321696in}{1.189215in}}{\pgfqpoint{4.326740in}{1.194258in}}%
\pgfpathcurveto{\pgfqpoint{4.331783in}{1.199302in}}{\pgfqpoint{4.334617in}{1.206144in}}{\pgfqpoint{4.334617in}{1.213276in}}%
\pgfpathcurveto{\pgfqpoint{4.334617in}{1.220409in}}{\pgfqpoint{4.331783in}{1.227251in}}{\pgfqpoint{4.326740in}{1.232295in}}%
\pgfpathcurveto{\pgfqpoint{4.321696in}{1.237338in}}{\pgfqpoint{4.314854in}{1.240172in}}{\pgfqpoint{4.307722in}{1.240172in}}%
\pgfpathcurveto{\pgfqpoint{4.300589in}{1.240172in}}{\pgfqpoint{4.293747in}{1.237338in}}{\pgfqpoint{4.288703in}{1.232295in}}%
\pgfpathcurveto{\pgfqpoint{4.283660in}{1.227251in}}{\pgfqpoint{4.280826in}{1.220409in}}{\pgfqpoint{4.280826in}{1.213276in}}%
\pgfpathcurveto{\pgfqpoint{4.280826in}{1.206144in}}{\pgfqpoint{4.283660in}{1.199302in}}{\pgfqpoint{4.288703in}{1.194258in}}%
\pgfpathcurveto{\pgfqpoint{4.293747in}{1.189215in}}{\pgfqpoint{4.300589in}{1.186381in}}{\pgfqpoint{4.307722in}{1.186381in}}%
\pgfpathclose%
\pgfusepath{stroke,fill}%
\end{pgfscope}%
\begin{pgfscope}%
\pgfpathrectangle{\pgfqpoint{2.867647in}{0.500000in}}{\pgfqpoint{1.764706in}{1.700000in}}%
\pgfusepath{clip}%
\pgfsetbuttcap%
\pgfsetroundjoin%
\definecolor{currentfill}{rgb}{0.979891,0.908948,0.848279}%
\pgfsetfillcolor{currentfill}%
\pgfsetlinewidth{0.311001pt}%
\definecolor{currentstroke}{rgb}{1.000000,1.000000,1.000000}%
\pgfsetstrokecolor{currentstroke}%
\pgfsetdash{}{0pt}%
\pgfpathmoveto{\pgfqpoint{4.208899in}{1.387267in}}%
\pgfpathcurveto{\pgfqpoint{4.216032in}{1.387267in}}{\pgfqpoint{4.222873in}{1.390101in}}{\pgfqpoint{4.227917in}{1.395145in}}%
\pgfpathcurveto{\pgfqpoint{4.232961in}{1.400188in}}{\pgfqpoint{4.235795in}{1.407030in}}{\pgfqpoint{4.235795in}{1.414163in}}%
\pgfpathcurveto{\pgfqpoint{4.235795in}{1.421296in}}{\pgfqpoint{4.232961in}{1.428137in}}{\pgfqpoint{4.227917in}{1.433181in}}%
\pgfpathcurveto{\pgfqpoint{4.222873in}{1.438225in}}{\pgfqpoint{4.216032in}{1.441058in}}{\pgfqpoint{4.208899in}{1.441058in}}%
\pgfpathcurveto{\pgfqpoint{4.201766in}{1.441058in}}{\pgfqpoint{4.194924in}{1.438225in}}{\pgfqpoint{4.189881in}{1.433181in}}%
\pgfpathcurveto{\pgfqpoint{4.184837in}{1.428137in}}{\pgfqpoint{4.182003in}{1.421296in}}{\pgfqpoint{4.182003in}{1.414163in}}%
\pgfpathcurveto{\pgfqpoint{4.182003in}{1.407030in}}{\pgfqpoint{4.184837in}{1.400188in}}{\pgfqpoint{4.189881in}{1.395145in}}%
\pgfpathcurveto{\pgfqpoint{4.194924in}{1.390101in}}{\pgfqpoint{4.201766in}{1.387267in}}{\pgfqpoint{4.208899in}{1.387267in}}%
\pgfpathclose%
\pgfusepath{stroke,fill}%
\end{pgfscope}%
\begin{pgfscope}%
\pgfpathrectangle{\pgfqpoint{2.867647in}{0.500000in}}{\pgfqpoint{1.764706in}{1.700000in}}%
\pgfusepath{clip}%
\pgfsetbuttcap%
\pgfsetroundjoin%
\definecolor{currentfill}{rgb}{0.975644,0.874038,0.797253}%
\pgfsetfillcolor{currentfill}%
\pgfsetlinewidth{0.311001pt}%
\definecolor{currentstroke}{rgb}{1.000000,1.000000,1.000000}%
\pgfsetstrokecolor{currentstroke}%
\pgfsetdash{}{0pt}%
\pgfpathmoveto{\pgfqpoint{4.090193in}{1.559655in}}%
\pgfpathcurveto{\pgfqpoint{4.097326in}{1.559655in}}{\pgfqpoint{4.104167in}{1.562489in}}{\pgfqpoint{4.109211in}{1.567533in}}%
\pgfpathcurveto{\pgfqpoint{4.114255in}{1.572577in}}{\pgfqpoint{4.117089in}{1.579418in}}{\pgfqpoint{4.117089in}{1.586551in}}%
\pgfpathcurveto{\pgfqpoint{4.117089in}{1.593684in}}{\pgfqpoint{4.114255in}{1.600526in}}{\pgfqpoint{4.109211in}{1.605569in}}%
\pgfpathcurveto{\pgfqpoint{4.104167in}{1.610613in}}{\pgfqpoint{4.097326in}{1.613447in}}{\pgfqpoint{4.090193in}{1.613447in}}%
\pgfpathcurveto{\pgfqpoint{4.083060in}{1.613447in}}{\pgfqpoint{4.076218in}{1.610613in}}{\pgfqpoint{4.071175in}{1.605569in}}%
\pgfpathcurveto{\pgfqpoint{4.066131in}{1.600526in}}{\pgfqpoint{4.063297in}{1.593684in}}{\pgfqpoint{4.063297in}{1.586551in}}%
\pgfpathcurveto{\pgfqpoint{4.063297in}{1.579418in}}{\pgfqpoint{4.066131in}{1.572577in}}{\pgfqpoint{4.071175in}{1.567533in}}%
\pgfpathcurveto{\pgfqpoint{4.076218in}{1.562489in}}{\pgfqpoint{4.083060in}{1.559655in}}{\pgfqpoint{4.090193in}{1.559655in}}%
\pgfpathclose%
\pgfusepath{stroke,fill}%
\end{pgfscope}%
\begin{pgfscope}%
\pgfpathrectangle{\pgfqpoint{2.867647in}{0.500000in}}{\pgfqpoint{1.764706in}{1.700000in}}%
\pgfusepath{clip}%
\pgfsetbuttcap%
\pgfsetroundjoin%
\definecolor{currentfill}{rgb}{0.972201,0.839051,0.745789}%
\pgfsetfillcolor{currentfill}%
\pgfsetlinewidth{0.311001pt}%
\definecolor{currentstroke}{rgb}{1.000000,1.000000,1.000000}%
\pgfsetstrokecolor{currentstroke}%
\pgfsetdash{}{0pt}%
\pgfpathmoveto{\pgfqpoint{4.106305in}{1.212891in}}%
\pgfpathcurveto{\pgfqpoint{4.113438in}{1.212891in}}{\pgfqpoint{4.120280in}{1.215724in}}{\pgfqpoint{4.125324in}{1.220768in}}%
\pgfpathcurveto{\pgfqpoint{4.130367in}{1.225812in}}{\pgfqpoint{4.133201in}{1.232653in}}{\pgfqpoint{4.133201in}{1.239786in}}%
\pgfpathcurveto{\pgfqpoint{4.133201in}{1.246919in}}{\pgfqpoint{4.130367in}{1.253761in}}{\pgfqpoint{4.125324in}{1.258804in}}%
\pgfpathcurveto{\pgfqpoint{4.120280in}{1.263848in}}{\pgfqpoint{4.113438in}{1.266682in}}{\pgfqpoint{4.106305in}{1.266682in}}%
\pgfpathcurveto{\pgfqpoint{4.099173in}{1.266682in}}{\pgfqpoint{4.092331in}{1.263848in}}{\pgfqpoint{4.087287in}{1.258804in}}%
\pgfpathcurveto{\pgfqpoint{4.082244in}{1.253761in}}{\pgfqpoint{4.079410in}{1.246919in}}{\pgfqpoint{4.079410in}{1.239786in}}%
\pgfpathcurveto{\pgfqpoint{4.079410in}{1.232653in}}{\pgfqpoint{4.082244in}{1.225812in}}{\pgfqpoint{4.087287in}{1.220768in}}%
\pgfpathcurveto{\pgfqpoint{4.092331in}{1.215724in}}{\pgfqpoint{4.099173in}{1.212891in}}{\pgfqpoint{4.106305in}{1.212891in}}%
\pgfpathclose%
\pgfusepath{stroke,fill}%
\end{pgfscope}%
\begin{pgfscope}%
\pgfpathrectangle{\pgfqpoint{2.867647in}{0.500000in}}{\pgfqpoint{1.764706in}{1.700000in}}%
\pgfusepath{clip}%
\pgfsetbuttcap%
\pgfsetroundjoin%
\definecolor{currentfill}{rgb}{0.964920,0.695342,0.545192}%
\pgfsetfillcolor{currentfill}%
\pgfsetlinewidth{0.311001pt}%
\definecolor{currentstroke}{rgb}{1.000000,1.000000,1.000000}%
\pgfsetstrokecolor{currentstroke}%
\pgfsetdash{}{0pt}%
\pgfpathmoveto{\pgfqpoint{4.071280in}{1.343840in}}%
\pgfpathcurveto{\pgfqpoint{4.078412in}{1.343840in}}{\pgfqpoint{4.085254in}{1.346674in}}{\pgfqpoint{4.090298in}{1.351718in}}%
\pgfpathcurveto{\pgfqpoint{4.095341in}{1.356762in}}{\pgfqpoint{4.098175in}{1.363603in}}{\pgfqpoint{4.098175in}{1.370736in}}%
\pgfpathcurveto{\pgfqpoint{4.098175in}{1.377869in}}{\pgfqpoint{4.095341in}{1.384711in}}{\pgfqpoint{4.090298in}{1.389754in}}%
\pgfpathcurveto{\pgfqpoint{4.085254in}{1.394798in}}{\pgfqpoint{4.078412in}{1.397632in}}{\pgfqpoint{4.071280in}{1.397632in}}%
\pgfpathcurveto{\pgfqpoint{4.064147in}{1.397632in}}{\pgfqpoint{4.057305in}{1.394798in}}{\pgfqpoint{4.052261in}{1.389754in}}%
\pgfpathcurveto{\pgfqpoint{4.047218in}{1.384711in}}{\pgfqpoint{4.044384in}{1.377869in}}{\pgfqpoint{4.044384in}{1.370736in}}%
\pgfpathcurveto{\pgfqpoint{4.044384in}{1.363603in}}{\pgfqpoint{4.047218in}{1.356762in}}{\pgfqpoint{4.052261in}{1.351718in}}%
\pgfpathcurveto{\pgfqpoint{4.057305in}{1.346674in}}{\pgfqpoint{4.064147in}{1.343840in}}{\pgfqpoint{4.071280in}{1.343840in}}%
\pgfpathclose%
\pgfusepath{stroke,fill}%
\end{pgfscope}%
\begin{pgfscope}%
\pgfpathrectangle{\pgfqpoint{2.867647in}{0.500000in}}{\pgfqpoint{1.764706in}{1.700000in}}%
\pgfusepath{clip}%
\pgfsetbuttcap%
\pgfsetroundjoin%
\definecolor{currentfill}{rgb}{0.979891,0.908948,0.848279}%
\pgfsetfillcolor{currentfill}%
\pgfsetlinewidth{0.311001pt}%
\definecolor{currentstroke}{rgb}{1.000000,1.000000,1.000000}%
\pgfsetstrokecolor{currentstroke}%
\pgfsetdash{}{0pt}%
\pgfpathmoveto{\pgfqpoint{4.147965in}{1.193964in}}%
\pgfpathcurveto{\pgfqpoint{4.155098in}{1.193964in}}{\pgfqpoint{4.161940in}{1.196798in}}{\pgfqpoint{4.166983in}{1.201842in}}%
\pgfpathcurveto{\pgfqpoint{4.172027in}{1.206886in}}{\pgfqpoint{4.174861in}{1.213727in}}{\pgfqpoint{4.174861in}{1.220860in}}%
\pgfpathcurveto{\pgfqpoint{4.174861in}{1.227993in}}{\pgfqpoint{4.172027in}{1.234835in}}{\pgfqpoint{4.166983in}{1.239878in}}%
\pgfpathcurveto{\pgfqpoint{4.161940in}{1.244922in}}{\pgfqpoint{4.155098in}{1.247756in}}{\pgfqpoint{4.147965in}{1.247756in}}%
\pgfpathcurveto{\pgfqpoint{4.140833in}{1.247756in}}{\pgfqpoint{4.133991in}{1.244922in}}{\pgfqpoint{4.128947in}{1.239878in}}%
\pgfpathcurveto{\pgfqpoint{4.123904in}{1.234835in}}{\pgfqpoint{4.121070in}{1.227993in}}{\pgfqpoint{4.121070in}{1.220860in}}%
\pgfpathcurveto{\pgfqpoint{4.121070in}{1.213727in}}{\pgfqpoint{4.123904in}{1.206886in}}{\pgfqpoint{4.128947in}{1.201842in}}%
\pgfpathcurveto{\pgfqpoint{4.133991in}{1.196798in}}{\pgfqpoint{4.140833in}{1.193964in}}{\pgfqpoint{4.147965in}{1.193964in}}%
\pgfpathclose%
\pgfusepath{stroke,fill}%
\end{pgfscope}%
\begin{pgfscope}%
\pgfpathrectangle{\pgfqpoint{2.867647in}{0.500000in}}{\pgfqpoint{1.764706in}{1.700000in}}%
\pgfusepath{clip}%
\pgfsetbuttcap%
\pgfsetroundjoin%
\definecolor{currentfill}{rgb}{0.960043,0.546576,0.387029}%
\pgfsetfillcolor{currentfill}%
\pgfsetlinewidth{0.311001pt}%
\definecolor{currentstroke}{rgb}{1.000000,1.000000,1.000000}%
\pgfsetstrokecolor{currentstroke}%
\pgfsetdash{}{0pt}%
\pgfpathmoveto{\pgfqpoint{4.308553in}{1.099610in}}%
\pgfpathcurveto{\pgfqpoint{4.315685in}{1.099610in}}{\pgfqpoint{4.322527in}{1.102444in}}{\pgfqpoint{4.327571in}{1.107488in}}%
\pgfpathcurveto{\pgfqpoint{4.332614in}{1.112531in}}{\pgfqpoint{4.335448in}{1.119373in}}{\pgfqpoint{4.335448in}{1.126506in}}%
\pgfpathcurveto{\pgfqpoint{4.335448in}{1.133639in}}{\pgfqpoint{4.332614in}{1.140480in}}{\pgfqpoint{4.327571in}{1.145524in}}%
\pgfpathcurveto{\pgfqpoint{4.322527in}{1.150568in}}{\pgfqpoint{4.315685in}{1.153402in}}{\pgfqpoint{4.308553in}{1.153402in}}%
\pgfpathcurveto{\pgfqpoint{4.301420in}{1.153402in}}{\pgfqpoint{4.294578in}{1.150568in}}{\pgfqpoint{4.289534in}{1.145524in}}%
\pgfpathcurveto{\pgfqpoint{4.284491in}{1.140480in}}{\pgfqpoint{4.281657in}{1.133639in}}{\pgfqpoint{4.281657in}{1.126506in}}%
\pgfpathcurveto{\pgfqpoint{4.281657in}{1.119373in}}{\pgfqpoint{4.284491in}{1.112531in}}{\pgfqpoint{4.289534in}{1.107488in}}%
\pgfpathcurveto{\pgfqpoint{4.294578in}{1.102444in}}{\pgfqpoint{4.301420in}{1.099610in}}{\pgfqpoint{4.308553in}{1.099610in}}%
\pgfpathclose%
\pgfusepath{stroke,fill}%
\end{pgfscope}%
\begin{pgfscope}%
\pgfpathrectangle{\pgfqpoint{2.867647in}{0.500000in}}{\pgfqpoint{1.764706in}{1.700000in}}%
\pgfusepath{clip}%
\pgfsetbuttcap%
\pgfsetroundjoin%
\definecolor{currentfill}{rgb}{0.979124,0.903132,0.839793}%
\pgfsetfillcolor{currentfill}%
\pgfsetlinewidth{0.311001pt}%
\definecolor{currentstroke}{rgb}{1.000000,1.000000,1.000000}%
\pgfsetstrokecolor{currentstroke}%
\pgfsetdash{}{0pt}%
\pgfpathmoveto{\pgfqpoint{4.227222in}{1.302836in}}%
\pgfpathcurveto{\pgfqpoint{4.234355in}{1.302836in}}{\pgfqpoint{4.241197in}{1.305670in}}{\pgfqpoint{4.246241in}{1.310713in}}%
\pgfpathcurveto{\pgfqpoint{4.251284in}{1.315757in}}{\pgfqpoint{4.254118in}{1.322599in}}{\pgfqpoint{4.254118in}{1.329731in}}%
\pgfpathcurveto{\pgfqpoint{4.254118in}{1.336864in}}{\pgfqpoint{4.251284in}{1.343706in}}{\pgfqpoint{4.246241in}{1.348749in}}%
\pgfpathcurveto{\pgfqpoint{4.241197in}{1.353793in}}{\pgfqpoint{4.234355in}{1.356627in}}{\pgfqpoint{4.227222in}{1.356627in}}%
\pgfpathcurveto{\pgfqpoint{4.220090in}{1.356627in}}{\pgfqpoint{4.213248in}{1.353793in}}{\pgfqpoint{4.208204in}{1.348749in}}%
\pgfpathcurveto{\pgfqpoint{4.203161in}{1.343706in}}{\pgfqpoint{4.200327in}{1.336864in}}{\pgfqpoint{4.200327in}{1.329731in}}%
\pgfpathcurveto{\pgfqpoint{4.200327in}{1.322599in}}{\pgfqpoint{4.203161in}{1.315757in}}{\pgfqpoint{4.208204in}{1.310713in}}%
\pgfpathcurveto{\pgfqpoint{4.213248in}{1.305670in}}{\pgfqpoint{4.220090in}{1.302836in}}{\pgfqpoint{4.227222in}{1.302836in}}%
\pgfpathclose%
\pgfusepath{stroke,fill}%
\end{pgfscope}%
\begin{pgfscope}%
\pgfpathrectangle{\pgfqpoint{2.867647in}{0.500000in}}{\pgfqpoint{1.764706in}{1.700000in}}%
\pgfusepath{clip}%
\pgfsetbuttcap%
\pgfsetroundjoin%
\definecolor{currentfill}{rgb}{0.981377,0.920617,0.865369}%
\pgfsetfillcolor{currentfill}%
\pgfsetlinewidth{0.311001pt}%
\definecolor{currentstroke}{rgb}{1.000000,1.000000,1.000000}%
\pgfsetstrokecolor{currentstroke}%
\pgfsetdash{}{0pt}%
\pgfpathmoveto{\pgfqpoint{4.181800in}{1.285264in}}%
\pgfpathcurveto{\pgfqpoint{4.188933in}{1.285264in}}{\pgfqpoint{4.195775in}{1.288098in}}{\pgfqpoint{4.200818in}{1.293142in}}%
\pgfpathcurveto{\pgfqpoint{4.205862in}{1.298185in}}{\pgfqpoint{4.208696in}{1.305027in}}{\pgfqpoint{4.208696in}{1.312160in}}%
\pgfpathcurveto{\pgfqpoint{4.208696in}{1.319293in}}{\pgfqpoint{4.205862in}{1.326134in}}{\pgfqpoint{4.200818in}{1.331178in}}%
\pgfpathcurveto{\pgfqpoint{4.195775in}{1.336222in}}{\pgfqpoint{4.188933in}{1.339056in}}{\pgfqpoint{4.181800in}{1.339056in}}%
\pgfpathcurveto{\pgfqpoint{4.174667in}{1.339056in}}{\pgfqpoint{4.167826in}{1.336222in}}{\pgfqpoint{4.162782in}{1.331178in}}%
\pgfpathcurveto{\pgfqpoint{4.157738in}{1.326134in}}{\pgfqpoint{4.154904in}{1.319293in}}{\pgfqpoint{4.154904in}{1.312160in}}%
\pgfpathcurveto{\pgfqpoint{4.154904in}{1.305027in}}{\pgfqpoint{4.157738in}{1.298185in}}{\pgfqpoint{4.162782in}{1.293142in}}%
\pgfpathcurveto{\pgfqpoint{4.167826in}{1.288098in}}{\pgfqpoint{4.174667in}{1.285264in}}{\pgfqpoint{4.181800in}{1.285264in}}%
\pgfpathclose%
\pgfusepath{stroke,fill}%
\end{pgfscope}%
\begin{pgfscope}%
\pgfpathrectangle{\pgfqpoint{2.867647in}{0.500000in}}{\pgfqpoint{1.764706in}{1.700000in}}%
\pgfusepath{clip}%
\pgfsetbuttcap%
\pgfsetroundjoin%
\definecolor{currentfill}{rgb}{0.968105,0.786346,0.667739}%
\pgfsetfillcolor{currentfill}%
\pgfsetlinewidth{0.311001pt}%
\definecolor{currentstroke}{rgb}{1.000000,1.000000,1.000000}%
\pgfsetstrokecolor{currentstroke}%
\pgfsetdash{}{0pt}%
\pgfpathmoveto{\pgfqpoint{4.030246in}{1.614249in}}%
\pgfpathcurveto{\pgfqpoint{4.037379in}{1.614249in}}{\pgfqpoint{4.044220in}{1.617083in}}{\pgfqpoint{4.049264in}{1.622127in}}%
\pgfpathcurveto{\pgfqpoint{4.054308in}{1.627170in}}{\pgfqpoint{4.057141in}{1.634012in}}{\pgfqpoint{4.057141in}{1.641145in}}%
\pgfpathcurveto{\pgfqpoint{4.057141in}{1.648278in}}{\pgfqpoint{4.054308in}{1.655119in}}{\pgfqpoint{4.049264in}{1.660163in}}%
\pgfpathcurveto{\pgfqpoint{4.044220in}{1.665207in}}{\pgfqpoint{4.037379in}{1.668041in}}{\pgfqpoint{4.030246in}{1.668041in}}%
\pgfpathcurveto{\pgfqpoint{4.023113in}{1.668041in}}{\pgfqpoint{4.016271in}{1.665207in}}{\pgfqpoint{4.011228in}{1.660163in}}%
\pgfpathcurveto{\pgfqpoint{4.006184in}{1.655119in}}{\pgfqpoint{4.003350in}{1.648278in}}{\pgfqpoint{4.003350in}{1.641145in}}%
\pgfpathcurveto{\pgfqpoint{4.003350in}{1.634012in}}{\pgfqpoint{4.006184in}{1.627170in}}{\pgfqpoint{4.011228in}{1.622127in}}%
\pgfpathcurveto{\pgfqpoint{4.016271in}{1.617083in}}{\pgfqpoint{4.023113in}{1.614249in}}{\pgfqpoint{4.030246in}{1.614249in}}%
\pgfpathclose%
\pgfusepath{stroke,fill}%
\end{pgfscope}%
\begin{pgfscope}%
\pgfpathrectangle{\pgfqpoint{2.867647in}{0.500000in}}{\pgfqpoint{1.764706in}{1.700000in}}%
\pgfusepath{clip}%
\pgfsetbuttcap%
\pgfsetroundjoin%
\definecolor{currentfill}{rgb}{0.946260,0.398132,0.274897}%
\pgfsetfillcolor{currentfill}%
\pgfsetlinewidth{0.311001pt}%
\definecolor{currentstroke}{rgb}{1.000000,1.000000,1.000000}%
\pgfsetstrokecolor{currentstroke}%
\pgfsetdash{}{0pt}%
\pgfpathmoveto{\pgfqpoint{3.879677in}{0.934242in}}%
\pgfpathcurveto{\pgfqpoint{3.886810in}{0.934242in}}{\pgfqpoint{3.893651in}{0.937076in}}{\pgfqpoint{3.898695in}{0.942120in}}%
\pgfpathcurveto{\pgfqpoint{3.903739in}{0.947164in}}{\pgfqpoint{3.906572in}{0.954005in}}{\pgfqpoint{3.906572in}{0.961138in}}%
\pgfpathcurveto{\pgfqpoint{3.906572in}{0.968271in}}{\pgfqpoint{3.903739in}{0.975113in}}{\pgfqpoint{3.898695in}{0.980156in}}%
\pgfpathcurveto{\pgfqpoint{3.893651in}{0.985200in}}{\pgfqpoint{3.886810in}{0.988034in}}{\pgfqpoint{3.879677in}{0.988034in}}%
\pgfpathcurveto{\pgfqpoint{3.872544in}{0.988034in}}{\pgfqpoint{3.865702in}{0.985200in}}{\pgfqpoint{3.860659in}{0.980156in}}%
\pgfpathcurveto{\pgfqpoint{3.855615in}{0.975113in}}{\pgfqpoint{3.852781in}{0.968271in}}{\pgfqpoint{3.852781in}{0.961138in}}%
\pgfpathcurveto{\pgfqpoint{3.852781in}{0.954005in}}{\pgfqpoint{3.855615in}{0.947164in}}{\pgfqpoint{3.860659in}{0.942120in}}%
\pgfpathcurveto{\pgfqpoint{3.865702in}{0.937076in}}{\pgfqpoint{3.872544in}{0.934242in}}{\pgfqpoint{3.879677in}{0.934242in}}%
\pgfpathclose%
\pgfusepath{stroke,fill}%
\end{pgfscope}%
\begin{pgfscope}%
\pgfpathrectangle{\pgfqpoint{2.867647in}{0.500000in}}{\pgfqpoint{1.764706in}{1.700000in}}%
\pgfusepath{clip}%
\pgfsetbuttcap%
\pgfsetroundjoin%
\definecolor{currentfill}{rgb}{0.973832,0.856556,0.771584}%
\pgfsetfillcolor{currentfill}%
\pgfsetlinewidth{0.311001pt}%
\definecolor{currentstroke}{rgb}{1.000000,1.000000,1.000000}%
\pgfsetstrokecolor{currentstroke}%
\pgfsetdash{}{0pt}%
\pgfpathmoveto{\pgfqpoint{4.125207in}{1.286213in}}%
\pgfpathcurveto{\pgfqpoint{4.132339in}{1.286213in}}{\pgfqpoint{4.139181in}{1.289047in}}{\pgfqpoint{4.144225in}{1.294090in}}%
\pgfpathcurveto{\pgfqpoint{4.149268in}{1.299134in}}{\pgfqpoint{4.152102in}{1.305976in}}{\pgfqpoint{4.152102in}{1.313109in}}%
\pgfpathcurveto{\pgfqpoint{4.152102in}{1.320241in}}{\pgfqpoint{4.149268in}{1.327083in}}{\pgfqpoint{4.144225in}{1.332127in}}%
\pgfpathcurveto{\pgfqpoint{4.139181in}{1.337170in}}{\pgfqpoint{4.132339in}{1.340004in}}{\pgfqpoint{4.125207in}{1.340004in}}%
\pgfpathcurveto{\pgfqpoint{4.118074in}{1.340004in}}{\pgfqpoint{4.111232in}{1.337170in}}{\pgfqpoint{4.106188in}{1.332127in}}%
\pgfpathcurveto{\pgfqpoint{4.101145in}{1.327083in}}{\pgfqpoint{4.098311in}{1.320241in}}{\pgfqpoint{4.098311in}{1.313109in}}%
\pgfpathcurveto{\pgfqpoint{4.098311in}{1.305976in}}{\pgfqpoint{4.101145in}{1.299134in}}{\pgfqpoint{4.106188in}{1.294090in}}%
\pgfpathcurveto{\pgfqpoint{4.111232in}{1.289047in}}{\pgfqpoint{4.118074in}{1.286213in}}{\pgfqpoint{4.125207in}{1.286213in}}%
\pgfpathclose%
\pgfusepath{stroke,fill}%
\end{pgfscope}%
\begin{pgfscope}%
\pgfpathrectangle{\pgfqpoint{2.867647in}{0.500000in}}{\pgfqpoint{1.764706in}{1.700000in}}%
\pgfusepath{clip}%
\pgfsetbuttcap%
\pgfsetroundjoin%
\definecolor{currentfill}{rgb}{0.959645,0.539840,0.380928}%
\pgfsetfillcolor{currentfill}%
\pgfsetlinewidth{0.311001pt}%
\definecolor{currentstroke}{rgb}{1.000000,1.000000,1.000000}%
\pgfsetstrokecolor{currentstroke}%
\pgfsetdash{}{0pt}%
\pgfpathmoveto{\pgfqpoint{4.024529in}{1.216757in}}%
\pgfpathcurveto{\pgfqpoint{4.031662in}{1.216757in}}{\pgfqpoint{4.038503in}{1.219591in}}{\pgfqpoint{4.043547in}{1.224634in}}%
\pgfpathcurveto{\pgfqpoint{4.048591in}{1.229678in}}{\pgfqpoint{4.051425in}{1.236520in}}{\pgfqpoint{4.051425in}{1.243653in}}%
\pgfpathcurveto{\pgfqpoint{4.051425in}{1.250785in}}{\pgfqpoint{4.048591in}{1.257627in}}{\pgfqpoint{4.043547in}{1.262671in}}%
\pgfpathcurveto{\pgfqpoint{4.038503in}{1.267714in}}{\pgfqpoint{4.031662in}{1.270548in}}{\pgfqpoint{4.024529in}{1.270548in}}%
\pgfpathcurveto{\pgfqpoint{4.017396in}{1.270548in}}{\pgfqpoint{4.010554in}{1.267714in}}{\pgfqpoint{4.005511in}{1.262671in}}%
\pgfpathcurveto{\pgfqpoint{4.000467in}{1.257627in}}{\pgfqpoint{3.997633in}{1.250785in}}{\pgfqpoint{3.997633in}{1.243653in}}%
\pgfpathcurveto{\pgfqpoint{3.997633in}{1.236520in}}{\pgfqpoint{4.000467in}{1.229678in}}{\pgfqpoint{4.005511in}{1.224634in}}%
\pgfpathcurveto{\pgfqpoint{4.010554in}{1.219591in}}{\pgfqpoint{4.017396in}{1.216757in}}{\pgfqpoint{4.024529in}{1.216757in}}%
\pgfpathclose%
\pgfusepath{stroke,fill}%
\end{pgfscope}%
\begin{pgfscope}%
\pgfpathrectangle{\pgfqpoint{2.867647in}{0.500000in}}{\pgfqpoint{1.764706in}{1.700000in}}%
\pgfusepath{clip}%
\pgfsetbuttcap%
\pgfsetroundjoin%
\definecolor{currentfill}{rgb}{0.965302,0.713942,0.568499}%
\pgfsetfillcolor{currentfill}%
\pgfsetlinewidth{0.311001pt}%
\definecolor{currentstroke}{rgb}{1.000000,1.000000,1.000000}%
\pgfsetstrokecolor{currentstroke}%
\pgfsetdash{}{0pt}%
\pgfpathmoveto{\pgfqpoint{4.091484in}{0.897594in}}%
\pgfpathcurveto{\pgfqpoint{4.098617in}{0.897594in}}{\pgfqpoint{4.105459in}{0.900428in}}{\pgfqpoint{4.110502in}{0.905472in}}%
\pgfpathcurveto{\pgfqpoint{4.115546in}{0.910515in}}{\pgfqpoint{4.118380in}{0.917357in}}{\pgfqpoint{4.118380in}{0.924490in}}%
\pgfpathcurveto{\pgfqpoint{4.118380in}{0.931623in}}{\pgfqpoint{4.115546in}{0.938464in}}{\pgfqpoint{4.110502in}{0.943508in}}%
\pgfpathcurveto{\pgfqpoint{4.105459in}{0.948552in}}{\pgfqpoint{4.098617in}{0.951386in}}{\pgfqpoint{4.091484in}{0.951386in}}%
\pgfpathcurveto{\pgfqpoint{4.084351in}{0.951386in}}{\pgfqpoint{4.077510in}{0.948552in}}{\pgfqpoint{4.072466in}{0.943508in}}%
\pgfpathcurveto{\pgfqpoint{4.067422in}{0.938464in}}{\pgfqpoint{4.064589in}{0.931623in}}{\pgfqpoint{4.064589in}{0.924490in}}%
\pgfpathcurveto{\pgfqpoint{4.064589in}{0.917357in}}{\pgfqpoint{4.067422in}{0.910515in}}{\pgfqpoint{4.072466in}{0.905472in}}%
\pgfpathcurveto{\pgfqpoint{4.077510in}{0.900428in}}{\pgfqpoint{4.084351in}{0.897594in}}{\pgfqpoint{4.091484in}{0.897594in}}%
\pgfpathclose%
\pgfusepath{stroke,fill}%
\end{pgfscope}%
\begin{pgfscope}%
\pgfpathrectangle{\pgfqpoint{2.867647in}{0.500000in}}{\pgfqpoint{1.764706in}{1.700000in}}%
\pgfusepath{clip}%
\pgfsetbuttcap%
\pgfsetroundjoin%
\definecolor{currentfill}{rgb}{0.980678,0.914765,0.856766}%
\pgfsetfillcolor{currentfill}%
\pgfsetlinewidth{0.311001pt}%
\definecolor{currentstroke}{rgb}{1.000000,1.000000,1.000000}%
\pgfsetstrokecolor{currentstroke}%
\pgfsetdash{}{0pt}%
\pgfpathmoveto{\pgfqpoint{4.185669in}{1.410184in}}%
\pgfpathcurveto{\pgfqpoint{4.192801in}{1.410184in}}{\pgfqpoint{4.199643in}{1.413018in}}{\pgfqpoint{4.204687in}{1.418061in}}%
\pgfpathcurveto{\pgfqpoint{4.209730in}{1.423105in}}{\pgfqpoint{4.212564in}{1.429947in}}{\pgfqpoint{4.212564in}{1.437080in}}%
\pgfpathcurveto{\pgfqpoint{4.212564in}{1.444212in}}{\pgfqpoint{4.209730in}{1.451054in}}{\pgfqpoint{4.204687in}{1.456098in}}%
\pgfpathcurveto{\pgfqpoint{4.199643in}{1.461141in}}{\pgfqpoint{4.192801in}{1.463975in}}{\pgfqpoint{4.185669in}{1.463975in}}%
\pgfpathcurveto{\pgfqpoint{4.178536in}{1.463975in}}{\pgfqpoint{4.171694in}{1.461141in}}{\pgfqpoint{4.166650in}{1.456098in}}%
\pgfpathcurveto{\pgfqpoint{4.161607in}{1.451054in}}{\pgfqpoint{4.158773in}{1.444212in}}{\pgfqpoint{4.158773in}{1.437080in}}%
\pgfpathcurveto{\pgfqpoint{4.158773in}{1.429947in}}{\pgfqpoint{4.161607in}{1.423105in}}{\pgfqpoint{4.166650in}{1.418061in}}%
\pgfpathcurveto{\pgfqpoint{4.171694in}{1.413018in}}{\pgfqpoint{4.178536in}{1.410184in}}{\pgfqpoint{4.185669in}{1.410184in}}%
\pgfpathclose%
\pgfusepath{stroke,fill}%
\end{pgfscope}%
\begin{pgfscope}%
\pgfpathrectangle{\pgfqpoint{2.867647in}{0.500000in}}{\pgfqpoint{1.764706in}{1.700000in}}%
\pgfusepath{clip}%
\pgfsetbuttcap%
\pgfsetroundjoin%
\definecolor{currentfill}{rgb}{0.963728,0.638439,0.479050}%
\pgfsetfillcolor{currentfill}%
\pgfsetlinewidth{0.311001pt}%
\definecolor{currentstroke}{rgb}{1.000000,1.000000,1.000000}%
\pgfsetstrokecolor{currentstroke}%
\pgfsetdash{}{0pt}%
\pgfpathmoveto{\pgfqpoint{4.198970in}{0.942693in}}%
\pgfpathcurveto{\pgfqpoint{4.206102in}{0.942693in}}{\pgfqpoint{4.212944in}{0.945526in}}{\pgfqpoint{4.217988in}{0.950570in}}%
\pgfpathcurveto{\pgfqpoint{4.223031in}{0.955614in}}{\pgfqpoint{4.225865in}{0.962455in}}{\pgfqpoint{4.225865in}{0.969588in}}%
\pgfpathcurveto{\pgfqpoint{4.225865in}{0.976721in}}{\pgfqpoint{4.223031in}{0.983563in}}{\pgfqpoint{4.217988in}{0.988606in}}%
\pgfpathcurveto{\pgfqpoint{4.212944in}{0.993650in}}{\pgfqpoint{4.206102in}{0.996484in}}{\pgfqpoint{4.198970in}{0.996484in}}%
\pgfpathcurveto{\pgfqpoint{4.191837in}{0.996484in}}{\pgfqpoint{4.184995in}{0.993650in}}{\pgfqpoint{4.179952in}{0.988606in}}%
\pgfpathcurveto{\pgfqpoint{4.174908in}{0.983563in}}{\pgfqpoint{4.172074in}{0.976721in}}{\pgfqpoint{4.172074in}{0.969588in}}%
\pgfpathcurveto{\pgfqpoint{4.172074in}{0.962455in}}{\pgfqpoint{4.174908in}{0.955614in}}{\pgfqpoint{4.179952in}{0.950570in}}%
\pgfpathcurveto{\pgfqpoint{4.184995in}{0.945526in}}{\pgfqpoint{4.191837in}{0.942693in}}{\pgfqpoint{4.198970in}{0.942693in}}%
\pgfpathclose%
\pgfusepath{stroke,fill}%
\end{pgfscope}%
\begin{pgfscope}%
\pgfpathrectangle{\pgfqpoint{2.867647in}{0.500000in}}{\pgfqpoint{1.764706in}{1.700000in}}%
\pgfusepath{clip}%
\pgfsetbuttcap%
\pgfsetroundjoin%
\definecolor{currentfill}{rgb}{0.981377,0.920617,0.865369}%
\pgfsetfillcolor{currentfill}%
\pgfsetlinewidth{0.311001pt}%
\definecolor{currentstroke}{rgb}{1.000000,1.000000,1.000000}%
\pgfsetstrokecolor{currentstroke}%
\pgfsetdash{}{0pt}%
\pgfpathmoveto{\pgfqpoint{4.170014in}{1.258471in}}%
\pgfpathcurveto{\pgfqpoint{4.177147in}{1.258471in}}{\pgfqpoint{4.183989in}{1.261305in}}{\pgfqpoint{4.189032in}{1.266349in}}%
\pgfpathcurveto{\pgfqpoint{4.194076in}{1.271393in}}{\pgfqpoint{4.196910in}{1.278234in}}{\pgfqpoint{4.196910in}{1.285367in}}%
\pgfpathcurveto{\pgfqpoint{4.196910in}{1.292500in}}{\pgfqpoint{4.194076in}{1.299342in}}{\pgfqpoint{4.189032in}{1.304385in}}%
\pgfpathcurveto{\pgfqpoint{4.183989in}{1.309429in}}{\pgfqpoint{4.177147in}{1.312263in}}{\pgfqpoint{4.170014in}{1.312263in}}%
\pgfpathcurveto{\pgfqpoint{4.162881in}{1.312263in}}{\pgfqpoint{4.156040in}{1.309429in}}{\pgfqpoint{4.150996in}{1.304385in}}%
\pgfpathcurveto{\pgfqpoint{4.145952in}{1.299342in}}{\pgfqpoint{4.143118in}{1.292500in}}{\pgfqpoint{4.143118in}{1.285367in}}%
\pgfpathcurveto{\pgfqpoint{4.143118in}{1.278234in}}{\pgfqpoint{4.145952in}{1.271393in}}{\pgfqpoint{4.150996in}{1.266349in}}%
\pgfpathcurveto{\pgfqpoint{4.156040in}{1.261305in}}{\pgfqpoint{4.162881in}{1.258471in}}{\pgfqpoint{4.170014in}{1.258471in}}%
\pgfpathclose%
\pgfusepath{stroke,fill}%
\end{pgfscope}%
\begin{pgfscope}%
\pgfpathrectangle{\pgfqpoint{2.867647in}{0.500000in}}{\pgfqpoint{1.764706in}{1.700000in}}%
\pgfusepath{clip}%
\pgfsetbuttcap%
\pgfsetroundjoin%
\definecolor{currentfill}{rgb}{0.976961,0.885681,0.814303}%
\pgfsetfillcolor{currentfill}%
\pgfsetlinewidth{0.311001pt}%
\definecolor{currentstroke}{rgb}{1.000000,1.000000,1.000000}%
\pgfsetstrokecolor{currentstroke}%
\pgfsetdash{}{0pt}%
\pgfpathmoveto{\pgfqpoint{4.114301in}{1.122090in}}%
\pgfpathcurveto{\pgfqpoint{4.121433in}{1.122090in}}{\pgfqpoint{4.128275in}{1.124923in}}{\pgfqpoint{4.133319in}{1.129967in}}%
\pgfpathcurveto{\pgfqpoint{4.138362in}{1.135011in}}{\pgfqpoint{4.141196in}{1.141852in}}{\pgfqpoint{4.141196in}{1.148985in}}%
\pgfpathcurveto{\pgfqpoint{4.141196in}{1.156118in}}{\pgfqpoint{4.138362in}{1.162960in}}{\pgfqpoint{4.133319in}{1.168003in}}%
\pgfpathcurveto{\pgfqpoint{4.128275in}{1.173047in}}{\pgfqpoint{4.121433in}{1.175881in}}{\pgfqpoint{4.114301in}{1.175881in}}%
\pgfpathcurveto{\pgfqpoint{4.107168in}{1.175881in}}{\pgfqpoint{4.100326in}{1.173047in}}{\pgfqpoint{4.095283in}{1.168003in}}%
\pgfpathcurveto{\pgfqpoint{4.090239in}{1.162960in}}{\pgfqpoint{4.087405in}{1.156118in}}{\pgfqpoint{4.087405in}{1.148985in}}%
\pgfpathcurveto{\pgfqpoint{4.087405in}{1.141852in}}{\pgfqpoint{4.090239in}{1.135011in}}{\pgfqpoint{4.095283in}{1.129967in}}%
\pgfpathcurveto{\pgfqpoint{4.100326in}{1.124923in}}{\pgfqpoint{4.107168in}{1.122090in}}{\pgfqpoint{4.114301in}{1.122090in}}%
\pgfpathclose%
\pgfusepath{stroke,fill}%
\end{pgfscope}%
\begin{pgfscope}%
\pgfpathrectangle{\pgfqpoint{2.867647in}{0.500000in}}{\pgfqpoint{1.764706in}{1.700000in}}%
\pgfusepath{clip}%
\pgfsetbuttcap%
\pgfsetroundjoin%
\definecolor{currentfill}{rgb}{0.966120,0.744512,0.608720}%
\pgfsetfillcolor{currentfill}%
\pgfsetlinewidth{0.311001pt}%
\definecolor{currentstroke}{rgb}{1.000000,1.000000,1.000000}%
\pgfsetstrokecolor{currentstroke}%
\pgfsetdash{}{0pt}%
\pgfpathmoveto{\pgfqpoint{4.151229in}{0.940338in}}%
\pgfpathcurveto{\pgfqpoint{4.158361in}{0.940338in}}{\pgfqpoint{4.165203in}{0.943172in}}{\pgfqpoint{4.170247in}{0.948216in}}%
\pgfpathcurveto{\pgfqpoint{4.175290in}{0.953260in}}{\pgfqpoint{4.178124in}{0.960101in}}{\pgfqpoint{4.178124in}{0.967234in}}%
\pgfpathcurveto{\pgfqpoint{4.178124in}{0.974367in}}{\pgfqpoint{4.175290in}{0.981208in}}{\pgfqpoint{4.170247in}{0.986252in}}%
\pgfpathcurveto{\pgfqpoint{4.165203in}{0.991296in}}{\pgfqpoint{4.158361in}{0.994130in}}{\pgfqpoint{4.151229in}{0.994130in}}%
\pgfpathcurveto{\pgfqpoint{4.144096in}{0.994130in}}{\pgfqpoint{4.137254in}{0.991296in}}{\pgfqpoint{4.132210in}{0.986252in}}%
\pgfpathcurveto{\pgfqpoint{4.127167in}{0.981208in}}{\pgfqpoint{4.124333in}{0.974367in}}{\pgfqpoint{4.124333in}{0.967234in}}%
\pgfpathcurveto{\pgfqpoint{4.124333in}{0.960101in}}{\pgfqpoint{4.127167in}{0.953260in}}{\pgfqpoint{4.132210in}{0.948216in}}%
\pgfpathcurveto{\pgfqpoint{4.137254in}{0.943172in}}{\pgfqpoint{4.144096in}{0.940338in}}{\pgfqpoint{4.151229in}{0.940338in}}%
\pgfpathclose%
\pgfusepath{stroke,fill}%
\end{pgfscope}%
\begin{pgfscope}%
\pgfpathrectangle{\pgfqpoint{2.867647in}{0.500000in}}{\pgfqpoint{1.764706in}{1.700000in}}%
\pgfusepath{clip}%
\pgfsetbuttcap%
\pgfsetroundjoin%
\definecolor{currentfill}{rgb}{0.966328,0.750560,0.616961}%
\pgfsetfillcolor{currentfill}%
\pgfsetlinewidth{0.311001pt}%
\definecolor{currentstroke}{rgb}{1.000000,1.000000,1.000000}%
\pgfsetstrokecolor{currentstroke}%
\pgfsetdash{}{0pt}%
\pgfpathmoveto{\pgfqpoint{4.230755in}{1.036954in}}%
\pgfpathcurveto{\pgfqpoint{4.237888in}{1.036954in}}{\pgfqpoint{4.244730in}{1.039788in}}{\pgfqpoint{4.249774in}{1.044832in}}%
\pgfpathcurveto{\pgfqpoint{4.254817in}{1.049876in}}{\pgfqpoint{4.257651in}{1.056717in}}{\pgfqpoint{4.257651in}{1.063850in}}%
\pgfpathcurveto{\pgfqpoint{4.257651in}{1.070983in}}{\pgfqpoint{4.254817in}{1.077825in}}{\pgfqpoint{4.249774in}{1.082868in}}%
\pgfpathcurveto{\pgfqpoint{4.244730in}{1.087912in}}{\pgfqpoint{4.237888in}{1.090746in}}{\pgfqpoint{4.230755in}{1.090746in}}%
\pgfpathcurveto{\pgfqpoint{4.223623in}{1.090746in}}{\pgfqpoint{4.216781in}{1.087912in}}{\pgfqpoint{4.211737in}{1.082868in}}%
\pgfpathcurveto{\pgfqpoint{4.206694in}{1.077825in}}{\pgfqpoint{4.203860in}{1.070983in}}{\pgfqpoint{4.203860in}{1.063850in}}%
\pgfpathcurveto{\pgfqpoint{4.203860in}{1.056717in}}{\pgfqpoint{4.206694in}{1.049876in}}{\pgfqpoint{4.211737in}{1.044832in}}%
\pgfpathcurveto{\pgfqpoint{4.216781in}{1.039788in}}{\pgfqpoint{4.223623in}{1.036954in}}{\pgfqpoint{4.230755in}{1.036954in}}%
\pgfpathclose%
\pgfusepath{stroke,fill}%
\end{pgfscope}%
\begin{pgfscope}%
\pgfpathrectangle{\pgfqpoint{2.867647in}{0.500000in}}{\pgfqpoint{1.764706in}{1.700000in}}%
\pgfusepath{clip}%
\pgfsetbuttcap%
\pgfsetroundjoin%
\definecolor{currentfill}{rgb}{0.981377,0.920617,0.865369}%
\pgfsetfillcolor{currentfill}%
\pgfsetlinewidth{0.311001pt}%
\definecolor{currentstroke}{rgb}{1.000000,1.000000,1.000000}%
\pgfsetstrokecolor{currentstroke}%
\pgfsetdash{}{0pt}%
\pgfpathmoveto{\pgfqpoint{4.201593in}{1.255883in}}%
\pgfpathcurveto{\pgfqpoint{4.208726in}{1.255883in}}{\pgfqpoint{4.215568in}{1.258717in}}{\pgfqpoint{4.220611in}{1.263761in}}%
\pgfpathcurveto{\pgfqpoint{4.225655in}{1.268804in}}{\pgfqpoint{4.228489in}{1.275646in}}{\pgfqpoint{4.228489in}{1.282779in}}%
\pgfpathcurveto{\pgfqpoint{4.228489in}{1.289912in}}{\pgfqpoint{4.225655in}{1.296753in}}{\pgfqpoint{4.220611in}{1.301797in}}%
\pgfpathcurveto{\pgfqpoint{4.215568in}{1.306841in}}{\pgfqpoint{4.208726in}{1.309675in}}{\pgfqpoint{4.201593in}{1.309675in}}%
\pgfpathcurveto{\pgfqpoint{4.194461in}{1.309675in}}{\pgfqpoint{4.187619in}{1.306841in}}{\pgfqpoint{4.182575in}{1.301797in}}%
\pgfpathcurveto{\pgfqpoint{4.177532in}{1.296753in}}{\pgfqpoint{4.174698in}{1.289912in}}{\pgfqpoint{4.174698in}{1.282779in}}%
\pgfpathcurveto{\pgfqpoint{4.174698in}{1.275646in}}{\pgfqpoint{4.177532in}{1.268804in}}{\pgfqpoint{4.182575in}{1.263761in}}%
\pgfpathcurveto{\pgfqpoint{4.187619in}{1.258717in}}{\pgfqpoint{4.194461in}{1.255883in}}{\pgfqpoint{4.201593in}{1.255883in}}%
\pgfpathclose%
\pgfusepath{stroke,fill}%
\end{pgfscope}%
\begin{pgfscope}%
\pgfpathrectangle{\pgfqpoint{2.867647in}{0.500000in}}{\pgfqpoint{1.764706in}{1.700000in}}%
\pgfusepath{clip}%
\pgfsetbuttcap%
\pgfsetroundjoin%
\definecolor{currentfill}{rgb}{0.968105,0.786346,0.667739}%
\pgfsetfillcolor{currentfill}%
\pgfsetlinewidth{0.311001pt}%
\definecolor{currentstroke}{rgb}{1.000000,1.000000,1.000000}%
\pgfsetstrokecolor{currentstroke}%
\pgfsetdash{}{0pt}%
\pgfpathmoveto{\pgfqpoint{4.108127in}{0.937763in}}%
\pgfpathcurveto{\pgfqpoint{4.115260in}{0.937763in}}{\pgfqpoint{4.122101in}{0.940597in}}{\pgfqpoint{4.127145in}{0.945641in}}%
\pgfpathcurveto{\pgfqpoint{4.132189in}{0.950684in}}{\pgfqpoint{4.135023in}{0.957526in}}{\pgfqpoint{4.135023in}{0.964659in}}%
\pgfpathcurveto{\pgfqpoint{4.135023in}{0.971792in}}{\pgfqpoint{4.132189in}{0.978633in}}{\pgfqpoint{4.127145in}{0.983677in}}%
\pgfpathcurveto{\pgfqpoint{4.122101in}{0.988721in}}{\pgfqpoint{4.115260in}{0.991555in}}{\pgfqpoint{4.108127in}{0.991555in}}%
\pgfpathcurveto{\pgfqpoint{4.100994in}{0.991555in}}{\pgfqpoint{4.094153in}{0.988721in}}{\pgfqpoint{4.089109in}{0.983677in}}%
\pgfpathcurveto{\pgfqpoint{4.084065in}{0.978633in}}{\pgfqpoint{4.081231in}{0.971792in}}{\pgfqpoint{4.081231in}{0.964659in}}%
\pgfpathcurveto{\pgfqpoint{4.081231in}{0.957526in}}{\pgfqpoint{4.084065in}{0.950684in}}{\pgfqpoint{4.089109in}{0.945641in}}%
\pgfpathcurveto{\pgfqpoint{4.094153in}{0.940597in}}{\pgfqpoint{4.100994in}{0.937763in}}{\pgfqpoint{4.108127in}{0.937763in}}%
\pgfpathclose%
\pgfusepath{stroke,fill}%
\end{pgfscope}%
\begin{pgfscope}%
\pgfpathrectangle{\pgfqpoint{2.867647in}{0.500000in}}{\pgfqpoint{1.764706in}{1.700000in}}%
\pgfusepath{clip}%
\pgfsetbuttcap%
\pgfsetroundjoin%
\definecolor{currentfill}{rgb}{0.965753,0.732351,0.592427}%
\pgfsetfillcolor{currentfill}%
\pgfsetlinewidth{0.311001pt}%
\definecolor{currentstroke}{rgb}{1.000000,1.000000,1.000000}%
\pgfsetstrokecolor{currentstroke}%
\pgfsetdash{}{0pt}%
\pgfpathmoveto{\pgfqpoint{4.008066in}{0.935544in}}%
\pgfpathcurveto{\pgfqpoint{4.015198in}{0.935544in}}{\pgfqpoint{4.022040in}{0.938378in}}{\pgfqpoint{4.027084in}{0.943422in}}%
\pgfpathcurveto{\pgfqpoint{4.032127in}{0.948465in}}{\pgfqpoint{4.034961in}{0.955307in}}{\pgfqpoint{4.034961in}{0.962440in}}%
\pgfpathcurveto{\pgfqpoint{4.034961in}{0.969573in}}{\pgfqpoint{4.032127in}{0.976414in}}{\pgfqpoint{4.027084in}{0.981458in}}%
\pgfpathcurveto{\pgfqpoint{4.022040in}{0.986502in}}{\pgfqpoint{4.015198in}{0.989336in}}{\pgfqpoint{4.008066in}{0.989336in}}%
\pgfpathcurveto{\pgfqpoint{4.000933in}{0.989336in}}{\pgfqpoint{3.994091in}{0.986502in}}{\pgfqpoint{3.989047in}{0.981458in}}%
\pgfpathcurveto{\pgfqpoint{3.984004in}{0.976414in}}{\pgfqpoint{3.981170in}{0.969573in}}{\pgfqpoint{3.981170in}{0.962440in}}%
\pgfpathcurveto{\pgfqpoint{3.981170in}{0.955307in}}{\pgfqpoint{3.984004in}{0.948465in}}{\pgfqpoint{3.989047in}{0.943422in}}%
\pgfpathcurveto{\pgfqpoint{3.994091in}{0.938378in}}{\pgfqpoint{4.000933in}{0.935544in}}{\pgfqpoint{4.008066in}{0.935544in}}%
\pgfpathclose%
\pgfusepath{stroke,fill}%
\end{pgfscope}%
\begin{pgfscope}%
\pgfpathrectangle{\pgfqpoint{2.867647in}{0.500000in}}{\pgfqpoint{1.764706in}{1.700000in}}%
\pgfusepath{clip}%
\pgfsetbuttcap%
\pgfsetroundjoin%
\definecolor{currentfill}{rgb}{0.530589,0.116624,0.355860}%
\pgfsetfillcolor{currentfill}%
\pgfsetlinewidth{0.311001pt}%
\definecolor{currentstroke}{rgb}{1.000000,1.000000,1.000000}%
\pgfsetstrokecolor{currentstroke}%
\pgfsetdash{}{0pt}%
\pgfpathmoveto{\pgfqpoint{3.891880in}{1.458336in}}%
\pgfpathcurveto{\pgfqpoint{3.899013in}{1.458336in}}{\pgfqpoint{3.905855in}{1.461170in}}{\pgfqpoint{3.910898in}{1.466214in}}%
\pgfpathcurveto{\pgfqpoint{3.915942in}{1.471257in}}{\pgfqpoint{3.918776in}{1.478099in}}{\pgfqpoint{3.918776in}{1.485232in}}%
\pgfpathcurveto{\pgfqpoint{3.918776in}{1.492365in}}{\pgfqpoint{3.915942in}{1.499206in}}{\pgfqpoint{3.910898in}{1.504250in}}%
\pgfpathcurveto{\pgfqpoint{3.905855in}{1.509294in}}{\pgfqpoint{3.899013in}{1.512127in}}{\pgfqpoint{3.891880in}{1.512127in}}%
\pgfpathcurveto{\pgfqpoint{3.884747in}{1.512127in}}{\pgfqpoint{3.877906in}{1.509294in}}{\pgfqpoint{3.872862in}{1.504250in}}%
\pgfpathcurveto{\pgfqpoint{3.867818in}{1.499206in}}{\pgfqpoint{3.864984in}{1.492365in}}{\pgfqpoint{3.864984in}{1.485232in}}%
\pgfpathcurveto{\pgfqpoint{3.864984in}{1.478099in}}{\pgfqpoint{3.867818in}{1.471257in}}{\pgfqpoint{3.872862in}{1.466214in}}%
\pgfpathcurveto{\pgfqpoint{3.877906in}{1.461170in}}{\pgfqpoint{3.884747in}{1.458336in}}{\pgfqpoint{3.891880in}{1.458336in}}%
\pgfpathclose%
\pgfusepath{stroke,fill}%
\end{pgfscope}%
\begin{pgfscope}%
\pgfpathrectangle{\pgfqpoint{2.867647in}{0.500000in}}{\pgfqpoint{1.764706in}{1.700000in}}%
\pgfusepath{clip}%
\pgfsetbuttcap%
\pgfsetroundjoin%
\definecolor{currentfill}{rgb}{0.979891,0.908948,0.848279}%
\pgfsetfillcolor{currentfill}%
\pgfsetlinewidth{0.311001pt}%
\definecolor{currentstroke}{rgb}{1.000000,1.000000,1.000000}%
\pgfsetstrokecolor{currentstroke}%
\pgfsetdash{}{0pt}%
\pgfpathmoveto{\pgfqpoint{4.198923in}{1.179104in}}%
\pgfpathcurveto{\pgfqpoint{4.206055in}{1.179104in}}{\pgfqpoint{4.212897in}{1.181938in}}{\pgfqpoint{4.217941in}{1.186982in}}%
\pgfpathcurveto{\pgfqpoint{4.222984in}{1.192026in}}{\pgfqpoint{4.225818in}{1.198867in}}{\pgfqpoint{4.225818in}{1.206000in}}%
\pgfpathcurveto{\pgfqpoint{4.225818in}{1.213133in}}{\pgfqpoint{4.222984in}{1.219975in}}{\pgfqpoint{4.217941in}{1.225018in}}%
\pgfpathcurveto{\pgfqpoint{4.212897in}{1.230062in}}{\pgfqpoint{4.206055in}{1.232896in}}{\pgfqpoint{4.198923in}{1.232896in}}%
\pgfpathcurveto{\pgfqpoint{4.191790in}{1.232896in}}{\pgfqpoint{4.184948in}{1.230062in}}{\pgfqpoint{4.179904in}{1.225018in}}%
\pgfpathcurveto{\pgfqpoint{4.174861in}{1.219975in}}{\pgfqpoint{4.172027in}{1.213133in}}{\pgfqpoint{4.172027in}{1.206000in}}%
\pgfpathcurveto{\pgfqpoint{4.172027in}{1.198867in}}{\pgfqpoint{4.174861in}{1.192026in}}{\pgfqpoint{4.179904in}{1.186982in}}%
\pgfpathcurveto{\pgfqpoint{4.184948in}{1.181938in}}{\pgfqpoint{4.191790in}{1.179104in}}{\pgfqpoint{4.198923in}{1.179104in}}%
\pgfpathclose%
\pgfusepath{stroke,fill}%
\end{pgfscope}%
\begin{pgfscope}%
\pgfpathrectangle{\pgfqpoint{2.867647in}{0.500000in}}{\pgfqpoint{1.764706in}{1.700000in}}%
\pgfusepath{clip}%
\pgfsetbuttcap%
\pgfsetroundjoin%
\definecolor{currentfill}{rgb}{0.953126,0.456614,0.312398}%
\pgfsetfillcolor{currentfill}%
\pgfsetlinewidth{0.311001pt}%
\definecolor{currentstroke}{rgb}{1.000000,1.000000,1.000000}%
\pgfsetstrokecolor{currentstroke}%
\pgfsetdash{}{0pt}%
\pgfpathmoveto{\pgfqpoint{4.340889in}{1.466027in}}%
\pgfpathcurveto{\pgfqpoint{4.348022in}{1.466027in}}{\pgfqpoint{4.354864in}{1.468861in}}{\pgfqpoint{4.359907in}{1.473904in}}%
\pgfpathcurveto{\pgfqpoint{4.364951in}{1.478948in}}{\pgfqpoint{4.367785in}{1.485790in}}{\pgfqpoint{4.367785in}{1.492922in}}%
\pgfpathcurveto{\pgfqpoint{4.367785in}{1.500055in}}{\pgfqpoint{4.364951in}{1.506897in}}{\pgfqpoint{4.359907in}{1.511941in}}%
\pgfpathcurveto{\pgfqpoint{4.354864in}{1.516984in}}{\pgfqpoint{4.348022in}{1.519818in}}{\pgfqpoint{4.340889in}{1.519818in}}%
\pgfpathcurveto{\pgfqpoint{4.333756in}{1.519818in}}{\pgfqpoint{4.326915in}{1.516984in}}{\pgfqpoint{4.321871in}{1.511941in}}%
\pgfpathcurveto{\pgfqpoint{4.316827in}{1.506897in}}{\pgfqpoint{4.313993in}{1.500055in}}{\pgfqpoint{4.313993in}{1.492922in}}%
\pgfpathcurveto{\pgfqpoint{4.313993in}{1.485790in}}{\pgfqpoint{4.316827in}{1.478948in}}{\pgfqpoint{4.321871in}{1.473904in}}%
\pgfpathcurveto{\pgfqpoint{4.326915in}{1.468861in}}{\pgfqpoint{4.333756in}{1.466027in}}{\pgfqpoint{4.340889in}{1.466027in}}%
\pgfpathclose%
\pgfusepath{stroke,fill}%
\end{pgfscope}%
\begin{pgfscope}%
\pgfpathrectangle{\pgfqpoint{2.867647in}{0.500000in}}{\pgfqpoint{1.764706in}{1.700000in}}%
\pgfusepath{clip}%
\pgfsetbuttcap%
\pgfsetroundjoin%
\definecolor{currentfill}{rgb}{0.980678,0.914765,0.856766}%
\pgfsetfillcolor{currentfill}%
\pgfsetlinewidth{0.311001pt}%
\definecolor{currentstroke}{rgb}{1.000000,1.000000,1.000000}%
\pgfsetstrokecolor{currentstroke}%
\pgfsetdash{}{0pt}%
\pgfpathmoveto{\pgfqpoint{4.175164in}{1.327115in}}%
\pgfpathcurveto{\pgfqpoint{4.182297in}{1.327115in}}{\pgfqpoint{4.189139in}{1.329949in}}{\pgfqpoint{4.194182in}{1.334992in}}%
\pgfpathcurveto{\pgfqpoint{4.199226in}{1.340036in}}{\pgfqpoint{4.202060in}{1.346878in}}{\pgfqpoint{4.202060in}{1.354010in}}%
\pgfpathcurveto{\pgfqpoint{4.202060in}{1.361143in}}{\pgfqpoint{4.199226in}{1.367985in}}{\pgfqpoint{4.194182in}{1.373029in}}%
\pgfpathcurveto{\pgfqpoint{4.189139in}{1.378072in}}{\pgfqpoint{4.182297in}{1.380906in}}{\pgfqpoint{4.175164in}{1.380906in}}%
\pgfpathcurveto{\pgfqpoint{4.168031in}{1.380906in}}{\pgfqpoint{4.161190in}{1.378072in}}{\pgfqpoint{4.156146in}{1.373029in}}%
\pgfpathcurveto{\pgfqpoint{4.151102in}{1.367985in}}{\pgfqpoint{4.148268in}{1.361143in}}{\pgfqpoint{4.148268in}{1.354010in}}%
\pgfpathcurveto{\pgfqpoint{4.148268in}{1.346878in}}{\pgfqpoint{4.151102in}{1.340036in}}{\pgfqpoint{4.156146in}{1.334992in}}%
\pgfpathcurveto{\pgfqpoint{4.161190in}{1.329949in}}{\pgfqpoint{4.168031in}{1.327115in}}{\pgfqpoint{4.175164in}{1.327115in}}%
\pgfpathclose%
\pgfusepath{stroke,fill}%
\end{pgfscope}%
\begin{pgfscope}%
\pgfpathrectangle{\pgfqpoint{2.867647in}{0.500000in}}{\pgfqpoint{1.764706in}{1.700000in}}%
\pgfusepath{clip}%
\pgfsetbuttcap%
\pgfsetroundjoin%
\definecolor{currentfill}{rgb}{0.852817,0.156578,0.279098}%
\pgfsetfillcolor{currentfill}%
\pgfsetlinewidth{0.311001pt}%
\definecolor{currentstroke}{rgb}{1.000000,1.000000,1.000000}%
\pgfsetstrokecolor{currentstroke}%
\pgfsetdash{}{0pt}%
\pgfpathmoveto{\pgfqpoint{4.396033in}{1.399019in}}%
\pgfpathcurveto{\pgfqpoint{4.403166in}{1.399019in}}{\pgfqpoint{4.410008in}{1.401853in}}{\pgfqpoint{4.415052in}{1.406896in}}%
\pgfpathcurveto{\pgfqpoint{4.420095in}{1.411940in}}{\pgfqpoint{4.422929in}{1.418782in}}{\pgfqpoint{4.422929in}{1.425915in}}%
\pgfpathcurveto{\pgfqpoint{4.422929in}{1.433047in}}{\pgfqpoint{4.420095in}{1.439889in}}{\pgfqpoint{4.415052in}{1.444933in}}%
\pgfpathcurveto{\pgfqpoint{4.410008in}{1.449976in}}{\pgfqpoint{4.403166in}{1.452810in}}{\pgfqpoint{4.396033in}{1.452810in}}%
\pgfpathcurveto{\pgfqpoint{4.388901in}{1.452810in}}{\pgfqpoint{4.382059in}{1.449976in}}{\pgfqpoint{4.377015in}{1.444933in}}%
\pgfpathcurveto{\pgfqpoint{4.371972in}{1.439889in}}{\pgfqpoint{4.369138in}{1.433047in}}{\pgfqpoint{4.369138in}{1.425915in}}%
\pgfpathcurveto{\pgfqpoint{4.369138in}{1.418782in}}{\pgfqpoint{4.371972in}{1.411940in}}{\pgfqpoint{4.377015in}{1.406896in}}%
\pgfpathcurveto{\pgfqpoint{4.382059in}{1.401853in}}{\pgfqpoint{4.388901in}{1.399019in}}{\pgfqpoint{4.396033in}{1.399019in}}%
\pgfpathclose%
\pgfusepath{stroke,fill}%
\end{pgfscope}%
\begin{pgfscope}%
\pgfpathrectangle{\pgfqpoint{2.867647in}{0.500000in}}{\pgfqpoint{1.764706in}{1.700000in}}%
\pgfusepath{clip}%
\pgfsetbuttcap%
\pgfsetroundjoin%
\definecolor{currentfill}{rgb}{0.979124,0.903132,0.839793}%
\pgfsetfillcolor{currentfill}%
\pgfsetlinewidth{0.311001pt}%
\definecolor{currentstroke}{rgb}{1.000000,1.000000,1.000000}%
\pgfsetstrokecolor{currentstroke}%
\pgfsetdash{}{0pt}%
\pgfpathmoveto{\pgfqpoint{4.143945in}{1.453702in}}%
\pgfpathcurveto{\pgfqpoint{4.151078in}{1.453702in}}{\pgfqpoint{4.157920in}{1.456536in}}{\pgfqpoint{4.162963in}{1.461580in}}%
\pgfpathcurveto{\pgfqpoint{4.168007in}{1.466623in}}{\pgfqpoint{4.170841in}{1.473465in}}{\pgfqpoint{4.170841in}{1.480598in}}%
\pgfpathcurveto{\pgfqpoint{4.170841in}{1.487731in}}{\pgfqpoint{4.168007in}{1.494572in}}{\pgfqpoint{4.162963in}{1.499616in}}%
\pgfpathcurveto{\pgfqpoint{4.157920in}{1.504660in}}{\pgfqpoint{4.151078in}{1.507494in}}{\pgfqpoint{4.143945in}{1.507494in}}%
\pgfpathcurveto{\pgfqpoint{4.136812in}{1.507494in}}{\pgfqpoint{4.129971in}{1.504660in}}{\pgfqpoint{4.124927in}{1.499616in}}%
\pgfpathcurveto{\pgfqpoint{4.119883in}{1.494572in}}{\pgfqpoint{4.117049in}{1.487731in}}{\pgfqpoint{4.117049in}{1.480598in}}%
\pgfpathcurveto{\pgfqpoint{4.117049in}{1.473465in}}{\pgfqpoint{4.119883in}{1.466623in}}{\pgfqpoint{4.124927in}{1.461580in}}%
\pgfpathcurveto{\pgfqpoint{4.129971in}{1.456536in}}{\pgfqpoint{4.136812in}{1.453702in}}{\pgfqpoint{4.143945in}{1.453702in}}%
\pgfpathclose%
\pgfusepath{stroke,fill}%
\end{pgfscope}%
\begin{pgfscope}%
\pgfpathrectangle{\pgfqpoint{2.867647in}{0.500000in}}{\pgfqpoint{1.764706in}{1.700000in}}%
\pgfusepath{clip}%
\pgfsetbuttcap%
\pgfsetroundjoin%
\definecolor{currentfill}{rgb}{0.972201,0.839051,0.745789}%
\pgfsetfillcolor{currentfill}%
\pgfsetlinewidth{0.311001pt}%
\definecolor{currentstroke}{rgb}{1.000000,1.000000,1.000000}%
\pgfsetstrokecolor{currentstroke}%
\pgfsetdash{}{0pt}%
\pgfpathmoveto{\pgfqpoint{4.058098in}{1.619329in}}%
\pgfpathcurveto{\pgfqpoint{4.065231in}{1.619329in}}{\pgfqpoint{4.072072in}{1.622163in}}{\pgfqpoint{4.077116in}{1.627207in}}%
\pgfpathcurveto{\pgfqpoint{4.082160in}{1.632250in}}{\pgfqpoint{4.084994in}{1.639092in}}{\pgfqpoint{4.084994in}{1.646225in}}%
\pgfpathcurveto{\pgfqpoint{4.084994in}{1.653358in}}{\pgfqpoint{4.082160in}{1.660199in}}{\pgfqpoint{4.077116in}{1.665243in}}%
\pgfpathcurveto{\pgfqpoint{4.072072in}{1.670287in}}{\pgfqpoint{4.065231in}{1.673120in}}{\pgfqpoint{4.058098in}{1.673120in}}%
\pgfpathcurveto{\pgfqpoint{4.050965in}{1.673120in}}{\pgfqpoint{4.044123in}{1.670287in}}{\pgfqpoint{4.039080in}{1.665243in}}%
\pgfpathcurveto{\pgfqpoint{4.034036in}{1.660199in}}{\pgfqpoint{4.031202in}{1.653358in}}{\pgfqpoint{4.031202in}{1.646225in}}%
\pgfpathcurveto{\pgfqpoint{4.031202in}{1.639092in}}{\pgfqpoint{4.034036in}{1.632250in}}{\pgfqpoint{4.039080in}{1.627207in}}%
\pgfpathcurveto{\pgfqpoint{4.044123in}{1.622163in}}{\pgfqpoint{4.050965in}{1.619329in}}{\pgfqpoint{4.058098in}{1.619329in}}%
\pgfpathclose%
\pgfusepath{stroke,fill}%
\end{pgfscope}%
\begin{pgfscope}%
\pgfpathrectangle{\pgfqpoint{2.867647in}{0.500000in}}{\pgfqpoint{1.764706in}{1.700000in}}%
\pgfusepath{clip}%
\pgfsetbuttcap%
\pgfsetroundjoin%
\definecolor{currentfill}{rgb}{0.949145,0.420383,0.287810}%
\pgfsetfillcolor{currentfill}%
\pgfsetlinewidth{0.311001pt}%
\definecolor{currentstroke}{rgb}{1.000000,1.000000,1.000000}%
\pgfsetstrokecolor{currentstroke}%
\pgfsetdash{}{0pt}%
\pgfpathmoveto{\pgfqpoint{3.846397in}{1.761280in}}%
\pgfpathcurveto{\pgfqpoint{3.853529in}{1.761280in}}{\pgfqpoint{3.860371in}{1.764114in}}{\pgfqpoint{3.865415in}{1.769158in}}%
\pgfpathcurveto{\pgfqpoint{3.870458in}{1.774201in}}{\pgfqpoint{3.873292in}{1.781043in}}{\pgfqpoint{3.873292in}{1.788176in}}%
\pgfpathcurveto{\pgfqpoint{3.873292in}{1.795309in}}{\pgfqpoint{3.870458in}{1.802150in}}{\pgfqpoint{3.865415in}{1.807194in}}%
\pgfpathcurveto{\pgfqpoint{3.860371in}{1.812238in}}{\pgfqpoint{3.853529in}{1.815071in}}{\pgfqpoint{3.846397in}{1.815071in}}%
\pgfpathcurveto{\pgfqpoint{3.839264in}{1.815071in}}{\pgfqpoint{3.832422in}{1.812238in}}{\pgfqpoint{3.827379in}{1.807194in}}%
\pgfpathcurveto{\pgfqpoint{3.822335in}{1.802150in}}{\pgfqpoint{3.819501in}{1.795309in}}{\pgfqpoint{3.819501in}{1.788176in}}%
\pgfpathcurveto{\pgfqpoint{3.819501in}{1.781043in}}{\pgfqpoint{3.822335in}{1.774201in}}{\pgfqpoint{3.827379in}{1.769158in}}%
\pgfpathcurveto{\pgfqpoint{3.832422in}{1.764114in}}{\pgfqpoint{3.839264in}{1.761280in}}{\pgfqpoint{3.846397in}{1.761280in}}%
\pgfpathclose%
\pgfusepath{stroke,fill}%
\end{pgfscope}%
\begin{pgfscope}%
\pgfpathrectangle{\pgfqpoint{2.867647in}{0.500000in}}{\pgfqpoint{1.764706in}{1.700000in}}%
\pgfusepath{clip}%
\pgfsetbuttcap%
\pgfsetroundjoin%
\definecolor{currentfill}{rgb}{0.979124,0.903132,0.839793}%
\pgfsetfillcolor{currentfill}%
\pgfsetlinewidth{0.311001pt}%
\definecolor{currentstroke}{rgb}{1.000000,1.000000,1.000000}%
\pgfsetstrokecolor{currentstroke}%
\pgfsetdash{}{0pt}%
\pgfpathmoveto{\pgfqpoint{4.222920in}{1.305899in}}%
\pgfpathcurveto{\pgfqpoint{4.230053in}{1.305899in}}{\pgfqpoint{4.236895in}{1.308733in}}{\pgfqpoint{4.241939in}{1.313777in}}%
\pgfpathcurveto{\pgfqpoint{4.246982in}{1.318820in}}{\pgfqpoint{4.249816in}{1.325662in}}{\pgfqpoint{4.249816in}{1.332795in}}%
\pgfpathcurveto{\pgfqpoint{4.249816in}{1.339928in}}{\pgfqpoint{4.246982in}{1.346769in}}{\pgfqpoint{4.241939in}{1.351813in}}%
\pgfpathcurveto{\pgfqpoint{4.236895in}{1.356857in}}{\pgfqpoint{4.230053in}{1.359691in}}{\pgfqpoint{4.222920in}{1.359691in}}%
\pgfpathcurveto{\pgfqpoint{4.215788in}{1.359691in}}{\pgfqpoint{4.208946in}{1.356857in}}{\pgfqpoint{4.203902in}{1.351813in}}%
\pgfpathcurveto{\pgfqpoint{4.198859in}{1.346769in}}{\pgfqpoint{4.196025in}{1.339928in}}{\pgfqpoint{4.196025in}{1.332795in}}%
\pgfpathcurveto{\pgfqpoint{4.196025in}{1.325662in}}{\pgfqpoint{4.198859in}{1.318820in}}{\pgfqpoint{4.203902in}{1.313777in}}%
\pgfpathcurveto{\pgfqpoint{4.208946in}{1.308733in}}{\pgfqpoint{4.215788in}{1.305899in}}{\pgfqpoint{4.222920in}{1.305899in}}%
\pgfpathclose%
\pgfusepath{stroke,fill}%
\end{pgfscope}%
\begin{pgfscope}%
\pgfpathrectangle{\pgfqpoint{2.867647in}{0.500000in}}{\pgfqpoint{1.764706in}{1.700000in}}%
\pgfusepath{clip}%
\pgfsetbuttcap%
\pgfsetroundjoin%
\definecolor{currentfill}{rgb}{0.966560,0.756582,0.625273}%
\pgfsetfillcolor{currentfill}%
\pgfsetlinewidth{0.311001pt}%
\definecolor{currentstroke}{rgb}{1.000000,1.000000,1.000000}%
\pgfsetstrokecolor{currentstroke}%
\pgfsetdash{}{0pt}%
\pgfpathmoveto{\pgfqpoint{4.062391in}{1.466740in}}%
\pgfpathcurveto{\pgfqpoint{4.069524in}{1.466740in}}{\pgfqpoint{4.076365in}{1.469574in}}{\pgfqpoint{4.081409in}{1.474617in}}%
\pgfpathcurveto{\pgfqpoint{4.086452in}{1.479661in}}{\pgfqpoint{4.089286in}{1.486503in}}{\pgfqpoint{4.089286in}{1.493635in}}%
\pgfpathcurveto{\pgfqpoint{4.089286in}{1.500768in}}{\pgfqpoint{4.086452in}{1.507610in}}{\pgfqpoint{4.081409in}{1.512654in}}%
\pgfpathcurveto{\pgfqpoint{4.076365in}{1.517697in}}{\pgfqpoint{4.069524in}{1.520531in}}{\pgfqpoint{4.062391in}{1.520531in}}%
\pgfpathcurveto{\pgfqpoint{4.055258in}{1.520531in}}{\pgfqpoint{4.048416in}{1.517697in}}{\pgfqpoint{4.043373in}{1.512654in}}%
\pgfpathcurveto{\pgfqpoint{4.038329in}{1.507610in}}{\pgfqpoint{4.035495in}{1.500768in}}{\pgfqpoint{4.035495in}{1.493635in}}%
\pgfpathcurveto{\pgfqpoint{4.035495in}{1.486503in}}{\pgfqpoint{4.038329in}{1.479661in}}{\pgfqpoint{4.043373in}{1.474617in}}%
\pgfpathcurveto{\pgfqpoint{4.048416in}{1.469574in}}{\pgfqpoint{4.055258in}{1.466740in}}{\pgfqpoint{4.062391in}{1.466740in}}%
\pgfpathclose%
\pgfusepath{stroke,fill}%
\end{pgfscope}%
\begin{pgfscope}%
\pgfpathrectangle{\pgfqpoint{2.867647in}{0.500000in}}{\pgfqpoint{1.764706in}{1.700000in}}%
\pgfusepath{clip}%
\pgfsetbuttcap%
\pgfsetroundjoin%
\definecolor{currentfill}{rgb}{0.964558,0.676556,0.522514}%
\pgfsetfillcolor{currentfill}%
\pgfsetlinewidth{0.311001pt}%
\definecolor{currentstroke}{rgb}{1.000000,1.000000,1.000000}%
\pgfsetstrokecolor{currentstroke}%
\pgfsetdash{}{0pt}%
\pgfpathmoveto{\pgfqpoint{4.217889in}{1.662026in}}%
\pgfpathcurveto{\pgfqpoint{4.225022in}{1.662026in}}{\pgfqpoint{4.231864in}{1.664860in}}{\pgfqpoint{4.236908in}{1.669903in}}%
\pgfpathcurveto{\pgfqpoint{4.241951in}{1.674947in}}{\pgfqpoint{4.244785in}{1.681789in}}{\pgfqpoint{4.244785in}{1.688921in}}%
\pgfpathcurveto{\pgfqpoint{4.244785in}{1.696054in}}{\pgfqpoint{4.241951in}{1.702896in}}{\pgfqpoint{4.236908in}{1.707940in}}%
\pgfpathcurveto{\pgfqpoint{4.231864in}{1.712983in}}{\pgfqpoint{4.225022in}{1.715817in}}{\pgfqpoint{4.217889in}{1.715817in}}%
\pgfpathcurveto{\pgfqpoint{4.210757in}{1.715817in}}{\pgfqpoint{4.203915in}{1.712983in}}{\pgfqpoint{4.198871in}{1.707940in}}%
\pgfpathcurveto{\pgfqpoint{4.193828in}{1.702896in}}{\pgfqpoint{4.190994in}{1.696054in}}{\pgfqpoint{4.190994in}{1.688921in}}%
\pgfpathcurveto{\pgfqpoint{4.190994in}{1.681789in}}{\pgfqpoint{4.193828in}{1.674947in}}{\pgfqpoint{4.198871in}{1.669903in}}%
\pgfpathcurveto{\pgfqpoint{4.203915in}{1.664860in}}{\pgfqpoint{4.210757in}{1.662026in}}{\pgfqpoint{4.217889in}{1.662026in}}%
\pgfpathclose%
\pgfusepath{stroke,fill}%
\end{pgfscope}%
\begin{pgfscope}%
\pgfpathrectangle{\pgfqpoint{2.867647in}{0.500000in}}{\pgfqpoint{1.764706in}{1.700000in}}%
\pgfusepath{clip}%
\pgfsetbuttcap%
\pgfsetroundjoin%
\definecolor{currentfill}{rgb}{0.980678,0.914765,0.856766}%
\pgfsetfillcolor{currentfill}%
\pgfsetlinewidth{0.311001pt}%
\definecolor{currentstroke}{rgb}{1.000000,1.000000,1.000000}%
\pgfsetstrokecolor{currentstroke}%
\pgfsetdash{}{0pt}%
\pgfpathmoveto{\pgfqpoint{4.187577in}{1.156317in}}%
\pgfpathcurveto{\pgfqpoint{4.194710in}{1.156317in}}{\pgfqpoint{4.201552in}{1.159151in}}{\pgfqpoint{4.206596in}{1.164195in}}%
\pgfpathcurveto{\pgfqpoint{4.211639in}{1.169239in}}{\pgfqpoint{4.214473in}{1.176080in}}{\pgfqpoint{4.214473in}{1.183213in}}%
\pgfpathcurveto{\pgfqpoint{4.214473in}{1.190346in}}{\pgfqpoint{4.211639in}{1.197187in}}{\pgfqpoint{4.206596in}{1.202231in}}%
\pgfpathcurveto{\pgfqpoint{4.201552in}{1.207275in}}{\pgfqpoint{4.194710in}{1.210109in}}{\pgfqpoint{4.187577in}{1.210109in}}%
\pgfpathcurveto{\pgfqpoint{4.180445in}{1.210109in}}{\pgfqpoint{4.173603in}{1.207275in}}{\pgfqpoint{4.168559in}{1.202231in}}%
\pgfpathcurveto{\pgfqpoint{4.163516in}{1.197187in}}{\pgfqpoint{4.160682in}{1.190346in}}{\pgfqpoint{4.160682in}{1.183213in}}%
\pgfpathcurveto{\pgfqpoint{4.160682in}{1.176080in}}{\pgfqpoint{4.163516in}{1.169239in}}{\pgfqpoint{4.168559in}{1.164195in}}%
\pgfpathcurveto{\pgfqpoint{4.173603in}{1.159151in}}{\pgfqpoint{4.180445in}{1.156317in}}{\pgfqpoint{4.187577in}{1.156317in}}%
\pgfpathclose%
\pgfusepath{stroke,fill}%
\end{pgfscope}%
\begin{pgfscope}%
\pgfpathrectangle{\pgfqpoint{2.867647in}{0.500000in}}{\pgfqpoint{1.764706in}{1.700000in}}%
\pgfusepath{clip}%
\pgfsetbuttcap%
\pgfsetroundjoin%
\definecolor{currentfill}{rgb}{0.976961,0.885681,0.814303}%
\pgfsetfillcolor{currentfill}%
\pgfsetlinewidth{0.311001pt}%
\definecolor{currentstroke}{rgb}{1.000000,1.000000,1.000000}%
\pgfsetstrokecolor{currentstroke}%
\pgfsetdash{}{0pt}%
\pgfpathmoveto{\pgfqpoint{4.096646in}{1.561519in}}%
\pgfpathcurveto{\pgfqpoint{4.103779in}{1.561519in}}{\pgfqpoint{4.110620in}{1.564353in}}{\pgfqpoint{4.115664in}{1.569396in}}%
\pgfpathcurveto{\pgfqpoint{4.120708in}{1.574440in}}{\pgfqpoint{4.123542in}{1.581282in}}{\pgfqpoint{4.123542in}{1.588414in}}%
\pgfpathcurveto{\pgfqpoint{4.123542in}{1.595547in}}{\pgfqpoint{4.120708in}{1.602389in}}{\pgfqpoint{4.115664in}{1.607433in}}%
\pgfpathcurveto{\pgfqpoint{4.110620in}{1.612476in}}{\pgfqpoint{4.103779in}{1.615310in}}{\pgfqpoint{4.096646in}{1.615310in}}%
\pgfpathcurveto{\pgfqpoint{4.089513in}{1.615310in}}{\pgfqpoint{4.082671in}{1.612476in}}{\pgfqpoint{4.077628in}{1.607433in}}%
\pgfpathcurveto{\pgfqpoint{4.072584in}{1.602389in}}{\pgfqpoint{4.069750in}{1.595547in}}{\pgfqpoint{4.069750in}{1.588414in}}%
\pgfpathcurveto{\pgfqpoint{4.069750in}{1.581282in}}{\pgfqpoint{4.072584in}{1.574440in}}{\pgfqpoint{4.077628in}{1.569396in}}%
\pgfpathcurveto{\pgfqpoint{4.082671in}{1.564353in}}{\pgfqpoint{4.089513in}{1.561519in}}{\pgfqpoint{4.096646in}{1.561519in}}%
\pgfpathclose%
\pgfusepath{stroke,fill}%
\end{pgfscope}%
\begin{pgfscope}%
\pgfpathrectangle{\pgfqpoint{2.867647in}{0.500000in}}{\pgfqpoint{1.764706in}{1.700000in}}%
\pgfusepath{clip}%
\pgfsetbuttcap%
\pgfsetroundjoin%
\definecolor{currentfill}{rgb}{0.980678,0.914765,0.856766}%
\pgfsetfillcolor{currentfill}%
\pgfsetlinewidth{0.311001pt}%
\definecolor{currentstroke}{rgb}{1.000000,1.000000,1.000000}%
\pgfsetstrokecolor{currentstroke}%
\pgfsetdash{}{0pt}%
\pgfpathmoveto{\pgfqpoint{4.171913in}{1.444166in}}%
\pgfpathcurveto{\pgfqpoint{4.179046in}{1.444166in}}{\pgfqpoint{4.185888in}{1.447000in}}{\pgfqpoint{4.190931in}{1.452044in}}%
\pgfpathcurveto{\pgfqpoint{4.195975in}{1.457087in}}{\pgfqpoint{4.198809in}{1.463929in}}{\pgfqpoint{4.198809in}{1.471062in}}%
\pgfpathcurveto{\pgfqpoint{4.198809in}{1.478195in}}{\pgfqpoint{4.195975in}{1.485036in}}{\pgfqpoint{4.190931in}{1.490080in}}%
\pgfpathcurveto{\pgfqpoint{4.185888in}{1.495124in}}{\pgfqpoint{4.179046in}{1.497957in}}{\pgfqpoint{4.171913in}{1.497957in}}%
\pgfpathcurveto{\pgfqpoint{4.164780in}{1.497957in}}{\pgfqpoint{4.157939in}{1.495124in}}{\pgfqpoint{4.152895in}{1.490080in}}%
\pgfpathcurveto{\pgfqpoint{4.147851in}{1.485036in}}{\pgfqpoint{4.145017in}{1.478195in}}{\pgfqpoint{4.145017in}{1.471062in}}%
\pgfpathcurveto{\pgfqpoint{4.145017in}{1.463929in}}{\pgfqpoint{4.147851in}{1.457087in}}{\pgfqpoint{4.152895in}{1.452044in}}%
\pgfpathcurveto{\pgfqpoint{4.157939in}{1.447000in}}{\pgfqpoint{4.164780in}{1.444166in}}{\pgfqpoint{4.171913in}{1.444166in}}%
\pgfpathclose%
\pgfusepath{stroke,fill}%
\end{pgfscope}%
\begin{pgfscope}%
\pgfpathrectangle{\pgfqpoint{2.867647in}{0.500000in}}{\pgfqpoint{1.764706in}{1.700000in}}%
\pgfusepath{clip}%
\pgfsetbuttcap%
\pgfsetroundjoin%
\definecolor{currentfill}{rgb}{0.967092,0.768560,0.642079}%
\pgfsetfillcolor{currentfill}%
\pgfsetlinewidth{0.311001pt}%
\definecolor{currentstroke}{rgb}{1.000000,1.000000,1.000000}%
\pgfsetstrokecolor{currentstroke}%
\pgfsetdash{}{0pt}%
\pgfpathmoveto{\pgfqpoint{4.021814in}{1.609391in}}%
\pgfpathcurveto{\pgfqpoint{4.028947in}{1.609391in}}{\pgfqpoint{4.035788in}{1.612225in}}{\pgfqpoint{4.040832in}{1.617269in}}%
\pgfpathcurveto{\pgfqpoint{4.045876in}{1.622313in}}{\pgfqpoint{4.048710in}{1.629154in}}{\pgfqpoint{4.048710in}{1.636287in}}%
\pgfpathcurveto{\pgfqpoint{4.048710in}{1.643420in}}{\pgfqpoint{4.045876in}{1.650262in}}{\pgfqpoint{4.040832in}{1.655305in}}%
\pgfpathcurveto{\pgfqpoint{4.035788in}{1.660349in}}{\pgfqpoint{4.028947in}{1.663183in}}{\pgfqpoint{4.021814in}{1.663183in}}%
\pgfpathcurveto{\pgfqpoint{4.014681in}{1.663183in}}{\pgfqpoint{4.007840in}{1.660349in}}{\pgfqpoint{4.002796in}{1.655305in}}%
\pgfpathcurveto{\pgfqpoint{3.997752in}{1.650262in}}{\pgfqpoint{3.994918in}{1.643420in}}{\pgfqpoint{3.994918in}{1.636287in}}%
\pgfpathcurveto{\pgfqpoint{3.994918in}{1.629154in}}{\pgfqpoint{3.997752in}{1.622313in}}{\pgfqpoint{4.002796in}{1.617269in}}%
\pgfpathcurveto{\pgfqpoint{4.007840in}{1.612225in}}{\pgfqpoint{4.014681in}{1.609391in}}{\pgfqpoint{4.021814in}{1.609391in}}%
\pgfpathclose%
\pgfusepath{stroke,fill}%
\end{pgfscope}%
\begin{pgfscope}%
\pgfpathrectangle{\pgfqpoint{2.867647in}{0.500000in}}{\pgfqpoint{1.764706in}{1.700000in}}%
\pgfusepath{clip}%
\pgfsetbuttcap%
\pgfsetroundjoin%
\definecolor{currentfill}{rgb}{0.979124,0.903132,0.839793}%
\pgfsetfillcolor{currentfill}%
\pgfsetlinewidth{0.311001pt}%
\definecolor{currentstroke}{rgb}{1.000000,1.000000,1.000000}%
\pgfsetstrokecolor{currentstroke}%
\pgfsetdash{}{0pt}%
\pgfpathmoveto{\pgfqpoint{4.216593in}{1.214092in}}%
\pgfpathcurveto{\pgfqpoint{4.223726in}{1.214092in}}{\pgfqpoint{4.230568in}{1.216925in}}{\pgfqpoint{4.235612in}{1.221969in}}%
\pgfpathcurveto{\pgfqpoint{4.240655in}{1.227013in}}{\pgfqpoint{4.243489in}{1.233854in}}{\pgfqpoint{4.243489in}{1.240987in}}%
\pgfpathcurveto{\pgfqpoint{4.243489in}{1.248120in}}{\pgfqpoint{4.240655in}{1.254962in}}{\pgfqpoint{4.235612in}{1.260005in}}%
\pgfpathcurveto{\pgfqpoint{4.230568in}{1.265049in}}{\pgfqpoint{4.223726in}{1.267883in}}{\pgfqpoint{4.216593in}{1.267883in}}%
\pgfpathcurveto{\pgfqpoint{4.209461in}{1.267883in}}{\pgfqpoint{4.202619in}{1.265049in}}{\pgfqpoint{4.197575in}{1.260005in}}%
\pgfpathcurveto{\pgfqpoint{4.192532in}{1.254962in}}{\pgfqpoint{4.189698in}{1.248120in}}{\pgfqpoint{4.189698in}{1.240987in}}%
\pgfpathcurveto{\pgfqpoint{4.189698in}{1.233854in}}{\pgfqpoint{4.192532in}{1.227013in}}{\pgfqpoint{4.197575in}{1.221969in}}%
\pgfpathcurveto{\pgfqpoint{4.202619in}{1.216925in}}{\pgfqpoint{4.209461in}{1.214092in}}{\pgfqpoint{4.216593in}{1.214092in}}%
\pgfpathclose%
\pgfusepath{stroke,fill}%
\end{pgfscope}%
\begin{pgfscope}%
\pgfpathrectangle{\pgfqpoint{2.867647in}{0.500000in}}{\pgfqpoint{1.764706in}{1.700000in}}%
\pgfusepath{clip}%
\pgfsetbuttcap%
\pgfsetroundjoin%
\definecolor{currentfill}{rgb}{0.537262,0.115874,0.356429}%
\pgfsetfillcolor{currentfill}%
\pgfsetlinewidth{0.311001pt}%
\definecolor{currentstroke}{rgb}{1.000000,1.000000,1.000000}%
\pgfsetstrokecolor{currentstroke}%
\pgfsetdash{}{0pt}%
\pgfpathmoveto{\pgfqpoint{3.715290in}{0.925127in}}%
\pgfpathcurveto{\pgfqpoint{3.722422in}{0.925127in}}{\pgfqpoint{3.729264in}{0.927961in}}{\pgfqpoint{3.734308in}{0.933004in}}%
\pgfpathcurveto{\pgfqpoint{3.739351in}{0.938048in}}{\pgfqpoint{3.742185in}{0.944890in}}{\pgfqpoint{3.742185in}{0.952023in}}%
\pgfpathcurveto{\pgfqpoint{3.742185in}{0.959155in}}{\pgfqpoint{3.739351in}{0.965997in}}{\pgfqpoint{3.734308in}{0.971041in}}%
\pgfpathcurveto{\pgfqpoint{3.729264in}{0.976084in}}{\pgfqpoint{3.722422in}{0.978918in}}{\pgfqpoint{3.715290in}{0.978918in}}%
\pgfpathcurveto{\pgfqpoint{3.708157in}{0.978918in}}{\pgfqpoint{3.701315in}{0.976084in}}{\pgfqpoint{3.696271in}{0.971041in}}%
\pgfpathcurveto{\pgfqpoint{3.691228in}{0.965997in}}{\pgfqpoint{3.688394in}{0.959155in}}{\pgfqpoint{3.688394in}{0.952023in}}%
\pgfpathcurveto{\pgfqpoint{3.688394in}{0.944890in}}{\pgfqpoint{3.691228in}{0.938048in}}{\pgfqpoint{3.696271in}{0.933004in}}%
\pgfpathcurveto{\pgfqpoint{3.701315in}{0.927961in}}{\pgfqpoint{3.708157in}{0.925127in}}{\pgfqpoint{3.715290in}{0.925127in}}%
\pgfpathclose%
\pgfusepath{stroke,fill}%
\end{pgfscope}%
\begin{pgfscope}%
\pgfpathrectangle{\pgfqpoint{2.867647in}{0.500000in}}{\pgfqpoint{1.764706in}{1.700000in}}%
\pgfusepath{clip}%
\pgfsetbuttcap%
\pgfsetroundjoin%
\definecolor{currentfill}{rgb}{0.866416,0.173878,0.270708}%
\pgfsetfillcolor{currentfill}%
\pgfsetlinewidth{0.311001pt}%
\definecolor{currentstroke}{rgb}{1.000000,1.000000,1.000000}%
\pgfsetstrokecolor{currentstroke}%
\pgfsetdash{}{0pt}%
\pgfpathmoveto{\pgfqpoint{3.750691in}{1.793138in}}%
\pgfpathcurveto{\pgfqpoint{3.757824in}{1.793138in}}{\pgfqpoint{3.764665in}{1.795972in}}{\pgfqpoint{3.769709in}{1.801015in}}%
\pgfpathcurveto{\pgfqpoint{3.774753in}{1.806059in}}{\pgfqpoint{3.777587in}{1.812901in}}{\pgfqpoint{3.777587in}{1.820033in}}%
\pgfpathcurveto{\pgfqpoint{3.777587in}{1.827166in}}{\pgfqpoint{3.774753in}{1.834008in}}{\pgfqpoint{3.769709in}{1.839052in}}%
\pgfpathcurveto{\pgfqpoint{3.764665in}{1.844095in}}{\pgfqpoint{3.757824in}{1.846929in}}{\pgfqpoint{3.750691in}{1.846929in}}%
\pgfpathcurveto{\pgfqpoint{3.743558in}{1.846929in}}{\pgfqpoint{3.736716in}{1.844095in}}{\pgfqpoint{3.731673in}{1.839052in}}%
\pgfpathcurveto{\pgfqpoint{3.726629in}{1.834008in}}{\pgfqpoint{3.723795in}{1.827166in}}{\pgfqpoint{3.723795in}{1.820033in}}%
\pgfpathcurveto{\pgfqpoint{3.723795in}{1.812901in}}{\pgfqpoint{3.726629in}{1.806059in}}{\pgfqpoint{3.731673in}{1.801015in}}%
\pgfpathcurveto{\pgfqpoint{3.736716in}{1.795972in}}{\pgfqpoint{3.743558in}{1.793138in}}{\pgfqpoint{3.750691in}{1.793138in}}%
\pgfpathclose%
\pgfusepath{stroke,fill}%
\end{pgfscope}%
\begin{pgfscope}%
\pgfpathrectangle{\pgfqpoint{2.867647in}{0.500000in}}{\pgfqpoint{1.764706in}{1.700000in}}%
\pgfusepath{clip}%
\pgfsetbuttcap%
\pgfsetroundjoin%
\definecolor{currentfill}{rgb}{0.963190,0.619109,0.458249}%
\pgfsetfillcolor{currentfill}%
\pgfsetlinewidth{0.311001pt}%
\definecolor{currentstroke}{rgb}{1.000000,1.000000,1.000000}%
\pgfsetstrokecolor{currentstroke}%
\pgfsetdash{}{0pt}%
\pgfpathmoveto{\pgfqpoint{3.950944in}{1.746019in}}%
\pgfpathcurveto{\pgfqpoint{3.958077in}{1.746019in}}{\pgfqpoint{3.964919in}{1.748853in}}{\pgfqpoint{3.969962in}{1.753897in}}%
\pgfpathcurveto{\pgfqpoint{3.975006in}{1.758941in}}{\pgfqpoint{3.977840in}{1.765782in}}{\pgfqpoint{3.977840in}{1.772915in}}%
\pgfpathcurveto{\pgfqpoint{3.977840in}{1.780048in}}{\pgfqpoint{3.975006in}{1.786889in}}{\pgfqpoint{3.969962in}{1.791933in}}%
\pgfpathcurveto{\pgfqpoint{3.964919in}{1.796977in}}{\pgfqpoint{3.958077in}{1.799811in}}{\pgfqpoint{3.950944in}{1.799811in}}%
\pgfpathcurveto{\pgfqpoint{3.943811in}{1.799811in}}{\pgfqpoint{3.936970in}{1.796977in}}{\pgfqpoint{3.931926in}{1.791933in}}%
\pgfpathcurveto{\pgfqpoint{3.926882in}{1.786889in}}{\pgfqpoint{3.924048in}{1.780048in}}{\pgfqpoint{3.924048in}{1.772915in}}%
\pgfpathcurveto{\pgfqpoint{3.924048in}{1.765782in}}{\pgfqpoint{3.926882in}{1.758941in}}{\pgfqpoint{3.931926in}{1.753897in}}%
\pgfpathcurveto{\pgfqpoint{3.936970in}{1.748853in}}{\pgfqpoint{3.943811in}{1.746019in}}{\pgfqpoint{3.950944in}{1.746019in}}%
\pgfpathclose%
\pgfusepath{stroke,fill}%
\end{pgfscope}%
\begin{pgfscope}%
\pgfpathrectangle{\pgfqpoint{2.867647in}{0.500000in}}{\pgfqpoint{1.764706in}{1.700000in}}%
\pgfusepath{clip}%
\pgfsetbuttcap%
\pgfsetroundjoin%
\definecolor{currentfill}{rgb}{0.966560,0.756582,0.625273}%
\pgfsetfillcolor{currentfill}%
\pgfsetlinewidth{0.311001pt}%
\definecolor{currentstroke}{rgb}{1.000000,1.000000,1.000000}%
\pgfsetstrokecolor{currentstroke}%
\pgfsetdash{}{0pt}%
\pgfpathmoveto{\pgfqpoint{4.088130in}{1.348567in}}%
\pgfpathcurveto{\pgfqpoint{4.095263in}{1.348567in}}{\pgfqpoint{4.102105in}{1.351401in}}{\pgfqpoint{4.107148in}{1.356444in}}%
\pgfpathcurveto{\pgfqpoint{4.112192in}{1.361488in}}{\pgfqpoint{4.115026in}{1.368330in}}{\pgfqpoint{4.115026in}{1.375462in}}%
\pgfpathcurveto{\pgfqpoint{4.115026in}{1.382595in}}{\pgfqpoint{4.112192in}{1.389437in}}{\pgfqpoint{4.107148in}{1.394481in}}%
\pgfpathcurveto{\pgfqpoint{4.102105in}{1.399524in}}{\pgfqpoint{4.095263in}{1.402358in}}{\pgfqpoint{4.088130in}{1.402358in}}%
\pgfpathcurveto{\pgfqpoint{4.080997in}{1.402358in}}{\pgfqpoint{4.074156in}{1.399524in}}{\pgfqpoint{4.069112in}{1.394481in}}%
\pgfpathcurveto{\pgfqpoint{4.064068in}{1.389437in}}{\pgfqpoint{4.061235in}{1.382595in}}{\pgfqpoint{4.061235in}{1.375462in}}%
\pgfpathcurveto{\pgfqpoint{4.061235in}{1.368330in}}{\pgfqpoint{4.064068in}{1.361488in}}{\pgfqpoint{4.069112in}{1.356444in}}%
\pgfpathcurveto{\pgfqpoint{4.074156in}{1.351401in}}{\pgfqpoint{4.080997in}{1.348567in}}{\pgfqpoint{4.088130in}{1.348567in}}%
\pgfpathclose%
\pgfusepath{stroke,fill}%
\end{pgfscope}%
\begin{pgfscope}%
\pgfpathrectangle{\pgfqpoint{2.867647in}{0.500000in}}{\pgfqpoint{1.764706in}{1.700000in}}%
\pgfusepath{clip}%
\pgfsetbuttcap%
\pgfsetroundjoin%
\definecolor{currentfill}{rgb}{0.964306,0.663930,0.507747}%
\pgfsetfillcolor{currentfill}%
\pgfsetlinewidth{0.311001pt}%
\definecolor{currentstroke}{rgb}{1.000000,1.000000,1.000000}%
\pgfsetstrokecolor{currentstroke}%
\pgfsetdash{}{0pt}%
\pgfpathmoveto{\pgfqpoint{4.298409in}{1.482742in}}%
\pgfpathcurveto{\pgfqpoint{4.305541in}{1.482742in}}{\pgfqpoint{4.312383in}{1.485576in}}{\pgfqpoint{4.317427in}{1.490620in}}%
\pgfpathcurveto{\pgfqpoint{4.322470in}{1.495664in}}{\pgfqpoint{4.325304in}{1.502505in}}{\pgfqpoint{4.325304in}{1.509638in}}%
\pgfpathcurveto{\pgfqpoint{4.325304in}{1.516771in}}{\pgfqpoint{4.322470in}{1.523613in}}{\pgfqpoint{4.317427in}{1.528656in}}%
\pgfpathcurveto{\pgfqpoint{4.312383in}{1.533700in}}{\pgfqpoint{4.305541in}{1.536534in}}{\pgfqpoint{4.298409in}{1.536534in}}%
\pgfpathcurveto{\pgfqpoint{4.291276in}{1.536534in}}{\pgfqpoint{4.284434in}{1.533700in}}{\pgfqpoint{4.279390in}{1.528656in}}%
\pgfpathcurveto{\pgfqpoint{4.274347in}{1.523613in}}{\pgfqpoint{4.271513in}{1.516771in}}{\pgfqpoint{4.271513in}{1.509638in}}%
\pgfpathcurveto{\pgfqpoint{4.271513in}{1.502505in}}{\pgfqpoint{4.274347in}{1.495664in}}{\pgfqpoint{4.279390in}{1.490620in}}%
\pgfpathcurveto{\pgfqpoint{4.284434in}{1.485576in}}{\pgfqpoint{4.291276in}{1.482742in}}{\pgfqpoint{4.298409in}{1.482742in}}%
\pgfpathclose%
\pgfusepath{stroke,fill}%
\end{pgfscope}%
\begin{pgfscope}%
\pgfpathrectangle{\pgfqpoint{2.867647in}{0.500000in}}{\pgfqpoint{1.764706in}{1.700000in}}%
\pgfusepath{clip}%
\pgfsetbuttcap%
\pgfsetroundjoin%
\definecolor{currentfill}{rgb}{0.963728,0.638439,0.479050}%
\pgfsetfillcolor{currentfill}%
\pgfsetlinewidth{0.311001pt}%
\definecolor{currentstroke}{rgb}{1.000000,1.000000,1.000000}%
\pgfsetstrokecolor{currentstroke}%
\pgfsetdash{}{0pt}%
\pgfpathmoveto{\pgfqpoint{4.053061in}{1.250566in}}%
\pgfpathcurveto{\pgfqpoint{4.060193in}{1.250566in}}{\pgfqpoint{4.067035in}{1.253400in}}{\pgfqpoint{4.072079in}{1.258444in}}%
\pgfpathcurveto{\pgfqpoint{4.077122in}{1.263488in}}{\pgfqpoint{4.079956in}{1.270329in}}{\pgfqpoint{4.079956in}{1.277462in}}%
\pgfpathcurveto{\pgfqpoint{4.079956in}{1.284595in}}{\pgfqpoint{4.077122in}{1.291436in}}{\pgfqpoint{4.072079in}{1.296480in}}%
\pgfpathcurveto{\pgfqpoint{4.067035in}{1.301524in}}{\pgfqpoint{4.060193in}{1.304358in}}{\pgfqpoint{4.053061in}{1.304358in}}%
\pgfpathcurveto{\pgfqpoint{4.045928in}{1.304358in}}{\pgfqpoint{4.039086in}{1.301524in}}{\pgfqpoint{4.034043in}{1.296480in}}%
\pgfpathcurveto{\pgfqpoint{4.028999in}{1.291436in}}{\pgfqpoint{4.026165in}{1.284595in}}{\pgfqpoint{4.026165in}{1.277462in}}%
\pgfpathcurveto{\pgfqpoint{4.026165in}{1.270329in}}{\pgfqpoint{4.028999in}{1.263488in}}{\pgfqpoint{4.034043in}{1.258444in}}%
\pgfpathcurveto{\pgfqpoint{4.039086in}{1.253400in}}{\pgfqpoint{4.045928in}{1.250566in}}{\pgfqpoint{4.053061in}{1.250566in}}%
\pgfpathclose%
\pgfusepath{stroke,fill}%
\end{pgfscope}%
\begin{pgfscope}%
\pgfpathrectangle{\pgfqpoint{2.867647in}{0.500000in}}{\pgfqpoint{1.764706in}{1.700000in}}%
\pgfusepath{clip}%
\pgfsetbuttcap%
\pgfsetroundjoin%
\definecolor{currentfill}{rgb}{0.974412,0.862387,0.780156}%
\pgfsetfillcolor{currentfill}%
\pgfsetlinewidth{0.311001pt}%
\definecolor{currentstroke}{rgb}{1.000000,1.000000,1.000000}%
\pgfsetstrokecolor{currentstroke}%
\pgfsetdash{}{0pt}%
\pgfpathmoveto{\pgfqpoint{4.126834in}{1.005021in}}%
\pgfpathcurveto{\pgfqpoint{4.133967in}{1.005021in}}{\pgfqpoint{4.140809in}{1.007855in}}{\pgfqpoint{4.145852in}{1.012899in}}%
\pgfpathcurveto{\pgfqpoint{4.150896in}{1.017942in}}{\pgfqpoint{4.153730in}{1.024784in}}{\pgfqpoint{4.153730in}{1.031917in}}%
\pgfpathcurveto{\pgfqpoint{4.153730in}{1.039050in}}{\pgfqpoint{4.150896in}{1.045891in}}{\pgfqpoint{4.145852in}{1.050935in}}%
\pgfpathcurveto{\pgfqpoint{4.140809in}{1.055979in}}{\pgfqpoint{4.133967in}{1.058813in}}{\pgfqpoint{4.126834in}{1.058813in}}%
\pgfpathcurveto{\pgfqpoint{4.119701in}{1.058813in}}{\pgfqpoint{4.112860in}{1.055979in}}{\pgfqpoint{4.107816in}{1.050935in}}%
\pgfpathcurveto{\pgfqpoint{4.102772in}{1.045891in}}{\pgfqpoint{4.099938in}{1.039050in}}{\pgfqpoint{4.099938in}{1.031917in}}%
\pgfpathcurveto{\pgfqpoint{4.099938in}{1.024784in}}{\pgfqpoint{4.102772in}{1.017942in}}{\pgfqpoint{4.107816in}{1.012899in}}%
\pgfpathcurveto{\pgfqpoint{4.112860in}{1.007855in}}{\pgfqpoint{4.119701in}{1.005021in}}{\pgfqpoint{4.126834in}{1.005021in}}%
\pgfpathclose%
\pgfusepath{stroke,fill}%
\end{pgfscope}%
\begin{pgfscope}%
\pgfpathrectangle{\pgfqpoint{2.867647in}{0.500000in}}{\pgfqpoint{1.764706in}{1.700000in}}%
\pgfusepath{clip}%
\pgfsetbuttcap%
\pgfsetroundjoin%
\definecolor{currentfill}{rgb}{0.843354,0.145567,0.284808}%
\pgfsetfillcolor{currentfill}%
\pgfsetlinewidth{0.311001pt}%
\definecolor{currentstroke}{rgb}{1.000000,1.000000,1.000000}%
\pgfsetstrokecolor{currentstroke}%
\pgfsetdash{}{0pt}%
\pgfpathmoveto{\pgfqpoint{4.374691in}{1.104554in}}%
\pgfpathcurveto{\pgfqpoint{4.381824in}{1.104554in}}{\pgfqpoint{4.388665in}{1.107388in}}{\pgfqpoint{4.393709in}{1.112431in}}%
\pgfpathcurveto{\pgfqpoint{4.398753in}{1.117475in}}{\pgfqpoint{4.401587in}{1.124317in}}{\pgfqpoint{4.401587in}{1.131450in}}%
\pgfpathcurveto{\pgfqpoint{4.401587in}{1.138582in}}{\pgfqpoint{4.398753in}{1.145424in}}{\pgfqpoint{4.393709in}{1.150468in}}%
\pgfpathcurveto{\pgfqpoint{4.388665in}{1.155511in}}{\pgfqpoint{4.381824in}{1.158345in}}{\pgfqpoint{4.374691in}{1.158345in}}%
\pgfpathcurveto{\pgfqpoint{4.367558in}{1.158345in}}{\pgfqpoint{4.360717in}{1.155511in}}{\pgfqpoint{4.355673in}{1.150468in}}%
\pgfpathcurveto{\pgfqpoint{4.350629in}{1.145424in}}{\pgfqpoint{4.347795in}{1.138582in}}{\pgfqpoint{4.347795in}{1.131450in}}%
\pgfpathcurveto{\pgfqpoint{4.347795in}{1.124317in}}{\pgfqpoint{4.350629in}{1.117475in}}{\pgfqpoint{4.355673in}{1.112431in}}%
\pgfpathcurveto{\pgfqpoint{4.360717in}{1.107388in}}{\pgfqpoint{4.367558in}{1.104554in}}{\pgfqpoint{4.374691in}{1.104554in}}%
\pgfpathclose%
\pgfusepath{stroke,fill}%
\end{pgfscope}%
\begin{pgfscope}%
\pgfpathrectangle{\pgfqpoint{2.867647in}{0.500000in}}{\pgfqpoint{1.764706in}{1.700000in}}%
\pgfusepath{clip}%
\pgfsetbuttcap%
\pgfsetroundjoin%
\definecolor{currentfill}{rgb}{0.961433,0.573272,0.412036}%
\pgfsetfillcolor{currentfill}%
\pgfsetlinewidth{0.311001pt}%
\definecolor{currentstroke}{rgb}{1.000000,1.000000,1.000000}%
\pgfsetstrokecolor{currentstroke}%
\pgfsetdash{}{0pt}%
\pgfpathmoveto{\pgfqpoint{3.968071in}{1.597157in}}%
\pgfpathcurveto{\pgfqpoint{3.975203in}{1.597157in}}{\pgfqpoint{3.982045in}{1.599991in}}{\pgfqpoint{3.987089in}{1.605035in}}%
\pgfpathcurveto{\pgfqpoint{3.992132in}{1.610079in}}{\pgfqpoint{3.994966in}{1.616920in}}{\pgfqpoint{3.994966in}{1.624053in}}%
\pgfpathcurveto{\pgfqpoint{3.994966in}{1.631186in}}{\pgfqpoint{3.992132in}{1.638028in}}{\pgfqpoint{3.987089in}{1.643071in}}%
\pgfpathcurveto{\pgfqpoint{3.982045in}{1.648115in}}{\pgfqpoint{3.975203in}{1.650949in}}{\pgfqpoint{3.968071in}{1.650949in}}%
\pgfpathcurveto{\pgfqpoint{3.960938in}{1.650949in}}{\pgfqpoint{3.954096in}{1.648115in}}{\pgfqpoint{3.949052in}{1.643071in}}%
\pgfpathcurveto{\pgfqpoint{3.944009in}{1.638028in}}{\pgfqpoint{3.941175in}{1.631186in}}{\pgfqpoint{3.941175in}{1.624053in}}%
\pgfpathcurveto{\pgfqpoint{3.941175in}{1.616920in}}{\pgfqpoint{3.944009in}{1.610079in}}{\pgfqpoint{3.949052in}{1.605035in}}%
\pgfpathcurveto{\pgfqpoint{3.954096in}{1.599991in}}{\pgfqpoint{3.960938in}{1.597157in}}{\pgfqpoint{3.968071in}{1.597157in}}%
\pgfpathclose%
\pgfusepath{stroke,fill}%
\end{pgfscope}%
\begin{pgfscope}%
\pgfpathrectangle{\pgfqpoint{2.867647in}{0.500000in}}{\pgfqpoint{1.764706in}{1.700000in}}%
\pgfusepath{clip}%
\pgfsetbuttcap%
\pgfsetroundjoin%
\definecolor{currentfill}{rgb}{0.973271,0.850724,0.762998}%
\pgfsetfillcolor{currentfill}%
\pgfsetlinewidth{0.311001pt}%
\definecolor{currentstroke}{rgb}{1.000000,1.000000,1.000000}%
\pgfsetstrokecolor{currentstroke}%
\pgfsetdash{}{0pt}%
\pgfpathmoveto{\pgfqpoint{4.074718in}{1.056816in}}%
\pgfpathcurveto{\pgfqpoint{4.081851in}{1.056816in}}{\pgfqpoint{4.088693in}{1.059650in}}{\pgfqpoint{4.093736in}{1.064694in}}%
\pgfpathcurveto{\pgfqpoint{4.098780in}{1.069738in}}{\pgfqpoint{4.101614in}{1.076579in}}{\pgfqpoint{4.101614in}{1.083712in}}%
\pgfpathcurveto{\pgfqpoint{4.101614in}{1.090845in}}{\pgfqpoint{4.098780in}{1.097687in}}{\pgfqpoint{4.093736in}{1.102730in}}%
\pgfpathcurveto{\pgfqpoint{4.088693in}{1.107774in}}{\pgfqpoint{4.081851in}{1.110608in}}{\pgfqpoint{4.074718in}{1.110608in}}%
\pgfpathcurveto{\pgfqpoint{4.067585in}{1.110608in}}{\pgfqpoint{4.060744in}{1.107774in}}{\pgfqpoint{4.055700in}{1.102730in}}%
\pgfpathcurveto{\pgfqpoint{4.050656in}{1.097687in}}{\pgfqpoint{4.047822in}{1.090845in}}{\pgfqpoint{4.047822in}{1.083712in}}%
\pgfpathcurveto{\pgfqpoint{4.047822in}{1.076579in}}{\pgfqpoint{4.050656in}{1.069738in}}{\pgfqpoint{4.055700in}{1.064694in}}%
\pgfpathcurveto{\pgfqpoint{4.060744in}{1.059650in}}{\pgfqpoint{4.067585in}{1.056816in}}{\pgfqpoint{4.074718in}{1.056816in}}%
\pgfpathclose%
\pgfusepath{stroke,fill}%
\end{pgfscope}%
\begin{pgfscope}%
\pgfpathrectangle{\pgfqpoint{2.867647in}{0.500000in}}{\pgfqpoint{1.764706in}{1.700000in}}%
\pgfusepath{clip}%
\pgfsetbuttcap%
\pgfsetroundjoin%
\definecolor{currentfill}{rgb}{0.956268,0.491874,0.339856}%
\pgfsetfillcolor{currentfill}%
\pgfsetlinewidth{0.311001pt}%
\definecolor{currentstroke}{rgb}{1.000000,1.000000,1.000000}%
\pgfsetstrokecolor{currentstroke}%
\pgfsetdash{}{0pt}%
\pgfpathmoveto{\pgfqpoint{4.014007in}{1.838089in}}%
\pgfpathcurveto{\pgfqpoint{4.021140in}{1.838089in}}{\pgfqpoint{4.027981in}{1.840923in}}{\pgfqpoint{4.033025in}{1.845966in}}%
\pgfpathcurveto{\pgfqpoint{4.038069in}{1.851010in}}{\pgfqpoint{4.040902in}{1.857852in}}{\pgfqpoint{4.040902in}{1.864984in}}%
\pgfpathcurveto{\pgfqpoint{4.040902in}{1.872117in}}{\pgfqpoint{4.038069in}{1.878959in}}{\pgfqpoint{4.033025in}{1.884003in}}%
\pgfpathcurveto{\pgfqpoint{4.027981in}{1.889046in}}{\pgfqpoint{4.021140in}{1.891880in}}{\pgfqpoint{4.014007in}{1.891880in}}%
\pgfpathcurveto{\pgfqpoint{4.006874in}{1.891880in}}{\pgfqpoint{4.000032in}{1.889046in}}{\pgfqpoint{3.994989in}{1.884003in}}%
\pgfpathcurveto{\pgfqpoint{3.989945in}{1.878959in}}{\pgfqpoint{3.987111in}{1.872117in}}{\pgfqpoint{3.987111in}{1.864984in}}%
\pgfpathcurveto{\pgfqpoint{3.987111in}{1.857852in}}{\pgfqpoint{3.989945in}{1.851010in}}{\pgfqpoint{3.994989in}{1.845966in}}%
\pgfpathcurveto{\pgfqpoint{4.000032in}{1.840923in}}{\pgfqpoint{4.006874in}{1.838089in}}{\pgfqpoint{4.014007in}{1.838089in}}%
\pgfpathclose%
\pgfusepath{stroke,fill}%
\end{pgfscope}%
\begin{pgfscope}%
\pgfpathrectangle{\pgfqpoint{2.867647in}{0.500000in}}{\pgfqpoint{1.764706in}{1.700000in}}%
\pgfusepath{clip}%
\pgfsetbuttcap%
\pgfsetroundjoin%
\definecolor{currentfill}{rgb}{0.970718,0.821518,0.719872}%
\pgfsetfillcolor{currentfill}%
\pgfsetlinewidth{0.311001pt}%
\definecolor{currentstroke}{rgb}{1.000000,1.000000,1.000000}%
\pgfsetstrokecolor{currentstroke}%
\pgfsetdash{}{0pt}%
\pgfpathmoveto{\pgfqpoint{4.049184in}{1.005776in}}%
\pgfpathcurveto{\pgfqpoint{4.056316in}{1.005776in}}{\pgfqpoint{4.063158in}{1.008610in}}{\pgfqpoint{4.068202in}{1.013653in}}%
\pgfpathcurveto{\pgfqpoint{4.073245in}{1.018697in}}{\pgfqpoint{4.076079in}{1.025539in}}{\pgfqpoint{4.076079in}{1.032671in}}%
\pgfpathcurveto{\pgfqpoint{4.076079in}{1.039804in}}{\pgfqpoint{4.073245in}{1.046646in}}{\pgfqpoint{4.068202in}{1.051690in}}%
\pgfpathcurveto{\pgfqpoint{4.063158in}{1.056733in}}{\pgfqpoint{4.056316in}{1.059567in}}{\pgfqpoint{4.049184in}{1.059567in}}%
\pgfpathcurveto{\pgfqpoint{4.042051in}{1.059567in}}{\pgfqpoint{4.035209in}{1.056733in}}{\pgfqpoint{4.030165in}{1.051690in}}%
\pgfpathcurveto{\pgfqpoint{4.025122in}{1.046646in}}{\pgfqpoint{4.022288in}{1.039804in}}{\pgfqpoint{4.022288in}{1.032671in}}%
\pgfpathcurveto{\pgfqpoint{4.022288in}{1.025539in}}{\pgfqpoint{4.025122in}{1.018697in}}{\pgfqpoint{4.030165in}{1.013653in}}%
\pgfpathcurveto{\pgfqpoint{4.035209in}{1.008610in}}{\pgfqpoint{4.042051in}{1.005776in}}{\pgfqpoint{4.049184in}{1.005776in}}%
\pgfpathclose%
\pgfusepath{stroke,fill}%
\end{pgfscope}%
\begin{pgfscope}%
\pgfpathrectangle{\pgfqpoint{2.867647in}{0.500000in}}{\pgfqpoint{1.764706in}{1.700000in}}%
\pgfusepath{clip}%
\pgfsetbuttcap%
\pgfsetroundjoin%
\definecolor{currentfill}{rgb}{0.970718,0.821518,0.719872}%
\pgfsetfillcolor{currentfill}%
\pgfsetlinewidth{0.311001pt}%
\definecolor{currentstroke}{rgb}{1.000000,1.000000,1.000000}%
\pgfsetstrokecolor{currentstroke}%
\pgfsetdash{}{0pt}%
\pgfpathmoveto{\pgfqpoint{4.054893in}{1.051115in}}%
\pgfpathcurveto{\pgfqpoint{4.062026in}{1.051115in}}{\pgfqpoint{4.068867in}{1.053949in}}{\pgfqpoint{4.073911in}{1.058992in}}%
\pgfpathcurveto{\pgfqpoint{4.078955in}{1.064036in}}{\pgfqpoint{4.081788in}{1.070878in}}{\pgfqpoint{4.081788in}{1.078010in}}%
\pgfpathcurveto{\pgfqpoint{4.081788in}{1.085143in}}{\pgfqpoint{4.078955in}{1.091985in}}{\pgfqpoint{4.073911in}{1.097029in}}%
\pgfpathcurveto{\pgfqpoint{4.068867in}{1.102072in}}{\pgfqpoint{4.062026in}{1.104906in}}{\pgfqpoint{4.054893in}{1.104906in}}%
\pgfpathcurveto{\pgfqpoint{4.047760in}{1.104906in}}{\pgfqpoint{4.040918in}{1.102072in}}{\pgfqpoint{4.035875in}{1.097029in}}%
\pgfpathcurveto{\pgfqpoint{4.030831in}{1.091985in}}{\pgfqpoint{4.027997in}{1.085143in}}{\pgfqpoint{4.027997in}{1.078010in}}%
\pgfpathcurveto{\pgfqpoint{4.027997in}{1.070878in}}{\pgfqpoint{4.030831in}{1.064036in}}{\pgfqpoint{4.035875in}{1.058992in}}%
\pgfpathcurveto{\pgfqpoint{4.040918in}{1.053949in}}{\pgfqpoint{4.047760in}{1.051115in}}{\pgfqpoint{4.054893in}{1.051115in}}%
\pgfpathclose%
\pgfusepath{stroke,fill}%
\end{pgfscope}%
\begin{pgfscope}%
\pgfpathrectangle{\pgfqpoint{2.867647in}{0.500000in}}{\pgfqpoint{1.764706in}{1.700000in}}%
\pgfusepath{clip}%
\pgfsetbuttcap%
\pgfsetroundjoin%
\definecolor{currentfill}{rgb}{0.956817,0.498820,0.345554}%
\pgfsetfillcolor{currentfill}%
\pgfsetlinewidth{0.311001pt}%
\definecolor{currentstroke}{rgb}{1.000000,1.000000,1.000000}%
\pgfsetstrokecolor{currentstroke}%
\pgfsetdash{}{0pt}%
\pgfpathmoveto{\pgfqpoint{4.051393in}{0.822189in}}%
\pgfpathcurveto{\pgfqpoint{4.058526in}{0.822189in}}{\pgfqpoint{4.065368in}{0.825023in}}{\pgfqpoint{4.070411in}{0.830066in}}%
\pgfpathcurveto{\pgfqpoint{4.075455in}{0.835110in}}{\pgfqpoint{4.078289in}{0.841952in}}{\pgfqpoint{4.078289in}{0.849084in}}%
\pgfpathcurveto{\pgfqpoint{4.078289in}{0.856217in}}{\pgfqpoint{4.075455in}{0.863059in}}{\pgfqpoint{4.070411in}{0.868103in}}%
\pgfpathcurveto{\pgfqpoint{4.065368in}{0.873146in}}{\pgfqpoint{4.058526in}{0.875980in}}{\pgfqpoint{4.051393in}{0.875980in}}%
\pgfpathcurveto{\pgfqpoint{4.044260in}{0.875980in}}{\pgfqpoint{4.037419in}{0.873146in}}{\pgfqpoint{4.032375in}{0.868103in}}%
\pgfpathcurveto{\pgfqpoint{4.027331in}{0.863059in}}{\pgfqpoint{4.024497in}{0.856217in}}{\pgfqpoint{4.024497in}{0.849084in}}%
\pgfpathcurveto{\pgfqpoint{4.024497in}{0.841952in}}{\pgfqpoint{4.027331in}{0.835110in}}{\pgfqpoint{4.032375in}{0.830066in}}%
\pgfpathcurveto{\pgfqpoint{4.037419in}{0.825023in}}{\pgfqpoint{4.044260in}{0.822189in}}{\pgfqpoint{4.051393in}{0.822189in}}%
\pgfpathclose%
\pgfusepath{stroke,fill}%
\end{pgfscope}%
\begin{pgfscope}%
\pgfpathrectangle{\pgfqpoint{2.867647in}{0.500000in}}{\pgfqpoint{1.764706in}{1.700000in}}%
\pgfusepath{clip}%
\pgfsetbuttcap%
\pgfsetroundjoin%
\definecolor{currentfill}{rgb}{0.979124,0.903132,0.839793}%
\pgfsetfillcolor{currentfill}%
\pgfsetlinewidth{0.311001pt}%
\definecolor{currentstroke}{rgb}{1.000000,1.000000,1.000000}%
\pgfsetstrokecolor{currentstroke}%
\pgfsetdash{}{0pt}%
\pgfpathmoveto{\pgfqpoint{4.136090in}{1.480074in}}%
\pgfpathcurveto{\pgfqpoint{4.143223in}{1.480074in}}{\pgfqpoint{4.150065in}{1.482908in}}{\pgfqpoint{4.155108in}{1.487952in}}%
\pgfpathcurveto{\pgfqpoint{4.160152in}{1.492995in}}{\pgfqpoint{4.162986in}{1.499837in}}{\pgfqpoint{4.162986in}{1.506970in}}%
\pgfpathcurveto{\pgfqpoint{4.162986in}{1.514103in}}{\pgfqpoint{4.160152in}{1.520944in}}{\pgfqpoint{4.155108in}{1.525988in}}%
\pgfpathcurveto{\pgfqpoint{4.150065in}{1.531032in}}{\pgfqpoint{4.143223in}{1.533865in}}{\pgfqpoint{4.136090in}{1.533865in}}%
\pgfpathcurveto{\pgfqpoint{4.128957in}{1.533865in}}{\pgfqpoint{4.122116in}{1.531032in}}{\pgfqpoint{4.117072in}{1.525988in}}%
\pgfpathcurveto{\pgfqpoint{4.112028in}{1.520944in}}{\pgfqpoint{4.109194in}{1.514103in}}{\pgfqpoint{4.109194in}{1.506970in}}%
\pgfpathcurveto{\pgfqpoint{4.109194in}{1.499837in}}{\pgfqpoint{4.112028in}{1.492995in}}{\pgfqpoint{4.117072in}{1.487952in}}%
\pgfpathcurveto{\pgfqpoint{4.122116in}{1.482908in}}{\pgfqpoint{4.128957in}{1.480074in}}{\pgfqpoint{4.136090in}{1.480074in}}%
\pgfpathclose%
\pgfusepath{stroke,fill}%
\end{pgfscope}%
\begin{pgfscope}%
\pgfpathrectangle{\pgfqpoint{2.867647in}{0.500000in}}{\pgfqpoint{1.764706in}{1.700000in}}%
\pgfusepath{clip}%
\pgfsetbuttcap%
\pgfsetroundjoin%
\definecolor{currentfill}{rgb}{0.972726,0.844889,0.754401}%
\pgfsetfillcolor{currentfill}%
\pgfsetlinewidth{0.311001pt}%
\definecolor{currentstroke}{rgb}{1.000000,1.000000,1.000000}%
\pgfsetstrokecolor{currentstroke}%
\pgfsetdash{}{0pt}%
\pgfpathmoveto{\pgfqpoint{4.075926in}{0.995279in}}%
\pgfpathcurveto{\pgfqpoint{4.083059in}{0.995279in}}{\pgfqpoint{4.089901in}{0.998113in}}{\pgfqpoint{4.094944in}{1.003156in}}%
\pgfpathcurveto{\pgfqpoint{4.099988in}{1.008200in}}{\pgfqpoint{4.102822in}{1.015042in}}{\pgfqpoint{4.102822in}{1.022175in}}%
\pgfpathcurveto{\pgfqpoint{4.102822in}{1.029307in}}{\pgfqpoint{4.099988in}{1.036149in}}{\pgfqpoint{4.094944in}{1.041193in}}%
\pgfpathcurveto{\pgfqpoint{4.089901in}{1.046236in}}{\pgfqpoint{4.083059in}{1.049070in}}{\pgfqpoint{4.075926in}{1.049070in}}%
\pgfpathcurveto{\pgfqpoint{4.068793in}{1.049070in}}{\pgfqpoint{4.061952in}{1.046236in}}{\pgfqpoint{4.056908in}{1.041193in}}%
\pgfpathcurveto{\pgfqpoint{4.051864in}{1.036149in}}{\pgfqpoint{4.049030in}{1.029307in}}{\pgfqpoint{4.049030in}{1.022175in}}%
\pgfpathcurveto{\pgfqpoint{4.049030in}{1.015042in}}{\pgfqpoint{4.051864in}{1.008200in}}{\pgfqpoint{4.056908in}{1.003156in}}%
\pgfpathcurveto{\pgfqpoint{4.061952in}{0.998113in}}{\pgfqpoint{4.068793in}{0.995279in}}{\pgfqpoint{4.075926in}{0.995279in}}%
\pgfpathclose%
\pgfusepath{stroke,fill}%
\end{pgfscope}%
\begin{pgfscope}%
\pgfpathrectangle{\pgfqpoint{2.867647in}{0.500000in}}{\pgfqpoint{1.764706in}{1.700000in}}%
\pgfusepath{clip}%
\pgfsetbuttcap%
\pgfsetroundjoin%
\definecolor{currentfill}{rgb}{0.964173,0.657587,0.500469}%
\pgfsetfillcolor{currentfill}%
\pgfsetlinewidth{0.311001pt}%
\definecolor{currentstroke}{rgb}{1.000000,1.000000,1.000000}%
\pgfsetstrokecolor{currentstroke}%
\pgfsetdash{}{0pt}%
\pgfpathmoveto{\pgfqpoint{4.027664in}{1.488417in}}%
\pgfpathcurveto{\pgfqpoint{4.034797in}{1.488417in}}{\pgfqpoint{4.041639in}{1.491251in}}{\pgfqpoint{4.046683in}{1.496294in}}%
\pgfpathcurveto{\pgfqpoint{4.051726in}{1.501338in}}{\pgfqpoint{4.054560in}{1.508180in}}{\pgfqpoint{4.054560in}{1.515313in}}%
\pgfpathcurveto{\pgfqpoint{4.054560in}{1.522445in}}{\pgfqpoint{4.051726in}{1.529287in}}{\pgfqpoint{4.046683in}{1.534331in}}%
\pgfpathcurveto{\pgfqpoint{4.041639in}{1.539374in}}{\pgfqpoint{4.034797in}{1.542208in}}{\pgfqpoint{4.027664in}{1.542208in}}%
\pgfpathcurveto{\pgfqpoint{4.020532in}{1.542208in}}{\pgfqpoint{4.013690in}{1.539374in}}{\pgfqpoint{4.008646in}{1.534331in}}%
\pgfpathcurveto{\pgfqpoint{4.003603in}{1.529287in}}{\pgfqpoint{4.000769in}{1.522445in}}{\pgfqpoint{4.000769in}{1.515313in}}%
\pgfpathcurveto{\pgfqpoint{4.000769in}{1.508180in}}{\pgfqpoint{4.003603in}{1.501338in}}{\pgfqpoint{4.008646in}{1.496294in}}%
\pgfpathcurveto{\pgfqpoint{4.013690in}{1.491251in}}{\pgfqpoint{4.020532in}{1.488417in}}{\pgfqpoint{4.027664in}{1.488417in}}%
\pgfpathclose%
\pgfusepath{stroke,fill}%
\end{pgfscope}%
\begin{pgfscope}%
\pgfpathrectangle{\pgfqpoint{2.867647in}{0.500000in}}{\pgfqpoint{1.764706in}{1.700000in}}%
\pgfusepath{clip}%
\pgfsetbuttcap%
\pgfsetroundjoin%
\definecolor{currentfill}{rgb}{0.981377,0.920617,0.865369}%
\pgfsetfillcolor{currentfill}%
\pgfsetlinewidth{0.311001pt}%
\definecolor{currentstroke}{rgb}{1.000000,1.000000,1.000000}%
\pgfsetstrokecolor{currentstroke}%
\pgfsetdash{}{0pt}%
\pgfpathmoveto{\pgfqpoint{4.161183in}{1.202262in}}%
\pgfpathcurveto{\pgfqpoint{4.168316in}{1.202262in}}{\pgfqpoint{4.175157in}{1.205095in}}{\pgfqpoint{4.180201in}{1.210139in}}%
\pgfpathcurveto{\pgfqpoint{4.185245in}{1.215183in}}{\pgfqpoint{4.188079in}{1.222024in}}{\pgfqpoint{4.188079in}{1.229157in}}%
\pgfpathcurveto{\pgfqpoint{4.188079in}{1.236290in}}{\pgfqpoint{4.185245in}{1.243132in}}{\pgfqpoint{4.180201in}{1.248175in}}%
\pgfpathcurveto{\pgfqpoint{4.175157in}{1.253219in}}{\pgfqpoint{4.168316in}{1.256053in}}{\pgfqpoint{4.161183in}{1.256053in}}%
\pgfpathcurveto{\pgfqpoint{4.154050in}{1.256053in}}{\pgfqpoint{4.147208in}{1.253219in}}{\pgfqpoint{4.142165in}{1.248175in}}%
\pgfpathcurveto{\pgfqpoint{4.137121in}{1.243132in}}{\pgfqpoint{4.134287in}{1.236290in}}{\pgfqpoint{4.134287in}{1.229157in}}%
\pgfpathcurveto{\pgfqpoint{4.134287in}{1.222024in}}{\pgfqpoint{4.137121in}{1.215183in}}{\pgfqpoint{4.142165in}{1.210139in}}%
\pgfpathcurveto{\pgfqpoint{4.147208in}{1.205095in}}{\pgfqpoint{4.154050in}{1.202262in}}{\pgfqpoint{4.161183in}{1.202262in}}%
\pgfpathclose%
\pgfusepath{stroke,fill}%
\end{pgfscope}%
\begin{pgfscope}%
\pgfpathrectangle{\pgfqpoint{2.867647in}{0.500000in}}{\pgfqpoint{1.764706in}{1.700000in}}%
\pgfusepath{clip}%
\pgfsetbuttcap%
\pgfsetroundjoin%
\definecolor{currentfill}{rgb}{0.010608,0.018082,0.100187}%
\pgfsetfillcolor{currentfill}%
\pgfsetlinewidth{0.311001pt}%
\definecolor{currentstroke}{rgb}{1.000000,1.000000,1.000000}%
\pgfsetstrokecolor{currentstroke}%
\pgfsetdash{}{0pt}%
\pgfpathmoveto{\pgfqpoint{4.521085in}{1.056396in}}%
\pgfpathcurveto{\pgfqpoint{4.528218in}{1.056396in}}{\pgfqpoint{4.535059in}{1.059230in}}{\pgfqpoint{4.540103in}{1.064274in}}%
\pgfpathcurveto{\pgfqpoint{4.545147in}{1.069317in}}{\pgfqpoint{4.547981in}{1.076159in}}{\pgfqpoint{4.547981in}{1.083292in}}%
\pgfpathcurveto{\pgfqpoint{4.547981in}{1.090425in}}{\pgfqpoint{4.545147in}{1.097266in}}{\pgfqpoint{4.540103in}{1.102310in}}%
\pgfpathcurveto{\pgfqpoint{4.535059in}{1.107354in}}{\pgfqpoint{4.528218in}{1.110188in}}{\pgfqpoint{4.521085in}{1.110188in}}%
\pgfpathcurveto{\pgfqpoint{4.513952in}{1.110188in}}{\pgfqpoint{4.507110in}{1.107354in}}{\pgfqpoint{4.502067in}{1.102310in}}%
\pgfpathcurveto{\pgfqpoint{4.497023in}{1.097266in}}{\pgfqpoint{4.494189in}{1.090425in}}{\pgfqpoint{4.494189in}{1.083292in}}%
\pgfpathcurveto{\pgfqpoint{4.494189in}{1.076159in}}{\pgfqpoint{4.497023in}{1.069317in}}{\pgfqpoint{4.502067in}{1.064274in}}%
\pgfpathcurveto{\pgfqpoint{4.507110in}{1.059230in}}{\pgfqpoint{4.513952in}{1.056396in}}{\pgfqpoint{4.521085in}{1.056396in}}%
\pgfpathclose%
\pgfusepath{stroke,fill}%
\end{pgfscope}%
\begin{pgfscope}%
\pgfpathrectangle{\pgfqpoint{2.867647in}{0.500000in}}{\pgfqpoint{1.764706in}{1.700000in}}%
\pgfusepath{clip}%
\pgfsetbuttcap%
\pgfsetroundjoin%
\definecolor{currentfill}{rgb}{0.970255,0.815666,0.711203}%
\pgfsetfillcolor{currentfill}%
\pgfsetlinewidth{0.311001pt}%
\definecolor{currentstroke}{rgb}{1.000000,1.000000,1.000000}%
\pgfsetstrokecolor{currentstroke}%
\pgfsetdash{}{0pt}%
\pgfpathmoveto{\pgfqpoint{4.088576in}{0.950829in}}%
\pgfpathcurveto{\pgfqpoint{4.095709in}{0.950829in}}{\pgfqpoint{4.102551in}{0.953663in}}{\pgfqpoint{4.107594in}{0.958707in}}%
\pgfpathcurveto{\pgfqpoint{4.112638in}{0.963750in}}{\pgfqpoint{4.115472in}{0.970592in}}{\pgfqpoint{4.115472in}{0.977725in}}%
\pgfpathcurveto{\pgfqpoint{4.115472in}{0.984857in}}{\pgfqpoint{4.112638in}{0.991699in}}{\pgfqpoint{4.107594in}{0.996743in}}%
\pgfpathcurveto{\pgfqpoint{4.102551in}{1.001786in}}{\pgfqpoint{4.095709in}{1.004620in}}{\pgfqpoint{4.088576in}{1.004620in}}%
\pgfpathcurveto{\pgfqpoint{4.081443in}{1.004620in}}{\pgfqpoint{4.074602in}{1.001786in}}{\pgfqpoint{4.069558in}{0.996743in}}%
\pgfpathcurveto{\pgfqpoint{4.064514in}{0.991699in}}{\pgfqpoint{4.061680in}{0.984857in}}{\pgfqpoint{4.061680in}{0.977725in}}%
\pgfpathcurveto{\pgfqpoint{4.061680in}{0.970592in}}{\pgfqpoint{4.064514in}{0.963750in}}{\pgfqpoint{4.069558in}{0.958707in}}%
\pgfpathcurveto{\pgfqpoint{4.074602in}{0.953663in}}{\pgfqpoint{4.081443in}{0.950829in}}{\pgfqpoint{4.088576in}{0.950829in}}%
\pgfpathclose%
\pgfusepath{stroke,fill}%
\end{pgfscope}%
\begin{pgfscope}%
\pgfpathrectangle{\pgfqpoint{2.867647in}{0.500000in}}{\pgfqpoint{1.764706in}{1.700000in}}%
\pgfusepath{clip}%
\pgfsetbuttcap%
\pgfsetroundjoin%
\definecolor{currentfill}{rgb}{0.972201,0.839051,0.745789}%
\pgfsetfillcolor{currentfill}%
\pgfsetlinewidth{0.311001pt}%
\definecolor{currentstroke}{rgb}{1.000000,1.000000,1.000000}%
\pgfsetstrokecolor{currentstroke}%
\pgfsetdash{}{0pt}%
\pgfpathmoveto{\pgfqpoint{4.255464in}{1.207282in}}%
\pgfpathcurveto{\pgfqpoint{4.262597in}{1.207282in}}{\pgfqpoint{4.269438in}{1.210116in}}{\pgfqpoint{4.274482in}{1.215160in}}%
\pgfpathcurveto{\pgfqpoint{4.279526in}{1.220203in}}{\pgfqpoint{4.282360in}{1.227045in}}{\pgfqpoint{4.282360in}{1.234178in}}%
\pgfpathcurveto{\pgfqpoint{4.282360in}{1.241311in}}{\pgfqpoint{4.279526in}{1.248152in}}{\pgfqpoint{4.274482in}{1.253196in}}%
\pgfpathcurveto{\pgfqpoint{4.269438in}{1.258240in}}{\pgfqpoint{4.262597in}{1.261074in}}{\pgfqpoint{4.255464in}{1.261074in}}%
\pgfpathcurveto{\pgfqpoint{4.248331in}{1.261074in}}{\pgfqpoint{4.241489in}{1.258240in}}{\pgfqpoint{4.236446in}{1.253196in}}%
\pgfpathcurveto{\pgfqpoint{4.231402in}{1.248152in}}{\pgfqpoint{4.228568in}{1.241311in}}{\pgfqpoint{4.228568in}{1.234178in}}%
\pgfpathcurveto{\pgfqpoint{4.228568in}{1.227045in}}{\pgfqpoint{4.231402in}{1.220203in}}{\pgfqpoint{4.236446in}{1.215160in}}%
\pgfpathcurveto{\pgfqpoint{4.241489in}{1.210116in}}{\pgfqpoint{4.248331in}{1.207282in}}{\pgfqpoint{4.255464in}{1.207282in}}%
\pgfpathclose%
\pgfusepath{stroke,fill}%
\end{pgfscope}%
\begin{pgfscope}%
\pgfpathrectangle{\pgfqpoint{2.867647in}{0.500000in}}{\pgfqpoint{1.764706in}{1.700000in}}%
\pgfusepath{clip}%
\pgfsetbuttcap%
\pgfsetroundjoin%
\definecolor{currentfill}{rgb}{0.972726,0.844889,0.754401}%
\pgfsetfillcolor{currentfill}%
\pgfsetlinewidth{0.311001pt}%
\definecolor{currentstroke}{rgb}{1.000000,1.000000,1.000000}%
\pgfsetstrokecolor{currentstroke}%
\pgfsetdash{}{0pt}%
\pgfpathmoveto{\pgfqpoint{4.140811in}{0.994383in}}%
\pgfpathcurveto{\pgfqpoint{4.147944in}{0.994383in}}{\pgfqpoint{4.154786in}{0.997217in}}{\pgfqpoint{4.159830in}{1.002261in}}%
\pgfpathcurveto{\pgfqpoint{4.164873in}{1.007305in}}{\pgfqpoint{4.167707in}{1.014146in}}{\pgfqpoint{4.167707in}{1.021279in}}%
\pgfpathcurveto{\pgfqpoint{4.167707in}{1.028412in}}{\pgfqpoint{4.164873in}{1.035254in}}{\pgfqpoint{4.159830in}{1.040297in}}%
\pgfpathcurveto{\pgfqpoint{4.154786in}{1.045341in}}{\pgfqpoint{4.147944in}{1.048175in}}{\pgfqpoint{4.140811in}{1.048175in}}%
\pgfpathcurveto{\pgfqpoint{4.133679in}{1.048175in}}{\pgfqpoint{4.126837in}{1.045341in}}{\pgfqpoint{4.121793in}{1.040297in}}%
\pgfpathcurveto{\pgfqpoint{4.116750in}{1.035254in}}{\pgfqpoint{4.113916in}{1.028412in}}{\pgfqpoint{4.113916in}{1.021279in}}%
\pgfpathcurveto{\pgfqpoint{4.113916in}{1.014146in}}{\pgfqpoint{4.116750in}{1.007305in}}{\pgfqpoint{4.121793in}{1.002261in}}%
\pgfpathcurveto{\pgfqpoint{4.126837in}{0.997217in}}{\pgfqpoint{4.133679in}{0.994383in}}{\pgfqpoint{4.140811in}{0.994383in}}%
\pgfpathclose%
\pgfusepath{stroke,fill}%
\end{pgfscope}%
\begin{pgfscope}%
\pgfpathrectangle{\pgfqpoint{2.867647in}{0.500000in}}{\pgfqpoint{1.764706in}{1.700000in}}%
\pgfusepath{clip}%
\pgfsetbuttcap%
\pgfsetroundjoin%
\definecolor{currentfill}{rgb}{0.981377,0.920617,0.865369}%
\pgfsetfillcolor{currentfill}%
\pgfsetlinewidth{0.311001pt}%
\definecolor{currentstroke}{rgb}{1.000000,1.000000,1.000000}%
\pgfsetstrokecolor{currentstroke}%
\pgfsetdash{}{0pt}%
\pgfpathmoveto{\pgfqpoint{4.170932in}{1.242152in}}%
\pgfpathcurveto{\pgfqpoint{4.178065in}{1.242152in}}{\pgfqpoint{4.184907in}{1.244986in}}{\pgfqpoint{4.189950in}{1.250030in}}%
\pgfpathcurveto{\pgfqpoint{4.194994in}{1.255074in}}{\pgfqpoint{4.197828in}{1.261915in}}{\pgfqpoint{4.197828in}{1.269048in}}%
\pgfpathcurveto{\pgfqpoint{4.197828in}{1.276181in}}{\pgfqpoint{4.194994in}{1.283022in}}{\pgfqpoint{4.189950in}{1.288066in}}%
\pgfpathcurveto{\pgfqpoint{4.184907in}{1.293110in}}{\pgfqpoint{4.178065in}{1.295944in}}{\pgfqpoint{4.170932in}{1.295944in}}%
\pgfpathcurveto{\pgfqpoint{4.163799in}{1.295944in}}{\pgfqpoint{4.156958in}{1.293110in}}{\pgfqpoint{4.151914in}{1.288066in}}%
\pgfpathcurveto{\pgfqpoint{4.146870in}{1.283022in}}{\pgfqpoint{4.144036in}{1.276181in}}{\pgfqpoint{4.144036in}{1.269048in}}%
\pgfpathcurveto{\pgfqpoint{4.144036in}{1.261915in}}{\pgfqpoint{4.146870in}{1.255074in}}{\pgfqpoint{4.151914in}{1.250030in}}%
\pgfpathcurveto{\pgfqpoint{4.156958in}{1.244986in}}{\pgfqpoint{4.163799in}{1.242152in}}{\pgfqpoint{4.170932in}{1.242152in}}%
\pgfpathclose%
\pgfusepath{stroke,fill}%
\end{pgfscope}%
\begin{pgfscope}%
\pgfpathrectangle{\pgfqpoint{2.867647in}{0.500000in}}{\pgfqpoint{1.764706in}{1.700000in}}%
\pgfusepath{clip}%
\pgfsetbuttcap%
\pgfsetroundjoin%
\definecolor{currentfill}{rgb}{0.970718,0.821518,0.719872}%
\pgfsetfillcolor{currentfill}%
\pgfsetlinewidth{0.311001pt}%
\definecolor{currentstroke}{rgb}{1.000000,1.000000,1.000000}%
\pgfsetstrokecolor{currentstroke}%
\pgfsetdash{}{0pt}%
\pgfpathmoveto{\pgfqpoint{4.270565in}{1.283195in}}%
\pgfpathcurveto{\pgfqpoint{4.277698in}{1.283195in}}{\pgfqpoint{4.284540in}{1.286029in}}{\pgfqpoint{4.289583in}{1.291072in}}%
\pgfpathcurveto{\pgfqpoint{4.294627in}{1.296116in}}{\pgfqpoint{4.297461in}{1.302958in}}{\pgfqpoint{4.297461in}{1.310090in}}%
\pgfpathcurveto{\pgfqpoint{4.297461in}{1.317223in}}{\pgfqpoint{4.294627in}{1.324065in}}{\pgfqpoint{4.289583in}{1.329109in}}%
\pgfpathcurveto{\pgfqpoint{4.284540in}{1.334152in}}{\pgfqpoint{4.277698in}{1.336986in}}{\pgfqpoint{4.270565in}{1.336986in}}%
\pgfpathcurveto{\pgfqpoint{4.263432in}{1.336986in}}{\pgfqpoint{4.256591in}{1.334152in}}{\pgfqpoint{4.251547in}{1.329109in}}%
\pgfpathcurveto{\pgfqpoint{4.246503in}{1.324065in}}{\pgfqpoint{4.243669in}{1.317223in}}{\pgfqpoint{4.243669in}{1.310090in}}%
\pgfpathcurveto{\pgfqpoint{4.243669in}{1.302958in}}{\pgfqpoint{4.246503in}{1.296116in}}{\pgfqpoint{4.251547in}{1.291072in}}%
\pgfpathcurveto{\pgfqpoint{4.256591in}{1.286029in}}{\pgfqpoint{4.263432in}{1.283195in}}{\pgfqpoint{4.270565in}{1.283195in}}%
\pgfpathclose%
\pgfusepath{stroke,fill}%
\end{pgfscope}%
\begin{pgfscope}%
\pgfpathrectangle{\pgfqpoint{2.867647in}{0.500000in}}{\pgfqpoint{1.764706in}{1.700000in}}%
\pgfusepath{clip}%
\pgfsetbuttcap%
\pgfsetroundjoin%
\definecolor{currentfill}{rgb}{0.970255,0.815666,0.711203}%
\pgfsetfillcolor{currentfill}%
\pgfsetlinewidth{0.311001pt}%
\definecolor{currentstroke}{rgb}{1.000000,1.000000,1.000000}%
\pgfsetstrokecolor{currentstroke}%
\pgfsetdash{}{0pt}%
\pgfpathmoveto{\pgfqpoint{4.157730in}{1.660675in}}%
\pgfpathcurveto{\pgfqpoint{4.164863in}{1.660675in}}{\pgfqpoint{4.171705in}{1.663509in}}{\pgfqpoint{4.176749in}{1.668552in}}%
\pgfpathcurveto{\pgfqpoint{4.181792in}{1.673596in}}{\pgfqpoint{4.184626in}{1.680438in}}{\pgfqpoint{4.184626in}{1.687570in}}%
\pgfpathcurveto{\pgfqpoint{4.184626in}{1.694703in}}{\pgfqpoint{4.181792in}{1.701545in}}{\pgfqpoint{4.176749in}{1.706589in}}%
\pgfpathcurveto{\pgfqpoint{4.171705in}{1.711632in}}{\pgfqpoint{4.164863in}{1.714466in}}{\pgfqpoint{4.157730in}{1.714466in}}%
\pgfpathcurveto{\pgfqpoint{4.150598in}{1.714466in}}{\pgfqpoint{4.143756in}{1.711632in}}{\pgfqpoint{4.138712in}{1.706589in}}%
\pgfpathcurveto{\pgfqpoint{4.133669in}{1.701545in}}{\pgfqpoint{4.130835in}{1.694703in}}{\pgfqpoint{4.130835in}{1.687570in}}%
\pgfpathcurveto{\pgfqpoint{4.130835in}{1.680438in}}{\pgfqpoint{4.133669in}{1.673596in}}{\pgfqpoint{4.138712in}{1.668552in}}%
\pgfpathcurveto{\pgfqpoint{4.143756in}{1.663509in}}{\pgfqpoint{4.150598in}{1.660675in}}{\pgfqpoint{4.157730in}{1.660675in}}%
\pgfpathclose%
\pgfusepath{stroke,fill}%
\end{pgfscope}%
\begin{pgfscope}%
\pgfpathrectangle{\pgfqpoint{2.867647in}{0.500000in}}{\pgfqpoint{1.764706in}{1.700000in}}%
\pgfusepath{clip}%
\pgfsetbuttcap%
\pgfsetroundjoin%
\definecolor{currentfill}{rgb}{0.975644,0.874038,0.797253}%
\pgfsetfillcolor{currentfill}%
\pgfsetlinewidth{0.311001pt}%
\definecolor{currentstroke}{rgb}{1.000000,1.000000,1.000000}%
\pgfsetstrokecolor{currentstroke}%
\pgfsetdash{}{0pt}%
\pgfpathmoveto{\pgfqpoint{4.093156in}{1.066586in}}%
\pgfpathcurveto{\pgfqpoint{4.100289in}{1.066586in}}{\pgfqpoint{4.107130in}{1.069420in}}{\pgfqpoint{4.112174in}{1.074464in}}%
\pgfpathcurveto{\pgfqpoint{4.117218in}{1.079508in}}{\pgfqpoint{4.120052in}{1.086349in}}{\pgfqpoint{4.120052in}{1.093482in}}%
\pgfpathcurveto{\pgfqpoint{4.120052in}{1.100615in}}{\pgfqpoint{4.117218in}{1.107457in}}{\pgfqpoint{4.112174in}{1.112500in}}%
\pgfpathcurveto{\pgfqpoint{4.107130in}{1.117544in}}{\pgfqpoint{4.100289in}{1.120378in}}{\pgfqpoint{4.093156in}{1.120378in}}%
\pgfpathcurveto{\pgfqpoint{4.086023in}{1.120378in}}{\pgfqpoint{4.079181in}{1.117544in}}{\pgfqpoint{4.074138in}{1.112500in}}%
\pgfpathcurveto{\pgfqpoint{4.069094in}{1.107457in}}{\pgfqpoint{4.066260in}{1.100615in}}{\pgfqpoint{4.066260in}{1.093482in}}%
\pgfpathcurveto{\pgfqpoint{4.066260in}{1.086349in}}{\pgfqpoint{4.069094in}{1.079508in}}{\pgfqpoint{4.074138in}{1.074464in}}%
\pgfpathcurveto{\pgfqpoint{4.079181in}{1.069420in}}{\pgfqpoint{4.086023in}{1.066586in}}{\pgfqpoint{4.093156in}{1.066586in}}%
\pgfpathclose%
\pgfusepath{stroke,fill}%
\end{pgfscope}%
\begin{pgfscope}%
\pgfpathrectangle{\pgfqpoint{2.867647in}{0.500000in}}{\pgfqpoint{1.764706in}{1.700000in}}%
\pgfusepath{clip}%
\pgfsetbuttcap%
\pgfsetroundjoin%
\definecolor{currentfill}{rgb}{0.919781,0.275262,0.242460}%
\pgfsetfillcolor{currentfill}%
\pgfsetlinewidth{0.311001pt}%
\definecolor{currentstroke}{rgb}{1.000000,1.000000,1.000000}%
\pgfsetstrokecolor{currentstroke}%
\pgfsetdash{}{0pt}%
\pgfpathmoveto{\pgfqpoint{3.861147in}{1.630404in}}%
\pgfpathcurveto{\pgfqpoint{3.868280in}{1.630404in}}{\pgfqpoint{3.875122in}{1.633238in}}{\pgfqpoint{3.880166in}{1.638281in}}%
\pgfpathcurveto{\pgfqpoint{3.885209in}{1.643325in}}{\pgfqpoint{3.888043in}{1.650167in}}{\pgfqpoint{3.888043in}{1.657299in}}%
\pgfpathcurveto{\pgfqpoint{3.888043in}{1.664432in}}{\pgfqpoint{3.885209in}{1.671274in}}{\pgfqpoint{3.880166in}{1.676318in}}%
\pgfpathcurveto{\pgfqpoint{3.875122in}{1.681361in}}{\pgfqpoint{3.868280in}{1.684195in}}{\pgfqpoint{3.861147in}{1.684195in}}%
\pgfpathcurveto{\pgfqpoint{3.854015in}{1.684195in}}{\pgfqpoint{3.847173in}{1.681361in}}{\pgfqpoint{3.842129in}{1.676318in}}%
\pgfpathcurveto{\pgfqpoint{3.837086in}{1.671274in}}{\pgfqpoint{3.834252in}{1.664432in}}{\pgfqpoint{3.834252in}{1.657299in}}%
\pgfpathcurveto{\pgfqpoint{3.834252in}{1.650167in}}{\pgfqpoint{3.837086in}{1.643325in}}{\pgfqpoint{3.842129in}{1.638281in}}%
\pgfpathcurveto{\pgfqpoint{3.847173in}{1.633238in}}{\pgfqpoint{3.854015in}{1.630404in}}{\pgfqpoint{3.861147in}{1.630404in}}%
\pgfpathclose%
\pgfusepath{stroke,fill}%
\end{pgfscope}%
\begin{pgfscope}%
\pgfpathrectangle{\pgfqpoint{2.867647in}{0.500000in}}{\pgfqpoint{1.764706in}{1.700000in}}%
\pgfusepath{clip}%
\pgfsetbuttcap%
\pgfsetroundjoin%
\definecolor{currentfill}{rgb}{0.979891,0.908948,0.848279}%
\pgfsetfillcolor{currentfill}%
\pgfsetlinewidth{0.311001pt}%
\definecolor{currentstroke}{rgb}{1.000000,1.000000,1.000000}%
\pgfsetstrokecolor{currentstroke}%
\pgfsetdash{}{0pt}%
\pgfpathmoveto{\pgfqpoint{4.217445in}{1.305316in}}%
\pgfpathcurveto{\pgfqpoint{4.224577in}{1.305316in}}{\pgfqpoint{4.231419in}{1.308150in}}{\pgfqpoint{4.236463in}{1.313194in}}%
\pgfpathcurveto{\pgfqpoint{4.241506in}{1.318237in}}{\pgfqpoint{4.244340in}{1.325079in}}{\pgfqpoint{4.244340in}{1.332212in}}%
\pgfpathcurveto{\pgfqpoint{4.244340in}{1.339344in}}{\pgfqpoint{4.241506in}{1.346186in}}{\pgfqpoint{4.236463in}{1.351230in}}%
\pgfpathcurveto{\pgfqpoint{4.231419in}{1.356273in}}{\pgfqpoint{4.224577in}{1.359107in}}{\pgfqpoint{4.217445in}{1.359107in}}%
\pgfpathcurveto{\pgfqpoint{4.210312in}{1.359107in}}{\pgfqpoint{4.203470in}{1.356273in}}{\pgfqpoint{4.198427in}{1.351230in}}%
\pgfpathcurveto{\pgfqpoint{4.193383in}{1.346186in}}{\pgfqpoint{4.190549in}{1.339344in}}{\pgfqpoint{4.190549in}{1.332212in}}%
\pgfpathcurveto{\pgfqpoint{4.190549in}{1.325079in}}{\pgfqpoint{4.193383in}{1.318237in}}{\pgfqpoint{4.198427in}{1.313194in}}%
\pgfpathcurveto{\pgfqpoint{4.203470in}{1.308150in}}{\pgfqpoint{4.210312in}{1.305316in}}{\pgfqpoint{4.217445in}{1.305316in}}%
\pgfpathclose%
\pgfusepath{stroke,fill}%
\end{pgfscope}%
\begin{pgfscope}%
\pgfpathrectangle{\pgfqpoint{2.867647in}{0.500000in}}{\pgfqpoint{1.764706in}{1.700000in}}%
\pgfusepath{clip}%
\pgfsetbuttcap%
\pgfsetroundjoin%
\definecolor{currentfill}{rgb}{0.979891,0.908948,0.848279}%
\pgfsetfillcolor{currentfill}%
\pgfsetlinewidth{0.311001pt}%
\definecolor{currentstroke}{rgb}{1.000000,1.000000,1.000000}%
\pgfsetstrokecolor{currentstroke}%
\pgfsetdash{}{0pt}%
\pgfpathmoveto{\pgfqpoint{4.163197in}{1.412404in}}%
\pgfpathcurveto{\pgfqpoint{4.170330in}{1.412404in}}{\pgfqpoint{4.177172in}{1.415238in}}{\pgfqpoint{4.182215in}{1.420282in}}%
\pgfpathcurveto{\pgfqpoint{4.187259in}{1.425325in}}{\pgfqpoint{4.190093in}{1.432167in}}{\pgfqpoint{4.190093in}{1.439300in}}%
\pgfpathcurveto{\pgfqpoint{4.190093in}{1.446433in}}{\pgfqpoint{4.187259in}{1.453274in}}{\pgfqpoint{4.182215in}{1.458318in}}%
\pgfpathcurveto{\pgfqpoint{4.177172in}{1.463362in}}{\pgfqpoint{4.170330in}{1.466196in}}{\pgfqpoint{4.163197in}{1.466196in}}%
\pgfpathcurveto{\pgfqpoint{4.156064in}{1.466196in}}{\pgfqpoint{4.149223in}{1.463362in}}{\pgfqpoint{4.144179in}{1.458318in}}%
\pgfpathcurveto{\pgfqpoint{4.139135in}{1.453274in}}{\pgfqpoint{4.136302in}{1.446433in}}{\pgfqpoint{4.136302in}{1.439300in}}%
\pgfpathcurveto{\pgfqpoint{4.136302in}{1.432167in}}{\pgfqpoint{4.139135in}{1.425325in}}{\pgfqpoint{4.144179in}{1.420282in}}%
\pgfpathcurveto{\pgfqpoint{4.149223in}{1.415238in}}{\pgfqpoint{4.156064in}{1.412404in}}{\pgfqpoint{4.163197in}{1.412404in}}%
\pgfpathclose%
\pgfusepath{stroke,fill}%
\end{pgfscope}%
\begin{pgfscope}%
\pgfpathrectangle{\pgfqpoint{2.867647in}{0.500000in}}{\pgfqpoint{1.764706in}{1.700000in}}%
\pgfusepath{clip}%
\pgfsetbuttcap%
\pgfsetroundjoin%
\definecolor{currentfill}{rgb}{0.962283,0.593046,0.431453}%
\pgfsetfillcolor{currentfill}%
\pgfsetlinewidth{0.311001pt}%
\definecolor{currentstroke}{rgb}{1.000000,1.000000,1.000000}%
\pgfsetstrokecolor{currentstroke}%
\pgfsetdash{}{0pt}%
\pgfpathmoveto{\pgfqpoint{3.957380in}{1.775049in}}%
\pgfpathcurveto{\pgfqpoint{3.964513in}{1.775049in}}{\pgfqpoint{3.971354in}{1.777883in}}{\pgfqpoint{3.976398in}{1.782926in}}%
\pgfpathcurveto{\pgfqpoint{3.981442in}{1.787970in}}{\pgfqpoint{3.984275in}{1.794812in}}{\pgfqpoint{3.984275in}{1.801944in}}%
\pgfpathcurveto{\pgfqpoint{3.984275in}{1.809077in}}{\pgfqpoint{3.981442in}{1.815919in}}{\pgfqpoint{3.976398in}{1.820963in}}%
\pgfpathcurveto{\pgfqpoint{3.971354in}{1.826006in}}{\pgfqpoint{3.964513in}{1.828840in}}{\pgfqpoint{3.957380in}{1.828840in}}%
\pgfpathcurveto{\pgfqpoint{3.950247in}{1.828840in}}{\pgfqpoint{3.943405in}{1.826006in}}{\pgfqpoint{3.938362in}{1.820963in}}%
\pgfpathcurveto{\pgfqpoint{3.933318in}{1.815919in}}{\pgfqpoint{3.930484in}{1.809077in}}{\pgfqpoint{3.930484in}{1.801944in}}%
\pgfpathcurveto{\pgfqpoint{3.930484in}{1.794812in}}{\pgfqpoint{3.933318in}{1.787970in}}{\pgfqpoint{3.938362in}{1.782926in}}%
\pgfpathcurveto{\pgfqpoint{3.943405in}{1.777883in}}{\pgfqpoint{3.950247in}{1.775049in}}{\pgfqpoint{3.957380in}{1.775049in}}%
\pgfpathclose%
\pgfusepath{stroke,fill}%
\end{pgfscope}%
\begin{pgfscope}%
\pgfpathrectangle{\pgfqpoint{2.867647in}{0.500000in}}{\pgfqpoint{1.764706in}{1.700000in}}%
\pgfusepath{clip}%
\pgfsetbuttcap%
\pgfsetroundjoin%
\definecolor{currentfill}{rgb}{0.973271,0.850724,0.762998}%
\pgfsetfillcolor{currentfill}%
\pgfsetlinewidth{0.311001pt}%
\definecolor{currentstroke}{rgb}{1.000000,1.000000,1.000000}%
\pgfsetstrokecolor{currentstroke}%
\pgfsetdash{}{0pt}%
\pgfpathmoveto{\pgfqpoint{4.120681in}{1.387500in}}%
\pgfpathcurveto{\pgfqpoint{4.127814in}{1.387500in}}{\pgfqpoint{4.134656in}{1.390334in}}{\pgfqpoint{4.139700in}{1.395378in}}%
\pgfpathcurveto{\pgfqpoint{4.144743in}{1.400421in}}{\pgfqpoint{4.147577in}{1.407263in}}{\pgfqpoint{4.147577in}{1.414396in}}%
\pgfpathcurveto{\pgfqpoint{4.147577in}{1.421529in}}{\pgfqpoint{4.144743in}{1.428370in}}{\pgfqpoint{4.139700in}{1.433414in}}%
\pgfpathcurveto{\pgfqpoint{4.134656in}{1.438458in}}{\pgfqpoint{4.127814in}{1.441292in}}{\pgfqpoint{4.120681in}{1.441292in}}%
\pgfpathcurveto{\pgfqpoint{4.113549in}{1.441292in}}{\pgfqpoint{4.106707in}{1.438458in}}{\pgfqpoint{4.101663in}{1.433414in}}%
\pgfpathcurveto{\pgfqpoint{4.096620in}{1.428370in}}{\pgfqpoint{4.093786in}{1.421529in}}{\pgfqpoint{4.093786in}{1.414396in}}%
\pgfpathcurveto{\pgfqpoint{4.093786in}{1.407263in}}{\pgfqpoint{4.096620in}{1.400421in}}{\pgfqpoint{4.101663in}{1.395378in}}%
\pgfpathcurveto{\pgfqpoint{4.106707in}{1.390334in}}{\pgfqpoint{4.113549in}{1.387500in}}{\pgfqpoint{4.120681in}{1.387500in}}%
\pgfpathclose%
\pgfusepath{stroke,fill}%
\end{pgfscope}%
\begin{pgfscope}%
\pgfpathrectangle{\pgfqpoint{2.867647in}{0.500000in}}{\pgfqpoint{1.764706in}{1.700000in}}%
\pgfusepath{clip}%
\pgfsetbuttcap%
\pgfsetroundjoin%
\definecolor{currentfill}{rgb}{0.978376,0.897317,0.831308}%
\pgfsetfillcolor{currentfill}%
\pgfsetlinewidth{0.311001pt}%
\definecolor{currentstroke}{rgb}{1.000000,1.000000,1.000000}%
\pgfsetstrokecolor{currentstroke}%
\pgfsetdash{}{0pt}%
\pgfpathmoveto{\pgfqpoint{4.113277in}{1.557138in}}%
\pgfpathcurveto{\pgfqpoint{4.120410in}{1.557138in}}{\pgfqpoint{4.127252in}{1.559971in}}{\pgfqpoint{4.132296in}{1.565015in}}%
\pgfpathcurveto{\pgfqpoint{4.137339in}{1.570059in}}{\pgfqpoint{4.140173in}{1.576900in}}{\pgfqpoint{4.140173in}{1.584033in}}%
\pgfpathcurveto{\pgfqpoint{4.140173in}{1.591166in}}{\pgfqpoint{4.137339in}{1.598008in}}{\pgfqpoint{4.132296in}{1.603051in}}%
\pgfpathcurveto{\pgfqpoint{4.127252in}{1.608095in}}{\pgfqpoint{4.120410in}{1.610929in}}{\pgfqpoint{4.113277in}{1.610929in}}%
\pgfpathcurveto{\pgfqpoint{4.106145in}{1.610929in}}{\pgfqpoint{4.099303in}{1.608095in}}{\pgfqpoint{4.094259in}{1.603051in}}%
\pgfpathcurveto{\pgfqpoint{4.089216in}{1.598008in}}{\pgfqpoint{4.086382in}{1.591166in}}{\pgfqpoint{4.086382in}{1.584033in}}%
\pgfpathcurveto{\pgfqpoint{4.086382in}{1.576900in}}{\pgfqpoint{4.089216in}{1.570059in}}{\pgfqpoint{4.094259in}{1.565015in}}%
\pgfpathcurveto{\pgfqpoint{4.099303in}{1.559971in}}{\pgfqpoint{4.106145in}{1.557138in}}{\pgfqpoint{4.113277in}{1.557138in}}%
\pgfpathclose%
\pgfusepath{stroke,fill}%
\end{pgfscope}%
\begin{pgfscope}%
\pgfpathrectangle{\pgfqpoint{2.867647in}{0.500000in}}{\pgfqpoint{1.764706in}{1.700000in}}%
\pgfusepath{clip}%
\pgfsetbuttcap%
\pgfsetroundjoin%
\definecolor{currentfill}{rgb}{0.980678,0.914765,0.856766}%
\pgfsetfillcolor{currentfill}%
\pgfsetlinewidth{0.311001pt}%
\definecolor{currentstroke}{rgb}{1.000000,1.000000,1.000000}%
\pgfsetstrokecolor{currentstroke}%
\pgfsetdash{}{0pt}%
\pgfpathmoveto{\pgfqpoint{4.151933in}{1.492077in}}%
\pgfpathcurveto{\pgfqpoint{4.159065in}{1.492077in}}{\pgfqpoint{4.165907in}{1.494911in}}{\pgfqpoint{4.170951in}{1.499954in}}%
\pgfpathcurveto{\pgfqpoint{4.175994in}{1.504998in}}{\pgfqpoint{4.178828in}{1.511840in}}{\pgfqpoint{4.178828in}{1.518972in}}%
\pgfpathcurveto{\pgfqpoint{4.178828in}{1.526105in}}{\pgfqpoint{4.175994in}{1.532947in}}{\pgfqpoint{4.170951in}{1.537991in}}%
\pgfpathcurveto{\pgfqpoint{4.165907in}{1.543034in}}{\pgfqpoint{4.159065in}{1.545868in}}{\pgfqpoint{4.151933in}{1.545868in}}%
\pgfpathcurveto{\pgfqpoint{4.144800in}{1.545868in}}{\pgfqpoint{4.137958in}{1.543034in}}{\pgfqpoint{4.132915in}{1.537991in}}%
\pgfpathcurveto{\pgfqpoint{4.127871in}{1.532947in}}{\pgfqpoint{4.125037in}{1.526105in}}{\pgfqpoint{4.125037in}{1.518972in}}%
\pgfpathcurveto{\pgfqpoint{4.125037in}{1.511840in}}{\pgfqpoint{4.127871in}{1.504998in}}{\pgfqpoint{4.132915in}{1.499954in}}%
\pgfpathcurveto{\pgfqpoint{4.137958in}{1.494911in}}{\pgfqpoint{4.144800in}{1.492077in}}{\pgfqpoint{4.151933in}{1.492077in}}%
\pgfpathclose%
\pgfusepath{stroke,fill}%
\end{pgfscope}%
\begin{pgfscope}%
\pgfpathrectangle{\pgfqpoint{2.867647in}{0.500000in}}{\pgfqpoint{1.764706in}{1.700000in}}%
\pgfusepath{clip}%
\pgfsetbuttcap%
\pgfsetroundjoin%
\definecolor{currentfill}{rgb}{0.976287,0.879862,0.805788}%
\pgfsetfillcolor{currentfill}%
\pgfsetlinewidth{0.311001pt}%
\definecolor{currentstroke}{rgb}{1.000000,1.000000,1.000000}%
\pgfsetstrokecolor{currentstroke}%
\pgfsetdash{}{0pt}%
\pgfpathmoveto{\pgfqpoint{4.142897in}{1.618249in}}%
\pgfpathcurveto{\pgfqpoint{4.150029in}{1.618249in}}{\pgfqpoint{4.156871in}{1.621083in}}{\pgfqpoint{4.161915in}{1.626127in}}%
\pgfpathcurveto{\pgfqpoint{4.166958in}{1.631171in}}{\pgfqpoint{4.169792in}{1.638012in}}{\pgfqpoint{4.169792in}{1.645145in}}%
\pgfpathcurveto{\pgfqpoint{4.169792in}{1.652278in}}{\pgfqpoint{4.166958in}{1.659120in}}{\pgfqpoint{4.161915in}{1.664163in}}%
\pgfpathcurveto{\pgfqpoint{4.156871in}{1.669207in}}{\pgfqpoint{4.150029in}{1.672041in}}{\pgfqpoint{4.142897in}{1.672041in}}%
\pgfpathcurveto{\pgfqpoint{4.135764in}{1.672041in}}{\pgfqpoint{4.128922in}{1.669207in}}{\pgfqpoint{4.123878in}{1.664163in}}%
\pgfpathcurveto{\pgfqpoint{4.118835in}{1.659120in}}{\pgfqpoint{4.116001in}{1.652278in}}{\pgfqpoint{4.116001in}{1.645145in}}%
\pgfpathcurveto{\pgfqpoint{4.116001in}{1.638012in}}{\pgfqpoint{4.118835in}{1.631171in}}{\pgfqpoint{4.123878in}{1.626127in}}%
\pgfpathcurveto{\pgfqpoint{4.128922in}{1.621083in}}{\pgfqpoint{4.135764in}{1.618249in}}{\pgfqpoint{4.142897in}{1.618249in}}%
\pgfpathclose%
\pgfusepath{stroke,fill}%
\end{pgfscope}%
\begin{pgfscope}%
\pgfpathrectangle{\pgfqpoint{2.867647in}{0.500000in}}{\pgfqpoint{1.764706in}{1.700000in}}%
\pgfusepath{clip}%
\pgfsetbuttcap%
\pgfsetroundjoin%
\definecolor{currentfill}{rgb}{0.976961,0.885681,0.814303}%
\pgfsetfillcolor{currentfill}%
\pgfsetlinewidth{0.311001pt}%
\definecolor{currentstroke}{rgb}{1.000000,1.000000,1.000000}%
\pgfsetstrokecolor{currentstroke}%
\pgfsetdash{}{0pt}%
\pgfpathmoveto{\pgfqpoint{4.228125in}{1.205517in}}%
\pgfpathcurveto{\pgfqpoint{4.235257in}{1.205517in}}{\pgfqpoint{4.242099in}{1.208351in}}{\pgfqpoint{4.247143in}{1.213395in}}%
\pgfpathcurveto{\pgfqpoint{4.252186in}{1.218439in}}{\pgfqpoint{4.255020in}{1.225280in}}{\pgfqpoint{4.255020in}{1.232413in}}%
\pgfpathcurveto{\pgfqpoint{4.255020in}{1.239546in}}{\pgfqpoint{4.252186in}{1.246388in}}{\pgfqpoint{4.247143in}{1.251431in}}%
\pgfpathcurveto{\pgfqpoint{4.242099in}{1.256475in}}{\pgfqpoint{4.235257in}{1.259309in}}{\pgfqpoint{4.228125in}{1.259309in}}%
\pgfpathcurveto{\pgfqpoint{4.220992in}{1.259309in}}{\pgfqpoint{4.214150in}{1.256475in}}{\pgfqpoint{4.209106in}{1.251431in}}%
\pgfpathcurveto{\pgfqpoint{4.204063in}{1.246388in}}{\pgfqpoint{4.201229in}{1.239546in}}{\pgfqpoint{4.201229in}{1.232413in}}%
\pgfpathcurveto{\pgfqpoint{4.201229in}{1.225280in}}{\pgfqpoint{4.204063in}{1.218439in}}{\pgfqpoint{4.209106in}{1.213395in}}%
\pgfpathcurveto{\pgfqpoint{4.214150in}{1.208351in}}{\pgfqpoint{4.220992in}{1.205517in}}{\pgfqpoint{4.228125in}{1.205517in}}%
\pgfpathclose%
\pgfusepath{stroke,fill}%
\end{pgfscope}%
\begin{pgfscope}%
\pgfpathrectangle{\pgfqpoint{2.867647in}{0.500000in}}{\pgfqpoint{1.764706in}{1.700000in}}%
\pgfusepath{clip}%
\pgfsetbuttcap%
\pgfsetroundjoin%
\definecolor{currentfill}{rgb}{0.972726,0.844889,0.754401}%
\pgfsetfillcolor{currentfill}%
\pgfsetlinewidth{0.311001pt}%
\definecolor{currentstroke}{rgb}{1.000000,1.000000,1.000000}%
\pgfsetstrokecolor{currentstroke}%
\pgfsetdash{}{0pt}%
\pgfpathmoveto{\pgfqpoint{4.181314in}{1.030151in}}%
\pgfpathcurveto{\pgfqpoint{4.188447in}{1.030151in}}{\pgfqpoint{4.195289in}{1.032985in}}{\pgfqpoint{4.200332in}{1.038029in}}%
\pgfpathcurveto{\pgfqpoint{4.205376in}{1.043072in}}{\pgfqpoint{4.208210in}{1.049914in}}{\pgfqpoint{4.208210in}{1.057047in}}%
\pgfpathcurveto{\pgfqpoint{4.208210in}{1.064180in}}{\pgfqpoint{4.205376in}{1.071021in}}{\pgfqpoint{4.200332in}{1.076065in}}%
\pgfpathcurveto{\pgfqpoint{4.195289in}{1.081109in}}{\pgfqpoint{4.188447in}{1.083943in}}{\pgfqpoint{4.181314in}{1.083943in}}%
\pgfpathcurveto{\pgfqpoint{4.174181in}{1.083943in}}{\pgfqpoint{4.167340in}{1.081109in}}{\pgfqpoint{4.162296in}{1.076065in}}%
\pgfpathcurveto{\pgfqpoint{4.157252in}{1.071021in}}{\pgfqpoint{4.154418in}{1.064180in}}{\pgfqpoint{4.154418in}{1.057047in}}%
\pgfpathcurveto{\pgfqpoint{4.154418in}{1.049914in}}{\pgfqpoint{4.157252in}{1.043072in}}{\pgfqpoint{4.162296in}{1.038029in}}%
\pgfpathcurveto{\pgfqpoint{4.167340in}{1.032985in}}{\pgfqpoint{4.174181in}{1.030151in}}{\pgfqpoint{4.181314in}{1.030151in}}%
\pgfpathclose%
\pgfusepath{stroke,fill}%
\end{pgfscope}%
\begin{pgfscope}%
\pgfpathrectangle{\pgfqpoint{2.867647in}{0.500000in}}{\pgfqpoint{1.764706in}{1.700000in}}%
\pgfusepath{clip}%
\pgfsetbuttcap%
\pgfsetroundjoin%
\definecolor{currentfill}{rgb}{0.969803,0.809811,0.702523}%
\pgfsetfillcolor{currentfill}%
\pgfsetlinewidth{0.311001pt}%
\definecolor{currentstroke}{rgb}{1.000000,1.000000,1.000000}%
\pgfsetstrokecolor{currentstroke}%
\pgfsetdash{}{0pt}%
\pgfpathmoveto{\pgfqpoint{4.098957in}{1.701238in}}%
\pgfpathcurveto{\pgfqpoint{4.106089in}{1.701238in}}{\pgfqpoint{4.112931in}{1.704071in}}{\pgfqpoint{4.117975in}{1.709115in}}%
\pgfpathcurveto{\pgfqpoint{4.123018in}{1.714159in}}{\pgfqpoint{4.125852in}{1.721000in}}{\pgfqpoint{4.125852in}{1.728133in}}%
\pgfpathcurveto{\pgfqpoint{4.125852in}{1.735266in}}{\pgfqpoint{4.123018in}{1.742108in}}{\pgfqpoint{4.117975in}{1.747151in}}%
\pgfpathcurveto{\pgfqpoint{4.112931in}{1.752195in}}{\pgfqpoint{4.106089in}{1.755029in}}{\pgfqpoint{4.098957in}{1.755029in}}%
\pgfpathcurveto{\pgfqpoint{4.091824in}{1.755029in}}{\pgfqpoint{4.084982in}{1.752195in}}{\pgfqpoint{4.079938in}{1.747151in}}%
\pgfpathcurveto{\pgfqpoint{4.074895in}{1.742108in}}{\pgfqpoint{4.072061in}{1.735266in}}{\pgfqpoint{4.072061in}{1.728133in}}%
\pgfpathcurveto{\pgfqpoint{4.072061in}{1.721000in}}{\pgfqpoint{4.074895in}{1.714159in}}{\pgfqpoint{4.079938in}{1.709115in}}%
\pgfpathcurveto{\pgfqpoint{4.084982in}{1.704071in}}{\pgfqpoint{4.091824in}{1.701238in}}{\pgfqpoint{4.098957in}{1.701238in}}%
\pgfpathclose%
\pgfusepath{stroke,fill}%
\end{pgfscope}%
\begin{pgfscope}%
\pgfpathrectangle{\pgfqpoint{2.867647in}{0.500000in}}{\pgfqpoint{1.764706in}{1.700000in}}%
\pgfusepath{clip}%
\pgfsetbuttcap%
\pgfsetroundjoin%
\definecolor{currentfill}{rgb}{0.967735,0.780441,0.659127}%
\pgfsetfillcolor{currentfill}%
\pgfsetlinewidth{0.311001pt}%
\definecolor{currentstroke}{rgb}{1.000000,1.000000,1.000000}%
\pgfsetstrokecolor{currentstroke}%
\pgfsetdash{}{0pt}%
\pgfpathmoveto{\pgfqpoint{4.198468in}{1.634926in}}%
\pgfpathcurveto{\pgfqpoint{4.205601in}{1.634926in}}{\pgfqpoint{4.212442in}{1.637760in}}{\pgfqpoint{4.217486in}{1.642804in}}%
\pgfpathcurveto{\pgfqpoint{4.222530in}{1.647847in}}{\pgfqpoint{4.225364in}{1.654689in}}{\pgfqpoint{4.225364in}{1.661822in}}%
\pgfpathcurveto{\pgfqpoint{4.225364in}{1.668955in}}{\pgfqpoint{4.222530in}{1.675796in}}{\pgfqpoint{4.217486in}{1.680840in}}%
\pgfpathcurveto{\pgfqpoint{4.212442in}{1.685884in}}{\pgfqpoint{4.205601in}{1.688717in}}{\pgfqpoint{4.198468in}{1.688717in}}%
\pgfpathcurveto{\pgfqpoint{4.191335in}{1.688717in}}{\pgfqpoint{4.184494in}{1.685884in}}{\pgfqpoint{4.179450in}{1.680840in}}%
\pgfpathcurveto{\pgfqpoint{4.174406in}{1.675796in}}{\pgfqpoint{4.171572in}{1.668955in}}{\pgfqpoint{4.171572in}{1.661822in}}%
\pgfpathcurveto{\pgfqpoint{4.171572in}{1.654689in}}{\pgfqpoint{4.174406in}{1.647847in}}{\pgfqpoint{4.179450in}{1.642804in}}%
\pgfpathcurveto{\pgfqpoint{4.184494in}{1.637760in}}{\pgfqpoint{4.191335in}{1.634926in}}{\pgfqpoint{4.198468in}{1.634926in}}%
\pgfpathclose%
\pgfusepath{stroke,fill}%
\end{pgfscope}%
\begin{pgfscope}%
\pgfpathrectangle{\pgfqpoint{2.867647in}{0.500000in}}{\pgfqpoint{1.764706in}{1.700000in}}%
\pgfusepath{clip}%
\pgfsetbuttcap%
\pgfsetroundjoin%
\definecolor{currentfill}{rgb}{0.966560,0.756582,0.625273}%
\pgfsetfillcolor{currentfill}%
\pgfsetlinewidth{0.311001pt}%
\definecolor{currentstroke}{rgb}{1.000000,1.000000,1.000000}%
\pgfsetstrokecolor{currentstroke}%
\pgfsetdash{}{0pt}%
\pgfpathmoveto{\pgfqpoint{4.237587in}{1.055619in}}%
\pgfpathcurveto{\pgfqpoint{4.244720in}{1.055619in}}{\pgfqpoint{4.251561in}{1.058452in}}{\pgfqpoint{4.256605in}{1.063496in}}%
\pgfpathcurveto{\pgfqpoint{4.261649in}{1.068540in}}{\pgfqpoint{4.264483in}{1.075381in}}{\pgfqpoint{4.264483in}{1.082514in}}%
\pgfpathcurveto{\pgfqpoint{4.264483in}{1.089647in}}{\pgfqpoint{4.261649in}{1.096489in}}{\pgfqpoint{4.256605in}{1.101532in}}%
\pgfpathcurveto{\pgfqpoint{4.251561in}{1.106576in}}{\pgfqpoint{4.244720in}{1.109410in}}{\pgfqpoint{4.237587in}{1.109410in}}%
\pgfpathcurveto{\pgfqpoint{4.230454in}{1.109410in}}{\pgfqpoint{4.223612in}{1.106576in}}{\pgfqpoint{4.218569in}{1.101532in}}%
\pgfpathcurveto{\pgfqpoint{4.213525in}{1.096489in}}{\pgfqpoint{4.210691in}{1.089647in}}{\pgfqpoint{4.210691in}{1.082514in}}%
\pgfpathcurveto{\pgfqpoint{4.210691in}{1.075381in}}{\pgfqpoint{4.213525in}{1.068540in}}{\pgfqpoint{4.218569in}{1.063496in}}%
\pgfpathcurveto{\pgfqpoint{4.223612in}{1.058452in}}{\pgfqpoint{4.230454in}{1.055619in}}{\pgfqpoint{4.237587in}{1.055619in}}%
\pgfpathclose%
\pgfusepath{stroke,fill}%
\end{pgfscope}%
\begin{pgfscope}%
\pgfpathrectangle{\pgfqpoint{2.867647in}{0.500000in}}{\pgfqpoint{1.764706in}{1.700000in}}%
\pgfusepath{clip}%
\pgfsetbuttcap%
\pgfsetroundjoin%
\definecolor{currentfill}{rgb}{0.972201,0.839051,0.745789}%
\pgfsetfillcolor{currentfill}%
\pgfsetlinewidth{0.311001pt}%
\definecolor{currentstroke}{rgb}{1.000000,1.000000,1.000000}%
\pgfsetstrokecolor{currentstroke}%
\pgfsetdash{}{0pt}%
\pgfpathmoveto{\pgfqpoint{4.060710in}{1.585260in}}%
\pgfpathcurveto{\pgfqpoint{4.067843in}{1.585260in}}{\pgfqpoint{4.074684in}{1.588093in}}{\pgfqpoint{4.079728in}{1.593137in}}%
\pgfpathcurveto{\pgfqpoint{4.084772in}{1.598181in}}{\pgfqpoint{4.087606in}{1.605022in}}{\pgfqpoint{4.087606in}{1.612155in}}%
\pgfpathcurveto{\pgfqpoint{4.087606in}{1.619288in}}{\pgfqpoint{4.084772in}{1.626130in}}{\pgfqpoint{4.079728in}{1.631173in}}%
\pgfpathcurveto{\pgfqpoint{4.074684in}{1.636217in}}{\pgfqpoint{4.067843in}{1.639051in}}{\pgfqpoint{4.060710in}{1.639051in}}%
\pgfpathcurveto{\pgfqpoint{4.053577in}{1.639051in}}{\pgfqpoint{4.046735in}{1.636217in}}{\pgfqpoint{4.041692in}{1.631173in}}%
\pgfpathcurveto{\pgfqpoint{4.036648in}{1.626130in}}{\pgfqpoint{4.033814in}{1.619288in}}{\pgfqpoint{4.033814in}{1.612155in}}%
\pgfpathcurveto{\pgfqpoint{4.033814in}{1.605022in}}{\pgfqpoint{4.036648in}{1.598181in}}{\pgfqpoint{4.041692in}{1.593137in}}%
\pgfpathcurveto{\pgfqpoint{4.046735in}{1.588093in}}{\pgfqpoint{4.053577in}{1.585260in}}{\pgfqpoint{4.060710in}{1.585260in}}%
\pgfpathclose%
\pgfusepath{stroke,fill}%
\end{pgfscope}%
\begin{pgfscope}%
\pgfpathrectangle{\pgfqpoint{2.867647in}{0.500000in}}{\pgfqpoint{1.764706in}{1.700000in}}%
\pgfusepath{clip}%
\pgfsetbuttcap%
\pgfsetroundjoin%
\definecolor{currentfill}{rgb}{0.956268,0.491874,0.339856}%
\pgfsetfillcolor{currentfill}%
\pgfsetlinewidth{0.311001pt}%
\definecolor{currentstroke}{rgb}{1.000000,1.000000,1.000000}%
\pgfsetstrokecolor{currentstroke}%
\pgfsetdash{}{0pt}%
\pgfpathmoveto{\pgfqpoint{3.985804in}{1.509976in}}%
\pgfpathcurveto{\pgfqpoint{3.992936in}{1.509976in}}{\pgfqpoint{3.999778in}{1.512810in}}{\pgfqpoint{4.004822in}{1.517854in}}%
\pgfpathcurveto{\pgfqpoint{4.009865in}{1.522897in}}{\pgfqpoint{4.012699in}{1.529739in}}{\pgfqpoint{4.012699in}{1.536872in}}%
\pgfpathcurveto{\pgfqpoint{4.012699in}{1.544005in}}{\pgfqpoint{4.009865in}{1.550846in}}{\pgfqpoint{4.004822in}{1.555890in}}%
\pgfpathcurveto{\pgfqpoint{3.999778in}{1.560933in}}{\pgfqpoint{3.992936in}{1.563767in}}{\pgfqpoint{3.985804in}{1.563767in}}%
\pgfpathcurveto{\pgfqpoint{3.978671in}{1.563767in}}{\pgfqpoint{3.971829in}{1.560933in}}{\pgfqpoint{3.966785in}{1.555890in}}%
\pgfpathcurveto{\pgfqpoint{3.961742in}{1.550846in}}{\pgfqpoint{3.958908in}{1.544005in}}{\pgfqpoint{3.958908in}{1.536872in}}%
\pgfpathcurveto{\pgfqpoint{3.958908in}{1.529739in}}{\pgfqpoint{3.961742in}{1.522897in}}{\pgfqpoint{3.966785in}{1.517854in}}%
\pgfpathcurveto{\pgfqpoint{3.971829in}{1.512810in}}{\pgfqpoint{3.978671in}{1.509976in}}{\pgfqpoint{3.985804in}{1.509976in}}%
\pgfpathclose%
\pgfusepath{stroke,fill}%
\end{pgfscope}%
\begin{pgfscope}%
\pgfpathrectangle{\pgfqpoint{2.867647in}{0.500000in}}{\pgfqpoint{1.764706in}{1.700000in}}%
\pgfusepath{clip}%
\pgfsetbuttcap%
\pgfsetroundjoin%
\definecolor{currentfill}{rgb}{0.961115,0.566634,0.405693}%
\pgfsetfillcolor{currentfill}%
\pgfsetlinewidth{0.311001pt}%
\definecolor{currentstroke}{rgb}{1.000000,1.000000,1.000000}%
\pgfsetstrokecolor{currentstroke}%
\pgfsetdash{}{0pt}%
\pgfpathmoveto{\pgfqpoint{3.922497in}{1.683310in}}%
\pgfpathcurveto{\pgfqpoint{3.929630in}{1.683310in}}{\pgfqpoint{3.936472in}{1.686144in}}{\pgfqpoint{3.941515in}{1.691188in}}%
\pgfpathcurveto{\pgfqpoint{3.946559in}{1.696231in}}{\pgfqpoint{3.949393in}{1.703073in}}{\pgfqpoint{3.949393in}{1.710206in}}%
\pgfpathcurveto{\pgfqpoint{3.949393in}{1.717339in}}{\pgfqpoint{3.946559in}{1.724180in}}{\pgfqpoint{3.941515in}{1.729224in}}%
\pgfpathcurveto{\pgfqpoint{3.936472in}{1.734268in}}{\pgfqpoint{3.929630in}{1.737101in}}{\pgfqpoint{3.922497in}{1.737101in}}%
\pgfpathcurveto{\pgfqpoint{3.915364in}{1.737101in}}{\pgfqpoint{3.908523in}{1.734268in}}{\pgfqpoint{3.903479in}{1.729224in}}%
\pgfpathcurveto{\pgfqpoint{3.898435in}{1.724180in}}{\pgfqpoint{3.895601in}{1.717339in}}{\pgfqpoint{3.895601in}{1.710206in}}%
\pgfpathcurveto{\pgfqpoint{3.895601in}{1.703073in}}{\pgfqpoint{3.898435in}{1.696231in}}{\pgfqpoint{3.903479in}{1.691188in}}%
\pgfpathcurveto{\pgfqpoint{3.908523in}{1.686144in}}{\pgfqpoint{3.915364in}{1.683310in}}{\pgfqpoint{3.922497in}{1.683310in}}%
\pgfpathclose%
\pgfusepath{stroke,fill}%
\end{pgfscope}%
\begin{pgfscope}%
\pgfpathrectangle{\pgfqpoint{2.867647in}{0.500000in}}{\pgfqpoint{1.764706in}{1.700000in}}%
\pgfusepath{clip}%
\pgfsetbuttcap%
\pgfsetroundjoin%
\definecolor{currentfill}{rgb}{0.980678,0.914765,0.856766}%
\pgfsetfillcolor{currentfill}%
\pgfsetlinewidth{0.311001pt}%
\definecolor{currentstroke}{rgb}{1.000000,1.000000,1.000000}%
\pgfsetstrokecolor{currentstroke}%
\pgfsetdash{}{0pt}%
\pgfpathmoveto{\pgfqpoint{4.153676in}{1.157848in}}%
\pgfpathcurveto{\pgfqpoint{4.160809in}{1.157848in}}{\pgfqpoint{4.167651in}{1.160682in}}{\pgfqpoint{4.172694in}{1.165725in}}%
\pgfpathcurveto{\pgfqpoint{4.177738in}{1.170769in}}{\pgfqpoint{4.180572in}{1.177611in}}{\pgfqpoint{4.180572in}{1.184743in}}%
\pgfpathcurveto{\pgfqpoint{4.180572in}{1.191876in}}{\pgfqpoint{4.177738in}{1.198718in}}{\pgfqpoint{4.172694in}{1.203762in}}%
\pgfpathcurveto{\pgfqpoint{4.167651in}{1.208805in}}{\pgfqpoint{4.160809in}{1.211639in}}{\pgfqpoint{4.153676in}{1.211639in}}%
\pgfpathcurveto{\pgfqpoint{4.146543in}{1.211639in}}{\pgfqpoint{4.139702in}{1.208805in}}{\pgfqpoint{4.134658in}{1.203762in}}%
\pgfpathcurveto{\pgfqpoint{4.129614in}{1.198718in}}{\pgfqpoint{4.126780in}{1.191876in}}{\pgfqpoint{4.126780in}{1.184743in}}%
\pgfpathcurveto{\pgfqpoint{4.126780in}{1.177611in}}{\pgfqpoint{4.129614in}{1.170769in}}{\pgfqpoint{4.134658in}{1.165725in}}%
\pgfpathcurveto{\pgfqpoint{4.139702in}{1.160682in}}{\pgfqpoint{4.146543in}{1.157848in}}{\pgfqpoint{4.153676in}{1.157848in}}%
\pgfpathclose%
\pgfusepath{stroke,fill}%
\end{pgfscope}%
\begin{pgfscope}%
\pgfpathrectangle{\pgfqpoint{2.867647in}{0.500000in}}{\pgfqpoint{1.764706in}{1.700000in}}%
\pgfusepath{clip}%
\pgfsetbuttcap%
\pgfsetroundjoin%
\definecolor{currentfill}{rgb}{0.968105,0.786346,0.667739}%
\pgfsetfillcolor{currentfill}%
\pgfsetlinewidth{0.311001pt}%
\definecolor{currentstroke}{rgb}{1.000000,1.000000,1.000000}%
\pgfsetstrokecolor{currentstroke}%
\pgfsetdash{}{0pt}%
\pgfpathmoveto{\pgfqpoint{4.107957in}{0.935577in}}%
\pgfpathcurveto{\pgfqpoint{4.115090in}{0.935577in}}{\pgfqpoint{4.121932in}{0.938411in}}{\pgfqpoint{4.126975in}{0.943455in}}%
\pgfpathcurveto{\pgfqpoint{4.132019in}{0.948498in}}{\pgfqpoint{4.134853in}{0.955340in}}{\pgfqpoint{4.134853in}{0.962473in}}%
\pgfpathcurveto{\pgfqpoint{4.134853in}{0.969606in}}{\pgfqpoint{4.132019in}{0.976447in}}{\pgfqpoint{4.126975in}{0.981491in}}%
\pgfpathcurveto{\pgfqpoint{4.121932in}{0.986535in}}{\pgfqpoint{4.115090in}{0.989368in}}{\pgfqpoint{4.107957in}{0.989368in}}%
\pgfpathcurveto{\pgfqpoint{4.100824in}{0.989368in}}{\pgfqpoint{4.093983in}{0.986535in}}{\pgfqpoint{4.088939in}{0.981491in}}%
\pgfpathcurveto{\pgfqpoint{4.083895in}{0.976447in}}{\pgfqpoint{4.081062in}{0.969606in}}{\pgfqpoint{4.081062in}{0.962473in}}%
\pgfpathcurveto{\pgfqpoint{4.081062in}{0.955340in}}{\pgfqpoint{4.083895in}{0.948498in}}{\pgfqpoint{4.088939in}{0.943455in}}%
\pgfpathcurveto{\pgfqpoint{4.093983in}{0.938411in}}{\pgfqpoint{4.100824in}{0.935577in}}{\pgfqpoint{4.107957in}{0.935577in}}%
\pgfpathclose%
\pgfusepath{stroke,fill}%
\end{pgfscope}%
\begin{pgfscope}%
\pgfpathrectangle{\pgfqpoint{2.867647in}{0.500000in}}{\pgfqpoint{1.764706in}{1.700000in}}%
\pgfusepath{clip}%
\pgfsetbuttcap%
\pgfsetroundjoin%
\definecolor{currentfill}{rgb}{0.955103,0.477872,0.328626}%
\pgfsetfillcolor{currentfill}%
\pgfsetlinewidth{0.311001pt}%
\definecolor{currentstroke}{rgb}{1.000000,1.000000,1.000000}%
\pgfsetstrokecolor{currentstroke}%
\pgfsetdash{}{0pt}%
\pgfpathmoveto{\pgfqpoint{4.355439in}{1.287859in}}%
\pgfpathcurveto{\pgfqpoint{4.362572in}{1.287859in}}{\pgfqpoint{4.369414in}{1.290693in}}{\pgfqpoint{4.374457in}{1.295736in}}%
\pgfpathcurveto{\pgfqpoint{4.379501in}{1.300780in}}{\pgfqpoint{4.382335in}{1.307622in}}{\pgfqpoint{4.382335in}{1.314755in}}%
\pgfpathcurveto{\pgfqpoint{4.382335in}{1.321887in}}{\pgfqpoint{4.379501in}{1.328729in}}{\pgfqpoint{4.374457in}{1.333773in}}%
\pgfpathcurveto{\pgfqpoint{4.369414in}{1.338816in}}{\pgfqpoint{4.362572in}{1.341650in}}{\pgfqpoint{4.355439in}{1.341650in}}%
\pgfpathcurveto{\pgfqpoint{4.348306in}{1.341650in}}{\pgfqpoint{4.341465in}{1.338816in}}{\pgfqpoint{4.336421in}{1.333773in}}%
\pgfpathcurveto{\pgfqpoint{4.331378in}{1.328729in}}{\pgfqpoint{4.328544in}{1.321887in}}{\pgfqpoint{4.328544in}{1.314755in}}%
\pgfpathcurveto{\pgfqpoint{4.328544in}{1.307622in}}{\pgfqpoint{4.331378in}{1.300780in}}{\pgfqpoint{4.336421in}{1.295736in}}%
\pgfpathcurveto{\pgfqpoint{4.341465in}{1.290693in}}{\pgfqpoint{4.348306in}{1.287859in}}{\pgfqpoint{4.355439in}{1.287859in}}%
\pgfpathclose%
\pgfusepath{stroke,fill}%
\end{pgfscope}%
\begin{pgfscope}%
\pgfpathrectangle{\pgfqpoint{2.867647in}{0.500000in}}{\pgfqpoint{1.764706in}{1.700000in}}%
\pgfusepath{clip}%
\pgfsetbuttcap%
\pgfsetroundjoin%
\definecolor{currentfill}{rgb}{0.928830,0.305981,0.243623}%
\pgfsetfillcolor{currentfill}%
\pgfsetlinewidth{0.311001pt}%
\definecolor{currentstroke}{rgb}{1.000000,1.000000,1.000000}%
\pgfsetstrokecolor{currentstroke}%
\pgfsetdash{}{0pt}%
\pgfpathmoveto{\pgfqpoint{4.356598in}{1.490981in}}%
\pgfpathcurveto{\pgfqpoint{4.363731in}{1.490981in}}{\pgfqpoint{4.370572in}{1.493815in}}{\pgfqpoint{4.375616in}{1.498858in}}%
\pgfpathcurveto{\pgfqpoint{4.380660in}{1.503902in}}{\pgfqpoint{4.383493in}{1.510744in}}{\pgfqpoint{4.383493in}{1.517876in}}%
\pgfpathcurveto{\pgfqpoint{4.383493in}{1.525009in}}{\pgfqpoint{4.380660in}{1.531851in}}{\pgfqpoint{4.375616in}{1.536894in}}%
\pgfpathcurveto{\pgfqpoint{4.370572in}{1.541938in}}{\pgfqpoint{4.363731in}{1.544772in}}{\pgfqpoint{4.356598in}{1.544772in}}%
\pgfpathcurveto{\pgfqpoint{4.349465in}{1.544772in}}{\pgfqpoint{4.342623in}{1.541938in}}{\pgfqpoint{4.337580in}{1.536894in}}%
\pgfpathcurveto{\pgfqpoint{4.332536in}{1.531851in}}{\pgfqpoint{4.329702in}{1.525009in}}{\pgfqpoint{4.329702in}{1.517876in}}%
\pgfpathcurveto{\pgfqpoint{4.329702in}{1.510744in}}{\pgfqpoint{4.332536in}{1.503902in}}{\pgfqpoint{4.337580in}{1.498858in}}%
\pgfpathcurveto{\pgfqpoint{4.342623in}{1.493815in}}{\pgfqpoint{4.349465in}{1.490981in}}{\pgfqpoint{4.356598in}{1.490981in}}%
\pgfpathclose%
\pgfusepath{stroke,fill}%
\end{pgfscope}%
\begin{pgfscope}%
\pgfpathrectangle{\pgfqpoint{2.867647in}{0.500000in}}{\pgfqpoint{1.764706in}{1.700000in}}%
\pgfusepath{clip}%
\pgfsetbuttcap%
\pgfsetroundjoin%
\definecolor{currentfill}{rgb}{0.973832,0.856556,0.771584}%
\pgfsetfillcolor{currentfill}%
\pgfsetlinewidth{0.311001pt}%
\definecolor{currentstroke}{rgb}{1.000000,1.000000,1.000000}%
\pgfsetstrokecolor{currentstroke}%
\pgfsetdash{}{0pt}%
\pgfpathmoveto{\pgfqpoint{4.248830in}{1.411660in}}%
\pgfpathcurveto{\pgfqpoint{4.255963in}{1.411660in}}{\pgfqpoint{4.262804in}{1.414494in}}{\pgfqpoint{4.267848in}{1.419538in}}%
\pgfpathcurveto{\pgfqpoint{4.272892in}{1.424582in}}{\pgfqpoint{4.275725in}{1.431423in}}{\pgfqpoint{4.275725in}{1.438556in}}%
\pgfpathcurveto{\pgfqpoint{4.275725in}{1.445689in}}{\pgfqpoint{4.272892in}{1.452531in}}{\pgfqpoint{4.267848in}{1.457574in}}%
\pgfpathcurveto{\pgfqpoint{4.262804in}{1.462618in}}{\pgfqpoint{4.255963in}{1.465452in}}{\pgfqpoint{4.248830in}{1.465452in}}%
\pgfpathcurveto{\pgfqpoint{4.241697in}{1.465452in}}{\pgfqpoint{4.234855in}{1.462618in}}{\pgfqpoint{4.229812in}{1.457574in}}%
\pgfpathcurveto{\pgfqpoint{4.224768in}{1.452531in}}{\pgfqpoint{4.221934in}{1.445689in}}{\pgfqpoint{4.221934in}{1.438556in}}%
\pgfpathcurveto{\pgfqpoint{4.221934in}{1.431423in}}{\pgfqpoint{4.224768in}{1.424582in}}{\pgfqpoint{4.229812in}{1.419538in}}%
\pgfpathcurveto{\pgfqpoint{4.234855in}{1.414494in}}{\pgfqpoint{4.241697in}{1.411660in}}{\pgfqpoint{4.248830in}{1.411660in}}%
\pgfpathclose%
\pgfusepath{stroke,fill}%
\end{pgfscope}%
\begin{pgfscope}%
\pgfpathrectangle{\pgfqpoint{2.867647in}{0.500000in}}{\pgfqpoint{1.764706in}{1.700000in}}%
\pgfusepath{clip}%
\pgfsetbuttcap%
\pgfsetroundjoin%
\definecolor{currentfill}{rgb}{0.964920,0.695342,0.545192}%
\pgfsetfillcolor{currentfill}%
\pgfsetlinewidth{0.311001pt}%
\definecolor{currentstroke}{rgb}{1.000000,1.000000,1.000000}%
\pgfsetstrokecolor{currentstroke}%
\pgfsetdash{}{0pt}%
\pgfpathmoveto{\pgfqpoint{3.989269in}{1.638053in}}%
\pgfpathcurveto{\pgfqpoint{3.996402in}{1.638053in}}{\pgfqpoint{4.003244in}{1.640887in}}{\pgfqpoint{4.008287in}{1.645931in}}%
\pgfpathcurveto{\pgfqpoint{4.013331in}{1.650975in}}{\pgfqpoint{4.016165in}{1.657816in}}{\pgfqpoint{4.016165in}{1.664949in}}%
\pgfpathcurveto{\pgfqpoint{4.016165in}{1.672082in}}{\pgfqpoint{4.013331in}{1.678923in}}{\pgfqpoint{4.008287in}{1.683967in}}%
\pgfpathcurveto{\pgfqpoint{4.003244in}{1.689011in}}{\pgfqpoint{3.996402in}{1.691845in}}{\pgfqpoint{3.989269in}{1.691845in}}%
\pgfpathcurveto{\pgfqpoint{3.982136in}{1.691845in}}{\pgfqpoint{3.975295in}{1.689011in}}{\pgfqpoint{3.970251in}{1.683967in}}%
\pgfpathcurveto{\pgfqpoint{3.965207in}{1.678923in}}{\pgfqpoint{3.962373in}{1.672082in}}{\pgfqpoint{3.962373in}{1.664949in}}%
\pgfpathcurveto{\pgfqpoint{3.962373in}{1.657816in}}{\pgfqpoint{3.965207in}{1.650975in}}{\pgfqpoint{3.970251in}{1.645931in}}%
\pgfpathcurveto{\pgfqpoint{3.975295in}{1.640887in}}{\pgfqpoint{3.982136in}{1.638053in}}{\pgfqpoint{3.989269in}{1.638053in}}%
\pgfpathclose%
\pgfusepath{stroke,fill}%
\end{pgfscope}%
\begin{pgfscope}%
\pgfpathrectangle{\pgfqpoint{2.867647in}{0.500000in}}{\pgfqpoint{1.764706in}{1.700000in}}%
\pgfusepath{clip}%
\pgfsetbuttcap%
\pgfsetroundjoin%
\definecolor{currentfill}{rgb}{0.968509,0.792226,0.676405}%
\pgfsetfillcolor{currentfill}%
\pgfsetlinewidth{0.311001pt}%
\definecolor{currentstroke}{rgb}{1.000000,1.000000,1.000000}%
\pgfsetstrokecolor{currentstroke}%
\pgfsetdash{}{0pt}%
\pgfpathmoveto{\pgfqpoint{4.061003in}{1.708787in}}%
\pgfpathcurveto{\pgfqpoint{4.068136in}{1.708787in}}{\pgfqpoint{4.074978in}{1.711621in}}{\pgfqpoint{4.080021in}{1.716665in}}%
\pgfpathcurveto{\pgfqpoint{4.085065in}{1.721708in}}{\pgfqpoint{4.087899in}{1.728550in}}{\pgfqpoint{4.087899in}{1.735683in}}%
\pgfpathcurveto{\pgfqpoint{4.087899in}{1.742816in}}{\pgfqpoint{4.085065in}{1.749657in}}{\pgfqpoint{4.080021in}{1.754701in}}%
\pgfpathcurveto{\pgfqpoint{4.074978in}{1.759745in}}{\pgfqpoint{4.068136in}{1.762579in}}{\pgfqpoint{4.061003in}{1.762579in}}%
\pgfpathcurveto{\pgfqpoint{4.053870in}{1.762579in}}{\pgfqpoint{4.047029in}{1.759745in}}{\pgfqpoint{4.041985in}{1.754701in}}%
\pgfpathcurveto{\pgfqpoint{4.036941in}{1.749657in}}{\pgfqpoint{4.034107in}{1.742816in}}{\pgfqpoint{4.034107in}{1.735683in}}%
\pgfpathcurveto{\pgfqpoint{4.034107in}{1.728550in}}{\pgfqpoint{4.036941in}{1.721708in}}{\pgfqpoint{4.041985in}{1.716665in}}%
\pgfpathcurveto{\pgfqpoint{4.047029in}{1.711621in}}{\pgfqpoint{4.053870in}{1.708787in}}{\pgfqpoint{4.061003in}{1.708787in}}%
\pgfpathclose%
\pgfusepath{stroke,fill}%
\end{pgfscope}%
\begin{pgfscope}%
\pgfpathrectangle{\pgfqpoint{2.867647in}{0.500000in}}{\pgfqpoint{1.764706in}{1.700000in}}%
\pgfusepath{clip}%
\pgfsetbuttcap%
\pgfsetroundjoin%
\definecolor{currentfill}{rgb}{0.975018,0.868213,0.788710}%
\pgfsetfillcolor{currentfill}%
\pgfsetlinewidth{0.311001pt}%
\definecolor{currentstroke}{rgb}{1.000000,1.000000,1.000000}%
\pgfsetstrokecolor{currentstroke}%
\pgfsetdash{}{0pt}%
\pgfpathmoveto{\pgfqpoint{4.091204in}{1.062283in}}%
\pgfpathcurveto{\pgfqpoint{4.098336in}{1.062283in}}{\pgfqpoint{4.105178in}{1.065117in}}{\pgfqpoint{4.110222in}{1.070160in}}%
\pgfpathcurveto{\pgfqpoint{4.115265in}{1.075204in}}{\pgfqpoint{4.118099in}{1.082046in}}{\pgfqpoint{4.118099in}{1.089178in}}%
\pgfpathcurveto{\pgfqpoint{4.118099in}{1.096311in}}{\pgfqpoint{4.115265in}{1.103153in}}{\pgfqpoint{4.110222in}{1.108197in}}%
\pgfpathcurveto{\pgfqpoint{4.105178in}{1.113240in}}{\pgfqpoint{4.098336in}{1.116074in}}{\pgfqpoint{4.091204in}{1.116074in}}%
\pgfpathcurveto{\pgfqpoint{4.084071in}{1.116074in}}{\pgfqpoint{4.077229in}{1.113240in}}{\pgfqpoint{4.072185in}{1.108197in}}%
\pgfpathcurveto{\pgfqpoint{4.067142in}{1.103153in}}{\pgfqpoint{4.064308in}{1.096311in}}{\pgfqpoint{4.064308in}{1.089178in}}%
\pgfpathcurveto{\pgfqpoint{4.064308in}{1.082046in}}{\pgfqpoint{4.067142in}{1.075204in}}{\pgfqpoint{4.072185in}{1.070160in}}%
\pgfpathcurveto{\pgfqpoint{4.077229in}{1.065117in}}{\pgfqpoint{4.084071in}{1.062283in}}{\pgfqpoint{4.091204in}{1.062283in}}%
\pgfpathclose%
\pgfusepath{stroke,fill}%
\end{pgfscope}%
\begin{pgfscope}%
\pgfpathrectangle{\pgfqpoint{2.867647in}{0.500000in}}{\pgfqpoint{1.764706in}{1.700000in}}%
\pgfusepath{clip}%
\pgfsetbuttcap%
\pgfsetroundjoin%
\definecolor{currentfill}{rgb}{0.963728,0.638439,0.479050}%
\pgfsetfillcolor{currentfill}%
\pgfsetlinewidth{0.311001pt}%
\definecolor{currentstroke}{rgb}{1.000000,1.000000,1.000000}%
\pgfsetstrokecolor{currentstroke}%
\pgfsetdash{}{0pt}%
\pgfpathmoveto{\pgfqpoint{3.971424in}{1.760361in}}%
\pgfpathcurveto{\pgfqpoint{3.978557in}{1.760361in}}{\pgfqpoint{3.985398in}{1.763195in}}{\pgfqpoint{3.990442in}{1.768239in}}%
\pgfpathcurveto{\pgfqpoint{3.995486in}{1.773283in}}{\pgfqpoint{3.998320in}{1.780124in}}{\pgfqpoint{3.998320in}{1.787257in}}%
\pgfpathcurveto{\pgfqpoint{3.998320in}{1.794390in}}{\pgfqpoint{3.995486in}{1.801232in}}{\pgfqpoint{3.990442in}{1.806275in}}%
\pgfpathcurveto{\pgfqpoint{3.985398in}{1.811319in}}{\pgfqpoint{3.978557in}{1.814153in}}{\pgfqpoint{3.971424in}{1.814153in}}%
\pgfpathcurveto{\pgfqpoint{3.964291in}{1.814153in}}{\pgfqpoint{3.957449in}{1.811319in}}{\pgfqpoint{3.952406in}{1.806275in}}%
\pgfpathcurveto{\pgfqpoint{3.947362in}{1.801232in}}{\pgfqpoint{3.944528in}{1.794390in}}{\pgfqpoint{3.944528in}{1.787257in}}%
\pgfpathcurveto{\pgfqpoint{3.944528in}{1.780124in}}{\pgfqpoint{3.947362in}{1.773283in}}{\pgfqpoint{3.952406in}{1.768239in}}%
\pgfpathcurveto{\pgfqpoint{3.957449in}{1.763195in}}{\pgfqpoint{3.964291in}{1.760361in}}{\pgfqpoint{3.971424in}{1.760361in}}%
\pgfpathclose%
\pgfusepath{stroke,fill}%
\end{pgfscope}%
\begin{pgfscope}%
\pgfpathrectangle{\pgfqpoint{2.867647in}{0.500000in}}{\pgfqpoint{1.764706in}{1.700000in}}%
\pgfusepath{clip}%
\pgfsetbuttcap%
\pgfsetroundjoin%
\definecolor{currentfill}{rgb}{0.979124,0.903132,0.839793}%
\pgfsetfillcolor{currentfill}%
\pgfsetlinewidth{0.311001pt}%
\definecolor{currentstroke}{rgb}{1.000000,1.000000,1.000000}%
\pgfsetstrokecolor{currentstroke}%
\pgfsetdash{}{0pt}%
\pgfpathmoveto{\pgfqpoint{4.128099in}{1.135630in}}%
\pgfpathcurveto{\pgfqpoint{4.135232in}{1.135630in}}{\pgfqpoint{4.142073in}{1.138464in}}{\pgfqpoint{4.147117in}{1.143507in}}%
\pgfpathcurveto{\pgfqpoint{4.152161in}{1.148551in}}{\pgfqpoint{4.154995in}{1.155393in}}{\pgfqpoint{4.154995in}{1.162525in}}%
\pgfpathcurveto{\pgfqpoint{4.154995in}{1.169658in}}{\pgfqpoint{4.152161in}{1.176500in}}{\pgfqpoint{4.147117in}{1.181544in}}%
\pgfpathcurveto{\pgfqpoint{4.142073in}{1.186587in}}{\pgfqpoint{4.135232in}{1.189421in}}{\pgfqpoint{4.128099in}{1.189421in}}%
\pgfpathcurveto{\pgfqpoint{4.120966in}{1.189421in}}{\pgfqpoint{4.114124in}{1.186587in}}{\pgfqpoint{4.109081in}{1.181544in}}%
\pgfpathcurveto{\pgfqpoint{4.104037in}{1.176500in}}{\pgfqpoint{4.101203in}{1.169658in}}{\pgfqpoint{4.101203in}{1.162525in}}%
\pgfpathcurveto{\pgfqpoint{4.101203in}{1.155393in}}{\pgfqpoint{4.104037in}{1.148551in}}{\pgfqpoint{4.109081in}{1.143507in}}%
\pgfpathcurveto{\pgfqpoint{4.114124in}{1.138464in}}{\pgfqpoint{4.120966in}{1.135630in}}{\pgfqpoint{4.128099in}{1.135630in}}%
\pgfpathclose%
\pgfusepath{stroke,fill}%
\end{pgfscope}%
\begin{pgfscope}%
\pgfpathrectangle{\pgfqpoint{2.867647in}{0.500000in}}{\pgfqpoint{1.764706in}{1.700000in}}%
\pgfusepath{clip}%
\pgfsetbuttcap%
\pgfsetroundjoin%
\definecolor{currentfill}{rgb}{0.977657,0.891500,0.822809}%
\pgfsetfillcolor{currentfill}%
\pgfsetlinewidth{0.311001pt}%
\definecolor{currentstroke}{rgb}{1.000000,1.000000,1.000000}%
\pgfsetstrokecolor{currentstroke}%
\pgfsetdash{}{0pt}%
\pgfpathmoveto{\pgfqpoint{4.123760in}{1.046357in}}%
\pgfpathcurveto{\pgfqpoint{4.130893in}{1.046357in}}{\pgfqpoint{4.137734in}{1.049190in}}{\pgfqpoint{4.142778in}{1.054234in}}%
\pgfpathcurveto{\pgfqpoint{4.147822in}{1.059278in}}{\pgfqpoint{4.150655in}{1.066119in}}{\pgfqpoint{4.150655in}{1.073252in}}%
\pgfpathcurveto{\pgfqpoint{4.150655in}{1.080385in}}{\pgfqpoint{4.147822in}{1.087227in}}{\pgfqpoint{4.142778in}{1.092270in}}%
\pgfpathcurveto{\pgfqpoint{4.137734in}{1.097314in}}{\pgfqpoint{4.130893in}{1.100148in}}{\pgfqpoint{4.123760in}{1.100148in}}%
\pgfpathcurveto{\pgfqpoint{4.116627in}{1.100148in}}{\pgfqpoint{4.109785in}{1.097314in}}{\pgfqpoint{4.104742in}{1.092270in}}%
\pgfpathcurveto{\pgfqpoint{4.099698in}{1.087227in}}{\pgfqpoint{4.096864in}{1.080385in}}{\pgfqpoint{4.096864in}{1.073252in}}%
\pgfpathcurveto{\pgfqpoint{4.096864in}{1.066119in}}{\pgfqpoint{4.099698in}{1.059278in}}{\pgfqpoint{4.104742in}{1.054234in}}%
\pgfpathcurveto{\pgfqpoint{4.109785in}{1.049190in}}{\pgfqpoint{4.116627in}{1.046357in}}{\pgfqpoint{4.123760in}{1.046357in}}%
\pgfpathclose%
\pgfusepath{stroke,fill}%
\end{pgfscope}%
\begin{pgfscope}%
\pgfpathrectangle{\pgfqpoint{2.867647in}{0.500000in}}{\pgfqpoint{1.764706in}{1.700000in}}%
\pgfusepath{clip}%
\pgfsetbuttcap%
\pgfsetroundjoin%
\definecolor{currentfill}{rgb}{0.975018,0.868213,0.788710}%
\pgfsetfillcolor{currentfill}%
\pgfsetlinewidth{0.311001pt}%
\definecolor{currentstroke}{rgb}{1.000000,1.000000,1.000000}%
\pgfsetstrokecolor{currentstroke}%
\pgfsetdash{}{0pt}%
\pgfpathmoveto{\pgfqpoint{4.128682in}{1.253741in}}%
\pgfpathcurveto{\pgfqpoint{4.135815in}{1.253741in}}{\pgfqpoint{4.142656in}{1.256575in}}{\pgfqpoint{4.147700in}{1.261619in}}%
\pgfpathcurveto{\pgfqpoint{4.152744in}{1.266663in}}{\pgfqpoint{4.155578in}{1.273504in}}{\pgfqpoint{4.155578in}{1.280637in}}%
\pgfpathcurveto{\pgfqpoint{4.155578in}{1.287770in}}{\pgfqpoint{4.152744in}{1.294612in}}{\pgfqpoint{4.147700in}{1.299655in}}%
\pgfpathcurveto{\pgfqpoint{4.142656in}{1.304699in}}{\pgfqpoint{4.135815in}{1.307533in}}{\pgfqpoint{4.128682in}{1.307533in}}%
\pgfpathcurveto{\pgfqpoint{4.121549in}{1.307533in}}{\pgfqpoint{4.114707in}{1.304699in}}{\pgfqpoint{4.109664in}{1.299655in}}%
\pgfpathcurveto{\pgfqpoint{4.104620in}{1.294612in}}{\pgfqpoint{4.101786in}{1.287770in}}{\pgfqpoint{4.101786in}{1.280637in}}%
\pgfpathcurveto{\pgfqpoint{4.101786in}{1.273504in}}{\pgfqpoint{4.104620in}{1.266663in}}{\pgfqpoint{4.109664in}{1.261619in}}%
\pgfpathcurveto{\pgfqpoint{4.114707in}{1.256575in}}{\pgfqpoint{4.121549in}{1.253741in}}{\pgfqpoint{4.128682in}{1.253741in}}%
\pgfpathclose%
\pgfusepath{stroke,fill}%
\end{pgfscope}%
\begin{pgfscope}%
\pgfpathrectangle{\pgfqpoint{2.867647in}{0.500000in}}{\pgfqpoint{1.764706in}{1.700000in}}%
\pgfusepath{clip}%
\pgfsetbuttcap%
\pgfsetroundjoin%
\definecolor{currentfill}{rgb}{0.977657,0.891500,0.822809}%
\pgfsetfillcolor{currentfill}%
\pgfsetlinewidth{0.311001pt}%
\definecolor{currentstroke}{rgb}{1.000000,1.000000,1.000000}%
\pgfsetstrokecolor{currentstroke}%
\pgfsetdash{}{0pt}%
\pgfpathmoveto{\pgfqpoint{4.206646in}{1.496691in}}%
\pgfpathcurveto{\pgfqpoint{4.213778in}{1.496691in}}{\pgfqpoint{4.220620in}{1.499525in}}{\pgfqpoint{4.225664in}{1.504568in}}%
\pgfpathcurveto{\pgfqpoint{4.230707in}{1.509612in}}{\pgfqpoint{4.233541in}{1.516454in}}{\pgfqpoint{4.233541in}{1.523586in}}%
\pgfpathcurveto{\pgfqpoint{4.233541in}{1.530719in}}{\pgfqpoint{4.230707in}{1.537561in}}{\pgfqpoint{4.225664in}{1.542604in}}%
\pgfpathcurveto{\pgfqpoint{4.220620in}{1.547648in}}{\pgfqpoint{4.213778in}{1.550482in}}{\pgfqpoint{4.206646in}{1.550482in}}%
\pgfpathcurveto{\pgfqpoint{4.199513in}{1.550482in}}{\pgfqpoint{4.192671in}{1.547648in}}{\pgfqpoint{4.187627in}{1.542604in}}%
\pgfpathcurveto{\pgfqpoint{4.182584in}{1.537561in}}{\pgfqpoint{4.179750in}{1.530719in}}{\pgfqpoint{4.179750in}{1.523586in}}%
\pgfpathcurveto{\pgfqpoint{4.179750in}{1.516454in}}{\pgfqpoint{4.182584in}{1.509612in}}{\pgfqpoint{4.187627in}{1.504568in}}%
\pgfpathcurveto{\pgfqpoint{4.192671in}{1.499525in}}{\pgfqpoint{4.199513in}{1.496691in}}{\pgfqpoint{4.206646in}{1.496691in}}%
\pgfpathclose%
\pgfusepath{stroke,fill}%
\end{pgfscope}%
\begin{pgfscope}%
\pgfpathrectangle{\pgfqpoint{2.867647in}{0.500000in}}{\pgfqpoint{1.764706in}{1.700000in}}%
\pgfusepath{clip}%
\pgfsetbuttcap%
\pgfsetroundjoin%
\definecolor{currentfill}{rgb}{0.975018,0.868213,0.788710}%
\pgfsetfillcolor{currentfill}%
\pgfsetlinewidth{0.311001pt}%
\definecolor{currentstroke}{rgb}{1.000000,1.000000,1.000000}%
\pgfsetstrokecolor{currentstroke}%
\pgfsetdash{}{0pt}%
\pgfpathmoveto{\pgfqpoint{4.124253in}{1.229271in}}%
\pgfpathcurveto{\pgfqpoint{4.131386in}{1.229271in}}{\pgfqpoint{4.138228in}{1.232105in}}{\pgfqpoint{4.143271in}{1.237148in}}%
\pgfpathcurveto{\pgfqpoint{4.148315in}{1.242192in}}{\pgfqpoint{4.151149in}{1.249034in}}{\pgfqpoint{4.151149in}{1.256166in}}%
\pgfpathcurveto{\pgfqpoint{4.151149in}{1.263299in}}{\pgfqpoint{4.148315in}{1.270141in}}{\pgfqpoint{4.143271in}{1.275185in}}%
\pgfpathcurveto{\pgfqpoint{4.138228in}{1.280228in}}{\pgfqpoint{4.131386in}{1.283062in}}{\pgfqpoint{4.124253in}{1.283062in}}%
\pgfpathcurveto{\pgfqpoint{4.117120in}{1.283062in}}{\pgfqpoint{4.110279in}{1.280228in}}{\pgfqpoint{4.105235in}{1.275185in}}%
\pgfpathcurveto{\pgfqpoint{4.100191in}{1.270141in}}{\pgfqpoint{4.097358in}{1.263299in}}{\pgfqpoint{4.097358in}{1.256166in}}%
\pgfpathcurveto{\pgfqpoint{4.097358in}{1.249034in}}{\pgfqpoint{4.100191in}{1.242192in}}{\pgfqpoint{4.105235in}{1.237148in}}%
\pgfpathcurveto{\pgfqpoint{4.110279in}{1.232105in}}{\pgfqpoint{4.117120in}{1.229271in}}{\pgfqpoint{4.124253in}{1.229271in}}%
\pgfpathclose%
\pgfusepath{stroke,fill}%
\end{pgfscope}%
\begin{pgfscope}%
\pgfpathrectangle{\pgfqpoint{2.867647in}{0.500000in}}{\pgfqpoint{1.764706in}{1.700000in}}%
\pgfusepath{clip}%
\pgfsetbuttcap%
\pgfsetroundjoin%
\definecolor{currentfill}{rgb}{0.974412,0.862387,0.780156}%
\pgfsetfillcolor{currentfill}%
\pgfsetlinewidth{0.311001pt}%
\definecolor{currentstroke}{rgb}{1.000000,1.000000,1.000000}%
\pgfsetstrokecolor{currentstroke}%
\pgfsetdash{}{0pt}%
\pgfpathmoveto{\pgfqpoint{4.223033in}{1.506294in}}%
\pgfpathcurveto{\pgfqpoint{4.230166in}{1.506294in}}{\pgfqpoint{4.237008in}{1.509128in}}{\pgfqpoint{4.242051in}{1.514171in}}%
\pgfpathcurveto{\pgfqpoint{4.247095in}{1.519215in}}{\pgfqpoint{4.249929in}{1.526057in}}{\pgfqpoint{4.249929in}{1.533189in}}%
\pgfpathcurveto{\pgfqpoint{4.249929in}{1.540322in}}{\pgfqpoint{4.247095in}{1.547164in}}{\pgfqpoint{4.242051in}{1.552208in}}%
\pgfpathcurveto{\pgfqpoint{4.237008in}{1.557251in}}{\pgfqpoint{4.230166in}{1.560085in}}{\pgfqpoint{4.223033in}{1.560085in}}%
\pgfpathcurveto{\pgfqpoint{4.215900in}{1.560085in}}{\pgfqpoint{4.209059in}{1.557251in}}{\pgfqpoint{4.204015in}{1.552208in}}%
\pgfpathcurveto{\pgfqpoint{4.198971in}{1.547164in}}{\pgfqpoint{4.196137in}{1.540322in}}{\pgfqpoint{4.196137in}{1.533189in}}%
\pgfpathcurveto{\pgfqpoint{4.196137in}{1.526057in}}{\pgfqpoint{4.198971in}{1.519215in}}{\pgfqpoint{4.204015in}{1.514171in}}%
\pgfpathcurveto{\pgfqpoint{4.209059in}{1.509128in}}{\pgfqpoint{4.215900in}{1.506294in}}{\pgfqpoint{4.223033in}{1.506294in}}%
\pgfpathclose%
\pgfusepath{stroke,fill}%
\end{pgfscope}%
\begin{pgfscope}%
\pgfpathrectangle{\pgfqpoint{2.867647in}{0.500000in}}{\pgfqpoint{1.764706in}{1.700000in}}%
\pgfusepath{clip}%
\pgfsetbuttcap%
\pgfsetroundjoin%
\definecolor{currentfill}{rgb}{0.975018,0.868213,0.788710}%
\pgfsetfillcolor{currentfill}%
\pgfsetlinewidth{0.311001pt}%
\definecolor{currentstroke}{rgb}{1.000000,1.000000,1.000000}%
\pgfsetstrokecolor{currentstroke}%
\pgfsetdash{}{0pt}%
\pgfpathmoveto{\pgfqpoint{4.106161in}{1.008862in}}%
\pgfpathcurveto{\pgfqpoint{4.113294in}{1.008862in}}{\pgfqpoint{4.120135in}{1.011696in}}{\pgfqpoint{4.125179in}{1.016740in}}%
\pgfpathcurveto{\pgfqpoint{4.130223in}{1.021784in}}{\pgfqpoint{4.133057in}{1.028625in}}{\pgfqpoint{4.133057in}{1.035758in}}%
\pgfpathcurveto{\pgfqpoint{4.133057in}{1.042891in}}{\pgfqpoint{4.130223in}{1.049733in}}{\pgfqpoint{4.125179in}{1.054776in}}%
\pgfpathcurveto{\pgfqpoint{4.120135in}{1.059820in}}{\pgfqpoint{4.113294in}{1.062654in}}{\pgfqpoint{4.106161in}{1.062654in}}%
\pgfpathcurveto{\pgfqpoint{4.099028in}{1.062654in}}{\pgfqpoint{4.092186in}{1.059820in}}{\pgfqpoint{4.087143in}{1.054776in}}%
\pgfpathcurveto{\pgfqpoint{4.082099in}{1.049733in}}{\pgfqpoint{4.079265in}{1.042891in}}{\pgfqpoint{4.079265in}{1.035758in}}%
\pgfpathcurveto{\pgfqpoint{4.079265in}{1.028625in}}{\pgfqpoint{4.082099in}{1.021784in}}{\pgfqpoint{4.087143in}{1.016740in}}%
\pgfpathcurveto{\pgfqpoint{4.092186in}{1.011696in}}{\pgfqpoint{4.099028in}{1.008862in}}{\pgfqpoint{4.106161in}{1.008862in}}%
\pgfpathclose%
\pgfusepath{stroke,fill}%
\end{pgfscope}%
\begin{pgfscope}%
\pgfpathrectangle{\pgfqpoint{2.867647in}{0.500000in}}{\pgfqpoint{1.764706in}{1.700000in}}%
\pgfusepath{clip}%
\pgfsetbuttcap%
\pgfsetroundjoin%
\definecolor{currentfill}{rgb}{0.945204,0.390623,0.270949}%
\pgfsetfillcolor{currentfill}%
\pgfsetlinewidth{0.311001pt}%
\definecolor{currentstroke}{rgb}{1.000000,1.000000,1.000000}%
\pgfsetstrokecolor{currentstroke}%
\pgfsetdash{}{0pt}%
\pgfpathmoveto{\pgfqpoint{3.880428in}{0.892684in}}%
\pgfpathcurveto{\pgfqpoint{3.887561in}{0.892684in}}{\pgfqpoint{3.894403in}{0.895518in}}{\pgfqpoint{3.899446in}{0.900562in}}%
\pgfpathcurveto{\pgfqpoint{3.904490in}{0.905605in}}{\pgfqpoint{3.907324in}{0.912447in}}{\pgfqpoint{3.907324in}{0.919580in}}%
\pgfpathcurveto{\pgfqpoint{3.907324in}{0.926713in}}{\pgfqpoint{3.904490in}{0.933554in}}{\pgfqpoint{3.899446in}{0.938598in}}%
\pgfpathcurveto{\pgfqpoint{3.894403in}{0.943642in}}{\pgfqpoint{3.887561in}{0.946476in}}{\pgfqpoint{3.880428in}{0.946476in}}%
\pgfpathcurveto{\pgfqpoint{3.873295in}{0.946476in}}{\pgfqpoint{3.866454in}{0.943642in}}{\pgfqpoint{3.861410in}{0.938598in}}%
\pgfpathcurveto{\pgfqpoint{3.856366in}{0.933554in}}{\pgfqpoint{3.853532in}{0.926713in}}{\pgfqpoint{3.853532in}{0.919580in}}%
\pgfpathcurveto{\pgfqpoint{3.853532in}{0.912447in}}{\pgfqpoint{3.856366in}{0.905605in}}{\pgfqpoint{3.861410in}{0.900562in}}%
\pgfpathcurveto{\pgfqpoint{3.866454in}{0.895518in}}{\pgfqpoint{3.873295in}{0.892684in}}{\pgfqpoint{3.880428in}{0.892684in}}%
\pgfpathclose%
\pgfusepath{stroke,fill}%
\end{pgfscope}%
\begin{pgfscope}%
\pgfpathrectangle{\pgfqpoint{2.867647in}{0.500000in}}{\pgfqpoint{1.764706in}{1.700000in}}%
\pgfusepath{clip}%
\pgfsetbuttcap%
\pgfsetroundjoin%
\definecolor{currentfill}{rgb}{0.981377,0.920617,0.865369}%
\pgfsetfillcolor{currentfill}%
\pgfsetlinewidth{0.311001pt}%
\definecolor{currentstroke}{rgb}{1.000000,1.000000,1.000000}%
\pgfsetstrokecolor{currentstroke}%
\pgfsetdash{}{0pt}%
\pgfpathmoveto{\pgfqpoint{4.162350in}{1.184192in}}%
\pgfpathcurveto{\pgfqpoint{4.169482in}{1.184192in}}{\pgfqpoint{4.176324in}{1.187026in}}{\pgfqpoint{4.181368in}{1.192070in}}%
\pgfpathcurveto{\pgfqpoint{4.186411in}{1.197114in}}{\pgfqpoint{4.189245in}{1.203955in}}{\pgfqpoint{4.189245in}{1.211088in}}%
\pgfpathcurveto{\pgfqpoint{4.189245in}{1.218221in}}{\pgfqpoint{4.186411in}{1.225063in}}{\pgfqpoint{4.181368in}{1.230106in}}%
\pgfpathcurveto{\pgfqpoint{4.176324in}{1.235150in}}{\pgfqpoint{4.169482in}{1.237984in}}{\pgfqpoint{4.162350in}{1.237984in}}%
\pgfpathcurveto{\pgfqpoint{4.155217in}{1.237984in}}{\pgfqpoint{4.148375in}{1.235150in}}{\pgfqpoint{4.143331in}{1.230106in}}%
\pgfpathcurveto{\pgfqpoint{4.138288in}{1.225063in}}{\pgfqpoint{4.135454in}{1.218221in}}{\pgfqpoint{4.135454in}{1.211088in}}%
\pgfpathcurveto{\pgfqpoint{4.135454in}{1.203955in}}{\pgfqpoint{4.138288in}{1.197114in}}{\pgfqpoint{4.143331in}{1.192070in}}%
\pgfpathcurveto{\pgfqpoint{4.148375in}{1.187026in}}{\pgfqpoint{4.155217in}{1.184192in}}{\pgfqpoint{4.162350in}{1.184192in}}%
\pgfpathclose%
\pgfusepath{stroke,fill}%
\end{pgfscope}%
\begin{pgfscope}%
\pgfpathrectangle{\pgfqpoint{2.867647in}{0.500000in}}{\pgfqpoint{1.764706in}{1.700000in}}%
\pgfusepath{clip}%
\pgfsetbuttcap%
\pgfsetroundjoin%
\definecolor{currentfill}{rgb}{0.979124,0.903132,0.839793}%
\pgfsetfillcolor{currentfill}%
\pgfsetlinewidth{0.311001pt}%
\definecolor{currentstroke}{rgb}{1.000000,1.000000,1.000000}%
\pgfsetstrokecolor{currentstroke}%
\pgfsetdash{}{0pt}%
\pgfpathmoveto{\pgfqpoint{4.158827in}{1.323275in}}%
\pgfpathcurveto{\pgfqpoint{4.165959in}{1.323275in}}{\pgfqpoint{4.172801in}{1.326109in}}{\pgfqpoint{4.177845in}{1.331153in}}%
\pgfpathcurveto{\pgfqpoint{4.182888in}{1.336197in}}{\pgfqpoint{4.185722in}{1.343038in}}{\pgfqpoint{4.185722in}{1.350171in}}%
\pgfpathcurveto{\pgfqpoint{4.185722in}{1.357304in}}{\pgfqpoint{4.182888in}{1.364146in}}{\pgfqpoint{4.177845in}{1.369189in}}%
\pgfpathcurveto{\pgfqpoint{4.172801in}{1.374233in}}{\pgfqpoint{4.165959in}{1.377067in}}{\pgfqpoint{4.158827in}{1.377067in}}%
\pgfpathcurveto{\pgfqpoint{4.151694in}{1.377067in}}{\pgfqpoint{4.144852in}{1.374233in}}{\pgfqpoint{4.139808in}{1.369189in}}%
\pgfpathcurveto{\pgfqpoint{4.134765in}{1.364146in}}{\pgfqpoint{4.131931in}{1.357304in}}{\pgfqpoint{4.131931in}{1.350171in}}%
\pgfpathcurveto{\pgfqpoint{4.131931in}{1.343038in}}{\pgfqpoint{4.134765in}{1.336197in}}{\pgfqpoint{4.139808in}{1.331153in}}%
\pgfpathcurveto{\pgfqpoint{4.144852in}{1.326109in}}{\pgfqpoint{4.151694in}{1.323275in}}{\pgfqpoint{4.158827in}{1.323275in}}%
\pgfpathclose%
\pgfusepath{stroke,fill}%
\end{pgfscope}%
\begin{pgfscope}%
\pgfpathrectangle{\pgfqpoint{2.867647in}{0.500000in}}{\pgfqpoint{1.764706in}{1.700000in}}%
\pgfusepath{clip}%
\pgfsetbuttcap%
\pgfsetroundjoin%
\definecolor{currentfill}{rgb}{0.970255,0.815666,0.711203}%
\pgfsetfillcolor{currentfill}%
\pgfsetlinewidth{0.311001pt}%
\definecolor{currentstroke}{rgb}{1.000000,1.000000,1.000000}%
\pgfsetstrokecolor{currentstroke}%
\pgfsetdash{}{0pt}%
\pgfpathmoveto{\pgfqpoint{4.052824in}{0.971803in}}%
\pgfpathcurveto{\pgfqpoint{4.059957in}{0.971803in}}{\pgfqpoint{4.066799in}{0.974637in}}{\pgfqpoint{4.071842in}{0.979681in}}%
\pgfpathcurveto{\pgfqpoint{4.076886in}{0.984724in}}{\pgfqpoint{4.079720in}{0.991566in}}{\pgfqpoint{4.079720in}{0.998699in}}%
\pgfpathcurveto{\pgfqpoint{4.079720in}{1.005832in}}{\pgfqpoint{4.076886in}{1.012673in}}{\pgfqpoint{4.071842in}{1.017717in}}%
\pgfpathcurveto{\pgfqpoint{4.066799in}{1.022761in}}{\pgfqpoint{4.059957in}{1.025595in}}{\pgfqpoint{4.052824in}{1.025595in}}%
\pgfpathcurveto{\pgfqpoint{4.045691in}{1.025595in}}{\pgfqpoint{4.038850in}{1.022761in}}{\pgfqpoint{4.033806in}{1.017717in}}%
\pgfpathcurveto{\pgfqpoint{4.028762in}{1.012673in}}{\pgfqpoint{4.025928in}{1.005832in}}{\pgfqpoint{4.025928in}{0.998699in}}%
\pgfpathcurveto{\pgfqpoint{4.025928in}{0.991566in}}{\pgfqpoint{4.028762in}{0.984724in}}{\pgfqpoint{4.033806in}{0.979681in}}%
\pgfpathcurveto{\pgfqpoint{4.038850in}{0.974637in}}{\pgfqpoint{4.045691in}{0.971803in}}{\pgfqpoint{4.052824in}{0.971803in}}%
\pgfpathclose%
\pgfusepath{stroke,fill}%
\end{pgfscope}%
\begin{pgfscope}%
\pgfpathrectangle{\pgfqpoint{2.867647in}{0.500000in}}{\pgfqpoint{1.764706in}{1.700000in}}%
\pgfusepath{clip}%
\pgfsetbuttcap%
\pgfsetroundjoin%
\definecolor{currentfill}{rgb}{0.971694,0.833208,0.737161}%
\pgfsetfillcolor{currentfill}%
\pgfsetlinewidth{0.311001pt}%
\definecolor{currentstroke}{rgb}{1.000000,1.000000,1.000000}%
\pgfsetstrokecolor{currentstroke}%
\pgfsetdash{}{0pt}%
\pgfpathmoveto{\pgfqpoint{4.099117in}{1.436741in}}%
\pgfpathcurveto{\pgfqpoint{4.106250in}{1.436741in}}{\pgfqpoint{4.113091in}{1.439574in}}{\pgfqpoint{4.118135in}{1.444618in}}%
\pgfpathcurveto{\pgfqpoint{4.123179in}{1.449662in}}{\pgfqpoint{4.126013in}{1.456503in}}{\pgfqpoint{4.126013in}{1.463636in}}%
\pgfpathcurveto{\pgfqpoint{4.126013in}{1.470769in}}{\pgfqpoint{4.123179in}{1.477611in}}{\pgfqpoint{4.118135in}{1.482654in}}%
\pgfpathcurveto{\pgfqpoint{4.113091in}{1.487698in}}{\pgfqpoint{4.106250in}{1.490532in}}{\pgfqpoint{4.099117in}{1.490532in}}%
\pgfpathcurveto{\pgfqpoint{4.091984in}{1.490532in}}{\pgfqpoint{4.085142in}{1.487698in}}{\pgfqpoint{4.080099in}{1.482654in}}%
\pgfpathcurveto{\pgfqpoint{4.075055in}{1.477611in}}{\pgfqpoint{4.072221in}{1.470769in}}{\pgfqpoint{4.072221in}{1.463636in}}%
\pgfpathcurveto{\pgfqpoint{4.072221in}{1.456503in}}{\pgfqpoint{4.075055in}{1.449662in}}{\pgfqpoint{4.080099in}{1.444618in}}%
\pgfpathcurveto{\pgfqpoint{4.085142in}{1.439574in}}{\pgfqpoint{4.091984in}{1.436741in}}{\pgfqpoint{4.099117in}{1.436741in}}%
\pgfpathclose%
\pgfusepath{stroke,fill}%
\end{pgfscope}%
\begin{pgfscope}%
\pgfpathrectangle{\pgfqpoint{2.867647in}{0.500000in}}{\pgfqpoint{1.764706in}{1.700000in}}%
\pgfusepath{clip}%
\pgfsetbuttcap%
\pgfsetroundjoin%
\definecolor{currentfill}{rgb}{0.978376,0.897317,0.831308}%
\pgfsetfillcolor{currentfill}%
\pgfsetlinewidth{0.311001pt}%
\definecolor{currentstroke}{rgb}{1.000000,1.000000,1.000000}%
\pgfsetstrokecolor{currentstroke}%
\pgfsetdash{}{0pt}%
\pgfpathmoveto{\pgfqpoint{4.153123in}{1.073337in}}%
\pgfpathcurveto{\pgfqpoint{4.160256in}{1.073337in}}{\pgfqpoint{4.167097in}{1.076171in}}{\pgfqpoint{4.172141in}{1.081214in}}%
\pgfpathcurveto{\pgfqpoint{4.177185in}{1.086258in}}{\pgfqpoint{4.180019in}{1.093100in}}{\pgfqpoint{4.180019in}{1.100233in}}%
\pgfpathcurveto{\pgfqpoint{4.180019in}{1.107365in}}{\pgfqpoint{4.177185in}{1.114207in}}{\pgfqpoint{4.172141in}{1.119251in}}%
\pgfpathcurveto{\pgfqpoint{4.167097in}{1.124294in}}{\pgfqpoint{4.160256in}{1.127128in}}{\pgfqpoint{4.153123in}{1.127128in}}%
\pgfpathcurveto{\pgfqpoint{4.145990in}{1.127128in}}{\pgfqpoint{4.139148in}{1.124294in}}{\pgfqpoint{4.134105in}{1.119251in}}%
\pgfpathcurveto{\pgfqpoint{4.129061in}{1.114207in}}{\pgfqpoint{4.126227in}{1.107365in}}{\pgfqpoint{4.126227in}{1.100233in}}%
\pgfpathcurveto{\pgfqpoint{4.126227in}{1.093100in}}{\pgfqpoint{4.129061in}{1.086258in}}{\pgfqpoint{4.134105in}{1.081214in}}%
\pgfpathcurveto{\pgfqpoint{4.139148in}{1.076171in}}{\pgfqpoint{4.145990in}{1.073337in}}{\pgfqpoint{4.153123in}{1.073337in}}%
\pgfpathclose%
\pgfusepath{stroke,fill}%
\end{pgfscope}%
\begin{pgfscope}%
\pgfpathrectangle{\pgfqpoint{2.867647in}{0.500000in}}{\pgfqpoint{1.764706in}{1.700000in}}%
\pgfusepath{clip}%
\pgfsetbuttcap%
\pgfsetroundjoin%
\definecolor{currentfill}{rgb}{0.964306,0.663930,0.507747}%
\pgfsetfillcolor{currentfill}%
\pgfsetlinewidth{0.311001pt}%
\definecolor{currentstroke}{rgb}{1.000000,1.000000,1.000000}%
\pgfsetstrokecolor{currentstroke}%
\pgfsetdash{}{0pt}%
\pgfpathmoveto{\pgfqpoint{3.972920in}{0.979045in}}%
\pgfpathcurveto{\pgfqpoint{3.980053in}{0.979045in}}{\pgfqpoint{3.986894in}{0.981879in}}{\pgfqpoint{3.991938in}{0.986923in}}%
\pgfpathcurveto{\pgfqpoint{3.996982in}{0.991966in}}{\pgfqpoint{3.999815in}{0.998808in}}{\pgfqpoint{3.999815in}{1.005941in}}%
\pgfpathcurveto{\pgfqpoint{3.999815in}{1.013074in}}{\pgfqpoint{3.996982in}{1.019915in}}{\pgfqpoint{3.991938in}{1.024959in}}%
\pgfpathcurveto{\pgfqpoint{3.986894in}{1.030003in}}{\pgfqpoint{3.980053in}{1.032837in}}{\pgfqpoint{3.972920in}{1.032837in}}%
\pgfpathcurveto{\pgfqpoint{3.965787in}{1.032837in}}{\pgfqpoint{3.958945in}{1.030003in}}{\pgfqpoint{3.953902in}{1.024959in}}%
\pgfpathcurveto{\pgfqpoint{3.948858in}{1.019915in}}{\pgfqpoint{3.946024in}{1.013074in}}{\pgfqpoint{3.946024in}{1.005941in}}%
\pgfpathcurveto{\pgfqpoint{3.946024in}{0.998808in}}{\pgfqpoint{3.948858in}{0.991966in}}{\pgfqpoint{3.953902in}{0.986923in}}%
\pgfpathcurveto{\pgfqpoint{3.958945in}{0.981879in}}{\pgfqpoint{3.965787in}{0.979045in}}{\pgfqpoint{3.972920in}{0.979045in}}%
\pgfpathclose%
\pgfusepath{stroke,fill}%
\end{pgfscope}%
\begin{pgfscope}%
\pgfpathrectangle{\pgfqpoint{2.867647in}{0.500000in}}{\pgfqpoint{1.764706in}{1.700000in}}%
\pgfusepath{clip}%
\pgfsetbuttcap%
\pgfsetroundjoin%
\definecolor{currentfill}{rgb}{0.965592,0.726236,0.584384}%
\pgfsetfillcolor{currentfill}%
\pgfsetlinewidth{0.311001pt}%
\definecolor{currentstroke}{rgb}{1.000000,1.000000,1.000000}%
\pgfsetstrokecolor{currentstroke}%
\pgfsetdash{}{0pt}%
\pgfpathmoveto{\pgfqpoint{4.076525in}{1.266583in}}%
\pgfpathcurveto{\pgfqpoint{4.083658in}{1.266583in}}{\pgfqpoint{4.090499in}{1.269417in}}{\pgfqpoint{4.095543in}{1.274460in}}%
\pgfpathcurveto{\pgfqpoint{4.100587in}{1.279504in}}{\pgfqpoint{4.103421in}{1.286346in}}{\pgfqpoint{4.103421in}{1.293479in}}%
\pgfpathcurveto{\pgfqpoint{4.103421in}{1.300611in}}{\pgfqpoint{4.100587in}{1.307453in}}{\pgfqpoint{4.095543in}{1.312497in}}%
\pgfpathcurveto{\pgfqpoint{4.090499in}{1.317540in}}{\pgfqpoint{4.083658in}{1.320374in}}{\pgfqpoint{4.076525in}{1.320374in}}%
\pgfpathcurveto{\pgfqpoint{4.069392in}{1.320374in}}{\pgfqpoint{4.062550in}{1.317540in}}{\pgfqpoint{4.057507in}{1.312497in}}%
\pgfpathcurveto{\pgfqpoint{4.052463in}{1.307453in}}{\pgfqpoint{4.049629in}{1.300611in}}{\pgfqpoint{4.049629in}{1.293479in}}%
\pgfpathcurveto{\pgfqpoint{4.049629in}{1.286346in}}{\pgfqpoint{4.052463in}{1.279504in}}{\pgfqpoint{4.057507in}{1.274460in}}%
\pgfpathcurveto{\pgfqpoint{4.062550in}{1.269417in}}{\pgfqpoint{4.069392in}{1.266583in}}{\pgfqpoint{4.076525in}{1.266583in}}%
\pgfpathclose%
\pgfusepath{stroke,fill}%
\end{pgfscope}%
\begin{pgfscope}%
\pgfpathrectangle{\pgfqpoint{2.867647in}{0.500000in}}{\pgfqpoint{1.764706in}{1.700000in}}%
\pgfusepath{clip}%
\pgfsetbuttcap%
\pgfsetroundjoin%
\definecolor{currentfill}{rgb}{0.980678,0.914765,0.856766}%
\pgfsetfillcolor{currentfill}%
\pgfsetlinewidth{0.311001pt}%
\definecolor{currentstroke}{rgb}{1.000000,1.000000,1.000000}%
\pgfsetstrokecolor{currentstroke}%
\pgfsetdash{}{0pt}%
\pgfpathmoveto{\pgfqpoint{4.177275in}{1.152435in}}%
\pgfpathcurveto{\pgfqpoint{4.184408in}{1.152435in}}{\pgfqpoint{4.191249in}{1.155269in}}{\pgfqpoint{4.196293in}{1.160313in}}%
\pgfpathcurveto{\pgfqpoint{4.201337in}{1.165357in}}{\pgfqpoint{4.204171in}{1.172198in}}{\pgfqpoint{4.204171in}{1.179331in}}%
\pgfpathcurveto{\pgfqpoint{4.204171in}{1.186464in}}{\pgfqpoint{4.201337in}{1.193305in}}{\pgfqpoint{4.196293in}{1.198349in}}%
\pgfpathcurveto{\pgfqpoint{4.191249in}{1.203393in}}{\pgfqpoint{4.184408in}{1.206227in}}{\pgfqpoint{4.177275in}{1.206227in}}%
\pgfpathcurveto{\pgfqpoint{4.170142in}{1.206227in}}{\pgfqpoint{4.163301in}{1.203393in}}{\pgfqpoint{4.158257in}{1.198349in}}%
\pgfpathcurveto{\pgfqpoint{4.153213in}{1.193305in}}{\pgfqpoint{4.150379in}{1.186464in}}{\pgfqpoint{4.150379in}{1.179331in}}%
\pgfpathcurveto{\pgfqpoint{4.150379in}{1.172198in}}{\pgfqpoint{4.153213in}{1.165357in}}{\pgfqpoint{4.158257in}{1.160313in}}%
\pgfpathcurveto{\pgfqpoint{4.163301in}{1.155269in}}{\pgfqpoint{4.170142in}{1.152435in}}{\pgfqpoint{4.177275in}{1.152435in}}%
\pgfpathclose%
\pgfusepath{stroke,fill}%
\end{pgfscope}%
\begin{pgfscope}%
\pgfpathrectangle{\pgfqpoint{2.867647in}{0.500000in}}{\pgfqpoint{1.764706in}{1.700000in}}%
\pgfusepath{clip}%
\pgfsetbuttcap%
\pgfsetroundjoin%
\definecolor{currentfill}{rgb}{0.964679,0.682838,0.530002}%
\pgfsetfillcolor{currentfill}%
\pgfsetlinewidth{0.311001pt}%
\definecolor{currentstroke}{rgb}{1.000000,1.000000,1.000000}%
\pgfsetstrokecolor{currentstroke}%
\pgfsetdash{}{0pt}%
\pgfpathmoveto{\pgfqpoint{4.119709in}{1.755763in}}%
\pgfpathcurveto{\pgfqpoint{4.126842in}{1.755763in}}{\pgfqpoint{4.133684in}{1.758597in}}{\pgfqpoint{4.138727in}{1.763640in}}%
\pgfpathcurveto{\pgfqpoint{4.143771in}{1.768684in}}{\pgfqpoint{4.146605in}{1.775526in}}{\pgfqpoint{4.146605in}{1.782658in}}%
\pgfpathcurveto{\pgfqpoint{4.146605in}{1.789791in}}{\pgfqpoint{4.143771in}{1.796633in}}{\pgfqpoint{4.138727in}{1.801677in}}%
\pgfpathcurveto{\pgfqpoint{4.133684in}{1.806720in}}{\pgfqpoint{4.126842in}{1.809554in}}{\pgfqpoint{4.119709in}{1.809554in}}%
\pgfpathcurveto{\pgfqpoint{4.112576in}{1.809554in}}{\pgfqpoint{4.105735in}{1.806720in}}{\pgfqpoint{4.100691in}{1.801677in}}%
\pgfpathcurveto{\pgfqpoint{4.095647in}{1.796633in}}{\pgfqpoint{4.092813in}{1.789791in}}{\pgfqpoint{4.092813in}{1.782658in}}%
\pgfpathcurveto{\pgfqpoint{4.092813in}{1.775526in}}{\pgfqpoint{4.095647in}{1.768684in}}{\pgfqpoint{4.100691in}{1.763640in}}%
\pgfpathcurveto{\pgfqpoint{4.105735in}{1.758597in}}{\pgfqpoint{4.112576in}{1.755763in}}{\pgfqpoint{4.119709in}{1.755763in}}%
\pgfpathclose%
\pgfusepath{stroke,fill}%
\end{pgfscope}%
\begin{pgfscope}%
\pgfpathrectangle{\pgfqpoint{2.867647in}{0.500000in}}{\pgfqpoint{1.764706in}{1.700000in}}%
\pgfusepath{clip}%
\pgfsetbuttcap%
\pgfsetroundjoin%
\definecolor{currentfill}{rgb}{0.980678,0.914765,0.856766}%
\pgfsetfillcolor{currentfill}%
\pgfsetlinewidth{0.311001pt}%
\definecolor{currentstroke}{rgb}{1.000000,1.000000,1.000000}%
\pgfsetstrokecolor{currentstroke}%
\pgfsetdash{}{0pt}%
\pgfpathmoveto{\pgfqpoint{4.159043in}{1.510954in}}%
\pgfpathcurveto{\pgfqpoint{4.166176in}{1.510954in}}{\pgfqpoint{4.173018in}{1.513787in}}{\pgfqpoint{4.178062in}{1.518831in}}%
\pgfpathcurveto{\pgfqpoint{4.183105in}{1.523875in}}{\pgfqpoint{4.185939in}{1.530716in}}{\pgfqpoint{4.185939in}{1.537849in}}%
\pgfpathcurveto{\pgfqpoint{4.185939in}{1.544982in}}{\pgfqpoint{4.183105in}{1.551824in}}{\pgfqpoint{4.178062in}{1.556867in}}%
\pgfpathcurveto{\pgfqpoint{4.173018in}{1.561911in}}{\pgfqpoint{4.166176in}{1.564745in}}{\pgfqpoint{4.159043in}{1.564745in}}%
\pgfpathcurveto{\pgfqpoint{4.151911in}{1.564745in}}{\pgfqpoint{4.145069in}{1.561911in}}{\pgfqpoint{4.140025in}{1.556867in}}%
\pgfpathcurveto{\pgfqpoint{4.134982in}{1.551824in}}{\pgfqpoint{4.132148in}{1.544982in}}{\pgfqpoint{4.132148in}{1.537849in}}%
\pgfpathcurveto{\pgfqpoint{4.132148in}{1.530716in}}{\pgfqpoint{4.134982in}{1.523875in}}{\pgfqpoint{4.140025in}{1.518831in}}%
\pgfpathcurveto{\pgfqpoint{4.145069in}{1.513787in}}{\pgfqpoint{4.151911in}{1.510954in}}{\pgfqpoint{4.159043in}{1.510954in}}%
\pgfpathclose%
\pgfusepath{stroke,fill}%
\end{pgfscope}%
\begin{pgfscope}%
\pgfpathrectangle{\pgfqpoint{2.867647in}{0.500000in}}{\pgfqpoint{1.764706in}{1.700000in}}%
\pgfusepath{clip}%
\pgfsetbuttcap%
\pgfsetroundjoin%
\definecolor{currentfill}{rgb}{0.971694,0.833208,0.737161}%
\pgfsetfillcolor{currentfill}%
\pgfsetlinewidth{0.311001pt}%
\definecolor{currentstroke}{rgb}{1.000000,1.000000,1.000000}%
\pgfsetstrokecolor{currentstroke}%
\pgfsetdash{}{0pt}%
\pgfpathmoveto{\pgfqpoint{4.057441in}{1.023740in}}%
\pgfpathcurveto{\pgfqpoint{4.064573in}{1.023740in}}{\pgfqpoint{4.071415in}{1.026574in}}{\pgfqpoint{4.076459in}{1.031617in}}%
\pgfpathcurveto{\pgfqpoint{4.081502in}{1.036661in}}{\pgfqpoint{4.084336in}{1.043503in}}{\pgfqpoint{4.084336in}{1.050636in}}%
\pgfpathcurveto{\pgfqpoint{4.084336in}{1.057768in}}{\pgfqpoint{4.081502in}{1.064610in}}{\pgfqpoint{4.076459in}{1.069654in}}%
\pgfpathcurveto{\pgfqpoint{4.071415in}{1.074697in}}{\pgfqpoint{4.064573in}{1.077531in}}{\pgfqpoint{4.057441in}{1.077531in}}%
\pgfpathcurveto{\pgfqpoint{4.050308in}{1.077531in}}{\pgfqpoint{4.043466in}{1.074697in}}{\pgfqpoint{4.038422in}{1.069654in}}%
\pgfpathcurveto{\pgfqpoint{4.033379in}{1.064610in}}{\pgfqpoint{4.030545in}{1.057768in}}{\pgfqpoint{4.030545in}{1.050636in}}%
\pgfpathcurveto{\pgfqpoint{4.030545in}{1.043503in}}{\pgfqpoint{4.033379in}{1.036661in}}{\pgfqpoint{4.038422in}{1.031617in}}%
\pgfpathcurveto{\pgfqpoint{4.043466in}{1.026574in}}{\pgfqpoint{4.050308in}{1.023740in}}{\pgfqpoint{4.057441in}{1.023740in}}%
\pgfpathclose%
\pgfusepath{stroke,fill}%
\end{pgfscope}%
\begin{pgfscope}%
\pgfpathrectangle{\pgfqpoint{2.867647in}{0.500000in}}{\pgfqpoint{1.764706in}{1.700000in}}%
\pgfusepath{clip}%
\pgfsetbuttcap%
\pgfsetroundjoin%
\definecolor{currentfill}{rgb}{0.975018,0.868213,0.788710}%
\pgfsetfillcolor{currentfill}%
\pgfsetlinewidth{0.311001pt}%
\definecolor{currentstroke}{rgb}{1.000000,1.000000,1.000000}%
\pgfsetstrokecolor{currentstroke}%
\pgfsetdash{}{0pt}%
\pgfpathmoveto{\pgfqpoint{4.183319in}{1.055656in}}%
\pgfpathcurveto{\pgfqpoint{4.190451in}{1.055656in}}{\pgfqpoint{4.197293in}{1.058490in}}{\pgfqpoint{4.202337in}{1.063533in}}%
\pgfpathcurveto{\pgfqpoint{4.207380in}{1.068577in}}{\pgfqpoint{4.210214in}{1.075419in}}{\pgfqpoint{4.210214in}{1.082552in}}%
\pgfpathcurveto{\pgfqpoint{4.210214in}{1.089684in}}{\pgfqpoint{4.207380in}{1.096526in}}{\pgfqpoint{4.202337in}{1.101570in}}%
\pgfpathcurveto{\pgfqpoint{4.197293in}{1.106613in}}{\pgfqpoint{4.190451in}{1.109447in}}{\pgfqpoint{4.183319in}{1.109447in}}%
\pgfpathcurveto{\pgfqpoint{4.176186in}{1.109447in}}{\pgfqpoint{4.169344in}{1.106613in}}{\pgfqpoint{4.164300in}{1.101570in}}%
\pgfpathcurveto{\pgfqpoint{4.159257in}{1.096526in}}{\pgfqpoint{4.156423in}{1.089684in}}{\pgfqpoint{4.156423in}{1.082552in}}%
\pgfpathcurveto{\pgfqpoint{4.156423in}{1.075419in}}{\pgfqpoint{4.159257in}{1.068577in}}{\pgfqpoint{4.164300in}{1.063533in}}%
\pgfpathcurveto{\pgfqpoint{4.169344in}{1.058490in}}{\pgfqpoint{4.176186in}{1.055656in}}{\pgfqpoint{4.183319in}{1.055656in}}%
\pgfpathclose%
\pgfusepath{stroke,fill}%
\end{pgfscope}%
\begin{pgfscope}%
\pgfpathrectangle{\pgfqpoint{2.867647in}{0.500000in}}{\pgfqpoint{1.764706in}{1.700000in}}%
\pgfusepath{clip}%
\pgfsetbuttcap%
\pgfsetroundjoin%
\definecolor{currentfill}{rgb}{0.979891,0.908948,0.848279}%
\pgfsetfillcolor{currentfill}%
\pgfsetlinewidth{0.311001pt}%
\definecolor{currentstroke}{rgb}{1.000000,1.000000,1.000000}%
\pgfsetstrokecolor{currentstroke}%
\pgfsetdash{}{0pt}%
\pgfpathmoveto{\pgfqpoint{4.145051in}{1.553348in}}%
\pgfpathcurveto{\pgfqpoint{4.152184in}{1.553348in}}{\pgfqpoint{4.159026in}{1.556182in}}{\pgfqpoint{4.164069in}{1.561226in}}%
\pgfpathcurveto{\pgfqpoint{4.169113in}{1.566270in}}{\pgfqpoint{4.171947in}{1.573111in}}{\pgfqpoint{4.171947in}{1.580244in}}%
\pgfpathcurveto{\pgfqpoint{4.171947in}{1.587377in}}{\pgfqpoint{4.169113in}{1.594219in}}{\pgfqpoint{4.164069in}{1.599262in}}%
\pgfpathcurveto{\pgfqpoint{4.159026in}{1.604306in}}{\pgfqpoint{4.152184in}{1.607140in}}{\pgfqpoint{4.145051in}{1.607140in}}%
\pgfpathcurveto{\pgfqpoint{4.137918in}{1.607140in}}{\pgfqpoint{4.131077in}{1.604306in}}{\pgfqpoint{4.126033in}{1.599262in}}%
\pgfpathcurveto{\pgfqpoint{4.120989in}{1.594219in}}{\pgfqpoint{4.118156in}{1.587377in}}{\pgfqpoint{4.118156in}{1.580244in}}%
\pgfpathcurveto{\pgfqpoint{4.118156in}{1.573111in}}{\pgfqpoint{4.120989in}{1.566270in}}{\pgfqpoint{4.126033in}{1.561226in}}%
\pgfpathcurveto{\pgfqpoint{4.131077in}{1.556182in}}{\pgfqpoint{4.137918in}{1.553348in}}{\pgfqpoint{4.145051in}{1.553348in}}%
\pgfpathclose%
\pgfusepath{stroke,fill}%
\end{pgfscope}%
\begin{pgfscope}%
\pgfpathrectangle{\pgfqpoint{2.867647in}{0.500000in}}{\pgfqpoint{1.764706in}{1.700000in}}%
\pgfusepath{clip}%
\pgfsetbuttcap%
\pgfsetroundjoin%
\definecolor{currentfill}{rgb}{0.979124,0.903132,0.839793}%
\pgfsetfillcolor{currentfill}%
\pgfsetlinewidth{0.311001pt}%
\definecolor{currentstroke}{rgb}{1.000000,1.000000,1.000000}%
\pgfsetstrokecolor{currentstroke}%
\pgfsetdash{}{0pt}%
\pgfpathmoveto{\pgfqpoint{4.157276in}{1.082136in}}%
\pgfpathcurveto{\pgfqpoint{4.164409in}{1.082136in}}{\pgfqpoint{4.171251in}{1.084970in}}{\pgfqpoint{4.176295in}{1.090013in}}%
\pgfpathcurveto{\pgfqpoint{4.181338in}{1.095057in}}{\pgfqpoint{4.184172in}{1.101899in}}{\pgfqpoint{4.184172in}{1.109032in}}%
\pgfpathcurveto{\pgfqpoint{4.184172in}{1.116164in}}{\pgfqpoint{4.181338in}{1.123006in}}{\pgfqpoint{4.176295in}{1.128050in}}%
\pgfpathcurveto{\pgfqpoint{4.171251in}{1.133093in}}{\pgfqpoint{4.164409in}{1.135927in}}{\pgfqpoint{4.157276in}{1.135927in}}%
\pgfpathcurveto{\pgfqpoint{4.150144in}{1.135927in}}{\pgfqpoint{4.143302in}{1.133093in}}{\pgfqpoint{4.138258in}{1.128050in}}%
\pgfpathcurveto{\pgfqpoint{4.133215in}{1.123006in}}{\pgfqpoint{4.130381in}{1.116164in}}{\pgfqpoint{4.130381in}{1.109032in}}%
\pgfpathcurveto{\pgfqpoint{4.130381in}{1.101899in}}{\pgfqpoint{4.133215in}{1.095057in}}{\pgfqpoint{4.138258in}{1.090013in}}%
\pgfpathcurveto{\pgfqpoint{4.143302in}{1.084970in}}{\pgfqpoint{4.150144in}{1.082136in}}{\pgfqpoint{4.157276in}{1.082136in}}%
\pgfpathclose%
\pgfusepath{stroke,fill}%
\end{pgfscope}%
\begin{pgfscope}%
\pgfpathrectangle{\pgfqpoint{2.867647in}{0.500000in}}{\pgfqpoint{1.764706in}{1.700000in}}%
\pgfusepath{clip}%
\pgfsetbuttcap%
\pgfsetroundjoin%
\definecolor{currentfill}{rgb}{0.972201,0.839051,0.745789}%
\pgfsetfillcolor{currentfill}%
\pgfsetlinewidth{0.311001pt}%
\definecolor{currentstroke}{rgb}{1.000000,1.000000,1.000000}%
\pgfsetstrokecolor{currentstroke}%
\pgfsetdash{}{0pt}%
\pgfpathmoveto{\pgfqpoint{4.118432in}{1.322010in}}%
\pgfpathcurveto{\pgfqpoint{4.125564in}{1.322010in}}{\pgfqpoint{4.132406in}{1.324844in}}{\pgfqpoint{4.137450in}{1.329887in}}%
\pgfpathcurveto{\pgfqpoint{4.142493in}{1.334931in}}{\pgfqpoint{4.145327in}{1.341773in}}{\pgfqpoint{4.145327in}{1.348905in}}%
\pgfpathcurveto{\pgfqpoint{4.145327in}{1.356038in}}{\pgfqpoint{4.142493in}{1.362880in}}{\pgfqpoint{4.137450in}{1.367924in}}%
\pgfpathcurveto{\pgfqpoint{4.132406in}{1.372967in}}{\pgfqpoint{4.125564in}{1.375801in}}{\pgfqpoint{4.118432in}{1.375801in}}%
\pgfpathcurveto{\pgfqpoint{4.111299in}{1.375801in}}{\pgfqpoint{4.104457in}{1.372967in}}{\pgfqpoint{4.099414in}{1.367924in}}%
\pgfpathcurveto{\pgfqpoint{4.094370in}{1.362880in}}{\pgfqpoint{4.091536in}{1.356038in}}{\pgfqpoint{4.091536in}{1.348905in}}%
\pgfpathcurveto{\pgfqpoint{4.091536in}{1.341773in}}{\pgfqpoint{4.094370in}{1.334931in}}{\pgfqpoint{4.099414in}{1.329887in}}%
\pgfpathcurveto{\pgfqpoint{4.104457in}{1.324844in}}{\pgfqpoint{4.111299in}{1.322010in}}{\pgfqpoint{4.118432in}{1.322010in}}%
\pgfpathclose%
\pgfusepath{stroke,fill}%
\end{pgfscope}%
\begin{pgfscope}%
\pgfpathrectangle{\pgfqpoint{2.867647in}{0.500000in}}{\pgfqpoint{1.764706in}{1.700000in}}%
\pgfusepath{clip}%
\pgfsetbuttcap%
\pgfsetroundjoin%
\definecolor{currentfill}{rgb}{0.980678,0.914765,0.856766}%
\pgfsetfillcolor{currentfill}%
\pgfsetlinewidth{0.311001pt}%
\definecolor{currentstroke}{rgb}{1.000000,1.000000,1.000000}%
\pgfsetstrokecolor{currentstroke}%
\pgfsetdash{}{0pt}%
\pgfpathmoveto{\pgfqpoint{4.180862in}{1.378420in}}%
\pgfpathcurveto{\pgfqpoint{4.187995in}{1.378420in}}{\pgfqpoint{4.194837in}{1.381254in}}{\pgfqpoint{4.199881in}{1.386298in}}%
\pgfpathcurveto{\pgfqpoint{4.204924in}{1.391341in}}{\pgfqpoint{4.207758in}{1.398183in}}{\pgfqpoint{4.207758in}{1.405316in}}%
\pgfpathcurveto{\pgfqpoint{4.207758in}{1.412449in}}{\pgfqpoint{4.204924in}{1.419290in}}{\pgfqpoint{4.199881in}{1.424334in}}%
\pgfpathcurveto{\pgfqpoint{4.194837in}{1.429378in}}{\pgfqpoint{4.187995in}{1.432211in}}{\pgfqpoint{4.180862in}{1.432211in}}%
\pgfpathcurveto{\pgfqpoint{4.173730in}{1.432211in}}{\pgfqpoint{4.166888in}{1.429378in}}{\pgfqpoint{4.161844in}{1.424334in}}%
\pgfpathcurveto{\pgfqpoint{4.156801in}{1.419290in}}{\pgfqpoint{4.153967in}{1.412449in}}{\pgfqpoint{4.153967in}{1.405316in}}%
\pgfpathcurveto{\pgfqpoint{4.153967in}{1.398183in}}{\pgfqpoint{4.156801in}{1.391341in}}{\pgfqpoint{4.161844in}{1.386298in}}%
\pgfpathcurveto{\pgfqpoint{4.166888in}{1.381254in}}{\pgfqpoint{4.173730in}{1.378420in}}{\pgfqpoint{4.180862in}{1.378420in}}%
\pgfpathclose%
\pgfusepath{stroke,fill}%
\end{pgfscope}%
\begin{pgfscope}%
\pgfpathrectangle{\pgfqpoint{2.867647in}{0.500000in}}{\pgfqpoint{1.764706in}{1.700000in}}%
\pgfusepath{clip}%
\pgfsetbuttcap%
\pgfsetroundjoin%
\definecolor{currentfill}{rgb}{0.973271,0.850724,0.762998}%
\pgfsetfillcolor{currentfill}%
\pgfsetlinewidth{0.311001pt}%
\definecolor{currentstroke}{rgb}{1.000000,1.000000,1.000000}%
\pgfsetstrokecolor{currentstroke}%
\pgfsetdash{}{0pt}%
\pgfpathmoveto{\pgfqpoint{4.159080in}{1.013145in}}%
\pgfpathcurveto{\pgfqpoint{4.166212in}{1.013145in}}{\pgfqpoint{4.173054in}{1.015979in}}{\pgfqpoint{4.178098in}{1.021022in}}%
\pgfpathcurveto{\pgfqpoint{4.183141in}{1.026066in}}{\pgfqpoint{4.185975in}{1.032908in}}{\pgfqpoint{4.185975in}{1.040041in}}%
\pgfpathcurveto{\pgfqpoint{4.185975in}{1.047173in}}{\pgfqpoint{4.183141in}{1.054015in}}{\pgfqpoint{4.178098in}{1.059059in}}%
\pgfpathcurveto{\pgfqpoint{4.173054in}{1.064102in}}{\pgfqpoint{4.166212in}{1.066936in}}{\pgfqpoint{4.159080in}{1.066936in}}%
\pgfpathcurveto{\pgfqpoint{4.151947in}{1.066936in}}{\pgfqpoint{4.145105in}{1.064102in}}{\pgfqpoint{4.140061in}{1.059059in}}%
\pgfpathcurveto{\pgfqpoint{4.135018in}{1.054015in}}{\pgfqpoint{4.132184in}{1.047173in}}{\pgfqpoint{4.132184in}{1.040041in}}%
\pgfpathcurveto{\pgfqpoint{4.132184in}{1.032908in}}{\pgfqpoint{4.135018in}{1.026066in}}{\pgfqpoint{4.140061in}{1.021022in}}%
\pgfpathcurveto{\pgfqpoint{4.145105in}{1.015979in}}{\pgfqpoint{4.151947in}{1.013145in}}{\pgfqpoint{4.159080in}{1.013145in}}%
\pgfpathclose%
\pgfusepath{stroke,fill}%
\end{pgfscope}%
\begin{pgfscope}%
\pgfpathrectangle{\pgfqpoint{2.867647in}{0.500000in}}{\pgfqpoint{1.764706in}{1.700000in}}%
\pgfusepath{clip}%
\pgfsetbuttcap%
\pgfsetroundjoin%
\definecolor{currentfill}{rgb}{0.883342,0.198306,0.260142}%
\pgfsetfillcolor{currentfill}%
\pgfsetlinewidth{0.311001pt}%
\definecolor{currentstroke}{rgb}{1.000000,1.000000,1.000000}%
\pgfsetstrokecolor{currentstroke}%
\pgfsetdash{}{0pt}%
\pgfpathmoveto{\pgfqpoint{3.878473in}{0.799224in}}%
\pgfpathcurveto{\pgfqpoint{3.885606in}{0.799224in}}{\pgfqpoint{3.892448in}{0.802058in}}{\pgfqpoint{3.897491in}{0.807101in}}%
\pgfpathcurveto{\pgfqpoint{3.902535in}{0.812145in}}{\pgfqpoint{3.905369in}{0.818987in}}{\pgfqpoint{3.905369in}{0.826119in}}%
\pgfpathcurveto{\pgfqpoint{3.905369in}{0.833252in}}{\pgfqpoint{3.902535in}{0.840094in}}{\pgfqpoint{3.897491in}{0.845138in}}%
\pgfpathcurveto{\pgfqpoint{3.892448in}{0.850181in}}{\pgfqpoint{3.885606in}{0.853015in}}{\pgfqpoint{3.878473in}{0.853015in}}%
\pgfpathcurveto{\pgfqpoint{3.871340in}{0.853015in}}{\pgfqpoint{3.864499in}{0.850181in}}{\pgfqpoint{3.859455in}{0.845138in}}%
\pgfpathcurveto{\pgfqpoint{3.854411in}{0.840094in}}{\pgfqpoint{3.851577in}{0.833252in}}{\pgfqpoint{3.851577in}{0.826119in}}%
\pgfpathcurveto{\pgfqpoint{3.851577in}{0.818987in}}{\pgfqpoint{3.854411in}{0.812145in}}{\pgfqpoint{3.859455in}{0.807101in}}%
\pgfpathcurveto{\pgfqpoint{3.864499in}{0.802058in}}{\pgfqpoint{3.871340in}{0.799224in}}{\pgfqpoint{3.878473in}{0.799224in}}%
\pgfpathclose%
\pgfusepath{stroke,fill}%
\end{pgfscope}%
\begin{pgfscope}%
\pgfpathrectangle{\pgfqpoint{2.867647in}{0.500000in}}{\pgfqpoint{1.764706in}{1.700000in}}%
\pgfusepath{clip}%
\pgfsetbuttcap%
\pgfsetroundjoin%
\definecolor{currentfill}{rgb}{0.969359,0.803954,0.693832}%
\pgfsetfillcolor{currentfill}%
\pgfsetlinewidth{0.311001pt}%
\definecolor{currentstroke}{rgb}{1.000000,1.000000,1.000000}%
\pgfsetstrokecolor{currentstroke}%
\pgfsetdash{}{0pt}%
\pgfpathmoveto{\pgfqpoint{4.119399in}{0.951664in}}%
\pgfpathcurveto{\pgfqpoint{4.126532in}{0.951664in}}{\pgfqpoint{4.133374in}{0.954498in}}{\pgfqpoint{4.138418in}{0.959541in}}%
\pgfpathcurveto{\pgfqpoint{4.143461in}{0.964585in}}{\pgfqpoint{4.146295in}{0.971427in}}{\pgfqpoint{4.146295in}{0.978559in}}%
\pgfpathcurveto{\pgfqpoint{4.146295in}{0.985692in}}{\pgfqpoint{4.143461in}{0.992534in}}{\pgfqpoint{4.138418in}{0.997577in}}%
\pgfpathcurveto{\pgfqpoint{4.133374in}{1.002621in}}{\pgfqpoint{4.126532in}{1.005455in}}{\pgfqpoint{4.119399in}{1.005455in}}%
\pgfpathcurveto{\pgfqpoint{4.112267in}{1.005455in}}{\pgfqpoint{4.105425in}{1.002621in}}{\pgfqpoint{4.100381in}{0.997577in}}%
\pgfpathcurveto{\pgfqpoint{4.095338in}{0.992534in}}{\pgfqpoint{4.092504in}{0.985692in}}{\pgfqpoint{4.092504in}{0.978559in}}%
\pgfpathcurveto{\pgfqpoint{4.092504in}{0.971427in}}{\pgfqpoint{4.095338in}{0.964585in}}{\pgfqpoint{4.100381in}{0.959541in}}%
\pgfpathcurveto{\pgfqpoint{4.105425in}{0.954498in}}{\pgfqpoint{4.112267in}{0.951664in}}{\pgfqpoint{4.119399in}{0.951664in}}%
\pgfpathclose%
\pgfusepath{stroke,fill}%
\end{pgfscope}%
\begin{pgfscope}%
\pgfpathrectangle{\pgfqpoint{2.867647in}{0.500000in}}{\pgfqpoint{1.764706in}{1.700000in}}%
\pgfusepath{clip}%
\pgfsetbuttcap%
\pgfsetroundjoin%
\definecolor{currentfill}{rgb}{0.980678,0.914765,0.856766}%
\pgfsetfillcolor{currentfill}%
\pgfsetlinewidth{0.311001pt}%
\definecolor{currentstroke}{rgb}{1.000000,1.000000,1.000000}%
\pgfsetstrokecolor{currentstroke}%
\pgfsetdash{}{0pt}%
\pgfpathmoveto{\pgfqpoint{4.176311in}{1.326808in}}%
\pgfpathcurveto{\pgfqpoint{4.183444in}{1.326808in}}{\pgfqpoint{4.190286in}{1.329642in}}{\pgfqpoint{4.195329in}{1.334685in}}%
\pgfpathcurveto{\pgfqpoint{4.200373in}{1.339729in}}{\pgfqpoint{4.203207in}{1.346571in}}{\pgfqpoint{4.203207in}{1.353704in}}%
\pgfpathcurveto{\pgfqpoint{4.203207in}{1.360836in}}{\pgfqpoint{4.200373in}{1.367678in}}{\pgfqpoint{4.195329in}{1.372722in}}%
\pgfpathcurveto{\pgfqpoint{4.190286in}{1.377765in}}{\pgfqpoint{4.183444in}{1.380599in}}{\pgfqpoint{4.176311in}{1.380599in}}%
\pgfpathcurveto{\pgfqpoint{4.169178in}{1.380599in}}{\pgfqpoint{4.162337in}{1.377765in}}{\pgfqpoint{4.157293in}{1.372722in}}%
\pgfpathcurveto{\pgfqpoint{4.152249in}{1.367678in}}{\pgfqpoint{4.149415in}{1.360836in}}{\pgfqpoint{4.149415in}{1.353704in}}%
\pgfpathcurveto{\pgfqpoint{4.149415in}{1.346571in}}{\pgfqpoint{4.152249in}{1.339729in}}{\pgfqpoint{4.157293in}{1.334685in}}%
\pgfpathcurveto{\pgfqpoint{4.162337in}{1.329642in}}{\pgfqpoint{4.169178in}{1.326808in}}{\pgfqpoint{4.176311in}{1.326808in}}%
\pgfpathclose%
\pgfusepath{stroke,fill}%
\end{pgfscope}%
\begin{pgfscope}%
\pgfpathrectangle{\pgfqpoint{2.867647in}{0.500000in}}{\pgfqpoint{1.764706in}{1.700000in}}%
\pgfusepath{clip}%
\pgfsetbuttcap%
\pgfsetroundjoin%
\definecolor{currentfill}{rgb}{0.976961,0.885681,0.814303}%
\pgfsetfillcolor{currentfill}%
\pgfsetlinewidth{0.311001pt}%
\definecolor{currentstroke}{rgb}{1.000000,1.000000,1.000000}%
\pgfsetstrokecolor{currentstroke}%
\pgfsetdash{}{0pt}%
\pgfpathmoveto{\pgfqpoint{4.240186in}{1.335114in}}%
\pgfpathcurveto{\pgfqpoint{4.247319in}{1.335114in}}{\pgfqpoint{4.254160in}{1.337947in}}{\pgfqpoint{4.259204in}{1.342991in}}%
\pgfpathcurveto{\pgfqpoint{4.264248in}{1.348035in}}{\pgfqpoint{4.267082in}{1.354876in}}{\pgfqpoint{4.267082in}{1.362009in}}%
\pgfpathcurveto{\pgfqpoint{4.267082in}{1.369142in}}{\pgfqpoint{4.264248in}{1.375984in}}{\pgfqpoint{4.259204in}{1.381027in}}%
\pgfpathcurveto{\pgfqpoint{4.254160in}{1.386071in}}{\pgfqpoint{4.247319in}{1.388905in}}{\pgfqpoint{4.240186in}{1.388905in}}%
\pgfpathcurveto{\pgfqpoint{4.233053in}{1.388905in}}{\pgfqpoint{4.226211in}{1.386071in}}{\pgfqpoint{4.221168in}{1.381027in}}%
\pgfpathcurveto{\pgfqpoint{4.216124in}{1.375984in}}{\pgfqpoint{4.213290in}{1.369142in}}{\pgfqpoint{4.213290in}{1.362009in}}%
\pgfpathcurveto{\pgfqpoint{4.213290in}{1.354876in}}{\pgfqpoint{4.216124in}{1.348035in}}{\pgfqpoint{4.221168in}{1.342991in}}%
\pgfpathcurveto{\pgfqpoint{4.226211in}{1.337947in}}{\pgfqpoint{4.233053in}{1.335114in}}{\pgfqpoint{4.240186in}{1.335114in}}%
\pgfpathclose%
\pgfusepath{stroke,fill}%
\end{pgfscope}%
\begin{pgfscope}%
\pgfpathrectangle{\pgfqpoint{2.867647in}{0.500000in}}{\pgfqpoint{1.764706in}{1.700000in}}%
\pgfusepath{clip}%
\pgfsetbuttcap%
\pgfsetroundjoin%
\definecolor{currentfill}{rgb}{0.978376,0.897317,0.831308}%
\pgfsetfillcolor{currentfill}%
\pgfsetlinewidth{0.311001pt}%
\definecolor{currentstroke}{rgb}{1.000000,1.000000,1.000000}%
\pgfsetstrokecolor{currentstroke}%
\pgfsetdash{}{0pt}%
\pgfpathmoveto{\pgfqpoint{4.111183in}{1.557541in}}%
\pgfpathcurveto{\pgfqpoint{4.118315in}{1.557541in}}{\pgfqpoint{4.125157in}{1.560375in}}{\pgfqpoint{4.130201in}{1.565418in}}%
\pgfpathcurveto{\pgfqpoint{4.135244in}{1.570462in}}{\pgfqpoint{4.138078in}{1.577304in}}{\pgfqpoint{4.138078in}{1.584436in}}%
\pgfpathcurveto{\pgfqpoint{4.138078in}{1.591569in}}{\pgfqpoint{4.135244in}{1.598411in}}{\pgfqpoint{4.130201in}{1.603455in}}%
\pgfpathcurveto{\pgfqpoint{4.125157in}{1.608498in}}{\pgfqpoint{4.118315in}{1.611332in}}{\pgfqpoint{4.111183in}{1.611332in}}%
\pgfpathcurveto{\pgfqpoint{4.104050in}{1.611332in}}{\pgfqpoint{4.097208in}{1.608498in}}{\pgfqpoint{4.092165in}{1.603455in}}%
\pgfpathcurveto{\pgfqpoint{4.087121in}{1.598411in}}{\pgfqpoint{4.084287in}{1.591569in}}{\pgfqpoint{4.084287in}{1.584436in}}%
\pgfpathcurveto{\pgfqpoint{4.084287in}{1.577304in}}{\pgfqpoint{4.087121in}{1.570462in}}{\pgfqpoint{4.092165in}{1.565418in}}%
\pgfpathcurveto{\pgfqpoint{4.097208in}{1.560375in}}{\pgfqpoint{4.104050in}{1.557541in}}{\pgfqpoint{4.111183in}{1.557541in}}%
\pgfpathclose%
\pgfusepath{stroke,fill}%
\end{pgfscope}%
\begin{pgfscope}%
\pgfpathrectangle{\pgfqpoint{2.867647in}{0.500000in}}{\pgfqpoint{1.764706in}{1.700000in}}%
\pgfusepath{clip}%
\pgfsetbuttcap%
\pgfsetroundjoin%
\definecolor{currentfill}{rgb}{0.965592,0.726236,0.584384}%
\pgfsetfillcolor{currentfill}%
\pgfsetlinewidth{0.311001pt}%
\definecolor{currentstroke}{rgb}{1.000000,1.000000,1.000000}%
\pgfsetstrokecolor{currentstroke}%
\pgfsetdash{}{0pt}%
\pgfpathmoveto{\pgfqpoint{4.027307in}{1.544067in}}%
\pgfpathcurveto{\pgfqpoint{4.034440in}{1.544067in}}{\pgfqpoint{4.041282in}{1.546901in}}{\pgfqpoint{4.046325in}{1.551945in}}%
\pgfpathcurveto{\pgfqpoint{4.051369in}{1.556988in}}{\pgfqpoint{4.054203in}{1.563830in}}{\pgfqpoint{4.054203in}{1.570963in}}%
\pgfpathcurveto{\pgfqpoint{4.054203in}{1.578095in}}{\pgfqpoint{4.051369in}{1.584937in}}{\pgfqpoint{4.046325in}{1.589981in}}%
\pgfpathcurveto{\pgfqpoint{4.041282in}{1.595024in}}{\pgfqpoint{4.034440in}{1.597858in}}{\pgfqpoint{4.027307in}{1.597858in}}%
\pgfpathcurveto{\pgfqpoint{4.020174in}{1.597858in}}{\pgfqpoint{4.013333in}{1.595024in}}{\pgfqpoint{4.008289in}{1.589981in}}%
\pgfpathcurveto{\pgfqpoint{4.003245in}{1.584937in}}{\pgfqpoint{4.000412in}{1.578095in}}{\pgfqpoint{4.000412in}{1.570963in}}%
\pgfpathcurveto{\pgfqpoint{4.000412in}{1.563830in}}{\pgfqpoint{4.003245in}{1.556988in}}{\pgfqpoint{4.008289in}{1.551945in}}%
\pgfpathcurveto{\pgfqpoint{4.013333in}{1.546901in}}{\pgfqpoint{4.020174in}{1.544067in}}{\pgfqpoint{4.027307in}{1.544067in}}%
\pgfpathclose%
\pgfusepath{stroke,fill}%
\end{pgfscope}%
\begin{pgfscope}%
\pgfpathrectangle{\pgfqpoint{2.867647in}{0.500000in}}{\pgfqpoint{1.764706in}{1.700000in}}%
\pgfusepath{clip}%
\pgfsetbuttcap%
\pgfsetroundjoin%
\definecolor{currentfill}{rgb}{0.976961,0.885681,0.814303}%
\pgfsetfillcolor{currentfill}%
\pgfsetlinewidth{0.311001pt}%
\definecolor{currentstroke}{rgb}{1.000000,1.000000,1.000000}%
\pgfsetstrokecolor{currentstroke}%
\pgfsetdash{}{0pt}%
\pgfpathmoveto{\pgfqpoint{4.130504in}{1.451262in}}%
\pgfpathcurveto{\pgfqpoint{4.137637in}{1.451262in}}{\pgfqpoint{4.144478in}{1.454096in}}{\pgfqpoint{4.149522in}{1.459139in}}%
\pgfpathcurveto{\pgfqpoint{4.154566in}{1.464183in}}{\pgfqpoint{4.157399in}{1.471025in}}{\pgfqpoint{4.157399in}{1.478158in}}%
\pgfpathcurveto{\pgfqpoint{4.157399in}{1.485290in}}{\pgfqpoint{4.154566in}{1.492132in}}{\pgfqpoint{4.149522in}{1.497176in}}%
\pgfpathcurveto{\pgfqpoint{4.144478in}{1.502219in}}{\pgfqpoint{4.137637in}{1.505053in}}{\pgfqpoint{4.130504in}{1.505053in}}%
\pgfpathcurveto{\pgfqpoint{4.123371in}{1.505053in}}{\pgfqpoint{4.116529in}{1.502219in}}{\pgfqpoint{4.111486in}{1.497176in}}%
\pgfpathcurveto{\pgfqpoint{4.106442in}{1.492132in}}{\pgfqpoint{4.103608in}{1.485290in}}{\pgfqpoint{4.103608in}{1.478158in}}%
\pgfpathcurveto{\pgfqpoint{4.103608in}{1.471025in}}{\pgfqpoint{4.106442in}{1.464183in}}{\pgfqpoint{4.111486in}{1.459139in}}%
\pgfpathcurveto{\pgfqpoint{4.116529in}{1.454096in}}{\pgfqpoint{4.123371in}{1.451262in}}{\pgfqpoint{4.130504in}{1.451262in}}%
\pgfpathclose%
\pgfusepath{stroke,fill}%
\end{pgfscope}%
\begin{pgfscope}%
\pgfpathrectangle{\pgfqpoint{2.867647in}{0.500000in}}{\pgfqpoint{1.764706in}{1.700000in}}%
\pgfusepath{clip}%
\pgfsetbuttcap%
\pgfsetroundjoin%
\definecolor{currentfill}{rgb}{0.964433,0.670254,0.515093}%
\pgfsetfillcolor{currentfill}%
\pgfsetlinewidth{0.311001pt}%
\definecolor{currentstroke}{rgb}{1.000000,1.000000,1.000000}%
\pgfsetstrokecolor{currentstroke}%
\pgfsetdash{}{0pt}%
\pgfpathmoveto{\pgfqpoint{4.255954in}{1.040576in}}%
\pgfpathcurveto{\pgfqpoint{4.263086in}{1.040576in}}{\pgfqpoint{4.269928in}{1.043410in}}{\pgfqpoint{4.274972in}{1.048453in}}%
\pgfpathcurveto{\pgfqpoint{4.280015in}{1.053497in}}{\pgfqpoint{4.282849in}{1.060339in}}{\pgfqpoint{4.282849in}{1.067471in}}%
\pgfpathcurveto{\pgfqpoint{4.282849in}{1.074604in}}{\pgfqpoint{4.280015in}{1.081446in}}{\pgfqpoint{4.274972in}{1.086490in}}%
\pgfpathcurveto{\pgfqpoint{4.269928in}{1.091533in}}{\pgfqpoint{4.263086in}{1.094367in}}{\pgfqpoint{4.255954in}{1.094367in}}%
\pgfpathcurveto{\pgfqpoint{4.248821in}{1.094367in}}{\pgfqpoint{4.241979in}{1.091533in}}{\pgfqpoint{4.236935in}{1.086490in}}%
\pgfpathcurveto{\pgfqpoint{4.231892in}{1.081446in}}{\pgfqpoint{4.229058in}{1.074604in}}{\pgfqpoint{4.229058in}{1.067471in}}%
\pgfpathcurveto{\pgfqpoint{4.229058in}{1.060339in}}{\pgfqpoint{4.231892in}{1.053497in}}{\pgfqpoint{4.236935in}{1.048453in}}%
\pgfpathcurveto{\pgfqpoint{4.241979in}{1.043410in}}{\pgfqpoint{4.248821in}{1.040576in}}{\pgfqpoint{4.255954in}{1.040576in}}%
\pgfpathclose%
\pgfusepath{stroke,fill}%
\end{pgfscope}%
\begin{pgfscope}%
\pgfpathrectangle{\pgfqpoint{2.867647in}{0.500000in}}{\pgfqpoint{1.764706in}{1.700000in}}%
\pgfusepath{clip}%
\pgfsetbuttcap%
\pgfsetroundjoin%
\definecolor{currentfill}{rgb}{0.953816,0.463738,0.317699}%
\pgfsetfillcolor{currentfill}%
\pgfsetlinewidth{0.311001pt}%
\definecolor{currentstroke}{rgb}{1.000000,1.000000,1.000000}%
\pgfsetstrokecolor{currentstroke}%
\pgfsetdash{}{0pt}%
\pgfpathmoveto{\pgfqpoint{4.034638in}{1.846216in}}%
\pgfpathcurveto{\pgfqpoint{4.041771in}{1.846216in}}{\pgfqpoint{4.048613in}{1.849050in}}{\pgfqpoint{4.053657in}{1.854094in}}%
\pgfpathcurveto{\pgfqpoint{4.058700in}{1.859138in}}{\pgfqpoint{4.061534in}{1.865979in}}{\pgfqpoint{4.061534in}{1.873112in}}%
\pgfpathcurveto{\pgfqpoint{4.061534in}{1.880245in}}{\pgfqpoint{4.058700in}{1.887087in}}{\pgfqpoint{4.053657in}{1.892130in}}%
\pgfpathcurveto{\pgfqpoint{4.048613in}{1.897174in}}{\pgfqpoint{4.041771in}{1.900008in}}{\pgfqpoint{4.034638in}{1.900008in}}%
\pgfpathcurveto{\pgfqpoint{4.027506in}{1.900008in}}{\pgfqpoint{4.020664in}{1.897174in}}{\pgfqpoint{4.015620in}{1.892130in}}%
\pgfpathcurveto{\pgfqpoint{4.010577in}{1.887087in}}{\pgfqpoint{4.007743in}{1.880245in}}{\pgfqpoint{4.007743in}{1.873112in}}%
\pgfpathcurveto{\pgfqpoint{4.007743in}{1.865979in}}{\pgfqpoint{4.010577in}{1.859138in}}{\pgfqpoint{4.015620in}{1.854094in}}%
\pgfpathcurveto{\pgfqpoint{4.020664in}{1.849050in}}{\pgfqpoint{4.027506in}{1.846216in}}{\pgfqpoint{4.034638in}{1.846216in}}%
\pgfpathclose%
\pgfusepath{stroke,fill}%
\end{pgfscope}%
\begin{pgfscope}%
\pgfpathrectangle{\pgfqpoint{2.867647in}{0.500000in}}{\pgfqpoint{1.764706in}{1.700000in}}%
\pgfusepath{clip}%
\pgfsetbuttcap%
\pgfsetroundjoin%
\definecolor{currentfill}{rgb}{0.976287,0.879862,0.805788}%
\pgfsetfillcolor{currentfill}%
\pgfsetlinewidth{0.311001pt}%
\definecolor{currentstroke}{rgb}{1.000000,1.000000,1.000000}%
\pgfsetstrokecolor{currentstroke}%
\pgfsetdash{}{0pt}%
\pgfpathmoveto{\pgfqpoint{4.220765in}{1.463308in}}%
\pgfpathcurveto{\pgfqpoint{4.227898in}{1.463308in}}{\pgfqpoint{4.234740in}{1.466142in}}{\pgfqpoint{4.239784in}{1.471186in}}%
\pgfpathcurveto{\pgfqpoint{4.244827in}{1.476229in}}{\pgfqpoint{4.247661in}{1.483071in}}{\pgfqpoint{4.247661in}{1.490204in}}%
\pgfpathcurveto{\pgfqpoint{4.247661in}{1.497337in}}{\pgfqpoint{4.244827in}{1.504178in}}{\pgfqpoint{4.239784in}{1.509222in}}%
\pgfpathcurveto{\pgfqpoint{4.234740in}{1.514266in}}{\pgfqpoint{4.227898in}{1.517099in}}{\pgfqpoint{4.220765in}{1.517099in}}%
\pgfpathcurveto{\pgfqpoint{4.213633in}{1.517099in}}{\pgfqpoint{4.206791in}{1.514266in}}{\pgfqpoint{4.201747in}{1.509222in}}%
\pgfpathcurveto{\pgfqpoint{4.196704in}{1.504178in}}{\pgfqpoint{4.193870in}{1.497337in}}{\pgfqpoint{4.193870in}{1.490204in}}%
\pgfpathcurveto{\pgfqpoint{4.193870in}{1.483071in}}{\pgfqpoint{4.196704in}{1.476229in}}{\pgfqpoint{4.201747in}{1.471186in}}%
\pgfpathcurveto{\pgfqpoint{4.206791in}{1.466142in}}{\pgfqpoint{4.213633in}{1.463308in}}{\pgfqpoint{4.220765in}{1.463308in}}%
\pgfpathclose%
\pgfusepath{stroke,fill}%
\end{pgfscope}%
\begin{pgfscope}%
\pgfpathrectangle{\pgfqpoint{2.867647in}{0.500000in}}{\pgfqpoint{1.764706in}{1.700000in}}%
\pgfusepath{clip}%
\pgfsetbuttcap%
\pgfsetroundjoin%
\definecolor{currentfill}{rgb}{0.970718,0.821518,0.719872}%
\pgfsetfillcolor{currentfill}%
\pgfsetlinewidth{0.311001pt}%
\definecolor{currentstroke}{rgb}{1.000000,1.000000,1.000000}%
\pgfsetstrokecolor{currentstroke}%
\pgfsetdash{}{0pt}%
\pgfpathmoveto{\pgfqpoint{4.095294in}{1.199921in}}%
\pgfpathcurveto{\pgfqpoint{4.102427in}{1.199921in}}{\pgfqpoint{4.109268in}{1.202755in}}{\pgfqpoint{4.114312in}{1.207799in}}%
\pgfpathcurveto{\pgfqpoint{4.119356in}{1.212843in}}{\pgfqpoint{4.122190in}{1.219684in}}{\pgfqpoint{4.122190in}{1.226817in}}%
\pgfpathcurveto{\pgfqpoint{4.122190in}{1.233950in}}{\pgfqpoint{4.119356in}{1.240792in}}{\pgfqpoint{4.114312in}{1.245835in}}%
\pgfpathcurveto{\pgfqpoint{4.109268in}{1.250879in}}{\pgfqpoint{4.102427in}{1.253713in}}{\pgfqpoint{4.095294in}{1.253713in}}%
\pgfpathcurveto{\pgfqpoint{4.088161in}{1.253713in}}{\pgfqpoint{4.081320in}{1.250879in}}{\pgfqpoint{4.076276in}{1.245835in}}%
\pgfpathcurveto{\pgfqpoint{4.071232in}{1.240792in}}{\pgfqpoint{4.068398in}{1.233950in}}{\pgfqpoint{4.068398in}{1.226817in}}%
\pgfpathcurveto{\pgfqpoint{4.068398in}{1.219684in}}{\pgfqpoint{4.071232in}{1.212843in}}{\pgfqpoint{4.076276in}{1.207799in}}%
\pgfpathcurveto{\pgfqpoint{4.081320in}{1.202755in}}{\pgfqpoint{4.088161in}{1.199921in}}{\pgfqpoint{4.095294in}{1.199921in}}%
\pgfpathclose%
\pgfusepath{stroke,fill}%
\end{pgfscope}%
\begin{pgfscope}%
\pgfpathrectangle{\pgfqpoint{2.867647in}{0.500000in}}{\pgfqpoint{1.764706in}{1.700000in}}%
\pgfusepath{clip}%
\pgfsetbuttcap%
\pgfsetroundjoin%
\definecolor{currentfill}{rgb}{0.965302,0.713942,0.568499}%
\pgfsetfillcolor{currentfill}%
\pgfsetlinewidth{0.311001pt}%
\definecolor{currentstroke}{rgb}{1.000000,1.000000,1.000000}%
\pgfsetstrokecolor{currentstroke}%
\pgfsetdash{}{0pt}%
\pgfpathmoveto{\pgfqpoint{4.141263in}{0.921521in}}%
\pgfpathcurveto{\pgfqpoint{4.148396in}{0.921521in}}{\pgfqpoint{4.155238in}{0.924355in}}{\pgfqpoint{4.160281in}{0.929398in}}%
\pgfpathcurveto{\pgfqpoint{4.165325in}{0.934442in}}{\pgfqpoint{4.168159in}{0.941284in}}{\pgfqpoint{4.168159in}{0.948416in}}%
\pgfpathcurveto{\pgfqpoint{4.168159in}{0.955549in}}{\pgfqpoint{4.165325in}{0.962391in}}{\pgfqpoint{4.160281in}{0.967435in}}%
\pgfpathcurveto{\pgfqpoint{4.155238in}{0.972478in}}{\pgfqpoint{4.148396in}{0.975312in}}{\pgfqpoint{4.141263in}{0.975312in}}%
\pgfpathcurveto{\pgfqpoint{4.134130in}{0.975312in}}{\pgfqpoint{4.127289in}{0.972478in}}{\pgfqpoint{4.122245in}{0.967435in}}%
\pgfpathcurveto{\pgfqpoint{4.117201in}{0.962391in}}{\pgfqpoint{4.114368in}{0.955549in}}{\pgfqpoint{4.114368in}{0.948416in}}%
\pgfpathcurveto{\pgfqpoint{4.114368in}{0.941284in}}{\pgfqpoint{4.117201in}{0.934442in}}{\pgfqpoint{4.122245in}{0.929398in}}%
\pgfpathcurveto{\pgfqpoint{4.127289in}{0.924355in}}{\pgfqpoint{4.134130in}{0.921521in}}{\pgfqpoint{4.141263in}{0.921521in}}%
\pgfpathclose%
\pgfusepath{stroke,fill}%
\end{pgfscope}%
\begin{pgfscope}%
\pgfpathrectangle{\pgfqpoint{2.867647in}{0.500000in}}{\pgfqpoint{1.764706in}{1.700000in}}%
\pgfusepath{clip}%
\pgfsetbuttcap%
\pgfsetroundjoin%
\definecolor{currentfill}{rgb}{0.955103,0.477872,0.328626}%
\pgfsetfillcolor{currentfill}%
\pgfsetlinewidth{0.311001pt}%
\definecolor{currentstroke}{rgb}{1.000000,1.000000,1.000000}%
\pgfsetstrokecolor{currentstroke}%
\pgfsetdash{}{0pt}%
\pgfpathmoveto{\pgfqpoint{4.346382in}{1.206532in}}%
\pgfpathcurveto{\pgfqpoint{4.353515in}{1.206532in}}{\pgfqpoint{4.360357in}{1.209366in}}{\pgfqpoint{4.365400in}{1.214410in}}%
\pgfpathcurveto{\pgfqpoint{4.370444in}{1.219453in}}{\pgfqpoint{4.373278in}{1.226295in}}{\pgfqpoint{4.373278in}{1.233428in}}%
\pgfpathcurveto{\pgfqpoint{4.373278in}{1.240561in}}{\pgfqpoint{4.370444in}{1.247402in}}{\pgfqpoint{4.365400in}{1.252446in}}%
\pgfpathcurveto{\pgfqpoint{4.360357in}{1.257490in}}{\pgfqpoint{4.353515in}{1.260323in}}{\pgfqpoint{4.346382in}{1.260323in}}%
\pgfpathcurveto{\pgfqpoint{4.339250in}{1.260323in}}{\pgfqpoint{4.332408in}{1.257490in}}{\pgfqpoint{4.327364in}{1.252446in}}%
\pgfpathcurveto{\pgfqpoint{4.322321in}{1.247402in}}{\pgfqpoint{4.319487in}{1.240561in}}{\pgfqpoint{4.319487in}{1.233428in}}%
\pgfpathcurveto{\pgfqpoint{4.319487in}{1.226295in}}{\pgfqpoint{4.322321in}{1.219453in}}{\pgfqpoint{4.327364in}{1.214410in}}%
\pgfpathcurveto{\pgfqpoint{4.332408in}{1.209366in}}{\pgfqpoint{4.339250in}{1.206532in}}{\pgfqpoint{4.346382in}{1.206532in}}%
\pgfpathclose%
\pgfusepath{stroke,fill}%
\end{pgfscope}%
\begin{pgfscope}%
\pgfpathrectangle{\pgfqpoint{2.867647in}{0.500000in}}{\pgfqpoint{1.764706in}{1.700000in}}%
\pgfusepath{clip}%
\pgfsetbuttcap%
\pgfsetroundjoin%
\definecolor{currentfill}{rgb}{0.973832,0.856556,0.771584}%
\pgfsetfillcolor{currentfill}%
\pgfsetlinewidth{0.311001pt}%
\definecolor{currentstroke}{rgb}{1.000000,1.000000,1.000000}%
\pgfsetstrokecolor{currentstroke}%
\pgfsetdash{}{0pt}%
\pgfpathmoveto{\pgfqpoint{4.245610in}{1.419599in}}%
\pgfpathcurveto{\pgfqpoint{4.252743in}{1.419599in}}{\pgfqpoint{4.259584in}{1.422433in}}{\pgfqpoint{4.264628in}{1.427477in}}%
\pgfpathcurveto{\pgfqpoint{4.269672in}{1.432520in}}{\pgfqpoint{4.272505in}{1.439362in}}{\pgfqpoint{4.272505in}{1.446495in}}%
\pgfpathcurveto{\pgfqpoint{4.272505in}{1.453628in}}{\pgfqpoint{4.269672in}{1.460469in}}{\pgfqpoint{4.264628in}{1.465513in}}%
\pgfpathcurveto{\pgfqpoint{4.259584in}{1.470557in}}{\pgfqpoint{4.252743in}{1.473391in}}{\pgfqpoint{4.245610in}{1.473391in}}%
\pgfpathcurveto{\pgfqpoint{4.238477in}{1.473391in}}{\pgfqpoint{4.231635in}{1.470557in}}{\pgfqpoint{4.226592in}{1.465513in}}%
\pgfpathcurveto{\pgfqpoint{4.221548in}{1.460469in}}{\pgfqpoint{4.218714in}{1.453628in}}{\pgfqpoint{4.218714in}{1.446495in}}%
\pgfpathcurveto{\pgfqpoint{4.218714in}{1.439362in}}{\pgfqpoint{4.221548in}{1.432520in}}{\pgfqpoint{4.226592in}{1.427477in}}%
\pgfpathcurveto{\pgfqpoint{4.231635in}{1.422433in}}{\pgfqpoint{4.238477in}{1.419599in}}{\pgfqpoint{4.245610in}{1.419599in}}%
\pgfpathclose%
\pgfusepath{stroke,fill}%
\end{pgfscope}%
\begin{pgfscope}%
\pgfpathrectangle{\pgfqpoint{2.867647in}{0.500000in}}{\pgfqpoint{1.764706in}{1.700000in}}%
\pgfusepath{clip}%
\pgfsetbuttcap%
\pgfsetroundjoin%
\definecolor{currentfill}{rgb}{0.976961,0.885681,0.814303}%
\pgfsetfillcolor{currentfill}%
\pgfsetlinewidth{0.311001pt}%
\definecolor{currentstroke}{rgb}{1.000000,1.000000,1.000000}%
\pgfsetstrokecolor{currentstroke}%
\pgfsetdash{}{0pt}%
\pgfpathmoveto{\pgfqpoint{4.236130in}{1.260524in}}%
\pgfpathcurveto{\pgfqpoint{4.243263in}{1.260524in}}{\pgfqpoint{4.250105in}{1.263358in}}{\pgfqpoint{4.255148in}{1.268401in}}%
\pgfpathcurveto{\pgfqpoint{4.260192in}{1.273445in}}{\pgfqpoint{4.263026in}{1.280287in}}{\pgfqpoint{4.263026in}{1.287419in}}%
\pgfpathcurveto{\pgfqpoint{4.263026in}{1.294552in}}{\pgfqpoint{4.260192in}{1.301394in}}{\pgfqpoint{4.255148in}{1.306438in}}%
\pgfpathcurveto{\pgfqpoint{4.250105in}{1.311481in}}{\pgfqpoint{4.243263in}{1.314315in}}{\pgfqpoint{4.236130in}{1.314315in}}%
\pgfpathcurveto{\pgfqpoint{4.228997in}{1.314315in}}{\pgfqpoint{4.222156in}{1.311481in}}{\pgfqpoint{4.217112in}{1.306438in}}%
\pgfpathcurveto{\pgfqpoint{4.212068in}{1.301394in}}{\pgfqpoint{4.209234in}{1.294552in}}{\pgfqpoint{4.209234in}{1.287419in}}%
\pgfpathcurveto{\pgfqpoint{4.209234in}{1.280287in}}{\pgfqpoint{4.212068in}{1.273445in}}{\pgfqpoint{4.217112in}{1.268401in}}%
\pgfpathcurveto{\pgfqpoint{4.222156in}{1.263358in}}{\pgfqpoint{4.228997in}{1.260524in}}{\pgfqpoint{4.236130in}{1.260524in}}%
\pgfpathclose%
\pgfusepath{stroke,fill}%
\end{pgfscope}%
\begin{pgfscope}%
\pgfpathrectangle{\pgfqpoint{2.867647in}{0.500000in}}{\pgfqpoint{1.764706in}{1.700000in}}%
\pgfusepath{clip}%
\pgfsetbuttcap%
\pgfsetroundjoin%
\definecolor{currentfill}{rgb}{0.651546,0.096802,0.354541}%
\pgfsetfillcolor{currentfill}%
\pgfsetlinewidth{0.311001pt}%
\definecolor{currentstroke}{rgb}{1.000000,1.000000,1.000000}%
\pgfsetstrokecolor{currentstroke}%
\pgfsetdash{}{0pt}%
\pgfpathmoveto{\pgfqpoint{4.248843in}{0.800867in}}%
\pgfpathcurveto{\pgfqpoint{4.255976in}{0.800867in}}{\pgfqpoint{4.262818in}{0.803701in}}{\pgfqpoint{4.267861in}{0.808745in}}%
\pgfpathcurveto{\pgfqpoint{4.272905in}{0.813789in}}{\pgfqpoint{4.275739in}{0.820630in}}{\pgfqpoint{4.275739in}{0.827763in}}%
\pgfpathcurveto{\pgfqpoint{4.275739in}{0.834896in}}{\pgfqpoint{4.272905in}{0.841737in}}{\pgfqpoint{4.267861in}{0.846781in}}%
\pgfpathcurveto{\pgfqpoint{4.262818in}{0.851825in}}{\pgfqpoint{4.255976in}{0.854659in}}{\pgfqpoint{4.248843in}{0.854659in}}%
\pgfpathcurveto{\pgfqpoint{4.241710in}{0.854659in}}{\pgfqpoint{4.234869in}{0.851825in}}{\pgfqpoint{4.229825in}{0.846781in}}%
\pgfpathcurveto{\pgfqpoint{4.224781in}{0.841737in}}{\pgfqpoint{4.221948in}{0.834896in}}{\pgfqpoint{4.221948in}{0.827763in}}%
\pgfpathcurveto{\pgfqpoint{4.221948in}{0.820630in}}{\pgfqpoint{4.224781in}{0.813789in}}{\pgfqpoint{4.229825in}{0.808745in}}%
\pgfpathcurveto{\pgfqpoint{4.234869in}{0.803701in}}{\pgfqpoint{4.241710in}{0.800867in}}{\pgfqpoint{4.248843in}{0.800867in}}%
\pgfpathclose%
\pgfusepath{stroke,fill}%
\end{pgfscope}%
\begin{pgfscope}%
\pgfpathrectangle{\pgfqpoint{2.867647in}{0.500000in}}{\pgfqpoint{1.764706in}{1.700000in}}%
\pgfusepath{clip}%
\pgfsetbuttcap%
\pgfsetroundjoin%
\definecolor{currentfill}{rgb}{0.971202,0.827364,0.728520}%
\pgfsetfillcolor{currentfill}%
\pgfsetlinewidth{0.311001pt}%
\definecolor{currentstroke}{rgb}{1.000000,1.000000,1.000000}%
\pgfsetstrokecolor{currentstroke}%
\pgfsetdash{}{0pt}%
\pgfpathmoveto{\pgfqpoint{4.081087in}{1.687593in}}%
\pgfpathcurveto{\pgfqpoint{4.088220in}{1.687593in}}{\pgfqpoint{4.095061in}{1.690427in}}{\pgfqpoint{4.100105in}{1.695470in}}%
\pgfpathcurveto{\pgfqpoint{4.105149in}{1.700514in}}{\pgfqpoint{4.107983in}{1.707356in}}{\pgfqpoint{4.107983in}{1.714488in}}%
\pgfpathcurveto{\pgfqpoint{4.107983in}{1.721621in}}{\pgfqpoint{4.105149in}{1.728463in}}{\pgfqpoint{4.100105in}{1.733507in}}%
\pgfpathcurveto{\pgfqpoint{4.095061in}{1.738550in}}{\pgfqpoint{4.088220in}{1.741384in}}{\pgfqpoint{4.081087in}{1.741384in}}%
\pgfpathcurveto{\pgfqpoint{4.073954in}{1.741384in}}{\pgfqpoint{4.067112in}{1.738550in}}{\pgfqpoint{4.062069in}{1.733507in}}%
\pgfpathcurveto{\pgfqpoint{4.057025in}{1.728463in}}{\pgfqpoint{4.054191in}{1.721621in}}{\pgfqpoint{4.054191in}{1.714488in}}%
\pgfpathcurveto{\pgfqpoint{4.054191in}{1.707356in}}{\pgfqpoint{4.057025in}{1.700514in}}{\pgfqpoint{4.062069in}{1.695470in}}%
\pgfpathcurveto{\pgfqpoint{4.067112in}{1.690427in}}{\pgfqpoint{4.073954in}{1.687593in}}{\pgfqpoint{4.081087in}{1.687593in}}%
\pgfpathclose%
\pgfusepath{stroke,fill}%
\end{pgfscope}%
\begin{pgfscope}%
\pgfpathrectangle{\pgfqpoint{2.867647in}{0.500000in}}{\pgfqpoint{1.764706in}{1.700000in}}%
\pgfusepath{clip}%
\pgfsetbuttcap%
\pgfsetroundjoin%
\definecolor{currentfill}{rgb}{0.964433,0.670254,0.515093}%
\pgfsetfillcolor{currentfill}%
\pgfsetlinewidth{0.311001pt}%
\definecolor{currentstroke}{rgb}{1.000000,1.000000,1.000000}%
\pgfsetstrokecolor{currentstroke}%
\pgfsetdash{}{0pt}%
\pgfpathmoveto{\pgfqpoint{3.966591in}{1.689523in}}%
\pgfpathcurveto{\pgfqpoint{3.973724in}{1.689523in}}{\pgfqpoint{3.980566in}{1.692357in}}{\pgfqpoint{3.985609in}{1.697400in}}%
\pgfpathcurveto{\pgfqpoint{3.990653in}{1.702444in}}{\pgfqpoint{3.993487in}{1.709286in}}{\pgfqpoint{3.993487in}{1.716419in}}%
\pgfpathcurveto{\pgfqpoint{3.993487in}{1.723551in}}{\pgfqpoint{3.990653in}{1.730393in}}{\pgfqpoint{3.985609in}{1.735437in}}%
\pgfpathcurveto{\pgfqpoint{3.980566in}{1.740480in}}{\pgfqpoint{3.973724in}{1.743314in}}{\pgfqpoint{3.966591in}{1.743314in}}%
\pgfpathcurveto{\pgfqpoint{3.959458in}{1.743314in}}{\pgfqpoint{3.952617in}{1.740480in}}{\pgfqpoint{3.947573in}{1.735437in}}%
\pgfpathcurveto{\pgfqpoint{3.942529in}{1.730393in}}{\pgfqpoint{3.939695in}{1.723551in}}{\pgfqpoint{3.939695in}{1.716419in}}%
\pgfpathcurveto{\pgfqpoint{3.939695in}{1.709286in}}{\pgfqpoint{3.942529in}{1.702444in}}{\pgfqpoint{3.947573in}{1.697400in}}%
\pgfpathcurveto{\pgfqpoint{3.952617in}{1.692357in}}{\pgfqpoint{3.959458in}{1.689523in}}{\pgfqpoint{3.966591in}{1.689523in}}%
\pgfpathclose%
\pgfusepath{stroke,fill}%
\end{pgfscope}%
\begin{pgfscope}%
\pgfpathrectangle{\pgfqpoint{2.867647in}{0.500000in}}{\pgfqpoint{1.764706in}{1.700000in}}%
\pgfusepath{clip}%
\pgfsetbuttcap%
\pgfsetroundjoin%
\definecolor{currentfill}{rgb}{0.970255,0.815666,0.711203}%
\pgfsetfillcolor{currentfill}%
\pgfsetlinewidth{0.311001pt}%
\definecolor{currentstroke}{rgb}{1.000000,1.000000,1.000000}%
\pgfsetstrokecolor{currentstroke}%
\pgfsetdash{}{0pt}%
\pgfpathmoveto{\pgfqpoint{4.270431in}{1.373709in}}%
\pgfpathcurveto{\pgfqpoint{4.277564in}{1.373709in}}{\pgfqpoint{4.284406in}{1.376543in}}{\pgfqpoint{4.289449in}{1.381586in}}%
\pgfpathcurveto{\pgfqpoint{4.294493in}{1.386630in}}{\pgfqpoint{4.297327in}{1.393472in}}{\pgfqpoint{4.297327in}{1.400605in}}%
\pgfpathcurveto{\pgfqpoint{4.297327in}{1.407737in}}{\pgfqpoint{4.294493in}{1.414579in}}{\pgfqpoint{4.289449in}{1.419623in}}%
\pgfpathcurveto{\pgfqpoint{4.284406in}{1.424666in}}{\pgfqpoint{4.277564in}{1.427500in}}{\pgfqpoint{4.270431in}{1.427500in}}%
\pgfpathcurveto{\pgfqpoint{4.263298in}{1.427500in}}{\pgfqpoint{4.256457in}{1.424666in}}{\pgfqpoint{4.251413in}{1.419623in}}%
\pgfpathcurveto{\pgfqpoint{4.246369in}{1.414579in}}{\pgfqpoint{4.243535in}{1.407737in}}{\pgfqpoint{4.243535in}{1.400605in}}%
\pgfpathcurveto{\pgfqpoint{4.243535in}{1.393472in}}{\pgfqpoint{4.246369in}{1.386630in}}{\pgfqpoint{4.251413in}{1.381586in}}%
\pgfpathcurveto{\pgfqpoint{4.256457in}{1.376543in}}{\pgfqpoint{4.263298in}{1.373709in}}{\pgfqpoint{4.270431in}{1.373709in}}%
\pgfpathclose%
\pgfusepath{stroke,fill}%
\end{pgfscope}%
\begin{pgfscope}%
\pgfpathrectangle{\pgfqpoint{2.867647in}{0.500000in}}{\pgfqpoint{1.764706in}{1.700000in}}%
\pgfusepath{clip}%
\pgfsetbuttcap%
\pgfsetroundjoin%
\definecolor{currentfill}{rgb}{0.965302,0.713942,0.568499}%
\pgfsetfillcolor{currentfill}%
\pgfsetlinewidth{0.311001pt}%
\definecolor{currentstroke}{rgb}{1.000000,1.000000,1.000000}%
\pgfsetstrokecolor{currentstroke}%
\pgfsetdash{}{0pt}%
\pgfpathmoveto{\pgfqpoint{4.002663in}{1.729696in}}%
\pgfpathcurveto{\pgfqpoint{4.009796in}{1.729696in}}{\pgfqpoint{4.016637in}{1.732530in}}{\pgfqpoint{4.021681in}{1.737574in}}%
\pgfpathcurveto{\pgfqpoint{4.026725in}{1.742617in}}{\pgfqpoint{4.029559in}{1.749459in}}{\pgfqpoint{4.029559in}{1.756592in}}%
\pgfpathcurveto{\pgfqpoint{4.029559in}{1.763725in}}{\pgfqpoint{4.026725in}{1.770566in}}{\pgfqpoint{4.021681in}{1.775610in}}%
\pgfpathcurveto{\pgfqpoint{4.016637in}{1.780654in}}{\pgfqpoint{4.009796in}{1.783488in}}{\pgfqpoint{4.002663in}{1.783488in}}%
\pgfpathcurveto{\pgfqpoint{3.995530in}{1.783488in}}{\pgfqpoint{3.988689in}{1.780654in}}{\pgfqpoint{3.983645in}{1.775610in}}%
\pgfpathcurveto{\pgfqpoint{3.978601in}{1.770566in}}{\pgfqpoint{3.975767in}{1.763725in}}{\pgfqpoint{3.975767in}{1.756592in}}%
\pgfpathcurveto{\pgfqpoint{3.975767in}{1.749459in}}{\pgfqpoint{3.978601in}{1.742617in}}{\pgfqpoint{3.983645in}{1.737574in}}%
\pgfpathcurveto{\pgfqpoint{3.988689in}{1.732530in}}{\pgfqpoint{3.995530in}{1.729696in}}{\pgfqpoint{4.002663in}{1.729696in}}%
\pgfpathclose%
\pgfusepath{stroke,fill}%
\end{pgfscope}%
\begin{pgfscope}%
\pgfpathrectangle{\pgfqpoint{2.867647in}{0.500000in}}{\pgfqpoint{1.764706in}{1.700000in}}%
\pgfusepath{clip}%
\pgfsetbuttcap%
\pgfsetroundjoin%
\definecolor{currentfill}{rgb}{0.981377,0.920617,0.865369}%
\pgfsetfillcolor{currentfill}%
\pgfsetlinewidth{0.311001pt}%
\definecolor{currentstroke}{rgb}{1.000000,1.000000,1.000000}%
\pgfsetstrokecolor{currentstroke}%
\pgfsetdash{}{0pt}%
\pgfpathmoveto{\pgfqpoint{4.193320in}{1.285589in}}%
\pgfpathcurveto{\pgfqpoint{4.200453in}{1.285589in}}{\pgfqpoint{4.207295in}{1.288423in}}{\pgfqpoint{4.212338in}{1.293467in}}%
\pgfpathcurveto{\pgfqpoint{4.217382in}{1.298510in}}{\pgfqpoint{4.220216in}{1.305352in}}{\pgfqpoint{4.220216in}{1.312485in}}%
\pgfpathcurveto{\pgfqpoint{4.220216in}{1.319618in}}{\pgfqpoint{4.217382in}{1.326459in}}{\pgfqpoint{4.212338in}{1.331503in}}%
\pgfpathcurveto{\pgfqpoint{4.207295in}{1.336547in}}{\pgfqpoint{4.200453in}{1.339380in}}{\pgfqpoint{4.193320in}{1.339380in}}%
\pgfpathcurveto{\pgfqpoint{4.186187in}{1.339380in}}{\pgfqpoint{4.179346in}{1.336547in}}{\pgfqpoint{4.174302in}{1.331503in}}%
\pgfpathcurveto{\pgfqpoint{4.169258in}{1.326459in}}{\pgfqpoint{4.166425in}{1.319618in}}{\pgfqpoint{4.166425in}{1.312485in}}%
\pgfpathcurveto{\pgfqpoint{4.166425in}{1.305352in}}{\pgfqpoint{4.169258in}{1.298510in}}{\pgfqpoint{4.174302in}{1.293467in}}%
\pgfpathcurveto{\pgfqpoint{4.179346in}{1.288423in}}{\pgfqpoint{4.186187in}{1.285589in}}{\pgfqpoint{4.193320in}{1.285589in}}%
\pgfpathclose%
\pgfusepath{stroke,fill}%
\end{pgfscope}%
\begin{pgfscope}%
\pgfpathrectangle{\pgfqpoint{2.867647in}{0.500000in}}{\pgfqpoint{1.764706in}{1.700000in}}%
\pgfusepath{clip}%
\pgfsetbuttcap%
\pgfsetroundjoin%
\definecolor{currentfill}{rgb}{0.978376,0.897317,0.831308}%
\pgfsetfillcolor{currentfill}%
\pgfsetlinewidth{0.311001pt}%
\definecolor{currentstroke}{rgb}{1.000000,1.000000,1.000000}%
\pgfsetstrokecolor{currentstroke}%
\pgfsetdash{}{0pt}%
\pgfpathmoveto{\pgfqpoint{4.219442in}{1.195936in}}%
\pgfpathcurveto{\pgfqpoint{4.226575in}{1.195936in}}{\pgfqpoint{4.233417in}{1.198770in}}{\pgfqpoint{4.238460in}{1.203813in}}%
\pgfpathcurveto{\pgfqpoint{4.243504in}{1.208857in}}{\pgfqpoint{4.246338in}{1.215699in}}{\pgfqpoint{4.246338in}{1.222832in}}%
\pgfpathcurveto{\pgfqpoint{4.246338in}{1.229964in}}{\pgfqpoint{4.243504in}{1.236806in}}{\pgfqpoint{4.238460in}{1.241850in}}%
\pgfpathcurveto{\pgfqpoint{4.233417in}{1.246893in}}{\pgfqpoint{4.226575in}{1.249727in}}{\pgfqpoint{4.219442in}{1.249727in}}%
\pgfpathcurveto{\pgfqpoint{4.212309in}{1.249727in}}{\pgfqpoint{4.205468in}{1.246893in}}{\pgfqpoint{4.200424in}{1.241850in}}%
\pgfpathcurveto{\pgfqpoint{4.195380in}{1.236806in}}{\pgfqpoint{4.192547in}{1.229964in}}{\pgfqpoint{4.192547in}{1.222832in}}%
\pgfpathcurveto{\pgfqpoint{4.192547in}{1.215699in}}{\pgfqpoint{4.195380in}{1.208857in}}{\pgfqpoint{4.200424in}{1.203813in}}%
\pgfpathcurveto{\pgfqpoint{4.205468in}{1.198770in}}{\pgfqpoint{4.212309in}{1.195936in}}{\pgfqpoint{4.219442in}{1.195936in}}%
\pgfpathclose%
\pgfusepath{stroke,fill}%
\end{pgfscope}%
\begin{pgfscope}%
\pgfpathrectangle{\pgfqpoint{2.867647in}{0.500000in}}{\pgfqpoint{1.764706in}{1.700000in}}%
\pgfusepath{clip}%
\pgfsetbuttcap%
\pgfsetroundjoin%
\definecolor{currentfill}{rgb}{0.975018,0.868213,0.788710}%
\pgfsetfillcolor{currentfill}%
\pgfsetlinewidth{0.311001pt}%
\definecolor{currentstroke}{rgb}{1.000000,1.000000,1.000000}%
\pgfsetstrokecolor{currentstroke}%
\pgfsetdash{}{0pt}%
\pgfpathmoveto{\pgfqpoint{4.122787in}{1.222289in}}%
\pgfpathcurveto{\pgfqpoint{4.129920in}{1.222289in}}{\pgfqpoint{4.136761in}{1.225123in}}{\pgfqpoint{4.141805in}{1.230167in}}%
\pgfpathcurveto{\pgfqpoint{4.146849in}{1.235210in}}{\pgfqpoint{4.149683in}{1.242052in}}{\pgfqpoint{4.149683in}{1.249185in}}%
\pgfpathcurveto{\pgfqpoint{4.149683in}{1.256318in}}{\pgfqpoint{4.146849in}{1.263159in}}{\pgfqpoint{4.141805in}{1.268203in}}%
\pgfpathcurveto{\pgfqpoint{4.136761in}{1.273247in}}{\pgfqpoint{4.129920in}{1.276081in}}{\pgfqpoint{4.122787in}{1.276081in}}%
\pgfpathcurveto{\pgfqpoint{4.115654in}{1.276081in}}{\pgfqpoint{4.108812in}{1.273247in}}{\pgfqpoint{4.103769in}{1.268203in}}%
\pgfpathcurveto{\pgfqpoint{4.098725in}{1.263159in}}{\pgfqpoint{4.095891in}{1.256318in}}{\pgfqpoint{4.095891in}{1.249185in}}%
\pgfpathcurveto{\pgfqpoint{4.095891in}{1.242052in}}{\pgfqpoint{4.098725in}{1.235210in}}{\pgfqpoint{4.103769in}{1.230167in}}%
\pgfpathcurveto{\pgfqpoint{4.108812in}{1.225123in}}{\pgfqpoint{4.115654in}{1.222289in}}{\pgfqpoint{4.122787in}{1.222289in}}%
\pgfpathclose%
\pgfusepath{stroke,fill}%
\end{pgfscope}%
\begin{pgfscope}%
\pgfpathrectangle{\pgfqpoint{2.867647in}{0.500000in}}{\pgfqpoint{1.764706in}{1.700000in}}%
\pgfusepath{clip}%
\pgfsetbuttcap%
\pgfsetroundjoin%
\definecolor{currentfill}{rgb}{0.980678,0.914765,0.856766}%
\pgfsetfillcolor{currentfill}%
\pgfsetlinewidth{0.311001pt}%
\definecolor{currentstroke}{rgb}{1.000000,1.000000,1.000000}%
\pgfsetstrokecolor{currentstroke}%
\pgfsetdash{}{0pt}%
\pgfpathmoveto{\pgfqpoint{4.158854in}{1.484211in}}%
\pgfpathcurveto{\pgfqpoint{4.165987in}{1.484211in}}{\pgfqpoint{4.172829in}{1.487045in}}{\pgfqpoint{4.177872in}{1.492089in}}%
\pgfpathcurveto{\pgfqpoint{4.182916in}{1.497132in}}{\pgfqpoint{4.185750in}{1.503974in}}{\pgfqpoint{4.185750in}{1.511107in}}%
\pgfpathcurveto{\pgfqpoint{4.185750in}{1.518240in}}{\pgfqpoint{4.182916in}{1.525081in}}{\pgfqpoint{4.177872in}{1.530125in}}%
\pgfpathcurveto{\pgfqpoint{4.172829in}{1.535169in}}{\pgfqpoint{4.165987in}{1.538003in}}{\pgfqpoint{4.158854in}{1.538003in}}%
\pgfpathcurveto{\pgfqpoint{4.151721in}{1.538003in}}{\pgfqpoint{4.144880in}{1.535169in}}{\pgfqpoint{4.139836in}{1.530125in}}%
\pgfpathcurveto{\pgfqpoint{4.134792in}{1.525081in}}{\pgfqpoint{4.131958in}{1.518240in}}{\pgfqpoint{4.131958in}{1.511107in}}%
\pgfpathcurveto{\pgfqpoint{4.131958in}{1.503974in}}{\pgfqpoint{4.134792in}{1.497132in}}{\pgfqpoint{4.139836in}{1.492089in}}%
\pgfpathcurveto{\pgfqpoint{4.144880in}{1.487045in}}{\pgfqpoint{4.151721in}{1.484211in}}{\pgfqpoint{4.158854in}{1.484211in}}%
\pgfpathclose%
\pgfusepath{stroke,fill}%
\end{pgfscope}%
\begin{pgfscope}%
\pgfpathrectangle{\pgfqpoint{2.867647in}{0.500000in}}{\pgfqpoint{1.764706in}{1.700000in}}%
\pgfusepath{clip}%
\pgfsetbuttcap%
\pgfsetroundjoin%
\definecolor{currentfill}{rgb}{0.981377,0.920617,0.865369}%
\pgfsetfillcolor{currentfill}%
\pgfsetlinewidth{0.311001pt}%
\definecolor{currentstroke}{rgb}{1.000000,1.000000,1.000000}%
\pgfsetstrokecolor{currentstroke}%
\pgfsetdash{}{0pt}%
\pgfpathmoveto{\pgfqpoint{4.194106in}{1.207905in}}%
\pgfpathcurveto{\pgfqpoint{4.201239in}{1.207905in}}{\pgfqpoint{4.208080in}{1.210739in}}{\pgfqpoint{4.213124in}{1.215782in}}%
\pgfpathcurveto{\pgfqpoint{4.218168in}{1.220826in}}{\pgfqpoint{4.221001in}{1.227668in}}{\pgfqpoint{4.221001in}{1.234801in}}%
\pgfpathcurveto{\pgfqpoint{4.221001in}{1.241933in}}{\pgfqpoint{4.218168in}{1.248775in}}{\pgfqpoint{4.213124in}{1.253819in}}%
\pgfpathcurveto{\pgfqpoint{4.208080in}{1.258862in}}{\pgfqpoint{4.201239in}{1.261696in}}{\pgfqpoint{4.194106in}{1.261696in}}%
\pgfpathcurveto{\pgfqpoint{4.186973in}{1.261696in}}{\pgfqpoint{4.180131in}{1.258862in}}{\pgfqpoint{4.175088in}{1.253819in}}%
\pgfpathcurveto{\pgfqpoint{4.170044in}{1.248775in}}{\pgfqpoint{4.167210in}{1.241933in}}{\pgfqpoint{4.167210in}{1.234801in}}%
\pgfpathcurveto{\pgfqpoint{4.167210in}{1.227668in}}{\pgfqpoint{4.170044in}{1.220826in}}{\pgfqpoint{4.175088in}{1.215782in}}%
\pgfpathcurveto{\pgfqpoint{4.180131in}{1.210739in}}{\pgfqpoint{4.186973in}{1.207905in}}{\pgfqpoint{4.194106in}{1.207905in}}%
\pgfpathclose%
\pgfusepath{stroke,fill}%
\end{pgfscope}%
\begin{pgfscope}%
\pgfpathrectangle{\pgfqpoint{2.867647in}{0.500000in}}{\pgfqpoint{1.764706in}{1.700000in}}%
\pgfusepath{clip}%
\pgfsetbuttcap%
\pgfsetroundjoin%
\definecolor{currentfill}{rgb}{0.969359,0.803954,0.693832}%
\pgfsetfillcolor{currentfill}%
\pgfsetlinewidth{0.311001pt}%
\definecolor{currentstroke}{rgb}{1.000000,1.000000,1.000000}%
\pgfsetstrokecolor{currentstroke}%
\pgfsetdash{}{0pt}%
\pgfpathmoveto{\pgfqpoint{4.073034in}{1.480478in}}%
\pgfpathcurveto{\pgfqpoint{4.080167in}{1.480478in}}{\pgfqpoint{4.087009in}{1.483312in}}{\pgfqpoint{4.092052in}{1.488356in}}%
\pgfpathcurveto{\pgfqpoint{4.097096in}{1.493400in}}{\pgfqpoint{4.099930in}{1.500241in}}{\pgfqpoint{4.099930in}{1.507374in}}%
\pgfpathcurveto{\pgfqpoint{4.099930in}{1.514507in}}{\pgfqpoint{4.097096in}{1.521348in}}{\pgfqpoint{4.092052in}{1.526392in}}%
\pgfpathcurveto{\pgfqpoint{4.087009in}{1.531436in}}{\pgfqpoint{4.080167in}{1.534270in}}{\pgfqpoint{4.073034in}{1.534270in}}%
\pgfpathcurveto{\pgfqpoint{4.065901in}{1.534270in}}{\pgfqpoint{4.059060in}{1.531436in}}{\pgfqpoint{4.054016in}{1.526392in}}%
\pgfpathcurveto{\pgfqpoint{4.048972in}{1.521348in}}{\pgfqpoint{4.046138in}{1.514507in}}{\pgfqpoint{4.046138in}{1.507374in}}%
\pgfpathcurveto{\pgfqpoint{4.046138in}{1.500241in}}{\pgfqpoint{4.048972in}{1.493400in}}{\pgfqpoint{4.054016in}{1.488356in}}%
\pgfpathcurveto{\pgfqpoint{4.059060in}{1.483312in}}{\pgfqpoint{4.065901in}{1.480478in}}{\pgfqpoint{4.073034in}{1.480478in}}%
\pgfpathclose%
\pgfusepath{stroke,fill}%
\end{pgfscope}%
\begin{pgfscope}%
\pgfpathrectangle{\pgfqpoint{2.867647in}{0.500000in}}{\pgfqpoint{1.764706in}{1.700000in}}%
\pgfusepath{clip}%
\pgfsetbuttcap%
\pgfsetroundjoin%
\definecolor{currentfill}{rgb}{0.975644,0.874038,0.797253}%
\pgfsetfillcolor{currentfill}%
\pgfsetlinewidth{0.311001pt}%
\definecolor{currentstroke}{rgb}{1.000000,1.000000,1.000000}%
\pgfsetstrokecolor{currentstroke}%
\pgfsetdash{}{0pt}%
\pgfpathmoveto{\pgfqpoint{4.235152in}{1.414171in}}%
\pgfpathcurveto{\pgfqpoint{4.242285in}{1.414171in}}{\pgfqpoint{4.249126in}{1.417005in}}{\pgfqpoint{4.254170in}{1.422048in}}%
\pgfpathcurveto{\pgfqpoint{4.259214in}{1.427092in}}{\pgfqpoint{4.262048in}{1.433934in}}{\pgfqpoint{4.262048in}{1.441066in}}%
\pgfpathcurveto{\pgfqpoint{4.262048in}{1.448199in}}{\pgfqpoint{4.259214in}{1.455041in}}{\pgfqpoint{4.254170in}{1.460085in}}%
\pgfpathcurveto{\pgfqpoint{4.249126in}{1.465128in}}{\pgfqpoint{4.242285in}{1.467962in}}{\pgfqpoint{4.235152in}{1.467962in}}%
\pgfpathcurveto{\pgfqpoint{4.228019in}{1.467962in}}{\pgfqpoint{4.221177in}{1.465128in}}{\pgfqpoint{4.216134in}{1.460085in}}%
\pgfpathcurveto{\pgfqpoint{4.211090in}{1.455041in}}{\pgfqpoint{4.208256in}{1.448199in}}{\pgfqpoint{4.208256in}{1.441066in}}%
\pgfpathcurveto{\pgfqpoint{4.208256in}{1.433934in}}{\pgfqpoint{4.211090in}{1.427092in}}{\pgfqpoint{4.216134in}{1.422048in}}%
\pgfpathcurveto{\pgfqpoint{4.221177in}{1.417005in}}{\pgfqpoint{4.228019in}{1.414171in}}{\pgfqpoint{4.235152in}{1.414171in}}%
\pgfpathclose%
\pgfusepath{stroke,fill}%
\end{pgfscope}%
\begin{pgfscope}%
\pgfpathrectangle{\pgfqpoint{2.867647in}{0.500000in}}{\pgfqpoint{1.764706in}{1.700000in}}%
\pgfusepath{clip}%
\pgfsetbuttcap%
\pgfsetroundjoin%
\definecolor{currentfill}{rgb}{0.972726,0.844889,0.754401}%
\pgfsetfillcolor{currentfill}%
\pgfsetlinewidth{0.311001pt}%
\definecolor{currentstroke}{rgb}{1.000000,1.000000,1.000000}%
\pgfsetstrokecolor{currentstroke}%
\pgfsetdash{}{0pt}%
\pgfpathmoveto{\pgfqpoint{4.170747in}{1.019465in}}%
\pgfpathcurveto{\pgfqpoint{4.177880in}{1.019465in}}{\pgfqpoint{4.184721in}{1.022299in}}{\pgfqpoint{4.189765in}{1.027343in}}%
\pgfpathcurveto{\pgfqpoint{4.194809in}{1.032386in}}{\pgfqpoint{4.197643in}{1.039228in}}{\pgfqpoint{4.197643in}{1.046361in}}%
\pgfpathcurveto{\pgfqpoint{4.197643in}{1.053494in}}{\pgfqpoint{4.194809in}{1.060335in}}{\pgfqpoint{4.189765in}{1.065379in}}%
\pgfpathcurveto{\pgfqpoint{4.184721in}{1.070423in}}{\pgfqpoint{4.177880in}{1.073257in}}{\pgfqpoint{4.170747in}{1.073257in}}%
\pgfpathcurveto{\pgfqpoint{4.163614in}{1.073257in}}{\pgfqpoint{4.156772in}{1.070423in}}{\pgfqpoint{4.151729in}{1.065379in}}%
\pgfpathcurveto{\pgfqpoint{4.146685in}{1.060335in}}{\pgfqpoint{4.143851in}{1.053494in}}{\pgfqpoint{4.143851in}{1.046361in}}%
\pgfpathcurveto{\pgfqpoint{4.143851in}{1.039228in}}{\pgfqpoint{4.146685in}{1.032386in}}{\pgfqpoint{4.151729in}{1.027343in}}%
\pgfpathcurveto{\pgfqpoint{4.156772in}{1.022299in}}{\pgfqpoint{4.163614in}{1.019465in}}{\pgfqpoint{4.170747in}{1.019465in}}%
\pgfpathclose%
\pgfusepath{stroke,fill}%
\end{pgfscope}%
\begin{pgfscope}%
\pgfpathrectangle{\pgfqpoint{2.867647in}{0.500000in}}{\pgfqpoint{1.764706in}{1.700000in}}%
\pgfusepath{clip}%
\pgfsetbuttcap%
\pgfsetroundjoin%
\definecolor{currentfill}{rgb}{0.966120,0.744512,0.608720}%
\pgfsetfillcolor{currentfill}%
\pgfsetlinewidth{0.311001pt}%
\definecolor{currentstroke}{rgb}{1.000000,1.000000,1.000000}%
\pgfsetstrokecolor{currentstroke}%
\pgfsetdash{}{0pt}%
\pgfpathmoveto{\pgfqpoint{4.263488in}{1.103176in}}%
\pgfpathcurveto{\pgfqpoint{4.270621in}{1.103176in}}{\pgfqpoint{4.277463in}{1.106010in}}{\pgfqpoint{4.282507in}{1.111053in}}%
\pgfpathcurveto{\pgfqpoint{4.287550in}{1.116097in}}{\pgfqpoint{4.290384in}{1.122939in}}{\pgfqpoint{4.290384in}{1.130071in}}%
\pgfpathcurveto{\pgfqpoint{4.290384in}{1.137204in}}{\pgfqpoint{4.287550in}{1.144046in}}{\pgfqpoint{4.282507in}{1.149090in}}%
\pgfpathcurveto{\pgfqpoint{4.277463in}{1.154133in}}{\pgfqpoint{4.270621in}{1.156967in}}{\pgfqpoint{4.263488in}{1.156967in}}%
\pgfpathcurveto{\pgfqpoint{4.256356in}{1.156967in}}{\pgfqpoint{4.249514in}{1.154133in}}{\pgfqpoint{4.244470in}{1.149090in}}%
\pgfpathcurveto{\pgfqpoint{4.239427in}{1.144046in}}{\pgfqpoint{4.236593in}{1.137204in}}{\pgfqpoint{4.236593in}{1.130071in}}%
\pgfpathcurveto{\pgfqpoint{4.236593in}{1.122939in}}{\pgfqpoint{4.239427in}{1.116097in}}{\pgfqpoint{4.244470in}{1.111053in}}%
\pgfpathcurveto{\pgfqpoint{4.249514in}{1.106010in}}{\pgfqpoint{4.256356in}{1.103176in}}{\pgfqpoint{4.263488in}{1.103176in}}%
\pgfpathclose%
\pgfusepath{stroke,fill}%
\end{pgfscope}%
\begin{pgfscope}%
\pgfpathrectangle{\pgfqpoint{2.867647in}{0.500000in}}{\pgfqpoint{1.764706in}{1.700000in}}%
\pgfusepath{clip}%
\pgfsetbuttcap%
\pgfsetroundjoin%
\definecolor{currentfill}{rgb}{0.967735,0.780441,0.659127}%
\pgfsetfillcolor{currentfill}%
\pgfsetlinewidth{0.311001pt}%
\definecolor{currentstroke}{rgb}{1.000000,1.000000,1.000000}%
\pgfsetstrokecolor{currentstroke}%
\pgfsetdash{}{0pt}%
\pgfpathmoveto{\pgfqpoint{4.031140in}{1.597240in}}%
\pgfpathcurveto{\pgfqpoint{4.038273in}{1.597240in}}{\pgfqpoint{4.045114in}{1.600074in}}{\pgfqpoint{4.050158in}{1.605117in}}%
\pgfpathcurveto{\pgfqpoint{4.055201in}{1.610161in}}{\pgfqpoint{4.058035in}{1.617003in}}{\pgfqpoint{4.058035in}{1.624136in}}%
\pgfpathcurveto{\pgfqpoint{4.058035in}{1.631268in}}{\pgfqpoint{4.055201in}{1.638110in}}{\pgfqpoint{4.050158in}{1.643154in}}%
\pgfpathcurveto{\pgfqpoint{4.045114in}{1.648197in}}{\pgfqpoint{4.038273in}{1.651031in}}{\pgfqpoint{4.031140in}{1.651031in}}%
\pgfpathcurveto{\pgfqpoint{4.024007in}{1.651031in}}{\pgfqpoint{4.017165in}{1.648197in}}{\pgfqpoint{4.012122in}{1.643154in}}%
\pgfpathcurveto{\pgfqpoint{4.007078in}{1.638110in}}{\pgfqpoint{4.004244in}{1.631268in}}{\pgfqpoint{4.004244in}{1.624136in}}%
\pgfpathcurveto{\pgfqpoint{4.004244in}{1.617003in}}{\pgfqpoint{4.007078in}{1.610161in}}{\pgfqpoint{4.012122in}{1.605117in}}%
\pgfpathcurveto{\pgfqpoint{4.017165in}{1.600074in}}{\pgfqpoint{4.024007in}{1.597240in}}{\pgfqpoint{4.031140in}{1.597240in}}%
\pgfpathclose%
\pgfusepath{stroke,fill}%
\end{pgfscope}%
\begin{pgfscope}%
\pgfpathrectangle{\pgfqpoint{2.867647in}{0.500000in}}{\pgfqpoint{1.764706in}{1.700000in}}%
\pgfusepath{clip}%
\pgfsetbuttcap%
\pgfsetroundjoin%
\definecolor{currentfill}{rgb}{0.979891,0.908948,0.848279}%
\pgfsetfillcolor{currentfill}%
\pgfsetlinewidth{0.311001pt}%
\definecolor{currentstroke}{rgb}{1.000000,1.000000,1.000000}%
\pgfsetstrokecolor{currentstroke}%
\pgfsetdash{}{0pt}%
\pgfpathmoveto{\pgfqpoint{4.139478in}{1.565280in}}%
\pgfpathcurveto{\pgfqpoint{4.146611in}{1.565280in}}{\pgfqpoint{4.153453in}{1.568114in}}{\pgfqpoint{4.158497in}{1.573158in}}%
\pgfpathcurveto{\pgfqpoint{4.163540in}{1.578201in}}{\pgfqpoint{4.166374in}{1.585043in}}{\pgfqpoint{4.166374in}{1.592176in}}%
\pgfpathcurveto{\pgfqpoint{4.166374in}{1.599309in}}{\pgfqpoint{4.163540in}{1.606150in}}{\pgfqpoint{4.158497in}{1.611194in}}%
\pgfpathcurveto{\pgfqpoint{4.153453in}{1.616238in}}{\pgfqpoint{4.146611in}{1.619072in}}{\pgfqpoint{4.139478in}{1.619072in}}%
\pgfpathcurveto{\pgfqpoint{4.132346in}{1.619072in}}{\pgfqpoint{4.125504in}{1.616238in}}{\pgfqpoint{4.120460in}{1.611194in}}%
\pgfpathcurveto{\pgfqpoint{4.115417in}{1.606150in}}{\pgfqpoint{4.112583in}{1.599309in}}{\pgfqpoint{4.112583in}{1.592176in}}%
\pgfpathcurveto{\pgfqpoint{4.112583in}{1.585043in}}{\pgfqpoint{4.115417in}{1.578201in}}{\pgfqpoint{4.120460in}{1.573158in}}%
\pgfpathcurveto{\pgfqpoint{4.125504in}{1.568114in}}{\pgfqpoint{4.132346in}{1.565280in}}{\pgfqpoint{4.139478in}{1.565280in}}%
\pgfpathclose%
\pgfusepath{stroke,fill}%
\end{pgfscope}%
\begin{pgfscope}%
\pgfpathrectangle{\pgfqpoint{2.867647in}{0.500000in}}{\pgfqpoint{1.764706in}{1.700000in}}%
\pgfusepath{clip}%
\pgfsetbuttcap%
\pgfsetroundjoin%
\definecolor{currentfill}{rgb}{0.980678,0.914765,0.856766}%
\pgfsetfillcolor{currentfill}%
\pgfsetlinewidth{0.311001pt}%
\definecolor{currentstroke}{rgb}{1.000000,1.000000,1.000000}%
\pgfsetstrokecolor{currentstroke}%
\pgfsetdash{}{0pt}%
\pgfpathmoveto{\pgfqpoint{4.162452in}{1.254082in}}%
\pgfpathcurveto{\pgfqpoint{4.169585in}{1.254082in}}{\pgfqpoint{4.176427in}{1.256916in}}{\pgfqpoint{4.181470in}{1.261959in}}%
\pgfpathcurveto{\pgfqpoint{4.186514in}{1.267003in}}{\pgfqpoint{4.189348in}{1.273845in}}{\pgfqpoint{4.189348in}{1.280977in}}%
\pgfpathcurveto{\pgfqpoint{4.189348in}{1.288110in}}{\pgfqpoint{4.186514in}{1.294952in}}{\pgfqpoint{4.181470in}{1.299996in}}%
\pgfpathcurveto{\pgfqpoint{4.176427in}{1.305039in}}{\pgfqpoint{4.169585in}{1.307873in}}{\pgfqpoint{4.162452in}{1.307873in}}%
\pgfpathcurveto{\pgfqpoint{4.155319in}{1.307873in}}{\pgfqpoint{4.148478in}{1.305039in}}{\pgfqpoint{4.143434in}{1.299996in}}%
\pgfpathcurveto{\pgfqpoint{4.138391in}{1.294952in}}{\pgfqpoint{4.135557in}{1.288110in}}{\pgfqpoint{4.135557in}{1.280977in}}%
\pgfpathcurveto{\pgfqpoint{4.135557in}{1.273845in}}{\pgfqpoint{4.138391in}{1.267003in}}{\pgfqpoint{4.143434in}{1.261959in}}%
\pgfpathcurveto{\pgfqpoint{4.148478in}{1.256916in}}{\pgfqpoint{4.155319in}{1.254082in}}{\pgfqpoint{4.162452in}{1.254082in}}%
\pgfpathclose%
\pgfusepath{stroke,fill}%
\end{pgfscope}%
\begin{pgfscope}%
\pgfpathrectangle{\pgfqpoint{2.867647in}{0.500000in}}{\pgfqpoint{1.764706in}{1.700000in}}%
\pgfusepath{clip}%
\pgfsetbuttcap%
\pgfsetroundjoin%
\definecolor{currentfill}{rgb}{0.980678,0.914765,0.856766}%
\pgfsetfillcolor{currentfill}%
\pgfsetlinewidth{0.311001pt}%
\definecolor{currentstroke}{rgb}{1.000000,1.000000,1.000000}%
\pgfsetstrokecolor{currentstroke}%
\pgfsetdash{}{0pt}%
\pgfpathmoveto{\pgfqpoint{4.200794in}{1.347812in}}%
\pgfpathcurveto{\pgfqpoint{4.207927in}{1.347812in}}{\pgfqpoint{4.214769in}{1.350646in}}{\pgfqpoint{4.219812in}{1.355689in}}%
\pgfpathcurveto{\pgfqpoint{4.224856in}{1.360733in}}{\pgfqpoint{4.227690in}{1.367575in}}{\pgfqpoint{4.227690in}{1.374707in}}%
\pgfpathcurveto{\pgfqpoint{4.227690in}{1.381840in}}{\pgfqpoint{4.224856in}{1.388682in}}{\pgfqpoint{4.219812in}{1.393726in}}%
\pgfpathcurveto{\pgfqpoint{4.214769in}{1.398769in}}{\pgfqpoint{4.207927in}{1.401603in}}{\pgfqpoint{4.200794in}{1.401603in}}%
\pgfpathcurveto{\pgfqpoint{4.193661in}{1.401603in}}{\pgfqpoint{4.186820in}{1.398769in}}{\pgfqpoint{4.181776in}{1.393726in}}%
\pgfpathcurveto{\pgfqpoint{4.176732in}{1.388682in}}{\pgfqpoint{4.173898in}{1.381840in}}{\pgfqpoint{4.173898in}{1.374707in}}%
\pgfpathcurveto{\pgfqpoint{4.173898in}{1.367575in}}{\pgfqpoint{4.176732in}{1.360733in}}{\pgfqpoint{4.181776in}{1.355689in}}%
\pgfpathcurveto{\pgfqpoint{4.186820in}{1.350646in}}{\pgfqpoint{4.193661in}{1.347812in}}{\pgfqpoint{4.200794in}{1.347812in}}%
\pgfpathclose%
\pgfusepath{stroke,fill}%
\end{pgfscope}%
\begin{pgfscope}%
\pgfpathrectangle{\pgfqpoint{2.867647in}{0.500000in}}{\pgfqpoint{1.764706in}{1.700000in}}%
\pgfusepath{clip}%
\pgfsetbuttcap%
\pgfsetroundjoin%
\definecolor{currentfill}{rgb}{0.978376,0.897317,0.831308}%
\pgfsetfillcolor{currentfill}%
\pgfsetlinewidth{0.311001pt}%
\definecolor{currentstroke}{rgb}{1.000000,1.000000,1.000000}%
\pgfsetstrokecolor{currentstroke}%
\pgfsetdash{}{0pt}%
\pgfpathmoveto{\pgfqpoint{4.133620in}{1.068295in}}%
\pgfpathcurveto{\pgfqpoint{4.140753in}{1.068295in}}{\pgfqpoint{4.147594in}{1.071129in}}{\pgfqpoint{4.152638in}{1.076172in}}%
\pgfpathcurveto{\pgfqpoint{4.157682in}{1.081216in}}{\pgfqpoint{4.160515in}{1.088058in}}{\pgfqpoint{4.160515in}{1.095190in}}%
\pgfpathcurveto{\pgfqpoint{4.160515in}{1.102323in}}{\pgfqpoint{4.157682in}{1.109165in}}{\pgfqpoint{4.152638in}{1.114208in}}%
\pgfpathcurveto{\pgfqpoint{4.147594in}{1.119252in}}{\pgfqpoint{4.140753in}{1.122086in}}{\pgfqpoint{4.133620in}{1.122086in}}%
\pgfpathcurveto{\pgfqpoint{4.126487in}{1.122086in}}{\pgfqpoint{4.119645in}{1.119252in}}{\pgfqpoint{4.114602in}{1.114208in}}%
\pgfpathcurveto{\pgfqpoint{4.109558in}{1.109165in}}{\pgfqpoint{4.106724in}{1.102323in}}{\pgfqpoint{4.106724in}{1.095190in}}%
\pgfpathcurveto{\pgfqpoint{4.106724in}{1.088058in}}{\pgfqpoint{4.109558in}{1.081216in}}{\pgfqpoint{4.114602in}{1.076172in}}%
\pgfpathcurveto{\pgfqpoint{4.119645in}{1.071129in}}{\pgfqpoint{4.126487in}{1.068295in}}{\pgfqpoint{4.133620in}{1.068295in}}%
\pgfpathclose%
\pgfusepath{stroke,fill}%
\end{pgfscope}%
\begin{pgfscope}%
\pgfpathrectangle{\pgfqpoint{2.867647in}{0.500000in}}{\pgfqpoint{1.764706in}{1.700000in}}%
\pgfusepath{clip}%
\pgfsetbuttcap%
\pgfsetroundjoin%
\definecolor{currentfill}{rgb}{0.979124,0.903132,0.839793}%
\pgfsetfillcolor{currentfill}%
\pgfsetlinewidth{0.311001pt}%
\definecolor{currentstroke}{rgb}{1.000000,1.000000,1.000000}%
\pgfsetstrokecolor{currentstroke}%
\pgfsetdash{}{0pt}%
\pgfpathmoveto{\pgfqpoint{4.160281in}{1.348185in}}%
\pgfpathcurveto{\pgfqpoint{4.167413in}{1.348185in}}{\pgfqpoint{4.174255in}{1.351019in}}{\pgfqpoint{4.179299in}{1.356063in}}%
\pgfpathcurveto{\pgfqpoint{4.184342in}{1.361106in}}{\pgfqpoint{4.187176in}{1.367948in}}{\pgfqpoint{4.187176in}{1.375081in}}%
\pgfpathcurveto{\pgfqpoint{4.187176in}{1.382214in}}{\pgfqpoint{4.184342in}{1.389055in}}{\pgfqpoint{4.179299in}{1.394099in}}%
\pgfpathcurveto{\pgfqpoint{4.174255in}{1.399143in}}{\pgfqpoint{4.167413in}{1.401976in}}{\pgfqpoint{4.160281in}{1.401976in}}%
\pgfpathcurveto{\pgfqpoint{4.153148in}{1.401976in}}{\pgfqpoint{4.146306in}{1.399143in}}{\pgfqpoint{4.141262in}{1.394099in}}%
\pgfpathcurveto{\pgfqpoint{4.136219in}{1.389055in}}{\pgfqpoint{4.133385in}{1.382214in}}{\pgfqpoint{4.133385in}{1.375081in}}%
\pgfpathcurveto{\pgfqpoint{4.133385in}{1.367948in}}{\pgfqpoint{4.136219in}{1.361106in}}{\pgfqpoint{4.141262in}{1.356063in}}%
\pgfpathcurveto{\pgfqpoint{4.146306in}{1.351019in}}{\pgfqpoint{4.153148in}{1.348185in}}{\pgfqpoint{4.160281in}{1.348185in}}%
\pgfpathclose%
\pgfusepath{stroke,fill}%
\end{pgfscope}%
\begin{pgfscope}%
\pgfpathrectangle{\pgfqpoint{2.867647in}{0.500000in}}{\pgfqpoint{1.764706in}{1.700000in}}%
\pgfusepath{clip}%
\pgfsetbuttcap%
\pgfsetroundjoin%
\definecolor{currentfill}{rgb}{0.974412,0.862387,0.780156}%
\pgfsetfillcolor{currentfill}%
\pgfsetlinewidth{0.311001pt}%
\definecolor{currentstroke}{rgb}{1.000000,1.000000,1.000000}%
\pgfsetstrokecolor{currentstroke}%
\pgfsetdash{}{0pt}%
\pgfpathmoveto{\pgfqpoint{4.226959in}{1.496034in}}%
\pgfpathcurveto{\pgfqpoint{4.234092in}{1.496034in}}{\pgfqpoint{4.240933in}{1.498868in}}{\pgfqpoint{4.245977in}{1.503912in}}%
\pgfpathcurveto{\pgfqpoint{4.251021in}{1.508956in}}{\pgfqpoint{4.253854in}{1.515797in}}{\pgfqpoint{4.253854in}{1.522930in}}%
\pgfpathcurveto{\pgfqpoint{4.253854in}{1.530063in}}{\pgfqpoint{4.251021in}{1.536904in}}{\pgfqpoint{4.245977in}{1.541948in}}%
\pgfpathcurveto{\pgfqpoint{4.240933in}{1.546992in}}{\pgfqpoint{4.234092in}{1.549826in}}{\pgfqpoint{4.226959in}{1.549826in}}%
\pgfpathcurveto{\pgfqpoint{4.219826in}{1.549826in}}{\pgfqpoint{4.212984in}{1.546992in}}{\pgfqpoint{4.207941in}{1.541948in}}%
\pgfpathcurveto{\pgfqpoint{4.202897in}{1.536904in}}{\pgfqpoint{4.200063in}{1.530063in}}{\pgfqpoint{4.200063in}{1.522930in}}%
\pgfpathcurveto{\pgfqpoint{4.200063in}{1.515797in}}{\pgfqpoint{4.202897in}{1.508956in}}{\pgfqpoint{4.207941in}{1.503912in}}%
\pgfpathcurveto{\pgfqpoint{4.212984in}{1.498868in}}{\pgfqpoint{4.219826in}{1.496034in}}{\pgfqpoint{4.226959in}{1.496034in}}%
\pgfpathclose%
\pgfusepath{stroke,fill}%
\end{pgfscope}%
\begin{pgfscope}%
\pgfpathrectangle{\pgfqpoint{2.867647in}{0.500000in}}{\pgfqpoint{1.764706in}{1.700000in}}%
\pgfusepath{clip}%
\pgfsetbuttcap%
\pgfsetroundjoin%
\definecolor{currentfill}{rgb}{0.972201,0.839051,0.745789}%
\pgfsetfillcolor{currentfill}%
\pgfsetlinewidth{0.311001pt}%
\definecolor{currentstroke}{rgb}{1.000000,1.000000,1.000000}%
\pgfsetstrokecolor{currentstroke}%
\pgfsetdash{}{0pt}%
\pgfpathmoveto{\pgfqpoint{4.152680in}{1.647580in}}%
\pgfpathcurveto{\pgfqpoint{4.159813in}{1.647580in}}{\pgfqpoint{4.166655in}{1.650414in}}{\pgfqpoint{4.171698in}{1.655457in}}%
\pgfpathcurveto{\pgfqpoint{4.176742in}{1.660501in}}{\pgfqpoint{4.179576in}{1.667343in}}{\pgfqpoint{4.179576in}{1.674475in}}%
\pgfpathcurveto{\pgfqpoint{4.179576in}{1.681608in}}{\pgfqpoint{4.176742in}{1.688450in}}{\pgfqpoint{4.171698in}{1.693493in}}%
\pgfpathcurveto{\pgfqpoint{4.166655in}{1.698537in}}{\pgfqpoint{4.159813in}{1.701371in}}{\pgfqpoint{4.152680in}{1.701371in}}%
\pgfpathcurveto{\pgfqpoint{4.145547in}{1.701371in}}{\pgfqpoint{4.138706in}{1.698537in}}{\pgfqpoint{4.133662in}{1.693493in}}%
\pgfpathcurveto{\pgfqpoint{4.128618in}{1.688450in}}{\pgfqpoint{4.125784in}{1.681608in}}{\pgfqpoint{4.125784in}{1.674475in}}%
\pgfpathcurveto{\pgfqpoint{4.125784in}{1.667343in}}{\pgfqpoint{4.128618in}{1.660501in}}{\pgfqpoint{4.133662in}{1.655457in}}%
\pgfpathcurveto{\pgfqpoint{4.138706in}{1.650414in}}{\pgfqpoint{4.145547in}{1.647580in}}{\pgfqpoint{4.152680in}{1.647580in}}%
\pgfpathclose%
\pgfusepath{stroke,fill}%
\end{pgfscope}%
\begin{pgfscope}%
\pgfpathrectangle{\pgfqpoint{2.867647in}{0.500000in}}{\pgfqpoint{1.764706in}{1.700000in}}%
\pgfusepath{clip}%
\pgfsetbuttcap%
\pgfsetroundjoin%
\definecolor{currentfill}{rgb}{0.979124,0.903132,0.839793}%
\pgfsetfillcolor{currentfill}%
\pgfsetlinewidth{0.311001pt}%
\definecolor{currentstroke}{rgb}{1.000000,1.000000,1.000000}%
\pgfsetstrokecolor{currentstroke}%
\pgfsetdash{}{0pt}%
\pgfpathmoveto{\pgfqpoint{4.150679in}{1.268130in}}%
\pgfpathcurveto{\pgfqpoint{4.157812in}{1.268130in}}{\pgfqpoint{4.164654in}{1.270964in}}{\pgfqpoint{4.169697in}{1.276008in}}%
\pgfpathcurveto{\pgfqpoint{4.174741in}{1.281051in}}{\pgfqpoint{4.177575in}{1.287893in}}{\pgfqpoint{4.177575in}{1.295026in}}%
\pgfpathcurveto{\pgfqpoint{4.177575in}{1.302159in}}{\pgfqpoint{4.174741in}{1.309000in}}{\pgfqpoint{4.169697in}{1.314044in}}%
\pgfpathcurveto{\pgfqpoint{4.164654in}{1.319088in}}{\pgfqpoint{4.157812in}{1.321921in}}{\pgfqpoint{4.150679in}{1.321921in}}%
\pgfpathcurveto{\pgfqpoint{4.143546in}{1.321921in}}{\pgfqpoint{4.136705in}{1.319088in}}{\pgfqpoint{4.131661in}{1.314044in}}%
\pgfpathcurveto{\pgfqpoint{4.126617in}{1.309000in}}{\pgfqpoint{4.123784in}{1.302159in}}{\pgfqpoint{4.123784in}{1.295026in}}%
\pgfpathcurveto{\pgfqpoint{4.123784in}{1.287893in}}{\pgfqpoint{4.126617in}{1.281051in}}{\pgfqpoint{4.131661in}{1.276008in}}%
\pgfpathcurveto{\pgfqpoint{4.136705in}{1.270964in}}{\pgfqpoint{4.143546in}{1.268130in}}{\pgfqpoint{4.150679in}{1.268130in}}%
\pgfpathclose%
\pgfusepath{stroke,fill}%
\end{pgfscope}%
\begin{pgfscope}%
\pgfpathrectangle{\pgfqpoint{2.867647in}{0.500000in}}{\pgfqpoint{1.764706in}{1.700000in}}%
\pgfusepath{clip}%
\pgfsetbuttcap%
\pgfsetroundjoin%
\definecolor{currentfill}{rgb}{0.981377,0.920617,0.865369}%
\pgfsetfillcolor{currentfill}%
\pgfsetlinewidth{0.311001pt}%
\definecolor{currentstroke}{rgb}{1.000000,1.000000,1.000000}%
\pgfsetstrokecolor{currentstroke}%
\pgfsetdash{}{0pt}%
\pgfpathmoveto{\pgfqpoint{4.191142in}{1.203787in}}%
\pgfpathcurveto{\pgfqpoint{4.198275in}{1.203787in}}{\pgfqpoint{4.205116in}{1.206620in}}{\pgfqpoint{4.210160in}{1.211664in}}%
\pgfpathcurveto{\pgfqpoint{4.215204in}{1.216708in}}{\pgfqpoint{4.218038in}{1.223549in}}{\pgfqpoint{4.218038in}{1.230682in}}%
\pgfpathcurveto{\pgfqpoint{4.218038in}{1.237815in}}{\pgfqpoint{4.215204in}{1.244657in}}{\pgfqpoint{4.210160in}{1.249700in}}%
\pgfpathcurveto{\pgfqpoint{4.205116in}{1.254744in}}{\pgfqpoint{4.198275in}{1.257578in}}{\pgfqpoint{4.191142in}{1.257578in}}%
\pgfpathcurveto{\pgfqpoint{4.184009in}{1.257578in}}{\pgfqpoint{4.177167in}{1.254744in}}{\pgfqpoint{4.172124in}{1.249700in}}%
\pgfpathcurveto{\pgfqpoint{4.167080in}{1.244657in}}{\pgfqpoint{4.164246in}{1.237815in}}{\pgfqpoint{4.164246in}{1.230682in}}%
\pgfpathcurveto{\pgfqpoint{4.164246in}{1.223549in}}{\pgfqpoint{4.167080in}{1.216708in}}{\pgfqpoint{4.172124in}{1.211664in}}%
\pgfpathcurveto{\pgfqpoint{4.177167in}{1.206620in}}{\pgfqpoint{4.184009in}{1.203787in}}{\pgfqpoint{4.191142in}{1.203787in}}%
\pgfpathclose%
\pgfusepath{stroke,fill}%
\end{pgfscope}%
\begin{pgfscope}%
\pgfpathrectangle{\pgfqpoint{2.867647in}{0.500000in}}{\pgfqpoint{1.764706in}{1.700000in}}%
\pgfusepath{clip}%
\pgfsetbuttcap%
\pgfsetroundjoin%
\definecolor{currentfill}{rgb}{0.972201,0.839051,0.745789}%
\pgfsetfillcolor{currentfill}%
\pgfsetlinewidth{0.311001pt}%
\definecolor{currentstroke}{rgb}{1.000000,1.000000,1.000000}%
\pgfsetstrokecolor{currentstroke}%
\pgfsetdash{}{0pt}%
\pgfpathmoveto{\pgfqpoint{4.243684in}{1.152588in}}%
\pgfpathcurveto{\pgfqpoint{4.250817in}{1.152588in}}{\pgfqpoint{4.257659in}{1.155422in}}{\pgfqpoint{4.262702in}{1.160466in}}%
\pgfpathcurveto{\pgfqpoint{4.267746in}{1.165509in}}{\pgfqpoint{4.270580in}{1.172351in}}{\pgfqpoint{4.270580in}{1.179484in}}%
\pgfpathcurveto{\pgfqpoint{4.270580in}{1.186617in}}{\pgfqpoint{4.267746in}{1.193458in}}{\pgfqpoint{4.262702in}{1.198502in}}%
\pgfpathcurveto{\pgfqpoint{4.257659in}{1.203546in}}{\pgfqpoint{4.250817in}{1.206380in}}{\pgfqpoint{4.243684in}{1.206380in}}%
\pgfpathcurveto{\pgfqpoint{4.236551in}{1.206380in}}{\pgfqpoint{4.229710in}{1.203546in}}{\pgfqpoint{4.224666in}{1.198502in}}%
\pgfpathcurveto{\pgfqpoint{4.219622in}{1.193458in}}{\pgfqpoint{4.216789in}{1.186617in}}{\pgfqpoint{4.216789in}{1.179484in}}%
\pgfpathcurveto{\pgfqpoint{4.216789in}{1.172351in}}{\pgfqpoint{4.219622in}{1.165509in}}{\pgfqpoint{4.224666in}{1.160466in}}%
\pgfpathcurveto{\pgfqpoint{4.229710in}{1.155422in}}{\pgfqpoint{4.236551in}{1.152588in}}{\pgfqpoint{4.243684in}{1.152588in}}%
\pgfpathclose%
\pgfusepath{stroke,fill}%
\end{pgfscope}%
\begin{pgfscope}%
\pgfpathrectangle{\pgfqpoint{2.867647in}{0.500000in}}{\pgfqpoint{1.764706in}{1.700000in}}%
\pgfusepath{clip}%
\pgfsetbuttcap%
\pgfsetroundjoin%
\definecolor{currentfill}{rgb}{0.972201,0.839051,0.745789}%
\pgfsetfillcolor{currentfill}%
\pgfsetlinewidth{0.311001pt}%
\definecolor{currentstroke}{rgb}{1.000000,1.000000,1.000000}%
\pgfsetstrokecolor{currentstroke}%
\pgfsetdash{}{0pt}%
\pgfpathmoveto{\pgfqpoint{4.065748in}{0.994879in}}%
\pgfpathcurveto{\pgfqpoint{4.072881in}{0.994879in}}{\pgfqpoint{4.079723in}{0.997713in}}{\pgfqpoint{4.084766in}{1.002757in}}%
\pgfpathcurveto{\pgfqpoint{4.089810in}{1.007800in}}{\pgfqpoint{4.092644in}{1.014642in}}{\pgfqpoint{4.092644in}{1.021775in}}%
\pgfpathcurveto{\pgfqpoint{4.092644in}{1.028908in}}{\pgfqpoint{4.089810in}{1.035749in}}{\pgfqpoint{4.084766in}{1.040793in}}%
\pgfpathcurveto{\pgfqpoint{4.079723in}{1.045837in}}{\pgfqpoint{4.072881in}{1.048671in}}{\pgfqpoint{4.065748in}{1.048671in}}%
\pgfpathcurveto{\pgfqpoint{4.058615in}{1.048671in}}{\pgfqpoint{4.051774in}{1.045837in}}{\pgfqpoint{4.046730in}{1.040793in}}%
\pgfpathcurveto{\pgfqpoint{4.041686in}{1.035749in}}{\pgfqpoint{4.038852in}{1.028908in}}{\pgfqpoint{4.038852in}{1.021775in}}%
\pgfpathcurveto{\pgfqpoint{4.038852in}{1.014642in}}{\pgfqpoint{4.041686in}{1.007800in}}{\pgfqpoint{4.046730in}{1.002757in}}%
\pgfpathcurveto{\pgfqpoint{4.051774in}{0.997713in}}{\pgfqpoint{4.058615in}{0.994879in}}{\pgfqpoint{4.065748in}{0.994879in}}%
\pgfpathclose%
\pgfusepath{stroke,fill}%
\end{pgfscope}%
\begin{pgfscope}%
\pgfpathrectangle{\pgfqpoint{2.867647in}{0.500000in}}{\pgfqpoint{1.764706in}{1.700000in}}%
\pgfusepath{clip}%
\pgfsetbuttcap%
\pgfsetroundjoin%
\definecolor{currentfill}{rgb}{0.934351,0.329284,0.247753}%
\pgfsetfillcolor{currentfill}%
\pgfsetlinewidth{0.311001pt}%
\definecolor{currentstroke}{rgb}{1.000000,1.000000,1.000000}%
\pgfsetstrokecolor{currentstroke}%
\pgfsetdash{}{0pt}%
\pgfpathmoveto{\pgfqpoint{4.361951in}{1.173055in}}%
\pgfpathcurveto{\pgfqpoint{4.369084in}{1.173055in}}{\pgfqpoint{4.375926in}{1.175889in}}{\pgfqpoint{4.380970in}{1.180932in}}%
\pgfpathcurveto{\pgfqpoint{4.386013in}{1.185976in}}{\pgfqpoint{4.388847in}{1.192818in}}{\pgfqpoint{4.388847in}{1.199951in}}%
\pgfpathcurveto{\pgfqpoint{4.388847in}{1.207083in}}{\pgfqpoint{4.386013in}{1.213925in}}{\pgfqpoint{4.380970in}{1.218969in}}%
\pgfpathcurveto{\pgfqpoint{4.375926in}{1.224012in}}{\pgfqpoint{4.369084in}{1.226846in}}{\pgfqpoint{4.361951in}{1.226846in}}%
\pgfpathcurveto{\pgfqpoint{4.354819in}{1.226846in}}{\pgfqpoint{4.347977in}{1.224012in}}{\pgfqpoint{4.342933in}{1.218969in}}%
\pgfpathcurveto{\pgfqpoint{4.337890in}{1.213925in}}{\pgfqpoint{4.335056in}{1.207083in}}{\pgfqpoint{4.335056in}{1.199951in}}%
\pgfpathcurveto{\pgfqpoint{4.335056in}{1.192818in}}{\pgfqpoint{4.337890in}{1.185976in}}{\pgfqpoint{4.342933in}{1.180932in}}%
\pgfpathcurveto{\pgfqpoint{4.347977in}{1.175889in}}{\pgfqpoint{4.354819in}{1.173055in}}{\pgfqpoint{4.361951in}{1.173055in}}%
\pgfpathclose%
\pgfusepath{stroke,fill}%
\end{pgfscope}%
\begin{pgfscope}%
\pgfpathrectangle{\pgfqpoint{2.867647in}{0.500000in}}{\pgfqpoint{1.764706in}{1.700000in}}%
\pgfusepath{clip}%
\pgfsetbuttcap%
\pgfsetroundjoin%
\definecolor{currentfill}{rgb}{0.978376,0.897317,0.831308}%
\pgfsetfillcolor{currentfill}%
\pgfsetlinewidth{0.311001pt}%
\definecolor{currentstroke}{rgb}{1.000000,1.000000,1.000000}%
\pgfsetstrokecolor{currentstroke}%
\pgfsetdash{}{0pt}%
\pgfpathmoveto{\pgfqpoint{4.148557in}{1.388635in}}%
\pgfpathcurveto{\pgfqpoint{4.155690in}{1.388635in}}{\pgfqpoint{4.162532in}{1.391468in}}{\pgfqpoint{4.167575in}{1.396512in}}%
\pgfpathcurveto{\pgfqpoint{4.172619in}{1.401556in}}{\pgfqpoint{4.175453in}{1.408397in}}{\pgfqpoint{4.175453in}{1.415530in}}%
\pgfpathcurveto{\pgfqpoint{4.175453in}{1.422663in}}{\pgfqpoint{4.172619in}{1.429505in}}{\pgfqpoint{4.167575in}{1.434548in}}%
\pgfpathcurveto{\pgfqpoint{4.162532in}{1.439592in}}{\pgfqpoint{4.155690in}{1.442426in}}{\pgfqpoint{4.148557in}{1.442426in}}%
\pgfpathcurveto{\pgfqpoint{4.141424in}{1.442426in}}{\pgfqpoint{4.134583in}{1.439592in}}{\pgfqpoint{4.129539in}{1.434548in}}%
\pgfpathcurveto{\pgfqpoint{4.124495in}{1.429505in}}{\pgfqpoint{4.121662in}{1.422663in}}{\pgfqpoint{4.121662in}{1.415530in}}%
\pgfpathcurveto{\pgfqpoint{4.121662in}{1.408397in}}{\pgfqpoint{4.124495in}{1.401556in}}{\pgfqpoint{4.129539in}{1.396512in}}%
\pgfpathcurveto{\pgfqpoint{4.134583in}{1.391468in}}{\pgfqpoint{4.141424in}{1.388635in}}{\pgfqpoint{4.148557in}{1.388635in}}%
\pgfpathclose%
\pgfusepath{stroke,fill}%
\end{pgfscope}%
\begin{pgfscope}%
\pgfpathrectangle{\pgfqpoint{2.867647in}{0.500000in}}{\pgfqpoint{1.764706in}{1.700000in}}%
\pgfusepath{clip}%
\pgfsetbuttcap%
\pgfsetroundjoin%
\definecolor{currentfill}{rgb}{0.979891,0.908948,0.848279}%
\pgfsetfillcolor{currentfill}%
\pgfsetlinewidth{0.311001pt}%
\definecolor{currentstroke}{rgb}{1.000000,1.000000,1.000000}%
\pgfsetstrokecolor{currentstroke}%
\pgfsetdash{}{0pt}%
\pgfpathmoveto{\pgfqpoint{4.196995in}{1.440356in}}%
\pgfpathcurveto{\pgfqpoint{4.204128in}{1.440356in}}{\pgfqpoint{4.210969in}{1.443190in}}{\pgfqpoint{4.216013in}{1.448234in}}%
\pgfpathcurveto{\pgfqpoint{4.221057in}{1.453278in}}{\pgfqpoint{4.223890in}{1.460119in}}{\pgfqpoint{4.223890in}{1.467252in}}%
\pgfpathcurveto{\pgfqpoint{4.223890in}{1.474385in}}{\pgfqpoint{4.221057in}{1.481227in}}{\pgfqpoint{4.216013in}{1.486270in}}%
\pgfpathcurveto{\pgfqpoint{4.210969in}{1.491314in}}{\pgfqpoint{4.204128in}{1.494148in}}{\pgfqpoint{4.196995in}{1.494148in}}%
\pgfpathcurveto{\pgfqpoint{4.189862in}{1.494148in}}{\pgfqpoint{4.183020in}{1.491314in}}{\pgfqpoint{4.177977in}{1.486270in}}%
\pgfpathcurveto{\pgfqpoint{4.172933in}{1.481227in}}{\pgfqpoint{4.170099in}{1.474385in}}{\pgfqpoint{4.170099in}{1.467252in}}%
\pgfpathcurveto{\pgfqpoint{4.170099in}{1.460119in}}{\pgfqpoint{4.172933in}{1.453278in}}{\pgfqpoint{4.177977in}{1.448234in}}%
\pgfpathcurveto{\pgfqpoint{4.183020in}{1.443190in}}{\pgfqpoint{4.189862in}{1.440356in}}{\pgfqpoint{4.196995in}{1.440356in}}%
\pgfpathclose%
\pgfusepath{stroke,fill}%
\end{pgfscope}%
\begin{pgfscope}%
\pgfpathrectangle{\pgfqpoint{2.867647in}{0.500000in}}{\pgfqpoint{1.764706in}{1.700000in}}%
\pgfusepath{clip}%
\pgfsetbuttcap%
\pgfsetroundjoin%
\definecolor{currentfill}{rgb}{0.977657,0.891500,0.822809}%
\pgfsetfillcolor{currentfill}%
\pgfsetlinewidth{0.311001pt}%
\definecolor{currentstroke}{rgb}{1.000000,1.000000,1.000000}%
\pgfsetstrokecolor{currentstroke}%
\pgfsetdash{}{0pt}%
\pgfpathmoveto{\pgfqpoint{4.114962in}{1.504170in}}%
\pgfpathcurveto{\pgfqpoint{4.122095in}{1.504170in}}{\pgfqpoint{4.128936in}{1.507004in}}{\pgfqpoint{4.133980in}{1.512047in}}%
\pgfpathcurveto{\pgfqpoint{4.139024in}{1.517091in}}{\pgfqpoint{4.141857in}{1.523933in}}{\pgfqpoint{4.141857in}{1.531066in}}%
\pgfpathcurveto{\pgfqpoint{4.141857in}{1.538198in}}{\pgfqpoint{4.139024in}{1.545040in}}{\pgfqpoint{4.133980in}{1.550084in}}%
\pgfpathcurveto{\pgfqpoint{4.128936in}{1.555127in}}{\pgfqpoint{4.122095in}{1.557961in}}{\pgfqpoint{4.114962in}{1.557961in}}%
\pgfpathcurveto{\pgfqpoint{4.107829in}{1.557961in}}{\pgfqpoint{4.100987in}{1.555127in}}{\pgfqpoint{4.095944in}{1.550084in}}%
\pgfpathcurveto{\pgfqpoint{4.090900in}{1.545040in}}{\pgfqpoint{4.088066in}{1.538198in}}{\pgfqpoint{4.088066in}{1.531066in}}%
\pgfpathcurveto{\pgfqpoint{4.088066in}{1.523933in}}{\pgfqpoint{4.090900in}{1.517091in}}{\pgfqpoint{4.095944in}{1.512047in}}%
\pgfpathcurveto{\pgfqpoint{4.100987in}{1.507004in}}{\pgfqpoint{4.107829in}{1.504170in}}{\pgfqpoint{4.114962in}{1.504170in}}%
\pgfpathclose%
\pgfusepath{stroke,fill}%
\end{pgfscope}%
\begin{pgfscope}%
\pgfpathrectangle{\pgfqpoint{2.867647in}{0.500000in}}{\pgfqpoint{1.764706in}{1.700000in}}%
\pgfusepath{clip}%
\pgfsetbuttcap%
\pgfsetroundjoin%
\definecolor{currentfill}{rgb}{0.962532,0.599594,0.438051}%
\pgfsetfillcolor{currentfill}%
\pgfsetlinewidth{0.311001pt}%
\definecolor{currentstroke}{rgb}{1.000000,1.000000,1.000000}%
\pgfsetstrokecolor{currentstroke}%
\pgfsetdash{}{0pt}%
\pgfpathmoveto{\pgfqpoint{4.318573in}{1.453392in}}%
\pgfpathcurveto{\pgfqpoint{4.325706in}{1.453392in}}{\pgfqpoint{4.332547in}{1.456226in}}{\pgfqpoint{4.337591in}{1.461269in}}%
\pgfpathcurveto{\pgfqpoint{4.342635in}{1.466313in}}{\pgfqpoint{4.345468in}{1.473155in}}{\pgfqpoint{4.345468in}{1.480287in}}%
\pgfpathcurveto{\pgfqpoint{4.345468in}{1.487420in}}{\pgfqpoint{4.342635in}{1.494262in}}{\pgfqpoint{4.337591in}{1.499306in}}%
\pgfpathcurveto{\pgfqpoint{4.332547in}{1.504349in}}{\pgfqpoint{4.325706in}{1.507183in}}{\pgfqpoint{4.318573in}{1.507183in}}%
\pgfpathcurveto{\pgfqpoint{4.311440in}{1.507183in}}{\pgfqpoint{4.304598in}{1.504349in}}{\pgfqpoint{4.299555in}{1.499306in}}%
\pgfpathcurveto{\pgfqpoint{4.294511in}{1.494262in}}{\pgfqpoint{4.291677in}{1.487420in}}{\pgfqpoint{4.291677in}{1.480287in}}%
\pgfpathcurveto{\pgfqpoint{4.291677in}{1.473155in}}{\pgfqpoint{4.294511in}{1.466313in}}{\pgfqpoint{4.299555in}{1.461269in}}%
\pgfpathcurveto{\pgfqpoint{4.304598in}{1.456226in}}{\pgfqpoint{4.311440in}{1.453392in}}{\pgfqpoint{4.318573in}{1.453392in}}%
\pgfpathclose%
\pgfusepath{stroke,fill}%
\end{pgfscope}%
\begin{pgfscope}%
\pgfpathrectangle{\pgfqpoint{2.867647in}{0.500000in}}{\pgfqpoint{1.764706in}{1.700000in}}%
\pgfusepath{clip}%
\pgfsetbuttcap%
\pgfsetroundjoin%
\definecolor{currentfill}{rgb}{0.973832,0.856556,0.771584}%
\pgfsetfillcolor{currentfill}%
\pgfsetlinewidth{0.311001pt}%
\definecolor{currentstroke}{rgb}{1.000000,1.000000,1.000000}%
\pgfsetstrokecolor{currentstroke}%
\pgfsetdash{}{0pt}%
\pgfpathmoveto{\pgfqpoint{4.111516in}{1.206918in}}%
\pgfpathcurveto{\pgfqpoint{4.118649in}{1.206918in}}{\pgfqpoint{4.125491in}{1.209752in}}{\pgfqpoint{4.130535in}{1.214796in}}%
\pgfpathcurveto{\pgfqpoint{4.135578in}{1.219839in}}{\pgfqpoint{4.138412in}{1.226681in}}{\pgfqpoint{4.138412in}{1.233814in}}%
\pgfpathcurveto{\pgfqpoint{4.138412in}{1.240947in}}{\pgfqpoint{4.135578in}{1.247788in}}{\pgfqpoint{4.130535in}{1.252832in}}%
\pgfpathcurveto{\pgfqpoint{4.125491in}{1.257876in}}{\pgfqpoint{4.118649in}{1.260709in}}{\pgfqpoint{4.111516in}{1.260709in}}%
\pgfpathcurveto{\pgfqpoint{4.104384in}{1.260709in}}{\pgfqpoint{4.097542in}{1.257876in}}{\pgfqpoint{4.092498in}{1.252832in}}%
\pgfpathcurveto{\pgfqpoint{4.087455in}{1.247788in}}{\pgfqpoint{4.084621in}{1.240947in}}{\pgfqpoint{4.084621in}{1.233814in}}%
\pgfpathcurveto{\pgfqpoint{4.084621in}{1.226681in}}{\pgfqpoint{4.087455in}{1.219839in}}{\pgfqpoint{4.092498in}{1.214796in}}%
\pgfpathcurveto{\pgfqpoint{4.097542in}{1.209752in}}{\pgfqpoint{4.104384in}{1.206918in}}{\pgfqpoint{4.111516in}{1.206918in}}%
\pgfpathclose%
\pgfusepath{stroke,fill}%
\end{pgfscope}%
\begin{pgfscope}%
\pgfpathrectangle{\pgfqpoint{2.867647in}{0.500000in}}{\pgfqpoint{1.764706in}{1.700000in}}%
\pgfusepath{clip}%
\pgfsetbuttcap%
\pgfsetroundjoin%
\definecolor{currentfill}{rgb}{0.966328,0.750560,0.616961}%
\pgfsetfillcolor{currentfill}%
\pgfsetlinewidth{0.311001pt}%
\definecolor{currentstroke}{rgb}{1.000000,1.000000,1.000000}%
\pgfsetstrokecolor{currentstroke}%
\pgfsetdash{}{0pt}%
\pgfpathmoveto{\pgfqpoint{4.069878in}{1.436804in}}%
\pgfpathcurveto{\pgfqpoint{4.077011in}{1.436804in}}{\pgfqpoint{4.083853in}{1.439638in}}{\pgfqpoint{4.088896in}{1.444682in}}%
\pgfpathcurveto{\pgfqpoint{4.093940in}{1.449725in}}{\pgfqpoint{4.096774in}{1.456567in}}{\pgfqpoint{4.096774in}{1.463700in}}%
\pgfpathcurveto{\pgfqpoint{4.096774in}{1.470833in}}{\pgfqpoint{4.093940in}{1.477674in}}{\pgfqpoint{4.088896in}{1.482718in}}%
\pgfpathcurveto{\pgfqpoint{4.083853in}{1.487762in}}{\pgfqpoint{4.077011in}{1.490595in}}{\pgfqpoint{4.069878in}{1.490595in}}%
\pgfpathcurveto{\pgfqpoint{4.062745in}{1.490595in}}{\pgfqpoint{4.055904in}{1.487762in}}{\pgfqpoint{4.050860in}{1.482718in}}%
\pgfpathcurveto{\pgfqpoint{4.045816in}{1.477674in}}{\pgfqpoint{4.042983in}{1.470833in}}{\pgfqpoint{4.042983in}{1.463700in}}%
\pgfpathcurveto{\pgfqpoint{4.042983in}{1.456567in}}{\pgfqpoint{4.045816in}{1.449725in}}{\pgfqpoint{4.050860in}{1.444682in}}%
\pgfpathcurveto{\pgfqpoint{4.055904in}{1.439638in}}{\pgfqpoint{4.062745in}{1.436804in}}{\pgfqpoint{4.069878in}{1.436804in}}%
\pgfpathclose%
\pgfusepath{stroke,fill}%
\end{pgfscope}%
\begin{pgfscope}%
\pgfpathrectangle{\pgfqpoint{2.867647in}{0.500000in}}{\pgfqpoint{1.764706in}{1.700000in}}%
\pgfusepath{clip}%
\pgfsetbuttcap%
\pgfsetroundjoin%
\definecolor{currentfill}{rgb}{0.973832,0.856556,0.771584}%
\pgfsetfillcolor{currentfill}%
\pgfsetlinewidth{0.311001pt}%
\definecolor{currentstroke}{rgb}{1.000000,1.000000,1.000000}%
\pgfsetstrokecolor{currentstroke}%
\pgfsetdash{}{0pt}%
\pgfpathmoveto{\pgfqpoint{4.081728in}{1.531038in}}%
\pgfpathcurveto{\pgfqpoint{4.088861in}{1.531038in}}{\pgfqpoint{4.095702in}{1.533872in}}{\pgfqpoint{4.100746in}{1.538915in}}%
\pgfpathcurveto{\pgfqpoint{4.105790in}{1.543959in}}{\pgfqpoint{4.108624in}{1.550801in}}{\pgfqpoint{4.108624in}{1.557934in}}%
\pgfpathcurveto{\pgfqpoint{4.108624in}{1.565066in}}{\pgfqpoint{4.105790in}{1.571908in}}{\pgfqpoint{4.100746in}{1.576952in}}%
\pgfpathcurveto{\pgfqpoint{4.095702in}{1.581995in}}{\pgfqpoint{4.088861in}{1.584829in}}{\pgfqpoint{4.081728in}{1.584829in}}%
\pgfpathcurveto{\pgfqpoint{4.074595in}{1.584829in}}{\pgfqpoint{4.067753in}{1.581995in}}{\pgfqpoint{4.062710in}{1.576952in}}%
\pgfpathcurveto{\pgfqpoint{4.057666in}{1.571908in}}{\pgfqpoint{4.054832in}{1.565066in}}{\pgfqpoint{4.054832in}{1.557934in}}%
\pgfpathcurveto{\pgfqpoint{4.054832in}{1.550801in}}{\pgfqpoint{4.057666in}{1.543959in}}{\pgfqpoint{4.062710in}{1.538915in}}%
\pgfpathcurveto{\pgfqpoint{4.067753in}{1.533872in}}{\pgfqpoint{4.074595in}{1.531038in}}{\pgfqpoint{4.081728in}{1.531038in}}%
\pgfpathclose%
\pgfusepath{stroke,fill}%
\end{pgfscope}%
\begin{pgfscope}%
\pgfpathrectangle{\pgfqpoint{2.867647in}{0.500000in}}{\pgfqpoint{1.764706in}{1.700000in}}%
\pgfusepath{clip}%
\pgfsetbuttcap%
\pgfsetroundjoin%
\definecolor{currentfill}{rgb}{0.979891,0.908948,0.848279}%
\pgfsetfillcolor{currentfill}%
\pgfsetlinewidth{0.311001pt}%
\definecolor{currentstroke}{rgb}{1.000000,1.000000,1.000000}%
\pgfsetstrokecolor{currentstroke}%
\pgfsetdash{}{0pt}%
\pgfpathmoveto{\pgfqpoint{4.144555in}{1.108754in}}%
\pgfpathcurveto{\pgfqpoint{4.151688in}{1.108754in}}{\pgfqpoint{4.158530in}{1.111588in}}{\pgfqpoint{4.163573in}{1.116631in}}%
\pgfpathcurveto{\pgfqpoint{4.168617in}{1.121675in}}{\pgfqpoint{4.171451in}{1.128517in}}{\pgfqpoint{4.171451in}{1.135650in}}%
\pgfpathcurveto{\pgfqpoint{4.171451in}{1.142782in}}{\pgfqpoint{4.168617in}{1.149624in}}{\pgfqpoint{4.163573in}{1.154668in}}%
\pgfpathcurveto{\pgfqpoint{4.158530in}{1.159711in}}{\pgfqpoint{4.151688in}{1.162545in}}{\pgfqpoint{4.144555in}{1.162545in}}%
\pgfpathcurveto{\pgfqpoint{4.137422in}{1.162545in}}{\pgfqpoint{4.130581in}{1.159711in}}{\pgfqpoint{4.125537in}{1.154668in}}%
\pgfpathcurveto{\pgfqpoint{4.120493in}{1.149624in}}{\pgfqpoint{4.117659in}{1.142782in}}{\pgfqpoint{4.117659in}{1.135650in}}%
\pgfpathcurveto{\pgfqpoint{4.117659in}{1.128517in}}{\pgfqpoint{4.120493in}{1.121675in}}{\pgfqpoint{4.125537in}{1.116631in}}%
\pgfpathcurveto{\pgfqpoint{4.130581in}{1.111588in}}{\pgfqpoint{4.137422in}{1.108754in}}{\pgfqpoint{4.144555in}{1.108754in}}%
\pgfpathclose%
\pgfusepath{stroke,fill}%
\end{pgfscope}%
\begin{pgfscope}%
\pgfpathrectangle{\pgfqpoint{2.867647in}{0.500000in}}{\pgfqpoint{1.764706in}{1.700000in}}%
\pgfusepath{clip}%
\pgfsetbuttcap%
\pgfsetroundjoin%
\definecolor{currentfill}{rgb}{0.980678,0.914765,0.856766}%
\pgfsetfillcolor{currentfill}%
\pgfsetlinewidth{0.311001pt}%
\definecolor{currentstroke}{rgb}{1.000000,1.000000,1.000000}%
\pgfsetstrokecolor{currentstroke}%
\pgfsetdash{}{0pt}%
\pgfpathmoveto{\pgfqpoint{4.165375in}{1.413545in}}%
\pgfpathcurveto{\pgfqpoint{4.172507in}{1.413545in}}{\pgfqpoint{4.179349in}{1.416379in}}{\pgfqpoint{4.184393in}{1.421423in}}%
\pgfpathcurveto{\pgfqpoint{4.189436in}{1.426467in}}{\pgfqpoint{4.192270in}{1.433308in}}{\pgfqpoint{4.192270in}{1.440441in}}%
\pgfpathcurveto{\pgfqpoint{4.192270in}{1.447574in}}{\pgfqpoint{4.189436in}{1.454415in}}{\pgfqpoint{4.184393in}{1.459459in}}%
\pgfpathcurveto{\pgfqpoint{4.179349in}{1.464503in}}{\pgfqpoint{4.172507in}{1.467337in}}{\pgfqpoint{4.165375in}{1.467337in}}%
\pgfpathcurveto{\pgfqpoint{4.158242in}{1.467337in}}{\pgfqpoint{4.151400in}{1.464503in}}{\pgfqpoint{4.146356in}{1.459459in}}%
\pgfpathcurveto{\pgfqpoint{4.141313in}{1.454415in}}{\pgfqpoint{4.138479in}{1.447574in}}{\pgfqpoint{4.138479in}{1.440441in}}%
\pgfpathcurveto{\pgfqpoint{4.138479in}{1.433308in}}{\pgfqpoint{4.141313in}{1.426467in}}{\pgfqpoint{4.146356in}{1.421423in}}%
\pgfpathcurveto{\pgfqpoint{4.151400in}{1.416379in}}{\pgfqpoint{4.158242in}{1.413545in}}{\pgfqpoint{4.165375in}{1.413545in}}%
\pgfpathclose%
\pgfusepath{stroke,fill}%
\end{pgfscope}%
\begin{pgfscope}%
\pgfpathrectangle{\pgfqpoint{2.867647in}{0.500000in}}{\pgfqpoint{1.764706in}{1.700000in}}%
\pgfusepath{clip}%
\pgfsetbuttcap%
\pgfsetroundjoin%
\definecolor{currentfill}{rgb}{0.978376,0.897317,0.831308}%
\pgfsetfillcolor{currentfill}%
\pgfsetlinewidth{0.311001pt}%
\definecolor{currentstroke}{rgb}{1.000000,1.000000,1.000000}%
\pgfsetstrokecolor{currentstroke}%
\pgfsetdash{}{0pt}%
\pgfpathmoveto{\pgfqpoint{4.151383in}{1.400825in}}%
\pgfpathcurveto{\pgfqpoint{4.158516in}{1.400825in}}{\pgfqpoint{4.165357in}{1.403659in}}{\pgfqpoint{4.170401in}{1.408703in}}%
\pgfpathcurveto{\pgfqpoint{4.175445in}{1.413747in}}{\pgfqpoint{4.178279in}{1.420588in}}{\pgfqpoint{4.178279in}{1.427721in}}%
\pgfpathcurveto{\pgfqpoint{4.178279in}{1.434854in}}{\pgfqpoint{4.175445in}{1.441696in}}{\pgfqpoint{4.170401in}{1.446739in}}%
\pgfpathcurveto{\pgfqpoint{4.165357in}{1.451783in}}{\pgfqpoint{4.158516in}{1.454617in}}{\pgfqpoint{4.151383in}{1.454617in}}%
\pgfpathcurveto{\pgfqpoint{4.144250in}{1.454617in}}{\pgfqpoint{4.137408in}{1.451783in}}{\pgfqpoint{4.132365in}{1.446739in}}%
\pgfpathcurveto{\pgfqpoint{4.127321in}{1.441696in}}{\pgfqpoint{4.124487in}{1.434854in}}{\pgfqpoint{4.124487in}{1.427721in}}%
\pgfpathcurveto{\pgfqpoint{4.124487in}{1.420588in}}{\pgfqpoint{4.127321in}{1.413747in}}{\pgfqpoint{4.132365in}{1.408703in}}%
\pgfpathcurveto{\pgfqpoint{4.137408in}{1.403659in}}{\pgfqpoint{4.144250in}{1.400825in}}{\pgfqpoint{4.151383in}{1.400825in}}%
\pgfpathclose%
\pgfusepath{stroke,fill}%
\end{pgfscope}%
\begin{pgfscope}%
\pgfpathrectangle{\pgfqpoint{2.867647in}{0.500000in}}{\pgfqpoint{1.764706in}{1.700000in}}%
\pgfusepath{clip}%
\pgfsetbuttcap%
\pgfsetroundjoin%
\definecolor{currentfill}{rgb}{0.959229,0.533075,0.374889}%
\pgfsetfillcolor{currentfill}%
\pgfsetlinewidth{0.311001pt}%
\definecolor{currentstroke}{rgb}{1.000000,1.000000,1.000000}%
\pgfsetstrokecolor{currentstroke}%
\pgfsetdash{}{0pt}%
\pgfpathmoveto{\pgfqpoint{4.115367in}{1.804998in}}%
\pgfpathcurveto{\pgfqpoint{4.122500in}{1.804998in}}{\pgfqpoint{4.129341in}{1.807832in}}{\pgfqpoint{4.134385in}{1.812875in}}%
\pgfpathcurveto{\pgfqpoint{4.139429in}{1.817919in}}{\pgfqpoint{4.142262in}{1.824761in}}{\pgfqpoint{4.142262in}{1.831893in}}%
\pgfpathcurveto{\pgfqpoint{4.142262in}{1.839026in}}{\pgfqpoint{4.139429in}{1.845868in}}{\pgfqpoint{4.134385in}{1.850911in}}%
\pgfpathcurveto{\pgfqpoint{4.129341in}{1.855955in}}{\pgfqpoint{4.122500in}{1.858789in}}{\pgfqpoint{4.115367in}{1.858789in}}%
\pgfpathcurveto{\pgfqpoint{4.108234in}{1.858789in}}{\pgfqpoint{4.101392in}{1.855955in}}{\pgfqpoint{4.096349in}{1.850911in}}%
\pgfpathcurveto{\pgfqpoint{4.091305in}{1.845868in}}{\pgfqpoint{4.088471in}{1.839026in}}{\pgfqpoint{4.088471in}{1.831893in}}%
\pgfpathcurveto{\pgfqpoint{4.088471in}{1.824761in}}{\pgfqpoint{4.091305in}{1.817919in}}{\pgfqpoint{4.096349in}{1.812875in}}%
\pgfpathcurveto{\pgfqpoint{4.101392in}{1.807832in}}{\pgfqpoint{4.108234in}{1.804998in}}{\pgfqpoint{4.115367in}{1.804998in}}%
\pgfpathclose%
\pgfusepath{stroke,fill}%
\end{pgfscope}%
\begin{pgfscope}%
\pgfpathrectangle{\pgfqpoint{2.867647in}{0.500000in}}{\pgfqpoint{1.764706in}{1.700000in}}%
\pgfusepath{clip}%
\pgfsetbuttcap%
\pgfsetroundjoin%
\definecolor{currentfill}{rgb}{0.969359,0.803954,0.693832}%
\pgfsetfillcolor{currentfill}%
\pgfsetlinewidth{0.311001pt}%
\definecolor{currentstroke}{rgb}{1.000000,1.000000,1.000000}%
\pgfsetstrokecolor{currentstroke}%
\pgfsetdash{}{0pt}%
\pgfpathmoveto{\pgfqpoint{4.216524in}{1.590563in}}%
\pgfpathcurveto{\pgfqpoint{4.223657in}{1.590563in}}{\pgfqpoint{4.230499in}{1.593397in}}{\pgfqpoint{4.235542in}{1.598440in}}%
\pgfpathcurveto{\pgfqpoint{4.240586in}{1.603484in}}{\pgfqpoint{4.243420in}{1.610326in}}{\pgfqpoint{4.243420in}{1.617458in}}%
\pgfpathcurveto{\pgfqpoint{4.243420in}{1.624591in}}{\pgfqpoint{4.240586in}{1.631433in}}{\pgfqpoint{4.235542in}{1.636477in}}%
\pgfpathcurveto{\pgfqpoint{4.230499in}{1.641520in}}{\pgfqpoint{4.223657in}{1.644354in}}{\pgfqpoint{4.216524in}{1.644354in}}%
\pgfpathcurveto{\pgfqpoint{4.209391in}{1.644354in}}{\pgfqpoint{4.202550in}{1.641520in}}{\pgfqpoint{4.197506in}{1.636477in}}%
\pgfpathcurveto{\pgfqpoint{4.192462in}{1.631433in}}{\pgfqpoint{4.189628in}{1.624591in}}{\pgfqpoint{4.189628in}{1.617458in}}%
\pgfpathcurveto{\pgfqpoint{4.189628in}{1.610326in}}{\pgfqpoint{4.192462in}{1.603484in}}{\pgfqpoint{4.197506in}{1.598440in}}%
\pgfpathcurveto{\pgfqpoint{4.202550in}{1.593397in}}{\pgfqpoint{4.209391in}{1.590563in}}{\pgfqpoint{4.216524in}{1.590563in}}%
\pgfpathclose%
\pgfusepath{stroke,fill}%
\end{pgfscope}%
\begin{pgfscope}%
\pgfpathrectangle{\pgfqpoint{2.867647in}{0.500000in}}{\pgfqpoint{1.764706in}{1.700000in}}%
\pgfusepath{clip}%
\pgfsetbuttcap%
\pgfsetroundjoin%
\definecolor{currentfill}{rgb}{0.953816,0.463738,0.317699}%
\pgfsetfillcolor{currentfill}%
\pgfsetlinewidth{0.311001pt}%
\definecolor{currentstroke}{rgb}{1.000000,1.000000,1.000000}%
\pgfsetstrokecolor{currentstroke}%
\pgfsetdash{}{0pt}%
\pgfpathmoveto{\pgfqpoint{3.977513in}{1.837716in}}%
\pgfpathcurveto{\pgfqpoint{3.984645in}{1.837716in}}{\pgfqpoint{3.991487in}{1.840550in}}{\pgfqpoint{3.996531in}{1.845593in}}%
\pgfpathcurveto{\pgfqpoint{4.001574in}{1.850637in}}{\pgfqpoint{4.004408in}{1.857479in}}{\pgfqpoint{4.004408in}{1.864612in}}%
\pgfpathcurveto{\pgfqpoint{4.004408in}{1.871744in}}{\pgfqpoint{4.001574in}{1.878586in}}{\pgfqpoint{3.996531in}{1.883630in}}%
\pgfpathcurveto{\pgfqpoint{3.991487in}{1.888673in}}{\pgfqpoint{3.984645in}{1.891507in}}{\pgfqpoint{3.977513in}{1.891507in}}%
\pgfpathcurveto{\pgfqpoint{3.970380in}{1.891507in}}{\pgfqpoint{3.963538in}{1.888673in}}{\pgfqpoint{3.958494in}{1.883630in}}%
\pgfpathcurveto{\pgfqpoint{3.953451in}{1.878586in}}{\pgfqpoint{3.950617in}{1.871744in}}{\pgfqpoint{3.950617in}{1.864612in}}%
\pgfpathcurveto{\pgfqpoint{3.950617in}{1.857479in}}{\pgfqpoint{3.953451in}{1.850637in}}{\pgfqpoint{3.958494in}{1.845593in}}%
\pgfpathcurveto{\pgfqpoint{3.963538in}{1.840550in}}{\pgfqpoint{3.970380in}{1.837716in}}{\pgfqpoint{3.977513in}{1.837716in}}%
\pgfpathclose%
\pgfusepath{stroke,fill}%
\end{pgfscope}%
\begin{pgfscope}%
\pgfpathrectangle{\pgfqpoint{2.867647in}{0.500000in}}{\pgfqpoint{1.764706in}{1.700000in}}%
\pgfusepath{clip}%
\pgfsetbuttcap%
\pgfsetroundjoin%
\definecolor{currentfill}{rgb}{0.966328,0.750560,0.616961}%
\pgfsetfillcolor{currentfill}%
\pgfsetlinewidth{0.311001pt}%
\definecolor{currentstroke}{rgb}{1.000000,1.000000,1.000000}%
\pgfsetstrokecolor{currentstroke}%
\pgfsetdash{}{0pt}%
\pgfpathmoveto{\pgfqpoint{4.098853in}{1.736533in}}%
\pgfpathcurveto{\pgfqpoint{4.105986in}{1.736533in}}{\pgfqpoint{4.112828in}{1.739367in}}{\pgfqpoint{4.117871in}{1.744410in}}%
\pgfpathcurveto{\pgfqpoint{4.122915in}{1.749454in}}{\pgfqpoint{4.125749in}{1.756296in}}{\pgfqpoint{4.125749in}{1.763429in}}%
\pgfpathcurveto{\pgfqpoint{4.125749in}{1.770561in}}{\pgfqpoint{4.122915in}{1.777403in}}{\pgfqpoint{4.117871in}{1.782447in}}%
\pgfpathcurveto{\pgfqpoint{4.112828in}{1.787490in}}{\pgfqpoint{4.105986in}{1.790324in}}{\pgfqpoint{4.098853in}{1.790324in}}%
\pgfpathcurveto{\pgfqpoint{4.091720in}{1.790324in}}{\pgfqpoint{4.084879in}{1.787490in}}{\pgfqpoint{4.079835in}{1.782447in}}%
\pgfpathcurveto{\pgfqpoint{4.074791in}{1.777403in}}{\pgfqpoint{4.071957in}{1.770561in}}{\pgfqpoint{4.071957in}{1.763429in}}%
\pgfpathcurveto{\pgfqpoint{4.071957in}{1.756296in}}{\pgfqpoint{4.074791in}{1.749454in}}{\pgfqpoint{4.079835in}{1.744410in}}%
\pgfpathcurveto{\pgfqpoint{4.084879in}{1.739367in}}{\pgfqpoint{4.091720in}{1.736533in}}{\pgfqpoint{4.098853in}{1.736533in}}%
\pgfpathclose%
\pgfusepath{stroke,fill}%
\end{pgfscope}%
\begin{pgfscope}%
\pgfpathrectangle{\pgfqpoint{2.867647in}{0.500000in}}{\pgfqpoint{1.764706in}{1.700000in}}%
\pgfusepath{clip}%
\pgfsetbuttcap%
\pgfsetroundjoin%
\definecolor{currentfill}{rgb}{0.970718,0.821518,0.719872}%
\pgfsetfillcolor{currentfill}%
\pgfsetlinewidth{0.311001pt}%
\definecolor{currentstroke}{rgb}{1.000000,1.000000,1.000000}%
\pgfsetstrokecolor{currentstroke}%
\pgfsetdash{}{0pt}%
\pgfpathmoveto{\pgfqpoint{4.259708in}{1.188110in}}%
\pgfpathcurveto{\pgfqpoint{4.266840in}{1.188110in}}{\pgfqpoint{4.273682in}{1.190944in}}{\pgfqpoint{4.278726in}{1.195987in}}%
\pgfpathcurveto{\pgfqpoint{4.283769in}{1.201031in}}{\pgfqpoint{4.286603in}{1.207873in}}{\pgfqpoint{4.286603in}{1.215005in}}%
\pgfpathcurveto{\pgfqpoint{4.286603in}{1.222138in}}{\pgfqpoint{4.283769in}{1.228980in}}{\pgfqpoint{4.278726in}{1.234024in}}%
\pgfpathcurveto{\pgfqpoint{4.273682in}{1.239067in}}{\pgfqpoint{4.266840in}{1.241901in}}{\pgfqpoint{4.259708in}{1.241901in}}%
\pgfpathcurveto{\pgfqpoint{4.252575in}{1.241901in}}{\pgfqpoint{4.245733in}{1.239067in}}{\pgfqpoint{4.240689in}{1.234024in}}%
\pgfpathcurveto{\pgfqpoint{4.235646in}{1.228980in}}{\pgfqpoint{4.232812in}{1.222138in}}{\pgfqpoint{4.232812in}{1.215005in}}%
\pgfpathcurveto{\pgfqpoint{4.232812in}{1.207873in}}{\pgfqpoint{4.235646in}{1.201031in}}{\pgfqpoint{4.240689in}{1.195987in}}%
\pgfpathcurveto{\pgfqpoint{4.245733in}{1.190944in}}{\pgfqpoint{4.252575in}{1.188110in}}{\pgfqpoint{4.259708in}{1.188110in}}%
\pgfpathclose%
\pgfusepath{stroke,fill}%
\end{pgfscope}%
\begin{pgfscope}%
\pgfpathrectangle{\pgfqpoint{2.867647in}{0.500000in}}{\pgfqpoint{1.764706in}{1.700000in}}%
\pgfusepath{clip}%
\pgfsetbuttcap%
\pgfsetroundjoin%
\definecolor{currentfill}{rgb}{0.979891,0.908948,0.848279}%
\pgfsetfillcolor{currentfill}%
\pgfsetlinewidth{0.311001pt}%
\definecolor{currentstroke}{rgb}{1.000000,1.000000,1.000000}%
\pgfsetstrokecolor{currentstroke}%
\pgfsetdash{}{0pt}%
\pgfpathmoveto{\pgfqpoint{4.220370in}{1.288935in}}%
\pgfpathcurveto{\pgfqpoint{4.227503in}{1.288935in}}{\pgfqpoint{4.234344in}{1.291769in}}{\pgfqpoint{4.239388in}{1.296812in}}%
\pgfpathcurveto{\pgfqpoint{4.244432in}{1.301856in}}{\pgfqpoint{4.247266in}{1.308698in}}{\pgfqpoint{4.247266in}{1.315831in}}%
\pgfpathcurveto{\pgfqpoint{4.247266in}{1.322963in}}{\pgfqpoint{4.244432in}{1.329805in}}{\pgfqpoint{4.239388in}{1.334849in}}%
\pgfpathcurveto{\pgfqpoint{4.234344in}{1.339892in}}{\pgfqpoint{4.227503in}{1.342726in}}{\pgfqpoint{4.220370in}{1.342726in}}%
\pgfpathcurveto{\pgfqpoint{4.213237in}{1.342726in}}{\pgfqpoint{4.206396in}{1.339892in}}{\pgfqpoint{4.201352in}{1.334849in}}%
\pgfpathcurveto{\pgfqpoint{4.196308in}{1.329805in}}{\pgfqpoint{4.193474in}{1.322963in}}{\pgfqpoint{4.193474in}{1.315831in}}%
\pgfpathcurveto{\pgfqpoint{4.193474in}{1.308698in}}{\pgfqpoint{4.196308in}{1.301856in}}{\pgfqpoint{4.201352in}{1.296812in}}%
\pgfpathcurveto{\pgfqpoint{4.206396in}{1.291769in}}{\pgfqpoint{4.213237in}{1.288935in}}{\pgfqpoint{4.220370in}{1.288935in}}%
\pgfpathclose%
\pgfusepath{stroke,fill}%
\end{pgfscope}%
\begin{pgfscope}%
\pgfpathrectangle{\pgfqpoint{2.867647in}{0.500000in}}{\pgfqpoint{1.764706in}{1.700000in}}%
\pgfusepath{clip}%
\pgfsetbuttcap%
\pgfsetroundjoin%
\definecolor{currentfill}{rgb}{0.972201,0.839051,0.745789}%
\pgfsetfillcolor{currentfill}%
\pgfsetlinewidth{0.311001pt}%
\definecolor{currentstroke}{rgb}{1.000000,1.000000,1.000000}%
\pgfsetstrokecolor{currentstroke}%
\pgfsetdash{}{0pt}%
\pgfpathmoveto{\pgfqpoint{4.263434in}{1.337658in}}%
\pgfpathcurveto{\pgfqpoint{4.270566in}{1.337658in}}{\pgfqpoint{4.277408in}{1.340492in}}{\pgfqpoint{4.282452in}{1.345536in}}%
\pgfpathcurveto{\pgfqpoint{4.287495in}{1.350580in}}{\pgfqpoint{4.290329in}{1.357421in}}{\pgfqpoint{4.290329in}{1.364554in}}%
\pgfpathcurveto{\pgfqpoint{4.290329in}{1.371687in}}{\pgfqpoint{4.287495in}{1.378529in}}{\pgfqpoint{4.282452in}{1.383572in}}%
\pgfpathcurveto{\pgfqpoint{4.277408in}{1.388616in}}{\pgfqpoint{4.270566in}{1.391450in}}{\pgfqpoint{4.263434in}{1.391450in}}%
\pgfpathcurveto{\pgfqpoint{4.256301in}{1.391450in}}{\pgfqpoint{4.249459in}{1.388616in}}{\pgfqpoint{4.244415in}{1.383572in}}%
\pgfpathcurveto{\pgfqpoint{4.239372in}{1.378529in}}{\pgfqpoint{4.236538in}{1.371687in}}{\pgfqpoint{4.236538in}{1.364554in}}%
\pgfpathcurveto{\pgfqpoint{4.236538in}{1.357421in}}{\pgfqpoint{4.239372in}{1.350580in}}{\pgfqpoint{4.244415in}{1.345536in}}%
\pgfpathcurveto{\pgfqpoint{4.249459in}{1.340492in}}{\pgfqpoint{4.256301in}{1.337658in}}{\pgfqpoint{4.263434in}{1.337658in}}%
\pgfpathclose%
\pgfusepath{stroke,fill}%
\end{pgfscope}%
\begin{pgfscope}%
\pgfpathrectangle{\pgfqpoint{2.867647in}{0.500000in}}{\pgfqpoint{1.764706in}{1.700000in}}%
\pgfusepath{clip}%
\pgfsetbuttcap%
\pgfsetroundjoin%
\definecolor{currentfill}{rgb}{0.975644,0.874038,0.797253}%
\pgfsetfillcolor{currentfill}%
\pgfsetlinewidth{0.311001pt}%
\definecolor{currentstroke}{rgb}{1.000000,1.000000,1.000000}%
\pgfsetstrokecolor{currentstroke}%
\pgfsetdash{}{0pt}%
\pgfpathmoveto{\pgfqpoint{4.093608in}{1.540190in}}%
\pgfpathcurveto{\pgfqpoint{4.100741in}{1.540190in}}{\pgfqpoint{4.107583in}{1.543024in}}{\pgfqpoint{4.112626in}{1.548068in}}%
\pgfpathcurveto{\pgfqpoint{4.117670in}{1.553112in}}{\pgfqpoint{4.120504in}{1.559953in}}{\pgfqpoint{4.120504in}{1.567086in}}%
\pgfpathcurveto{\pgfqpoint{4.120504in}{1.574219in}}{\pgfqpoint{4.117670in}{1.581060in}}{\pgfqpoint{4.112626in}{1.586104in}}%
\pgfpathcurveto{\pgfqpoint{4.107583in}{1.591148in}}{\pgfqpoint{4.100741in}{1.593982in}}{\pgfqpoint{4.093608in}{1.593982in}}%
\pgfpathcurveto{\pgfqpoint{4.086475in}{1.593982in}}{\pgfqpoint{4.079634in}{1.591148in}}{\pgfqpoint{4.074590in}{1.586104in}}%
\pgfpathcurveto{\pgfqpoint{4.069546in}{1.581060in}}{\pgfqpoint{4.066713in}{1.574219in}}{\pgfqpoint{4.066713in}{1.567086in}}%
\pgfpathcurveto{\pgfqpoint{4.066713in}{1.559953in}}{\pgfqpoint{4.069546in}{1.553112in}}{\pgfqpoint{4.074590in}{1.548068in}}%
\pgfpathcurveto{\pgfqpoint{4.079634in}{1.543024in}}{\pgfqpoint{4.086475in}{1.540190in}}{\pgfqpoint{4.093608in}{1.540190in}}%
\pgfpathclose%
\pgfusepath{stroke,fill}%
\end{pgfscope}%
\begin{pgfscope}%
\pgfpathrectangle{\pgfqpoint{2.867647in}{0.500000in}}{\pgfqpoint{1.764706in}{1.700000in}}%
\pgfusepath{clip}%
\pgfsetbuttcap%
\pgfsetroundjoin%
\definecolor{currentfill}{rgb}{0.976287,0.879862,0.805788}%
\pgfsetfillcolor{currentfill}%
\pgfsetlinewidth{0.311001pt}%
\definecolor{currentstroke}{rgb}{1.000000,1.000000,1.000000}%
\pgfsetstrokecolor{currentstroke}%
\pgfsetdash{}{0pt}%
\pgfpathmoveto{\pgfqpoint{4.242778in}{1.319982in}}%
\pgfpathcurveto{\pgfqpoint{4.249911in}{1.319982in}}{\pgfqpoint{4.256752in}{1.322815in}}{\pgfqpoint{4.261796in}{1.327859in}}%
\pgfpathcurveto{\pgfqpoint{4.266840in}{1.332903in}}{\pgfqpoint{4.269673in}{1.339744in}}{\pgfqpoint{4.269673in}{1.346877in}}%
\pgfpathcurveto{\pgfqpoint{4.269673in}{1.354010in}}{\pgfqpoint{4.266840in}{1.360852in}}{\pgfqpoint{4.261796in}{1.365895in}}%
\pgfpathcurveto{\pgfqpoint{4.256752in}{1.370939in}}{\pgfqpoint{4.249911in}{1.373773in}}{\pgfqpoint{4.242778in}{1.373773in}}%
\pgfpathcurveto{\pgfqpoint{4.235645in}{1.373773in}}{\pgfqpoint{4.228803in}{1.370939in}}{\pgfqpoint{4.223760in}{1.365895in}}%
\pgfpathcurveto{\pgfqpoint{4.218716in}{1.360852in}}{\pgfqpoint{4.215882in}{1.354010in}}{\pgfqpoint{4.215882in}{1.346877in}}%
\pgfpathcurveto{\pgfqpoint{4.215882in}{1.339744in}}{\pgfqpoint{4.218716in}{1.332903in}}{\pgfqpoint{4.223760in}{1.327859in}}%
\pgfpathcurveto{\pgfqpoint{4.228803in}{1.322815in}}{\pgfqpoint{4.235645in}{1.319982in}}{\pgfqpoint{4.242778in}{1.319982in}}%
\pgfpathclose%
\pgfusepath{stroke,fill}%
\end{pgfscope}%
\begin{pgfscope}%
\pgfpathrectangle{\pgfqpoint{2.867647in}{0.500000in}}{\pgfqpoint{1.764706in}{1.700000in}}%
\pgfusepath{clip}%
\pgfsetbuttcap%
\pgfsetroundjoin%
\definecolor{currentfill}{rgb}{0.980678,0.914765,0.856766}%
\pgfsetfillcolor{currentfill}%
\pgfsetlinewidth{0.311001pt}%
\definecolor{currentstroke}{rgb}{1.000000,1.000000,1.000000}%
\pgfsetstrokecolor{currentstroke}%
\pgfsetdash{}{0pt}%
\pgfpathmoveto{\pgfqpoint{4.168401in}{1.455724in}}%
\pgfpathcurveto{\pgfqpoint{4.175534in}{1.455724in}}{\pgfqpoint{4.182376in}{1.458557in}}{\pgfqpoint{4.187419in}{1.463601in}}%
\pgfpathcurveto{\pgfqpoint{4.192463in}{1.468645in}}{\pgfqpoint{4.195297in}{1.475486in}}{\pgfqpoint{4.195297in}{1.482619in}}%
\pgfpathcurveto{\pgfqpoint{4.195297in}{1.489752in}}{\pgfqpoint{4.192463in}{1.496594in}}{\pgfqpoint{4.187419in}{1.501637in}}%
\pgfpathcurveto{\pgfqpoint{4.182376in}{1.506681in}}{\pgfqpoint{4.175534in}{1.509515in}}{\pgfqpoint{4.168401in}{1.509515in}}%
\pgfpathcurveto{\pgfqpoint{4.161268in}{1.509515in}}{\pgfqpoint{4.154427in}{1.506681in}}{\pgfqpoint{4.149383in}{1.501637in}}%
\pgfpathcurveto{\pgfqpoint{4.144339in}{1.496594in}}{\pgfqpoint{4.141506in}{1.489752in}}{\pgfqpoint{4.141506in}{1.482619in}}%
\pgfpathcurveto{\pgfqpoint{4.141506in}{1.475486in}}{\pgfqpoint{4.144339in}{1.468645in}}{\pgfqpoint{4.149383in}{1.463601in}}%
\pgfpathcurveto{\pgfqpoint{4.154427in}{1.458557in}}{\pgfqpoint{4.161268in}{1.455724in}}{\pgfqpoint{4.168401in}{1.455724in}}%
\pgfpathclose%
\pgfusepath{stroke,fill}%
\end{pgfscope}%
\begin{pgfscope}%
\pgfpathrectangle{\pgfqpoint{2.867647in}{0.500000in}}{\pgfqpoint{1.764706in}{1.700000in}}%
\pgfusepath{clip}%
\pgfsetbuttcap%
\pgfsetroundjoin%
\definecolor{currentfill}{rgb}{0.976287,0.879862,0.805788}%
\pgfsetfillcolor{currentfill}%
\pgfsetlinewidth{0.311001pt}%
\definecolor{currentstroke}{rgb}{1.000000,1.000000,1.000000}%
\pgfsetstrokecolor{currentstroke}%
\pgfsetdash{}{0pt}%
\pgfpathmoveto{\pgfqpoint{4.203535in}{1.113350in}}%
\pgfpathcurveto{\pgfqpoint{4.210667in}{1.113350in}}{\pgfqpoint{4.217509in}{1.116184in}}{\pgfqpoint{4.222553in}{1.121227in}}%
\pgfpathcurveto{\pgfqpoint{4.227596in}{1.126271in}}{\pgfqpoint{4.230430in}{1.133113in}}{\pgfqpoint{4.230430in}{1.140245in}}%
\pgfpathcurveto{\pgfqpoint{4.230430in}{1.147378in}}{\pgfqpoint{4.227596in}{1.154220in}}{\pgfqpoint{4.222553in}{1.159264in}}%
\pgfpathcurveto{\pgfqpoint{4.217509in}{1.164307in}}{\pgfqpoint{4.210667in}{1.167141in}}{\pgfqpoint{4.203535in}{1.167141in}}%
\pgfpathcurveto{\pgfqpoint{4.196402in}{1.167141in}}{\pgfqpoint{4.189560in}{1.164307in}}{\pgfqpoint{4.184516in}{1.159264in}}%
\pgfpathcurveto{\pgfqpoint{4.179473in}{1.154220in}}{\pgfqpoint{4.176639in}{1.147378in}}{\pgfqpoint{4.176639in}{1.140245in}}%
\pgfpathcurveto{\pgfqpoint{4.176639in}{1.133113in}}{\pgfqpoint{4.179473in}{1.126271in}}{\pgfqpoint{4.184516in}{1.121227in}}%
\pgfpathcurveto{\pgfqpoint{4.189560in}{1.116184in}}{\pgfqpoint{4.196402in}{1.113350in}}{\pgfqpoint{4.203535in}{1.113350in}}%
\pgfpathclose%
\pgfusepath{stroke,fill}%
\end{pgfscope}%
\begin{pgfscope}%
\pgfpathrectangle{\pgfqpoint{2.867647in}{0.500000in}}{\pgfqpoint{1.764706in}{1.700000in}}%
\pgfusepath{clip}%
\pgfsetbuttcap%
\pgfsetroundjoin%
\definecolor{currentfill}{rgb}{0.979891,0.908948,0.848279}%
\pgfsetfillcolor{currentfill}%
\pgfsetlinewidth{0.311001pt}%
\definecolor{currentstroke}{rgb}{1.000000,1.000000,1.000000}%
\pgfsetstrokecolor{currentstroke}%
\pgfsetdash{}{0pt}%
\pgfpathmoveto{\pgfqpoint{4.209469in}{1.211824in}}%
\pgfpathcurveto{\pgfqpoint{4.216602in}{1.211824in}}{\pgfqpoint{4.223443in}{1.214658in}}{\pgfqpoint{4.228487in}{1.219702in}}%
\pgfpathcurveto{\pgfqpoint{4.233531in}{1.224745in}}{\pgfqpoint{4.236365in}{1.231587in}}{\pgfqpoint{4.236365in}{1.238720in}}%
\pgfpathcurveto{\pgfqpoint{4.236365in}{1.245853in}}{\pgfqpoint{4.233531in}{1.252694in}}{\pgfqpoint{4.228487in}{1.257738in}}%
\pgfpathcurveto{\pgfqpoint{4.223443in}{1.262782in}}{\pgfqpoint{4.216602in}{1.265616in}}{\pgfqpoint{4.209469in}{1.265616in}}%
\pgfpathcurveto{\pgfqpoint{4.202336in}{1.265616in}}{\pgfqpoint{4.195494in}{1.262782in}}{\pgfqpoint{4.190451in}{1.257738in}}%
\pgfpathcurveto{\pgfqpoint{4.185407in}{1.252694in}}{\pgfqpoint{4.182573in}{1.245853in}}{\pgfqpoint{4.182573in}{1.238720in}}%
\pgfpathcurveto{\pgfqpoint{4.182573in}{1.231587in}}{\pgfqpoint{4.185407in}{1.224745in}}{\pgfqpoint{4.190451in}{1.219702in}}%
\pgfpathcurveto{\pgfqpoint{4.195494in}{1.214658in}}{\pgfqpoint{4.202336in}{1.211824in}}{\pgfqpoint{4.209469in}{1.211824in}}%
\pgfpathclose%
\pgfusepath{stroke,fill}%
\end{pgfscope}%
\begin{pgfscope}%
\pgfpathrectangle{\pgfqpoint{2.867647in}{0.500000in}}{\pgfqpoint{1.764706in}{1.700000in}}%
\pgfusepath{clip}%
\pgfsetbuttcap%
\pgfsetroundjoin%
\definecolor{currentfill}{rgb}{0.965440,0.720101,0.576404}%
\pgfsetfillcolor{currentfill}%
\pgfsetlinewidth{0.311001pt}%
\definecolor{currentstroke}{rgb}{1.000000,1.000000,1.000000}%
\pgfsetstrokecolor{currentstroke}%
\pgfsetdash{}{0pt}%
\pgfpathmoveto{\pgfqpoint{4.008795in}{1.038588in}}%
\pgfpathcurveto{\pgfqpoint{4.015928in}{1.038588in}}{\pgfqpoint{4.022769in}{1.041422in}}{\pgfqpoint{4.027813in}{1.046465in}}%
\pgfpathcurveto{\pgfqpoint{4.032857in}{1.051509in}}{\pgfqpoint{4.035690in}{1.058351in}}{\pgfqpoint{4.035690in}{1.065484in}}%
\pgfpathcurveto{\pgfqpoint{4.035690in}{1.072616in}}{\pgfqpoint{4.032857in}{1.079458in}}{\pgfqpoint{4.027813in}{1.084502in}}%
\pgfpathcurveto{\pgfqpoint{4.022769in}{1.089545in}}{\pgfqpoint{4.015928in}{1.092379in}}{\pgfqpoint{4.008795in}{1.092379in}}%
\pgfpathcurveto{\pgfqpoint{4.001662in}{1.092379in}}{\pgfqpoint{3.994820in}{1.089545in}}{\pgfqpoint{3.989777in}{1.084502in}}%
\pgfpathcurveto{\pgfqpoint{3.984733in}{1.079458in}}{\pgfqpoint{3.981899in}{1.072616in}}{\pgfqpoint{3.981899in}{1.065484in}}%
\pgfpathcurveto{\pgfqpoint{3.981899in}{1.058351in}}{\pgfqpoint{3.984733in}{1.051509in}}{\pgfqpoint{3.989777in}{1.046465in}}%
\pgfpathcurveto{\pgfqpoint{3.994820in}{1.041422in}}{\pgfqpoint{4.001662in}{1.038588in}}{\pgfqpoint{4.008795in}{1.038588in}}%
\pgfpathclose%
\pgfusepath{stroke,fill}%
\end{pgfscope}%
\begin{pgfscope}%
\pgfpathrectangle{\pgfqpoint{2.867647in}{0.500000in}}{\pgfqpoint{1.764706in}{1.700000in}}%
\pgfusepath{clip}%
\pgfsetbuttcap%
\pgfsetroundjoin%
\definecolor{currentfill}{rgb}{0.979891,0.908948,0.848279}%
\pgfsetfillcolor{currentfill}%
\pgfsetlinewidth{0.311001pt}%
\definecolor{currentstroke}{rgb}{1.000000,1.000000,1.000000}%
\pgfsetstrokecolor{currentstroke}%
\pgfsetdash{}{0pt}%
\pgfpathmoveto{\pgfqpoint{4.212497in}{1.384527in}}%
\pgfpathcurveto{\pgfqpoint{4.219629in}{1.384527in}}{\pgfqpoint{4.226471in}{1.387361in}}{\pgfqpoint{4.231515in}{1.392405in}}%
\pgfpathcurveto{\pgfqpoint{4.236558in}{1.397448in}}{\pgfqpoint{4.239392in}{1.404290in}}{\pgfqpoint{4.239392in}{1.411423in}}%
\pgfpathcurveto{\pgfqpoint{4.239392in}{1.418556in}}{\pgfqpoint{4.236558in}{1.425397in}}{\pgfqpoint{4.231515in}{1.430441in}}%
\pgfpathcurveto{\pgfqpoint{4.226471in}{1.435485in}}{\pgfqpoint{4.219629in}{1.438319in}}{\pgfqpoint{4.212497in}{1.438319in}}%
\pgfpathcurveto{\pgfqpoint{4.205364in}{1.438319in}}{\pgfqpoint{4.198522in}{1.435485in}}{\pgfqpoint{4.193479in}{1.430441in}}%
\pgfpathcurveto{\pgfqpoint{4.188435in}{1.425397in}}{\pgfqpoint{4.185601in}{1.418556in}}{\pgfqpoint{4.185601in}{1.411423in}}%
\pgfpathcurveto{\pgfqpoint{4.185601in}{1.404290in}}{\pgfqpoint{4.188435in}{1.397448in}}{\pgfqpoint{4.193479in}{1.392405in}}%
\pgfpathcurveto{\pgfqpoint{4.198522in}{1.387361in}}{\pgfqpoint{4.205364in}{1.384527in}}{\pgfqpoint{4.212497in}{1.384527in}}%
\pgfpathclose%
\pgfusepath{stroke,fill}%
\end{pgfscope}%
\begin{pgfscope}%
\pgfpathrectangle{\pgfqpoint{2.867647in}{0.500000in}}{\pgfqpoint{1.764706in}{1.700000in}}%
\pgfusepath{clip}%
\pgfsetbuttcap%
\pgfsetroundjoin%
\definecolor{currentfill}{rgb}{0.979891,0.908948,0.848279}%
\pgfsetfillcolor{currentfill}%
\pgfsetlinewidth{0.311001pt}%
\definecolor{currentstroke}{rgb}{1.000000,1.000000,1.000000}%
\pgfsetstrokecolor{currentstroke}%
\pgfsetdash{}{0pt}%
\pgfpathmoveto{\pgfqpoint{4.151370in}{1.558441in}}%
\pgfpathcurveto{\pgfqpoint{4.158503in}{1.558441in}}{\pgfqpoint{4.165345in}{1.561275in}}{\pgfqpoint{4.170389in}{1.566319in}}%
\pgfpathcurveto{\pgfqpoint{4.175432in}{1.571362in}}{\pgfqpoint{4.178266in}{1.578204in}}{\pgfqpoint{4.178266in}{1.585337in}}%
\pgfpathcurveto{\pgfqpoint{4.178266in}{1.592470in}}{\pgfqpoint{4.175432in}{1.599311in}}{\pgfqpoint{4.170389in}{1.604355in}}%
\pgfpathcurveto{\pgfqpoint{4.165345in}{1.609399in}}{\pgfqpoint{4.158503in}{1.612233in}}{\pgfqpoint{4.151370in}{1.612233in}}%
\pgfpathcurveto{\pgfqpoint{4.144238in}{1.612233in}}{\pgfqpoint{4.137396in}{1.609399in}}{\pgfqpoint{4.132352in}{1.604355in}}%
\pgfpathcurveto{\pgfqpoint{4.127309in}{1.599311in}}{\pgfqpoint{4.124475in}{1.592470in}}{\pgfqpoint{4.124475in}{1.585337in}}%
\pgfpathcurveto{\pgfqpoint{4.124475in}{1.578204in}}{\pgfqpoint{4.127309in}{1.571362in}}{\pgfqpoint{4.132352in}{1.566319in}}%
\pgfpathcurveto{\pgfqpoint{4.137396in}{1.561275in}}{\pgfqpoint{4.144238in}{1.558441in}}{\pgfqpoint{4.151370in}{1.558441in}}%
\pgfpathclose%
\pgfusepath{stroke,fill}%
\end{pgfscope}%
\begin{pgfscope}%
\pgfpathrectangle{\pgfqpoint{2.867647in}{0.500000in}}{\pgfqpoint{1.764706in}{1.700000in}}%
\pgfusepath{clip}%
\pgfsetbuttcap%
\pgfsetroundjoin%
\definecolor{currentfill}{rgb}{0.879259,0.192033,0.262681}%
\pgfsetfillcolor{currentfill}%
\pgfsetlinewidth{0.311001pt}%
\definecolor{currentstroke}{rgb}{1.000000,1.000000,1.000000}%
\pgfsetstrokecolor{currentstroke}%
\pgfsetdash{}{0pt}%
\pgfpathmoveto{\pgfqpoint{4.193924in}{1.818305in}}%
\pgfpathcurveto{\pgfqpoint{4.201057in}{1.818305in}}{\pgfqpoint{4.207899in}{1.821139in}}{\pgfqpoint{4.212942in}{1.826183in}}%
\pgfpathcurveto{\pgfqpoint{4.217986in}{1.831226in}}{\pgfqpoint{4.220820in}{1.838068in}}{\pgfqpoint{4.220820in}{1.845201in}}%
\pgfpathcurveto{\pgfqpoint{4.220820in}{1.852334in}}{\pgfqpoint{4.217986in}{1.859175in}}{\pgfqpoint{4.212942in}{1.864219in}}%
\pgfpathcurveto{\pgfqpoint{4.207899in}{1.869263in}}{\pgfqpoint{4.201057in}{1.872097in}}{\pgfqpoint{4.193924in}{1.872097in}}%
\pgfpathcurveto{\pgfqpoint{4.186791in}{1.872097in}}{\pgfqpoint{4.179950in}{1.869263in}}{\pgfqpoint{4.174906in}{1.864219in}}%
\pgfpathcurveto{\pgfqpoint{4.169862in}{1.859175in}}{\pgfqpoint{4.167028in}{1.852334in}}{\pgfqpoint{4.167028in}{1.845201in}}%
\pgfpathcurveto{\pgfqpoint{4.167028in}{1.838068in}}{\pgfqpoint{4.169862in}{1.831226in}}{\pgfqpoint{4.174906in}{1.826183in}}%
\pgfpathcurveto{\pgfqpoint{4.179950in}{1.821139in}}{\pgfqpoint{4.186791in}{1.818305in}}{\pgfqpoint{4.193924in}{1.818305in}}%
\pgfpathclose%
\pgfusepath{stroke,fill}%
\end{pgfscope}%
\begin{pgfscope}%
\pgfpathrectangle{\pgfqpoint{2.867647in}{0.500000in}}{\pgfqpoint{1.764706in}{1.700000in}}%
\pgfusepath{clip}%
\pgfsetbuttcap%
\pgfsetroundjoin%
\definecolor{currentfill}{rgb}{0.973271,0.850724,0.762998}%
\pgfsetfillcolor{currentfill}%
\pgfsetlinewidth{0.311001pt}%
\definecolor{currentstroke}{rgb}{1.000000,1.000000,1.000000}%
\pgfsetstrokecolor{currentstroke}%
\pgfsetdash{}{0pt}%
\pgfpathmoveto{\pgfqpoint{4.103943in}{0.985043in}}%
\pgfpathcurveto{\pgfqpoint{4.111076in}{0.985043in}}{\pgfqpoint{4.117918in}{0.987877in}}{\pgfqpoint{4.122961in}{0.992920in}}%
\pgfpathcurveto{\pgfqpoint{4.128005in}{0.997964in}}{\pgfqpoint{4.130839in}{1.004806in}}{\pgfqpoint{4.130839in}{1.011938in}}%
\pgfpathcurveto{\pgfqpoint{4.130839in}{1.019071in}}{\pgfqpoint{4.128005in}{1.025913in}}{\pgfqpoint{4.122961in}{1.030957in}}%
\pgfpathcurveto{\pgfqpoint{4.117918in}{1.036000in}}{\pgfqpoint{4.111076in}{1.038834in}}{\pgfqpoint{4.103943in}{1.038834in}}%
\pgfpathcurveto{\pgfqpoint{4.096810in}{1.038834in}}{\pgfqpoint{4.089969in}{1.036000in}}{\pgfqpoint{4.084925in}{1.030957in}}%
\pgfpathcurveto{\pgfqpoint{4.079881in}{1.025913in}}{\pgfqpoint{4.077048in}{1.019071in}}{\pgfqpoint{4.077048in}{1.011938in}}%
\pgfpathcurveto{\pgfqpoint{4.077048in}{1.004806in}}{\pgfqpoint{4.079881in}{0.997964in}}{\pgfqpoint{4.084925in}{0.992920in}}%
\pgfpathcurveto{\pgfqpoint{4.089969in}{0.987877in}}{\pgfqpoint{4.096810in}{0.985043in}}{\pgfqpoint{4.103943in}{0.985043in}}%
\pgfpathclose%
\pgfusepath{stroke,fill}%
\end{pgfscope}%
\begin{pgfscope}%
\pgfpathrectangle{\pgfqpoint{2.867647in}{0.500000in}}{\pgfqpoint{1.764706in}{1.700000in}}%
\pgfusepath{clip}%
\pgfsetbuttcap%
\pgfsetroundjoin%
\definecolor{currentfill}{rgb}{0.981377,0.920617,0.865369}%
\pgfsetfillcolor{currentfill}%
\pgfsetlinewidth{0.311001pt}%
\definecolor{currentstroke}{rgb}{1.000000,1.000000,1.000000}%
\pgfsetstrokecolor{currentstroke}%
\pgfsetdash{}{0pt}%
\pgfpathmoveto{\pgfqpoint{4.200255in}{1.323715in}}%
\pgfpathcurveto{\pgfqpoint{4.207388in}{1.323715in}}{\pgfqpoint{4.214230in}{1.326549in}}{\pgfqpoint{4.219273in}{1.331593in}}%
\pgfpathcurveto{\pgfqpoint{4.224317in}{1.336637in}}{\pgfqpoint{4.227151in}{1.343478in}}{\pgfqpoint{4.227151in}{1.350611in}}%
\pgfpathcurveto{\pgfqpoint{4.227151in}{1.357744in}}{\pgfqpoint{4.224317in}{1.364586in}}{\pgfqpoint{4.219273in}{1.369629in}}%
\pgfpathcurveto{\pgfqpoint{4.214230in}{1.374673in}}{\pgfqpoint{4.207388in}{1.377507in}}{\pgfqpoint{4.200255in}{1.377507in}}%
\pgfpathcurveto{\pgfqpoint{4.193122in}{1.377507in}}{\pgfqpoint{4.186281in}{1.374673in}}{\pgfqpoint{4.181237in}{1.369629in}}%
\pgfpathcurveto{\pgfqpoint{4.176193in}{1.364586in}}{\pgfqpoint{4.173360in}{1.357744in}}{\pgfqpoint{4.173360in}{1.350611in}}%
\pgfpathcurveto{\pgfqpoint{4.173360in}{1.343478in}}{\pgfqpoint{4.176193in}{1.336637in}}{\pgfqpoint{4.181237in}{1.331593in}}%
\pgfpathcurveto{\pgfqpoint{4.186281in}{1.326549in}}{\pgfqpoint{4.193122in}{1.323715in}}{\pgfqpoint{4.200255in}{1.323715in}}%
\pgfpathclose%
\pgfusepath{stroke,fill}%
\end{pgfscope}%
\begin{pgfscope}%
\pgfpathrectangle{\pgfqpoint{2.867647in}{0.500000in}}{\pgfqpoint{1.764706in}{1.700000in}}%
\pgfusepath{clip}%
\pgfsetbuttcap%
\pgfsetroundjoin%
\definecolor{currentfill}{rgb}{0.979124,0.903132,0.839793}%
\pgfsetfillcolor{currentfill}%
\pgfsetlinewidth{0.311001pt}%
\definecolor{currentstroke}{rgb}{1.000000,1.000000,1.000000}%
\pgfsetstrokecolor{currentstroke}%
\pgfsetdash{}{0pt}%
\pgfpathmoveto{\pgfqpoint{4.133204in}{1.110023in}}%
\pgfpathcurveto{\pgfqpoint{4.140336in}{1.110023in}}{\pgfqpoint{4.147178in}{1.112857in}}{\pgfqpoint{4.152222in}{1.117900in}}%
\pgfpathcurveto{\pgfqpoint{4.157265in}{1.122944in}}{\pgfqpoint{4.160099in}{1.129786in}}{\pgfqpoint{4.160099in}{1.136919in}}%
\pgfpathcurveto{\pgfqpoint{4.160099in}{1.144051in}}{\pgfqpoint{4.157265in}{1.150893in}}{\pgfqpoint{4.152222in}{1.155937in}}%
\pgfpathcurveto{\pgfqpoint{4.147178in}{1.160980in}}{\pgfqpoint{4.140336in}{1.163814in}}{\pgfqpoint{4.133204in}{1.163814in}}%
\pgfpathcurveto{\pgfqpoint{4.126071in}{1.163814in}}{\pgfqpoint{4.119229in}{1.160980in}}{\pgfqpoint{4.114185in}{1.155937in}}%
\pgfpathcurveto{\pgfqpoint{4.109142in}{1.150893in}}{\pgfqpoint{4.106308in}{1.144051in}}{\pgfqpoint{4.106308in}{1.136919in}}%
\pgfpathcurveto{\pgfqpoint{4.106308in}{1.129786in}}{\pgfqpoint{4.109142in}{1.122944in}}{\pgfqpoint{4.114185in}{1.117900in}}%
\pgfpathcurveto{\pgfqpoint{4.119229in}{1.112857in}}{\pgfqpoint{4.126071in}{1.110023in}}{\pgfqpoint{4.133204in}{1.110023in}}%
\pgfpathclose%
\pgfusepath{stroke,fill}%
\end{pgfscope}%
\begin{pgfscope}%
\pgfpathrectangle{\pgfqpoint{2.867647in}{0.500000in}}{\pgfqpoint{1.764706in}{1.700000in}}%
\pgfusepath{clip}%
\pgfsetbuttcap%
\pgfsetroundjoin%
\definecolor{currentfill}{rgb}{0.975644,0.874038,0.797253}%
\pgfsetfillcolor{currentfill}%
\pgfsetlinewidth{0.311001pt}%
\definecolor{currentstroke}{rgb}{1.000000,1.000000,1.000000}%
\pgfsetstrokecolor{currentstroke}%
\pgfsetdash{}{0pt}%
\pgfpathmoveto{\pgfqpoint{4.093638in}{1.541129in}}%
\pgfpathcurveto{\pgfqpoint{4.100771in}{1.541129in}}{\pgfqpoint{4.107612in}{1.543962in}}{\pgfqpoint{4.112656in}{1.549006in}}%
\pgfpathcurveto{\pgfqpoint{4.117699in}{1.554050in}}{\pgfqpoint{4.120533in}{1.560891in}}{\pgfqpoint{4.120533in}{1.568024in}}%
\pgfpathcurveto{\pgfqpoint{4.120533in}{1.575157in}}{\pgfqpoint{4.117699in}{1.581999in}}{\pgfqpoint{4.112656in}{1.587042in}}%
\pgfpathcurveto{\pgfqpoint{4.107612in}{1.592086in}}{\pgfqpoint{4.100771in}{1.594920in}}{\pgfqpoint{4.093638in}{1.594920in}}%
\pgfpathcurveto{\pgfqpoint{4.086505in}{1.594920in}}{\pgfqpoint{4.079663in}{1.592086in}}{\pgfqpoint{4.074620in}{1.587042in}}%
\pgfpathcurveto{\pgfqpoint{4.069576in}{1.581999in}}{\pgfqpoint{4.066742in}{1.575157in}}{\pgfqpoint{4.066742in}{1.568024in}}%
\pgfpathcurveto{\pgfqpoint{4.066742in}{1.560891in}}{\pgfqpoint{4.069576in}{1.554050in}}{\pgfqpoint{4.074620in}{1.549006in}}%
\pgfpathcurveto{\pgfqpoint{4.079663in}{1.543962in}}{\pgfqpoint{4.086505in}{1.541129in}}{\pgfqpoint{4.093638in}{1.541129in}}%
\pgfpathclose%
\pgfusepath{stroke,fill}%
\end{pgfscope}%
\begin{pgfscope}%
\pgfpathrectangle{\pgfqpoint{2.867647in}{0.500000in}}{\pgfqpoint{1.764706in}{1.700000in}}%
\pgfusepath{clip}%
\pgfsetbuttcap%
\pgfsetroundjoin%
\definecolor{currentfill}{rgb}{0.979891,0.908948,0.848279}%
\pgfsetfillcolor{currentfill}%
\pgfsetlinewidth{0.311001pt}%
\definecolor{currentstroke}{rgb}{1.000000,1.000000,1.000000}%
\pgfsetstrokecolor{currentstroke}%
\pgfsetdash{}{0pt}%
\pgfpathmoveto{\pgfqpoint{4.150994in}{1.552612in}}%
\pgfpathcurveto{\pgfqpoint{4.158126in}{1.552612in}}{\pgfqpoint{4.164968in}{1.555446in}}{\pgfqpoint{4.170012in}{1.560490in}}%
\pgfpathcurveto{\pgfqpoint{4.175055in}{1.565533in}}{\pgfqpoint{4.177889in}{1.572375in}}{\pgfqpoint{4.177889in}{1.579508in}}%
\pgfpathcurveto{\pgfqpoint{4.177889in}{1.586641in}}{\pgfqpoint{4.175055in}{1.593482in}}{\pgfqpoint{4.170012in}{1.598526in}}%
\pgfpathcurveto{\pgfqpoint{4.164968in}{1.603570in}}{\pgfqpoint{4.158126in}{1.606404in}}{\pgfqpoint{4.150994in}{1.606404in}}%
\pgfpathcurveto{\pgfqpoint{4.143861in}{1.606404in}}{\pgfqpoint{4.137019in}{1.603570in}}{\pgfqpoint{4.131975in}{1.598526in}}%
\pgfpathcurveto{\pgfqpoint{4.126932in}{1.593482in}}{\pgfqpoint{4.124098in}{1.586641in}}{\pgfqpoint{4.124098in}{1.579508in}}%
\pgfpathcurveto{\pgfqpoint{4.124098in}{1.572375in}}{\pgfqpoint{4.126932in}{1.565533in}}{\pgfqpoint{4.131975in}{1.560490in}}%
\pgfpathcurveto{\pgfqpoint{4.137019in}{1.555446in}}{\pgfqpoint{4.143861in}{1.552612in}}{\pgfqpoint{4.150994in}{1.552612in}}%
\pgfpathclose%
\pgfusepath{stroke,fill}%
\end{pgfscope}%
\begin{pgfscope}%
\pgfpathrectangle{\pgfqpoint{2.867647in}{0.500000in}}{\pgfqpoint{1.764706in}{1.700000in}}%
\pgfusepath{clip}%
\pgfsetbuttcap%
\pgfsetroundjoin%
\definecolor{currentfill}{rgb}{0.967398,0.774513,0.650573}%
\pgfsetfillcolor{currentfill}%
\pgfsetlinewidth{0.311001pt}%
\definecolor{currentstroke}{rgb}{1.000000,1.000000,1.000000}%
\pgfsetstrokecolor{currentstroke}%
\pgfsetdash{}{0pt}%
\pgfpathmoveto{\pgfqpoint{4.246358in}{1.554495in}}%
\pgfpathcurveto{\pgfqpoint{4.253491in}{1.554495in}}{\pgfqpoint{4.260333in}{1.557329in}}{\pgfqpoint{4.265376in}{1.562372in}}%
\pgfpathcurveto{\pgfqpoint{4.270420in}{1.567416in}}{\pgfqpoint{4.273254in}{1.574258in}}{\pgfqpoint{4.273254in}{1.581391in}}%
\pgfpathcurveto{\pgfqpoint{4.273254in}{1.588523in}}{\pgfqpoint{4.270420in}{1.595365in}}{\pgfqpoint{4.265376in}{1.600409in}}%
\pgfpathcurveto{\pgfqpoint{4.260333in}{1.605452in}}{\pgfqpoint{4.253491in}{1.608286in}}{\pgfqpoint{4.246358in}{1.608286in}}%
\pgfpathcurveto{\pgfqpoint{4.239226in}{1.608286in}}{\pgfqpoint{4.232384in}{1.605452in}}{\pgfqpoint{4.227340in}{1.600409in}}%
\pgfpathcurveto{\pgfqpoint{4.222297in}{1.595365in}}{\pgfqpoint{4.219463in}{1.588523in}}{\pgfqpoint{4.219463in}{1.581391in}}%
\pgfpathcurveto{\pgfqpoint{4.219463in}{1.574258in}}{\pgfqpoint{4.222297in}{1.567416in}}{\pgfqpoint{4.227340in}{1.562372in}}%
\pgfpathcurveto{\pgfqpoint{4.232384in}{1.557329in}}{\pgfqpoint{4.239226in}{1.554495in}}{\pgfqpoint{4.246358in}{1.554495in}}%
\pgfpathclose%
\pgfusepath{stroke,fill}%
\end{pgfscope}%
\begin{pgfscope}%
\pgfpathrectangle{\pgfqpoint{2.867647in}{0.500000in}}{\pgfqpoint{1.764706in}{1.700000in}}%
\pgfusepath{clip}%
\pgfsetbuttcap%
\pgfsetroundjoin%
\definecolor{currentfill}{rgb}{0.975644,0.874038,0.797253}%
\pgfsetfillcolor{currentfill}%
\pgfsetlinewidth{0.311001pt}%
\definecolor{currentstroke}{rgb}{1.000000,1.000000,1.000000}%
\pgfsetstrokecolor{currentstroke}%
\pgfsetdash{}{0pt}%
\pgfpathmoveto{\pgfqpoint{4.095952in}{1.036228in}}%
\pgfpathcurveto{\pgfqpoint{4.103085in}{1.036228in}}{\pgfqpoint{4.109927in}{1.039062in}}{\pgfqpoint{4.114971in}{1.044106in}}%
\pgfpathcurveto{\pgfqpoint{4.120014in}{1.049149in}}{\pgfqpoint{4.122848in}{1.055991in}}{\pgfqpoint{4.122848in}{1.063124in}}%
\pgfpathcurveto{\pgfqpoint{4.122848in}{1.070257in}}{\pgfqpoint{4.120014in}{1.077098in}}{\pgfqpoint{4.114971in}{1.082142in}}%
\pgfpathcurveto{\pgfqpoint{4.109927in}{1.087186in}}{\pgfqpoint{4.103085in}{1.090020in}}{\pgfqpoint{4.095952in}{1.090020in}}%
\pgfpathcurveto{\pgfqpoint{4.088820in}{1.090020in}}{\pgfqpoint{4.081978in}{1.087186in}}{\pgfqpoint{4.076934in}{1.082142in}}%
\pgfpathcurveto{\pgfqpoint{4.071891in}{1.077098in}}{\pgfqpoint{4.069057in}{1.070257in}}{\pgfqpoint{4.069057in}{1.063124in}}%
\pgfpathcurveto{\pgfqpoint{4.069057in}{1.055991in}}{\pgfqpoint{4.071891in}{1.049149in}}{\pgfqpoint{4.076934in}{1.044106in}}%
\pgfpathcurveto{\pgfqpoint{4.081978in}{1.039062in}}{\pgfqpoint{4.088820in}{1.036228in}}{\pgfqpoint{4.095952in}{1.036228in}}%
\pgfpathclose%
\pgfusepath{stroke,fill}%
\end{pgfscope}%
\begin{pgfscope}%
\pgfpathrectangle{\pgfqpoint{2.867647in}{0.500000in}}{\pgfqpoint{1.764706in}{1.700000in}}%
\pgfusepath{clip}%
\pgfsetbuttcap%
\pgfsetroundjoin%
\definecolor{currentfill}{rgb}{0.980678,0.914765,0.856766}%
\pgfsetfillcolor{currentfill}%
\pgfsetlinewidth{0.311001pt}%
\definecolor{currentstroke}{rgb}{1.000000,1.000000,1.000000}%
\pgfsetstrokecolor{currentstroke}%
\pgfsetdash{}{0pt}%
\pgfpathmoveto{\pgfqpoint{4.184954in}{1.466175in}}%
\pgfpathcurveto{\pgfqpoint{4.192087in}{1.466175in}}{\pgfqpoint{4.198929in}{1.469009in}}{\pgfqpoint{4.203973in}{1.474052in}}%
\pgfpathcurveto{\pgfqpoint{4.209016in}{1.479096in}}{\pgfqpoint{4.211850in}{1.485938in}}{\pgfqpoint{4.211850in}{1.493071in}}%
\pgfpathcurveto{\pgfqpoint{4.211850in}{1.500203in}}{\pgfqpoint{4.209016in}{1.507045in}}{\pgfqpoint{4.203973in}{1.512089in}}%
\pgfpathcurveto{\pgfqpoint{4.198929in}{1.517132in}}{\pgfqpoint{4.192087in}{1.519966in}}{\pgfqpoint{4.184954in}{1.519966in}}%
\pgfpathcurveto{\pgfqpoint{4.177822in}{1.519966in}}{\pgfqpoint{4.170980in}{1.517132in}}{\pgfqpoint{4.165936in}{1.512089in}}%
\pgfpathcurveto{\pgfqpoint{4.160893in}{1.507045in}}{\pgfqpoint{4.158059in}{1.500203in}}{\pgfqpoint{4.158059in}{1.493071in}}%
\pgfpathcurveto{\pgfqpoint{4.158059in}{1.485938in}}{\pgfqpoint{4.160893in}{1.479096in}}{\pgfqpoint{4.165936in}{1.474052in}}%
\pgfpathcurveto{\pgfqpoint{4.170980in}{1.469009in}}{\pgfqpoint{4.177822in}{1.466175in}}{\pgfqpoint{4.184954in}{1.466175in}}%
\pgfpathclose%
\pgfusepath{stroke,fill}%
\end{pgfscope}%
\begin{pgfscope}%
\pgfpathrectangle{\pgfqpoint{2.867647in}{0.500000in}}{\pgfqpoint{1.764706in}{1.700000in}}%
\pgfusepath{clip}%
\pgfsetbuttcap%
\pgfsetroundjoin%
\definecolor{currentfill}{rgb}{0.971202,0.827364,0.728520}%
\pgfsetfillcolor{currentfill}%
\pgfsetlinewidth{0.311001pt}%
\definecolor{currentstroke}{rgb}{1.000000,1.000000,1.000000}%
\pgfsetstrokecolor{currentstroke}%
\pgfsetdash{}{0pt}%
\pgfpathmoveto{\pgfqpoint{4.113585in}{1.327941in}}%
\pgfpathcurveto{\pgfqpoint{4.120717in}{1.327941in}}{\pgfqpoint{4.127559in}{1.330775in}}{\pgfqpoint{4.132603in}{1.335819in}}%
\pgfpathcurveto{\pgfqpoint{4.137646in}{1.340862in}}{\pgfqpoint{4.140480in}{1.347704in}}{\pgfqpoint{4.140480in}{1.354837in}}%
\pgfpathcurveto{\pgfqpoint{4.140480in}{1.361970in}}{\pgfqpoint{4.137646in}{1.368811in}}{\pgfqpoint{4.132603in}{1.373855in}}%
\pgfpathcurveto{\pgfqpoint{4.127559in}{1.378899in}}{\pgfqpoint{4.120717in}{1.381733in}}{\pgfqpoint{4.113585in}{1.381733in}}%
\pgfpathcurveto{\pgfqpoint{4.106452in}{1.381733in}}{\pgfqpoint{4.099610in}{1.378899in}}{\pgfqpoint{4.094566in}{1.373855in}}%
\pgfpathcurveto{\pgfqpoint{4.089523in}{1.368811in}}{\pgfqpoint{4.086689in}{1.361970in}}{\pgfqpoint{4.086689in}{1.354837in}}%
\pgfpathcurveto{\pgfqpoint{4.086689in}{1.347704in}}{\pgfqpoint{4.089523in}{1.340862in}}{\pgfqpoint{4.094566in}{1.335819in}}%
\pgfpathcurveto{\pgfqpoint{4.099610in}{1.330775in}}{\pgfqpoint{4.106452in}{1.327941in}}{\pgfqpoint{4.113585in}{1.327941in}}%
\pgfpathclose%
\pgfusepath{stroke,fill}%
\end{pgfscope}%
\begin{pgfscope}%
\pgfpathrectangle{\pgfqpoint{2.867647in}{0.500000in}}{\pgfqpoint{1.764706in}{1.700000in}}%
\pgfusepath{clip}%
\pgfsetbuttcap%
\pgfsetroundjoin%
\definecolor{currentfill}{rgb}{0.967398,0.774513,0.650573}%
\pgfsetfillcolor{currentfill}%
\pgfsetlinewidth{0.311001pt}%
\definecolor{currentstroke}{rgb}{1.000000,1.000000,1.000000}%
\pgfsetstrokecolor{currentstroke}%
\pgfsetdash{}{0pt}%
\pgfpathmoveto{\pgfqpoint{4.276314in}{1.181712in}}%
\pgfpathcurveto{\pgfqpoint{4.283446in}{1.181712in}}{\pgfqpoint{4.290288in}{1.184546in}}{\pgfqpoint{4.295332in}{1.189590in}}%
\pgfpathcurveto{\pgfqpoint{4.300375in}{1.194634in}}{\pgfqpoint{4.303209in}{1.201475in}}{\pgfqpoint{4.303209in}{1.208608in}}%
\pgfpathcurveto{\pgfqpoint{4.303209in}{1.215741in}}{\pgfqpoint{4.300375in}{1.222583in}}{\pgfqpoint{4.295332in}{1.227626in}}%
\pgfpathcurveto{\pgfqpoint{4.290288in}{1.232670in}}{\pgfqpoint{4.283446in}{1.235504in}}{\pgfqpoint{4.276314in}{1.235504in}}%
\pgfpathcurveto{\pgfqpoint{4.269181in}{1.235504in}}{\pgfqpoint{4.262339in}{1.232670in}}{\pgfqpoint{4.257295in}{1.227626in}}%
\pgfpathcurveto{\pgfqpoint{4.252252in}{1.222583in}}{\pgfqpoint{4.249418in}{1.215741in}}{\pgfqpoint{4.249418in}{1.208608in}}%
\pgfpathcurveto{\pgfqpoint{4.249418in}{1.201475in}}{\pgfqpoint{4.252252in}{1.194634in}}{\pgfqpoint{4.257295in}{1.189590in}}%
\pgfpathcurveto{\pgfqpoint{4.262339in}{1.184546in}}{\pgfqpoint{4.269181in}{1.181712in}}{\pgfqpoint{4.276314in}{1.181712in}}%
\pgfpathclose%
\pgfusepath{stroke,fill}%
\end{pgfscope}%
\begin{pgfscope}%
\pgfpathrectangle{\pgfqpoint{2.867647in}{0.500000in}}{\pgfqpoint{1.764706in}{1.700000in}}%
\pgfusepath{clip}%
\pgfsetbuttcap%
\pgfsetroundjoin%
\definecolor{currentfill}{rgb}{0.980678,0.914765,0.856766}%
\pgfsetfillcolor{currentfill}%
\pgfsetlinewidth{0.311001pt}%
\definecolor{currentstroke}{rgb}{1.000000,1.000000,1.000000}%
\pgfsetstrokecolor{currentstroke}%
\pgfsetdash{}{0pt}%
\pgfpathmoveto{\pgfqpoint{4.212677in}{1.246193in}}%
\pgfpathcurveto{\pgfqpoint{4.219810in}{1.246193in}}{\pgfqpoint{4.226651in}{1.249027in}}{\pgfqpoint{4.231695in}{1.254071in}}%
\pgfpathcurveto{\pgfqpoint{4.236739in}{1.259115in}}{\pgfqpoint{4.239573in}{1.265956in}}{\pgfqpoint{4.239573in}{1.273089in}}%
\pgfpathcurveto{\pgfqpoint{4.239573in}{1.280222in}}{\pgfqpoint{4.236739in}{1.287064in}}{\pgfqpoint{4.231695in}{1.292107in}}%
\pgfpathcurveto{\pgfqpoint{4.226651in}{1.297151in}}{\pgfqpoint{4.219810in}{1.299985in}}{\pgfqpoint{4.212677in}{1.299985in}}%
\pgfpathcurveto{\pgfqpoint{4.205544in}{1.299985in}}{\pgfqpoint{4.198703in}{1.297151in}}{\pgfqpoint{4.193659in}{1.292107in}}%
\pgfpathcurveto{\pgfqpoint{4.188615in}{1.287064in}}{\pgfqpoint{4.185781in}{1.280222in}}{\pgfqpoint{4.185781in}{1.273089in}}%
\pgfpathcurveto{\pgfqpoint{4.185781in}{1.265956in}}{\pgfqpoint{4.188615in}{1.259115in}}{\pgfqpoint{4.193659in}{1.254071in}}%
\pgfpathcurveto{\pgfqpoint{4.198703in}{1.249027in}}{\pgfqpoint{4.205544in}{1.246193in}}{\pgfqpoint{4.212677in}{1.246193in}}%
\pgfpathclose%
\pgfusepath{stroke,fill}%
\end{pgfscope}%
\begin{pgfscope}%
\pgfpathrectangle{\pgfqpoint{2.867647in}{0.500000in}}{\pgfqpoint{1.764706in}{1.700000in}}%
\pgfusepath{clip}%
\pgfsetbuttcap%
\pgfsetroundjoin%
\definecolor{currentfill}{rgb}{0.970255,0.815666,0.711203}%
\pgfsetfillcolor{currentfill}%
\pgfsetlinewidth{0.311001pt}%
\definecolor{currentstroke}{rgb}{1.000000,1.000000,1.000000}%
\pgfsetstrokecolor{currentstroke}%
\pgfsetdash{}{0pt}%
\pgfpathmoveto{\pgfqpoint{4.269994in}{1.391124in}}%
\pgfpathcurveto{\pgfqpoint{4.277127in}{1.391124in}}{\pgfqpoint{4.283968in}{1.393958in}}{\pgfqpoint{4.289012in}{1.399002in}}%
\pgfpathcurveto{\pgfqpoint{4.294056in}{1.404045in}}{\pgfqpoint{4.296890in}{1.410887in}}{\pgfqpoint{4.296890in}{1.418020in}}%
\pgfpathcurveto{\pgfqpoint{4.296890in}{1.425153in}}{\pgfqpoint{4.294056in}{1.431994in}}{\pgfqpoint{4.289012in}{1.437038in}}%
\pgfpathcurveto{\pgfqpoint{4.283968in}{1.442082in}}{\pgfqpoint{4.277127in}{1.444916in}}{\pgfqpoint{4.269994in}{1.444916in}}%
\pgfpathcurveto{\pgfqpoint{4.262861in}{1.444916in}}{\pgfqpoint{4.256019in}{1.442082in}}{\pgfqpoint{4.250976in}{1.437038in}}%
\pgfpathcurveto{\pgfqpoint{4.245932in}{1.431994in}}{\pgfqpoint{4.243098in}{1.425153in}}{\pgfqpoint{4.243098in}{1.418020in}}%
\pgfpathcurveto{\pgfqpoint{4.243098in}{1.410887in}}{\pgfqpoint{4.245932in}{1.404045in}}{\pgfqpoint{4.250976in}{1.399002in}}%
\pgfpathcurveto{\pgfqpoint{4.256019in}{1.393958in}}{\pgfqpoint{4.262861in}{1.391124in}}{\pgfqpoint{4.269994in}{1.391124in}}%
\pgfpathclose%
\pgfusepath{stroke,fill}%
\end{pgfscope}%
\begin{pgfscope}%
\pgfpathrectangle{\pgfqpoint{2.867647in}{0.500000in}}{\pgfqpoint{1.764706in}{1.700000in}}%
\pgfusepath{clip}%
\pgfsetbuttcap%
\pgfsetroundjoin%
\definecolor{currentfill}{rgb}{0.965302,0.713942,0.568499}%
\pgfsetfillcolor{currentfill}%
\pgfsetlinewidth{0.311001pt}%
\definecolor{currentstroke}{rgb}{1.000000,1.000000,1.000000}%
\pgfsetstrokecolor{currentstroke}%
\pgfsetdash{}{0pt}%
\pgfpathmoveto{\pgfqpoint{4.064166in}{1.415700in}}%
\pgfpathcurveto{\pgfqpoint{4.071299in}{1.415700in}}{\pgfqpoint{4.078141in}{1.418534in}}{\pgfqpoint{4.083184in}{1.423577in}}%
\pgfpathcurveto{\pgfqpoint{4.088228in}{1.428621in}}{\pgfqpoint{4.091062in}{1.435463in}}{\pgfqpoint{4.091062in}{1.442596in}}%
\pgfpathcurveto{\pgfqpoint{4.091062in}{1.449728in}}{\pgfqpoint{4.088228in}{1.456570in}}{\pgfqpoint{4.083184in}{1.461614in}}%
\pgfpathcurveto{\pgfqpoint{4.078141in}{1.466657in}}{\pgfqpoint{4.071299in}{1.469491in}}{\pgfqpoint{4.064166in}{1.469491in}}%
\pgfpathcurveto{\pgfqpoint{4.057033in}{1.469491in}}{\pgfqpoint{4.050192in}{1.466657in}}{\pgfqpoint{4.045148in}{1.461614in}}%
\pgfpathcurveto{\pgfqpoint{4.040104in}{1.456570in}}{\pgfqpoint{4.037271in}{1.449728in}}{\pgfqpoint{4.037271in}{1.442596in}}%
\pgfpathcurveto{\pgfqpoint{4.037271in}{1.435463in}}{\pgfqpoint{4.040104in}{1.428621in}}{\pgfqpoint{4.045148in}{1.423577in}}%
\pgfpathcurveto{\pgfqpoint{4.050192in}{1.418534in}}{\pgfqpoint{4.057033in}{1.415700in}}{\pgfqpoint{4.064166in}{1.415700in}}%
\pgfpathclose%
\pgfusepath{stroke,fill}%
\end{pgfscope}%
\begin{pgfscope}%
\pgfpathrectangle{\pgfqpoint{2.867647in}{0.500000in}}{\pgfqpoint{1.764706in}{1.700000in}}%
\pgfusepath{clip}%
\pgfsetbuttcap%
\pgfsetroundjoin%
\definecolor{currentfill}{rgb}{0.980678,0.914765,0.856766}%
\pgfsetfillcolor{currentfill}%
\pgfsetlinewidth{0.311001pt}%
\definecolor{currentstroke}{rgb}{1.000000,1.000000,1.000000}%
\pgfsetstrokecolor{currentstroke}%
\pgfsetdash{}{0pt}%
\pgfpathmoveto{\pgfqpoint{4.180872in}{1.401568in}}%
\pgfpathcurveto{\pgfqpoint{4.188005in}{1.401568in}}{\pgfqpoint{4.194847in}{1.404402in}}{\pgfqpoint{4.199891in}{1.409446in}}%
\pgfpathcurveto{\pgfqpoint{4.204934in}{1.414490in}}{\pgfqpoint{4.207768in}{1.421331in}}{\pgfqpoint{4.207768in}{1.428464in}}%
\pgfpathcurveto{\pgfqpoint{4.207768in}{1.435597in}}{\pgfqpoint{4.204934in}{1.442439in}}{\pgfqpoint{4.199891in}{1.447482in}}%
\pgfpathcurveto{\pgfqpoint{4.194847in}{1.452526in}}{\pgfqpoint{4.188005in}{1.455360in}}{\pgfqpoint{4.180872in}{1.455360in}}%
\pgfpathcurveto{\pgfqpoint{4.173740in}{1.455360in}}{\pgfqpoint{4.166898in}{1.452526in}}{\pgfqpoint{4.161854in}{1.447482in}}%
\pgfpathcurveto{\pgfqpoint{4.156811in}{1.442439in}}{\pgfqpoint{4.153977in}{1.435597in}}{\pgfqpoint{4.153977in}{1.428464in}}%
\pgfpathcurveto{\pgfqpoint{4.153977in}{1.421331in}}{\pgfqpoint{4.156811in}{1.414490in}}{\pgfqpoint{4.161854in}{1.409446in}}%
\pgfpathcurveto{\pgfqpoint{4.166898in}{1.404402in}}{\pgfqpoint{4.173740in}{1.401568in}}{\pgfqpoint{4.180872in}{1.401568in}}%
\pgfpathclose%
\pgfusepath{stroke,fill}%
\end{pgfscope}%
\begin{pgfscope}%
\pgfpathrectangle{\pgfqpoint{2.867647in}{0.500000in}}{\pgfqpoint{1.764706in}{1.700000in}}%
\pgfusepath{clip}%
\pgfsetbuttcap%
\pgfsetroundjoin%
\definecolor{currentfill}{rgb}{0.960421,0.553286,0.393191}%
\pgfsetfillcolor{currentfill}%
\pgfsetlinewidth{0.311001pt}%
\definecolor{currentstroke}{rgb}{1.000000,1.000000,1.000000}%
\pgfsetstrokecolor{currentstroke}%
\pgfsetdash{}{0pt}%
\pgfpathmoveto{\pgfqpoint{4.012238in}{0.835984in}}%
\pgfpathcurveto{\pgfqpoint{4.019371in}{0.835984in}}{\pgfqpoint{4.026212in}{0.838818in}}{\pgfqpoint{4.031256in}{0.843862in}}%
\pgfpathcurveto{\pgfqpoint{4.036300in}{0.848905in}}{\pgfqpoint{4.039134in}{0.855747in}}{\pgfqpoint{4.039134in}{0.862880in}}%
\pgfpathcurveto{\pgfqpoint{4.039134in}{0.870013in}}{\pgfqpoint{4.036300in}{0.876854in}}{\pgfqpoint{4.031256in}{0.881898in}}%
\pgfpathcurveto{\pgfqpoint{4.026212in}{0.886942in}}{\pgfqpoint{4.019371in}{0.889776in}}{\pgfqpoint{4.012238in}{0.889776in}}%
\pgfpathcurveto{\pgfqpoint{4.005105in}{0.889776in}}{\pgfqpoint{3.998264in}{0.886942in}}{\pgfqpoint{3.993220in}{0.881898in}}%
\pgfpathcurveto{\pgfqpoint{3.988176in}{0.876854in}}{\pgfqpoint{3.985342in}{0.870013in}}{\pgfqpoint{3.985342in}{0.862880in}}%
\pgfpathcurveto{\pgfqpoint{3.985342in}{0.855747in}}{\pgfqpoint{3.988176in}{0.848905in}}{\pgfqpoint{3.993220in}{0.843862in}}%
\pgfpathcurveto{\pgfqpoint{3.998264in}{0.838818in}}{\pgfqpoint{4.005105in}{0.835984in}}{\pgfqpoint{4.012238in}{0.835984in}}%
\pgfpathclose%
\pgfusepath{stroke,fill}%
\end{pgfscope}%
\begin{pgfscope}%
\pgfpathrectangle{\pgfqpoint{2.867647in}{0.500000in}}{\pgfqpoint{1.764706in}{1.700000in}}%
\pgfusepath{clip}%
\pgfsetbuttcap%
\pgfsetroundjoin%
\definecolor{currentfill}{rgb}{0.978376,0.897317,0.831308}%
\pgfsetfillcolor{currentfill}%
\pgfsetlinewidth{0.311001pt}%
\definecolor{currentstroke}{rgb}{1.000000,1.000000,1.000000}%
\pgfsetstrokecolor{currentstroke}%
\pgfsetdash{}{0pt}%
\pgfpathmoveto{\pgfqpoint{4.111476in}{1.537800in}}%
\pgfpathcurveto{\pgfqpoint{4.118608in}{1.537800in}}{\pgfqpoint{4.125450in}{1.540634in}}{\pgfqpoint{4.130494in}{1.545677in}}%
\pgfpathcurveto{\pgfqpoint{4.135537in}{1.550721in}}{\pgfqpoint{4.138371in}{1.557563in}}{\pgfqpoint{4.138371in}{1.564695in}}%
\pgfpathcurveto{\pgfqpoint{4.138371in}{1.571828in}}{\pgfqpoint{4.135537in}{1.578670in}}{\pgfqpoint{4.130494in}{1.583714in}}%
\pgfpathcurveto{\pgfqpoint{4.125450in}{1.588757in}}{\pgfqpoint{4.118608in}{1.591591in}}{\pgfqpoint{4.111476in}{1.591591in}}%
\pgfpathcurveto{\pgfqpoint{4.104343in}{1.591591in}}{\pgfqpoint{4.097501in}{1.588757in}}{\pgfqpoint{4.092457in}{1.583714in}}%
\pgfpathcurveto{\pgfqpoint{4.087414in}{1.578670in}}{\pgfqpoint{4.084580in}{1.571828in}}{\pgfqpoint{4.084580in}{1.564695in}}%
\pgfpathcurveto{\pgfqpoint{4.084580in}{1.557563in}}{\pgfqpoint{4.087414in}{1.550721in}}{\pgfqpoint{4.092457in}{1.545677in}}%
\pgfpathcurveto{\pgfqpoint{4.097501in}{1.540634in}}{\pgfqpoint{4.104343in}{1.537800in}}{\pgfqpoint{4.111476in}{1.537800in}}%
\pgfpathclose%
\pgfusepath{stroke,fill}%
\end{pgfscope}%
\begin{pgfscope}%
\pgfpathrectangle{\pgfqpoint{2.867647in}{0.500000in}}{\pgfqpoint{1.764706in}{1.700000in}}%
\pgfusepath{clip}%
\pgfsetbuttcap%
\pgfsetroundjoin%
\definecolor{currentfill}{rgb}{0.969803,0.809811,0.702523}%
\pgfsetfillcolor{currentfill}%
\pgfsetlinewidth{0.311001pt}%
\definecolor{currentstroke}{rgb}{1.000000,1.000000,1.000000}%
\pgfsetstrokecolor{currentstroke}%
\pgfsetdash{}{0pt}%
\pgfpathmoveto{\pgfqpoint{4.264296in}{1.189747in}}%
\pgfpathcurveto{\pgfqpoint{4.271429in}{1.189747in}}{\pgfqpoint{4.278271in}{1.192581in}}{\pgfqpoint{4.283314in}{1.197624in}}%
\pgfpathcurveto{\pgfqpoint{4.288358in}{1.202668in}}{\pgfqpoint{4.291192in}{1.209510in}}{\pgfqpoint{4.291192in}{1.216642in}}%
\pgfpathcurveto{\pgfqpoint{4.291192in}{1.223775in}}{\pgfqpoint{4.288358in}{1.230617in}}{\pgfqpoint{4.283314in}{1.235661in}}%
\pgfpathcurveto{\pgfqpoint{4.278271in}{1.240704in}}{\pgfqpoint{4.271429in}{1.243538in}}{\pgfqpoint{4.264296in}{1.243538in}}%
\pgfpathcurveto{\pgfqpoint{4.257163in}{1.243538in}}{\pgfqpoint{4.250322in}{1.240704in}}{\pgfqpoint{4.245278in}{1.235661in}}%
\pgfpathcurveto{\pgfqpoint{4.240234in}{1.230617in}}{\pgfqpoint{4.237400in}{1.223775in}}{\pgfqpoint{4.237400in}{1.216642in}}%
\pgfpathcurveto{\pgfqpoint{4.237400in}{1.209510in}}{\pgfqpoint{4.240234in}{1.202668in}}{\pgfqpoint{4.245278in}{1.197624in}}%
\pgfpathcurveto{\pgfqpoint{4.250322in}{1.192581in}}{\pgfqpoint{4.257163in}{1.189747in}}{\pgfqpoint{4.264296in}{1.189747in}}%
\pgfpathclose%
\pgfusepath{stroke,fill}%
\end{pgfscope}%
\begin{pgfscope}%
\pgfpathrectangle{\pgfqpoint{2.867647in}{0.500000in}}{\pgfqpoint{1.764706in}{1.700000in}}%
\pgfusepath{clip}%
\pgfsetbuttcap%
\pgfsetroundjoin%
\definecolor{currentfill}{rgb}{0.977657,0.891500,0.822809}%
\pgfsetfillcolor{currentfill}%
\pgfsetlinewidth{0.311001pt}%
\definecolor{currentstroke}{rgb}{1.000000,1.000000,1.000000}%
\pgfsetstrokecolor{currentstroke}%
\pgfsetdash{}{0pt}%
\pgfpathmoveto{\pgfqpoint{4.130499in}{1.463710in}}%
\pgfpathcurveto{\pgfqpoint{4.137632in}{1.463710in}}{\pgfqpoint{4.144474in}{1.466544in}}{\pgfqpoint{4.149517in}{1.471588in}}%
\pgfpathcurveto{\pgfqpoint{4.154561in}{1.476632in}}{\pgfqpoint{4.157395in}{1.483473in}}{\pgfqpoint{4.157395in}{1.490606in}}%
\pgfpathcurveto{\pgfqpoint{4.157395in}{1.497739in}}{\pgfqpoint{4.154561in}{1.504581in}}{\pgfqpoint{4.149517in}{1.509624in}}%
\pgfpathcurveto{\pgfqpoint{4.144474in}{1.514668in}}{\pgfqpoint{4.137632in}{1.517502in}}{\pgfqpoint{4.130499in}{1.517502in}}%
\pgfpathcurveto{\pgfqpoint{4.123366in}{1.517502in}}{\pgfqpoint{4.116525in}{1.514668in}}{\pgfqpoint{4.111481in}{1.509624in}}%
\pgfpathcurveto{\pgfqpoint{4.106437in}{1.504581in}}{\pgfqpoint{4.103603in}{1.497739in}}{\pgfqpoint{4.103603in}{1.490606in}}%
\pgfpathcurveto{\pgfqpoint{4.103603in}{1.483473in}}{\pgfqpoint{4.106437in}{1.476632in}}{\pgfqpoint{4.111481in}{1.471588in}}%
\pgfpathcurveto{\pgfqpoint{4.116525in}{1.466544in}}{\pgfqpoint{4.123366in}{1.463710in}}{\pgfqpoint{4.130499in}{1.463710in}}%
\pgfpathclose%
\pgfusepath{stroke,fill}%
\end{pgfscope}%
\begin{pgfscope}%
\pgfpathrectangle{\pgfqpoint{2.867647in}{0.500000in}}{\pgfqpoint{1.764706in}{1.700000in}}%
\pgfusepath{clip}%
\pgfsetbuttcap%
\pgfsetroundjoin%
\definecolor{currentfill}{rgb}{0.970255,0.815666,0.711203}%
\pgfsetfillcolor{currentfill}%
\pgfsetlinewidth{0.311001pt}%
\definecolor{currentstroke}{rgb}{1.000000,1.000000,1.000000}%
\pgfsetstrokecolor{currentstroke}%
\pgfsetdash{}{0pt}%
\pgfpathmoveto{\pgfqpoint{4.268697in}{1.406747in}}%
\pgfpathcurveto{\pgfqpoint{4.275830in}{1.406747in}}{\pgfqpoint{4.282672in}{1.409581in}}{\pgfqpoint{4.287715in}{1.414624in}}%
\pgfpathcurveto{\pgfqpoint{4.292759in}{1.419668in}}{\pgfqpoint{4.295593in}{1.426510in}}{\pgfqpoint{4.295593in}{1.433643in}}%
\pgfpathcurveto{\pgfqpoint{4.295593in}{1.440775in}}{\pgfqpoint{4.292759in}{1.447617in}}{\pgfqpoint{4.287715in}{1.452661in}}%
\pgfpathcurveto{\pgfqpoint{4.282672in}{1.457704in}}{\pgfqpoint{4.275830in}{1.460538in}}{\pgfqpoint{4.268697in}{1.460538in}}%
\pgfpathcurveto{\pgfqpoint{4.261564in}{1.460538in}}{\pgfqpoint{4.254723in}{1.457704in}}{\pgfqpoint{4.249679in}{1.452661in}}%
\pgfpathcurveto{\pgfqpoint{4.244635in}{1.447617in}}{\pgfqpoint{4.241802in}{1.440775in}}{\pgfqpoint{4.241802in}{1.433643in}}%
\pgfpathcurveto{\pgfqpoint{4.241802in}{1.426510in}}{\pgfqpoint{4.244635in}{1.419668in}}{\pgfqpoint{4.249679in}{1.414624in}}%
\pgfpathcurveto{\pgfqpoint{4.254723in}{1.409581in}}{\pgfqpoint{4.261564in}{1.406747in}}{\pgfqpoint{4.268697in}{1.406747in}}%
\pgfpathclose%
\pgfusepath{stroke,fill}%
\end{pgfscope}%
\begin{pgfscope}%
\pgfpathrectangle{\pgfqpoint{2.867647in}{0.500000in}}{\pgfqpoint{1.764706in}{1.700000in}}%
\pgfusepath{clip}%
\pgfsetbuttcap%
\pgfsetroundjoin%
\definecolor{currentfill}{rgb}{0.971202,0.827364,0.728520}%
\pgfsetfillcolor{currentfill}%
\pgfsetlinewidth{0.311001pt}%
\definecolor{currentstroke}{rgb}{1.000000,1.000000,1.000000}%
\pgfsetstrokecolor{currentstroke}%
\pgfsetdash{}{0pt}%
\pgfpathmoveto{\pgfqpoint{4.260257in}{1.422128in}}%
\pgfpathcurveto{\pgfqpoint{4.267390in}{1.422128in}}{\pgfqpoint{4.274231in}{1.424962in}}{\pgfqpoint{4.279275in}{1.430006in}}%
\pgfpathcurveto{\pgfqpoint{4.284319in}{1.435050in}}{\pgfqpoint{4.287153in}{1.441891in}}{\pgfqpoint{4.287153in}{1.449024in}}%
\pgfpathcurveto{\pgfqpoint{4.287153in}{1.456157in}}{\pgfqpoint{4.284319in}{1.462998in}}{\pgfqpoint{4.279275in}{1.468042in}}%
\pgfpathcurveto{\pgfqpoint{4.274231in}{1.473086in}}{\pgfqpoint{4.267390in}{1.475920in}}{\pgfqpoint{4.260257in}{1.475920in}}%
\pgfpathcurveto{\pgfqpoint{4.253124in}{1.475920in}}{\pgfqpoint{4.246282in}{1.473086in}}{\pgfqpoint{4.241239in}{1.468042in}}%
\pgfpathcurveto{\pgfqpoint{4.236195in}{1.462998in}}{\pgfqpoint{4.233361in}{1.456157in}}{\pgfqpoint{4.233361in}{1.449024in}}%
\pgfpathcurveto{\pgfqpoint{4.233361in}{1.441891in}}{\pgfqpoint{4.236195in}{1.435050in}}{\pgfqpoint{4.241239in}{1.430006in}}%
\pgfpathcurveto{\pgfqpoint{4.246282in}{1.424962in}}{\pgfqpoint{4.253124in}{1.422128in}}{\pgfqpoint{4.260257in}{1.422128in}}%
\pgfpathclose%
\pgfusepath{stroke,fill}%
\end{pgfscope}%
\begin{pgfscope}%
\pgfpathrectangle{\pgfqpoint{2.867647in}{0.500000in}}{\pgfqpoint{1.764706in}{1.700000in}}%
\pgfusepath{clip}%
\pgfsetbuttcap%
\pgfsetroundjoin%
\definecolor{currentfill}{rgb}{0.965592,0.726236,0.584384}%
\pgfsetfillcolor{currentfill}%
\pgfsetlinewidth{0.311001pt}%
\definecolor{currentstroke}{rgb}{1.000000,1.000000,1.000000}%
\pgfsetstrokecolor{currentstroke}%
\pgfsetdash{}{0pt}%
\pgfpathmoveto{\pgfqpoint{4.079543in}{1.287999in}}%
\pgfpathcurveto{\pgfqpoint{4.086675in}{1.287999in}}{\pgfqpoint{4.093517in}{1.290833in}}{\pgfqpoint{4.098561in}{1.295877in}}%
\pgfpathcurveto{\pgfqpoint{4.103604in}{1.300921in}}{\pgfqpoint{4.106438in}{1.307762in}}{\pgfqpoint{4.106438in}{1.314895in}}%
\pgfpathcurveto{\pgfqpoint{4.106438in}{1.322028in}}{\pgfqpoint{4.103604in}{1.328870in}}{\pgfqpoint{4.098561in}{1.333913in}}%
\pgfpathcurveto{\pgfqpoint{4.093517in}{1.338957in}}{\pgfqpoint{4.086675in}{1.341791in}}{\pgfqpoint{4.079543in}{1.341791in}}%
\pgfpathcurveto{\pgfqpoint{4.072410in}{1.341791in}}{\pgfqpoint{4.065568in}{1.338957in}}{\pgfqpoint{4.060524in}{1.333913in}}%
\pgfpathcurveto{\pgfqpoint{4.055481in}{1.328870in}}{\pgfqpoint{4.052647in}{1.322028in}}{\pgfqpoint{4.052647in}{1.314895in}}%
\pgfpathcurveto{\pgfqpoint{4.052647in}{1.307762in}}{\pgfqpoint{4.055481in}{1.300921in}}{\pgfqpoint{4.060524in}{1.295877in}}%
\pgfpathcurveto{\pgfqpoint{4.065568in}{1.290833in}}{\pgfqpoint{4.072410in}{1.287999in}}{\pgfqpoint{4.079543in}{1.287999in}}%
\pgfpathclose%
\pgfusepath{stroke,fill}%
\end{pgfscope}%
\begin{pgfscope}%
\pgfpathrectangle{\pgfqpoint{2.867647in}{0.500000in}}{\pgfqpoint{1.764706in}{1.700000in}}%
\pgfusepath{clip}%
\pgfsetbuttcap%
\pgfsetroundjoin%
\definecolor{currentfill}{rgb}{0.976287,0.879862,0.805788}%
\pgfsetfillcolor{currentfill}%
\pgfsetlinewidth{0.311001pt}%
\definecolor{currentstroke}{rgb}{1.000000,1.000000,1.000000}%
\pgfsetstrokecolor{currentstroke}%
\pgfsetdash{}{0pt}%
\pgfpathmoveto{\pgfqpoint{4.112560in}{1.486369in}}%
\pgfpathcurveto{\pgfqpoint{4.119692in}{1.486369in}}{\pgfqpoint{4.126534in}{1.489203in}}{\pgfqpoint{4.131578in}{1.494247in}}%
\pgfpathcurveto{\pgfqpoint{4.136621in}{1.499290in}}{\pgfqpoint{4.139455in}{1.506132in}}{\pgfqpoint{4.139455in}{1.513265in}}%
\pgfpathcurveto{\pgfqpoint{4.139455in}{1.520398in}}{\pgfqpoint{4.136621in}{1.527239in}}{\pgfqpoint{4.131578in}{1.532283in}}%
\pgfpathcurveto{\pgfqpoint{4.126534in}{1.537327in}}{\pgfqpoint{4.119692in}{1.540161in}}{\pgfqpoint{4.112560in}{1.540161in}}%
\pgfpathcurveto{\pgfqpoint{4.105427in}{1.540161in}}{\pgfqpoint{4.098585in}{1.537327in}}{\pgfqpoint{4.093541in}{1.532283in}}%
\pgfpathcurveto{\pgfqpoint{4.088498in}{1.527239in}}{\pgfqpoint{4.085664in}{1.520398in}}{\pgfqpoint{4.085664in}{1.513265in}}%
\pgfpathcurveto{\pgfqpoint{4.085664in}{1.506132in}}{\pgfqpoint{4.088498in}{1.499290in}}{\pgfqpoint{4.093541in}{1.494247in}}%
\pgfpathcurveto{\pgfqpoint{4.098585in}{1.489203in}}{\pgfqpoint{4.105427in}{1.486369in}}{\pgfqpoint{4.112560in}{1.486369in}}%
\pgfpathclose%
\pgfusepath{stroke,fill}%
\end{pgfscope}%
\begin{pgfscope}%
\pgfpathrectangle{\pgfqpoint{2.867647in}{0.500000in}}{\pgfqpoint{1.764706in}{1.700000in}}%
\pgfusepath{clip}%
\pgfsetbuttcap%
\pgfsetroundjoin%
\definecolor{currentfill}{rgb}{0.976287,0.879862,0.805788}%
\pgfsetfillcolor{currentfill}%
\pgfsetlinewidth{0.311001pt}%
\definecolor{currentstroke}{rgb}{1.000000,1.000000,1.000000}%
\pgfsetstrokecolor{currentstroke}%
\pgfsetdash{}{0pt}%
\pgfpathmoveto{\pgfqpoint{4.138274in}{1.275374in}}%
\pgfpathcurveto{\pgfqpoint{4.145407in}{1.275374in}}{\pgfqpoint{4.152249in}{1.278208in}}{\pgfqpoint{4.157293in}{1.283252in}}%
\pgfpathcurveto{\pgfqpoint{4.162336in}{1.288296in}}{\pgfqpoint{4.165170in}{1.295137in}}{\pgfqpoint{4.165170in}{1.302270in}}%
\pgfpathcurveto{\pgfqpoint{4.165170in}{1.309403in}}{\pgfqpoint{4.162336in}{1.316245in}}{\pgfqpoint{4.157293in}{1.321288in}}%
\pgfpathcurveto{\pgfqpoint{4.152249in}{1.326332in}}{\pgfqpoint{4.145407in}{1.329166in}}{\pgfqpoint{4.138274in}{1.329166in}}%
\pgfpathcurveto{\pgfqpoint{4.131142in}{1.329166in}}{\pgfqpoint{4.124300in}{1.326332in}}{\pgfqpoint{4.119256in}{1.321288in}}%
\pgfpathcurveto{\pgfqpoint{4.114213in}{1.316245in}}{\pgfqpoint{4.111379in}{1.309403in}}{\pgfqpoint{4.111379in}{1.302270in}}%
\pgfpathcurveto{\pgfqpoint{4.111379in}{1.295137in}}{\pgfqpoint{4.114213in}{1.288296in}}{\pgfqpoint{4.119256in}{1.283252in}}%
\pgfpathcurveto{\pgfqpoint{4.124300in}{1.278208in}}{\pgfqpoint{4.131142in}{1.275374in}}{\pgfqpoint{4.138274in}{1.275374in}}%
\pgfpathclose%
\pgfusepath{stroke,fill}%
\end{pgfscope}%
\begin{pgfscope}%
\pgfpathrectangle{\pgfqpoint{2.867647in}{0.500000in}}{\pgfqpoint{1.764706in}{1.700000in}}%
\pgfusepath{clip}%
\pgfsetbuttcap%
\pgfsetroundjoin%
\definecolor{currentfill}{rgb}{0.974412,0.862387,0.780156}%
\pgfsetfillcolor{currentfill}%
\pgfsetlinewidth{0.311001pt}%
\definecolor{currentstroke}{rgb}{1.000000,1.000000,1.000000}%
\pgfsetstrokecolor{currentstroke}%
\pgfsetdash{}{0pt}%
\pgfpathmoveto{\pgfqpoint{4.127956in}{1.363023in}}%
\pgfpathcurveto{\pgfqpoint{4.135088in}{1.363023in}}{\pgfqpoint{4.141930in}{1.365857in}}{\pgfqpoint{4.146974in}{1.370900in}}%
\pgfpathcurveto{\pgfqpoint{4.152017in}{1.375944in}}{\pgfqpoint{4.154851in}{1.382785in}}{\pgfqpoint{4.154851in}{1.389918in}}%
\pgfpathcurveto{\pgfqpoint{4.154851in}{1.397051in}}{\pgfqpoint{4.152017in}{1.403893in}}{\pgfqpoint{4.146974in}{1.408936in}}%
\pgfpathcurveto{\pgfqpoint{4.141930in}{1.413980in}}{\pgfqpoint{4.135088in}{1.416814in}}{\pgfqpoint{4.127956in}{1.416814in}}%
\pgfpathcurveto{\pgfqpoint{4.120823in}{1.416814in}}{\pgfqpoint{4.113981in}{1.413980in}}{\pgfqpoint{4.108937in}{1.408936in}}%
\pgfpathcurveto{\pgfqpoint{4.103894in}{1.403893in}}{\pgfqpoint{4.101060in}{1.397051in}}{\pgfqpoint{4.101060in}{1.389918in}}%
\pgfpathcurveto{\pgfqpoint{4.101060in}{1.382785in}}{\pgfqpoint{4.103894in}{1.375944in}}{\pgfqpoint{4.108937in}{1.370900in}}%
\pgfpathcurveto{\pgfqpoint{4.113981in}{1.365857in}}{\pgfqpoint{4.120823in}{1.363023in}}{\pgfqpoint{4.127956in}{1.363023in}}%
\pgfpathclose%
\pgfusepath{stroke,fill}%
\end{pgfscope}%
\begin{pgfscope}%
\pgfpathrectangle{\pgfqpoint{2.867647in}{0.500000in}}{\pgfqpoint{1.764706in}{1.700000in}}%
\pgfusepath{clip}%
\pgfsetbuttcap%
\pgfsetroundjoin%
\definecolor{currentfill}{rgb}{0.965169,0.707764,0.560659}%
\pgfsetfillcolor{currentfill}%
\pgfsetlinewidth{0.311001pt}%
\definecolor{currentstroke}{rgb}{1.000000,1.000000,1.000000}%
\pgfsetstrokecolor{currentstroke}%
\pgfsetdash{}{0pt}%
\pgfpathmoveto{\pgfqpoint{3.989662in}{1.711756in}}%
\pgfpathcurveto{\pgfqpoint{3.996794in}{1.711756in}}{\pgfqpoint{4.003636in}{1.714590in}}{\pgfqpoint{4.008680in}{1.719633in}}%
\pgfpathcurveto{\pgfqpoint{4.013723in}{1.724677in}}{\pgfqpoint{4.016557in}{1.731519in}}{\pgfqpoint{4.016557in}{1.738652in}}%
\pgfpathcurveto{\pgfqpoint{4.016557in}{1.745784in}}{\pgfqpoint{4.013723in}{1.752626in}}{\pgfqpoint{4.008680in}{1.757670in}}%
\pgfpathcurveto{\pgfqpoint{4.003636in}{1.762713in}}{\pgfqpoint{3.996794in}{1.765547in}}{\pgfqpoint{3.989662in}{1.765547in}}%
\pgfpathcurveto{\pgfqpoint{3.982529in}{1.765547in}}{\pgfqpoint{3.975687in}{1.762713in}}{\pgfqpoint{3.970643in}{1.757670in}}%
\pgfpathcurveto{\pgfqpoint{3.965600in}{1.752626in}}{\pgfqpoint{3.962766in}{1.745784in}}{\pgfqpoint{3.962766in}{1.738652in}}%
\pgfpathcurveto{\pgfqpoint{3.962766in}{1.731519in}}{\pgfqpoint{3.965600in}{1.724677in}}{\pgfqpoint{3.970643in}{1.719633in}}%
\pgfpathcurveto{\pgfqpoint{3.975687in}{1.714590in}}{\pgfqpoint{3.982529in}{1.711756in}}{\pgfqpoint{3.989662in}{1.711756in}}%
\pgfpathclose%
\pgfusepath{stroke,fill}%
\end{pgfscope}%
\begin{pgfscope}%
\pgfpathrectangle{\pgfqpoint{2.867647in}{0.500000in}}{\pgfqpoint{1.764706in}{1.700000in}}%
\pgfusepath{clip}%
\pgfsetbuttcap%
\pgfsetroundjoin%
\definecolor{currentfill}{rgb}{0.961115,0.566634,0.405693}%
\pgfsetfillcolor{currentfill}%
\pgfsetlinewidth{0.311001pt}%
\definecolor{currentstroke}{rgb}{1.000000,1.000000,1.000000}%
\pgfsetstrokecolor{currentstroke}%
\pgfsetdash{}{0pt}%
\pgfpathmoveto{\pgfqpoint{4.311570in}{1.509396in}}%
\pgfpathcurveto{\pgfqpoint{4.318703in}{1.509396in}}{\pgfqpoint{4.325544in}{1.512230in}}{\pgfqpoint{4.330588in}{1.517273in}}%
\pgfpathcurveto{\pgfqpoint{4.335632in}{1.522317in}}{\pgfqpoint{4.338466in}{1.529159in}}{\pgfqpoint{4.338466in}{1.536291in}}%
\pgfpathcurveto{\pgfqpoint{4.338466in}{1.543424in}}{\pgfqpoint{4.335632in}{1.550266in}}{\pgfqpoint{4.330588in}{1.555310in}}%
\pgfpathcurveto{\pgfqpoint{4.325544in}{1.560353in}}{\pgfqpoint{4.318703in}{1.563187in}}{\pgfqpoint{4.311570in}{1.563187in}}%
\pgfpathcurveto{\pgfqpoint{4.304437in}{1.563187in}}{\pgfqpoint{4.297596in}{1.560353in}}{\pgfqpoint{4.292552in}{1.555310in}}%
\pgfpathcurveto{\pgfqpoint{4.287508in}{1.550266in}}{\pgfqpoint{4.284674in}{1.543424in}}{\pgfqpoint{4.284674in}{1.536291in}}%
\pgfpathcurveto{\pgfqpoint{4.284674in}{1.529159in}}{\pgfqpoint{4.287508in}{1.522317in}}{\pgfqpoint{4.292552in}{1.517273in}}%
\pgfpathcurveto{\pgfqpoint{4.297596in}{1.512230in}}{\pgfqpoint{4.304437in}{1.509396in}}{\pgfqpoint{4.311570in}{1.509396in}}%
\pgfpathclose%
\pgfusepath{stroke,fill}%
\end{pgfscope}%
\begin{pgfscope}%
\pgfpathrectangle{\pgfqpoint{2.867647in}{0.500000in}}{\pgfqpoint{1.764706in}{1.700000in}}%
\pgfusepath{clip}%
\pgfsetbuttcap%
\pgfsetroundjoin%
\definecolor{currentfill}{rgb}{0.963559,0.632016,0.472047}%
\pgfsetfillcolor{currentfill}%
\pgfsetlinewidth{0.311001pt}%
\definecolor{currentstroke}{rgb}{1.000000,1.000000,1.000000}%
\pgfsetstrokecolor{currentstroke}%
\pgfsetdash{}{0pt}%
\pgfpathmoveto{\pgfqpoint{3.981074in}{0.885577in}}%
\pgfpathcurveto{\pgfqpoint{3.988207in}{0.885577in}}{\pgfqpoint{3.995049in}{0.888411in}}{\pgfqpoint{4.000092in}{0.893455in}}%
\pgfpathcurveto{\pgfqpoint{4.005136in}{0.898498in}}{\pgfqpoint{4.007970in}{0.905340in}}{\pgfqpoint{4.007970in}{0.912473in}}%
\pgfpathcurveto{\pgfqpoint{4.007970in}{0.919606in}}{\pgfqpoint{4.005136in}{0.926447in}}{\pgfqpoint{4.000092in}{0.931491in}}%
\pgfpathcurveto{\pgfqpoint{3.995049in}{0.936535in}}{\pgfqpoint{3.988207in}{0.939369in}}{\pgfqpoint{3.981074in}{0.939369in}}%
\pgfpathcurveto{\pgfqpoint{3.973941in}{0.939369in}}{\pgfqpoint{3.967100in}{0.936535in}}{\pgfqpoint{3.962056in}{0.931491in}}%
\pgfpathcurveto{\pgfqpoint{3.957012in}{0.926447in}}{\pgfqpoint{3.954178in}{0.919606in}}{\pgfqpoint{3.954178in}{0.912473in}}%
\pgfpathcurveto{\pgfqpoint{3.954178in}{0.905340in}}{\pgfqpoint{3.957012in}{0.898498in}}{\pgfqpoint{3.962056in}{0.893455in}}%
\pgfpathcurveto{\pgfqpoint{3.967100in}{0.888411in}}{\pgfqpoint{3.973941in}{0.885577in}}{\pgfqpoint{3.981074in}{0.885577in}}%
\pgfpathclose%
\pgfusepath{stroke,fill}%
\end{pgfscope}%
\begin{pgfscope}%
\pgfpathrectangle{\pgfqpoint{2.867647in}{0.500000in}}{\pgfqpoint{1.764706in}{1.700000in}}%
\pgfusepath{clip}%
\pgfsetbuttcap%
\pgfsetroundjoin%
\definecolor{currentfill}{rgb}{0.964173,0.657587,0.500469}%
\pgfsetfillcolor{currentfill}%
\pgfsetlinewidth{0.311001pt}%
\definecolor{currentstroke}{rgb}{1.000000,1.000000,1.000000}%
\pgfsetstrokecolor{currentstroke}%
\pgfsetdash{}{0pt}%
\pgfpathmoveto{\pgfqpoint{4.212099in}{1.678625in}}%
\pgfpathcurveto{\pgfqpoint{4.219232in}{1.678625in}}{\pgfqpoint{4.226073in}{1.681458in}}{\pgfqpoint{4.231117in}{1.686502in}}%
\pgfpathcurveto{\pgfqpoint{4.236161in}{1.691546in}}{\pgfqpoint{4.238994in}{1.698387in}}{\pgfqpoint{4.238994in}{1.705520in}}%
\pgfpathcurveto{\pgfqpoint{4.238994in}{1.712653in}}{\pgfqpoint{4.236161in}{1.719495in}}{\pgfqpoint{4.231117in}{1.724538in}}%
\pgfpathcurveto{\pgfqpoint{4.226073in}{1.729582in}}{\pgfqpoint{4.219232in}{1.732416in}}{\pgfqpoint{4.212099in}{1.732416in}}%
\pgfpathcurveto{\pgfqpoint{4.204966in}{1.732416in}}{\pgfqpoint{4.198124in}{1.729582in}}{\pgfqpoint{4.193081in}{1.724538in}}%
\pgfpathcurveto{\pgfqpoint{4.188037in}{1.719495in}}{\pgfqpoint{4.185203in}{1.712653in}}{\pgfqpoint{4.185203in}{1.705520in}}%
\pgfpathcurveto{\pgfqpoint{4.185203in}{1.698387in}}{\pgfqpoint{4.188037in}{1.691546in}}{\pgfqpoint{4.193081in}{1.686502in}}%
\pgfpathcurveto{\pgfqpoint{4.198124in}{1.681458in}}{\pgfqpoint{4.204966in}{1.678625in}}{\pgfqpoint{4.212099in}{1.678625in}}%
\pgfpathclose%
\pgfusepath{stroke,fill}%
\end{pgfscope}%
\begin{pgfscope}%
\pgfpathrectangle{\pgfqpoint{2.867647in}{0.500000in}}{\pgfqpoint{1.764706in}{1.700000in}}%
\pgfusepath{clip}%
\pgfsetbuttcap%
\pgfsetroundjoin%
\definecolor{currentfill}{rgb}{0.965928,0.738443,0.600540}%
\pgfsetfillcolor{currentfill}%
\pgfsetlinewidth{0.311001pt}%
\definecolor{currentstroke}{rgb}{1.000000,1.000000,1.000000}%
\pgfsetstrokecolor{currentstroke}%
\pgfsetdash{}{0pt}%
\pgfpathmoveto{\pgfqpoint{4.045322in}{1.745042in}}%
\pgfpathcurveto{\pgfqpoint{4.052455in}{1.745042in}}{\pgfqpoint{4.059297in}{1.747876in}}{\pgfqpoint{4.064340in}{1.752919in}}%
\pgfpathcurveto{\pgfqpoint{4.069384in}{1.757963in}}{\pgfqpoint{4.072218in}{1.764805in}}{\pgfqpoint{4.072218in}{1.771938in}}%
\pgfpathcurveto{\pgfqpoint{4.072218in}{1.779070in}}{\pgfqpoint{4.069384in}{1.785912in}}{\pgfqpoint{4.064340in}{1.790956in}}%
\pgfpathcurveto{\pgfqpoint{4.059297in}{1.795999in}}{\pgfqpoint{4.052455in}{1.798833in}}{\pgfqpoint{4.045322in}{1.798833in}}%
\pgfpathcurveto{\pgfqpoint{4.038189in}{1.798833in}}{\pgfqpoint{4.031348in}{1.795999in}}{\pgfqpoint{4.026304in}{1.790956in}}%
\pgfpathcurveto{\pgfqpoint{4.021260in}{1.785912in}}{\pgfqpoint{4.018426in}{1.779070in}}{\pgfqpoint{4.018426in}{1.771938in}}%
\pgfpathcurveto{\pgfqpoint{4.018426in}{1.764805in}}{\pgfqpoint{4.021260in}{1.757963in}}{\pgfqpoint{4.026304in}{1.752919in}}%
\pgfpathcurveto{\pgfqpoint{4.031348in}{1.747876in}}{\pgfqpoint{4.038189in}{1.745042in}}{\pgfqpoint{4.045322in}{1.745042in}}%
\pgfpathclose%
\pgfusepath{stroke,fill}%
\end{pgfscope}%
\begin{pgfscope}%
\pgfpathrectangle{\pgfqpoint{2.867647in}{0.500000in}}{\pgfqpoint{1.764706in}{1.700000in}}%
\pgfusepath{clip}%
\pgfsetbuttcap%
\pgfsetroundjoin%
\definecolor{currentfill}{rgb}{0.978376,0.897317,0.831308}%
\pgfsetfillcolor{currentfill}%
\pgfsetlinewidth{0.311001pt}%
\definecolor{currentstroke}{rgb}{1.000000,1.000000,1.000000}%
\pgfsetstrokecolor{currentstroke}%
\pgfsetdash{}{0pt}%
\pgfpathmoveto{\pgfqpoint{4.150439in}{1.290332in}}%
\pgfpathcurveto{\pgfqpoint{4.157571in}{1.290332in}}{\pgfqpoint{4.164413in}{1.293166in}}{\pgfqpoint{4.169457in}{1.298209in}}%
\pgfpathcurveto{\pgfqpoint{4.174500in}{1.303253in}}{\pgfqpoint{4.177334in}{1.310095in}}{\pgfqpoint{4.177334in}{1.317228in}}%
\pgfpathcurveto{\pgfqpoint{4.177334in}{1.324360in}}{\pgfqpoint{4.174500in}{1.331202in}}{\pgfqpoint{4.169457in}{1.336246in}}%
\pgfpathcurveto{\pgfqpoint{4.164413in}{1.341289in}}{\pgfqpoint{4.157571in}{1.344123in}}{\pgfqpoint{4.150439in}{1.344123in}}%
\pgfpathcurveto{\pgfqpoint{4.143306in}{1.344123in}}{\pgfqpoint{4.136464in}{1.341289in}}{\pgfqpoint{4.131420in}{1.336246in}}%
\pgfpathcurveto{\pgfqpoint{4.126377in}{1.331202in}}{\pgfqpoint{4.123543in}{1.324360in}}{\pgfqpoint{4.123543in}{1.317228in}}%
\pgfpathcurveto{\pgfqpoint{4.123543in}{1.310095in}}{\pgfqpoint{4.126377in}{1.303253in}}{\pgfqpoint{4.131420in}{1.298209in}}%
\pgfpathcurveto{\pgfqpoint{4.136464in}{1.293166in}}{\pgfqpoint{4.143306in}{1.290332in}}{\pgfqpoint{4.150439in}{1.290332in}}%
\pgfpathclose%
\pgfusepath{stroke,fill}%
\end{pgfscope}%
\begin{pgfscope}%
\pgfpathrectangle{\pgfqpoint{2.867647in}{0.500000in}}{\pgfqpoint{1.764706in}{1.700000in}}%
\pgfusepath{clip}%
\pgfsetbuttcap%
\pgfsetroundjoin%
\definecolor{currentfill}{rgb}{0.975018,0.868213,0.788710}%
\pgfsetfillcolor{currentfill}%
\pgfsetlinewidth{0.311001pt}%
\definecolor{currentstroke}{rgb}{1.000000,1.000000,1.000000}%
\pgfsetstrokecolor{currentstroke}%
\pgfsetdash{}{0pt}%
\pgfpathmoveto{\pgfqpoint{4.201834in}{1.550881in}}%
\pgfpathcurveto{\pgfqpoint{4.208967in}{1.550881in}}{\pgfqpoint{4.215809in}{1.553714in}}{\pgfqpoint{4.220853in}{1.558758in}}%
\pgfpathcurveto{\pgfqpoint{4.225896in}{1.563802in}}{\pgfqpoint{4.228730in}{1.570643in}}{\pgfqpoint{4.228730in}{1.577776in}}%
\pgfpathcurveto{\pgfqpoint{4.228730in}{1.584909in}}{\pgfqpoint{4.225896in}{1.591751in}}{\pgfqpoint{4.220853in}{1.596794in}}%
\pgfpathcurveto{\pgfqpoint{4.215809in}{1.601838in}}{\pgfqpoint{4.208967in}{1.604672in}}{\pgfqpoint{4.201834in}{1.604672in}}%
\pgfpathcurveto{\pgfqpoint{4.194702in}{1.604672in}}{\pgfqpoint{4.187860in}{1.601838in}}{\pgfqpoint{4.182816in}{1.596794in}}%
\pgfpathcurveto{\pgfqpoint{4.177773in}{1.591751in}}{\pgfqpoint{4.174939in}{1.584909in}}{\pgfqpoint{4.174939in}{1.577776in}}%
\pgfpathcurveto{\pgfqpoint{4.174939in}{1.570643in}}{\pgfqpoint{4.177773in}{1.563802in}}{\pgfqpoint{4.182816in}{1.558758in}}%
\pgfpathcurveto{\pgfqpoint{4.187860in}{1.553714in}}{\pgfqpoint{4.194702in}{1.550881in}}{\pgfqpoint{4.201834in}{1.550881in}}%
\pgfpathclose%
\pgfusepath{stroke,fill}%
\end{pgfscope}%
\begin{pgfscope}%
\pgfpathrectangle{\pgfqpoint{2.867647in}{0.500000in}}{\pgfqpoint{1.764706in}{1.700000in}}%
\pgfusepath{clip}%
\pgfsetbuttcap%
\pgfsetroundjoin%
\definecolor{currentfill}{rgb}{0.973832,0.856556,0.771584}%
\pgfsetfillcolor{currentfill}%
\pgfsetlinewidth{0.311001pt}%
\definecolor{currentstroke}{rgb}{1.000000,1.000000,1.000000}%
\pgfsetstrokecolor{currentstroke}%
\pgfsetdash{}{0pt}%
\pgfpathmoveto{\pgfqpoint{4.101732in}{1.663324in}}%
\pgfpathcurveto{\pgfqpoint{4.108865in}{1.663324in}}{\pgfqpoint{4.115706in}{1.666158in}}{\pgfqpoint{4.120750in}{1.671202in}}%
\pgfpathcurveto{\pgfqpoint{4.125794in}{1.676245in}}{\pgfqpoint{4.128628in}{1.683087in}}{\pgfqpoint{4.128628in}{1.690220in}}%
\pgfpathcurveto{\pgfqpoint{4.128628in}{1.697353in}}{\pgfqpoint{4.125794in}{1.704194in}}{\pgfqpoint{4.120750in}{1.709238in}}%
\pgfpathcurveto{\pgfqpoint{4.115706in}{1.714282in}}{\pgfqpoint{4.108865in}{1.717116in}}{\pgfqpoint{4.101732in}{1.717116in}}%
\pgfpathcurveto{\pgfqpoint{4.094599in}{1.717116in}}{\pgfqpoint{4.087757in}{1.714282in}}{\pgfqpoint{4.082714in}{1.709238in}}%
\pgfpathcurveto{\pgfqpoint{4.077670in}{1.704194in}}{\pgfqpoint{4.074836in}{1.697353in}}{\pgfqpoint{4.074836in}{1.690220in}}%
\pgfpathcurveto{\pgfqpoint{4.074836in}{1.683087in}}{\pgfqpoint{4.077670in}{1.676245in}}{\pgfqpoint{4.082714in}{1.671202in}}%
\pgfpathcurveto{\pgfqpoint{4.087757in}{1.666158in}}{\pgfqpoint{4.094599in}{1.663324in}}{\pgfqpoint{4.101732in}{1.663324in}}%
\pgfpathclose%
\pgfusepath{stroke,fill}%
\end{pgfscope}%
\begin{pgfscope}%
\pgfpathrectangle{\pgfqpoint{2.867647in}{0.500000in}}{\pgfqpoint{1.764706in}{1.700000in}}%
\pgfusepath{clip}%
\pgfsetbuttcap%
\pgfsetroundjoin%
\definecolor{currentfill}{rgb}{0.979891,0.908948,0.848279}%
\pgfsetfillcolor{currentfill}%
\pgfsetlinewidth{0.311001pt}%
\definecolor{currentstroke}{rgb}{1.000000,1.000000,1.000000}%
\pgfsetstrokecolor{currentstroke}%
\pgfsetdash{}{0pt}%
\pgfpathmoveto{\pgfqpoint{4.156744in}{1.544129in}}%
\pgfpathcurveto{\pgfqpoint{4.163877in}{1.544129in}}{\pgfqpoint{4.170719in}{1.546962in}}{\pgfqpoint{4.175762in}{1.552006in}}%
\pgfpathcurveto{\pgfqpoint{4.180806in}{1.557050in}}{\pgfqpoint{4.183640in}{1.563891in}}{\pgfqpoint{4.183640in}{1.571024in}}%
\pgfpathcurveto{\pgfqpoint{4.183640in}{1.578157in}}{\pgfqpoint{4.180806in}{1.584999in}}{\pgfqpoint{4.175762in}{1.590042in}}%
\pgfpathcurveto{\pgfqpoint{4.170719in}{1.595086in}}{\pgfqpoint{4.163877in}{1.597920in}}{\pgfqpoint{4.156744in}{1.597920in}}%
\pgfpathcurveto{\pgfqpoint{4.149611in}{1.597920in}}{\pgfqpoint{4.142770in}{1.595086in}}{\pgfqpoint{4.137726in}{1.590042in}}%
\pgfpathcurveto{\pgfqpoint{4.132682in}{1.584999in}}{\pgfqpoint{4.129848in}{1.578157in}}{\pgfqpoint{4.129848in}{1.571024in}}%
\pgfpathcurveto{\pgfqpoint{4.129848in}{1.563891in}}{\pgfqpoint{4.132682in}{1.557050in}}{\pgfqpoint{4.137726in}{1.552006in}}%
\pgfpathcurveto{\pgfqpoint{4.142770in}{1.546962in}}{\pgfqpoint{4.149611in}{1.544129in}}{\pgfqpoint{4.156744in}{1.544129in}}%
\pgfpathclose%
\pgfusepath{stroke,fill}%
\end{pgfscope}%
\begin{pgfscope}%
\pgfpathrectangle{\pgfqpoint{2.867647in}{0.500000in}}{\pgfqpoint{1.764706in}{1.700000in}}%
\pgfusepath{clip}%
\pgfsetbuttcap%
\pgfsetroundjoin%
\definecolor{currentfill}{rgb}{0.971694,0.833208,0.737161}%
\pgfsetfillcolor{currentfill}%
\pgfsetlinewidth{0.311001pt}%
\definecolor{currentstroke}{rgb}{1.000000,1.000000,1.000000}%
\pgfsetstrokecolor{currentstroke}%
\pgfsetdash{}{0pt}%
\pgfpathmoveto{\pgfqpoint{4.115729in}{0.969793in}}%
\pgfpathcurveto{\pgfqpoint{4.122861in}{0.969793in}}{\pgfqpoint{4.129703in}{0.972627in}}{\pgfqpoint{4.134747in}{0.977670in}}%
\pgfpathcurveto{\pgfqpoint{4.139790in}{0.982714in}}{\pgfqpoint{4.142624in}{0.989556in}}{\pgfqpoint{4.142624in}{0.996688in}}%
\pgfpathcurveto{\pgfqpoint{4.142624in}{1.003821in}}{\pgfqpoint{4.139790in}{1.010663in}}{\pgfqpoint{4.134747in}{1.015707in}}%
\pgfpathcurveto{\pgfqpoint{4.129703in}{1.020750in}}{\pgfqpoint{4.122861in}{1.023584in}}{\pgfqpoint{4.115729in}{1.023584in}}%
\pgfpathcurveto{\pgfqpoint{4.108596in}{1.023584in}}{\pgfqpoint{4.101754in}{1.020750in}}{\pgfqpoint{4.096711in}{1.015707in}}%
\pgfpathcurveto{\pgfqpoint{4.091667in}{1.010663in}}{\pgfqpoint{4.088833in}{1.003821in}}{\pgfqpoint{4.088833in}{0.996688in}}%
\pgfpathcurveto{\pgfqpoint{4.088833in}{0.989556in}}{\pgfqpoint{4.091667in}{0.982714in}}{\pgfqpoint{4.096711in}{0.977670in}}%
\pgfpathcurveto{\pgfqpoint{4.101754in}{0.972627in}}{\pgfqpoint{4.108596in}{0.969793in}}{\pgfqpoint{4.115729in}{0.969793in}}%
\pgfpathclose%
\pgfusepath{stroke,fill}%
\end{pgfscope}%
\begin{pgfscope}%
\pgfpathrectangle{\pgfqpoint{2.867647in}{0.500000in}}{\pgfqpoint{1.764706in}{1.700000in}}%
\pgfusepath{clip}%
\pgfsetbuttcap%
\pgfsetroundjoin%
\definecolor{currentfill}{rgb}{0.981377,0.920617,0.865369}%
\pgfsetfillcolor{currentfill}%
\pgfsetlinewidth{0.311001pt}%
\definecolor{currentstroke}{rgb}{1.000000,1.000000,1.000000}%
\pgfsetstrokecolor{currentstroke}%
\pgfsetdash{}{0pt}%
\pgfpathmoveto{\pgfqpoint{4.173588in}{1.247979in}}%
\pgfpathcurveto{\pgfqpoint{4.180721in}{1.247979in}}{\pgfqpoint{4.187563in}{1.250813in}}{\pgfqpoint{4.192606in}{1.255856in}}%
\pgfpathcurveto{\pgfqpoint{4.197650in}{1.260900in}}{\pgfqpoint{4.200484in}{1.267742in}}{\pgfqpoint{4.200484in}{1.274875in}}%
\pgfpathcurveto{\pgfqpoint{4.200484in}{1.282007in}}{\pgfqpoint{4.197650in}{1.288849in}}{\pgfqpoint{4.192606in}{1.293893in}}%
\pgfpathcurveto{\pgfqpoint{4.187563in}{1.298936in}}{\pgfqpoint{4.180721in}{1.301770in}}{\pgfqpoint{4.173588in}{1.301770in}}%
\pgfpathcurveto{\pgfqpoint{4.166455in}{1.301770in}}{\pgfqpoint{4.159614in}{1.298936in}}{\pgfqpoint{4.154570in}{1.293893in}}%
\pgfpathcurveto{\pgfqpoint{4.149526in}{1.288849in}}{\pgfqpoint{4.146693in}{1.282007in}}{\pgfqpoint{4.146693in}{1.274875in}}%
\pgfpathcurveto{\pgfqpoint{4.146693in}{1.267742in}}{\pgfqpoint{4.149526in}{1.260900in}}{\pgfqpoint{4.154570in}{1.255856in}}%
\pgfpathcurveto{\pgfqpoint{4.159614in}{1.250813in}}{\pgfqpoint{4.166455in}{1.247979in}}{\pgfqpoint{4.173588in}{1.247979in}}%
\pgfpathclose%
\pgfusepath{stroke,fill}%
\end{pgfscope}%
\begin{pgfscope}%
\pgfpathrectangle{\pgfqpoint{2.867647in}{0.500000in}}{\pgfqpoint{1.764706in}{1.700000in}}%
\pgfusepath{clip}%
\pgfsetbuttcap%
\pgfsetroundjoin%
\definecolor{currentfill}{rgb}{0.979124,0.903132,0.839793}%
\pgfsetfillcolor{currentfill}%
\pgfsetlinewidth{0.311001pt}%
\definecolor{currentstroke}{rgb}{1.000000,1.000000,1.000000}%
\pgfsetstrokecolor{currentstroke}%
\pgfsetdash{}{0pt}%
\pgfpathmoveto{\pgfqpoint{4.216075in}{1.402236in}}%
\pgfpathcurveto{\pgfqpoint{4.223207in}{1.402236in}}{\pgfqpoint{4.230049in}{1.405069in}}{\pgfqpoint{4.235093in}{1.410113in}}%
\pgfpathcurveto{\pgfqpoint{4.240136in}{1.415157in}}{\pgfqpoint{4.242970in}{1.421998in}}{\pgfqpoint{4.242970in}{1.429131in}}%
\pgfpathcurveto{\pgfqpoint{4.242970in}{1.436264in}}{\pgfqpoint{4.240136in}{1.443106in}}{\pgfqpoint{4.235093in}{1.448149in}}%
\pgfpathcurveto{\pgfqpoint{4.230049in}{1.453193in}}{\pgfqpoint{4.223207in}{1.456027in}}{\pgfqpoint{4.216075in}{1.456027in}}%
\pgfpathcurveto{\pgfqpoint{4.208942in}{1.456027in}}{\pgfqpoint{4.202100in}{1.453193in}}{\pgfqpoint{4.197056in}{1.448149in}}%
\pgfpathcurveto{\pgfqpoint{4.192013in}{1.443106in}}{\pgfqpoint{4.189179in}{1.436264in}}{\pgfqpoint{4.189179in}{1.429131in}}%
\pgfpathcurveto{\pgfqpoint{4.189179in}{1.421998in}}{\pgfqpoint{4.192013in}{1.415157in}}{\pgfqpoint{4.197056in}{1.410113in}}%
\pgfpathcurveto{\pgfqpoint{4.202100in}{1.405069in}}{\pgfqpoint{4.208942in}{1.402236in}}{\pgfqpoint{4.216075in}{1.402236in}}%
\pgfpathclose%
\pgfusepath{stroke,fill}%
\end{pgfscope}%
\begin{pgfscope}%
\pgfpathrectangle{\pgfqpoint{2.867647in}{0.500000in}}{\pgfqpoint{1.764706in}{1.700000in}}%
\pgfusepath{clip}%
\pgfsetbuttcap%
\pgfsetroundjoin%
\definecolor{currentfill}{rgb}{0.980678,0.914765,0.856766}%
\pgfsetfillcolor{currentfill}%
\pgfsetlinewidth{0.311001pt}%
\definecolor{currentstroke}{rgb}{1.000000,1.000000,1.000000}%
\pgfsetstrokecolor{currentstroke}%
\pgfsetdash{}{0pt}%
\pgfpathmoveto{\pgfqpoint{4.183712in}{1.164848in}}%
\pgfpathcurveto{\pgfqpoint{4.190845in}{1.164848in}}{\pgfqpoint{4.197686in}{1.167682in}}{\pgfqpoint{4.202730in}{1.172726in}}%
\pgfpathcurveto{\pgfqpoint{4.207774in}{1.177769in}}{\pgfqpoint{4.210608in}{1.184611in}}{\pgfqpoint{4.210608in}{1.191744in}}%
\pgfpathcurveto{\pgfqpoint{4.210608in}{1.198877in}}{\pgfqpoint{4.207774in}{1.205718in}}{\pgfqpoint{4.202730in}{1.210762in}}%
\pgfpathcurveto{\pgfqpoint{4.197686in}{1.215806in}}{\pgfqpoint{4.190845in}{1.218640in}}{\pgfqpoint{4.183712in}{1.218640in}}%
\pgfpathcurveto{\pgfqpoint{4.176579in}{1.218640in}}{\pgfqpoint{4.169737in}{1.215806in}}{\pgfqpoint{4.164694in}{1.210762in}}%
\pgfpathcurveto{\pgfqpoint{4.159650in}{1.205718in}}{\pgfqpoint{4.156816in}{1.198877in}}{\pgfqpoint{4.156816in}{1.191744in}}%
\pgfpathcurveto{\pgfqpoint{4.156816in}{1.184611in}}{\pgfqpoint{4.159650in}{1.177769in}}{\pgfqpoint{4.164694in}{1.172726in}}%
\pgfpathcurveto{\pgfqpoint{4.169737in}{1.167682in}}{\pgfqpoint{4.176579in}{1.164848in}}{\pgfqpoint{4.183712in}{1.164848in}}%
\pgfpathclose%
\pgfusepath{stroke,fill}%
\end{pgfscope}%
\begin{pgfscope}%
\pgfpathrectangle{\pgfqpoint{2.867647in}{0.500000in}}{\pgfqpoint{1.764706in}{1.700000in}}%
\pgfusepath{clip}%
\pgfsetbuttcap%
\pgfsetroundjoin%
\definecolor{currentfill}{rgb}{0.966328,0.750560,0.616961}%
\pgfsetfillcolor{currentfill}%
\pgfsetlinewidth{0.311001pt}%
\definecolor{currentstroke}{rgb}{1.000000,1.000000,1.000000}%
\pgfsetstrokecolor{currentstroke}%
\pgfsetdash{}{0pt}%
\pgfpathmoveto{\pgfqpoint{4.272971in}{1.495869in}}%
\pgfpathcurveto{\pgfqpoint{4.280104in}{1.495869in}}{\pgfqpoint{4.286945in}{1.498703in}}{\pgfqpoint{4.291989in}{1.503747in}}%
\pgfpathcurveto{\pgfqpoint{4.297033in}{1.508791in}}{\pgfqpoint{4.299867in}{1.515632in}}{\pgfqpoint{4.299867in}{1.522765in}}%
\pgfpathcurveto{\pgfqpoint{4.299867in}{1.529898in}}{\pgfqpoint{4.297033in}{1.536740in}}{\pgfqpoint{4.291989in}{1.541783in}}%
\pgfpathcurveto{\pgfqpoint{4.286945in}{1.546827in}}{\pgfqpoint{4.280104in}{1.549661in}}{\pgfqpoint{4.272971in}{1.549661in}}%
\pgfpathcurveto{\pgfqpoint{4.265838in}{1.549661in}}{\pgfqpoint{4.258996in}{1.546827in}}{\pgfqpoint{4.253953in}{1.541783in}}%
\pgfpathcurveto{\pgfqpoint{4.248909in}{1.536740in}}{\pgfqpoint{4.246075in}{1.529898in}}{\pgfqpoint{4.246075in}{1.522765in}}%
\pgfpathcurveto{\pgfqpoint{4.246075in}{1.515632in}}{\pgfqpoint{4.248909in}{1.508791in}}{\pgfqpoint{4.253953in}{1.503747in}}%
\pgfpathcurveto{\pgfqpoint{4.258996in}{1.498703in}}{\pgfqpoint{4.265838in}{1.495869in}}{\pgfqpoint{4.272971in}{1.495869in}}%
\pgfpathclose%
\pgfusepath{stroke,fill}%
\end{pgfscope}%
\begin{pgfscope}%
\pgfpathrectangle{\pgfqpoint{2.867647in}{0.500000in}}{\pgfqpoint{1.764706in}{1.700000in}}%
\pgfusepath{clip}%
\pgfsetbuttcap%
\pgfsetroundjoin%
\definecolor{currentfill}{rgb}{0.976961,0.885681,0.814303}%
\pgfsetfillcolor{currentfill}%
\pgfsetlinewidth{0.311001pt}%
\definecolor{currentstroke}{rgb}{1.000000,1.000000,1.000000}%
\pgfsetstrokecolor{currentstroke}%
\pgfsetdash{}{0pt}%
\pgfpathmoveto{\pgfqpoint{4.108124in}{1.076315in}}%
\pgfpathcurveto{\pgfqpoint{4.115257in}{1.076315in}}{\pgfqpoint{4.122099in}{1.079149in}}{\pgfqpoint{4.127142in}{1.084193in}}%
\pgfpathcurveto{\pgfqpoint{4.132186in}{1.089236in}}{\pgfqpoint{4.135020in}{1.096078in}}{\pgfqpoint{4.135020in}{1.103211in}}%
\pgfpathcurveto{\pgfqpoint{4.135020in}{1.110344in}}{\pgfqpoint{4.132186in}{1.117185in}}{\pgfqpoint{4.127142in}{1.122229in}}%
\pgfpathcurveto{\pgfqpoint{4.122099in}{1.127273in}}{\pgfqpoint{4.115257in}{1.130107in}}{\pgfqpoint{4.108124in}{1.130107in}}%
\pgfpathcurveto{\pgfqpoint{4.100991in}{1.130107in}}{\pgfqpoint{4.094150in}{1.127273in}}{\pgfqpoint{4.089106in}{1.122229in}}%
\pgfpathcurveto{\pgfqpoint{4.084062in}{1.117185in}}{\pgfqpoint{4.081229in}{1.110344in}}{\pgfqpoint{4.081229in}{1.103211in}}%
\pgfpathcurveto{\pgfqpoint{4.081229in}{1.096078in}}{\pgfqpoint{4.084062in}{1.089236in}}{\pgfqpoint{4.089106in}{1.084193in}}%
\pgfpathcurveto{\pgfqpoint{4.094150in}{1.079149in}}{\pgfqpoint{4.100991in}{1.076315in}}{\pgfqpoint{4.108124in}{1.076315in}}%
\pgfpathclose%
\pgfusepath{stroke,fill}%
\end{pgfscope}%
\begin{pgfscope}%
\pgfpathrectangle{\pgfqpoint{2.867647in}{0.500000in}}{\pgfqpoint{1.764706in}{1.700000in}}%
\pgfusepath{clip}%
\pgfsetbuttcap%
\pgfsetroundjoin%
\definecolor{currentfill}{rgb}{0.971694,0.833208,0.737161}%
\pgfsetfillcolor{currentfill}%
\pgfsetlinewidth{0.311001pt}%
\definecolor{currentstroke}{rgb}{1.000000,1.000000,1.000000}%
\pgfsetstrokecolor{currentstroke}%
\pgfsetdash{}{0pt}%
\pgfpathmoveto{\pgfqpoint{4.109296in}{1.243539in}}%
\pgfpathcurveto{\pgfqpoint{4.116429in}{1.243539in}}{\pgfqpoint{4.123270in}{1.246373in}}{\pgfqpoint{4.128314in}{1.251416in}}%
\pgfpathcurveto{\pgfqpoint{4.133358in}{1.256460in}}{\pgfqpoint{4.136192in}{1.263302in}}{\pgfqpoint{4.136192in}{1.270434in}}%
\pgfpathcurveto{\pgfqpoint{4.136192in}{1.277567in}}{\pgfqpoint{4.133358in}{1.284409in}}{\pgfqpoint{4.128314in}{1.289453in}}%
\pgfpathcurveto{\pgfqpoint{4.123270in}{1.294496in}}{\pgfqpoint{4.116429in}{1.297330in}}{\pgfqpoint{4.109296in}{1.297330in}}%
\pgfpathcurveto{\pgfqpoint{4.102163in}{1.297330in}}{\pgfqpoint{4.095321in}{1.294496in}}{\pgfqpoint{4.090278in}{1.289453in}}%
\pgfpathcurveto{\pgfqpoint{4.085234in}{1.284409in}}{\pgfqpoint{4.082400in}{1.277567in}}{\pgfqpoint{4.082400in}{1.270434in}}%
\pgfpathcurveto{\pgfqpoint{4.082400in}{1.263302in}}{\pgfqpoint{4.085234in}{1.256460in}}{\pgfqpoint{4.090278in}{1.251416in}}%
\pgfpathcurveto{\pgfqpoint{4.095321in}{1.246373in}}{\pgfqpoint{4.102163in}{1.243539in}}{\pgfqpoint{4.109296in}{1.243539in}}%
\pgfpathclose%
\pgfusepath{stroke,fill}%
\end{pgfscope}%
\begin{pgfscope}%
\pgfpathrectangle{\pgfqpoint{2.867647in}{0.500000in}}{\pgfqpoint{1.764706in}{1.700000in}}%
\pgfusepath{clip}%
\pgfsetbuttcap%
\pgfsetroundjoin%
\definecolor{currentfill}{rgb}{0.979124,0.903132,0.839793}%
\pgfsetfillcolor{currentfill}%
\pgfsetlinewidth{0.311001pt}%
\definecolor{currentstroke}{rgb}{1.000000,1.000000,1.000000}%
\pgfsetstrokecolor{currentstroke}%
\pgfsetdash{}{0pt}%
\pgfpathmoveto{\pgfqpoint{4.139353in}{1.469877in}}%
\pgfpathcurveto{\pgfqpoint{4.146485in}{1.469877in}}{\pgfqpoint{4.153327in}{1.472711in}}{\pgfqpoint{4.158371in}{1.477754in}}%
\pgfpathcurveto{\pgfqpoint{4.163414in}{1.482798in}}{\pgfqpoint{4.166248in}{1.489640in}}{\pgfqpoint{4.166248in}{1.496772in}}%
\pgfpathcurveto{\pgfqpoint{4.166248in}{1.503905in}}{\pgfqpoint{4.163414in}{1.510747in}}{\pgfqpoint{4.158371in}{1.515791in}}%
\pgfpathcurveto{\pgfqpoint{4.153327in}{1.520834in}}{\pgfqpoint{4.146485in}{1.523668in}}{\pgfqpoint{4.139353in}{1.523668in}}%
\pgfpathcurveto{\pgfqpoint{4.132220in}{1.523668in}}{\pgfqpoint{4.125378in}{1.520834in}}{\pgfqpoint{4.120334in}{1.515791in}}%
\pgfpathcurveto{\pgfqpoint{4.115291in}{1.510747in}}{\pgfqpoint{4.112457in}{1.503905in}}{\pgfqpoint{4.112457in}{1.496772in}}%
\pgfpathcurveto{\pgfqpoint{4.112457in}{1.489640in}}{\pgfqpoint{4.115291in}{1.482798in}}{\pgfqpoint{4.120334in}{1.477754in}}%
\pgfpathcurveto{\pgfqpoint{4.125378in}{1.472711in}}{\pgfqpoint{4.132220in}{1.469877in}}{\pgfqpoint{4.139353in}{1.469877in}}%
\pgfpathclose%
\pgfusepath{stroke,fill}%
\end{pgfscope}%
\begin{pgfscope}%
\pgfpathrectangle{\pgfqpoint{2.867647in}{0.500000in}}{\pgfqpoint{1.764706in}{1.700000in}}%
\pgfusepath{clip}%
\pgfsetbuttcap%
\pgfsetroundjoin%
\definecolor{currentfill}{rgb}{0.963379,0.625574,0.465113}%
\pgfsetfillcolor{currentfill}%
\pgfsetlinewidth{0.311001pt}%
\definecolor{currentstroke}{rgb}{1.000000,1.000000,1.000000}%
\pgfsetstrokecolor{currentstroke}%
\pgfsetdash{}{0pt}%
\pgfpathmoveto{\pgfqpoint{4.119670in}{0.876453in}}%
\pgfpathcurveto{\pgfqpoint{4.126803in}{0.876453in}}{\pgfqpoint{4.133645in}{0.879287in}}{\pgfqpoint{4.138688in}{0.884331in}}%
\pgfpathcurveto{\pgfqpoint{4.143732in}{0.889375in}}{\pgfqpoint{4.146566in}{0.896216in}}{\pgfqpoint{4.146566in}{0.903349in}}%
\pgfpathcurveto{\pgfqpoint{4.146566in}{0.910482in}}{\pgfqpoint{4.143732in}{0.917324in}}{\pgfqpoint{4.138688in}{0.922367in}}%
\pgfpathcurveto{\pgfqpoint{4.133645in}{0.927411in}}{\pgfqpoint{4.126803in}{0.930245in}}{\pgfqpoint{4.119670in}{0.930245in}}%
\pgfpathcurveto{\pgfqpoint{4.112537in}{0.930245in}}{\pgfqpoint{4.105696in}{0.927411in}}{\pgfqpoint{4.100652in}{0.922367in}}%
\pgfpathcurveto{\pgfqpoint{4.095608in}{0.917324in}}{\pgfqpoint{4.092775in}{0.910482in}}{\pgfqpoint{4.092775in}{0.903349in}}%
\pgfpathcurveto{\pgfqpoint{4.092775in}{0.896216in}}{\pgfqpoint{4.095608in}{0.889375in}}{\pgfqpoint{4.100652in}{0.884331in}}%
\pgfpathcurveto{\pgfqpoint{4.105696in}{0.879287in}}{\pgfqpoint{4.112537in}{0.876453in}}{\pgfqpoint{4.119670in}{0.876453in}}%
\pgfpathclose%
\pgfusepath{stroke,fill}%
\end{pgfscope}%
\begin{pgfscope}%
\pgfpathrectangle{\pgfqpoint{2.867647in}{0.500000in}}{\pgfqpoint{1.764706in}{1.700000in}}%
\pgfusepath{clip}%
\pgfsetbuttcap%
\pgfsetroundjoin%
\definecolor{currentfill}{rgb}{0.962283,0.593046,0.431453}%
\pgfsetfillcolor{currentfill}%
\pgfsetlinewidth{0.311001pt}%
\definecolor{currentstroke}{rgb}{1.000000,1.000000,1.000000}%
\pgfsetstrokecolor{currentstroke}%
\pgfsetdash{}{0pt}%
\pgfpathmoveto{\pgfqpoint{3.926500in}{1.712782in}}%
\pgfpathcurveto{\pgfqpoint{3.933633in}{1.712782in}}{\pgfqpoint{3.940475in}{1.715616in}}{\pgfqpoint{3.945518in}{1.720659in}}%
\pgfpathcurveto{\pgfqpoint{3.950562in}{1.725703in}}{\pgfqpoint{3.953396in}{1.732545in}}{\pgfqpoint{3.953396in}{1.739678in}}%
\pgfpathcurveto{\pgfqpoint{3.953396in}{1.746810in}}{\pgfqpoint{3.950562in}{1.753652in}}{\pgfqpoint{3.945518in}{1.758696in}}%
\pgfpathcurveto{\pgfqpoint{3.940475in}{1.763739in}}{\pgfqpoint{3.933633in}{1.766573in}}{\pgfqpoint{3.926500in}{1.766573in}}%
\pgfpathcurveto{\pgfqpoint{3.919367in}{1.766573in}}{\pgfqpoint{3.912526in}{1.763739in}}{\pgfqpoint{3.907482in}{1.758696in}}%
\pgfpathcurveto{\pgfqpoint{3.902438in}{1.753652in}}{\pgfqpoint{3.899604in}{1.746810in}}{\pgfqpoint{3.899604in}{1.739678in}}%
\pgfpathcurveto{\pgfqpoint{3.899604in}{1.732545in}}{\pgfqpoint{3.902438in}{1.725703in}}{\pgfqpoint{3.907482in}{1.720659in}}%
\pgfpathcurveto{\pgfqpoint{3.912526in}{1.715616in}}{\pgfqpoint{3.919367in}{1.712782in}}{\pgfqpoint{3.926500in}{1.712782in}}%
\pgfpathclose%
\pgfusepath{stroke,fill}%
\end{pgfscope}%
\begin{pgfscope}%
\pgfpathrectangle{\pgfqpoint{2.867647in}{0.500000in}}{\pgfqpoint{1.764706in}{1.700000in}}%
\pgfusepath{clip}%
\pgfsetbuttcap%
\pgfsetroundjoin%
\definecolor{currentfill}{rgb}{0.964920,0.695342,0.545192}%
\pgfsetfillcolor{currentfill}%
\pgfsetlinewidth{0.311001pt}%
\definecolor{currentstroke}{rgb}{1.000000,1.000000,1.000000}%
\pgfsetstrokecolor{currentstroke}%
\pgfsetdash{}{0pt}%
\pgfpathmoveto{\pgfqpoint{4.011709in}{1.562684in}}%
\pgfpathcurveto{\pgfqpoint{4.018842in}{1.562684in}}{\pgfqpoint{4.025683in}{1.565518in}}{\pgfqpoint{4.030727in}{1.570562in}}%
\pgfpathcurveto{\pgfqpoint{4.035771in}{1.575605in}}{\pgfqpoint{4.038605in}{1.582447in}}{\pgfqpoint{4.038605in}{1.589580in}}%
\pgfpathcurveto{\pgfqpoint{4.038605in}{1.596713in}}{\pgfqpoint{4.035771in}{1.603554in}}{\pgfqpoint{4.030727in}{1.608598in}}%
\pgfpathcurveto{\pgfqpoint{4.025683in}{1.613641in}}{\pgfqpoint{4.018842in}{1.616475in}}{\pgfqpoint{4.011709in}{1.616475in}}%
\pgfpathcurveto{\pgfqpoint{4.004576in}{1.616475in}}{\pgfqpoint{3.997735in}{1.613641in}}{\pgfqpoint{3.992691in}{1.608598in}}%
\pgfpathcurveto{\pgfqpoint{3.987647in}{1.603554in}}{\pgfqpoint{3.984813in}{1.596713in}}{\pgfqpoint{3.984813in}{1.589580in}}%
\pgfpathcurveto{\pgfqpoint{3.984813in}{1.582447in}}{\pgfqpoint{3.987647in}{1.575605in}}{\pgfqpoint{3.992691in}{1.570562in}}%
\pgfpathcurveto{\pgfqpoint{3.997735in}{1.565518in}}{\pgfqpoint{4.004576in}{1.562684in}}{\pgfqpoint{4.011709in}{1.562684in}}%
\pgfpathclose%
\pgfusepath{stroke,fill}%
\end{pgfscope}%
\begin{pgfscope}%
\pgfpathrectangle{\pgfqpoint{2.867647in}{0.500000in}}{\pgfqpoint{1.764706in}{1.700000in}}%
\pgfusepath{clip}%
\pgfsetbuttcap%
\pgfsetroundjoin%
\definecolor{currentfill}{rgb}{0.962283,0.593046,0.431453}%
\pgfsetfillcolor{currentfill}%
\pgfsetlinewidth{0.311001pt}%
\definecolor{currentstroke}{rgb}{1.000000,1.000000,1.000000}%
\pgfsetstrokecolor{currentstroke}%
\pgfsetdash{}{0pt}%
\pgfpathmoveto{\pgfqpoint{3.950130in}{0.980244in}}%
\pgfpathcurveto{\pgfqpoint{3.957263in}{0.980244in}}{\pgfqpoint{3.964105in}{0.983078in}}{\pgfqpoint{3.969149in}{0.988121in}}%
\pgfpathcurveto{\pgfqpoint{3.974192in}{0.993165in}}{\pgfqpoint{3.977026in}{1.000007in}}{\pgfqpoint{3.977026in}{1.007140in}}%
\pgfpathcurveto{\pgfqpoint{3.977026in}{1.014272in}}{\pgfqpoint{3.974192in}{1.021114in}}{\pgfqpoint{3.969149in}{1.026158in}}%
\pgfpathcurveto{\pgfqpoint{3.964105in}{1.031201in}}{\pgfqpoint{3.957263in}{1.034035in}}{\pgfqpoint{3.950130in}{1.034035in}}%
\pgfpathcurveto{\pgfqpoint{3.942998in}{1.034035in}}{\pgfqpoint{3.936156in}{1.031201in}}{\pgfqpoint{3.931112in}{1.026158in}}%
\pgfpathcurveto{\pgfqpoint{3.926069in}{1.021114in}}{\pgfqpoint{3.923235in}{1.014272in}}{\pgfqpoint{3.923235in}{1.007140in}}%
\pgfpathcurveto{\pgfqpoint{3.923235in}{1.000007in}}{\pgfqpoint{3.926069in}{0.993165in}}{\pgfqpoint{3.931112in}{0.988121in}}%
\pgfpathcurveto{\pgfqpoint{3.936156in}{0.983078in}}{\pgfqpoint{3.942998in}{0.980244in}}{\pgfqpoint{3.950130in}{0.980244in}}%
\pgfpathclose%
\pgfusepath{stroke,fill}%
\end{pgfscope}%
\begin{pgfscope}%
\pgfpathrectangle{\pgfqpoint{2.867647in}{0.500000in}}{\pgfqpoint{1.764706in}{1.700000in}}%
\pgfusepath{clip}%
\pgfsetbuttcap%
\pgfsetroundjoin%
\definecolor{currentfill}{rgb}{0.979124,0.903132,0.839793}%
\pgfsetfillcolor{currentfill}%
\pgfsetlinewidth{0.311001pt}%
\definecolor{currentstroke}{rgb}{1.000000,1.000000,1.000000}%
\pgfsetstrokecolor{currentstroke}%
\pgfsetdash{}{0pt}%
\pgfpathmoveto{\pgfqpoint{4.149116in}{1.237437in}}%
\pgfpathcurveto{\pgfqpoint{4.156249in}{1.237437in}}{\pgfqpoint{4.163090in}{1.240271in}}{\pgfqpoint{4.168134in}{1.245315in}}%
\pgfpathcurveto{\pgfqpoint{4.173178in}{1.250358in}}{\pgfqpoint{4.176012in}{1.257200in}}{\pgfqpoint{4.176012in}{1.264333in}}%
\pgfpathcurveto{\pgfqpoint{4.176012in}{1.271466in}}{\pgfqpoint{4.173178in}{1.278307in}}{\pgfqpoint{4.168134in}{1.283351in}}%
\pgfpathcurveto{\pgfqpoint{4.163090in}{1.288395in}}{\pgfqpoint{4.156249in}{1.291229in}}{\pgfqpoint{4.149116in}{1.291229in}}%
\pgfpathcurveto{\pgfqpoint{4.141983in}{1.291229in}}{\pgfqpoint{4.135141in}{1.288395in}}{\pgfqpoint{4.130098in}{1.283351in}}%
\pgfpathcurveto{\pgfqpoint{4.125054in}{1.278307in}}{\pgfqpoint{4.122220in}{1.271466in}}{\pgfqpoint{4.122220in}{1.264333in}}%
\pgfpathcurveto{\pgfqpoint{4.122220in}{1.257200in}}{\pgfqpoint{4.125054in}{1.250358in}}{\pgfqpoint{4.130098in}{1.245315in}}%
\pgfpathcurveto{\pgfqpoint{4.135141in}{1.240271in}}{\pgfqpoint{4.141983in}{1.237437in}}{\pgfqpoint{4.149116in}{1.237437in}}%
\pgfpathclose%
\pgfusepath{stroke,fill}%
\end{pgfscope}%
\begin{pgfscope}%
\pgfpathrectangle{\pgfqpoint{2.867647in}{0.500000in}}{\pgfqpoint{1.764706in}{1.700000in}}%
\pgfusepath{clip}%
\pgfsetbuttcap%
\pgfsetroundjoin%
\definecolor{currentfill}{rgb}{0.981377,0.920617,0.865369}%
\pgfsetfillcolor{currentfill}%
\pgfsetlinewidth{0.311001pt}%
\definecolor{currentstroke}{rgb}{1.000000,1.000000,1.000000}%
\pgfsetstrokecolor{currentstroke}%
\pgfsetdash{}{0pt}%
\pgfpathmoveto{\pgfqpoint{4.175290in}{1.179239in}}%
\pgfpathcurveto{\pgfqpoint{4.182423in}{1.179239in}}{\pgfqpoint{4.189264in}{1.182073in}}{\pgfqpoint{4.194308in}{1.187116in}}%
\pgfpathcurveto{\pgfqpoint{4.199352in}{1.192160in}}{\pgfqpoint{4.202186in}{1.199002in}}{\pgfqpoint{4.202186in}{1.206135in}}%
\pgfpathcurveto{\pgfqpoint{4.202186in}{1.213267in}}{\pgfqpoint{4.199352in}{1.220109in}}{\pgfqpoint{4.194308in}{1.225153in}}%
\pgfpathcurveto{\pgfqpoint{4.189264in}{1.230196in}}{\pgfqpoint{4.182423in}{1.233030in}}{\pgfqpoint{4.175290in}{1.233030in}}%
\pgfpathcurveto{\pgfqpoint{4.168157in}{1.233030in}}{\pgfqpoint{4.161315in}{1.230196in}}{\pgfqpoint{4.156272in}{1.225153in}}%
\pgfpathcurveto{\pgfqpoint{4.151228in}{1.220109in}}{\pgfqpoint{4.148394in}{1.213267in}}{\pgfqpoint{4.148394in}{1.206135in}}%
\pgfpathcurveto{\pgfqpoint{4.148394in}{1.199002in}}{\pgfqpoint{4.151228in}{1.192160in}}{\pgfqpoint{4.156272in}{1.187116in}}%
\pgfpathcurveto{\pgfqpoint{4.161315in}{1.182073in}}{\pgfqpoint{4.168157in}{1.179239in}}{\pgfqpoint{4.175290in}{1.179239in}}%
\pgfpathclose%
\pgfusepath{stroke,fill}%
\end{pgfscope}%
\begin{pgfscope}%
\pgfpathrectangle{\pgfqpoint{2.867647in}{0.500000in}}{\pgfqpoint{1.764706in}{1.700000in}}%
\pgfusepath{clip}%
\pgfsetbuttcap%
\pgfsetroundjoin%
\definecolor{currentfill}{rgb}{0.964920,0.695342,0.545192}%
\pgfsetfillcolor{currentfill}%
\pgfsetlinewidth{0.311001pt}%
\definecolor{currentstroke}{rgb}{1.000000,1.000000,1.000000}%
\pgfsetstrokecolor{currentstroke}%
\pgfsetdash{}{0pt}%
\pgfpathmoveto{\pgfqpoint{4.303158in}{1.416448in}}%
\pgfpathcurveto{\pgfqpoint{4.310291in}{1.416448in}}{\pgfqpoint{4.317132in}{1.419282in}}{\pgfqpoint{4.322176in}{1.424325in}}%
\pgfpathcurveto{\pgfqpoint{4.327220in}{1.429369in}}{\pgfqpoint{4.330053in}{1.436211in}}{\pgfqpoint{4.330053in}{1.443344in}}%
\pgfpathcurveto{\pgfqpoint{4.330053in}{1.450476in}}{\pgfqpoint{4.327220in}{1.457318in}}{\pgfqpoint{4.322176in}{1.462362in}}%
\pgfpathcurveto{\pgfqpoint{4.317132in}{1.467405in}}{\pgfqpoint{4.310291in}{1.470239in}}{\pgfqpoint{4.303158in}{1.470239in}}%
\pgfpathcurveto{\pgfqpoint{4.296025in}{1.470239in}}{\pgfqpoint{4.289183in}{1.467405in}}{\pgfqpoint{4.284140in}{1.462362in}}%
\pgfpathcurveto{\pgfqpoint{4.279096in}{1.457318in}}{\pgfqpoint{4.276262in}{1.450476in}}{\pgfqpoint{4.276262in}{1.443344in}}%
\pgfpathcurveto{\pgfqpoint{4.276262in}{1.436211in}}{\pgfqpoint{4.279096in}{1.429369in}}{\pgfqpoint{4.284140in}{1.424325in}}%
\pgfpathcurveto{\pgfqpoint{4.289183in}{1.419282in}}{\pgfqpoint{4.296025in}{1.416448in}}{\pgfqpoint{4.303158in}{1.416448in}}%
\pgfpathclose%
\pgfusepath{stroke,fill}%
\end{pgfscope}%
\begin{pgfscope}%
\pgfpathrectangle{\pgfqpoint{2.867647in}{0.500000in}}{\pgfqpoint{1.764706in}{1.700000in}}%
\pgfusepath{clip}%
\pgfsetbuttcap%
\pgfsetroundjoin%
\definecolor{currentfill}{rgb}{0.963379,0.625574,0.465113}%
\pgfsetfillcolor{currentfill}%
\pgfsetlinewidth{0.311001pt}%
\definecolor{currentstroke}{rgb}{1.000000,1.000000,1.000000}%
\pgfsetstrokecolor{currentstroke}%
\pgfsetdash{}{0pt}%
\pgfpathmoveto{\pgfqpoint{4.238299in}{0.987868in}}%
\pgfpathcurveto{\pgfqpoint{4.245432in}{0.987868in}}{\pgfqpoint{4.252274in}{0.990701in}}{\pgfqpoint{4.257318in}{0.995745in}}%
\pgfpathcurveto{\pgfqpoint{4.262361in}{1.000789in}}{\pgfqpoint{4.265195in}{1.007630in}}{\pgfqpoint{4.265195in}{1.014763in}}%
\pgfpathcurveto{\pgfqpoint{4.265195in}{1.021896in}}{\pgfqpoint{4.262361in}{1.028738in}}{\pgfqpoint{4.257318in}{1.033781in}}%
\pgfpathcurveto{\pgfqpoint{4.252274in}{1.038825in}}{\pgfqpoint{4.245432in}{1.041659in}}{\pgfqpoint{4.238299in}{1.041659in}}%
\pgfpathcurveto{\pgfqpoint{4.231167in}{1.041659in}}{\pgfqpoint{4.224325in}{1.038825in}}{\pgfqpoint{4.219281in}{1.033781in}}%
\pgfpathcurveto{\pgfqpoint{4.214238in}{1.028738in}}{\pgfqpoint{4.211404in}{1.021896in}}{\pgfqpoint{4.211404in}{1.014763in}}%
\pgfpathcurveto{\pgfqpoint{4.211404in}{1.007630in}}{\pgfqpoint{4.214238in}{1.000789in}}{\pgfqpoint{4.219281in}{0.995745in}}%
\pgfpathcurveto{\pgfqpoint{4.224325in}{0.990701in}}{\pgfqpoint{4.231167in}{0.987868in}}{\pgfqpoint{4.238299in}{0.987868in}}%
\pgfpathclose%
\pgfusepath{stroke,fill}%
\end{pgfscope}%
\begin{pgfscope}%
\pgfpathrectangle{\pgfqpoint{2.867647in}{0.500000in}}{\pgfqpoint{1.764706in}{1.700000in}}%
\pgfusepath{clip}%
\pgfsetbuttcap%
\pgfsetroundjoin%
\definecolor{currentfill}{rgb}{0.875073,0.185874,0.265297}%
\pgfsetfillcolor{currentfill}%
\pgfsetlinewidth{0.311001pt}%
\definecolor{currentstroke}{rgb}{1.000000,1.000000,1.000000}%
\pgfsetstrokecolor{currentstroke}%
\pgfsetdash{}{0pt}%
\pgfpathmoveto{\pgfqpoint{4.393489in}{1.382523in}}%
\pgfpathcurveto{\pgfqpoint{4.400622in}{1.382523in}}{\pgfqpoint{4.407464in}{1.385357in}}{\pgfqpoint{4.412507in}{1.390401in}}%
\pgfpathcurveto{\pgfqpoint{4.417551in}{1.395444in}}{\pgfqpoint{4.420385in}{1.402286in}}{\pgfqpoint{4.420385in}{1.409419in}}%
\pgfpathcurveto{\pgfqpoint{4.420385in}{1.416552in}}{\pgfqpoint{4.417551in}{1.423393in}}{\pgfqpoint{4.412507in}{1.428437in}}%
\pgfpathcurveto{\pgfqpoint{4.407464in}{1.433481in}}{\pgfqpoint{4.400622in}{1.436314in}}{\pgfqpoint{4.393489in}{1.436314in}}%
\pgfpathcurveto{\pgfqpoint{4.386356in}{1.436314in}}{\pgfqpoint{4.379515in}{1.433481in}}{\pgfqpoint{4.374471in}{1.428437in}}%
\pgfpathcurveto{\pgfqpoint{4.369427in}{1.423393in}}{\pgfqpoint{4.366593in}{1.416552in}}{\pgfqpoint{4.366593in}{1.409419in}}%
\pgfpathcurveto{\pgfqpoint{4.366593in}{1.402286in}}{\pgfqpoint{4.369427in}{1.395444in}}{\pgfqpoint{4.374471in}{1.390401in}}%
\pgfpathcurveto{\pgfqpoint{4.379515in}{1.385357in}}{\pgfqpoint{4.386356in}{1.382523in}}{\pgfqpoint{4.393489in}{1.382523in}}%
\pgfpathclose%
\pgfusepath{stroke,fill}%
\end{pgfscope}%
\begin{pgfscope}%
\pgfpathrectangle{\pgfqpoint{2.867647in}{0.500000in}}{\pgfqpoint{1.764706in}{1.700000in}}%
\pgfusepath{clip}%
\pgfsetbuttcap%
\pgfsetroundjoin%
\definecolor{currentfill}{rgb}{0.972201,0.839051,0.745789}%
\pgfsetfillcolor{currentfill}%
\pgfsetlinewidth{0.311001pt}%
\definecolor{currentstroke}{rgb}{1.000000,1.000000,1.000000}%
\pgfsetstrokecolor{currentstroke}%
\pgfsetdash{}{0pt}%
\pgfpathmoveto{\pgfqpoint{4.083507in}{1.673962in}}%
\pgfpathcurveto{\pgfqpoint{4.090640in}{1.673962in}}{\pgfqpoint{4.097482in}{1.676796in}}{\pgfqpoint{4.102525in}{1.681840in}}%
\pgfpathcurveto{\pgfqpoint{4.107569in}{1.686883in}}{\pgfqpoint{4.110403in}{1.693725in}}{\pgfqpoint{4.110403in}{1.700858in}}%
\pgfpathcurveto{\pgfqpoint{4.110403in}{1.707991in}}{\pgfqpoint{4.107569in}{1.714832in}}{\pgfqpoint{4.102525in}{1.719876in}}%
\pgfpathcurveto{\pgfqpoint{4.097482in}{1.724920in}}{\pgfqpoint{4.090640in}{1.727753in}}{\pgfqpoint{4.083507in}{1.727753in}}%
\pgfpathcurveto{\pgfqpoint{4.076374in}{1.727753in}}{\pgfqpoint{4.069533in}{1.724920in}}{\pgfqpoint{4.064489in}{1.719876in}}%
\pgfpathcurveto{\pgfqpoint{4.059445in}{1.714832in}}{\pgfqpoint{4.056611in}{1.707991in}}{\pgfqpoint{4.056611in}{1.700858in}}%
\pgfpathcurveto{\pgfqpoint{4.056611in}{1.693725in}}{\pgfqpoint{4.059445in}{1.686883in}}{\pgfqpoint{4.064489in}{1.681840in}}%
\pgfpathcurveto{\pgfqpoint{4.069533in}{1.676796in}}{\pgfqpoint{4.076374in}{1.673962in}}{\pgfqpoint{4.083507in}{1.673962in}}%
\pgfpathclose%
\pgfusepath{stroke,fill}%
\end{pgfscope}%
\begin{pgfscope}%
\pgfpathrectangle{\pgfqpoint{2.867647in}{0.500000in}}{\pgfqpoint{1.764706in}{1.700000in}}%
\pgfusepath{clip}%
\pgfsetbuttcap%
\pgfsetroundjoin%
\definecolor{currentfill}{rgb}{0.963884,0.644842,0.486120}%
\pgfsetfillcolor{currentfill}%
\pgfsetlinewidth{0.311001pt}%
\definecolor{currentstroke}{rgb}{1.000000,1.000000,1.000000}%
\pgfsetstrokecolor{currentstroke}%
\pgfsetdash{}{0pt}%
\pgfpathmoveto{\pgfqpoint{4.229158in}{1.660890in}}%
\pgfpathcurveto{\pgfqpoint{4.236291in}{1.660890in}}{\pgfqpoint{4.243133in}{1.663724in}}{\pgfqpoint{4.248177in}{1.668768in}}%
\pgfpathcurveto{\pgfqpoint{4.253220in}{1.673812in}}{\pgfqpoint{4.256054in}{1.680653in}}{\pgfqpoint{4.256054in}{1.687786in}}%
\pgfpathcurveto{\pgfqpoint{4.256054in}{1.694919in}}{\pgfqpoint{4.253220in}{1.701761in}}{\pgfqpoint{4.248177in}{1.706804in}}%
\pgfpathcurveto{\pgfqpoint{4.243133in}{1.711848in}}{\pgfqpoint{4.236291in}{1.714682in}}{\pgfqpoint{4.229158in}{1.714682in}}%
\pgfpathcurveto{\pgfqpoint{4.222026in}{1.714682in}}{\pgfqpoint{4.215184in}{1.711848in}}{\pgfqpoint{4.210140in}{1.706804in}}%
\pgfpathcurveto{\pgfqpoint{4.205097in}{1.701761in}}{\pgfqpoint{4.202263in}{1.694919in}}{\pgfqpoint{4.202263in}{1.687786in}}%
\pgfpathcurveto{\pgfqpoint{4.202263in}{1.680653in}}{\pgfqpoint{4.205097in}{1.673812in}}{\pgfqpoint{4.210140in}{1.668768in}}%
\pgfpathcurveto{\pgfqpoint{4.215184in}{1.663724in}}{\pgfqpoint{4.222026in}{1.660890in}}{\pgfqpoint{4.229158in}{1.660890in}}%
\pgfpathclose%
\pgfusepath{stroke,fill}%
\end{pgfscope}%
\begin{pgfscope}%
\pgfpathrectangle{\pgfqpoint{2.867647in}{0.500000in}}{\pgfqpoint{1.764706in}{1.700000in}}%
\pgfusepath{clip}%
\pgfsetbuttcap%
\pgfsetroundjoin%
\definecolor{currentfill}{rgb}{0.973271,0.850724,0.762998}%
\pgfsetfillcolor{currentfill}%
\pgfsetlinewidth{0.311001pt}%
\definecolor{currentstroke}{rgb}{1.000000,1.000000,1.000000}%
\pgfsetstrokecolor{currentstroke}%
\pgfsetdash{}{0pt}%
\pgfpathmoveto{\pgfqpoint{4.194551in}{1.585645in}}%
\pgfpathcurveto{\pgfqpoint{4.201684in}{1.585645in}}{\pgfqpoint{4.208526in}{1.588479in}}{\pgfqpoint{4.213569in}{1.593523in}}%
\pgfpathcurveto{\pgfqpoint{4.218613in}{1.598566in}}{\pgfqpoint{4.221447in}{1.605408in}}{\pgfqpoint{4.221447in}{1.612541in}}%
\pgfpathcurveto{\pgfqpoint{4.221447in}{1.619674in}}{\pgfqpoint{4.218613in}{1.626515in}}{\pgfqpoint{4.213569in}{1.631559in}}%
\pgfpathcurveto{\pgfqpoint{4.208526in}{1.636603in}}{\pgfqpoint{4.201684in}{1.639436in}}{\pgfqpoint{4.194551in}{1.639436in}}%
\pgfpathcurveto{\pgfqpoint{4.187418in}{1.639436in}}{\pgfqpoint{4.180577in}{1.636603in}}{\pgfqpoint{4.175533in}{1.631559in}}%
\pgfpathcurveto{\pgfqpoint{4.170489in}{1.626515in}}{\pgfqpoint{4.167655in}{1.619674in}}{\pgfqpoint{4.167655in}{1.612541in}}%
\pgfpathcurveto{\pgfqpoint{4.167655in}{1.605408in}}{\pgfqpoint{4.170489in}{1.598566in}}{\pgfqpoint{4.175533in}{1.593523in}}%
\pgfpathcurveto{\pgfqpoint{4.180577in}{1.588479in}}{\pgfqpoint{4.187418in}{1.585645in}}{\pgfqpoint{4.194551in}{1.585645in}}%
\pgfpathclose%
\pgfusepath{stroke,fill}%
\end{pgfscope}%
\begin{pgfscope}%
\pgfpathrectangle{\pgfqpoint{2.867647in}{0.500000in}}{\pgfqpoint{1.764706in}{1.700000in}}%
\pgfusepath{clip}%
\pgfsetbuttcap%
\pgfsetroundjoin%
\definecolor{currentfill}{rgb}{0.965753,0.732351,0.592427}%
\pgfsetfillcolor{currentfill}%
\pgfsetlinewidth{0.311001pt}%
\definecolor{currentstroke}{rgb}{1.000000,1.000000,1.000000}%
\pgfsetstrokecolor{currentstroke}%
\pgfsetdash{}{0pt}%
\pgfpathmoveto{\pgfqpoint{4.151809in}{0.935297in}}%
\pgfpathcurveto{\pgfqpoint{4.158942in}{0.935297in}}{\pgfqpoint{4.165783in}{0.938131in}}{\pgfqpoint{4.170827in}{0.943175in}}%
\pgfpathcurveto{\pgfqpoint{4.175871in}{0.948219in}}{\pgfqpoint{4.178704in}{0.955060in}}{\pgfqpoint{4.178704in}{0.962193in}}%
\pgfpathcurveto{\pgfqpoint{4.178704in}{0.969326in}}{\pgfqpoint{4.175871in}{0.976168in}}{\pgfqpoint{4.170827in}{0.981211in}}%
\pgfpathcurveto{\pgfqpoint{4.165783in}{0.986255in}}{\pgfqpoint{4.158942in}{0.989089in}}{\pgfqpoint{4.151809in}{0.989089in}}%
\pgfpathcurveto{\pgfqpoint{4.144676in}{0.989089in}}{\pgfqpoint{4.137834in}{0.986255in}}{\pgfqpoint{4.132791in}{0.981211in}}%
\pgfpathcurveto{\pgfqpoint{4.127747in}{0.976168in}}{\pgfqpoint{4.124913in}{0.969326in}}{\pgfqpoint{4.124913in}{0.962193in}}%
\pgfpathcurveto{\pgfqpoint{4.124913in}{0.955060in}}{\pgfqpoint{4.127747in}{0.948219in}}{\pgfqpoint{4.132791in}{0.943175in}}%
\pgfpathcurveto{\pgfqpoint{4.137834in}{0.938131in}}{\pgfqpoint{4.144676in}{0.935297in}}{\pgfqpoint{4.151809in}{0.935297in}}%
\pgfpathclose%
\pgfusepath{stroke,fill}%
\end{pgfscope}%
\begin{pgfscope}%
\pgfpathrectangle{\pgfqpoint{2.867647in}{0.500000in}}{\pgfqpoint{1.764706in}{1.700000in}}%
\pgfusepath{clip}%
\pgfsetbuttcap%
\pgfsetroundjoin%
\definecolor{currentfill}{rgb}{0.971202,0.827364,0.728520}%
\pgfsetfillcolor{currentfill}%
\pgfsetlinewidth{0.311001pt}%
\definecolor{currentstroke}{rgb}{1.000000,1.000000,1.000000}%
\pgfsetstrokecolor{currentstroke}%
\pgfsetdash{}{0pt}%
\pgfpathmoveto{\pgfqpoint{4.112955in}{1.359712in}}%
\pgfpathcurveto{\pgfqpoint{4.120088in}{1.359712in}}{\pgfqpoint{4.126929in}{1.362546in}}{\pgfqpoint{4.131973in}{1.367589in}}%
\pgfpathcurveto{\pgfqpoint{4.137017in}{1.372633in}}{\pgfqpoint{4.139851in}{1.379475in}}{\pgfqpoint{4.139851in}{1.386608in}}%
\pgfpathcurveto{\pgfqpoint{4.139851in}{1.393740in}}{\pgfqpoint{4.137017in}{1.400582in}}{\pgfqpoint{4.131973in}{1.405626in}}%
\pgfpathcurveto{\pgfqpoint{4.126929in}{1.410669in}}{\pgfqpoint{4.120088in}{1.413503in}}{\pgfqpoint{4.112955in}{1.413503in}}%
\pgfpathcurveto{\pgfqpoint{4.105822in}{1.413503in}}{\pgfqpoint{4.098980in}{1.410669in}}{\pgfqpoint{4.093937in}{1.405626in}}%
\pgfpathcurveto{\pgfqpoint{4.088893in}{1.400582in}}{\pgfqpoint{4.086059in}{1.393740in}}{\pgfqpoint{4.086059in}{1.386608in}}%
\pgfpathcurveto{\pgfqpoint{4.086059in}{1.379475in}}{\pgfqpoint{4.088893in}{1.372633in}}{\pgfqpoint{4.093937in}{1.367589in}}%
\pgfpathcurveto{\pgfqpoint{4.098980in}{1.362546in}}{\pgfqpoint{4.105822in}{1.359712in}}{\pgfqpoint{4.112955in}{1.359712in}}%
\pgfpathclose%
\pgfusepath{stroke,fill}%
\end{pgfscope}%
\begin{pgfscope}%
\pgfpathrectangle{\pgfqpoint{2.867647in}{0.500000in}}{\pgfqpoint{1.764706in}{1.700000in}}%
\pgfusepath{clip}%
\pgfsetbuttcap%
\pgfsetroundjoin%
\definecolor{currentfill}{rgb}{0.964173,0.657587,0.500469}%
\pgfsetfillcolor{currentfill}%
\pgfsetlinewidth{0.311001pt}%
\definecolor{currentstroke}{rgb}{1.000000,1.000000,1.000000}%
\pgfsetstrokecolor{currentstroke}%
\pgfsetdash{}{0pt}%
\pgfpathmoveto{\pgfqpoint{4.064851in}{0.870674in}}%
\pgfpathcurveto{\pgfqpoint{4.071984in}{0.870674in}}{\pgfqpoint{4.078826in}{0.873508in}}{\pgfqpoint{4.083869in}{0.878552in}}%
\pgfpathcurveto{\pgfqpoint{4.088913in}{0.883595in}}{\pgfqpoint{4.091747in}{0.890437in}}{\pgfqpoint{4.091747in}{0.897570in}}%
\pgfpathcurveto{\pgfqpoint{4.091747in}{0.904703in}}{\pgfqpoint{4.088913in}{0.911544in}}{\pgfqpoint{4.083869in}{0.916588in}}%
\pgfpathcurveto{\pgfqpoint{4.078826in}{0.921632in}}{\pgfqpoint{4.071984in}{0.924466in}}{\pgfqpoint{4.064851in}{0.924466in}}%
\pgfpathcurveto{\pgfqpoint{4.057718in}{0.924466in}}{\pgfqpoint{4.050877in}{0.921632in}}{\pgfqpoint{4.045833in}{0.916588in}}%
\pgfpathcurveto{\pgfqpoint{4.040789in}{0.911544in}}{\pgfqpoint{4.037955in}{0.904703in}}{\pgfqpoint{4.037955in}{0.897570in}}%
\pgfpathcurveto{\pgfqpoint{4.037955in}{0.890437in}}{\pgfqpoint{4.040789in}{0.883595in}}{\pgfqpoint{4.045833in}{0.878552in}}%
\pgfpathcurveto{\pgfqpoint{4.050877in}{0.873508in}}{\pgfqpoint{4.057718in}{0.870674in}}{\pgfqpoint{4.064851in}{0.870674in}}%
\pgfpathclose%
\pgfusepath{stroke,fill}%
\end{pgfscope}%
\begin{pgfscope}%
\pgfpathrectangle{\pgfqpoint{2.867647in}{0.500000in}}{\pgfqpoint{1.764706in}{1.700000in}}%
\pgfusepath{clip}%
\pgfsetbuttcap%
\pgfsetroundjoin%
\definecolor{currentfill}{rgb}{0.966812,0.762584,0.633643}%
\pgfsetfillcolor{currentfill}%
\pgfsetlinewidth{0.311001pt}%
\definecolor{currentstroke}{rgb}{1.000000,1.000000,1.000000}%
\pgfsetstrokecolor{currentstroke}%
\pgfsetdash{}{0pt}%
\pgfpathmoveto{\pgfqpoint{4.084903in}{0.917930in}}%
\pgfpathcurveto{\pgfqpoint{4.092035in}{0.917930in}}{\pgfqpoint{4.098877in}{0.920764in}}{\pgfqpoint{4.103921in}{0.925808in}}%
\pgfpathcurveto{\pgfqpoint{4.108964in}{0.930852in}}{\pgfqpoint{4.111798in}{0.937693in}}{\pgfqpoint{4.111798in}{0.944826in}}%
\pgfpathcurveto{\pgfqpoint{4.111798in}{0.951959in}}{\pgfqpoint{4.108964in}{0.958801in}}{\pgfqpoint{4.103921in}{0.963844in}}%
\pgfpathcurveto{\pgfqpoint{4.098877in}{0.968888in}}{\pgfqpoint{4.092035in}{0.971722in}}{\pgfqpoint{4.084903in}{0.971722in}}%
\pgfpathcurveto{\pgfqpoint{4.077770in}{0.971722in}}{\pgfqpoint{4.070928in}{0.968888in}}{\pgfqpoint{4.065885in}{0.963844in}}%
\pgfpathcurveto{\pgfqpoint{4.060841in}{0.958801in}}{\pgfqpoint{4.058007in}{0.951959in}}{\pgfqpoint{4.058007in}{0.944826in}}%
\pgfpathcurveto{\pgfqpoint{4.058007in}{0.937693in}}{\pgfqpoint{4.060841in}{0.930852in}}{\pgfqpoint{4.065885in}{0.925808in}}%
\pgfpathcurveto{\pgfqpoint{4.070928in}{0.920764in}}{\pgfqpoint{4.077770in}{0.917930in}}{\pgfqpoint{4.084903in}{0.917930in}}%
\pgfpathclose%
\pgfusepath{stroke,fill}%
\end{pgfscope}%
\begin{pgfscope}%
\pgfpathrectangle{\pgfqpoint{2.867647in}{0.500000in}}{\pgfqpoint{1.764706in}{1.700000in}}%
\pgfusepath{clip}%
\pgfsetbuttcap%
\pgfsetroundjoin%
\definecolor{currentfill}{rgb}{0.976287,0.879862,0.805788}%
\pgfsetfillcolor{currentfill}%
\pgfsetlinewidth{0.311001pt}%
\definecolor{currentstroke}{rgb}{1.000000,1.000000,1.000000}%
\pgfsetstrokecolor{currentstroke}%
\pgfsetdash{}{0pt}%
\pgfpathmoveto{\pgfqpoint{4.241317in}{1.278534in}}%
\pgfpathcurveto{\pgfqpoint{4.248450in}{1.278534in}}{\pgfqpoint{4.255291in}{1.281368in}}{\pgfqpoint{4.260335in}{1.286411in}}%
\pgfpathcurveto{\pgfqpoint{4.265379in}{1.291455in}}{\pgfqpoint{4.268213in}{1.298297in}}{\pgfqpoint{4.268213in}{1.305430in}}%
\pgfpathcurveto{\pgfqpoint{4.268213in}{1.312562in}}{\pgfqpoint{4.265379in}{1.319404in}}{\pgfqpoint{4.260335in}{1.324448in}}%
\pgfpathcurveto{\pgfqpoint{4.255291in}{1.329491in}}{\pgfqpoint{4.248450in}{1.332325in}}{\pgfqpoint{4.241317in}{1.332325in}}%
\pgfpathcurveto{\pgfqpoint{4.234184in}{1.332325in}}{\pgfqpoint{4.227342in}{1.329491in}}{\pgfqpoint{4.222299in}{1.324448in}}%
\pgfpathcurveto{\pgfqpoint{4.217255in}{1.319404in}}{\pgfqpoint{4.214421in}{1.312562in}}{\pgfqpoint{4.214421in}{1.305430in}}%
\pgfpathcurveto{\pgfqpoint{4.214421in}{1.298297in}}{\pgfqpoint{4.217255in}{1.291455in}}{\pgfqpoint{4.222299in}{1.286411in}}%
\pgfpathcurveto{\pgfqpoint{4.227342in}{1.281368in}}{\pgfqpoint{4.234184in}{1.278534in}}{\pgfqpoint{4.241317in}{1.278534in}}%
\pgfpathclose%
\pgfusepath{stroke,fill}%
\end{pgfscope}%
\begin{pgfscope}%
\pgfpathrectangle{\pgfqpoint{2.867647in}{0.500000in}}{\pgfqpoint{1.764706in}{1.700000in}}%
\pgfusepath{clip}%
\pgfsetbuttcap%
\pgfsetroundjoin%
\definecolor{currentfill}{rgb}{0.961734,0.579886,0.418445}%
\pgfsetfillcolor{currentfill}%
\pgfsetlinewidth{0.311001pt}%
\definecolor{currentstroke}{rgb}{1.000000,1.000000,1.000000}%
\pgfsetstrokecolor{currentstroke}%
\pgfsetdash{}{0pt}%
\pgfpathmoveto{\pgfqpoint{3.943479in}{0.973786in}}%
\pgfpathcurveto{\pgfqpoint{3.950612in}{0.973786in}}{\pgfqpoint{3.957454in}{0.976620in}}{\pgfqpoint{3.962497in}{0.981664in}}%
\pgfpathcurveto{\pgfqpoint{3.967541in}{0.986707in}}{\pgfqpoint{3.970375in}{0.993549in}}{\pgfqpoint{3.970375in}{1.000682in}}%
\pgfpathcurveto{\pgfqpoint{3.970375in}{1.007815in}}{\pgfqpoint{3.967541in}{1.014656in}}{\pgfqpoint{3.962497in}{1.019700in}}%
\pgfpathcurveto{\pgfqpoint{3.957454in}{1.024744in}}{\pgfqpoint{3.950612in}{1.027577in}}{\pgfqpoint{3.943479in}{1.027577in}}%
\pgfpathcurveto{\pgfqpoint{3.936346in}{1.027577in}}{\pgfqpoint{3.929505in}{1.024744in}}{\pgfqpoint{3.924461in}{1.019700in}}%
\pgfpathcurveto{\pgfqpoint{3.919417in}{1.014656in}}{\pgfqpoint{3.916583in}{1.007815in}}{\pgfqpoint{3.916583in}{1.000682in}}%
\pgfpathcurveto{\pgfqpoint{3.916583in}{0.993549in}}{\pgfqpoint{3.919417in}{0.986707in}}{\pgfqpoint{3.924461in}{0.981664in}}%
\pgfpathcurveto{\pgfqpoint{3.929505in}{0.976620in}}{\pgfqpoint{3.936346in}{0.973786in}}{\pgfqpoint{3.943479in}{0.973786in}}%
\pgfpathclose%
\pgfusepath{stroke,fill}%
\end{pgfscope}%
\begin{pgfscope}%
\pgfpathrectangle{\pgfqpoint{2.867647in}{0.500000in}}{\pgfqpoint{1.764706in}{1.700000in}}%
\pgfusepath{clip}%
\pgfsetbuttcap%
\pgfsetroundjoin%
\definecolor{currentfill}{rgb}{0.977657,0.891500,0.822809}%
\pgfsetfillcolor{currentfill}%
\pgfsetlinewidth{0.311001pt}%
\definecolor{currentstroke}{rgb}{1.000000,1.000000,1.000000}%
\pgfsetstrokecolor{currentstroke}%
\pgfsetdash{}{0pt}%
\pgfpathmoveto{\pgfqpoint{4.132576in}{1.613720in}}%
\pgfpathcurveto{\pgfqpoint{4.139709in}{1.613720in}}{\pgfqpoint{4.146551in}{1.616554in}}{\pgfqpoint{4.151595in}{1.621597in}}%
\pgfpathcurveto{\pgfqpoint{4.156638in}{1.626641in}}{\pgfqpoint{4.159472in}{1.633483in}}{\pgfqpoint{4.159472in}{1.640615in}}%
\pgfpathcurveto{\pgfqpoint{4.159472in}{1.647748in}}{\pgfqpoint{4.156638in}{1.654590in}}{\pgfqpoint{4.151595in}{1.659634in}}%
\pgfpathcurveto{\pgfqpoint{4.146551in}{1.664677in}}{\pgfqpoint{4.139709in}{1.667511in}}{\pgfqpoint{4.132576in}{1.667511in}}%
\pgfpathcurveto{\pgfqpoint{4.125444in}{1.667511in}}{\pgfqpoint{4.118602in}{1.664677in}}{\pgfqpoint{4.113558in}{1.659634in}}%
\pgfpathcurveto{\pgfqpoint{4.108515in}{1.654590in}}{\pgfqpoint{4.105681in}{1.647748in}}{\pgfqpoint{4.105681in}{1.640615in}}%
\pgfpathcurveto{\pgfqpoint{4.105681in}{1.633483in}}{\pgfqpoint{4.108515in}{1.626641in}}{\pgfqpoint{4.113558in}{1.621597in}}%
\pgfpathcurveto{\pgfqpoint{4.118602in}{1.616554in}}{\pgfqpoint{4.125444in}{1.613720in}}{\pgfqpoint{4.132576in}{1.613720in}}%
\pgfpathclose%
\pgfusepath{stroke,fill}%
\end{pgfscope}%
\begin{pgfscope}%
\pgfpathrectangle{\pgfqpoint{2.867647in}{0.500000in}}{\pgfqpoint{1.764706in}{1.700000in}}%
\pgfusepath{clip}%
\pgfsetbuttcap%
\pgfsetroundjoin%
\definecolor{currentfill}{rgb}{0.967735,0.780441,0.659127}%
\pgfsetfillcolor{currentfill}%
\pgfsetlinewidth{0.311001pt}%
\definecolor{currentstroke}{rgb}{1.000000,1.000000,1.000000}%
\pgfsetstrokecolor{currentstroke}%
\pgfsetdash{}{0pt}%
\pgfpathmoveto{\pgfqpoint{4.036337in}{0.943475in}}%
\pgfpathcurveto{\pgfqpoint{4.043470in}{0.943475in}}{\pgfqpoint{4.050312in}{0.946309in}}{\pgfqpoint{4.055355in}{0.951353in}}%
\pgfpathcurveto{\pgfqpoint{4.060399in}{0.956396in}}{\pgfqpoint{4.063233in}{0.963238in}}{\pgfqpoint{4.063233in}{0.970371in}}%
\pgfpathcurveto{\pgfqpoint{4.063233in}{0.977504in}}{\pgfqpoint{4.060399in}{0.984345in}}{\pgfqpoint{4.055355in}{0.989389in}}%
\pgfpathcurveto{\pgfqpoint{4.050312in}{0.994432in}}{\pgfqpoint{4.043470in}{0.997266in}}{\pgfqpoint{4.036337in}{0.997266in}}%
\pgfpathcurveto{\pgfqpoint{4.029204in}{0.997266in}}{\pgfqpoint{4.022363in}{0.994432in}}{\pgfqpoint{4.017319in}{0.989389in}}%
\pgfpathcurveto{\pgfqpoint{4.012275in}{0.984345in}}{\pgfqpoint{4.009441in}{0.977504in}}{\pgfqpoint{4.009441in}{0.970371in}}%
\pgfpathcurveto{\pgfqpoint{4.009441in}{0.963238in}}{\pgfqpoint{4.012275in}{0.956396in}}{\pgfqpoint{4.017319in}{0.951353in}}%
\pgfpathcurveto{\pgfqpoint{4.022363in}{0.946309in}}{\pgfqpoint{4.029204in}{0.943475in}}{\pgfqpoint{4.036337in}{0.943475in}}%
\pgfpathclose%
\pgfusepath{stroke,fill}%
\end{pgfscope}%
\begin{pgfscope}%
\pgfpathrectangle{\pgfqpoint{2.867647in}{0.500000in}}{\pgfqpoint{1.764706in}{1.700000in}}%
\pgfusepath{clip}%
\pgfsetbuttcap%
\pgfsetroundjoin%
\definecolor{currentfill}{rgb}{0.767484,0.092136,0.323905}%
\pgfsetfillcolor{currentfill}%
\pgfsetlinewidth{0.311001pt}%
\definecolor{currentstroke}{rgb}{1.000000,1.000000,1.000000}%
\pgfsetstrokecolor{currentstroke}%
\pgfsetdash{}{0pt}%
\pgfpathmoveto{\pgfqpoint{3.827418in}{1.886444in}}%
\pgfpathcurveto{\pgfqpoint{3.834551in}{1.886444in}}{\pgfqpoint{3.841393in}{1.889277in}}{\pgfqpoint{3.846436in}{1.894321in}}%
\pgfpathcurveto{\pgfqpoint{3.851480in}{1.899365in}}{\pgfqpoint{3.854314in}{1.906206in}}{\pgfqpoint{3.854314in}{1.913339in}}%
\pgfpathcurveto{\pgfqpoint{3.854314in}{1.920472in}}{\pgfqpoint{3.851480in}{1.927314in}}{\pgfqpoint{3.846436in}{1.932357in}}%
\pgfpathcurveto{\pgfqpoint{3.841393in}{1.937401in}}{\pgfqpoint{3.834551in}{1.940235in}}{\pgfqpoint{3.827418in}{1.940235in}}%
\pgfpathcurveto{\pgfqpoint{3.820285in}{1.940235in}}{\pgfqpoint{3.813444in}{1.937401in}}{\pgfqpoint{3.808400in}{1.932357in}}%
\pgfpathcurveto{\pgfqpoint{3.803356in}{1.927314in}}{\pgfqpoint{3.800523in}{1.920472in}}{\pgfqpoint{3.800523in}{1.913339in}}%
\pgfpathcurveto{\pgfqpoint{3.800523in}{1.906206in}}{\pgfqpoint{3.803356in}{1.899365in}}{\pgfqpoint{3.808400in}{1.894321in}}%
\pgfpathcurveto{\pgfqpoint{3.813444in}{1.889277in}}{\pgfqpoint{3.820285in}{1.886444in}}{\pgfqpoint{3.827418in}{1.886444in}}%
\pgfpathclose%
\pgfusepath{stroke,fill}%
\end{pgfscope}%
\begin{pgfscope}%
\pgfpathrectangle{\pgfqpoint{2.867647in}{0.500000in}}{\pgfqpoint{1.764706in}{1.700000in}}%
\pgfusepath{clip}%
\pgfsetbuttcap%
\pgfsetroundjoin%
\definecolor{currentfill}{rgb}{0.977657,0.891500,0.822809}%
\pgfsetfillcolor{currentfill}%
\pgfsetlinewidth{0.311001pt}%
\definecolor{currentstroke}{rgb}{1.000000,1.000000,1.000000}%
\pgfsetstrokecolor{currentstroke}%
\pgfsetdash{}{0pt}%
\pgfpathmoveto{\pgfqpoint{4.160943in}{1.581375in}}%
\pgfpathcurveto{\pgfqpoint{4.168076in}{1.581375in}}{\pgfqpoint{4.174918in}{1.584209in}}{\pgfqpoint{4.179961in}{1.589252in}}%
\pgfpathcurveto{\pgfqpoint{4.185005in}{1.594296in}}{\pgfqpoint{4.187839in}{1.601138in}}{\pgfqpoint{4.187839in}{1.608270in}}%
\pgfpathcurveto{\pgfqpoint{4.187839in}{1.615403in}}{\pgfqpoint{4.185005in}{1.622245in}}{\pgfqpoint{4.179961in}{1.627288in}}%
\pgfpathcurveto{\pgfqpoint{4.174918in}{1.632332in}}{\pgfqpoint{4.168076in}{1.635166in}}{\pgfqpoint{4.160943in}{1.635166in}}%
\pgfpathcurveto{\pgfqpoint{4.153810in}{1.635166in}}{\pgfqpoint{4.146969in}{1.632332in}}{\pgfqpoint{4.141925in}{1.627288in}}%
\pgfpathcurveto{\pgfqpoint{4.136881in}{1.622245in}}{\pgfqpoint{4.134047in}{1.615403in}}{\pgfqpoint{4.134047in}{1.608270in}}%
\pgfpathcurveto{\pgfqpoint{4.134047in}{1.601138in}}{\pgfqpoint{4.136881in}{1.594296in}}{\pgfqpoint{4.141925in}{1.589252in}}%
\pgfpathcurveto{\pgfqpoint{4.146969in}{1.584209in}}{\pgfqpoint{4.153810in}{1.581375in}}{\pgfqpoint{4.160943in}{1.581375in}}%
\pgfpathclose%
\pgfusepath{stroke,fill}%
\end{pgfscope}%
\begin{pgfscope}%
\pgfpathrectangle{\pgfqpoint{2.867647in}{0.500000in}}{\pgfqpoint{1.764706in}{1.700000in}}%
\pgfusepath{clip}%
\pgfsetbuttcap%
\pgfsetroundjoin%
\definecolor{currentfill}{rgb}{0.963559,0.632016,0.472047}%
\pgfsetfillcolor{currentfill}%
\pgfsetlinewidth{0.311001pt}%
\definecolor{currentstroke}{rgb}{1.000000,1.000000,1.000000}%
\pgfsetstrokecolor{currentstroke}%
\pgfsetdash{}{0pt}%
\pgfpathmoveto{\pgfqpoint{4.280470in}{1.070509in}}%
\pgfpathcurveto{\pgfqpoint{4.287603in}{1.070509in}}{\pgfqpoint{4.294444in}{1.073342in}}{\pgfqpoint{4.299488in}{1.078386in}}%
\pgfpathcurveto{\pgfqpoint{4.304532in}{1.083430in}}{\pgfqpoint{4.307366in}{1.090271in}}{\pgfqpoint{4.307366in}{1.097404in}}%
\pgfpathcurveto{\pgfqpoint{4.307366in}{1.104537in}}{\pgfqpoint{4.304532in}{1.111379in}}{\pgfqpoint{4.299488in}{1.116422in}}%
\pgfpathcurveto{\pgfqpoint{4.294444in}{1.121466in}}{\pgfqpoint{4.287603in}{1.124300in}}{\pgfqpoint{4.280470in}{1.124300in}}%
\pgfpathcurveto{\pgfqpoint{4.273337in}{1.124300in}}{\pgfqpoint{4.266495in}{1.121466in}}{\pgfqpoint{4.261452in}{1.116422in}}%
\pgfpathcurveto{\pgfqpoint{4.256408in}{1.111379in}}{\pgfqpoint{4.253574in}{1.104537in}}{\pgfqpoint{4.253574in}{1.097404in}}%
\pgfpathcurveto{\pgfqpoint{4.253574in}{1.090271in}}{\pgfqpoint{4.256408in}{1.083430in}}{\pgfqpoint{4.261452in}{1.078386in}}%
\pgfpathcurveto{\pgfqpoint{4.266495in}{1.073342in}}{\pgfqpoint{4.273337in}{1.070509in}}{\pgfqpoint{4.280470in}{1.070509in}}%
\pgfpathclose%
\pgfusepath{stroke,fill}%
\end{pgfscope}%
\begin{pgfscope}%
\pgfpathrectangle{\pgfqpoint{2.867647in}{0.500000in}}{\pgfqpoint{1.764706in}{1.700000in}}%
\pgfusepath{clip}%
\pgfsetbuttcap%
\pgfsetroundjoin%
\definecolor{currentfill}{rgb}{0.962985,0.612625,0.451451}%
\pgfsetfillcolor{currentfill}%
\pgfsetlinewidth{0.311001pt}%
\definecolor{currentstroke}{rgb}{1.000000,1.000000,1.000000}%
\pgfsetstrokecolor{currentstroke}%
\pgfsetdash{}{0pt}%
\pgfpathmoveto{\pgfqpoint{4.032734in}{1.801290in}}%
\pgfpathcurveto{\pgfqpoint{4.039867in}{1.801290in}}{\pgfqpoint{4.046709in}{1.804124in}}{\pgfqpoint{4.051752in}{1.809168in}}%
\pgfpathcurveto{\pgfqpoint{4.056796in}{1.814212in}}{\pgfqpoint{4.059630in}{1.821053in}}{\pgfqpoint{4.059630in}{1.828186in}}%
\pgfpathcurveto{\pgfqpoint{4.059630in}{1.835319in}}{\pgfqpoint{4.056796in}{1.842161in}}{\pgfqpoint{4.051752in}{1.847204in}}%
\pgfpathcurveto{\pgfqpoint{4.046709in}{1.852248in}}{\pgfqpoint{4.039867in}{1.855082in}}{\pgfqpoint{4.032734in}{1.855082in}}%
\pgfpathcurveto{\pgfqpoint{4.025601in}{1.855082in}}{\pgfqpoint{4.018760in}{1.852248in}}{\pgfqpoint{4.013716in}{1.847204in}}%
\pgfpathcurveto{\pgfqpoint{4.008672in}{1.842161in}}{\pgfqpoint{4.005838in}{1.835319in}}{\pgfqpoint{4.005838in}{1.828186in}}%
\pgfpathcurveto{\pgfqpoint{4.005838in}{1.821053in}}{\pgfqpoint{4.008672in}{1.814212in}}{\pgfqpoint{4.013716in}{1.809168in}}%
\pgfpathcurveto{\pgfqpoint{4.018760in}{1.804124in}}{\pgfqpoint{4.025601in}{1.801290in}}{\pgfqpoint{4.032734in}{1.801290in}}%
\pgfpathclose%
\pgfusepath{stroke,fill}%
\end{pgfscope}%
\begin{pgfscope}%
\pgfpathrectangle{\pgfqpoint{2.867647in}{0.500000in}}{\pgfqpoint{1.764706in}{1.700000in}}%
\pgfusepath{clip}%
\pgfsetbuttcap%
\pgfsetroundjoin%
\definecolor{currentfill}{rgb}{0.972201,0.839051,0.745789}%
\pgfsetfillcolor{currentfill}%
\pgfsetlinewidth{0.311001pt}%
\definecolor{currentstroke}{rgb}{1.000000,1.000000,1.000000}%
\pgfsetstrokecolor{currentstroke}%
\pgfsetdash{}{0pt}%
\pgfpathmoveto{\pgfqpoint{4.088530in}{1.149067in}}%
\pgfpathcurveto{\pgfqpoint{4.095663in}{1.149067in}}{\pgfqpoint{4.102504in}{1.151901in}}{\pgfqpoint{4.107548in}{1.156945in}}%
\pgfpathcurveto{\pgfqpoint{4.112592in}{1.161989in}}{\pgfqpoint{4.115426in}{1.168830in}}{\pgfqpoint{4.115426in}{1.175963in}}%
\pgfpathcurveto{\pgfqpoint{4.115426in}{1.183096in}}{\pgfqpoint{4.112592in}{1.189938in}}{\pgfqpoint{4.107548in}{1.194981in}}%
\pgfpathcurveto{\pgfqpoint{4.102504in}{1.200025in}}{\pgfqpoint{4.095663in}{1.202859in}}{\pgfqpoint{4.088530in}{1.202859in}}%
\pgfpathcurveto{\pgfqpoint{4.081397in}{1.202859in}}{\pgfqpoint{4.074555in}{1.200025in}}{\pgfqpoint{4.069512in}{1.194981in}}%
\pgfpathcurveto{\pgfqpoint{4.064468in}{1.189938in}}{\pgfqpoint{4.061634in}{1.183096in}}{\pgfqpoint{4.061634in}{1.175963in}}%
\pgfpathcurveto{\pgfqpoint{4.061634in}{1.168830in}}{\pgfqpoint{4.064468in}{1.161989in}}{\pgfqpoint{4.069512in}{1.156945in}}%
\pgfpathcurveto{\pgfqpoint{4.074555in}{1.151901in}}{\pgfqpoint{4.081397in}{1.149067in}}{\pgfqpoint{4.088530in}{1.149067in}}%
\pgfpathclose%
\pgfusepath{stroke,fill}%
\end{pgfscope}%
\begin{pgfscope}%
\pgfpathrectangle{\pgfqpoint{2.867647in}{0.500000in}}{\pgfqpoint{1.764706in}{1.700000in}}%
\pgfusepath{clip}%
\pgfsetbuttcap%
\pgfsetroundjoin%
\definecolor{currentfill}{rgb}{0.970255,0.815666,0.711203}%
\pgfsetfillcolor{currentfill}%
\pgfsetlinewidth{0.311001pt}%
\definecolor{currentstroke}{rgb}{1.000000,1.000000,1.000000}%
\pgfsetstrokecolor{currentstroke}%
\pgfsetdash{}{0pt}%
\pgfpathmoveto{\pgfqpoint{4.116621in}{1.692101in}}%
\pgfpathcurveto{\pgfqpoint{4.123753in}{1.692101in}}{\pgfqpoint{4.130595in}{1.694935in}}{\pgfqpoint{4.135639in}{1.699979in}}%
\pgfpathcurveto{\pgfqpoint{4.140682in}{1.705022in}}{\pgfqpoint{4.143516in}{1.711864in}}{\pgfqpoint{4.143516in}{1.718997in}}%
\pgfpathcurveto{\pgfqpoint{4.143516in}{1.726130in}}{\pgfqpoint{4.140682in}{1.732971in}}{\pgfqpoint{4.135639in}{1.738015in}}%
\pgfpathcurveto{\pgfqpoint{4.130595in}{1.743059in}}{\pgfqpoint{4.123753in}{1.745893in}}{\pgfqpoint{4.116621in}{1.745893in}}%
\pgfpathcurveto{\pgfqpoint{4.109488in}{1.745893in}}{\pgfqpoint{4.102646in}{1.743059in}}{\pgfqpoint{4.097602in}{1.738015in}}%
\pgfpathcurveto{\pgfqpoint{4.092559in}{1.732971in}}{\pgfqpoint{4.089725in}{1.726130in}}{\pgfqpoint{4.089725in}{1.718997in}}%
\pgfpathcurveto{\pgfqpoint{4.089725in}{1.711864in}}{\pgfqpoint{4.092559in}{1.705022in}}{\pgfqpoint{4.097602in}{1.699979in}}%
\pgfpathcurveto{\pgfqpoint{4.102646in}{1.694935in}}{\pgfqpoint{4.109488in}{1.692101in}}{\pgfqpoint{4.116621in}{1.692101in}}%
\pgfpathclose%
\pgfusepath{stroke,fill}%
\end{pgfscope}%
\begin{pgfscope}%
\pgfpathrectangle{\pgfqpoint{2.867647in}{0.500000in}}{\pgfqpoint{1.764706in}{1.700000in}}%
\pgfusepath{clip}%
\pgfsetbuttcap%
\pgfsetroundjoin%
\definecolor{currentfill}{rgb}{0.971202,0.827364,0.728520}%
\pgfsetfillcolor{currentfill}%
\pgfsetlinewidth{0.311001pt}%
\definecolor{currentstroke}{rgb}{1.000000,1.000000,1.000000}%
\pgfsetstrokecolor{currentstroke}%
\pgfsetdash{}{0pt}%
\pgfpathmoveto{\pgfqpoint{4.191973in}{1.031662in}}%
\pgfpathcurveto{\pgfqpoint{4.199106in}{1.031662in}}{\pgfqpoint{4.205948in}{1.034496in}}{\pgfqpoint{4.210991in}{1.039540in}}%
\pgfpathcurveto{\pgfqpoint{4.216035in}{1.044583in}}{\pgfqpoint{4.218869in}{1.051425in}}{\pgfqpoint{4.218869in}{1.058558in}}%
\pgfpathcurveto{\pgfqpoint{4.218869in}{1.065691in}}{\pgfqpoint{4.216035in}{1.072532in}}{\pgfqpoint{4.210991in}{1.077576in}}%
\pgfpathcurveto{\pgfqpoint{4.205948in}{1.082620in}}{\pgfqpoint{4.199106in}{1.085454in}}{\pgfqpoint{4.191973in}{1.085454in}}%
\pgfpathcurveto{\pgfqpoint{4.184841in}{1.085454in}}{\pgfqpoint{4.177999in}{1.082620in}}{\pgfqpoint{4.172955in}{1.077576in}}%
\pgfpathcurveto{\pgfqpoint{4.167912in}{1.072532in}}{\pgfqpoint{4.165078in}{1.065691in}}{\pgfqpoint{4.165078in}{1.058558in}}%
\pgfpathcurveto{\pgfqpoint{4.165078in}{1.051425in}}{\pgfqpoint{4.167912in}{1.044583in}}{\pgfqpoint{4.172955in}{1.039540in}}%
\pgfpathcurveto{\pgfqpoint{4.177999in}{1.034496in}}{\pgfqpoint{4.184841in}{1.031662in}}{\pgfqpoint{4.191973in}{1.031662in}}%
\pgfpathclose%
\pgfusepath{stroke,fill}%
\end{pgfscope}%
\begin{pgfscope}%
\pgfpathrectangle{\pgfqpoint{2.867647in}{0.500000in}}{\pgfqpoint{1.764706in}{1.700000in}}%
\pgfusepath{clip}%
\pgfsetbuttcap%
\pgfsetroundjoin%
\definecolor{currentfill}{rgb}{0.978376,0.897317,0.831308}%
\pgfsetfillcolor{currentfill}%
\pgfsetlinewidth{0.311001pt}%
\definecolor{currentstroke}{rgb}{1.000000,1.000000,1.000000}%
\pgfsetstrokecolor{currentstroke}%
\pgfsetdash{}{0pt}%
\pgfpathmoveto{\pgfqpoint{4.120749in}{1.112735in}}%
\pgfpathcurveto{\pgfqpoint{4.127882in}{1.112735in}}{\pgfqpoint{4.134724in}{1.115569in}}{\pgfqpoint{4.139768in}{1.120613in}}%
\pgfpathcurveto{\pgfqpoint{4.144811in}{1.125657in}}{\pgfqpoint{4.147645in}{1.132498in}}{\pgfqpoint{4.147645in}{1.139631in}}%
\pgfpathcurveto{\pgfqpoint{4.147645in}{1.146764in}}{\pgfqpoint{4.144811in}{1.153606in}}{\pgfqpoint{4.139768in}{1.158649in}}%
\pgfpathcurveto{\pgfqpoint{4.134724in}{1.163693in}}{\pgfqpoint{4.127882in}{1.166527in}}{\pgfqpoint{4.120749in}{1.166527in}}%
\pgfpathcurveto{\pgfqpoint{4.113617in}{1.166527in}}{\pgfqpoint{4.106775in}{1.163693in}}{\pgfqpoint{4.101731in}{1.158649in}}%
\pgfpathcurveto{\pgfqpoint{4.096688in}{1.153606in}}{\pgfqpoint{4.093854in}{1.146764in}}{\pgfqpoint{4.093854in}{1.139631in}}%
\pgfpathcurveto{\pgfqpoint{4.093854in}{1.132498in}}{\pgfqpoint{4.096688in}{1.125657in}}{\pgfqpoint{4.101731in}{1.120613in}}%
\pgfpathcurveto{\pgfqpoint{4.106775in}{1.115569in}}{\pgfqpoint{4.113617in}{1.112735in}}{\pgfqpoint{4.120749in}{1.112735in}}%
\pgfpathclose%
\pgfusepath{stroke,fill}%
\end{pgfscope}%
\begin{pgfscope}%
\pgfpathrectangle{\pgfqpoint{2.867647in}{0.500000in}}{\pgfqpoint{1.764706in}{1.700000in}}%
\pgfusepath{clip}%
\pgfsetbuttcap%
\pgfsetroundjoin%
\definecolor{currentfill}{rgb}{0.976287,0.879862,0.805788}%
\pgfsetfillcolor{currentfill}%
\pgfsetlinewidth{0.311001pt}%
\definecolor{currentstroke}{rgb}{1.000000,1.000000,1.000000}%
\pgfsetstrokecolor{currentstroke}%
\pgfsetdash{}{0pt}%
\pgfpathmoveto{\pgfqpoint{4.212854in}{1.129480in}}%
\pgfpathcurveto{\pgfqpoint{4.219987in}{1.129480in}}{\pgfqpoint{4.226829in}{1.132314in}}{\pgfqpoint{4.231873in}{1.137358in}}%
\pgfpathcurveto{\pgfqpoint{4.236916in}{1.142401in}}{\pgfqpoint{4.239750in}{1.149243in}}{\pgfqpoint{4.239750in}{1.156376in}}%
\pgfpathcurveto{\pgfqpoint{4.239750in}{1.163509in}}{\pgfqpoint{4.236916in}{1.170350in}}{\pgfqpoint{4.231873in}{1.175394in}}%
\pgfpathcurveto{\pgfqpoint{4.226829in}{1.180438in}}{\pgfqpoint{4.219987in}{1.183272in}}{\pgfqpoint{4.212854in}{1.183272in}}%
\pgfpathcurveto{\pgfqpoint{4.205722in}{1.183272in}}{\pgfqpoint{4.198880in}{1.180438in}}{\pgfqpoint{4.193836in}{1.175394in}}%
\pgfpathcurveto{\pgfqpoint{4.188793in}{1.170350in}}{\pgfqpoint{4.185959in}{1.163509in}}{\pgfqpoint{4.185959in}{1.156376in}}%
\pgfpathcurveto{\pgfqpoint{4.185959in}{1.149243in}}{\pgfqpoint{4.188793in}{1.142401in}}{\pgfqpoint{4.193836in}{1.137358in}}%
\pgfpathcurveto{\pgfqpoint{4.198880in}{1.132314in}}{\pgfqpoint{4.205722in}{1.129480in}}{\pgfqpoint{4.212854in}{1.129480in}}%
\pgfpathclose%
\pgfusepath{stroke,fill}%
\end{pgfscope}%
\begin{pgfscope}%
\pgfpathrectangle{\pgfqpoint{2.867647in}{0.500000in}}{\pgfqpoint{1.764706in}{1.700000in}}%
\pgfusepath{clip}%
\pgfsetbuttcap%
\pgfsetroundjoin%
\definecolor{currentfill}{rgb}{0.965302,0.713942,0.568499}%
\pgfsetfillcolor{currentfill}%
\pgfsetlinewidth{0.311001pt}%
\definecolor{currentstroke}{rgb}{1.000000,1.000000,1.000000}%
\pgfsetstrokecolor{currentstroke}%
\pgfsetdash{}{0pt}%
\pgfpathmoveto{\pgfqpoint{3.994124in}{0.954508in}}%
\pgfpathcurveto{\pgfqpoint{4.001257in}{0.954508in}}{\pgfqpoint{4.008099in}{0.957342in}}{\pgfqpoint{4.013142in}{0.962385in}}%
\pgfpathcurveto{\pgfqpoint{4.018186in}{0.967429in}}{\pgfqpoint{4.021020in}{0.974271in}}{\pgfqpoint{4.021020in}{0.981403in}}%
\pgfpathcurveto{\pgfqpoint{4.021020in}{0.988536in}}{\pgfqpoint{4.018186in}{0.995378in}}{\pgfqpoint{4.013142in}{1.000422in}}%
\pgfpathcurveto{\pgfqpoint{4.008099in}{1.005465in}}{\pgfqpoint{4.001257in}{1.008299in}}{\pgfqpoint{3.994124in}{1.008299in}}%
\pgfpathcurveto{\pgfqpoint{3.986991in}{1.008299in}}{\pgfqpoint{3.980150in}{1.005465in}}{\pgfqpoint{3.975106in}{1.000422in}}%
\pgfpathcurveto{\pgfqpoint{3.970062in}{0.995378in}}{\pgfqpoint{3.967228in}{0.988536in}}{\pgfqpoint{3.967228in}{0.981403in}}%
\pgfpathcurveto{\pgfqpoint{3.967228in}{0.974271in}}{\pgfqpoint{3.970062in}{0.967429in}}{\pgfqpoint{3.975106in}{0.962385in}}%
\pgfpathcurveto{\pgfqpoint{3.980150in}{0.957342in}}{\pgfqpoint{3.986991in}{0.954508in}}{\pgfqpoint{3.994124in}{0.954508in}}%
\pgfpathclose%
\pgfusepath{stroke,fill}%
\end{pgfscope}%
\begin{pgfscope}%
\pgfpathrectangle{\pgfqpoint{2.867647in}{0.500000in}}{\pgfqpoint{1.764706in}{1.700000in}}%
\pgfusepath{clip}%
\pgfsetbuttcap%
\pgfsetroundjoin%
\definecolor{currentfill}{rgb}{0.981377,0.920617,0.865369}%
\pgfsetfillcolor{currentfill}%
\pgfsetlinewidth{0.311001pt}%
\definecolor{currentstroke}{rgb}{1.000000,1.000000,1.000000}%
\pgfsetstrokecolor{currentstroke}%
\pgfsetdash{}{0pt}%
\pgfpathmoveto{\pgfqpoint{4.172326in}{1.285774in}}%
\pgfpathcurveto{\pgfqpoint{4.179458in}{1.285774in}}{\pgfqpoint{4.186300in}{1.288608in}}{\pgfqpoint{4.191344in}{1.293652in}}%
\pgfpathcurveto{\pgfqpoint{4.196387in}{1.298696in}}{\pgfqpoint{4.199221in}{1.305537in}}{\pgfqpoint{4.199221in}{1.312670in}}%
\pgfpathcurveto{\pgfqpoint{4.199221in}{1.319803in}}{\pgfqpoint{4.196387in}{1.326645in}}{\pgfqpoint{4.191344in}{1.331688in}}%
\pgfpathcurveto{\pgfqpoint{4.186300in}{1.336732in}}{\pgfqpoint{4.179458in}{1.339566in}}{\pgfqpoint{4.172326in}{1.339566in}}%
\pgfpathcurveto{\pgfqpoint{4.165193in}{1.339566in}}{\pgfqpoint{4.158351in}{1.336732in}}{\pgfqpoint{4.153307in}{1.331688in}}%
\pgfpathcurveto{\pgfqpoint{4.148264in}{1.326645in}}{\pgfqpoint{4.145430in}{1.319803in}}{\pgfqpoint{4.145430in}{1.312670in}}%
\pgfpathcurveto{\pgfqpoint{4.145430in}{1.305537in}}{\pgfqpoint{4.148264in}{1.298696in}}{\pgfqpoint{4.153307in}{1.293652in}}%
\pgfpathcurveto{\pgfqpoint{4.158351in}{1.288608in}}{\pgfqpoint{4.165193in}{1.285774in}}{\pgfqpoint{4.172326in}{1.285774in}}%
\pgfpathclose%
\pgfusepath{stroke,fill}%
\end{pgfscope}%
\begin{pgfscope}%
\pgfpathrectangle{\pgfqpoint{2.867647in}{0.500000in}}{\pgfqpoint{1.764706in}{1.700000in}}%
\pgfusepath{clip}%
\pgfsetbuttcap%
\pgfsetroundjoin%
\definecolor{currentfill}{rgb}{0.980678,0.914765,0.856766}%
\pgfsetfillcolor{currentfill}%
\pgfsetlinewidth{0.311001pt}%
\definecolor{currentstroke}{rgb}{1.000000,1.000000,1.000000}%
\pgfsetstrokecolor{currentstroke}%
\pgfsetdash{}{0pt}%
\pgfpathmoveto{\pgfqpoint{4.208166in}{1.350920in}}%
\pgfpathcurveto{\pgfqpoint{4.215299in}{1.350920in}}{\pgfqpoint{4.222140in}{1.353754in}}{\pgfqpoint{4.227184in}{1.358798in}}%
\pgfpathcurveto{\pgfqpoint{4.232228in}{1.363841in}}{\pgfqpoint{4.235062in}{1.370683in}}{\pgfqpoint{4.235062in}{1.377816in}}%
\pgfpathcurveto{\pgfqpoint{4.235062in}{1.384949in}}{\pgfqpoint{4.232228in}{1.391790in}}{\pgfqpoint{4.227184in}{1.396834in}}%
\pgfpathcurveto{\pgfqpoint{4.222140in}{1.401878in}}{\pgfqpoint{4.215299in}{1.404711in}}{\pgfqpoint{4.208166in}{1.404711in}}%
\pgfpathcurveto{\pgfqpoint{4.201033in}{1.404711in}}{\pgfqpoint{4.194192in}{1.401878in}}{\pgfqpoint{4.189148in}{1.396834in}}%
\pgfpathcurveto{\pgfqpoint{4.184104in}{1.391790in}}{\pgfqpoint{4.181270in}{1.384949in}}{\pgfqpoint{4.181270in}{1.377816in}}%
\pgfpathcurveto{\pgfqpoint{4.181270in}{1.370683in}}{\pgfqpoint{4.184104in}{1.363841in}}{\pgfqpoint{4.189148in}{1.358798in}}%
\pgfpathcurveto{\pgfqpoint{4.194192in}{1.353754in}}{\pgfqpoint{4.201033in}{1.350920in}}{\pgfqpoint{4.208166in}{1.350920in}}%
\pgfpathclose%
\pgfusepath{stroke,fill}%
\end{pgfscope}%
\begin{pgfscope}%
\pgfpathrectangle{\pgfqpoint{2.867647in}{0.500000in}}{\pgfqpoint{1.764706in}{1.700000in}}%
\pgfusepath{clip}%
\pgfsetbuttcap%
\pgfsetroundjoin%
\definecolor{currentfill}{rgb}{0.957848,0.512613,0.357119}%
\pgfsetfillcolor{currentfill}%
\pgfsetlinewidth{0.311001pt}%
\definecolor{currentstroke}{rgb}{1.000000,1.000000,1.000000}%
\pgfsetstrokecolor{currentstroke}%
\pgfsetdash{}{0pt}%
\pgfpathmoveto{\pgfqpoint{3.890088in}{1.705433in}}%
\pgfpathcurveto{\pgfqpoint{3.897221in}{1.705433in}}{\pgfqpoint{3.904063in}{1.708267in}}{\pgfqpoint{3.909107in}{1.713310in}}%
\pgfpathcurveto{\pgfqpoint{3.914150in}{1.718354in}}{\pgfqpoint{3.916984in}{1.725196in}}{\pgfqpoint{3.916984in}{1.732328in}}%
\pgfpathcurveto{\pgfqpoint{3.916984in}{1.739461in}}{\pgfqpoint{3.914150in}{1.746303in}}{\pgfqpoint{3.909107in}{1.751347in}}%
\pgfpathcurveto{\pgfqpoint{3.904063in}{1.756390in}}{\pgfqpoint{3.897221in}{1.759224in}}{\pgfqpoint{3.890088in}{1.759224in}}%
\pgfpathcurveto{\pgfqpoint{3.882956in}{1.759224in}}{\pgfqpoint{3.876114in}{1.756390in}}{\pgfqpoint{3.871070in}{1.751347in}}%
\pgfpathcurveto{\pgfqpoint{3.866027in}{1.746303in}}{\pgfqpoint{3.863193in}{1.739461in}}{\pgfqpoint{3.863193in}{1.732328in}}%
\pgfpathcurveto{\pgfqpoint{3.863193in}{1.725196in}}{\pgfqpoint{3.866027in}{1.718354in}}{\pgfqpoint{3.871070in}{1.713310in}}%
\pgfpathcurveto{\pgfqpoint{3.876114in}{1.708267in}}{\pgfqpoint{3.882956in}{1.705433in}}{\pgfqpoint{3.890088in}{1.705433in}}%
\pgfpathclose%
\pgfusepath{stroke,fill}%
\end{pgfscope}%
\begin{pgfscope}%
\pgfpathrectangle{\pgfqpoint{2.867647in}{0.500000in}}{\pgfqpoint{1.764706in}{1.700000in}}%
\pgfusepath{clip}%
\pgfsetbuttcap%
\pgfsetroundjoin%
\definecolor{currentfill}{rgb}{0.981377,0.920617,0.865369}%
\pgfsetfillcolor{currentfill}%
\pgfsetlinewidth{0.311001pt}%
\definecolor{currentstroke}{rgb}{1.000000,1.000000,1.000000}%
\pgfsetstrokecolor{currentstroke}%
\pgfsetdash{}{0pt}%
\pgfpathmoveto{\pgfqpoint{4.177991in}{1.173814in}}%
\pgfpathcurveto{\pgfqpoint{4.185124in}{1.173814in}}{\pgfqpoint{4.191965in}{1.176647in}}{\pgfqpoint{4.197009in}{1.181691in}}%
\pgfpathcurveto{\pgfqpoint{4.202053in}{1.186735in}}{\pgfqpoint{4.204887in}{1.193576in}}{\pgfqpoint{4.204887in}{1.200709in}}%
\pgfpathcurveto{\pgfqpoint{4.204887in}{1.207842in}}{\pgfqpoint{4.202053in}{1.214684in}}{\pgfqpoint{4.197009in}{1.219727in}}%
\pgfpathcurveto{\pgfqpoint{4.191965in}{1.224771in}}{\pgfqpoint{4.185124in}{1.227605in}}{\pgfqpoint{4.177991in}{1.227605in}}%
\pgfpathcurveto{\pgfqpoint{4.170858in}{1.227605in}}{\pgfqpoint{4.164016in}{1.224771in}}{\pgfqpoint{4.158973in}{1.219727in}}%
\pgfpathcurveto{\pgfqpoint{4.153929in}{1.214684in}}{\pgfqpoint{4.151095in}{1.207842in}}{\pgfqpoint{4.151095in}{1.200709in}}%
\pgfpathcurveto{\pgfqpoint{4.151095in}{1.193576in}}{\pgfqpoint{4.153929in}{1.186735in}}{\pgfqpoint{4.158973in}{1.181691in}}%
\pgfpathcurveto{\pgfqpoint{4.164016in}{1.176647in}}{\pgfqpoint{4.170858in}{1.173814in}}{\pgfqpoint{4.177991in}{1.173814in}}%
\pgfpathclose%
\pgfusepath{stroke,fill}%
\end{pgfscope}%
\begin{pgfscope}%
\pgfpathrectangle{\pgfqpoint{2.867647in}{0.500000in}}{\pgfqpoint{1.764706in}{1.700000in}}%
\pgfusepath{clip}%
\pgfsetbuttcap%
\pgfsetroundjoin%
\definecolor{currentfill}{rgb}{0.974412,0.862387,0.780156}%
\pgfsetfillcolor{currentfill}%
\pgfsetlinewidth{0.311001pt}%
\definecolor{currentstroke}{rgb}{1.000000,1.000000,1.000000}%
\pgfsetstrokecolor{currentstroke}%
\pgfsetdash{}{0pt}%
\pgfpathmoveto{\pgfqpoint{4.209417in}{1.543876in}}%
\pgfpathcurveto{\pgfqpoint{4.216549in}{1.543876in}}{\pgfqpoint{4.223391in}{1.546710in}}{\pgfqpoint{4.228435in}{1.551754in}}%
\pgfpathcurveto{\pgfqpoint{4.233478in}{1.556797in}}{\pgfqpoint{4.236312in}{1.563639in}}{\pgfqpoint{4.236312in}{1.570772in}}%
\pgfpathcurveto{\pgfqpoint{4.236312in}{1.577905in}}{\pgfqpoint{4.233478in}{1.584746in}}{\pgfqpoint{4.228435in}{1.589790in}}%
\pgfpathcurveto{\pgfqpoint{4.223391in}{1.594834in}}{\pgfqpoint{4.216549in}{1.597668in}}{\pgfqpoint{4.209417in}{1.597668in}}%
\pgfpathcurveto{\pgfqpoint{4.202284in}{1.597668in}}{\pgfqpoint{4.195442in}{1.594834in}}{\pgfqpoint{4.190398in}{1.589790in}}%
\pgfpathcurveto{\pgfqpoint{4.185355in}{1.584746in}}{\pgfqpoint{4.182521in}{1.577905in}}{\pgfqpoint{4.182521in}{1.570772in}}%
\pgfpathcurveto{\pgfqpoint{4.182521in}{1.563639in}}{\pgfqpoint{4.185355in}{1.556797in}}{\pgfqpoint{4.190398in}{1.551754in}}%
\pgfpathcurveto{\pgfqpoint{4.195442in}{1.546710in}}{\pgfqpoint{4.202284in}{1.543876in}}{\pgfqpoint{4.209417in}{1.543876in}}%
\pgfpathclose%
\pgfusepath{stroke,fill}%
\end{pgfscope}%
\begin{pgfscope}%
\pgfpathrectangle{\pgfqpoint{2.867647in}{0.500000in}}{\pgfqpoint{1.764706in}{1.700000in}}%
\pgfusepath{clip}%
\pgfsetbuttcap%
\pgfsetroundjoin%
\definecolor{currentfill}{rgb}{0.974412,0.862387,0.780156}%
\pgfsetfillcolor{currentfill}%
\pgfsetlinewidth{0.311001pt}%
\definecolor{currentstroke}{rgb}{1.000000,1.000000,1.000000}%
\pgfsetstrokecolor{currentstroke}%
\pgfsetdash{}{0pt}%
\pgfpathmoveto{\pgfqpoint{4.081393in}{1.559638in}}%
\pgfpathcurveto{\pgfqpoint{4.088526in}{1.559638in}}{\pgfqpoint{4.095368in}{1.562472in}}{\pgfqpoint{4.100411in}{1.567515in}}%
\pgfpathcurveto{\pgfqpoint{4.105455in}{1.572559in}}{\pgfqpoint{4.108289in}{1.579401in}}{\pgfqpoint{4.108289in}{1.586534in}}%
\pgfpathcurveto{\pgfqpoint{4.108289in}{1.593666in}}{\pgfqpoint{4.105455in}{1.600508in}}{\pgfqpoint{4.100411in}{1.605552in}}%
\pgfpathcurveto{\pgfqpoint{4.095368in}{1.610595in}}{\pgfqpoint{4.088526in}{1.613429in}}{\pgfqpoint{4.081393in}{1.613429in}}%
\pgfpathcurveto{\pgfqpoint{4.074260in}{1.613429in}}{\pgfqpoint{4.067419in}{1.610595in}}{\pgfqpoint{4.062375in}{1.605552in}}%
\pgfpathcurveto{\pgfqpoint{4.057331in}{1.600508in}}{\pgfqpoint{4.054497in}{1.593666in}}{\pgfqpoint{4.054497in}{1.586534in}}%
\pgfpathcurveto{\pgfqpoint{4.054497in}{1.579401in}}{\pgfqpoint{4.057331in}{1.572559in}}{\pgfqpoint{4.062375in}{1.567515in}}%
\pgfpathcurveto{\pgfqpoint{4.067419in}{1.562472in}}{\pgfqpoint{4.074260in}{1.559638in}}{\pgfqpoint{4.081393in}{1.559638in}}%
\pgfpathclose%
\pgfusepath{stroke,fill}%
\end{pgfscope}%
\begin{pgfscope}%
\pgfpathrectangle{\pgfqpoint{2.867647in}{0.500000in}}{\pgfqpoint{1.764706in}{1.700000in}}%
\pgfusepath{clip}%
\pgfsetbuttcap%
\pgfsetroundjoin%
\definecolor{currentfill}{rgb}{0.971202,0.827364,0.728520}%
\pgfsetfillcolor{currentfill}%
\pgfsetlinewidth{0.311001pt}%
\definecolor{currentstroke}{rgb}{1.000000,1.000000,1.000000}%
\pgfsetstrokecolor{currentstroke}%
\pgfsetdash{}{0pt}%
\pgfpathmoveto{\pgfqpoint{4.270549in}{1.323648in}}%
\pgfpathcurveto{\pgfqpoint{4.277682in}{1.323648in}}{\pgfqpoint{4.284524in}{1.326481in}}{\pgfqpoint{4.289567in}{1.331525in}}%
\pgfpathcurveto{\pgfqpoint{4.294611in}{1.336569in}}{\pgfqpoint{4.297445in}{1.343410in}}{\pgfqpoint{4.297445in}{1.350543in}}%
\pgfpathcurveto{\pgfqpoint{4.297445in}{1.357676in}}{\pgfqpoint{4.294611in}{1.364518in}}{\pgfqpoint{4.289567in}{1.369561in}}%
\pgfpathcurveto{\pgfqpoint{4.284524in}{1.374605in}}{\pgfqpoint{4.277682in}{1.377439in}}{\pgfqpoint{4.270549in}{1.377439in}}%
\pgfpathcurveto{\pgfqpoint{4.263417in}{1.377439in}}{\pgfqpoint{4.256575in}{1.374605in}}{\pgfqpoint{4.251531in}{1.369561in}}%
\pgfpathcurveto{\pgfqpoint{4.246488in}{1.364518in}}{\pgfqpoint{4.243654in}{1.357676in}}{\pgfqpoint{4.243654in}{1.350543in}}%
\pgfpathcurveto{\pgfqpoint{4.243654in}{1.343410in}}{\pgfqpoint{4.246488in}{1.336569in}}{\pgfqpoint{4.251531in}{1.331525in}}%
\pgfpathcurveto{\pgfqpoint{4.256575in}{1.326481in}}{\pgfqpoint{4.263417in}{1.323648in}}{\pgfqpoint{4.270549in}{1.323648in}}%
\pgfpathclose%
\pgfusepath{stroke,fill}%
\end{pgfscope}%
\begin{pgfscope}%
\pgfpathrectangle{\pgfqpoint{2.867647in}{0.500000in}}{\pgfqpoint{1.764706in}{1.700000in}}%
\pgfusepath{clip}%
\pgfsetbuttcap%
\pgfsetroundjoin%
\definecolor{currentfill}{rgb}{0.978376,0.897317,0.831308}%
\pgfsetfillcolor{currentfill}%
\pgfsetlinewidth{0.311001pt}%
\definecolor{currentstroke}{rgb}{1.000000,1.000000,1.000000}%
\pgfsetstrokecolor{currentstroke}%
\pgfsetdash{}{0pt}%
\pgfpathmoveto{\pgfqpoint{4.127947in}{1.169239in}}%
\pgfpathcurveto{\pgfqpoint{4.135080in}{1.169239in}}{\pgfqpoint{4.141922in}{1.172072in}}{\pgfqpoint{4.146965in}{1.177116in}}%
\pgfpathcurveto{\pgfqpoint{4.152009in}{1.182160in}}{\pgfqpoint{4.154843in}{1.189001in}}{\pgfqpoint{4.154843in}{1.196134in}}%
\pgfpathcurveto{\pgfqpoint{4.154843in}{1.203267in}}{\pgfqpoint{4.152009in}{1.210109in}}{\pgfqpoint{4.146965in}{1.215152in}}%
\pgfpathcurveto{\pgfqpoint{4.141922in}{1.220196in}}{\pgfqpoint{4.135080in}{1.223030in}}{\pgfqpoint{4.127947in}{1.223030in}}%
\pgfpathcurveto{\pgfqpoint{4.120814in}{1.223030in}}{\pgfqpoint{4.113973in}{1.220196in}}{\pgfqpoint{4.108929in}{1.215152in}}%
\pgfpathcurveto{\pgfqpoint{4.103885in}{1.210109in}}{\pgfqpoint{4.101052in}{1.203267in}}{\pgfqpoint{4.101052in}{1.196134in}}%
\pgfpathcurveto{\pgfqpoint{4.101052in}{1.189001in}}{\pgfqpoint{4.103885in}{1.182160in}}{\pgfqpoint{4.108929in}{1.177116in}}%
\pgfpathcurveto{\pgfqpoint{4.113973in}{1.172072in}}{\pgfqpoint{4.120814in}{1.169239in}}{\pgfqpoint{4.127947in}{1.169239in}}%
\pgfpathclose%
\pgfusepath{stroke,fill}%
\end{pgfscope}%
\begin{pgfscope}%
\pgfpathrectangle{\pgfqpoint{2.867647in}{0.500000in}}{\pgfqpoint{1.764706in}{1.700000in}}%
\pgfusepath{clip}%
\pgfsetbuttcap%
\pgfsetroundjoin%
\definecolor{currentfill}{rgb}{0.975644,0.874038,0.797253}%
\pgfsetfillcolor{currentfill}%
\pgfsetlinewidth{0.311001pt}%
\definecolor{currentstroke}{rgb}{1.000000,1.000000,1.000000}%
\pgfsetstrokecolor{currentstroke}%
\pgfsetdash{}{0pt}%
\pgfpathmoveto{\pgfqpoint{4.206919in}{1.530798in}}%
\pgfpathcurveto{\pgfqpoint{4.214052in}{1.530798in}}{\pgfqpoint{4.220894in}{1.533631in}}{\pgfqpoint{4.225937in}{1.538675in}}%
\pgfpathcurveto{\pgfqpoint{4.230981in}{1.543719in}}{\pgfqpoint{4.233815in}{1.550560in}}{\pgfqpoint{4.233815in}{1.557693in}}%
\pgfpathcurveto{\pgfqpoint{4.233815in}{1.564826in}}{\pgfqpoint{4.230981in}{1.571668in}}{\pgfqpoint{4.225937in}{1.576711in}}%
\pgfpathcurveto{\pgfqpoint{4.220894in}{1.581755in}}{\pgfqpoint{4.214052in}{1.584589in}}{\pgfqpoint{4.206919in}{1.584589in}}%
\pgfpathcurveto{\pgfqpoint{4.199786in}{1.584589in}}{\pgfqpoint{4.192945in}{1.581755in}}{\pgfqpoint{4.187901in}{1.576711in}}%
\pgfpathcurveto{\pgfqpoint{4.182857in}{1.571668in}}{\pgfqpoint{4.180023in}{1.564826in}}{\pgfqpoint{4.180023in}{1.557693in}}%
\pgfpathcurveto{\pgfqpoint{4.180023in}{1.550560in}}{\pgfqpoint{4.182857in}{1.543719in}}{\pgfqpoint{4.187901in}{1.538675in}}%
\pgfpathcurveto{\pgfqpoint{4.192945in}{1.533631in}}{\pgfqpoint{4.199786in}{1.530798in}}{\pgfqpoint{4.206919in}{1.530798in}}%
\pgfpathclose%
\pgfusepath{stroke,fill}%
\end{pgfscope}%
\begin{pgfscope}%
\pgfpathrectangle{\pgfqpoint{2.867647in}{0.500000in}}{\pgfqpoint{1.764706in}{1.700000in}}%
\pgfusepath{clip}%
\pgfsetbuttcap%
\pgfsetroundjoin%
\definecolor{currentfill}{rgb}{0.964799,0.689101,0.537560}%
\pgfsetfillcolor{currentfill}%
\pgfsetlinewidth{0.311001pt}%
\definecolor{currentstroke}{rgb}{1.000000,1.000000,1.000000}%
\pgfsetstrokecolor{currentstroke}%
\pgfsetdash{}{0pt}%
\pgfpathmoveto{\pgfqpoint{3.983067in}{0.950421in}}%
\pgfpathcurveto{\pgfqpoint{3.990200in}{0.950421in}}{\pgfqpoint{3.997042in}{0.953255in}}{\pgfqpoint{4.002085in}{0.958299in}}%
\pgfpathcurveto{\pgfqpoint{4.007129in}{0.963343in}}{\pgfqpoint{4.009963in}{0.970184in}}{\pgfqpoint{4.009963in}{0.977317in}}%
\pgfpathcurveto{\pgfqpoint{4.009963in}{0.984450in}}{\pgfqpoint{4.007129in}{0.991292in}}{\pgfqpoint{4.002085in}{0.996335in}}%
\pgfpathcurveto{\pgfqpoint{3.997042in}{1.001379in}}{\pgfqpoint{3.990200in}{1.004213in}}{\pgfqpoint{3.983067in}{1.004213in}}%
\pgfpathcurveto{\pgfqpoint{3.975934in}{1.004213in}}{\pgfqpoint{3.969093in}{1.001379in}}{\pgfqpoint{3.964049in}{0.996335in}}%
\pgfpathcurveto{\pgfqpoint{3.959005in}{0.991292in}}{\pgfqpoint{3.956172in}{0.984450in}}{\pgfqpoint{3.956172in}{0.977317in}}%
\pgfpathcurveto{\pgfqpoint{3.956172in}{0.970184in}}{\pgfqpoint{3.959005in}{0.963343in}}{\pgfqpoint{3.964049in}{0.958299in}}%
\pgfpathcurveto{\pgfqpoint{3.969093in}{0.953255in}}{\pgfqpoint{3.975934in}{0.950421in}}{\pgfqpoint{3.983067in}{0.950421in}}%
\pgfpathclose%
\pgfusepath{stroke,fill}%
\end{pgfscope}%
\begin{pgfscope}%
\pgfpathrectangle{\pgfqpoint{2.867647in}{0.500000in}}{\pgfqpoint{1.764706in}{1.700000in}}%
\pgfusepath{clip}%
\pgfsetbuttcap%
\pgfsetroundjoin%
\definecolor{currentfill}{rgb}{0.977657,0.891500,0.822809}%
\pgfsetfillcolor{currentfill}%
\pgfsetlinewidth{0.311001pt}%
\definecolor{currentstroke}{rgb}{1.000000,1.000000,1.000000}%
\pgfsetstrokecolor{currentstroke}%
\pgfsetdash{}{0pt}%
\pgfpathmoveto{\pgfqpoint{4.207408in}{1.491689in}}%
\pgfpathcurveto{\pgfqpoint{4.214540in}{1.491689in}}{\pgfqpoint{4.221382in}{1.494523in}}{\pgfqpoint{4.226426in}{1.499567in}}%
\pgfpathcurveto{\pgfqpoint{4.231469in}{1.504610in}}{\pgfqpoint{4.234303in}{1.511452in}}{\pgfqpoint{4.234303in}{1.518585in}}%
\pgfpathcurveto{\pgfqpoint{4.234303in}{1.525718in}}{\pgfqpoint{4.231469in}{1.532559in}}{\pgfqpoint{4.226426in}{1.537603in}}%
\pgfpathcurveto{\pgfqpoint{4.221382in}{1.542647in}}{\pgfqpoint{4.214540in}{1.545480in}}{\pgfqpoint{4.207408in}{1.545480in}}%
\pgfpathcurveto{\pgfqpoint{4.200275in}{1.545480in}}{\pgfqpoint{4.193433in}{1.542647in}}{\pgfqpoint{4.188389in}{1.537603in}}%
\pgfpathcurveto{\pgfqpoint{4.183346in}{1.532559in}}{\pgfqpoint{4.180512in}{1.525718in}}{\pgfqpoint{4.180512in}{1.518585in}}%
\pgfpathcurveto{\pgfqpoint{4.180512in}{1.511452in}}{\pgfqpoint{4.183346in}{1.504610in}}{\pgfqpoint{4.188389in}{1.499567in}}%
\pgfpathcurveto{\pgfqpoint{4.193433in}{1.494523in}}{\pgfqpoint{4.200275in}{1.491689in}}{\pgfqpoint{4.207408in}{1.491689in}}%
\pgfpathclose%
\pgfusepath{stroke,fill}%
\end{pgfscope}%
\begin{pgfscope}%
\pgfpathrectangle{\pgfqpoint{2.867647in}{0.500000in}}{\pgfqpoint{1.764706in}{1.700000in}}%
\pgfusepath{clip}%
\pgfsetbuttcap%
\pgfsetroundjoin%
\definecolor{currentfill}{rgb}{0.968105,0.786346,0.667739}%
\pgfsetfillcolor{currentfill}%
\pgfsetlinewidth{0.311001pt}%
\definecolor{currentstroke}{rgb}{1.000000,1.000000,1.000000}%
\pgfsetstrokecolor{currentstroke}%
\pgfsetdash{}{0pt}%
\pgfpathmoveto{\pgfqpoint{4.057569in}{1.714892in}}%
\pgfpathcurveto{\pgfqpoint{4.064702in}{1.714892in}}{\pgfqpoint{4.071544in}{1.717726in}}{\pgfqpoint{4.076587in}{1.722769in}}%
\pgfpathcurveto{\pgfqpoint{4.081631in}{1.727813in}}{\pgfqpoint{4.084465in}{1.734654in}}{\pgfqpoint{4.084465in}{1.741787in}}%
\pgfpathcurveto{\pgfqpoint{4.084465in}{1.748920in}}{\pgfqpoint{4.081631in}{1.755762in}}{\pgfqpoint{4.076587in}{1.760805in}}%
\pgfpathcurveto{\pgfqpoint{4.071544in}{1.765849in}}{\pgfqpoint{4.064702in}{1.768683in}}{\pgfqpoint{4.057569in}{1.768683in}}%
\pgfpathcurveto{\pgfqpoint{4.050436in}{1.768683in}}{\pgfqpoint{4.043595in}{1.765849in}}{\pgfqpoint{4.038551in}{1.760805in}}%
\pgfpathcurveto{\pgfqpoint{4.033507in}{1.755762in}}{\pgfqpoint{4.030673in}{1.748920in}}{\pgfqpoint{4.030673in}{1.741787in}}%
\pgfpathcurveto{\pgfqpoint{4.030673in}{1.734654in}}{\pgfqpoint{4.033507in}{1.727813in}}{\pgfqpoint{4.038551in}{1.722769in}}%
\pgfpathcurveto{\pgfqpoint{4.043595in}{1.717726in}}{\pgfqpoint{4.050436in}{1.714892in}}{\pgfqpoint{4.057569in}{1.714892in}}%
\pgfpathclose%
\pgfusepath{stroke,fill}%
\end{pgfscope}%
\begin{pgfscope}%
\pgfpathrectangle{\pgfqpoint{2.867647in}{0.500000in}}{\pgfqpoint{1.764706in}{1.700000in}}%
\pgfusepath{clip}%
\pgfsetbuttcap%
\pgfsetroundjoin%
\definecolor{currentfill}{rgb}{0.976961,0.885681,0.814303}%
\pgfsetfillcolor{currentfill}%
\pgfsetlinewidth{0.311001pt}%
\definecolor{currentstroke}{rgb}{1.000000,1.000000,1.000000}%
\pgfsetstrokecolor{currentstroke}%
\pgfsetdash{}{0pt}%
\pgfpathmoveto{\pgfqpoint{4.202729in}{1.522208in}}%
\pgfpathcurveto{\pgfqpoint{4.209862in}{1.522208in}}{\pgfqpoint{4.216703in}{1.525042in}}{\pgfqpoint{4.221747in}{1.530086in}}%
\pgfpathcurveto{\pgfqpoint{4.226791in}{1.535130in}}{\pgfqpoint{4.229625in}{1.541971in}}{\pgfqpoint{4.229625in}{1.549104in}}%
\pgfpathcurveto{\pgfqpoint{4.229625in}{1.556237in}}{\pgfqpoint{4.226791in}{1.563079in}}{\pgfqpoint{4.221747in}{1.568122in}}%
\pgfpathcurveto{\pgfqpoint{4.216703in}{1.573166in}}{\pgfqpoint{4.209862in}{1.576000in}}{\pgfqpoint{4.202729in}{1.576000in}}%
\pgfpathcurveto{\pgfqpoint{4.195596in}{1.576000in}}{\pgfqpoint{4.188754in}{1.573166in}}{\pgfqpoint{4.183711in}{1.568122in}}%
\pgfpathcurveto{\pgfqpoint{4.178667in}{1.563079in}}{\pgfqpoint{4.175833in}{1.556237in}}{\pgfqpoint{4.175833in}{1.549104in}}%
\pgfpathcurveto{\pgfqpoint{4.175833in}{1.541971in}}{\pgfqpoint{4.178667in}{1.535130in}}{\pgfqpoint{4.183711in}{1.530086in}}%
\pgfpathcurveto{\pgfqpoint{4.188754in}{1.525042in}}{\pgfqpoint{4.195596in}{1.522208in}}{\pgfqpoint{4.202729in}{1.522208in}}%
\pgfpathclose%
\pgfusepath{stroke,fill}%
\end{pgfscope}%
\begin{pgfscope}%
\pgfpathrectangle{\pgfqpoint{2.867647in}{0.500000in}}{\pgfqpoint{1.764706in}{1.700000in}}%
\pgfusepath{clip}%
\pgfsetbuttcap%
\pgfsetroundjoin%
\definecolor{currentfill}{rgb}{0.965753,0.732351,0.592427}%
\pgfsetfillcolor{currentfill}%
\pgfsetlinewidth{0.311001pt}%
\definecolor{currentstroke}{rgb}{1.000000,1.000000,1.000000}%
\pgfsetstrokecolor{currentstroke}%
\pgfsetdash{}{0pt}%
\pgfpathmoveto{\pgfqpoint{4.044223in}{1.496411in}}%
\pgfpathcurveto{\pgfqpoint{4.051356in}{1.496411in}}{\pgfqpoint{4.058197in}{1.499245in}}{\pgfqpoint{4.063241in}{1.504288in}}%
\pgfpathcurveto{\pgfqpoint{4.068285in}{1.509332in}}{\pgfqpoint{4.071119in}{1.516174in}}{\pgfqpoint{4.071119in}{1.523306in}}%
\pgfpathcurveto{\pgfqpoint{4.071119in}{1.530439in}}{\pgfqpoint{4.068285in}{1.537281in}}{\pgfqpoint{4.063241in}{1.542325in}}%
\pgfpathcurveto{\pgfqpoint{4.058197in}{1.547368in}}{\pgfqpoint{4.051356in}{1.550202in}}{\pgfqpoint{4.044223in}{1.550202in}}%
\pgfpathcurveto{\pgfqpoint{4.037090in}{1.550202in}}{\pgfqpoint{4.030248in}{1.547368in}}{\pgfqpoint{4.025205in}{1.542325in}}%
\pgfpathcurveto{\pgfqpoint{4.020161in}{1.537281in}}{\pgfqpoint{4.017327in}{1.530439in}}{\pgfqpoint{4.017327in}{1.523306in}}%
\pgfpathcurveto{\pgfqpoint{4.017327in}{1.516174in}}{\pgfqpoint{4.020161in}{1.509332in}}{\pgfqpoint{4.025205in}{1.504288in}}%
\pgfpathcurveto{\pgfqpoint{4.030248in}{1.499245in}}{\pgfqpoint{4.037090in}{1.496411in}}{\pgfqpoint{4.044223in}{1.496411in}}%
\pgfpathclose%
\pgfusepath{stroke,fill}%
\end{pgfscope}%
\begin{pgfscope}%
\pgfpathrectangle{\pgfqpoint{2.867647in}{0.500000in}}{\pgfqpoint{1.764706in}{1.700000in}}%
\pgfusepath{clip}%
\pgfsetbuttcap%
\pgfsetroundjoin%
\definecolor{currentfill}{rgb}{0.975644,0.874038,0.797253}%
\pgfsetfillcolor{currentfill}%
\pgfsetlinewidth{0.311001pt}%
\definecolor{currentstroke}{rgb}{1.000000,1.000000,1.000000}%
\pgfsetstrokecolor{currentstroke}%
\pgfsetdash{}{0pt}%
\pgfpathmoveto{\pgfqpoint{4.232413in}{1.440542in}}%
\pgfpathcurveto{\pgfqpoint{4.239546in}{1.440542in}}{\pgfqpoint{4.246388in}{1.443376in}}{\pgfqpoint{4.251432in}{1.448420in}}%
\pgfpathcurveto{\pgfqpoint{4.256475in}{1.453463in}}{\pgfqpoint{4.259309in}{1.460305in}}{\pgfqpoint{4.259309in}{1.467438in}}%
\pgfpathcurveto{\pgfqpoint{4.259309in}{1.474571in}}{\pgfqpoint{4.256475in}{1.481412in}}{\pgfqpoint{4.251432in}{1.486456in}}%
\pgfpathcurveto{\pgfqpoint{4.246388in}{1.491500in}}{\pgfqpoint{4.239546in}{1.494334in}}{\pgfqpoint{4.232413in}{1.494334in}}%
\pgfpathcurveto{\pgfqpoint{4.225281in}{1.494334in}}{\pgfqpoint{4.218439in}{1.491500in}}{\pgfqpoint{4.213395in}{1.486456in}}%
\pgfpathcurveto{\pgfqpoint{4.208352in}{1.481412in}}{\pgfqpoint{4.205518in}{1.474571in}}{\pgfqpoint{4.205518in}{1.467438in}}%
\pgfpathcurveto{\pgfqpoint{4.205518in}{1.460305in}}{\pgfqpoint{4.208352in}{1.453463in}}{\pgfqpoint{4.213395in}{1.448420in}}%
\pgfpathcurveto{\pgfqpoint{4.218439in}{1.443376in}}{\pgfqpoint{4.225281in}{1.440542in}}{\pgfqpoint{4.232413in}{1.440542in}}%
\pgfpathclose%
\pgfusepath{stroke,fill}%
\end{pgfscope}%
\begin{pgfscope}%
\pgfpathrectangle{\pgfqpoint{2.867647in}{0.500000in}}{\pgfqpoint{1.764706in}{1.700000in}}%
\pgfusepath{clip}%
\pgfsetbuttcap%
\pgfsetroundjoin%
\definecolor{currentfill}{rgb}{0.973832,0.856556,0.771584}%
\pgfsetfillcolor{currentfill}%
\pgfsetlinewidth{0.311001pt}%
\definecolor{currentstroke}{rgb}{1.000000,1.000000,1.000000}%
\pgfsetstrokecolor{currentstroke}%
\pgfsetdash{}{0pt}%
\pgfpathmoveto{\pgfqpoint{4.124397in}{1.361892in}}%
\pgfpathcurveto{\pgfqpoint{4.131530in}{1.361892in}}{\pgfqpoint{4.138372in}{1.364726in}}{\pgfqpoint{4.143415in}{1.369770in}}%
\pgfpathcurveto{\pgfqpoint{4.148459in}{1.374813in}}{\pgfqpoint{4.151293in}{1.381655in}}{\pgfqpoint{4.151293in}{1.388788in}}%
\pgfpathcurveto{\pgfqpoint{4.151293in}{1.395921in}}{\pgfqpoint{4.148459in}{1.402762in}}{\pgfqpoint{4.143415in}{1.407806in}}%
\pgfpathcurveto{\pgfqpoint{4.138372in}{1.412850in}}{\pgfqpoint{4.131530in}{1.415684in}}{\pgfqpoint{4.124397in}{1.415684in}}%
\pgfpathcurveto{\pgfqpoint{4.117264in}{1.415684in}}{\pgfqpoint{4.110423in}{1.412850in}}{\pgfqpoint{4.105379in}{1.407806in}}%
\pgfpathcurveto{\pgfqpoint{4.100335in}{1.402762in}}{\pgfqpoint{4.097501in}{1.395921in}}{\pgfqpoint{4.097501in}{1.388788in}}%
\pgfpathcurveto{\pgfqpoint{4.097501in}{1.381655in}}{\pgfqpoint{4.100335in}{1.374813in}}{\pgfqpoint{4.105379in}{1.369770in}}%
\pgfpathcurveto{\pgfqpoint{4.110423in}{1.364726in}}{\pgfqpoint{4.117264in}{1.361892in}}{\pgfqpoint{4.124397in}{1.361892in}}%
\pgfpathclose%
\pgfusepath{stroke,fill}%
\end{pgfscope}%
\begin{pgfscope}%
\pgfpathrectangle{\pgfqpoint{2.867647in}{0.500000in}}{\pgfqpoint{1.764706in}{1.700000in}}%
\pgfusepath{clip}%
\pgfsetbuttcap%
\pgfsetroundjoin%
\definecolor{currentfill}{rgb}{0.978376,0.897317,0.831308}%
\pgfsetfillcolor{currentfill}%
\pgfsetlinewidth{0.311001pt}%
\definecolor{currentstroke}{rgb}{1.000000,1.000000,1.000000}%
\pgfsetstrokecolor{currentstroke}%
\pgfsetdash{}{0pt}%
\pgfpathmoveto{\pgfqpoint{4.229109in}{1.320081in}}%
\pgfpathcurveto{\pgfqpoint{4.236242in}{1.320081in}}{\pgfqpoint{4.243084in}{1.322915in}}{\pgfqpoint{4.248128in}{1.327959in}}%
\pgfpathcurveto{\pgfqpoint{4.253171in}{1.333002in}}{\pgfqpoint{4.256005in}{1.339844in}}{\pgfqpoint{4.256005in}{1.346977in}}%
\pgfpathcurveto{\pgfqpoint{4.256005in}{1.354110in}}{\pgfqpoint{4.253171in}{1.360951in}}{\pgfqpoint{4.248128in}{1.365995in}}%
\pgfpathcurveto{\pgfqpoint{4.243084in}{1.371039in}}{\pgfqpoint{4.236242in}{1.373872in}}{\pgfqpoint{4.229109in}{1.373872in}}%
\pgfpathcurveto{\pgfqpoint{4.221977in}{1.373872in}}{\pgfqpoint{4.215135in}{1.371039in}}{\pgfqpoint{4.210091in}{1.365995in}}%
\pgfpathcurveto{\pgfqpoint{4.205048in}{1.360951in}}{\pgfqpoint{4.202214in}{1.354110in}}{\pgfqpoint{4.202214in}{1.346977in}}%
\pgfpathcurveto{\pgfqpoint{4.202214in}{1.339844in}}{\pgfqpoint{4.205048in}{1.333002in}}{\pgfqpoint{4.210091in}{1.327959in}}%
\pgfpathcurveto{\pgfqpoint{4.215135in}{1.322915in}}{\pgfqpoint{4.221977in}{1.320081in}}{\pgfqpoint{4.229109in}{1.320081in}}%
\pgfpathclose%
\pgfusepath{stroke,fill}%
\end{pgfscope}%
\begin{pgfscope}%
\pgfpathrectangle{\pgfqpoint{2.867647in}{0.500000in}}{\pgfqpoint{1.764706in}{1.700000in}}%
\pgfusepath{clip}%
\pgfsetbuttcap%
\pgfsetroundjoin%
\definecolor{currentfill}{rgb}{0.946260,0.398132,0.274897}%
\pgfsetfillcolor{currentfill}%
\pgfsetlinewidth{0.311001pt}%
\definecolor{currentstroke}{rgb}{1.000000,1.000000,1.000000}%
\pgfsetstrokecolor{currentstroke}%
\pgfsetdash{}{0pt}%
\pgfpathmoveto{\pgfqpoint{3.854414in}{1.675737in}}%
\pgfpathcurveto{\pgfqpoint{3.861547in}{1.675737in}}{\pgfqpoint{3.868389in}{1.678570in}}{\pgfqpoint{3.873432in}{1.683614in}}%
\pgfpathcurveto{\pgfqpoint{3.878476in}{1.688658in}}{\pgfqpoint{3.881310in}{1.695499in}}{\pgfqpoint{3.881310in}{1.702632in}}%
\pgfpathcurveto{\pgfqpoint{3.881310in}{1.709765in}}{\pgfqpoint{3.878476in}{1.716607in}}{\pgfqpoint{3.873432in}{1.721650in}}%
\pgfpathcurveto{\pgfqpoint{3.868389in}{1.726694in}}{\pgfqpoint{3.861547in}{1.729528in}}{\pgfqpoint{3.854414in}{1.729528in}}%
\pgfpathcurveto{\pgfqpoint{3.847282in}{1.729528in}}{\pgfqpoint{3.840440in}{1.726694in}}{\pgfqpoint{3.835396in}{1.721650in}}%
\pgfpathcurveto{\pgfqpoint{3.830353in}{1.716607in}}{\pgfqpoint{3.827519in}{1.709765in}}{\pgfqpoint{3.827519in}{1.702632in}}%
\pgfpathcurveto{\pgfqpoint{3.827519in}{1.695499in}}{\pgfqpoint{3.830353in}{1.688658in}}{\pgfqpoint{3.835396in}{1.683614in}}%
\pgfpathcurveto{\pgfqpoint{3.840440in}{1.678570in}}{\pgfqpoint{3.847282in}{1.675737in}}{\pgfqpoint{3.854414in}{1.675737in}}%
\pgfpathclose%
\pgfusepath{stroke,fill}%
\end{pgfscope}%
\begin{pgfscope}%
\pgfpathrectangle{\pgfqpoint{2.867647in}{0.500000in}}{\pgfqpoint{1.764706in}{1.700000in}}%
\pgfusepath{clip}%
\pgfsetbuttcap%
\pgfsetroundjoin%
\definecolor{currentfill}{rgb}{0.980678,0.914765,0.856766}%
\pgfsetfillcolor{currentfill}%
\pgfsetlinewidth{0.311001pt}%
\definecolor{currentstroke}{rgb}{1.000000,1.000000,1.000000}%
\pgfsetstrokecolor{currentstroke}%
\pgfsetdash{}{0pt}%
\pgfpathmoveto{\pgfqpoint{4.156428in}{1.150584in}}%
\pgfpathcurveto{\pgfqpoint{4.163561in}{1.150584in}}{\pgfqpoint{4.170402in}{1.153418in}}{\pgfqpoint{4.175446in}{1.158462in}}%
\pgfpathcurveto{\pgfqpoint{4.180490in}{1.163506in}}{\pgfqpoint{4.183324in}{1.170347in}}{\pgfqpoint{4.183324in}{1.177480in}}%
\pgfpathcurveto{\pgfqpoint{4.183324in}{1.184613in}}{\pgfqpoint{4.180490in}{1.191455in}}{\pgfqpoint{4.175446in}{1.196498in}}%
\pgfpathcurveto{\pgfqpoint{4.170402in}{1.201542in}}{\pgfqpoint{4.163561in}{1.204376in}}{\pgfqpoint{4.156428in}{1.204376in}}%
\pgfpathcurveto{\pgfqpoint{4.149295in}{1.204376in}}{\pgfqpoint{4.142453in}{1.201542in}}{\pgfqpoint{4.137410in}{1.196498in}}%
\pgfpathcurveto{\pgfqpoint{4.132366in}{1.191455in}}{\pgfqpoint{4.129532in}{1.184613in}}{\pgfqpoint{4.129532in}{1.177480in}}%
\pgfpathcurveto{\pgfqpoint{4.129532in}{1.170347in}}{\pgfqpoint{4.132366in}{1.163506in}}{\pgfqpoint{4.137410in}{1.158462in}}%
\pgfpathcurveto{\pgfqpoint{4.142453in}{1.153418in}}{\pgfqpoint{4.149295in}{1.150584in}}{\pgfqpoint{4.156428in}{1.150584in}}%
\pgfpathclose%
\pgfusepath{stroke,fill}%
\end{pgfscope}%
\begin{pgfscope}%
\pgfpathrectangle{\pgfqpoint{2.867647in}{0.500000in}}{\pgfqpoint{1.764706in}{1.700000in}}%
\pgfusepath{clip}%
\pgfsetbuttcap%
\pgfsetroundjoin%
\definecolor{currentfill}{rgb}{0.981377,0.920617,0.865369}%
\pgfsetfillcolor{currentfill}%
\pgfsetlinewidth{0.311001pt}%
\definecolor{currentstroke}{rgb}{1.000000,1.000000,1.000000}%
\pgfsetstrokecolor{currentstroke}%
\pgfsetdash{}{0pt}%
\pgfpathmoveto{\pgfqpoint{4.183765in}{1.299145in}}%
\pgfpathcurveto{\pgfqpoint{4.190898in}{1.299145in}}{\pgfqpoint{4.197740in}{1.301979in}}{\pgfqpoint{4.202783in}{1.307022in}}%
\pgfpathcurveto{\pgfqpoint{4.207827in}{1.312066in}}{\pgfqpoint{4.210661in}{1.318908in}}{\pgfqpoint{4.210661in}{1.326040in}}%
\pgfpathcurveto{\pgfqpoint{4.210661in}{1.333173in}}{\pgfqpoint{4.207827in}{1.340015in}}{\pgfqpoint{4.202783in}{1.345058in}}%
\pgfpathcurveto{\pgfqpoint{4.197740in}{1.350102in}}{\pgfqpoint{4.190898in}{1.352936in}}{\pgfqpoint{4.183765in}{1.352936in}}%
\pgfpathcurveto{\pgfqpoint{4.176632in}{1.352936in}}{\pgfqpoint{4.169791in}{1.350102in}}{\pgfqpoint{4.164747in}{1.345058in}}%
\pgfpathcurveto{\pgfqpoint{4.159703in}{1.340015in}}{\pgfqpoint{4.156870in}{1.333173in}}{\pgfqpoint{4.156870in}{1.326040in}}%
\pgfpathcurveto{\pgfqpoint{4.156870in}{1.318908in}}{\pgfqpoint{4.159703in}{1.312066in}}{\pgfqpoint{4.164747in}{1.307022in}}%
\pgfpathcurveto{\pgfqpoint{4.169791in}{1.301979in}}{\pgfqpoint{4.176632in}{1.299145in}}{\pgfqpoint{4.183765in}{1.299145in}}%
\pgfpathclose%
\pgfusepath{stroke,fill}%
\end{pgfscope}%
\begin{pgfscope}%
\pgfpathrectangle{\pgfqpoint{2.867647in}{0.500000in}}{\pgfqpoint{1.764706in}{1.700000in}}%
\pgfusepath{clip}%
\pgfsetbuttcap%
\pgfsetroundjoin%
\definecolor{currentfill}{rgb}{0.973832,0.856556,0.771584}%
\pgfsetfillcolor{currentfill}%
\pgfsetlinewidth{0.311001pt}%
\definecolor{currentstroke}{rgb}{1.000000,1.000000,1.000000}%
\pgfsetstrokecolor{currentstroke}%
\pgfsetdash{}{0pt}%
\pgfpathmoveto{\pgfqpoint{4.160816in}{1.017460in}}%
\pgfpathcurveto{\pgfqpoint{4.167949in}{1.017460in}}{\pgfqpoint{4.174791in}{1.020294in}}{\pgfqpoint{4.179835in}{1.025337in}}%
\pgfpathcurveto{\pgfqpoint{4.184878in}{1.030381in}}{\pgfqpoint{4.187712in}{1.037223in}}{\pgfqpoint{4.187712in}{1.044356in}}%
\pgfpathcurveto{\pgfqpoint{4.187712in}{1.051488in}}{\pgfqpoint{4.184878in}{1.058330in}}{\pgfqpoint{4.179835in}{1.063374in}}%
\pgfpathcurveto{\pgfqpoint{4.174791in}{1.068417in}}{\pgfqpoint{4.167949in}{1.071251in}}{\pgfqpoint{4.160816in}{1.071251in}}%
\pgfpathcurveto{\pgfqpoint{4.153684in}{1.071251in}}{\pgfqpoint{4.146842in}{1.068417in}}{\pgfqpoint{4.141798in}{1.063374in}}%
\pgfpathcurveto{\pgfqpoint{4.136755in}{1.058330in}}{\pgfqpoint{4.133921in}{1.051488in}}{\pgfqpoint{4.133921in}{1.044356in}}%
\pgfpathcurveto{\pgfqpoint{4.133921in}{1.037223in}}{\pgfqpoint{4.136755in}{1.030381in}}{\pgfqpoint{4.141798in}{1.025337in}}%
\pgfpathcurveto{\pgfqpoint{4.146842in}{1.020294in}}{\pgfqpoint{4.153684in}{1.017460in}}{\pgfqpoint{4.160816in}{1.017460in}}%
\pgfpathclose%
\pgfusepath{stroke,fill}%
\end{pgfscope}%
\begin{pgfscope}%
\pgfpathrectangle{\pgfqpoint{2.867647in}{0.500000in}}{\pgfqpoint{1.764706in}{1.700000in}}%
\pgfusepath{clip}%
\pgfsetbuttcap%
\pgfsetroundjoin%
\definecolor{currentfill}{rgb}{0.981377,0.920617,0.865369}%
\pgfsetfillcolor{currentfill}%
\pgfsetlinewidth{0.311001pt}%
\definecolor{currentstroke}{rgb}{1.000000,1.000000,1.000000}%
\pgfsetstrokecolor{currentstroke}%
\pgfsetdash{}{0pt}%
\pgfpathmoveto{\pgfqpoint{4.184765in}{1.220926in}}%
\pgfpathcurveto{\pgfqpoint{4.191897in}{1.220926in}}{\pgfqpoint{4.198739in}{1.223760in}}{\pgfqpoint{4.203783in}{1.228804in}}%
\pgfpathcurveto{\pgfqpoint{4.208826in}{1.233848in}}{\pgfqpoint{4.211660in}{1.240689in}}{\pgfqpoint{4.211660in}{1.247822in}}%
\pgfpathcurveto{\pgfqpoint{4.211660in}{1.254955in}}{\pgfqpoint{4.208826in}{1.261797in}}{\pgfqpoint{4.203783in}{1.266840in}}%
\pgfpathcurveto{\pgfqpoint{4.198739in}{1.271884in}}{\pgfqpoint{4.191897in}{1.274718in}}{\pgfqpoint{4.184765in}{1.274718in}}%
\pgfpathcurveto{\pgfqpoint{4.177632in}{1.274718in}}{\pgfqpoint{4.170790in}{1.271884in}}{\pgfqpoint{4.165746in}{1.266840in}}%
\pgfpathcurveto{\pgfqpoint{4.160703in}{1.261797in}}{\pgfqpoint{4.157869in}{1.254955in}}{\pgfqpoint{4.157869in}{1.247822in}}%
\pgfpathcurveto{\pgfqpoint{4.157869in}{1.240689in}}{\pgfqpoint{4.160703in}{1.233848in}}{\pgfqpoint{4.165746in}{1.228804in}}%
\pgfpathcurveto{\pgfqpoint{4.170790in}{1.223760in}}{\pgfqpoint{4.177632in}{1.220926in}}{\pgfqpoint{4.184765in}{1.220926in}}%
\pgfpathclose%
\pgfusepath{stroke,fill}%
\end{pgfscope}%
\begin{pgfscope}%
\pgfpathrectangle{\pgfqpoint{2.867647in}{0.500000in}}{\pgfqpoint{1.764706in}{1.700000in}}%
\pgfusepath{clip}%
\pgfsetbuttcap%
\pgfsetroundjoin%
\definecolor{currentfill}{rgb}{0.966560,0.756582,0.625273}%
\pgfsetfillcolor{currentfill}%
\pgfsetlinewidth{0.311001pt}%
\definecolor{currentstroke}{rgb}{1.000000,1.000000,1.000000}%
\pgfsetstrokecolor{currentstroke}%
\pgfsetdash{}{0pt}%
\pgfpathmoveto{\pgfqpoint{4.013001in}{0.997864in}}%
\pgfpathcurveto{\pgfqpoint{4.020134in}{0.997864in}}{\pgfqpoint{4.026975in}{1.000698in}}{\pgfqpoint{4.032019in}{1.005741in}}%
\pgfpathcurveto{\pgfqpoint{4.037063in}{1.010785in}}{\pgfqpoint{4.039897in}{1.017627in}}{\pgfqpoint{4.039897in}{1.024760in}}%
\pgfpathcurveto{\pgfqpoint{4.039897in}{1.031892in}}{\pgfqpoint{4.037063in}{1.038734in}}{\pgfqpoint{4.032019in}{1.043778in}}%
\pgfpathcurveto{\pgfqpoint{4.026975in}{1.048821in}}{\pgfqpoint{4.020134in}{1.051655in}}{\pgfqpoint{4.013001in}{1.051655in}}%
\pgfpathcurveto{\pgfqpoint{4.005868in}{1.051655in}}{\pgfqpoint{3.999026in}{1.048821in}}{\pgfqpoint{3.993983in}{1.043778in}}%
\pgfpathcurveto{\pgfqpoint{3.988939in}{1.038734in}}{\pgfqpoint{3.986105in}{1.031892in}}{\pgfqpoint{3.986105in}{1.024760in}}%
\pgfpathcurveto{\pgfqpoint{3.986105in}{1.017627in}}{\pgfqpoint{3.988939in}{1.010785in}}{\pgfqpoint{3.993983in}{1.005741in}}%
\pgfpathcurveto{\pgfqpoint{3.999026in}{1.000698in}}{\pgfqpoint{4.005868in}{0.997864in}}{\pgfqpoint{4.013001in}{0.997864in}}%
\pgfpathclose%
\pgfusepath{stroke,fill}%
\end{pgfscope}%
\begin{pgfscope}%
\pgfpathrectangle{\pgfqpoint{2.867647in}{0.500000in}}{\pgfqpoint{1.764706in}{1.700000in}}%
\pgfusepath{clip}%
\pgfsetbuttcap%
\pgfsetroundjoin%
\definecolor{currentfill}{rgb}{0.964679,0.682838,0.530002}%
\pgfsetfillcolor{currentfill}%
\pgfsetlinewidth{0.311001pt}%
\definecolor{currentstroke}{rgb}{1.000000,1.000000,1.000000}%
\pgfsetstrokecolor{currentstroke}%
\pgfsetdash{}{0pt}%
\pgfpathmoveto{\pgfqpoint{4.212181in}{1.666826in}}%
\pgfpathcurveto{\pgfqpoint{4.219314in}{1.666826in}}{\pgfqpoint{4.226155in}{1.669660in}}{\pgfqpoint{4.231199in}{1.674703in}}%
\pgfpathcurveto{\pgfqpoint{4.236243in}{1.679747in}}{\pgfqpoint{4.239076in}{1.686589in}}{\pgfqpoint{4.239076in}{1.693721in}}%
\pgfpathcurveto{\pgfqpoint{4.239076in}{1.700854in}}{\pgfqpoint{4.236243in}{1.707696in}}{\pgfqpoint{4.231199in}{1.712740in}}%
\pgfpathcurveto{\pgfqpoint{4.226155in}{1.717783in}}{\pgfqpoint{4.219314in}{1.720617in}}{\pgfqpoint{4.212181in}{1.720617in}}%
\pgfpathcurveto{\pgfqpoint{4.205048in}{1.720617in}}{\pgfqpoint{4.198206in}{1.717783in}}{\pgfqpoint{4.193163in}{1.712740in}}%
\pgfpathcurveto{\pgfqpoint{4.188119in}{1.707696in}}{\pgfqpoint{4.185285in}{1.700854in}}{\pgfqpoint{4.185285in}{1.693721in}}%
\pgfpathcurveto{\pgfqpoint{4.185285in}{1.686589in}}{\pgfqpoint{4.188119in}{1.679747in}}{\pgfqpoint{4.193163in}{1.674703in}}%
\pgfpathcurveto{\pgfqpoint{4.198206in}{1.669660in}}{\pgfqpoint{4.205048in}{1.666826in}}{\pgfqpoint{4.212181in}{1.666826in}}%
\pgfpathclose%
\pgfusepath{stroke,fill}%
\end{pgfscope}%
\begin{pgfscope}%
\pgfpathrectangle{\pgfqpoint{2.867647in}{0.500000in}}{\pgfqpoint{1.764706in}{1.700000in}}%
\pgfusepath{clip}%
\pgfsetbuttcap%
\pgfsetroundjoin%
\definecolor{currentfill}{rgb}{0.981377,0.920617,0.865369}%
\pgfsetfillcolor{currentfill}%
\pgfsetlinewidth{0.311001pt}%
\definecolor{currentstroke}{rgb}{1.000000,1.000000,1.000000}%
\pgfsetstrokecolor{currentstroke}%
\pgfsetdash{}{0pt}%
\pgfpathmoveto{\pgfqpoint{4.195064in}{1.359640in}}%
\pgfpathcurveto{\pgfqpoint{4.202197in}{1.359640in}}{\pgfqpoint{4.209038in}{1.362474in}}{\pgfqpoint{4.214082in}{1.367518in}}%
\pgfpathcurveto{\pgfqpoint{4.219125in}{1.372561in}}{\pgfqpoint{4.221959in}{1.379403in}}{\pgfqpoint{4.221959in}{1.386536in}}%
\pgfpathcurveto{\pgfqpoint{4.221959in}{1.393669in}}{\pgfqpoint{4.219125in}{1.400510in}}{\pgfqpoint{4.214082in}{1.405554in}}%
\pgfpathcurveto{\pgfqpoint{4.209038in}{1.410598in}}{\pgfqpoint{4.202197in}{1.413432in}}{\pgfqpoint{4.195064in}{1.413432in}}%
\pgfpathcurveto{\pgfqpoint{4.187931in}{1.413432in}}{\pgfqpoint{4.181089in}{1.410598in}}{\pgfqpoint{4.176046in}{1.405554in}}%
\pgfpathcurveto{\pgfqpoint{4.171002in}{1.400510in}}{\pgfqpoint{4.168168in}{1.393669in}}{\pgfqpoint{4.168168in}{1.386536in}}%
\pgfpathcurveto{\pgfqpoint{4.168168in}{1.379403in}}{\pgfqpoint{4.171002in}{1.372561in}}{\pgfqpoint{4.176046in}{1.367518in}}%
\pgfpathcurveto{\pgfqpoint{4.181089in}{1.362474in}}{\pgfqpoint{4.187931in}{1.359640in}}{\pgfqpoint{4.195064in}{1.359640in}}%
\pgfpathclose%
\pgfusepath{stroke,fill}%
\end{pgfscope}%
\begin{pgfscope}%
\pgfpathrectangle{\pgfqpoint{2.867647in}{0.500000in}}{\pgfqpoint{1.764706in}{1.700000in}}%
\pgfusepath{clip}%
\pgfsetbuttcap%
\pgfsetroundjoin%
\definecolor{currentfill}{rgb}{0.964799,0.689101,0.537560}%
\pgfsetfillcolor{currentfill}%
\pgfsetlinewidth{0.311001pt}%
\definecolor{currentstroke}{rgb}{1.000000,1.000000,1.000000}%
\pgfsetstrokecolor{currentstroke}%
\pgfsetdash{}{0pt}%
\pgfpathmoveto{\pgfqpoint{4.311419in}{1.263683in}}%
\pgfpathcurveto{\pgfqpoint{4.318552in}{1.263683in}}{\pgfqpoint{4.325394in}{1.266517in}}{\pgfqpoint{4.330437in}{1.271561in}}%
\pgfpathcurveto{\pgfqpoint{4.335481in}{1.276605in}}{\pgfqpoint{4.338315in}{1.283446in}}{\pgfqpoint{4.338315in}{1.290579in}}%
\pgfpathcurveto{\pgfqpoint{4.338315in}{1.297712in}}{\pgfqpoint{4.335481in}{1.304554in}}{\pgfqpoint{4.330437in}{1.309597in}}%
\pgfpathcurveto{\pgfqpoint{4.325394in}{1.314641in}}{\pgfqpoint{4.318552in}{1.317475in}}{\pgfqpoint{4.311419in}{1.317475in}}%
\pgfpathcurveto{\pgfqpoint{4.304286in}{1.317475in}}{\pgfqpoint{4.297445in}{1.314641in}}{\pgfqpoint{4.292401in}{1.309597in}}%
\pgfpathcurveto{\pgfqpoint{4.287358in}{1.304554in}}{\pgfqpoint{4.284524in}{1.297712in}}{\pgfqpoint{4.284524in}{1.290579in}}%
\pgfpathcurveto{\pgfqpoint{4.284524in}{1.283446in}}{\pgfqpoint{4.287358in}{1.276605in}}{\pgfqpoint{4.292401in}{1.271561in}}%
\pgfpathcurveto{\pgfqpoint{4.297445in}{1.266517in}}{\pgfqpoint{4.304286in}{1.263683in}}{\pgfqpoint{4.311419in}{1.263683in}}%
\pgfpathclose%
\pgfusepath{stroke,fill}%
\end{pgfscope}%
\begin{pgfscope}%
\pgfpathrectangle{\pgfqpoint{2.867647in}{0.500000in}}{\pgfqpoint{1.764706in}{1.700000in}}%
\pgfusepath{clip}%
\pgfsetbuttcap%
\pgfsetroundjoin%
\definecolor{currentfill}{rgb}{0.966560,0.756582,0.625273}%
\pgfsetfillcolor{currentfill}%
\pgfsetlinewidth{0.311001pt}%
\definecolor{currentstroke}{rgb}{1.000000,1.000000,1.000000}%
\pgfsetstrokecolor{currentstroke}%
\pgfsetdash{}{0pt}%
\pgfpathmoveto{\pgfqpoint{4.032361in}{0.924208in}}%
\pgfpathcurveto{\pgfqpoint{4.039494in}{0.924208in}}{\pgfqpoint{4.046335in}{0.927042in}}{\pgfqpoint{4.051379in}{0.932085in}}%
\pgfpathcurveto{\pgfqpoint{4.056423in}{0.937129in}}{\pgfqpoint{4.059257in}{0.943971in}}{\pgfqpoint{4.059257in}{0.951104in}}%
\pgfpathcurveto{\pgfqpoint{4.059257in}{0.958236in}}{\pgfqpoint{4.056423in}{0.965078in}}{\pgfqpoint{4.051379in}{0.970122in}}%
\pgfpathcurveto{\pgfqpoint{4.046335in}{0.975165in}}{\pgfqpoint{4.039494in}{0.977999in}}{\pgfqpoint{4.032361in}{0.977999in}}%
\pgfpathcurveto{\pgfqpoint{4.025228in}{0.977999in}}{\pgfqpoint{4.018387in}{0.975165in}}{\pgfqpoint{4.013343in}{0.970122in}}%
\pgfpathcurveto{\pgfqpoint{4.008299in}{0.965078in}}{\pgfqpoint{4.005465in}{0.958236in}}{\pgfqpoint{4.005465in}{0.951104in}}%
\pgfpathcurveto{\pgfqpoint{4.005465in}{0.943971in}}{\pgfqpoint{4.008299in}{0.937129in}}{\pgfqpoint{4.013343in}{0.932085in}}%
\pgfpathcurveto{\pgfqpoint{4.018387in}{0.927042in}}{\pgfqpoint{4.025228in}{0.924208in}}{\pgfqpoint{4.032361in}{0.924208in}}%
\pgfpathclose%
\pgfusepath{stroke,fill}%
\end{pgfscope}%
\begin{pgfscope}%
\pgfpathrectangle{\pgfqpoint{2.867647in}{0.500000in}}{\pgfqpoint{1.764706in}{1.700000in}}%
\pgfusepath{clip}%
\pgfsetbuttcap%
\pgfsetroundjoin%
\definecolor{currentfill}{rgb}{0.977657,0.891500,0.822809}%
\pgfsetfillcolor{currentfill}%
\pgfsetlinewidth{0.311001pt}%
\definecolor{currentstroke}{rgb}{1.000000,1.000000,1.000000}%
\pgfsetstrokecolor{currentstroke}%
\pgfsetdash{}{0pt}%
\pgfpathmoveto{\pgfqpoint{4.146924in}{1.365491in}}%
\pgfpathcurveto{\pgfqpoint{4.154057in}{1.365491in}}{\pgfqpoint{4.160899in}{1.368325in}}{\pgfqpoint{4.165942in}{1.373368in}}%
\pgfpathcurveto{\pgfqpoint{4.170986in}{1.378412in}}{\pgfqpoint{4.173820in}{1.385254in}}{\pgfqpoint{4.173820in}{1.392387in}}%
\pgfpathcurveto{\pgfqpoint{4.173820in}{1.399519in}}{\pgfqpoint{4.170986in}{1.406361in}}{\pgfqpoint{4.165942in}{1.411405in}}%
\pgfpathcurveto{\pgfqpoint{4.160899in}{1.416448in}}{\pgfqpoint{4.154057in}{1.419282in}}{\pgfqpoint{4.146924in}{1.419282in}}%
\pgfpathcurveto{\pgfqpoint{4.139791in}{1.419282in}}{\pgfqpoint{4.132950in}{1.416448in}}{\pgfqpoint{4.127906in}{1.411405in}}%
\pgfpathcurveto{\pgfqpoint{4.122862in}{1.406361in}}{\pgfqpoint{4.120028in}{1.399519in}}{\pgfqpoint{4.120028in}{1.392387in}}%
\pgfpathcurveto{\pgfqpoint{4.120028in}{1.385254in}}{\pgfqpoint{4.122862in}{1.378412in}}{\pgfqpoint{4.127906in}{1.373368in}}%
\pgfpathcurveto{\pgfqpoint{4.132950in}{1.368325in}}{\pgfqpoint{4.139791in}{1.365491in}}{\pgfqpoint{4.146924in}{1.365491in}}%
\pgfpathclose%
\pgfusepath{stroke,fill}%
\end{pgfscope}%
\begin{pgfscope}%
\pgfpathrectangle{\pgfqpoint{2.867647in}{0.500000in}}{\pgfqpoint{1.764706in}{1.700000in}}%
\pgfusepath{clip}%
\pgfsetbuttcap%
\pgfsetroundjoin%
\definecolor{currentfill}{rgb}{0.981377,0.920617,0.865369}%
\pgfsetfillcolor{currentfill}%
\pgfsetlinewidth{0.311001pt}%
\definecolor{currentstroke}{rgb}{1.000000,1.000000,1.000000}%
\pgfsetstrokecolor{currentstroke}%
\pgfsetdash{}{0pt}%
\pgfpathmoveto{\pgfqpoint{4.204115in}{1.314433in}}%
\pgfpathcurveto{\pgfqpoint{4.211248in}{1.314433in}}{\pgfqpoint{4.218090in}{1.317266in}}{\pgfqpoint{4.223133in}{1.322310in}}%
\pgfpathcurveto{\pgfqpoint{4.228177in}{1.327354in}}{\pgfqpoint{4.231011in}{1.334195in}}{\pgfqpoint{4.231011in}{1.341328in}}%
\pgfpathcurveto{\pgfqpoint{4.231011in}{1.348461in}}{\pgfqpoint{4.228177in}{1.355303in}}{\pgfqpoint{4.223133in}{1.360346in}}%
\pgfpathcurveto{\pgfqpoint{4.218090in}{1.365390in}}{\pgfqpoint{4.211248in}{1.368224in}}{\pgfqpoint{4.204115in}{1.368224in}}%
\pgfpathcurveto{\pgfqpoint{4.196982in}{1.368224in}}{\pgfqpoint{4.190141in}{1.365390in}}{\pgfqpoint{4.185097in}{1.360346in}}%
\pgfpathcurveto{\pgfqpoint{4.180053in}{1.355303in}}{\pgfqpoint{4.177220in}{1.348461in}}{\pgfqpoint{4.177220in}{1.341328in}}%
\pgfpathcurveto{\pgfqpoint{4.177220in}{1.334195in}}{\pgfqpoint{4.180053in}{1.327354in}}{\pgfqpoint{4.185097in}{1.322310in}}%
\pgfpathcurveto{\pgfqpoint{4.190141in}{1.317266in}}{\pgfqpoint{4.196982in}{1.314433in}}{\pgfqpoint{4.204115in}{1.314433in}}%
\pgfpathclose%
\pgfusepath{stroke,fill}%
\end{pgfscope}%
\begin{pgfscope}%
\pgfpathrectangle{\pgfqpoint{2.867647in}{0.500000in}}{\pgfqpoint{1.764706in}{1.700000in}}%
\pgfusepath{clip}%
\pgfsetbuttcap%
\pgfsetroundjoin%
\definecolor{currentfill}{rgb}{0.962765,0.606121,0.444717}%
\pgfsetfillcolor{currentfill}%
\pgfsetlinewidth{0.311001pt}%
\definecolor{currentstroke}{rgb}{1.000000,1.000000,1.000000}%
\pgfsetstrokecolor{currentstroke}%
\pgfsetdash{}{0pt}%
\pgfpathmoveto{\pgfqpoint{4.321468in}{1.429374in}}%
\pgfpathcurveto{\pgfqpoint{4.328600in}{1.429374in}}{\pgfqpoint{4.335442in}{1.432208in}}{\pgfqpoint{4.340486in}{1.437252in}}%
\pgfpathcurveto{\pgfqpoint{4.345529in}{1.442295in}}{\pgfqpoint{4.348363in}{1.449137in}}{\pgfqpoint{4.348363in}{1.456270in}}%
\pgfpathcurveto{\pgfqpoint{4.348363in}{1.463403in}}{\pgfqpoint{4.345529in}{1.470244in}}{\pgfqpoint{4.340486in}{1.475288in}}%
\pgfpathcurveto{\pgfqpoint{4.335442in}{1.480332in}}{\pgfqpoint{4.328600in}{1.483166in}}{\pgfqpoint{4.321468in}{1.483166in}}%
\pgfpathcurveto{\pgfqpoint{4.314335in}{1.483166in}}{\pgfqpoint{4.307493in}{1.480332in}}{\pgfqpoint{4.302449in}{1.475288in}}%
\pgfpathcurveto{\pgfqpoint{4.297406in}{1.470244in}}{\pgfqpoint{4.294572in}{1.463403in}}{\pgfqpoint{4.294572in}{1.456270in}}%
\pgfpathcurveto{\pgfqpoint{4.294572in}{1.449137in}}{\pgfqpoint{4.297406in}{1.442295in}}{\pgfqpoint{4.302449in}{1.437252in}}%
\pgfpathcurveto{\pgfqpoint{4.307493in}{1.432208in}}{\pgfqpoint{4.314335in}{1.429374in}}{\pgfqpoint{4.321468in}{1.429374in}}%
\pgfpathclose%
\pgfusepath{stroke,fill}%
\end{pgfscope}%
\begin{pgfscope}%
\pgfpathrectangle{\pgfqpoint{2.867647in}{0.500000in}}{\pgfqpoint{1.764706in}{1.700000in}}%
\pgfusepath{clip}%
\pgfsetbuttcap%
\pgfsetroundjoin%
\definecolor{currentfill}{rgb}{0.974412,0.862387,0.780156}%
\pgfsetfillcolor{currentfill}%
\pgfsetlinewidth{0.311001pt}%
\definecolor{currentstroke}{rgb}{1.000000,1.000000,1.000000}%
\pgfsetstrokecolor{currentstroke}%
\pgfsetdash{}{0pt}%
\pgfpathmoveto{\pgfqpoint{4.131930in}{1.003482in}}%
\pgfpathcurveto{\pgfqpoint{4.139063in}{1.003482in}}{\pgfqpoint{4.145904in}{1.006316in}}{\pgfqpoint{4.150948in}{1.011359in}}%
\pgfpathcurveto{\pgfqpoint{4.155992in}{1.016403in}}{\pgfqpoint{4.158826in}{1.023245in}}{\pgfqpoint{4.158826in}{1.030378in}}%
\pgfpathcurveto{\pgfqpoint{4.158826in}{1.037510in}}{\pgfqpoint{4.155992in}{1.044352in}}{\pgfqpoint{4.150948in}{1.049396in}}%
\pgfpathcurveto{\pgfqpoint{4.145904in}{1.054439in}}{\pgfqpoint{4.139063in}{1.057273in}}{\pgfqpoint{4.131930in}{1.057273in}}%
\pgfpathcurveto{\pgfqpoint{4.124797in}{1.057273in}}{\pgfqpoint{4.117955in}{1.054439in}}{\pgfqpoint{4.112912in}{1.049396in}}%
\pgfpathcurveto{\pgfqpoint{4.107868in}{1.044352in}}{\pgfqpoint{4.105034in}{1.037510in}}{\pgfqpoint{4.105034in}{1.030378in}}%
\pgfpathcurveto{\pgfqpoint{4.105034in}{1.023245in}}{\pgfqpoint{4.107868in}{1.016403in}}{\pgfqpoint{4.112912in}{1.011359in}}%
\pgfpathcurveto{\pgfqpoint{4.117955in}{1.006316in}}{\pgfqpoint{4.124797in}{1.003482in}}{\pgfqpoint{4.131930in}{1.003482in}}%
\pgfpathclose%
\pgfusepath{stroke,fill}%
\end{pgfscope}%
\begin{pgfscope}%
\pgfpathrectangle{\pgfqpoint{2.867647in}{0.500000in}}{\pgfqpoint{1.764706in}{1.700000in}}%
\pgfusepath{clip}%
\pgfsetbuttcap%
\pgfsetroundjoin%
\definecolor{currentfill}{rgb}{0.979124,0.903132,0.839793}%
\pgfsetfillcolor{currentfill}%
\pgfsetlinewidth{0.311001pt}%
\definecolor{currentstroke}{rgb}{1.000000,1.000000,1.000000}%
\pgfsetstrokecolor{currentstroke}%
\pgfsetdash{}{0pt}%
\pgfpathmoveto{\pgfqpoint{4.138557in}{1.077730in}}%
\pgfpathcurveto{\pgfqpoint{4.145690in}{1.077730in}}{\pgfqpoint{4.152532in}{1.080564in}}{\pgfqpoint{4.157575in}{1.085608in}}%
\pgfpathcurveto{\pgfqpoint{4.162619in}{1.090651in}}{\pgfqpoint{4.165453in}{1.097493in}}{\pgfqpoint{4.165453in}{1.104626in}}%
\pgfpathcurveto{\pgfqpoint{4.165453in}{1.111759in}}{\pgfqpoint{4.162619in}{1.118600in}}{\pgfqpoint{4.157575in}{1.123644in}}%
\pgfpathcurveto{\pgfqpoint{4.152532in}{1.128688in}}{\pgfqpoint{4.145690in}{1.131522in}}{\pgfqpoint{4.138557in}{1.131522in}}%
\pgfpathcurveto{\pgfqpoint{4.131424in}{1.131522in}}{\pgfqpoint{4.124583in}{1.128688in}}{\pgfqpoint{4.119539in}{1.123644in}}%
\pgfpathcurveto{\pgfqpoint{4.114495in}{1.118600in}}{\pgfqpoint{4.111662in}{1.111759in}}{\pgfqpoint{4.111662in}{1.104626in}}%
\pgfpathcurveto{\pgfqpoint{4.111662in}{1.097493in}}{\pgfqpoint{4.114495in}{1.090651in}}{\pgfqpoint{4.119539in}{1.085608in}}%
\pgfpathcurveto{\pgfqpoint{4.124583in}{1.080564in}}{\pgfqpoint{4.131424in}{1.077730in}}{\pgfqpoint{4.138557in}{1.077730in}}%
\pgfpathclose%
\pgfusepath{stroke,fill}%
\end{pgfscope}%
\begin{pgfscope}%
\pgfpathrectangle{\pgfqpoint{2.867647in}{0.500000in}}{\pgfqpoint{1.764706in}{1.700000in}}%
\pgfusepath{clip}%
\pgfsetbuttcap%
\pgfsetroundjoin%
\definecolor{currentfill}{rgb}{0.967398,0.774513,0.650573}%
\pgfsetfillcolor{currentfill}%
\pgfsetlinewidth{0.311001pt}%
\definecolor{currentstroke}{rgb}{1.000000,1.000000,1.000000}%
\pgfsetstrokecolor{currentstroke}%
\pgfsetdash{}{0pt}%
\pgfpathmoveto{\pgfqpoint{4.023596in}{1.009233in}}%
\pgfpathcurveto{\pgfqpoint{4.030729in}{1.009233in}}{\pgfqpoint{4.037571in}{1.012067in}}{\pgfqpoint{4.042615in}{1.017111in}}%
\pgfpathcurveto{\pgfqpoint{4.047658in}{1.022154in}}{\pgfqpoint{4.050492in}{1.028996in}}{\pgfqpoint{4.050492in}{1.036129in}}%
\pgfpathcurveto{\pgfqpoint{4.050492in}{1.043262in}}{\pgfqpoint{4.047658in}{1.050103in}}{\pgfqpoint{4.042615in}{1.055147in}}%
\pgfpathcurveto{\pgfqpoint{4.037571in}{1.060191in}}{\pgfqpoint{4.030729in}{1.063024in}}{\pgfqpoint{4.023596in}{1.063024in}}%
\pgfpathcurveto{\pgfqpoint{4.016464in}{1.063024in}}{\pgfqpoint{4.009622in}{1.060191in}}{\pgfqpoint{4.004578in}{1.055147in}}%
\pgfpathcurveto{\pgfqpoint{3.999535in}{1.050103in}}{\pgfqpoint{3.996701in}{1.043262in}}{\pgfqpoint{3.996701in}{1.036129in}}%
\pgfpathcurveto{\pgfqpoint{3.996701in}{1.028996in}}{\pgfqpoint{3.999535in}{1.022154in}}{\pgfqpoint{4.004578in}{1.017111in}}%
\pgfpathcurveto{\pgfqpoint{4.009622in}{1.012067in}}{\pgfqpoint{4.016464in}{1.009233in}}{\pgfqpoint{4.023596in}{1.009233in}}%
\pgfpathclose%
\pgfusepath{stroke,fill}%
\end{pgfscope}%
\begin{pgfscope}%
\pgfpathrectangle{\pgfqpoint{2.867647in}{0.500000in}}{\pgfqpoint{1.764706in}{1.700000in}}%
\pgfusepath{clip}%
\pgfsetbuttcap%
\pgfsetroundjoin%
\definecolor{currentfill}{rgb}{0.965169,0.707764,0.560659}%
\pgfsetfillcolor{currentfill}%
\pgfsetlinewidth{0.311001pt}%
\definecolor{currentstroke}{rgb}{1.000000,1.000000,1.000000}%
\pgfsetstrokecolor{currentstroke}%
\pgfsetdash{}{0pt}%
\pgfpathmoveto{\pgfqpoint{4.273591in}{1.101094in}}%
\pgfpathcurveto{\pgfqpoint{4.280724in}{1.101094in}}{\pgfqpoint{4.287565in}{1.103928in}}{\pgfqpoint{4.292609in}{1.108972in}}%
\pgfpathcurveto{\pgfqpoint{4.297653in}{1.114015in}}{\pgfqpoint{4.300487in}{1.120857in}}{\pgfqpoint{4.300487in}{1.127990in}}%
\pgfpathcurveto{\pgfqpoint{4.300487in}{1.135123in}}{\pgfqpoint{4.297653in}{1.141964in}}{\pgfqpoint{4.292609in}{1.147008in}}%
\pgfpathcurveto{\pgfqpoint{4.287565in}{1.152052in}}{\pgfqpoint{4.280724in}{1.154886in}}{\pgfqpoint{4.273591in}{1.154886in}}%
\pgfpathcurveto{\pgfqpoint{4.266458in}{1.154886in}}{\pgfqpoint{4.259617in}{1.152052in}}{\pgfqpoint{4.254573in}{1.147008in}}%
\pgfpathcurveto{\pgfqpoint{4.249529in}{1.141964in}}{\pgfqpoint{4.246695in}{1.135123in}}{\pgfqpoint{4.246695in}{1.127990in}}%
\pgfpathcurveto{\pgfqpoint{4.246695in}{1.120857in}}{\pgfqpoint{4.249529in}{1.114015in}}{\pgfqpoint{4.254573in}{1.108972in}}%
\pgfpathcurveto{\pgfqpoint{4.259617in}{1.103928in}}{\pgfqpoint{4.266458in}{1.101094in}}{\pgfqpoint{4.273591in}{1.101094in}}%
\pgfpathclose%
\pgfusepath{stroke,fill}%
\end{pgfscope}%
\begin{pgfscope}%
\pgfpathrectangle{\pgfqpoint{2.867647in}{0.500000in}}{\pgfqpoint{1.764706in}{1.700000in}}%
\pgfusepath{clip}%
\pgfsetbuttcap%
\pgfsetroundjoin%
\definecolor{currentfill}{rgb}{0.975018,0.868213,0.788710}%
\pgfsetfillcolor{currentfill}%
\pgfsetlinewidth{0.311001pt}%
\definecolor{currentstroke}{rgb}{1.000000,1.000000,1.000000}%
\pgfsetstrokecolor{currentstroke}%
\pgfsetdash{}{0pt}%
\pgfpathmoveto{\pgfqpoint{4.202639in}{1.548906in}}%
\pgfpathcurveto{\pgfqpoint{4.209772in}{1.548906in}}{\pgfqpoint{4.216614in}{1.551740in}}{\pgfqpoint{4.221657in}{1.556783in}}%
\pgfpathcurveto{\pgfqpoint{4.226701in}{1.561827in}}{\pgfqpoint{4.229535in}{1.568669in}}{\pgfqpoint{4.229535in}{1.575801in}}%
\pgfpathcurveto{\pgfqpoint{4.229535in}{1.582934in}}{\pgfqpoint{4.226701in}{1.589776in}}{\pgfqpoint{4.221657in}{1.594820in}}%
\pgfpathcurveto{\pgfqpoint{4.216614in}{1.599863in}}{\pgfqpoint{4.209772in}{1.602697in}}{\pgfqpoint{4.202639in}{1.602697in}}%
\pgfpathcurveto{\pgfqpoint{4.195506in}{1.602697in}}{\pgfqpoint{4.188665in}{1.599863in}}{\pgfqpoint{4.183621in}{1.594820in}}%
\pgfpathcurveto{\pgfqpoint{4.178577in}{1.589776in}}{\pgfqpoint{4.175743in}{1.582934in}}{\pgfqpoint{4.175743in}{1.575801in}}%
\pgfpathcurveto{\pgfqpoint{4.175743in}{1.568669in}}{\pgfqpoint{4.178577in}{1.561827in}}{\pgfqpoint{4.183621in}{1.556783in}}%
\pgfpathcurveto{\pgfqpoint{4.188665in}{1.551740in}}{\pgfqpoint{4.195506in}{1.548906in}}{\pgfqpoint{4.202639in}{1.548906in}}%
\pgfpathclose%
\pgfusepath{stroke,fill}%
\end{pgfscope}%
\begin{pgfscope}%
\pgfpathrectangle{\pgfqpoint{2.867647in}{0.500000in}}{\pgfqpoint{1.764706in}{1.700000in}}%
\pgfusepath{clip}%
\pgfsetbuttcap%
\pgfsetroundjoin%
\definecolor{currentfill}{rgb}{0.911533,0.252926,0.244703}%
\pgfsetfillcolor{currentfill}%
\pgfsetlinewidth{0.311001pt}%
\definecolor{currentstroke}{rgb}{1.000000,1.000000,1.000000}%
\pgfsetstrokecolor{currentstroke}%
\pgfsetdash{}{0pt}%
\pgfpathmoveto{\pgfqpoint{4.213998in}{0.851459in}}%
\pgfpathcurveto{\pgfqpoint{4.221131in}{0.851459in}}{\pgfqpoint{4.227972in}{0.854293in}}{\pgfqpoint{4.233016in}{0.859336in}}%
\pgfpathcurveto{\pgfqpoint{4.238060in}{0.864380in}}{\pgfqpoint{4.240893in}{0.871222in}}{\pgfqpoint{4.240893in}{0.878354in}}%
\pgfpathcurveto{\pgfqpoint{4.240893in}{0.885487in}}{\pgfqpoint{4.238060in}{0.892329in}}{\pgfqpoint{4.233016in}{0.897373in}}%
\pgfpathcurveto{\pgfqpoint{4.227972in}{0.902416in}}{\pgfqpoint{4.221131in}{0.905250in}}{\pgfqpoint{4.213998in}{0.905250in}}%
\pgfpathcurveto{\pgfqpoint{4.206865in}{0.905250in}}{\pgfqpoint{4.200023in}{0.902416in}}{\pgfqpoint{4.194980in}{0.897373in}}%
\pgfpathcurveto{\pgfqpoint{4.189936in}{0.892329in}}{\pgfqpoint{4.187102in}{0.885487in}}{\pgfqpoint{4.187102in}{0.878354in}}%
\pgfpathcurveto{\pgfqpoint{4.187102in}{0.871222in}}{\pgfqpoint{4.189936in}{0.864380in}}{\pgfqpoint{4.194980in}{0.859336in}}%
\pgfpathcurveto{\pgfqpoint{4.200023in}{0.854293in}}{\pgfqpoint{4.206865in}{0.851459in}}{\pgfqpoint{4.213998in}{0.851459in}}%
\pgfpathclose%
\pgfusepath{stroke,fill}%
\end{pgfscope}%
\begin{pgfscope}%
\pgfpathrectangle{\pgfqpoint{2.867647in}{0.500000in}}{\pgfqpoint{1.764706in}{1.700000in}}%
\pgfusepath{clip}%
\pgfsetbuttcap%
\pgfsetroundjoin%
\definecolor{currentfill}{rgb}{0.968931,0.798091,0.685123}%
\pgfsetfillcolor{currentfill}%
\pgfsetlinewidth{0.311001pt}%
\definecolor{currentstroke}{rgb}{1.000000,1.000000,1.000000}%
\pgfsetstrokecolor{currentstroke}%
\pgfsetdash{}{0pt}%
\pgfpathmoveto{\pgfqpoint{4.072678in}{0.940749in}}%
\pgfpathcurveto{\pgfqpoint{4.079811in}{0.940749in}}{\pgfqpoint{4.086653in}{0.943583in}}{\pgfqpoint{4.091696in}{0.948626in}}%
\pgfpathcurveto{\pgfqpoint{4.096740in}{0.953670in}}{\pgfqpoint{4.099574in}{0.960512in}}{\pgfqpoint{4.099574in}{0.967645in}}%
\pgfpathcurveto{\pgfqpoint{4.099574in}{0.974777in}}{\pgfqpoint{4.096740in}{0.981619in}}{\pgfqpoint{4.091696in}{0.986663in}}%
\pgfpathcurveto{\pgfqpoint{4.086653in}{0.991706in}}{\pgfqpoint{4.079811in}{0.994540in}}{\pgfqpoint{4.072678in}{0.994540in}}%
\pgfpathcurveto{\pgfqpoint{4.065545in}{0.994540in}}{\pgfqpoint{4.058704in}{0.991706in}}{\pgfqpoint{4.053660in}{0.986663in}}%
\pgfpathcurveto{\pgfqpoint{4.048616in}{0.981619in}}{\pgfqpoint{4.045782in}{0.974777in}}{\pgfqpoint{4.045782in}{0.967645in}}%
\pgfpathcurveto{\pgfqpoint{4.045782in}{0.960512in}}{\pgfqpoint{4.048616in}{0.953670in}}{\pgfqpoint{4.053660in}{0.948626in}}%
\pgfpathcurveto{\pgfqpoint{4.058704in}{0.943583in}}{\pgfqpoint{4.065545in}{0.940749in}}{\pgfqpoint{4.072678in}{0.940749in}}%
\pgfpathclose%
\pgfusepath{stroke,fill}%
\end{pgfscope}%
\begin{pgfscope}%
\pgfpathrectangle{\pgfqpoint{2.867647in}{0.500000in}}{\pgfqpoint{1.764706in}{1.700000in}}%
\pgfusepath{clip}%
\pgfsetbuttcap%
\pgfsetroundjoin%
\definecolor{currentfill}{rgb}{0.965592,0.726236,0.584384}%
\pgfsetfillcolor{currentfill}%
\pgfsetlinewidth{0.311001pt}%
\definecolor{currentstroke}{rgb}{1.000000,1.000000,1.000000}%
\pgfsetstrokecolor{currentstroke}%
\pgfsetdash{}{0pt}%
\pgfpathmoveto{\pgfqpoint{3.993773in}{1.679416in}}%
\pgfpathcurveto{\pgfqpoint{4.000906in}{1.679416in}}{\pgfqpoint{4.007748in}{1.682250in}}{\pgfqpoint{4.012792in}{1.687293in}}%
\pgfpathcurveto{\pgfqpoint{4.017835in}{1.692337in}}{\pgfqpoint{4.020669in}{1.699179in}}{\pgfqpoint{4.020669in}{1.706312in}}%
\pgfpathcurveto{\pgfqpoint{4.020669in}{1.713444in}}{\pgfqpoint{4.017835in}{1.720286in}}{\pgfqpoint{4.012792in}{1.725330in}}%
\pgfpathcurveto{\pgfqpoint{4.007748in}{1.730373in}}{\pgfqpoint{4.000906in}{1.733207in}}{\pgfqpoint{3.993773in}{1.733207in}}%
\pgfpathcurveto{\pgfqpoint{3.986641in}{1.733207in}}{\pgfqpoint{3.979799in}{1.730373in}}{\pgfqpoint{3.974755in}{1.725330in}}%
\pgfpathcurveto{\pgfqpoint{3.969712in}{1.720286in}}{\pgfqpoint{3.966878in}{1.713444in}}{\pgfqpoint{3.966878in}{1.706312in}}%
\pgfpathcurveto{\pgfqpoint{3.966878in}{1.699179in}}{\pgfqpoint{3.969712in}{1.692337in}}{\pgfqpoint{3.974755in}{1.687293in}}%
\pgfpathcurveto{\pgfqpoint{3.979799in}{1.682250in}}{\pgfqpoint{3.986641in}{1.679416in}}{\pgfqpoint{3.993773in}{1.679416in}}%
\pgfpathclose%
\pgfusepath{stroke,fill}%
\end{pgfscope}%
\begin{pgfscope}%
\pgfpathrectangle{\pgfqpoint{2.867647in}{0.500000in}}{\pgfqpoint{1.764706in}{1.700000in}}%
\pgfusepath{clip}%
\pgfsetbuttcap%
\pgfsetroundjoin%
\definecolor{currentfill}{rgb}{0.955103,0.477872,0.328626}%
\pgfsetfillcolor{currentfill}%
\pgfsetlinewidth{0.311001pt}%
\definecolor{currentstroke}{rgb}{1.000000,1.000000,1.000000}%
\pgfsetstrokecolor{currentstroke}%
\pgfsetdash{}{0pt}%
\pgfpathmoveto{\pgfqpoint{4.325058in}{1.112073in}}%
\pgfpathcurveto{\pgfqpoint{4.332191in}{1.112073in}}{\pgfqpoint{4.339033in}{1.114907in}}{\pgfqpoint{4.344076in}{1.119950in}}%
\pgfpathcurveto{\pgfqpoint{4.349120in}{1.124994in}}{\pgfqpoint{4.351954in}{1.131836in}}{\pgfqpoint{4.351954in}{1.138969in}}%
\pgfpathcurveto{\pgfqpoint{4.351954in}{1.146101in}}{\pgfqpoint{4.349120in}{1.152943in}}{\pgfqpoint{4.344076in}{1.157987in}}%
\pgfpathcurveto{\pgfqpoint{4.339033in}{1.163030in}}{\pgfqpoint{4.332191in}{1.165864in}}{\pgfqpoint{4.325058in}{1.165864in}}%
\pgfpathcurveto{\pgfqpoint{4.317925in}{1.165864in}}{\pgfqpoint{4.311084in}{1.163030in}}{\pgfqpoint{4.306040in}{1.157987in}}%
\pgfpathcurveto{\pgfqpoint{4.300996in}{1.152943in}}{\pgfqpoint{4.298162in}{1.146101in}}{\pgfqpoint{4.298162in}{1.138969in}}%
\pgfpathcurveto{\pgfqpoint{4.298162in}{1.131836in}}{\pgfqpoint{4.300996in}{1.124994in}}{\pgfqpoint{4.306040in}{1.119950in}}%
\pgfpathcurveto{\pgfqpoint{4.311084in}{1.114907in}}{\pgfqpoint{4.317925in}{1.112073in}}{\pgfqpoint{4.325058in}{1.112073in}}%
\pgfpathclose%
\pgfusepath{stroke,fill}%
\end{pgfscope}%
\begin{pgfscope}%
\pgfpathrectangle{\pgfqpoint{2.867647in}{0.500000in}}{\pgfqpoint{1.764706in}{1.700000in}}%
\pgfusepath{clip}%
\pgfsetbuttcap%
\pgfsetroundjoin%
\definecolor{currentfill}{rgb}{0.919781,0.275262,0.242460}%
\pgfsetfillcolor{currentfill}%
\pgfsetlinewidth{0.311001pt}%
\definecolor{currentstroke}{rgb}{1.000000,1.000000,1.000000}%
\pgfsetstrokecolor{currentstroke}%
\pgfsetdash{}{0pt}%
\pgfpathmoveto{\pgfqpoint{4.203433in}{0.846512in}}%
\pgfpathcurveto{\pgfqpoint{4.210566in}{0.846512in}}{\pgfqpoint{4.217408in}{0.849346in}}{\pgfqpoint{4.222452in}{0.854389in}}%
\pgfpathcurveto{\pgfqpoint{4.227495in}{0.859433in}}{\pgfqpoint{4.230329in}{0.866275in}}{\pgfqpoint{4.230329in}{0.873408in}}%
\pgfpathcurveto{\pgfqpoint{4.230329in}{0.880540in}}{\pgfqpoint{4.227495in}{0.887382in}}{\pgfqpoint{4.222452in}{0.892426in}}%
\pgfpathcurveto{\pgfqpoint{4.217408in}{0.897469in}}{\pgfqpoint{4.210566in}{0.900303in}}{\pgfqpoint{4.203433in}{0.900303in}}%
\pgfpathcurveto{\pgfqpoint{4.196301in}{0.900303in}}{\pgfqpoint{4.189459in}{0.897469in}}{\pgfqpoint{4.184415in}{0.892426in}}%
\pgfpathcurveto{\pgfqpoint{4.179372in}{0.887382in}}{\pgfqpoint{4.176538in}{0.880540in}}{\pgfqpoint{4.176538in}{0.873408in}}%
\pgfpathcurveto{\pgfqpoint{4.176538in}{0.866275in}}{\pgfqpoint{4.179372in}{0.859433in}}{\pgfqpoint{4.184415in}{0.854389in}}%
\pgfpathcurveto{\pgfqpoint{4.189459in}{0.849346in}}{\pgfqpoint{4.196301in}{0.846512in}}{\pgfqpoint{4.203433in}{0.846512in}}%
\pgfpathclose%
\pgfusepath{stroke,fill}%
\end{pgfscope}%
\begin{pgfscope}%
\pgfpathrectangle{\pgfqpoint{2.867647in}{0.500000in}}{\pgfqpoint{1.764706in}{1.700000in}}%
\pgfusepath{clip}%
\pgfsetbuttcap%
\pgfsetroundjoin%
\definecolor{currentfill}{rgb}{0.975018,0.868213,0.788710}%
\pgfsetfillcolor{currentfill}%
\pgfsetlinewidth{0.311001pt}%
\definecolor{currentstroke}{rgb}{1.000000,1.000000,1.000000}%
\pgfsetstrokecolor{currentstroke}%
\pgfsetdash{}{0pt}%
\pgfpathmoveto{\pgfqpoint{4.105218in}{1.147664in}}%
\pgfpathcurveto{\pgfqpoint{4.112350in}{1.147664in}}{\pgfqpoint{4.119192in}{1.150497in}}{\pgfqpoint{4.124236in}{1.155541in}}%
\pgfpathcurveto{\pgfqpoint{4.129279in}{1.160585in}}{\pgfqpoint{4.132113in}{1.167426in}}{\pgfqpoint{4.132113in}{1.174559in}}%
\pgfpathcurveto{\pgfqpoint{4.132113in}{1.181692in}}{\pgfqpoint{4.129279in}{1.188534in}}{\pgfqpoint{4.124236in}{1.193577in}}%
\pgfpathcurveto{\pgfqpoint{4.119192in}{1.198621in}}{\pgfqpoint{4.112350in}{1.201455in}}{\pgfqpoint{4.105218in}{1.201455in}}%
\pgfpathcurveto{\pgfqpoint{4.098085in}{1.201455in}}{\pgfqpoint{4.091243in}{1.198621in}}{\pgfqpoint{4.086199in}{1.193577in}}%
\pgfpathcurveto{\pgfqpoint{4.081156in}{1.188534in}}{\pgfqpoint{4.078322in}{1.181692in}}{\pgfqpoint{4.078322in}{1.174559in}}%
\pgfpathcurveto{\pgfqpoint{4.078322in}{1.167426in}}{\pgfqpoint{4.081156in}{1.160585in}}{\pgfqpoint{4.086199in}{1.155541in}}%
\pgfpathcurveto{\pgfqpoint{4.091243in}{1.150497in}}{\pgfqpoint{4.098085in}{1.147664in}}{\pgfqpoint{4.105218in}{1.147664in}}%
\pgfpathclose%
\pgfusepath{stroke,fill}%
\end{pgfscope}%
\begin{pgfscope}%
\pgfpathrectangle{\pgfqpoint{2.867647in}{0.500000in}}{\pgfqpoint{1.764706in}{1.700000in}}%
\pgfusepath{clip}%
\pgfsetbuttcap%
\pgfsetroundjoin%
\definecolor{currentfill}{rgb}{0.975018,0.868213,0.788710}%
\pgfsetfillcolor{currentfill}%
\pgfsetlinewidth{0.311001pt}%
\definecolor{currentstroke}{rgb}{1.000000,1.000000,1.000000}%
\pgfsetstrokecolor{currentstroke}%
\pgfsetdash{}{0pt}%
\pgfpathmoveto{\pgfqpoint{4.250987in}{1.296803in}}%
\pgfpathcurveto{\pgfqpoint{4.258120in}{1.296803in}}{\pgfqpoint{4.264962in}{1.299637in}}{\pgfqpoint{4.270005in}{1.304680in}}%
\pgfpathcurveto{\pgfqpoint{4.275049in}{1.309724in}}{\pgfqpoint{4.277883in}{1.316566in}}{\pgfqpoint{4.277883in}{1.323699in}}%
\pgfpathcurveto{\pgfqpoint{4.277883in}{1.330831in}}{\pgfqpoint{4.275049in}{1.337673in}}{\pgfqpoint{4.270005in}{1.342717in}}%
\pgfpathcurveto{\pgfqpoint{4.264962in}{1.347760in}}{\pgfqpoint{4.258120in}{1.350594in}}{\pgfqpoint{4.250987in}{1.350594in}}%
\pgfpathcurveto{\pgfqpoint{4.243854in}{1.350594in}}{\pgfqpoint{4.237013in}{1.347760in}}{\pgfqpoint{4.231969in}{1.342717in}}%
\pgfpathcurveto{\pgfqpoint{4.226925in}{1.337673in}}{\pgfqpoint{4.224091in}{1.330831in}}{\pgfqpoint{4.224091in}{1.323699in}}%
\pgfpathcurveto{\pgfqpoint{4.224091in}{1.316566in}}{\pgfqpoint{4.226925in}{1.309724in}}{\pgfqpoint{4.231969in}{1.304680in}}%
\pgfpathcurveto{\pgfqpoint{4.237013in}{1.299637in}}{\pgfqpoint{4.243854in}{1.296803in}}{\pgfqpoint{4.250987in}{1.296803in}}%
\pgfpathclose%
\pgfusepath{stroke,fill}%
\end{pgfscope}%
\begin{pgfscope}%
\pgfpathrectangle{\pgfqpoint{2.867647in}{0.500000in}}{\pgfqpoint{1.764706in}{1.700000in}}%
\pgfusepath{clip}%
\pgfsetbuttcap%
\pgfsetroundjoin%
\definecolor{currentfill}{rgb}{0.972201,0.839051,0.745789}%
\pgfsetfillcolor{currentfill}%
\pgfsetlinewidth{0.311001pt}%
\definecolor{currentstroke}{rgb}{1.000000,1.000000,1.000000}%
\pgfsetstrokecolor{currentstroke}%
\pgfsetdash{}{0pt}%
\pgfpathmoveto{\pgfqpoint{4.113258in}{1.390884in}}%
\pgfpathcurveto{\pgfqpoint{4.120391in}{1.390884in}}{\pgfqpoint{4.127232in}{1.393718in}}{\pgfqpoint{4.132276in}{1.398762in}}%
\pgfpathcurveto{\pgfqpoint{4.137320in}{1.403805in}}{\pgfqpoint{4.140153in}{1.410647in}}{\pgfqpoint{4.140153in}{1.417780in}}%
\pgfpathcurveto{\pgfqpoint{4.140153in}{1.424913in}}{\pgfqpoint{4.137320in}{1.431754in}}{\pgfqpoint{4.132276in}{1.436798in}}%
\pgfpathcurveto{\pgfqpoint{4.127232in}{1.441842in}}{\pgfqpoint{4.120391in}{1.444676in}}{\pgfqpoint{4.113258in}{1.444676in}}%
\pgfpathcurveto{\pgfqpoint{4.106125in}{1.444676in}}{\pgfqpoint{4.099283in}{1.441842in}}{\pgfqpoint{4.094240in}{1.436798in}}%
\pgfpathcurveto{\pgfqpoint{4.089196in}{1.431754in}}{\pgfqpoint{4.086362in}{1.424913in}}{\pgfqpoint{4.086362in}{1.417780in}}%
\pgfpathcurveto{\pgfqpoint{4.086362in}{1.410647in}}{\pgfqpoint{4.089196in}{1.403805in}}{\pgfqpoint{4.094240in}{1.398762in}}%
\pgfpathcurveto{\pgfqpoint{4.099283in}{1.393718in}}{\pgfqpoint{4.106125in}{1.390884in}}{\pgfqpoint{4.113258in}{1.390884in}}%
\pgfpathclose%
\pgfusepath{stroke,fill}%
\end{pgfscope}%
\begin{pgfscope}%
\pgfpathrectangle{\pgfqpoint{2.867647in}{0.500000in}}{\pgfqpoint{1.764706in}{1.700000in}}%
\pgfusepath{clip}%
\pgfsetbuttcap%
\pgfsetroundjoin%
\definecolor{currentfill}{rgb}{0.979891,0.908948,0.848279}%
\pgfsetfillcolor{currentfill}%
\pgfsetlinewidth{0.311001pt}%
\definecolor{currentstroke}{rgb}{1.000000,1.000000,1.000000}%
\pgfsetstrokecolor{currentstroke}%
\pgfsetdash{}{0pt}%
\pgfpathmoveto{\pgfqpoint{4.204016in}{1.184738in}}%
\pgfpathcurveto{\pgfqpoint{4.211149in}{1.184738in}}{\pgfqpoint{4.217990in}{1.187572in}}{\pgfqpoint{4.223034in}{1.192616in}}%
\pgfpathcurveto{\pgfqpoint{4.228078in}{1.197659in}}{\pgfqpoint{4.230911in}{1.204501in}}{\pgfqpoint{4.230911in}{1.211634in}}%
\pgfpathcurveto{\pgfqpoint{4.230911in}{1.218767in}}{\pgfqpoint{4.228078in}{1.225608in}}{\pgfqpoint{4.223034in}{1.230652in}}%
\pgfpathcurveto{\pgfqpoint{4.217990in}{1.235696in}}{\pgfqpoint{4.211149in}{1.238530in}}{\pgfqpoint{4.204016in}{1.238530in}}%
\pgfpathcurveto{\pgfqpoint{4.196883in}{1.238530in}}{\pgfqpoint{4.190041in}{1.235696in}}{\pgfqpoint{4.184998in}{1.230652in}}%
\pgfpathcurveto{\pgfqpoint{4.179954in}{1.225608in}}{\pgfqpoint{4.177120in}{1.218767in}}{\pgfqpoint{4.177120in}{1.211634in}}%
\pgfpathcurveto{\pgfqpoint{4.177120in}{1.204501in}}{\pgfqpoint{4.179954in}{1.197659in}}{\pgfqpoint{4.184998in}{1.192616in}}%
\pgfpathcurveto{\pgfqpoint{4.190041in}{1.187572in}}{\pgfqpoint{4.196883in}{1.184738in}}{\pgfqpoint{4.204016in}{1.184738in}}%
\pgfpathclose%
\pgfusepath{stroke,fill}%
\end{pgfscope}%
\begin{pgfscope}%
\pgfpathrectangle{\pgfqpoint{2.867647in}{0.500000in}}{\pgfqpoint{1.764706in}{1.700000in}}%
\pgfusepath{clip}%
\pgfsetbuttcap%
\pgfsetroundjoin%
\definecolor{currentfill}{rgb}{0.981377,0.920617,0.865369}%
\pgfsetfillcolor{currentfill}%
\pgfsetlinewidth{0.311001pt}%
\definecolor{currentstroke}{rgb}{1.000000,1.000000,1.000000}%
\pgfsetstrokecolor{currentstroke}%
\pgfsetdash{}{0pt}%
\pgfpathmoveto{\pgfqpoint{4.182328in}{1.246946in}}%
\pgfpathcurveto{\pgfqpoint{4.189461in}{1.246946in}}{\pgfqpoint{4.196303in}{1.249780in}}{\pgfqpoint{4.201346in}{1.254824in}}%
\pgfpathcurveto{\pgfqpoint{4.206390in}{1.259867in}}{\pgfqpoint{4.209224in}{1.266709in}}{\pgfqpoint{4.209224in}{1.273842in}}%
\pgfpathcurveto{\pgfqpoint{4.209224in}{1.280975in}}{\pgfqpoint{4.206390in}{1.287816in}}{\pgfqpoint{4.201346in}{1.292860in}}%
\pgfpathcurveto{\pgfqpoint{4.196303in}{1.297904in}}{\pgfqpoint{4.189461in}{1.300737in}}{\pgfqpoint{4.182328in}{1.300737in}}%
\pgfpathcurveto{\pgfqpoint{4.175195in}{1.300737in}}{\pgfqpoint{4.168354in}{1.297904in}}{\pgfqpoint{4.163310in}{1.292860in}}%
\pgfpathcurveto{\pgfqpoint{4.158266in}{1.287816in}}{\pgfqpoint{4.155432in}{1.280975in}}{\pgfqpoint{4.155432in}{1.273842in}}%
\pgfpathcurveto{\pgfqpoint{4.155432in}{1.266709in}}{\pgfqpoint{4.158266in}{1.259867in}}{\pgfqpoint{4.163310in}{1.254824in}}%
\pgfpathcurveto{\pgfqpoint{4.168354in}{1.249780in}}{\pgfqpoint{4.175195in}{1.246946in}}{\pgfqpoint{4.182328in}{1.246946in}}%
\pgfpathclose%
\pgfusepath{stroke,fill}%
\end{pgfscope}%
\begin{pgfscope}%
\pgfpathrectangle{\pgfqpoint{2.867647in}{0.500000in}}{\pgfqpoint{1.764706in}{1.700000in}}%
\pgfusepath{clip}%
\pgfsetbuttcap%
\pgfsetroundjoin%
\definecolor{currentfill}{rgb}{0.965592,0.726236,0.584384}%
\pgfsetfillcolor{currentfill}%
\pgfsetlinewidth{0.311001pt}%
\definecolor{currentstroke}{rgb}{1.000000,1.000000,1.000000}%
\pgfsetstrokecolor{currentstroke}%
\pgfsetdash{}{0pt}%
\pgfpathmoveto{\pgfqpoint{4.283421in}{1.477842in}}%
\pgfpathcurveto{\pgfqpoint{4.290554in}{1.477842in}}{\pgfqpoint{4.297395in}{1.480676in}}{\pgfqpoint{4.302439in}{1.485719in}}%
\pgfpathcurveto{\pgfqpoint{4.307483in}{1.490763in}}{\pgfqpoint{4.310317in}{1.497605in}}{\pgfqpoint{4.310317in}{1.504737in}}%
\pgfpathcurveto{\pgfqpoint{4.310317in}{1.511870in}}{\pgfqpoint{4.307483in}{1.518712in}}{\pgfqpoint{4.302439in}{1.523756in}}%
\pgfpathcurveto{\pgfqpoint{4.297395in}{1.528799in}}{\pgfqpoint{4.290554in}{1.531633in}}{\pgfqpoint{4.283421in}{1.531633in}}%
\pgfpathcurveto{\pgfqpoint{4.276288in}{1.531633in}}{\pgfqpoint{4.269446in}{1.528799in}}{\pgfqpoint{4.264403in}{1.523756in}}%
\pgfpathcurveto{\pgfqpoint{4.259359in}{1.518712in}}{\pgfqpoint{4.256525in}{1.511870in}}{\pgfqpoint{4.256525in}{1.504737in}}%
\pgfpathcurveto{\pgfqpoint{4.256525in}{1.497605in}}{\pgfqpoint{4.259359in}{1.490763in}}{\pgfqpoint{4.264403in}{1.485719in}}%
\pgfpathcurveto{\pgfqpoint{4.269446in}{1.480676in}}{\pgfqpoint{4.276288in}{1.477842in}}{\pgfqpoint{4.283421in}{1.477842in}}%
\pgfpathclose%
\pgfusepath{stroke,fill}%
\end{pgfscope}%
\begin{pgfscope}%
\pgfpathrectangle{\pgfqpoint{2.867647in}{0.500000in}}{\pgfqpoint{1.764706in}{1.700000in}}%
\pgfusepath{clip}%
\pgfsetbuttcap%
\pgfsetroundjoin%
\definecolor{currentfill}{rgb}{0.967092,0.768560,0.642079}%
\pgfsetfillcolor{currentfill}%
\pgfsetlinewidth{0.311001pt}%
\definecolor{currentstroke}{rgb}{1.000000,1.000000,1.000000}%
\pgfsetstrokecolor{currentstroke}%
\pgfsetdash{}{0pt}%
\pgfpathmoveto{\pgfqpoint{4.118584in}{1.717421in}}%
\pgfpathcurveto{\pgfqpoint{4.125717in}{1.717421in}}{\pgfqpoint{4.132558in}{1.720254in}}{\pgfqpoint{4.137602in}{1.725298in}}%
\pgfpathcurveto{\pgfqpoint{4.142646in}{1.730342in}}{\pgfqpoint{4.145479in}{1.737183in}}{\pgfqpoint{4.145479in}{1.744316in}}%
\pgfpathcurveto{\pgfqpoint{4.145479in}{1.751449in}}{\pgfqpoint{4.142646in}{1.758291in}}{\pgfqpoint{4.137602in}{1.763334in}}%
\pgfpathcurveto{\pgfqpoint{4.132558in}{1.768378in}}{\pgfqpoint{4.125717in}{1.771212in}}{\pgfqpoint{4.118584in}{1.771212in}}%
\pgfpathcurveto{\pgfqpoint{4.111451in}{1.771212in}}{\pgfqpoint{4.104609in}{1.768378in}}{\pgfqpoint{4.099566in}{1.763334in}}%
\pgfpathcurveto{\pgfqpoint{4.094522in}{1.758291in}}{\pgfqpoint{4.091688in}{1.751449in}}{\pgfqpoint{4.091688in}{1.744316in}}%
\pgfpathcurveto{\pgfqpoint{4.091688in}{1.737183in}}{\pgfqpoint{4.094522in}{1.730342in}}{\pgfqpoint{4.099566in}{1.725298in}}%
\pgfpathcurveto{\pgfqpoint{4.104609in}{1.720254in}}{\pgfqpoint{4.111451in}{1.717421in}}{\pgfqpoint{4.118584in}{1.717421in}}%
\pgfpathclose%
\pgfusepath{stroke,fill}%
\end{pgfscope}%
\begin{pgfscope}%
\pgfpathrectangle{\pgfqpoint{2.867647in}{0.500000in}}{\pgfqpoint{1.764706in}{1.700000in}}%
\pgfusepath{clip}%
\pgfsetbuttcap%
\pgfsetroundjoin%
\definecolor{currentfill}{rgb}{0.979891,0.908948,0.848279}%
\pgfsetfillcolor{currentfill}%
\pgfsetlinewidth{0.311001pt}%
\definecolor{currentstroke}{rgb}{1.000000,1.000000,1.000000}%
\pgfsetstrokecolor{currentstroke}%
\pgfsetdash{}{0pt}%
\pgfpathmoveto{\pgfqpoint{4.207115in}{1.397165in}}%
\pgfpathcurveto{\pgfqpoint{4.214247in}{1.397165in}}{\pgfqpoint{4.221089in}{1.399999in}}{\pgfqpoint{4.226133in}{1.405042in}}%
\pgfpathcurveto{\pgfqpoint{4.231176in}{1.410086in}}{\pgfqpoint{4.234010in}{1.416928in}}{\pgfqpoint{4.234010in}{1.424060in}}%
\pgfpathcurveto{\pgfqpoint{4.234010in}{1.431193in}}{\pgfqpoint{4.231176in}{1.438035in}}{\pgfqpoint{4.226133in}{1.443079in}}%
\pgfpathcurveto{\pgfqpoint{4.221089in}{1.448122in}}{\pgfqpoint{4.214247in}{1.450956in}}{\pgfqpoint{4.207115in}{1.450956in}}%
\pgfpathcurveto{\pgfqpoint{4.199982in}{1.450956in}}{\pgfqpoint{4.193140in}{1.448122in}}{\pgfqpoint{4.188096in}{1.443079in}}%
\pgfpathcurveto{\pgfqpoint{4.183053in}{1.438035in}}{\pgfqpoint{4.180219in}{1.431193in}}{\pgfqpoint{4.180219in}{1.424060in}}%
\pgfpathcurveto{\pgfqpoint{4.180219in}{1.416928in}}{\pgfqpoint{4.183053in}{1.410086in}}{\pgfqpoint{4.188096in}{1.405042in}}%
\pgfpathcurveto{\pgfqpoint{4.193140in}{1.399999in}}{\pgfqpoint{4.199982in}{1.397165in}}{\pgfqpoint{4.207115in}{1.397165in}}%
\pgfpathclose%
\pgfusepath{stroke,fill}%
\end{pgfscope}%
\begin{pgfscope}%
\pgfpathrectangle{\pgfqpoint{2.867647in}{0.500000in}}{\pgfqpoint{1.764706in}{1.700000in}}%
\pgfusepath{clip}%
\pgfsetbuttcap%
\pgfsetroundjoin%
\definecolor{currentfill}{rgb}{0.976287,0.879862,0.805788}%
\pgfsetfillcolor{currentfill}%
\pgfsetlinewidth{0.311001pt}%
\definecolor{currentstroke}{rgb}{1.000000,1.000000,1.000000}%
\pgfsetstrokecolor{currentstroke}%
\pgfsetdash{}{0pt}%
\pgfpathmoveto{\pgfqpoint{4.133385in}{1.621427in}}%
\pgfpathcurveto{\pgfqpoint{4.140518in}{1.621427in}}{\pgfqpoint{4.147360in}{1.624261in}}{\pgfqpoint{4.152403in}{1.629305in}}%
\pgfpathcurveto{\pgfqpoint{4.157447in}{1.634348in}}{\pgfqpoint{4.160281in}{1.641190in}}{\pgfqpoint{4.160281in}{1.648323in}}%
\pgfpathcurveto{\pgfqpoint{4.160281in}{1.655456in}}{\pgfqpoint{4.157447in}{1.662297in}}{\pgfqpoint{4.152403in}{1.667341in}}%
\pgfpathcurveto{\pgfqpoint{4.147360in}{1.672385in}}{\pgfqpoint{4.140518in}{1.675219in}}{\pgfqpoint{4.133385in}{1.675219in}}%
\pgfpathcurveto{\pgfqpoint{4.126252in}{1.675219in}}{\pgfqpoint{4.119411in}{1.672385in}}{\pgfqpoint{4.114367in}{1.667341in}}%
\pgfpathcurveto{\pgfqpoint{4.109323in}{1.662297in}}{\pgfqpoint{4.106489in}{1.655456in}}{\pgfqpoint{4.106489in}{1.648323in}}%
\pgfpathcurveto{\pgfqpoint{4.106489in}{1.641190in}}{\pgfqpoint{4.109323in}{1.634348in}}{\pgfqpoint{4.114367in}{1.629305in}}%
\pgfpathcurveto{\pgfqpoint{4.119411in}{1.624261in}}{\pgfqpoint{4.126252in}{1.621427in}}{\pgfqpoint{4.133385in}{1.621427in}}%
\pgfpathclose%
\pgfusepath{stroke,fill}%
\end{pgfscope}%
\begin{pgfscope}%
\pgfpathrectangle{\pgfqpoint{2.867647in}{0.500000in}}{\pgfqpoint{1.764706in}{1.700000in}}%
\pgfusepath{clip}%
\pgfsetbuttcap%
\pgfsetroundjoin%
\definecolor{currentfill}{rgb}{0.963728,0.638439,0.479050}%
\pgfsetfillcolor{currentfill}%
\pgfsetlinewidth{0.311001pt}%
\definecolor{currentstroke}{rgb}{1.000000,1.000000,1.000000}%
\pgfsetstrokecolor{currentstroke}%
\pgfsetdash{}{0pt}%
\pgfpathmoveto{\pgfqpoint{4.318834in}{1.405127in}}%
\pgfpathcurveto{\pgfqpoint{4.325966in}{1.405127in}}{\pgfqpoint{4.332808in}{1.407961in}}{\pgfqpoint{4.337852in}{1.413004in}}%
\pgfpathcurveto{\pgfqpoint{4.342895in}{1.418048in}}{\pgfqpoint{4.345729in}{1.424890in}}{\pgfqpoint{4.345729in}{1.432022in}}%
\pgfpathcurveto{\pgfqpoint{4.345729in}{1.439155in}}{\pgfqpoint{4.342895in}{1.445997in}}{\pgfqpoint{4.337852in}{1.451041in}}%
\pgfpathcurveto{\pgfqpoint{4.332808in}{1.456084in}}{\pgfqpoint{4.325966in}{1.458918in}}{\pgfqpoint{4.318834in}{1.458918in}}%
\pgfpathcurveto{\pgfqpoint{4.311701in}{1.458918in}}{\pgfqpoint{4.304859in}{1.456084in}}{\pgfqpoint{4.299815in}{1.451041in}}%
\pgfpathcurveto{\pgfqpoint{4.294772in}{1.445997in}}{\pgfqpoint{4.291938in}{1.439155in}}{\pgfqpoint{4.291938in}{1.432022in}}%
\pgfpathcurveto{\pgfqpoint{4.291938in}{1.424890in}}{\pgfqpoint{4.294772in}{1.418048in}}{\pgfqpoint{4.299815in}{1.413004in}}%
\pgfpathcurveto{\pgfqpoint{4.304859in}{1.407961in}}{\pgfqpoint{4.311701in}{1.405127in}}{\pgfqpoint{4.318834in}{1.405127in}}%
\pgfpathclose%
\pgfusepath{stroke,fill}%
\end{pgfscope}%
\begin{pgfscope}%
\pgfpathrectangle{\pgfqpoint{2.867647in}{0.500000in}}{\pgfqpoint{1.764706in}{1.700000in}}%
\pgfusepath{clip}%
\pgfsetbuttcap%
\pgfsetroundjoin%
\definecolor{currentfill}{rgb}{0.966120,0.744512,0.608720}%
\pgfsetfillcolor{currentfill}%
\pgfsetlinewidth{0.311001pt}%
\definecolor{currentstroke}{rgb}{1.000000,1.000000,1.000000}%
\pgfsetstrokecolor{currentstroke}%
\pgfsetdash{}{0pt}%
\pgfpathmoveto{\pgfqpoint{4.078773in}{1.249671in}}%
\pgfpathcurveto{\pgfqpoint{4.085906in}{1.249671in}}{\pgfqpoint{4.092748in}{1.252505in}}{\pgfqpoint{4.097791in}{1.257548in}}%
\pgfpathcurveto{\pgfqpoint{4.102835in}{1.262592in}}{\pgfqpoint{4.105669in}{1.269434in}}{\pgfqpoint{4.105669in}{1.276566in}}%
\pgfpathcurveto{\pgfqpoint{4.105669in}{1.283699in}}{\pgfqpoint{4.102835in}{1.290541in}}{\pgfqpoint{4.097791in}{1.295585in}}%
\pgfpathcurveto{\pgfqpoint{4.092748in}{1.300628in}}{\pgfqpoint{4.085906in}{1.303462in}}{\pgfqpoint{4.078773in}{1.303462in}}%
\pgfpathcurveto{\pgfqpoint{4.071640in}{1.303462in}}{\pgfqpoint{4.064799in}{1.300628in}}{\pgfqpoint{4.059755in}{1.295585in}}%
\pgfpathcurveto{\pgfqpoint{4.054711in}{1.290541in}}{\pgfqpoint{4.051877in}{1.283699in}}{\pgfqpoint{4.051877in}{1.276566in}}%
\pgfpathcurveto{\pgfqpoint{4.051877in}{1.269434in}}{\pgfqpoint{4.054711in}{1.262592in}}{\pgfqpoint{4.059755in}{1.257548in}}%
\pgfpathcurveto{\pgfqpoint{4.064799in}{1.252505in}}{\pgfqpoint{4.071640in}{1.249671in}}{\pgfqpoint{4.078773in}{1.249671in}}%
\pgfpathclose%
\pgfusepath{stroke,fill}%
\end{pgfscope}%
\begin{pgfscope}%
\pgfpathrectangle{\pgfqpoint{2.867647in}{0.500000in}}{\pgfqpoint{1.764706in}{1.700000in}}%
\pgfusepath{clip}%
\pgfsetbuttcap%
\pgfsetroundjoin%
\definecolor{currentfill}{rgb}{0.973832,0.856556,0.771584}%
\pgfsetfillcolor{currentfill}%
\pgfsetlinewidth{0.311001pt}%
\definecolor{currentstroke}{rgb}{1.000000,1.000000,1.000000}%
\pgfsetstrokecolor{currentstroke}%
\pgfsetdash{}{0pt}%
\pgfpathmoveto{\pgfqpoint{4.117227in}{1.655301in}}%
\pgfpathcurveto{\pgfqpoint{4.124360in}{1.655301in}}{\pgfqpoint{4.131202in}{1.658135in}}{\pgfqpoint{4.136245in}{1.663179in}}%
\pgfpathcurveto{\pgfqpoint{4.141289in}{1.668223in}}{\pgfqpoint{4.144123in}{1.675064in}}{\pgfqpoint{4.144123in}{1.682197in}}%
\pgfpathcurveto{\pgfqpoint{4.144123in}{1.689330in}}{\pgfqpoint{4.141289in}{1.696172in}}{\pgfqpoint{4.136245in}{1.701215in}}%
\pgfpathcurveto{\pgfqpoint{4.131202in}{1.706259in}}{\pgfqpoint{4.124360in}{1.709093in}}{\pgfqpoint{4.117227in}{1.709093in}}%
\pgfpathcurveto{\pgfqpoint{4.110094in}{1.709093in}}{\pgfqpoint{4.103253in}{1.706259in}}{\pgfqpoint{4.098209in}{1.701215in}}%
\pgfpathcurveto{\pgfqpoint{4.093165in}{1.696172in}}{\pgfqpoint{4.090331in}{1.689330in}}{\pgfqpoint{4.090331in}{1.682197in}}%
\pgfpathcurveto{\pgfqpoint{4.090331in}{1.675064in}}{\pgfqpoint{4.093165in}{1.668223in}}{\pgfqpoint{4.098209in}{1.663179in}}%
\pgfpathcurveto{\pgfqpoint{4.103253in}{1.658135in}}{\pgfqpoint{4.110094in}{1.655301in}}{\pgfqpoint{4.117227in}{1.655301in}}%
\pgfpathclose%
\pgfusepath{stroke,fill}%
\end{pgfscope}%
\begin{pgfscope}%
\pgfpathrectangle{\pgfqpoint{2.867647in}{0.500000in}}{\pgfqpoint{1.764706in}{1.700000in}}%
\pgfusepath{clip}%
\pgfsetbuttcap%
\pgfsetroundjoin%
\definecolor{currentfill}{rgb}{0.981377,0.920617,0.865369}%
\pgfsetfillcolor{currentfill}%
\pgfsetlinewidth{0.311001pt}%
\definecolor{currentstroke}{rgb}{1.000000,1.000000,1.000000}%
\pgfsetstrokecolor{currentstroke}%
\pgfsetdash{}{0pt}%
\pgfpathmoveto{\pgfqpoint{4.181411in}{1.288115in}}%
\pgfpathcurveto{\pgfqpoint{4.188544in}{1.288115in}}{\pgfqpoint{4.195385in}{1.290949in}}{\pgfqpoint{4.200429in}{1.295993in}}%
\pgfpathcurveto{\pgfqpoint{4.205473in}{1.301037in}}{\pgfqpoint{4.208307in}{1.307878in}}{\pgfqpoint{4.208307in}{1.315011in}}%
\pgfpathcurveto{\pgfqpoint{4.208307in}{1.322144in}}{\pgfqpoint{4.205473in}{1.328985in}}{\pgfqpoint{4.200429in}{1.334029in}}%
\pgfpathcurveto{\pgfqpoint{4.195385in}{1.339073in}}{\pgfqpoint{4.188544in}{1.341907in}}{\pgfqpoint{4.181411in}{1.341907in}}%
\pgfpathcurveto{\pgfqpoint{4.174278in}{1.341907in}}{\pgfqpoint{4.167436in}{1.339073in}}{\pgfqpoint{4.162393in}{1.334029in}}%
\pgfpathcurveto{\pgfqpoint{4.157349in}{1.328985in}}{\pgfqpoint{4.154515in}{1.322144in}}{\pgfqpoint{4.154515in}{1.315011in}}%
\pgfpathcurveto{\pgfqpoint{4.154515in}{1.307878in}}{\pgfqpoint{4.157349in}{1.301037in}}{\pgfqpoint{4.162393in}{1.295993in}}%
\pgfpathcurveto{\pgfqpoint{4.167436in}{1.290949in}}{\pgfqpoint{4.174278in}{1.288115in}}{\pgfqpoint{4.181411in}{1.288115in}}%
\pgfpathclose%
\pgfusepath{stroke,fill}%
\end{pgfscope}%
\begin{pgfscope}%
\pgfpathrectangle{\pgfqpoint{2.867647in}{0.500000in}}{\pgfqpoint{1.764706in}{1.700000in}}%
\pgfusepath{clip}%
\pgfsetbuttcap%
\pgfsetroundjoin%
\definecolor{currentfill}{rgb}{0.966120,0.744512,0.608720}%
\pgfsetfillcolor{currentfill}%
\pgfsetlinewidth{0.311001pt}%
\definecolor{currentstroke}{rgb}{1.000000,1.000000,1.000000}%
\pgfsetstrokecolor{currentstroke}%
\pgfsetdash{}{0pt}%
\pgfpathmoveto{\pgfqpoint{4.008277in}{1.629419in}}%
\pgfpathcurveto{\pgfqpoint{4.015409in}{1.629419in}}{\pgfqpoint{4.022251in}{1.632253in}}{\pgfqpoint{4.027295in}{1.637297in}}%
\pgfpathcurveto{\pgfqpoint{4.032338in}{1.642340in}}{\pgfqpoint{4.035172in}{1.649182in}}{\pgfqpoint{4.035172in}{1.656315in}}%
\pgfpathcurveto{\pgfqpoint{4.035172in}{1.663448in}}{\pgfqpoint{4.032338in}{1.670289in}}{\pgfqpoint{4.027295in}{1.675333in}}%
\pgfpathcurveto{\pgfqpoint{4.022251in}{1.680377in}}{\pgfqpoint{4.015409in}{1.683211in}}{\pgfqpoint{4.008277in}{1.683211in}}%
\pgfpathcurveto{\pgfqpoint{4.001144in}{1.683211in}}{\pgfqpoint{3.994302in}{1.680377in}}{\pgfqpoint{3.989258in}{1.675333in}}%
\pgfpathcurveto{\pgfqpoint{3.984215in}{1.670289in}}{\pgfqpoint{3.981381in}{1.663448in}}{\pgfqpoint{3.981381in}{1.656315in}}%
\pgfpathcurveto{\pgfqpoint{3.981381in}{1.649182in}}{\pgfqpoint{3.984215in}{1.642340in}}{\pgfqpoint{3.989258in}{1.637297in}}%
\pgfpathcurveto{\pgfqpoint{3.994302in}{1.632253in}}{\pgfqpoint{4.001144in}{1.629419in}}{\pgfqpoint{4.008277in}{1.629419in}}%
\pgfpathclose%
\pgfusepath{stroke,fill}%
\end{pgfscope}%
\begin{pgfscope}%
\pgfpathrectangle{\pgfqpoint{2.867647in}{0.500000in}}{\pgfqpoint{1.764706in}{1.700000in}}%
\pgfusepath{clip}%
\pgfsetbuttcap%
\pgfsetroundjoin%
\definecolor{currentfill}{rgb}{0.981377,0.920617,0.865369}%
\pgfsetfillcolor{currentfill}%
\pgfsetlinewidth{0.311001pt}%
\definecolor{currentstroke}{rgb}{1.000000,1.000000,1.000000}%
\pgfsetstrokecolor{currentstroke}%
\pgfsetdash{}{0pt}%
\pgfpathmoveto{\pgfqpoint{4.189176in}{1.202434in}}%
\pgfpathcurveto{\pgfqpoint{4.196309in}{1.202434in}}{\pgfqpoint{4.203151in}{1.205268in}}{\pgfqpoint{4.208194in}{1.210312in}}%
\pgfpathcurveto{\pgfqpoint{4.213238in}{1.215355in}}{\pgfqpoint{4.216072in}{1.222197in}}{\pgfqpoint{4.216072in}{1.229330in}}%
\pgfpathcurveto{\pgfqpoint{4.216072in}{1.236463in}}{\pgfqpoint{4.213238in}{1.243304in}}{\pgfqpoint{4.208194in}{1.248348in}}%
\pgfpathcurveto{\pgfqpoint{4.203151in}{1.253392in}}{\pgfqpoint{4.196309in}{1.256225in}}{\pgfqpoint{4.189176in}{1.256225in}}%
\pgfpathcurveto{\pgfqpoint{4.182043in}{1.256225in}}{\pgfqpoint{4.175202in}{1.253392in}}{\pgfqpoint{4.170158in}{1.248348in}}%
\pgfpathcurveto{\pgfqpoint{4.165114in}{1.243304in}}{\pgfqpoint{4.162280in}{1.236463in}}{\pgfqpoint{4.162280in}{1.229330in}}%
\pgfpathcurveto{\pgfqpoint{4.162280in}{1.222197in}}{\pgfqpoint{4.165114in}{1.215355in}}{\pgfqpoint{4.170158in}{1.210312in}}%
\pgfpathcurveto{\pgfqpoint{4.175202in}{1.205268in}}{\pgfqpoint{4.182043in}{1.202434in}}{\pgfqpoint{4.189176in}{1.202434in}}%
\pgfpathclose%
\pgfusepath{stroke,fill}%
\end{pgfscope}%
\begin{pgfscope}%
\pgfpathrectangle{\pgfqpoint{2.867647in}{0.500000in}}{\pgfqpoint{1.764706in}{1.700000in}}%
\pgfusepath{clip}%
\pgfsetbuttcap%
\pgfsetroundjoin%
\definecolor{currentfill}{rgb}{0.976287,0.879862,0.805788}%
\pgfsetfillcolor{currentfill}%
\pgfsetlinewidth{0.311001pt}%
\definecolor{currentstroke}{rgb}{1.000000,1.000000,1.000000}%
\pgfsetstrokecolor{currentstroke}%
\pgfsetdash{}{0pt}%
\pgfpathmoveto{\pgfqpoint{4.197659in}{1.537350in}}%
\pgfpathcurveto{\pgfqpoint{4.204792in}{1.537350in}}{\pgfqpoint{4.211634in}{1.540184in}}{\pgfqpoint{4.216677in}{1.545227in}}%
\pgfpathcurveto{\pgfqpoint{4.221721in}{1.550271in}}{\pgfqpoint{4.224555in}{1.557113in}}{\pgfqpoint{4.224555in}{1.564245in}}%
\pgfpathcurveto{\pgfqpoint{4.224555in}{1.571378in}}{\pgfqpoint{4.221721in}{1.578220in}}{\pgfqpoint{4.216677in}{1.583264in}}%
\pgfpathcurveto{\pgfqpoint{4.211634in}{1.588307in}}{\pgfqpoint{4.204792in}{1.591141in}}{\pgfqpoint{4.197659in}{1.591141in}}%
\pgfpathcurveto{\pgfqpoint{4.190526in}{1.591141in}}{\pgfqpoint{4.183685in}{1.588307in}}{\pgfqpoint{4.178641in}{1.583264in}}%
\pgfpathcurveto{\pgfqpoint{4.173597in}{1.578220in}}{\pgfqpoint{4.170763in}{1.571378in}}{\pgfqpoint{4.170763in}{1.564245in}}%
\pgfpathcurveto{\pgfqpoint{4.170763in}{1.557113in}}{\pgfqpoint{4.173597in}{1.550271in}}{\pgfqpoint{4.178641in}{1.545227in}}%
\pgfpathcurveto{\pgfqpoint{4.183685in}{1.540184in}}{\pgfqpoint{4.190526in}{1.537350in}}{\pgfqpoint{4.197659in}{1.537350in}}%
\pgfpathclose%
\pgfusepath{stroke,fill}%
\end{pgfscope}%
\begin{pgfscope}%
\pgfpathrectangle{\pgfqpoint{2.867647in}{0.500000in}}{\pgfqpoint{1.764706in}{1.700000in}}%
\pgfusepath{clip}%
\pgfsetbuttcap%
\pgfsetroundjoin%
\definecolor{currentfill}{rgb}{0.981377,0.920617,0.865369}%
\pgfsetfillcolor{currentfill}%
\pgfsetlinewidth{0.311001pt}%
\definecolor{currentstroke}{rgb}{1.000000,1.000000,1.000000}%
\pgfsetstrokecolor{currentstroke}%
\pgfsetdash{}{0pt}%
\pgfpathmoveto{\pgfqpoint{4.170091in}{1.239240in}}%
\pgfpathcurveto{\pgfqpoint{4.177224in}{1.239240in}}{\pgfqpoint{4.184066in}{1.242074in}}{\pgfqpoint{4.189110in}{1.247117in}}%
\pgfpathcurveto{\pgfqpoint{4.194153in}{1.252161in}}{\pgfqpoint{4.196987in}{1.259003in}}{\pgfqpoint{4.196987in}{1.266136in}}%
\pgfpathcurveto{\pgfqpoint{4.196987in}{1.273268in}}{\pgfqpoint{4.194153in}{1.280110in}}{\pgfqpoint{4.189110in}{1.285154in}}%
\pgfpathcurveto{\pgfqpoint{4.184066in}{1.290197in}}{\pgfqpoint{4.177224in}{1.293031in}}{\pgfqpoint{4.170091in}{1.293031in}}%
\pgfpathcurveto{\pgfqpoint{4.162959in}{1.293031in}}{\pgfqpoint{4.156117in}{1.290197in}}{\pgfqpoint{4.151073in}{1.285154in}}%
\pgfpathcurveto{\pgfqpoint{4.146030in}{1.280110in}}{\pgfqpoint{4.143196in}{1.273268in}}{\pgfqpoint{4.143196in}{1.266136in}}%
\pgfpathcurveto{\pgfqpoint{4.143196in}{1.259003in}}{\pgfqpoint{4.146030in}{1.252161in}}{\pgfqpoint{4.151073in}{1.247117in}}%
\pgfpathcurveto{\pgfqpoint{4.156117in}{1.242074in}}{\pgfqpoint{4.162959in}{1.239240in}}{\pgfqpoint{4.170091in}{1.239240in}}%
\pgfpathclose%
\pgfusepath{stroke,fill}%
\end{pgfscope}%
\begin{pgfscope}%
\pgfpathrectangle{\pgfqpoint{2.867647in}{0.500000in}}{\pgfqpoint{1.764706in}{1.700000in}}%
\pgfusepath{clip}%
\pgfsetbuttcap%
\pgfsetroundjoin%
\definecolor{currentfill}{rgb}{0.980678,0.914765,0.856766}%
\pgfsetfillcolor{currentfill}%
\pgfsetlinewidth{0.311001pt}%
\definecolor{currentstroke}{rgb}{1.000000,1.000000,1.000000}%
\pgfsetstrokecolor{currentstroke}%
\pgfsetdash{}{0pt}%
\pgfpathmoveto{\pgfqpoint{4.204486in}{1.213346in}}%
\pgfpathcurveto{\pgfqpoint{4.211619in}{1.213346in}}{\pgfqpoint{4.218460in}{1.216180in}}{\pgfqpoint{4.223504in}{1.221224in}}%
\pgfpathcurveto{\pgfqpoint{4.228548in}{1.226267in}}{\pgfqpoint{4.231382in}{1.233109in}}{\pgfqpoint{4.231382in}{1.240242in}}%
\pgfpathcurveto{\pgfqpoint{4.231382in}{1.247375in}}{\pgfqpoint{4.228548in}{1.254216in}}{\pgfqpoint{4.223504in}{1.259260in}}%
\pgfpathcurveto{\pgfqpoint{4.218460in}{1.264303in}}{\pgfqpoint{4.211619in}{1.267137in}}{\pgfqpoint{4.204486in}{1.267137in}}%
\pgfpathcurveto{\pgfqpoint{4.197353in}{1.267137in}}{\pgfqpoint{4.190511in}{1.264303in}}{\pgfqpoint{4.185468in}{1.259260in}}%
\pgfpathcurveto{\pgfqpoint{4.180424in}{1.254216in}}{\pgfqpoint{4.177590in}{1.247375in}}{\pgfqpoint{4.177590in}{1.240242in}}%
\pgfpathcurveto{\pgfqpoint{4.177590in}{1.233109in}}{\pgfqpoint{4.180424in}{1.226267in}}{\pgfqpoint{4.185468in}{1.221224in}}%
\pgfpathcurveto{\pgfqpoint{4.190511in}{1.216180in}}{\pgfqpoint{4.197353in}{1.213346in}}{\pgfqpoint{4.204486in}{1.213346in}}%
\pgfpathclose%
\pgfusepath{stroke,fill}%
\end{pgfscope}%
\begin{pgfscope}%
\pgfpathrectangle{\pgfqpoint{2.867647in}{0.500000in}}{\pgfqpoint{1.764706in}{1.700000in}}%
\pgfusepath{clip}%
\pgfsetbuttcap%
\pgfsetroundjoin%
\definecolor{currentfill}{rgb}{0.979891,0.908948,0.848279}%
\pgfsetfillcolor{currentfill}%
\pgfsetlinewidth{0.311001pt}%
\definecolor{currentstroke}{rgb}{1.000000,1.000000,1.000000}%
\pgfsetstrokecolor{currentstroke}%
\pgfsetdash{}{0pt}%
\pgfpathmoveto{\pgfqpoint{4.207031in}{1.210703in}}%
\pgfpathcurveto{\pgfqpoint{4.214164in}{1.210703in}}{\pgfqpoint{4.221006in}{1.213536in}}{\pgfqpoint{4.226049in}{1.218580in}}%
\pgfpathcurveto{\pgfqpoint{4.231093in}{1.223624in}}{\pgfqpoint{4.233927in}{1.230465in}}{\pgfqpoint{4.233927in}{1.237598in}}%
\pgfpathcurveto{\pgfqpoint{4.233927in}{1.244731in}}{\pgfqpoint{4.231093in}{1.251573in}}{\pgfqpoint{4.226049in}{1.256616in}}%
\pgfpathcurveto{\pgfqpoint{4.221006in}{1.261660in}}{\pgfqpoint{4.214164in}{1.264494in}}{\pgfqpoint{4.207031in}{1.264494in}}%
\pgfpathcurveto{\pgfqpoint{4.199898in}{1.264494in}}{\pgfqpoint{4.193057in}{1.261660in}}{\pgfqpoint{4.188013in}{1.256616in}}%
\pgfpathcurveto{\pgfqpoint{4.182969in}{1.251573in}}{\pgfqpoint{4.180136in}{1.244731in}}{\pgfqpoint{4.180136in}{1.237598in}}%
\pgfpathcurveto{\pgfqpoint{4.180136in}{1.230465in}}{\pgfqpoint{4.182969in}{1.223624in}}{\pgfqpoint{4.188013in}{1.218580in}}%
\pgfpathcurveto{\pgfqpoint{4.193057in}{1.213536in}}{\pgfqpoint{4.199898in}{1.210703in}}{\pgfqpoint{4.207031in}{1.210703in}}%
\pgfpathclose%
\pgfusepath{stroke,fill}%
\end{pgfscope}%
\begin{pgfscope}%
\pgfpathrectangle{\pgfqpoint{2.867647in}{0.500000in}}{\pgfqpoint{1.764706in}{1.700000in}}%
\pgfusepath{clip}%
\pgfsetbuttcap%
\pgfsetroundjoin%
\definecolor{currentfill}{rgb}{0.964306,0.663930,0.507747}%
\pgfsetfillcolor{currentfill}%
\pgfsetlinewidth{0.311001pt}%
\definecolor{currentstroke}{rgb}{1.000000,1.000000,1.000000}%
\pgfsetstrokecolor{currentstroke}%
\pgfsetdash{}{0pt}%
\pgfpathmoveto{\pgfqpoint{3.985039in}{1.025818in}}%
\pgfpathcurveto{\pgfqpoint{3.992171in}{1.025818in}}{\pgfqpoint{3.999013in}{1.028652in}}{\pgfqpoint{4.004057in}{1.033695in}}%
\pgfpathcurveto{\pgfqpoint{4.009100in}{1.038739in}}{\pgfqpoint{4.011934in}{1.045581in}}{\pgfqpoint{4.011934in}{1.052713in}}%
\pgfpathcurveto{\pgfqpoint{4.011934in}{1.059846in}}{\pgfqpoint{4.009100in}{1.066688in}}{\pgfqpoint{4.004057in}{1.071732in}}%
\pgfpathcurveto{\pgfqpoint{3.999013in}{1.076775in}}{\pgfqpoint{3.992171in}{1.079609in}}{\pgfqpoint{3.985039in}{1.079609in}}%
\pgfpathcurveto{\pgfqpoint{3.977906in}{1.079609in}}{\pgfqpoint{3.971064in}{1.076775in}}{\pgfqpoint{3.966020in}{1.071732in}}%
\pgfpathcurveto{\pgfqpoint{3.960977in}{1.066688in}}{\pgfqpoint{3.958143in}{1.059846in}}{\pgfqpoint{3.958143in}{1.052713in}}%
\pgfpathcurveto{\pgfqpoint{3.958143in}{1.045581in}}{\pgfqpoint{3.960977in}{1.038739in}}{\pgfqpoint{3.966020in}{1.033695in}}%
\pgfpathcurveto{\pgfqpoint{3.971064in}{1.028652in}}{\pgfqpoint{3.977906in}{1.025818in}}{\pgfqpoint{3.985039in}{1.025818in}}%
\pgfpathclose%
\pgfusepath{stroke,fill}%
\end{pgfscope}%
\begin{pgfscope}%
\pgfpathrectangle{\pgfqpoint{2.867647in}{0.500000in}}{\pgfqpoint{1.764706in}{1.700000in}}%
\pgfusepath{clip}%
\pgfsetbuttcap%
\pgfsetroundjoin%
\definecolor{currentfill}{rgb}{0.966328,0.750560,0.616961}%
\pgfsetfillcolor{currentfill}%
\pgfsetlinewidth{0.311001pt}%
\definecolor{currentstroke}{rgb}{1.000000,1.000000,1.000000}%
\pgfsetstrokecolor{currentstroke}%
\pgfsetdash{}{0pt}%
\pgfpathmoveto{\pgfqpoint{4.296190in}{1.333079in}}%
\pgfpathcurveto{\pgfqpoint{4.303323in}{1.333079in}}{\pgfqpoint{4.310165in}{1.335913in}}{\pgfqpoint{4.315208in}{1.340956in}}%
\pgfpathcurveto{\pgfqpoint{4.320252in}{1.346000in}}{\pgfqpoint{4.323086in}{1.352842in}}{\pgfqpoint{4.323086in}{1.359975in}}%
\pgfpathcurveto{\pgfqpoint{4.323086in}{1.367107in}}{\pgfqpoint{4.320252in}{1.373949in}}{\pgfqpoint{4.315208in}{1.378993in}}%
\pgfpathcurveto{\pgfqpoint{4.310165in}{1.384036in}}{\pgfqpoint{4.303323in}{1.386870in}}{\pgfqpoint{4.296190in}{1.386870in}}%
\pgfpathcurveto{\pgfqpoint{4.289057in}{1.386870in}}{\pgfqpoint{4.282216in}{1.384036in}}{\pgfqpoint{4.277172in}{1.378993in}}%
\pgfpathcurveto{\pgfqpoint{4.272128in}{1.373949in}}{\pgfqpoint{4.269294in}{1.367107in}}{\pgfqpoint{4.269294in}{1.359975in}}%
\pgfpathcurveto{\pgfqpoint{4.269294in}{1.352842in}}{\pgfqpoint{4.272128in}{1.346000in}}{\pgfqpoint{4.277172in}{1.340956in}}%
\pgfpathcurveto{\pgfqpoint{4.282216in}{1.335913in}}{\pgfqpoint{4.289057in}{1.333079in}}{\pgfqpoint{4.296190in}{1.333079in}}%
\pgfpathclose%
\pgfusepath{stroke,fill}%
\end{pgfscope}%
\begin{pgfscope}%
\pgfpathrectangle{\pgfqpoint{2.867647in}{0.500000in}}{\pgfqpoint{1.764706in}{1.700000in}}%
\pgfusepath{clip}%
\pgfsetbuttcap%
\pgfsetroundjoin%
\definecolor{currentfill}{rgb}{0.981377,0.920617,0.865369}%
\pgfsetfillcolor{currentfill}%
\pgfsetlinewidth{0.311001pt}%
\definecolor{currentstroke}{rgb}{1.000000,1.000000,1.000000}%
\pgfsetstrokecolor{currentstroke}%
\pgfsetdash{}{0pt}%
\pgfpathmoveto{\pgfqpoint{4.194522in}{1.205091in}}%
\pgfpathcurveto{\pgfqpoint{4.201654in}{1.205091in}}{\pgfqpoint{4.208496in}{1.207925in}}{\pgfqpoint{4.213540in}{1.212968in}}%
\pgfpathcurveto{\pgfqpoint{4.218583in}{1.218012in}}{\pgfqpoint{4.221417in}{1.224854in}}{\pgfqpoint{4.221417in}{1.231987in}}%
\pgfpathcurveto{\pgfqpoint{4.221417in}{1.239119in}}{\pgfqpoint{4.218583in}{1.245961in}}{\pgfqpoint{4.213540in}{1.251005in}}%
\pgfpathcurveto{\pgfqpoint{4.208496in}{1.256048in}}{\pgfqpoint{4.201654in}{1.258882in}}{\pgfqpoint{4.194522in}{1.258882in}}%
\pgfpathcurveto{\pgfqpoint{4.187389in}{1.258882in}}{\pgfqpoint{4.180547in}{1.256048in}}{\pgfqpoint{4.175503in}{1.251005in}}%
\pgfpathcurveto{\pgfqpoint{4.170460in}{1.245961in}}{\pgfqpoint{4.167626in}{1.239119in}}{\pgfqpoint{4.167626in}{1.231987in}}%
\pgfpathcurveto{\pgfqpoint{4.167626in}{1.224854in}}{\pgfqpoint{4.170460in}{1.218012in}}{\pgfqpoint{4.175503in}{1.212968in}}%
\pgfpathcurveto{\pgfqpoint{4.180547in}{1.207925in}}{\pgfqpoint{4.187389in}{1.205091in}}{\pgfqpoint{4.194522in}{1.205091in}}%
\pgfpathclose%
\pgfusepath{stroke,fill}%
\end{pgfscope}%
\begin{pgfscope}%
\pgfpathrectangle{\pgfqpoint{2.867647in}{0.500000in}}{\pgfqpoint{1.764706in}{1.700000in}}%
\pgfusepath{clip}%
\pgfsetbuttcap%
\pgfsetroundjoin%
\definecolor{currentfill}{rgb}{0.965928,0.738443,0.600540}%
\pgfsetfillcolor{currentfill}%
\pgfsetlinewidth{0.311001pt}%
\definecolor{currentstroke}{rgb}{1.000000,1.000000,1.000000}%
\pgfsetstrokecolor{currentstroke}%
\pgfsetdash{}{0pt}%
\pgfpathmoveto{\pgfqpoint{4.035205in}{1.529507in}}%
\pgfpathcurveto{\pgfqpoint{4.042338in}{1.529507in}}{\pgfqpoint{4.049180in}{1.532341in}}{\pgfqpoint{4.054224in}{1.537385in}}%
\pgfpathcurveto{\pgfqpoint{4.059267in}{1.542428in}}{\pgfqpoint{4.062101in}{1.549270in}}{\pgfqpoint{4.062101in}{1.556403in}}%
\pgfpathcurveto{\pgfqpoint{4.062101in}{1.563536in}}{\pgfqpoint{4.059267in}{1.570377in}}{\pgfqpoint{4.054224in}{1.575421in}}%
\pgfpathcurveto{\pgfqpoint{4.049180in}{1.580465in}}{\pgfqpoint{4.042338in}{1.583299in}}{\pgfqpoint{4.035205in}{1.583299in}}%
\pgfpathcurveto{\pgfqpoint{4.028073in}{1.583299in}}{\pgfqpoint{4.021231in}{1.580465in}}{\pgfqpoint{4.016187in}{1.575421in}}%
\pgfpathcurveto{\pgfqpoint{4.011144in}{1.570377in}}{\pgfqpoint{4.008310in}{1.563536in}}{\pgfqpoint{4.008310in}{1.556403in}}%
\pgfpathcurveto{\pgfqpoint{4.008310in}{1.549270in}}{\pgfqpoint{4.011144in}{1.542428in}}{\pgfqpoint{4.016187in}{1.537385in}}%
\pgfpathcurveto{\pgfqpoint{4.021231in}{1.532341in}}{\pgfqpoint{4.028073in}{1.529507in}}{\pgfqpoint{4.035205in}{1.529507in}}%
\pgfpathclose%
\pgfusepath{stroke,fill}%
\end{pgfscope}%
\begin{pgfscope}%
\pgfpathrectangle{\pgfqpoint{2.867647in}{0.500000in}}{\pgfqpoint{1.764706in}{1.700000in}}%
\pgfusepath{clip}%
\pgfsetbuttcap%
\pgfsetroundjoin%
\definecolor{currentfill}{rgb}{0.979124,0.903132,0.839793}%
\pgfsetfillcolor{currentfill}%
\pgfsetlinewidth{0.311001pt}%
\definecolor{currentstroke}{rgb}{1.000000,1.000000,1.000000}%
\pgfsetstrokecolor{currentstroke}%
\pgfsetdash{}{0pt}%
\pgfpathmoveto{\pgfqpoint{4.223745in}{1.343170in}}%
\pgfpathcurveto{\pgfqpoint{4.230878in}{1.343170in}}{\pgfqpoint{4.237720in}{1.346004in}}{\pgfqpoint{4.242763in}{1.351047in}}%
\pgfpathcurveto{\pgfqpoint{4.247807in}{1.356091in}}{\pgfqpoint{4.250641in}{1.362932in}}{\pgfqpoint{4.250641in}{1.370065in}}%
\pgfpathcurveto{\pgfqpoint{4.250641in}{1.377198in}}{\pgfqpoint{4.247807in}{1.384040in}}{\pgfqpoint{4.242763in}{1.389083in}}%
\pgfpathcurveto{\pgfqpoint{4.237720in}{1.394127in}}{\pgfqpoint{4.230878in}{1.396961in}}{\pgfqpoint{4.223745in}{1.396961in}}%
\pgfpathcurveto{\pgfqpoint{4.216612in}{1.396961in}}{\pgfqpoint{4.209771in}{1.394127in}}{\pgfqpoint{4.204727in}{1.389083in}}%
\pgfpathcurveto{\pgfqpoint{4.199684in}{1.384040in}}{\pgfqpoint{4.196850in}{1.377198in}}{\pgfqpoint{4.196850in}{1.370065in}}%
\pgfpathcurveto{\pgfqpoint{4.196850in}{1.362932in}}{\pgfqpoint{4.199684in}{1.356091in}}{\pgfqpoint{4.204727in}{1.351047in}}%
\pgfpathcurveto{\pgfqpoint{4.209771in}{1.346004in}}{\pgfqpoint{4.216612in}{1.343170in}}{\pgfqpoint{4.223745in}{1.343170in}}%
\pgfpathclose%
\pgfusepath{stroke,fill}%
\end{pgfscope}%
\begin{pgfscope}%
\pgfpathrectangle{\pgfqpoint{2.867647in}{0.500000in}}{\pgfqpoint{1.764706in}{1.700000in}}%
\pgfusepath{clip}%
\pgfsetbuttcap%
\pgfsetroundjoin%
\definecolor{currentfill}{rgb}{0.976961,0.885681,0.814303}%
\pgfsetfillcolor{currentfill}%
\pgfsetlinewidth{0.311001pt}%
\definecolor{currentstroke}{rgb}{1.000000,1.000000,1.000000}%
\pgfsetstrokecolor{currentstroke}%
\pgfsetdash{}{0pt}%
\pgfpathmoveto{\pgfqpoint{4.214626in}{1.148178in}}%
\pgfpathcurveto{\pgfqpoint{4.221759in}{1.148178in}}{\pgfqpoint{4.228600in}{1.151011in}}{\pgfqpoint{4.233644in}{1.156055in}}%
\pgfpathcurveto{\pgfqpoint{4.238688in}{1.161099in}}{\pgfqpoint{4.241522in}{1.167940in}}{\pgfqpoint{4.241522in}{1.175073in}}%
\pgfpathcurveto{\pgfqpoint{4.241522in}{1.182206in}}{\pgfqpoint{4.238688in}{1.189048in}}{\pgfqpoint{4.233644in}{1.194091in}}%
\pgfpathcurveto{\pgfqpoint{4.228600in}{1.199135in}}{\pgfqpoint{4.221759in}{1.201969in}}{\pgfqpoint{4.214626in}{1.201969in}}%
\pgfpathcurveto{\pgfqpoint{4.207493in}{1.201969in}}{\pgfqpoint{4.200651in}{1.199135in}}{\pgfqpoint{4.195608in}{1.194091in}}%
\pgfpathcurveto{\pgfqpoint{4.190564in}{1.189048in}}{\pgfqpoint{4.187730in}{1.182206in}}{\pgfqpoint{4.187730in}{1.175073in}}%
\pgfpathcurveto{\pgfqpoint{4.187730in}{1.167940in}}{\pgfqpoint{4.190564in}{1.161099in}}{\pgfqpoint{4.195608in}{1.156055in}}%
\pgfpathcurveto{\pgfqpoint{4.200651in}{1.151011in}}{\pgfqpoint{4.207493in}{1.148178in}}{\pgfqpoint{4.214626in}{1.148178in}}%
\pgfpathclose%
\pgfusepath{stroke,fill}%
\end{pgfscope}%
\begin{pgfscope}%
\pgfpathrectangle{\pgfqpoint{2.867647in}{0.500000in}}{\pgfqpoint{1.764706in}{1.700000in}}%
\pgfusepath{clip}%
\pgfsetbuttcap%
\pgfsetroundjoin%
\definecolor{currentfill}{rgb}{0.969803,0.809811,0.702523}%
\pgfsetfillcolor{currentfill}%
\pgfsetlinewidth{0.311001pt}%
\definecolor{currentstroke}{rgb}{1.000000,1.000000,1.000000}%
\pgfsetstrokecolor{currentstroke}%
\pgfsetdash{}{0pt}%
\pgfpathmoveto{\pgfqpoint{4.270116in}{1.216605in}}%
\pgfpathcurveto{\pgfqpoint{4.277249in}{1.216605in}}{\pgfqpoint{4.284091in}{1.219439in}}{\pgfqpoint{4.289134in}{1.224483in}}%
\pgfpathcurveto{\pgfqpoint{4.294178in}{1.229526in}}{\pgfqpoint{4.297012in}{1.236368in}}{\pgfqpoint{4.297012in}{1.243501in}}%
\pgfpathcurveto{\pgfqpoint{4.297012in}{1.250633in}}{\pgfqpoint{4.294178in}{1.257475in}}{\pgfqpoint{4.289134in}{1.262519in}}%
\pgfpathcurveto{\pgfqpoint{4.284091in}{1.267562in}}{\pgfqpoint{4.277249in}{1.270396in}}{\pgfqpoint{4.270116in}{1.270396in}}%
\pgfpathcurveto{\pgfqpoint{4.262983in}{1.270396in}}{\pgfqpoint{4.256142in}{1.267562in}}{\pgfqpoint{4.251098in}{1.262519in}}%
\pgfpathcurveto{\pgfqpoint{4.246054in}{1.257475in}}{\pgfqpoint{4.243220in}{1.250633in}}{\pgfqpoint{4.243220in}{1.243501in}}%
\pgfpathcurveto{\pgfqpoint{4.243220in}{1.236368in}}{\pgfqpoint{4.246054in}{1.229526in}}{\pgfqpoint{4.251098in}{1.224483in}}%
\pgfpathcurveto{\pgfqpoint{4.256142in}{1.219439in}}{\pgfqpoint{4.262983in}{1.216605in}}{\pgfqpoint{4.270116in}{1.216605in}}%
\pgfpathclose%
\pgfusepath{stroke,fill}%
\end{pgfscope}%
\begin{pgfscope}%
\pgfpathrectangle{\pgfqpoint{2.867647in}{0.500000in}}{\pgfqpoint{1.764706in}{1.700000in}}%
\pgfusepath{clip}%
\pgfsetbuttcap%
\pgfsetroundjoin%
\definecolor{currentfill}{rgb}{0.972726,0.844889,0.754401}%
\pgfsetfillcolor{currentfill}%
\pgfsetlinewidth{0.311001pt}%
\definecolor{currentstroke}{rgb}{1.000000,1.000000,1.000000}%
\pgfsetstrokecolor{currentstroke}%
\pgfsetdash{}{0pt}%
\pgfpathmoveto{\pgfqpoint{4.118914in}{1.288620in}}%
\pgfpathcurveto{\pgfqpoint{4.126047in}{1.288620in}}{\pgfqpoint{4.132889in}{1.291454in}}{\pgfqpoint{4.137932in}{1.296497in}}%
\pgfpathcurveto{\pgfqpoint{4.142976in}{1.301541in}}{\pgfqpoint{4.145810in}{1.308383in}}{\pgfqpoint{4.145810in}{1.315515in}}%
\pgfpathcurveto{\pgfqpoint{4.145810in}{1.322648in}}{\pgfqpoint{4.142976in}{1.329490in}}{\pgfqpoint{4.137932in}{1.334534in}}%
\pgfpathcurveto{\pgfqpoint{4.132889in}{1.339577in}}{\pgfqpoint{4.126047in}{1.342411in}}{\pgfqpoint{4.118914in}{1.342411in}}%
\pgfpathcurveto{\pgfqpoint{4.111781in}{1.342411in}}{\pgfqpoint{4.104940in}{1.339577in}}{\pgfqpoint{4.099896in}{1.334534in}}%
\pgfpathcurveto{\pgfqpoint{4.094852in}{1.329490in}}{\pgfqpoint{4.092018in}{1.322648in}}{\pgfqpoint{4.092018in}{1.315515in}}%
\pgfpathcurveto{\pgfqpoint{4.092018in}{1.308383in}}{\pgfqpoint{4.094852in}{1.301541in}}{\pgfqpoint{4.099896in}{1.296497in}}%
\pgfpathcurveto{\pgfqpoint{4.104940in}{1.291454in}}{\pgfqpoint{4.111781in}{1.288620in}}{\pgfqpoint{4.118914in}{1.288620in}}%
\pgfpathclose%
\pgfusepath{stroke,fill}%
\end{pgfscope}%
\begin{pgfscope}%
\pgfpathrectangle{\pgfqpoint{2.867647in}{0.500000in}}{\pgfqpoint{1.764706in}{1.700000in}}%
\pgfusepath{clip}%
\pgfsetbuttcap%
\pgfsetroundjoin%
\definecolor{currentfill}{rgb}{0.980678,0.914765,0.856766}%
\pgfsetfillcolor{currentfill}%
\pgfsetlinewidth{0.311001pt}%
\definecolor{currentstroke}{rgb}{1.000000,1.000000,1.000000}%
\pgfsetstrokecolor{currentstroke}%
\pgfsetdash{}{0pt}%
\pgfpathmoveto{\pgfqpoint{4.204843in}{1.368130in}}%
\pgfpathcurveto{\pgfqpoint{4.211976in}{1.368130in}}{\pgfqpoint{4.218817in}{1.370964in}}{\pgfqpoint{4.223861in}{1.376007in}}%
\pgfpathcurveto{\pgfqpoint{4.228905in}{1.381051in}}{\pgfqpoint{4.231738in}{1.387893in}}{\pgfqpoint{4.231738in}{1.395025in}}%
\pgfpathcurveto{\pgfqpoint{4.231738in}{1.402158in}}{\pgfqpoint{4.228905in}{1.409000in}}{\pgfqpoint{4.223861in}{1.414044in}}%
\pgfpathcurveto{\pgfqpoint{4.218817in}{1.419087in}}{\pgfqpoint{4.211976in}{1.421921in}}{\pgfqpoint{4.204843in}{1.421921in}}%
\pgfpathcurveto{\pgfqpoint{4.197710in}{1.421921in}}{\pgfqpoint{4.190868in}{1.419087in}}{\pgfqpoint{4.185825in}{1.414044in}}%
\pgfpathcurveto{\pgfqpoint{4.180781in}{1.409000in}}{\pgfqpoint{4.177947in}{1.402158in}}{\pgfqpoint{4.177947in}{1.395025in}}%
\pgfpathcurveto{\pgfqpoint{4.177947in}{1.387893in}}{\pgfqpoint{4.180781in}{1.381051in}}{\pgfqpoint{4.185825in}{1.376007in}}%
\pgfpathcurveto{\pgfqpoint{4.190868in}{1.370964in}}{\pgfqpoint{4.197710in}{1.368130in}}{\pgfqpoint{4.204843in}{1.368130in}}%
\pgfpathclose%
\pgfusepath{stroke,fill}%
\end{pgfscope}%
\begin{pgfscope}%
\pgfpathrectangle{\pgfqpoint{2.867647in}{0.500000in}}{\pgfqpoint{1.764706in}{1.700000in}}%
\pgfusepath{clip}%
\pgfsetbuttcap%
\pgfsetroundjoin%
\definecolor{currentfill}{rgb}{0.967735,0.780441,0.659127}%
\pgfsetfillcolor{currentfill}%
\pgfsetlinewidth{0.311001pt}%
\definecolor{currentstroke}{rgb}{1.000000,1.000000,1.000000}%
\pgfsetstrokecolor{currentstroke}%
\pgfsetdash{}{0pt}%
\pgfpathmoveto{\pgfqpoint{4.051243in}{1.714619in}}%
\pgfpathcurveto{\pgfqpoint{4.058376in}{1.714619in}}{\pgfqpoint{4.065218in}{1.717453in}}{\pgfqpoint{4.070261in}{1.722497in}}%
\pgfpathcurveto{\pgfqpoint{4.075305in}{1.727541in}}{\pgfqpoint{4.078139in}{1.734382in}}{\pgfqpoint{4.078139in}{1.741515in}}%
\pgfpathcurveto{\pgfqpoint{4.078139in}{1.748648in}}{\pgfqpoint{4.075305in}{1.755489in}}{\pgfqpoint{4.070261in}{1.760533in}}%
\pgfpathcurveto{\pgfqpoint{4.065218in}{1.765577in}}{\pgfqpoint{4.058376in}{1.768411in}}{\pgfqpoint{4.051243in}{1.768411in}}%
\pgfpathcurveto{\pgfqpoint{4.044110in}{1.768411in}}{\pgfqpoint{4.037269in}{1.765577in}}{\pgfqpoint{4.032225in}{1.760533in}}%
\pgfpathcurveto{\pgfqpoint{4.027181in}{1.755489in}}{\pgfqpoint{4.024347in}{1.748648in}}{\pgfqpoint{4.024347in}{1.741515in}}%
\pgfpathcurveto{\pgfqpoint{4.024347in}{1.734382in}}{\pgfqpoint{4.027181in}{1.727541in}}{\pgfqpoint{4.032225in}{1.722497in}}%
\pgfpathcurveto{\pgfqpoint{4.037269in}{1.717453in}}{\pgfqpoint{4.044110in}{1.714619in}}{\pgfqpoint{4.051243in}{1.714619in}}%
\pgfpathclose%
\pgfusepath{stroke,fill}%
\end{pgfscope}%
\begin{pgfscope}%
\pgfpathrectangle{\pgfqpoint{2.867647in}{0.500000in}}{\pgfqpoint{1.764706in}{1.700000in}}%
\pgfusepath{clip}%
\pgfsetbuttcap%
\pgfsetroundjoin%
\definecolor{currentfill}{rgb}{0.966328,0.750560,0.616961}%
\pgfsetfillcolor{currentfill}%
\pgfsetlinewidth{0.311001pt}%
\definecolor{currentstroke}{rgb}{1.000000,1.000000,1.000000}%
\pgfsetstrokecolor{currentstroke}%
\pgfsetdash{}{0pt}%
\pgfpathmoveto{\pgfqpoint{4.249366in}{1.074060in}}%
\pgfpathcurveto{\pgfqpoint{4.256499in}{1.074060in}}{\pgfqpoint{4.263341in}{1.076894in}}{\pgfqpoint{4.268385in}{1.081938in}}%
\pgfpathcurveto{\pgfqpoint{4.273428in}{1.086982in}}{\pgfqpoint{4.276262in}{1.093823in}}{\pgfqpoint{4.276262in}{1.100956in}}%
\pgfpathcurveto{\pgfqpoint{4.276262in}{1.108089in}}{\pgfqpoint{4.273428in}{1.114931in}}{\pgfqpoint{4.268385in}{1.119974in}}%
\pgfpathcurveto{\pgfqpoint{4.263341in}{1.125018in}}{\pgfqpoint{4.256499in}{1.127852in}}{\pgfqpoint{4.249366in}{1.127852in}}%
\pgfpathcurveto{\pgfqpoint{4.242234in}{1.127852in}}{\pgfqpoint{4.235392in}{1.125018in}}{\pgfqpoint{4.230348in}{1.119974in}}%
\pgfpathcurveto{\pgfqpoint{4.225305in}{1.114931in}}{\pgfqpoint{4.222471in}{1.108089in}}{\pgfqpoint{4.222471in}{1.100956in}}%
\pgfpathcurveto{\pgfqpoint{4.222471in}{1.093823in}}{\pgfqpoint{4.225305in}{1.086982in}}{\pgfqpoint{4.230348in}{1.081938in}}%
\pgfpathcurveto{\pgfqpoint{4.235392in}{1.076894in}}{\pgfqpoint{4.242234in}{1.074060in}}{\pgfqpoint{4.249366in}{1.074060in}}%
\pgfpathclose%
\pgfusepath{stroke,fill}%
\end{pgfscope}%
\begin{pgfscope}%
\pgfpathrectangle{\pgfqpoint{2.867647in}{0.500000in}}{\pgfqpoint{1.764706in}{1.700000in}}%
\pgfusepath{clip}%
\pgfsetbuttcap%
\pgfsetroundjoin%
\definecolor{currentfill}{rgb}{0.961433,0.573272,0.412036}%
\pgfsetfillcolor{currentfill}%
\pgfsetlinewidth{0.311001pt}%
\definecolor{currentstroke}{rgb}{1.000000,1.000000,1.000000}%
\pgfsetstrokecolor{currentstroke}%
\pgfsetdash{}{0pt}%
\pgfpathmoveto{\pgfqpoint{3.994146in}{1.533217in}}%
\pgfpathcurveto{\pgfqpoint{4.001279in}{1.533217in}}{\pgfqpoint{4.008121in}{1.536051in}}{\pgfqpoint{4.013164in}{1.541095in}}%
\pgfpathcurveto{\pgfqpoint{4.018208in}{1.546138in}}{\pgfqpoint{4.021042in}{1.552980in}}{\pgfqpoint{4.021042in}{1.560113in}}%
\pgfpathcurveto{\pgfqpoint{4.021042in}{1.567246in}}{\pgfqpoint{4.018208in}{1.574087in}}{\pgfqpoint{4.013164in}{1.579131in}}%
\pgfpathcurveto{\pgfqpoint{4.008121in}{1.584175in}}{\pgfqpoint{4.001279in}{1.587009in}}{\pgfqpoint{3.994146in}{1.587009in}}%
\pgfpathcurveto{\pgfqpoint{3.987013in}{1.587009in}}{\pgfqpoint{3.980172in}{1.584175in}}{\pgfqpoint{3.975128in}{1.579131in}}%
\pgfpathcurveto{\pgfqpoint{3.970084in}{1.574087in}}{\pgfqpoint{3.967250in}{1.567246in}}{\pgfqpoint{3.967250in}{1.560113in}}%
\pgfpathcurveto{\pgfqpoint{3.967250in}{1.552980in}}{\pgfqpoint{3.970084in}{1.546138in}}{\pgfqpoint{3.975128in}{1.541095in}}%
\pgfpathcurveto{\pgfqpoint{3.980172in}{1.536051in}}{\pgfqpoint{3.987013in}{1.533217in}}{\pgfqpoint{3.994146in}{1.533217in}}%
\pgfpathclose%
\pgfusepath{stroke,fill}%
\end{pgfscope}%
\begin{pgfscope}%
\pgfpathrectangle{\pgfqpoint{2.867647in}{0.500000in}}{\pgfqpoint{1.764706in}{1.700000in}}%
\pgfusepath{clip}%
\pgfsetbuttcap%
\pgfsetroundjoin%
\definecolor{currentfill}{rgb}{0.975644,0.874038,0.797253}%
\pgfsetfillcolor{currentfill}%
\pgfsetlinewidth{0.311001pt}%
\definecolor{currentstroke}{rgb}{1.000000,1.000000,1.000000}%
\pgfsetstrokecolor{currentstroke}%
\pgfsetdash{}{0pt}%
\pgfpathmoveto{\pgfqpoint{4.231865in}{1.183252in}}%
\pgfpathcurveto{\pgfqpoint{4.238998in}{1.183252in}}{\pgfqpoint{4.245840in}{1.186086in}}{\pgfqpoint{4.250883in}{1.191130in}}%
\pgfpathcurveto{\pgfqpoint{4.255927in}{1.196173in}}{\pgfqpoint{4.258761in}{1.203015in}}{\pgfqpoint{4.258761in}{1.210148in}}%
\pgfpathcurveto{\pgfqpoint{4.258761in}{1.217280in}}{\pgfqpoint{4.255927in}{1.224122in}}{\pgfqpoint{4.250883in}{1.229166in}}%
\pgfpathcurveto{\pgfqpoint{4.245840in}{1.234209in}}{\pgfqpoint{4.238998in}{1.237043in}}{\pgfqpoint{4.231865in}{1.237043in}}%
\pgfpathcurveto{\pgfqpoint{4.224732in}{1.237043in}}{\pgfqpoint{4.217891in}{1.234209in}}{\pgfqpoint{4.212847in}{1.229166in}}%
\pgfpathcurveto{\pgfqpoint{4.207803in}{1.224122in}}{\pgfqpoint{4.204969in}{1.217280in}}{\pgfqpoint{4.204969in}{1.210148in}}%
\pgfpathcurveto{\pgfqpoint{4.204969in}{1.203015in}}{\pgfqpoint{4.207803in}{1.196173in}}{\pgfqpoint{4.212847in}{1.191130in}}%
\pgfpathcurveto{\pgfqpoint{4.217891in}{1.186086in}}{\pgfqpoint{4.224732in}{1.183252in}}{\pgfqpoint{4.231865in}{1.183252in}}%
\pgfpathclose%
\pgfusepath{stroke,fill}%
\end{pgfscope}%
\begin{pgfscope}%
\pgfpathrectangle{\pgfqpoint{2.867647in}{0.500000in}}{\pgfqpoint{1.764706in}{1.700000in}}%
\pgfusepath{clip}%
\pgfsetbuttcap%
\pgfsetroundjoin%
\definecolor{currentfill}{rgb}{0.977657,0.891500,0.822809}%
\pgfsetfillcolor{currentfill}%
\pgfsetlinewidth{0.311001pt}%
\definecolor{currentstroke}{rgb}{1.000000,1.000000,1.000000}%
\pgfsetstrokecolor{currentstroke}%
\pgfsetdash{}{0pt}%
\pgfpathmoveto{\pgfqpoint{4.198328in}{1.118544in}}%
\pgfpathcurveto{\pgfqpoint{4.205461in}{1.118544in}}{\pgfqpoint{4.212303in}{1.121378in}}{\pgfqpoint{4.217346in}{1.126422in}}%
\pgfpathcurveto{\pgfqpoint{4.222390in}{1.131466in}}{\pgfqpoint{4.225224in}{1.138307in}}{\pgfqpoint{4.225224in}{1.145440in}}%
\pgfpathcurveto{\pgfqpoint{4.225224in}{1.152573in}}{\pgfqpoint{4.222390in}{1.159414in}}{\pgfqpoint{4.217346in}{1.164458in}}%
\pgfpathcurveto{\pgfqpoint{4.212303in}{1.169502in}}{\pgfqpoint{4.205461in}{1.172336in}}{\pgfqpoint{4.198328in}{1.172336in}}%
\pgfpathcurveto{\pgfqpoint{4.191195in}{1.172336in}}{\pgfqpoint{4.184354in}{1.169502in}}{\pgfqpoint{4.179310in}{1.164458in}}%
\pgfpathcurveto{\pgfqpoint{4.174266in}{1.159414in}}{\pgfqpoint{4.171432in}{1.152573in}}{\pgfqpoint{4.171432in}{1.145440in}}%
\pgfpathcurveto{\pgfqpoint{4.171432in}{1.138307in}}{\pgfqpoint{4.174266in}{1.131466in}}{\pgfqpoint{4.179310in}{1.126422in}}%
\pgfpathcurveto{\pgfqpoint{4.184354in}{1.121378in}}{\pgfqpoint{4.191195in}{1.118544in}}{\pgfqpoint{4.198328in}{1.118544in}}%
\pgfpathclose%
\pgfusepath{stroke,fill}%
\end{pgfscope}%
\begin{pgfscope}%
\pgfpathrectangle{\pgfqpoint{2.867647in}{0.500000in}}{\pgfqpoint{1.764706in}{1.700000in}}%
\pgfusepath{clip}%
\pgfsetbuttcap%
\pgfsetroundjoin%
\definecolor{currentfill}{rgb}{0.980678,0.914765,0.856766}%
\pgfsetfillcolor{currentfill}%
\pgfsetlinewidth{0.311001pt}%
\definecolor{currentstroke}{rgb}{1.000000,1.000000,1.000000}%
\pgfsetstrokecolor{currentstroke}%
\pgfsetdash{}{0pt}%
\pgfpathmoveto{\pgfqpoint{4.168816in}{1.144054in}}%
\pgfpathcurveto{\pgfqpoint{4.175949in}{1.144054in}}{\pgfqpoint{4.182790in}{1.146888in}}{\pgfqpoint{4.187834in}{1.151931in}}%
\pgfpathcurveto{\pgfqpoint{4.192878in}{1.156975in}}{\pgfqpoint{4.195712in}{1.163817in}}{\pgfqpoint{4.195712in}{1.170950in}}%
\pgfpathcurveto{\pgfqpoint{4.195712in}{1.178082in}}{\pgfqpoint{4.192878in}{1.184924in}}{\pgfqpoint{4.187834in}{1.189968in}}%
\pgfpathcurveto{\pgfqpoint{4.182790in}{1.195011in}}{\pgfqpoint{4.175949in}{1.197845in}}{\pgfqpoint{4.168816in}{1.197845in}}%
\pgfpathcurveto{\pgfqpoint{4.161683in}{1.197845in}}{\pgfqpoint{4.154842in}{1.195011in}}{\pgfqpoint{4.149798in}{1.189968in}}%
\pgfpathcurveto{\pgfqpoint{4.144754in}{1.184924in}}{\pgfqpoint{4.141920in}{1.178082in}}{\pgfqpoint{4.141920in}{1.170950in}}%
\pgfpathcurveto{\pgfqpoint{4.141920in}{1.163817in}}{\pgfqpoint{4.144754in}{1.156975in}}{\pgfqpoint{4.149798in}{1.151931in}}%
\pgfpathcurveto{\pgfqpoint{4.154842in}{1.146888in}}{\pgfqpoint{4.161683in}{1.144054in}}{\pgfqpoint{4.168816in}{1.144054in}}%
\pgfpathclose%
\pgfusepath{stroke,fill}%
\end{pgfscope}%
\begin{pgfscope}%
\pgfpathrectangle{\pgfqpoint{2.867647in}{0.500000in}}{\pgfqpoint{1.764706in}{1.700000in}}%
\pgfusepath{clip}%
\pgfsetbuttcap%
\pgfsetroundjoin%
\definecolor{currentfill}{rgb}{0.969803,0.809811,0.702523}%
\pgfsetfillcolor{currentfill}%
\pgfsetlinewidth{0.311001pt}%
\definecolor{currentstroke}{rgb}{1.000000,1.000000,1.000000}%
\pgfsetstrokecolor{currentstroke}%
\pgfsetdash{}{0pt}%
\pgfpathmoveto{\pgfqpoint{4.100613in}{1.384475in}}%
\pgfpathcurveto{\pgfqpoint{4.107746in}{1.384475in}}{\pgfqpoint{4.114588in}{1.387309in}}{\pgfqpoint{4.119631in}{1.392353in}}%
\pgfpathcurveto{\pgfqpoint{4.124675in}{1.397397in}}{\pgfqpoint{4.127509in}{1.404238in}}{\pgfqpoint{4.127509in}{1.411371in}}%
\pgfpathcurveto{\pgfqpoint{4.127509in}{1.418504in}}{\pgfqpoint{4.124675in}{1.425346in}}{\pgfqpoint{4.119631in}{1.430389in}}%
\pgfpathcurveto{\pgfqpoint{4.114588in}{1.435433in}}{\pgfqpoint{4.107746in}{1.438267in}}{\pgfqpoint{4.100613in}{1.438267in}}%
\pgfpathcurveto{\pgfqpoint{4.093480in}{1.438267in}}{\pgfqpoint{4.086639in}{1.435433in}}{\pgfqpoint{4.081595in}{1.430389in}}%
\pgfpathcurveto{\pgfqpoint{4.076551in}{1.425346in}}{\pgfqpoint{4.073717in}{1.418504in}}{\pgfqpoint{4.073717in}{1.411371in}}%
\pgfpathcurveto{\pgfqpoint{4.073717in}{1.404238in}}{\pgfqpoint{4.076551in}{1.397397in}}{\pgfqpoint{4.081595in}{1.392353in}}%
\pgfpathcurveto{\pgfqpoint{4.086639in}{1.387309in}}{\pgfqpoint{4.093480in}{1.384475in}}{\pgfqpoint{4.100613in}{1.384475in}}%
\pgfpathclose%
\pgfusepath{stroke,fill}%
\end{pgfscope}%
\begin{pgfscope}%
\pgfpathrectangle{\pgfqpoint{2.867647in}{0.500000in}}{\pgfqpoint{1.764706in}{1.700000in}}%
\pgfusepath{clip}%
\pgfsetbuttcap%
\pgfsetroundjoin%
\definecolor{currentfill}{rgb}{0.965440,0.720101,0.576404}%
\pgfsetfillcolor{currentfill}%
\pgfsetlinewidth{0.311001pt}%
\definecolor{currentstroke}{rgb}{1.000000,1.000000,1.000000}%
\pgfsetstrokecolor{currentstroke}%
\pgfsetdash{}{0pt}%
\pgfpathmoveto{\pgfqpoint{4.292910in}{1.437717in}}%
\pgfpathcurveto{\pgfqpoint{4.300043in}{1.437717in}}{\pgfqpoint{4.306885in}{1.440551in}}{\pgfqpoint{4.311928in}{1.445595in}}%
\pgfpathcurveto{\pgfqpoint{4.316972in}{1.450638in}}{\pgfqpoint{4.319806in}{1.457480in}}{\pgfqpoint{4.319806in}{1.464613in}}%
\pgfpathcurveto{\pgfqpoint{4.319806in}{1.471746in}}{\pgfqpoint{4.316972in}{1.478587in}}{\pgfqpoint{4.311928in}{1.483631in}}%
\pgfpathcurveto{\pgfqpoint{4.306885in}{1.488675in}}{\pgfqpoint{4.300043in}{1.491509in}}{\pgfqpoint{4.292910in}{1.491509in}}%
\pgfpathcurveto{\pgfqpoint{4.285777in}{1.491509in}}{\pgfqpoint{4.278936in}{1.488675in}}{\pgfqpoint{4.273892in}{1.483631in}}%
\pgfpathcurveto{\pgfqpoint{4.268848in}{1.478587in}}{\pgfqpoint{4.266015in}{1.471746in}}{\pgfqpoint{4.266015in}{1.464613in}}%
\pgfpathcurveto{\pgfqpoint{4.266015in}{1.457480in}}{\pgfqpoint{4.268848in}{1.450638in}}{\pgfqpoint{4.273892in}{1.445595in}}%
\pgfpathcurveto{\pgfqpoint{4.278936in}{1.440551in}}{\pgfqpoint{4.285777in}{1.437717in}}{\pgfqpoint{4.292910in}{1.437717in}}%
\pgfpathclose%
\pgfusepath{stroke,fill}%
\end{pgfscope}%
\begin{pgfscope}%
\pgfpathrectangle{\pgfqpoint{2.867647in}{0.500000in}}{\pgfqpoint{1.764706in}{1.700000in}}%
\pgfusepath{clip}%
\pgfsetbuttcap%
\pgfsetroundjoin%
\definecolor{currentfill}{rgb}{0.964799,0.689101,0.537560}%
\pgfsetfillcolor{currentfill}%
\pgfsetlinewidth{0.311001pt}%
\definecolor{currentstroke}{rgb}{1.000000,1.000000,1.000000}%
\pgfsetstrokecolor{currentstroke}%
\pgfsetdash{}{0pt}%
\pgfpathmoveto{\pgfqpoint{4.280729in}{1.109963in}}%
\pgfpathcurveto{\pgfqpoint{4.287861in}{1.109963in}}{\pgfqpoint{4.294703in}{1.112797in}}{\pgfqpoint{4.299747in}{1.117840in}}%
\pgfpathcurveto{\pgfqpoint{4.304790in}{1.122884in}}{\pgfqpoint{4.307624in}{1.129726in}}{\pgfqpoint{4.307624in}{1.136858in}}%
\pgfpathcurveto{\pgfqpoint{4.307624in}{1.143991in}}{\pgfqpoint{4.304790in}{1.150833in}}{\pgfqpoint{4.299747in}{1.155877in}}%
\pgfpathcurveto{\pgfqpoint{4.294703in}{1.160920in}}{\pgfqpoint{4.287861in}{1.163754in}}{\pgfqpoint{4.280729in}{1.163754in}}%
\pgfpathcurveto{\pgfqpoint{4.273596in}{1.163754in}}{\pgfqpoint{4.266754in}{1.160920in}}{\pgfqpoint{4.261710in}{1.155877in}}%
\pgfpathcurveto{\pgfqpoint{4.256667in}{1.150833in}}{\pgfqpoint{4.253833in}{1.143991in}}{\pgfqpoint{4.253833in}{1.136858in}}%
\pgfpathcurveto{\pgfqpoint{4.253833in}{1.129726in}}{\pgfqpoint{4.256667in}{1.122884in}}{\pgfqpoint{4.261710in}{1.117840in}}%
\pgfpathcurveto{\pgfqpoint{4.266754in}{1.112797in}}{\pgfqpoint{4.273596in}{1.109963in}}{\pgfqpoint{4.280729in}{1.109963in}}%
\pgfpathclose%
\pgfusepath{stroke,fill}%
\end{pgfscope}%
\begin{pgfscope}%
\pgfpathrectangle{\pgfqpoint{2.867647in}{0.500000in}}{\pgfqpoint{1.764706in}{1.700000in}}%
\pgfusepath{clip}%
\pgfsetbuttcap%
\pgfsetroundjoin%
\definecolor{currentfill}{rgb}{0.968931,0.798091,0.685123}%
\pgfsetfillcolor{currentfill}%
\pgfsetlinewidth{0.311001pt}%
\definecolor{currentstroke}{rgb}{1.000000,1.000000,1.000000}%
\pgfsetstrokecolor{currentstroke}%
\pgfsetdash{}{0pt}%
\pgfpathmoveto{\pgfqpoint{4.189974in}{1.006762in}}%
\pgfpathcurveto{\pgfqpoint{4.197107in}{1.006762in}}{\pgfqpoint{4.203949in}{1.009596in}}{\pgfqpoint{4.208992in}{1.014640in}}%
\pgfpathcurveto{\pgfqpoint{4.214036in}{1.019684in}}{\pgfqpoint{4.216870in}{1.026525in}}{\pgfqpoint{4.216870in}{1.033658in}}%
\pgfpathcurveto{\pgfqpoint{4.216870in}{1.040791in}}{\pgfqpoint{4.214036in}{1.047633in}}{\pgfqpoint{4.208992in}{1.052676in}}%
\pgfpathcurveto{\pgfqpoint{4.203949in}{1.057720in}}{\pgfqpoint{4.197107in}{1.060554in}}{\pgfqpoint{4.189974in}{1.060554in}}%
\pgfpathcurveto{\pgfqpoint{4.182841in}{1.060554in}}{\pgfqpoint{4.176000in}{1.057720in}}{\pgfqpoint{4.170956in}{1.052676in}}%
\pgfpathcurveto{\pgfqpoint{4.165912in}{1.047633in}}{\pgfqpoint{4.163078in}{1.040791in}}{\pgfqpoint{4.163078in}{1.033658in}}%
\pgfpathcurveto{\pgfqpoint{4.163078in}{1.026525in}}{\pgfqpoint{4.165912in}{1.019684in}}{\pgfqpoint{4.170956in}{1.014640in}}%
\pgfpathcurveto{\pgfqpoint{4.176000in}{1.009596in}}{\pgfqpoint{4.182841in}{1.006762in}}{\pgfqpoint{4.189974in}{1.006762in}}%
\pgfpathclose%
\pgfusepath{stroke,fill}%
\end{pgfscope}%
\begin{pgfscope}%
\pgfpathrectangle{\pgfqpoint{2.867647in}{0.500000in}}{\pgfqpoint{1.764706in}{1.700000in}}%
\pgfusepath{clip}%
\pgfsetbuttcap%
\pgfsetroundjoin%
\definecolor{currentfill}{rgb}{0.979124,0.903132,0.839793}%
\pgfsetfillcolor{currentfill}%
\pgfsetlinewidth{0.311001pt}%
\definecolor{currentstroke}{rgb}{1.000000,1.000000,1.000000}%
\pgfsetstrokecolor{currentstroke}%
\pgfsetdash{}{0pt}%
\pgfpathmoveto{\pgfqpoint{4.128843in}{1.108284in}}%
\pgfpathcurveto{\pgfqpoint{4.135976in}{1.108284in}}{\pgfqpoint{4.142817in}{1.111118in}}{\pgfqpoint{4.147861in}{1.116162in}}%
\pgfpathcurveto{\pgfqpoint{4.152905in}{1.121205in}}{\pgfqpoint{4.155739in}{1.128047in}}{\pgfqpoint{4.155739in}{1.135180in}}%
\pgfpathcurveto{\pgfqpoint{4.155739in}{1.142313in}}{\pgfqpoint{4.152905in}{1.149154in}}{\pgfqpoint{4.147861in}{1.154198in}}%
\pgfpathcurveto{\pgfqpoint{4.142817in}{1.159242in}}{\pgfqpoint{4.135976in}{1.162076in}}{\pgfqpoint{4.128843in}{1.162076in}}%
\pgfpathcurveto{\pgfqpoint{4.121710in}{1.162076in}}{\pgfqpoint{4.114868in}{1.159242in}}{\pgfqpoint{4.109825in}{1.154198in}}%
\pgfpathcurveto{\pgfqpoint{4.104781in}{1.149154in}}{\pgfqpoint{4.101947in}{1.142313in}}{\pgfqpoint{4.101947in}{1.135180in}}%
\pgfpathcurveto{\pgfqpoint{4.101947in}{1.128047in}}{\pgfqpoint{4.104781in}{1.121205in}}{\pgfqpoint{4.109825in}{1.116162in}}%
\pgfpathcurveto{\pgfqpoint{4.114868in}{1.111118in}}{\pgfqpoint{4.121710in}{1.108284in}}{\pgfqpoint{4.128843in}{1.108284in}}%
\pgfpathclose%
\pgfusepath{stroke,fill}%
\end{pgfscope}%
\begin{pgfscope}%
\pgfpathrectangle{\pgfqpoint{2.867647in}{0.500000in}}{\pgfqpoint{1.764706in}{1.700000in}}%
\pgfusepath{clip}%
\pgfsetbuttcap%
\pgfsetroundjoin%
\definecolor{currentfill}{rgb}{0.945204,0.390623,0.270949}%
\pgfsetfillcolor{currentfill}%
\pgfsetlinewidth{0.311001pt}%
\definecolor{currentstroke}{rgb}{1.000000,1.000000,1.000000}%
\pgfsetstrokecolor{currentstroke}%
\pgfsetdash{}{0pt}%
\pgfpathmoveto{\pgfqpoint{3.864782in}{1.663727in}}%
\pgfpathcurveto{\pgfqpoint{3.871915in}{1.663727in}}{\pgfqpoint{3.878756in}{1.666560in}}{\pgfqpoint{3.883800in}{1.671604in}}%
\pgfpathcurveto{\pgfqpoint{3.888844in}{1.676648in}}{\pgfqpoint{3.891677in}{1.683489in}}{\pgfqpoint{3.891677in}{1.690622in}}%
\pgfpathcurveto{\pgfqpoint{3.891677in}{1.697755in}}{\pgfqpoint{3.888844in}{1.704597in}}{\pgfqpoint{3.883800in}{1.709640in}}%
\pgfpathcurveto{\pgfqpoint{3.878756in}{1.714684in}}{\pgfqpoint{3.871915in}{1.717518in}}{\pgfqpoint{3.864782in}{1.717518in}}%
\pgfpathcurveto{\pgfqpoint{3.857649in}{1.717518in}}{\pgfqpoint{3.850807in}{1.714684in}}{\pgfqpoint{3.845764in}{1.709640in}}%
\pgfpathcurveto{\pgfqpoint{3.840720in}{1.704597in}}{\pgfqpoint{3.837886in}{1.697755in}}{\pgfqpoint{3.837886in}{1.690622in}}%
\pgfpathcurveto{\pgfqpoint{3.837886in}{1.683489in}}{\pgfqpoint{3.840720in}{1.676648in}}{\pgfqpoint{3.845764in}{1.671604in}}%
\pgfpathcurveto{\pgfqpoint{3.850807in}{1.666560in}}{\pgfqpoint{3.857649in}{1.663727in}}{\pgfqpoint{3.864782in}{1.663727in}}%
\pgfpathclose%
\pgfusepath{stroke,fill}%
\end{pgfscope}%
\begin{pgfscope}%
\pgfpathrectangle{\pgfqpoint{2.867647in}{0.500000in}}{\pgfqpoint{1.764706in}{1.700000in}}%
\pgfusepath{clip}%
\pgfsetbuttcap%
\pgfsetroundjoin%
\definecolor{currentfill}{rgb}{0.965753,0.732351,0.592427}%
\pgfsetfillcolor{currentfill}%
\pgfsetlinewidth{0.311001pt}%
\definecolor{currentstroke}{rgb}{1.000000,1.000000,1.000000}%
\pgfsetstrokecolor{currentstroke}%
\pgfsetdash{}{0pt}%
\pgfpathmoveto{\pgfqpoint{4.051010in}{0.901712in}}%
\pgfpathcurveto{\pgfqpoint{4.058143in}{0.901712in}}{\pgfqpoint{4.064984in}{0.904545in}}{\pgfqpoint{4.070028in}{0.909589in}}%
\pgfpathcurveto{\pgfqpoint{4.075072in}{0.914633in}}{\pgfqpoint{4.077906in}{0.921474in}}{\pgfqpoint{4.077906in}{0.928607in}}%
\pgfpathcurveto{\pgfqpoint{4.077906in}{0.935740in}}{\pgfqpoint{4.075072in}{0.942582in}}{\pgfqpoint{4.070028in}{0.947625in}}%
\pgfpathcurveto{\pgfqpoint{4.064984in}{0.952669in}}{\pgfqpoint{4.058143in}{0.955503in}}{\pgfqpoint{4.051010in}{0.955503in}}%
\pgfpathcurveto{\pgfqpoint{4.043877in}{0.955503in}}{\pgfqpoint{4.037035in}{0.952669in}}{\pgfqpoint{4.031992in}{0.947625in}}%
\pgfpathcurveto{\pgfqpoint{4.026948in}{0.942582in}}{\pgfqpoint{4.024114in}{0.935740in}}{\pgfqpoint{4.024114in}{0.928607in}}%
\pgfpathcurveto{\pgfqpoint{4.024114in}{0.921474in}}{\pgfqpoint{4.026948in}{0.914633in}}{\pgfqpoint{4.031992in}{0.909589in}}%
\pgfpathcurveto{\pgfqpoint{4.037035in}{0.904545in}}{\pgfqpoint{4.043877in}{0.901712in}}{\pgfqpoint{4.051010in}{0.901712in}}%
\pgfpathclose%
\pgfusepath{stroke,fill}%
\end{pgfscope}%
\begin{pgfscope}%
\pgfpathrectangle{\pgfqpoint{2.867647in}{0.500000in}}{\pgfqpoint{1.764706in}{1.700000in}}%
\pgfusepath{clip}%
\pgfsetbuttcap%
\pgfsetroundjoin%
\definecolor{currentfill}{rgb}{0.977657,0.891500,0.822809}%
\pgfsetfillcolor{currentfill}%
\pgfsetlinewidth{0.311001pt}%
\definecolor{currentstroke}{rgb}{1.000000,1.000000,1.000000}%
\pgfsetstrokecolor{currentstroke}%
\pgfsetdash{}{0pt}%
\pgfpathmoveto{\pgfqpoint{4.109643in}{1.599028in}}%
\pgfpathcurveto{\pgfqpoint{4.116776in}{1.599028in}}{\pgfqpoint{4.123617in}{1.601862in}}{\pgfqpoint{4.128661in}{1.606906in}}%
\pgfpathcurveto{\pgfqpoint{4.133705in}{1.611949in}}{\pgfqpoint{4.136539in}{1.618791in}}{\pgfqpoint{4.136539in}{1.625924in}}%
\pgfpathcurveto{\pgfqpoint{4.136539in}{1.633056in}}{\pgfqpoint{4.133705in}{1.639898in}}{\pgfqpoint{4.128661in}{1.644942in}}%
\pgfpathcurveto{\pgfqpoint{4.123617in}{1.649985in}}{\pgfqpoint{4.116776in}{1.652819in}}{\pgfqpoint{4.109643in}{1.652819in}}%
\pgfpathcurveto{\pgfqpoint{4.102510in}{1.652819in}}{\pgfqpoint{4.095668in}{1.649985in}}{\pgfqpoint{4.090625in}{1.644942in}}%
\pgfpathcurveto{\pgfqpoint{4.085581in}{1.639898in}}{\pgfqpoint{4.082747in}{1.633056in}}{\pgfqpoint{4.082747in}{1.625924in}}%
\pgfpathcurveto{\pgfqpoint{4.082747in}{1.618791in}}{\pgfqpoint{4.085581in}{1.611949in}}{\pgfqpoint{4.090625in}{1.606906in}}%
\pgfpathcurveto{\pgfqpoint{4.095668in}{1.601862in}}{\pgfqpoint{4.102510in}{1.599028in}}{\pgfqpoint{4.109643in}{1.599028in}}%
\pgfpathclose%
\pgfusepath{stroke,fill}%
\end{pgfscope}%
\begin{pgfscope}%
\pgfpathrectangle{\pgfqpoint{2.867647in}{0.500000in}}{\pgfqpoint{1.764706in}{1.700000in}}%
\pgfusepath{clip}%
\pgfsetbuttcap%
\pgfsetroundjoin%
\definecolor{currentfill}{rgb}{0.970255,0.815666,0.711203}%
\pgfsetfillcolor{currentfill}%
\pgfsetlinewidth{0.311001pt}%
\definecolor{currentstroke}{rgb}{1.000000,1.000000,1.000000}%
\pgfsetstrokecolor{currentstroke}%
\pgfsetdash{}{0pt}%
\pgfpathmoveto{\pgfqpoint{4.125261in}{0.961333in}}%
\pgfpathcurveto{\pgfqpoint{4.132394in}{0.961333in}}{\pgfqpoint{4.139235in}{0.964167in}}{\pgfqpoint{4.144279in}{0.969211in}}%
\pgfpathcurveto{\pgfqpoint{4.149323in}{0.974254in}}{\pgfqpoint{4.152157in}{0.981096in}}{\pgfqpoint{4.152157in}{0.988229in}}%
\pgfpathcurveto{\pgfqpoint{4.152157in}{0.995362in}}{\pgfqpoint{4.149323in}{1.002203in}}{\pgfqpoint{4.144279in}{1.007247in}}%
\pgfpathcurveto{\pgfqpoint{4.139235in}{1.012291in}}{\pgfqpoint{4.132394in}{1.015125in}}{\pgfqpoint{4.125261in}{1.015125in}}%
\pgfpathcurveto{\pgfqpoint{4.118128in}{1.015125in}}{\pgfqpoint{4.111286in}{1.012291in}}{\pgfqpoint{4.106243in}{1.007247in}}%
\pgfpathcurveto{\pgfqpoint{4.101199in}{1.002203in}}{\pgfqpoint{4.098365in}{0.995362in}}{\pgfqpoint{4.098365in}{0.988229in}}%
\pgfpathcurveto{\pgfqpoint{4.098365in}{0.981096in}}{\pgfqpoint{4.101199in}{0.974254in}}{\pgfqpoint{4.106243in}{0.969211in}}%
\pgfpathcurveto{\pgfqpoint{4.111286in}{0.964167in}}{\pgfqpoint{4.118128in}{0.961333in}}{\pgfqpoint{4.125261in}{0.961333in}}%
\pgfpathclose%
\pgfusepath{stroke,fill}%
\end{pgfscope}%
\begin{pgfscope}%
\pgfpathrectangle{\pgfqpoint{2.867647in}{0.500000in}}{\pgfqpoint{1.764706in}{1.700000in}}%
\pgfusepath{clip}%
\pgfsetbuttcap%
\pgfsetroundjoin%
\definecolor{currentfill}{rgb}{0.971202,0.827364,0.728520}%
\pgfsetfillcolor{currentfill}%
\pgfsetlinewidth{0.311001pt}%
\definecolor{currentstroke}{rgb}{1.000000,1.000000,1.000000}%
\pgfsetstrokecolor{currentstroke}%
\pgfsetdash{}{0pt}%
\pgfpathmoveto{\pgfqpoint{4.093205in}{1.182710in}}%
\pgfpathcurveto{\pgfqpoint{4.100338in}{1.182710in}}{\pgfqpoint{4.107179in}{1.185544in}}{\pgfqpoint{4.112223in}{1.190588in}}%
\pgfpathcurveto{\pgfqpoint{4.117267in}{1.195631in}}{\pgfqpoint{4.120101in}{1.202473in}}{\pgfqpoint{4.120101in}{1.209606in}}%
\pgfpathcurveto{\pgfqpoint{4.120101in}{1.216739in}}{\pgfqpoint{4.117267in}{1.223580in}}{\pgfqpoint{4.112223in}{1.228624in}}%
\pgfpathcurveto{\pgfqpoint{4.107179in}{1.233668in}}{\pgfqpoint{4.100338in}{1.236501in}}{\pgfqpoint{4.093205in}{1.236501in}}%
\pgfpathcurveto{\pgfqpoint{4.086072in}{1.236501in}}{\pgfqpoint{4.079230in}{1.233668in}}{\pgfqpoint{4.074187in}{1.228624in}}%
\pgfpathcurveto{\pgfqpoint{4.069143in}{1.223580in}}{\pgfqpoint{4.066309in}{1.216739in}}{\pgfqpoint{4.066309in}{1.209606in}}%
\pgfpathcurveto{\pgfqpoint{4.066309in}{1.202473in}}{\pgfqpoint{4.069143in}{1.195631in}}{\pgfqpoint{4.074187in}{1.190588in}}%
\pgfpathcurveto{\pgfqpoint{4.079230in}{1.185544in}}{\pgfqpoint{4.086072in}{1.182710in}}{\pgfqpoint{4.093205in}{1.182710in}}%
\pgfpathclose%
\pgfusepath{stroke,fill}%
\end{pgfscope}%
\begin{pgfscope}%
\pgfpathrectangle{\pgfqpoint{2.867647in}{0.500000in}}{\pgfqpoint{1.764706in}{1.700000in}}%
\pgfusepath{clip}%
\pgfsetbuttcap%
\pgfsetroundjoin%
\definecolor{currentfill}{rgb}{0.976287,0.879862,0.805788}%
\pgfsetfillcolor{currentfill}%
\pgfsetlinewidth{0.311001pt}%
\definecolor{currentstroke}{rgb}{1.000000,1.000000,1.000000}%
\pgfsetstrokecolor{currentstroke}%
\pgfsetdash{}{0pt}%
\pgfpathmoveto{\pgfqpoint{4.102472in}{1.516476in}}%
\pgfpathcurveto{\pgfqpoint{4.109605in}{1.516476in}}{\pgfqpoint{4.116446in}{1.519310in}}{\pgfqpoint{4.121490in}{1.524354in}}%
\pgfpathcurveto{\pgfqpoint{4.126534in}{1.529398in}}{\pgfqpoint{4.129368in}{1.536239in}}{\pgfqpoint{4.129368in}{1.543372in}}%
\pgfpathcurveto{\pgfqpoint{4.129368in}{1.550505in}}{\pgfqpoint{4.126534in}{1.557347in}}{\pgfqpoint{4.121490in}{1.562390in}}%
\pgfpathcurveto{\pgfqpoint{4.116446in}{1.567434in}}{\pgfqpoint{4.109605in}{1.570268in}}{\pgfqpoint{4.102472in}{1.570268in}}%
\pgfpathcurveto{\pgfqpoint{4.095339in}{1.570268in}}{\pgfqpoint{4.088498in}{1.567434in}}{\pgfqpoint{4.083454in}{1.562390in}}%
\pgfpathcurveto{\pgfqpoint{4.078410in}{1.557347in}}{\pgfqpoint{4.075576in}{1.550505in}}{\pgfqpoint{4.075576in}{1.543372in}}%
\pgfpathcurveto{\pgfqpoint{4.075576in}{1.536239in}}{\pgfqpoint{4.078410in}{1.529398in}}{\pgfqpoint{4.083454in}{1.524354in}}%
\pgfpathcurveto{\pgfqpoint{4.088498in}{1.519310in}}{\pgfqpoint{4.095339in}{1.516476in}}{\pgfqpoint{4.102472in}{1.516476in}}%
\pgfpathclose%
\pgfusepath{stroke,fill}%
\end{pgfscope}%
\begin{pgfscope}%
\pgfpathrectangle{\pgfqpoint{2.867647in}{0.500000in}}{\pgfqpoint{1.764706in}{1.700000in}}%
\pgfusepath{clip}%
\pgfsetbuttcap%
\pgfsetroundjoin%
\definecolor{currentfill}{rgb}{0.971694,0.833208,0.737161}%
\pgfsetfillcolor{currentfill}%
\pgfsetlinewidth{0.311001pt}%
\definecolor{currentstroke}{rgb}{1.000000,1.000000,1.000000}%
\pgfsetstrokecolor{currentstroke}%
\pgfsetdash{}{0pt}%
\pgfpathmoveto{\pgfqpoint{4.081353in}{1.679036in}}%
\pgfpathcurveto{\pgfqpoint{4.088486in}{1.679036in}}{\pgfqpoint{4.095328in}{1.681870in}}{\pgfqpoint{4.100371in}{1.686914in}}%
\pgfpathcurveto{\pgfqpoint{4.105415in}{1.691958in}}{\pgfqpoint{4.108249in}{1.698799in}}{\pgfqpoint{4.108249in}{1.705932in}}%
\pgfpathcurveto{\pgfqpoint{4.108249in}{1.713065in}}{\pgfqpoint{4.105415in}{1.719907in}}{\pgfqpoint{4.100371in}{1.724950in}}%
\pgfpathcurveto{\pgfqpoint{4.095328in}{1.729994in}}{\pgfqpoint{4.088486in}{1.732828in}}{\pgfqpoint{4.081353in}{1.732828in}}%
\pgfpathcurveto{\pgfqpoint{4.074220in}{1.732828in}}{\pgfqpoint{4.067379in}{1.729994in}}{\pgfqpoint{4.062335in}{1.724950in}}%
\pgfpathcurveto{\pgfqpoint{4.057291in}{1.719907in}}{\pgfqpoint{4.054458in}{1.713065in}}{\pgfqpoint{4.054458in}{1.705932in}}%
\pgfpathcurveto{\pgfqpoint{4.054458in}{1.698799in}}{\pgfqpoint{4.057291in}{1.691958in}}{\pgfqpoint{4.062335in}{1.686914in}}%
\pgfpathcurveto{\pgfqpoint{4.067379in}{1.681870in}}{\pgfqpoint{4.074220in}{1.679036in}}{\pgfqpoint{4.081353in}{1.679036in}}%
\pgfpathclose%
\pgfusepath{stroke,fill}%
\end{pgfscope}%
\begin{pgfscope}%
\pgfpathrectangle{\pgfqpoint{2.867647in}{0.500000in}}{\pgfqpoint{1.764706in}{1.700000in}}%
\pgfusepath{clip}%
\pgfsetbuttcap%
\pgfsetroundjoin%
\definecolor{currentfill}{rgb}{0.977657,0.891500,0.822809}%
\pgfsetfillcolor{currentfill}%
\pgfsetlinewidth{0.311001pt}%
\definecolor{currentstroke}{rgb}{1.000000,1.000000,1.000000}%
\pgfsetstrokecolor{currentstroke}%
\pgfsetdash{}{0pt}%
\pgfpathmoveto{\pgfqpoint{4.149015in}{1.355692in}}%
\pgfpathcurveto{\pgfqpoint{4.156148in}{1.355692in}}{\pgfqpoint{4.162990in}{1.358526in}}{\pgfqpoint{4.168033in}{1.363570in}}%
\pgfpathcurveto{\pgfqpoint{4.173077in}{1.368613in}}{\pgfqpoint{4.175911in}{1.375455in}}{\pgfqpoint{4.175911in}{1.382588in}}%
\pgfpathcurveto{\pgfqpoint{4.175911in}{1.389721in}}{\pgfqpoint{4.173077in}{1.396562in}}{\pgfqpoint{4.168033in}{1.401606in}}%
\pgfpathcurveto{\pgfqpoint{4.162990in}{1.406649in}}{\pgfqpoint{4.156148in}{1.409483in}}{\pgfqpoint{4.149015in}{1.409483in}}%
\pgfpathcurveto{\pgfqpoint{4.141882in}{1.409483in}}{\pgfqpoint{4.135041in}{1.406649in}}{\pgfqpoint{4.129997in}{1.401606in}}%
\pgfpathcurveto{\pgfqpoint{4.124953in}{1.396562in}}{\pgfqpoint{4.122120in}{1.389721in}}{\pgfqpoint{4.122120in}{1.382588in}}%
\pgfpathcurveto{\pgfqpoint{4.122120in}{1.375455in}}{\pgfqpoint{4.124953in}{1.368613in}}{\pgfqpoint{4.129997in}{1.363570in}}%
\pgfpathcurveto{\pgfqpoint{4.135041in}{1.358526in}}{\pgfqpoint{4.141882in}{1.355692in}}{\pgfqpoint{4.149015in}{1.355692in}}%
\pgfpathclose%
\pgfusepath{stroke,fill}%
\end{pgfscope}%
\begin{pgfscope}%
\pgfpathrectangle{\pgfqpoint{2.867647in}{0.500000in}}{\pgfqpoint{1.764706in}{1.700000in}}%
\pgfusepath{clip}%
\pgfsetbuttcap%
\pgfsetroundjoin%
\definecolor{currentfill}{rgb}{0.963884,0.644842,0.486120}%
\pgfsetfillcolor{currentfill}%
\pgfsetlinewidth{0.311001pt}%
\definecolor{currentstroke}{rgb}{1.000000,1.000000,1.000000}%
\pgfsetstrokecolor{currentstroke}%
\pgfsetdash{}{0pt}%
\pgfpathmoveto{\pgfqpoint{4.066867in}{1.789128in}}%
\pgfpathcurveto{\pgfqpoint{4.074000in}{1.789128in}}{\pgfqpoint{4.080841in}{1.791962in}}{\pgfqpoint{4.085885in}{1.797005in}}%
\pgfpathcurveto{\pgfqpoint{4.090929in}{1.802049in}}{\pgfqpoint{4.093763in}{1.808891in}}{\pgfqpoint{4.093763in}{1.816023in}}%
\pgfpathcurveto{\pgfqpoint{4.093763in}{1.823156in}}{\pgfqpoint{4.090929in}{1.829998in}}{\pgfqpoint{4.085885in}{1.835042in}}%
\pgfpathcurveto{\pgfqpoint{4.080841in}{1.840085in}}{\pgfqpoint{4.074000in}{1.842919in}}{\pgfqpoint{4.066867in}{1.842919in}}%
\pgfpathcurveto{\pgfqpoint{4.059734in}{1.842919in}}{\pgfqpoint{4.052893in}{1.840085in}}{\pgfqpoint{4.047849in}{1.835042in}}%
\pgfpathcurveto{\pgfqpoint{4.042805in}{1.829998in}}{\pgfqpoint{4.039971in}{1.823156in}}{\pgfqpoint{4.039971in}{1.816023in}}%
\pgfpathcurveto{\pgfqpoint{4.039971in}{1.808891in}}{\pgfqpoint{4.042805in}{1.802049in}}{\pgfqpoint{4.047849in}{1.797005in}}%
\pgfpathcurveto{\pgfqpoint{4.052893in}{1.791962in}}{\pgfqpoint{4.059734in}{1.789128in}}{\pgfqpoint{4.066867in}{1.789128in}}%
\pgfpathclose%
\pgfusepath{stroke,fill}%
\end{pgfscope}%
\begin{pgfscope}%
\pgfpathrectangle{\pgfqpoint{2.867647in}{0.500000in}}{\pgfqpoint{1.764706in}{1.700000in}}%
\pgfusepath{clip}%
\pgfsetbuttcap%
\pgfsetroundjoin%
\definecolor{currentfill}{rgb}{0.977657,0.891500,0.822809}%
\pgfsetfillcolor{currentfill}%
\pgfsetlinewidth{0.311001pt}%
\definecolor{currentstroke}{rgb}{1.000000,1.000000,1.000000}%
\pgfsetstrokecolor{currentstroke}%
\pgfsetdash{}{0pt}%
\pgfpathmoveto{\pgfqpoint{4.227534in}{1.396939in}}%
\pgfpathcurveto{\pgfqpoint{4.234667in}{1.396939in}}{\pgfqpoint{4.241509in}{1.399773in}}{\pgfqpoint{4.246552in}{1.404816in}}%
\pgfpathcurveto{\pgfqpoint{4.251596in}{1.409860in}}{\pgfqpoint{4.254430in}{1.416702in}}{\pgfqpoint{4.254430in}{1.423834in}}%
\pgfpathcurveto{\pgfqpoint{4.254430in}{1.430967in}}{\pgfqpoint{4.251596in}{1.437809in}}{\pgfqpoint{4.246552in}{1.442853in}}%
\pgfpathcurveto{\pgfqpoint{4.241509in}{1.447896in}}{\pgfqpoint{4.234667in}{1.450730in}}{\pgfqpoint{4.227534in}{1.450730in}}%
\pgfpathcurveto{\pgfqpoint{4.220401in}{1.450730in}}{\pgfqpoint{4.213560in}{1.447896in}}{\pgfqpoint{4.208516in}{1.442853in}}%
\pgfpathcurveto{\pgfqpoint{4.203472in}{1.437809in}}{\pgfqpoint{4.200638in}{1.430967in}}{\pgfqpoint{4.200638in}{1.423834in}}%
\pgfpathcurveto{\pgfqpoint{4.200638in}{1.416702in}}{\pgfqpoint{4.203472in}{1.409860in}}{\pgfqpoint{4.208516in}{1.404816in}}%
\pgfpathcurveto{\pgfqpoint{4.213560in}{1.399773in}}{\pgfqpoint{4.220401in}{1.396939in}}{\pgfqpoint{4.227534in}{1.396939in}}%
\pgfpathclose%
\pgfusepath{stroke,fill}%
\end{pgfscope}%
\begin{pgfscope}%
\pgfpathrectangle{\pgfqpoint{2.867647in}{0.500000in}}{\pgfqpoint{1.764706in}{1.700000in}}%
\pgfusepath{clip}%
\pgfsetbuttcap%
\pgfsetroundjoin%
\definecolor{currentfill}{rgb}{0.965440,0.720101,0.576404}%
\pgfsetfillcolor{currentfill}%
\pgfsetlinewidth{0.311001pt}%
\definecolor{currentstroke}{rgb}{1.000000,1.000000,1.000000}%
\pgfsetstrokecolor{currentstroke}%
\pgfsetdash{}{0pt}%
\pgfpathmoveto{\pgfqpoint{4.054686in}{1.454945in}}%
\pgfpathcurveto{\pgfqpoint{4.061818in}{1.454945in}}{\pgfqpoint{4.068660in}{1.457779in}}{\pgfqpoint{4.073704in}{1.462823in}}%
\pgfpathcurveto{\pgfqpoint{4.078747in}{1.467866in}}{\pgfqpoint{4.081581in}{1.474708in}}{\pgfqpoint{4.081581in}{1.481841in}}%
\pgfpathcurveto{\pgfqpoint{4.081581in}{1.488974in}}{\pgfqpoint{4.078747in}{1.495815in}}{\pgfqpoint{4.073704in}{1.500859in}}%
\pgfpathcurveto{\pgfqpoint{4.068660in}{1.505903in}}{\pgfqpoint{4.061818in}{1.508736in}}{\pgfqpoint{4.054686in}{1.508736in}}%
\pgfpathcurveto{\pgfqpoint{4.047553in}{1.508736in}}{\pgfqpoint{4.040711in}{1.505903in}}{\pgfqpoint{4.035667in}{1.500859in}}%
\pgfpathcurveto{\pgfqpoint{4.030624in}{1.495815in}}{\pgfqpoint{4.027790in}{1.488974in}}{\pgfqpoint{4.027790in}{1.481841in}}%
\pgfpathcurveto{\pgfqpoint{4.027790in}{1.474708in}}{\pgfqpoint{4.030624in}{1.467866in}}{\pgfqpoint{4.035667in}{1.462823in}}%
\pgfpathcurveto{\pgfqpoint{4.040711in}{1.457779in}}{\pgfqpoint{4.047553in}{1.454945in}}{\pgfqpoint{4.054686in}{1.454945in}}%
\pgfpathclose%
\pgfusepath{stroke,fill}%
\end{pgfscope}%
\begin{pgfscope}%
\pgfpathrectangle{\pgfqpoint{2.867647in}{0.500000in}}{\pgfqpoint{1.764706in}{1.700000in}}%
\pgfusepath{clip}%
\pgfsetbuttcap%
\pgfsetroundjoin%
\definecolor{currentfill}{rgb}{0.979124,0.903132,0.839793}%
\pgfsetfillcolor{currentfill}%
\pgfsetlinewidth{0.311001pt}%
\definecolor{currentstroke}{rgb}{1.000000,1.000000,1.000000}%
\pgfsetstrokecolor{currentstroke}%
\pgfsetdash{}{0pt}%
\pgfpathmoveto{\pgfqpoint{4.141196in}{1.212924in}}%
\pgfpathcurveto{\pgfqpoint{4.148328in}{1.212924in}}{\pgfqpoint{4.155170in}{1.215758in}}{\pgfqpoint{4.160214in}{1.220802in}}%
\pgfpathcurveto{\pgfqpoint{4.165257in}{1.225845in}}{\pgfqpoint{4.168091in}{1.232687in}}{\pgfqpoint{4.168091in}{1.239820in}}%
\pgfpathcurveto{\pgfqpoint{4.168091in}{1.246953in}}{\pgfqpoint{4.165257in}{1.253794in}}{\pgfqpoint{4.160214in}{1.258838in}}%
\pgfpathcurveto{\pgfqpoint{4.155170in}{1.263882in}}{\pgfqpoint{4.148328in}{1.266716in}}{\pgfqpoint{4.141196in}{1.266716in}}%
\pgfpathcurveto{\pgfqpoint{4.134063in}{1.266716in}}{\pgfqpoint{4.127221in}{1.263882in}}{\pgfqpoint{4.122177in}{1.258838in}}%
\pgfpathcurveto{\pgfqpoint{4.117134in}{1.253794in}}{\pgfqpoint{4.114300in}{1.246953in}}{\pgfqpoint{4.114300in}{1.239820in}}%
\pgfpathcurveto{\pgfqpoint{4.114300in}{1.232687in}}{\pgfqpoint{4.117134in}{1.225845in}}{\pgfqpoint{4.122177in}{1.220802in}}%
\pgfpathcurveto{\pgfqpoint{4.127221in}{1.215758in}}{\pgfqpoint{4.134063in}{1.212924in}}{\pgfqpoint{4.141196in}{1.212924in}}%
\pgfpathclose%
\pgfusepath{stroke,fill}%
\end{pgfscope}%
\begin{pgfscope}%
\pgfpathrectangle{\pgfqpoint{2.867647in}{0.500000in}}{\pgfqpoint{1.764706in}{1.700000in}}%
\pgfusepath{clip}%
\pgfsetbuttcap%
\pgfsetroundjoin%
\definecolor{currentfill}{rgb}{0.972201,0.839051,0.745789}%
\pgfsetfillcolor{currentfill}%
\pgfsetlinewidth{0.311001pt}%
\definecolor{currentstroke}{rgb}{1.000000,1.000000,1.000000}%
\pgfsetstrokecolor{currentstroke}%
\pgfsetdash{}{0pt}%
\pgfpathmoveto{\pgfqpoint{4.254273in}{1.199614in}}%
\pgfpathcurveto{\pgfqpoint{4.261406in}{1.199614in}}{\pgfqpoint{4.268247in}{1.202448in}}{\pgfqpoint{4.273291in}{1.207492in}}%
\pgfpathcurveto{\pgfqpoint{4.278335in}{1.212535in}}{\pgfqpoint{4.281169in}{1.219377in}}{\pgfqpoint{4.281169in}{1.226510in}}%
\pgfpathcurveto{\pgfqpoint{4.281169in}{1.233643in}}{\pgfqpoint{4.278335in}{1.240484in}}{\pgfqpoint{4.273291in}{1.245528in}}%
\pgfpathcurveto{\pgfqpoint{4.268247in}{1.250572in}}{\pgfqpoint{4.261406in}{1.253406in}}{\pgfqpoint{4.254273in}{1.253406in}}%
\pgfpathcurveto{\pgfqpoint{4.247140in}{1.253406in}}{\pgfqpoint{4.240299in}{1.250572in}}{\pgfqpoint{4.235255in}{1.245528in}}%
\pgfpathcurveto{\pgfqpoint{4.230211in}{1.240484in}}{\pgfqpoint{4.227377in}{1.233643in}}{\pgfqpoint{4.227377in}{1.226510in}}%
\pgfpathcurveto{\pgfqpoint{4.227377in}{1.219377in}}{\pgfqpoint{4.230211in}{1.212535in}}{\pgfqpoint{4.235255in}{1.207492in}}%
\pgfpathcurveto{\pgfqpoint{4.240299in}{1.202448in}}{\pgfqpoint{4.247140in}{1.199614in}}{\pgfqpoint{4.254273in}{1.199614in}}%
\pgfpathclose%
\pgfusepath{stroke,fill}%
\end{pgfscope}%
\begin{pgfscope}%
\pgfpathrectangle{\pgfqpoint{2.867647in}{0.500000in}}{\pgfqpoint{1.764706in}{1.700000in}}%
\pgfusepath{clip}%
\pgfsetbuttcap%
\pgfsetroundjoin%
\definecolor{currentfill}{rgb}{0.964799,0.689101,0.537560}%
\pgfsetfillcolor{currentfill}%
\pgfsetlinewidth{0.311001pt}%
\definecolor{currentstroke}{rgb}{1.000000,1.000000,1.000000}%
\pgfsetstrokecolor{currentstroke}%
\pgfsetdash{}{0pt}%
\pgfpathmoveto{\pgfqpoint{4.007771in}{1.571396in}}%
\pgfpathcurveto{\pgfqpoint{4.014904in}{1.571396in}}{\pgfqpoint{4.021745in}{1.574230in}}{\pgfqpoint{4.026789in}{1.579273in}}%
\pgfpathcurveto{\pgfqpoint{4.031833in}{1.584317in}}{\pgfqpoint{4.034666in}{1.591159in}}{\pgfqpoint{4.034666in}{1.598291in}}%
\pgfpathcurveto{\pgfqpoint{4.034666in}{1.605424in}}{\pgfqpoint{4.031833in}{1.612266in}}{\pgfqpoint{4.026789in}{1.617310in}}%
\pgfpathcurveto{\pgfqpoint{4.021745in}{1.622353in}}{\pgfqpoint{4.014904in}{1.625187in}}{\pgfqpoint{4.007771in}{1.625187in}}%
\pgfpathcurveto{\pgfqpoint{4.000638in}{1.625187in}}{\pgfqpoint{3.993796in}{1.622353in}}{\pgfqpoint{3.988753in}{1.617310in}}%
\pgfpathcurveto{\pgfqpoint{3.983709in}{1.612266in}}{\pgfqpoint{3.980875in}{1.605424in}}{\pgfqpoint{3.980875in}{1.598291in}}%
\pgfpathcurveto{\pgfqpoint{3.980875in}{1.591159in}}{\pgfqpoint{3.983709in}{1.584317in}}{\pgfqpoint{3.988753in}{1.579273in}}%
\pgfpathcurveto{\pgfqpoint{3.993796in}{1.574230in}}{\pgfqpoint{4.000638in}{1.571396in}}{\pgfqpoint{4.007771in}{1.571396in}}%
\pgfpathclose%
\pgfusepath{stroke,fill}%
\end{pgfscope}%
\begin{pgfscope}%
\pgfpathrectangle{\pgfqpoint{2.867647in}{0.500000in}}{\pgfqpoint{1.764706in}{1.700000in}}%
\pgfusepath{clip}%
\pgfsetbuttcap%
\pgfsetroundjoin%
\definecolor{currentfill}{rgb}{0.973271,0.850724,0.762998}%
\pgfsetfillcolor{currentfill}%
\pgfsetlinewidth{0.311001pt}%
\definecolor{currentstroke}{rgb}{1.000000,1.000000,1.000000}%
\pgfsetstrokecolor{currentstroke}%
\pgfsetdash{}{0pt}%
\pgfpathmoveto{\pgfqpoint{4.123161in}{1.282893in}}%
\pgfpathcurveto{\pgfqpoint{4.130294in}{1.282893in}}{\pgfqpoint{4.137136in}{1.285727in}}{\pgfqpoint{4.142179in}{1.290770in}}%
\pgfpathcurveto{\pgfqpoint{4.147223in}{1.295814in}}{\pgfqpoint{4.150057in}{1.302655in}}{\pgfqpoint{4.150057in}{1.309788in}}%
\pgfpathcurveto{\pgfqpoint{4.150057in}{1.316921in}}{\pgfqpoint{4.147223in}{1.323763in}}{\pgfqpoint{4.142179in}{1.328806in}}%
\pgfpathcurveto{\pgfqpoint{4.137136in}{1.333850in}}{\pgfqpoint{4.130294in}{1.336684in}}{\pgfqpoint{4.123161in}{1.336684in}}%
\pgfpathcurveto{\pgfqpoint{4.116028in}{1.336684in}}{\pgfqpoint{4.109187in}{1.333850in}}{\pgfqpoint{4.104143in}{1.328806in}}%
\pgfpathcurveto{\pgfqpoint{4.099099in}{1.323763in}}{\pgfqpoint{4.096266in}{1.316921in}}{\pgfqpoint{4.096266in}{1.309788in}}%
\pgfpathcurveto{\pgfqpoint{4.096266in}{1.302655in}}{\pgfqpoint{4.099099in}{1.295814in}}{\pgfqpoint{4.104143in}{1.290770in}}%
\pgfpathcurveto{\pgfqpoint{4.109187in}{1.285727in}}{\pgfqpoint{4.116028in}{1.282893in}}{\pgfqpoint{4.123161in}{1.282893in}}%
\pgfpathclose%
\pgfusepath{stroke,fill}%
\end{pgfscope}%
\begin{pgfscope}%
\pgfpathrectangle{\pgfqpoint{2.867647in}{0.500000in}}{\pgfqpoint{1.764706in}{1.700000in}}%
\pgfusepath{clip}%
\pgfsetbuttcap%
\pgfsetroundjoin%
\definecolor{currentfill}{rgb}{0.979124,0.903132,0.839793}%
\pgfsetfillcolor{currentfill}%
\pgfsetlinewidth{0.311001pt}%
\definecolor{currentstroke}{rgb}{1.000000,1.000000,1.000000}%
\pgfsetstrokecolor{currentstroke}%
\pgfsetdash{}{0pt}%
\pgfpathmoveto{\pgfqpoint{4.224448in}{1.291320in}}%
\pgfpathcurveto{\pgfqpoint{4.231581in}{1.291320in}}{\pgfqpoint{4.238422in}{1.294154in}}{\pgfqpoint{4.243466in}{1.299198in}}%
\pgfpathcurveto{\pgfqpoint{4.248510in}{1.304241in}}{\pgfqpoint{4.251343in}{1.311083in}}{\pgfqpoint{4.251343in}{1.318216in}}%
\pgfpathcurveto{\pgfqpoint{4.251343in}{1.325349in}}{\pgfqpoint{4.248510in}{1.332190in}}{\pgfqpoint{4.243466in}{1.337234in}}%
\pgfpathcurveto{\pgfqpoint{4.238422in}{1.342278in}}{\pgfqpoint{4.231581in}{1.345112in}}{\pgfqpoint{4.224448in}{1.345112in}}%
\pgfpathcurveto{\pgfqpoint{4.217315in}{1.345112in}}{\pgfqpoint{4.210473in}{1.342278in}}{\pgfqpoint{4.205430in}{1.337234in}}%
\pgfpathcurveto{\pgfqpoint{4.200386in}{1.332190in}}{\pgfqpoint{4.197552in}{1.325349in}}{\pgfqpoint{4.197552in}{1.318216in}}%
\pgfpathcurveto{\pgfqpoint{4.197552in}{1.311083in}}{\pgfqpoint{4.200386in}{1.304241in}}{\pgfqpoint{4.205430in}{1.299198in}}%
\pgfpathcurveto{\pgfqpoint{4.210473in}{1.294154in}}{\pgfqpoint{4.217315in}{1.291320in}}{\pgfqpoint{4.224448in}{1.291320in}}%
\pgfpathclose%
\pgfusepath{stroke,fill}%
\end{pgfscope}%
\begin{pgfscope}%
\pgfpathrectangle{\pgfqpoint{2.867647in}{0.500000in}}{\pgfqpoint{1.764706in}{1.700000in}}%
\pgfusepath{clip}%
\pgfsetbuttcap%
\pgfsetroundjoin%
\definecolor{currentfill}{rgb}{0.973271,0.850724,0.762998}%
\pgfsetfillcolor{currentfill}%
\pgfsetlinewidth{0.311001pt}%
\definecolor{currentstroke}{rgb}{1.000000,1.000000,1.000000}%
\pgfsetstrokecolor{currentstroke}%
\pgfsetdash{}{0pt}%
\pgfpathmoveto{\pgfqpoint{4.086158in}{1.118711in}}%
\pgfpathcurveto{\pgfqpoint{4.093291in}{1.118711in}}{\pgfqpoint{4.100132in}{1.121545in}}{\pgfqpoint{4.105176in}{1.126589in}}%
\pgfpathcurveto{\pgfqpoint{4.110220in}{1.131633in}}{\pgfqpoint{4.113054in}{1.138474in}}{\pgfqpoint{4.113054in}{1.145607in}}%
\pgfpathcurveto{\pgfqpoint{4.113054in}{1.152740in}}{\pgfqpoint{4.110220in}{1.159582in}}{\pgfqpoint{4.105176in}{1.164625in}}%
\pgfpathcurveto{\pgfqpoint{4.100132in}{1.169669in}}{\pgfqpoint{4.093291in}{1.172503in}}{\pgfqpoint{4.086158in}{1.172503in}}%
\pgfpathcurveto{\pgfqpoint{4.079025in}{1.172503in}}{\pgfqpoint{4.072183in}{1.169669in}}{\pgfqpoint{4.067140in}{1.164625in}}%
\pgfpathcurveto{\pgfqpoint{4.062096in}{1.159582in}}{\pgfqpoint{4.059262in}{1.152740in}}{\pgfqpoint{4.059262in}{1.145607in}}%
\pgfpathcurveto{\pgfqpoint{4.059262in}{1.138474in}}{\pgfqpoint{4.062096in}{1.131633in}}{\pgfqpoint{4.067140in}{1.126589in}}%
\pgfpathcurveto{\pgfqpoint{4.072183in}{1.121545in}}{\pgfqpoint{4.079025in}{1.118711in}}{\pgfqpoint{4.086158in}{1.118711in}}%
\pgfpathclose%
\pgfusepath{stroke,fill}%
\end{pgfscope}%
\begin{pgfscope}%
\pgfpathrectangle{\pgfqpoint{2.867647in}{0.500000in}}{\pgfqpoint{1.764706in}{1.700000in}}%
\pgfusepath{clip}%
\pgfsetbuttcap%
\pgfsetroundjoin%
\definecolor{currentfill}{rgb}{0.964032,0.651225,0.493258}%
\pgfsetfillcolor{currentfill}%
\pgfsetlinewidth{0.311001pt}%
\definecolor{currentstroke}{rgb}{1.000000,1.000000,1.000000}%
\pgfsetstrokecolor{currentstroke}%
\pgfsetdash{}{0pt}%
\pgfpathmoveto{\pgfqpoint{4.043746in}{1.788851in}}%
\pgfpathcurveto{\pgfqpoint{4.050879in}{1.788851in}}{\pgfqpoint{4.057721in}{1.791685in}}{\pgfqpoint{4.062764in}{1.796729in}}%
\pgfpathcurveto{\pgfqpoint{4.067808in}{1.801772in}}{\pgfqpoint{4.070642in}{1.808614in}}{\pgfqpoint{4.070642in}{1.815747in}}%
\pgfpathcurveto{\pgfqpoint{4.070642in}{1.822880in}}{\pgfqpoint{4.067808in}{1.829721in}}{\pgfqpoint{4.062764in}{1.834765in}}%
\pgfpathcurveto{\pgfqpoint{4.057721in}{1.839809in}}{\pgfqpoint{4.050879in}{1.842643in}}{\pgfqpoint{4.043746in}{1.842643in}}%
\pgfpathcurveto{\pgfqpoint{4.036613in}{1.842643in}}{\pgfqpoint{4.029772in}{1.839809in}}{\pgfqpoint{4.024728in}{1.834765in}}%
\pgfpathcurveto{\pgfqpoint{4.019684in}{1.829721in}}{\pgfqpoint{4.016851in}{1.822880in}}{\pgfqpoint{4.016851in}{1.815747in}}%
\pgfpathcurveto{\pgfqpoint{4.016851in}{1.808614in}}{\pgfqpoint{4.019684in}{1.801772in}}{\pgfqpoint{4.024728in}{1.796729in}}%
\pgfpathcurveto{\pgfqpoint{4.029772in}{1.791685in}}{\pgfqpoint{4.036613in}{1.788851in}}{\pgfqpoint{4.043746in}{1.788851in}}%
\pgfpathclose%
\pgfusepath{stroke,fill}%
\end{pgfscope}%
\begin{pgfscope}%
\pgfpathrectangle{\pgfqpoint{2.867647in}{0.500000in}}{\pgfqpoint{1.764706in}{1.700000in}}%
\pgfusepath{clip}%
\pgfsetbuttcap%
\pgfsetroundjoin%
\definecolor{currentfill}{rgb}{0.978376,0.897317,0.831308}%
\pgfsetfillcolor{currentfill}%
\pgfsetlinewidth{0.311001pt}%
\definecolor{currentstroke}{rgb}{1.000000,1.000000,1.000000}%
\pgfsetstrokecolor{currentstroke}%
\pgfsetdash{}{0pt}%
\pgfpathmoveto{\pgfqpoint{4.180737in}{1.546126in}}%
\pgfpathcurveto{\pgfqpoint{4.187869in}{1.546126in}}{\pgfqpoint{4.194711in}{1.548959in}}{\pgfqpoint{4.199755in}{1.554003in}}%
\pgfpathcurveto{\pgfqpoint{4.204798in}{1.559047in}}{\pgfqpoint{4.207632in}{1.565888in}}{\pgfqpoint{4.207632in}{1.573021in}}%
\pgfpathcurveto{\pgfqpoint{4.207632in}{1.580154in}}{\pgfqpoint{4.204798in}{1.586996in}}{\pgfqpoint{4.199755in}{1.592039in}}%
\pgfpathcurveto{\pgfqpoint{4.194711in}{1.597083in}}{\pgfqpoint{4.187869in}{1.599917in}}{\pgfqpoint{4.180737in}{1.599917in}}%
\pgfpathcurveto{\pgfqpoint{4.173604in}{1.599917in}}{\pgfqpoint{4.166762in}{1.597083in}}{\pgfqpoint{4.161718in}{1.592039in}}%
\pgfpathcurveto{\pgfqpoint{4.156675in}{1.586996in}}{\pgfqpoint{4.153841in}{1.580154in}}{\pgfqpoint{4.153841in}{1.573021in}}%
\pgfpathcurveto{\pgfqpoint{4.153841in}{1.565888in}}{\pgfqpoint{4.156675in}{1.559047in}}{\pgfqpoint{4.161718in}{1.554003in}}%
\pgfpathcurveto{\pgfqpoint{4.166762in}{1.548959in}}{\pgfqpoint{4.173604in}{1.546126in}}{\pgfqpoint{4.180737in}{1.546126in}}%
\pgfpathclose%
\pgfusepath{stroke,fill}%
\end{pgfscope}%
\begin{pgfscope}%
\pgfpathrectangle{\pgfqpoint{2.867647in}{0.500000in}}{\pgfqpoint{1.764706in}{1.700000in}}%
\pgfusepath{clip}%
\pgfsetbuttcap%
\pgfsetroundjoin%
\definecolor{currentfill}{rgb}{0.960778,0.559972,0.399412}%
\pgfsetfillcolor{currentfill}%
\pgfsetlinewidth{0.311001pt}%
\definecolor{currentstroke}{rgb}{1.000000,1.000000,1.000000}%
\pgfsetstrokecolor{currentstroke}%
\pgfsetdash{}{0pt}%
\pgfpathmoveto{\pgfqpoint{3.973358in}{1.577723in}}%
\pgfpathcurveto{\pgfqpoint{3.980491in}{1.577723in}}{\pgfqpoint{3.987332in}{1.580557in}}{\pgfqpoint{3.992376in}{1.585601in}}%
\pgfpathcurveto{\pgfqpoint{3.997420in}{1.590644in}}{\pgfqpoint{4.000254in}{1.597486in}}{\pgfqpoint{4.000254in}{1.604619in}}%
\pgfpathcurveto{\pgfqpoint{4.000254in}{1.611752in}}{\pgfqpoint{3.997420in}{1.618593in}}{\pgfqpoint{3.992376in}{1.623637in}}%
\pgfpathcurveto{\pgfqpoint{3.987332in}{1.628681in}}{\pgfqpoint{3.980491in}{1.631515in}}{\pgfqpoint{3.973358in}{1.631515in}}%
\pgfpathcurveto{\pgfqpoint{3.966225in}{1.631515in}}{\pgfqpoint{3.959384in}{1.628681in}}{\pgfqpoint{3.954340in}{1.623637in}}%
\pgfpathcurveto{\pgfqpoint{3.949296in}{1.618593in}}{\pgfqpoint{3.946462in}{1.611752in}}{\pgfqpoint{3.946462in}{1.604619in}}%
\pgfpathcurveto{\pgfqpoint{3.946462in}{1.597486in}}{\pgfqpoint{3.949296in}{1.590644in}}{\pgfqpoint{3.954340in}{1.585601in}}%
\pgfpathcurveto{\pgfqpoint{3.959384in}{1.580557in}}{\pgfqpoint{3.966225in}{1.577723in}}{\pgfqpoint{3.973358in}{1.577723in}}%
\pgfpathclose%
\pgfusepath{stroke,fill}%
\end{pgfscope}%
\begin{pgfscope}%
\pgfpathrectangle{\pgfqpoint{2.867647in}{0.500000in}}{\pgfqpoint{1.764706in}{1.700000in}}%
\pgfusepath{clip}%
\pgfsetbuttcap%
\pgfsetroundjoin%
\definecolor{currentfill}{rgb}{0.975644,0.874038,0.797253}%
\pgfsetfillcolor{currentfill}%
\pgfsetlinewidth{0.311001pt}%
\definecolor{currentstroke}{rgb}{1.000000,1.000000,1.000000}%
\pgfsetstrokecolor{currentstroke}%
\pgfsetdash{}{0pt}%
\pgfpathmoveto{\pgfqpoint{4.085648in}{1.566119in}}%
\pgfpathcurveto{\pgfqpoint{4.092780in}{1.566119in}}{\pgfqpoint{4.099622in}{1.568953in}}{\pgfqpoint{4.104666in}{1.573997in}}%
\pgfpathcurveto{\pgfqpoint{4.109709in}{1.579041in}}{\pgfqpoint{4.112543in}{1.585882in}}{\pgfqpoint{4.112543in}{1.593015in}}%
\pgfpathcurveto{\pgfqpoint{4.112543in}{1.600148in}}{\pgfqpoint{4.109709in}{1.606990in}}{\pgfqpoint{4.104666in}{1.612033in}}%
\pgfpathcurveto{\pgfqpoint{4.099622in}{1.617077in}}{\pgfqpoint{4.092780in}{1.619911in}}{\pgfqpoint{4.085648in}{1.619911in}}%
\pgfpathcurveto{\pgfqpoint{4.078515in}{1.619911in}}{\pgfqpoint{4.071673in}{1.617077in}}{\pgfqpoint{4.066630in}{1.612033in}}%
\pgfpathcurveto{\pgfqpoint{4.061586in}{1.606990in}}{\pgfqpoint{4.058752in}{1.600148in}}{\pgfqpoint{4.058752in}{1.593015in}}%
\pgfpathcurveto{\pgfqpoint{4.058752in}{1.585882in}}{\pgfqpoint{4.061586in}{1.579041in}}{\pgfqpoint{4.066630in}{1.573997in}}%
\pgfpathcurveto{\pgfqpoint{4.071673in}{1.568953in}}{\pgfqpoint{4.078515in}{1.566119in}}{\pgfqpoint{4.085648in}{1.566119in}}%
\pgfpathclose%
\pgfusepath{stroke,fill}%
\end{pgfscope}%
\begin{pgfscope}%
\pgfpathrectangle{\pgfqpoint{2.867647in}{0.500000in}}{\pgfqpoint{1.764706in}{1.700000in}}%
\pgfusepath{clip}%
\pgfsetbuttcap%
\pgfsetroundjoin%
\definecolor{currentfill}{rgb}{0.963728,0.638439,0.479050}%
\pgfsetfillcolor{currentfill}%
\pgfsetlinewidth{0.311001pt}%
\definecolor{currentstroke}{rgb}{1.000000,1.000000,1.000000}%
\pgfsetstrokecolor{currentstroke}%
\pgfsetdash{}{0pt}%
\pgfpathmoveto{\pgfqpoint{3.962808in}{0.929403in}}%
\pgfpathcurveto{\pgfqpoint{3.969941in}{0.929403in}}{\pgfqpoint{3.976782in}{0.932237in}}{\pgfqpoint{3.981826in}{0.937280in}}%
\pgfpathcurveto{\pgfqpoint{3.986870in}{0.942324in}}{\pgfqpoint{3.989703in}{0.949166in}}{\pgfqpoint{3.989703in}{0.956299in}}%
\pgfpathcurveto{\pgfqpoint{3.989703in}{0.963431in}}{\pgfqpoint{3.986870in}{0.970273in}}{\pgfqpoint{3.981826in}{0.975317in}}%
\pgfpathcurveto{\pgfqpoint{3.976782in}{0.980360in}}{\pgfqpoint{3.969941in}{0.983194in}}{\pgfqpoint{3.962808in}{0.983194in}}%
\pgfpathcurveto{\pgfqpoint{3.955675in}{0.983194in}}{\pgfqpoint{3.948833in}{0.980360in}}{\pgfqpoint{3.943790in}{0.975317in}}%
\pgfpathcurveto{\pgfqpoint{3.938746in}{0.970273in}}{\pgfqpoint{3.935912in}{0.963431in}}{\pgfqpoint{3.935912in}{0.956299in}}%
\pgfpathcurveto{\pgfqpoint{3.935912in}{0.949166in}}{\pgfqpoint{3.938746in}{0.942324in}}{\pgfqpoint{3.943790in}{0.937280in}}%
\pgfpathcurveto{\pgfqpoint{3.948833in}{0.932237in}}{\pgfqpoint{3.955675in}{0.929403in}}{\pgfqpoint{3.962808in}{0.929403in}}%
\pgfpathclose%
\pgfusepath{stroke,fill}%
\end{pgfscope}%
\begin{pgfscope}%
\pgfpathrectangle{\pgfqpoint{2.867647in}{0.500000in}}{\pgfqpoint{1.764706in}{1.700000in}}%
\pgfusepath{clip}%
\pgfsetbuttcap%
\pgfsetroundjoin%
\definecolor{currentfill}{rgb}{0.978376,0.897317,0.831308}%
\pgfsetfillcolor{currentfill}%
\pgfsetlinewidth{0.311001pt}%
\definecolor{currentstroke}{rgb}{1.000000,1.000000,1.000000}%
\pgfsetstrokecolor{currentstroke}%
\pgfsetdash{}{0pt}%
\pgfpathmoveto{\pgfqpoint{4.213788in}{1.446696in}}%
\pgfpathcurveto{\pgfqpoint{4.220920in}{1.446696in}}{\pgfqpoint{4.227762in}{1.449530in}}{\pgfqpoint{4.232806in}{1.454574in}}%
\pgfpathcurveto{\pgfqpoint{4.237849in}{1.459617in}}{\pgfqpoint{4.240683in}{1.466459in}}{\pgfqpoint{4.240683in}{1.473592in}}%
\pgfpathcurveto{\pgfqpoint{4.240683in}{1.480725in}}{\pgfqpoint{4.237849in}{1.487566in}}{\pgfqpoint{4.232806in}{1.492610in}}%
\pgfpathcurveto{\pgfqpoint{4.227762in}{1.497654in}}{\pgfqpoint{4.220920in}{1.500487in}}{\pgfqpoint{4.213788in}{1.500487in}}%
\pgfpathcurveto{\pgfqpoint{4.206655in}{1.500487in}}{\pgfqpoint{4.199813in}{1.497654in}}{\pgfqpoint{4.194769in}{1.492610in}}%
\pgfpathcurveto{\pgfqpoint{4.189726in}{1.487566in}}{\pgfqpoint{4.186892in}{1.480725in}}{\pgfqpoint{4.186892in}{1.473592in}}%
\pgfpathcurveto{\pgfqpoint{4.186892in}{1.466459in}}{\pgfqpoint{4.189726in}{1.459617in}}{\pgfqpoint{4.194769in}{1.454574in}}%
\pgfpathcurveto{\pgfqpoint{4.199813in}{1.449530in}}{\pgfqpoint{4.206655in}{1.446696in}}{\pgfqpoint{4.213788in}{1.446696in}}%
\pgfpathclose%
\pgfusepath{stroke,fill}%
\end{pgfscope}%
\begin{pgfscope}%
\pgfpathrectangle{\pgfqpoint{2.867647in}{0.500000in}}{\pgfqpoint{1.764706in}{1.700000in}}%
\pgfusepath{clip}%
\pgfsetbuttcap%
\pgfsetroundjoin%
\definecolor{currentfill}{rgb}{0.960421,0.553286,0.393191}%
\pgfsetfillcolor{currentfill}%
\pgfsetlinewidth{0.311001pt}%
\definecolor{currentstroke}{rgb}{1.000000,1.000000,1.000000}%
\pgfsetstrokecolor{currentstroke}%
\pgfsetdash{}{0pt}%
\pgfpathmoveto{\pgfqpoint{4.055650in}{1.819714in}}%
\pgfpathcurveto{\pgfqpoint{4.062782in}{1.819714in}}{\pgfqpoint{4.069624in}{1.822548in}}{\pgfqpoint{4.074668in}{1.827592in}}%
\pgfpathcurveto{\pgfqpoint{4.079711in}{1.832635in}}{\pgfqpoint{4.082545in}{1.839477in}}{\pgfqpoint{4.082545in}{1.846610in}}%
\pgfpathcurveto{\pgfqpoint{4.082545in}{1.853743in}}{\pgfqpoint{4.079711in}{1.860584in}}{\pgfqpoint{4.074668in}{1.865628in}}%
\pgfpathcurveto{\pgfqpoint{4.069624in}{1.870672in}}{\pgfqpoint{4.062782in}{1.873505in}}{\pgfqpoint{4.055650in}{1.873505in}}%
\pgfpathcurveto{\pgfqpoint{4.048517in}{1.873505in}}{\pgfqpoint{4.041675in}{1.870672in}}{\pgfqpoint{4.036631in}{1.865628in}}%
\pgfpathcurveto{\pgfqpoint{4.031588in}{1.860584in}}{\pgfqpoint{4.028754in}{1.853743in}}{\pgfqpoint{4.028754in}{1.846610in}}%
\pgfpathcurveto{\pgfqpoint{4.028754in}{1.839477in}}{\pgfqpoint{4.031588in}{1.832635in}}{\pgfqpoint{4.036631in}{1.827592in}}%
\pgfpathcurveto{\pgfqpoint{4.041675in}{1.822548in}}{\pgfqpoint{4.048517in}{1.819714in}}{\pgfqpoint{4.055650in}{1.819714in}}%
\pgfpathclose%
\pgfusepath{stroke,fill}%
\end{pgfscope}%
\begin{pgfscope}%
\pgfpathrectangle{\pgfqpoint{2.867647in}{0.500000in}}{\pgfqpoint{1.764706in}{1.700000in}}%
\pgfusepath{clip}%
\pgfsetbuttcap%
\pgfsetroundjoin%
\definecolor{currentfill}{rgb}{0.960043,0.546576,0.387029}%
\pgfsetfillcolor{currentfill}%
\pgfsetlinewidth{0.311001pt}%
\definecolor{currentstroke}{rgb}{1.000000,1.000000,1.000000}%
\pgfsetstrokecolor{currentstroke}%
\pgfsetdash{}{0pt}%
\pgfpathmoveto{\pgfqpoint{4.217202in}{0.932862in}}%
\pgfpathcurveto{\pgfqpoint{4.224335in}{0.932862in}}{\pgfqpoint{4.231176in}{0.935696in}}{\pgfqpoint{4.236220in}{0.940740in}}%
\pgfpathcurveto{\pgfqpoint{4.241263in}{0.945783in}}{\pgfqpoint{4.244097in}{0.952625in}}{\pgfqpoint{4.244097in}{0.959758in}}%
\pgfpathcurveto{\pgfqpoint{4.244097in}{0.966890in}}{\pgfqpoint{4.241263in}{0.973732in}}{\pgfqpoint{4.236220in}{0.978776in}}%
\pgfpathcurveto{\pgfqpoint{4.231176in}{0.983819in}}{\pgfqpoint{4.224335in}{0.986653in}}{\pgfqpoint{4.217202in}{0.986653in}}%
\pgfpathcurveto{\pgfqpoint{4.210069in}{0.986653in}}{\pgfqpoint{4.203227in}{0.983819in}}{\pgfqpoint{4.198184in}{0.978776in}}%
\pgfpathcurveto{\pgfqpoint{4.193140in}{0.973732in}}{\pgfqpoint{4.190306in}{0.966890in}}{\pgfqpoint{4.190306in}{0.959758in}}%
\pgfpathcurveto{\pgfqpoint{4.190306in}{0.952625in}}{\pgfqpoint{4.193140in}{0.945783in}}{\pgfqpoint{4.198184in}{0.940740in}}%
\pgfpathcurveto{\pgfqpoint{4.203227in}{0.935696in}}{\pgfqpoint{4.210069in}{0.932862in}}{\pgfqpoint{4.217202in}{0.932862in}}%
\pgfpathclose%
\pgfusepath{stroke,fill}%
\end{pgfscope}%
\begin{pgfscope}%
\pgfpathrectangle{\pgfqpoint{2.867647in}{0.500000in}}{\pgfqpoint{1.764706in}{1.700000in}}%
\pgfusepath{clip}%
\pgfsetbuttcap%
\pgfsetroundjoin%
\definecolor{currentfill}{rgb}{0.980678,0.914765,0.856766}%
\pgfsetfillcolor{currentfill}%
\pgfsetlinewidth{0.311001pt}%
\definecolor{currentstroke}{rgb}{1.000000,1.000000,1.000000}%
\pgfsetstrokecolor{currentstroke}%
\pgfsetdash{}{0pt}%
\pgfpathmoveto{\pgfqpoint{4.155285in}{1.197697in}}%
\pgfpathcurveto{\pgfqpoint{4.162418in}{1.197697in}}{\pgfqpoint{4.169260in}{1.200531in}}{\pgfqpoint{4.174303in}{1.205575in}}%
\pgfpathcurveto{\pgfqpoint{4.179347in}{1.210619in}}{\pgfqpoint{4.182181in}{1.217460in}}{\pgfqpoint{4.182181in}{1.224593in}}%
\pgfpathcurveto{\pgfqpoint{4.182181in}{1.231726in}}{\pgfqpoint{4.179347in}{1.238568in}}{\pgfqpoint{4.174303in}{1.243611in}}%
\pgfpathcurveto{\pgfqpoint{4.169260in}{1.248655in}}{\pgfqpoint{4.162418in}{1.251489in}}{\pgfqpoint{4.155285in}{1.251489in}}%
\pgfpathcurveto{\pgfqpoint{4.148152in}{1.251489in}}{\pgfqpoint{4.141311in}{1.248655in}}{\pgfqpoint{4.136267in}{1.243611in}}%
\pgfpathcurveto{\pgfqpoint{4.131223in}{1.238568in}}{\pgfqpoint{4.128389in}{1.231726in}}{\pgfqpoint{4.128389in}{1.224593in}}%
\pgfpathcurveto{\pgfqpoint{4.128389in}{1.217460in}}{\pgfqpoint{4.131223in}{1.210619in}}{\pgfqpoint{4.136267in}{1.205575in}}%
\pgfpathcurveto{\pgfqpoint{4.141311in}{1.200531in}}{\pgfqpoint{4.148152in}{1.197697in}}{\pgfqpoint{4.155285in}{1.197697in}}%
\pgfpathclose%
\pgfusepath{stroke,fill}%
\end{pgfscope}%
\begin{pgfscope}%
\pgfpathrectangle{\pgfqpoint{2.867647in}{0.500000in}}{\pgfqpoint{1.764706in}{1.700000in}}%
\pgfusepath{clip}%
\pgfsetbuttcap%
\pgfsetroundjoin%
\definecolor{currentfill}{rgb}{0.976287,0.879862,0.805788}%
\pgfsetfillcolor{currentfill}%
\pgfsetlinewidth{0.311001pt}%
\definecolor{currentstroke}{rgb}{1.000000,1.000000,1.000000}%
\pgfsetstrokecolor{currentstroke}%
\pgfsetdash{}{0pt}%
\pgfpathmoveto{\pgfqpoint{4.157356in}{1.601164in}}%
\pgfpathcurveto{\pgfqpoint{4.164488in}{1.601164in}}{\pgfqpoint{4.171330in}{1.603998in}}{\pgfqpoint{4.176374in}{1.609042in}}%
\pgfpathcurveto{\pgfqpoint{4.181417in}{1.614086in}}{\pgfqpoint{4.184251in}{1.620927in}}{\pgfqpoint{4.184251in}{1.628060in}}%
\pgfpathcurveto{\pgfqpoint{4.184251in}{1.635193in}}{\pgfqpoint{4.181417in}{1.642034in}}{\pgfqpoint{4.176374in}{1.647078in}}%
\pgfpathcurveto{\pgfqpoint{4.171330in}{1.652122in}}{\pgfqpoint{4.164488in}{1.654956in}}{\pgfqpoint{4.157356in}{1.654956in}}%
\pgfpathcurveto{\pgfqpoint{4.150223in}{1.654956in}}{\pgfqpoint{4.143381in}{1.652122in}}{\pgfqpoint{4.138337in}{1.647078in}}%
\pgfpathcurveto{\pgfqpoint{4.133294in}{1.642034in}}{\pgfqpoint{4.130460in}{1.635193in}}{\pgfqpoint{4.130460in}{1.628060in}}%
\pgfpathcurveto{\pgfqpoint{4.130460in}{1.620927in}}{\pgfqpoint{4.133294in}{1.614086in}}{\pgfqpoint{4.138337in}{1.609042in}}%
\pgfpathcurveto{\pgfqpoint{4.143381in}{1.603998in}}{\pgfqpoint{4.150223in}{1.601164in}}{\pgfqpoint{4.157356in}{1.601164in}}%
\pgfpathclose%
\pgfusepath{stroke,fill}%
\end{pgfscope}%
\begin{pgfscope}%
\pgfpathrectangle{\pgfqpoint{2.867647in}{0.500000in}}{\pgfqpoint{1.764706in}{1.700000in}}%
\pgfusepath{clip}%
\pgfsetbuttcap%
\pgfsetroundjoin%
\definecolor{currentfill}{rgb}{0.979124,0.903132,0.839793}%
\pgfsetfillcolor{currentfill}%
\pgfsetlinewidth{0.311001pt}%
\definecolor{currentstroke}{rgb}{1.000000,1.000000,1.000000}%
\pgfsetstrokecolor{currentstroke}%
\pgfsetdash{}{0pt}%
\pgfpathmoveto{\pgfqpoint{4.136040in}{1.182728in}}%
\pgfpathcurveto{\pgfqpoint{4.143173in}{1.182728in}}{\pgfqpoint{4.150014in}{1.185562in}}{\pgfqpoint{4.155058in}{1.190606in}}%
\pgfpathcurveto{\pgfqpoint{4.160102in}{1.195649in}}{\pgfqpoint{4.162936in}{1.202491in}}{\pgfqpoint{4.162936in}{1.209624in}}%
\pgfpathcurveto{\pgfqpoint{4.162936in}{1.216757in}}{\pgfqpoint{4.160102in}{1.223598in}}{\pgfqpoint{4.155058in}{1.228642in}}%
\pgfpathcurveto{\pgfqpoint{4.150014in}{1.233686in}}{\pgfqpoint{4.143173in}{1.236519in}}{\pgfqpoint{4.136040in}{1.236519in}}%
\pgfpathcurveto{\pgfqpoint{4.128907in}{1.236519in}}{\pgfqpoint{4.122066in}{1.233686in}}{\pgfqpoint{4.117022in}{1.228642in}}%
\pgfpathcurveto{\pgfqpoint{4.111978in}{1.223598in}}{\pgfqpoint{4.109144in}{1.216757in}}{\pgfqpoint{4.109144in}{1.209624in}}%
\pgfpathcurveto{\pgfqpoint{4.109144in}{1.202491in}}{\pgfqpoint{4.111978in}{1.195649in}}{\pgfqpoint{4.117022in}{1.190606in}}%
\pgfpathcurveto{\pgfqpoint{4.122066in}{1.185562in}}{\pgfqpoint{4.128907in}{1.182728in}}{\pgfqpoint{4.136040in}{1.182728in}}%
\pgfpathclose%
\pgfusepath{stroke,fill}%
\end{pgfscope}%
\begin{pgfscope}%
\pgfpathrectangle{\pgfqpoint{2.867647in}{0.500000in}}{\pgfqpoint{1.764706in}{1.700000in}}%
\pgfusepath{clip}%
\pgfsetbuttcap%
\pgfsetroundjoin%
\definecolor{currentfill}{rgb}{0.980678,0.914765,0.856766}%
\pgfsetfillcolor{currentfill}%
\pgfsetlinewidth{0.311001pt}%
\definecolor{currentstroke}{rgb}{1.000000,1.000000,1.000000}%
\pgfsetstrokecolor{currentstroke}%
\pgfsetdash{}{0pt}%
\pgfpathmoveto{\pgfqpoint{4.171343in}{1.410195in}}%
\pgfpathcurveto{\pgfqpoint{4.178476in}{1.410195in}}{\pgfqpoint{4.185318in}{1.413028in}}{\pgfqpoint{4.190361in}{1.418072in}}%
\pgfpathcurveto{\pgfqpoint{4.195405in}{1.423116in}}{\pgfqpoint{4.198239in}{1.429957in}}{\pgfqpoint{4.198239in}{1.437090in}}%
\pgfpathcurveto{\pgfqpoint{4.198239in}{1.444223in}}{\pgfqpoint{4.195405in}{1.451065in}}{\pgfqpoint{4.190361in}{1.456108in}}%
\pgfpathcurveto{\pgfqpoint{4.185318in}{1.461152in}}{\pgfqpoint{4.178476in}{1.463986in}}{\pgfqpoint{4.171343in}{1.463986in}}%
\pgfpathcurveto{\pgfqpoint{4.164210in}{1.463986in}}{\pgfqpoint{4.157369in}{1.461152in}}{\pgfqpoint{4.152325in}{1.456108in}}%
\pgfpathcurveto{\pgfqpoint{4.147281in}{1.451065in}}{\pgfqpoint{4.144448in}{1.444223in}}{\pgfqpoint{4.144448in}{1.437090in}}%
\pgfpathcurveto{\pgfqpoint{4.144448in}{1.429957in}}{\pgfqpoint{4.147281in}{1.423116in}}{\pgfqpoint{4.152325in}{1.418072in}}%
\pgfpathcurveto{\pgfqpoint{4.157369in}{1.413028in}}{\pgfqpoint{4.164210in}{1.410195in}}{\pgfqpoint{4.171343in}{1.410195in}}%
\pgfpathclose%
\pgfusepath{stroke,fill}%
\end{pgfscope}%
\begin{pgfscope}%
\pgfpathrectangle{\pgfqpoint{2.867647in}{0.500000in}}{\pgfqpoint{1.764706in}{1.700000in}}%
\pgfusepath{clip}%
\pgfsetbuttcap%
\pgfsetroundjoin%
\definecolor{currentfill}{rgb}{0.950017,0.427714,0.292447}%
\pgfsetfillcolor{currentfill}%
\pgfsetlinewidth{0.311001pt}%
\definecolor{currentstroke}{rgb}{1.000000,1.000000,1.000000}%
\pgfsetstrokecolor{currentstroke}%
\pgfsetdash{}{0pt}%
\pgfpathmoveto{\pgfqpoint{3.856480in}{1.693386in}}%
\pgfpathcurveto{\pgfqpoint{3.863613in}{1.693386in}}{\pgfqpoint{3.870455in}{1.696220in}}{\pgfqpoint{3.875499in}{1.701264in}}%
\pgfpathcurveto{\pgfqpoint{3.880542in}{1.706307in}}{\pgfqpoint{3.883376in}{1.713149in}}{\pgfqpoint{3.883376in}{1.720282in}}%
\pgfpathcurveto{\pgfqpoint{3.883376in}{1.727415in}}{\pgfqpoint{3.880542in}{1.734256in}}{\pgfqpoint{3.875499in}{1.739300in}}%
\pgfpathcurveto{\pgfqpoint{3.870455in}{1.744344in}}{\pgfqpoint{3.863613in}{1.747178in}}{\pgfqpoint{3.856480in}{1.747178in}}%
\pgfpathcurveto{\pgfqpoint{3.849348in}{1.747178in}}{\pgfqpoint{3.842506in}{1.744344in}}{\pgfqpoint{3.837462in}{1.739300in}}%
\pgfpathcurveto{\pgfqpoint{3.832419in}{1.734256in}}{\pgfqpoint{3.829585in}{1.727415in}}{\pgfqpoint{3.829585in}{1.720282in}}%
\pgfpathcurveto{\pgfqpoint{3.829585in}{1.713149in}}{\pgfqpoint{3.832419in}{1.706307in}}{\pgfqpoint{3.837462in}{1.701264in}}%
\pgfpathcurveto{\pgfqpoint{3.842506in}{1.696220in}}{\pgfqpoint{3.849348in}{1.693386in}}{\pgfqpoint{3.856480in}{1.693386in}}%
\pgfpathclose%
\pgfusepath{stroke,fill}%
\end{pgfscope}%
\begin{pgfscope}%
\pgfpathrectangle{\pgfqpoint{2.867647in}{0.500000in}}{\pgfqpoint{1.764706in}{1.700000in}}%
\pgfusepath{clip}%
\pgfsetbuttcap%
\pgfsetroundjoin%
\definecolor{currentfill}{rgb}{0.965928,0.738443,0.600540}%
\pgfsetfillcolor{currentfill}%
\pgfsetlinewidth{0.311001pt}%
\definecolor{currentstroke}{rgb}{1.000000,1.000000,1.000000}%
\pgfsetstrokecolor{currentstroke}%
\pgfsetdash{}{0pt}%
\pgfpathmoveto{\pgfqpoint{4.050945in}{1.481072in}}%
\pgfpathcurveto{\pgfqpoint{4.058078in}{1.481072in}}{\pgfqpoint{4.064919in}{1.483906in}}{\pgfqpoint{4.069963in}{1.488950in}}%
\pgfpathcurveto{\pgfqpoint{4.075007in}{1.493994in}}{\pgfqpoint{4.077841in}{1.500835in}}{\pgfqpoint{4.077841in}{1.507968in}}%
\pgfpathcurveto{\pgfqpoint{4.077841in}{1.515101in}}{\pgfqpoint{4.075007in}{1.521943in}}{\pgfqpoint{4.069963in}{1.526986in}}%
\pgfpathcurveto{\pgfqpoint{4.064919in}{1.532030in}}{\pgfqpoint{4.058078in}{1.534864in}}{\pgfqpoint{4.050945in}{1.534864in}}%
\pgfpathcurveto{\pgfqpoint{4.043812in}{1.534864in}}{\pgfqpoint{4.036970in}{1.532030in}}{\pgfqpoint{4.031927in}{1.526986in}}%
\pgfpathcurveto{\pgfqpoint{4.026883in}{1.521943in}}{\pgfqpoint{4.024049in}{1.515101in}}{\pgfqpoint{4.024049in}{1.507968in}}%
\pgfpathcurveto{\pgfqpoint{4.024049in}{1.500835in}}{\pgfqpoint{4.026883in}{1.493994in}}{\pgfqpoint{4.031927in}{1.488950in}}%
\pgfpathcurveto{\pgfqpoint{4.036970in}{1.483906in}}{\pgfqpoint{4.043812in}{1.481072in}}{\pgfqpoint{4.050945in}{1.481072in}}%
\pgfpathclose%
\pgfusepath{stroke,fill}%
\end{pgfscope}%
\begin{pgfscope}%
\pgfpathrectangle{\pgfqpoint{2.867647in}{0.500000in}}{\pgfqpoint{1.764706in}{1.700000in}}%
\pgfusepath{clip}%
\pgfsetbuttcap%
\pgfsetroundjoin%
\definecolor{currentfill}{rgb}{0.955697,0.484891,0.334214}%
\pgfsetfillcolor{currentfill}%
\pgfsetlinewidth{0.311001pt}%
\definecolor{currentstroke}{rgb}{1.000000,1.000000,1.000000}%
\pgfsetstrokecolor{currentstroke}%
\pgfsetdash{}{0pt}%
\pgfpathmoveto{\pgfqpoint{4.334617in}{1.473032in}}%
\pgfpathcurveto{\pgfqpoint{4.341750in}{1.473032in}}{\pgfqpoint{4.348592in}{1.475866in}}{\pgfqpoint{4.353635in}{1.480910in}}%
\pgfpathcurveto{\pgfqpoint{4.358679in}{1.485954in}}{\pgfqpoint{4.361513in}{1.492795in}}{\pgfqpoint{4.361513in}{1.499928in}}%
\pgfpathcurveto{\pgfqpoint{4.361513in}{1.507061in}}{\pgfqpoint{4.358679in}{1.513903in}}{\pgfqpoint{4.353635in}{1.518946in}}%
\pgfpathcurveto{\pgfqpoint{4.348592in}{1.523990in}}{\pgfqpoint{4.341750in}{1.526824in}}{\pgfqpoint{4.334617in}{1.526824in}}%
\pgfpathcurveto{\pgfqpoint{4.327484in}{1.526824in}}{\pgfqpoint{4.320643in}{1.523990in}}{\pgfqpoint{4.315599in}{1.518946in}}%
\pgfpathcurveto{\pgfqpoint{4.310555in}{1.513903in}}{\pgfqpoint{4.307722in}{1.507061in}}{\pgfqpoint{4.307722in}{1.499928in}}%
\pgfpathcurveto{\pgfqpoint{4.307722in}{1.492795in}}{\pgfqpoint{4.310555in}{1.485954in}}{\pgfqpoint{4.315599in}{1.480910in}}%
\pgfpathcurveto{\pgfqpoint{4.320643in}{1.475866in}}{\pgfqpoint{4.327484in}{1.473032in}}{\pgfqpoint{4.334617in}{1.473032in}}%
\pgfpathclose%
\pgfusepath{stroke,fill}%
\end{pgfscope}%
\begin{pgfscope}%
\pgfpathrectangle{\pgfqpoint{2.867647in}{0.500000in}}{\pgfqpoint{1.764706in}{1.700000in}}%
\pgfusepath{clip}%
\pgfsetbuttcap%
\pgfsetroundjoin%
\definecolor{currentfill}{rgb}{0.976961,0.885681,0.814303}%
\pgfsetfillcolor{currentfill}%
\pgfsetlinewidth{0.311001pt}%
\definecolor{currentstroke}{rgb}{1.000000,1.000000,1.000000}%
\pgfsetstrokecolor{currentstroke}%
\pgfsetdash{}{0pt}%
\pgfpathmoveto{\pgfqpoint{4.192906in}{1.541476in}}%
\pgfpathcurveto{\pgfqpoint{4.200039in}{1.541476in}}{\pgfqpoint{4.206881in}{1.544310in}}{\pgfqpoint{4.211925in}{1.549354in}}%
\pgfpathcurveto{\pgfqpoint{4.216968in}{1.554398in}}{\pgfqpoint{4.219802in}{1.561239in}}{\pgfqpoint{4.219802in}{1.568372in}}%
\pgfpathcurveto{\pgfqpoint{4.219802in}{1.575505in}}{\pgfqpoint{4.216968in}{1.582347in}}{\pgfqpoint{4.211925in}{1.587390in}}%
\pgfpathcurveto{\pgfqpoint{4.206881in}{1.592434in}}{\pgfqpoint{4.200039in}{1.595268in}}{\pgfqpoint{4.192906in}{1.595268in}}%
\pgfpathcurveto{\pgfqpoint{4.185774in}{1.595268in}}{\pgfqpoint{4.178932in}{1.592434in}}{\pgfqpoint{4.173888in}{1.587390in}}%
\pgfpathcurveto{\pgfqpoint{4.168845in}{1.582347in}}{\pgfqpoint{4.166011in}{1.575505in}}{\pgfqpoint{4.166011in}{1.568372in}}%
\pgfpathcurveto{\pgfqpoint{4.166011in}{1.561239in}}{\pgfqpoint{4.168845in}{1.554398in}}{\pgfqpoint{4.173888in}{1.549354in}}%
\pgfpathcurveto{\pgfqpoint{4.178932in}{1.544310in}}{\pgfqpoint{4.185774in}{1.541476in}}{\pgfqpoint{4.192906in}{1.541476in}}%
\pgfpathclose%
\pgfusepath{stroke,fill}%
\end{pgfscope}%
\begin{pgfscope}%
\pgfpathrectangle{\pgfqpoint{2.867647in}{0.500000in}}{\pgfqpoint{1.764706in}{1.700000in}}%
\pgfusepath{clip}%
\pgfsetbuttcap%
\pgfsetroundjoin%
\definecolor{currentfill}{rgb}{0.969359,0.803954,0.693832}%
\pgfsetfillcolor{currentfill}%
\pgfsetlinewidth{0.311001pt}%
\definecolor{currentstroke}{rgb}{1.000000,1.000000,1.000000}%
\pgfsetstrokecolor{currentstroke}%
\pgfsetdash{}{0pt}%
\pgfpathmoveto{\pgfqpoint{4.049833in}{0.962051in}}%
\pgfpathcurveto{\pgfqpoint{4.056966in}{0.962051in}}{\pgfqpoint{4.063808in}{0.964885in}}{\pgfqpoint{4.068851in}{0.969928in}}%
\pgfpathcurveto{\pgfqpoint{4.073895in}{0.974972in}}{\pgfqpoint{4.076729in}{0.981814in}}{\pgfqpoint{4.076729in}{0.988946in}}%
\pgfpathcurveto{\pgfqpoint{4.076729in}{0.996079in}}{\pgfqpoint{4.073895in}{1.002921in}}{\pgfqpoint{4.068851in}{1.007965in}}%
\pgfpathcurveto{\pgfqpoint{4.063808in}{1.013008in}}{\pgfqpoint{4.056966in}{1.015842in}}{\pgfqpoint{4.049833in}{1.015842in}}%
\pgfpathcurveto{\pgfqpoint{4.042700in}{1.015842in}}{\pgfqpoint{4.035859in}{1.013008in}}{\pgfqpoint{4.030815in}{1.007965in}}%
\pgfpathcurveto{\pgfqpoint{4.025771in}{1.002921in}}{\pgfqpoint{4.022937in}{0.996079in}}{\pgfqpoint{4.022937in}{0.988946in}}%
\pgfpathcurveto{\pgfqpoint{4.022937in}{0.981814in}}{\pgfqpoint{4.025771in}{0.974972in}}{\pgfqpoint{4.030815in}{0.969928in}}%
\pgfpathcurveto{\pgfqpoint{4.035859in}{0.964885in}}{\pgfqpoint{4.042700in}{0.962051in}}{\pgfqpoint{4.049833in}{0.962051in}}%
\pgfpathclose%
\pgfusepath{stroke,fill}%
\end{pgfscope}%
\begin{pgfscope}%
\pgfpathrectangle{\pgfqpoint{2.867647in}{0.500000in}}{\pgfqpoint{1.764706in}{1.700000in}}%
\pgfusepath{clip}%
\pgfsetbuttcap%
\pgfsetroundjoin%
\definecolor{currentfill}{rgb}{0.965440,0.720101,0.576404}%
\pgfsetfillcolor{currentfill}%
\pgfsetlinewidth{0.311001pt}%
\definecolor{currentstroke}{rgb}{1.000000,1.000000,1.000000}%
\pgfsetstrokecolor{currentstroke}%
\pgfsetdash{}{0pt}%
\pgfpathmoveto{\pgfqpoint{3.993522in}{1.687889in}}%
\pgfpathcurveto{\pgfqpoint{4.000655in}{1.687889in}}{\pgfqpoint{4.007497in}{1.690723in}}{\pgfqpoint{4.012540in}{1.695767in}}%
\pgfpathcurveto{\pgfqpoint{4.017584in}{1.700810in}}{\pgfqpoint{4.020418in}{1.707652in}}{\pgfqpoint{4.020418in}{1.714785in}}%
\pgfpathcurveto{\pgfqpoint{4.020418in}{1.721918in}}{\pgfqpoint{4.017584in}{1.728759in}}{\pgfqpoint{4.012540in}{1.733803in}}%
\pgfpathcurveto{\pgfqpoint{4.007497in}{1.738847in}}{\pgfqpoint{4.000655in}{1.741681in}}{\pgfqpoint{3.993522in}{1.741681in}}%
\pgfpathcurveto{\pgfqpoint{3.986389in}{1.741681in}}{\pgfqpoint{3.979548in}{1.738847in}}{\pgfqpoint{3.974504in}{1.733803in}}%
\pgfpathcurveto{\pgfqpoint{3.969460in}{1.728759in}}{\pgfqpoint{3.966626in}{1.721918in}}{\pgfqpoint{3.966626in}{1.714785in}}%
\pgfpathcurveto{\pgfqpoint{3.966626in}{1.707652in}}{\pgfqpoint{3.969460in}{1.700810in}}{\pgfqpoint{3.974504in}{1.695767in}}%
\pgfpathcurveto{\pgfqpoint{3.979548in}{1.690723in}}{\pgfqpoint{3.986389in}{1.687889in}}{\pgfqpoint{3.993522in}{1.687889in}}%
\pgfpathclose%
\pgfusepath{stroke,fill}%
\end{pgfscope}%
\begin{pgfscope}%
\pgfpathrectangle{\pgfqpoint{2.867647in}{0.500000in}}{\pgfqpoint{1.764706in}{1.700000in}}%
\pgfusepath{clip}%
\pgfsetbuttcap%
\pgfsetroundjoin%
\definecolor{currentfill}{rgb}{0.980678,0.914765,0.856766}%
\pgfsetfillcolor{currentfill}%
\pgfsetlinewidth{0.311001pt}%
\definecolor{currentstroke}{rgb}{1.000000,1.000000,1.000000}%
\pgfsetstrokecolor{currentstroke}%
\pgfsetdash{}{0pt}%
\pgfpathmoveto{\pgfqpoint{4.201481in}{1.399062in}}%
\pgfpathcurveto{\pgfqpoint{4.208614in}{1.399062in}}{\pgfqpoint{4.215456in}{1.401896in}}{\pgfqpoint{4.220499in}{1.406940in}}%
\pgfpathcurveto{\pgfqpoint{4.225543in}{1.411983in}}{\pgfqpoint{4.228377in}{1.418825in}}{\pgfqpoint{4.228377in}{1.425958in}}%
\pgfpathcurveto{\pgfqpoint{4.228377in}{1.433091in}}{\pgfqpoint{4.225543in}{1.439932in}}{\pgfqpoint{4.220499in}{1.444976in}}%
\pgfpathcurveto{\pgfqpoint{4.215456in}{1.450020in}}{\pgfqpoint{4.208614in}{1.452853in}}{\pgfqpoint{4.201481in}{1.452853in}}%
\pgfpathcurveto{\pgfqpoint{4.194348in}{1.452853in}}{\pgfqpoint{4.187507in}{1.450020in}}{\pgfqpoint{4.182463in}{1.444976in}}%
\pgfpathcurveto{\pgfqpoint{4.177419in}{1.439932in}}{\pgfqpoint{4.174586in}{1.433091in}}{\pgfqpoint{4.174586in}{1.425958in}}%
\pgfpathcurveto{\pgfqpoint{4.174586in}{1.418825in}}{\pgfqpoint{4.177419in}{1.411983in}}{\pgfqpoint{4.182463in}{1.406940in}}%
\pgfpathcurveto{\pgfqpoint{4.187507in}{1.401896in}}{\pgfqpoint{4.194348in}{1.399062in}}{\pgfqpoint{4.201481in}{1.399062in}}%
\pgfpathclose%
\pgfusepath{stroke,fill}%
\end{pgfscope}%
\begin{pgfscope}%
\pgfpathrectangle{\pgfqpoint{2.867647in}{0.500000in}}{\pgfqpoint{1.764706in}{1.700000in}}%
\pgfusepath{clip}%
\pgfsetbuttcap%
\pgfsetroundjoin%
\definecolor{currentfill}{rgb}{0.917171,0.267738,0.242941}%
\pgfsetfillcolor{currentfill}%
\pgfsetlinewidth{0.311001pt}%
\definecolor{currentstroke}{rgb}{1.000000,1.000000,1.000000}%
\pgfsetstrokecolor{currentstroke}%
\pgfsetdash{}{0pt}%
\pgfpathmoveto{\pgfqpoint{3.851070in}{1.636093in}}%
\pgfpathcurveto{\pgfqpoint{3.858202in}{1.636093in}}{\pgfqpoint{3.865044in}{1.638927in}}{\pgfqpoint{3.870088in}{1.643971in}}%
\pgfpathcurveto{\pgfqpoint{3.875131in}{1.649014in}}{\pgfqpoint{3.877965in}{1.655856in}}{\pgfqpoint{3.877965in}{1.662989in}}%
\pgfpathcurveto{\pgfqpoint{3.877965in}{1.670122in}}{\pgfqpoint{3.875131in}{1.676963in}}{\pgfqpoint{3.870088in}{1.682007in}}%
\pgfpathcurveto{\pgfqpoint{3.865044in}{1.687051in}}{\pgfqpoint{3.858202in}{1.689885in}}{\pgfqpoint{3.851070in}{1.689885in}}%
\pgfpathcurveto{\pgfqpoint{3.843937in}{1.689885in}}{\pgfqpoint{3.837095in}{1.687051in}}{\pgfqpoint{3.832051in}{1.682007in}}%
\pgfpathcurveto{\pgfqpoint{3.827008in}{1.676963in}}{\pgfqpoint{3.824174in}{1.670122in}}{\pgfqpoint{3.824174in}{1.662989in}}%
\pgfpathcurveto{\pgfqpoint{3.824174in}{1.655856in}}{\pgfqpoint{3.827008in}{1.649014in}}{\pgfqpoint{3.832051in}{1.643971in}}%
\pgfpathcurveto{\pgfqpoint{3.837095in}{1.638927in}}{\pgfqpoint{3.843937in}{1.636093in}}{\pgfqpoint{3.851070in}{1.636093in}}%
\pgfpathclose%
\pgfusepath{stroke,fill}%
\end{pgfscope}%
\begin{pgfscope}%
\pgfpathrectangle{\pgfqpoint{2.867647in}{0.500000in}}{\pgfqpoint{1.764706in}{1.700000in}}%
\pgfusepath{clip}%
\pgfsetbuttcap%
\pgfsetroundjoin%
\definecolor{currentfill}{rgb}{0.974412,0.862387,0.780156}%
\pgfsetfillcolor{currentfill}%
\pgfsetlinewidth{0.311001pt}%
\definecolor{currentstroke}{rgb}{1.000000,1.000000,1.000000}%
\pgfsetstrokecolor{currentstroke}%
\pgfsetdash{}{0pt}%
\pgfpathmoveto{\pgfqpoint{4.089389in}{1.520311in}}%
\pgfpathcurveto{\pgfqpoint{4.096522in}{1.520311in}}{\pgfqpoint{4.103364in}{1.523145in}}{\pgfqpoint{4.108407in}{1.528188in}}%
\pgfpathcurveto{\pgfqpoint{4.113451in}{1.533232in}}{\pgfqpoint{4.116285in}{1.540074in}}{\pgfqpoint{4.116285in}{1.547206in}}%
\pgfpathcurveto{\pgfqpoint{4.116285in}{1.554339in}}{\pgfqpoint{4.113451in}{1.561181in}}{\pgfqpoint{4.108407in}{1.566225in}}%
\pgfpathcurveto{\pgfqpoint{4.103364in}{1.571268in}}{\pgfqpoint{4.096522in}{1.574102in}}{\pgfqpoint{4.089389in}{1.574102in}}%
\pgfpathcurveto{\pgfqpoint{4.082256in}{1.574102in}}{\pgfqpoint{4.075415in}{1.571268in}}{\pgfqpoint{4.070371in}{1.566225in}}%
\pgfpathcurveto{\pgfqpoint{4.065327in}{1.561181in}}{\pgfqpoint{4.062493in}{1.554339in}}{\pgfqpoint{4.062493in}{1.547206in}}%
\pgfpathcurveto{\pgfqpoint{4.062493in}{1.540074in}}{\pgfqpoint{4.065327in}{1.533232in}}{\pgfqpoint{4.070371in}{1.528188in}}%
\pgfpathcurveto{\pgfqpoint{4.075415in}{1.523145in}}{\pgfqpoint{4.082256in}{1.520311in}}{\pgfqpoint{4.089389in}{1.520311in}}%
\pgfpathclose%
\pgfusepath{stroke,fill}%
\end{pgfscope}%
\begin{pgfscope}%
\pgfpathrectangle{\pgfqpoint{2.867647in}{0.500000in}}{\pgfqpoint{1.764706in}{1.700000in}}%
\pgfusepath{clip}%
\pgfsetbuttcap%
\pgfsetroundjoin%
\definecolor{currentfill}{rgb}{0.975018,0.868213,0.788710}%
\pgfsetfillcolor{currentfill}%
\pgfsetlinewidth{0.311001pt}%
\definecolor{currentstroke}{rgb}{1.000000,1.000000,1.000000}%
\pgfsetstrokecolor{currentstroke}%
\pgfsetdash{}{0pt}%
\pgfpathmoveto{\pgfqpoint{4.088023in}{1.064462in}}%
\pgfpathcurveto{\pgfqpoint{4.095156in}{1.064462in}}{\pgfqpoint{4.101998in}{1.067296in}}{\pgfqpoint{4.107042in}{1.072340in}}%
\pgfpathcurveto{\pgfqpoint{4.112085in}{1.077384in}}{\pgfqpoint{4.114919in}{1.084225in}}{\pgfqpoint{4.114919in}{1.091358in}}%
\pgfpathcurveto{\pgfqpoint{4.114919in}{1.098491in}}{\pgfqpoint{4.112085in}{1.105333in}}{\pgfqpoint{4.107042in}{1.110376in}}%
\pgfpathcurveto{\pgfqpoint{4.101998in}{1.115420in}}{\pgfqpoint{4.095156in}{1.118254in}}{\pgfqpoint{4.088023in}{1.118254in}}%
\pgfpathcurveto{\pgfqpoint{4.080891in}{1.118254in}}{\pgfqpoint{4.074049in}{1.115420in}}{\pgfqpoint{4.069005in}{1.110376in}}%
\pgfpathcurveto{\pgfqpoint{4.063962in}{1.105333in}}{\pgfqpoint{4.061128in}{1.098491in}}{\pgfqpoint{4.061128in}{1.091358in}}%
\pgfpathcurveto{\pgfqpoint{4.061128in}{1.084225in}}{\pgfqpoint{4.063962in}{1.077384in}}{\pgfqpoint{4.069005in}{1.072340in}}%
\pgfpathcurveto{\pgfqpoint{4.074049in}{1.067296in}}{\pgfqpoint{4.080891in}{1.064462in}}{\pgfqpoint{4.088023in}{1.064462in}}%
\pgfpathclose%
\pgfusepath{stroke,fill}%
\end{pgfscope}%
\begin{pgfscope}%
\pgfpathrectangle{\pgfqpoint{2.867647in}{0.500000in}}{\pgfqpoint{1.764706in}{1.700000in}}%
\pgfusepath{clip}%
\pgfsetbuttcap%
\pgfsetroundjoin%
\definecolor{currentfill}{rgb}{0.973271,0.850724,0.762998}%
\pgfsetfillcolor{currentfill}%
\pgfsetlinewidth{0.311001pt}%
\definecolor{currentstroke}{rgb}{1.000000,1.000000,1.000000}%
\pgfsetstrokecolor{currentstroke}%
\pgfsetdash{}{0pt}%
\pgfpathmoveto{\pgfqpoint{4.099654in}{0.989224in}}%
\pgfpathcurveto{\pgfqpoint{4.106787in}{0.989224in}}{\pgfqpoint{4.113629in}{0.992058in}}{\pgfqpoint{4.118672in}{0.997101in}}%
\pgfpathcurveto{\pgfqpoint{4.123716in}{1.002145in}}{\pgfqpoint{4.126550in}{1.008987in}}{\pgfqpoint{4.126550in}{1.016119in}}%
\pgfpathcurveto{\pgfqpoint{4.126550in}{1.023252in}}{\pgfqpoint{4.123716in}{1.030094in}}{\pgfqpoint{4.118672in}{1.035138in}}%
\pgfpathcurveto{\pgfqpoint{4.113629in}{1.040181in}}{\pgfqpoint{4.106787in}{1.043015in}}{\pgfqpoint{4.099654in}{1.043015in}}%
\pgfpathcurveto{\pgfqpoint{4.092521in}{1.043015in}}{\pgfqpoint{4.085680in}{1.040181in}}{\pgfqpoint{4.080636in}{1.035138in}}%
\pgfpathcurveto{\pgfqpoint{4.075592in}{1.030094in}}{\pgfqpoint{4.072759in}{1.023252in}}{\pgfqpoint{4.072759in}{1.016119in}}%
\pgfpathcurveto{\pgfqpoint{4.072759in}{1.008987in}}{\pgfqpoint{4.075592in}{1.002145in}}{\pgfqpoint{4.080636in}{0.997101in}}%
\pgfpathcurveto{\pgfqpoint{4.085680in}{0.992058in}}{\pgfqpoint{4.092521in}{0.989224in}}{\pgfqpoint{4.099654in}{0.989224in}}%
\pgfpathclose%
\pgfusepath{stroke,fill}%
\end{pgfscope}%
\begin{pgfscope}%
\pgfpathrectangle{\pgfqpoint{2.867647in}{0.500000in}}{\pgfqpoint{1.764706in}{1.700000in}}%
\pgfusepath{clip}%
\pgfsetbuttcap%
\pgfsetroundjoin%
\definecolor{currentfill}{rgb}{0.965753,0.732351,0.592427}%
\pgfsetfillcolor{currentfill}%
\pgfsetlinewidth{0.311001pt}%
\definecolor{currentstroke}{rgb}{1.000000,1.000000,1.000000}%
\pgfsetstrokecolor{currentstroke}%
\pgfsetdash{}{0pt}%
\pgfpathmoveto{\pgfqpoint{4.021943in}{1.568642in}}%
\pgfpathcurveto{\pgfqpoint{4.029076in}{1.568642in}}{\pgfqpoint{4.035918in}{1.571476in}}{\pgfqpoint{4.040962in}{1.576520in}}%
\pgfpathcurveto{\pgfqpoint{4.046005in}{1.581563in}}{\pgfqpoint{4.048839in}{1.588405in}}{\pgfqpoint{4.048839in}{1.595538in}}%
\pgfpathcurveto{\pgfqpoint{4.048839in}{1.602671in}}{\pgfqpoint{4.046005in}{1.609512in}}{\pgfqpoint{4.040962in}{1.614556in}}%
\pgfpathcurveto{\pgfqpoint{4.035918in}{1.619600in}}{\pgfqpoint{4.029076in}{1.622434in}}{\pgfqpoint{4.021943in}{1.622434in}}%
\pgfpathcurveto{\pgfqpoint{4.014811in}{1.622434in}}{\pgfqpoint{4.007969in}{1.619600in}}{\pgfqpoint{4.002925in}{1.614556in}}%
\pgfpathcurveto{\pgfqpoint{3.997882in}{1.609512in}}{\pgfqpoint{3.995048in}{1.602671in}}{\pgfqpoint{3.995048in}{1.595538in}}%
\pgfpathcurveto{\pgfqpoint{3.995048in}{1.588405in}}{\pgfqpoint{3.997882in}{1.581563in}}{\pgfqpoint{4.002925in}{1.576520in}}%
\pgfpathcurveto{\pgfqpoint{4.007969in}{1.571476in}}{\pgfqpoint{4.014811in}{1.568642in}}{\pgfqpoint{4.021943in}{1.568642in}}%
\pgfpathclose%
\pgfusepath{stroke,fill}%
\end{pgfscope}%
\begin{pgfscope}%
\pgfpathrectangle{\pgfqpoint{2.867647in}{0.500000in}}{\pgfqpoint{1.764706in}{1.700000in}}%
\pgfusepath{clip}%
\pgfsetbuttcap%
\pgfsetroundjoin%
\definecolor{currentfill}{rgb}{0.965042,0.701564,0.552889}%
\pgfsetfillcolor{currentfill}%
\pgfsetlinewidth{0.311001pt}%
\definecolor{currentstroke}{rgb}{1.000000,1.000000,1.000000}%
\pgfsetstrokecolor{currentstroke}%
\pgfsetdash{}{0pt}%
\pgfpathmoveto{\pgfqpoint{4.041387in}{0.888710in}}%
\pgfpathcurveto{\pgfqpoint{4.048520in}{0.888710in}}{\pgfqpoint{4.055362in}{0.891544in}}{\pgfqpoint{4.060405in}{0.896588in}}%
\pgfpathcurveto{\pgfqpoint{4.065449in}{0.901632in}}{\pgfqpoint{4.068283in}{0.908473in}}{\pgfqpoint{4.068283in}{0.915606in}}%
\pgfpathcurveto{\pgfqpoint{4.068283in}{0.922739in}}{\pgfqpoint{4.065449in}{0.929581in}}{\pgfqpoint{4.060405in}{0.934624in}}%
\pgfpathcurveto{\pgfqpoint{4.055362in}{0.939668in}}{\pgfqpoint{4.048520in}{0.942502in}}{\pgfqpoint{4.041387in}{0.942502in}}%
\pgfpathcurveto{\pgfqpoint{4.034254in}{0.942502in}}{\pgfqpoint{4.027413in}{0.939668in}}{\pgfqpoint{4.022369in}{0.934624in}}%
\pgfpathcurveto{\pgfqpoint{4.017325in}{0.929581in}}{\pgfqpoint{4.014492in}{0.922739in}}{\pgfqpoint{4.014492in}{0.915606in}}%
\pgfpathcurveto{\pgfqpoint{4.014492in}{0.908473in}}{\pgfqpoint{4.017325in}{0.901632in}}{\pgfqpoint{4.022369in}{0.896588in}}%
\pgfpathcurveto{\pgfqpoint{4.027413in}{0.891544in}}{\pgfqpoint{4.034254in}{0.888710in}}{\pgfqpoint{4.041387in}{0.888710in}}%
\pgfpathclose%
\pgfusepath{stroke,fill}%
\end{pgfscope}%
\begin{pgfscope}%
\pgfpathrectangle{\pgfqpoint{2.867647in}{0.500000in}}{\pgfqpoint{1.764706in}{1.700000in}}%
\pgfusepath{clip}%
\pgfsetbuttcap%
\pgfsetroundjoin%
\definecolor{currentfill}{rgb}{0.974412,0.862387,0.780156}%
\pgfsetfillcolor{currentfill}%
\pgfsetlinewidth{0.311001pt}%
\definecolor{currentstroke}{rgb}{1.000000,1.000000,1.000000}%
\pgfsetstrokecolor{currentstroke}%
\pgfsetdash{}{0pt}%
\pgfpathmoveto{\pgfqpoint{4.252741in}{1.342064in}}%
\pgfpathcurveto{\pgfqpoint{4.259874in}{1.342064in}}{\pgfqpoint{4.266716in}{1.344898in}}{\pgfqpoint{4.271759in}{1.349941in}}%
\pgfpathcurveto{\pgfqpoint{4.276803in}{1.354985in}}{\pgfqpoint{4.279637in}{1.361827in}}{\pgfqpoint{4.279637in}{1.368960in}}%
\pgfpathcurveto{\pgfqpoint{4.279637in}{1.376092in}}{\pgfqpoint{4.276803in}{1.382934in}}{\pgfqpoint{4.271759in}{1.387978in}}%
\pgfpathcurveto{\pgfqpoint{4.266716in}{1.393021in}}{\pgfqpoint{4.259874in}{1.395855in}}{\pgfqpoint{4.252741in}{1.395855in}}%
\pgfpathcurveto{\pgfqpoint{4.245608in}{1.395855in}}{\pgfqpoint{4.238767in}{1.393021in}}{\pgfqpoint{4.233723in}{1.387978in}}%
\pgfpathcurveto{\pgfqpoint{4.228679in}{1.382934in}}{\pgfqpoint{4.225845in}{1.376092in}}{\pgfqpoint{4.225845in}{1.368960in}}%
\pgfpathcurveto{\pgfqpoint{4.225845in}{1.361827in}}{\pgfqpoint{4.228679in}{1.354985in}}{\pgfqpoint{4.233723in}{1.349941in}}%
\pgfpathcurveto{\pgfqpoint{4.238767in}{1.344898in}}{\pgfqpoint{4.245608in}{1.342064in}}{\pgfqpoint{4.252741in}{1.342064in}}%
\pgfpathclose%
\pgfusepath{stroke,fill}%
\end{pgfscope}%
\begin{pgfscope}%
\pgfpathrectangle{\pgfqpoint{2.867647in}{0.500000in}}{\pgfqpoint{1.764706in}{1.700000in}}%
\pgfusepath{clip}%
\pgfsetbuttcap%
\pgfsetroundjoin%
\definecolor{currentfill}{rgb}{0.964799,0.689101,0.537560}%
\pgfsetfillcolor{currentfill}%
\pgfsetlinewidth{0.311001pt}%
\definecolor{currentstroke}{rgb}{1.000000,1.000000,1.000000}%
\pgfsetstrokecolor{currentstroke}%
\pgfsetdash{}{0pt}%
\pgfpathmoveto{\pgfqpoint{3.982955in}{1.650498in}}%
\pgfpathcurveto{\pgfqpoint{3.990088in}{1.650498in}}{\pgfqpoint{3.996930in}{1.653332in}}{\pgfqpoint{4.001973in}{1.658376in}}%
\pgfpathcurveto{\pgfqpoint{4.007017in}{1.663419in}}{\pgfqpoint{4.009851in}{1.670261in}}{\pgfqpoint{4.009851in}{1.677394in}}%
\pgfpathcurveto{\pgfqpoint{4.009851in}{1.684527in}}{\pgfqpoint{4.007017in}{1.691368in}}{\pgfqpoint{4.001973in}{1.696412in}}%
\pgfpathcurveto{\pgfqpoint{3.996930in}{1.701456in}}{\pgfqpoint{3.990088in}{1.704290in}}{\pgfqpoint{3.982955in}{1.704290in}}%
\pgfpathcurveto{\pgfqpoint{3.975822in}{1.704290in}}{\pgfqpoint{3.968981in}{1.701456in}}{\pgfqpoint{3.963937in}{1.696412in}}%
\pgfpathcurveto{\pgfqpoint{3.958893in}{1.691368in}}{\pgfqpoint{3.956059in}{1.684527in}}{\pgfqpoint{3.956059in}{1.677394in}}%
\pgfpathcurveto{\pgfqpoint{3.956059in}{1.670261in}}{\pgfqpoint{3.958893in}{1.663419in}}{\pgfqpoint{3.963937in}{1.658376in}}%
\pgfpathcurveto{\pgfqpoint{3.968981in}{1.653332in}}{\pgfqpoint{3.975822in}{1.650498in}}{\pgfqpoint{3.982955in}{1.650498in}}%
\pgfpathclose%
\pgfusepath{stroke,fill}%
\end{pgfscope}%
\begin{pgfscope}%
\pgfpathrectangle{\pgfqpoint{2.867647in}{0.500000in}}{\pgfqpoint{1.764706in}{1.700000in}}%
\pgfusepath{clip}%
\pgfsetbuttcap%
\pgfsetroundjoin%
\definecolor{currentfill}{rgb}{0.975644,0.874038,0.797253}%
\pgfsetfillcolor{currentfill}%
\pgfsetlinewidth{0.311001pt}%
\definecolor{currentstroke}{rgb}{1.000000,1.000000,1.000000}%
\pgfsetstrokecolor{currentstroke}%
\pgfsetdash{}{0pt}%
\pgfpathmoveto{\pgfqpoint{4.102372in}{1.027514in}}%
\pgfpathcurveto{\pgfqpoint{4.109505in}{1.027514in}}{\pgfqpoint{4.116347in}{1.030348in}}{\pgfqpoint{4.121390in}{1.035391in}}%
\pgfpathcurveto{\pgfqpoint{4.126434in}{1.040435in}}{\pgfqpoint{4.129268in}{1.047277in}}{\pgfqpoint{4.129268in}{1.054409in}}%
\pgfpathcurveto{\pgfqpoint{4.129268in}{1.061542in}}{\pgfqpoint{4.126434in}{1.068384in}}{\pgfqpoint{4.121390in}{1.073428in}}%
\pgfpathcurveto{\pgfqpoint{4.116347in}{1.078471in}}{\pgfqpoint{4.109505in}{1.081305in}}{\pgfqpoint{4.102372in}{1.081305in}}%
\pgfpathcurveto{\pgfqpoint{4.095239in}{1.081305in}}{\pgfqpoint{4.088398in}{1.078471in}}{\pgfqpoint{4.083354in}{1.073428in}}%
\pgfpathcurveto{\pgfqpoint{4.078310in}{1.068384in}}{\pgfqpoint{4.075476in}{1.061542in}}{\pgfqpoint{4.075476in}{1.054409in}}%
\pgfpathcurveto{\pgfqpoint{4.075476in}{1.047277in}}{\pgfqpoint{4.078310in}{1.040435in}}{\pgfqpoint{4.083354in}{1.035391in}}%
\pgfpathcurveto{\pgfqpoint{4.088398in}{1.030348in}}{\pgfqpoint{4.095239in}{1.027514in}}{\pgfqpoint{4.102372in}{1.027514in}}%
\pgfpathclose%
\pgfusepath{stroke,fill}%
\end{pgfscope}%
\begin{pgfscope}%
\pgfpathrectangle{\pgfqpoint{2.867647in}{0.500000in}}{\pgfqpoint{1.764706in}{1.700000in}}%
\pgfusepath{clip}%
\pgfsetbuttcap%
\pgfsetroundjoin%
\definecolor{currentfill}{rgb}{0.970255,0.815666,0.711203}%
\pgfsetfillcolor{currentfill}%
\pgfsetlinewidth{0.311001pt}%
\definecolor{currentstroke}{rgb}{1.000000,1.000000,1.000000}%
\pgfsetstrokecolor{currentstroke}%
\pgfsetdash{}{0pt}%
\pgfpathmoveto{\pgfqpoint{4.098661in}{1.224088in}}%
\pgfpathcurveto{\pgfqpoint{4.105794in}{1.224088in}}{\pgfqpoint{4.112636in}{1.226922in}}{\pgfqpoint{4.117679in}{1.231965in}}%
\pgfpathcurveto{\pgfqpoint{4.122723in}{1.237009in}}{\pgfqpoint{4.125557in}{1.243851in}}{\pgfqpoint{4.125557in}{1.250983in}}%
\pgfpathcurveto{\pgfqpoint{4.125557in}{1.258116in}}{\pgfqpoint{4.122723in}{1.264958in}}{\pgfqpoint{4.117679in}{1.270002in}}%
\pgfpathcurveto{\pgfqpoint{4.112636in}{1.275045in}}{\pgfqpoint{4.105794in}{1.277879in}}{\pgfqpoint{4.098661in}{1.277879in}}%
\pgfpathcurveto{\pgfqpoint{4.091528in}{1.277879in}}{\pgfqpoint{4.084687in}{1.275045in}}{\pgfqpoint{4.079643in}{1.270002in}}%
\pgfpathcurveto{\pgfqpoint{4.074599in}{1.264958in}}{\pgfqpoint{4.071766in}{1.258116in}}{\pgfqpoint{4.071766in}{1.250983in}}%
\pgfpathcurveto{\pgfqpoint{4.071766in}{1.243851in}}{\pgfqpoint{4.074599in}{1.237009in}}{\pgfqpoint{4.079643in}{1.231965in}}%
\pgfpathcurveto{\pgfqpoint{4.084687in}{1.226922in}}{\pgfqpoint{4.091528in}{1.224088in}}{\pgfqpoint{4.098661in}{1.224088in}}%
\pgfpathclose%
\pgfusepath{stroke,fill}%
\end{pgfscope}%
\begin{pgfscope}%
\pgfpathrectangle{\pgfqpoint{2.867647in}{0.500000in}}{\pgfqpoint{1.764706in}{1.700000in}}%
\pgfusepath{clip}%
\pgfsetbuttcap%
\pgfsetroundjoin%
\definecolor{currentfill}{rgb}{0.971202,0.827364,0.728520}%
\pgfsetfillcolor{currentfill}%
\pgfsetlinewidth{0.311001pt}%
\definecolor{currentstroke}{rgb}{1.000000,1.000000,1.000000}%
\pgfsetstrokecolor{currentstroke}%
\pgfsetdash{}{0pt}%
\pgfpathmoveto{\pgfqpoint{4.052408in}{1.586127in}}%
\pgfpathcurveto{\pgfqpoint{4.059541in}{1.586127in}}{\pgfqpoint{4.066382in}{1.588960in}}{\pgfqpoint{4.071426in}{1.594004in}}%
\pgfpathcurveto{\pgfqpoint{4.076470in}{1.599048in}}{\pgfqpoint{4.079304in}{1.605889in}}{\pgfqpoint{4.079304in}{1.613022in}}%
\pgfpathcurveto{\pgfqpoint{4.079304in}{1.620155in}}{\pgfqpoint{4.076470in}{1.626997in}}{\pgfqpoint{4.071426in}{1.632040in}}%
\pgfpathcurveto{\pgfqpoint{4.066382in}{1.637084in}}{\pgfqpoint{4.059541in}{1.639918in}}{\pgfqpoint{4.052408in}{1.639918in}}%
\pgfpathcurveto{\pgfqpoint{4.045275in}{1.639918in}}{\pgfqpoint{4.038433in}{1.637084in}}{\pgfqpoint{4.033390in}{1.632040in}}%
\pgfpathcurveto{\pgfqpoint{4.028346in}{1.626997in}}{\pgfqpoint{4.025512in}{1.620155in}}{\pgfqpoint{4.025512in}{1.613022in}}%
\pgfpathcurveto{\pgfqpoint{4.025512in}{1.605889in}}{\pgfqpoint{4.028346in}{1.599048in}}{\pgfqpoint{4.033390in}{1.594004in}}%
\pgfpathcurveto{\pgfqpoint{4.038433in}{1.588960in}}{\pgfqpoint{4.045275in}{1.586127in}}{\pgfqpoint{4.052408in}{1.586127in}}%
\pgfpathclose%
\pgfusepath{stroke,fill}%
\end{pgfscope}%
\begin{pgfscope}%
\pgfpathrectangle{\pgfqpoint{2.867647in}{0.500000in}}{\pgfqpoint{1.764706in}{1.700000in}}%
\pgfusepath{clip}%
\pgfsetbuttcap%
\pgfsetroundjoin%
\definecolor{currentfill}{rgb}{0.965042,0.701564,0.552889}%
\pgfsetfillcolor{currentfill}%
\pgfsetlinewidth{0.311001pt}%
\definecolor{currentstroke}{rgb}{1.000000,1.000000,1.000000}%
\pgfsetstrokecolor{currentstroke}%
\pgfsetdash{}{0pt}%
\pgfpathmoveto{\pgfqpoint{3.987823in}{0.957319in}}%
\pgfpathcurveto{\pgfqpoint{3.994955in}{0.957319in}}{\pgfqpoint{4.001797in}{0.960153in}}{\pgfqpoint{4.006841in}{0.965197in}}%
\pgfpathcurveto{\pgfqpoint{4.011884in}{0.970241in}}{\pgfqpoint{4.014718in}{0.977082in}}{\pgfqpoint{4.014718in}{0.984215in}}%
\pgfpathcurveto{\pgfqpoint{4.014718in}{0.991348in}}{\pgfqpoint{4.011884in}{0.998190in}}{\pgfqpoint{4.006841in}{1.003233in}}%
\pgfpathcurveto{\pgfqpoint{4.001797in}{1.008277in}}{\pgfqpoint{3.994955in}{1.011111in}}{\pgfqpoint{3.987823in}{1.011111in}}%
\pgfpathcurveto{\pgfqpoint{3.980690in}{1.011111in}}{\pgfqpoint{3.973848in}{1.008277in}}{\pgfqpoint{3.968804in}{1.003233in}}%
\pgfpathcurveto{\pgfqpoint{3.963761in}{0.998190in}}{\pgfqpoint{3.960927in}{0.991348in}}{\pgfqpoint{3.960927in}{0.984215in}}%
\pgfpathcurveto{\pgfqpoint{3.960927in}{0.977082in}}{\pgfqpoint{3.963761in}{0.970241in}}{\pgfqpoint{3.968804in}{0.965197in}}%
\pgfpathcurveto{\pgfqpoint{3.973848in}{0.960153in}}{\pgfqpoint{3.980690in}{0.957319in}}{\pgfqpoint{3.987823in}{0.957319in}}%
\pgfpathclose%
\pgfusepath{stroke,fill}%
\end{pgfscope}%
\begin{pgfscope}%
\pgfpathrectangle{\pgfqpoint{2.867647in}{0.500000in}}{\pgfqpoint{1.764706in}{1.700000in}}%
\pgfusepath{clip}%
\pgfsetbuttcap%
\pgfsetroundjoin%
\definecolor{currentfill}{rgb}{0.964433,0.670254,0.515093}%
\pgfsetfillcolor{currentfill}%
\pgfsetlinewidth{0.311001pt}%
\definecolor{currentstroke}{rgb}{1.000000,1.000000,1.000000}%
\pgfsetstrokecolor{currentstroke}%
\pgfsetdash{}{0pt}%
\pgfpathmoveto{\pgfqpoint{4.014435in}{0.882637in}}%
\pgfpathcurveto{\pgfqpoint{4.021567in}{0.882637in}}{\pgfqpoint{4.028409in}{0.885471in}}{\pgfqpoint{4.033453in}{0.890515in}}%
\pgfpathcurveto{\pgfqpoint{4.038496in}{0.895559in}}{\pgfqpoint{4.041330in}{0.902400in}}{\pgfqpoint{4.041330in}{0.909533in}}%
\pgfpathcurveto{\pgfqpoint{4.041330in}{0.916666in}}{\pgfqpoint{4.038496in}{0.923508in}}{\pgfqpoint{4.033453in}{0.928551in}}%
\pgfpathcurveto{\pgfqpoint{4.028409in}{0.933595in}}{\pgfqpoint{4.021567in}{0.936429in}}{\pgfqpoint{4.014435in}{0.936429in}}%
\pgfpathcurveto{\pgfqpoint{4.007302in}{0.936429in}}{\pgfqpoint{4.000460in}{0.933595in}}{\pgfqpoint{3.995416in}{0.928551in}}%
\pgfpathcurveto{\pgfqpoint{3.990373in}{0.923508in}}{\pgfqpoint{3.987539in}{0.916666in}}{\pgfqpoint{3.987539in}{0.909533in}}%
\pgfpathcurveto{\pgfqpoint{3.987539in}{0.902400in}}{\pgfqpoint{3.990373in}{0.895559in}}{\pgfqpoint{3.995416in}{0.890515in}}%
\pgfpathcurveto{\pgfqpoint{4.000460in}{0.885471in}}{\pgfqpoint{4.007302in}{0.882637in}}{\pgfqpoint{4.014435in}{0.882637in}}%
\pgfpathclose%
\pgfusepath{stroke,fill}%
\end{pgfscope}%
\begin{pgfscope}%
\pgfpathrectangle{\pgfqpoint{2.867647in}{0.500000in}}{\pgfqpoint{1.764706in}{1.700000in}}%
\pgfusepath{clip}%
\pgfsetbuttcap%
\pgfsetroundjoin%
\definecolor{currentfill}{rgb}{0.980678,0.914765,0.856766}%
\pgfsetfillcolor{currentfill}%
\pgfsetlinewidth{0.311001pt}%
\definecolor{currentstroke}{rgb}{1.000000,1.000000,1.000000}%
\pgfsetstrokecolor{currentstroke}%
\pgfsetdash{}{0pt}%
\pgfpathmoveto{\pgfqpoint{4.196539in}{1.200270in}}%
\pgfpathcurveto{\pgfqpoint{4.203672in}{1.200270in}}{\pgfqpoint{4.210513in}{1.203104in}}{\pgfqpoint{4.215557in}{1.208148in}}%
\pgfpathcurveto{\pgfqpoint{4.220601in}{1.213191in}}{\pgfqpoint{4.223434in}{1.220033in}}{\pgfqpoint{4.223434in}{1.227166in}}%
\pgfpathcurveto{\pgfqpoint{4.223434in}{1.234299in}}{\pgfqpoint{4.220601in}{1.241140in}}{\pgfqpoint{4.215557in}{1.246184in}}%
\pgfpathcurveto{\pgfqpoint{4.210513in}{1.251228in}}{\pgfqpoint{4.203672in}{1.254062in}}{\pgfqpoint{4.196539in}{1.254062in}}%
\pgfpathcurveto{\pgfqpoint{4.189406in}{1.254062in}}{\pgfqpoint{4.182564in}{1.251228in}}{\pgfqpoint{4.177521in}{1.246184in}}%
\pgfpathcurveto{\pgfqpoint{4.172477in}{1.241140in}}{\pgfqpoint{4.169643in}{1.234299in}}{\pgfqpoint{4.169643in}{1.227166in}}%
\pgfpathcurveto{\pgfqpoint{4.169643in}{1.220033in}}{\pgfqpoint{4.172477in}{1.213191in}}{\pgfqpoint{4.177521in}{1.208148in}}%
\pgfpathcurveto{\pgfqpoint{4.182564in}{1.203104in}}{\pgfqpoint{4.189406in}{1.200270in}}{\pgfqpoint{4.196539in}{1.200270in}}%
\pgfpathclose%
\pgfusepath{stroke,fill}%
\end{pgfscope}%
\begin{pgfscope}%
\pgfpathrectangle{\pgfqpoint{2.867647in}{0.500000in}}{\pgfqpoint{1.764706in}{1.700000in}}%
\pgfusepath{clip}%
\pgfsetbuttcap%
\pgfsetroundjoin%
\definecolor{currentfill}{rgb}{0.968931,0.798091,0.685123}%
\pgfsetfillcolor{currentfill}%
\pgfsetlinewidth{0.311001pt}%
\definecolor{currentstroke}{rgb}{1.000000,1.000000,1.000000}%
\pgfsetstrokecolor{currentstroke}%
\pgfsetdash{}{0pt}%
\pgfpathmoveto{\pgfqpoint{4.225187in}{1.059273in}}%
\pgfpathcurveto{\pgfqpoint{4.232320in}{1.059273in}}{\pgfqpoint{4.239162in}{1.062107in}}{\pgfqpoint{4.244206in}{1.067150in}}%
\pgfpathcurveto{\pgfqpoint{4.249249in}{1.072194in}}{\pgfqpoint{4.252083in}{1.079036in}}{\pgfqpoint{4.252083in}{1.086168in}}%
\pgfpathcurveto{\pgfqpoint{4.252083in}{1.093301in}}{\pgfqpoint{4.249249in}{1.100143in}}{\pgfqpoint{4.244206in}{1.105187in}}%
\pgfpathcurveto{\pgfqpoint{4.239162in}{1.110230in}}{\pgfqpoint{4.232320in}{1.113064in}}{\pgfqpoint{4.225187in}{1.113064in}}%
\pgfpathcurveto{\pgfqpoint{4.218055in}{1.113064in}}{\pgfqpoint{4.211213in}{1.110230in}}{\pgfqpoint{4.206169in}{1.105187in}}%
\pgfpathcurveto{\pgfqpoint{4.201126in}{1.100143in}}{\pgfqpoint{4.198292in}{1.093301in}}{\pgfqpoint{4.198292in}{1.086168in}}%
\pgfpathcurveto{\pgfqpoint{4.198292in}{1.079036in}}{\pgfqpoint{4.201126in}{1.072194in}}{\pgfqpoint{4.206169in}{1.067150in}}%
\pgfpathcurveto{\pgfqpoint{4.211213in}{1.062107in}}{\pgfqpoint{4.218055in}{1.059273in}}{\pgfqpoint{4.225187in}{1.059273in}}%
\pgfpathclose%
\pgfusepath{stroke,fill}%
\end{pgfscope}%
\begin{pgfscope}%
\pgfpathrectangle{\pgfqpoint{2.867647in}{0.500000in}}{\pgfqpoint{1.764706in}{1.700000in}}%
\pgfusepath{clip}%
\pgfsetbuttcap%
\pgfsetroundjoin%
\definecolor{currentfill}{rgb}{0.963559,0.632016,0.472047}%
\pgfsetfillcolor{currentfill}%
\pgfsetlinewidth{0.311001pt}%
\definecolor{currentstroke}{rgb}{1.000000,1.000000,1.000000}%
\pgfsetstrokecolor{currentstroke}%
\pgfsetdash{}{0pt}%
\pgfpathmoveto{\pgfqpoint{4.044432in}{1.418235in}}%
\pgfpathcurveto{\pgfqpoint{4.051565in}{1.418235in}}{\pgfqpoint{4.058407in}{1.421069in}}{\pgfqpoint{4.063450in}{1.426113in}}%
\pgfpathcurveto{\pgfqpoint{4.068494in}{1.431156in}}{\pgfqpoint{4.071328in}{1.437998in}}{\pgfqpoint{4.071328in}{1.445131in}}%
\pgfpathcurveto{\pgfqpoint{4.071328in}{1.452263in}}{\pgfqpoint{4.068494in}{1.459105in}}{\pgfqpoint{4.063450in}{1.464149in}}%
\pgfpathcurveto{\pgfqpoint{4.058407in}{1.469192in}}{\pgfqpoint{4.051565in}{1.472026in}}{\pgfqpoint{4.044432in}{1.472026in}}%
\pgfpathcurveto{\pgfqpoint{4.037299in}{1.472026in}}{\pgfqpoint{4.030458in}{1.469192in}}{\pgfqpoint{4.025414in}{1.464149in}}%
\pgfpathcurveto{\pgfqpoint{4.020370in}{1.459105in}}{\pgfqpoint{4.017536in}{1.452263in}}{\pgfqpoint{4.017536in}{1.445131in}}%
\pgfpathcurveto{\pgfqpoint{4.017536in}{1.437998in}}{\pgfqpoint{4.020370in}{1.431156in}}{\pgfqpoint{4.025414in}{1.426113in}}%
\pgfpathcurveto{\pgfqpoint{4.030458in}{1.421069in}}{\pgfqpoint{4.037299in}{1.418235in}}{\pgfqpoint{4.044432in}{1.418235in}}%
\pgfpathclose%
\pgfusepath{stroke,fill}%
\end{pgfscope}%
\begin{pgfscope}%
\pgfpathrectangle{\pgfqpoint{2.867647in}{0.500000in}}{\pgfqpoint{1.764706in}{1.700000in}}%
\pgfusepath{clip}%
\pgfsetbuttcap%
\pgfsetroundjoin%
\definecolor{currentfill}{rgb}{0.965592,0.726236,0.584384}%
\pgfsetfillcolor{currentfill}%
\pgfsetlinewidth{0.311001pt}%
\definecolor{currentstroke}{rgb}{1.000000,1.000000,1.000000}%
\pgfsetstrokecolor{currentstroke}%
\pgfsetdash{}{0pt}%
\pgfpathmoveto{\pgfqpoint{4.042310in}{1.497998in}}%
\pgfpathcurveto{\pgfqpoint{4.049443in}{1.497998in}}{\pgfqpoint{4.056285in}{1.500832in}}{\pgfqpoint{4.061328in}{1.505875in}}%
\pgfpathcurveto{\pgfqpoint{4.066372in}{1.510919in}}{\pgfqpoint{4.069206in}{1.517761in}}{\pgfqpoint{4.069206in}{1.524894in}}%
\pgfpathcurveto{\pgfqpoint{4.069206in}{1.532026in}}{\pgfqpoint{4.066372in}{1.538868in}}{\pgfqpoint{4.061328in}{1.543912in}}%
\pgfpathcurveto{\pgfqpoint{4.056285in}{1.548955in}}{\pgfqpoint{4.049443in}{1.551789in}}{\pgfqpoint{4.042310in}{1.551789in}}%
\pgfpathcurveto{\pgfqpoint{4.035177in}{1.551789in}}{\pgfqpoint{4.028336in}{1.548955in}}{\pgfqpoint{4.023292in}{1.543912in}}%
\pgfpathcurveto{\pgfqpoint{4.018248in}{1.538868in}}{\pgfqpoint{4.015415in}{1.532026in}}{\pgfqpoint{4.015415in}{1.524894in}}%
\pgfpathcurveto{\pgfqpoint{4.015415in}{1.517761in}}{\pgfqpoint{4.018248in}{1.510919in}}{\pgfqpoint{4.023292in}{1.505875in}}%
\pgfpathcurveto{\pgfqpoint{4.028336in}{1.500832in}}{\pgfqpoint{4.035177in}{1.497998in}}{\pgfqpoint{4.042310in}{1.497998in}}%
\pgfpathclose%
\pgfusepath{stroke,fill}%
\end{pgfscope}%
\begin{pgfscope}%
\pgfpathrectangle{\pgfqpoint{2.867647in}{0.500000in}}{\pgfqpoint{1.764706in}{1.700000in}}%
\pgfusepath{clip}%
\pgfsetbuttcap%
\pgfsetroundjoin%
\definecolor{currentfill}{rgb}{0.975644,0.874038,0.797253}%
\pgfsetfillcolor{currentfill}%
\pgfsetlinewidth{0.311001pt}%
\definecolor{currentstroke}{rgb}{1.000000,1.000000,1.000000}%
\pgfsetstrokecolor{currentstroke}%
\pgfsetdash{}{0pt}%
\pgfpathmoveto{\pgfqpoint{4.241464in}{1.371223in}}%
\pgfpathcurveto{\pgfqpoint{4.248597in}{1.371223in}}{\pgfqpoint{4.255438in}{1.374057in}}{\pgfqpoint{4.260482in}{1.379100in}}%
\pgfpathcurveto{\pgfqpoint{4.265526in}{1.384144in}}{\pgfqpoint{4.268360in}{1.390986in}}{\pgfqpoint{4.268360in}{1.398118in}}%
\pgfpathcurveto{\pgfqpoint{4.268360in}{1.405251in}}{\pgfqpoint{4.265526in}{1.412093in}}{\pgfqpoint{4.260482in}{1.417137in}}%
\pgfpathcurveto{\pgfqpoint{4.255438in}{1.422180in}}{\pgfqpoint{4.248597in}{1.425014in}}{\pgfqpoint{4.241464in}{1.425014in}}%
\pgfpathcurveto{\pgfqpoint{4.234331in}{1.425014in}}{\pgfqpoint{4.227489in}{1.422180in}}{\pgfqpoint{4.222446in}{1.417137in}}%
\pgfpathcurveto{\pgfqpoint{4.217402in}{1.412093in}}{\pgfqpoint{4.214568in}{1.405251in}}{\pgfqpoint{4.214568in}{1.398118in}}%
\pgfpathcurveto{\pgfqpoint{4.214568in}{1.390986in}}{\pgfqpoint{4.217402in}{1.384144in}}{\pgfqpoint{4.222446in}{1.379100in}}%
\pgfpathcurveto{\pgfqpoint{4.227489in}{1.374057in}}{\pgfqpoint{4.234331in}{1.371223in}}{\pgfqpoint{4.241464in}{1.371223in}}%
\pgfpathclose%
\pgfusepath{stroke,fill}%
\end{pgfscope}%
\begin{pgfscope}%
\pgfpathrectangle{\pgfqpoint{2.867647in}{0.500000in}}{\pgfqpoint{1.764706in}{1.700000in}}%
\pgfusepath{clip}%
\pgfsetbuttcap%
\pgfsetroundjoin%
\definecolor{currentfill}{rgb}{0.964032,0.651225,0.493258}%
\pgfsetfillcolor{currentfill}%
\pgfsetlinewidth{0.311001pt}%
\definecolor{currentstroke}{rgb}{1.000000,1.000000,1.000000}%
\pgfsetstrokecolor{currentstroke}%
\pgfsetdash{}{0pt}%
\pgfpathmoveto{\pgfqpoint{4.064212in}{0.868232in}}%
\pgfpathcurveto{\pgfqpoint{4.071345in}{0.868232in}}{\pgfqpoint{4.078186in}{0.871066in}}{\pgfqpoint{4.083230in}{0.876110in}}%
\pgfpathcurveto{\pgfqpoint{4.088274in}{0.881154in}}{\pgfqpoint{4.091107in}{0.887995in}}{\pgfqpoint{4.091107in}{0.895128in}}%
\pgfpathcurveto{\pgfqpoint{4.091107in}{0.902261in}}{\pgfqpoint{4.088274in}{0.909103in}}{\pgfqpoint{4.083230in}{0.914146in}}%
\pgfpathcurveto{\pgfqpoint{4.078186in}{0.919190in}}{\pgfqpoint{4.071345in}{0.922024in}}{\pgfqpoint{4.064212in}{0.922024in}}%
\pgfpathcurveto{\pgfqpoint{4.057079in}{0.922024in}}{\pgfqpoint{4.050237in}{0.919190in}}{\pgfqpoint{4.045194in}{0.914146in}}%
\pgfpathcurveto{\pgfqpoint{4.040150in}{0.909103in}}{\pgfqpoint{4.037316in}{0.902261in}}{\pgfqpoint{4.037316in}{0.895128in}}%
\pgfpathcurveto{\pgfqpoint{4.037316in}{0.887995in}}{\pgfqpoint{4.040150in}{0.881154in}}{\pgfqpoint{4.045194in}{0.876110in}}%
\pgfpathcurveto{\pgfqpoint{4.050237in}{0.871066in}}{\pgfqpoint{4.057079in}{0.868232in}}{\pgfqpoint{4.064212in}{0.868232in}}%
\pgfpathclose%
\pgfusepath{stroke,fill}%
\end{pgfscope}%
\begin{pgfscope}%
\pgfpathrectangle{\pgfqpoint{2.867647in}{0.500000in}}{\pgfqpoint{1.764706in}{1.700000in}}%
\pgfusepath{clip}%
\pgfsetbuttcap%
\pgfsetroundjoin%
\definecolor{currentfill}{rgb}{0.978376,0.897317,0.831308}%
\pgfsetfillcolor{currentfill}%
\pgfsetlinewidth{0.311001pt}%
\definecolor{currentstroke}{rgb}{1.000000,1.000000,1.000000}%
\pgfsetstrokecolor{currentstroke}%
\pgfsetdash{}{0pt}%
\pgfpathmoveto{\pgfqpoint{4.185434in}{1.526793in}}%
\pgfpathcurveto{\pgfqpoint{4.192567in}{1.526793in}}{\pgfqpoint{4.199409in}{1.529627in}}{\pgfqpoint{4.204452in}{1.534671in}}%
\pgfpathcurveto{\pgfqpoint{4.209496in}{1.539714in}}{\pgfqpoint{4.212330in}{1.546556in}}{\pgfqpoint{4.212330in}{1.553689in}}%
\pgfpathcurveto{\pgfqpoint{4.212330in}{1.560822in}}{\pgfqpoint{4.209496in}{1.567663in}}{\pgfqpoint{4.204452in}{1.572707in}}%
\pgfpathcurveto{\pgfqpoint{4.199409in}{1.577751in}}{\pgfqpoint{4.192567in}{1.580585in}}{\pgfqpoint{4.185434in}{1.580585in}}%
\pgfpathcurveto{\pgfqpoint{4.178301in}{1.580585in}}{\pgfqpoint{4.171460in}{1.577751in}}{\pgfqpoint{4.166416in}{1.572707in}}%
\pgfpathcurveto{\pgfqpoint{4.161372in}{1.567663in}}{\pgfqpoint{4.158538in}{1.560822in}}{\pgfqpoint{4.158538in}{1.553689in}}%
\pgfpathcurveto{\pgfqpoint{4.158538in}{1.546556in}}{\pgfqpoint{4.161372in}{1.539714in}}{\pgfqpoint{4.166416in}{1.534671in}}%
\pgfpathcurveto{\pgfqpoint{4.171460in}{1.529627in}}{\pgfqpoint{4.178301in}{1.526793in}}{\pgfqpoint{4.185434in}{1.526793in}}%
\pgfpathclose%
\pgfusepath{stroke,fill}%
\end{pgfscope}%
\begin{pgfscope}%
\pgfpathrectangle{\pgfqpoint{2.867647in}{0.500000in}}{\pgfqpoint{1.764706in}{1.700000in}}%
\pgfusepath{clip}%
\pgfsetbuttcap%
\pgfsetroundjoin%
\definecolor{currentfill}{rgb}{0.972726,0.844889,0.754401}%
\pgfsetfillcolor{currentfill}%
\pgfsetlinewidth{0.311001pt}%
\definecolor{currentstroke}{rgb}{1.000000,1.000000,1.000000}%
\pgfsetstrokecolor{currentstroke}%
\pgfsetdash{}{0pt}%
\pgfpathmoveto{\pgfqpoint{4.116695in}{1.270399in}}%
\pgfpathcurveto{\pgfqpoint{4.123828in}{1.270399in}}{\pgfqpoint{4.130669in}{1.273233in}}{\pgfqpoint{4.135713in}{1.278276in}}%
\pgfpathcurveto{\pgfqpoint{4.140757in}{1.283320in}}{\pgfqpoint{4.143591in}{1.290162in}}{\pgfqpoint{4.143591in}{1.297295in}}%
\pgfpathcurveto{\pgfqpoint{4.143591in}{1.304427in}}{\pgfqpoint{4.140757in}{1.311269in}}{\pgfqpoint{4.135713in}{1.316313in}}%
\pgfpathcurveto{\pgfqpoint{4.130669in}{1.321356in}}{\pgfqpoint{4.123828in}{1.324190in}}{\pgfqpoint{4.116695in}{1.324190in}}%
\pgfpathcurveto{\pgfqpoint{4.109562in}{1.324190in}}{\pgfqpoint{4.102720in}{1.321356in}}{\pgfqpoint{4.097677in}{1.316313in}}%
\pgfpathcurveto{\pgfqpoint{4.092633in}{1.311269in}}{\pgfqpoint{4.089799in}{1.304427in}}{\pgfqpoint{4.089799in}{1.297295in}}%
\pgfpathcurveto{\pgfqpoint{4.089799in}{1.290162in}}{\pgfqpoint{4.092633in}{1.283320in}}{\pgfqpoint{4.097677in}{1.278276in}}%
\pgfpathcurveto{\pgfqpoint{4.102720in}{1.273233in}}{\pgfqpoint{4.109562in}{1.270399in}}{\pgfqpoint{4.116695in}{1.270399in}}%
\pgfpathclose%
\pgfusepath{stroke,fill}%
\end{pgfscope}%
\begin{pgfscope}%
\pgfpathrectangle{\pgfqpoint{2.867647in}{0.500000in}}{\pgfqpoint{1.764706in}{1.700000in}}%
\pgfusepath{clip}%
\pgfsetbuttcap%
\pgfsetroundjoin%
\definecolor{currentfill}{rgb}{0.953126,0.456614,0.312398}%
\pgfsetfillcolor{currentfill}%
\pgfsetlinewidth{0.311001pt}%
\definecolor{currentstroke}{rgb}{1.000000,1.000000,1.000000}%
\pgfsetstrokecolor{currentstroke}%
\pgfsetdash{}{0pt}%
\pgfpathmoveto{\pgfqpoint{4.318277in}{1.555247in}}%
\pgfpathcurveto{\pgfqpoint{4.325410in}{1.555247in}}{\pgfqpoint{4.332251in}{1.558081in}}{\pgfqpoint{4.337295in}{1.563125in}}%
\pgfpathcurveto{\pgfqpoint{4.342339in}{1.568168in}}{\pgfqpoint{4.345173in}{1.575010in}}{\pgfqpoint{4.345173in}{1.582143in}}%
\pgfpathcurveto{\pgfqpoint{4.345173in}{1.589276in}}{\pgfqpoint{4.342339in}{1.596117in}}{\pgfqpoint{4.337295in}{1.601161in}}%
\pgfpathcurveto{\pgfqpoint{4.332251in}{1.606205in}}{\pgfqpoint{4.325410in}{1.609038in}}{\pgfqpoint{4.318277in}{1.609038in}}%
\pgfpathcurveto{\pgfqpoint{4.311144in}{1.609038in}}{\pgfqpoint{4.304302in}{1.606205in}}{\pgfqpoint{4.299259in}{1.601161in}}%
\pgfpathcurveto{\pgfqpoint{4.294215in}{1.596117in}}{\pgfqpoint{4.291381in}{1.589276in}}{\pgfqpoint{4.291381in}{1.582143in}}%
\pgfpathcurveto{\pgfqpoint{4.291381in}{1.575010in}}{\pgfqpoint{4.294215in}{1.568168in}}{\pgfqpoint{4.299259in}{1.563125in}}%
\pgfpathcurveto{\pgfqpoint{4.304302in}{1.558081in}}{\pgfqpoint{4.311144in}{1.555247in}}{\pgfqpoint{4.318277in}{1.555247in}}%
\pgfpathclose%
\pgfusepath{stroke,fill}%
\end{pgfscope}%
\begin{pgfscope}%
\pgfpathrectangle{\pgfqpoint{2.867647in}{0.500000in}}{\pgfqpoint{1.764706in}{1.700000in}}%
\pgfusepath{clip}%
\pgfsetbuttcap%
\pgfsetroundjoin%
\definecolor{currentfill}{rgb}{0.976287,0.879862,0.805788}%
\pgfsetfillcolor{currentfill}%
\pgfsetlinewidth{0.311001pt}%
\definecolor{currentstroke}{rgb}{1.000000,1.000000,1.000000}%
\pgfsetstrokecolor{currentstroke}%
\pgfsetdash{}{0pt}%
\pgfpathmoveto{\pgfqpoint{4.115904in}{1.632864in}}%
\pgfpathcurveto{\pgfqpoint{4.123037in}{1.632864in}}{\pgfqpoint{4.129878in}{1.635698in}}{\pgfqpoint{4.134922in}{1.640742in}}%
\pgfpathcurveto{\pgfqpoint{4.139966in}{1.645785in}}{\pgfqpoint{4.142800in}{1.652627in}}{\pgfqpoint{4.142800in}{1.659760in}}%
\pgfpathcurveto{\pgfqpoint{4.142800in}{1.666893in}}{\pgfqpoint{4.139966in}{1.673734in}}{\pgfqpoint{4.134922in}{1.678778in}}%
\pgfpathcurveto{\pgfqpoint{4.129878in}{1.683822in}}{\pgfqpoint{4.123037in}{1.686656in}}{\pgfqpoint{4.115904in}{1.686656in}}%
\pgfpathcurveto{\pgfqpoint{4.108771in}{1.686656in}}{\pgfqpoint{4.101930in}{1.683822in}}{\pgfqpoint{4.096886in}{1.678778in}}%
\pgfpathcurveto{\pgfqpoint{4.091842in}{1.673734in}}{\pgfqpoint{4.089008in}{1.666893in}}{\pgfqpoint{4.089008in}{1.659760in}}%
\pgfpathcurveto{\pgfqpoint{4.089008in}{1.652627in}}{\pgfqpoint{4.091842in}{1.645785in}}{\pgfqpoint{4.096886in}{1.640742in}}%
\pgfpathcurveto{\pgfqpoint{4.101930in}{1.635698in}}{\pgfqpoint{4.108771in}{1.632864in}}{\pgfqpoint{4.115904in}{1.632864in}}%
\pgfpathclose%
\pgfusepath{stroke,fill}%
\end{pgfscope}%
\begin{pgfscope}%
\pgfpathrectangle{\pgfqpoint{2.867647in}{0.500000in}}{\pgfqpoint{1.764706in}{1.700000in}}%
\pgfusepath{clip}%
\pgfsetbuttcap%
\pgfsetroundjoin%
\definecolor{currentfill}{rgb}{0.977657,0.891500,0.822809}%
\pgfsetfillcolor{currentfill}%
\pgfsetlinewidth{0.311001pt}%
\definecolor{currentstroke}{rgb}{1.000000,1.000000,1.000000}%
\pgfsetstrokecolor{currentstroke}%
\pgfsetdash{}{0pt}%
\pgfpathmoveto{\pgfqpoint{4.146966in}{1.295065in}}%
\pgfpathcurveto{\pgfqpoint{4.154099in}{1.295065in}}{\pgfqpoint{4.160941in}{1.297899in}}{\pgfqpoint{4.165985in}{1.302942in}}%
\pgfpathcurveto{\pgfqpoint{4.171028in}{1.307986in}}{\pgfqpoint{4.173862in}{1.314828in}}{\pgfqpoint{4.173862in}{1.321960in}}%
\pgfpathcurveto{\pgfqpoint{4.173862in}{1.329093in}}{\pgfqpoint{4.171028in}{1.335935in}}{\pgfqpoint{4.165985in}{1.340978in}}%
\pgfpathcurveto{\pgfqpoint{4.160941in}{1.346022in}}{\pgfqpoint{4.154099in}{1.348856in}}{\pgfqpoint{4.146966in}{1.348856in}}%
\pgfpathcurveto{\pgfqpoint{4.139834in}{1.348856in}}{\pgfqpoint{4.132992in}{1.346022in}}{\pgfqpoint{4.127948in}{1.340978in}}%
\pgfpathcurveto{\pgfqpoint{4.122905in}{1.335935in}}{\pgfqpoint{4.120071in}{1.329093in}}{\pgfqpoint{4.120071in}{1.321960in}}%
\pgfpathcurveto{\pgfqpoint{4.120071in}{1.314828in}}{\pgfqpoint{4.122905in}{1.307986in}}{\pgfqpoint{4.127948in}{1.302942in}}%
\pgfpathcurveto{\pgfqpoint{4.132992in}{1.297899in}}{\pgfqpoint{4.139834in}{1.295065in}}{\pgfqpoint{4.146966in}{1.295065in}}%
\pgfpathclose%
\pgfusepath{stroke,fill}%
\end{pgfscope}%
\begin{pgfscope}%
\pgfpathrectangle{\pgfqpoint{2.867647in}{0.500000in}}{\pgfqpoint{1.764706in}{1.700000in}}%
\pgfusepath{clip}%
\pgfsetbuttcap%
\pgfsetroundjoin%
\definecolor{currentfill}{rgb}{0.970718,0.821518,0.719872}%
\pgfsetfillcolor{currentfill}%
\pgfsetlinewidth{0.311001pt}%
\definecolor{currentstroke}{rgb}{1.000000,1.000000,1.000000}%
\pgfsetstrokecolor{currentstroke}%
\pgfsetdash{}{0pt}%
\pgfpathmoveto{\pgfqpoint{4.176249in}{1.006661in}}%
\pgfpathcurveto{\pgfqpoint{4.183382in}{1.006661in}}{\pgfqpoint{4.190224in}{1.009495in}}{\pgfqpoint{4.195267in}{1.014538in}}%
\pgfpathcurveto{\pgfqpoint{4.200311in}{1.019582in}}{\pgfqpoint{4.203145in}{1.026424in}}{\pgfqpoint{4.203145in}{1.033557in}}%
\pgfpathcurveto{\pgfqpoint{4.203145in}{1.040689in}}{\pgfqpoint{4.200311in}{1.047531in}}{\pgfqpoint{4.195267in}{1.052575in}}%
\pgfpathcurveto{\pgfqpoint{4.190224in}{1.057618in}}{\pgfqpoint{4.183382in}{1.060452in}}{\pgfqpoint{4.176249in}{1.060452in}}%
\pgfpathcurveto{\pgfqpoint{4.169116in}{1.060452in}}{\pgfqpoint{4.162275in}{1.057618in}}{\pgfqpoint{4.157231in}{1.052575in}}%
\pgfpathcurveto{\pgfqpoint{4.152187in}{1.047531in}}{\pgfqpoint{4.149353in}{1.040689in}}{\pgfqpoint{4.149353in}{1.033557in}}%
\pgfpathcurveto{\pgfqpoint{4.149353in}{1.026424in}}{\pgfqpoint{4.152187in}{1.019582in}}{\pgfqpoint{4.157231in}{1.014538in}}%
\pgfpathcurveto{\pgfqpoint{4.162275in}{1.009495in}}{\pgfqpoint{4.169116in}{1.006661in}}{\pgfqpoint{4.176249in}{1.006661in}}%
\pgfpathclose%
\pgfusepath{stroke,fill}%
\end{pgfscope}%
\begin{pgfscope}%
\pgfpathrectangle{\pgfqpoint{2.867647in}{0.500000in}}{\pgfqpoint{1.764706in}{1.700000in}}%
\pgfusepath{clip}%
\pgfsetbuttcap%
\pgfsetroundjoin%
\definecolor{currentfill}{rgb}{0.979891,0.908948,0.848279}%
\pgfsetfillcolor{currentfill}%
\pgfsetlinewidth{0.311001pt}%
\definecolor{currentstroke}{rgb}{1.000000,1.000000,1.000000}%
\pgfsetstrokecolor{currentstroke}%
\pgfsetdash{}{0pt}%
\pgfpathmoveto{\pgfqpoint{4.216997in}{1.332568in}}%
\pgfpathcurveto{\pgfqpoint{4.224129in}{1.332568in}}{\pgfqpoint{4.230971in}{1.335402in}}{\pgfqpoint{4.236015in}{1.340446in}}%
\pgfpathcurveto{\pgfqpoint{4.241058in}{1.345489in}}{\pgfqpoint{4.243892in}{1.352331in}}{\pgfqpoint{4.243892in}{1.359464in}}%
\pgfpathcurveto{\pgfqpoint{4.243892in}{1.366597in}}{\pgfqpoint{4.241058in}{1.373438in}}{\pgfqpoint{4.236015in}{1.378482in}}%
\pgfpathcurveto{\pgfqpoint{4.230971in}{1.383526in}}{\pgfqpoint{4.224129in}{1.386360in}}{\pgfqpoint{4.216997in}{1.386360in}}%
\pgfpathcurveto{\pgfqpoint{4.209864in}{1.386360in}}{\pgfqpoint{4.203022in}{1.383526in}}{\pgfqpoint{4.197978in}{1.378482in}}%
\pgfpathcurveto{\pgfqpoint{4.192935in}{1.373438in}}{\pgfqpoint{4.190101in}{1.366597in}}{\pgfqpoint{4.190101in}{1.359464in}}%
\pgfpathcurveto{\pgfqpoint{4.190101in}{1.352331in}}{\pgfqpoint{4.192935in}{1.345489in}}{\pgfqpoint{4.197978in}{1.340446in}}%
\pgfpathcurveto{\pgfqpoint{4.203022in}{1.335402in}}{\pgfqpoint{4.209864in}{1.332568in}}{\pgfqpoint{4.216997in}{1.332568in}}%
\pgfpathclose%
\pgfusepath{stroke,fill}%
\end{pgfscope}%
\begin{pgfscope}%
\pgfpathrectangle{\pgfqpoint{2.867647in}{0.500000in}}{\pgfqpoint{1.764706in}{1.700000in}}%
\pgfusepath{clip}%
\pgfsetbuttcap%
\pgfsetroundjoin%
\definecolor{currentfill}{rgb}{0.981377,0.920617,0.865369}%
\pgfsetfillcolor{currentfill}%
\pgfsetlinewidth{0.311001pt}%
\definecolor{currentstroke}{rgb}{1.000000,1.000000,1.000000}%
\pgfsetstrokecolor{currentstroke}%
\pgfsetdash{}{0pt}%
\pgfpathmoveto{\pgfqpoint{4.183228in}{1.360904in}}%
\pgfpathcurveto{\pgfqpoint{4.190361in}{1.360904in}}{\pgfqpoint{4.197202in}{1.363737in}}{\pgfqpoint{4.202246in}{1.368781in}}%
\pgfpathcurveto{\pgfqpoint{4.207290in}{1.373825in}}{\pgfqpoint{4.210124in}{1.380666in}}{\pgfqpoint{4.210124in}{1.387799in}}%
\pgfpathcurveto{\pgfqpoint{4.210124in}{1.394932in}}{\pgfqpoint{4.207290in}{1.401774in}}{\pgfqpoint{4.202246in}{1.406817in}}%
\pgfpathcurveto{\pgfqpoint{4.197202in}{1.411861in}}{\pgfqpoint{4.190361in}{1.414695in}}{\pgfqpoint{4.183228in}{1.414695in}}%
\pgfpathcurveto{\pgfqpoint{4.176095in}{1.414695in}}{\pgfqpoint{4.169254in}{1.411861in}}{\pgfqpoint{4.164210in}{1.406817in}}%
\pgfpathcurveto{\pgfqpoint{4.159166in}{1.401774in}}{\pgfqpoint{4.156332in}{1.394932in}}{\pgfqpoint{4.156332in}{1.387799in}}%
\pgfpathcurveto{\pgfqpoint{4.156332in}{1.380666in}}{\pgfqpoint{4.159166in}{1.373825in}}{\pgfqpoint{4.164210in}{1.368781in}}%
\pgfpathcurveto{\pgfqpoint{4.169254in}{1.363737in}}{\pgfqpoint{4.176095in}{1.360904in}}{\pgfqpoint{4.183228in}{1.360904in}}%
\pgfpathclose%
\pgfusepath{stroke,fill}%
\end{pgfscope}%
\begin{pgfscope}%
\pgfpathrectangle{\pgfqpoint{2.867647in}{0.500000in}}{\pgfqpoint{1.764706in}{1.700000in}}%
\pgfusepath{clip}%
\pgfsetbuttcap%
\pgfsetroundjoin%
\definecolor{currentfill}{rgb}{0.979891,0.908948,0.848279}%
\pgfsetfillcolor{currentfill}%
\pgfsetlinewidth{0.311001pt}%
\definecolor{currentstroke}{rgb}{1.000000,1.000000,1.000000}%
\pgfsetstrokecolor{currentstroke}%
\pgfsetdash{}{0pt}%
\pgfpathmoveto{\pgfqpoint{4.183990in}{1.138415in}}%
\pgfpathcurveto{\pgfqpoint{4.191123in}{1.138415in}}{\pgfqpoint{4.197964in}{1.141249in}}{\pgfqpoint{4.203008in}{1.146292in}}%
\pgfpathcurveto{\pgfqpoint{4.208052in}{1.151336in}}{\pgfqpoint{4.210886in}{1.158178in}}{\pgfqpoint{4.210886in}{1.165311in}}%
\pgfpathcurveto{\pgfqpoint{4.210886in}{1.172443in}}{\pgfqpoint{4.208052in}{1.179285in}}{\pgfqpoint{4.203008in}{1.184329in}}%
\pgfpathcurveto{\pgfqpoint{4.197964in}{1.189372in}}{\pgfqpoint{4.191123in}{1.192206in}}{\pgfqpoint{4.183990in}{1.192206in}}%
\pgfpathcurveto{\pgfqpoint{4.176857in}{1.192206in}}{\pgfqpoint{4.170016in}{1.189372in}}{\pgfqpoint{4.164972in}{1.184329in}}%
\pgfpathcurveto{\pgfqpoint{4.159928in}{1.179285in}}{\pgfqpoint{4.157094in}{1.172443in}}{\pgfqpoint{4.157094in}{1.165311in}}%
\pgfpathcurveto{\pgfqpoint{4.157094in}{1.158178in}}{\pgfqpoint{4.159928in}{1.151336in}}{\pgfqpoint{4.164972in}{1.146292in}}%
\pgfpathcurveto{\pgfqpoint{4.170016in}{1.141249in}}{\pgfqpoint{4.176857in}{1.138415in}}{\pgfqpoint{4.183990in}{1.138415in}}%
\pgfpathclose%
\pgfusepath{stroke,fill}%
\end{pgfscope}%
\begin{pgfscope}%
\pgfpathrectangle{\pgfqpoint{2.867647in}{0.500000in}}{\pgfqpoint{1.764706in}{1.700000in}}%
\pgfusepath{clip}%
\pgfsetbuttcap%
\pgfsetroundjoin%
\definecolor{currentfill}{rgb}{0.962018,0.586477,0.424918}%
\pgfsetfillcolor{currentfill}%
\pgfsetlinewidth{0.311001pt}%
\definecolor{currentstroke}{rgb}{1.000000,1.000000,1.000000}%
\pgfsetstrokecolor{currentstroke}%
\pgfsetdash{}{0pt}%
\pgfpathmoveto{\pgfqpoint{4.038013in}{0.844310in}}%
\pgfpathcurveto{\pgfqpoint{4.045146in}{0.844310in}}{\pgfqpoint{4.051988in}{0.847144in}}{\pgfqpoint{4.057031in}{0.852188in}}%
\pgfpathcurveto{\pgfqpoint{4.062075in}{0.857231in}}{\pgfqpoint{4.064909in}{0.864073in}}{\pgfqpoint{4.064909in}{0.871206in}}%
\pgfpathcurveto{\pgfqpoint{4.064909in}{0.878339in}}{\pgfqpoint{4.062075in}{0.885180in}}{\pgfqpoint{4.057031in}{0.890224in}}%
\pgfpathcurveto{\pgfqpoint{4.051988in}{0.895268in}}{\pgfqpoint{4.045146in}{0.898101in}}{\pgfqpoint{4.038013in}{0.898101in}}%
\pgfpathcurveto{\pgfqpoint{4.030881in}{0.898101in}}{\pgfqpoint{4.024039in}{0.895268in}}{\pgfqpoint{4.018995in}{0.890224in}}%
\pgfpathcurveto{\pgfqpoint{4.013952in}{0.885180in}}{\pgfqpoint{4.011118in}{0.878339in}}{\pgfqpoint{4.011118in}{0.871206in}}%
\pgfpathcurveto{\pgfqpoint{4.011118in}{0.864073in}}{\pgfqpoint{4.013952in}{0.857231in}}{\pgfqpoint{4.018995in}{0.852188in}}%
\pgfpathcurveto{\pgfqpoint{4.024039in}{0.847144in}}{\pgfqpoint{4.030881in}{0.844310in}}{\pgfqpoint{4.038013in}{0.844310in}}%
\pgfpathclose%
\pgfusepath{stroke,fill}%
\end{pgfscope}%
\begin{pgfscope}%
\pgfpathrectangle{\pgfqpoint{2.867647in}{0.500000in}}{\pgfqpoint{1.764706in}{1.700000in}}%
\pgfusepath{clip}%
\pgfsetbuttcap%
\pgfsetroundjoin%
\definecolor{currentfill}{rgb}{0.961433,0.573272,0.412036}%
\pgfsetfillcolor{currentfill}%
\pgfsetlinewidth{0.311001pt}%
\definecolor{currentstroke}{rgb}{1.000000,1.000000,1.000000}%
\pgfsetstrokecolor{currentstroke}%
\pgfsetdash{}{0pt}%
\pgfpathmoveto{\pgfqpoint{4.013218in}{1.160515in}}%
\pgfpathcurveto{\pgfqpoint{4.020350in}{1.160515in}}{\pgfqpoint{4.027192in}{1.163349in}}{\pgfqpoint{4.032236in}{1.168393in}}%
\pgfpathcurveto{\pgfqpoint{4.037279in}{1.173436in}}{\pgfqpoint{4.040113in}{1.180278in}}{\pgfqpoint{4.040113in}{1.187411in}}%
\pgfpathcurveto{\pgfqpoint{4.040113in}{1.194544in}}{\pgfqpoint{4.037279in}{1.201385in}}{\pgfqpoint{4.032236in}{1.206429in}}%
\pgfpathcurveto{\pgfqpoint{4.027192in}{1.211473in}}{\pgfqpoint{4.020350in}{1.214307in}}{\pgfqpoint{4.013218in}{1.214307in}}%
\pgfpathcurveto{\pgfqpoint{4.006085in}{1.214307in}}{\pgfqpoint{3.999243in}{1.211473in}}{\pgfqpoint{3.994199in}{1.206429in}}%
\pgfpathcurveto{\pgfqpoint{3.989156in}{1.201385in}}{\pgfqpoint{3.986322in}{1.194544in}}{\pgfqpoint{3.986322in}{1.187411in}}%
\pgfpathcurveto{\pgfqpoint{3.986322in}{1.180278in}}{\pgfqpoint{3.989156in}{1.173436in}}{\pgfqpoint{3.994199in}{1.168393in}}%
\pgfpathcurveto{\pgfqpoint{3.999243in}{1.163349in}}{\pgfqpoint{4.006085in}{1.160515in}}{\pgfqpoint{4.013218in}{1.160515in}}%
\pgfpathclose%
\pgfusepath{stroke,fill}%
\end{pgfscope}%
\begin{pgfscope}%
\pgfpathrectangle{\pgfqpoint{2.867647in}{0.500000in}}{\pgfqpoint{1.764706in}{1.700000in}}%
\pgfusepath{clip}%
\pgfsetbuttcap%
\pgfsetroundjoin%
\definecolor{currentfill}{rgb}{0.976961,0.885681,0.814303}%
\pgfsetfillcolor{currentfill}%
\pgfsetlinewidth{0.311001pt}%
\definecolor{currentstroke}{rgb}{1.000000,1.000000,1.000000}%
\pgfsetstrokecolor{currentstroke}%
\pgfsetdash{}{0pt}%
\pgfpathmoveto{\pgfqpoint{4.109427in}{1.114615in}}%
\pgfpathcurveto{\pgfqpoint{4.116560in}{1.114615in}}{\pgfqpoint{4.123401in}{1.117449in}}{\pgfqpoint{4.128445in}{1.122493in}}%
\pgfpathcurveto{\pgfqpoint{4.133489in}{1.127536in}}{\pgfqpoint{4.136322in}{1.134378in}}{\pgfqpoint{4.136322in}{1.141511in}}%
\pgfpathcurveto{\pgfqpoint{4.136322in}{1.148644in}}{\pgfqpoint{4.133489in}{1.155485in}}{\pgfqpoint{4.128445in}{1.160529in}}%
\pgfpathcurveto{\pgfqpoint{4.123401in}{1.165572in}}{\pgfqpoint{4.116560in}{1.168406in}}{\pgfqpoint{4.109427in}{1.168406in}}%
\pgfpathcurveto{\pgfqpoint{4.102294in}{1.168406in}}{\pgfqpoint{4.095452in}{1.165572in}}{\pgfqpoint{4.090409in}{1.160529in}}%
\pgfpathcurveto{\pgfqpoint{4.085365in}{1.155485in}}{\pgfqpoint{4.082531in}{1.148644in}}{\pgfqpoint{4.082531in}{1.141511in}}%
\pgfpathcurveto{\pgfqpoint{4.082531in}{1.134378in}}{\pgfqpoint{4.085365in}{1.127536in}}{\pgfqpoint{4.090409in}{1.122493in}}%
\pgfpathcurveto{\pgfqpoint{4.095452in}{1.117449in}}{\pgfqpoint{4.102294in}{1.114615in}}{\pgfqpoint{4.109427in}{1.114615in}}%
\pgfpathclose%
\pgfusepath{stroke,fill}%
\end{pgfscope}%
\begin{pgfscope}%
\pgfpathrectangle{\pgfqpoint{2.867647in}{0.500000in}}{\pgfqpoint{1.764706in}{1.700000in}}%
\pgfusepath{clip}%
\pgfsetbuttcap%
\pgfsetroundjoin%
\definecolor{currentfill}{rgb}{0.966812,0.762584,0.633643}%
\pgfsetfillcolor{currentfill}%
\pgfsetlinewidth{0.311001pt}%
\definecolor{currentstroke}{rgb}{1.000000,1.000000,1.000000}%
\pgfsetstrokecolor{currentstroke}%
\pgfsetdash{}{0pt}%
\pgfpathmoveto{\pgfqpoint{4.034183in}{1.076044in}}%
\pgfpathcurveto{\pgfqpoint{4.041316in}{1.076044in}}{\pgfqpoint{4.048157in}{1.078878in}}{\pgfqpoint{4.053201in}{1.083922in}}%
\pgfpathcurveto{\pgfqpoint{4.058245in}{1.088966in}}{\pgfqpoint{4.061078in}{1.095807in}}{\pgfqpoint{4.061078in}{1.102940in}}%
\pgfpathcurveto{\pgfqpoint{4.061078in}{1.110073in}}{\pgfqpoint{4.058245in}{1.116915in}}{\pgfqpoint{4.053201in}{1.121958in}}%
\pgfpathcurveto{\pgfqpoint{4.048157in}{1.127002in}}{\pgfqpoint{4.041316in}{1.129836in}}{\pgfqpoint{4.034183in}{1.129836in}}%
\pgfpathcurveto{\pgfqpoint{4.027050in}{1.129836in}}{\pgfqpoint{4.020208in}{1.127002in}}{\pgfqpoint{4.015165in}{1.121958in}}%
\pgfpathcurveto{\pgfqpoint{4.010121in}{1.116915in}}{\pgfqpoint{4.007287in}{1.110073in}}{\pgfqpoint{4.007287in}{1.102940in}}%
\pgfpathcurveto{\pgfqpoint{4.007287in}{1.095807in}}{\pgfqpoint{4.010121in}{1.088966in}}{\pgfqpoint{4.015165in}{1.083922in}}%
\pgfpathcurveto{\pgfqpoint{4.020208in}{1.078878in}}{\pgfqpoint{4.027050in}{1.076044in}}{\pgfqpoint{4.034183in}{1.076044in}}%
\pgfpathclose%
\pgfusepath{stroke,fill}%
\end{pgfscope}%
\begin{pgfscope}%
\pgfpathrectangle{\pgfqpoint{2.867647in}{0.500000in}}{\pgfqpoint{1.764706in}{1.700000in}}%
\pgfusepath{clip}%
\pgfsetbuttcap%
\pgfsetroundjoin%
\definecolor{currentfill}{rgb}{0.967735,0.780441,0.659127}%
\pgfsetfillcolor{currentfill}%
\pgfsetlinewidth{0.311001pt}%
\definecolor{currentstroke}{rgb}{1.000000,1.000000,1.000000}%
\pgfsetstrokecolor{currentstroke}%
\pgfsetdash{}{0pt}%
\pgfpathmoveto{\pgfqpoint{4.126847in}{1.705365in}}%
\pgfpathcurveto{\pgfqpoint{4.133980in}{1.705365in}}{\pgfqpoint{4.140822in}{1.708199in}}{\pgfqpoint{4.145865in}{1.713243in}}%
\pgfpathcurveto{\pgfqpoint{4.150909in}{1.718287in}}{\pgfqpoint{4.153743in}{1.725128in}}{\pgfqpoint{4.153743in}{1.732261in}}%
\pgfpathcurveto{\pgfqpoint{4.153743in}{1.739394in}}{\pgfqpoint{4.150909in}{1.746236in}}{\pgfqpoint{4.145865in}{1.751279in}}%
\pgfpathcurveto{\pgfqpoint{4.140822in}{1.756323in}}{\pgfqpoint{4.133980in}{1.759157in}}{\pgfqpoint{4.126847in}{1.759157in}}%
\pgfpathcurveto{\pgfqpoint{4.119714in}{1.759157in}}{\pgfqpoint{4.112873in}{1.756323in}}{\pgfqpoint{4.107829in}{1.751279in}}%
\pgfpathcurveto{\pgfqpoint{4.102785in}{1.746236in}}{\pgfqpoint{4.099951in}{1.739394in}}{\pgfqpoint{4.099951in}{1.732261in}}%
\pgfpathcurveto{\pgfqpoint{4.099951in}{1.725128in}}{\pgfqpoint{4.102785in}{1.718287in}}{\pgfqpoint{4.107829in}{1.713243in}}%
\pgfpathcurveto{\pgfqpoint{4.112873in}{1.708199in}}{\pgfqpoint{4.119714in}{1.705365in}}{\pgfqpoint{4.126847in}{1.705365in}}%
\pgfpathclose%
\pgfusepath{stroke,fill}%
\end{pgfscope}%
\begin{pgfscope}%
\pgfpathrectangle{\pgfqpoint{2.867647in}{0.500000in}}{\pgfqpoint{1.764706in}{1.700000in}}%
\pgfusepath{clip}%
\pgfsetbuttcap%
\pgfsetroundjoin%
\definecolor{currentfill}{rgb}{0.891169,0.211218,0.255359}%
\pgfsetfillcolor{currentfill}%
\pgfsetlinewidth{0.311001pt}%
\definecolor{currentstroke}{rgb}{1.000000,1.000000,1.000000}%
\pgfsetstrokecolor{currentstroke}%
\pgfsetdash{}{0pt}%
\pgfpathmoveto{\pgfqpoint{3.931396in}{1.886020in}}%
\pgfpathcurveto{\pgfqpoint{3.938529in}{1.886020in}}{\pgfqpoint{3.945371in}{1.888854in}}{\pgfqpoint{3.950415in}{1.893898in}}%
\pgfpathcurveto{\pgfqpoint{3.955458in}{1.898942in}}{\pgfqpoint{3.958292in}{1.905783in}}{\pgfqpoint{3.958292in}{1.912916in}}%
\pgfpathcurveto{\pgfqpoint{3.958292in}{1.920049in}}{\pgfqpoint{3.955458in}{1.926891in}}{\pgfqpoint{3.950415in}{1.931934in}}%
\pgfpathcurveto{\pgfqpoint{3.945371in}{1.936978in}}{\pgfqpoint{3.938529in}{1.939812in}}{\pgfqpoint{3.931396in}{1.939812in}}%
\pgfpathcurveto{\pgfqpoint{3.924264in}{1.939812in}}{\pgfqpoint{3.917422in}{1.936978in}}{\pgfqpoint{3.912378in}{1.931934in}}%
\pgfpathcurveto{\pgfqpoint{3.907335in}{1.926891in}}{\pgfqpoint{3.904501in}{1.920049in}}{\pgfqpoint{3.904501in}{1.912916in}}%
\pgfpathcurveto{\pgfqpoint{3.904501in}{1.905783in}}{\pgfqpoint{3.907335in}{1.898942in}}{\pgfqpoint{3.912378in}{1.893898in}}%
\pgfpathcurveto{\pgfqpoint{3.917422in}{1.888854in}}{\pgfqpoint{3.924264in}{1.886020in}}{\pgfqpoint{3.931396in}{1.886020in}}%
\pgfpathclose%
\pgfusepath{stroke,fill}%
\end{pgfscope}%
\begin{pgfscope}%
\pgfpathrectangle{\pgfqpoint{2.867647in}{0.500000in}}{\pgfqpoint{1.764706in}{1.700000in}}%
\pgfusepath{clip}%
\pgfsetbuttcap%
\pgfsetroundjoin%
\definecolor{currentfill}{rgb}{0.965928,0.738443,0.600540}%
\pgfsetfillcolor{currentfill}%
\pgfsetlinewidth{0.311001pt}%
\definecolor{currentstroke}{rgb}{1.000000,1.000000,1.000000}%
\pgfsetstrokecolor{currentstroke}%
\pgfsetdash{}{0pt}%
\pgfpathmoveto{\pgfqpoint{4.061141in}{1.177790in}}%
\pgfpathcurveto{\pgfqpoint{4.068274in}{1.177790in}}{\pgfqpoint{4.075115in}{1.180624in}}{\pgfqpoint{4.080159in}{1.185668in}}%
\pgfpathcurveto{\pgfqpoint{4.085203in}{1.190711in}}{\pgfqpoint{4.088036in}{1.197553in}}{\pgfqpoint{4.088036in}{1.204686in}}%
\pgfpathcurveto{\pgfqpoint{4.088036in}{1.211819in}}{\pgfqpoint{4.085203in}{1.218660in}}{\pgfqpoint{4.080159in}{1.223704in}}%
\pgfpathcurveto{\pgfqpoint{4.075115in}{1.228748in}}{\pgfqpoint{4.068274in}{1.231581in}}{\pgfqpoint{4.061141in}{1.231581in}}%
\pgfpathcurveto{\pgfqpoint{4.054008in}{1.231581in}}{\pgfqpoint{4.047166in}{1.228748in}}{\pgfqpoint{4.042123in}{1.223704in}}%
\pgfpathcurveto{\pgfqpoint{4.037079in}{1.218660in}}{\pgfqpoint{4.034245in}{1.211819in}}{\pgfqpoint{4.034245in}{1.204686in}}%
\pgfpathcurveto{\pgfqpoint{4.034245in}{1.197553in}}{\pgfqpoint{4.037079in}{1.190711in}}{\pgfqpoint{4.042123in}{1.185668in}}%
\pgfpathcurveto{\pgfqpoint{4.047166in}{1.180624in}}{\pgfqpoint{4.054008in}{1.177790in}}{\pgfqpoint{4.061141in}{1.177790in}}%
\pgfpathclose%
\pgfusepath{stroke,fill}%
\end{pgfscope}%
\begin{pgfscope}%
\pgfpathrectangle{\pgfqpoint{2.867647in}{0.500000in}}{\pgfqpoint{1.764706in}{1.700000in}}%
\pgfusepath{clip}%
\pgfsetbuttcap%
\pgfsetroundjoin%
\definecolor{currentfill}{rgb}{0.979891,0.908948,0.848279}%
\pgfsetfillcolor{currentfill}%
\pgfsetlinewidth{0.311001pt}%
\definecolor{currentstroke}{rgb}{1.000000,1.000000,1.000000}%
\pgfsetstrokecolor{currentstroke}%
\pgfsetdash{}{0pt}%
\pgfpathmoveto{\pgfqpoint{4.162982in}{1.400761in}}%
\pgfpathcurveto{\pgfqpoint{4.170115in}{1.400761in}}{\pgfqpoint{4.176957in}{1.403595in}}{\pgfqpoint{4.182001in}{1.408639in}}%
\pgfpathcurveto{\pgfqpoint{4.187044in}{1.413683in}}{\pgfqpoint{4.189878in}{1.420524in}}{\pgfqpoint{4.189878in}{1.427657in}}%
\pgfpathcurveto{\pgfqpoint{4.189878in}{1.434790in}}{\pgfqpoint{4.187044in}{1.441631in}}{\pgfqpoint{4.182001in}{1.446675in}}%
\pgfpathcurveto{\pgfqpoint{4.176957in}{1.451719in}}{\pgfqpoint{4.170115in}{1.454553in}}{\pgfqpoint{4.162982in}{1.454553in}}%
\pgfpathcurveto{\pgfqpoint{4.155850in}{1.454553in}}{\pgfqpoint{4.149008in}{1.451719in}}{\pgfqpoint{4.143964in}{1.446675in}}%
\pgfpathcurveto{\pgfqpoint{4.138921in}{1.441631in}}{\pgfqpoint{4.136087in}{1.434790in}}{\pgfqpoint{4.136087in}{1.427657in}}%
\pgfpathcurveto{\pgfqpoint{4.136087in}{1.420524in}}{\pgfqpoint{4.138921in}{1.413683in}}{\pgfqpoint{4.143964in}{1.408639in}}%
\pgfpathcurveto{\pgfqpoint{4.149008in}{1.403595in}}{\pgfqpoint{4.155850in}{1.400761in}}{\pgfqpoint{4.162982in}{1.400761in}}%
\pgfpathclose%
\pgfusepath{stroke,fill}%
\end{pgfscope}%
\begin{pgfscope}%
\pgfpathrectangle{\pgfqpoint{2.867647in}{0.500000in}}{\pgfqpoint{1.764706in}{1.700000in}}%
\pgfusepath{clip}%
\pgfsetbuttcap%
\pgfsetroundjoin%
\definecolor{currentfill}{rgb}{0.980678,0.914765,0.856766}%
\pgfsetfillcolor{currentfill}%
\pgfsetlinewidth{0.311001pt}%
\definecolor{currentstroke}{rgb}{1.000000,1.000000,1.000000}%
\pgfsetstrokecolor{currentstroke}%
\pgfsetdash{}{0pt}%
\pgfpathmoveto{\pgfqpoint{4.210492in}{1.325467in}}%
\pgfpathcurveto{\pgfqpoint{4.217624in}{1.325467in}}{\pgfqpoint{4.224466in}{1.328301in}}{\pgfqpoint{4.229510in}{1.333345in}}%
\pgfpathcurveto{\pgfqpoint{4.234553in}{1.338388in}}{\pgfqpoint{4.237387in}{1.345230in}}{\pgfqpoint{4.237387in}{1.352363in}}%
\pgfpathcurveto{\pgfqpoint{4.237387in}{1.359496in}}{\pgfqpoint{4.234553in}{1.366337in}}{\pgfqpoint{4.229510in}{1.371381in}}%
\pgfpathcurveto{\pgfqpoint{4.224466in}{1.376425in}}{\pgfqpoint{4.217624in}{1.379259in}}{\pgfqpoint{4.210492in}{1.379259in}}%
\pgfpathcurveto{\pgfqpoint{4.203359in}{1.379259in}}{\pgfqpoint{4.196517in}{1.376425in}}{\pgfqpoint{4.191473in}{1.371381in}}%
\pgfpathcurveto{\pgfqpoint{4.186430in}{1.366337in}}{\pgfqpoint{4.183596in}{1.359496in}}{\pgfqpoint{4.183596in}{1.352363in}}%
\pgfpathcurveto{\pgfqpoint{4.183596in}{1.345230in}}{\pgfqpoint{4.186430in}{1.338388in}}{\pgfqpoint{4.191473in}{1.333345in}}%
\pgfpathcurveto{\pgfqpoint{4.196517in}{1.328301in}}{\pgfqpoint{4.203359in}{1.325467in}}{\pgfqpoint{4.210492in}{1.325467in}}%
\pgfpathclose%
\pgfusepath{stroke,fill}%
\end{pgfscope}%
\begin{pgfscope}%
\pgfpathrectangle{\pgfqpoint{2.867647in}{0.500000in}}{\pgfqpoint{1.764706in}{1.700000in}}%
\pgfusepath{clip}%
\pgfsetbuttcap%
\pgfsetroundjoin%
\definecolor{currentfill}{rgb}{0.973271,0.850724,0.762998}%
\pgfsetfillcolor{currentfill}%
\pgfsetlinewidth{0.311001pt}%
\definecolor{currentstroke}{rgb}{1.000000,1.000000,1.000000}%
\pgfsetstrokecolor{currentstroke}%
\pgfsetdash{}{0pt}%
\pgfpathmoveto{\pgfqpoint{4.238880in}{1.473040in}}%
\pgfpathcurveto{\pgfqpoint{4.246012in}{1.473040in}}{\pgfqpoint{4.252854in}{1.475874in}}{\pgfqpoint{4.257898in}{1.480918in}}%
\pgfpathcurveto{\pgfqpoint{4.262941in}{1.485961in}}{\pgfqpoint{4.265775in}{1.492803in}}{\pgfqpoint{4.265775in}{1.499936in}}%
\pgfpathcurveto{\pgfqpoint{4.265775in}{1.507069in}}{\pgfqpoint{4.262941in}{1.513910in}}{\pgfqpoint{4.257898in}{1.518954in}}%
\pgfpathcurveto{\pgfqpoint{4.252854in}{1.523998in}}{\pgfqpoint{4.246012in}{1.526832in}}{\pgfqpoint{4.238880in}{1.526832in}}%
\pgfpathcurveto{\pgfqpoint{4.231747in}{1.526832in}}{\pgfqpoint{4.224905in}{1.523998in}}{\pgfqpoint{4.219862in}{1.518954in}}%
\pgfpathcurveto{\pgfqpoint{4.214818in}{1.513910in}}{\pgfqpoint{4.211984in}{1.507069in}}{\pgfqpoint{4.211984in}{1.499936in}}%
\pgfpathcurveto{\pgfqpoint{4.211984in}{1.492803in}}{\pgfqpoint{4.214818in}{1.485961in}}{\pgfqpoint{4.219862in}{1.480918in}}%
\pgfpathcurveto{\pgfqpoint{4.224905in}{1.475874in}}{\pgfqpoint{4.231747in}{1.473040in}}{\pgfqpoint{4.238880in}{1.473040in}}%
\pgfpathclose%
\pgfusepath{stroke,fill}%
\end{pgfscope}%
\begin{pgfscope}%
\pgfpathrectangle{\pgfqpoint{2.867647in}{0.500000in}}{\pgfqpoint{1.764706in}{1.700000in}}%
\pgfusepath{clip}%
\pgfsetbuttcap%
\pgfsetroundjoin%
\definecolor{currentfill}{rgb}{0.980678,0.914765,0.856766}%
\pgfsetfillcolor{currentfill}%
\pgfsetlinewidth{0.311001pt}%
\definecolor{currentstroke}{rgb}{1.000000,1.000000,1.000000}%
\pgfsetstrokecolor{currentstroke}%
\pgfsetdash{}{0pt}%
\pgfpathmoveto{\pgfqpoint{4.158038in}{1.478605in}}%
\pgfpathcurveto{\pgfqpoint{4.165171in}{1.478605in}}{\pgfqpoint{4.172013in}{1.481439in}}{\pgfqpoint{4.177056in}{1.486482in}}%
\pgfpathcurveto{\pgfqpoint{4.182100in}{1.491526in}}{\pgfqpoint{4.184934in}{1.498368in}}{\pgfqpoint{4.184934in}{1.505500in}}%
\pgfpathcurveto{\pgfqpoint{4.184934in}{1.512633in}}{\pgfqpoint{4.182100in}{1.519475in}}{\pgfqpoint{4.177056in}{1.524519in}}%
\pgfpathcurveto{\pgfqpoint{4.172013in}{1.529562in}}{\pgfqpoint{4.165171in}{1.532396in}}{\pgfqpoint{4.158038in}{1.532396in}}%
\pgfpathcurveto{\pgfqpoint{4.150905in}{1.532396in}}{\pgfqpoint{4.144064in}{1.529562in}}{\pgfqpoint{4.139020in}{1.524519in}}%
\pgfpathcurveto{\pgfqpoint{4.133976in}{1.519475in}}{\pgfqpoint{4.131142in}{1.512633in}}{\pgfqpoint{4.131142in}{1.505500in}}%
\pgfpathcurveto{\pgfqpoint{4.131142in}{1.498368in}}{\pgfqpoint{4.133976in}{1.491526in}}{\pgfqpoint{4.139020in}{1.486482in}}%
\pgfpathcurveto{\pgfqpoint{4.144064in}{1.481439in}}{\pgfqpoint{4.150905in}{1.478605in}}{\pgfqpoint{4.158038in}{1.478605in}}%
\pgfpathclose%
\pgfusepath{stroke,fill}%
\end{pgfscope}%
\begin{pgfscope}%
\pgfpathrectangle{\pgfqpoint{2.867647in}{0.500000in}}{\pgfqpoint{1.764706in}{1.700000in}}%
\pgfusepath{clip}%
\pgfsetbuttcap%
\pgfsetroundjoin%
\definecolor{currentfill}{rgb}{0.980678,0.914765,0.856766}%
\pgfsetfillcolor{currentfill}%
\pgfsetlinewidth{0.311001pt}%
\definecolor{currentstroke}{rgb}{1.000000,1.000000,1.000000}%
\pgfsetstrokecolor{currentstroke}%
\pgfsetdash{}{0pt}%
\pgfpathmoveto{\pgfqpoint{4.166376in}{1.507833in}}%
\pgfpathcurveto{\pgfqpoint{4.173509in}{1.507833in}}{\pgfqpoint{4.180351in}{1.510667in}}{\pgfqpoint{4.185394in}{1.515710in}}%
\pgfpathcurveto{\pgfqpoint{4.190438in}{1.520754in}}{\pgfqpoint{4.193272in}{1.527596in}}{\pgfqpoint{4.193272in}{1.534728in}}%
\pgfpathcurveto{\pgfqpoint{4.193272in}{1.541861in}}{\pgfqpoint{4.190438in}{1.548703in}}{\pgfqpoint{4.185394in}{1.553747in}}%
\pgfpathcurveto{\pgfqpoint{4.180351in}{1.558790in}}{\pgfqpoint{4.173509in}{1.561624in}}{\pgfqpoint{4.166376in}{1.561624in}}%
\pgfpathcurveto{\pgfqpoint{4.159243in}{1.561624in}}{\pgfqpoint{4.152402in}{1.558790in}}{\pgfqpoint{4.147358in}{1.553747in}}%
\pgfpathcurveto{\pgfqpoint{4.142314in}{1.548703in}}{\pgfqpoint{4.139480in}{1.541861in}}{\pgfqpoint{4.139480in}{1.534728in}}%
\pgfpathcurveto{\pgfqpoint{4.139480in}{1.527596in}}{\pgfqpoint{4.142314in}{1.520754in}}{\pgfqpoint{4.147358in}{1.515710in}}%
\pgfpathcurveto{\pgfqpoint{4.152402in}{1.510667in}}{\pgfqpoint{4.159243in}{1.507833in}}{\pgfqpoint{4.166376in}{1.507833in}}%
\pgfpathclose%
\pgfusepath{stroke,fill}%
\end{pgfscope}%
\begin{pgfscope}%
\pgfpathrectangle{\pgfqpoint{2.867647in}{0.500000in}}{\pgfqpoint{1.764706in}{1.700000in}}%
\pgfusepath{clip}%
\pgfsetbuttcap%
\pgfsetroundjoin%
\definecolor{currentfill}{rgb}{0.976961,0.885681,0.814303}%
\pgfsetfillcolor{currentfill}%
\pgfsetlinewidth{0.311001pt}%
\definecolor{currentstroke}{rgb}{1.000000,1.000000,1.000000}%
\pgfsetstrokecolor{currentstroke}%
\pgfsetdash{}{0pt}%
\pgfpathmoveto{\pgfqpoint{4.220036in}{1.163196in}}%
\pgfpathcurveto{\pgfqpoint{4.227169in}{1.163196in}}{\pgfqpoint{4.234010in}{1.166029in}}{\pgfqpoint{4.239054in}{1.171073in}}%
\pgfpathcurveto{\pgfqpoint{4.244098in}{1.176117in}}{\pgfqpoint{4.246932in}{1.182958in}}{\pgfqpoint{4.246932in}{1.190091in}}%
\pgfpathcurveto{\pgfqpoint{4.246932in}{1.197224in}}{\pgfqpoint{4.244098in}{1.204066in}}{\pgfqpoint{4.239054in}{1.209109in}}%
\pgfpathcurveto{\pgfqpoint{4.234010in}{1.214153in}}{\pgfqpoint{4.227169in}{1.216987in}}{\pgfqpoint{4.220036in}{1.216987in}}%
\pgfpathcurveto{\pgfqpoint{4.212903in}{1.216987in}}{\pgfqpoint{4.206061in}{1.214153in}}{\pgfqpoint{4.201018in}{1.209109in}}%
\pgfpathcurveto{\pgfqpoint{4.195974in}{1.204066in}}{\pgfqpoint{4.193140in}{1.197224in}}{\pgfqpoint{4.193140in}{1.190091in}}%
\pgfpathcurveto{\pgfqpoint{4.193140in}{1.182958in}}{\pgfqpoint{4.195974in}{1.176117in}}{\pgfqpoint{4.201018in}{1.171073in}}%
\pgfpathcurveto{\pgfqpoint{4.206061in}{1.166029in}}{\pgfqpoint{4.212903in}{1.163196in}}{\pgfqpoint{4.220036in}{1.163196in}}%
\pgfpathclose%
\pgfusepath{stroke,fill}%
\end{pgfscope}%
\begin{pgfscope}%
\pgfpathrectangle{\pgfqpoint{2.867647in}{0.500000in}}{\pgfqpoint{1.764706in}{1.700000in}}%
\pgfusepath{clip}%
\pgfsetbuttcap%
\pgfsetroundjoin%
\definecolor{currentfill}{rgb}{0.968509,0.792226,0.676405}%
\pgfsetfillcolor{currentfill}%
\pgfsetlinewidth{0.311001pt}%
\definecolor{currentstroke}{rgb}{1.000000,1.000000,1.000000}%
\pgfsetstrokecolor{currentstroke}%
\pgfsetdash{}{0pt}%
\pgfpathmoveto{\pgfqpoint{4.280825in}{1.374241in}}%
\pgfpathcurveto{\pgfqpoint{4.287958in}{1.374241in}}{\pgfqpoint{4.294800in}{1.377075in}}{\pgfqpoint{4.299843in}{1.382118in}}%
\pgfpathcurveto{\pgfqpoint{4.304887in}{1.387162in}}{\pgfqpoint{4.307721in}{1.394004in}}{\pgfqpoint{4.307721in}{1.401136in}}%
\pgfpathcurveto{\pgfqpoint{4.307721in}{1.408269in}}{\pgfqpoint{4.304887in}{1.415111in}}{\pgfqpoint{4.299843in}{1.420155in}}%
\pgfpathcurveto{\pgfqpoint{4.294800in}{1.425198in}}{\pgfqpoint{4.287958in}{1.428032in}}{\pgfqpoint{4.280825in}{1.428032in}}%
\pgfpathcurveto{\pgfqpoint{4.273692in}{1.428032in}}{\pgfqpoint{4.266851in}{1.425198in}}{\pgfqpoint{4.261807in}{1.420155in}}%
\pgfpathcurveto{\pgfqpoint{4.256763in}{1.415111in}}{\pgfqpoint{4.253929in}{1.408269in}}{\pgfqpoint{4.253929in}{1.401136in}}%
\pgfpathcurveto{\pgfqpoint{4.253929in}{1.394004in}}{\pgfqpoint{4.256763in}{1.387162in}}{\pgfqpoint{4.261807in}{1.382118in}}%
\pgfpathcurveto{\pgfqpoint{4.266851in}{1.377075in}}{\pgfqpoint{4.273692in}{1.374241in}}{\pgfqpoint{4.280825in}{1.374241in}}%
\pgfpathclose%
\pgfusepath{stroke,fill}%
\end{pgfscope}%
\begin{pgfscope}%
\pgfpathrectangle{\pgfqpoint{2.867647in}{0.500000in}}{\pgfqpoint{1.764706in}{1.700000in}}%
\pgfusepath{clip}%
\pgfsetbuttcap%
\pgfsetroundjoin%
\definecolor{currentfill}{rgb}{0.980678,0.914765,0.856766}%
\pgfsetfillcolor{currentfill}%
\pgfsetlinewidth{0.311001pt}%
\definecolor{currentstroke}{rgb}{1.000000,1.000000,1.000000}%
\pgfsetstrokecolor{currentstroke}%
\pgfsetdash{}{0pt}%
\pgfpathmoveto{\pgfqpoint{4.208057in}{1.229158in}}%
\pgfpathcurveto{\pgfqpoint{4.215190in}{1.229158in}}{\pgfqpoint{4.222032in}{1.231992in}}{\pgfqpoint{4.227075in}{1.237036in}}%
\pgfpathcurveto{\pgfqpoint{4.232119in}{1.242080in}}{\pgfqpoint{4.234953in}{1.248921in}}{\pgfqpoint{4.234953in}{1.256054in}}%
\pgfpathcurveto{\pgfqpoint{4.234953in}{1.263187in}}{\pgfqpoint{4.232119in}{1.270028in}}{\pgfqpoint{4.227075in}{1.275072in}}%
\pgfpathcurveto{\pgfqpoint{4.222032in}{1.280116in}}{\pgfqpoint{4.215190in}{1.282950in}}{\pgfqpoint{4.208057in}{1.282950in}}%
\pgfpathcurveto{\pgfqpoint{4.200924in}{1.282950in}}{\pgfqpoint{4.194083in}{1.280116in}}{\pgfqpoint{4.189039in}{1.275072in}}%
\pgfpathcurveto{\pgfqpoint{4.183995in}{1.270028in}}{\pgfqpoint{4.181162in}{1.263187in}}{\pgfqpoint{4.181162in}{1.256054in}}%
\pgfpathcurveto{\pgfqpoint{4.181162in}{1.248921in}}{\pgfqpoint{4.183995in}{1.242080in}}{\pgfqpoint{4.189039in}{1.237036in}}%
\pgfpathcurveto{\pgfqpoint{4.194083in}{1.231992in}}{\pgfqpoint{4.200924in}{1.229158in}}{\pgfqpoint{4.208057in}{1.229158in}}%
\pgfpathclose%
\pgfusepath{stroke,fill}%
\end{pgfscope}%
\begin{pgfscope}%
\pgfpathrectangle{\pgfqpoint{2.867647in}{0.500000in}}{\pgfqpoint{1.764706in}{1.700000in}}%
\pgfusepath{clip}%
\pgfsetbuttcap%
\pgfsetroundjoin%
\definecolor{currentfill}{rgb}{0.964306,0.663930,0.507747}%
\pgfsetfillcolor{currentfill}%
\pgfsetlinewidth{0.311001pt}%
\definecolor{currentstroke}{rgb}{1.000000,1.000000,1.000000}%
\pgfsetstrokecolor{currentstroke}%
\pgfsetdash{}{0pt}%
\pgfpathmoveto{\pgfqpoint{4.085481in}{0.876249in}}%
\pgfpathcurveto{\pgfqpoint{4.092614in}{0.876249in}}{\pgfqpoint{4.099455in}{0.879083in}}{\pgfqpoint{4.104499in}{0.884127in}}%
\pgfpathcurveto{\pgfqpoint{4.109543in}{0.889171in}}{\pgfqpoint{4.112377in}{0.896012in}}{\pgfqpoint{4.112377in}{0.903145in}}%
\pgfpathcurveto{\pgfqpoint{4.112377in}{0.910278in}}{\pgfqpoint{4.109543in}{0.917120in}}{\pgfqpoint{4.104499in}{0.922163in}}%
\pgfpathcurveto{\pgfqpoint{4.099455in}{0.927207in}}{\pgfqpoint{4.092614in}{0.930041in}}{\pgfqpoint{4.085481in}{0.930041in}}%
\pgfpathcurveto{\pgfqpoint{4.078348in}{0.930041in}}{\pgfqpoint{4.071506in}{0.927207in}}{\pgfqpoint{4.066463in}{0.922163in}}%
\pgfpathcurveto{\pgfqpoint{4.061419in}{0.917120in}}{\pgfqpoint{4.058585in}{0.910278in}}{\pgfqpoint{4.058585in}{0.903145in}}%
\pgfpathcurveto{\pgfqpoint{4.058585in}{0.896012in}}{\pgfqpoint{4.061419in}{0.889171in}}{\pgfqpoint{4.066463in}{0.884127in}}%
\pgfpathcurveto{\pgfqpoint{4.071506in}{0.879083in}}{\pgfqpoint{4.078348in}{0.876249in}}{\pgfqpoint{4.085481in}{0.876249in}}%
\pgfpathclose%
\pgfusepath{stroke,fill}%
\end{pgfscope}%
\begin{pgfscope}%
\pgfpathrectangle{\pgfqpoint{2.867647in}{0.500000in}}{\pgfqpoint{1.764706in}{1.700000in}}%
\pgfusepath{clip}%
\pgfsetbuttcap%
\pgfsetroundjoin%
\definecolor{currentfill}{rgb}{0.953816,0.463738,0.317699}%
\pgfsetfillcolor{currentfill}%
\pgfsetlinewidth{0.311001pt}%
\definecolor{currentstroke}{rgb}{1.000000,1.000000,1.000000}%
\pgfsetstrokecolor{currentstroke}%
\pgfsetdash{}{0pt}%
\pgfpathmoveto{\pgfqpoint{3.906117in}{0.966014in}}%
\pgfpathcurveto{\pgfqpoint{3.913249in}{0.966014in}}{\pgfqpoint{3.920091in}{0.968848in}}{\pgfqpoint{3.925135in}{0.973891in}}%
\pgfpathcurveto{\pgfqpoint{3.930178in}{0.978935in}}{\pgfqpoint{3.933012in}{0.985777in}}{\pgfqpoint{3.933012in}{0.992909in}}%
\pgfpathcurveto{\pgfqpoint{3.933012in}{1.000042in}}{\pgfqpoint{3.930178in}{1.006884in}}{\pgfqpoint{3.925135in}{1.011928in}}%
\pgfpathcurveto{\pgfqpoint{3.920091in}{1.016971in}}{\pgfqpoint{3.913249in}{1.019805in}}{\pgfqpoint{3.906117in}{1.019805in}}%
\pgfpathcurveto{\pgfqpoint{3.898984in}{1.019805in}}{\pgfqpoint{3.892142in}{1.016971in}}{\pgfqpoint{3.887098in}{1.011928in}}%
\pgfpathcurveto{\pgfqpoint{3.882055in}{1.006884in}}{\pgfqpoint{3.879221in}{1.000042in}}{\pgfqpoint{3.879221in}{0.992909in}}%
\pgfpathcurveto{\pgfqpoint{3.879221in}{0.985777in}}{\pgfqpoint{3.882055in}{0.978935in}}{\pgfqpoint{3.887098in}{0.973891in}}%
\pgfpathcurveto{\pgfqpoint{3.892142in}{0.968848in}}{\pgfqpoint{3.898984in}{0.966014in}}{\pgfqpoint{3.906117in}{0.966014in}}%
\pgfpathclose%
\pgfusepath{stroke,fill}%
\end{pgfscope}%
\begin{pgfscope}%
\pgfpathrectangle{\pgfqpoint{2.867647in}{0.500000in}}{\pgfqpoint{1.764706in}{1.700000in}}%
\pgfusepath{clip}%
\pgfsetbuttcap%
\pgfsetroundjoin%
\definecolor{currentfill}{rgb}{0.981377,0.920617,0.865369}%
\pgfsetfillcolor{currentfill}%
\pgfsetlinewidth{0.311001pt}%
\definecolor{currentstroke}{rgb}{1.000000,1.000000,1.000000}%
\pgfsetstrokecolor{currentstroke}%
\pgfsetdash{}{0pt}%
\pgfpathmoveto{\pgfqpoint{4.189865in}{1.221631in}}%
\pgfpathcurveto{\pgfqpoint{4.196997in}{1.221631in}}{\pgfqpoint{4.203839in}{1.224465in}}{\pgfqpoint{4.208883in}{1.229508in}}%
\pgfpathcurveto{\pgfqpoint{4.213926in}{1.234552in}}{\pgfqpoint{4.216760in}{1.241394in}}{\pgfqpoint{4.216760in}{1.248526in}}%
\pgfpathcurveto{\pgfqpoint{4.216760in}{1.255659in}}{\pgfqpoint{4.213926in}{1.262501in}}{\pgfqpoint{4.208883in}{1.267545in}}%
\pgfpathcurveto{\pgfqpoint{4.203839in}{1.272588in}}{\pgfqpoint{4.196997in}{1.275422in}}{\pgfqpoint{4.189865in}{1.275422in}}%
\pgfpathcurveto{\pgfqpoint{4.182732in}{1.275422in}}{\pgfqpoint{4.175890in}{1.272588in}}{\pgfqpoint{4.170846in}{1.267545in}}%
\pgfpathcurveto{\pgfqpoint{4.165803in}{1.262501in}}{\pgfqpoint{4.162969in}{1.255659in}}{\pgfqpoint{4.162969in}{1.248526in}}%
\pgfpathcurveto{\pgfqpoint{4.162969in}{1.241394in}}{\pgfqpoint{4.165803in}{1.234552in}}{\pgfqpoint{4.170846in}{1.229508in}}%
\pgfpathcurveto{\pgfqpoint{4.175890in}{1.224465in}}{\pgfqpoint{4.182732in}{1.221631in}}{\pgfqpoint{4.189865in}{1.221631in}}%
\pgfpathclose%
\pgfusepath{stroke,fill}%
\end{pgfscope}%
\begin{pgfscope}%
\pgfpathrectangle{\pgfqpoint{2.867647in}{0.500000in}}{\pgfqpoint{1.764706in}{1.700000in}}%
\pgfusepath{clip}%
\pgfsetbuttcap%
\pgfsetroundjoin%
\definecolor{currentfill}{rgb}{0.972726,0.844889,0.754401}%
\pgfsetfillcolor{currentfill}%
\pgfsetlinewidth{0.311001pt}%
\definecolor{currentstroke}{rgb}{1.000000,1.000000,1.000000}%
\pgfsetstrokecolor{currentstroke}%
\pgfsetdash{}{0pt}%
\pgfpathmoveto{\pgfqpoint{4.233043in}{1.499555in}}%
\pgfpathcurveto{\pgfqpoint{4.240176in}{1.499555in}}{\pgfqpoint{4.247018in}{1.502389in}}{\pgfqpoint{4.252062in}{1.507432in}}%
\pgfpathcurveto{\pgfqpoint{4.257105in}{1.512476in}}{\pgfqpoint{4.259939in}{1.519318in}}{\pgfqpoint{4.259939in}{1.526450in}}%
\pgfpathcurveto{\pgfqpoint{4.259939in}{1.533583in}}{\pgfqpoint{4.257105in}{1.540425in}}{\pgfqpoint{4.252062in}{1.545469in}}%
\pgfpathcurveto{\pgfqpoint{4.247018in}{1.550512in}}{\pgfqpoint{4.240176in}{1.553346in}}{\pgfqpoint{4.233043in}{1.553346in}}%
\pgfpathcurveto{\pgfqpoint{4.225911in}{1.553346in}}{\pgfqpoint{4.219069in}{1.550512in}}{\pgfqpoint{4.214025in}{1.545469in}}%
\pgfpathcurveto{\pgfqpoint{4.208982in}{1.540425in}}{\pgfqpoint{4.206148in}{1.533583in}}{\pgfqpoint{4.206148in}{1.526450in}}%
\pgfpathcurveto{\pgfqpoint{4.206148in}{1.519318in}}{\pgfqpoint{4.208982in}{1.512476in}}{\pgfqpoint{4.214025in}{1.507432in}}%
\pgfpathcurveto{\pgfqpoint{4.219069in}{1.502389in}}{\pgfqpoint{4.225911in}{1.499555in}}{\pgfqpoint{4.233043in}{1.499555in}}%
\pgfpathclose%
\pgfusepath{stroke,fill}%
\end{pgfscope}%
\begin{pgfscope}%
\pgfpathrectangle{\pgfqpoint{2.867647in}{0.500000in}}{\pgfqpoint{1.764706in}{1.700000in}}%
\pgfusepath{clip}%
\pgfsetbuttcap%
\pgfsetroundjoin%
\definecolor{currentfill}{rgb}{0.980678,0.914765,0.856766}%
\pgfsetfillcolor{currentfill}%
\pgfsetlinewidth{0.311001pt}%
\definecolor{currentstroke}{rgb}{1.000000,1.000000,1.000000}%
\pgfsetstrokecolor{currentstroke}%
\pgfsetdash{}{0pt}%
\pgfpathmoveto{\pgfqpoint{4.192190in}{1.173849in}}%
\pgfpathcurveto{\pgfqpoint{4.199323in}{1.173849in}}{\pgfqpoint{4.206165in}{1.176683in}}{\pgfqpoint{4.211208in}{1.181726in}}%
\pgfpathcurveto{\pgfqpoint{4.216252in}{1.186770in}}{\pgfqpoint{4.219086in}{1.193612in}}{\pgfqpoint{4.219086in}{1.200745in}}%
\pgfpathcurveto{\pgfqpoint{4.219086in}{1.207877in}}{\pgfqpoint{4.216252in}{1.214719in}}{\pgfqpoint{4.211208in}{1.219763in}}%
\pgfpathcurveto{\pgfqpoint{4.206165in}{1.224806in}}{\pgfqpoint{4.199323in}{1.227640in}}{\pgfqpoint{4.192190in}{1.227640in}}%
\pgfpathcurveto{\pgfqpoint{4.185057in}{1.227640in}}{\pgfqpoint{4.178216in}{1.224806in}}{\pgfqpoint{4.173172in}{1.219763in}}%
\pgfpathcurveto{\pgfqpoint{4.168128in}{1.214719in}}{\pgfqpoint{4.165295in}{1.207877in}}{\pgfqpoint{4.165295in}{1.200745in}}%
\pgfpathcurveto{\pgfqpoint{4.165295in}{1.193612in}}{\pgfqpoint{4.168128in}{1.186770in}}{\pgfqpoint{4.173172in}{1.181726in}}%
\pgfpathcurveto{\pgfqpoint{4.178216in}{1.176683in}}{\pgfqpoint{4.185057in}{1.173849in}}{\pgfqpoint{4.192190in}{1.173849in}}%
\pgfpathclose%
\pgfusepath{stroke,fill}%
\end{pgfscope}%
\begin{pgfscope}%
\pgfpathrectangle{\pgfqpoint{2.867647in}{0.500000in}}{\pgfqpoint{1.764706in}{1.700000in}}%
\pgfusepath{clip}%
\pgfsetbuttcap%
\pgfsetroundjoin%
\definecolor{currentfill}{rgb}{0.979124,0.903132,0.839793}%
\pgfsetfillcolor{currentfill}%
\pgfsetlinewidth{0.311001pt}%
\definecolor{currentstroke}{rgb}{1.000000,1.000000,1.000000}%
\pgfsetstrokecolor{currentstroke}%
\pgfsetdash{}{0pt}%
\pgfpathmoveto{\pgfqpoint{4.148957in}{1.239018in}}%
\pgfpathcurveto{\pgfqpoint{4.156090in}{1.239018in}}{\pgfqpoint{4.162932in}{1.241852in}}{\pgfqpoint{4.167975in}{1.246895in}}%
\pgfpathcurveto{\pgfqpoint{4.173019in}{1.251939in}}{\pgfqpoint{4.175853in}{1.258781in}}{\pgfqpoint{4.175853in}{1.265913in}}%
\pgfpathcurveto{\pgfqpoint{4.175853in}{1.273046in}}{\pgfqpoint{4.173019in}{1.279888in}}{\pgfqpoint{4.167975in}{1.284932in}}%
\pgfpathcurveto{\pgfqpoint{4.162932in}{1.289975in}}{\pgfqpoint{4.156090in}{1.292809in}}{\pgfqpoint{4.148957in}{1.292809in}}%
\pgfpathcurveto{\pgfqpoint{4.141824in}{1.292809in}}{\pgfqpoint{4.134983in}{1.289975in}}{\pgfqpoint{4.129939in}{1.284932in}}%
\pgfpathcurveto{\pgfqpoint{4.124896in}{1.279888in}}{\pgfqpoint{4.122062in}{1.273046in}}{\pgfqpoint{4.122062in}{1.265913in}}%
\pgfpathcurveto{\pgfqpoint{4.122062in}{1.258781in}}{\pgfqpoint{4.124896in}{1.251939in}}{\pgfqpoint{4.129939in}{1.246895in}}%
\pgfpathcurveto{\pgfqpoint{4.134983in}{1.241852in}}{\pgfqpoint{4.141824in}{1.239018in}}{\pgfqpoint{4.148957in}{1.239018in}}%
\pgfpathclose%
\pgfusepath{stroke,fill}%
\end{pgfscope}%
\begin{pgfscope}%
\pgfpathrectangle{\pgfqpoint{2.867647in}{0.500000in}}{\pgfqpoint{1.764706in}{1.700000in}}%
\pgfusepath{clip}%
\pgfsetbuttcap%
\pgfsetroundjoin%
\definecolor{currentfill}{rgb}{0.968931,0.798091,0.685123}%
\pgfsetfillcolor{currentfill}%
\pgfsetlinewidth{0.311001pt}%
\definecolor{currentstroke}{rgb}{1.000000,1.000000,1.000000}%
\pgfsetstrokecolor{currentstroke}%
\pgfsetdash{}{0pt}%
\pgfpathmoveto{\pgfqpoint{4.156458in}{1.672903in}}%
\pgfpathcurveto{\pgfqpoint{4.163591in}{1.672903in}}{\pgfqpoint{4.170433in}{1.675737in}}{\pgfqpoint{4.175476in}{1.680781in}}%
\pgfpathcurveto{\pgfqpoint{4.180520in}{1.685825in}}{\pgfqpoint{4.183354in}{1.692666in}}{\pgfqpoint{4.183354in}{1.699799in}}%
\pgfpathcurveto{\pgfqpoint{4.183354in}{1.706932in}}{\pgfqpoint{4.180520in}{1.713774in}}{\pgfqpoint{4.175476in}{1.718817in}}%
\pgfpathcurveto{\pgfqpoint{4.170433in}{1.723861in}}{\pgfqpoint{4.163591in}{1.726695in}}{\pgfqpoint{4.156458in}{1.726695in}}%
\pgfpathcurveto{\pgfqpoint{4.149325in}{1.726695in}}{\pgfqpoint{4.142484in}{1.723861in}}{\pgfqpoint{4.137440in}{1.718817in}}%
\pgfpathcurveto{\pgfqpoint{4.132396in}{1.713774in}}{\pgfqpoint{4.129562in}{1.706932in}}{\pgfqpoint{4.129562in}{1.699799in}}%
\pgfpathcurveto{\pgfqpoint{4.129562in}{1.692666in}}{\pgfqpoint{4.132396in}{1.685825in}}{\pgfqpoint{4.137440in}{1.680781in}}%
\pgfpathcurveto{\pgfqpoint{4.142484in}{1.675737in}}{\pgfqpoint{4.149325in}{1.672903in}}{\pgfqpoint{4.156458in}{1.672903in}}%
\pgfpathclose%
\pgfusepath{stroke,fill}%
\end{pgfscope}%
\begin{pgfscope}%
\pgfpathrectangle{\pgfqpoint{2.867647in}{0.500000in}}{\pgfqpoint{1.764706in}{1.700000in}}%
\pgfusepath{clip}%
\pgfsetbuttcap%
\pgfsetroundjoin%
\definecolor{currentfill}{rgb}{0.973271,0.850724,0.762998}%
\pgfsetfillcolor{currentfill}%
\pgfsetlinewidth{0.311001pt}%
\definecolor{currentstroke}{rgb}{1.000000,1.000000,1.000000}%
\pgfsetstrokecolor{currentstroke}%
\pgfsetdash{}{0pt}%
\pgfpathmoveto{\pgfqpoint{4.068148in}{1.613457in}}%
\pgfpathcurveto{\pgfqpoint{4.075281in}{1.613457in}}{\pgfqpoint{4.082123in}{1.616290in}}{\pgfqpoint{4.087166in}{1.621334in}}%
\pgfpathcurveto{\pgfqpoint{4.092210in}{1.626378in}}{\pgfqpoint{4.095044in}{1.633219in}}{\pgfqpoint{4.095044in}{1.640352in}}%
\pgfpathcurveto{\pgfqpoint{4.095044in}{1.647485in}}{\pgfqpoint{4.092210in}{1.654327in}}{\pgfqpoint{4.087166in}{1.659370in}}%
\pgfpathcurveto{\pgfqpoint{4.082123in}{1.664414in}}{\pgfqpoint{4.075281in}{1.667248in}}{\pgfqpoint{4.068148in}{1.667248in}}%
\pgfpathcurveto{\pgfqpoint{4.061015in}{1.667248in}}{\pgfqpoint{4.054174in}{1.664414in}}{\pgfqpoint{4.049130in}{1.659370in}}%
\pgfpathcurveto{\pgfqpoint{4.044086in}{1.654327in}}{\pgfqpoint{4.041253in}{1.647485in}}{\pgfqpoint{4.041253in}{1.640352in}}%
\pgfpathcurveto{\pgfqpoint{4.041253in}{1.633219in}}{\pgfqpoint{4.044086in}{1.626378in}}{\pgfqpoint{4.049130in}{1.621334in}}%
\pgfpathcurveto{\pgfqpoint{4.054174in}{1.616290in}}{\pgfqpoint{4.061015in}{1.613457in}}{\pgfqpoint{4.068148in}{1.613457in}}%
\pgfpathclose%
\pgfusepath{stroke,fill}%
\end{pgfscope}%
\begin{pgfscope}%
\pgfpathrectangle{\pgfqpoint{2.867647in}{0.500000in}}{\pgfqpoint{1.764706in}{1.700000in}}%
\pgfusepath{clip}%
\pgfsetbuttcap%
\pgfsetroundjoin%
\definecolor{currentfill}{rgb}{0.966560,0.756582,0.625273}%
\pgfsetfillcolor{currentfill}%
\pgfsetlinewidth{0.311001pt}%
\definecolor{currentstroke}{rgb}{1.000000,1.000000,1.000000}%
\pgfsetstrokecolor{currentstroke}%
\pgfsetdash{}{0pt}%
\pgfpathmoveto{\pgfqpoint{4.177318in}{0.969063in}}%
\pgfpathcurveto{\pgfqpoint{4.184450in}{0.969063in}}{\pgfqpoint{4.191292in}{0.971897in}}{\pgfqpoint{4.196336in}{0.976941in}}%
\pgfpathcurveto{\pgfqpoint{4.201379in}{0.981984in}}{\pgfqpoint{4.204213in}{0.988826in}}{\pgfqpoint{4.204213in}{0.995959in}}%
\pgfpathcurveto{\pgfqpoint{4.204213in}{1.003092in}}{\pgfqpoint{4.201379in}{1.009933in}}{\pgfqpoint{4.196336in}{1.014977in}}%
\pgfpathcurveto{\pgfqpoint{4.191292in}{1.020021in}}{\pgfqpoint{4.184450in}{1.022855in}}{\pgfqpoint{4.177318in}{1.022855in}}%
\pgfpathcurveto{\pgfqpoint{4.170185in}{1.022855in}}{\pgfqpoint{4.163343in}{1.020021in}}{\pgfqpoint{4.158299in}{1.014977in}}%
\pgfpathcurveto{\pgfqpoint{4.153256in}{1.009933in}}{\pgfqpoint{4.150422in}{1.003092in}}{\pgfqpoint{4.150422in}{0.995959in}}%
\pgfpathcurveto{\pgfqpoint{4.150422in}{0.988826in}}{\pgfqpoint{4.153256in}{0.981984in}}{\pgfqpoint{4.158299in}{0.976941in}}%
\pgfpathcurveto{\pgfqpoint{4.163343in}{0.971897in}}{\pgfqpoint{4.170185in}{0.969063in}}{\pgfqpoint{4.177318in}{0.969063in}}%
\pgfpathclose%
\pgfusepath{stroke,fill}%
\end{pgfscope}%
\begin{pgfscope}%
\pgfpathrectangle{\pgfqpoint{2.867647in}{0.500000in}}{\pgfqpoint{1.764706in}{1.700000in}}%
\pgfusepath{clip}%
\pgfsetbuttcap%
\pgfsetroundjoin%
\definecolor{currentfill}{rgb}{0.960778,0.559972,0.399412}%
\pgfsetfillcolor{currentfill}%
\pgfsetlinewidth{0.311001pt}%
\definecolor{currentstroke}{rgb}{1.000000,1.000000,1.000000}%
\pgfsetstrokecolor{currentstroke}%
\pgfsetdash{}{0pt}%
\pgfpathmoveto{\pgfqpoint{3.943008in}{0.890650in}}%
\pgfpathcurveto{\pgfqpoint{3.950141in}{0.890650in}}{\pgfqpoint{3.956983in}{0.893483in}}{\pgfqpoint{3.962027in}{0.898527in}}%
\pgfpathcurveto{\pgfqpoint{3.967070in}{0.903571in}}{\pgfqpoint{3.969904in}{0.910412in}}{\pgfqpoint{3.969904in}{0.917545in}}%
\pgfpathcurveto{\pgfqpoint{3.969904in}{0.924678in}}{\pgfqpoint{3.967070in}{0.931520in}}{\pgfqpoint{3.962027in}{0.936563in}}%
\pgfpathcurveto{\pgfqpoint{3.956983in}{0.941607in}}{\pgfqpoint{3.950141in}{0.944441in}}{\pgfqpoint{3.943008in}{0.944441in}}%
\pgfpathcurveto{\pgfqpoint{3.935876in}{0.944441in}}{\pgfqpoint{3.929034in}{0.941607in}}{\pgfqpoint{3.923990in}{0.936563in}}%
\pgfpathcurveto{\pgfqpoint{3.918947in}{0.931520in}}{\pgfqpoint{3.916113in}{0.924678in}}{\pgfqpoint{3.916113in}{0.917545in}}%
\pgfpathcurveto{\pgfqpoint{3.916113in}{0.910412in}}{\pgfqpoint{3.918947in}{0.903571in}}{\pgfqpoint{3.923990in}{0.898527in}}%
\pgfpathcurveto{\pgfqpoint{3.929034in}{0.893483in}}{\pgfqpoint{3.935876in}{0.890650in}}{\pgfqpoint{3.943008in}{0.890650in}}%
\pgfpathclose%
\pgfusepath{stroke,fill}%
\end{pgfscope}%
\begin{pgfscope}%
\pgfpathrectangle{\pgfqpoint{2.867647in}{0.500000in}}{\pgfqpoint{1.764706in}{1.700000in}}%
\pgfusepath{clip}%
\pgfsetbuttcap%
\pgfsetroundjoin%
\definecolor{currentfill}{rgb}{0.969359,0.803954,0.693832}%
\pgfsetfillcolor{currentfill}%
\pgfsetlinewidth{0.311001pt}%
\definecolor{currentstroke}{rgb}{1.000000,1.000000,1.000000}%
\pgfsetstrokecolor{currentstroke}%
\pgfsetdash{}{0pt}%
\pgfpathmoveto{\pgfqpoint{4.271323in}{1.212387in}}%
\pgfpathcurveto{\pgfqpoint{4.278456in}{1.212387in}}{\pgfqpoint{4.285297in}{1.215221in}}{\pgfqpoint{4.290341in}{1.220265in}}%
\pgfpathcurveto{\pgfqpoint{4.295385in}{1.225309in}}{\pgfqpoint{4.298219in}{1.232150in}}{\pgfqpoint{4.298219in}{1.239283in}}%
\pgfpathcurveto{\pgfqpoint{4.298219in}{1.246416in}}{\pgfqpoint{4.295385in}{1.253258in}}{\pgfqpoint{4.290341in}{1.258301in}}%
\pgfpathcurveto{\pgfqpoint{4.285297in}{1.263345in}}{\pgfqpoint{4.278456in}{1.266179in}}{\pgfqpoint{4.271323in}{1.266179in}}%
\pgfpathcurveto{\pgfqpoint{4.264190in}{1.266179in}}{\pgfqpoint{4.257348in}{1.263345in}}{\pgfqpoint{4.252305in}{1.258301in}}%
\pgfpathcurveto{\pgfqpoint{4.247261in}{1.253258in}}{\pgfqpoint{4.244427in}{1.246416in}}{\pgfqpoint{4.244427in}{1.239283in}}%
\pgfpathcurveto{\pgfqpoint{4.244427in}{1.232150in}}{\pgfqpoint{4.247261in}{1.225309in}}{\pgfqpoint{4.252305in}{1.220265in}}%
\pgfpathcurveto{\pgfqpoint{4.257348in}{1.215221in}}{\pgfqpoint{4.264190in}{1.212387in}}{\pgfqpoint{4.271323in}{1.212387in}}%
\pgfpathclose%
\pgfusepath{stroke,fill}%
\end{pgfscope}%
\begin{pgfscope}%
\pgfpathrectangle{\pgfqpoint{2.867647in}{0.500000in}}{\pgfqpoint{1.764706in}{1.700000in}}%
\pgfusepath{clip}%
\pgfsetbuttcap%
\pgfsetroundjoin%
\definecolor{currentfill}{rgb}{0.979124,0.903132,0.839793}%
\pgfsetfillcolor{currentfill}%
\pgfsetlinewidth{0.311001pt}%
\definecolor{currentstroke}{rgb}{1.000000,1.000000,1.000000}%
\pgfsetstrokecolor{currentstroke}%
\pgfsetdash{}{0pt}%
\pgfpathmoveto{\pgfqpoint{4.148366in}{1.448843in}}%
\pgfpathcurveto{\pgfqpoint{4.155499in}{1.448843in}}{\pgfqpoint{4.162340in}{1.451677in}}{\pgfqpoint{4.167384in}{1.456721in}}%
\pgfpathcurveto{\pgfqpoint{4.172428in}{1.461765in}}{\pgfqpoint{4.175262in}{1.468606in}}{\pgfqpoint{4.175262in}{1.475739in}}%
\pgfpathcurveto{\pgfqpoint{4.175262in}{1.482872in}}{\pgfqpoint{4.172428in}{1.489713in}}{\pgfqpoint{4.167384in}{1.494757in}}%
\pgfpathcurveto{\pgfqpoint{4.162340in}{1.499801in}}{\pgfqpoint{4.155499in}{1.502635in}}{\pgfqpoint{4.148366in}{1.502635in}}%
\pgfpathcurveto{\pgfqpoint{4.141233in}{1.502635in}}{\pgfqpoint{4.134391in}{1.499801in}}{\pgfqpoint{4.129348in}{1.494757in}}%
\pgfpathcurveto{\pgfqpoint{4.124304in}{1.489713in}}{\pgfqpoint{4.121470in}{1.482872in}}{\pgfqpoint{4.121470in}{1.475739in}}%
\pgfpathcurveto{\pgfqpoint{4.121470in}{1.468606in}}{\pgfqpoint{4.124304in}{1.461765in}}{\pgfqpoint{4.129348in}{1.456721in}}%
\pgfpathcurveto{\pgfqpoint{4.134391in}{1.451677in}}{\pgfqpoint{4.141233in}{1.448843in}}{\pgfqpoint{4.148366in}{1.448843in}}%
\pgfpathclose%
\pgfusepath{stroke,fill}%
\end{pgfscope}%
\begin{pgfscope}%
\pgfpathrectangle{\pgfqpoint{2.867647in}{0.500000in}}{\pgfqpoint{1.764706in}{1.700000in}}%
\pgfusepath{clip}%
\pgfsetbuttcap%
\pgfsetroundjoin%
\definecolor{currentfill}{rgb}{0.965928,0.738443,0.600540}%
\pgfsetfillcolor{currentfill}%
\pgfsetlinewidth{0.311001pt}%
\definecolor{currentstroke}{rgb}{1.000000,1.000000,1.000000}%
\pgfsetstrokecolor{currentstroke}%
\pgfsetdash{}{0pt}%
\pgfpathmoveto{\pgfqpoint{4.043162in}{1.507874in}}%
\pgfpathcurveto{\pgfqpoint{4.050295in}{1.507874in}}{\pgfqpoint{4.057137in}{1.510708in}}{\pgfqpoint{4.062180in}{1.515751in}}%
\pgfpathcurveto{\pgfqpoint{4.067224in}{1.520795in}}{\pgfqpoint{4.070058in}{1.527637in}}{\pgfqpoint{4.070058in}{1.534770in}}%
\pgfpathcurveto{\pgfqpoint{4.070058in}{1.541902in}}{\pgfqpoint{4.067224in}{1.548744in}}{\pgfqpoint{4.062180in}{1.553788in}}%
\pgfpathcurveto{\pgfqpoint{4.057137in}{1.558831in}}{\pgfqpoint{4.050295in}{1.561665in}}{\pgfqpoint{4.043162in}{1.561665in}}%
\pgfpathcurveto{\pgfqpoint{4.036029in}{1.561665in}}{\pgfqpoint{4.029188in}{1.558831in}}{\pgfqpoint{4.024144in}{1.553788in}}%
\pgfpathcurveto{\pgfqpoint{4.019100in}{1.548744in}}{\pgfqpoint{4.016266in}{1.541902in}}{\pgfqpoint{4.016266in}{1.534770in}}%
\pgfpathcurveto{\pgfqpoint{4.016266in}{1.527637in}}{\pgfqpoint{4.019100in}{1.520795in}}{\pgfqpoint{4.024144in}{1.515751in}}%
\pgfpathcurveto{\pgfqpoint{4.029188in}{1.510708in}}{\pgfqpoint{4.036029in}{1.507874in}}{\pgfqpoint{4.043162in}{1.507874in}}%
\pgfpathclose%
\pgfusepath{stroke,fill}%
\end{pgfscope}%
\begin{pgfscope}%
\pgfpathrectangle{\pgfqpoint{2.867647in}{0.500000in}}{\pgfqpoint{1.764706in}{1.700000in}}%
\pgfusepath{clip}%
\pgfsetbuttcap%
\pgfsetroundjoin%
\definecolor{currentfill}{rgb}{0.958791,0.526283,0.368909}%
\pgfsetfillcolor{currentfill}%
\pgfsetlinewidth{0.311001pt}%
\definecolor{currentstroke}{rgb}{1.000000,1.000000,1.000000}%
\pgfsetstrokecolor{currentstroke}%
\pgfsetdash{}{0pt}%
\pgfpathmoveto{\pgfqpoint{3.961109in}{0.848518in}}%
\pgfpathcurveto{\pgfqpoint{3.968242in}{0.848518in}}{\pgfqpoint{3.975084in}{0.851352in}}{\pgfqpoint{3.980127in}{0.856395in}}%
\pgfpathcurveto{\pgfqpoint{3.985171in}{0.861439in}}{\pgfqpoint{3.988005in}{0.868281in}}{\pgfqpoint{3.988005in}{0.875413in}}%
\pgfpathcurveto{\pgfqpoint{3.988005in}{0.882546in}}{\pgfqpoint{3.985171in}{0.889388in}}{\pgfqpoint{3.980127in}{0.894432in}}%
\pgfpathcurveto{\pgfqpoint{3.975084in}{0.899475in}}{\pgfqpoint{3.968242in}{0.902309in}}{\pgfqpoint{3.961109in}{0.902309in}}%
\pgfpathcurveto{\pgfqpoint{3.953976in}{0.902309in}}{\pgfqpoint{3.947135in}{0.899475in}}{\pgfqpoint{3.942091in}{0.894432in}}%
\pgfpathcurveto{\pgfqpoint{3.937047in}{0.889388in}}{\pgfqpoint{3.934214in}{0.882546in}}{\pgfqpoint{3.934214in}{0.875413in}}%
\pgfpathcurveto{\pgfqpoint{3.934214in}{0.868281in}}{\pgfqpoint{3.937047in}{0.861439in}}{\pgfqpoint{3.942091in}{0.856395in}}%
\pgfpathcurveto{\pgfqpoint{3.947135in}{0.851352in}}{\pgfqpoint{3.953976in}{0.848518in}}{\pgfqpoint{3.961109in}{0.848518in}}%
\pgfpathclose%
\pgfusepath{stroke,fill}%
\end{pgfscope}%
\begin{pgfscope}%
\pgfpathrectangle{\pgfqpoint{2.867647in}{0.500000in}}{\pgfqpoint{1.764706in}{1.700000in}}%
\pgfusepath{clip}%
\pgfsetbuttcap%
\pgfsetroundjoin%
\definecolor{currentfill}{rgb}{0.958331,0.519463,0.362986}%
\pgfsetfillcolor{currentfill}%
\pgfsetlinewidth{0.311001pt}%
\definecolor{currentstroke}{rgb}{1.000000,1.000000,1.000000}%
\pgfsetstrokecolor{currentstroke}%
\pgfsetdash{}{0pt}%
\pgfpathmoveto{\pgfqpoint{4.347422in}{1.284743in}}%
\pgfpathcurveto{\pgfqpoint{4.354555in}{1.284743in}}{\pgfqpoint{4.361396in}{1.287577in}}{\pgfqpoint{4.366440in}{1.292620in}}%
\pgfpathcurveto{\pgfqpoint{4.371484in}{1.297664in}}{\pgfqpoint{4.374318in}{1.304506in}}{\pgfqpoint{4.374318in}{1.311638in}}%
\pgfpathcurveto{\pgfqpoint{4.374318in}{1.318771in}}{\pgfqpoint{4.371484in}{1.325613in}}{\pgfqpoint{4.366440in}{1.330657in}}%
\pgfpathcurveto{\pgfqpoint{4.361396in}{1.335700in}}{\pgfqpoint{4.354555in}{1.338534in}}{\pgfqpoint{4.347422in}{1.338534in}}%
\pgfpathcurveto{\pgfqpoint{4.340289in}{1.338534in}}{\pgfqpoint{4.333447in}{1.335700in}}{\pgfqpoint{4.328404in}{1.330657in}}%
\pgfpathcurveto{\pgfqpoint{4.323360in}{1.325613in}}{\pgfqpoint{4.320526in}{1.318771in}}{\pgfqpoint{4.320526in}{1.311638in}}%
\pgfpathcurveto{\pgfqpoint{4.320526in}{1.304506in}}{\pgfqpoint{4.323360in}{1.297664in}}{\pgfqpoint{4.328404in}{1.292620in}}%
\pgfpathcurveto{\pgfqpoint{4.333447in}{1.287577in}}{\pgfqpoint{4.340289in}{1.284743in}}{\pgfqpoint{4.347422in}{1.284743in}}%
\pgfpathclose%
\pgfusepath{stroke,fill}%
\end{pgfscope}%
\begin{pgfscope}%
\pgfpathrectangle{\pgfqpoint{2.867647in}{0.500000in}}{\pgfqpoint{1.764706in}{1.700000in}}%
\pgfusepath{clip}%
\pgfsetbuttcap%
\pgfsetroundjoin%
\definecolor{currentfill}{rgb}{0.971694,0.833208,0.737161}%
\pgfsetfillcolor{currentfill}%
\pgfsetlinewidth{0.311001pt}%
\definecolor{currentstroke}{rgb}{1.000000,1.000000,1.000000}%
\pgfsetstrokecolor{currentstroke}%
\pgfsetdash{}{0pt}%
\pgfpathmoveto{\pgfqpoint{4.115313in}{1.310493in}}%
\pgfpathcurveto{\pgfqpoint{4.122446in}{1.310493in}}{\pgfqpoint{4.129288in}{1.313327in}}{\pgfqpoint{4.134332in}{1.318371in}}%
\pgfpathcurveto{\pgfqpoint{4.139375in}{1.323415in}}{\pgfqpoint{4.142209in}{1.330256in}}{\pgfqpoint{4.142209in}{1.337389in}}%
\pgfpathcurveto{\pgfqpoint{4.142209in}{1.344522in}}{\pgfqpoint{4.139375in}{1.351364in}}{\pgfqpoint{4.134332in}{1.356407in}}%
\pgfpathcurveto{\pgfqpoint{4.129288in}{1.361451in}}{\pgfqpoint{4.122446in}{1.364285in}}{\pgfqpoint{4.115313in}{1.364285in}}%
\pgfpathcurveto{\pgfqpoint{4.108181in}{1.364285in}}{\pgfqpoint{4.101339in}{1.361451in}}{\pgfqpoint{4.096295in}{1.356407in}}%
\pgfpathcurveto{\pgfqpoint{4.091252in}{1.351364in}}{\pgfqpoint{4.088418in}{1.344522in}}{\pgfqpoint{4.088418in}{1.337389in}}%
\pgfpathcurveto{\pgfqpoint{4.088418in}{1.330256in}}{\pgfqpoint{4.091252in}{1.323415in}}{\pgfqpoint{4.096295in}{1.318371in}}%
\pgfpathcurveto{\pgfqpoint{4.101339in}{1.313327in}}{\pgfqpoint{4.108181in}{1.310493in}}{\pgfqpoint{4.115313in}{1.310493in}}%
\pgfpathclose%
\pgfusepath{stroke,fill}%
\end{pgfscope}%
\begin{pgfscope}%
\pgfpathrectangle{\pgfqpoint{2.867647in}{0.500000in}}{\pgfqpoint{1.764706in}{1.700000in}}%
\pgfusepath{clip}%
\pgfsetbuttcap%
\pgfsetroundjoin%
\definecolor{currentfill}{rgb}{0.969803,0.809811,0.702523}%
\pgfsetfillcolor{currentfill}%
\pgfsetlinewidth{0.311001pt}%
\definecolor{currentstroke}{rgb}{1.000000,1.000000,1.000000}%
\pgfsetstrokecolor{currentstroke}%
\pgfsetdash{}{0pt}%
\pgfpathmoveto{\pgfqpoint{4.084108in}{1.174295in}}%
\pgfpathcurveto{\pgfqpoint{4.091241in}{1.174295in}}{\pgfqpoint{4.098082in}{1.177129in}}{\pgfqpoint{4.103126in}{1.182173in}}%
\pgfpathcurveto{\pgfqpoint{4.108170in}{1.187216in}}{\pgfqpoint{4.111004in}{1.194058in}}{\pgfqpoint{4.111004in}{1.201191in}}%
\pgfpathcurveto{\pgfqpoint{4.111004in}{1.208324in}}{\pgfqpoint{4.108170in}{1.215165in}}{\pgfqpoint{4.103126in}{1.220209in}}%
\pgfpathcurveto{\pgfqpoint{4.098082in}{1.225253in}}{\pgfqpoint{4.091241in}{1.228087in}}{\pgfqpoint{4.084108in}{1.228087in}}%
\pgfpathcurveto{\pgfqpoint{4.076975in}{1.228087in}}{\pgfqpoint{4.070133in}{1.225253in}}{\pgfqpoint{4.065090in}{1.220209in}}%
\pgfpathcurveto{\pgfqpoint{4.060046in}{1.215165in}}{\pgfqpoint{4.057212in}{1.208324in}}{\pgfqpoint{4.057212in}{1.201191in}}%
\pgfpathcurveto{\pgfqpoint{4.057212in}{1.194058in}}{\pgfqpoint{4.060046in}{1.187216in}}{\pgfqpoint{4.065090in}{1.182173in}}%
\pgfpathcurveto{\pgfqpoint{4.070133in}{1.177129in}}{\pgfqpoint{4.076975in}{1.174295in}}{\pgfqpoint{4.084108in}{1.174295in}}%
\pgfpathclose%
\pgfusepath{stroke,fill}%
\end{pgfscope}%
\begin{pgfscope}%
\pgfpathrectangle{\pgfqpoint{2.867647in}{0.500000in}}{\pgfqpoint{1.764706in}{1.700000in}}%
\pgfusepath{clip}%
\pgfsetbuttcap%
\pgfsetroundjoin%
\definecolor{currentfill}{rgb}{0.971694,0.833208,0.737161}%
\pgfsetfillcolor{currentfill}%
\pgfsetlinewidth{0.311001pt}%
\definecolor{currentstroke}{rgb}{1.000000,1.000000,1.000000}%
\pgfsetstrokecolor{currentstroke}%
\pgfsetdash{}{0pt}%
\pgfpathmoveto{\pgfqpoint{4.105147in}{1.226792in}}%
\pgfpathcurveto{\pgfqpoint{4.112279in}{1.226792in}}{\pgfqpoint{4.119121in}{1.229626in}}{\pgfqpoint{4.124165in}{1.234670in}}%
\pgfpathcurveto{\pgfqpoint{4.129208in}{1.239714in}}{\pgfqpoint{4.132042in}{1.246555in}}{\pgfqpoint{4.132042in}{1.253688in}}%
\pgfpathcurveto{\pgfqpoint{4.132042in}{1.260821in}}{\pgfqpoint{4.129208in}{1.267663in}}{\pgfqpoint{4.124165in}{1.272706in}}%
\pgfpathcurveto{\pgfqpoint{4.119121in}{1.277750in}}{\pgfqpoint{4.112279in}{1.280584in}}{\pgfqpoint{4.105147in}{1.280584in}}%
\pgfpathcurveto{\pgfqpoint{4.098014in}{1.280584in}}{\pgfqpoint{4.091172in}{1.277750in}}{\pgfqpoint{4.086128in}{1.272706in}}%
\pgfpathcurveto{\pgfqpoint{4.081085in}{1.267663in}}{\pgfqpoint{4.078251in}{1.260821in}}{\pgfqpoint{4.078251in}{1.253688in}}%
\pgfpathcurveto{\pgfqpoint{4.078251in}{1.246555in}}{\pgfqpoint{4.081085in}{1.239714in}}{\pgfqpoint{4.086128in}{1.234670in}}%
\pgfpathcurveto{\pgfqpoint{4.091172in}{1.229626in}}{\pgfqpoint{4.098014in}{1.226792in}}{\pgfqpoint{4.105147in}{1.226792in}}%
\pgfpathclose%
\pgfusepath{stroke,fill}%
\end{pgfscope}%
\begin{pgfscope}%
\pgfpathrectangle{\pgfqpoint{2.867647in}{0.500000in}}{\pgfqpoint{1.764706in}{1.700000in}}%
\pgfusepath{clip}%
\pgfsetbuttcap%
\pgfsetroundjoin%
\definecolor{currentfill}{rgb}{0.978376,0.897317,0.831308}%
\pgfsetfillcolor{currentfill}%
\pgfsetlinewidth{0.311001pt}%
\definecolor{currentstroke}{rgb}{1.000000,1.000000,1.000000}%
\pgfsetstrokecolor{currentstroke}%
\pgfsetdash{}{0pt}%
\pgfpathmoveto{\pgfqpoint{4.171037in}{1.082016in}}%
\pgfpathcurveto{\pgfqpoint{4.178170in}{1.082016in}}{\pgfqpoint{4.185012in}{1.084850in}}{\pgfqpoint{4.190056in}{1.089894in}}%
\pgfpathcurveto{\pgfqpoint{4.195099in}{1.094937in}}{\pgfqpoint{4.197933in}{1.101779in}}{\pgfqpoint{4.197933in}{1.108912in}}%
\pgfpathcurveto{\pgfqpoint{4.197933in}{1.116045in}}{\pgfqpoint{4.195099in}{1.122886in}}{\pgfqpoint{4.190056in}{1.127930in}}%
\pgfpathcurveto{\pgfqpoint{4.185012in}{1.132974in}}{\pgfqpoint{4.178170in}{1.135807in}}{\pgfqpoint{4.171037in}{1.135807in}}%
\pgfpathcurveto{\pgfqpoint{4.163905in}{1.135807in}}{\pgfqpoint{4.157063in}{1.132974in}}{\pgfqpoint{4.152019in}{1.127930in}}%
\pgfpathcurveto{\pgfqpoint{4.146976in}{1.122886in}}{\pgfqpoint{4.144142in}{1.116045in}}{\pgfqpoint{4.144142in}{1.108912in}}%
\pgfpathcurveto{\pgfqpoint{4.144142in}{1.101779in}}{\pgfqpoint{4.146976in}{1.094937in}}{\pgfqpoint{4.152019in}{1.089894in}}%
\pgfpathcurveto{\pgfqpoint{4.157063in}{1.084850in}}{\pgfqpoint{4.163905in}{1.082016in}}{\pgfqpoint{4.171037in}{1.082016in}}%
\pgfpathclose%
\pgfusepath{stroke,fill}%
\end{pgfscope}%
\begin{pgfscope}%
\pgfpathrectangle{\pgfqpoint{2.867647in}{0.500000in}}{\pgfqpoint{1.764706in}{1.700000in}}%
\pgfusepath{clip}%
\pgfsetbuttcap%
\pgfsetroundjoin%
\definecolor{currentfill}{rgb}{0.978376,0.897317,0.831308}%
\pgfsetfillcolor{currentfill}%
\pgfsetlinewidth{0.311001pt}%
\definecolor{currentstroke}{rgb}{1.000000,1.000000,1.000000}%
\pgfsetstrokecolor{currentstroke}%
\pgfsetdash{}{0pt}%
\pgfpathmoveto{\pgfqpoint{4.145121in}{1.241408in}}%
\pgfpathcurveto{\pgfqpoint{4.152253in}{1.241408in}}{\pgfqpoint{4.159095in}{1.244242in}}{\pgfqpoint{4.164139in}{1.249286in}}%
\pgfpathcurveto{\pgfqpoint{4.169182in}{1.254330in}}{\pgfqpoint{4.172016in}{1.261171in}}{\pgfqpoint{4.172016in}{1.268304in}}%
\pgfpathcurveto{\pgfqpoint{4.172016in}{1.275437in}}{\pgfqpoint{4.169182in}{1.282278in}}{\pgfqpoint{4.164139in}{1.287322in}}%
\pgfpathcurveto{\pgfqpoint{4.159095in}{1.292366in}}{\pgfqpoint{4.152253in}{1.295200in}}{\pgfqpoint{4.145121in}{1.295200in}}%
\pgfpathcurveto{\pgfqpoint{4.137988in}{1.295200in}}{\pgfqpoint{4.131146in}{1.292366in}}{\pgfqpoint{4.126102in}{1.287322in}}%
\pgfpathcurveto{\pgfqpoint{4.121059in}{1.282278in}}{\pgfqpoint{4.118225in}{1.275437in}}{\pgfqpoint{4.118225in}{1.268304in}}%
\pgfpathcurveto{\pgfqpoint{4.118225in}{1.261171in}}{\pgfqpoint{4.121059in}{1.254330in}}{\pgfqpoint{4.126102in}{1.249286in}}%
\pgfpathcurveto{\pgfqpoint{4.131146in}{1.244242in}}{\pgfqpoint{4.137988in}{1.241408in}}{\pgfqpoint{4.145121in}{1.241408in}}%
\pgfpathclose%
\pgfusepath{stroke,fill}%
\end{pgfscope}%
\begin{pgfscope}%
\pgfpathrectangle{\pgfqpoint{2.867647in}{0.500000in}}{\pgfqpoint{1.764706in}{1.700000in}}%
\pgfusepath{clip}%
\pgfsetbuttcap%
\pgfsetroundjoin%
\definecolor{currentfill}{rgb}{0.966120,0.744512,0.608720}%
\pgfsetfillcolor{currentfill}%
\pgfsetlinewidth{0.311001pt}%
\definecolor{currentstroke}{rgb}{1.000000,1.000000,1.000000}%
\pgfsetstrokecolor{currentstroke}%
\pgfsetdash{}{0pt}%
\pgfpathmoveto{\pgfqpoint{4.073468in}{1.414479in}}%
\pgfpathcurveto{\pgfqpoint{4.080601in}{1.414479in}}{\pgfqpoint{4.087442in}{1.417313in}}{\pgfqpoint{4.092486in}{1.422357in}}%
\pgfpathcurveto{\pgfqpoint{4.097530in}{1.427400in}}{\pgfqpoint{4.100364in}{1.434242in}}{\pgfqpoint{4.100364in}{1.441375in}}%
\pgfpathcurveto{\pgfqpoint{4.100364in}{1.448508in}}{\pgfqpoint{4.097530in}{1.455349in}}{\pgfqpoint{4.092486in}{1.460393in}}%
\pgfpathcurveto{\pgfqpoint{4.087442in}{1.465436in}}{\pgfqpoint{4.080601in}{1.468270in}}{\pgfqpoint{4.073468in}{1.468270in}}%
\pgfpathcurveto{\pgfqpoint{4.066335in}{1.468270in}}{\pgfqpoint{4.059493in}{1.465436in}}{\pgfqpoint{4.054450in}{1.460393in}}%
\pgfpathcurveto{\pgfqpoint{4.049406in}{1.455349in}}{\pgfqpoint{4.046572in}{1.448508in}}{\pgfqpoint{4.046572in}{1.441375in}}%
\pgfpathcurveto{\pgfqpoint{4.046572in}{1.434242in}}{\pgfqpoint{4.049406in}{1.427400in}}{\pgfqpoint{4.054450in}{1.422357in}}%
\pgfpathcurveto{\pgfqpoint{4.059493in}{1.417313in}}{\pgfqpoint{4.066335in}{1.414479in}}{\pgfqpoint{4.073468in}{1.414479in}}%
\pgfpathclose%
\pgfusepath{stroke,fill}%
\end{pgfscope}%
\begin{pgfscope}%
\pgfpathrectangle{\pgfqpoint{2.867647in}{0.500000in}}{\pgfqpoint{1.764706in}{1.700000in}}%
\pgfusepath{clip}%
\pgfsetbuttcap%
\pgfsetroundjoin%
\definecolor{currentfill}{rgb}{0.979891,0.908948,0.848279}%
\pgfsetfillcolor{currentfill}%
\pgfsetlinewidth{0.311001pt}%
\definecolor{currentstroke}{rgb}{1.000000,1.000000,1.000000}%
\pgfsetstrokecolor{currentstroke}%
\pgfsetdash{}{0pt}%
\pgfpathmoveto{\pgfqpoint{4.203053in}{1.402705in}}%
\pgfpathcurveto{\pgfqpoint{4.210186in}{1.402705in}}{\pgfqpoint{4.217028in}{1.405539in}}{\pgfqpoint{4.222071in}{1.410583in}}%
\pgfpathcurveto{\pgfqpoint{4.227115in}{1.415626in}}{\pgfqpoint{4.229949in}{1.422468in}}{\pgfqpoint{4.229949in}{1.429601in}}%
\pgfpathcurveto{\pgfqpoint{4.229949in}{1.436734in}}{\pgfqpoint{4.227115in}{1.443575in}}{\pgfqpoint{4.222071in}{1.448619in}}%
\pgfpathcurveto{\pgfqpoint{4.217028in}{1.453663in}}{\pgfqpoint{4.210186in}{1.456497in}}{\pgfqpoint{4.203053in}{1.456497in}}%
\pgfpathcurveto{\pgfqpoint{4.195920in}{1.456497in}}{\pgfqpoint{4.189079in}{1.453663in}}{\pgfqpoint{4.184035in}{1.448619in}}%
\pgfpathcurveto{\pgfqpoint{4.178991in}{1.443575in}}{\pgfqpoint{4.176158in}{1.436734in}}{\pgfqpoint{4.176158in}{1.429601in}}%
\pgfpathcurveto{\pgfqpoint{4.176158in}{1.422468in}}{\pgfqpoint{4.178991in}{1.415626in}}{\pgfqpoint{4.184035in}{1.410583in}}%
\pgfpathcurveto{\pgfqpoint{4.189079in}{1.405539in}}{\pgfqpoint{4.195920in}{1.402705in}}{\pgfqpoint{4.203053in}{1.402705in}}%
\pgfpathclose%
\pgfusepath{stroke,fill}%
\end{pgfscope}%
\begin{pgfscope}%
\pgfpathrectangle{\pgfqpoint{2.867647in}{0.500000in}}{\pgfqpoint{1.764706in}{1.700000in}}%
\pgfusepath{clip}%
\pgfsetbuttcap%
\pgfsetroundjoin%
\definecolor{currentfill}{rgb}{0.972201,0.839051,0.745789}%
\pgfsetfillcolor{currentfill}%
\pgfsetlinewidth{0.311001pt}%
\definecolor{currentstroke}{rgb}{1.000000,1.000000,1.000000}%
\pgfsetstrokecolor{currentstroke}%
\pgfsetdash{}{0pt}%
\pgfpathmoveto{\pgfqpoint{4.060575in}{1.638107in}}%
\pgfpathcurveto{\pgfqpoint{4.067708in}{1.638107in}}{\pgfqpoint{4.074549in}{1.640940in}}{\pgfqpoint{4.079593in}{1.645984in}}%
\pgfpathcurveto{\pgfqpoint{4.084637in}{1.651028in}}{\pgfqpoint{4.087470in}{1.657869in}}{\pgfqpoint{4.087470in}{1.665002in}}%
\pgfpathcurveto{\pgfqpoint{4.087470in}{1.672135in}}{\pgfqpoint{4.084637in}{1.678977in}}{\pgfqpoint{4.079593in}{1.684020in}}%
\pgfpathcurveto{\pgfqpoint{4.074549in}{1.689064in}}{\pgfqpoint{4.067708in}{1.691898in}}{\pgfqpoint{4.060575in}{1.691898in}}%
\pgfpathcurveto{\pgfqpoint{4.053442in}{1.691898in}}{\pgfqpoint{4.046600in}{1.689064in}}{\pgfqpoint{4.041557in}{1.684020in}}%
\pgfpathcurveto{\pgfqpoint{4.036513in}{1.678977in}}{\pgfqpoint{4.033679in}{1.672135in}}{\pgfqpoint{4.033679in}{1.665002in}}%
\pgfpathcurveto{\pgfqpoint{4.033679in}{1.657869in}}{\pgfqpoint{4.036513in}{1.651028in}}{\pgfqpoint{4.041557in}{1.645984in}}%
\pgfpathcurveto{\pgfqpoint{4.046600in}{1.640940in}}{\pgfqpoint{4.053442in}{1.638107in}}{\pgfqpoint{4.060575in}{1.638107in}}%
\pgfpathclose%
\pgfusepath{stroke,fill}%
\end{pgfscope}%
\begin{pgfscope}%
\pgfpathrectangle{\pgfqpoint{2.867647in}{0.500000in}}{\pgfqpoint{1.764706in}{1.700000in}}%
\pgfusepath{clip}%
\pgfsetbuttcap%
\pgfsetroundjoin%
\definecolor{currentfill}{rgb}{0.979891,0.908948,0.848279}%
\pgfsetfillcolor{currentfill}%
\pgfsetlinewidth{0.311001pt}%
\definecolor{currentstroke}{rgb}{1.000000,1.000000,1.000000}%
\pgfsetstrokecolor{currentstroke}%
\pgfsetdash{}{0pt}%
\pgfpathmoveto{\pgfqpoint{4.183585in}{1.498836in}}%
\pgfpathcurveto{\pgfqpoint{4.190718in}{1.498836in}}{\pgfqpoint{4.197559in}{1.501670in}}{\pgfqpoint{4.202603in}{1.506713in}}%
\pgfpathcurveto{\pgfqpoint{4.207647in}{1.511757in}}{\pgfqpoint{4.210481in}{1.518599in}}{\pgfqpoint{4.210481in}{1.525731in}}%
\pgfpathcurveto{\pgfqpoint{4.210481in}{1.532864in}}{\pgfqpoint{4.207647in}{1.539706in}}{\pgfqpoint{4.202603in}{1.544749in}}%
\pgfpathcurveto{\pgfqpoint{4.197559in}{1.549793in}}{\pgfqpoint{4.190718in}{1.552627in}}{\pgfqpoint{4.183585in}{1.552627in}}%
\pgfpathcurveto{\pgfqpoint{4.176452in}{1.552627in}}{\pgfqpoint{4.169610in}{1.549793in}}{\pgfqpoint{4.164567in}{1.544749in}}%
\pgfpathcurveto{\pgfqpoint{4.159523in}{1.539706in}}{\pgfqpoint{4.156689in}{1.532864in}}{\pgfqpoint{4.156689in}{1.525731in}}%
\pgfpathcurveto{\pgfqpoint{4.156689in}{1.518599in}}{\pgfqpoint{4.159523in}{1.511757in}}{\pgfqpoint{4.164567in}{1.506713in}}%
\pgfpathcurveto{\pgfqpoint{4.169610in}{1.501670in}}{\pgfqpoint{4.176452in}{1.498836in}}{\pgfqpoint{4.183585in}{1.498836in}}%
\pgfpathclose%
\pgfusepath{stroke,fill}%
\end{pgfscope}%
\begin{pgfscope}%
\pgfpathrectangle{\pgfqpoint{2.867647in}{0.500000in}}{\pgfqpoint{1.764706in}{1.700000in}}%
\pgfusepath{clip}%
\pgfsetbuttcap%
\pgfsetroundjoin%
\definecolor{currentfill}{rgb}{0.967092,0.768560,0.642079}%
\pgfsetfillcolor{currentfill}%
\pgfsetlinewidth{0.311001pt}%
\definecolor{currentstroke}{rgb}{1.000000,1.000000,1.000000}%
\pgfsetstrokecolor{currentstroke}%
\pgfsetdash{}{0pt}%
\pgfpathmoveto{\pgfqpoint{4.170472in}{0.968255in}}%
\pgfpathcurveto{\pgfqpoint{4.177605in}{0.968255in}}{\pgfqpoint{4.184446in}{0.971089in}}{\pgfqpoint{4.189490in}{0.976132in}}%
\pgfpathcurveto{\pgfqpoint{4.194534in}{0.981176in}}{\pgfqpoint{4.197368in}{0.988018in}}{\pgfqpoint{4.197368in}{0.995151in}}%
\pgfpathcurveto{\pgfqpoint{4.197368in}{1.002283in}}{\pgfqpoint{4.194534in}{1.009125in}}{\pgfqpoint{4.189490in}{1.014169in}}%
\pgfpathcurveto{\pgfqpoint{4.184446in}{1.019212in}}{\pgfqpoint{4.177605in}{1.022046in}}{\pgfqpoint{4.170472in}{1.022046in}}%
\pgfpathcurveto{\pgfqpoint{4.163339in}{1.022046in}}{\pgfqpoint{4.156497in}{1.019212in}}{\pgfqpoint{4.151454in}{1.014169in}}%
\pgfpathcurveto{\pgfqpoint{4.146410in}{1.009125in}}{\pgfqpoint{4.143576in}{1.002283in}}{\pgfqpoint{4.143576in}{0.995151in}}%
\pgfpathcurveto{\pgfqpoint{4.143576in}{0.988018in}}{\pgfqpoint{4.146410in}{0.981176in}}{\pgfqpoint{4.151454in}{0.976132in}}%
\pgfpathcurveto{\pgfqpoint{4.156497in}{0.971089in}}{\pgfqpoint{4.163339in}{0.968255in}}{\pgfqpoint{4.170472in}{0.968255in}}%
\pgfpathclose%
\pgfusepath{stroke,fill}%
\end{pgfscope}%
\begin{pgfscope}%
\pgfpathrectangle{\pgfqpoint{2.867647in}{0.500000in}}{\pgfqpoint{1.764706in}{1.700000in}}%
\pgfusepath{clip}%
\pgfsetbuttcap%
\pgfsetroundjoin%
\definecolor{currentfill}{rgb}{0.961734,0.579886,0.418445}%
\pgfsetfillcolor{currentfill}%
\pgfsetlinewidth{0.311001pt}%
\definecolor{currentstroke}{rgb}{1.000000,1.000000,1.000000}%
\pgfsetstrokecolor{currentstroke}%
\pgfsetdash{}{0pt}%
\pgfpathmoveto{\pgfqpoint{4.323529in}{1.180306in}}%
\pgfpathcurveto{\pgfqpoint{4.330662in}{1.180306in}}{\pgfqpoint{4.337504in}{1.183140in}}{\pgfqpoint{4.342548in}{1.188183in}}%
\pgfpathcurveto{\pgfqpoint{4.347591in}{1.193227in}}{\pgfqpoint{4.350425in}{1.200069in}}{\pgfqpoint{4.350425in}{1.207201in}}%
\pgfpathcurveto{\pgfqpoint{4.350425in}{1.214334in}}{\pgfqpoint{4.347591in}{1.221176in}}{\pgfqpoint{4.342548in}{1.226220in}}%
\pgfpathcurveto{\pgfqpoint{4.337504in}{1.231263in}}{\pgfqpoint{4.330662in}{1.234097in}}{\pgfqpoint{4.323529in}{1.234097in}}%
\pgfpathcurveto{\pgfqpoint{4.316397in}{1.234097in}}{\pgfqpoint{4.309555in}{1.231263in}}{\pgfqpoint{4.304511in}{1.226220in}}%
\pgfpathcurveto{\pgfqpoint{4.299468in}{1.221176in}}{\pgfqpoint{4.296634in}{1.214334in}}{\pgfqpoint{4.296634in}{1.207201in}}%
\pgfpathcurveto{\pgfqpoint{4.296634in}{1.200069in}}{\pgfqpoint{4.299468in}{1.193227in}}{\pgfqpoint{4.304511in}{1.188183in}}%
\pgfpathcurveto{\pgfqpoint{4.309555in}{1.183140in}}{\pgfqpoint{4.316397in}{1.180306in}}{\pgfqpoint{4.323529in}{1.180306in}}%
\pgfpathclose%
\pgfusepath{stroke,fill}%
\end{pgfscope}%
\begin{pgfscope}%
\pgfpathrectangle{\pgfqpoint{2.867647in}{0.500000in}}{\pgfqpoint{1.764706in}{1.700000in}}%
\pgfusepath{clip}%
\pgfsetbuttcap%
\pgfsetroundjoin%
\definecolor{currentfill}{rgb}{0.969803,0.809811,0.702523}%
\pgfsetfillcolor{currentfill}%
\pgfsetlinewidth{0.311001pt}%
\definecolor{currentstroke}{rgb}{1.000000,1.000000,1.000000}%
\pgfsetstrokecolor{currentstroke}%
\pgfsetdash{}{0pt}%
\pgfpathmoveto{\pgfqpoint{4.204146in}{1.608033in}}%
\pgfpathcurveto{\pgfqpoint{4.211279in}{1.608033in}}{\pgfqpoint{4.218121in}{1.610867in}}{\pgfqpoint{4.223164in}{1.615910in}}%
\pgfpathcurveto{\pgfqpoint{4.228208in}{1.620954in}}{\pgfqpoint{4.231042in}{1.627796in}}{\pgfqpoint{4.231042in}{1.634928in}}%
\pgfpathcurveto{\pgfqpoint{4.231042in}{1.642061in}}{\pgfqpoint{4.228208in}{1.648903in}}{\pgfqpoint{4.223164in}{1.653946in}}%
\pgfpathcurveto{\pgfqpoint{4.218121in}{1.658990in}}{\pgfqpoint{4.211279in}{1.661824in}}{\pgfqpoint{4.204146in}{1.661824in}}%
\pgfpathcurveto{\pgfqpoint{4.197013in}{1.661824in}}{\pgfqpoint{4.190172in}{1.658990in}}{\pgfqpoint{4.185128in}{1.653946in}}%
\pgfpathcurveto{\pgfqpoint{4.180084in}{1.648903in}}{\pgfqpoint{4.177250in}{1.642061in}}{\pgfqpoint{4.177250in}{1.634928in}}%
\pgfpathcurveto{\pgfqpoint{4.177250in}{1.627796in}}{\pgfqpoint{4.180084in}{1.620954in}}{\pgfqpoint{4.185128in}{1.615910in}}%
\pgfpathcurveto{\pgfqpoint{4.190172in}{1.610867in}}{\pgfqpoint{4.197013in}{1.608033in}}{\pgfqpoint{4.204146in}{1.608033in}}%
\pgfpathclose%
\pgfusepath{stroke,fill}%
\end{pgfscope}%
\begin{pgfscope}%
\pgfpathrectangle{\pgfqpoint{2.867647in}{0.500000in}}{\pgfqpoint{1.764706in}{1.700000in}}%
\pgfusepath{clip}%
\pgfsetbuttcap%
\pgfsetroundjoin%
\definecolor{currentfill}{rgb}{0.978376,0.897317,0.831308}%
\pgfsetfillcolor{currentfill}%
\pgfsetlinewidth{0.311001pt}%
\definecolor{currentstroke}{rgb}{1.000000,1.000000,1.000000}%
\pgfsetstrokecolor{currentstroke}%
\pgfsetdash{}{0pt}%
\pgfpathmoveto{\pgfqpoint{4.150554in}{1.073952in}}%
\pgfpathcurveto{\pgfqpoint{4.157687in}{1.073952in}}{\pgfqpoint{4.164529in}{1.076786in}}{\pgfqpoint{4.169572in}{1.081829in}}%
\pgfpathcurveto{\pgfqpoint{4.174616in}{1.086873in}}{\pgfqpoint{4.177450in}{1.093715in}}{\pgfqpoint{4.177450in}{1.100847in}}%
\pgfpathcurveto{\pgfqpoint{4.177450in}{1.107980in}}{\pgfqpoint{4.174616in}{1.114822in}}{\pgfqpoint{4.169572in}{1.119866in}}%
\pgfpathcurveto{\pgfqpoint{4.164529in}{1.124909in}}{\pgfqpoint{4.157687in}{1.127743in}}{\pgfqpoint{4.150554in}{1.127743in}}%
\pgfpathcurveto{\pgfqpoint{4.143422in}{1.127743in}}{\pgfqpoint{4.136580in}{1.124909in}}{\pgfqpoint{4.131536in}{1.119866in}}%
\pgfpathcurveto{\pgfqpoint{4.126493in}{1.114822in}}{\pgfqpoint{4.123659in}{1.107980in}}{\pgfqpoint{4.123659in}{1.100847in}}%
\pgfpathcurveto{\pgfqpoint{4.123659in}{1.093715in}}{\pgfqpoint{4.126493in}{1.086873in}}{\pgfqpoint{4.131536in}{1.081829in}}%
\pgfpathcurveto{\pgfqpoint{4.136580in}{1.076786in}}{\pgfqpoint{4.143422in}{1.073952in}}{\pgfqpoint{4.150554in}{1.073952in}}%
\pgfpathclose%
\pgfusepath{stroke,fill}%
\end{pgfscope}%
\begin{pgfscope}%
\pgfpathrectangle{\pgfqpoint{2.867647in}{0.500000in}}{\pgfqpoint{1.764706in}{1.700000in}}%
\pgfusepath{clip}%
\pgfsetbuttcap%
\pgfsetroundjoin%
\definecolor{currentfill}{rgb}{0.979124,0.903132,0.839793}%
\pgfsetfillcolor{currentfill}%
\pgfsetlinewidth{0.311001pt}%
\definecolor{currentstroke}{rgb}{1.000000,1.000000,1.000000}%
\pgfsetstrokecolor{currentstroke}%
\pgfsetdash{}{0pt}%
\pgfpathmoveto{\pgfqpoint{4.139860in}{1.091397in}}%
\pgfpathcurveto{\pgfqpoint{4.146992in}{1.091397in}}{\pgfqpoint{4.153834in}{1.094231in}}{\pgfqpoint{4.158878in}{1.099274in}}%
\pgfpathcurveto{\pgfqpoint{4.163921in}{1.104318in}}{\pgfqpoint{4.166755in}{1.111160in}}{\pgfqpoint{4.166755in}{1.118293in}}%
\pgfpathcurveto{\pgfqpoint{4.166755in}{1.125425in}}{\pgfqpoint{4.163921in}{1.132267in}}{\pgfqpoint{4.158878in}{1.137311in}}%
\pgfpathcurveto{\pgfqpoint{4.153834in}{1.142354in}}{\pgfqpoint{4.146992in}{1.145188in}}{\pgfqpoint{4.139860in}{1.145188in}}%
\pgfpathcurveto{\pgfqpoint{4.132727in}{1.145188in}}{\pgfqpoint{4.125885in}{1.142354in}}{\pgfqpoint{4.120842in}{1.137311in}}%
\pgfpathcurveto{\pgfqpoint{4.115798in}{1.132267in}}{\pgfqpoint{4.112964in}{1.125425in}}{\pgfqpoint{4.112964in}{1.118293in}}%
\pgfpathcurveto{\pgfqpoint{4.112964in}{1.111160in}}{\pgfqpoint{4.115798in}{1.104318in}}{\pgfqpoint{4.120842in}{1.099274in}}%
\pgfpathcurveto{\pgfqpoint{4.125885in}{1.094231in}}{\pgfqpoint{4.132727in}{1.091397in}}{\pgfqpoint{4.139860in}{1.091397in}}%
\pgfpathclose%
\pgfusepath{stroke,fill}%
\end{pgfscope}%
\begin{pgfscope}%
\pgfpathrectangle{\pgfqpoint{2.867647in}{0.500000in}}{\pgfqpoint{1.764706in}{1.700000in}}%
\pgfusepath{clip}%
\pgfsetbuttcap%
\pgfsetroundjoin%
\definecolor{currentfill}{rgb}{0.971202,0.827364,0.728520}%
\pgfsetfillcolor{currentfill}%
\pgfsetlinewidth{0.311001pt}%
\definecolor{currentstroke}{rgb}{1.000000,1.000000,1.000000}%
\pgfsetstrokecolor{currentstroke}%
\pgfsetdash{}{0pt}%
\pgfpathmoveto{\pgfqpoint{4.099804in}{1.214432in}}%
\pgfpathcurveto{\pgfqpoint{4.106937in}{1.214432in}}{\pgfqpoint{4.113779in}{1.217266in}}{\pgfqpoint{4.118822in}{1.222310in}}%
\pgfpathcurveto{\pgfqpoint{4.123866in}{1.227353in}}{\pgfqpoint{4.126700in}{1.234195in}}{\pgfqpoint{4.126700in}{1.241328in}}%
\pgfpathcurveto{\pgfqpoint{4.126700in}{1.248461in}}{\pgfqpoint{4.123866in}{1.255302in}}{\pgfqpoint{4.118822in}{1.260346in}}%
\pgfpathcurveto{\pgfqpoint{4.113779in}{1.265390in}}{\pgfqpoint{4.106937in}{1.268224in}}{\pgfqpoint{4.099804in}{1.268224in}}%
\pgfpathcurveto{\pgfqpoint{4.092671in}{1.268224in}}{\pgfqpoint{4.085830in}{1.265390in}}{\pgfqpoint{4.080786in}{1.260346in}}%
\pgfpathcurveto{\pgfqpoint{4.075742in}{1.255302in}}{\pgfqpoint{4.072908in}{1.248461in}}{\pgfqpoint{4.072908in}{1.241328in}}%
\pgfpathcurveto{\pgfqpoint{4.072908in}{1.234195in}}{\pgfqpoint{4.075742in}{1.227353in}}{\pgfqpoint{4.080786in}{1.222310in}}%
\pgfpathcurveto{\pgfqpoint{4.085830in}{1.217266in}}{\pgfqpoint{4.092671in}{1.214432in}}{\pgfqpoint{4.099804in}{1.214432in}}%
\pgfpathclose%
\pgfusepath{stroke,fill}%
\end{pgfscope}%
\begin{pgfscope}%
\pgfpathrectangle{\pgfqpoint{2.867647in}{0.500000in}}{\pgfqpoint{1.764706in}{1.700000in}}%
\pgfusepath{clip}%
\pgfsetbuttcap%
\pgfsetroundjoin%
\definecolor{currentfill}{rgb}{0.972726,0.844889,0.754401}%
\pgfsetfillcolor{currentfill}%
\pgfsetlinewidth{0.311001pt}%
\definecolor{currentstroke}{rgb}{1.000000,1.000000,1.000000}%
\pgfsetstrokecolor{currentstroke}%
\pgfsetdash{}{0pt}%
\pgfpathmoveto{\pgfqpoint{4.253967in}{1.405524in}}%
\pgfpathcurveto{\pgfqpoint{4.261100in}{1.405524in}}{\pgfqpoint{4.267942in}{1.408358in}}{\pgfqpoint{4.272986in}{1.413402in}}%
\pgfpathcurveto{\pgfqpoint{4.278029in}{1.418446in}}{\pgfqpoint{4.280863in}{1.425287in}}{\pgfqpoint{4.280863in}{1.432420in}}%
\pgfpathcurveto{\pgfqpoint{4.280863in}{1.439553in}}{\pgfqpoint{4.278029in}{1.446395in}}{\pgfqpoint{4.272986in}{1.451438in}}%
\pgfpathcurveto{\pgfqpoint{4.267942in}{1.456482in}}{\pgfqpoint{4.261100in}{1.459316in}}{\pgfqpoint{4.253967in}{1.459316in}}%
\pgfpathcurveto{\pgfqpoint{4.246835in}{1.459316in}}{\pgfqpoint{4.239993in}{1.456482in}}{\pgfqpoint{4.234949in}{1.451438in}}%
\pgfpathcurveto{\pgfqpoint{4.229906in}{1.446395in}}{\pgfqpoint{4.227072in}{1.439553in}}{\pgfqpoint{4.227072in}{1.432420in}}%
\pgfpathcurveto{\pgfqpoint{4.227072in}{1.425287in}}{\pgfqpoint{4.229906in}{1.418446in}}{\pgfqpoint{4.234949in}{1.413402in}}%
\pgfpathcurveto{\pgfqpoint{4.239993in}{1.408358in}}{\pgfqpoint{4.246835in}{1.405524in}}{\pgfqpoint{4.253967in}{1.405524in}}%
\pgfpathclose%
\pgfusepath{stroke,fill}%
\end{pgfscope}%
\begin{pgfscope}%
\pgfpathrectangle{\pgfqpoint{2.867647in}{0.500000in}}{\pgfqpoint{1.764706in}{1.700000in}}%
\pgfusepath{clip}%
\pgfsetbuttcap%
\pgfsetroundjoin%
\definecolor{currentfill}{rgb}{0.965169,0.707764,0.560659}%
\pgfsetfillcolor{currentfill}%
\pgfsetlinewidth{0.311001pt}%
\definecolor{currentstroke}{rgb}{1.000000,1.000000,1.000000}%
\pgfsetstrokecolor{currentstroke}%
\pgfsetdash{}{0pt}%
\pgfpathmoveto{\pgfqpoint{4.297220in}{1.437131in}}%
\pgfpathcurveto{\pgfqpoint{4.304353in}{1.437131in}}{\pgfqpoint{4.311194in}{1.439965in}}{\pgfqpoint{4.316238in}{1.445009in}}%
\pgfpathcurveto{\pgfqpoint{4.321282in}{1.450052in}}{\pgfqpoint{4.324116in}{1.456894in}}{\pgfqpoint{4.324116in}{1.464027in}}%
\pgfpathcurveto{\pgfqpoint{4.324116in}{1.471160in}}{\pgfqpoint{4.321282in}{1.478001in}}{\pgfqpoint{4.316238in}{1.483045in}}%
\pgfpathcurveto{\pgfqpoint{4.311194in}{1.488089in}}{\pgfqpoint{4.304353in}{1.490923in}}{\pgfqpoint{4.297220in}{1.490923in}}%
\pgfpathcurveto{\pgfqpoint{4.290087in}{1.490923in}}{\pgfqpoint{4.283245in}{1.488089in}}{\pgfqpoint{4.278202in}{1.483045in}}%
\pgfpathcurveto{\pgfqpoint{4.273158in}{1.478001in}}{\pgfqpoint{4.270324in}{1.471160in}}{\pgfqpoint{4.270324in}{1.464027in}}%
\pgfpathcurveto{\pgfqpoint{4.270324in}{1.456894in}}{\pgfqpoint{4.273158in}{1.450052in}}{\pgfqpoint{4.278202in}{1.445009in}}%
\pgfpathcurveto{\pgfqpoint{4.283245in}{1.439965in}}{\pgfqpoint{4.290087in}{1.437131in}}{\pgfqpoint{4.297220in}{1.437131in}}%
\pgfpathclose%
\pgfusepath{stroke,fill}%
\end{pgfscope}%
\begin{pgfscope}%
\pgfpathrectangle{\pgfqpoint{2.867647in}{0.500000in}}{\pgfqpoint{1.764706in}{1.700000in}}%
\pgfusepath{clip}%
\pgfsetbuttcap%
\pgfsetroundjoin%
\definecolor{currentfill}{rgb}{0.979124,0.903132,0.839793}%
\pgfsetfillcolor{currentfill}%
\pgfsetlinewidth{0.311001pt}%
\definecolor{currentstroke}{rgb}{1.000000,1.000000,1.000000}%
\pgfsetstrokecolor{currentstroke}%
\pgfsetdash{}{0pt}%
\pgfpathmoveto{\pgfqpoint{4.114523in}{1.570367in}}%
\pgfpathcurveto{\pgfqpoint{4.121656in}{1.570367in}}{\pgfqpoint{4.128498in}{1.573200in}}{\pgfqpoint{4.133541in}{1.578244in}}%
\pgfpathcurveto{\pgfqpoint{4.138585in}{1.583288in}}{\pgfqpoint{4.141419in}{1.590129in}}{\pgfqpoint{4.141419in}{1.597262in}}%
\pgfpathcurveto{\pgfqpoint{4.141419in}{1.604395in}}{\pgfqpoint{4.138585in}{1.611237in}}{\pgfqpoint{4.133541in}{1.616280in}}%
\pgfpathcurveto{\pgfqpoint{4.128498in}{1.621324in}}{\pgfqpoint{4.121656in}{1.624158in}}{\pgfqpoint{4.114523in}{1.624158in}}%
\pgfpathcurveto{\pgfqpoint{4.107390in}{1.624158in}}{\pgfqpoint{4.100549in}{1.621324in}}{\pgfqpoint{4.095505in}{1.616280in}}%
\pgfpathcurveto{\pgfqpoint{4.090461in}{1.611237in}}{\pgfqpoint{4.087628in}{1.604395in}}{\pgfqpoint{4.087628in}{1.597262in}}%
\pgfpathcurveto{\pgfqpoint{4.087628in}{1.590129in}}{\pgfqpoint{4.090461in}{1.583288in}}{\pgfqpoint{4.095505in}{1.578244in}}%
\pgfpathcurveto{\pgfqpoint{4.100549in}{1.573200in}}{\pgfqpoint{4.107390in}{1.570367in}}{\pgfqpoint{4.114523in}{1.570367in}}%
\pgfpathclose%
\pgfusepath{stroke,fill}%
\end{pgfscope}%
\begin{pgfscope}%
\pgfpathrectangle{\pgfqpoint{2.867647in}{0.500000in}}{\pgfqpoint{1.764706in}{1.700000in}}%
\pgfusepath{clip}%
\pgfsetbuttcap%
\pgfsetroundjoin%
\definecolor{currentfill}{rgb}{0.975018,0.868213,0.788710}%
\pgfsetfillcolor{currentfill}%
\pgfsetlinewidth{0.311001pt}%
\definecolor{currentstroke}{rgb}{1.000000,1.000000,1.000000}%
\pgfsetstrokecolor{currentstroke}%
\pgfsetdash{}{0pt}%
\pgfpathmoveto{\pgfqpoint{4.118894in}{1.216171in}}%
\pgfpathcurveto{\pgfqpoint{4.126026in}{1.216171in}}{\pgfqpoint{4.132868in}{1.219005in}}{\pgfqpoint{4.137912in}{1.224048in}}%
\pgfpathcurveto{\pgfqpoint{4.142955in}{1.229092in}}{\pgfqpoint{4.145789in}{1.235934in}}{\pgfqpoint{4.145789in}{1.243067in}}%
\pgfpathcurveto{\pgfqpoint{4.145789in}{1.250199in}}{\pgfqpoint{4.142955in}{1.257041in}}{\pgfqpoint{4.137912in}{1.262085in}}%
\pgfpathcurveto{\pgfqpoint{4.132868in}{1.267128in}}{\pgfqpoint{4.126026in}{1.269962in}}{\pgfqpoint{4.118894in}{1.269962in}}%
\pgfpathcurveto{\pgfqpoint{4.111761in}{1.269962in}}{\pgfqpoint{4.104919in}{1.267128in}}{\pgfqpoint{4.099875in}{1.262085in}}%
\pgfpathcurveto{\pgfqpoint{4.094832in}{1.257041in}}{\pgfqpoint{4.091998in}{1.250199in}}{\pgfqpoint{4.091998in}{1.243067in}}%
\pgfpathcurveto{\pgfqpoint{4.091998in}{1.235934in}}{\pgfqpoint{4.094832in}{1.229092in}}{\pgfqpoint{4.099875in}{1.224048in}}%
\pgfpathcurveto{\pgfqpoint{4.104919in}{1.219005in}}{\pgfqpoint{4.111761in}{1.216171in}}{\pgfqpoint{4.118894in}{1.216171in}}%
\pgfpathclose%
\pgfusepath{stroke,fill}%
\end{pgfscope}%
\begin{pgfscope}%
\pgfpathrectangle{\pgfqpoint{2.867647in}{0.500000in}}{\pgfqpoint{1.764706in}{1.700000in}}%
\pgfusepath{clip}%
\pgfsetbuttcap%
\pgfsetroundjoin%
\definecolor{currentfill}{rgb}{0.977657,0.891500,0.822809}%
\pgfsetfillcolor{currentfill}%
\pgfsetlinewidth{0.311001pt}%
\definecolor{currentstroke}{rgb}{1.000000,1.000000,1.000000}%
\pgfsetstrokecolor{currentstroke}%
\pgfsetdash{}{0pt}%
\pgfpathmoveto{\pgfqpoint{4.155120in}{1.593610in}}%
\pgfpathcurveto{\pgfqpoint{4.162253in}{1.593610in}}{\pgfqpoint{4.169094in}{1.596444in}}{\pgfqpoint{4.174138in}{1.601487in}}%
\pgfpathcurveto{\pgfqpoint{4.179182in}{1.606531in}}{\pgfqpoint{4.182016in}{1.613373in}}{\pgfqpoint{4.182016in}{1.620506in}}%
\pgfpathcurveto{\pgfqpoint{4.182016in}{1.627638in}}{\pgfqpoint{4.179182in}{1.634480in}}{\pgfqpoint{4.174138in}{1.639524in}}%
\pgfpathcurveto{\pgfqpoint{4.169094in}{1.644567in}}{\pgfqpoint{4.162253in}{1.647401in}}{\pgfqpoint{4.155120in}{1.647401in}}%
\pgfpathcurveto{\pgfqpoint{4.147987in}{1.647401in}}{\pgfqpoint{4.141145in}{1.644567in}}{\pgfqpoint{4.136102in}{1.639524in}}%
\pgfpathcurveto{\pgfqpoint{4.131058in}{1.634480in}}{\pgfqpoint{4.128224in}{1.627638in}}{\pgfqpoint{4.128224in}{1.620506in}}%
\pgfpathcurveto{\pgfqpoint{4.128224in}{1.613373in}}{\pgfqpoint{4.131058in}{1.606531in}}{\pgfqpoint{4.136102in}{1.601487in}}%
\pgfpathcurveto{\pgfqpoint{4.141145in}{1.596444in}}{\pgfqpoint{4.147987in}{1.593610in}}{\pgfqpoint{4.155120in}{1.593610in}}%
\pgfpathclose%
\pgfusepath{stroke,fill}%
\end{pgfscope}%
\begin{pgfscope}%
\pgfpathrectangle{\pgfqpoint{2.867647in}{0.500000in}}{\pgfqpoint{1.764706in}{1.700000in}}%
\pgfusepath{clip}%
\pgfsetbuttcap%
\pgfsetroundjoin%
\definecolor{currentfill}{rgb}{0.976287,0.879862,0.805788}%
\pgfsetfillcolor{currentfill}%
\pgfsetlinewidth{0.311001pt}%
\definecolor{currentstroke}{rgb}{1.000000,1.000000,1.000000}%
\pgfsetstrokecolor{currentstroke}%
\pgfsetdash{}{0pt}%
\pgfpathmoveto{\pgfqpoint{4.200056in}{1.102382in}}%
\pgfpathcurveto{\pgfqpoint{4.207189in}{1.102382in}}{\pgfqpoint{4.214030in}{1.105216in}}{\pgfqpoint{4.219074in}{1.110259in}}%
\pgfpathcurveto{\pgfqpoint{4.224118in}{1.115303in}}{\pgfqpoint{4.226952in}{1.122145in}}{\pgfqpoint{4.226952in}{1.129278in}}%
\pgfpathcurveto{\pgfqpoint{4.226952in}{1.136410in}}{\pgfqpoint{4.224118in}{1.143252in}}{\pgfqpoint{4.219074in}{1.148296in}}%
\pgfpathcurveto{\pgfqpoint{4.214030in}{1.153339in}}{\pgfqpoint{4.207189in}{1.156173in}}{\pgfqpoint{4.200056in}{1.156173in}}%
\pgfpathcurveto{\pgfqpoint{4.192923in}{1.156173in}}{\pgfqpoint{4.186081in}{1.153339in}}{\pgfqpoint{4.181038in}{1.148296in}}%
\pgfpathcurveto{\pgfqpoint{4.175994in}{1.143252in}}{\pgfqpoint{4.173160in}{1.136410in}}{\pgfqpoint{4.173160in}{1.129278in}}%
\pgfpathcurveto{\pgfqpoint{4.173160in}{1.122145in}}{\pgfqpoint{4.175994in}{1.115303in}}{\pgfqpoint{4.181038in}{1.110259in}}%
\pgfpathcurveto{\pgfqpoint{4.186081in}{1.105216in}}{\pgfqpoint{4.192923in}{1.102382in}}{\pgfqpoint{4.200056in}{1.102382in}}%
\pgfpathclose%
\pgfusepath{stroke,fill}%
\end{pgfscope}%
\begin{pgfscope}%
\pgfpathrectangle{\pgfqpoint{2.867647in}{0.500000in}}{\pgfqpoint{1.764706in}{1.700000in}}%
\pgfusepath{clip}%
\pgfsetbuttcap%
\pgfsetroundjoin%
\definecolor{currentfill}{rgb}{0.977657,0.891500,0.822809}%
\pgfsetfillcolor{currentfill}%
\pgfsetlinewidth{0.311001pt}%
\definecolor{currentstroke}{rgb}{1.000000,1.000000,1.000000}%
\pgfsetstrokecolor{currentstroke}%
\pgfsetdash{}{0pt}%
\pgfpathmoveto{\pgfqpoint{4.204207in}{1.130346in}}%
\pgfpathcurveto{\pgfqpoint{4.211340in}{1.130346in}}{\pgfqpoint{4.218182in}{1.133180in}}{\pgfqpoint{4.223225in}{1.138224in}}%
\pgfpathcurveto{\pgfqpoint{4.228269in}{1.143268in}}{\pgfqpoint{4.231103in}{1.150109in}}{\pgfqpoint{4.231103in}{1.157242in}}%
\pgfpathcurveto{\pgfqpoint{4.231103in}{1.164375in}}{\pgfqpoint{4.228269in}{1.171217in}}{\pgfqpoint{4.223225in}{1.176260in}}%
\pgfpathcurveto{\pgfqpoint{4.218182in}{1.181304in}}{\pgfqpoint{4.211340in}{1.184138in}}{\pgfqpoint{4.204207in}{1.184138in}}%
\pgfpathcurveto{\pgfqpoint{4.197074in}{1.184138in}}{\pgfqpoint{4.190233in}{1.181304in}}{\pgfqpoint{4.185189in}{1.176260in}}%
\pgfpathcurveto{\pgfqpoint{4.180145in}{1.171217in}}{\pgfqpoint{4.177311in}{1.164375in}}{\pgfqpoint{4.177311in}{1.157242in}}%
\pgfpathcurveto{\pgfqpoint{4.177311in}{1.150109in}}{\pgfqpoint{4.180145in}{1.143268in}}{\pgfqpoint{4.185189in}{1.138224in}}%
\pgfpathcurveto{\pgfqpoint{4.190233in}{1.133180in}}{\pgfqpoint{4.197074in}{1.130346in}}{\pgfqpoint{4.204207in}{1.130346in}}%
\pgfpathclose%
\pgfusepath{stroke,fill}%
\end{pgfscope}%
\begin{pgfscope}%
\pgfpathrectangle{\pgfqpoint{2.867647in}{0.500000in}}{\pgfqpoint{1.764706in}{1.700000in}}%
\pgfusepath{clip}%
\pgfsetbuttcap%
\pgfsetroundjoin%
\definecolor{currentfill}{rgb}{0.981377,0.920617,0.865369}%
\pgfsetfillcolor{currentfill}%
\pgfsetlinewidth{0.311001pt}%
\definecolor{currentstroke}{rgb}{1.000000,1.000000,1.000000}%
\pgfsetstrokecolor{currentstroke}%
\pgfsetdash{}{0pt}%
\pgfpathmoveto{\pgfqpoint{4.174626in}{1.207723in}}%
\pgfpathcurveto{\pgfqpoint{4.181759in}{1.207723in}}{\pgfqpoint{4.188600in}{1.210557in}}{\pgfqpoint{4.193644in}{1.215600in}}%
\pgfpathcurveto{\pgfqpoint{4.198688in}{1.220644in}}{\pgfqpoint{4.201522in}{1.227486in}}{\pgfqpoint{4.201522in}{1.234618in}}%
\pgfpathcurveto{\pgfqpoint{4.201522in}{1.241751in}}{\pgfqpoint{4.198688in}{1.248593in}}{\pgfqpoint{4.193644in}{1.253637in}}%
\pgfpathcurveto{\pgfqpoint{4.188600in}{1.258680in}}{\pgfqpoint{4.181759in}{1.261514in}}{\pgfqpoint{4.174626in}{1.261514in}}%
\pgfpathcurveto{\pgfqpoint{4.167493in}{1.261514in}}{\pgfqpoint{4.160651in}{1.258680in}}{\pgfqpoint{4.155608in}{1.253637in}}%
\pgfpathcurveto{\pgfqpoint{4.150564in}{1.248593in}}{\pgfqpoint{4.147730in}{1.241751in}}{\pgfqpoint{4.147730in}{1.234618in}}%
\pgfpathcurveto{\pgfqpoint{4.147730in}{1.227486in}}{\pgfqpoint{4.150564in}{1.220644in}}{\pgfqpoint{4.155608in}{1.215600in}}%
\pgfpathcurveto{\pgfqpoint{4.160651in}{1.210557in}}{\pgfqpoint{4.167493in}{1.207723in}}{\pgfqpoint{4.174626in}{1.207723in}}%
\pgfpathclose%
\pgfusepath{stroke,fill}%
\end{pgfscope}%
\begin{pgfscope}%
\pgfpathrectangle{\pgfqpoint{2.867647in}{0.500000in}}{\pgfqpoint{1.764706in}{1.700000in}}%
\pgfusepath{clip}%
\pgfsetbuttcap%
\pgfsetroundjoin%
\definecolor{currentfill}{rgb}{0.964306,0.663930,0.507747}%
\pgfsetfillcolor{currentfill}%
\pgfsetlinewidth{0.311001pt}%
\definecolor{currentstroke}{rgb}{1.000000,1.000000,1.000000}%
\pgfsetstrokecolor{currentstroke}%
\pgfsetdash{}{0pt}%
\pgfpathmoveto{\pgfqpoint{4.010374in}{1.096874in}}%
\pgfpathcurveto{\pgfqpoint{4.017507in}{1.096874in}}{\pgfqpoint{4.024349in}{1.099708in}}{\pgfqpoint{4.029392in}{1.104752in}}%
\pgfpathcurveto{\pgfqpoint{4.034436in}{1.109796in}}{\pgfqpoint{4.037270in}{1.116637in}}{\pgfqpoint{4.037270in}{1.123770in}}%
\pgfpathcurveto{\pgfqpoint{4.037270in}{1.130903in}}{\pgfqpoint{4.034436in}{1.137745in}}{\pgfqpoint{4.029392in}{1.142788in}}%
\pgfpathcurveto{\pgfqpoint{4.024349in}{1.147832in}}{\pgfqpoint{4.017507in}{1.150666in}}{\pgfqpoint{4.010374in}{1.150666in}}%
\pgfpathcurveto{\pgfqpoint{4.003241in}{1.150666in}}{\pgfqpoint{3.996400in}{1.147832in}}{\pgfqpoint{3.991356in}{1.142788in}}%
\pgfpathcurveto{\pgfqpoint{3.986312in}{1.137745in}}{\pgfqpoint{3.983478in}{1.130903in}}{\pgfqpoint{3.983478in}{1.123770in}}%
\pgfpathcurveto{\pgfqpoint{3.983478in}{1.116637in}}{\pgfqpoint{3.986312in}{1.109796in}}{\pgfqpoint{3.991356in}{1.104752in}}%
\pgfpathcurveto{\pgfqpoint{3.996400in}{1.099708in}}{\pgfqpoint{4.003241in}{1.096874in}}{\pgfqpoint{4.010374in}{1.096874in}}%
\pgfpathclose%
\pgfusepath{stroke,fill}%
\end{pgfscope}%
\begin{pgfscope}%
\pgfpathrectangle{\pgfqpoint{2.867647in}{0.500000in}}{\pgfqpoint{1.764706in}{1.700000in}}%
\pgfusepath{clip}%
\pgfsetbuttcap%
\pgfsetroundjoin%
\definecolor{currentfill}{rgb}{0.971694,0.833208,0.737161}%
\pgfsetfillcolor{currentfill}%
\pgfsetlinewidth{0.311001pt}%
\definecolor{currentstroke}{rgb}{1.000000,1.000000,1.000000}%
\pgfsetstrokecolor{currentstroke}%
\pgfsetdash{}{0pt}%
\pgfpathmoveto{\pgfqpoint{4.092266in}{1.465770in}}%
\pgfpathcurveto{\pgfqpoint{4.099399in}{1.465770in}}{\pgfqpoint{4.106240in}{1.468604in}}{\pgfqpoint{4.111284in}{1.473648in}}%
\pgfpathcurveto{\pgfqpoint{4.116328in}{1.478691in}}{\pgfqpoint{4.119161in}{1.485533in}}{\pgfqpoint{4.119161in}{1.492666in}}%
\pgfpathcurveto{\pgfqpoint{4.119161in}{1.499799in}}{\pgfqpoint{4.116328in}{1.506640in}}{\pgfqpoint{4.111284in}{1.511684in}}%
\pgfpathcurveto{\pgfqpoint{4.106240in}{1.516728in}}{\pgfqpoint{4.099399in}{1.519561in}}{\pgfqpoint{4.092266in}{1.519561in}}%
\pgfpathcurveto{\pgfqpoint{4.085133in}{1.519561in}}{\pgfqpoint{4.078291in}{1.516728in}}{\pgfqpoint{4.073248in}{1.511684in}}%
\pgfpathcurveto{\pgfqpoint{4.068204in}{1.506640in}}{\pgfqpoint{4.065370in}{1.499799in}}{\pgfqpoint{4.065370in}{1.492666in}}%
\pgfpathcurveto{\pgfqpoint{4.065370in}{1.485533in}}{\pgfqpoint{4.068204in}{1.478691in}}{\pgfqpoint{4.073248in}{1.473648in}}%
\pgfpathcurveto{\pgfqpoint{4.078291in}{1.468604in}}{\pgfqpoint{4.085133in}{1.465770in}}{\pgfqpoint{4.092266in}{1.465770in}}%
\pgfpathclose%
\pgfusepath{stroke,fill}%
\end{pgfscope}%
\begin{pgfscope}%
\pgfpathrectangle{\pgfqpoint{2.867647in}{0.500000in}}{\pgfqpoint{1.764706in}{1.700000in}}%
\pgfusepath{clip}%
\pgfsetbuttcap%
\pgfsetroundjoin%
\definecolor{currentfill}{rgb}{0.963884,0.644842,0.486120}%
\pgfsetfillcolor{currentfill}%
\pgfsetlinewidth{0.311001pt}%
\definecolor{currentstroke}{rgb}{1.000000,1.000000,1.000000}%
\pgfsetstrokecolor{currentstroke}%
\pgfsetdash{}{0pt}%
\pgfpathmoveto{\pgfqpoint{4.014342in}{1.119976in}}%
\pgfpathcurveto{\pgfqpoint{4.021474in}{1.119976in}}{\pgfqpoint{4.028316in}{1.122810in}}{\pgfqpoint{4.033360in}{1.127854in}}%
\pgfpathcurveto{\pgfqpoint{4.038403in}{1.132898in}}{\pgfqpoint{4.041237in}{1.139739in}}{\pgfqpoint{4.041237in}{1.146872in}}%
\pgfpathcurveto{\pgfqpoint{4.041237in}{1.154005in}}{\pgfqpoint{4.038403in}{1.160847in}}{\pgfqpoint{4.033360in}{1.165890in}}%
\pgfpathcurveto{\pgfqpoint{4.028316in}{1.170934in}}{\pgfqpoint{4.021474in}{1.173768in}}{\pgfqpoint{4.014342in}{1.173768in}}%
\pgfpathcurveto{\pgfqpoint{4.007209in}{1.173768in}}{\pgfqpoint{4.000367in}{1.170934in}}{\pgfqpoint{3.995324in}{1.165890in}}%
\pgfpathcurveto{\pgfqpoint{3.990280in}{1.160847in}}{\pgfqpoint{3.987446in}{1.154005in}}{\pgfqpoint{3.987446in}{1.146872in}}%
\pgfpathcurveto{\pgfqpoint{3.987446in}{1.139739in}}{\pgfqpoint{3.990280in}{1.132898in}}{\pgfqpoint{3.995324in}{1.127854in}}%
\pgfpathcurveto{\pgfqpoint{4.000367in}{1.122810in}}{\pgfqpoint{4.007209in}{1.119976in}}{\pgfqpoint{4.014342in}{1.119976in}}%
\pgfpathclose%
\pgfusepath{stroke,fill}%
\end{pgfscope}%
\begin{pgfscope}%
\pgfpathrectangle{\pgfqpoint{2.867647in}{0.500000in}}{\pgfqpoint{1.764706in}{1.700000in}}%
\pgfusepath{clip}%
\pgfsetbuttcap%
\pgfsetroundjoin%
\definecolor{currentfill}{rgb}{0.963884,0.644842,0.486120}%
\pgfsetfillcolor{currentfill}%
\pgfsetlinewidth{0.311001pt}%
\definecolor{currentstroke}{rgb}{1.000000,1.000000,1.000000}%
\pgfsetstrokecolor{currentstroke}%
\pgfsetdash{}{0pt}%
\pgfpathmoveto{\pgfqpoint{4.061411in}{1.324269in}}%
\pgfpathcurveto{\pgfqpoint{4.068544in}{1.324269in}}{\pgfqpoint{4.075386in}{1.327102in}}{\pgfqpoint{4.080430in}{1.332146in}}%
\pgfpathcurveto{\pgfqpoint{4.085473in}{1.337190in}}{\pgfqpoint{4.088307in}{1.344031in}}{\pgfqpoint{4.088307in}{1.351164in}}%
\pgfpathcurveto{\pgfqpoint{4.088307in}{1.358297in}}{\pgfqpoint{4.085473in}{1.365139in}}{\pgfqpoint{4.080430in}{1.370182in}}%
\pgfpathcurveto{\pgfqpoint{4.075386in}{1.375226in}}{\pgfqpoint{4.068544in}{1.378060in}}{\pgfqpoint{4.061411in}{1.378060in}}%
\pgfpathcurveto{\pgfqpoint{4.054279in}{1.378060in}}{\pgfqpoint{4.047437in}{1.375226in}}{\pgfqpoint{4.042393in}{1.370182in}}%
\pgfpathcurveto{\pgfqpoint{4.037350in}{1.365139in}}{\pgfqpoint{4.034516in}{1.358297in}}{\pgfqpoint{4.034516in}{1.351164in}}%
\pgfpathcurveto{\pgfqpoint{4.034516in}{1.344031in}}{\pgfqpoint{4.037350in}{1.337190in}}{\pgfqpoint{4.042393in}{1.332146in}}%
\pgfpathcurveto{\pgfqpoint{4.047437in}{1.327102in}}{\pgfqpoint{4.054279in}{1.324269in}}{\pgfqpoint{4.061411in}{1.324269in}}%
\pgfpathclose%
\pgfusepath{stroke,fill}%
\end{pgfscope}%
\begin{pgfscope}%
\pgfpathrectangle{\pgfqpoint{2.867647in}{0.500000in}}{\pgfqpoint{1.764706in}{1.700000in}}%
\pgfusepath{clip}%
\pgfsetbuttcap%
\pgfsetroundjoin%
\definecolor{currentfill}{rgb}{0.944085,0.383081,0.267220}%
\pgfsetfillcolor{currentfill}%
\pgfsetlinewidth{0.311001pt}%
\definecolor{currentstroke}{rgb}{1.000000,1.000000,1.000000}%
\pgfsetstrokecolor{currentstroke}%
\pgfsetdash{}{0pt}%
\pgfpathmoveto{\pgfqpoint{4.305990in}{1.019293in}}%
\pgfpathcurveto{\pgfqpoint{4.313123in}{1.019293in}}{\pgfqpoint{4.319964in}{1.022127in}}{\pgfqpoint{4.325008in}{1.027171in}}%
\pgfpathcurveto{\pgfqpoint{4.330052in}{1.032214in}}{\pgfqpoint{4.332885in}{1.039056in}}{\pgfqpoint{4.332885in}{1.046189in}}%
\pgfpathcurveto{\pgfqpoint{4.332885in}{1.053322in}}{\pgfqpoint{4.330052in}{1.060163in}}{\pgfqpoint{4.325008in}{1.065207in}}%
\pgfpathcurveto{\pgfqpoint{4.319964in}{1.070251in}}{\pgfqpoint{4.313123in}{1.073085in}}{\pgfqpoint{4.305990in}{1.073085in}}%
\pgfpathcurveto{\pgfqpoint{4.298857in}{1.073085in}}{\pgfqpoint{4.292015in}{1.070251in}}{\pgfqpoint{4.286972in}{1.065207in}}%
\pgfpathcurveto{\pgfqpoint{4.281928in}{1.060163in}}{\pgfqpoint{4.279094in}{1.053322in}}{\pgfqpoint{4.279094in}{1.046189in}}%
\pgfpathcurveto{\pgfqpoint{4.279094in}{1.039056in}}{\pgfqpoint{4.281928in}{1.032214in}}{\pgfqpoint{4.286972in}{1.027171in}}%
\pgfpathcurveto{\pgfqpoint{4.292015in}{1.022127in}}{\pgfqpoint{4.298857in}{1.019293in}}{\pgfqpoint{4.305990in}{1.019293in}}%
\pgfpathclose%
\pgfusepath{stroke,fill}%
\end{pgfscope}%
\begin{pgfscope}%
\pgfpathrectangle{\pgfqpoint{2.867647in}{0.500000in}}{\pgfqpoint{1.764706in}{1.700000in}}%
\pgfusepath{clip}%
\pgfsetbuttcap%
\pgfsetroundjoin%
\definecolor{currentfill}{rgb}{0.977657,0.891500,0.822809}%
\pgfsetfillcolor{currentfill}%
\pgfsetlinewidth{0.311001pt}%
\definecolor{currentstroke}{rgb}{1.000000,1.000000,1.000000}%
\pgfsetstrokecolor{currentstroke}%
\pgfsetdash{}{0pt}%
\pgfpathmoveto{\pgfqpoint{4.134945in}{1.221755in}}%
\pgfpathcurveto{\pgfqpoint{4.142078in}{1.221755in}}{\pgfqpoint{4.148919in}{1.224589in}}{\pgfqpoint{4.153963in}{1.229633in}}%
\pgfpathcurveto{\pgfqpoint{4.159007in}{1.234676in}}{\pgfqpoint{4.161841in}{1.241518in}}{\pgfqpoint{4.161841in}{1.248651in}}%
\pgfpathcurveto{\pgfqpoint{4.161841in}{1.255784in}}{\pgfqpoint{4.159007in}{1.262625in}}{\pgfqpoint{4.153963in}{1.267669in}}%
\pgfpathcurveto{\pgfqpoint{4.148919in}{1.272712in}}{\pgfqpoint{4.142078in}{1.275546in}}{\pgfqpoint{4.134945in}{1.275546in}}%
\pgfpathcurveto{\pgfqpoint{4.127812in}{1.275546in}}{\pgfqpoint{4.120970in}{1.272712in}}{\pgfqpoint{4.115927in}{1.267669in}}%
\pgfpathcurveto{\pgfqpoint{4.110883in}{1.262625in}}{\pgfqpoint{4.108049in}{1.255784in}}{\pgfqpoint{4.108049in}{1.248651in}}%
\pgfpathcurveto{\pgfqpoint{4.108049in}{1.241518in}}{\pgfqpoint{4.110883in}{1.234676in}}{\pgfqpoint{4.115927in}{1.229633in}}%
\pgfpathcurveto{\pgfqpoint{4.120970in}{1.224589in}}{\pgfqpoint{4.127812in}{1.221755in}}{\pgfqpoint{4.134945in}{1.221755in}}%
\pgfpathclose%
\pgfusepath{stroke,fill}%
\end{pgfscope}%
\begin{pgfscope}%
\pgfpathrectangle{\pgfqpoint{2.867647in}{0.500000in}}{\pgfqpoint{1.764706in}{1.700000in}}%
\pgfusepath{clip}%
\pgfsetbuttcap%
\pgfsetroundjoin%
\definecolor{currentfill}{rgb}{0.978376,0.897317,0.831308}%
\pgfsetfillcolor{currentfill}%
\pgfsetlinewidth{0.311001pt}%
\definecolor{currentstroke}{rgb}{1.000000,1.000000,1.000000}%
\pgfsetstrokecolor{currentstroke}%
\pgfsetdash{}{0pt}%
\pgfpathmoveto{\pgfqpoint{4.138166in}{1.069198in}}%
\pgfpathcurveto{\pgfqpoint{4.145299in}{1.069198in}}{\pgfqpoint{4.152140in}{1.072032in}}{\pgfqpoint{4.157184in}{1.077075in}}%
\pgfpathcurveto{\pgfqpoint{4.162228in}{1.082119in}}{\pgfqpoint{4.165062in}{1.088961in}}{\pgfqpoint{4.165062in}{1.096093in}}%
\pgfpathcurveto{\pgfqpoint{4.165062in}{1.103226in}}{\pgfqpoint{4.162228in}{1.110068in}}{\pgfqpoint{4.157184in}{1.115112in}}%
\pgfpathcurveto{\pgfqpoint{4.152140in}{1.120155in}}{\pgfqpoint{4.145299in}{1.122989in}}{\pgfqpoint{4.138166in}{1.122989in}}%
\pgfpathcurveto{\pgfqpoint{4.131033in}{1.122989in}}{\pgfqpoint{4.124191in}{1.120155in}}{\pgfqpoint{4.119148in}{1.115112in}}%
\pgfpathcurveto{\pgfqpoint{4.114104in}{1.110068in}}{\pgfqpoint{4.111270in}{1.103226in}}{\pgfqpoint{4.111270in}{1.096093in}}%
\pgfpathcurveto{\pgfqpoint{4.111270in}{1.088961in}}{\pgfqpoint{4.114104in}{1.082119in}}{\pgfqpoint{4.119148in}{1.077075in}}%
\pgfpathcurveto{\pgfqpoint{4.124191in}{1.072032in}}{\pgfqpoint{4.131033in}{1.069198in}}{\pgfqpoint{4.138166in}{1.069198in}}%
\pgfpathclose%
\pgfusepath{stroke,fill}%
\end{pgfscope}%
\begin{pgfscope}%
\pgfpathrectangle{\pgfqpoint{2.867647in}{0.500000in}}{\pgfqpoint{1.764706in}{1.700000in}}%
\pgfusepath{clip}%
\pgfsetbuttcap%
\pgfsetroundjoin%
\definecolor{currentfill}{rgb}{0.930781,0.313740,0.244688}%
\pgfsetfillcolor{currentfill}%
\pgfsetlinewidth{0.311001pt}%
\definecolor{currentstroke}{rgb}{1.000000,1.000000,1.000000}%
\pgfsetstrokecolor{currentstroke}%
\pgfsetdash{}{0pt}%
\pgfpathmoveto{\pgfqpoint{4.182439in}{1.802346in}}%
\pgfpathcurveto{\pgfqpoint{4.189572in}{1.802346in}}{\pgfqpoint{4.196413in}{1.805180in}}{\pgfqpoint{4.201457in}{1.810224in}}%
\pgfpathcurveto{\pgfqpoint{4.206501in}{1.815267in}}{\pgfqpoint{4.209335in}{1.822109in}}{\pgfqpoint{4.209335in}{1.829242in}}%
\pgfpathcurveto{\pgfqpoint{4.209335in}{1.836375in}}{\pgfqpoint{4.206501in}{1.843216in}}{\pgfqpoint{4.201457in}{1.848260in}}%
\pgfpathcurveto{\pgfqpoint{4.196413in}{1.853303in}}{\pgfqpoint{4.189572in}{1.856137in}}{\pgfqpoint{4.182439in}{1.856137in}}%
\pgfpathcurveto{\pgfqpoint{4.175306in}{1.856137in}}{\pgfqpoint{4.168464in}{1.853303in}}{\pgfqpoint{4.163421in}{1.848260in}}%
\pgfpathcurveto{\pgfqpoint{4.158377in}{1.843216in}}{\pgfqpoint{4.155543in}{1.836375in}}{\pgfqpoint{4.155543in}{1.829242in}}%
\pgfpathcurveto{\pgfqpoint{4.155543in}{1.822109in}}{\pgfqpoint{4.158377in}{1.815267in}}{\pgfqpoint{4.163421in}{1.810224in}}%
\pgfpathcurveto{\pgfqpoint{4.168464in}{1.805180in}}{\pgfqpoint{4.175306in}{1.802346in}}{\pgfqpoint{4.182439in}{1.802346in}}%
\pgfpathclose%
\pgfusepath{stroke,fill}%
\end{pgfscope}%
\begin{pgfscope}%
\pgfpathrectangle{\pgfqpoint{2.867647in}{0.500000in}}{\pgfqpoint{1.764706in}{1.700000in}}%
\pgfusepath{clip}%
\pgfsetbuttcap%
\pgfsetroundjoin%
\definecolor{currentfill}{rgb}{0.977657,0.891500,0.822809}%
\pgfsetfillcolor{currentfill}%
\pgfsetlinewidth{0.311001pt}%
\definecolor{currentstroke}{rgb}{1.000000,1.000000,1.000000}%
\pgfsetstrokecolor{currentstroke}%
\pgfsetdash{}{0pt}%
\pgfpathmoveto{\pgfqpoint{4.118283in}{1.125998in}}%
\pgfpathcurveto{\pgfqpoint{4.125416in}{1.125998in}}{\pgfqpoint{4.132257in}{1.128832in}}{\pgfqpoint{4.137301in}{1.133876in}}%
\pgfpathcurveto{\pgfqpoint{4.142345in}{1.138920in}}{\pgfqpoint{4.145179in}{1.145761in}}{\pgfqpoint{4.145179in}{1.152894in}}%
\pgfpathcurveto{\pgfqpoint{4.145179in}{1.160027in}}{\pgfqpoint{4.142345in}{1.166869in}}{\pgfqpoint{4.137301in}{1.171912in}}%
\pgfpathcurveto{\pgfqpoint{4.132257in}{1.176956in}}{\pgfqpoint{4.125416in}{1.179790in}}{\pgfqpoint{4.118283in}{1.179790in}}%
\pgfpathcurveto{\pgfqpoint{4.111150in}{1.179790in}}{\pgfqpoint{4.104308in}{1.176956in}}{\pgfqpoint{4.099265in}{1.171912in}}%
\pgfpathcurveto{\pgfqpoint{4.094221in}{1.166869in}}{\pgfqpoint{4.091387in}{1.160027in}}{\pgfqpoint{4.091387in}{1.152894in}}%
\pgfpathcurveto{\pgfqpoint{4.091387in}{1.145761in}}{\pgfqpoint{4.094221in}{1.138920in}}{\pgfqpoint{4.099265in}{1.133876in}}%
\pgfpathcurveto{\pgfqpoint{4.104308in}{1.128832in}}{\pgfqpoint{4.111150in}{1.125998in}}{\pgfqpoint{4.118283in}{1.125998in}}%
\pgfpathclose%
\pgfusepath{stroke,fill}%
\end{pgfscope}%
\begin{pgfscope}%
\pgfpathrectangle{\pgfqpoint{2.867647in}{0.500000in}}{\pgfqpoint{1.764706in}{1.700000in}}%
\pgfusepath{clip}%
\pgfsetbuttcap%
\pgfsetroundjoin%
\definecolor{currentfill}{rgb}{0.976287,0.879862,0.805788}%
\pgfsetfillcolor{currentfill}%
\pgfsetlinewidth{0.311001pt}%
\definecolor{currentstroke}{rgb}{1.000000,1.000000,1.000000}%
\pgfsetstrokecolor{currentstroke}%
\pgfsetdash{}{0pt}%
\pgfpathmoveto{\pgfqpoint{4.223277in}{1.155645in}}%
\pgfpathcurveto{\pgfqpoint{4.230410in}{1.155645in}}{\pgfqpoint{4.237251in}{1.158478in}}{\pgfqpoint{4.242295in}{1.163522in}}%
\pgfpathcurveto{\pgfqpoint{4.247339in}{1.168566in}}{\pgfqpoint{4.250173in}{1.175407in}}{\pgfqpoint{4.250173in}{1.182540in}}%
\pgfpathcurveto{\pgfqpoint{4.250173in}{1.189673in}}{\pgfqpoint{4.247339in}{1.196515in}}{\pgfqpoint{4.242295in}{1.201558in}}%
\pgfpathcurveto{\pgfqpoint{4.237251in}{1.206602in}}{\pgfqpoint{4.230410in}{1.209436in}}{\pgfqpoint{4.223277in}{1.209436in}}%
\pgfpathcurveto{\pgfqpoint{4.216144in}{1.209436in}}{\pgfqpoint{4.209302in}{1.206602in}}{\pgfqpoint{4.204259in}{1.201558in}}%
\pgfpathcurveto{\pgfqpoint{4.199215in}{1.196515in}}{\pgfqpoint{4.196381in}{1.189673in}}{\pgfqpoint{4.196381in}{1.182540in}}%
\pgfpathcurveto{\pgfqpoint{4.196381in}{1.175407in}}{\pgfqpoint{4.199215in}{1.168566in}}{\pgfqpoint{4.204259in}{1.163522in}}%
\pgfpathcurveto{\pgfqpoint{4.209302in}{1.158478in}}{\pgfqpoint{4.216144in}{1.155645in}}{\pgfqpoint{4.223277in}{1.155645in}}%
\pgfpathclose%
\pgfusepath{stroke,fill}%
\end{pgfscope}%
\begin{pgfscope}%
\pgfpathrectangle{\pgfqpoint{2.867647in}{0.500000in}}{\pgfqpoint{1.764706in}{1.700000in}}%
\pgfusepath{clip}%
\pgfsetbuttcap%
\pgfsetroundjoin%
\definecolor{currentfill}{rgb}{0.962532,0.599594,0.438051}%
\pgfsetfillcolor{currentfill}%
\pgfsetlinewidth{0.311001pt}%
\definecolor{currentstroke}{rgb}{1.000000,1.000000,1.000000}%
\pgfsetstrokecolor{currentstroke}%
\pgfsetdash{}{0pt}%
\pgfpathmoveto{\pgfqpoint{4.049582in}{0.848812in}}%
\pgfpathcurveto{\pgfqpoint{4.056715in}{0.848812in}}{\pgfqpoint{4.063557in}{0.851646in}}{\pgfqpoint{4.068601in}{0.856690in}}%
\pgfpathcurveto{\pgfqpoint{4.073644in}{0.861734in}}{\pgfqpoint{4.076478in}{0.868575in}}{\pgfqpoint{4.076478in}{0.875708in}}%
\pgfpathcurveto{\pgfqpoint{4.076478in}{0.882841in}}{\pgfqpoint{4.073644in}{0.889683in}}{\pgfqpoint{4.068601in}{0.894726in}}%
\pgfpathcurveto{\pgfqpoint{4.063557in}{0.899770in}}{\pgfqpoint{4.056715in}{0.902604in}}{\pgfqpoint{4.049582in}{0.902604in}}%
\pgfpathcurveto{\pgfqpoint{4.042450in}{0.902604in}}{\pgfqpoint{4.035608in}{0.899770in}}{\pgfqpoint{4.030564in}{0.894726in}}%
\pgfpathcurveto{\pgfqpoint{4.025521in}{0.889683in}}{\pgfqpoint{4.022687in}{0.882841in}}{\pgfqpoint{4.022687in}{0.875708in}}%
\pgfpathcurveto{\pgfqpoint{4.022687in}{0.868575in}}{\pgfqpoint{4.025521in}{0.861734in}}{\pgfqpoint{4.030564in}{0.856690in}}%
\pgfpathcurveto{\pgfqpoint{4.035608in}{0.851646in}}{\pgfqpoint{4.042450in}{0.848812in}}{\pgfqpoint{4.049582in}{0.848812in}}%
\pgfpathclose%
\pgfusepath{stroke,fill}%
\end{pgfscope}%
\begin{pgfscope}%
\pgfpathrectangle{\pgfqpoint{2.867647in}{0.500000in}}{\pgfqpoint{1.764706in}{1.700000in}}%
\pgfusepath{clip}%
\pgfsetbuttcap%
\pgfsetroundjoin%
\definecolor{currentfill}{rgb}{0.977657,0.891500,0.822809}%
\pgfsetfillcolor{currentfill}%
\pgfsetlinewidth{0.311001pt}%
\definecolor{currentstroke}{rgb}{1.000000,1.000000,1.000000}%
\pgfsetstrokecolor{currentstroke}%
\pgfsetdash{}{0pt}%
\pgfpathmoveto{\pgfqpoint{4.110726in}{1.518433in}}%
\pgfpathcurveto{\pgfqpoint{4.117859in}{1.518433in}}{\pgfqpoint{4.124701in}{1.521267in}}{\pgfqpoint{4.129744in}{1.526311in}}%
\pgfpathcurveto{\pgfqpoint{4.134788in}{1.531354in}}{\pgfqpoint{4.137622in}{1.538196in}}{\pgfqpoint{4.137622in}{1.545329in}}%
\pgfpathcurveto{\pgfqpoint{4.137622in}{1.552462in}}{\pgfqpoint{4.134788in}{1.559303in}}{\pgfqpoint{4.129744in}{1.564347in}}%
\pgfpathcurveto{\pgfqpoint{4.124701in}{1.569391in}}{\pgfqpoint{4.117859in}{1.572225in}}{\pgfqpoint{4.110726in}{1.572225in}}%
\pgfpathcurveto{\pgfqpoint{4.103593in}{1.572225in}}{\pgfqpoint{4.096752in}{1.569391in}}{\pgfqpoint{4.091708in}{1.564347in}}%
\pgfpathcurveto{\pgfqpoint{4.086665in}{1.559303in}}{\pgfqpoint{4.083831in}{1.552462in}}{\pgfqpoint{4.083831in}{1.545329in}}%
\pgfpathcurveto{\pgfqpoint{4.083831in}{1.538196in}}{\pgfqpoint{4.086665in}{1.531354in}}{\pgfqpoint{4.091708in}{1.526311in}}%
\pgfpathcurveto{\pgfqpoint{4.096752in}{1.521267in}}{\pgfqpoint{4.103593in}{1.518433in}}{\pgfqpoint{4.110726in}{1.518433in}}%
\pgfpathclose%
\pgfusepath{stroke,fill}%
\end{pgfscope}%
\begin{pgfscope}%
\pgfpathrectangle{\pgfqpoint{2.867647in}{0.500000in}}{\pgfqpoint{1.764706in}{1.700000in}}%
\pgfusepath{clip}%
\pgfsetbuttcap%
\pgfsetroundjoin%
\definecolor{currentfill}{rgb}{0.969803,0.809811,0.702523}%
\pgfsetfillcolor{currentfill}%
\pgfsetlinewidth{0.311001pt}%
\definecolor{currentstroke}{rgb}{1.000000,1.000000,1.000000}%
\pgfsetstrokecolor{currentstroke}%
\pgfsetdash{}{0pt}%
\pgfpathmoveto{\pgfqpoint{4.083869in}{1.459075in}}%
\pgfpathcurveto{\pgfqpoint{4.091002in}{1.459075in}}{\pgfqpoint{4.097844in}{1.461909in}}{\pgfqpoint{4.102888in}{1.466953in}}%
\pgfpathcurveto{\pgfqpoint{4.107931in}{1.471996in}}{\pgfqpoint{4.110765in}{1.478838in}}{\pgfqpoint{4.110765in}{1.485971in}}%
\pgfpathcurveto{\pgfqpoint{4.110765in}{1.493104in}}{\pgfqpoint{4.107931in}{1.499945in}}{\pgfqpoint{4.102888in}{1.504989in}}%
\pgfpathcurveto{\pgfqpoint{4.097844in}{1.510033in}}{\pgfqpoint{4.091002in}{1.512867in}}{\pgfqpoint{4.083869in}{1.512867in}}%
\pgfpathcurveto{\pgfqpoint{4.076737in}{1.512867in}}{\pgfqpoint{4.069895in}{1.510033in}}{\pgfqpoint{4.064851in}{1.504989in}}%
\pgfpathcurveto{\pgfqpoint{4.059808in}{1.499945in}}{\pgfqpoint{4.056974in}{1.493104in}}{\pgfqpoint{4.056974in}{1.485971in}}%
\pgfpathcurveto{\pgfqpoint{4.056974in}{1.478838in}}{\pgfqpoint{4.059808in}{1.471996in}}{\pgfqpoint{4.064851in}{1.466953in}}%
\pgfpathcurveto{\pgfqpoint{4.069895in}{1.461909in}}{\pgfqpoint{4.076737in}{1.459075in}}{\pgfqpoint{4.083869in}{1.459075in}}%
\pgfpathclose%
\pgfusepath{stroke,fill}%
\end{pgfscope}%
\begin{pgfscope}%
\pgfpathrectangle{\pgfqpoint{2.867647in}{0.500000in}}{\pgfqpoint{1.764706in}{1.700000in}}%
\pgfusepath{clip}%
\pgfsetbuttcap%
\pgfsetroundjoin%
\definecolor{currentfill}{rgb}{0.926767,0.298233,0.242855}%
\pgfsetfillcolor{currentfill}%
\pgfsetlinewidth{0.311001pt}%
\definecolor{currentstroke}{rgb}{1.000000,1.000000,1.000000}%
\pgfsetstrokecolor{currentstroke}%
\pgfsetdash{}{0pt}%
\pgfpathmoveto{\pgfqpoint{3.782115in}{1.731313in}}%
\pgfpathcurveto{\pgfqpoint{3.789248in}{1.731313in}}{\pgfqpoint{3.796090in}{1.734147in}}{\pgfqpoint{3.801134in}{1.739191in}}%
\pgfpathcurveto{\pgfqpoint{3.806177in}{1.744234in}}{\pgfqpoint{3.809011in}{1.751076in}}{\pgfqpoint{3.809011in}{1.758209in}}%
\pgfpathcurveto{\pgfqpoint{3.809011in}{1.765342in}}{\pgfqpoint{3.806177in}{1.772183in}}{\pgfqpoint{3.801134in}{1.777227in}}%
\pgfpathcurveto{\pgfqpoint{3.796090in}{1.782271in}}{\pgfqpoint{3.789248in}{1.785104in}}{\pgfqpoint{3.782115in}{1.785104in}}%
\pgfpathcurveto{\pgfqpoint{3.774983in}{1.785104in}}{\pgfqpoint{3.768141in}{1.782271in}}{\pgfqpoint{3.763097in}{1.777227in}}%
\pgfpathcurveto{\pgfqpoint{3.758054in}{1.772183in}}{\pgfqpoint{3.755220in}{1.765342in}}{\pgfqpoint{3.755220in}{1.758209in}}%
\pgfpathcurveto{\pgfqpoint{3.755220in}{1.751076in}}{\pgfqpoint{3.758054in}{1.744234in}}{\pgfqpoint{3.763097in}{1.739191in}}%
\pgfpathcurveto{\pgfqpoint{3.768141in}{1.734147in}}{\pgfqpoint{3.774983in}{1.731313in}}{\pgfqpoint{3.782115in}{1.731313in}}%
\pgfpathclose%
\pgfusepath{stroke,fill}%
\end{pgfscope}%
\begin{pgfscope}%
\pgfpathrectangle{\pgfqpoint{2.867647in}{0.500000in}}{\pgfqpoint{1.764706in}{1.700000in}}%
\pgfusepath{clip}%
\pgfsetbuttcap%
\pgfsetroundjoin%
\definecolor{currentfill}{rgb}{0.964032,0.651225,0.493258}%
\pgfsetfillcolor{currentfill}%
\pgfsetlinewidth{0.311001pt}%
\definecolor{currentstroke}{rgb}{1.000000,1.000000,1.000000}%
\pgfsetstrokecolor{currentstroke}%
\pgfsetdash{}{0pt}%
\pgfpathmoveto{\pgfqpoint{3.958753in}{1.718965in}}%
\pgfpathcurveto{\pgfqpoint{3.965886in}{1.718965in}}{\pgfqpoint{3.972728in}{1.721798in}}{\pgfqpoint{3.977771in}{1.726842in}}%
\pgfpathcurveto{\pgfqpoint{3.982815in}{1.731886in}}{\pgfqpoint{3.985649in}{1.738727in}}{\pgfqpoint{3.985649in}{1.745860in}}%
\pgfpathcurveto{\pgfqpoint{3.985649in}{1.752993in}}{\pgfqpoint{3.982815in}{1.759835in}}{\pgfqpoint{3.977771in}{1.764878in}}%
\pgfpathcurveto{\pgfqpoint{3.972728in}{1.769922in}}{\pgfqpoint{3.965886in}{1.772756in}}{\pgfqpoint{3.958753in}{1.772756in}}%
\pgfpathcurveto{\pgfqpoint{3.951620in}{1.772756in}}{\pgfqpoint{3.944779in}{1.769922in}}{\pgfqpoint{3.939735in}{1.764878in}}%
\pgfpathcurveto{\pgfqpoint{3.934691in}{1.759835in}}{\pgfqpoint{3.931858in}{1.752993in}}{\pgfqpoint{3.931858in}{1.745860in}}%
\pgfpathcurveto{\pgfqpoint{3.931858in}{1.738727in}}{\pgfqpoint{3.934691in}{1.731886in}}{\pgfqpoint{3.939735in}{1.726842in}}%
\pgfpathcurveto{\pgfqpoint{3.944779in}{1.721798in}}{\pgfqpoint{3.951620in}{1.718965in}}{\pgfqpoint{3.958753in}{1.718965in}}%
\pgfpathclose%
\pgfusepath{stroke,fill}%
\end{pgfscope}%
\begin{pgfscope}%
\pgfpathrectangle{\pgfqpoint{2.867647in}{0.500000in}}{\pgfqpoint{1.764706in}{1.700000in}}%
\pgfusepath{clip}%
\pgfsetbuttcap%
\pgfsetroundjoin%
\definecolor{currentfill}{rgb}{0.972201,0.839051,0.745789}%
\pgfsetfillcolor{currentfill}%
\pgfsetlinewidth{0.311001pt}%
\definecolor{currentstroke}{rgb}{1.000000,1.000000,1.000000}%
\pgfsetstrokecolor{currentstroke}%
\pgfsetdash{}{0pt}%
\pgfpathmoveto{\pgfqpoint{4.248443in}{1.451463in}}%
\pgfpathcurveto{\pgfqpoint{4.255576in}{1.451463in}}{\pgfqpoint{4.262418in}{1.454297in}}{\pgfqpoint{4.267462in}{1.459341in}}%
\pgfpathcurveto{\pgfqpoint{4.272505in}{1.464384in}}{\pgfqpoint{4.275339in}{1.471226in}}{\pgfqpoint{4.275339in}{1.478359in}}%
\pgfpathcurveto{\pgfqpoint{4.275339in}{1.485492in}}{\pgfqpoint{4.272505in}{1.492333in}}{\pgfqpoint{4.267462in}{1.497377in}}%
\pgfpathcurveto{\pgfqpoint{4.262418in}{1.502421in}}{\pgfqpoint{4.255576in}{1.505255in}}{\pgfqpoint{4.248443in}{1.505255in}}%
\pgfpathcurveto{\pgfqpoint{4.241311in}{1.505255in}}{\pgfqpoint{4.234469in}{1.502421in}}{\pgfqpoint{4.229425in}{1.497377in}}%
\pgfpathcurveto{\pgfqpoint{4.224382in}{1.492333in}}{\pgfqpoint{4.221548in}{1.485492in}}{\pgfqpoint{4.221548in}{1.478359in}}%
\pgfpathcurveto{\pgfqpoint{4.221548in}{1.471226in}}{\pgfqpoint{4.224382in}{1.464384in}}{\pgfqpoint{4.229425in}{1.459341in}}%
\pgfpathcurveto{\pgfqpoint{4.234469in}{1.454297in}}{\pgfqpoint{4.241311in}{1.451463in}}{\pgfqpoint{4.248443in}{1.451463in}}%
\pgfpathclose%
\pgfusepath{stroke,fill}%
\end{pgfscope}%
\begin{pgfscope}%
\pgfpathrectangle{\pgfqpoint{2.867647in}{0.500000in}}{\pgfqpoint{1.764706in}{1.700000in}}%
\pgfusepath{clip}%
\pgfsetbuttcap%
\pgfsetroundjoin%
\definecolor{currentfill}{rgb}{0.981377,0.920617,0.865369}%
\pgfsetfillcolor{currentfill}%
\pgfsetlinewidth{0.311001pt}%
\definecolor{currentstroke}{rgb}{1.000000,1.000000,1.000000}%
\pgfsetstrokecolor{currentstroke}%
\pgfsetdash{}{0pt}%
\pgfpathmoveto{\pgfqpoint{4.180472in}{1.300705in}}%
\pgfpathcurveto{\pgfqpoint{4.187605in}{1.300705in}}{\pgfqpoint{4.194447in}{1.303539in}}{\pgfqpoint{4.199490in}{1.308583in}}%
\pgfpathcurveto{\pgfqpoint{4.204534in}{1.313626in}}{\pgfqpoint{4.207368in}{1.320468in}}{\pgfqpoint{4.207368in}{1.327601in}}%
\pgfpathcurveto{\pgfqpoint{4.207368in}{1.334734in}}{\pgfqpoint{4.204534in}{1.341575in}}{\pgfqpoint{4.199490in}{1.346619in}}%
\pgfpathcurveto{\pgfqpoint{4.194447in}{1.351663in}}{\pgfqpoint{4.187605in}{1.354497in}}{\pgfqpoint{4.180472in}{1.354497in}}%
\pgfpathcurveto{\pgfqpoint{4.173339in}{1.354497in}}{\pgfqpoint{4.166498in}{1.351663in}}{\pgfqpoint{4.161454in}{1.346619in}}%
\pgfpathcurveto{\pgfqpoint{4.156410in}{1.341575in}}{\pgfqpoint{4.153576in}{1.334734in}}{\pgfqpoint{4.153576in}{1.327601in}}%
\pgfpathcurveto{\pgfqpoint{4.153576in}{1.320468in}}{\pgfqpoint{4.156410in}{1.313626in}}{\pgfqpoint{4.161454in}{1.308583in}}%
\pgfpathcurveto{\pgfqpoint{4.166498in}{1.303539in}}{\pgfqpoint{4.173339in}{1.300705in}}{\pgfqpoint{4.180472in}{1.300705in}}%
\pgfpathclose%
\pgfusepath{stroke,fill}%
\end{pgfscope}%
\begin{pgfscope}%
\pgfpathrectangle{\pgfqpoint{2.867647in}{0.500000in}}{\pgfqpoint{1.764706in}{1.700000in}}%
\pgfusepath{clip}%
\pgfsetbuttcap%
\pgfsetroundjoin%
\definecolor{currentfill}{rgb}{0.980678,0.914765,0.856766}%
\pgfsetfillcolor{currentfill}%
\pgfsetlinewidth{0.311001pt}%
\definecolor{currentstroke}{rgb}{1.000000,1.000000,1.000000}%
\pgfsetstrokecolor{currentstroke}%
\pgfsetdash{}{0pt}%
\pgfpathmoveto{\pgfqpoint{4.172419in}{1.311550in}}%
\pgfpathcurveto{\pgfqpoint{4.179552in}{1.311550in}}{\pgfqpoint{4.186393in}{1.314384in}}{\pgfqpoint{4.191437in}{1.319428in}}%
\pgfpathcurveto{\pgfqpoint{4.196481in}{1.324471in}}{\pgfqpoint{4.199315in}{1.331313in}}{\pgfqpoint{4.199315in}{1.338446in}}%
\pgfpathcurveto{\pgfqpoint{4.199315in}{1.345579in}}{\pgfqpoint{4.196481in}{1.352420in}}{\pgfqpoint{4.191437in}{1.357464in}}%
\pgfpathcurveto{\pgfqpoint{4.186393in}{1.362508in}}{\pgfqpoint{4.179552in}{1.365342in}}{\pgfqpoint{4.172419in}{1.365342in}}%
\pgfpathcurveto{\pgfqpoint{4.165286in}{1.365342in}}{\pgfqpoint{4.158444in}{1.362508in}}{\pgfqpoint{4.153401in}{1.357464in}}%
\pgfpathcurveto{\pgfqpoint{4.148357in}{1.352420in}}{\pgfqpoint{4.145523in}{1.345579in}}{\pgfqpoint{4.145523in}{1.338446in}}%
\pgfpathcurveto{\pgfqpoint{4.145523in}{1.331313in}}{\pgfqpoint{4.148357in}{1.324471in}}{\pgfqpoint{4.153401in}{1.319428in}}%
\pgfpathcurveto{\pgfqpoint{4.158444in}{1.314384in}}{\pgfqpoint{4.165286in}{1.311550in}}{\pgfqpoint{4.172419in}{1.311550in}}%
\pgfpathclose%
\pgfusepath{stroke,fill}%
\end{pgfscope}%
\begin{pgfscope}%
\pgfpathrectangle{\pgfqpoint{2.867647in}{0.500000in}}{\pgfqpoint{1.764706in}{1.700000in}}%
\pgfusepath{clip}%
\pgfsetbuttcap%
\pgfsetroundjoin%
\definecolor{currentfill}{rgb}{0.965592,0.726236,0.584384}%
\pgfsetfillcolor{currentfill}%
\pgfsetlinewidth{0.311001pt}%
\definecolor{currentstroke}{rgb}{1.000000,1.000000,1.000000}%
\pgfsetstrokecolor{currentstroke}%
\pgfsetdash{}{0pt}%
\pgfpathmoveto{\pgfqpoint{4.303957in}{1.297392in}}%
\pgfpathcurveto{\pgfqpoint{4.311090in}{1.297392in}}{\pgfqpoint{4.317932in}{1.300226in}}{\pgfqpoint{4.322976in}{1.305269in}}%
\pgfpathcurveto{\pgfqpoint{4.328019in}{1.310313in}}{\pgfqpoint{4.330853in}{1.317155in}}{\pgfqpoint{4.330853in}{1.324288in}}%
\pgfpathcurveto{\pgfqpoint{4.330853in}{1.331420in}}{\pgfqpoint{4.328019in}{1.338262in}}{\pgfqpoint{4.322976in}{1.343306in}}%
\pgfpathcurveto{\pgfqpoint{4.317932in}{1.348349in}}{\pgfqpoint{4.311090in}{1.351183in}}{\pgfqpoint{4.303957in}{1.351183in}}%
\pgfpathcurveto{\pgfqpoint{4.296825in}{1.351183in}}{\pgfqpoint{4.289983in}{1.348349in}}{\pgfqpoint{4.284939in}{1.343306in}}%
\pgfpathcurveto{\pgfqpoint{4.279896in}{1.338262in}}{\pgfqpoint{4.277062in}{1.331420in}}{\pgfqpoint{4.277062in}{1.324288in}}%
\pgfpathcurveto{\pgfqpoint{4.277062in}{1.317155in}}{\pgfqpoint{4.279896in}{1.310313in}}{\pgfqpoint{4.284939in}{1.305269in}}%
\pgfpathcurveto{\pgfqpoint{4.289983in}{1.300226in}}{\pgfqpoint{4.296825in}{1.297392in}}{\pgfqpoint{4.303957in}{1.297392in}}%
\pgfpathclose%
\pgfusepath{stroke,fill}%
\end{pgfscope}%
\begin{pgfscope}%
\pgfpathrectangle{\pgfqpoint{2.867647in}{0.500000in}}{\pgfqpoint{1.764706in}{1.700000in}}%
\pgfusepath{clip}%
\pgfsetbuttcap%
\pgfsetroundjoin%
\definecolor{currentfill}{rgb}{0.976287,0.879862,0.805788}%
\pgfsetfillcolor{currentfill}%
\pgfsetlinewidth{0.311001pt}%
\definecolor{currentstroke}{rgb}{1.000000,1.000000,1.000000}%
\pgfsetstrokecolor{currentstroke}%
\pgfsetdash{}{0pt}%
\pgfpathmoveto{\pgfqpoint{4.121139in}{1.466035in}}%
\pgfpathcurveto{\pgfqpoint{4.128272in}{1.466035in}}{\pgfqpoint{4.135114in}{1.468869in}}{\pgfqpoint{4.140157in}{1.473913in}}%
\pgfpathcurveto{\pgfqpoint{4.145201in}{1.478957in}}{\pgfqpoint{4.148035in}{1.485798in}}{\pgfqpoint{4.148035in}{1.492931in}}%
\pgfpathcurveto{\pgfqpoint{4.148035in}{1.500064in}}{\pgfqpoint{4.145201in}{1.506906in}}{\pgfqpoint{4.140157in}{1.511949in}}%
\pgfpathcurveto{\pgfqpoint{4.135114in}{1.516993in}}{\pgfqpoint{4.128272in}{1.519827in}}{\pgfqpoint{4.121139in}{1.519827in}}%
\pgfpathcurveto{\pgfqpoint{4.114006in}{1.519827in}}{\pgfqpoint{4.107165in}{1.516993in}}{\pgfqpoint{4.102121in}{1.511949in}}%
\pgfpathcurveto{\pgfqpoint{4.097077in}{1.506906in}}{\pgfqpoint{4.094244in}{1.500064in}}{\pgfqpoint{4.094244in}{1.492931in}}%
\pgfpathcurveto{\pgfqpoint{4.094244in}{1.485798in}}{\pgfqpoint{4.097077in}{1.478957in}}{\pgfqpoint{4.102121in}{1.473913in}}%
\pgfpathcurveto{\pgfqpoint{4.107165in}{1.468869in}}{\pgfqpoint{4.114006in}{1.466035in}}{\pgfqpoint{4.121139in}{1.466035in}}%
\pgfpathclose%
\pgfusepath{stroke,fill}%
\end{pgfscope}%
\begin{pgfscope}%
\pgfpathrectangle{\pgfqpoint{2.867647in}{0.500000in}}{\pgfqpoint{1.764706in}{1.700000in}}%
\pgfusepath{clip}%
\pgfsetbuttcap%
\pgfsetroundjoin%
\definecolor{currentfill}{rgb}{0.970255,0.815666,0.711203}%
\pgfsetfillcolor{currentfill}%
\pgfsetlinewidth{0.311001pt}%
\definecolor{currentstroke}{rgb}{1.000000,1.000000,1.000000}%
\pgfsetstrokecolor{currentstroke}%
\pgfsetdash{}{0pt}%
\pgfpathmoveto{\pgfqpoint{4.046326in}{1.020909in}}%
\pgfpathcurveto{\pgfqpoint{4.053459in}{1.020909in}}{\pgfqpoint{4.060300in}{1.023743in}}{\pgfqpoint{4.065344in}{1.028786in}}%
\pgfpathcurveto{\pgfqpoint{4.070388in}{1.033830in}}{\pgfqpoint{4.073222in}{1.040672in}}{\pgfqpoint{4.073222in}{1.047804in}}%
\pgfpathcurveto{\pgfqpoint{4.073222in}{1.054937in}}{\pgfqpoint{4.070388in}{1.061779in}}{\pgfqpoint{4.065344in}{1.066822in}}%
\pgfpathcurveto{\pgfqpoint{4.060300in}{1.071866in}}{\pgfqpoint{4.053459in}{1.074700in}}{\pgfqpoint{4.046326in}{1.074700in}}%
\pgfpathcurveto{\pgfqpoint{4.039193in}{1.074700in}}{\pgfqpoint{4.032351in}{1.071866in}}{\pgfqpoint{4.027308in}{1.066822in}}%
\pgfpathcurveto{\pgfqpoint{4.022264in}{1.061779in}}{\pgfqpoint{4.019430in}{1.054937in}}{\pgfqpoint{4.019430in}{1.047804in}}%
\pgfpathcurveto{\pgfqpoint{4.019430in}{1.040672in}}{\pgfqpoint{4.022264in}{1.033830in}}{\pgfqpoint{4.027308in}{1.028786in}}%
\pgfpathcurveto{\pgfqpoint{4.032351in}{1.023743in}}{\pgfqpoint{4.039193in}{1.020909in}}{\pgfqpoint{4.046326in}{1.020909in}}%
\pgfpathclose%
\pgfusepath{stroke,fill}%
\end{pgfscope}%
\begin{pgfscope}%
\pgfpathrectangle{\pgfqpoint{2.867647in}{0.500000in}}{\pgfqpoint{1.764706in}{1.700000in}}%
\pgfusepath{clip}%
\pgfsetbuttcap%
\pgfsetroundjoin%
\definecolor{currentfill}{rgb}{0.965169,0.707764,0.560659}%
\pgfsetfillcolor{currentfill}%
\pgfsetlinewidth{0.311001pt}%
\definecolor{currentstroke}{rgb}{1.000000,1.000000,1.000000}%
\pgfsetstrokecolor{currentstroke}%
\pgfsetdash{}{0pt}%
\pgfpathmoveto{\pgfqpoint{4.006875in}{1.738030in}}%
\pgfpathcurveto{\pgfqpoint{4.014008in}{1.738030in}}{\pgfqpoint{4.020849in}{1.740864in}}{\pgfqpoint{4.025893in}{1.745908in}}%
\pgfpathcurveto{\pgfqpoint{4.030937in}{1.750951in}}{\pgfqpoint{4.033771in}{1.757793in}}{\pgfqpoint{4.033771in}{1.764926in}}%
\pgfpathcurveto{\pgfqpoint{4.033771in}{1.772059in}}{\pgfqpoint{4.030937in}{1.778900in}}{\pgfqpoint{4.025893in}{1.783944in}}%
\pgfpathcurveto{\pgfqpoint{4.020849in}{1.788988in}}{\pgfqpoint{4.014008in}{1.791821in}}{\pgfqpoint{4.006875in}{1.791821in}}%
\pgfpathcurveto{\pgfqpoint{3.999742in}{1.791821in}}{\pgfqpoint{3.992900in}{1.788988in}}{\pgfqpoint{3.987857in}{1.783944in}}%
\pgfpathcurveto{\pgfqpoint{3.982813in}{1.778900in}}{\pgfqpoint{3.979979in}{1.772059in}}{\pgfqpoint{3.979979in}{1.764926in}}%
\pgfpathcurveto{\pgfqpoint{3.979979in}{1.757793in}}{\pgfqpoint{3.982813in}{1.750951in}}{\pgfqpoint{3.987857in}{1.745908in}}%
\pgfpathcurveto{\pgfqpoint{3.992900in}{1.740864in}}{\pgfqpoint{3.999742in}{1.738030in}}{\pgfqpoint{4.006875in}{1.738030in}}%
\pgfpathclose%
\pgfusepath{stroke,fill}%
\end{pgfscope}%
\begin{pgfscope}%
\pgfpathrectangle{\pgfqpoint{2.867647in}{0.500000in}}{\pgfqpoint{1.764706in}{1.700000in}}%
\pgfusepath{clip}%
\pgfsetbuttcap%
\pgfsetroundjoin%
\definecolor{currentfill}{rgb}{0.980678,0.914765,0.856766}%
\pgfsetfillcolor{currentfill}%
\pgfsetlinewidth{0.311001pt}%
\definecolor{currentstroke}{rgb}{1.000000,1.000000,1.000000}%
\pgfsetstrokecolor{currentstroke}%
\pgfsetdash{}{0pt}%
\pgfpathmoveto{\pgfqpoint{4.176328in}{1.384041in}}%
\pgfpathcurveto{\pgfqpoint{4.183461in}{1.384041in}}{\pgfqpoint{4.190303in}{1.386875in}}{\pgfqpoint{4.195346in}{1.391919in}}%
\pgfpathcurveto{\pgfqpoint{4.200390in}{1.396962in}}{\pgfqpoint{4.203224in}{1.403804in}}{\pgfqpoint{4.203224in}{1.410937in}}%
\pgfpathcurveto{\pgfqpoint{4.203224in}{1.418070in}}{\pgfqpoint{4.200390in}{1.424911in}}{\pgfqpoint{4.195346in}{1.429955in}}%
\pgfpathcurveto{\pgfqpoint{4.190303in}{1.434999in}}{\pgfqpoint{4.183461in}{1.437833in}}{\pgfqpoint{4.176328in}{1.437833in}}%
\pgfpathcurveto{\pgfqpoint{4.169195in}{1.437833in}}{\pgfqpoint{4.162354in}{1.434999in}}{\pgfqpoint{4.157310in}{1.429955in}}%
\pgfpathcurveto{\pgfqpoint{4.152266in}{1.424911in}}{\pgfqpoint{4.149433in}{1.418070in}}{\pgfqpoint{4.149433in}{1.410937in}}%
\pgfpathcurveto{\pgfqpoint{4.149433in}{1.403804in}}{\pgfqpoint{4.152266in}{1.396962in}}{\pgfqpoint{4.157310in}{1.391919in}}%
\pgfpathcurveto{\pgfqpoint{4.162354in}{1.386875in}}{\pgfqpoint{4.169195in}{1.384041in}}{\pgfqpoint{4.176328in}{1.384041in}}%
\pgfpathclose%
\pgfusepath{stroke,fill}%
\end{pgfscope}%
\begin{pgfscope}%
\pgfpathrectangle{\pgfqpoint{2.867647in}{0.500000in}}{\pgfqpoint{1.764706in}{1.700000in}}%
\pgfusepath{clip}%
\pgfsetbuttcap%
\pgfsetroundjoin%
\definecolor{currentfill}{rgb}{0.980678,0.914765,0.856766}%
\pgfsetfillcolor{currentfill}%
\pgfsetlinewidth{0.311001pt}%
\definecolor{currentstroke}{rgb}{1.000000,1.000000,1.000000}%
\pgfsetstrokecolor{currentstroke}%
\pgfsetdash{}{0pt}%
\pgfpathmoveto{\pgfqpoint{4.163953in}{1.430443in}}%
\pgfpathcurveto{\pgfqpoint{4.171086in}{1.430443in}}{\pgfqpoint{4.177927in}{1.433277in}}{\pgfqpoint{4.182971in}{1.438321in}}%
\pgfpathcurveto{\pgfqpoint{4.188015in}{1.443365in}}{\pgfqpoint{4.190849in}{1.450206in}}{\pgfqpoint{4.190849in}{1.457339in}}%
\pgfpathcurveto{\pgfqpoint{4.190849in}{1.464472in}}{\pgfqpoint{4.188015in}{1.471314in}}{\pgfqpoint{4.182971in}{1.476357in}}%
\pgfpathcurveto{\pgfqpoint{4.177927in}{1.481401in}}{\pgfqpoint{4.171086in}{1.484235in}}{\pgfqpoint{4.163953in}{1.484235in}}%
\pgfpathcurveto{\pgfqpoint{4.156820in}{1.484235in}}{\pgfqpoint{4.149978in}{1.481401in}}{\pgfqpoint{4.144935in}{1.476357in}}%
\pgfpathcurveto{\pgfqpoint{4.139891in}{1.471314in}}{\pgfqpoint{4.137057in}{1.464472in}}{\pgfqpoint{4.137057in}{1.457339in}}%
\pgfpathcurveto{\pgfqpoint{4.137057in}{1.450206in}}{\pgfqpoint{4.139891in}{1.443365in}}{\pgfqpoint{4.144935in}{1.438321in}}%
\pgfpathcurveto{\pgfqpoint{4.149978in}{1.433277in}}{\pgfqpoint{4.156820in}{1.430443in}}{\pgfqpoint{4.163953in}{1.430443in}}%
\pgfpathclose%
\pgfusepath{stroke,fill}%
\end{pgfscope}%
\begin{pgfscope}%
\pgfpathrectangle{\pgfqpoint{2.867647in}{0.500000in}}{\pgfqpoint{1.764706in}{1.700000in}}%
\pgfusepath{clip}%
\pgfsetbuttcap%
\pgfsetroundjoin%
\definecolor{currentfill}{rgb}{0.964173,0.657587,0.500469}%
\pgfsetfillcolor{currentfill}%
\pgfsetlinewidth{0.311001pt}%
\definecolor{currentstroke}{rgb}{1.000000,1.000000,1.000000}%
\pgfsetstrokecolor{currentstroke}%
\pgfsetdash{}{0pt}%
\pgfpathmoveto{\pgfqpoint{4.265703in}{1.051964in}}%
\pgfpathcurveto{\pgfqpoint{4.272836in}{1.051964in}}{\pgfqpoint{4.279678in}{1.054798in}}{\pgfqpoint{4.284721in}{1.059842in}}%
\pgfpathcurveto{\pgfqpoint{4.289765in}{1.064886in}}{\pgfqpoint{4.292599in}{1.071727in}}{\pgfqpoint{4.292599in}{1.078860in}}%
\pgfpathcurveto{\pgfqpoint{4.292599in}{1.085993in}}{\pgfqpoint{4.289765in}{1.092835in}}{\pgfqpoint{4.284721in}{1.097878in}}%
\pgfpathcurveto{\pgfqpoint{4.279678in}{1.102922in}}{\pgfqpoint{4.272836in}{1.105756in}}{\pgfqpoint{4.265703in}{1.105756in}}%
\pgfpathcurveto{\pgfqpoint{4.258570in}{1.105756in}}{\pgfqpoint{4.251729in}{1.102922in}}{\pgfqpoint{4.246685in}{1.097878in}}%
\pgfpathcurveto{\pgfqpoint{4.241641in}{1.092835in}}{\pgfqpoint{4.238807in}{1.085993in}}{\pgfqpoint{4.238807in}{1.078860in}}%
\pgfpathcurveto{\pgfqpoint{4.238807in}{1.071727in}}{\pgfqpoint{4.241641in}{1.064886in}}{\pgfqpoint{4.246685in}{1.059842in}}%
\pgfpathcurveto{\pgfqpoint{4.251729in}{1.054798in}}{\pgfqpoint{4.258570in}{1.051964in}}{\pgfqpoint{4.265703in}{1.051964in}}%
\pgfpathclose%
\pgfusepath{stroke,fill}%
\end{pgfscope}%
\begin{pgfscope}%
\pgfpathrectangle{\pgfqpoint{2.867647in}{0.500000in}}{\pgfqpoint{1.764706in}{1.700000in}}%
\pgfusepath{clip}%
\pgfsetbuttcap%
\pgfsetroundjoin%
\definecolor{currentfill}{rgb}{0.970255,0.815666,0.711203}%
\pgfsetfillcolor{currentfill}%
\pgfsetlinewidth{0.311001pt}%
\definecolor{currentstroke}{rgb}{1.000000,1.000000,1.000000}%
\pgfsetstrokecolor{currentstroke}%
\pgfsetdash{}{0pt}%
\pgfpathmoveto{\pgfqpoint{4.271967in}{1.252049in}}%
\pgfpathcurveto{\pgfqpoint{4.279100in}{1.252049in}}{\pgfqpoint{4.285941in}{1.254883in}}{\pgfqpoint{4.290985in}{1.259927in}}%
\pgfpathcurveto{\pgfqpoint{4.296029in}{1.264970in}}{\pgfqpoint{4.298862in}{1.271812in}}{\pgfqpoint{4.298862in}{1.278945in}}%
\pgfpathcurveto{\pgfqpoint{4.298862in}{1.286078in}}{\pgfqpoint{4.296029in}{1.292919in}}{\pgfqpoint{4.290985in}{1.297963in}}%
\pgfpathcurveto{\pgfqpoint{4.285941in}{1.303007in}}{\pgfqpoint{4.279100in}{1.305841in}}{\pgfqpoint{4.271967in}{1.305841in}}%
\pgfpathcurveto{\pgfqpoint{4.264834in}{1.305841in}}{\pgfqpoint{4.257992in}{1.303007in}}{\pgfqpoint{4.252949in}{1.297963in}}%
\pgfpathcurveto{\pgfqpoint{4.247905in}{1.292919in}}{\pgfqpoint{4.245071in}{1.286078in}}{\pgfqpoint{4.245071in}{1.278945in}}%
\pgfpathcurveto{\pgfqpoint{4.245071in}{1.271812in}}{\pgfqpoint{4.247905in}{1.264970in}}{\pgfqpoint{4.252949in}{1.259927in}}%
\pgfpathcurveto{\pgfqpoint{4.257992in}{1.254883in}}{\pgfqpoint{4.264834in}{1.252049in}}{\pgfqpoint{4.271967in}{1.252049in}}%
\pgfpathclose%
\pgfusepath{stroke,fill}%
\end{pgfscope}%
\begin{pgfscope}%
\pgfpathrectangle{\pgfqpoint{2.867647in}{0.500000in}}{\pgfqpoint{1.764706in}{1.700000in}}%
\pgfusepath{clip}%
\pgfsetbuttcap%
\pgfsetroundjoin%
\definecolor{currentfill}{rgb}{0.965440,0.720101,0.576404}%
\pgfsetfillcolor{currentfill}%
\pgfsetlinewidth{0.311001pt}%
\definecolor{currentstroke}{rgb}{1.000000,1.000000,1.000000}%
\pgfsetstrokecolor{currentstroke}%
\pgfsetdash{}{0pt}%
\pgfpathmoveto{\pgfqpoint{4.131413in}{0.915730in}}%
\pgfpathcurveto{\pgfqpoint{4.138546in}{0.915730in}}{\pgfqpoint{4.145388in}{0.918564in}}{\pgfqpoint{4.150431in}{0.923608in}}%
\pgfpathcurveto{\pgfqpoint{4.155475in}{0.928651in}}{\pgfqpoint{4.158309in}{0.935493in}}{\pgfqpoint{4.158309in}{0.942626in}}%
\pgfpathcurveto{\pgfqpoint{4.158309in}{0.949758in}}{\pgfqpoint{4.155475in}{0.956600in}}{\pgfqpoint{4.150431in}{0.961644in}}%
\pgfpathcurveto{\pgfqpoint{4.145388in}{0.966687in}}{\pgfqpoint{4.138546in}{0.969521in}}{\pgfqpoint{4.131413in}{0.969521in}}%
\pgfpathcurveto{\pgfqpoint{4.124280in}{0.969521in}}{\pgfqpoint{4.117439in}{0.966687in}}{\pgfqpoint{4.112395in}{0.961644in}}%
\pgfpathcurveto{\pgfqpoint{4.107351in}{0.956600in}}{\pgfqpoint{4.104517in}{0.949758in}}{\pgfqpoint{4.104517in}{0.942626in}}%
\pgfpathcurveto{\pgfqpoint{4.104517in}{0.935493in}}{\pgfqpoint{4.107351in}{0.928651in}}{\pgfqpoint{4.112395in}{0.923608in}}%
\pgfpathcurveto{\pgfqpoint{4.117439in}{0.918564in}}{\pgfqpoint{4.124280in}{0.915730in}}{\pgfqpoint{4.131413in}{0.915730in}}%
\pgfpathclose%
\pgfusepath{stroke,fill}%
\end{pgfscope}%
\begin{pgfscope}%
\pgfpathrectangle{\pgfqpoint{2.867647in}{0.500000in}}{\pgfqpoint{1.764706in}{1.700000in}}%
\pgfusepath{clip}%
\pgfsetbuttcap%
\pgfsetroundjoin%
\definecolor{currentfill}{rgb}{0.976961,0.885681,0.814303}%
\pgfsetfillcolor{currentfill}%
\pgfsetlinewidth{0.311001pt}%
\definecolor{currentstroke}{rgb}{1.000000,1.000000,1.000000}%
\pgfsetstrokecolor{currentstroke}%
\pgfsetdash{}{0pt}%
\pgfpathmoveto{\pgfqpoint{4.109295in}{1.507871in}}%
\pgfpathcurveto{\pgfqpoint{4.116428in}{1.507871in}}{\pgfqpoint{4.123270in}{1.510705in}}{\pgfqpoint{4.128314in}{1.515749in}}%
\pgfpathcurveto{\pgfqpoint{4.133357in}{1.520793in}}{\pgfqpoint{4.136191in}{1.527634in}}{\pgfqpoint{4.136191in}{1.534767in}}%
\pgfpathcurveto{\pgfqpoint{4.136191in}{1.541900in}}{\pgfqpoint{4.133357in}{1.548741in}}{\pgfqpoint{4.128314in}{1.553785in}}%
\pgfpathcurveto{\pgfqpoint{4.123270in}{1.558829in}}{\pgfqpoint{4.116428in}{1.561663in}}{\pgfqpoint{4.109295in}{1.561663in}}%
\pgfpathcurveto{\pgfqpoint{4.102163in}{1.561663in}}{\pgfqpoint{4.095321in}{1.558829in}}{\pgfqpoint{4.090277in}{1.553785in}}%
\pgfpathcurveto{\pgfqpoint{4.085234in}{1.548741in}}{\pgfqpoint{4.082400in}{1.541900in}}{\pgfqpoint{4.082400in}{1.534767in}}%
\pgfpathcurveto{\pgfqpoint{4.082400in}{1.527634in}}{\pgfqpoint{4.085234in}{1.520793in}}{\pgfqpoint{4.090277in}{1.515749in}}%
\pgfpathcurveto{\pgfqpoint{4.095321in}{1.510705in}}{\pgfqpoint{4.102163in}{1.507871in}}{\pgfqpoint{4.109295in}{1.507871in}}%
\pgfpathclose%
\pgfusepath{stroke,fill}%
\end{pgfscope}%
\begin{pgfscope}%
\pgfpathrectangle{\pgfqpoint{2.867647in}{0.500000in}}{\pgfqpoint{1.764706in}{1.700000in}}%
\pgfusepath{clip}%
\pgfsetbuttcap%
\pgfsetroundjoin%
\definecolor{currentfill}{rgb}{0.980678,0.914765,0.856766}%
\pgfsetfillcolor{currentfill}%
\pgfsetlinewidth{0.311001pt}%
\definecolor{currentstroke}{rgb}{1.000000,1.000000,1.000000}%
\pgfsetstrokecolor{currentstroke}%
\pgfsetdash{}{0pt}%
\pgfpathmoveto{\pgfqpoint{4.171184in}{1.484786in}}%
\pgfpathcurveto{\pgfqpoint{4.178317in}{1.484786in}}{\pgfqpoint{4.185158in}{1.487620in}}{\pgfqpoint{4.190202in}{1.492663in}}%
\pgfpathcurveto{\pgfqpoint{4.195246in}{1.497707in}}{\pgfqpoint{4.198080in}{1.504549in}}{\pgfqpoint{4.198080in}{1.511681in}}%
\pgfpathcurveto{\pgfqpoint{4.198080in}{1.518814in}}{\pgfqpoint{4.195246in}{1.525656in}}{\pgfqpoint{4.190202in}{1.530700in}}%
\pgfpathcurveto{\pgfqpoint{4.185158in}{1.535743in}}{\pgfqpoint{4.178317in}{1.538577in}}{\pgfqpoint{4.171184in}{1.538577in}}%
\pgfpathcurveto{\pgfqpoint{4.164051in}{1.538577in}}{\pgfqpoint{4.157210in}{1.535743in}}{\pgfqpoint{4.152166in}{1.530700in}}%
\pgfpathcurveto{\pgfqpoint{4.147122in}{1.525656in}}{\pgfqpoint{4.144288in}{1.518814in}}{\pgfqpoint{4.144288in}{1.511681in}}%
\pgfpathcurveto{\pgfqpoint{4.144288in}{1.504549in}}{\pgfqpoint{4.147122in}{1.497707in}}{\pgfqpoint{4.152166in}{1.492663in}}%
\pgfpathcurveto{\pgfqpoint{4.157210in}{1.487620in}}{\pgfqpoint{4.164051in}{1.484786in}}{\pgfqpoint{4.171184in}{1.484786in}}%
\pgfpathclose%
\pgfusepath{stroke,fill}%
\end{pgfscope}%
\begin{pgfscope}%
\pgfpathrectangle{\pgfqpoint{2.867647in}{0.500000in}}{\pgfqpoint{1.764706in}{1.700000in}}%
\pgfusepath{clip}%
\pgfsetbuttcap%
\pgfsetroundjoin%
\definecolor{currentfill}{rgb}{0.977657,0.891500,0.822809}%
\pgfsetfillcolor{currentfill}%
\pgfsetlinewidth{0.311001pt}%
\definecolor{currentstroke}{rgb}{1.000000,1.000000,1.000000}%
\pgfsetstrokecolor{currentstroke}%
\pgfsetdash{}{0pt}%
\pgfpathmoveto{\pgfqpoint{4.147219in}{1.328715in}}%
\pgfpathcurveto{\pgfqpoint{4.154352in}{1.328715in}}{\pgfqpoint{4.161193in}{1.331549in}}{\pgfqpoint{4.166237in}{1.336593in}}%
\pgfpathcurveto{\pgfqpoint{4.171281in}{1.341636in}}{\pgfqpoint{4.174114in}{1.348478in}}{\pgfqpoint{4.174114in}{1.355611in}}%
\pgfpathcurveto{\pgfqpoint{4.174114in}{1.362743in}}{\pgfqpoint{4.171281in}{1.369585in}}{\pgfqpoint{4.166237in}{1.374629in}}%
\pgfpathcurveto{\pgfqpoint{4.161193in}{1.379672in}}{\pgfqpoint{4.154352in}{1.382506in}}{\pgfqpoint{4.147219in}{1.382506in}}%
\pgfpathcurveto{\pgfqpoint{4.140086in}{1.382506in}}{\pgfqpoint{4.133244in}{1.379672in}}{\pgfqpoint{4.128201in}{1.374629in}}%
\pgfpathcurveto{\pgfqpoint{4.123157in}{1.369585in}}{\pgfqpoint{4.120323in}{1.362743in}}{\pgfqpoint{4.120323in}{1.355611in}}%
\pgfpathcurveto{\pgfqpoint{4.120323in}{1.348478in}}{\pgfqpoint{4.123157in}{1.341636in}}{\pgfqpoint{4.128201in}{1.336593in}}%
\pgfpathcurveto{\pgfqpoint{4.133244in}{1.331549in}}{\pgfqpoint{4.140086in}{1.328715in}}{\pgfqpoint{4.147219in}{1.328715in}}%
\pgfpathclose%
\pgfusepath{stroke,fill}%
\end{pgfscope}%
\begin{pgfscope}%
\pgfpathrectangle{\pgfqpoint{2.867647in}{0.500000in}}{\pgfqpoint{1.764706in}{1.700000in}}%
\pgfusepath{clip}%
\pgfsetbuttcap%
\pgfsetroundjoin%
\definecolor{currentfill}{rgb}{0.958791,0.526283,0.368909}%
\pgfsetfillcolor{currentfill}%
\pgfsetlinewidth{0.311001pt}%
\definecolor{currentstroke}{rgb}{1.000000,1.000000,1.000000}%
\pgfsetstrokecolor{currentstroke}%
\pgfsetdash{}{0pt}%
\pgfpathmoveto{\pgfqpoint{3.981475in}{1.110834in}}%
\pgfpathcurveto{\pgfqpoint{3.988608in}{1.110834in}}{\pgfqpoint{3.995450in}{1.113668in}}{\pgfqpoint{4.000493in}{1.118711in}}%
\pgfpathcurveto{\pgfqpoint{4.005537in}{1.123755in}}{\pgfqpoint{4.008371in}{1.130597in}}{\pgfqpoint{4.008371in}{1.137729in}}%
\pgfpathcurveto{\pgfqpoint{4.008371in}{1.144862in}}{\pgfqpoint{4.005537in}{1.151704in}}{\pgfqpoint{4.000493in}{1.156748in}}%
\pgfpathcurveto{\pgfqpoint{3.995450in}{1.161791in}}{\pgfqpoint{3.988608in}{1.164625in}}{\pgfqpoint{3.981475in}{1.164625in}}%
\pgfpathcurveto{\pgfqpoint{3.974342in}{1.164625in}}{\pgfqpoint{3.967501in}{1.161791in}}{\pgfqpoint{3.962457in}{1.156748in}}%
\pgfpathcurveto{\pgfqpoint{3.957413in}{1.151704in}}{\pgfqpoint{3.954579in}{1.144862in}}{\pgfqpoint{3.954579in}{1.137729in}}%
\pgfpathcurveto{\pgfqpoint{3.954579in}{1.130597in}}{\pgfqpoint{3.957413in}{1.123755in}}{\pgfqpoint{3.962457in}{1.118711in}}%
\pgfpathcurveto{\pgfqpoint{3.967501in}{1.113668in}}{\pgfqpoint{3.974342in}{1.110834in}}{\pgfqpoint{3.981475in}{1.110834in}}%
\pgfpathclose%
\pgfusepath{stroke,fill}%
\end{pgfscope}%
\begin{pgfscope}%
\pgfpathrectangle{\pgfqpoint{2.867647in}{0.500000in}}{\pgfqpoint{1.764706in}{1.700000in}}%
\pgfusepath{clip}%
\pgfsetbuttcap%
\pgfsetroundjoin%
\definecolor{currentfill}{rgb}{0.964173,0.657587,0.500469}%
\pgfsetfillcolor{currentfill}%
\pgfsetlinewidth{0.311001pt}%
\definecolor{currentstroke}{rgb}{1.000000,1.000000,1.000000}%
\pgfsetstrokecolor{currentstroke}%
\pgfsetdash{}{0pt}%
\pgfpathmoveto{\pgfqpoint{3.977289in}{1.002513in}}%
\pgfpathcurveto{\pgfqpoint{3.984422in}{1.002513in}}{\pgfqpoint{3.991264in}{1.005347in}}{\pgfqpoint{3.996307in}{1.010391in}}%
\pgfpathcurveto{\pgfqpoint{4.001351in}{1.015435in}}{\pgfqpoint{4.004185in}{1.022276in}}{\pgfqpoint{4.004185in}{1.029409in}}%
\pgfpathcurveto{\pgfqpoint{4.004185in}{1.036542in}}{\pgfqpoint{4.001351in}{1.043384in}}{\pgfqpoint{3.996307in}{1.048427in}}%
\pgfpathcurveto{\pgfqpoint{3.991264in}{1.053471in}}{\pgfqpoint{3.984422in}{1.056305in}}{\pgfqpoint{3.977289in}{1.056305in}}%
\pgfpathcurveto{\pgfqpoint{3.970156in}{1.056305in}}{\pgfqpoint{3.963315in}{1.053471in}}{\pgfqpoint{3.958271in}{1.048427in}}%
\pgfpathcurveto{\pgfqpoint{3.953227in}{1.043384in}}{\pgfqpoint{3.950393in}{1.036542in}}{\pgfqpoint{3.950393in}{1.029409in}}%
\pgfpathcurveto{\pgfqpoint{3.950393in}{1.022276in}}{\pgfqpoint{3.953227in}{1.015435in}}{\pgfqpoint{3.958271in}{1.010391in}}%
\pgfpathcurveto{\pgfqpoint{3.963315in}{1.005347in}}{\pgfqpoint{3.970156in}{1.002513in}}{\pgfqpoint{3.977289in}{1.002513in}}%
\pgfpathclose%
\pgfusepath{stroke,fill}%
\end{pgfscope}%
\begin{pgfscope}%
\pgfpathrectangle{\pgfqpoint{2.867647in}{0.500000in}}{\pgfqpoint{1.764706in}{1.700000in}}%
\pgfusepath{clip}%
\pgfsetbuttcap%
\pgfsetroundjoin%
\definecolor{currentfill}{rgb}{0.973832,0.856556,0.771584}%
\pgfsetfillcolor{currentfill}%
\pgfsetlinewidth{0.311001pt}%
\definecolor{currentstroke}{rgb}{1.000000,1.000000,1.000000}%
\pgfsetstrokecolor{currentstroke}%
\pgfsetdash{}{0pt}%
\pgfpathmoveto{\pgfqpoint{4.078968in}{1.037327in}}%
\pgfpathcurveto{\pgfqpoint{4.086101in}{1.037327in}}{\pgfqpoint{4.092942in}{1.040161in}}{\pgfqpoint{4.097986in}{1.045205in}}%
\pgfpathcurveto{\pgfqpoint{4.103030in}{1.050248in}}{\pgfqpoint{4.105864in}{1.057090in}}{\pgfqpoint{4.105864in}{1.064223in}}%
\pgfpathcurveto{\pgfqpoint{4.105864in}{1.071356in}}{\pgfqpoint{4.103030in}{1.078197in}}{\pgfqpoint{4.097986in}{1.083241in}}%
\pgfpathcurveto{\pgfqpoint{4.092942in}{1.088285in}}{\pgfqpoint{4.086101in}{1.091118in}}{\pgfqpoint{4.078968in}{1.091118in}}%
\pgfpathcurveto{\pgfqpoint{4.071835in}{1.091118in}}{\pgfqpoint{4.064994in}{1.088285in}}{\pgfqpoint{4.059950in}{1.083241in}}%
\pgfpathcurveto{\pgfqpoint{4.054906in}{1.078197in}}{\pgfqpoint{4.052072in}{1.071356in}}{\pgfqpoint{4.052072in}{1.064223in}}%
\pgfpathcurveto{\pgfqpoint{4.052072in}{1.057090in}}{\pgfqpoint{4.054906in}{1.050248in}}{\pgfqpoint{4.059950in}{1.045205in}}%
\pgfpathcurveto{\pgfqpoint{4.064994in}{1.040161in}}{\pgfqpoint{4.071835in}{1.037327in}}{\pgfqpoint{4.078968in}{1.037327in}}%
\pgfpathclose%
\pgfusepath{stroke,fill}%
\end{pgfscope}%
\begin{pgfscope}%
\pgfpathrectangle{\pgfqpoint{2.867647in}{0.500000in}}{\pgfqpoint{1.764706in}{1.700000in}}%
\pgfusepath{clip}%
\pgfsetbuttcap%
\pgfsetroundjoin%
\definecolor{currentfill}{rgb}{0.980678,0.914765,0.856766}%
\pgfsetfillcolor{currentfill}%
\pgfsetlinewidth{0.311001pt}%
\definecolor{currentstroke}{rgb}{1.000000,1.000000,1.000000}%
\pgfsetstrokecolor{currentstroke}%
\pgfsetdash{}{0pt}%
\pgfpathmoveto{\pgfqpoint{4.153678in}{1.497133in}}%
\pgfpathcurveto{\pgfqpoint{4.160811in}{1.497133in}}{\pgfqpoint{4.167653in}{1.499967in}}{\pgfqpoint{4.172696in}{1.505011in}}%
\pgfpathcurveto{\pgfqpoint{4.177740in}{1.510055in}}{\pgfqpoint{4.180574in}{1.516896in}}{\pgfqpoint{4.180574in}{1.524029in}}%
\pgfpathcurveto{\pgfqpoint{4.180574in}{1.531162in}}{\pgfqpoint{4.177740in}{1.538004in}}{\pgfqpoint{4.172696in}{1.543047in}}%
\pgfpathcurveto{\pgfqpoint{4.167653in}{1.548091in}}{\pgfqpoint{4.160811in}{1.550925in}}{\pgfqpoint{4.153678in}{1.550925in}}%
\pgfpathcurveto{\pgfqpoint{4.146545in}{1.550925in}}{\pgfqpoint{4.139704in}{1.548091in}}{\pgfqpoint{4.134660in}{1.543047in}}%
\pgfpathcurveto{\pgfqpoint{4.129616in}{1.538004in}}{\pgfqpoint{4.126782in}{1.531162in}}{\pgfqpoint{4.126782in}{1.524029in}}%
\pgfpathcurveto{\pgfqpoint{4.126782in}{1.516896in}}{\pgfqpoint{4.129616in}{1.510055in}}{\pgfqpoint{4.134660in}{1.505011in}}%
\pgfpathcurveto{\pgfqpoint{4.139704in}{1.499967in}}{\pgfqpoint{4.146545in}{1.497133in}}{\pgfqpoint{4.153678in}{1.497133in}}%
\pgfpathclose%
\pgfusepath{stroke,fill}%
\end{pgfscope}%
\begin{pgfscope}%
\pgfpathrectangle{\pgfqpoint{2.867647in}{0.500000in}}{\pgfqpoint{1.764706in}{1.700000in}}%
\pgfusepath{clip}%
\pgfsetbuttcap%
\pgfsetroundjoin%
\definecolor{currentfill}{rgb}{0.976287,0.879862,0.805788}%
\pgfsetfillcolor{currentfill}%
\pgfsetlinewidth{0.311001pt}%
\definecolor{currentstroke}{rgb}{1.000000,1.000000,1.000000}%
\pgfsetstrokecolor{currentstroke}%
\pgfsetdash{}{0pt}%
\pgfpathmoveto{\pgfqpoint{4.219867in}{1.473654in}}%
\pgfpathcurveto{\pgfqpoint{4.227000in}{1.473654in}}{\pgfqpoint{4.233842in}{1.476488in}}{\pgfqpoint{4.238885in}{1.481532in}}%
\pgfpathcurveto{\pgfqpoint{4.243929in}{1.486575in}}{\pgfqpoint{4.246763in}{1.493417in}}{\pgfqpoint{4.246763in}{1.500550in}}%
\pgfpathcurveto{\pgfqpoint{4.246763in}{1.507683in}}{\pgfqpoint{4.243929in}{1.514524in}}{\pgfqpoint{4.238885in}{1.519568in}}%
\pgfpathcurveto{\pgfqpoint{4.233842in}{1.524612in}}{\pgfqpoint{4.227000in}{1.527446in}}{\pgfqpoint{4.219867in}{1.527446in}}%
\pgfpathcurveto{\pgfqpoint{4.212734in}{1.527446in}}{\pgfqpoint{4.205893in}{1.524612in}}{\pgfqpoint{4.200849in}{1.519568in}}%
\pgfpathcurveto{\pgfqpoint{4.195806in}{1.514524in}}{\pgfqpoint{4.192972in}{1.507683in}}{\pgfqpoint{4.192972in}{1.500550in}}%
\pgfpathcurveto{\pgfqpoint{4.192972in}{1.493417in}}{\pgfqpoint{4.195806in}{1.486575in}}{\pgfqpoint{4.200849in}{1.481532in}}%
\pgfpathcurveto{\pgfqpoint{4.205893in}{1.476488in}}{\pgfqpoint{4.212734in}{1.473654in}}{\pgfqpoint{4.219867in}{1.473654in}}%
\pgfpathclose%
\pgfusepath{stroke,fill}%
\end{pgfscope}%
\begin{pgfscope}%
\pgfpathrectangle{\pgfqpoint{2.867647in}{0.500000in}}{\pgfqpoint{1.764706in}{1.700000in}}%
\pgfusepath{clip}%
\pgfsetbuttcap%
\pgfsetroundjoin%
\definecolor{currentfill}{rgb}{0.970718,0.821518,0.719872}%
\pgfsetfillcolor{currentfill}%
\pgfsetlinewidth{0.311001pt}%
\definecolor{currentstroke}{rgb}{1.000000,1.000000,1.000000}%
\pgfsetstrokecolor{currentstroke}%
\pgfsetdash{}{0pt}%
\pgfpathmoveto{\pgfqpoint{4.212543in}{1.584986in}}%
\pgfpathcurveto{\pgfqpoint{4.219676in}{1.584986in}}{\pgfqpoint{4.226518in}{1.587820in}}{\pgfqpoint{4.231562in}{1.592864in}}%
\pgfpathcurveto{\pgfqpoint{4.236605in}{1.597907in}}{\pgfqpoint{4.239439in}{1.604749in}}{\pgfqpoint{4.239439in}{1.611882in}}%
\pgfpathcurveto{\pgfqpoint{4.239439in}{1.619015in}}{\pgfqpoint{4.236605in}{1.625856in}}{\pgfqpoint{4.231562in}{1.630900in}}%
\pgfpathcurveto{\pgfqpoint{4.226518in}{1.635944in}}{\pgfqpoint{4.219676in}{1.638777in}}{\pgfqpoint{4.212543in}{1.638777in}}%
\pgfpathcurveto{\pgfqpoint{4.205411in}{1.638777in}}{\pgfqpoint{4.198569in}{1.635944in}}{\pgfqpoint{4.193525in}{1.630900in}}%
\pgfpathcurveto{\pgfqpoint{4.188482in}{1.625856in}}{\pgfqpoint{4.185648in}{1.619015in}}{\pgfqpoint{4.185648in}{1.611882in}}%
\pgfpathcurveto{\pgfqpoint{4.185648in}{1.604749in}}{\pgfqpoint{4.188482in}{1.597907in}}{\pgfqpoint{4.193525in}{1.592864in}}%
\pgfpathcurveto{\pgfqpoint{4.198569in}{1.587820in}}{\pgfqpoint{4.205411in}{1.584986in}}{\pgfqpoint{4.212543in}{1.584986in}}%
\pgfpathclose%
\pgfusepath{stroke,fill}%
\end{pgfscope}%
\begin{pgfscope}%
\pgfpathrectangle{\pgfqpoint{2.867647in}{0.500000in}}{\pgfqpoint{1.764706in}{1.700000in}}%
\pgfusepath{clip}%
\pgfsetbuttcap%
\pgfsetroundjoin%
\definecolor{currentfill}{rgb}{0.962985,0.612625,0.451451}%
\pgfsetfillcolor{currentfill}%
\pgfsetlinewidth{0.311001pt}%
\definecolor{currentstroke}{rgb}{1.000000,1.000000,1.000000}%
\pgfsetstrokecolor{currentstroke}%
\pgfsetdash{}{0pt}%
\pgfpathmoveto{\pgfqpoint{3.997391in}{1.792805in}}%
\pgfpathcurveto{\pgfqpoint{4.004524in}{1.792805in}}{\pgfqpoint{4.011365in}{1.795639in}}{\pgfqpoint{4.016409in}{1.800682in}}%
\pgfpathcurveto{\pgfqpoint{4.021453in}{1.805726in}}{\pgfqpoint{4.024287in}{1.812568in}}{\pgfqpoint{4.024287in}{1.819700in}}%
\pgfpathcurveto{\pgfqpoint{4.024287in}{1.826833in}}{\pgfqpoint{4.021453in}{1.833675in}}{\pgfqpoint{4.016409in}{1.838718in}}%
\pgfpathcurveto{\pgfqpoint{4.011365in}{1.843762in}}{\pgfqpoint{4.004524in}{1.846596in}}{\pgfqpoint{3.997391in}{1.846596in}}%
\pgfpathcurveto{\pgfqpoint{3.990258in}{1.846596in}}{\pgfqpoint{3.983416in}{1.843762in}}{\pgfqpoint{3.978373in}{1.838718in}}%
\pgfpathcurveto{\pgfqpoint{3.973329in}{1.833675in}}{\pgfqpoint{3.970495in}{1.826833in}}{\pgfqpoint{3.970495in}{1.819700in}}%
\pgfpathcurveto{\pgfqpoint{3.970495in}{1.812568in}}{\pgfqpoint{3.973329in}{1.805726in}}{\pgfqpoint{3.978373in}{1.800682in}}%
\pgfpathcurveto{\pgfqpoint{3.983416in}{1.795639in}}{\pgfqpoint{3.990258in}{1.792805in}}{\pgfqpoint{3.997391in}{1.792805in}}%
\pgfpathclose%
\pgfusepath{stroke,fill}%
\end{pgfscope}%
\begin{pgfscope}%
\pgfpathrectangle{\pgfqpoint{2.867647in}{0.500000in}}{\pgfqpoint{1.764706in}{1.700000in}}%
\pgfusepath{clip}%
\pgfsetbuttcap%
\pgfsetroundjoin%
\definecolor{currentfill}{rgb}{0.968931,0.798091,0.685123}%
\pgfsetfillcolor{currentfill}%
\pgfsetlinewidth{0.311001pt}%
\definecolor{currentstroke}{rgb}{1.000000,1.000000,1.000000}%
\pgfsetstrokecolor{currentstroke}%
\pgfsetdash{}{0pt}%
\pgfpathmoveto{\pgfqpoint{4.237678in}{1.550358in}}%
\pgfpathcurveto{\pgfqpoint{4.244811in}{1.550358in}}{\pgfqpoint{4.251653in}{1.553192in}}{\pgfqpoint{4.256696in}{1.558236in}}%
\pgfpathcurveto{\pgfqpoint{4.261740in}{1.563279in}}{\pgfqpoint{4.264574in}{1.570121in}}{\pgfqpoint{4.264574in}{1.577254in}}%
\pgfpathcurveto{\pgfqpoint{4.264574in}{1.584387in}}{\pgfqpoint{4.261740in}{1.591228in}}{\pgfqpoint{4.256696in}{1.596272in}}%
\pgfpathcurveto{\pgfqpoint{4.251653in}{1.601316in}}{\pgfqpoint{4.244811in}{1.604150in}}{\pgfqpoint{4.237678in}{1.604150in}}%
\pgfpathcurveto{\pgfqpoint{4.230545in}{1.604150in}}{\pgfqpoint{4.223704in}{1.601316in}}{\pgfqpoint{4.218660in}{1.596272in}}%
\pgfpathcurveto{\pgfqpoint{4.213616in}{1.591228in}}{\pgfqpoint{4.210783in}{1.584387in}}{\pgfqpoint{4.210783in}{1.577254in}}%
\pgfpathcurveto{\pgfqpoint{4.210783in}{1.570121in}}{\pgfqpoint{4.213616in}{1.563279in}}{\pgfqpoint{4.218660in}{1.558236in}}%
\pgfpathcurveto{\pgfqpoint{4.223704in}{1.553192in}}{\pgfqpoint{4.230545in}{1.550358in}}{\pgfqpoint{4.237678in}{1.550358in}}%
\pgfpathclose%
\pgfusepath{stroke,fill}%
\end{pgfscope}%
\begin{pgfscope}%
\pgfpathrectangle{\pgfqpoint{2.867647in}{0.500000in}}{\pgfqpoint{1.764706in}{1.700000in}}%
\pgfusepath{clip}%
\pgfsetbuttcap%
\pgfsetroundjoin%
\definecolor{currentfill}{rgb}{0.970718,0.821518,0.719872}%
\pgfsetfillcolor{currentfill}%
\pgfsetlinewidth{0.311001pt}%
\definecolor{currentstroke}{rgb}{1.000000,1.000000,1.000000}%
\pgfsetstrokecolor{currentstroke}%
\pgfsetdash{}{0pt}%
\pgfpathmoveto{\pgfqpoint{4.060939in}{1.075960in}}%
\pgfpathcurveto{\pgfqpoint{4.068072in}{1.075960in}}{\pgfqpoint{4.074914in}{1.078794in}}{\pgfqpoint{4.079958in}{1.083838in}}%
\pgfpathcurveto{\pgfqpoint{4.085001in}{1.088881in}}{\pgfqpoint{4.087835in}{1.095723in}}{\pgfqpoint{4.087835in}{1.102856in}}%
\pgfpathcurveto{\pgfqpoint{4.087835in}{1.109989in}}{\pgfqpoint{4.085001in}{1.116830in}}{\pgfqpoint{4.079958in}{1.121874in}}%
\pgfpathcurveto{\pgfqpoint{4.074914in}{1.126918in}}{\pgfqpoint{4.068072in}{1.129752in}}{\pgfqpoint{4.060939in}{1.129752in}}%
\pgfpathcurveto{\pgfqpoint{4.053807in}{1.129752in}}{\pgfqpoint{4.046965in}{1.126918in}}{\pgfqpoint{4.041921in}{1.121874in}}%
\pgfpathcurveto{\pgfqpoint{4.036878in}{1.116830in}}{\pgfqpoint{4.034044in}{1.109989in}}{\pgfqpoint{4.034044in}{1.102856in}}%
\pgfpathcurveto{\pgfqpoint{4.034044in}{1.095723in}}{\pgfqpoint{4.036878in}{1.088881in}}{\pgfqpoint{4.041921in}{1.083838in}}%
\pgfpathcurveto{\pgfqpoint{4.046965in}{1.078794in}}{\pgfqpoint{4.053807in}{1.075960in}}{\pgfqpoint{4.060939in}{1.075960in}}%
\pgfpathclose%
\pgfusepath{stroke,fill}%
\end{pgfscope}%
\begin{pgfscope}%
\pgfpathrectangle{\pgfqpoint{2.867647in}{0.500000in}}{\pgfqpoint{1.764706in}{1.700000in}}%
\pgfusepath{clip}%
\pgfsetbuttcap%
\pgfsetroundjoin%
\definecolor{currentfill}{rgb}{0.960778,0.559972,0.399412}%
\pgfsetfillcolor{currentfill}%
\pgfsetlinewidth{0.311001pt}%
\definecolor{currentstroke}{rgb}{1.000000,1.000000,1.000000}%
\pgfsetstrokecolor{currentstroke}%
\pgfsetdash{}{0pt}%
\pgfpathmoveto{\pgfqpoint{3.965556in}{1.055945in}}%
\pgfpathcurveto{\pgfqpoint{3.972689in}{1.055945in}}{\pgfqpoint{3.979531in}{1.058779in}}{\pgfqpoint{3.984574in}{1.063822in}}%
\pgfpathcurveto{\pgfqpoint{3.989618in}{1.068866in}}{\pgfqpoint{3.992452in}{1.075708in}}{\pgfqpoint{3.992452in}{1.082840in}}%
\pgfpathcurveto{\pgfqpoint{3.992452in}{1.089973in}}{\pgfqpoint{3.989618in}{1.096815in}}{\pgfqpoint{3.984574in}{1.101859in}}%
\pgfpathcurveto{\pgfqpoint{3.979531in}{1.106902in}}{\pgfqpoint{3.972689in}{1.109736in}}{\pgfqpoint{3.965556in}{1.109736in}}%
\pgfpathcurveto{\pgfqpoint{3.958423in}{1.109736in}}{\pgfqpoint{3.951582in}{1.106902in}}{\pgfqpoint{3.946538in}{1.101859in}}%
\pgfpathcurveto{\pgfqpoint{3.941494in}{1.096815in}}{\pgfqpoint{3.938661in}{1.089973in}}{\pgfqpoint{3.938661in}{1.082840in}}%
\pgfpathcurveto{\pgfqpoint{3.938661in}{1.075708in}}{\pgfqpoint{3.941494in}{1.068866in}}{\pgfqpoint{3.946538in}{1.063822in}}%
\pgfpathcurveto{\pgfqpoint{3.951582in}{1.058779in}}{\pgfqpoint{3.958423in}{1.055945in}}{\pgfqpoint{3.965556in}{1.055945in}}%
\pgfpathclose%
\pgfusepath{stroke,fill}%
\end{pgfscope}%
\begin{pgfscope}%
\pgfpathrectangle{\pgfqpoint{2.867647in}{0.500000in}}{\pgfqpoint{1.764706in}{1.700000in}}%
\pgfusepath{clip}%
\pgfsetbuttcap%
\pgfsetroundjoin%
\definecolor{currentfill}{rgb}{0.979124,0.903132,0.839793}%
\pgfsetfillcolor{currentfill}%
\pgfsetlinewidth{0.311001pt}%
\definecolor{currentstroke}{rgb}{1.000000,1.000000,1.000000}%
\pgfsetstrokecolor{currentstroke}%
\pgfsetdash{}{0pt}%
\pgfpathmoveto{\pgfqpoint{4.214507in}{1.203695in}}%
\pgfpathcurveto{\pgfqpoint{4.221640in}{1.203695in}}{\pgfqpoint{4.228481in}{1.206528in}}{\pgfqpoint{4.233525in}{1.211572in}}%
\pgfpathcurveto{\pgfqpoint{4.238569in}{1.216616in}}{\pgfqpoint{4.241402in}{1.223457in}}{\pgfqpoint{4.241402in}{1.230590in}}%
\pgfpathcurveto{\pgfqpoint{4.241402in}{1.237723in}}{\pgfqpoint{4.238569in}{1.244565in}}{\pgfqpoint{4.233525in}{1.249608in}}%
\pgfpathcurveto{\pgfqpoint{4.228481in}{1.254652in}}{\pgfqpoint{4.221640in}{1.257486in}}{\pgfqpoint{4.214507in}{1.257486in}}%
\pgfpathcurveto{\pgfqpoint{4.207374in}{1.257486in}}{\pgfqpoint{4.200532in}{1.254652in}}{\pgfqpoint{4.195489in}{1.249608in}}%
\pgfpathcurveto{\pgfqpoint{4.190445in}{1.244565in}}{\pgfqpoint{4.187611in}{1.237723in}}{\pgfqpoint{4.187611in}{1.230590in}}%
\pgfpathcurveto{\pgfqpoint{4.187611in}{1.223457in}}{\pgfqpoint{4.190445in}{1.216616in}}{\pgfqpoint{4.195489in}{1.211572in}}%
\pgfpathcurveto{\pgfqpoint{4.200532in}{1.206528in}}{\pgfqpoint{4.207374in}{1.203695in}}{\pgfqpoint{4.214507in}{1.203695in}}%
\pgfpathclose%
\pgfusepath{stroke,fill}%
\end{pgfscope}%
\begin{pgfscope}%
\pgfpathrectangle{\pgfqpoint{2.867647in}{0.500000in}}{\pgfqpoint{1.764706in}{1.700000in}}%
\pgfusepath{clip}%
\pgfsetbuttcap%
\pgfsetroundjoin%
\definecolor{currentfill}{rgb}{0.966812,0.762584,0.633643}%
\pgfsetfillcolor{currentfill}%
\pgfsetlinewidth{0.311001pt}%
\definecolor{currentstroke}{rgb}{1.000000,1.000000,1.000000}%
\pgfsetstrokecolor{currentstroke}%
\pgfsetdash{}{0pt}%
\pgfpathmoveto{\pgfqpoint{4.252301in}{1.549505in}}%
\pgfpathcurveto{\pgfqpoint{4.259434in}{1.549505in}}{\pgfqpoint{4.266275in}{1.552339in}}{\pgfqpoint{4.271319in}{1.557383in}}%
\pgfpathcurveto{\pgfqpoint{4.276363in}{1.562426in}}{\pgfqpoint{4.279197in}{1.569268in}}{\pgfqpoint{4.279197in}{1.576401in}}%
\pgfpathcurveto{\pgfqpoint{4.279197in}{1.583534in}}{\pgfqpoint{4.276363in}{1.590375in}}{\pgfqpoint{4.271319in}{1.595419in}}%
\pgfpathcurveto{\pgfqpoint{4.266275in}{1.600463in}}{\pgfqpoint{4.259434in}{1.603297in}}{\pgfqpoint{4.252301in}{1.603297in}}%
\pgfpathcurveto{\pgfqpoint{4.245168in}{1.603297in}}{\pgfqpoint{4.238326in}{1.600463in}}{\pgfqpoint{4.233283in}{1.595419in}}%
\pgfpathcurveto{\pgfqpoint{4.228239in}{1.590375in}}{\pgfqpoint{4.225405in}{1.583534in}}{\pgfqpoint{4.225405in}{1.576401in}}%
\pgfpathcurveto{\pgfqpoint{4.225405in}{1.569268in}}{\pgfqpoint{4.228239in}{1.562426in}}{\pgfqpoint{4.233283in}{1.557383in}}%
\pgfpathcurveto{\pgfqpoint{4.238326in}{1.552339in}}{\pgfqpoint{4.245168in}{1.549505in}}{\pgfqpoint{4.252301in}{1.549505in}}%
\pgfpathclose%
\pgfusepath{stroke,fill}%
\end{pgfscope}%
\begin{pgfscope}%
\pgfpathrectangle{\pgfqpoint{2.867647in}{0.500000in}}{\pgfqpoint{1.764706in}{1.700000in}}%
\pgfusepath{clip}%
\pgfsetbuttcap%
\pgfsetroundjoin%
\definecolor{currentfill}{rgb}{0.965302,0.713942,0.568499}%
\pgfsetfillcolor{currentfill}%
\pgfsetlinewidth{0.311001pt}%
\definecolor{currentstroke}{rgb}{1.000000,1.000000,1.000000}%
\pgfsetstrokecolor{currentstroke}%
\pgfsetdash{}{0pt}%
\pgfpathmoveto{\pgfqpoint{4.001873in}{0.926317in}}%
\pgfpathcurveto{\pgfqpoint{4.009006in}{0.926317in}}{\pgfqpoint{4.015847in}{0.929151in}}{\pgfqpoint{4.020891in}{0.934195in}}%
\pgfpathcurveto{\pgfqpoint{4.025935in}{0.939239in}}{\pgfqpoint{4.028768in}{0.946080in}}{\pgfqpoint{4.028768in}{0.953213in}}%
\pgfpathcurveto{\pgfqpoint{4.028768in}{0.960346in}}{\pgfqpoint{4.025935in}{0.967188in}}{\pgfqpoint{4.020891in}{0.972231in}}%
\pgfpathcurveto{\pgfqpoint{4.015847in}{0.977275in}}{\pgfqpoint{4.009006in}{0.980109in}}{\pgfqpoint{4.001873in}{0.980109in}}%
\pgfpathcurveto{\pgfqpoint{3.994740in}{0.980109in}}{\pgfqpoint{3.987898in}{0.977275in}}{\pgfqpoint{3.982855in}{0.972231in}}%
\pgfpathcurveto{\pgfqpoint{3.977811in}{0.967188in}}{\pgfqpoint{3.974977in}{0.960346in}}{\pgfqpoint{3.974977in}{0.953213in}}%
\pgfpathcurveto{\pgfqpoint{3.974977in}{0.946080in}}{\pgfqpoint{3.977811in}{0.939239in}}{\pgfqpoint{3.982855in}{0.934195in}}%
\pgfpathcurveto{\pgfqpoint{3.987898in}{0.929151in}}{\pgfqpoint{3.994740in}{0.926317in}}{\pgfqpoint{4.001873in}{0.926317in}}%
\pgfpathclose%
\pgfusepath{stroke,fill}%
\end{pgfscope}%
\begin{pgfscope}%
\pgfpathrectangle{\pgfqpoint{2.867647in}{0.500000in}}{\pgfqpoint{1.764706in}{1.700000in}}%
\pgfusepath{clip}%
\pgfsetbuttcap%
\pgfsetroundjoin%
\definecolor{currentfill}{rgb}{0.971694,0.833208,0.737161}%
\pgfsetfillcolor{currentfill}%
\pgfsetlinewidth{0.311001pt}%
\definecolor{currentstroke}{rgb}{1.000000,1.000000,1.000000}%
\pgfsetstrokecolor{currentstroke}%
\pgfsetdash{}{0pt}%
\pgfpathmoveto{\pgfqpoint{4.064808in}{0.987098in}}%
\pgfpathcurveto{\pgfqpoint{4.071941in}{0.987098in}}{\pgfqpoint{4.078782in}{0.989932in}}{\pgfqpoint{4.083826in}{0.994976in}}%
\pgfpathcurveto{\pgfqpoint{4.088870in}{1.000019in}}{\pgfqpoint{4.091704in}{1.006861in}}{\pgfqpoint{4.091704in}{1.013994in}}%
\pgfpathcurveto{\pgfqpoint{4.091704in}{1.021127in}}{\pgfqpoint{4.088870in}{1.027968in}}{\pgfqpoint{4.083826in}{1.033012in}}%
\pgfpathcurveto{\pgfqpoint{4.078782in}{1.038056in}}{\pgfqpoint{4.071941in}{1.040890in}}{\pgfqpoint{4.064808in}{1.040890in}}%
\pgfpathcurveto{\pgfqpoint{4.057675in}{1.040890in}}{\pgfqpoint{4.050833in}{1.038056in}}{\pgfqpoint{4.045790in}{1.033012in}}%
\pgfpathcurveto{\pgfqpoint{4.040746in}{1.027968in}}{\pgfqpoint{4.037912in}{1.021127in}}{\pgfqpoint{4.037912in}{1.013994in}}%
\pgfpathcurveto{\pgfqpoint{4.037912in}{1.006861in}}{\pgfqpoint{4.040746in}{1.000019in}}{\pgfqpoint{4.045790in}{0.994976in}}%
\pgfpathcurveto{\pgfqpoint{4.050833in}{0.989932in}}{\pgfqpoint{4.057675in}{0.987098in}}{\pgfqpoint{4.064808in}{0.987098in}}%
\pgfpathclose%
\pgfusepath{stroke,fill}%
\end{pgfscope}%
\begin{pgfscope}%
\pgfpathrectangle{\pgfqpoint{2.867647in}{0.500000in}}{\pgfqpoint{1.764706in}{1.700000in}}%
\pgfusepath{clip}%
\pgfsetbuttcap%
\pgfsetroundjoin%
\definecolor{currentfill}{rgb}{0.974412,0.862387,0.780156}%
\pgfsetfillcolor{currentfill}%
\pgfsetlinewidth{0.311001pt}%
\definecolor{currentstroke}{rgb}{1.000000,1.000000,1.000000}%
\pgfsetstrokecolor{currentstroke}%
\pgfsetdash{}{0pt}%
\pgfpathmoveto{\pgfqpoint{4.117685in}{1.430597in}}%
\pgfpathcurveto{\pgfqpoint{4.124818in}{1.430597in}}{\pgfqpoint{4.131660in}{1.433431in}}{\pgfqpoint{4.136703in}{1.438475in}}%
\pgfpathcurveto{\pgfqpoint{4.141747in}{1.443518in}}{\pgfqpoint{4.144581in}{1.450360in}}{\pgfqpoint{4.144581in}{1.457493in}}%
\pgfpathcurveto{\pgfqpoint{4.144581in}{1.464626in}}{\pgfqpoint{4.141747in}{1.471467in}}{\pgfqpoint{4.136703in}{1.476511in}}%
\pgfpathcurveto{\pgfqpoint{4.131660in}{1.481555in}}{\pgfqpoint{4.124818in}{1.484389in}}{\pgfqpoint{4.117685in}{1.484389in}}%
\pgfpathcurveto{\pgfqpoint{4.110552in}{1.484389in}}{\pgfqpoint{4.103711in}{1.481555in}}{\pgfqpoint{4.098667in}{1.476511in}}%
\pgfpathcurveto{\pgfqpoint{4.093623in}{1.471467in}}{\pgfqpoint{4.090789in}{1.464626in}}{\pgfqpoint{4.090789in}{1.457493in}}%
\pgfpathcurveto{\pgfqpoint{4.090789in}{1.450360in}}{\pgfqpoint{4.093623in}{1.443518in}}{\pgfqpoint{4.098667in}{1.438475in}}%
\pgfpathcurveto{\pgfqpoint{4.103711in}{1.433431in}}{\pgfqpoint{4.110552in}{1.430597in}}{\pgfqpoint{4.117685in}{1.430597in}}%
\pgfpathclose%
\pgfusepath{stroke,fill}%
\end{pgfscope}%
\begin{pgfscope}%
\pgfpathrectangle{\pgfqpoint{2.867647in}{0.500000in}}{\pgfqpoint{1.764706in}{1.700000in}}%
\pgfusepath{clip}%
\pgfsetbuttcap%
\pgfsetroundjoin%
\definecolor{currentfill}{rgb}{0.962532,0.599594,0.438051}%
\pgfsetfillcolor{currentfill}%
\pgfsetlinewidth{0.311001pt}%
\definecolor{currentstroke}{rgb}{1.000000,1.000000,1.000000}%
\pgfsetstrokecolor{currentstroke}%
\pgfsetdash{}{0pt}%
\pgfpathmoveto{\pgfqpoint{3.948791in}{0.919858in}}%
\pgfpathcurveto{\pgfqpoint{3.955924in}{0.919858in}}{\pgfqpoint{3.962766in}{0.922692in}}{\pgfqpoint{3.967809in}{0.927735in}}%
\pgfpathcurveto{\pgfqpoint{3.972853in}{0.932779in}}{\pgfqpoint{3.975687in}{0.939620in}}{\pgfqpoint{3.975687in}{0.946753in}}%
\pgfpathcurveto{\pgfqpoint{3.975687in}{0.953886in}}{\pgfqpoint{3.972853in}{0.960728in}}{\pgfqpoint{3.967809in}{0.965771in}}%
\pgfpathcurveto{\pgfqpoint{3.962766in}{0.970815in}}{\pgfqpoint{3.955924in}{0.973649in}}{\pgfqpoint{3.948791in}{0.973649in}}%
\pgfpathcurveto{\pgfqpoint{3.941658in}{0.973649in}}{\pgfqpoint{3.934817in}{0.970815in}}{\pgfqpoint{3.929773in}{0.965771in}}%
\pgfpathcurveto{\pgfqpoint{3.924729in}{0.960728in}}{\pgfqpoint{3.921895in}{0.953886in}}{\pgfqpoint{3.921895in}{0.946753in}}%
\pgfpathcurveto{\pgfqpoint{3.921895in}{0.939620in}}{\pgfqpoint{3.924729in}{0.932779in}}{\pgfqpoint{3.929773in}{0.927735in}}%
\pgfpathcurveto{\pgfqpoint{3.934817in}{0.922692in}}{\pgfqpoint{3.941658in}{0.919858in}}{\pgfqpoint{3.948791in}{0.919858in}}%
\pgfpathclose%
\pgfusepath{stroke,fill}%
\end{pgfscope}%
\begin{pgfscope}%
\pgfpathrectangle{\pgfqpoint{2.867647in}{0.500000in}}{\pgfqpoint{1.764706in}{1.700000in}}%
\pgfusepath{clip}%
\pgfsetbuttcap%
\pgfsetroundjoin%
\definecolor{currentfill}{rgb}{0.970255,0.815666,0.711203}%
\pgfsetfillcolor{currentfill}%
\pgfsetlinewidth{0.311001pt}%
\definecolor{currentstroke}{rgb}{1.000000,1.000000,1.000000}%
\pgfsetstrokecolor{currentstroke}%
\pgfsetdash{}{0pt}%
\pgfpathmoveto{\pgfqpoint{4.163659in}{1.654988in}}%
\pgfpathcurveto{\pgfqpoint{4.170792in}{1.654988in}}{\pgfqpoint{4.177634in}{1.657822in}}{\pgfqpoint{4.182677in}{1.662866in}}%
\pgfpathcurveto{\pgfqpoint{4.187721in}{1.667909in}}{\pgfqpoint{4.190555in}{1.674751in}}{\pgfqpoint{4.190555in}{1.681884in}}%
\pgfpathcurveto{\pgfqpoint{4.190555in}{1.689017in}}{\pgfqpoint{4.187721in}{1.695858in}}{\pgfqpoint{4.182677in}{1.700902in}}%
\pgfpathcurveto{\pgfqpoint{4.177634in}{1.705946in}}{\pgfqpoint{4.170792in}{1.708780in}}{\pgfqpoint{4.163659in}{1.708780in}}%
\pgfpathcurveto{\pgfqpoint{4.156526in}{1.708780in}}{\pgfqpoint{4.149685in}{1.705946in}}{\pgfqpoint{4.144641in}{1.700902in}}%
\pgfpathcurveto{\pgfqpoint{4.139597in}{1.695858in}}{\pgfqpoint{4.136763in}{1.689017in}}{\pgfqpoint{4.136763in}{1.681884in}}%
\pgfpathcurveto{\pgfqpoint{4.136763in}{1.674751in}}{\pgfqpoint{4.139597in}{1.667909in}}{\pgfqpoint{4.144641in}{1.662866in}}%
\pgfpathcurveto{\pgfqpoint{4.149685in}{1.657822in}}{\pgfqpoint{4.156526in}{1.654988in}}{\pgfqpoint{4.163659in}{1.654988in}}%
\pgfpathclose%
\pgfusepath{stroke,fill}%
\end{pgfscope}%
\begin{pgfscope}%
\pgfpathrectangle{\pgfqpoint{2.867647in}{0.500000in}}{\pgfqpoint{1.764706in}{1.700000in}}%
\pgfusepath{clip}%
\pgfsetbuttcap%
\pgfsetroundjoin%
\definecolor{currentfill}{rgb}{0.980678,0.914765,0.856766}%
\pgfsetfillcolor{currentfill}%
\pgfsetlinewidth{0.311001pt}%
\definecolor{currentstroke}{rgb}{1.000000,1.000000,1.000000}%
\pgfsetstrokecolor{currentstroke}%
\pgfsetdash{}{0pt}%
\pgfpathmoveto{\pgfqpoint{4.177829in}{1.161914in}}%
\pgfpathcurveto{\pgfqpoint{4.184962in}{1.161914in}}{\pgfqpoint{4.191803in}{1.164748in}}{\pgfqpoint{4.196847in}{1.169792in}}%
\pgfpathcurveto{\pgfqpoint{4.201891in}{1.174835in}}{\pgfqpoint{4.204725in}{1.181677in}}{\pgfqpoint{4.204725in}{1.188810in}}%
\pgfpathcurveto{\pgfqpoint{4.204725in}{1.195943in}}{\pgfqpoint{4.201891in}{1.202784in}}{\pgfqpoint{4.196847in}{1.207828in}}%
\pgfpathcurveto{\pgfqpoint{4.191803in}{1.212872in}}{\pgfqpoint{4.184962in}{1.215706in}}{\pgfqpoint{4.177829in}{1.215706in}}%
\pgfpathcurveto{\pgfqpoint{4.170696in}{1.215706in}}{\pgfqpoint{4.163855in}{1.212872in}}{\pgfqpoint{4.158811in}{1.207828in}}%
\pgfpathcurveto{\pgfqpoint{4.153767in}{1.202784in}}{\pgfqpoint{4.150933in}{1.195943in}}{\pgfqpoint{4.150933in}{1.188810in}}%
\pgfpathcurveto{\pgfqpoint{4.150933in}{1.181677in}}{\pgfqpoint{4.153767in}{1.174835in}}{\pgfqpoint{4.158811in}{1.169792in}}%
\pgfpathcurveto{\pgfqpoint{4.163855in}{1.164748in}}{\pgfqpoint{4.170696in}{1.161914in}}{\pgfqpoint{4.177829in}{1.161914in}}%
\pgfpathclose%
\pgfusepath{stroke,fill}%
\end{pgfscope}%
\begin{pgfscope}%
\pgfpathrectangle{\pgfqpoint{2.867647in}{0.500000in}}{\pgfqpoint{1.764706in}{1.700000in}}%
\pgfusepath{clip}%
\pgfsetbuttcap%
\pgfsetroundjoin%
\definecolor{currentfill}{rgb}{0.977657,0.891500,0.822809}%
\pgfsetfillcolor{currentfill}%
\pgfsetlinewidth{0.311001pt}%
\definecolor{currentstroke}{rgb}{1.000000,1.000000,1.000000}%
\pgfsetstrokecolor{currentstroke}%
\pgfsetdash{}{0pt}%
\pgfpathmoveto{\pgfqpoint{4.146872in}{1.055321in}}%
\pgfpathcurveto{\pgfqpoint{4.154005in}{1.055321in}}{\pgfqpoint{4.160846in}{1.058155in}}{\pgfqpoint{4.165890in}{1.063198in}}%
\pgfpathcurveto{\pgfqpoint{4.170934in}{1.068242in}}{\pgfqpoint{4.173768in}{1.075084in}}{\pgfqpoint{4.173768in}{1.082216in}}%
\pgfpathcurveto{\pgfqpoint{4.173768in}{1.089349in}}{\pgfqpoint{4.170934in}{1.096191in}}{\pgfqpoint{4.165890in}{1.101235in}}%
\pgfpathcurveto{\pgfqpoint{4.160846in}{1.106278in}}{\pgfqpoint{4.154005in}{1.109112in}}{\pgfqpoint{4.146872in}{1.109112in}}%
\pgfpathcurveto{\pgfqpoint{4.139739in}{1.109112in}}{\pgfqpoint{4.132897in}{1.106278in}}{\pgfqpoint{4.127854in}{1.101235in}}%
\pgfpathcurveto{\pgfqpoint{4.122810in}{1.096191in}}{\pgfqpoint{4.119976in}{1.089349in}}{\pgfqpoint{4.119976in}{1.082216in}}%
\pgfpathcurveto{\pgfqpoint{4.119976in}{1.075084in}}{\pgfqpoint{4.122810in}{1.068242in}}{\pgfqpoint{4.127854in}{1.063198in}}%
\pgfpathcurveto{\pgfqpoint{4.132897in}{1.058155in}}{\pgfqpoint{4.139739in}{1.055321in}}{\pgfqpoint{4.146872in}{1.055321in}}%
\pgfpathclose%
\pgfusepath{stroke,fill}%
\end{pgfscope}%
\begin{pgfscope}%
\pgfpathrectangle{\pgfqpoint{2.867647in}{0.500000in}}{\pgfqpoint{1.764706in}{1.700000in}}%
\pgfusepath{clip}%
\pgfsetbuttcap%
\pgfsetroundjoin%
\definecolor{currentfill}{rgb}{0.953816,0.463738,0.317699}%
\pgfsetfillcolor{currentfill}%
\pgfsetlinewidth{0.311001pt}%
\definecolor{currentstroke}{rgb}{1.000000,1.000000,1.000000}%
\pgfsetstrokecolor{currentstroke}%
\pgfsetdash{}{0pt}%
\pgfpathmoveto{\pgfqpoint{3.991373in}{1.164643in}}%
\pgfpathcurveto{\pgfqpoint{3.998506in}{1.164643in}}{\pgfqpoint{4.005347in}{1.167476in}}{\pgfqpoint{4.010391in}{1.172520in}}%
\pgfpathcurveto{\pgfqpoint{4.015435in}{1.177564in}}{\pgfqpoint{4.018269in}{1.184405in}}{\pgfqpoint{4.018269in}{1.191538in}}%
\pgfpathcurveto{\pgfqpoint{4.018269in}{1.198671in}}{\pgfqpoint{4.015435in}{1.205513in}}{\pgfqpoint{4.010391in}{1.210556in}}%
\pgfpathcurveto{\pgfqpoint{4.005347in}{1.215600in}}{\pgfqpoint{3.998506in}{1.218434in}}{\pgfqpoint{3.991373in}{1.218434in}}%
\pgfpathcurveto{\pgfqpoint{3.984240in}{1.218434in}}{\pgfqpoint{3.977398in}{1.215600in}}{\pgfqpoint{3.972355in}{1.210556in}}%
\pgfpathcurveto{\pgfqpoint{3.967311in}{1.205513in}}{\pgfqpoint{3.964477in}{1.198671in}}{\pgfqpoint{3.964477in}{1.191538in}}%
\pgfpathcurveto{\pgfqpoint{3.964477in}{1.184405in}}{\pgfqpoint{3.967311in}{1.177564in}}{\pgfqpoint{3.972355in}{1.172520in}}%
\pgfpathcurveto{\pgfqpoint{3.977398in}{1.167476in}}{\pgfqpoint{3.984240in}{1.164643in}}{\pgfqpoint{3.991373in}{1.164643in}}%
\pgfpathclose%
\pgfusepath{stroke,fill}%
\end{pgfscope}%
\begin{pgfscope}%
\pgfpathrectangle{\pgfqpoint{2.867647in}{0.500000in}}{\pgfqpoint{1.764706in}{1.700000in}}%
\pgfusepath{clip}%
\pgfsetbuttcap%
\pgfsetroundjoin%
\definecolor{currentfill}{rgb}{0.953816,0.463738,0.317699}%
\pgfsetfillcolor{currentfill}%
\pgfsetlinewidth{0.311001pt}%
\definecolor{currentstroke}{rgb}{1.000000,1.000000,1.000000}%
\pgfsetstrokecolor{currentstroke}%
\pgfsetdash{}{0pt}%
\pgfpathmoveto{\pgfqpoint{4.227271in}{0.918567in}}%
\pgfpathcurveto{\pgfqpoint{4.234404in}{0.918567in}}{\pgfqpoint{4.241246in}{0.921401in}}{\pgfqpoint{4.246290in}{0.926445in}}%
\pgfpathcurveto{\pgfqpoint{4.251333in}{0.931488in}}{\pgfqpoint{4.254167in}{0.938330in}}{\pgfqpoint{4.254167in}{0.945463in}}%
\pgfpathcurveto{\pgfqpoint{4.254167in}{0.952596in}}{\pgfqpoint{4.251333in}{0.959437in}}{\pgfqpoint{4.246290in}{0.964481in}}%
\pgfpathcurveto{\pgfqpoint{4.241246in}{0.969525in}}{\pgfqpoint{4.234404in}{0.972359in}}{\pgfqpoint{4.227271in}{0.972359in}}%
\pgfpathcurveto{\pgfqpoint{4.220139in}{0.972359in}}{\pgfqpoint{4.213297in}{0.969525in}}{\pgfqpoint{4.208253in}{0.964481in}}%
\pgfpathcurveto{\pgfqpoint{4.203210in}{0.959437in}}{\pgfqpoint{4.200376in}{0.952596in}}{\pgfqpoint{4.200376in}{0.945463in}}%
\pgfpathcurveto{\pgfqpoint{4.200376in}{0.938330in}}{\pgfqpoint{4.203210in}{0.931488in}}{\pgfqpoint{4.208253in}{0.926445in}}%
\pgfpathcurveto{\pgfqpoint{4.213297in}{0.921401in}}{\pgfqpoint{4.220139in}{0.918567in}}{\pgfqpoint{4.227271in}{0.918567in}}%
\pgfpathclose%
\pgfusepath{stroke,fill}%
\end{pgfscope}%
\begin{pgfscope}%
\pgfpathrectangle{\pgfqpoint{2.867647in}{0.500000in}}{\pgfqpoint{1.764706in}{1.700000in}}%
\pgfusepath{clip}%
\pgfsetbuttcap%
\pgfsetroundjoin%
\definecolor{currentfill}{rgb}{0.967735,0.780441,0.659127}%
\pgfsetfillcolor{currentfill}%
\pgfsetlinewidth{0.311001pt}%
\definecolor{currentstroke}{rgb}{1.000000,1.000000,1.000000}%
\pgfsetstrokecolor{currentstroke}%
\pgfsetdash{}{0pt}%
\pgfpathmoveto{\pgfqpoint{4.236986in}{1.569901in}}%
\pgfpathcurveto{\pgfqpoint{4.244119in}{1.569901in}}{\pgfqpoint{4.250960in}{1.572735in}}{\pgfqpoint{4.256004in}{1.577778in}}%
\pgfpathcurveto{\pgfqpoint{4.261048in}{1.582822in}}{\pgfqpoint{4.263882in}{1.589664in}}{\pgfqpoint{4.263882in}{1.596797in}}%
\pgfpathcurveto{\pgfqpoint{4.263882in}{1.603929in}}{\pgfqpoint{4.261048in}{1.610771in}}{\pgfqpoint{4.256004in}{1.615815in}}%
\pgfpathcurveto{\pgfqpoint{4.250960in}{1.620858in}}{\pgfqpoint{4.244119in}{1.623692in}}{\pgfqpoint{4.236986in}{1.623692in}}%
\pgfpathcurveto{\pgfqpoint{4.229853in}{1.623692in}}{\pgfqpoint{4.223011in}{1.620858in}}{\pgfqpoint{4.217968in}{1.615815in}}%
\pgfpathcurveto{\pgfqpoint{4.212924in}{1.610771in}}{\pgfqpoint{4.210090in}{1.603929in}}{\pgfqpoint{4.210090in}{1.596797in}}%
\pgfpathcurveto{\pgfqpoint{4.210090in}{1.589664in}}{\pgfqpoint{4.212924in}{1.582822in}}{\pgfqpoint{4.217968in}{1.577778in}}%
\pgfpathcurveto{\pgfqpoint{4.223011in}{1.572735in}}{\pgfqpoint{4.229853in}{1.569901in}}{\pgfqpoint{4.236986in}{1.569901in}}%
\pgfpathclose%
\pgfusepath{stroke,fill}%
\end{pgfscope}%
\begin{pgfscope}%
\pgfpathrectangle{\pgfqpoint{2.867647in}{0.500000in}}{\pgfqpoint{1.764706in}{1.700000in}}%
\pgfusepath{clip}%
\pgfsetbuttcap%
\pgfsetroundjoin%
\definecolor{currentfill}{rgb}{0.979891,0.908948,0.848279}%
\pgfsetfillcolor{currentfill}%
\pgfsetlinewidth{0.311001pt}%
\definecolor{currentstroke}{rgb}{1.000000,1.000000,1.000000}%
\pgfsetstrokecolor{currentstroke}%
\pgfsetdash{}{0pt}%
\pgfpathmoveto{\pgfqpoint{4.138895in}{1.139224in}}%
\pgfpathcurveto{\pgfqpoint{4.146028in}{1.139224in}}{\pgfqpoint{4.152869in}{1.142057in}}{\pgfqpoint{4.157913in}{1.147101in}}%
\pgfpathcurveto{\pgfqpoint{4.162957in}{1.152145in}}{\pgfqpoint{4.165791in}{1.158986in}}{\pgfqpoint{4.165791in}{1.166119in}}%
\pgfpathcurveto{\pgfqpoint{4.165791in}{1.173252in}}{\pgfqpoint{4.162957in}{1.180094in}}{\pgfqpoint{4.157913in}{1.185137in}}%
\pgfpathcurveto{\pgfqpoint{4.152869in}{1.190181in}}{\pgfqpoint{4.146028in}{1.193015in}}{\pgfqpoint{4.138895in}{1.193015in}}%
\pgfpathcurveto{\pgfqpoint{4.131762in}{1.193015in}}{\pgfqpoint{4.124920in}{1.190181in}}{\pgfqpoint{4.119877in}{1.185137in}}%
\pgfpathcurveto{\pgfqpoint{4.114833in}{1.180094in}}{\pgfqpoint{4.111999in}{1.173252in}}{\pgfqpoint{4.111999in}{1.166119in}}%
\pgfpathcurveto{\pgfqpoint{4.111999in}{1.158986in}}{\pgfqpoint{4.114833in}{1.152145in}}{\pgfqpoint{4.119877in}{1.147101in}}%
\pgfpathcurveto{\pgfqpoint{4.124920in}{1.142057in}}{\pgfqpoint{4.131762in}{1.139224in}}{\pgfqpoint{4.138895in}{1.139224in}}%
\pgfpathclose%
\pgfusepath{stroke,fill}%
\end{pgfscope}%
\begin{pgfscope}%
\pgfpathrectangle{\pgfqpoint{2.867647in}{0.500000in}}{\pgfqpoint{1.764706in}{1.700000in}}%
\pgfusepath{clip}%
\pgfsetbuttcap%
\pgfsetroundjoin%
\definecolor{currentfill}{rgb}{0.973271,0.850724,0.762998}%
\pgfsetfillcolor{currentfill}%
\pgfsetlinewidth{0.311001pt}%
\definecolor{currentstroke}{rgb}{1.000000,1.000000,1.000000}%
\pgfsetstrokecolor{currentstroke}%
\pgfsetdash{}{0pt}%
\pgfpathmoveto{\pgfqpoint{4.107146in}{1.193531in}}%
\pgfpathcurveto{\pgfqpoint{4.114279in}{1.193531in}}{\pgfqpoint{4.121121in}{1.196365in}}{\pgfqpoint{4.126164in}{1.201409in}}%
\pgfpathcurveto{\pgfqpoint{4.131208in}{1.206452in}}{\pgfqpoint{4.134042in}{1.213294in}}{\pgfqpoint{4.134042in}{1.220427in}}%
\pgfpathcurveto{\pgfqpoint{4.134042in}{1.227560in}}{\pgfqpoint{4.131208in}{1.234401in}}{\pgfqpoint{4.126164in}{1.239445in}}%
\pgfpathcurveto{\pgfqpoint{4.121121in}{1.244489in}}{\pgfqpoint{4.114279in}{1.247322in}}{\pgfqpoint{4.107146in}{1.247322in}}%
\pgfpathcurveto{\pgfqpoint{4.100013in}{1.247322in}}{\pgfqpoint{4.093172in}{1.244489in}}{\pgfqpoint{4.088128in}{1.239445in}}%
\pgfpathcurveto{\pgfqpoint{4.083084in}{1.234401in}}{\pgfqpoint{4.080251in}{1.227560in}}{\pgfqpoint{4.080251in}{1.220427in}}%
\pgfpathcurveto{\pgfqpoint{4.080251in}{1.213294in}}{\pgfqpoint{4.083084in}{1.206452in}}{\pgfqpoint{4.088128in}{1.201409in}}%
\pgfpathcurveto{\pgfqpoint{4.093172in}{1.196365in}}{\pgfqpoint{4.100013in}{1.193531in}}{\pgfqpoint{4.107146in}{1.193531in}}%
\pgfpathclose%
\pgfusepath{stroke,fill}%
\end{pgfscope}%
\begin{pgfscope}%
\pgfpathrectangle{\pgfqpoint{2.867647in}{0.500000in}}{\pgfqpoint{1.764706in}{1.700000in}}%
\pgfusepath{clip}%
\pgfsetbuttcap%
\pgfsetroundjoin%
\definecolor{currentfill}{rgb}{0.979891,0.908948,0.848279}%
\pgfsetfillcolor{currentfill}%
\pgfsetlinewidth{0.311001pt}%
\definecolor{currentstroke}{rgb}{1.000000,1.000000,1.000000}%
\pgfsetstrokecolor{currentstroke}%
\pgfsetdash{}{0pt}%
\pgfpathmoveto{\pgfqpoint{4.220020in}{1.279385in}}%
\pgfpathcurveto{\pgfqpoint{4.227153in}{1.279385in}}{\pgfqpoint{4.233995in}{1.282218in}}{\pgfqpoint{4.239038in}{1.287262in}}%
\pgfpathcurveto{\pgfqpoint{4.244082in}{1.292306in}}{\pgfqpoint{4.246916in}{1.299147in}}{\pgfqpoint{4.246916in}{1.306280in}}%
\pgfpathcurveto{\pgfqpoint{4.246916in}{1.313413in}}{\pgfqpoint{4.244082in}{1.320255in}}{\pgfqpoint{4.239038in}{1.325298in}}%
\pgfpathcurveto{\pgfqpoint{4.233995in}{1.330342in}}{\pgfqpoint{4.227153in}{1.333176in}}{\pgfqpoint{4.220020in}{1.333176in}}%
\pgfpathcurveto{\pgfqpoint{4.212887in}{1.333176in}}{\pgfqpoint{4.206046in}{1.330342in}}{\pgfqpoint{4.201002in}{1.325298in}}%
\pgfpathcurveto{\pgfqpoint{4.195959in}{1.320255in}}{\pgfqpoint{4.193125in}{1.313413in}}{\pgfqpoint{4.193125in}{1.306280in}}%
\pgfpathcurveto{\pgfqpoint{4.193125in}{1.299147in}}{\pgfqpoint{4.195959in}{1.292306in}}{\pgfqpoint{4.201002in}{1.287262in}}%
\pgfpathcurveto{\pgfqpoint{4.206046in}{1.282218in}}{\pgfqpoint{4.212887in}{1.279385in}}{\pgfqpoint{4.220020in}{1.279385in}}%
\pgfpathclose%
\pgfusepath{stroke,fill}%
\end{pgfscope}%
\begin{pgfscope}%
\pgfpathrectangle{\pgfqpoint{2.867647in}{0.500000in}}{\pgfqpoint{1.764706in}{1.700000in}}%
\pgfusepath{clip}%
\pgfsetbuttcap%
\pgfsetroundjoin%
\definecolor{currentfill}{rgb}{0.965592,0.726236,0.584384}%
\pgfsetfillcolor{currentfill}%
\pgfsetlinewidth{0.311001pt}%
\definecolor{currentstroke}{rgb}{1.000000,1.000000,1.000000}%
\pgfsetstrokecolor{currentstroke}%
\pgfsetdash{}{0pt}%
\pgfpathmoveto{\pgfqpoint{4.028622in}{1.742040in}}%
\pgfpathcurveto{\pgfqpoint{4.035755in}{1.742040in}}{\pgfqpoint{4.042597in}{1.744874in}}{\pgfqpoint{4.047641in}{1.749917in}}%
\pgfpathcurveto{\pgfqpoint{4.052684in}{1.754961in}}{\pgfqpoint{4.055518in}{1.761803in}}{\pgfqpoint{4.055518in}{1.768935in}}%
\pgfpathcurveto{\pgfqpoint{4.055518in}{1.776068in}}{\pgfqpoint{4.052684in}{1.782910in}}{\pgfqpoint{4.047641in}{1.787954in}}%
\pgfpathcurveto{\pgfqpoint{4.042597in}{1.792997in}}{\pgfqpoint{4.035755in}{1.795831in}}{\pgfqpoint{4.028622in}{1.795831in}}%
\pgfpathcurveto{\pgfqpoint{4.021490in}{1.795831in}}{\pgfqpoint{4.014648in}{1.792997in}}{\pgfqpoint{4.009604in}{1.787954in}}%
\pgfpathcurveto{\pgfqpoint{4.004561in}{1.782910in}}{\pgfqpoint{4.001727in}{1.776068in}}{\pgfqpoint{4.001727in}{1.768935in}}%
\pgfpathcurveto{\pgfqpoint{4.001727in}{1.761803in}}{\pgfqpoint{4.004561in}{1.754961in}}{\pgfqpoint{4.009604in}{1.749917in}}%
\pgfpathcurveto{\pgfqpoint{4.014648in}{1.744874in}}{\pgfqpoint{4.021490in}{1.742040in}}{\pgfqpoint{4.028622in}{1.742040in}}%
\pgfpathclose%
\pgfusepath{stroke,fill}%
\end{pgfscope}%
\begin{pgfscope}%
\pgfpathrectangle{\pgfqpoint{2.867647in}{0.500000in}}{\pgfqpoint{1.764706in}{1.700000in}}%
\pgfusepath{clip}%
\pgfsetbuttcap%
\pgfsetroundjoin%
\definecolor{currentfill}{rgb}{0.964558,0.676556,0.522514}%
\pgfsetfillcolor{currentfill}%
\pgfsetlinewidth{0.311001pt}%
\definecolor{currentstroke}{rgb}{1.000000,1.000000,1.000000}%
\pgfsetstrokecolor{currentstroke}%
\pgfsetdash{}{0pt}%
\pgfpathmoveto{\pgfqpoint{4.128845in}{1.751645in}}%
\pgfpathcurveto{\pgfqpoint{4.135978in}{1.751645in}}{\pgfqpoint{4.142820in}{1.754479in}}{\pgfqpoint{4.147863in}{1.759523in}}%
\pgfpathcurveto{\pgfqpoint{4.152907in}{1.764567in}}{\pgfqpoint{4.155741in}{1.771408in}}{\pgfqpoint{4.155741in}{1.778541in}}%
\pgfpathcurveto{\pgfqpoint{4.155741in}{1.785674in}}{\pgfqpoint{4.152907in}{1.792516in}}{\pgfqpoint{4.147863in}{1.797559in}}%
\pgfpathcurveto{\pgfqpoint{4.142820in}{1.802603in}}{\pgfqpoint{4.135978in}{1.805437in}}{\pgfqpoint{4.128845in}{1.805437in}}%
\pgfpathcurveto{\pgfqpoint{4.121712in}{1.805437in}}{\pgfqpoint{4.114871in}{1.802603in}}{\pgfqpoint{4.109827in}{1.797559in}}%
\pgfpathcurveto{\pgfqpoint{4.104784in}{1.792516in}}{\pgfqpoint{4.101950in}{1.785674in}}{\pgfqpoint{4.101950in}{1.778541in}}%
\pgfpathcurveto{\pgfqpoint{4.101950in}{1.771408in}}{\pgfqpoint{4.104784in}{1.764567in}}{\pgfqpoint{4.109827in}{1.759523in}}%
\pgfpathcurveto{\pgfqpoint{4.114871in}{1.754479in}}{\pgfqpoint{4.121712in}{1.751645in}}{\pgfqpoint{4.128845in}{1.751645in}}%
\pgfpathclose%
\pgfusepath{stroke,fill}%
\end{pgfscope}%
\begin{pgfscope}%
\pgfpathrectangle{\pgfqpoint{2.867647in}{0.500000in}}{\pgfqpoint{1.764706in}{1.700000in}}%
\pgfusepath{clip}%
\pgfsetbuttcap%
\pgfsetroundjoin%
\definecolor{currentfill}{rgb}{0.152013,0.081591,0.202705}%
\pgfsetfillcolor{currentfill}%
\pgfsetlinewidth{0.803000pt}%
\definecolor{currentstroke}{rgb}{0.152013,0.081591,0.202705}%
\pgfsetstrokecolor{currentstroke}%
\pgfsetdash{}{0pt}%
\pgfsys@defobject{currentmarker}{\pgfqpoint{-0.033333in}{-0.033333in}}{\pgfqpoint{0.033333in}{0.033333in}}{%
\pgfpathmoveto{\pgfqpoint{0.000000in}{-0.033333in}}%
\pgfpathcurveto{\pgfqpoint{0.008840in}{-0.033333in}}{\pgfqpoint{0.017319in}{-0.029821in}}{\pgfqpoint{0.023570in}{-0.023570in}}%
\pgfpathcurveto{\pgfqpoint{0.029821in}{-0.017319in}}{\pgfqpoint{0.033333in}{-0.008840in}}{\pgfqpoint{0.033333in}{0.000000in}}%
\pgfpathcurveto{\pgfqpoint{0.033333in}{0.008840in}}{\pgfqpoint{0.029821in}{0.017319in}}{\pgfqpoint{0.023570in}{0.023570in}}%
\pgfpathcurveto{\pgfqpoint{0.017319in}{0.029821in}}{\pgfqpoint{0.008840in}{0.033333in}}{\pgfqpoint{0.000000in}{0.033333in}}%
\pgfpathcurveto{\pgfqpoint{-0.008840in}{0.033333in}}{\pgfqpoint{-0.017319in}{0.029821in}}{\pgfqpoint{-0.023570in}{0.023570in}}%
\pgfpathcurveto{\pgfqpoint{-0.029821in}{0.017319in}}{\pgfqpoint{-0.033333in}{0.008840in}}{\pgfqpoint{-0.033333in}{0.000000in}}%
\pgfpathcurveto{\pgfqpoint{-0.033333in}{-0.008840in}}{\pgfqpoint{-0.029821in}{-0.017319in}}{\pgfqpoint{-0.023570in}{-0.023570in}}%
\pgfpathcurveto{\pgfqpoint{-0.017319in}{-0.029821in}}{\pgfqpoint{-0.008840in}{-0.033333in}}{\pgfqpoint{0.000000in}{-0.033333in}}%
\pgfpathclose%
\pgfusepath{stroke,fill}%
}%
\end{pgfscope}%
\begin{pgfscope}%
\pgfpathrectangle{\pgfqpoint{2.867647in}{0.500000in}}{\pgfqpoint{1.764706in}{1.700000in}}%
\pgfusepath{clip}%
\pgfsetbuttcap%
\pgfsetroundjoin%
\definecolor{currentfill}{rgb}{0.419253,0.121738,0.336404}%
\pgfsetfillcolor{currentfill}%
\pgfsetlinewidth{0.803000pt}%
\definecolor{currentstroke}{rgb}{0.419253,0.121738,0.336404}%
\pgfsetstrokecolor{currentstroke}%
\pgfsetdash{}{0pt}%
\pgfsys@defobject{currentmarker}{\pgfqpoint{-0.033333in}{-0.033333in}}{\pgfqpoint{0.033333in}{0.033333in}}{%
\pgfpathmoveto{\pgfqpoint{0.000000in}{-0.033333in}}%
\pgfpathcurveto{\pgfqpoint{0.008840in}{-0.033333in}}{\pgfqpoint{0.017319in}{-0.029821in}}{\pgfqpoint{0.023570in}{-0.023570in}}%
\pgfpathcurveto{\pgfqpoint{0.029821in}{-0.017319in}}{\pgfqpoint{0.033333in}{-0.008840in}}{\pgfqpoint{0.033333in}{0.000000in}}%
\pgfpathcurveto{\pgfqpoint{0.033333in}{0.008840in}}{\pgfqpoint{0.029821in}{0.017319in}}{\pgfqpoint{0.023570in}{0.023570in}}%
\pgfpathcurveto{\pgfqpoint{0.017319in}{0.029821in}}{\pgfqpoint{0.008840in}{0.033333in}}{\pgfqpoint{0.000000in}{0.033333in}}%
\pgfpathcurveto{\pgfqpoint{-0.008840in}{0.033333in}}{\pgfqpoint{-0.017319in}{0.029821in}}{\pgfqpoint{-0.023570in}{0.023570in}}%
\pgfpathcurveto{\pgfqpoint{-0.029821in}{0.017319in}}{\pgfqpoint{-0.033333in}{0.008840in}}{\pgfqpoint{-0.033333in}{0.000000in}}%
\pgfpathcurveto{\pgfqpoint{-0.033333in}{-0.008840in}}{\pgfqpoint{-0.029821in}{-0.017319in}}{\pgfqpoint{-0.023570in}{-0.023570in}}%
\pgfpathcurveto{\pgfqpoint{-0.017319in}{-0.029821in}}{\pgfqpoint{-0.008840in}{-0.033333in}}{\pgfqpoint{0.000000in}{-0.033333in}}%
\pgfpathclose%
\pgfusepath{stroke,fill}%
}%
\end{pgfscope}%
\begin{pgfscope}%
\pgfpathrectangle{\pgfqpoint{2.867647in}{0.500000in}}{\pgfqpoint{1.764706in}{1.700000in}}%
\pgfusepath{clip}%
\pgfsetbuttcap%
\pgfsetroundjoin%
\definecolor{currentfill}{rgb}{0.711081,0.087612,0.343064}%
\pgfsetfillcolor{currentfill}%
\pgfsetlinewidth{0.803000pt}%
\definecolor{currentstroke}{rgb}{0.711081,0.087612,0.343064}%
\pgfsetstrokecolor{currentstroke}%
\pgfsetdash{}{0pt}%
\pgfsys@defobject{currentmarker}{\pgfqpoint{-0.033333in}{-0.033333in}}{\pgfqpoint{0.033333in}{0.033333in}}{%
\pgfpathmoveto{\pgfqpoint{0.000000in}{-0.033333in}}%
\pgfpathcurveto{\pgfqpoint{0.008840in}{-0.033333in}}{\pgfqpoint{0.017319in}{-0.029821in}}{\pgfqpoint{0.023570in}{-0.023570in}}%
\pgfpathcurveto{\pgfqpoint{0.029821in}{-0.017319in}}{\pgfqpoint{0.033333in}{-0.008840in}}{\pgfqpoint{0.033333in}{0.000000in}}%
\pgfpathcurveto{\pgfqpoint{0.033333in}{0.008840in}}{\pgfqpoint{0.029821in}{0.017319in}}{\pgfqpoint{0.023570in}{0.023570in}}%
\pgfpathcurveto{\pgfqpoint{0.017319in}{0.029821in}}{\pgfqpoint{0.008840in}{0.033333in}}{\pgfqpoint{0.000000in}{0.033333in}}%
\pgfpathcurveto{\pgfqpoint{-0.008840in}{0.033333in}}{\pgfqpoint{-0.017319in}{0.029821in}}{\pgfqpoint{-0.023570in}{0.023570in}}%
\pgfpathcurveto{\pgfqpoint{-0.029821in}{0.017319in}}{\pgfqpoint{-0.033333in}{0.008840in}}{\pgfqpoint{-0.033333in}{0.000000in}}%
\pgfpathcurveto{\pgfqpoint{-0.033333in}{-0.008840in}}{\pgfqpoint{-0.029821in}{-0.017319in}}{\pgfqpoint{-0.023570in}{-0.023570in}}%
\pgfpathcurveto{\pgfqpoint{-0.017319in}{-0.029821in}}{\pgfqpoint{-0.008840in}{-0.033333in}}{\pgfqpoint{0.000000in}{-0.033333in}}%
\pgfpathclose%
\pgfusepath{stroke,fill}%
}%
\end{pgfscope}%
\begin{pgfscope}%
\pgfpathrectangle{\pgfqpoint{2.867647in}{0.500000in}}{\pgfqpoint{1.764706in}{1.700000in}}%
\pgfusepath{clip}%
\pgfsetbuttcap%
\pgfsetroundjoin%
\definecolor{currentfill}{rgb}{0.922239,0.282873,0.242296}%
\pgfsetfillcolor{currentfill}%
\pgfsetlinewidth{0.803000pt}%
\definecolor{currentstroke}{rgb}{0.922239,0.282873,0.242296}%
\pgfsetstrokecolor{currentstroke}%
\pgfsetdash{}{0pt}%
\pgfsys@defobject{currentmarker}{\pgfqpoint{-0.033333in}{-0.033333in}}{\pgfqpoint{0.033333in}{0.033333in}}{%
\pgfpathmoveto{\pgfqpoint{0.000000in}{-0.033333in}}%
\pgfpathcurveto{\pgfqpoint{0.008840in}{-0.033333in}}{\pgfqpoint{0.017319in}{-0.029821in}}{\pgfqpoint{0.023570in}{-0.023570in}}%
\pgfpathcurveto{\pgfqpoint{0.029821in}{-0.017319in}}{\pgfqpoint{0.033333in}{-0.008840in}}{\pgfqpoint{0.033333in}{0.000000in}}%
\pgfpathcurveto{\pgfqpoint{0.033333in}{0.008840in}}{\pgfqpoint{0.029821in}{0.017319in}}{\pgfqpoint{0.023570in}{0.023570in}}%
\pgfpathcurveto{\pgfqpoint{0.017319in}{0.029821in}}{\pgfqpoint{0.008840in}{0.033333in}}{\pgfqpoint{0.000000in}{0.033333in}}%
\pgfpathcurveto{\pgfqpoint{-0.008840in}{0.033333in}}{\pgfqpoint{-0.017319in}{0.029821in}}{\pgfqpoint{-0.023570in}{0.023570in}}%
\pgfpathcurveto{\pgfqpoint{-0.029821in}{0.017319in}}{\pgfqpoint{-0.033333in}{0.008840in}}{\pgfqpoint{-0.033333in}{0.000000in}}%
\pgfpathcurveto{\pgfqpoint{-0.033333in}{-0.008840in}}{\pgfqpoint{-0.029821in}{-0.017319in}}{\pgfqpoint{-0.023570in}{-0.023570in}}%
\pgfpathcurveto{\pgfqpoint{-0.017319in}{-0.029821in}}{\pgfqpoint{-0.008840in}{-0.033333in}}{\pgfqpoint{0.000000in}{-0.033333in}}%
\pgfpathclose%
\pgfusepath{stroke,fill}%
}%
\end{pgfscope}%
\begin{pgfscope}%
\pgfpathrectangle{\pgfqpoint{2.867647in}{0.500000in}}{\pgfqpoint{1.764706in}{1.700000in}}%
\pgfusepath{clip}%
\pgfsetbuttcap%
\pgfsetroundjoin%
\definecolor{currentfill}{rgb}{0.962532,0.599594,0.438051}%
\pgfsetfillcolor{currentfill}%
\pgfsetlinewidth{0.803000pt}%
\definecolor{currentstroke}{rgb}{0.962532,0.599594,0.438051}%
\pgfsetstrokecolor{currentstroke}%
\pgfsetdash{}{0pt}%
\pgfsys@defobject{currentmarker}{\pgfqpoint{-0.033333in}{-0.033333in}}{\pgfqpoint{0.033333in}{0.033333in}}{%
\pgfpathmoveto{\pgfqpoint{0.000000in}{-0.033333in}}%
\pgfpathcurveto{\pgfqpoint{0.008840in}{-0.033333in}}{\pgfqpoint{0.017319in}{-0.029821in}}{\pgfqpoint{0.023570in}{-0.023570in}}%
\pgfpathcurveto{\pgfqpoint{0.029821in}{-0.017319in}}{\pgfqpoint{0.033333in}{-0.008840in}}{\pgfqpoint{0.033333in}{0.000000in}}%
\pgfpathcurveto{\pgfqpoint{0.033333in}{0.008840in}}{\pgfqpoint{0.029821in}{0.017319in}}{\pgfqpoint{0.023570in}{0.023570in}}%
\pgfpathcurveto{\pgfqpoint{0.017319in}{0.029821in}}{\pgfqpoint{0.008840in}{0.033333in}}{\pgfqpoint{0.000000in}{0.033333in}}%
\pgfpathcurveto{\pgfqpoint{-0.008840in}{0.033333in}}{\pgfqpoint{-0.017319in}{0.029821in}}{\pgfqpoint{-0.023570in}{0.023570in}}%
\pgfpathcurveto{\pgfqpoint{-0.029821in}{0.017319in}}{\pgfqpoint{-0.033333in}{0.008840in}}{\pgfqpoint{-0.033333in}{0.000000in}}%
\pgfpathcurveto{\pgfqpoint{-0.033333in}{-0.008840in}}{\pgfqpoint{-0.029821in}{-0.017319in}}{\pgfqpoint{-0.023570in}{-0.023570in}}%
\pgfpathcurveto{\pgfqpoint{-0.017319in}{-0.029821in}}{\pgfqpoint{-0.008840in}{-0.033333in}}{\pgfqpoint{0.000000in}{-0.033333in}}%
\pgfpathclose%
\pgfusepath{stroke,fill}%
}%
\end{pgfscope}%
\begin{pgfscope}%
\pgfpathrectangle{\pgfqpoint{2.867647in}{0.500000in}}{\pgfqpoint{1.764706in}{1.700000in}}%
\pgfusepath{clip}%
\pgfsetbuttcap%
\pgfsetroundjoin%
\definecolor{currentfill}{rgb}{0.975644,0.874038,0.797253}%
\pgfsetfillcolor{currentfill}%
\pgfsetlinewidth{0.803000pt}%
\definecolor{currentstroke}{rgb}{0.975644,0.874038,0.797253}%
\pgfsetstrokecolor{currentstroke}%
\pgfsetdash{}{0pt}%
\pgfsys@defobject{currentmarker}{\pgfqpoint{-0.033333in}{-0.033333in}}{\pgfqpoint{0.033333in}{0.033333in}}{%
\pgfpathmoveto{\pgfqpoint{0.000000in}{-0.033333in}}%
\pgfpathcurveto{\pgfqpoint{0.008840in}{-0.033333in}}{\pgfqpoint{0.017319in}{-0.029821in}}{\pgfqpoint{0.023570in}{-0.023570in}}%
\pgfpathcurveto{\pgfqpoint{0.029821in}{-0.017319in}}{\pgfqpoint{0.033333in}{-0.008840in}}{\pgfqpoint{0.033333in}{0.000000in}}%
\pgfpathcurveto{\pgfqpoint{0.033333in}{0.008840in}}{\pgfqpoint{0.029821in}{0.017319in}}{\pgfqpoint{0.023570in}{0.023570in}}%
\pgfpathcurveto{\pgfqpoint{0.017319in}{0.029821in}}{\pgfqpoint{0.008840in}{0.033333in}}{\pgfqpoint{0.000000in}{0.033333in}}%
\pgfpathcurveto{\pgfqpoint{-0.008840in}{0.033333in}}{\pgfqpoint{-0.017319in}{0.029821in}}{\pgfqpoint{-0.023570in}{0.023570in}}%
\pgfpathcurveto{\pgfqpoint{-0.029821in}{0.017319in}}{\pgfqpoint{-0.033333in}{0.008840in}}{\pgfqpoint{-0.033333in}{0.000000in}}%
\pgfpathcurveto{\pgfqpoint{-0.033333in}{-0.008840in}}{\pgfqpoint{-0.029821in}{-0.017319in}}{\pgfqpoint{-0.023570in}{-0.023570in}}%
\pgfpathcurveto{\pgfqpoint{-0.017319in}{-0.029821in}}{\pgfqpoint{-0.008840in}{-0.033333in}}{\pgfqpoint{0.000000in}{-0.033333in}}%
\pgfpathclose%
\pgfusepath{stroke,fill}%
}%
\end{pgfscope}%
\begin{pgfscope}%
\pgfsetrectcap%
\pgfsetmiterjoin%
\pgfsetlinewidth{1.003750pt}%
\definecolor{currentstroke}{rgb}{0.150000,0.150000,0.150000}%
\pgfsetstrokecolor{currentstroke}%
\pgfsetdash{}{0pt}%
\pgfpathmoveto{\pgfqpoint{2.867647in}{0.500000in}}%
\pgfpathlineto{\pgfqpoint{2.867647in}{2.200000in}}%
\pgfusepath{stroke}%
\end{pgfscope}%
\begin{pgfscope}%
\pgfsetrectcap%
\pgfsetmiterjoin%
\pgfsetlinewidth{1.003750pt}%
\definecolor{currentstroke}{rgb}{0.150000,0.150000,0.150000}%
\pgfsetstrokecolor{currentstroke}%
\pgfsetdash{}{0pt}%
\pgfpathmoveto{\pgfqpoint{4.632353in}{0.500000in}}%
\pgfpathlineto{\pgfqpoint{4.632353in}{2.200000in}}%
\pgfusepath{stroke}%
\end{pgfscope}%
\begin{pgfscope}%
\pgfsetrectcap%
\pgfsetmiterjoin%
\pgfsetlinewidth{1.003750pt}%
\definecolor{currentstroke}{rgb}{0.150000,0.150000,0.150000}%
\pgfsetstrokecolor{currentstroke}%
\pgfsetdash{}{0pt}%
\pgfpathmoveto{\pgfqpoint{2.867647in}{0.500000in}}%
\pgfpathlineto{\pgfqpoint{4.632353in}{0.500000in}}%
\pgfusepath{stroke}%
\end{pgfscope}%
\begin{pgfscope}%
\pgfsetrectcap%
\pgfsetmiterjoin%
\pgfsetlinewidth{1.003750pt}%
\definecolor{currentstroke}{rgb}{0.150000,0.150000,0.150000}%
\pgfsetstrokecolor{currentstroke}%
\pgfsetdash{}{0pt}%
\pgfpathmoveto{\pgfqpoint{2.867647in}{2.200000in}}%
\pgfpathlineto{\pgfqpoint{4.632353in}{2.200000in}}%
\pgfusepath{stroke}%
\end{pgfscope}%
\begin{pgfscope}%
\definecolor{textcolor}{rgb}{0.150000,0.150000,0.150000}%
\pgfsetstrokecolor{textcolor}%
\pgfsetfillcolor{textcolor}%
\pgftext[x=3.750000in,y=2.283333in,,base]{\color{textcolor}\rmfamily\fontsize{9.600000}{11.520000}\selectfont Iteration 5000}%
\end{pgfscope}%
\begin{pgfscope}%
\pgfsetbuttcap%
\pgfsetmiterjoin%
\definecolor{currentfill}{rgb}{1.000000,1.000000,1.000000}%
\pgfsetfillcolor{currentfill}%
\pgfsetlinewidth{0.000000pt}%
\definecolor{currentstroke}{rgb}{0.000000,0.000000,0.000000}%
\pgfsetstrokecolor{currentstroke}%
\pgfsetstrokeopacity{0.000000}%
\pgfsetdash{}{0pt}%
\pgfpathmoveto{\pgfqpoint{4.985294in}{0.500000in}}%
\pgfpathlineto{\pgfqpoint{6.750000in}{0.500000in}}%
\pgfpathlineto{\pgfqpoint{6.750000in}{2.200000in}}%
\pgfpathlineto{\pgfqpoint{4.985294in}{2.200000in}}%
\pgfpathclose%
\pgfusepath{fill}%
\end{pgfscope}%
\begin{pgfscope}%
\pgfpathrectangle{\pgfqpoint{4.985294in}{0.500000in}}{\pgfqpoint{1.764706in}{1.700000in}}%
\pgfusepath{clip}%
\pgfsetbuttcap%
\pgfsetroundjoin%
\definecolor{currentfill}{rgb}{0.975018,0.868213,0.788710}%
\pgfsetfillcolor{currentfill}%
\pgfsetlinewidth{0.311001pt}%
\definecolor{currentstroke}{rgb}{1.000000,1.000000,1.000000}%
\pgfsetstrokecolor{currentstroke}%
\pgfsetdash{}{0pt}%
\pgfpathmoveto{\pgfqpoint{5.403671in}{1.150775in}}%
\pgfpathcurveto{\pgfqpoint{5.410804in}{1.150775in}}{\pgfqpoint{5.417645in}{1.153609in}}{\pgfqpoint{5.422689in}{1.158653in}}%
\pgfpathcurveto{\pgfqpoint{5.427733in}{1.163696in}}{\pgfqpoint{5.430567in}{1.170538in}}{\pgfqpoint{5.430567in}{1.177671in}}%
\pgfpathcurveto{\pgfqpoint{5.430567in}{1.184804in}}{\pgfqpoint{5.427733in}{1.191645in}}{\pgfqpoint{5.422689in}{1.196689in}}%
\pgfpathcurveto{\pgfqpoint{5.417645in}{1.201733in}}{\pgfqpoint{5.410804in}{1.204567in}}{\pgfqpoint{5.403671in}{1.204567in}}%
\pgfpathcurveto{\pgfqpoint{5.396538in}{1.204567in}}{\pgfqpoint{5.389696in}{1.201733in}}{\pgfqpoint{5.384653in}{1.196689in}}%
\pgfpathcurveto{\pgfqpoint{5.379609in}{1.191645in}}{\pgfqpoint{5.376775in}{1.184804in}}{\pgfqpoint{5.376775in}{1.177671in}}%
\pgfpathcurveto{\pgfqpoint{5.376775in}{1.170538in}}{\pgfqpoint{5.379609in}{1.163696in}}{\pgfqpoint{5.384653in}{1.158653in}}%
\pgfpathcurveto{\pgfqpoint{5.389696in}{1.153609in}}{\pgfqpoint{5.396538in}{1.150775in}}{\pgfqpoint{5.403671in}{1.150775in}}%
\pgfpathclose%
\pgfusepath{stroke,fill}%
\end{pgfscope}%
\begin{pgfscope}%
\pgfpathrectangle{\pgfqpoint{4.985294in}{0.500000in}}{\pgfqpoint{1.764706in}{1.700000in}}%
\pgfusepath{clip}%
\pgfsetbuttcap%
\pgfsetroundjoin%
\definecolor{currentfill}{rgb}{0.970255,0.815666,0.711203}%
\pgfsetfillcolor{currentfill}%
\pgfsetlinewidth{0.311001pt}%
\definecolor{currentstroke}{rgb}{1.000000,1.000000,1.000000}%
\pgfsetstrokecolor{currentstroke}%
\pgfsetdash{}{0pt}%
\pgfpathmoveto{\pgfqpoint{5.359327in}{1.404620in}}%
\pgfpathcurveto{\pgfqpoint{5.366460in}{1.404620in}}{\pgfqpoint{5.373301in}{1.407454in}}{\pgfqpoint{5.378345in}{1.412497in}}%
\pgfpathcurveto{\pgfqpoint{5.383389in}{1.417541in}}{\pgfqpoint{5.386222in}{1.424383in}}{\pgfqpoint{5.386222in}{1.431516in}}%
\pgfpathcurveto{\pgfqpoint{5.386222in}{1.438648in}}{\pgfqpoint{5.383389in}{1.445490in}}{\pgfqpoint{5.378345in}{1.450534in}}%
\pgfpathcurveto{\pgfqpoint{5.373301in}{1.455577in}}{\pgfqpoint{5.366460in}{1.458411in}}{\pgfqpoint{5.359327in}{1.458411in}}%
\pgfpathcurveto{\pgfqpoint{5.352194in}{1.458411in}}{\pgfqpoint{5.345352in}{1.455577in}}{\pgfqpoint{5.340309in}{1.450534in}}%
\pgfpathcurveto{\pgfqpoint{5.335265in}{1.445490in}}{\pgfqpoint{5.332431in}{1.438648in}}{\pgfqpoint{5.332431in}{1.431516in}}%
\pgfpathcurveto{\pgfqpoint{5.332431in}{1.424383in}}{\pgfqpoint{5.335265in}{1.417541in}}{\pgfqpoint{5.340309in}{1.412497in}}%
\pgfpathcurveto{\pgfqpoint{5.345352in}{1.407454in}}{\pgfqpoint{5.352194in}{1.404620in}}{\pgfqpoint{5.359327in}{1.404620in}}%
\pgfpathclose%
\pgfusepath{stroke,fill}%
\end{pgfscope}%
\begin{pgfscope}%
\pgfpathrectangle{\pgfqpoint{4.985294in}{0.500000in}}{\pgfqpoint{1.764706in}{1.700000in}}%
\pgfusepath{clip}%
\pgfsetbuttcap%
\pgfsetroundjoin%
\definecolor{currentfill}{rgb}{0.981377,0.920617,0.865369}%
\pgfsetfillcolor{currentfill}%
\pgfsetlinewidth{0.311001pt}%
\definecolor{currentstroke}{rgb}{1.000000,1.000000,1.000000}%
\pgfsetstrokecolor{currentstroke}%
\pgfsetdash{}{0pt}%
\pgfpathmoveto{\pgfqpoint{6.305872in}{1.282297in}}%
\pgfpathcurveto{\pgfqpoint{6.313005in}{1.282297in}}{\pgfqpoint{6.319847in}{1.285131in}}{\pgfqpoint{6.324891in}{1.290175in}}%
\pgfpathcurveto{\pgfqpoint{6.329934in}{1.295218in}}{\pgfqpoint{6.332768in}{1.302060in}}{\pgfqpoint{6.332768in}{1.309193in}}%
\pgfpathcurveto{\pgfqpoint{6.332768in}{1.316326in}}{\pgfqpoint{6.329934in}{1.323167in}}{\pgfqpoint{6.324891in}{1.328211in}}%
\pgfpathcurveto{\pgfqpoint{6.319847in}{1.333255in}}{\pgfqpoint{6.313005in}{1.336088in}}{\pgfqpoint{6.305872in}{1.336088in}}%
\pgfpathcurveto{\pgfqpoint{6.298740in}{1.336088in}}{\pgfqpoint{6.291898in}{1.333255in}}{\pgfqpoint{6.286854in}{1.328211in}}%
\pgfpathcurveto{\pgfqpoint{6.281811in}{1.323167in}}{\pgfqpoint{6.278977in}{1.316326in}}{\pgfqpoint{6.278977in}{1.309193in}}%
\pgfpathcurveto{\pgfqpoint{6.278977in}{1.302060in}}{\pgfqpoint{6.281811in}{1.295218in}}{\pgfqpoint{6.286854in}{1.290175in}}%
\pgfpathcurveto{\pgfqpoint{6.291898in}{1.285131in}}{\pgfqpoint{6.298740in}{1.282297in}}{\pgfqpoint{6.305872in}{1.282297in}}%
\pgfpathclose%
\pgfusepath{stroke,fill}%
\end{pgfscope}%
\begin{pgfscope}%
\pgfpathrectangle{\pgfqpoint{4.985294in}{0.500000in}}{\pgfqpoint{1.764706in}{1.700000in}}%
\pgfusepath{clip}%
\pgfsetbuttcap%
\pgfsetroundjoin%
\definecolor{currentfill}{rgb}{0.981377,0.920617,0.865369}%
\pgfsetfillcolor{currentfill}%
\pgfsetlinewidth{0.311001pt}%
\definecolor{currentstroke}{rgb}{1.000000,1.000000,1.000000}%
\pgfsetstrokecolor{currentstroke}%
\pgfsetdash{}{0pt}%
\pgfpathmoveto{\pgfqpoint{6.318422in}{1.324510in}}%
\pgfpathcurveto{\pgfqpoint{6.325555in}{1.324510in}}{\pgfqpoint{6.332397in}{1.327344in}}{\pgfqpoint{6.337440in}{1.332387in}}%
\pgfpathcurveto{\pgfqpoint{6.342484in}{1.337431in}}{\pgfqpoint{6.345318in}{1.344273in}}{\pgfqpoint{6.345318in}{1.351405in}}%
\pgfpathcurveto{\pgfqpoint{6.345318in}{1.358538in}}{\pgfqpoint{6.342484in}{1.365380in}}{\pgfqpoint{6.337440in}{1.370424in}}%
\pgfpathcurveto{\pgfqpoint{6.332397in}{1.375467in}}{\pgfqpoint{6.325555in}{1.378301in}}{\pgfqpoint{6.318422in}{1.378301in}}%
\pgfpathcurveto{\pgfqpoint{6.311289in}{1.378301in}}{\pgfqpoint{6.304448in}{1.375467in}}{\pgfqpoint{6.299404in}{1.370424in}}%
\pgfpathcurveto{\pgfqpoint{6.294360in}{1.365380in}}{\pgfqpoint{6.291527in}{1.358538in}}{\pgfqpoint{6.291527in}{1.351405in}}%
\pgfpathcurveto{\pgfqpoint{6.291527in}{1.344273in}}{\pgfqpoint{6.294360in}{1.337431in}}{\pgfqpoint{6.299404in}{1.332387in}}%
\pgfpathcurveto{\pgfqpoint{6.304448in}{1.327344in}}{\pgfqpoint{6.311289in}{1.324510in}}{\pgfqpoint{6.318422in}{1.324510in}}%
\pgfpathclose%
\pgfusepath{stroke,fill}%
\end{pgfscope}%
\begin{pgfscope}%
\pgfpathrectangle{\pgfqpoint{4.985294in}{0.500000in}}{\pgfqpoint{1.764706in}{1.700000in}}%
\pgfusepath{clip}%
\pgfsetbuttcap%
\pgfsetroundjoin%
\definecolor{currentfill}{rgb}{0.974412,0.862387,0.780156}%
\pgfsetfillcolor{currentfill}%
\pgfsetlinewidth{0.311001pt}%
\definecolor{currentstroke}{rgb}{1.000000,1.000000,1.000000}%
\pgfsetstrokecolor{currentstroke}%
\pgfsetdash{}{0pt}%
\pgfpathmoveto{\pgfqpoint{5.472169in}{1.378189in}}%
\pgfpathcurveto{\pgfqpoint{5.479301in}{1.378189in}}{\pgfqpoint{5.486143in}{1.381023in}}{\pgfqpoint{5.491187in}{1.386066in}}%
\pgfpathcurveto{\pgfqpoint{5.496230in}{1.391110in}}{\pgfqpoint{5.499064in}{1.397952in}}{\pgfqpoint{5.499064in}{1.405084in}}%
\pgfpathcurveto{\pgfqpoint{5.499064in}{1.412217in}}{\pgfqpoint{5.496230in}{1.419059in}}{\pgfqpoint{5.491187in}{1.424103in}}%
\pgfpathcurveto{\pgfqpoint{5.486143in}{1.429146in}}{\pgfqpoint{5.479301in}{1.431980in}}{\pgfqpoint{5.472169in}{1.431980in}}%
\pgfpathcurveto{\pgfqpoint{5.465036in}{1.431980in}}{\pgfqpoint{5.458194in}{1.429146in}}{\pgfqpoint{5.453150in}{1.424103in}}%
\pgfpathcurveto{\pgfqpoint{5.448107in}{1.419059in}}{\pgfqpoint{5.445273in}{1.412217in}}{\pgfqpoint{5.445273in}{1.405084in}}%
\pgfpathcurveto{\pgfqpoint{5.445273in}{1.397952in}}{\pgfqpoint{5.448107in}{1.391110in}}{\pgfqpoint{5.453150in}{1.386066in}}%
\pgfpathcurveto{\pgfqpoint{5.458194in}{1.381023in}}{\pgfqpoint{5.465036in}{1.378189in}}{\pgfqpoint{5.472169in}{1.378189in}}%
\pgfpathclose%
\pgfusepath{stroke,fill}%
\end{pgfscope}%
\begin{pgfscope}%
\pgfpathrectangle{\pgfqpoint{4.985294in}{0.500000in}}{\pgfqpoint{1.764706in}{1.700000in}}%
\pgfusepath{clip}%
\pgfsetbuttcap%
\pgfsetroundjoin%
\definecolor{currentfill}{rgb}{0.969359,0.803954,0.693832}%
\pgfsetfillcolor{currentfill}%
\pgfsetlinewidth{0.311001pt}%
\definecolor{currentstroke}{rgb}{1.000000,1.000000,1.000000}%
\pgfsetstrokecolor{currentstroke}%
\pgfsetdash{}{0pt}%
\pgfpathmoveto{\pgfqpoint{5.545355in}{1.048068in}}%
\pgfpathcurveto{\pgfqpoint{5.552487in}{1.048068in}}{\pgfqpoint{5.559329in}{1.050902in}}{\pgfqpoint{5.564373in}{1.055946in}}%
\pgfpathcurveto{\pgfqpoint{5.569416in}{1.060990in}}{\pgfqpoint{5.572250in}{1.067831in}}{\pgfqpoint{5.572250in}{1.074964in}}%
\pgfpathcurveto{\pgfqpoint{5.572250in}{1.082097in}}{\pgfqpoint{5.569416in}{1.088939in}}{\pgfqpoint{5.564373in}{1.093982in}}%
\pgfpathcurveto{\pgfqpoint{5.559329in}{1.099026in}}{\pgfqpoint{5.552487in}{1.101860in}}{\pgfqpoint{5.545355in}{1.101860in}}%
\pgfpathcurveto{\pgfqpoint{5.538222in}{1.101860in}}{\pgfqpoint{5.531380in}{1.099026in}}{\pgfqpoint{5.526336in}{1.093982in}}%
\pgfpathcurveto{\pgfqpoint{5.521293in}{1.088939in}}{\pgfqpoint{5.518459in}{1.082097in}}{\pgfqpoint{5.518459in}{1.074964in}}%
\pgfpathcurveto{\pgfqpoint{5.518459in}{1.067831in}}{\pgfqpoint{5.521293in}{1.060990in}}{\pgfqpoint{5.526336in}{1.055946in}}%
\pgfpathcurveto{\pgfqpoint{5.531380in}{1.050902in}}{\pgfqpoint{5.538222in}{1.048068in}}{\pgfqpoint{5.545355in}{1.048068in}}%
\pgfpathclose%
\pgfusepath{stroke,fill}%
\end{pgfscope}%
\begin{pgfscope}%
\pgfpathrectangle{\pgfqpoint{4.985294in}{0.500000in}}{\pgfqpoint{1.764706in}{1.700000in}}%
\pgfusepath{clip}%
\pgfsetbuttcap%
\pgfsetroundjoin%
\definecolor{currentfill}{rgb}{0.973832,0.856556,0.771584}%
\pgfsetfillcolor{currentfill}%
\pgfsetlinewidth{0.311001pt}%
\definecolor{currentstroke}{rgb}{1.000000,1.000000,1.000000}%
\pgfsetstrokecolor{currentstroke}%
\pgfsetdash{}{0pt}%
\pgfpathmoveto{\pgfqpoint{5.429727in}{1.080916in}}%
\pgfpathcurveto{\pgfqpoint{5.436860in}{1.080916in}}{\pgfqpoint{5.443702in}{1.083750in}}{\pgfqpoint{5.448745in}{1.088794in}}%
\pgfpathcurveto{\pgfqpoint{5.453789in}{1.093837in}}{\pgfqpoint{5.456623in}{1.100679in}}{\pgfqpoint{5.456623in}{1.107812in}}%
\pgfpathcurveto{\pgfqpoint{5.456623in}{1.114945in}}{\pgfqpoint{5.453789in}{1.121786in}}{\pgfqpoint{5.448745in}{1.126830in}}%
\pgfpathcurveto{\pgfqpoint{5.443702in}{1.131874in}}{\pgfqpoint{5.436860in}{1.134708in}}{\pgfqpoint{5.429727in}{1.134708in}}%
\pgfpathcurveto{\pgfqpoint{5.422594in}{1.134708in}}{\pgfqpoint{5.415753in}{1.131874in}}{\pgfqpoint{5.410709in}{1.126830in}}%
\pgfpathcurveto{\pgfqpoint{5.405666in}{1.121786in}}{\pgfqpoint{5.402832in}{1.114945in}}{\pgfqpoint{5.402832in}{1.107812in}}%
\pgfpathcurveto{\pgfqpoint{5.402832in}{1.100679in}}{\pgfqpoint{5.405666in}{1.093837in}}{\pgfqpoint{5.410709in}{1.088794in}}%
\pgfpathcurveto{\pgfqpoint{5.415753in}{1.083750in}}{\pgfqpoint{5.422594in}{1.080916in}}{\pgfqpoint{5.429727in}{1.080916in}}%
\pgfpathclose%
\pgfusepath{stroke,fill}%
\end{pgfscope}%
\begin{pgfscope}%
\pgfpathrectangle{\pgfqpoint{4.985294in}{0.500000in}}{\pgfqpoint{1.764706in}{1.700000in}}%
\pgfusepath{clip}%
\pgfsetbuttcap%
\pgfsetroundjoin%
\definecolor{currentfill}{rgb}{0.964799,0.689101,0.537560}%
\pgfsetfillcolor{currentfill}%
\pgfsetlinewidth{0.311001pt}%
\definecolor{currentstroke}{rgb}{1.000000,1.000000,1.000000}%
\pgfsetstrokecolor{currentstroke}%
\pgfsetdash{}{0pt}%
\pgfpathmoveto{\pgfqpoint{5.590912in}{1.052564in}}%
\pgfpathcurveto{\pgfqpoint{5.598045in}{1.052564in}}{\pgfqpoint{5.604886in}{1.055398in}}{\pgfqpoint{5.609930in}{1.060441in}}%
\pgfpathcurveto{\pgfqpoint{5.614974in}{1.065485in}}{\pgfqpoint{5.617807in}{1.072327in}}{\pgfqpoint{5.617807in}{1.079459in}}%
\pgfpathcurveto{\pgfqpoint{5.617807in}{1.086592in}}{\pgfqpoint{5.614974in}{1.093434in}}{\pgfqpoint{5.609930in}{1.098478in}}%
\pgfpathcurveto{\pgfqpoint{5.604886in}{1.103521in}}{\pgfqpoint{5.598045in}{1.106355in}}{\pgfqpoint{5.590912in}{1.106355in}}%
\pgfpathcurveto{\pgfqpoint{5.583779in}{1.106355in}}{\pgfqpoint{5.576937in}{1.103521in}}{\pgfqpoint{5.571894in}{1.098478in}}%
\pgfpathcurveto{\pgfqpoint{5.566850in}{1.093434in}}{\pgfqpoint{5.564016in}{1.086592in}}{\pgfqpoint{5.564016in}{1.079459in}}%
\pgfpathcurveto{\pgfqpoint{5.564016in}{1.072327in}}{\pgfqpoint{5.566850in}{1.065485in}}{\pgfqpoint{5.571894in}{1.060441in}}%
\pgfpathcurveto{\pgfqpoint{5.576937in}{1.055398in}}{\pgfqpoint{5.583779in}{1.052564in}}{\pgfqpoint{5.590912in}{1.052564in}}%
\pgfpathclose%
\pgfusepath{stroke,fill}%
\end{pgfscope}%
\begin{pgfscope}%
\pgfpathrectangle{\pgfqpoint{4.985294in}{0.500000in}}{\pgfqpoint{1.764706in}{1.700000in}}%
\pgfusepath{clip}%
\pgfsetbuttcap%
\pgfsetroundjoin%
\definecolor{currentfill}{rgb}{0.979124,0.903132,0.839793}%
\pgfsetfillcolor{currentfill}%
\pgfsetlinewidth{0.311001pt}%
\definecolor{currentstroke}{rgb}{1.000000,1.000000,1.000000}%
\pgfsetstrokecolor{currentstroke}%
\pgfsetdash{}{0pt}%
\pgfpathmoveto{\pgfqpoint{5.399878in}{1.388926in}}%
\pgfpathcurveto{\pgfqpoint{5.407011in}{1.388926in}}{\pgfqpoint{5.413853in}{1.391760in}}{\pgfqpoint{5.418896in}{1.396804in}}%
\pgfpathcurveto{\pgfqpoint{5.423940in}{1.401848in}}{\pgfqpoint{5.426774in}{1.408689in}}{\pgfqpoint{5.426774in}{1.415822in}}%
\pgfpathcurveto{\pgfqpoint{5.426774in}{1.422955in}}{\pgfqpoint{5.423940in}{1.429796in}}{\pgfqpoint{5.418896in}{1.434840in}}%
\pgfpathcurveto{\pgfqpoint{5.413853in}{1.439884in}}{\pgfqpoint{5.407011in}{1.442718in}}{\pgfqpoint{5.399878in}{1.442718in}}%
\pgfpathcurveto{\pgfqpoint{5.392745in}{1.442718in}}{\pgfqpoint{5.385904in}{1.439884in}}{\pgfqpoint{5.380860in}{1.434840in}}%
\pgfpathcurveto{\pgfqpoint{5.375816in}{1.429796in}}{\pgfqpoint{5.372982in}{1.422955in}}{\pgfqpoint{5.372982in}{1.415822in}}%
\pgfpathcurveto{\pgfqpoint{5.372982in}{1.408689in}}{\pgfqpoint{5.375816in}{1.401848in}}{\pgfqpoint{5.380860in}{1.396804in}}%
\pgfpathcurveto{\pgfqpoint{5.385904in}{1.391760in}}{\pgfqpoint{5.392745in}{1.388926in}}{\pgfqpoint{5.399878in}{1.388926in}}%
\pgfpathclose%
\pgfusepath{stroke,fill}%
\end{pgfscope}%
\begin{pgfscope}%
\pgfpathrectangle{\pgfqpoint{4.985294in}{0.500000in}}{\pgfqpoint{1.764706in}{1.700000in}}%
\pgfusepath{clip}%
\pgfsetbuttcap%
\pgfsetroundjoin%
\definecolor{currentfill}{rgb}{0.964679,0.682838,0.530002}%
\pgfsetfillcolor{currentfill}%
\pgfsetlinewidth{0.311001pt}%
\definecolor{currentstroke}{rgb}{1.000000,1.000000,1.000000}%
\pgfsetstrokecolor{currentstroke}%
\pgfsetdash{}{0pt}%
\pgfpathmoveto{\pgfqpoint{6.261419in}{1.705667in}}%
\pgfpathcurveto{\pgfqpoint{6.268552in}{1.705667in}}{\pgfqpoint{6.275394in}{1.708501in}}{\pgfqpoint{6.280437in}{1.713545in}}%
\pgfpathcurveto{\pgfqpoint{6.285481in}{1.718588in}}{\pgfqpoint{6.288315in}{1.725430in}}{\pgfqpoint{6.288315in}{1.732563in}}%
\pgfpathcurveto{\pgfqpoint{6.288315in}{1.739696in}}{\pgfqpoint{6.285481in}{1.746537in}}{\pgfqpoint{6.280437in}{1.751581in}}%
\pgfpathcurveto{\pgfqpoint{6.275394in}{1.756625in}}{\pgfqpoint{6.268552in}{1.759459in}}{\pgfqpoint{6.261419in}{1.759459in}}%
\pgfpathcurveto{\pgfqpoint{6.254286in}{1.759459in}}{\pgfqpoint{6.247445in}{1.756625in}}{\pgfqpoint{6.242401in}{1.751581in}}%
\pgfpathcurveto{\pgfqpoint{6.237357in}{1.746537in}}{\pgfqpoint{6.234523in}{1.739696in}}{\pgfqpoint{6.234523in}{1.732563in}}%
\pgfpathcurveto{\pgfqpoint{6.234523in}{1.725430in}}{\pgfqpoint{6.237357in}{1.718588in}}{\pgfqpoint{6.242401in}{1.713545in}}%
\pgfpathcurveto{\pgfqpoint{6.247445in}{1.708501in}}{\pgfqpoint{6.254286in}{1.705667in}}{\pgfqpoint{6.261419in}{1.705667in}}%
\pgfpathclose%
\pgfusepath{stroke,fill}%
\end{pgfscope}%
\begin{pgfscope}%
\pgfpathrectangle{\pgfqpoint{4.985294in}{0.500000in}}{\pgfqpoint{1.764706in}{1.700000in}}%
\pgfusepath{clip}%
\pgfsetbuttcap%
\pgfsetroundjoin%
\definecolor{currentfill}{rgb}{0.974412,0.862387,0.780156}%
\pgfsetfillcolor{currentfill}%
\pgfsetlinewidth{0.311001pt}%
\definecolor{currentstroke}{rgb}{1.000000,1.000000,1.000000}%
\pgfsetstrokecolor{currentstroke}%
\pgfsetdash{}{0pt}%
\pgfpathmoveto{\pgfqpoint{6.257146in}{1.262994in}}%
\pgfpathcurveto{\pgfqpoint{6.264279in}{1.262994in}}{\pgfqpoint{6.271120in}{1.265828in}}{\pgfqpoint{6.276164in}{1.270872in}}%
\pgfpathcurveto{\pgfqpoint{6.281208in}{1.275916in}}{\pgfqpoint{6.284042in}{1.282757in}}{\pgfqpoint{6.284042in}{1.289890in}}%
\pgfpathcurveto{\pgfqpoint{6.284042in}{1.297023in}}{\pgfqpoint{6.281208in}{1.303865in}}{\pgfqpoint{6.276164in}{1.308908in}}%
\pgfpathcurveto{\pgfqpoint{6.271120in}{1.313952in}}{\pgfqpoint{6.264279in}{1.316786in}}{\pgfqpoint{6.257146in}{1.316786in}}%
\pgfpathcurveto{\pgfqpoint{6.250013in}{1.316786in}}{\pgfqpoint{6.243171in}{1.313952in}}{\pgfqpoint{6.238128in}{1.308908in}}%
\pgfpathcurveto{\pgfqpoint{6.233084in}{1.303865in}}{\pgfqpoint{6.230250in}{1.297023in}}{\pgfqpoint{6.230250in}{1.289890in}}%
\pgfpathcurveto{\pgfqpoint{6.230250in}{1.282757in}}{\pgfqpoint{6.233084in}{1.275916in}}{\pgfqpoint{6.238128in}{1.270872in}}%
\pgfpathcurveto{\pgfqpoint{6.243171in}{1.265828in}}{\pgfqpoint{6.250013in}{1.262994in}}{\pgfqpoint{6.257146in}{1.262994in}}%
\pgfpathclose%
\pgfusepath{stroke,fill}%
\end{pgfscope}%
\begin{pgfscope}%
\pgfpathrectangle{\pgfqpoint{4.985294in}{0.500000in}}{\pgfqpoint{1.764706in}{1.700000in}}%
\pgfusepath{clip}%
\pgfsetbuttcap%
\pgfsetroundjoin%
\definecolor{currentfill}{rgb}{0.966120,0.744512,0.608720}%
\pgfsetfillcolor{currentfill}%
\pgfsetlinewidth{0.311001pt}%
\definecolor{currentstroke}{rgb}{1.000000,1.000000,1.000000}%
\pgfsetstrokecolor{currentstroke}%
\pgfsetdash{}{0pt}%
\pgfpathmoveto{\pgfqpoint{6.151101in}{1.675981in}}%
\pgfpathcurveto{\pgfqpoint{6.158234in}{1.675981in}}{\pgfqpoint{6.165075in}{1.678815in}}{\pgfqpoint{6.170119in}{1.683859in}}%
\pgfpathcurveto{\pgfqpoint{6.175163in}{1.688902in}}{\pgfqpoint{6.177997in}{1.695744in}}{\pgfqpoint{6.177997in}{1.702877in}}%
\pgfpathcurveto{\pgfqpoint{6.177997in}{1.710010in}}{\pgfqpoint{6.175163in}{1.716851in}}{\pgfqpoint{6.170119in}{1.721895in}}%
\pgfpathcurveto{\pgfqpoint{6.165075in}{1.726939in}}{\pgfqpoint{6.158234in}{1.729772in}}{\pgfqpoint{6.151101in}{1.729772in}}%
\pgfpathcurveto{\pgfqpoint{6.143968in}{1.729772in}}{\pgfqpoint{6.137127in}{1.726939in}}{\pgfqpoint{6.132083in}{1.721895in}}%
\pgfpathcurveto{\pgfqpoint{6.127039in}{1.716851in}}{\pgfqpoint{6.124205in}{1.710010in}}{\pgfqpoint{6.124205in}{1.702877in}}%
\pgfpathcurveto{\pgfqpoint{6.124205in}{1.695744in}}{\pgfqpoint{6.127039in}{1.688902in}}{\pgfqpoint{6.132083in}{1.683859in}}%
\pgfpathcurveto{\pgfqpoint{6.137127in}{1.678815in}}{\pgfqpoint{6.143968in}{1.675981in}}{\pgfqpoint{6.151101in}{1.675981in}}%
\pgfpathclose%
\pgfusepath{stroke,fill}%
\end{pgfscope}%
\begin{pgfscope}%
\pgfpathrectangle{\pgfqpoint{4.985294in}{0.500000in}}{\pgfqpoint{1.764706in}{1.700000in}}%
\pgfusepath{clip}%
\pgfsetbuttcap%
\pgfsetroundjoin%
\definecolor{currentfill}{rgb}{0.976287,0.879862,0.805788}%
\pgfsetfillcolor{currentfill}%
\pgfsetlinewidth{0.311001pt}%
\definecolor{currentstroke}{rgb}{1.000000,1.000000,1.000000}%
\pgfsetstrokecolor{currentstroke}%
\pgfsetdash{}{0pt}%
\pgfpathmoveto{\pgfqpoint{6.319739in}{1.512849in}}%
\pgfpathcurveto{\pgfqpoint{6.326872in}{1.512849in}}{\pgfqpoint{6.333714in}{1.515683in}}{\pgfqpoint{6.338758in}{1.520727in}}%
\pgfpathcurveto{\pgfqpoint{6.343801in}{1.525771in}}{\pgfqpoint{6.346635in}{1.532612in}}{\pgfqpoint{6.346635in}{1.539745in}}%
\pgfpathcurveto{\pgfqpoint{6.346635in}{1.546878in}}{\pgfqpoint{6.343801in}{1.553720in}}{\pgfqpoint{6.338758in}{1.558763in}}%
\pgfpathcurveto{\pgfqpoint{6.333714in}{1.563807in}}{\pgfqpoint{6.326872in}{1.566641in}}{\pgfqpoint{6.319739in}{1.566641in}}%
\pgfpathcurveto{\pgfqpoint{6.312607in}{1.566641in}}{\pgfqpoint{6.305765in}{1.563807in}}{\pgfqpoint{6.300721in}{1.558763in}}%
\pgfpathcurveto{\pgfqpoint{6.295678in}{1.553720in}}{\pgfqpoint{6.292844in}{1.546878in}}{\pgfqpoint{6.292844in}{1.539745in}}%
\pgfpathcurveto{\pgfqpoint{6.292844in}{1.532612in}}{\pgfqpoint{6.295678in}{1.525771in}}{\pgfqpoint{6.300721in}{1.520727in}}%
\pgfpathcurveto{\pgfqpoint{6.305765in}{1.515683in}}{\pgfqpoint{6.312607in}{1.512849in}}{\pgfqpoint{6.319739in}{1.512849in}}%
\pgfpathclose%
\pgfusepath{stroke,fill}%
\end{pgfscope}%
\begin{pgfscope}%
\pgfpathrectangle{\pgfqpoint{4.985294in}{0.500000in}}{\pgfqpoint{1.764706in}{1.700000in}}%
\pgfusepath{clip}%
\pgfsetbuttcap%
\pgfsetroundjoin%
\definecolor{currentfill}{rgb}{0.972726,0.844889,0.754401}%
\pgfsetfillcolor{currentfill}%
\pgfsetlinewidth{0.311001pt}%
\definecolor{currentstroke}{rgb}{1.000000,1.000000,1.000000}%
\pgfsetstrokecolor{currentstroke}%
\pgfsetdash{}{0pt}%
\pgfpathmoveto{\pgfqpoint{6.321735in}{1.548804in}}%
\pgfpathcurveto{\pgfqpoint{6.328868in}{1.548804in}}{\pgfqpoint{6.335709in}{1.551638in}}{\pgfqpoint{6.340753in}{1.556682in}}%
\pgfpathcurveto{\pgfqpoint{6.345797in}{1.561726in}}{\pgfqpoint{6.348631in}{1.568567in}}{\pgfqpoint{6.348631in}{1.575700in}}%
\pgfpathcurveto{\pgfqpoint{6.348631in}{1.582833in}}{\pgfqpoint{6.345797in}{1.589675in}}{\pgfqpoint{6.340753in}{1.594718in}}%
\pgfpathcurveto{\pgfqpoint{6.335709in}{1.599762in}}{\pgfqpoint{6.328868in}{1.602596in}}{\pgfqpoint{6.321735in}{1.602596in}}%
\pgfpathcurveto{\pgfqpoint{6.314602in}{1.602596in}}{\pgfqpoint{6.307761in}{1.599762in}}{\pgfqpoint{6.302717in}{1.594718in}}%
\pgfpathcurveto{\pgfqpoint{6.297673in}{1.589675in}}{\pgfqpoint{6.294839in}{1.582833in}}{\pgfqpoint{6.294839in}{1.575700in}}%
\pgfpathcurveto{\pgfqpoint{6.294839in}{1.568567in}}{\pgfqpoint{6.297673in}{1.561726in}}{\pgfqpoint{6.302717in}{1.556682in}}%
\pgfpathcurveto{\pgfqpoint{6.307761in}{1.551638in}}{\pgfqpoint{6.314602in}{1.548804in}}{\pgfqpoint{6.321735in}{1.548804in}}%
\pgfpathclose%
\pgfusepath{stroke,fill}%
\end{pgfscope}%
\begin{pgfscope}%
\pgfpathrectangle{\pgfqpoint{4.985294in}{0.500000in}}{\pgfqpoint{1.764706in}{1.700000in}}%
\pgfusepath{clip}%
\pgfsetbuttcap%
\pgfsetroundjoin%
\definecolor{currentfill}{rgb}{0.972201,0.839051,0.745789}%
\pgfsetfillcolor{currentfill}%
\pgfsetlinewidth{0.311001pt}%
\definecolor{currentstroke}{rgb}{1.000000,1.000000,1.000000}%
\pgfsetstrokecolor{currentstroke}%
\pgfsetdash{}{0pt}%
\pgfpathmoveto{\pgfqpoint{5.407152in}{1.527428in}}%
\pgfpathcurveto{\pgfqpoint{5.414284in}{1.527428in}}{\pgfqpoint{5.421126in}{1.530262in}}{\pgfqpoint{5.426170in}{1.535306in}}%
\pgfpathcurveto{\pgfqpoint{5.431213in}{1.540349in}}{\pgfqpoint{5.434047in}{1.547191in}}{\pgfqpoint{5.434047in}{1.554324in}}%
\pgfpathcurveto{\pgfqpoint{5.434047in}{1.561457in}}{\pgfqpoint{5.431213in}{1.568298in}}{\pgfqpoint{5.426170in}{1.573342in}}%
\pgfpathcurveto{\pgfqpoint{5.421126in}{1.578386in}}{\pgfqpoint{5.414284in}{1.581220in}}{\pgfqpoint{5.407152in}{1.581220in}}%
\pgfpathcurveto{\pgfqpoint{5.400019in}{1.581220in}}{\pgfqpoint{5.393177in}{1.578386in}}{\pgfqpoint{5.388133in}{1.573342in}}%
\pgfpathcurveto{\pgfqpoint{5.383090in}{1.568298in}}{\pgfqpoint{5.380256in}{1.561457in}}{\pgfqpoint{5.380256in}{1.554324in}}%
\pgfpathcurveto{\pgfqpoint{5.380256in}{1.547191in}}{\pgfqpoint{5.383090in}{1.540349in}}{\pgfqpoint{5.388133in}{1.535306in}}%
\pgfpathcurveto{\pgfqpoint{5.393177in}{1.530262in}}{\pgfqpoint{5.400019in}{1.527428in}}{\pgfqpoint{5.407152in}{1.527428in}}%
\pgfpathclose%
\pgfusepath{stroke,fill}%
\end{pgfscope}%
\begin{pgfscope}%
\pgfpathrectangle{\pgfqpoint{4.985294in}{0.500000in}}{\pgfqpoint{1.764706in}{1.700000in}}%
\pgfusepath{clip}%
\pgfsetbuttcap%
\pgfsetroundjoin%
\definecolor{currentfill}{rgb}{0.974412,0.862387,0.780156}%
\pgfsetfillcolor{currentfill}%
\pgfsetlinewidth{0.311001pt}%
\definecolor{currentstroke}{rgb}{1.000000,1.000000,1.000000}%
\pgfsetstrokecolor{currentstroke}%
\pgfsetdash{}{0pt}%
\pgfpathmoveto{\pgfqpoint{6.267352in}{1.356329in}}%
\pgfpathcurveto{\pgfqpoint{6.274484in}{1.356329in}}{\pgfqpoint{6.281326in}{1.359163in}}{\pgfqpoint{6.286370in}{1.364207in}}%
\pgfpathcurveto{\pgfqpoint{6.291413in}{1.369250in}}{\pgfqpoint{6.294247in}{1.376092in}}{\pgfqpoint{6.294247in}{1.383225in}}%
\pgfpathcurveto{\pgfqpoint{6.294247in}{1.390358in}}{\pgfqpoint{6.291413in}{1.397199in}}{\pgfqpoint{6.286370in}{1.402243in}}%
\pgfpathcurveto{\pgfqpoint{6.281326in}{1.407287in}}{\pgfqpoint{6.274484in}{1.410121in}}{\pgfqpoint{6.267352in}{1.410121in}}%
\pgfpathcurveto{\pgfqpoint{6.260219in}{1.410121in}}{\pgfqpoint{6.253377in}{1.407287in}}{\pgfqpoint{6.248333in}{1.402243in}}%
\pgfpathcurveto{\pgfqpoint{6.243290in}{1.397199in}}{\pgfqpoint{6.240456in}{1.390358in}}{\pgfqpoint{6.240456in}{1.383225in}}%
\pgfpathcurveto{\pgfqpoint{6.240456in}{1.376092in}}{\pgfqpoint{6.243290in}{1.369250in}}{\pgfqpoint{6.248333in}{1.364207in}}%
\pgfpathcurveto{\pgfqpoint{6.253377in}{1.359163in}}{\pgfqpoint{6.260219in}{1.356329in}}{\pgfqpoint{6.267352in}{1.356329in}}%
\pgfpathclose%
\pgfusepath{stroke,fill}%
\end{pgfscope}%
\begin{pgfscope}%
\pgfpathrectangle{\pgfqpoint{4.985294in}{0.500000in}}{\pgfqpoint{1.764706in}{1.700000in}}%
\pgfusepath{clip}%
\pgfsetbuttcap%
\pgfsetroundjoin%
\definecolor{currentfill}{rgb}{0.975018,0.868213,0.788710}%
\pgfsetfillcolor{currentfill}%
\pgfsetlinewidth{0.311001pt}%
\definecolor{currentstroke}{rgb}{1.000000,1.000000,1.000000}%
\pgfsetstrokecolor{currentstroke}%
\pgfsetdash{}{0pt}%
\pgfpathmoveto{\pgfqpoint{5.481175in}{1.069169in}}%
\pgfpathcurveto{\pgfqpoint{5.488307in}{1.069169in}}{\pgfqpoint{5.495149in}{1.072002in}}{\pgfqpoint{5.500193in}{1.077046in}}%
\pgfpathcurveto{\pgfqpoint{5.505236in}{1.082090in}}{\pgfqpoint{5.508070in}{1.088931in}}{\pgfqpoint{5.508070in}{1.096064in}}%
\pgfpathcurveto{\pgfqpoint{5.508070in}{1.103197in}}{\pgfqpoint{5.505236in}{1.110039in}}{\pgfqpoint{5.500193in}{1.115082in}}%
\pgfpathcurveto{\pgfqpoint{5.495149in}{1.120126in}}{\pgfqpoint{5.488307in}{1.122960in}}{\pgfqpoint{5.481175in}{1.122960in}}%
\pgfpathcurveto{\pgfqpoint{5.474042in}{1.122960in}}{\pgfqpoint{5.467200in}{1.120126in}}{\pgfqpoint{5.462157in}{1.115082in}}%
\pgfpathcurveto{\pgfqpoint{5.457113in}{1.110039in}}{\pgfqpoint{5.454279in}{1.103197in}}{\pgfqpoint{5.454279in}{1.096064in}}%
\pgfpathcurveto{\pgfqpoint{5.454279in}{1.088931in}}{\pgfqpoint{5.457113in}{1.082090in}}{\pgfqpoint{5.462157in}{1.077046in}}%
\pgfpathcurveto{\pgfqpoint{5.467200in}{1.072002in}}{\pgfqpoint{5.474042in}{1.069169in}}{\pgfqpoint{5.481175in}{1.069169in}}%
\pgfpathclose%
\pgfusepath{stroke,fill}%
\end{pgfscope}%
\begin{pgfscope}%
\pgfpathrectangle{\pgfqpoint{4.985294in}{0.500000in}}{\pgfqpoint{1.764706in}{1.700000in}}%
\pgfusepath{clip}%
\pgfsetbuttcap%
\pgfsetroundjoin%
\definecolor{currentfill}{rgb}{0.966812,0.762584,0.633643}%
\pgfsetfillcolor{currentfill}%
\pgfsetlinewidth{0.311001pt}%
\definecolor{currentstroke}{rgb}{1.000000,1.000000,1.000000}%
\pgfsetstrokecolor{currentstroke}%
\pgfsetdash{}{0pt}%
\pgfpathmoveto{\pgfqpoint{6.213787in}{1.341245in}}%
\pgfpathcurveto{\pgfqpoint{6.220920in}{1.341245in}}{\pgfqpoint{6.227762in}{1.344079in}}{\pgfqpoint{6.232805in}{1.349122in}}%
\pgfpathcurveto{\pgfqpoint{6.237849in}{1.354166in}}{\pgfqpoint{6.240683in}{1.361008in}}{\pgfqpoint{6.240683in}{1.368141in}}%
\pgfpathcurveto{\pgfqpoint{6.240683in}{1.375273in}}{\pgfqpoint{6.237849in}{1.382115in}}{\pgfqpoint{6.232805in}{1.387159in}}%
\pgfpathcurveto{\pgfqpoint{6.227762in}{1.392202in}}{\pgfqpoint{6.220920in}{1.395036in}}{\pgfqpoint{6.213787in}{1.395036in}}%
\pgfpathcurveto{\pgfqpoint{6.206655in}{1.395036in}}{\pgfqpoint{6.199813in}{1.392202in}}{\pgfqpoint{6.194769in}{1.387159in}}%
\pgfpathcurveto{\pgfqpoint{6.189726in}{1.382115in}}{\pgfqpoint{6.186892in}{1.375273in}}{\pgfqpoint{6.186892in}{1.368141in}}%
\pgfpathcurveto{\pgfqpoint{6.186892in}{1.361008in}}{\pgfqpoint{6.189726in}{1.354166in}}{\pgfqpoint{6.194769in}{1.349122in}}%
\pgfpathcurveto{\pgfqpoint{6.199813in}{1.344079in}}{\pgfqpoint{6.206655in}{1.341245in}}{\pgfqpoint{6.213787in}{1.341245in}}%
\pgfpathclose%
\pgfusepath{stroke,fill}%
\end{pgfscope}%
\begin{pgfscope}%
\pgfpathrectangle{\pgfqpoint{4.985294in}{0.500000in}}{\pgfqpoint{1.764706in}{1.700000in}}%
\pgfusepath{clip}%
\pgfsetbuttcap%
\pgfsetroundjoin%
\definecolor{currentfill}{rgb}{0.980678,0.914765,0.856766}%
\pgfsetfillcolor{currentfill}%
\pgfsetlinewidth{0.311001pt}%
\definecolor{currentstroke}{rgb}{1.000000,1.000000,1.000000}%
\pgfsetstrokecolor{currentstroke}%
\pgfsetdash{}{0pt}%
\pgfpathmoveto{\pgfqpoint{5.417167in}{1.260622in}}%
\pgfpathcurveto{\pgfqpoint{5.424299in}{1.260622in}}{\pgfqpoint{5.431141in}{1.263456in}}{\pgfqpoint{5.436185in}{1.268500in}}%
\pgfpathcurveto{\pgfqpoint{5.441228in}{1.273544in}}{\pgfqpoint{5.444062in}{1.280385in}}{\pgfqpoint{5.444062in}{1.287518in}}%
\pgfpathcurveto{\pgfqpoint{5.444062in}{1.294651in}}{\pgfqpoint{5.441228in}{1.301492in}}{\pgfqpoint{5.436185in}{1.306536in}}%
\pgfpathcurveto{\pgfqpoint{5.431141in}{1.311580in}}{\pgfqpoint{5.424299in}{1.314414in}}{\pgfqpoint{5.417167in}{1.314414in}}%
\pgfpathcurveto{\pgfqpoint{5.410034in}{1.314414in}}{\pgfqpoint{5.403192in}{1.311580in}}{\pgfqpoint{5.398148in}{1.306536in}}%
\pgfpathcurveto{\pgfqpoint{5.393105in}{1.301492in}}{\pgfqpoint{5.390271in}{1.294651in}}{\pgfqpoint{5.390271in}{1.287518in}}%
\pgfpathcurveto{\pgfqpoint{5.390271in}{1.280385in}}{\pgfqpoint{5.393105in}{1.273544in}}{\pgfqpoint{5.398148in}{1.268500in}}%
\pgfpathcurveto{\pgfqpoint{5.403192in}{1.263456in}}{\pgfqpoint{5.410034in}{1.260622in}}{\pgfqpoint{5.417167in}{1.260622in}}%
\pgfpathclose%
\pgfusepath{stroke,fill}%
\end{pgfscope}%
\begin{pgfscope}%
\pgfpathrectangle{\pgfqpoint{4.985294in}{0.500000in}}{\pgfqpoint{1.764706in}{1.700000in}}%
\pgfusepath{clip}%
\pgfsetbuttcap%
\pgfsetroundjoin%
\definecolor{currentfill}{rgb}{0.966328,0.750560,0.616961}%
\pgfsetfillcolor{currentfill}%
\pgfsetlinewidth{0.311001pt}%
\definecolor{currentstroke}{rgb}{1.000000,1.000000,1.000000}%
\pgfsetstrokecolor{currentstroke}%
\pgfsetdash{}{0pt}%
\pgfpathmoveto{\pgfqpoint{6.396110in}{1.222688in}}%
\pgfpathcurveto{\pgfqpoint{6.403243in}{1.222688in}}{\pgfqpoint{6.410085in}{1.225522in}}{\pgfqpoint{6.415128in}{1.230565in}}%
\pgfpathcurveto{\pgfqpoint{6.420172in}{1.235609in}}{\pgfqpoint{6.423006in}{1.242451in}}{\pgfqpoint{6.423006in}{1.249583in}}%
\pgfpathcurveto{\pgfqpoint{6.423006in}{1.256716in}}{\pgfqpoint{6.420172in}{1.263558in}}{\pgfqpoint{6.415128in}{1.268602in}}%
\pgfpathcurveto{\pgfqpoint{6.410085in}{1.273645in}}{\pgfqpoint{6.403243in}{1.276479in}}{\pgfqpoint{6.396110in}{1.276479in}}%
\pgfpathcurveto{\pgfqpoint{6.388977in}{1.276479in}}{\pgfqpoint{6.382136in}{1.273645in}}{\pgfqpoint{6.377092in}{1.268602in}}%
\pgfpathcurveto{\pgfqpoint{6.372048in}{1.263558in}}{\pgfqpoint{6.369214in}{1.256716in}}{\pgfqpoint{6.369214in}{1.249583in}}%
\pgfpathcurveto{\pgfqpoint{6.369214in}{1.242451in}}{\pgfqpoint{6.372048in}{1.235609in}}{\pgfqpoint{6.377092in}{1.230565in}}%
\pgfpathcurveto{\pgfqpoint{6.382136in}{1.225522in}}{\pgfqpoint{6.388977in}{1.222688in}}{\pgfqpoint{6.396110in}{1.222688in}}%
\pgfpathclose%
\pgfusepath{stroke,fill}%
\end{pgfscope}%
\begin{pgfscope}%
\pgfpathrectangle{\pgfqpoint{4.985294in}{0.500000in}}{\pgfqpoint{1.764706in}{1.700000in}}%
\pgfusepath{clip}%
\pgfsetbuttcap%
\pgfsetroundjoin%
\definecolor{currentfill}{rgb}{0.964306,0.663930,0.507747}%
\pgfsetfillcolor{currentfill}%
\pgfsetlinewidth{0.311001pt}%
\definecolor{currentstroke}{rgb}{1.000000,1.000000,1.000000}%
\pgfsetstrokecolor{currentstroke}%
\pgfsetdash{}{0pt}%
\pgfpathmoveto{\pgfqpoint{5.316745in}{1.361010in}}%
\pgfpathcurveto{\pgfqpoint{5.323878in}{1.361010in}}{\pgfqpoint{5.330719in}{1.363844in}}{\pgfqpoint{5.335763in}{1.368887in}}%
\pgfpathcurveto{\pgfqpoint{5.340807in}{1.373931in}}{\pgfqpoint{5.343641in}{1.380773in}}{\pgfqpoint{5.343641in}{1.387905in}}%
\pgfpathcurveto{\pgfqpoint{5.343641in}{1.395038in}}{\pgfqpoint{5.340807in}{1.401880in}}{\pgfqpoint{5.335763in}{1.406923in}}%
\pgfpathcurveto{\pgfqpoint{5.330719in}{1.411967in}}{\pgfqpoint{5.323878in}{1.414801in}}{\pgfqpoint{5.316745in}{1.414801in}}%
\pgfpathcurveto{\pgfqpoint{5.309612in}{1.414801in}}{\pgfqpoint{5.302770in}{1.411967in}}{\pgfqpoint{5.297727in}{1.406923in}}%
\pgfpathcurveto{\pgfqpoint{5.292683in}{1.401880in}}{\pgfqpoint{5.289849in}{1.395038in}}{\pgfqpoint{5.289849in}{1.387905in}}%
\pgfpathcurveto{\pgfqpoint{5.289849in}{1.380773in}}{\pgfqpoint{5.292683in}{1.373931in}}{\pgfqpoint{5.297727in}{1.368887in}}%
\pgfpathcurveto{\pgfqpoint{5.302770in}{1.363844in}}{\pgfqpoint{5.309612in}{1.361010in}}{\pgfqpoint{5.316745in}{1.361010in}}%
\pgfpathclose%
\pgfusepath{stroke,fill}%
\end{pgfscope}%
\begin{pgfscope}%
\pgfpathrectangle{\pgfqpoint{4.985294in}{0.500000in}}{\pgfqpoint{1.764706in}{1.700000in}}%
\pgfusepath{clip}%
\pgfsetbuttcap%
\pgfsetroundjoin%
\definecolor{currentfill}{rgb}{0.974412,0.862387,0.780156}%
\pgfsetfillcolor{currentfill}%
\pgfsetlinewidth{0.311001pt}%
\definecolor{currentstroke}{rgb}{1.000000,1.000000,1.000000}%
\pgfsetstrokecolor{currentstroke}%
\pgfsetdash{}{0pt}%
\pgfpathmoveto{\pgfqpoint{6.254064in}{1.238895in}}%
\pgfpathcurveto{\pgfqpoint{6.261197in}{1.238895in}}{\pgfqpoint{6.268038in}{1.241729in}}{\pgfqpoint{6.273082in}{1.246773in}}%
\pgfpathcurveto{\pgfqpoint{6.278126in}{1.251816in}}{\pgfqpoint{6.280960in}{1.258658in}}{\pgfqpoint{6.280960in}{1.265791in}}%
\pgfpathcurveto{\pgfqpoint{6.280960in}{1.272924in}}{\pgfqpoint{6.278126in}{1.279765in}}{\pgfqpoint{6.273082in}{1.284809in}}%
\pgfpathcurveto{\pgfqpoint{6.268038in}{1.289853in}}{\pgfqpoint{6.261197in}{1.292686in}}{\pgfqpoint{6.254064in}{1.292686in}}%
\pgfpathcurveto{\pgfqpoint{6.246931in}{1.292686in}}{\pgfqpoint{6.240089in}{1.289853in}}{\pgfqpoint{6.235046in}{1.284809in}}%
\pgfpathcurveto{\pgfqpoint{6.230002in}{1.279765in}}{\pgfqpoint{6.227168in}{1.272924in}}{\pgfqpoint{6.227168in}{1.265791in}}%
\pgfpathcurveto{\pgfqpoint{6.227168in}{1.258658in}}{\pgfqpoint{6.230002in}{1.251816in}}{\pgfqpoint{6.235046in}{1.246773in}}%
\pgfpathcurveto{\pgfqpoint{6.240089in}{1.241729in}}{\pgfqpoint{6.246931in}{1.238895in}}{\pgfqpoint{6.254064in}{1.238895in}}%
\pgfpathclose%
\pgfusepath{stroke,fill}%
\end{pgfscope}%
\begin{pgfscope}%
\pgfpathrectangle{\pgfqpoint{4.985294in}{0.500000in}}{\pgfqpoint{1.764706in}{1.700000in}}%
\pgfusepath{clip}%
\pgfsetbuttcap%
\pgfsetroundjoin%
\definecolor{currentfill}{rgb}{0.965042,0.701564,0.552889}%
\pgfsetfillcolor{currentfill}%
\pgfsetlinewidth{0.311001pt}%
\definecolor{currentstroke}{rgb}{1.000000,1.000000,1.000000}%
\pgfsetstrokecolor{currentstroke}%
\pgfsetdash{}{0pt}%
\pgfpathmoveto{\pgfqpoint{5.342834in}{1.464395in}}%
\pgfpathcurveto{\pgfqpoint{5.349967in}{1.464395in}}{\pgfqpoint{5.356809in}{1.467229in}}{\pgfqpoint{5.361852in}{1.472273in}}%
\pgfpathcurveto{\pgfqpoint{5.366896in}{1.477316in}}{\pgfqpoint{5.369730in}{1.484158in}}{\pgfqpoint{5.369730in}{1.491291in}}%
\pgfpathcurveto{\pgfqpoint{5.369730in}{1.498424in}}{\pgfqpoint{5.366896in}{1.505265in}}{\pgfqpoint{5.361852in}{1.510309in}}%
\pgfpathcurveto{\pgfqpoint{5.356809in}{1.515353in}}{\pgfqpoint{5.349967in}{1.518186in}}{\pgfqpoint{5.342834in}{1.518186in}}%
\pgfpathcurveto{\pgfqpoint{5.335701in}{1.518186in}}{\pgfqpoint{5.328860in}{1.515353in}}{\pgfqpoint{5.323816in}{1.510309in}}%
\pgfpathcurveto{\pgfqpoint{5.318772in}{1.505265in}}{\pgfqpoint{5.315938in}{1.498424in}}{\pgfqpoint{5.315938in}{1.491291in}}%
\pgfpathcurveto{\pgfqpoint{5.315938in}{1.484158in}}{\pgfqpoint{5.318772in}{1.477316in}}{\pgfqpoint{5.323816in}{1.472273in}}%
\pgfpathcurveto{\pgfqpoint{5.328860in}{1.467229in}}{\pgfqpoint{5.335701in}{1.464395in}}{\pgfqpoint{5.342834in}{1.464395in}}%
\pgfpathclose%
\pgfusepath{stroke,fill}%
\end{pgfscope}%
\begin{pgfscope}%
\pgfpathrectangle{\pgfqpoint{4.985294in}{0.500000in}}{\pgfqpoint{1.764706in}{1.700000in}}%
\pgfusepath{clip}%
\pgfsetbuttcap%
\pgfsetroundjoin%
\definecolor{currentfill}{rgb}{0.966812,0.762584,0.633643}%
\pgfsetfillcolor{currentfill}%
\pgfsetlinewidth{0.311001pt}%
\definecolor{currentstroke}{rgb}{1.000000,1.000000,1.000000}%
\pgfsetstrokecolor{currentstroke}%
\pgfsetdash{}{0pt}%
\pgfpathmoveto{\pgfqpoint{5.379329in}{1.092762in}}%
\pgfpathcurveto{\pgfqpoint{5.386462in}{1.092762in}}{\pgfqpoint{5.393304in}{1.095596in}}{\pgfqpoint{5.398347in}{1.100639in}}%
\pgfpathcurveto{\pgfqpoint{5.403391in}{1.105683in}}{\pgfqpoint{5.406225in}{1.112525in}}{\pgfqpoint{5.406225in}{1.119658in}}%
\pgfpathcurveto{\pgfqpoint{5.406225in}{1.126790in}}{\pgfqpoint{5.403391in}{1.133632in}}{\pgfqpoint{5.398347in}{1.138676in}}%
\pgfpathcurveto{\pgfqpoint{5.393304in}{1.143719in}}{\pgfqpoint{5.386462in}{1.146553in}}{\pgfqpoint{5.379329in}{1.146553in}}%
\pgfpathcurveto{\pgfqpoint{5.372196in}{1.146553in}}{\pgfqpoint{5.365355in}{1.143719in}}{\pgfqpoint{5.360311in}{1.138676in}}%
\pgfpathcurveto{\pgfqpoint{5.355267in}{1.133632in}}{\pgfqpoint{5.352433in}{1.126790in}}{\pgfqpoint{5.352433in}{1.119658in}}%
\pgfpathcurveto{\pgfqpoint{5.352433in}{1.112525in}}{\pgfqpoint{5.355267in}{1.105683in}}{\pgfqpoint{5.360311in}{1.100639in}}%
\pgfpathcurveto{\pgfqpoint{5.365355in}{1.095596in}}{\pgfqpoint{5.372196in}{1.092762in}}{\pgfqpoint{5.379329in}{1.092762in}}%
\pgfpathclose%
\pgfusepath{stroke,fill}%
\end{pgfscope}%
\begin{pgfscope}%
\pgfpathrectangle{\pgfqpoint{4.985294in}{0.500000in}}{\pgfqpoint{1.764706in}{1.700000in}}%
\pgfusepath{clip}%
\pgfsetbuttcap%
\pgfsetroundjoin%
\definecolor{currentfill}{rgb}{0.966328,0.750560,0.616961}%
\pgfsetfillcolor{currentfill}%
\pgfsetlinewidth{0.311001pt}%
\definecolor{currentstroke}{rgb}{1.000000,1.000000,1.000000}%
\pgfsetstrokecolor{currentstroke}%
\pgfsetdash{}{0pt}%
\pgfpathmoveto{\pgfqpoint{5.481322in}{1.690445in}}%
\pgfpathcurveto{\pgfqpoint{5.488455in}{1.690445in}}{\pgfqpoint{5.495296in}{1.693279in}}{\pgfqpoint{5.500340in}{1.698323in}}%
\pgfpathcurveto{\pgfqpoint{5.505384in}{1.703367in}}{\pgfqpoint{5.508218in}{1.710208in}}{\pgfqpoint{5.508218in}{1.717341in}}%
\pgfpathcurveto{\pgfqpoint{5.508218in}{1.724474in}}{\pgfqpoint{5.505384in}{1.731316in}}{\pgfqpoint{5.500340in}{1.736359in}}%
\pgfpathcurveto{\pgfqpoint{5.495296in}{1.741403in}}{\pgfqpoint{5.488455in}{1.744237in}}{\pgfqpoint{5.481322in}{1.744237in}}%
\pgfpathcurveto{\pgfqpoint{5.474189in}{1.744237in}}{\pgfqpoint{5.467347in}{1.741403in}}{\pgfqpoint{5.462304in}{1.736359in}}%
\pgfpathcurveto{\pgfqpoint{5.457260in}{1.731316in}}{\pgfqpoint{5.454426in}{1.724474in}}{\pgfqpoint{5.454426in}{1.717341in}}%
\pgfpathcurveto{\pgfqpoint{5.454426in}{1.710208in}}{\pgfqpoint{5.457260in}{1.703367in}}{\pgfqpoint{5.462304in}{1.698323in}}%
\pgfpathcurveto{\pgfqpoint{5.467347in}{1.693279in}}{\pgfqpoint{5.474189in}{1.690445in}}{\pgfqpoint{5.481322in}{1.690445in}}%
\pgfpathclose%
\pgfusepath{stroke,fill}%
\end{pgfscope}%
\begin{pgfscope}%
\pgfpathrectangle{\pgfqpoint{4.985294in}{0.500000in}}{\pgfqpoint{1.764706in}{1.700000in}}%
\pgfusepath{clip}%
\pgfsetbuttcap%
\pgfsetroundjoin%
\definecolor{currentfill}{rgb}{0.977657,0.891500,0.822809}%
\pgfsetfillcolor{currentfill}%
\pgfsetlinewidth{0.311001pt}%
\definecolor{currentstroke}{rgb}{1.000000,1.000000,1.000000}%
\pgfsetstrokecolor{currentstroke}%
\pgfsetdash{}{0pt}%
\pgfpathmoveto{\pgfqpoint{6.302109in}{1.547110in}}%
\pgfpathcurveto{\pgfqpoint{6.309242in}{1.547110in}}{\pgfqpoint{6.316084in}{1.549944in}}{\pgfqpoint{6.321128in}{1.554988in}}%
\pgfpathcurveto{\pgfqpoint{6.326171in}{1.560032in}}{\pgfqpoint{6.329005in}{1.566873in}}{\pgfqpoint{6.329005in}{1.574006in}}%
\pgfpathcurveto{\pgfqpoint{6.329005in}{1.581139in}}{\pgfqpoint{6.326171in}{1.587981in}}{\pgfqpoint{6.321128in}{1.593024in}}%
\pgfpathcurveto{\pgfqpoint{6.316084in}{1.598068in}}{\pgfqpoint{6.309242in}{1.600902in}}{\pgfqpoint{6.302109in}{1.600902in}}%
\pgfpathcurveto{\pgfqpoint{6.294977in}{1.600902in}}{\pgfqpoint{6.288135in}{1.598068in}}{\pgfqpoint{6.283091in}{1.593024in}}%
\pgfpathcurveto{\pgfqpoint{6.278048in}{1.587981in}}{\pgfqpoint{6.275214in}{1.581139in}}{\pgfqpoint{6.275214in}{1.574006in}}%
\pgfpathcurveto{\pgfqpoint{6.275214in}{1.566873in}}{\pgfqpoint{6.278048in}{1.560032in}}{\pgfqpoint{6.283091in}{1.554988in}}%
\pgfpathcurveto{\pgfqpoint{6.288135in}{1.549944in}}{\pgfqpoint{6.294977in}{1.547110in}}{\pgfqpoint{6.302109in}{1.547110in}}%
\pgfpathclose%
\pgfusepath{stroke,fill}%
\end{pgfscope}%
\begin{pgfscope}%
\pgfpathrectangle{\pgfqpoint{4.985294in}{0.500000in}}{\pgfqpoint{1.764706in}{1.700000in}}%
\pgfusepath{clip}%
\pgfsetbuttcap%
\pgfsetroundjoin%
\definecolor{currentfill}{rgb}{0.978376,0.897317,0.831308}%
\pgfsetfillcolor{currentfill}%
\pgfsetlinewidth{0.311001pt}%
\definecolor{currentstroke}{rgb}{1.000000,1.000000,1.000000}%
\pgfsetstrokecolor{currentstroke}%
\pgfsetdash{}{0pt}%
\pgfpathmoveto{\pgfqpoint{6.302270in}{1.157587in}}%
\pgfpathcurveto{\pgfqpoint{6.309402in}{1.157587in}}{\pgfqpoint{6.316244in}{1.160421in}}{\pgfqpoint{6.321288in}{1.165464in}}%
\pgfpathcurveto{\pgfqpoint{6.326331in}{1.170508in}}{\pgfqpoint{6.329165in}{1.177350in}}{\pgfqpoint{6.329165in}{1.184483in}}%
\pgfpathcurveto{\pgfqpoint{6.329165in}{1.191615in}}{\pgfqpoint{6.326331in}{1.198457in}}{\pgfqpoint{6.321288in}{1.203501in}}%
\pgfpathcurveto{\pgfqpoint{6.316244in}{1.208544in}}{\pgfqpoint{6.309402in}{1.211378in}}{\pgfqpoint{6.302270in}{1.211378in}}%
\pgfpathcurveto{\pgfqpoint{6.295137in}{1.211378in}}{\pgfqpoint{6.288295in}{1.208544in}}{\pgfqpoint{6.283251in}{1.203501in}}%
\pgfpathcurveto{\pgfqpoint{6.278208in}{1.198457in}}{\pgfqpoint{6.275374in}{1.191615in}}{\pgfqpoint{6.275374in}{1.184483in}}%
\pgfpathcurveto{\pgfqpoint{6.275374in}{1.177350in}}{\pgfqpoint{6.278208in}{1.170508in}}{\pgfqpoint{6.283251in}{1.165464in}}%
\pgfpathcurveto{\pgfqpoint{6.288295in}{1.160421in}}{\pgfqpoint{6.295137in}{1.157587in}}{\pgfqpoint{6.302270in}{1.157587in}}%
\pgfpathclose%
\pgfusepath{stroke,fill}%
\end{pgfscope}%
\begin{pgfscope}%
\pgfpathrectangle{\pgfqpoint{4.985294in}{0.500000in}}{\pgfqpoint{1.764706in}{1.700000in}}%
\pgfusepath{clip}%
\pgfsetbuttcap%
\pgfsetroundjoin%
\definecolor{currentfill}{rgb}{0.967092,0.768560,0.642079}%
\pgfsetfillcolor{currentfill}%
\pgfsetlinewidth{0.311001pt}%
\definecolor{currentstroke}{rgb}{1.000000,1.000000,1.000000}%
\pgfsetstrokecolor{currentstroke}%
\pgfsetdash{}{0pt}%
\pgfpathmoveto{\pgfqpoint{6.193024in}{1.701674in}}%
\pgfpathcurveto{\pgfqpoint{6.200157in}{1.701674in}}{\pgfqpoint{6.206999in}{1.704508in}}{\pgfqpoint{6.212043in}{1.709551in}}%
\pgfpathcurveto{\pgfqpoint{6.217086in}{1.714595in}}{\pgfqpoint{6.219920in}{1.721437in}}{\pgfqpoint{6.219920in}{1.728570in}}%
\pgfpathcurveto{\pgfqpoint{6.219920in}{1.735702in}}{\pgfqpoint{6.217086in}{1.742544in}}{\pgfqpoint{6.212043in}{1.747588in}}%
\pgfpathcurveto{\pgfqpoint{6.206999in}{1.752631in}}{\pgfqpoint{6.200157in}{1.755465in}}{\pgfqpoint{6.193024in}{1.755465in}}%
\pgfpathcurveto{\pgfqpoint{6.185892in}{1.755465in}}{\pgfqpoint{6.179050in}{1.752631in}}{\pgfqpoint{6.174006in}{1.747588in}}%
\pgfpathcurveto{\pgfqpoint{6.168963in}{1.742544in}}{\pgfqpoint{6.166129in}{1.735702in}}{\pgfqpoint{6.166129in}{1.728570in}}%
\pgfpathcurveto{\pgfqpoint{6.166129in}{1.721437in}}{\pgfqpoint{6.168963in}{1.714595in}}{\pgfqpoint{6.174006in}{1.709551in}}%
\pgfpathcurveto{\pgfqpoint{6.179050in}{1.704508in}}{\pgfqpoint{6.185892in}{1.701674in}}{\pgfqpoint{6.193024in}{1.701674in}}%
\pgfpathclose%
\pgfusepath{stroke,fill}%
\end{pgfscope}%
\begin{pgfscope}%
\pgfpathrectangle{\pgfqpoint{4.985294in}{0.500000in}}{\pgfqpoint{1.764706in}{1.700000in}}%
\pgfusepath{clip}%
\pgfsetbuttcap%
\pgfsetroundjoin%
\definecolor{currentfill}{rgb}{0.974412,0.862387,0.780156}%
\pgfsetfillcolor{currentfill}%
\pgfsetlinewidth{0.311001pt}%
\definecolor{currentstroke}{rgb}{1.000000,1.000000,1.000000}%
\pgfsetstrokecolor{currentstroke}%
\pgfsetdash{}{0pt}%
\pgfpathmoveto{\pgfqpoint{5.368989in}{1.300279in}}%
\pgfpathcurveto{\pgfqpoint{5.376122in}{1.300279in}}{\pgfqpoint{5.382964in}{1.303113in}}{\pgfqpoint{5.388007in}{1.308156in}}%
\pgfpathcurveto{\pgfqpoint{5.393051in}{1.313200in}}{\pgfqpoint{5.395885in}{1.320042in}}{\pgfqpoint{5.395885in}{1.327175in}}%
\pgfpathcurveto{\pgfqpoint{5.395885in}{1.334307in}}{\pgfqpoint{5.393051in}{1.341149in}}{\pgfqpoint{5.388007in}{1.346193in}}%
\pgfpathcurveto{\pgfqpoint{5.382964in}{1.351236in}}{\pgfqpoint{5.376122in}{1.354070in}}{\pgfqpoint{5.368989in}{1.354070in}}%
\pgfpathcurveto{\pgfqpoint{5.361856in}{1.354070in}}{\pgfqpoint{5.355015in}{1.351236in}}{\pgfqpoint{5.349971in}{1.346193in}}%
\pgfpathcurveto{\pgfqpoint{5.344927in}{1.341149in}}{\pgfqpoint{5.342093in}{1.334307in}}{\pgfqpoint{5.342093in}{1.327175in}}%
\pgfpathcurveto{\pgfqpoint{5.342093in}{1.320042in}}{\pgfqpoint{5.344927in}{1.313200in}}{\pgfqpoint{5.349971in}{1.308156in}}%
\pgfpathcurveto{\pgfqpoint{5.355015in}{1.303113in}}{\pgfqpoint{5.361856in}{1.300279in}}{\pgfqpoint{5.368989in}{1.300279in}}%
\pgfpathclose%
\pgfusepath{stroke,fill}%
\end{pgfscope}%
\begin{pgfscope}%
\pgfpathrectangle{\pgfqpoint{4.985294in}{0.500000in}}{\pgfqpoint{1.764706in}{1.700000in}}%
\pgfusepath{clip}%
\pgfsetbuttcap%
\pgfsetroundjoin%
\definecolor{currentfill}{rgb}{0.966560,0.756582,0.625273}%
\pgfsetfillcolor{currentfill}%
\pgfsetlinewidth{0.311001pt}%
\definecolor{currentstroke}{rgb}{1.000000,1.000000,1.000000}%
\pgfsetstrokecolor{currentstroke}%
\pgfsetdash{}{0pt}%
\pgfpathmoveto{\pgfqpoint{6.396464in}{1.364303in}}%
\pgfpathcurveto{\pgfqpoint{6.403597in}{1.364303in}}{\pgfqpoint{6.410439in}{1.367137in}}{\pgfqpoint{6.415483in}{1.372181in}}%
\pgfpathcurveto{\pgfqpoint{6.420526in}{1.377224in}}{\pgfqpoint{6.423360in}{1.384066in}}{\pgfqpoint{6.423360in}{1.391199in}}%
\pgfpathcurveto{\pgfqpoint{6.423360in}{1.398332in}}{\pgfqpoint{6.420526in}{1.405173in}}{\pgfqpoint{6.415483in}{1.410217in}}%
\pgfpathcurveto{\pgfqpoint{6.410439in}{1.415261in}}{\pgfqpoint{6.403597in}{1.418095in}}{\pgfqpoint{6.396464in}{1.418095in}}%
\pgfpathcurveto{\pgfqpoint{6.389332in}{1.418095in}}{\pgfqpoint{6.382490in}{1.415261in}}{\pgfqpoint{6.377446in}{1.410217in}}%
\pgfpathcurveto{\pgfqpoint{6.372403in}{1.405173in}}{\pgfqpoint{6.369569in}{1.398332in}}{\pgfqpoint{6.369569in}{1.391199in}}%
\pgfpathcurveto{\pgfqpoint{6.369569in}{1.384066in}}{\pgfqpoint{6.372403in}{1.377224in}}{\pgfqpoint{6.377446in}{1.372181in}}%
\pgfpathcurveto{\pgfqpoint{6.382490in}{1.367137in}}{\pgfqpoint{6.389332in}{1.364303in}}{\pgfqpoint{6.396464in}{1.364303in}}%
\pgfpathclose%
\pgfusepath{stroke,fill}%
\end{pgfscope}%
\begin{pgfscope}%
\pgfpathrectangle{\pgfqpoint{4.985294in}{0.500000in}}{\pgfqpoint{1.764706in}{1.700000in}}%
\pgfusepath{clip}%
\pgfsetbuttcap%
\pgfsetroundjoin%
\definecolor{currentfill}{rgb}{0.968105,0.786346,0.667739}%
\pgfsetfillcolor{currentfill}%
\pgfsetlinewidth{0.311001pt}%
\definecolor{currentstroke}{rgb}{1.000000,1.000000,1.000000}%
\pgfsetstrokecolor{currentstroke}%
\pgfsetdash{}{0pt}%
\pgfpathmoveto{\pgfqpoint{5.572375in}{0.976922in}}%
\pgfpathcurveto{\pgfqpoint{5.579508in}{0.976922in}}{\pgfqpoint{5.586349in}{0.979756in}}{\pgfqpoint{5.591393in}{0.984800in}}%
\pgfpathcurveto{\pgfqpoint{5.596437in}{0.989844in}}{\pgfqpoint{5.599271in}{0.996685in}}{\pgfqpoint{5.599271in}{1.003818in}}%
\pgfpathcurveto{\pgfqpoint{5.599271in}{1.010951in}}{\pgfqpoint{5.596437in}{1.017793in}}{\pgfqpoint{5.591393in}{1.022836in}}%
\pgfpathcurveto{\pgfqpoint{5.586349in}{1.027880in}}{\pgfqpoint{5.579508in}{1.030714in}}{\pgfqpoint{5.572375in}{1.030714in}}%
\pgfpathcurveto{\pgfqpoint{5.565242in}{1.030714in}}{\pgfqpoint{5.558400in}{1.027880in}}{\pgfqpoint{5.553357in}{1.022836in}}%
\pgfpathcurveto{\pgfqpoint{5.548313in}{1.017793in}}{\pgfqpoint{5.545479in}{1.010951in}}{\pgfqpoint{5.545479in}{1.003818in}}%
\pgfpathcurveto{\pgfqpoint{5.545479in}{0.996685in}}{\pgfqpoint{5.548313in}{0.989844in}}{\pgfqpoint{5.553357in}{0.984800in}}%
\pgfpathcurveto{\pgfqpoint{5.558400in}{0.979756in}}{\pgfqpoint{5.565242in}{0.976922in}}{\pgfqpoint{5.572375in}{0.976922in}}%
\pgfpathclose%
\pgfusepath{stroke,fill}%
\end{pgfscope}%
\begin{pgfscope}%
\pgfpathrectangle{\pgfqpoint{4.985294in}{0.500000in}}{\pgfqpoint{1.764706in}{1.700000in}}%
\pgfusepath{clip}%
\pgfsetbuttcap%
\pgfsetroundjoin%
\definecolor{currentfill}{rgb}{0.966560,0.756582,0.625273}%
\pgfsetfillcolor{currentfill}%
\pgfsetlinewidth{0.311001pt}%
\definecolor{currentstroke}{rgb}{1.000000,1.000000,1.000000}%
\pgfsetstrokecolor{currentstroke}%
\pgfsetdash{}{0pt}%
\pgfpathmoveto{\pgfqpoint{6.197005in}{1.484604in}}%
\pgfpathcurveto{\pgfqpoint{6.204138in}{1.484604in}}{\pgfqpoint{6.210980in}{1.487438in}}{\pgfqpoint{6.216023in}{1.492481in}}%
\pgfpathcurveto{\pgfqpoint{6.221067in}{1.497525in}}{\pgfqpoint{6.223901in}{1.504367in}}{\pgfqpoint{6.223901in}{1.511500in}}%
\pgfpathcurveto{\pgfqpoint{6.223901in}{1.518632in}}{\pgfqpoint{6.221067in}{1.525474in}}{\pgfqpoint{6.216023in}{1.530518in}}%
\pgfpathcurveto{\pgfqpoint{6.210980in}{1.535561in}}{\pgfqpoint{6.204138in}{1.538395in}}{\pgfqpoint{6.197005in}{1.538395in}}%
\pgfpathcurveto{\pgfqpoint{6.189872in}{1.538395in}}{\pgfqpoint{6.183031in}{1.535561in}}{\pgfqpoint{6.177987in}{1.530518in}}%
\pgfpathcurveto{\pgfqpoint{6.172943in}{1.525474in}}{\pgfqpoint{6.170109in}{1.518632in}}{\pgfqpoint{6.170109in}{1.511500in}}%
\pgfpathcurveto{\pgfqpoint{6.170109in}{1.504367in}}{\pgfqpoint{6.172943in}{1.497525in}}{\pgfqpoint{6.177987in}{1.492481in}}%
\pgfpathcurveto{\pgfqpoint{6.183031in}{1.487438in}}{\pgfqpoint{6.189872in}{1.484604in}}{\pgfqpoint{6.197005in}{1.484604in}}%
\pgfpathclose%
\pgfusepath{stroke,fill}%
\end{pgfscope}%
\begin{pgfscope}%
\pgfpathrectangle{\pgfqpoint{4.985294in}{0.500000in}}{\pgfqpoint{1.764706in}{1.700000in}}%
\pgfusepath{clip}%
\pgfsetbuttcap%
\pgfsetroundjoin%
\definecolor{currentfill}{rgb}{0.980678,0.914765,0.856766}%
\pgfsetfillcolor{currentfill}%
\pgfsetlinewidth{0.311001pt}%
\definecolor{currentstroke}{rgb}{1.000000,1.000000,1.000000}%
\pgfsetstrokecolor{currentstroke}%
\pgfsetdash{}{0pt}%
\pgfpathmoveto{\pgfqpoint{5.416981in}{1.335389in}}%
\pgfpathcurveto{\pgfqpoint{5.424114in}{1.335389in}}{\pgfqpoint{5.430956in}{1.338223in}}{\pgfqpoint{5.435999in}{1.343267in}}%
\pgfpathcurveto{\pgfqpoint{5.441043in}{1.348311in}}{\pgfqpoint{5.443877in}{1.355152in}}{\pgfqpoint{5.443877in}{1.362285in}}%
\pgfpathcurveto{\pgfqpoint{5.443877in}{1.369418in}}{\pgfqpoint{5.441043in}{1.376260in}}{\pgfqpoint{5.435999in}{1.381303in}}%
\pgfpathcurveto{\pgfqpoint{5.430956in}{1.386347in}}{\pgfqpoint{5.424114in}{1.389181in}}{\pgfqpoint{5.416981in}{1.389181in}}%
\pgfpathcurveto{\pgfqpoint{5.409848in}{1.389181in}}{\pgfqpoint{5.403007in}{1.386347in}}{\pgfqpoint{5.397963in}{1.381303in}}%
\pgfpathcurveto{\pgfqpoint{5.392919in}{1.376260in}}{\pgfqpoint{5.390085in}{1.369418in}}{\pgfqpoint{5.390085in}{1.362285in}}%
\pgfpathcurveto{\pgfqpoint{5.390085in}{1.355152in}}{\pgfqpoint{5.392919in}{1.348311in}}{\pgfqpoint{5.397963in}{1.343267in}}%
\pgfpathcurveto{\pgfqpoint{5.403007in}{1.338223in}}{\pgfqpoint{5.409848in}{1.335389in}}{\pgfqpoint{5.416981in}{1.335389in}}%
\pgfpathclose%
\pgfusepath{stroke,fill}%
\end{pgfscope}%
\begin{pgfscope}%
\pgfpathrectangle{\pgfqpoint{4.985294in}{0.500000in}}{\pgfqpoint{1.764706in}{1.700000in}}%
\pgfusepath{clip}%
\pgfsetbuttcap%
\pgfsetroundjoin%
\definecolor{currentfill}{rgb}{0.965169,0.707764,0.560659}%
\pgfsetfillcolor{currentfill}%
\pgfsetlinewidth{0.311001pt}%
\definecolor{currentstroke}{rgb}{1.000000,1.000000,1.000000}%
\pgfsetstrokecolor{currentstroke}%
\pgfsetdash{}{0pt}%
\pgfpathmoveto{\pgfqpoint{5.353733in}{1.500389in}}%
\pgfpathcurveto{\pgfqpoint{5.360865in}{1.500389in}}{\pgfqpoint{5.367707in}{1.503223in}}{\pgfqpoint{5.372751in}{1.508267in}}%
\pgfpathcurveto{\pgfqpoint{5.377794in}{1.513310in}}{\pgfqpoint{5.380628in}{1.520152in}}{\pgfqpoint{5.380628in}{1.527285in}}%
\pgfpathcurveto{\pgfqpoint{5.380628in}{1.534418in}}{\pgfqpoint{5.377794in}{1.541259in}}{\pgfqpoint{5.372751in}{1.546303in}}%
\pgfpathcurveto{\pgfqpoint{5.367707in}{1.551347in}}{\pgfqpoint{5.360865in}{1.554181in}}{\pgfqpoint{5.353733in}{1.554181in}}%
\pgfpathcurveto{\pgfqpoint{5.346600in}{1.554181in}}{\pgfqpoint{5.339758in}{1.551347in}}{\pgfqpoint{5.334714in}{1.546303in}}%
\pgfpathcurveto{\pgfqpoint{5.329671in}{1.541259in}}{\pgfqpoint{5.326837in}{1.534418in}}{\pgfqpoint{5.326837in}{1.527285in}}%
\pgfpathcurveto{\pgfqpoint{5.326837in}{1.520152in}}{\pgfqpoint{5.329671in}{1.513310in}}{\pgfqpoint{5.334714in}{1.508267in}}%
\pgfpathcurveto{\pgfqpoint{5.339758in}{1.503223in}}{\pgfqpoint{5.346600in}{1.500389in}}{\pgfqpoint{5.353733in}{1.500389in}}%
\pgfpathclose%
\pgfusepath{stroke,fill}%
\end{pgfscope}%
\begin{pgfscope}%
\pgfpathrectangle{\pgfqpoint{4.985294in}{0.500000in}}{\pgfqpoint{1.764706in}{1.700000in}}%
\pgfusepath{clip}%
\pgfsetbuttcap%
\pgfsetroundjoin%
\definecolor{currentfill}{rgb}{0.972726,0.844889,0.754401}%
\pgfsetfillcolor{currentfill}%
\pgfsetlinewidth{0.311001pt}%
\definecolor{currentstroke}{rgb}{1.000000,1.000000,1.000000}%
\pgfsetstrokecolor{currentstroke}%
\pgfsetdash{}{0pt}%
\pgfpathmoveto{\pgfqpoint{6.243358in}{1.234142in}}%
\pgfpathcurveto{\pgfqpoint{6.250491in}{1.234142in}}{\pgfqpoint{6.257333in}{1.236976in}}{\pgfqpoint{6.262377in}{1.242020in}}%
\pgfpathcurveto{\pgfqpoint{6.267420in}{1.247064in}}{\pgfqpoint{6.270254in}{1.253905in}}{\pgfqpoint{6.270254in}{1.261038in}}%
\pgfpathcurveto{\pgfqpoint{6.270254in}{1.268171in}}{\pgfqpoint{6.267420in}{1.275013in}}{\pgfqpoint{6.262377in}{1.280056in}}%
\pgfpathcurveto{\pgfqpoint{6.257333in}{1.285100in}}{\pgfqpoint{6.250491in}{1.287934in}}{\pgfqpoint{6.243358in}{1.287934in}}%
\pgfpathcurveto{\pgfqpoint{6.236226in}{1.287934in}}{\pgfqpoint{6.229384in}{1.285100in}}{\pgfqpoint{6.224340in}{1.280056in}}%
\pgfpathcurveto{\pgfqpoint{6.219297in}{1.275013in}}{\pgfqpoint{6.216463in}{1.268171in}}{\pgfqpoint{6.216463in}{1.261038in}}%
\pgfpathcurveto{\pgfqpoint{6.216463in}{1.253905in}}{\pgfqpoint{6.219297in}{1.247064in}}{\pgfqpoint{6.224340in}{1.242020in}}%
\pgfpathcurveto{\pgfqpoint{6.229384in}{1.236976in}}{\pgfqpoint{6.236226in}{1.234142in}}{\pgfqpoint{6.243358in}{1.234142in}}%
\pgfpathclose%
\pgfusepath{stroke,fill}%
\end{pgfscope}%
\begin{pgfscope}%
\pgfpathrectangle{\pgfqpoint{4.985294in}{0.500000in}}{\pgfqpoint{1.764706in}{1.700000in}}%
\pgfusepath{clip}%
\pgfsetbuttcap%
\pgfsetroundjoin%
\definecolor{currentfill}{rgb}{0.978376,0.897317,0.831308}%
\pgfsetfillcolor{currentfill}%
\pgfsetlinewidth{0.311001pt}%
\definecolor{currentstroke}{rgb}{1.000000,1.000000,1.000000}%
\pgfsetstrokecolor{currentstroke}%
\pgfsetdash{}{0pt}%
\pgfpathmoveto{\pgfqpoint{6.283662in}{1.394943in}}%
\pgfpathcurveto{\pgfqpoint{6.290795in}{1.394943in}}{\pgfqpoint{6.297637in}{1.397777in}}{\pgfqpoint{6.302681in}{1.402820in}}%
\pgfpathcurveto{\pgfqpoint{6.307724in}{1.407864in}}{\pgfqpoint{6.310558in}{1.414706in}}{\pgfqpoint{6.310558in}{1.421839in}}%
\pgfpathcurveto{\pgfqpoint{6.310558in}{1.428971in}}{\pgfqpoint{6.307724in}{1.435813in}}{\pgfqpoint{6.302681in}{1.440857in}}%
\pgfpathcurveto{\pgfqpoint{6.297637in}{1.445900in}}{\pgfqpoint{6.290795in}{1.448734in}}{\pgfqpoint{6.283662in}{1.448734in}}%
\pgfpathcurveto{\pgfqpoint{6.276530in}{1.448734in}}{\pgfqpoint{6.269688in}{1.445900in}}{\pgfqpoint{6.264644in}{1.440857in}}%
\pgfpathcurveto{\pgfqpoint{6.259601in}{1.435813in}}{\pgfqpoint{6.256767in}{1.428971in}}{\pgfqpoint{6.256767in}{1.421839in}}%
\pgfpathcurveto{\pgfqpoint{6.256767in}{1.414706in}}{\pgfqpoint{6.259601in}{1.407864in}}{\pgfqpoint{6.264644in}{1.402820in}}%
\pgfpathcurveto{\pgfqpoint{6.269688in}{1.397777in}}{\pgfqpoint{6.276530in}{1.394943in}}{\pgfqpoint{6.283662in}{1.394943in}}%
\pgfpathclose%
\pgfusepath{stroke,fill}%
\end{pgfscope}%
\begin{pgfscope}%
\pgfpathrectangle{\pgfqpoint{4.985294in}{0.500000in}}{\pgfqpoint{1.764706in}{1.700000in}}%
\pgfusepath{clip}%
\pgfsetbuttcap%
\pgfsetroundjoin%
\definecolor{currentfill}{rgb}{0.964433,0.670254,0.515093}%
\pgfsetfillcolor{currentfill}%
\pgfsetlinewidth{0.311001pt}%
\definecolor{currentstroke}{rgb}{1.000000,1.000000,1.000000}%
\pgfsetstrokecolor{currentstroke}%
\pgfsetdash{}{0pt}%
\pgfpathmoveto{\pgfqpoint{6.172999in}{1.190455in}}%
\pgfpathcurveto{\pgfqpoint{6.180132in}{1.190455in}}{\pgfqpoint{6.186974in}{1.193289in}}{\pgfqpoint{6.192018in}{1.198333in}}%
\pgfpathcurveto{\pgfqpoint{6.197061in}{1.203376in}}{\pgfqpoint{6.199895in}{1.210218in}}{\pgfqpoint{6.199895in}{1.217351in}}%
\pgfpathcurveto{\pgfqpoint{6.199895in}{1.224484in}}{\pgfqpoint{6.197061in}{1.231325in}}{\pgfqpoint{6.192018in}{1.236369in}}%
\pgfpathcurveto{\pgfqpoint{6.186974in}{1.241413in}}{\pgfqpoint{6.180132in}{1.244246in}}{\pgfqpoint{6.172999in}{1.244246in}}%
\pgfpathcurveto{\pgfqpoint{6.165867in}{1.244246in}}{\pgfqpoint{6.159025in}{1.241413in}}{\pgfqpoint{6.153981in}{1.236369in}}%
\pgfpathcurveto{\pgfqpoint{6.148938in}{1.231325in}}{\pgfqpoint{6.146104in}{1.224484in}}{\pgfqpoint{6.146104in}{1.217351in}}%
\pgfpathcurveto{\pgfqpoint{6.146104in}{1.210218in}}{\pgfqpoint{6.148938in}{1.203376in}}{\pgfqpoint{6.153981in}{1.198333in}}%
\pgfpathcurveto{\pgfqpoint{6.159025in}{1.193289in}}{\pgfqpoint{6.165867in}{1.190455in}}{\pgfqpoint{6.172999in}{1.190455in}}%
\pgfpathclose%
\pgfusepath{stroke,fill}%
\end{pgfscope}%
\begin{pgfscope}%
\pgfpathrectangle{\pgfqpoint{4.985294in}{0.500000in}}{\pgfqpoint{1.764706in}{1.700000in}}%
\pgfusepath{clip}%
\pgfsetbuttcap%
\pgfsetroundjoin%
\definecolor{currentfill}{rgb}{0.967735,0.780441,0.659127}%
\pgfsetfillcolor{currentfill}%
\pgfsetlinewidth{0.311001pt}%
\definecolor{currentstroke}{rgb}{1.000000,1.000000,1.000000}%
\pgfsetstrokecolor{currentstroke}%
\pgfsetdash{}{0pt}%
\pgfpathmoveto{\pgfqpoint{6.279418in}{1.659419in}}%
\pgfpathcurveto{\pgfqpoint{6.286551in}{1.659419in}}{\pgfqpoint{6.293392in}{1.662253in}}{\pgfqpoint{6.298436in}{1.667297in}}%
\pgfpathcurveto{\pgfqpoint{6.303480in}{1.672341in}}{\pgfqpoint{6.306314in}{1.679182in}}{\pgfqpoint{6.306314in}{1.686315in}}%
\pgfpathcurveto{\pgfqpoint{6.306314in}{1.693448in}}{\pgfqpoint{6.303480in}{1.700290in}}{\pgfqpoint{6.298436in}{1.705333in}}%
\pgfpathcurveto{\pgfqpoint{6.293392in}{1.710377in}}{\pgfqpoint{6.286551in}{1.713211in}}{\pgfqpoint{6.279418in}{1.713211in}}%
\pgfpathcurveto{\pgfqpoint{6.272285in}{1.713211in}}{\pgfqpoint{6.265443in}{1.710377in}}{\pgfqpoint{6.260400in}{1.705333in}}%
\pgfpathcurveto{\pgfqpoint{6.255356in}{1.700290in}}{\pgfqpoint{6.252522in}{1.693448in}}{\pgfqpoint{6.252522in}{1.686315in}}%
\pgfpathcurveto{\pgfqpoint{6.252522in}{1.679182in}}{\pgfqpoint{6.255356in}{1.672341in}}{\pgfqpoint{6.260400in}{1.667297in}}%
\pgfpathcurveto{\pgfqpoint{6.265443in}{1.662253in}}{\pgfqpoint{6.272285in}{1.659419in}}{\pgfqpoint{6.279418in}{1.659419in}}%
\pgfpathclose%
\pgfusepath{stroke,fill}%
\end{pgfscope}%
\begin{pgfscope}%
\pgfpathrectangle{\pgfqpoint{4.985294in}{0.500000in}}{\pgfqpoint{1.764706in}{1.700000in}}%
\pgfusepath{clip}%
\pgfsetbuttcap%
\pgfsetroundjoin%
\definecolor{currentfill}{rgb}{0.954476,0.470822,0.323110}%
\pgfsetfillcolor{currentfill}%
\pgfsetlinewidth{0.311001pt}%
\definecolor{currentstroke}{rgb}{1.000000,1.000000,1.000000}%
\pgfsetstrokecolor{currentstroke}%
\pgfsetdash{}{0pt}%
\pgfpathmoveto{\pgfqpoint{6.215020in}{1.784042in}}%
\pgfpathcurveto{\pgfqpoint{6.222152in}{1.784042in}}{\pgfqpoint{6.228994in}{1.786875in}}{\pgfqpoint{6.234038in}{1.791919in}}%
\pgfpathcurveto{\pgfqpoint{6.239081in}{1.796963in}}{\pgfqpoint{6.241915in}{1.803804in}}{\pgfqpoint{6.241915in}{1.810937in}}%
\pgfpathcurveto{\pgfqpoint{6.241915in}{1.818070in}}{\pgfqpoint{6.239081in}{1.824912in}}{\pgfqpoint{6.234038in}{1.829955in}}%
\pgfpathcurveto{\pgfqpoint{6.228994in}{1.834999in}}{\pgfqpoint{6.222152in}{1.837833in}}{\pgfqpoint{6.215020in}{1.837833in}}%
\pgfpathcurveto{\pgfqpoint{6.207887in}{1.837833in}}{\pgfqpoint{6.201045in}{1.834999in}}{\pgfqpoint{6.196001in}{1.829955in}}%
\pgfpathcurveto{\pgfqpoint{6.190958in}{1.824912in}}{\pgfqpoint{6.188124in}{1.818070in}}{\pgfqpoint{6.188124in}{1.810937in}}%
\pgfpathcurveto{\pgfqpoint{6.188124in}{1.803804in}}{\pgfqpoint{6.190958in}{1.796963in}}{\pgfqpoint{6.196001in}{1.791919in}}%
\pgfpathcurveto{\pgfqpoint{6.201045in}{1.786875in}}{\pgfqpoint{6.207887in}{1.784042in}}{\pgfqpoint{6.215020in}{1.784042in}}%
\pgfpathclose%
\pgfusepath{stroke,fill}%
\end{pgfscope}%
\begin{pgfscope}%
\pgfpathrectangle{\pgfqpoint{4.985294in}{0.500000in}}{\pgfqpoint{1.764706in}{1.700000in}}%
\pgfusepath{clip}%
\pgfsetbuttcap%
\pgfsetroundjoin%
\definecolor{currentfill}{rgb}{0.973832,0.856556,0.771584}%
\pgfsetfillcolor{currentfill}%
\pgfsetlinewidth{0.311001pt}%
\definecolor{currentstroke}{rgb}{1.000000,1.000000,1.000000}%
\pgfsetstrokecolor{currentstroke}%
\pgfsetdash{}{0pt}%
\pgfpathmoveto{\pgfqpoint{6.199109in}{1.608898in}}%
\pgfpathcurveto{\pgfqpoint{6.206242in}{1.608898in}}{\pgfqpoint{6.213084in}{1.611732in}}{\pgfqpoint{6.218128in}{1.616776in}}%
\pgfpathcurveto{\pgfqpoint{6.223171in}{1.621819in}}{\pgfqpoint{6.226005in}{1.628661in}}{\pgfqpoint{6.226005in}{1.635794in}}%
\pgfpathcurveto{\pgfqpoint{6.226005in}{1.642927in}}{\pgfqpoint{6.223171in}{1.649768in}}{\pgfqpoint{6.218128in}{1.654812in}}%
\pgfpathcurveto{\pgfqpoint{6.213084in}{1.659856in}}{\pgfqpoint{6.206242in}{1.662689in}}{\pgfqpoint{6.199109in}{1.662689in}}%
\pgfpathcurveto{\pgfqpoint{6.191977in}{1.662689in}}{\pgfqpoint{6.185135in}{1.659856in}}{\pgfqpoint{6.180091in}{1.654812in}}%
\pgfpathcurveto{\pgfqpoint{6.175048in}{1.649768in}}{\pgfqpoint{6.172214in}{1.642927in}}{\pgfqpoint{6.172214in}{1.635794in}}%
\pgfpathcurveto{\pgfqpoint{6.172214in}{1.628661in}}{\pgfqpoint{6.175048in}{1.621819in}}{\pgfqpoint{6.180091in}{1.616776in}}%
\pgfpathcurveto{\pgfqpoint{6.185135in}{1.611732in}}{\pgfqpoint{6.191977in}{1.608898in}}{\pgfqpoint{6.199109in}{1.608898in}}%
\pgfpathclose%
\pgfusepath{stroke,fill}%
\end{pgfscope}%
\begin{pgfscope}%
\pgfpathrectangle{\pgfqpoint{4.985294in}{0.500000in}}{\pgfqpoint{1.764706in}{1.700000in}}%
\pgfusepath{clip}%
\pgfsetbuttcap%
\pgfsetroundjoin%
\definecolor{currentfill}{rgb}{0.979891,0.908948,0.848279}%
\pgfsetfillcolor{currentfill}%
\pgfsetlinewidth{0.311001pt}%
\definecolor{currentstroke}{rgb}{1.000000,1.000000,1.000000}%
\pgfsetstrokecolor{currentstroke}%
\pgfsetdash{}{0pt}%
\pgfpathmoveto{\pgfqpoint{5.398956in}{1.322001in}}%
\pgfpathcurveto{\pgfqpoint{5.406089in}{1.322001in}}{\pgfqpoint{5.412931in}{1.324835in}}{\pgfqpoint{5.417974in}{1.329878in}}%
\pgfpathcurveto{\pgfqpoint{5.423018in}{1.334922in}}{\pgfqpoint{5.425852in}{1.341764in}}{\pgfqpoint{5.425852in}{1.348896in}}%
\pgfpathcurveto{\pgfqpoint{5.425852in}{1.356029in}}{\pgfqpoint{5.423018in}{1.362871in}}{\pgfqpoint{5.417974in}{1.367915in}}%
\pgfpathcurveto{\pgfqpoint{5.412931in}{1.372958in}}{\pgfqpoint{5.406089in}{1.375792in}}{\pgfqpoint{5.398956in}{1.375792in}}%
\pgfpathcurveto{\pgfqpoint{5.391823in}{1.375792in}}{\pgfqpoint{5.384982in}{1.372958in}}{\pgfqpoint{5.379938in}{1.367915in}}%
\pgfpathcurveto{\pgfqpoint{5.374894in}{1.362871in}}{\pgfqpoint{5.372061in}{1.356029in}}{\pgfqpoint{5.372061in}{1.348896in}}%
\pgfpathcurveto{\pgfqpoint{5.372061in}{1.341764in}}{\pgfqpoint{5.374894in}{1.334922in}}{\pgfqpoint{5.379938in}{1.329878in}}%
\pgfpathcurveto{\pgfqpoint{5.384982in}{1.324835in}}{\pgfqpoint{5.391823in}{1.322001in}}{\pgfqpoint{5.398956in}{1.322001in}}%
\pgfpathclose%
\pgfusepath{stroke,fill}%
\end{pgfscope}%
\begin{pgfscope}%
\pgfpathrectangle{\pgfqpoint{4.985294in}{0.500000in}}{\pgfqpoint{1.764706in}{1.700000in}}%
\pgfusepath{clip}%
\pgfsetbuttcap%
\pgfsetroundjoin%
\definecolor{currentfill}{rgb}{0.976961,0.885681,0.814303}%
\pgfsetfillcolor{currentfill}%
\pgfsetlinewidth{0.311001pt}%
\definecolor{currentstroke}{rgb}{1.000000,1.000000,1.000000}%
\pgfsetstrokecolor{currentstroke}%
\pgfsetdash{}{0pt}%
\pgfpathmoveto{\pgfqpoint{5.464696in}{1.168226in}}%
\pgfpathcurveto{\pgfqpoint{5.471829in}{1.168226in}}{\pgfqpoint{5.478670in}{1.171060in}}{\pgfqpoint{5.483714in}{1.176103in}}%
\pgfpathcurveto{\pgfqpoint{5.488758in}{1.181147in}}{\pgfqpoint{5.491592in}{1.187989in}}{\pgfqpoint{5.491592in}{1.195121in}}%
\pgfpathcurveto{\pgfqpoint{5.491592in}{1.202254in}}{\pgfqpoint{5.488758in}{1.209096in}}{\pgfqpoint{5.483714in}{1.214140in}}%
\pgfpathcurveto{\pgfqpoint{5.478670in}{1.219183in}}{\pgfqpoint{5.471829in}{1.222017in}}{\pgfqpoint{5.464696in}{1.222017in}}%
\pgfpathcurveto{\pgfqpoint{5.457563in}{1.222017in}}{\pgfqpoint{5.450721in}{1.219183in}}{\pgfqpoint{5.445678in}{1.214140in}}%
\pgfpathcurveto{\pgfqpoint{5.440634in}{1.209096in}}{\pgfqpoint{5.437800in}{1.202254in}}{\pgfqpoint{5.437800in}{1.195121in}}%
\pgfpathcurveto{\pgfqpoint{5.437800in}{1.187989in}}{\pgfqpoint{5.440634in}{1.181147in}}{\pgfqpoint{5.445678in}{1.176103in}}%
\pgfpathcurveto{\pgfqpoint{5.450721in}{1.171060in}}{\pgfqpoint{5.457563in}{1.168226in}}{\pgfqpoint{5.464696in}{1.168226in}}%
\pgfpathclose%
\pgfusepath{stroke,fill}%
\end{pgfscope}%
\begin{pgfscope}%
\pgfpathrectangle{\pgfqpoint{4.985294in}{0.500000in}}{\pgfqpoint{1.764706in}{1.700000in}}%
\pgfusepath{clip}%
\pgfsetbuttcap%
\pgfsetroundjoin%
\definecolor{currentfill}{rgb}{0.969359,0.803954,0.693832}%
\pgfsetfillcolor{currentfill}%
\pgfsetlinewidth{0.311001pt}%
\definecolor{currentstroke}{rgb}{1.000000,1.000000,1.000000}%
\pgfsetstrokecolor{currentstroke}%
\pgfsetdash{}{0pt}%
\pgfpathmoveto{\pgfqpoint{6.227027in}{1.279937in}}%
\pgfpathcurveto{\pgfqpoint{6.234160in}{1.279937in}}{\pgfqpoint{6.241002in}{1.282771in}}{\pgfqpoint{6.246045in}{1.287815in}}%
\pgfpathcurveto{\pgfqpoint{6.251089in}{1.292858in}}{\pgfqpoint{6.253923in}{1.299700in}}{\pgfqpoint{6.253923in}{1.306833in}}%
\pgfpathcurveto{\pgfqpoint{6.253923in}{1.313966in}}{\pgfqpoint{6.251089in}{1.320807in}}{\pgfqpoint{6.246045in}{1.325851in}}%
\pgfpathcurveto{\pgfqpoint{6.241002in}{1.330895in}}{\pgfqpoint{6.234160in}{1.333728in}}{\pgfqpoint{6.227027in}{1.333728in}}%
\pgfpathcurveto{\pgfqpoint{6.219894in}{1.333728in}}{\pgfqpoint{6.213053in}{1.330895in}}{\pgfqpoint{6.208009in}{1.325851in}}%
\pgfpathcurveto{\pgfqpoint{6.202965in}{1.320807in}}{\pgfqpoint{6.200132in}{1.313966in}}{\pgfqpoint{6.200132in}{1.306833in}}%
\pgfpathcurveto{\pgfqpoint{6.200132in}{1.299700in}}{\pgfqpoint{6.202965in}{1.292858in}}{\pgfqpoint{6.208009in}{1.287815in}}%
\pgfpathcurveto{\pgfqpoint{6.213053in}{1.282771in}}{\pgfqpoint{6.219894in}{1.279937in}}{\pgfqpoint{6.227027in}{1.279937in}}%
\pgfpathclose%
\pgfusepath{stroke,fill}%
\end{pgfscope}%
\begin{pgfscope}%
\pgfpathrectangle{\pgfqpoint{4.985294in}{0.500000in}}{\pgfqpoint{1.764706in}{1.700000in}}%
\pgfusepath{clip}%
\pgfsetbuttcap%
\pgfsetroundjoin%
\definecolor{currentfill}{rgb}{0.965928,0.738443,0.600540}%
\pgfsetfillcolor{currentfill}%
\pgfsetlinewidth{0.311001pt}%
\definecolor{currentstroke}{rgb}{1.000000,1.000000,1.000000}%
\pgfsetstrokecolor{currentstroke}%
\pgfsetdash{}{0pt}%
\pgfpathmoveto{\pgfqpoint{5.516626in}{1.238231in}}%
\pgfpathcurveto{\pgfqpoint{5.523759in}{1.238231in}}{\pgfqpoint{5.530601in}{1.241065in}}{\pgfqpoint{5.535645in}{1.246109in}}%
\pgfpathcurveto{\pgfqpoint{5.540688in}{1.251152in}}{\pgfqpoint{5.543522in}{1.257994in}}{\pgfqpoint{5.543522in}{1.265127in}}%
\pgfpathcurveto{\pgfqpoint{5.543522in}{1.272260in}}{\pgfqpoint{5.540688in}{1.279101in}}{\pgfqpoint{5.535645in}{1.284145in}}%
\pgfpathcurveto{\pgfqpoint{5.530601in}{1.289189in}}{\pgfqpoint{5.523759in}{1.292023in}}{\pgfqpoint{5.516626in}{1.292023in}}%
\pgfpathcurveto{\pgfqpoint{5.509494in}{1.292023in}}{\pgfqpoint{5.502652in}{1.289189in}}{\pgfqpoint{5.497608in}{1.284145in}}%
\pgfpathcurveto{\pgfqpoint{5.492565in}{1.279101in}}{\pgfqpoint{5.489731in}{1.272260in}}{\pgfqpoint{5.489731in}{1.265127in}}%
\pgfpathcurveto{\pgfqpoint{5.489731in}{1.257994in}}{\pgfqpoint{5.492565in}{1.251152in}}{\pgfqpoint{5.497608in}{1.246109in}}%
\pgfpathcurveto{\pgfqpoint{5.502652in}{1.241065in}}{\pgfqpoint{5.509494in}{1.238231in}}{\pgfqpoint{5.516626in}{1.238231in}}%
\pgfpathclose%
\pgfusepath{stroke,fill}%
\end{pgfscope}%
\begin{pgfscope}%
\pgfpathrectangle{\pgfqpoint{4.985294in}{0.500000in}}{\pgfqpoint{1.764706in}{1.700000in}}%
\pgfusepath{clip}%
\pgfsetbuttcap%
\pgfsetroundjoin%
\definecolor{currentfill}{rgb}{0.970718,0.821518,0.719872}%
\pgfsetfillcolor{currentfill}%
\pgfsetlinewidth{0.311001pt}%
\definecolor{currentstroke}{rgb}{1.000000,1.000000,1.000000}%
\pgfsetstrokecolor{currentstroke}%
\pgfsetdash{}{0pt}%
\pgfpathmoveto{\pgfqpoint{5.468991in}{1.631141in}}%
\pgfpathcurveto{\pgfqpoint{5.476123in}{1.631141in}}{\pgfqpoint{5.482965in}{1.633975in}}{\pgfqpoint{5.488009in}{1.639018in}}%
\pgfpathcurveto{\pgfqpoint{5.493052in}{1.644062in}}{\pgfqpoint{5.495886in}{1.650904in}}{\pgfqpoint{5.495886in}{1.658036in}}%
\pgfpathcurveto{\pgfqpoint{5.495886in}{1.665169in}}{\pgfqpoint{5.493052in}{1.672011in}}{\pgfqpoint{5.488009in}{1.677055in}}%
\pgfpathcurveto{\pgfqpoint{5.482965in}{1.682098in}}{\pgfqpoint{5.476123in}{1.684932in}}{\pgfqpoint{5.468991in}{1.684932in}}%
\pgfpathcurveto{\pgfqpoint{5.461858in}{1.684932in}}{\pgfqpoint{5.455016in}{1.682098in}}{\pgfqpoint{5.449972in}{1.677055in}}%
\pgfpathcurveto{\pgfqpoint{5.444929in}{1.672011in}}{\pgfqpoint{5.442095in}{1.665169in}}{\pgfqpoint{5.442095in}{1.658036in}}%
\pgfpathcurveto{\pgfqpoint{5.442095in}{1.650904in}}{\pgfqpoint{5.444929in}{1.644062in}}{\pgfqpoint{5.449972in}{1.639018in}}%
\pgfpathcurveto{\pgfqpoint{5.455016in}{1.633975in}}{\pgfqpoint{5.461858in}{1.631141in}}{\pgfqpoint{5.468991in}{1.631141in}}%
\pgfpathclose%
\pgfusepath{stroke,fill}%
\end{pgfscope}%
\begin{pgfscope}%
\pgfpathrectangle{\pgfqpoint{4.985294in}{0.500000in}}{\pgfqpoint{1.764706in}{1.700000in}}%
\pgfusepath{clip}%
\pgfsetbuttcap%
\pgfsetroundjoin%
\definecolor{currentfill}{rgb}{0.974412,0.862387,0.780156}%
\pgfsetfillcolor{currentfill}%
\pgfsetlinewidth{0.311001pt}%
\definecolor{currentstroke}{rgb}{1.000000,1.000000,1.000000}%
\pgfsetstrokecolor{currentstroke}%
\pgfsetdash{}{0pt}%
\pgfpathmoveto{\pgfqpoint{5.498481in}{1.048977in}}%
\pgfpathcurveto{\pgfqpoint{5.505614in}{1.048977in}}{\pgfqpoint{5.512456in}{1.051811in}}{\pgfqpoint{5.517499in}{1.056854in}}%
\pgfpathcurveto{\pgfqpoint{5.522543in}{1.061898in}}{\pgfqpoint{5.525377in}{1.068740in}}{\pgfqpoint{5.525377in}{1.075872in}}%
\pgfpathcurveto{\pgfqpoint{5.525377in}{1.083005in}}{\pgfqpoint{5.522543in}{1.089847in}}{\pgfqpoint{5.517499in}{1.094891in}}%
\pgfpathcurveto{\pgfqpoint{5.512456in}{1.099934in}}{\pgfqpoint{5.505614in}{1.102768in}}{\pgfqpoint{5.498481in}{1.102768in}}%
\pgfpathcurveto{\pgfqpoint{5.491348in}{1.102768in}}{\pgfqpoint{5.484507in}{1.099934in}}{\pgfqpoint{5.479463in}{1.094891in}}%
\pgfpathcurveto{\pgfqpoint{5.474419in}{1.089847in}}{\pgfqpoint{5.471585in}{1.083005in}}{\pgfqpoint{5.471585in}{1.075872in}}%
\pgfpathcurveto{\pgfqpoint{5.471585in}{1.068740in}}{\pgfqpoint{5.474419in}{1.061898in}}{\pgfqpoint{5.479463in}{1.056854in}}%
\pgfpathcurveto{\pgfqpoint{5.484507in}{1.051811in}}{\pgfqpoint{5.491348in}{1.048977in}}{\pgfqpoint{5.498481in}{1.048977in}}%
\pgfpathclose%
\pgfusepath{stroke,fill}%
\end{pgfscope}%
\begin{pgfscope}%
\pgfpathrectangle{\pgfqpoint{4.985294in}{0.500000in}}{\pgfqpoint{1.764706in}{1.700000in}}%
\pgfusepath{clip}%
\pgfsetbuttcap%
\pgfsetroundjoin%
\definecolor{currentfill}{rgb}{0.980678,0.914765,0.856766}%
\pgfsetfillcolor{currentfill}%
\pgfsetlinewidth{0.311001pt}%
\definecolor{currentstroke}{rgb}{1.000000,1.000000,1.000000}%
\pgfsetstrokecolor{currentstroke}%
\pgfsetdash{}{0pt}%
\pgfpathmoveto{\pgfqpoint{6.327667in}{1.280532in}}%
\pgfpathcurveto{\pgfqpoint{6.334800in}{1.280532in}}{\pgfqpoint{6.341642in}{1.283366in}}{\pgfqpoint{6.346685in}{1.288410in}}%
\pgfpathcurveto{\pgfqpoint{6.351729in}{1.293454in}}{\pgfqpoint{6.354563in}{1.300295in}}{\pgfqpoint{6.354563in}{1.307428in}}%
\pgfpathcurveto{\pgfqpoint{6.354563in}{1.314561in}}{\pgfqpoint{6.351729in}{1.321403in}}{\pgfqpoint{6.346685in}{1.326446in}}%
\pgfpathcurveto{\pgfqpoint{6.341642in}{1.331490in}}{\pgfqpoint{6.334800in}{1.334324in}}{\pgfqpoint{6.327667in}{1.334324in}}%
\pgfpathcurveto{\pgfqpoint{6.320534in}{1.334324in}}{\pgfqpoint{6.313693in}{1.331490in}}{\pgfqpoint{6.308649in}{1.326446in}}%
\pgfpathcurveto{\pgfqpoint{6.303605in}{1.321403in}}{\pgfqpoint{6.300771in}{1.314561in}}{\pgfqpoint{6.300771in}{1.307428in}}%
\pgfpathcurveto{\pgfqpoint{6.300771in}{1.300295in}}{\pgfqpoint{6.303605in}{1.293454in}}{\pgfqpoint{6.308649in}{1.288410in}}%
\pgfpathcurveto{\pgfqpoint{6.313693in}{1.283366in}}{\pgfqpoint{6.320534in}{1.280532in}}{\pgfqpoint{6.327667in}{1.280532in}}%
\pgfpathclose%
\pgfusepath{stroke,fill}%
\end{pgfscope}%
\begin{pgfscope}%
\pgfpathrectangle{\pgfqpoint{4.985294in}{0.500000in}}{\pgfqpoint{1.764706in}{1.700000in}}%
\pgfusepath{clip}%
\pgfsetbuttcap%
\pgfsetroundjoin%
\definecolor{currentfill}{rgb}{0.951650,0.442241,0.302145}%
\pgfsetfillcolor{currentfill}%
\pgfsetlinewidth{0.311001pt}%
\definecolor{currentstroke}{rgb}{1.000000,1.000000,1.000000}%
\pgfsetstrokecolor{currentstroke}%
\pgfsetdash{}{0pt}%
\pgfpathmoveto{\pgfqpoint{5.644689in}{1.776829in}}%
\pgfpathcurveto{\pgfqpoint{5.651822in}{1.776829in}}{\pgfqpoint{5.658664in}{1.779663in}}{\pgfqpoint{5.663707in}{1.784706in}}%
\pgfpathcurveto{\pgfqpoint{5.668751in}{1.789750in}}{\pgfqpoint{5.671585in}{1.796592in}}{\pgfqpoint{5.671585in}{1.803724in}}%
\pgfpathcurveto{\pgfqpoint{5.671585in}{1.810857in}}{\pgfqpoint{5.668751in}{1.817699in}}{\pgfqpoint{5.663707in}{1.822742in}}%
\pgfpathcurveto{\pgfqpoint{5.658664in}{1.827786in}}{\pgfqpoint{5.651822in}{1.830620in}}{\pgfqpoint{5.644689in}{1.830620in}}%
\pgfpathcurveto{\pgfqpoint{5.637557in}{1.830620in}}{\pgfqpoint{5.630715in}{1.827786in}}{\pgfqpoint{5.625671in}{1.822742in}}%
\pgfpathcurveto{\pgfqpoint{5.620628in}{1.817699in}}{\pgfqpoint{5.617794in}{1.810857in}}{\pgfqpoint{5.617794in}{1.803724in}}%
\pgfpathcurveto{\pgfqpoint{5.617794in}{1.796592in}}{\pgfqpoint{5.620628in}{1.789750in}}{\pgfqpoint{5.625671in}{1.784706in}}%
\pgfpathcurveto{\pgfqpoint{5.630715in}{1.779663in}}{\pgfqpoint{5.637557in}{1.776829in}}{\pgfqpoint{5.644689in}{1.776829in}}%
\pgfpathclose%
\pgfusepath{stroke,fill}%
\end{pgfscope}%
\begin{pgfscope}%
\pgfpathrectangle{\pgfqpoint{4.985294in}{0.500000in}}{\pgfqpoint{1.764706in}{1.700000in}}%
\pgfusepath{clip}%
\pgfsetbuttcap%
\pgfsetroundjoin%
\definecolor{currentfill}{rgb}{0.975018,0.868213,0.788710}%
\pgfsetfillcolor{currentfill}%
\pgfsetlinewidth{0.311001pt}%
\definecolor{currentstroke}{rgb}{1.000000,1.000000,1.000000}%
\pgfsetstrokecolor{currentstroke}%
\pgfsetdash{}{0pt}%
\pgfpathmoveto{\pgfqpoint{5.488415in}{1.074860in}}%
\pgfpathcurveto{\pgfqpoint{5.495548in}{1.074860in}}{\pgfqpoint{5.502390in}{1.077694in}}{\pgfqpoint{5.507434in}{1.082738in}}%
\pgfpathcurveto{\pgfqpoint{5.512477in}{1.087782in}}{\pgfqpoint{5.515311in}{1.094623in}}{\pgfqpoint{5.515311in}{1.101756in}}%
\pgfpathcurveto{\pgfqpoint{5.515311in}{1.108889in}}{\pgfqpoint{5.512477in}{1.115731in}}{\pgfqpoint{5.507434in}{1.120774in}}%
\pgfpathcurveto{\pgfqpoint{5.502390in}{1.125818in}}{\pgfqpoint{5.495548in}{1.128652in}}{\pgfqpoint{5.488415in}{1.128652in}}%
\pgfpathcurveto{\pgfqpoint{5.481283in}{1.128652in}}{\pgfqpoint{5.474441in}{1.125818in}}{\pgfqpoint{5.469397in}{1.120774in}}%
\pgfpathcurveto{\pgfqpoint{5.464354in}{1.115731in}}{\pgfqpoint{5.461520in}{1.108889in}}{\pgfqpoint{5.461520in}{1.101756in}}%
\pgfpathcurveto{\pgfqpoint{5.461520in}{1.094623in}}{\pgfqpoint{5.464354in}{1.087782in}}{\pgfqpoint{5.469397in}{1.082738in}}%
\pgfpathcurveto{\pgfqpoint{5.474441in}{1.077694in}}{\pgfqpoint{5.481283in}{1.074860in}}{\pgfqpoint{5.488415in}{1.074860in}}%
\pgfpathclose%
\pgfusepath{stroke,fill}%
\end{pgfscope}%
\begin{pgfscope}%
\pgfpathrectangle{\pgfqpoint{4.985294in}{0.500000in}}{\pgfqpoint{1.764706in}{1.700000in}}%
\pgfusepath{clip}%
\pgfsetbuttcap%
\pgfsetroundjoin%
\definecolor{currentfill}{rgb}{0.979891,0.908948,0.848279}%
\pgfsetfillcolor{currentfill}%
\pgfsetlinewidth{0.311001pt}%
\definecolor{currentstroke}{rgb}{1.000000,1.000000,1.000000}%
\pgfsetstrokecolor{currentstroke}%
\pgfsetdash{}{0pt}%
\pgfpathmoveto{\pgfqpoint{6.321714in}{1.445073in}}%
\pgfpathcurveto{\pgfqpoint{6.328847in}{1.445073in}}{\pgfqpoint{6.335688in}{1.447906in}}{\pgfqpoint{6.340732in}{1.452950in}}%
\pgfpathcurveto{\pgfqpoint{6.345776in}{1.457994in}}{\pgfqpoint{6.348609in}{1.464835in}}{\pgfqpoint{6.348609in}{1.471968in}}%
\pgfpathcurveto{\pgfqpoint{6.348609in}{1.479101in}}{\pgfqpoint{6.345776in}{1.485943in}}{\pgfqpoint{6.340732in}{1.490986in}}%
\pgfpathcurveto{\pgfqpoint{6.335688in}{1.496030in}}{\pgfqpoint{6.328847in}{1.498864in}}{\pgfqpoint{6.321714in}{1.498864in}}%
\pgfpathcurveto{\pgfqpoint{6.314581in}{1.498864in}}{\pgfqpoint{6.307739in}{1.496030in}}{\pgfqpoint{6.302696in}{1.490986in}}%
\pgfpathcurveto{\pgfqpoint{6.297652in}{1.485943in}}{\pgfqpoint{6.294818in}{1.479101in}}{\pgfqpoint{6.294818in}{1.471968in}}%
\pgfpathcurveto{\pgfqpoint{6.294818in}{1.464835in}}{\pgfqpoint{6.297652in}{1.457994in}}{\pgfqpoint{6.302696in}{1.452950in}}%
\pgfpathcurveto{\pgfqpoint{6.307739in}{1.447906in}}{\pgfqpoint{6.314581in}{1.445073in}}{\pgfqpoint{6.321714in}{1.445073in}}%
\pgfpathclose%
\pgfusepath{stroke,fill}%
\end{pgfscope}%
\begin{pgfscope}%
\pgfpathrectangle{\pgfqpoint{4.985294in}{0.500000in}}{\pgfqpoint{1.764706in}{1.700000in}}%
\pgfusepath{clip}%
\pgfsetbuttcap%
\pgfsetroundjoin%
\definecolor{currentfill}{rgb}{0.960421,0.553286,0.393191}%
\pgfsetfillcolor{currentfill}%
\pgfsetlinewidth{0.311001pt}%
\definecolor{currentstroke}{rgb}{1.000000,1.000000,1.000000}%
\pgfsetstrokecolor{currentstroke}%
\pgfsetdash{}{0pt}%
\pgfpathmoveto{\pgfqpoint{6.129092in}{1.114444in}}%
\pgfpathcurveto{\pgfqpoint{6.136225in}{1.114444in}}{\pgfqpoint{6.143067in}{1.117278in}}{\pgfqpoint{6.148111in}{1.122322in}}%
\pgfpathcurveto{\pgfqpoint{6.153154in}{1.127366in}}{\pgfqpoint{6.155988in}{1.134207in}}{\pgfqpoint{6.155988in}{1.141340in}}%
\pgfpathcurveto{\pgfqpoint{6.155988in}{1.148473in}}{\pgfqpoint{6.153154in}{1.155315in}}{\pgfqpoint{6.148111in}{1.160358in}}%
\pgfpathcurveto{\pgfqpoint{6.143067in}{1.165402in}}{\pgfqpoint{6.136225in}{1.168236in}}{\pgfqpoint{6.129092in}{1.168236in}}%
\pgfpathcurveto{\pgfqpoint{6.121960in}{1.168236in}}{\pgfqpoint{6.115118in}{1.165402in}}{\pgfqpoint{6.110074in}{1.160358in}}%
\pgfpathcurveto{\pgfqpoint{6.105031in}{1.155315in}}{\pgfqpoint{6.102197in}{1.148473in}}{\pgfqpoint{6.102197in}{1.141340in}}%
\pgfpathcurveto{\pgfqpoint{6.102197in}{1.134207in}}{\pgfqpoint{6.105031in}{1.127366in}}{\pgfqpoint{6.110074in}{1.122322in}}%
\pgfpathcurveto{\pgfqpoint{6.115118in}{1.117278in}}{\pgfqpoint{6.121960in}{1.114444in}}{\pgfqpoint{6.129092in}{1.114444in}}%
\pgfpathclose%
\pgfusepath{stroke,fill}%
\end{pgfscope}%
\begin{pgfscope}%
\pgfpathrectangle{\pgfqpoint{4.985294in}{0.500000in}}{\pgfqpoint{1.764706in}{1.700000in}}%
\pgfusepath{clip}%
\pgfsetbuttcap%
\pgfsetroundjoin%
\definecolor{currentfill}{rgb}{0.966328,0.750560,0.616961}%
\pgfsetfillcolor{currentfill}%
\pgfsetlinewidth{0.311001pt}%
\definecolor{currentstroke}{rgb}{1.000000,1.000000,1.000000}%
\pgfsetstrokecolor{currentstroke}%
\pgfsetdash{}{0pt}%
\pgfpathmoveto{\pgfqpoint{5.568716in}{1.627130in}}%
\pgfpathcurveto{\pgfqpoint{5.575848in}{1.627130in}}{\pgfqpoint{5.582690in}{1.629963in}}{\pgfqpoint{5.587734in}{1.635007in}}%
\pgfpathcurveto{\pgfqpoint{5.592777in}{1.640051in}}{\pgfqpoint{5.595611in}{1.646892in}}{\pgfqpoint{5.595611in}{1.654025in}}%
\pgfpathcurveto{\pgfqpoint{5.595611in}{1.661158in}}{\pgfqpoint{5.592777in}{1.668000in}}{\pgfqpoint{5.587734in}{1.673043in}}%
\pgfpathcurveto{\pgfqpoint{5.582690in}{1.678087in}}{\pgfqpoint{5.575848in}{1.680921in}}{\pgfqpoint{5.568716in}{1.680921in}}%
\pgfpathcurveto{\pgfqpoint{5.561583in}{1.680921in}}{\pgfqpoint{5.554741in}{1.678087in}}{\pgfqpoint{5.549697in}{1.673043in}}%
\pgfpathcurveto{\pgfqpoint{5.544654in}{1.668000in}}{\pgfqpoint{5.541820in}{1.661158in}}{\pgfqpoint{5.541820in}{1.654025in}}%
\pgfpathcurveto{\pgfqpoint{5.541820in}{1.646892in}}{\pgfqpoint{5.544654in}{1.640051in}}{\pgfqpoint{5.549697in}{1.635007in}}%
\pgfpathcurveto{\pgfqpoint{5.554741in}{1.629963in}}{\pgfqpoint{5.561583in}{1.627130in}}{\pgfqpoint{5.568716in}{1.627130in}}%
\pgfpathclose%
\pgfusepath{stroke,fill}%
\end{pgfscope}%
\begin{pgfscope}%
\pgfpathrectangle{\pgfqpoint{4.985294in}{0.500000in}}{\pgfqpoint{1.764706in}{1.700000in}}%
\pgfusepath{clip}%
\pgfsetbuttcap%
\pgfsetroundjoin%
\definecolor{currentfill}{rgb}{0.967735,0.780441,0.659127}%
\pgfsetfillcolor{currentfill}%
\pgfsetlinewidth{0.311001pt}%
\definecolor{currentstroke}{rgb}{1.000000,1.000000,1.000000}%
\pgfsetstrokecolor{currentstroke}%
\pgfsetdash{}{0pt}%
\pgfpathmoveto{\pgfqpoint{6.353087in}{1.521984in}}%
\pgfpathcurveto{\pgfqpoint{6.360220in}{1.521984in}}{\pgfqpoint{6.367062in}{1.524818in}}{\pgfqpoint{6.372105in}{1.529862in}}%
\pgfpathcurveto{\pgfqpoint{6.377149in}{1.534906in}}{\pgfqpoint{6.379983in}{1.541747in}}{\pgfqpoint{6.379983in}{1.548880in}}%
\pgfpathcurveto{\pgfqpoint{6.379983in}{1.556013in}}{\pgfqpoint{6.377149in}{1.562854in}}{\pgfqpoint{6.372105in}{1.567898in}}%
\pgfpathcurveto{\pgfqpoint{6.367062in}{1.572942in}}{\pgfqpoint{6.360220in}{1.575776in}}{\pgfqpoint{6.353087in}{1.575776in}}%
\pgfpathcurveto{\pgfqpoint{6.345954in}{1.575776in}}{\pgfqpoint{6.339113in}{1.572942in}}{\pgfqpoint{6.334069in}{1.567898in}}%
\pgfpathcurveto{\pgfqpoint{6.329025in}{1.562854in}}{\pgfqpoint{6.326192in}{1.556013in}}{\pgfqpoint{6.326192in}{1.548880in}}%
\pgfpathcurveto{\pgfqpoint{6.326192in}{1.541747in}}{\pgfqpoint{6.329025in}{1.534906in}}{\pgfqpoint{6.334069in}{1.529862in}}%
\pgfpathcurveto{\pgfqpoint{6.339113in}{1.524818in}}{\pgfqpoint{6.345954in}{1.521984in}}{\pgfqpoint{6.353087in}{1.521984in}}%
\pgfpathclose%
\pgfusepath{stroke,fill}%
\end{pgfscope}%
\begin{pgfscope}%
\pgfpathrectangle{\pgfqpoint{4.985294in}{0.500000in}}{\pgfqpoint{1.764706in}{1.700000in}}%
\pgfusepath{clip}%
\pgfsetbuttcap%
\pgfsetroundjoin%
\definecolor{currentfill}{rgb}{0.965042,0.701564,0.552889}%
\pgfsetfillcolor{currentfill}%
\pgfsetlinewidth{0.311001pt}%
\definecolor{currentstroke}{rgb}{1.000000,1.000000,1.000000}%
\pgfsetstrokecolor{currentstroke}%
\pgfsetdash{}{0pt}%
\pgfpathmoveto{\pgfqpoint{5.520802in}{0.902437in}}%
\pgfpathcurveto{\pgfqpoint{5.527934in}{0.902437in}}{\pgfqpoint{5.534776in}{0.905271in}}{\pgfqpoint{5.539820in}{0.910314in}}%
\pgfpathcurveto{\pgfqpoint{5.544863in}{0.915358in}}{\pgfqpoint{5.547697in}{0.922200in}}{\pgfqpoint{5.547697in}{0.929332in}}%
\pgfpathcurveto{\pgfqpoint{5.547697in}{0.936465in}}{\pgfqpoint{5.544863in}{0.943307in}}{\pgfqpoint{5.539820in}{0.948350in}}%
\pgfpathcurveto{\pgfqpoint{5.534776in}{0.953394in}}{\pgfqpoint{5.527934in}{0.956228in}}{\pgfqpoint{5.520802in}{0.956228in}}%
\pgfpathcurveto{\pgfqpoint{5.513669in}{0.956228in}}{\pgfqpoint{5.506827in}{0.953394in}}{\pgfqpoint{5.501784in}{0.948350in}}%
\pgfpathcurveto{\pgfqpoint{5.496740in}{0.943307in}}{\pgfqpoint{5.493906in}{0.936465in}}{\pgfqpoint{5.493906in}{0.929332in}}%
\pgfpathcurveto{\pgfqpoint{5.493906in}{0.922200in}}{\pgfqpoint{5.496740in}{0.915358in}}{\pgfqpoint{5.501784in}{0.910314in}}%
\pgfpathcurveto{\pgfqpoint{5.506827in}{0.905271in}}{\pgfqpoint{5.513669in}{0.902437in}}{\pgfqpoint{5.520802in}{0.902437in}}%
\pgfpathclose%
\pgfusepath{stroke,fill}%
\end{pgfscope}%
\begin{pgfscope}%
\pgfpathrectangle{\pgfqpoint{4.985294in}{0.500000in}}{\pgfqpoint{1.764706in}{1.700000in}}%
\pgfusepath{clip}%
\pgfsetbuttcap%
\pgfsetroundjoin%
\definecolor{currentfill}{rgb}{0.979891,0.908948,0.848279}%
\pgfsetfillcolor{currentfill}%
\pgfsetlinewidth{0.311001pt}%
\definecolor{currentstroke}{rgb}{1.000000,1.000000,1.000000}%
\pgfsetstrokecolor{currentstroke}%
\pgfsetdash{}{0pt}%
\pgfpathmoveto{\pgfqpoint{6.284954in}{1.504717in}}%
\pgfpathcurveto{\pgfqpoint{6.292087in}{1.504717in}}{\pgfqpoint{6.298929in}{1.507551in}}{\pgfqpoint{6.303972in}{1.512595in}}%
\pgfpathcurveto{\pgfqpoint{6.309016in}{1.517639in}}{\pgfqpoint{6.311850in}{1.524480in}}{\pgfqpoint{6.311850in}{1.531613in}}%
\pgfpathcurveto{\pgfqpoint{6.311850in}{1.538746in}}{\pgfqpoint{6.309016in}{1.545588in}}{\pgfqpoint{6.303972in}{1.550631in}}%
\pgfpathcurveto{\pgfqpoint{6.298929in}{1.555675in}}{\pgfqpoint{6.292087in}{1.558509in}}{\pgfqpoint{6.284954in}{1.558509in}}%
\pgfpathcurveto{\pgfqpoint{6.277821in}{1.558509in}}{\pgfqpoint{6.270980in}{1.555675in}}{\pgfqpoint{6.265936in}{1.550631in}}%
\pgfpathcurveto{\pgfqpoint{6.260892in}{1.545588in}}{\pgfqpoint{6.258058in}{1.538746in}}{\pgfqpoint{6.258058in}{1.531613in}}%
\pgfpathcurveto{\pgfqpoint{6.258058in}{1.524480in}}{\pgfqpoint{6.260892in}{1.517639in}}{\pgfqpoint{6.265936in}{1.512595in}}%
\pgfpathcurveto{\pgfqpoint{6.270980in}{1.507551in}}{\pgfqpoint{6.277821in}{1.504717in}}{\pgfqpoint{6.284954in}{1.504717in}}%
\pgfpathclose%
\pgfusepath{stroke,fill}%
\end{pgfscope}%
\begin{pgfscope}%
\pgfpathrectangle{\pgfqpoint{4.985294in}{0.500000in}}{\pgfqpoint{1.764706in}{1.700000in}}%
\pgfusepath{clip}%
\pgfsetbuttcap%
\pgfsetroundjoin%
\definecolor{currentfill}{rgb}{0.972201,0.839051,0.745789}%
\pgfsetfillcolor{currentfill}%
\pgfsetlinewidth{0.311001pt}%
\definecolor{currentstroke}{rgb}{1.000000,1.000000,1.000000}%
\pgfsetstrokecolor{currentstroke}%
\pgfsetdash{}{0pt}%
\pgfpathmoveto{\pgfqpoint{5.413315in}{1.088541in}}%
\pgfpathcurveto{\pgfqpoint{5.420447in}{1.088541in}}{\pgfqpoint{5.427289in}{1.091375in}}{\pgfqpoint{5.432333in}{1.096418in}}%
\pgfpathcurveto{\pgfqpoint{5.437376in}{1.101462in}}{\pgfqpoint{5.440210in}{1.108304in}}{\pgfqpoint{5.440210in}{1.115436in}}%
\pgfpathcurveto{\pgfqpoint{5.440210in}{1.122569in}}{\pgfqpoint{5.437376in}{1.129411in}}{\pgfqpoint{5.432333in}{1.134455in}}%
\pgfpathcurveto{\pgfqpoint{5.427289in}{1.139498in}}{\pgfqpoint{5.420447in}{1.142332in}}{\pgfqpoint{5.413315in}{1.142332in}}%
\pgfpathcurveto{\pgfqpoint{5.406182in}{1.142332in}}{\pgfqpoint{5.399340in}{1.139498in}}{\pgfqpoint{5.394297in}{1.134455in}}%
\pgfpathcurveto{\pgfqpoint{5.389253in}{1.129411in}}{\pgfqpoint{5.386419in}{1.122569in}}{\pgfqpoint{5.386419in}{1.115436in}}%
\pgfpathcurveto{\pgfqpoint{5.386419in}{1.108304in}}{\pgfqpoint{5.389253in}{1.101462in}}{\pgfqpoint{5.394297in}{1.096418in}}%
\pgfpathcurveto{\pgfqpoint{5.399340in}{1.091375in}}{\pgfqpoint{5.406182in}{1.088541in}}{\pgfqpoint{5.413315in}{1.088541in}}%
\pgfpathclose%
\pgfusepath{stroke,fill}%
\end{pgfscope}%
\begin{pgfscope}%
\pgfpathrectangle{\pgfqpoint{4.985294in}{0.500000in}}{\pgfqpoint{1.764706in}{1.700000in}}%
\pgfusepath{clip}%
\pgfsetbuttcap%
\pgfsetroundjoin%
\definecolor{currentfill}{rgb}{0.962018,0.586477,0.424918}%
\pgfsetfillcolor{currentfill}%
\pgfsetlinewidth{0.311001pt}%
\definecolor{currentstroke}{rgb}{1.000000,1.000000,1.000000}%
\pgfsetstrokecolor{currentstroke}%
\pgfsetdash{}{0pt}%
\pgfpathmoveto{\pgfqpoint{6.159784in}{1.451380in}}%
\pgfpathcurveto{\pgfqpoint{6.166917in}{1.451380in}}{\pgfqpoint{6.173759in}{1.454214in}}{\pgfqpoint{6.178803in}{1.459257in}}%
\pgfpathcurveto{\pgfqpoint{6.183846in}{1.464301in}}{\pgfqpoint{6.186680in}{1.471143in}}{\pgfqpoint{6.186680in}{1.478275in}}%
\pgfpathcurveto{\pgfqpoint{6.186680in}{1.485408in}}{\pgfqpoint{6.183846in}{1.492250in}}{\pgfqpoint{6.178803in}{1.497294in}}%
\pgfpathcurveto{\pgfqpoint{6.173759in}{1.502337in}}{\pgfqpoint{6.166917in}{1.505171in}}{\pgfqpoint{6.159784in}{1.505171in}}%
\pgfpathcurveto{\pgfqpoint{6.152652in}{1.505171in}}{\pgfqpoint{6.145810in}{1.502337in}}{\pgfqpoint{6.140766in}{1.497294in}}%
\pgfpathcurveto{\pgfqpoint{6.135723in}{1.492250in}}{\pgfqpoint{6.132889in}{1.485408in}}{\pgfqpoint{6.132889in}{1.478275in}}%
\pgfpathcurveto{\pgfqpoint{6.132889in}{1.471143in}}{\pgfqpoint{6.135723in}{1.464301in}}{\pgfqpoint{6.140766in}{1.459257in}}%
\pgfpathcurveto{\pgfqpoint{6.145810in}{1.454214in}}{\pgfqpoint{6.152652in}{1.451380in}}{\pgfqpoint{6.159784in}{1.451380in}}%
\pgfpathclose%
\pgfusepath{stroke,fill}%
\end{pgfscope}%
\begin{pgfscope}%
\pgfpathrectangle{\pgfqpoint{4.985294in}{0.500000in}}{\pgfqpoint{1.764706in}{1.700000in}}%
\pgfusepath{clip}%
\pgfsetbuttcap%
\pgfsetroundjoin%
\definecolor{currentfill}{rgb}{0.968105,0.786346,0.667739}%
\pgfsetfillcolor{currentfill}%
\pgfsetlinewidth{0.311001pt}%
\definecolor{currentstroke}{rgb}{1.000000,1.000000,1.000000}%
\pgfsetstrokecolor{currentstroke}%
\pgfsetdash{}{0pt}%
\pgfpathmoveto{\pgfqpoint{6.391320in}{1.330049in}}%
\pgfpathcurveto{\pgfqpoint{6.398453in}{1.330049in}}{\pgfqpoint{6.405295in}{1.332883in}}{\pgfqpoint{6.410338in}{1.337927in}}%
\pgfpathcurveto{\pgfqpoint{6.415382in}{1.342971in}}{\pgfqpoint{6.418216in}{1.349812in}}{\pgfqpoint{6.418216in}{1.356945in}}%
\pgfpathcurveto{\pgfqpoint{6.418216in}{1.364078in}}{\pgfqpoint{6.415382in}{1.370920in}}{\pgfqpoint{6.410338in}{1.375963in}}%
\pgfpathcurveto{\pgfqpoint{6.405295in}{1.381007in}}{\pgfqpoint{6.398453in}{1.383841in}}{\pgfqpoint{6.391320in}{1.383841in}}%
\pgfpathcurveto{\pgfqpoint{6.384187in}{1.383841in}}{\pgfqpoint{6.377346in}{1.381007in}}{\pgfqpoint{6.372302in}{1.375963in}}%
\pgfpathcurveto{\pgfqpoint{6.367258in}{1.370920in}}{\pgfqpoint{6.364424in}{1.364078in}}{\pgfqpoint{6.364424in}{1.356945in}}%
\pgfpathcurveto{\pgfqpoint{6.364424in}{1.349812in}}{\pgfqpoint{6.367258in}{1.342971in}}{\pgfqpoint{6.372302in}{1.337927in}}%
\pgfpathcurveto{\pgfqpoint{6.377346in}{1.332883in}}{\pgfqpoint{6.384187in}{1.330049in}}{\pgfqpoint{6.391320in}{1.330049in}}%
\pgfpathclose%
\pgfusepath{stroke,fill}%
\end{pgfscope}%
\begin{pgfscope}%
\pgfpathrectangle{\pgfqpoint{4.985294in}{0.500000in}}{\pgfqpoint{1.764706in}{1.700000in}}%
\pgfusepath{clip}%
\pgfsetbuttcap%
\pgfsetroundjoin%
\definecolor{currentfill}{rgb}{0.908486,0.245685,0.245983}%
\pgfsetfillcolor{currentfill}%
\pgfsetlinewidth{0.311001pt}%
\definecolor{currentstroke}{rgb}{1.000000,1.000000,1.000000}%
\pgfsetstrokecolor{currentstroke}%
\pgfsetdash{}{0pt}%
\pgfpathmoveto{\pgfqpoint{5.261407in}{1.416670in}}%
\pgfpathcurveto{\pgfqpoint{5.268540in}{1.416670in}}{\pgfqpoint{5.275382in}{1.419504in}}{\pgfqpoint{5.280425in}{1.424548in}}%
\pgfpathcurveto{\pgfqpoint{5.285469in}{1.429591in}}{\pgfqpoint{5.288303in}{1.436433in}}{\pgfqpoint{5.288303in}{1.443566in}}%
\pgfpathcurveto{\pgfqpoint{5.288303in}{1.450698in}}{\pgfqpoint{5.285469in}{1.457540in}}{\pgfqpoint{5.280425in}{1.462584in}}%
\pgfpathcurveto{\pgfqpoint{5.275382in}{1.467627in}}{\pgfqpoint{5.268540in}{1.470461in}}{\pgfqpoint{5.261407in}{1.470461in}}%
\pgfpathcurveto{\pgfqpoint{5.254274in}{1.470461in}}{\pgfqpoint{5.247433in}{1.467627in}}{\pgfqpoint{5.242389in}{1.462584in}}%
\pgfpathcurveto{\pgfqpoint{5.237345in}{1.457540in}}{\pgfqpoint{5.234511in}{1.450698in}}{\pgfqpoint{5.234511in}{1.443566in}}%
\pgfpathcurveto{\pgfqpoint{5.234511in}{1.436433in}}{\pgfqpoint{5.237345in}{1.429591in}}{\pgfqpoint{5.242389in}{1.424548in}}%
\pgfpathcurveto{\pgfqpoint{5.247433in}{1.419504in}}{\pgfqpoint{5.254274in}{1.416670in}}{\pgfqpoint{5.261407in}{1.416670in}}%
\pgfpathclose%
\pgfusepath{stroke,fill}%
\end{pgfscope}%
\begin{pgfscope}%
\pgfpathrectangle{\pgfqpoint{4.985294in}{0.500000in}}{\pgfqpoint{1.764706in}{1.700000in}}%
\pgfusepath{clip}%
\pgfsetbuttcap%
\pgfsetroundjoin%
\definecolor{currentfill}{rgb}{0.976287,0.879862,0.805788}%
\pgfsetfillcolor{currentfill}%
\pgfsetlinewidth{0.311001pt}%
\definecolor{currentstroke}{rgb}{1.000000,1.000000,1.000000}%
\pgfsetstrokecolor{currentstroke}%
\pgfsetdash{}{0pt}%
\pgfpathmoveto{\pgfqpoint{6.350490in}{1.390077in}}%
\pgfpathcurveto{\pgfqpoint{6.357623in}{1.390077in}}{\pgfqpoint{6.364465in}{1.392911in}}{\pgfqpoint{6.369509in}{1.397955in}}%
\pgfpathcurveto{\pgfqpoint{6.374552in}{1.402998in}}{\pgfqpoint{6.377386in}{1.409840in}}{\pgfqpoint{6.377386in}{1.416973in}}%
\pgfpathcurveto{\pgfqpoint{6.377386in}{1.424106in}}{\pgfqpoint{6.374552in}{1.430947in}}{\pgfqpoint{6.369509in}{1.435991in}}%
\pgfpathcurveto{\pgfqpoint{6.364465in}{1.441035in}}{\pgfqpoint{6.357623in}{1.443869in}}{\pgfqpoint{6.350490in}{1.443869in}}%
\pgfpathcurveto{\pgfqpoint{6.343358in}{1.443869in}}{\pgfqpoint{6.336516in}{1.441035in}}{\pgfqpoint{6.331472in}{1.435991in}}%
\pgfpathcurveto{\pgfqpoint{6.326429in}{1.430947in}}{\pgfqpoint{6.323595in}{1.424106in}}{\pgfqpoint{6.323595in}{1.416973in}}%
\pgfpathcurveto{\pgfqpoint{6.323595in}{1.409840in}}{\pgfqpoint{6.326429in}{1.402998in}}{\pgfqpoint{6.331472in}{1.397955in}}%
\pgfpathcurveto{\pgfqpoint{6.336516in}{1.392911in}}{\pgfqpoint{6.343358in}{1.390077in}}{\pgfqpoint{6.350490in}{1.390077in}}%
\pgfpathclose%
\pgfusepath{stroke,fill}%
\end{pgfscope}%
\begin{pgfscope}%
\pgfpathrectangle{\pgfqpoint{4.985294in}{0.500000in}}{\pgfqpoint{1.764706in}{1.700000in}}%
\pgfusepath{clip}%
\pgfsetbuttcap%
\pgfsetroundjoin%
\definecolor{currentfill}{rgb}{0.975018,0.868213,0.788710}%
\pgfsetfillcolor{currentfill}%
\pgfsetlinewidth{0.311001pt}%
\definecolor{currentstroke}{rgb}{1.000000,1.000000,1.000000}%
\pgfsetstrokecolor{currentstroke}%
\pgfsetdash{}{0pt}%
\pgfpathmoveto{\pgfqpoint{5.400853in}{1.159472in}}%
\pgfpathcurveto{\pgfqpoint{5.407986in}{1.159472in}}{\pgfqpoint{5.414827in}{1.162306in}}{\pgfqpoint{5.419871in}{1.167350in}}%
\pgfpathcurveto{\pgfqpoint{5.424915in}{1.172393in}}{\pgfqpoint{5.427749in}{1.179235in}}{\pgfqpoint{5.427749in}{1.186368in}}%
\pgfpathcurveto{\pgfqpoint{5.427749in}{1.193501in}}{\pgfqpoint{5.424915in}{1.200342in}}{\pgfqpoint{5.419871in}{1.205386in}}%
\pgfpathcurveto{\pgfqpoint{5.414827in}{1.210430in}}{\pgfqpoint{5.407986in}{1.213264in}}{\pgfqpoint{5.400853in}{1.213264in}}%
\pgfpathcurveto{\pgfqpoint{5.393720in}{1.213264in}}{\pgfqpoint{5.386878in}{1.210430in}}{\pgfqpoint{5.381835in}{1.205386in}}%
\pgfpathcurveto{\pgfqpoint{5.376791in}{1.200342in}}{\pgfqpoint{5.373957in}{1.193501in}}{\pgfqpoint{5.373957in}{1.186368in}}%
\pgfpathcurveto{\pgfqpoint{5.373957in}{1.179235in}}{\pgfqpoint{5.376791in}{1.172393in}}{\pgfqpoint{5.381835in}{1.167350in}}%
\pgfpathcurveto{\pgfqpoint{5.386878in}{1.162306in}}{\pgfqpoint{5.393720in}{1.159472in}}{\pgfqpoint{5.400853in}{1.159472in}}%
\pgfpathclose%
\pgfusepath{stroke,fill}%
\end{pgfscope}%
\begin{pgfscope}%
\pgfpathrectangle{\pgfqpoint{4.985294in}{0.500000in}}{\pgfqpoint{1.764706in}{1.700000in}}%
\pgfusepath{clip}%
\pgfsetbuttcap%
\pgfsetroundjoin%
\definecolor{currentfill}{rgb}{0.967092,0.768560,0.642079}%
\pgfsetfillcolor{currentfill}%
\pgfsetlinewidth{0.311001pt}%
\definecolor{currentstroke}{rgb}{1.000000,1.000000,1.000000}%
\pgfsetstrokecolor{currentstroke}%
\pgfsetdash{}{0pt}%
\pgfpathmoveto{\pgfqpoint{5.573990in}{0.949655in}}%
\pgfpathcurveto{\pgfqpoint{5.581122in}{0.949655in}}{\pgfqpoint{5.587964in}{0.952489in}}{\pgfqpoint{5.593008in}{0.957533in}}%
\pgfpathcurveto{\pgfqpoint{5.598051in}{0.962577in}}{\pgfqpoint{5.600885in}{0.969418in}}{\pgfqpoint{5.600885in}{0.976551in}}%
\pgfpathcurveto{\pgfqpoint{5.600885in}{0.983684in}}{\pgfqpoint{5.598051in}{0.990526in}}{\pgfqpoint{5.593008in}{0.995569in}}%
\pgfpathcurveto{\pgfqpoint{5.587964in}{1.000613in}}{\pgfqpoint{5.581122in}{1.003447in}}{\pgfqpoint{5.573990in}{1.003447in}}%
\pgfpathcurveto{\pgfqpoint{5.566857in}{1.003447in}}{\pgfqpoint{5.560015in}{1.000613in}}{\pgfqpoint{5.554972in}{0.995569in}}%
\pgfpathcurveto{\pgfqpoint{5.549928in}{0.990526in}}{\pgfqpoint{5.547094in}{0.983684in}}{\pgfqpoint{5.547094in}{0.976551in}}%
\pgfpathcurveto{\pgfqpoint{5.547094in}{0.969418in}}{\pgfqpoint{5.549928in}{0.962577in}}{\pgfqpoint{5.554972in}{0.957533in}}%
\pgfpathcurveto{\pgfqpoint{5.560015in}{0.952489in}}{\pgfqpoint{5.566857in}{0.949655in}}{\pgfqpoint{5.573990in}{0.949655in}}%
\pgfpathclose%
\pgfusepath{stroke,fill}%
\end{pgfscope}%
\begin{pgfscope}%
\pgfpathrectangle{\pgfqpoint{4.985294in}{0.500000in}}{\pgfqpoint{1.764706in}{1.700000in}}%
\pgfusepath{clip}%
\pgfsetbuttcap%
\pgfsetroundjoin%
\definecolor{currentfill}{rgb}{0.958331,0.519463,0.362986}%
\pgfsetfillcolor{currentfill}%
\pgfsetlinewidth{0.311001pt}%
\definecolor{currentstroke}{rgb}{1.000000,1.000000,1.000000}%
\pgfsetstrokecolor{currentstroke}%
\pgfsetdash{}{0pt}%
\pgfpathmoveto{\pgfqpoint{5.294095in}{1.224311in}}%
\pgfpathcurveto{\pgfqpoint{5.301228in}{1.224311in}}{\pgfqpoint{5.308070in}{1.227145in}}{\pgfqpoint{5.313113in}{1.232189in}}%
\pgfpathcurveto{\pgfqpoint{5.318157in}{1.237233in}}{\pgfqpoint{5.320991in}{1.244074in}}{\pgfqpoint{5.320991in}{1.251207in}}%
\pgfpathcurveto{\pgfqpoint{5.320991in}{1.258340in}}{\pgfqpoint{5.318157in}{1.265182in}}{\pgfqpoint{5.313113in}{1.270225in}}%
\pgfpathcurveto{\pgfqpoint{5.308070in}{1.275269in}}{\pgfqpoint{5.301228in}{1.278103in}}{\pgfqpoint{5.294095in}{1.278103in}}%
\pgfpathcurveto{\pgfqpoint{5.286962in}{1.278103in}}{\pgfqpoint{5.280121in}{1.275269in}}{\pgfqpoint{5.275077in}{1.270225in}}%
\pgfpathcurveto{\pgfqpoint{5.270033in}{1.265182in}}{\pgfqpoint{5.267200in}{1.258340in}}{\pgfqpoint{5.267200in}{1.251207in}}%
\pgfpathcurveto{\pgfqpoint{5.267200in}{1.244074in}}{\pgfqpoint{5.270033in}{1.237233in}}{\pgfqpoint{5.275077in}{1.232189in}}%
\pgfpathcurveto{\pgfqpoint{5.280121in}{1.227145in}}{\pgfqpoint{5.286962in}{1.224311in}}{\pgfqpoint{5.294095in}{1.224311in}}%
\pgfpathclose%
\pgfusepath{stroke,fill}%
\end{pgfscope}%
\begin{pgfscope}%
\pgfpathrectangle{\pgfqpoint{4.985294in}{0.500000in}}{\pgfqpoint{1.764706in}{1.700000in}}%
\pgfusepath{clip}%
\pgfsetbuttcap%
\pgfsetroundjoin%
\definecolor{currentfill}{rgb}{0.974412,0.862387,0.780156}%
\pgfsetfillcolor{currentfill}%
\pgfsetlinewidth{0.311001pt}%
\definecolor{currentstroke}{rgb}{1.000000,1.000000,1.000000}%
\pgfsetstrokecolor{currentstroke}%
\pgfsetdash{}{0pt}%
\pgfpathmoveto{\pgfqpoint{5.470977in}{1.569105in}}%
\pgfpathcurveto{\pgfqpoint{5.478110in}{1.569105in}}{\pgfqpoint{5.484952in}{1.571939in}}{\pgfqpoint{5.489996in}{1.576983in}}%
\pgfpathcurveto{\pgfqpoint{5.495039in}{1.582026in}}{\pgfqpoint{5.497873in}{1.588868in}}{\pgfqpoint{5.497873in}{1.596001in}}%
\pgfpathcurveto{\pgfqpoint{5.497873in}{1.603134in}}{\pgfqpoint{5.495039in}{1.609975in}}{\pgfqpoint{5.489996in}{1.615019in}}%
\pgfpathcurveto{\pgfqpoint{5.484952in}{1.620063in}}{\pgfqpoint{5.478110in}{1.622896in}}{\pgfqpoint{5.470977in}{1.622896in}}%
\pgfpathcurveto{\pgfqpoint{5.463845in}{1.622896in}}{\pgfqpoint{5.457003in}{1.620063in}}{\pgfqpoint{5.451959in}{1.615019in}}%
\pgfpathcurveto{\pgfqpoint{5.446916in}{1.609975in}}{\pgfqpoint{5.444082in}{1.603134in}}{\pgfqpoint{5.444082in}{1.596001in}}%
\pgfpathcurveto{\pgfqpoint{5.444082in}{1.588868in}}{\pgfqpoint{5.446916in}{1.582026in}}{\pgfqpoint{5.451959in}{1.576983in}}%
\pgfpathcurveto{\pgfqpoint{5.457003in}{1.571939in}}{\pgfqpoint{5.463845in}{1.569105in}}{\pgfqpoint{5.470977in}{1.569105in}}%
\pgfpathclose%
\pgfusepath{stroke,fill}%
\end{pgfscope}%
\begin{pgfscope}%
\pgfpathrectangle{\pgfqpoint{4.985294in}{0.500000in}}{\pgfqpoint{1.764706in}{1.700000in}}%
\pgfusepath{clip}%
\pgfsetbuttcap%
\pgfsetroundjoin%
\definecolor{currentfill}{rgb}{0.980678,0.914765,0.856766}%
\pgfsetfillcolor{currentfill}%
\pgfsetlinewidth{0.311001pt}%
\definecolor{currentstroke}{rgb}{1.000000,1.000000,1.000000}%
\pgfsetstrokecolor{currentstroke}%
\pgfsetdash{}{0pt}%
\pgfpathmoveto{\pgfqpoint{6.285763in}{1.503303in}}%
\pgfpathcurveto{\pgfqpoint{6.292896in}{1.503303in}}{\pgfqpoint{6.299737in}{1.506136in}}{\pgfqpoint{6.304781in}{1.511180in}}%
\pgfpathcurveto{\pgfqpoint{6.309825in}{1.516224in}}{\pgfqpoint{6.312658in}{1.523065in}}{\pgfqpoint{6.312658in}{1.530198in}}%
\pgfpathcurveto{\pgfqpoint{6.312658in}{1.537331in}}{\pgfqpoint{6.309825in}{1.544173in}}{\pgfqpoint{6.304781in}{1.549216in}}%
\pgfpathcurveto{\pgfqpoint{6.299737in}{1.554260in}}{\pgfqpoint{6.292896in}{1.557094in}}{\pgfqpoint{6.285763in}{1.557094in}}%
\pgfpathcurveto{\pgfqpoint{6.278630in}{1.557094in}}{\pgfqpoint{6.271788in}{1.554260in}}{\pgfqpoint{6.266745in}{1.549216in}}%
\pgfpathcurveto{\pgfqpoint{6.261701in}{1.544173in}}{\pgfqpoint{6.258867in}{1.537331in}}{\pgfqpoint{6.258867in}{1.530198in}}%
\pgfpathcurveto{\pgfqpoint{6.258867in}{1.523065in}}{\pgfqpoint{6.261701in}{1.516224in}}{\pgfqpoint{6.266745in}{1.511180in}}%
\pgfpathcurveto{\pgfqpoint{6.271788in}{1.506136in}}{\pgfqpoint{6.278630in}{1.503303in}}{\pgfqpoint{6.285763in}{1.503303in}}%
\pgfpathclose%
\pgfusepath{stroke,fill}%
\end{pgfscope}%
\begin{pgfscope}%
\pgfpathrectangle{\pgfqpoint{4.985294in}{0.500000in}}{\pgfqpoint{1.764706in}{1.700000in}}%
\pgfusepath{clip}%
\pgfsetbuttcap%
\pgfsetroundjoin%
\definecolor{currentfill}{rgb}{0.962018,0.586477,0.424918}%
\pgfsetfillcolor{currentfill}%
\pgfsetlinewidth{0.311001pt}%
\definecolor{currentstroke}{rgb}{1.000000,1.000000,1.000000}%
\pgfsetstrokecolor{currentstroke}%
\pgfsetdash{}{0pt}%
\pgfpathmoveto{\pgfqpoint{6.421481in}{1.427596in}}%
\pgfpathcurveto{\pgfqpoint{6.428614in}{1.427596in}}{\pgfqpoint{6.435455in}{1.430430in}}{\pgfqpoint{6.440499in}{1.435473in}}%
\pgfpathcurveto{\pgfqpoint{6.445543in}{1.440517in}}{\pgfqpoint{6.448376in}{1.447359in}}{\pgfqpoint{6.448376in}{1.454491in}}%
\pgfpathcurveto{\pgfqpoint{6.448376in}{1.461624in}}{\pgfqpoint{6.445543in}{1.468466in}}{\pgfqpoint{6.440499in}{1.473509in}}%
\pgfpathcurveto{\pgfqpoint{6.435455in}{1.478553in}}{\pgfqpoint{6.428614in}{1.481387in}}{\pgfqpoint{6.421481in}{1.481387in}}%
\pgfpathcurveto{\pgfqpoint{6.414348in}{1.481387in}}{\pgfqpoint{6.407506in}{1.478553in}}{\pgfqpoint{6.402463in}{1.473509in}}%
\pgfpathcurveto{\pgfqpoint{6.397419in}{1.468466in}}{\pgfqpoint{6.394585in}{1.461624in}}{\pgfqpoint{6.394585in}{1.454491in}}%
\pgfpathcurveto{\pgfqpoint{6.394585in}{1.447359in}}{\pgfqpoint{6.397419in}{1.440517in}}{\pgfqpoint{6.402463in}{1.435473in}}%
\pgfpathcurveto{\pgfqpoint{6.407506in}{1.430430in}}{\pgfqpoint{6.414348in}{1.427596in}}{\pgfqpoint{6.421481in}{1.427596in}}%
\pgfpathclose%
\pgfusepath{stroke,fill}%
\end{pgfscope}%
\begin{pgfscope}%
\pgfpathrectangle{\pgfqpoint{4.985294in}{0.500000in}}{\pgfqpoint{1.764706in}{1.700000in}}%
\pgfusepath{clip}%
\pgfsetbuttcap%
\pgfsetroundjoin%
\definecolor{currentfill}{rgb}{0.978376,0.897317,0.831308}%
\pgfsetfillcolor{currentfill}%
\pgfsetlinewidth{0.311001pt}%
\definecolor{currentstroke}{rgb}{1.000000,1.000000,1.000000}%
\pgfsetstrokecolor{currentstroke}%
\pgfsetdash{}{0pt}%
\pgfpathmoveto{\pgfqpoint{6.285429in}{1.365108in}}%
\pgfpathcurveto{\pgfqpoint{6.292562in}{1.365108in}}{\pgfqpoint{6.299403in}{1.367942in}}{\pgfqpoint{6.304447in}{1.372985in}}%
\pgfpathcurveto{\pgfqpoint{6.309491in}{1.378029in}}{\pgfqpoint{6.312325in}{1.384871in}}{\pgfqpoint{6.312325in}{1.392004in}}%
\pgfpathcurveto{\pgfqpoint{6.312325in}{1.399136in}}{\pgfqpoint{6.309491in}{1.405978in}}{\pgfqpoint{6.304447in}{1.411022in}}%
\pgfpathcurveto{\pgfqpoint{6.299403in}{1.416065in}}{\pgfqpoint{6.292562in}{1.418899in}}{\pgfqpoint{6.285429in}{1.418899in}}%
\pgfpathcurveto{\pgfqpoint{6.278296in}{1.418899in}}{\pgfqpoint{6.271454in}{1.416065in}}{\pgfqpoint{6.266411in}{1.411022in}}%
\pgfpathcurveto{\pgfqpoint{6.261367in}{1.405978in}}{\pgfqpoint{6.258533in}{1.399136in}}{\pgfqpoint{6.258533in}{1.392004in}}%
\pgfpathcurveto{\pgfqpoint{6.258533in}{1.384871in}}{\pgfqpoint{6.261367in}{1.378029in}}{\pgfqpoint{6.266411in}{1.372985in}}%
\pgfpathcurveto{\pgfqpoint{6.271454in}{1.367942in}}{\pgfqpoint{6.278296in}{1.365108in}}{\pgfqpoint{6.285429in}{1.365108in}}%
\pgfpathclose%
\pgfusepath{stroke,fill}%
\end{pgfscope}%
\begin{pgfscope}%
\pgfpathrectangle{\pgfqpoint{4.985294in}{0.500000in}}{\pgfqpoint{1.764706in}{1.700000in}}%
\pgfusepath{clip}%
\pgfsetbuttcap%
\pgfsetroundjoin%
\definecolor{currentfill}{rgb}{0.974412,0.862387,0.780156}%
\pgfsetfillcolor{currentfill}%
\pgfsetlinewidth{0.311001pt}%
\definecolor{currentstroke}{rgb}{1.000000,1.000000,1.000000}%
\pgfsetstrokecolor{currentstroke}%
\pgfsetdash{}{0pt}%
\pgfpathmoveto{\pgfqpoint{6.365643in}{1.354245in}}%
\pgfpathcurveto{\pgfqpoint{6.372775in}{1.354245in}}{\pgfqpoint{6.379617in}{1.357079in}}{\pgfqpoint{6.384661in}{1.362122in}}%
\pgfpathcurveto{\pgfqpoint{6.389704in}{1.367166in}}{\pgfqpoint{6.392538in}{1.374008in}}{\pgfqpoint{6.392538in}{1.381141in}}%
\pgfpathcurveto{\pgfqpoint{6.392538in}{1.388273in}}{\pgfqpoint{6.389704in}{1.395115in}}{\pgfqpoint{6.384661in}{1.400159in}}%
\pgfpathcurveto{\pgfqpoint{6.379617in}{1.405202in}}{\pgfqpoint{6.372775in}{1.408036in}}{\pgfqpoint{6.365643in}{1.408036in}}%
\pgfpathcurveto{\pgfqpoint{6.358510in}{1.408036in}}{\pgfqpoint{6.351668in}{1.405202in}}{\pgfqpoint{6.346625in}{1.400159in}}%
\pgfpathcurveto{\pgfqpoint{6.341581in}{1.395115in}}{\pgfqpoint{6.338747in}{1.388273in}}{\pgfqpoint{6.338747in}{1.381141in}}%
\pgfpathcurveto{\pgfqpoint{6.338747in}{1.374008in}}{\pgfqpoint{6.341581in}{1.367166in}}{\pgfqpoint{6.346625in}{1.362122in}}%
\pgfpathcurveto{\pgfqpoint{6.351668in}{1.357079in}}{\pgfqpoint{6.358510in}{1.354245in}}{\pgfqpoint{6.365643in}{1.354245in}}%
\pgfpathclose%
\pgfusepath{stroke,fill}%
\end{pgfscope}%
\begin{pgfscope}%
\pgfpathrectangle{\pgfqpoint{4.985294in}{0.500000in}}{\pgfqpoint{1.764706in}{1.700000in}}%
\pgfusepath{clip}%
\pgfsetbuttcap%
\pgfsetroundjoin%
\definecolor{currentfill}{rgb}{0.973271,0.850724,0.762998}%
\pgfsetfillcolor{currentfill}%
\pgfsetlinewidth{0.311001pt}%
\definecolor{currentstroke}{rgb}{1.000000,1.000000,1.000000}%
\pgfsetstrokecolor{currentstroke}%
\pgfsetdash{}{0pt}%
\pgfpathmoveto{\pgfqpoint{6.216672in}{1.111229in}}%
\pgfpathcurveto{\pgfqpoint{6.223804in}{1.111229in}}{\pgfqpoint{6.230646in}{1.114063in}}{\pgfqpoint{6.235690in}{1.119107in}}%
\pgfpathcurveto{\pgfqpoint{6.240733in}{1.124150in}}{\pgfqpoint{6.243567in}{1.130992in}}{\pgfqpoint{6.243567in}{1.138125in}}%
\pgfpathcurveto{\pgfqpoint{6.243567in}{1.145258in}}{\pgfqpoint{6.240733in}{1.152099in}}{\pgfqpoint{6.235690in}{1.157143in}}%
\pgfpathcurveto{\pgfqpoint{6.230646in}{1.162187in}}{\pgfqpoint{6.223804in}{1.165021in}}{\pgfqpoint{6.216672in}{1.165021in}}%
\pgfpathcurveto{\pgfqpoint{6.209539in}{1.165021in}}{\pgfqpoint{6.202697in}{1.162187in}}{\pgfqpoint{6.197653in}{1.157143in}}%
\pgfpathcurveto{\pgfqpoint{6.192610in}{1.152099in}}{\pgfqpoint{6.189776in}{1.145258in}}{\pgfqpoint{6.189776in}{1.138125in}}%
\pgfpathcurveto{\pgfqpoint{6.189776in}{1.130992in}}{\pgfqpoint{6.192610in}{1.124150in}}{\pgfqpoint{6.197653in}{1.119107in}}%
\pgfpathcurveto{\pgfqpoint{6.202697in}{1.114063in}}{\pgfqpoint{6.209539in}{1.111229in}}{\pgfqpoint{6.216672in}{1.111229in}}%
\pgfpathclose%
\pgfusepath{stroke,fill}%
\end{pgfscope}%
\begin{pgfscope}%
\pgfpathrectangle{\pgfqpoint{4.985294in}{0.500000in}}{\pgfqpoint{1.764706in}{1.700000in}}%
\pgfusepath{clip}%
\pgfsetbuttcap%
\pgfsetroundjoin%
\definecolor{currentfill}{rgb}{0.980678,0.914765,0.856766}%
\pgfsetfillcolor{currentfill}%
\pgfsetlinewidth{0.311001pt}%
\definecolor{currentstroke}{rgb}{1.000000,1.000000,1.000000}%
\pgfsetstrokecolor{currentstroke}%
\pgfsetdash{}{0pt}%
\pgfpathmoveto{\pgfqpoint{6.296772in}{1.219692in}}%
\pgfpathcurveto{\pgfqpoint{6.303904in}{1.219692in}}{\pgfqpoint{6.310746in}{1.222526in}}{\pgfqpoint{6.315790in}{1.227569in}}%
\pgfpathcurveto{\pgfqpoint{6.320833in}{1.232613in}}{\pgfqpoint{6.323667in}{1.239454in}}{\pgfqpoint{6.323667in}{1.246587in}}%
\pgfpathcurveto{\pgfqpoint{6.323667in}{1.253720in}}{\pgfqpoint{6.320833in}{1.260562in}}{\pgfqpoint{6.315790in}{1.265605in}}%
\pgfpathcurveto{\pgfqpoint{6.310746in}{1.270649in}}{\pgfqpoint{6.303904in}{1.273483in}}{\pgfqpoint{6.296772in}{1.273483in}}%
\pgfpathcurveto{\pgfqpoint{6.289639in}{1.273483in}}{\pgfqpoint{6.282797in}{1.270649in}}{\pgfqpoint{6.277753in}{1.265605in}}%
\pgfpathcurveto{\pgfqpoint{6.272710in}{1.260562in}}{\pgfqpoint{6.269876in}{1.253720in}}{\pgfqpoint{6.269876in}{1.246587in}}%
\pgfpathcurveto{\pgfqpoint{6.269876in}{1.239454in}}{\pgfqpoint{6.272710in}{1.232613in}}{\pgfqpoint{6.277753in}{1.227569in}}%
\pgfpathcurveto{\pgfqpoint{6.282797in}{1.222526in}}{\pgfqpoint{6.289639in}{1.219692in}}{\pgfqpoint{6.296772in}{1.219692in}}%
\pgfpathclose%
\pgfusepath{stroke,fill}%
\end{pgfscope}%
\begin{pgfscope}%
\pgfpathrectangle{\pgfqpoint{4.985294in}{0.500000in}}{\pgfqpoint{1.764706in}{1.700000in}}%
\pgfusepath{clip}%
\pgfsetbuttcap%
\pgfsetroundjoin%
\definecolor{currentfill}{rgb}{0.969803,0.809811,0.702523}%
\pgfsetfillcolor{currentfill}%
\pgfsetlinewidth{0.311001pt}%
\definecolor{currentstroke}{rgb}{1.000000,1.000000,1.000000}%
\pgfsetstrokecolor{currentstroke}%
\pgfsetdash{}{0pt}%
\pgfpathmoveto{\pgfqpoint{6.355116in}{1.140893in}}%
\pgfpathcurveto{\pgfqpoint{6.362248in}{1.140893in}}{\pgfqpoint{6.369090in}{1.143727in}}{\pgfqpoint{6.374134in}{1.148770in}}%
\pgfpathcurveto{\pgfqpoint{6.379177in}{1.153814in}}{\pgfqpoint{6.382011in}{1.160655in}}{\pgfqpoint{6.382011in}{1.167788in}}%
\pgfpathcurveto{\pgfqpoint{6.382011in}{1.174921in}}{\pgfqpoint{6.379177in}{1.181763in}}{\pgfqpoint{6.374134in}{1.186806in}}%
\pgfpathcurveto{\pgfqpoint{6.369090in}{1.191850in}}{\pgfqpoint{6.362248in}{1.194684in}}{\pgfqpoint{6.355116in}{1.194684in}}%
\pgfpathcurveto{\pgfqpoint{6.347983in}{1.194684in}}{\pgfqpoint{6.341141in}{1.191850in}}{\pgfqpoint{6.336097in}{1.186806in}}%
\pgfpathcurveto{\pgfqpoint{6.331054in}{1.181763in}}{\pgfqpoint{6.328220in}{1.174921in}}{\pgfqpoint{6.328220in}{1.167788in}}%
\pgfpathcurveto{\pgfqpoint{6.328220in}{1.160655in}}{\pgfqpoint{6.331054in}{1.153814in}}{\pgfqpoint{6.336097in}{1.148770in}}%
\pgfpathcurveto{\pgfqpoint{6.341141in}{1.143727in}}{\pgfqpoint{6.347983in}{1.140893in}}{\pgfqpoint{6.355116in}{1.140893in}}%
\pgfpathclose%
\pgfusepath{stroke,fill}%
\end{pgfscope}%
\begin{pgfscope}%
\pgfpathrectangle{\pgfqpoint{4.985294in}{0.500000in}}{\pgfqpoint{1.764706in}{1.700000in}}%
\pgfusepath{clip}%
\pgfsetbuttcap%
\pgfsetroundjoin%
\definecolor{currentfill}{rgb}{0.969359,0.803954,0.693832}%
\pgfsetfillcolor{currentfill}%
\pgfsetlinewidth{0.311001pt}%
\definecolor{currentstroke}{rgb}{1.000000,1.000000,1.000000}%
\pgfsetstrokecolor{currentstroke}%
\pgfsetdash{}{0pt}%
\pgfpathmoveto{\pgfqpoint{6.347428in}{1.115877in}}%
\pgfpathcurveto{\pgfqpoint{6.354561in}{1.115877in}}{\pgfqpoint{6.361403in}{1.118711in}}{\pgfqpoint{6.366446in}{1.123754in}}%
\pgfpathcurveto{\pgfqpoint{6.371490in}{1.128798in}}{\pgfqpoint{6.374324in}{1.135640in}}{\pgfqpoint{6.374324in}{1.142772in}}%
\pgfpathcurveto{\pgfqpoint{6.374324in}{1.149905in}}{\pgfqpoint{6.371490in}{1.156747in}}{\pgfqpoint{6.366446in}{1.161791in}}%
\pgfpathcurveto{\pgfqpoint{6.361403in}{1.166834in}}{\pgfqpoint{6.354561in}{1.169668in}}{\pgfqpoint{6.347428in}{1.169668in}}%
\pgfpathcurveto{\pgfqpoint{6.340295in}{1.169668in}}{\pgfqpoint{6.333454in}{1.166834in}}{\pgfqpoint{6.328410in}{1.161791in}}%
\pgfpathcurveto{\pgfqpoint{6.323366in}{1.156747in}}{\pgfqpoint{6.320532in}{1.149905in}}{\pgfqpoint{6.320532in}{1.142772in}}%
\pgfpathcurveto{\pgfqpoint{6.320532in}{1.135640in}}{\pgfqpoint{6.323366in}{1.128798in}}{\pgfqpoint{6.328410in}{1.123754in}}%
\pgfpathcurveto{\pgfqpoint{6.333454in}{1.118711in}}{\pgfqpoint{6.340295in}{1.115877in}}{\pgfqpoint{6.347428in}{1.115877in}}%
\pgfpathclose%
\pgfusepath{stroke,fill}%
\end{pgfscope}%
\begin{pgfscope}%
\pgfpathrectangle{\pgfqpoint{4.985294in}{0.500000in}}{\pgfqpoint{1.764706in}{1.700000in}}%
\pgfusepath{clip}%
\pgfsetbuttcap%
\pgfsetroundjoin%
\definecolor{currentfill}{rgb}{0.964032,0.651225,0.493258}%
\pgfsetfillcolor{currentfill}%
\pgfsetlinewidth{0.311001pt}%
\definecolor{currentstroke}{rgb}{1.000000,1.000000,1.000000}%
\pgfsetstrokecolor{currentstroke}%
\pgfsetdash{}{0pt}%
\pgfpathmoveto{\pgfqpoint{6.296815in}{0.964171in}}%
\pgfpathcurveto{\pgfqpoint{6.303948in}{0.964171in}}{\pgfqpoint{6.310790in}{0.967004in}}{\pgfqpoint{6.315834in}{0.972048in}}%
\pgfpathcurveto{\pgfqpoint{6.320877in}{0.977092in}}{\pgfqpoint{6.323711in}{0.983933in}}{\pgfqpoint{6.323711in}{0.991066in}}%
\pgfpathcurveto{\pgfqpoint{6.323711in}{0.998199in}}{\pgfqpoint{6.320877in}{1.005041in}}{\pgfqpoint{6.315834in}{1.010084in}}%
\pgfpathcurveto{\pgfqpoint{6.310790in}{1.015128in}}{\pgfqpoint{6.303948in}{1.017962in}}{\pgfqpoint{6.296815in}{1.017962in}}%
\pgfpathcurveto{\pgfqpoint{6.289683in}{1.017962in}}{\pgfqpoint{6.282841in}{1.015128in}}{\pgfqpoint{6.277797in}{1.010084in}}%
\pgfpathcurveto{\pgfqpoint{6.272754in}{1.005041in}}{\pgfqpoint{6.269920in}{0.998199in}}{\pgfqpoint{6.269920in}{0.991066in}}%
\pgfpathcurveto{\pgfqpoint{6.269920in}{0.983933in}}{\pgfqpoint{6.272754in}{0.977092in}}{\pgfqpoint{6.277797in}{0.972048in}}%
\pgfpathcurveto{\pgfqpoint{6.282841in}{0.967004in}}{\pgfqpoint{6.289683in}{0.964171in}}{\pgfqpoint{6.296815in}{0.964171in}}%
\pgfpathclose%
\pgfusepath{stroke,fill}%
\end{pgfscope}%
\begin{pgfscope}%
\pgfpathrectangle{\pgfqpoint{4.985294in}{0.500000in}}{\pgfqpoint{1.764706in}{1.700000in}}%
\pgfusepath{clip}%
\pgfsetbuttcap%
\pgfsetroundjoin%
\definecolor{currentfill}{rgb}{0.964173,0.657587,0.500469}%
\pgfsetfillcolor{currentfill}%
\pgfsetlinewidth{0.311001pt}%
\definecolor{currentstroke}{rgb}{1.000000,1.000000,1.000000}%
\pgfsetstrokecolor{currentstroke}%
\pgfsetdash{}{0pt}%
\pgfpathmoveto{\pgfqpoint{6.161176in}{1.491277in}}%
\pgfpathcurveto{\pgfqpoint{6.168309in}{1.491277in}}{\pgfqpoint{6.175151in}{1.494111in}}{\pgfqpoint{6.180194in}{1.499155in}}%
\pgfpathcurveto{\pgfqpoint{6.185238in}{1.504198in}}{\pgfqpoint{6.188072in}{1.511040in}}{\pgfqpoint{6.188072in}{1.518173in}}%
\pgfpathcurveto{\pgfqpoint{6.188072in}{1.525306in}}{\pgfqpoint{6.185238in}{1.532147in}}{\pgfqpoint{6.180194in}{1.537191in}}%
\pgfpathcurveto{\pgfqpoint{6.175151in}{1.542235in}}{\pgfqpoint{6.168309in}{1.545069in}}{\pgfqpoint{6.161176in}{1.545069in}}%
\pgfpathcurveto{\pgfqpoint{6.154043in}{1.545069in}}{\pgfqpoint{6.147202in}{1.542235in}}{\pgfqpoint{6.142158in}{1.537191in}}%
\pgfpathcurveto{\pgfqpoint{6.137114in}{1.532147in}}{\pgfqpoint{6.134280in}{1.525306in}}{\pgfqpoint{6.134280in}{1.518173in}}%
\pgfpathcurveto{\pgfqpoint{6.134280in}{1.511040in}}{\pgfqpoint{6.137114in}{1.504198in}}{\pgfqpoint{6.142158in}{1.499155in}}%
\pgfpathcurveto{\pgfqpoint{6.147202in}{1.494111in}}{\pgfqpoint{6.154043in}{1.491277in}}{\pgfqpoint{6.161176in}{1.491277in}}%
\pgfpathclose%
\pgfusepath{stroke,fill}%
\end{pgfscope}%
\begin{pgfscope}%
\pgfpathrectangle{\pgfqpoint{4.985294in}{0.500000in}}{\pgfqpoint{1.764706in}{1.700000in}}%
\pgfusepath{clip}%
\pgfsetbuttcap%
\pgfsetroundjoin%
\definecolor{currentfill}{rgb}{0.704578,0.088213,0.344730}%
\pgfsetfillcolor{currentfill}%
\pgfsetlinewidth{0.311001pt}%
\definecolor{currentstroke}{rgb}{1.000000,1.000000,1.000000}%
\pgfsetstrokecolor{currentstroke}%
\pgfsetdash{}{0pt}%
\pgfpathmoveto{\pgfqpoint{5.218520in}{1.247678in}}%
\pgfpathcurveto{\pgfqpoint{5.225653in}{1.247678in}}{\pgfqpoint{5.232494in}{1.250512in}}{\pgfqpoint{5.237538in}{1.255555in}}%
\pgfpathcurveto{\pgfqpoint{5.242582in}{1.260599in}}{\pgfqpoint{5.245415in}{1.267441in}}{\pgfqpoint{5.245415in}{1.274574in}}%
\pgfpathcurveto{\pgfqpoint{5.245415in}{1.281706in}}{\pgfqpoint{5.242582in}{1.288548in}}{\pgfqpoint{5.237538in}{1.293592in}}%
\pgfpathcurveto{\pgfqpoint{5.232494in}{1.298635in}}{\pgfqpoint{5.225653in}{1.301469in}}{\pgfqpoint{5.218520in}{1.301469in}}%
\pgfpathcurveto{\pgfqpoint{5.211387in}{1.301469in}}{\pgfqpoint{5.204545in}{1.298635in}}{\pgfqpoint{5.199502in}{1.293592in}}%
\pgfpathcurveto{\pgfqpoint{5.194458in}{1.288548in}}{\pgfqpoint{5.191624in}{1.281706in}}{\pgfqpoint{5.191624in}{1.274574in}}%
\pgfpathcurveto{\pgfqpoint{5.191624in}{1.267441in}}{\pgfqpoint{5.194458in}{1.260599in}}{\pgfqpoint{5.199502in}{1.255555in}}%
\pgfpathcurveto{\pgfqpoint{5.204545in}{1.250512in}}{\pgfqpoint{5.211387in}{1.247678in}}{\pgfqpoint{5.218520in}{1.247678in}}%
\pgfpathclose%
\pgfusepath{stroke,fill}%
\end{pgfscope}%
\begin{pgfscope}%
\pgfpathrectangle{\pgfqpoint{4.985294in}{0.500000in}}{\pgfqpoint{1.764706in}{1.700000in}}%
\pgfusepath{clip}%
\pgfsetbuttcap%
\pgfsetroundjoin%
\definecolor{currentfill}{rgb}{0.971202,0.827364,0.728520}%
\pgfsetfillcolor{currentfill}%
\pgfsetlinewidth{0.311001pt}%
\definecolor{currentstroke}{rgb}{1.000000,1.000000,1.000000}%
\pgfsetstrokecolor{currentstroke}%
\pgfsetdash{}{0pt}%
\pgfpathmoveto{\pgfqpoint{6.260155in}{1.652365in}}%
\pgfpathcurveto{\pgfqpoint{6.267288in}{1.652365in}}{\pgfqpoint{6.274129in}{1.655198in}}{\pgfqpoint{6.279173in}{1.660242in}}%
\pgfpathcurveto{\pgfqpoint{6.284217in}{1.665286in}}{\pgfqpoint{6.287051in}{1.672127in}}{\pgfqpoint{6.287051in}{1.679260in}}%
\pgfpathcurveto{\pgfqpoint{6.287051in}{1.686393in}}{\pgfqpoint{6.284217in}{1.693235in}}{\pgfqpoint{6.279173in}{1.698278in}}%
\pgfpathcurveto{\pgfqpoint{6.274129in}{1.703322in}}{\pgfqpoint{6.267288in}{1.706156in}}{\pgfqpoint{6.260155in}{1.706156in}}%
\pgfpathcurveto{\pgfqpoint{6.253022in}{1.706156in}}{\pgfqpoint{6.246180in}{1.703322in}}{\pgfqpoint{6.241137in}{1.698278in}}%
\pgfpathcurveto{\pgfqpoint{6.236093in}{1.693235in}}{\pgfqpoint{6.233259in}{1.686393in}}{\pgfqpoint{6.233259in}{1.679260in}}%
\pgfpathcurveto{\pgfqpoint{6.233259in}{1.672127in}}{\pgfqpoint{6.236093in}{1.665286in}}{\pgfqpoint{6.241137in}{1.660242in}}%
\pgfpathcurveto{\pgfqpoint{6.246180in}{1.655198in}}{\pgfqpoint{6.253022in}{1.652365in}}{\pgfqpoint{6.260155in}{1.652365in}}%
\pgfpathclose%
\pgfusepath{stroke,fill}%
\end{pgfscope}%
\begin{pgfscope}%
\pgfpathrectangle{\pgfqpoint{4.985294in}{0.500000in}}{\pgfqpoint{1.764706in}{1.700000in}}%
\pgfusepath{clip}%
\pgfsetbuttcap%
\pgfsetroundjoin%
\definecolor{currentfill}{rgb}{0.963379,0.625574,0.465113}%
\pgfsetfillcolor{currentfill}%
\pgfsetlinewidth{0.311001pt}%
\definecolor{currentstroke}{rgb}{1.000000,1.000000,1.000000}%
\pgfsetstrokecolor{currentstroke}%
\pgfsetdash{}{0pt}%
\pgfpathmoveto{\pgfqpoint{6.102561in}{0.968675in}}%
\pgfpathcurveto{\pgfqpoint{6.109693in}{0.968675in}}{\pgfqpoint{6.116535in}{0.971509in}}{\pgfqpoint{6.121579in}{0.976552in}}%
\pgfpathcurveto{\pgfqpoint{6.126622in}{0.981596in}}{\pgfqpoint{6.129456in}{0.988438in}}{\pgfqpoint{6.129456in}{0.995570in}}%
\pgfpathcurveto{\pgfqpoint{6.129456in}{1.002703in}}{\pgfqpoint{6.126622in}{1.009545in}}{\pgfqpoint{6.121579in}{1.014589in}}%
\pgfpathcurveto{\pgfqpoint{6.116535in}{1.019632in}}{\pgfqpoint{6.109693in}{1.022466in}}{\pgfqpoint{6.102561in}{1.022466in}}%
\pgfpathcurveto{\pgfqpoint{6.095428in}{1.022466in}}{\pgfqpoint{6.088586in}{1.019632in}}{\pgfqpoint{6.083542in}{1.014589in}}%
\pgfpathcurveto{\pgfqpoint{6.078499in}{1.009545in}}{\pgfqpoint{6.075665in}{1.002703in}}{\pgfqpoint{6.075665in}{0.995570in}}%
\pgfpathcurveto{\pgfqpoint{6.075665in}{0.988438in}}{\pgfqpoint{6.078499in}{0.981596in}}{\pgfqpoint{6.083542in}{0.976552in}}%
\pgfpathcurveto{\pgfqpoint{6.088586in}{0.971509in}}{\pgfqpoint{6.095428in}{0.968675in}}{\pgfqpoint{6.102561in}{0.968675in}}%
\pgfpathclose%
\pgfusepath{stroke,fill}%
\end{pgfscope}%
\begin{pgfscope}%
\pgfpathrectangle{\pgfqpoint{4.985294in}{0.500000in}}{\pgfqpoint{1.764706in}{1.700000in}}%
\pgfusepath{clip}%
\pgfsetbuttcap%
\pgfsetroundjoin%
\definecolor{currentfill}{rgb}{0.966120,0.744512,0.608720}%
\pgfsetfillcolor{currentfill}%
\pgfsetlinewidth{0.311001pt}%
\definecolor{currentstroke}{rgb}{1.000000,1.000000,1.000000}%
\pgfsetstrokecolor{currentstroke}%
\pgfsetdash{}{0pt}%
\pgfpathmoveto{\pgfqpoint{5.597768in}{0.980445in}}%
\pgfpathcurveto{\pgfqpoint{5.604901in}{0.980445in}}{\pgfqpoint{5.611743in}{0.983279in}}{\pgfqpoint{5.616786in}{0.988322in}}%
\pgfpathcurveto{\pgfqpoint{5.621830in}{0.993366in}}{\pgfqpoint{5.624664in}{1.000208in}}{\pgfqpoint{5.624664in}{1.007341in}}%
\pgfpathcurveto{\pgfqpoint{5.624664in}{1.014473in}}{\pgfqpoint{5.621830in}{1.021315in}}{\pgfqpoint{5.616786in}{1.026359in}}%
\pgfpathcurveto{\pgfqpoint{5.611743in}{1.031402in}}{\pgfqpoint{5.604901in}{1.034236in}}{\pgfqpoint{5.597768in}{1.034236in}}%
\pgfpathcurveto{\pgfqpoint{5.590635in}{1.034236in}}{\pgfqpoint{5.583794in}{1.031402in}}{\pgfqpoint{5.578750in}{1.026359in}}%
\pgfpathcurveto{\pgfqpoint{5.573706in}{1.021315in}}{\pgfqpoint{5.570872in}{1.014473in}}{\pgfqpoint{5.570872in}{1.007341in}}%
\pgfpathcurveto{\pgfqpoint{5.570872in}{1.000208in}}{\pgfqpoint{5.573706in}{0.993366in}}{\pgfqpoint{5.578750in}{0.988322in}}%
\pgfpathcurveto{\pgfqpoint{5.583794in}{0.983279in}}{\pgfqpoint{5.590635in}{0.980445in}}{\pgfqpoint{5.597768in}{0.980445in}}%
\pgfpathclose%
\pgfusepath{stroke,fill}%
\end{pgfscope}%
\begin{pgfscope}%
\pgfpathrectangle{\pgfqpoint{4.985294in}{0.500000in}}{\pgfqpoint{1.764706in}{1.700000in}}%
\pgfusepath{clip}%
\pgfsetbuttcap%
\pgfsetroundjoin%
\definecolor{currentfill}{rgb}{0.962018,0.586477,0.424918}%
\pgfsetfillcolor{currentfill}%
\pgfsetlinewidth{0.311001pt}%
\definecolor{currentstroke}{rgb}{1.000000,1.000000,1.000000}%
\pgfsetstrokecolor{currentstroke}%
\pgfsetdash{}{0pt}%
\pgfpathmoveto{\pgfqpoint{6.130562in}{1.098837in}}%
\pgfpathcurveto{\pgfqpoint{6.137695in}{1.098837in}}{\pgfqpoint{6.144536in}{1.101671in}}{\pgfqpoint{6.149580in}{1.106715in}}%
\pgfpathcurveto{\pgfqpoint{6.154624in}{1.111759in}}{\pgfqpoint{6.157458in}{1.118600in}}{\pgfqpoint{6.157458in}{1.125733in}}%
\pgfpathcurveto{\pgfqpoint{6.157458in}{1.132866in}}{\pgfqpoint{6.154624in}{1.139708in}}{\pgfqpoint{6.149580in}{1.144751in}}%
\pgfpathcurveto{\pgfqpoint{6.144536in}{1.149795in}}{\pgfqpoint{6.137695in}{1.152629in}}{\pgfqpoint{6.130562in}{1.152629in}}%
\pgfpathcurveto{\pgfqpoint{6.123429in}{1.152629in}}{\pgfqpoint{6.116587in}{1.149795in}}{\pgfqpoint{6.111544in}{1.144751in}}%
\pgfpathcurveto{\pgfqpoint{6.106500in}{1.139708in}}{\pgfqpoint{6.103666in}{1.132866in}}{\pgfqpoint{6.103666in}{1.125733in}}%
\pgfpathcurveto{\pgfqpoint{6.103666in}{1.118600in}}{\pgfqpoint{6.106500in}{1.111759in}}{\pgfqpoint{6.111544in}{1.106715in}}%
\pgfpathcurveto{\pgfqpoint{6.116587in}{1.101671in}}{\pgfqpoint{6.123429in}{1.098837in}}{\pgfqpoint{6.130562in}{1.098837in}}%
\pgfpathclose%
\pgfusepath{stroke,fill}%
\end{pgfscope}%
\begin{pgfscope}%
\pgfpathrectangle{\pgfqpoint{4.985294in}{0.500000in}}{\pgfqpoint{1.764706in}{1.700000in}}%
\pgfusepath{clip}%
\pgfsetbuttcap%
\pgfsetroundjoin%
\definecolor{currentfill}{rgb}{0.975644,0.874038,0.797253}%
\pgfsetfillcolor{currentfill}%
\pgfsetlinewidth{0.311001pt}%
\definecolor{currentstroke}{rgb}{1.000000,1.000000,1.000000}%
\pgfsetstrokecolor{currentstroke}%
\pgfsetdash{}{0pt}%
\pgfpathmoveto{\pgfqpoint{6.238850in}{1.078545in}}%
\pgfpathcurveto{\pgfqpoint{6.245982in}{1.078545in}}{\pgfqpoint{6.252824in}{1.081379in}}{\pgfqpoint{6.257868in}{1.086422in}}%
\pgfpathcurveto{\pgfqpoint{6.262911in}{1.091466in}}{\pgfqpoint{6.265745in}{1.098308in}}{\pgfqpoint{6.265745in}{1.105441in}}%
\pgfpathcurveto{\pgfqpoint{6.265745in}{1.112573in}}{\pgfqpoint{6.262911in}{1.119415in}}{\pgfqpoint{6.257868in}{1.124459in}}%
\pgfpathcurveto{\pgfqpoint{6.252824in}{1.129502in}}{\pgfqpoint{6.245982in}{1.132336in}}{\pgfqpoint{6.238850in}{1.132336in}}%
\pgfpathcurveto{\pgfqpoint{6.231717in}{1.132336in}}{\pgfqpoint{6.224875in}{1.129502in}}{\pgfqpoint{6.219831in}{1.124459in}}%
\pgfpathcurveto{\pgfqpoint{6.214788in}{1.119415in}}{\pgfqpoint{6.211954in}{1.112573in}}{\pgfqpoint{6.211954in}{1.105441in}}%
\pgfpathcurveto{\pgfqpoint{6.211954in}{1.098308in}}{\pgfqpoint{6.214788in}{1.091466in}}{\pgfqpoint{6.219831in}{1.086422in}}%
\pgfpathcurveto{\pgfqpoint{6.224875in}{1.081379in}}{\pgfqpoint{6.231717in}{1.078545in}}{\pgfqpoint{6.238850in}{1.078545in}}%
\pgfpathclose%
\pgfusepath{stroke,fill}%
\end{pgfscope}%
\begin{pgfscope}%
\pgfpathrectangle{\pgfqpoint{4.985294in}{0.500000in}}{\pgfqpoint{1.764706in}{1.700000in}}%
\pgfusepath{clip}%
\pgfsetbuttcap%
\pgfsetroundjoin%
\definecolor{currentfill}{rgb}{0.977657,0.891500,0.822809}%
\pgfsetfillcolor{currentfill}%
\pgfsetlinewidth{0.311001pt}%
\definecolor{currentstroke}{rgb}{1.000000,1.000000,1.000000}%
\pgfsetstrokecolor{currentstroke}%
\pgfsetdash{}{0pt}%
\pgfpathmoveto{\pgfqpoint{6.264620in}{1.180219in}}%
\pgfpathcurveto{\pgfqpoint{6.271752in}{1.180219in}}{\pgfqpoint{6.278594in}{1.183053in}}{\pgfqpoint{6.283638in}{1.188097in}}%
\pgfpathcurveto{\pgfqpoint{6.288681in}{1.193141in}}{\pgfqpoint{6.291515in}{1.199982in}}{\pgfqpoint{6.291515in}{1.207115in}}%
\pgfpathcurveto{\pgfqpoint{6.291515in}{1.214248in}}{\pgfqpoint{6.288681in}{1.221089in}}{\pgfqpoint{6.283638in}{1.226133in}}%
\pgfpathcurveto{\pgfqpoint{6.278594in}{1.231177in}}{\pgfqpoint{6.271752in}{1.234011in}}{\pgfqpoint{6.264620in}{1.234011in}}%
\pgfpathcurveto{\pgfqpoint{6.257487in}{1.234011in}}{\pgfqpoint{6.250645in}{1.231177in}}{\pgfqpoint{6.245601in}{1.226133in}}%
\pgfpathcurveto{\pgfqpoint{6.240558in}{1.221089in}}{\pgfqpoint{6.237724in}{1.214248in}}{\pgfqpoint{6.237724in}{1.207115in}}%
\pgfpathcurveto{\pgfqpoint{6.237724in}{1.199982in}}{\pgfqpoint{6.240558in}{1.193141in}}{\pgfqpoint{6.245601in}{1.188097in}}%
\pgfpathcurveto{\pgfqpoint{6.250645in}{1.183053in}}{\pgfqpoint{6.257487in}{1.180219in}}{\pgfqpoint{6.264620in}{1.180219in}}%
\pgfpathclose%
\pgfusepath{stroke,fill}%
\end{pgfscope}%
\begin{pgfscope}%
\pgfpathrectangle{\pgfqpoint{4.985294in}{0.500000in}}{\pgfqpoint{1.764706in}{1.700000in}}%
\pgfusepath{clip}%
\pgfsetbuttcap%
\pgfsetroundjoin%
\definecolor{currentfill}{rgb}{0.967735,0.780441,0.659127}%
\pgfsetfillcolor{currentfill}%
\pgfsetlinewidth{0.311001pt}%
\definecolor{currentstroke}{rgb}{1.000000,1.000000,1.000000}%
\pgfsetstrokecolor{currentstroke}%
\pgfsetdash{}{0pt}%
\pgfpathmoveto{\pgfqpoint{5.440124in}{0.994462in}}%
\pgfpathcurveto{\pgfqpoint{5.447257in}{0.994462in}}{\pgfqpoint{5.454098in}{0.997296in}}{\pgfqpoint{5.459142in}{1.002339in}}%
\pgfpathcurveto{\pgfqpoint{5.464186in}{1.007383in}}{\pgfqpoint{5.467020in}{1.014225in}}{\pgfqpoint{5.467020in}{1.021358in}}%
\pgfpathcurveto{\pgfqpoint{5.467020in}{1.028490in}}{\pgfqpoint{5.464186in}{1.035332in}}{\pgfqpoint{5.459142in}{1.040376in}}%
\pgfpathcurveto{\pgfqpoint{5.454098in}{1.045419in}}{\pgfqpoint{5.447257in}{1.048253in}}{\pgfqpoint{5.440124in}{1.048253in}}%
\pgfpathcurveto{\pgfqpoint{5.432991in}{1.048253in}}{\pgfqpoint{5.426150in}{1.045419in}}{\pgfqpoint{5.421106in}{1.040376in}}%
\pgfpathcurveto{\pgfqpoint{5.416062in}{1.035332in}}{\pgfqpoint{5.413228in}{1.028490in}}{\pgfqpoint{5.413228in}{1.021358in}}%
\pgfpathcurveto{\pgfqpoint{5.413228in}{1.014225in}}{\pgfqpoint{5.416062in}{1.007383in}}{\pgfqpoint{5.421106in}{1.002339in}}%
\pgfpathcurveto{\pgfqpoint{5.426150in}{0.997296in}}{\pgfqpoint{5.432991in}{0.994462in}}{\pgfqpoint{5.440124in}{0.994462in}}%
\pgfpathclose%
\pgfusepath{stroke,fill}%
\end{pgfscope}%
\begin{pgfscope}%
\pgfpathrectangle{\pgfqpoint{4.985294in}{0.500000in}}{\pgfqpoint{1.764706in}{1.700000in}}%
\pgfusepath{clip}%
\pgfsetbuttcap%
\pgfsetroundjoin%
\definecolor{currentfill}{rgb}{0.974412,0.862387,0.780156}%
\pgfsetfillcolor{currentfill}%
\pgfsetlinewidth{0.311001pt}%
\definecolor{currentstroke}{rgb}{1.000000,1.000000,1.000000}%
\pgfsetstrokecolor{currentstroke}%
\pgfsetdash{}{0pt}%
\pgfpathmoveto{\pgfqpoint{6.367321in}{1.296454in}}%
\pgfpathcurveto{\pgfqpoint{6.374454in}{1.296454in}}{\pgfqpoint{6.381296in}{1.299287in}}{\pgfqpoint{6.386339in}{1.304331in}}%
\pgfpathcurveto{\pgfqpoint{6.391383in}{1.309375in}}{\pgfqpoint{6.394217in}{1.316216in}}{\pgfqpoint{6.394217in}{1.323349in}}%
\pgfpathcurveto{\pgfqpoint{6.394217in}{1.330482in}}{\pgfqpoint{6.391383in}{1.337324in}}{\pgfqpoint{6.386339in}{1.342367in}}%
\pgfpathcurveto{\pgfqpoint{6.381296in}{1.347411in}}{\pgfqpoint{6.374454in}{1.350245in}}{\pgfqpoint{6.367321in}{1.350245in}}%
\pgfpathcurveto{\pgfqpoint{6.360189in}{1.350245in}}{\pgfqpoint{6.353347in}{1.347411in}}{\pgfqpoint{6.348303in}{1.342367in}}%
\pgfpathcurveto{\pgfqpoint{6.343260in}{1.337324in}}{\pgfqpoint{6.340426in}{1.330482in}}{\pgfqpoint{6.340426in}{1.323349in}}%
\pgfpathcurveto{\pgfqpoint{6.340426in}{1.316216in}}{\pgfqpoint{6.343260in}{1.309375in}}{\pgfqpoint{6.348303in}{1.304331in}}%
\pgfpathcurveto{\pgfqpoint{6.353347in}{1.299287in}}{\pgfqpoint{6.360189in}{1.296454in}}{\pgfqpoint{6.367321in}{1.296454in}}%
\pgfpathclose%
\pgfusepath{stroke,fill}%
\end{pgfscope}%
\begin{pgfscope}%
\pgfpathrectangle{\pgfqpoint{4.985294in}{0.500000in}}{\pgfqpoint{1.764706in}{1.700000in}}%
\pgfusepath{clip}%
\pgfsetbuttcap%
\pgfsetroundjoin%
\definecolor{currentfill}{rgb}{0.976287,0.879862,0.805788}%
\pgfsetfillcolor{currentfill}%
\pgfsetlinewidth{0.311001pt}%
\definecolor{currentstroke}{rgb}{1.000000,1.000000,1.000000}%
\pgfsetstrokecolor{currentstroke}%
\pgfsetdash{}{0pt}%
\pgfpathmoveto{\pgfqpoint{5.471761in}{1.089156in}}%
\pgfpathcurveto{\pgfqpoint{5.478893in}{1.089156in}}{\pgfqpoint{5.485735in}{1.091990in}}{\pgfqpoint{5.490779in}{1.097034in}}%
\pgfpathcurveto{\pgfqpoint{5.495822in}{1.102078in}}{\pgfqpoint{5.498656in}{1.108919in}}{\pgfqpoint{5.498656in}{1.116052in}}%
\pgfpathcurveto{\pgfqpoint{5.498656in}{1.123185in}}{\pgfqpoint{5.495822in}{1.130026in}}{\pgfqpoint{5.490779in}{1.135070in}}%
\pgfpathcurveto{\pgfqpoint{5.485735in}{1.140114in}}{\pgfqpoint{5.478893in}{1.142948in}}{\pgfqpoint{5.471761in}{1.142948in}}%
\pgfpathcurveto{\pgfqpoint{5.464628in}{1.142948in}}{\pgfqpoint{5.457786in}{1.140114in}}{\pgfqpoint{5.452742in}{1.135070in}}%
\pgfpathcurveto{\pgfqpoint{5.447699in}{1.130026in}}{\pgfqpoint{5.444865in}{1.123185in}}{\pgfqpoint{5.444865in}{1.116052in}}%
\pgfpathcurveto{\pgfqpoint{5.444865in}{1.108919in}}{\pgfqpoint{5.447699in}{1.102078in}}{\pgfqpoint{5.452742in}{1.097034in}}%
\pgfpathcurveto{\pgfqpoint{5.457786in}{1.091990in}}{\pgfqpoint{5.464628in}{1.089156in}}{\pgfqpoint{5.471761in}{1.089156in}}%
\pgfpathclose%
\pgfusepath{stroke,fill}%
\end{pgfscope}%
\begin{pgfscope}%
\pgfpathrectangle{\pgfqpoint{4.985294in}{0.500000in}}{\pgfqpoint{1.764706in}{1.700000in}}%
\pgfusepath{clip}%
\pgfsetbuttcap%
\pgfsetroundjoin%
\definecolor{currentfill}{rgb}{0.969359,0.803954,0.693832}%
\pgfsetfillcolor{currentfill}%
\pgfsetlinewidth{0.311001pt}%
\definecolor{currentstroke}{rgb}{1.000000,1.000000,1.000000}%
\pgfsetstrokecolor{currentstroke}%
\pgfsetdash{}{0pt}%
\pgfpathmoveto{\pgfqpoint{6.385349in}{1.342184in}}%
\pgfpathcurveto{\pgfqpoint{6.392482in}{1.342184in}}{\pgfqpoint{6.399324in}{1.345018in}}{\pgfqpoint{6.404368in}{1.350062in}}%
\pgfpathcurveto{\pgfqpoint{6.409411in}{1.355105in}}{\pgfqpoint{6.412245in}{1.361947in}}{\pgfqpoint{6.412245in}{1.369080in}}%
\pgfpathcurveto{\pgfqpoint{6.412245in}{1.376213in}}{\pgfqpoint{6.409411in}{1.383054in}}{\pgfqpoint{6.404368in}{1.388098in}}%
\pgfpathcurveto{\pgfqpoint{6.399324in}{1.393142in}}{\pgfqpoint{6.392482in}{1.395976in}}{\pgfqpoint{6.385349in}{1.395976in}}%
\pgfpathcurveto{\pgfqpoint{6.378217in}{1.395976in}}{\pgfqpoint{6.371375in}{1.393142in}}{\pgfqpoint{6.366331in}{1.388098in}}%
\pgfpathcurveto{\pgfqpoint{6.361288in}{1.383054in}}{\pgfqpoint{6.358454in}{1.376213in}}{\pgfqpoint{6.358454in}{1.369080in}}%
\pgfpathcurveto{\pgfqpoint{6.358454in}{1.361947in}}{\pgfqpoint{6.361288in}{1.355105in}}{\pgfqpoint{6.366331in}{1.350062in}}%
\pgfpathcurveto{\pgfqpoint{6.371375in}{1.345018in}}{\pgfqpoint{6.378217in}{1.342184in}}{\pgfqpoint{6.385349in}{1.342184in}}%
\pgfpathclose%
\pgfusepath{stroke,fill}%
\end{pgfscope}%
\begin{pgfscope}%
\pgfpathrectangle{\pgfqpoint{4.985294in}{0.500000in}}{\pgfqpoint{1.764706in}{1.700000in}}%
\pgfusepath{clip}%
\pgfsetbuttcap%
\pgfsetroundjoin%
\definecolor{currentfill}{rgb}{0.965753,0.732351,0.592427}%
\pgfsetfillcolor{currentfill}%
\pgfsetlinewidth{0.311001pt}%
\definecolor{currentstroke}{rgb}{1.000000,1.000000,1.000000}%
\pgfsetstrokecolor{currentstroke}%
\pgfsetdash{}{0pt}%
\pgfpathmoveto{\pgfqpoint{6.405938in}{1.305676in}}%
\pgfpathcurveto{\pgfqpoint{6.413071in}{1.305676in}}{\pgfqpoint{6.419912in}{1.308509in}}{\pgfqpoint{6.424956in}{1.313553in}}%
\pgfpathcurveto{\pgfqpoint{6.430000in}{1.318597in}}{\pgfqpoint{6.432834in}{1.325438in}}{\pgfqpoint{6.432834in}{1.332571in}}%
\pgfpathcurveto{\pgfqpoint{6.432834in}{1.339704in}}{\pgfqpoint{6.430000in}{1.346546in}}{\pgfqpoint{6.424956in}{1.351589in}}%
\pgfpathcurveto{\pgfqpoint{6.419912in}{1.356633in}}{\pgfqpoint{6.413071in}{1.359467in}}{\pgfqpoint{6.405938in}{1.359467in}}%
\pgfpathcurveto{\pgfqpoint{6.398805in}{1.359467in}}{\pgfqpoint{6.391963in}{1.356633in}}{\pgfqpoint{6.386920in}{1.351589in}}%
\pgfpathcurveto{\pgfqpoint{6.381876in}{1.346546in}}{\pgfqpoint{6.379042in}{1.339704in}}{\pgfqpoint{6.379042in}{1.332571in}}%
\pgfpathcurveto{\pgfqpoint{6.379042in}{1.325438in}}{\pgfqpoint{6.381876in}{1.318597in}}{\pgfqpoint{6.386920in}{1.313553in}}%
\pgfpathcurveto{\pgfqpoint{6.391963in}{1.308509in}}{\pgfqpoint{6.398805in}{1.305676in}}{\pgfqpoint{6.405938in}{1.305676in}}%
\pgfpathclose%
\pgfusepath{stroke,fill}%
\end{pgfscope}%
\begin{pgfscope}%
\pgfpathrectangle{\pgfqpoint{4.985294in}{0.500000in}}{\pgfqpoint{1.764706in}{1.700000in}}%
\pgfusepath{clip}%
\pgfsetbuttcap%
\pgfsetroundjoin%
\definecolor{currentfill}{rgb}{0.970255,0.815666,0.711203}%
\pgfsetfillcolor{currentfill}%
\pgfsetlinewidth{0.311001pt}%
\definecolor{currentstroke}{rgb}{1.000000,1.000000,1.000000}%
\pgfsetstrokecolor{currentstroke}%
\pgfsetdash{}{0pt}%
\pgfpathmoveto{\pgfqpoint{5.523304in}{0.970303in}}%
\pgfpathcurveto{\pgfqpoint{5.530437in}{0.970303in}}{\pgfqpoint{5.537279in}{0.973137in}}{\pgfqpoint{5.542322in}{0.978180in}}%
\pgfpathcurveto{\pgfqpoint{5.547366in}{0.983224in}}{\pgfqpoint{5.550200in}{0.990066in}}{\pgfqpoint{5.550200in}{0.997199in}}%
\pgfpathcurveto{\pgfqpoint{5.550200in}{1.004331in}}{\pgfqpoint{5.547366in}{1.011173in}}{\pgfqpoint{5.542322in}{1.016217in}}%
\pgfpathcurveto{\pgfqpoint{5.537279in}{1.021260in}}{\pgfqpoint{5.530437in}{1.024094in}}{\pgfqpoint{5.523304in}{1.024094in}}%
\pgfpathcurveto{\pgfqpoint{5.516171in}{1.024094in}}{\pgfqpoint{5.509330in}{1.021260in}}{\pgfqpoint{5.504286in}{1.016217in}}%
\pgfpathcurveto{\pgfqpoint{5.499242in}{1.011173in}}{\pgfqpoint{5.496408in}{1.004331in}}{\pgfqpoint{5.496408in}{0.997199in}}%
\pgfpathcurveto{\pgfqpoint{5.496408in}{0.990066in}}{\pgfqpoint{5.499242in}{0.983224in}}{\pgfqpoint{5.504286in}{0.978180in}}%
\pgfpathcurveto{\pgfqpoint{5.509330in}{0.973137in}}{\pgfqpoint{5.516171in}{0.970303in}}{\pgfqpoint{5.523304in}{0.970303in}}%
\pgfpathclose%
\pgfusepath{stroke,fill}%
\end{pgfscope}%
\begin{pgfscope}%
\pgfpathrectangle{\pgfqpoint{4.985294in}{0.500000in}}{\pgfqpoint{1.764706in}{1.700000in}}%
\pgfusepath{clip}%
\pgfsetbuttcap%
\pgfsetroundjoin%
\definecolor{currentfill}{rgb}{0.963190,0.619109,0.458249}%
\pgfsetfillcolor{currentfill}%
\pgfsetlinewidth{0.311001pt}%
\definecolor{currentstroke}{rgb}{1.000000,1.000000,1.000000}%
\pgfsetstrokecolor{currentstroke}%
\pgfsetdash{}{0pt}%
\pgfpathmoveto{\pgfqpoint{6.176599in}{1.262943in}}%
\pgfpathcurveto{\pgfqpoint{6.183732in}{1.262943in}}{\pgfqpoint{6.190574in}{1.265777in}}{\pgfqpoint{6.195617in}{1.270820in}}%
\pgfpathcurveto{\pgfqpoint{6.200661in}{1.275864in}}{\pgfqpoint{6.203495in}{1.282706in}}{\pgfqpoint{6.203495in}{1.289838in}}%
\pgfpathcurveto{\pgfqpoint{6.203495in}{1.296971in}}{\pgfqpoint{6.200661in}{1.303813in}}{\pgfqpoint{6.195617in}{1.308857in}}%
\pgfpathcurveto{\pgfqpoint{6.190574in}{1.313900in}}{\pgfqpoint{6.183732in}{1.316734in}}{\pgfqpoint{6.176599in}{1.316734in}}%
\pgfpathcurveto{\pgfqpoint{6.169466in}{1.316734in}}{\pgfqpoint{6.162625in}{1.313900in}}{\pgfqpoint{6.157581in}{1.308857in}}%
\pgfpathcurveto{\pgfqpoint{6.152537in}{1.303813in}}{\pgfqpoint{6.149704in}{1.296971in}}{\pgfqpoint{6.149704in}{1.289838in}}%
\pgfpathcurveto{\pgfqpoint{6.149704in}{1.282706in}}{\pgfqpoint{6.152537in}{1.275864in}}{\pgfqpoint{6.157581in}{1.270820in}}%
\pgfpathcurveto{\pgfqpoint{6.162625in}{1.265777in}}{\pgfqpoint{6.169466in}{1.262943in}}{\pgfqpoint{6.176599in}{1.262943in}}%
\pgfpathclose%
\pgfusepath{stroke,fill}%
\end{pgfscope}%
\begin{pgfscope}%
\pgfpathrectangle{\pgfqpoint{4.985294in}{0.500000in}}{\pgfqpoint{1.764706in}{1.700000in}}%
\pgfusepath{clip}%
\pgfsetbuttcap%
\pgfsetroundjoin%
\definecolor{currentfill}{rgb}{0.963728,0.638439,0.479050}%
\pgfsetfillcolor{currentfill}%
\pgfsetlinewidth{0.311001pt}%
\definecolor{currentstroke}{rgb}{1.000000,1.000000,1.000000}%
\pgfsetstrokecolor{currentstroke}%
\pgfsetdash{}{0pt}%
\pgfpathmoveto{\pgfqpoint{6.130791in}{1.064658in}}%
\pgfpathcurveto{\pgfqpoint{6.137923in}{1.064658in}}{\pgfqpoint{6.144765in}{1.067492in}}{\pgfqpoint{6.149809in}{1.072535in}}%
\pgfpathcurveto{\pgfqpoint{6.154852in}{1.077579in}}{\pgfqpoint{6.157686in}{1.084421in}}{\pgfqpoint{6.157686in}{1.091554in}}%
\pgfpathcurveto{\pgfqpoint{6.157686in}{1.098686in}}{\pgfqpoint{6.154852in}{1.105528in}}{\pgfqpoint{6.149809in}{1.110572in}}%
\pgfpathcurveto{\pgfqpoint{6.144765in}{1.115615in}}{\pgfqpoint{6.137923in}{1.118449in}}{\pgfqpoint{6.130791in}{1.118449in}}%
\pgfpathcurveto{\pgfqpoint{6.123658in}{1.118449in}}{\pgfqpoint{6.116816in}{1.115615in}}{\pgfqpoint{6.111772in}{1.110572in}}%
\pgfpathcurveto{\pgfqpoint{6.106729in}{1.105528in}}{\pgfqpoint{6.103895in}{1.098686in}}{\pgfqpoint{6.103895in}{1.091554in}}%
\pgfpathcurveto{\pgfqpoint{6.103895in}{1.084421in}}{\pgfqpoint{6.106729in}{1.077579in}}{\pgfqpoint{6.111772in}{1.072535in}}%
\pgfpathcurveto{\pgfqpoint{6.116816in}{1.067492in}}{\pgfqpoint{6.123658in}{1.064658in}}{\pgfqpoint{6.130791in}{1.064658in}}%
\pgfpathclose%
\pgfusepath{stroke,fill}%
\end{pgfscope}%
\begin{pgfscope}%
\pgfpathrectangle{\pgfqpoint{4.985294in}{0.500000in}}{\pgfqpoint{1.764706in}{1.700000in}}%
\pgfusepath{clip}%
\pgfsetbuttcap%
\pgfsetroundjoin%
\definecolor{currentfill}{rgb}{0.979891,0.908948,0.848279}%
\pgfsetfillcolor{currentfill}%
\pgfsetlinewidth{0.311001pt}%
\definecolor{currentstroke}{rgb}{1.000000,1.000000,1.000000}%
\pgfsetstrokecolor{currentstroke}%
\pgfsetdash{}{0pt}%
\pgfpathmoveto{\pgfqpoint{6.326135in}{1.430789in}}%
\pgfpathcurveto{\pgfqpoint{6.333268in}{1.430789in}}{\pgfqpoint{6.340110in}{1.433623in}}{\pgfqpoint{6.345153in}{1.438667in}}%
\pgfpathcurveto{\pgfqpoint{6.350197in}{1.443711in}}{\pgfqpoint{6.353031in}{1.450552in}}{\pgfqpoint{6.353031in}{1.457685in}}%
\pgfpathcurveto{\pgfqpoint{6.353031in}{1.464818in}}{\pgfqpoint{6.350197in}{1.471660in}}{\pgfqpoint{6.345153in}{1.476703in}}%
\pgfpathcurveto{\pgfqpoint{6.340110in}{1.481747in}}{\pgfqpoint{6.333268in}{1.484581in}}{\pgfqpoint{6.326135in}{1.484581in}}%
\pgfpathcurveto{\pgfqpoint{6.319003in}{1.484581in}}{\pgfqpoint{6.312161in}{1.481747in}}{\pgfqpoint{6.307117in}{1.476703in}}%
\pgfpathcurveto{\pgfqpoint{6.302074in}{1.471660in}}{\pgfqpoint{6.299240in}{1.464818in}}{\pgfqpoint{6.299240in}{1.457685in}}%
\pgfpathcurveto{\pgfqpoint{6.299240in}{1.450552in}}{\pgfqpoint{6.302074in}{1.443711in}}{\pgfqpoint{6.307117in}{1.438667in}}%
\pgfpathcurveto{\pgfqpoint{6.312161in}{1.433623in}}{\pgfqpoint{6.319003in}{1.430789in}}{\pgfqpoint{6.326135in}{1.430789in}}%
\pgfpathclose%
\pgfusepath{stroke,fill}%
\end{pgfscope}%
\begin{pgfscope}%
\pgfpathrectangle{\pgfqpoint{4.985294in}{0.500000in}}{\pgfqpoint{1.764706in}{1.700000in}}%
\pgfusepath{clip}%
\pgfsetbuttcap%
\pgfsetroundjoin%
\definecolor{currentfill}{rgb}{0.948235,0.413004,0.283323}%
\pgfsetfillcolor{currentfill}%
\pgfsetlinewidth{0.311001pt}%
\definecolor{currentstroke}{rgb}{1.000000,1.000000,1.000000}%
\pgfsetstrokecolor{currentstroke}%
\pgfsetdash{}{0pt}%
\pgfpathmoveto{\pgfqpoint{6.079658in}{1.067668in}}%
\pgfpathcurveto{\pgfqpoint{6.086791in}{1.067668in}}{\pgfqpoint{6.093632in}{1.070502in}}{\pgfqpoint{6.098676in}{1.075545in}}%
\pgfpathcurveto{\pgfqpoint{6.103720in}{1.080589in}}{\pgfqpoint{6.106554in}{1.087431in}}{\pgfqpoint{6.106554in}{1.094563in}}%
\pgfpathcurveto{\pgfqpoint{6.106554in}{1.101696in}}{\pgfqpoint{6.103720in}{1.108538in}}{\pgfqpoint{6.098676in}{1.113582in}}%
\pgfpathcurveto{\pgfqpoint{6.093632in}{1.118625in}}{\pgfqpoint{6.086791in}{1.121459in}}{\pgfqpoint{6.079658in}{1.121459in}}%
\pgfpathcurveto{\pgfqpoint{6.072525in}{1.121459in}}{\pgfqpoint{6.065683in}{1.118625in}}{\pgfqpoint{6.060640in}{1.113582in}}%
\pgfpathcurveto{\pgfqpoint{6.055596in}{1.108538in}}{\pgfqpoint{6.052762in}{1.101696in}}{\pgfqpoint{6.052762in}{1.094563in}}%
\pgfpathcurveto{\pgfqpoint{6.052762in}{1.087431in}}{\pgfqpoint{6.055596in}{1.080589in}}{\pgfqpoint{6.060640in}{1.075545in}}%
\pgfpathcurveto{\pgfqpoint{6.065683in}{1.070502in}}{\pgfqpoint{6.072525in}{1.067668in}}{\pgfqpoint{6.079658in}{1.067668in}}%
\pgfpathclose%
\pgfusepath{stroke,fill}%
\end{pgfscope}%
\begin{pgfscope}%
\pgfpathrectangle{\pgfqpoint{4.985294in}{0.500000in}}{\pgfqpoint{1.764706in}{1.700000in}}%
\pgfusepath{clip}%
\pgfsetbuttcap%
\pgfsetroundjoin%
\definecolor{currentfill}{rgb}{0.973832,0.856556,0.771584}%
\pgfsetfillcolor{currentfill}%
\pgfsetlinewidth{0.311001pt}%
\definecolor{currentstroke}{rgb}{1.000000,1.000000,1.000000}%
\pgfsetstrokecolor{currentstroke}%
\pgfsetdash{}{0pt}%
\pgfpathmoveto{\pgfqpoint{6.238318in}{1.048033in}}%
\pgfpathcurveto{\pgfqpoint{6.245451in}{1.048033in}}{\pgfqpoint{6.252293in}{1.050867in}}{\pgfqpoint{6.257336in}{1.055911in}}%
\pgfpathcurveto{\pgfqpoint{6.262380in}{1.060954in}}{\pgfqpoint{6.265214in}{1.067796in}}{\pgfqpoint{6.265214in}{1.074929in}}%
\pgfpathcurveto{\pgfqpoint{6.265214in}{1.082062in}}{\pgfqpoint{6.262380in}{1.088903in}}{\pgfqpoint{6.257336in}{1.093947in}}%
\pgfpathcurveto{\pgfqpoint{6.252293in}{1.098991in}}{\pgfqpoint{6.245451in}{1.101825in}}{\pgfqpoint{6.238318in}{1.101825in}}%
\pgfpathcurveto{\pgfqpoint{6.231186in}{1.101825in}}{\pgfqpoint{6.224344in}{1.098991in}}{\pgfqpoint{6.219300in}{1.093947in}}%
\pgfpathcurveto{\pgfqpoint{6.214257in}{1.088903in}}{\pgfqpoint{6.211423in}{1.082062in}}{\pgfqpoint{6.211423in}{1.074929in}}%
\pgfpathcurveto{\pgfqpoint{6.211423in}{1.067796in}}{\pgfqpoint{6.214257in}{1.060954in}}{\pgfqpoint{6.219300in}{1.055911in}}%
\pgfpathcurveto{\pgfqpoint{6.224344in}{1.050867in}}{\pgfqpoint{6.231186in}{1.048033in}}{\pgfqpoint{6.238318in}{1.048033in}}%
\pgfpathclose%
\pgfusepath{stroke,fill}%
\end{pgfscope}%
\begin{pgfscope}%
\pgfpathrectangle{\pgfqpoint{4.985294in}{0.500000in}}{\pgfqpoint{1.764706in}{1.700000in}}%
\pgfusepath{clip}%
\pgfsetbuttcap%
\pgfsetroundjoin%
\definecolor{currentfill}{rgb}{0.975644,0.874038,0.797253}%
\pgfsetfillcolor{currentfill}%
\pgfsetlinewidth{0.311001pt}%
\definecolor{currentstroke}{rgb}{1.000000,1.000000,1.000000}%
\pgfsetstrokecolor{currentstroke}%
\pgfsetdash{}{0pt}%
\pgfpathmoveto{\pgfqpoint{6.289724in}{1.102790in}}%
\pgfpathcurveto{\pgfqpoint{6.296857in}{1.102790in}}{\pgfqpoint{6.303698in}{1.105624in}}{\pgfqpoint{6.308742in}{1.110668in}}%
\pgfpathcurveto{\pgfqpoint{6.313786in}{1.115711in}}{\pgfqpoint{6.316619in}{1.122553in}}{\pgfqpoint{6.316619in}{1.129686in}}%
\pgfpathcurveto{\pgfqpoint{6.316619in}{1.136819in}}{\pgfqpoint{6.313786in}{1.143660in}}{\pgfqpoint{6.308742in}{1.148704in}}%
\pgfpathcurveto{\pgfqpoint{6.303698in}{1.153748in}}{\pgfqpoint{6.296857in}{1.156582in}}{\pgfqpoint{6.289724in}{1.156582in}}%
\pgfpathcurveto{\pgfqpoint{6.282591in}{1.156582in}}{\pgfqpoint{6.275749in}{1.153748in}}{\pgfqpoint{6.270706in}{1.148704in}}%
\pgfpathcurveto{\pgfqpoint{6.265662in}{1.143660in}}{\pgfqpoint{6.262828in}{1.136819in}}{\pgfqpoint{6.262828in}{1.129686in}}%
\pgfpathcurveto{\pgfqpoint{6.262828in}{1.122553in}}{\pgfqpoint{6.265662in}{1.115711in}}{\pgfqpoint{6.270706in}{1.110668in}}%
\pgfpathcurveto{\pgfqpoint{6.275749in}{1.105624in}}{\pgfqpoint{6.282591in}{1.102790in}}{\pgfqpoint{6.289724in}{1.102790in}}%
\pgfpathclose%
\pgfusepath{stroke,fill}%
\end{pgfscope}%
\begin{pgfscope}%
\pgfpathrectangle{\pgfqpoint{4.985294in}{0.500000in}}{\pgfqpoint{1.764706in}{1.700000in}}%
\pgfusepath{clip}%
\pgfsetbuttcap%
\pgfsetroundjoin%
\definecolor{currentfill}{rgb}{0.961734,0.579886,0.418445}%
\pgfsetfillcolor{currentfill}%
\pgfsetlinewidth{0.311001pt}%
\definecolor{currentstroke}{rgb}{1.000000,1.000000,1.000000}%
\pgfsetstrokecolor{currentstroke}%
\pgfsetdash{}{0pt}%
\pgfpathmoveto{\pgfqpoint{6.270988in}{1.725357in}}%
\pgfpathcurveto{\pgfqpoint{6.278121in}{1.725357in}}{\pgfqpoint{6.284962in}{1.728191in}}{\pgfqpoint{6.290006in}{1.733235in}}%
\pgfpathcurveto{\pgfqpoint{6.295050in}{1.738278in}}{\pgfqpoint{6.297884in}{1.745120in}}{\pgfqpoint{6.297884in}{1.752253in}}%
\pgfpathcurveto{\pgfqpoint{6.297884in}{1.759386in}}{\pgfqpoint{6.295050in}{1.766227in}}{\pgfqpoint{6.290006in}{1.771271in}}%
\pgfpathcurveto{\pgfqpoint{6.284962in}{1.776315in}}{\pgfqpoint{6.278121in}{1.779148in}}{\pgfqpoint{6.270988in}{1.779148in}}%
\pgfpathcurveto{\pgfqpoint{6.263855in}{1.779148in}}{\pgfqpoint{6.257013in}{1.776315in}}{\pgfqpoint{6.251970in}{1.771271in}}%
\pgfpathcurveto{\pgfqpoint{6.246926in}{1.766227in}}{\pgfqpoint{6.244092in}{1.759386in}}{\pgfqpoint{6.244092in}{1.752253in}}%
\pgfpathcurveto{\pgfqpoint{6.244092in}{1.745120in}}{\pgfqpoint{6.246926in}{1.738278in}}{\pgfqpoint{6.251970in}{1.733235in}}%
\pgfpathcurveto{\pgfqpoint{6.257013in}{1.728191in}}{\pgfqpoint{6.263855in}{1.725357in}}{\pgfqpoint{6.270988in}{1.725357in}}%
\pgfpathclose%
\pgfusepath{stroke,fill}%
\end{pgfscope}%
\begin{pgfscope}%
\pgfpathrectangle{\pgfqpoint{4.985294in}{0.500000in}}{\pgfqpoint{1.764706in}{1.700000in}}%
\pgfusepath{clip}%
\pgfsetbuttcap%
\pgfsetroundjoin%
\definecolor{currentfill}{rgb}{0.979124,0.903132,0.839793}%
\pgfsetfillcolor{currentfill}%
\pgfsetlinewidth{0.311001pt}%
\definecolor{currentstroke}{rgb}{1.000000,1.000000,1.000000}%
\pgfsetstrokecolor{currentstroke}%
\pgfsetdash{}{0pt}%
\pgfpathmoveto{\pgfqpoint{6.324707in}{1.202565in}}%
\pgfpathcurveto{\pgfqpoint{6.331840in}{1.202565in}}{\pgfqpoint{6.338682in}{1.205399in}}{\pgfqpoint{6.343725in}{1.210442in}}%
\pgfpathcurveto{\pgfqpoint{6.348769in}{1.215486in}}{\pgfqpoint{6.351603in}{1.222328in}}{\pgfqpoint{6.351603in}{1.229460in}}%
\pgfpathcurveto{\pgfqpoint{6.351603in}{1.236593in}}{\pgfqpoint{6.348769in}{1.243435in}}{\pgfqpoint{6.343725in}{1.248479in}}%
\pgfpathcurveto{\pgfqpoint{6.338682in}{1.253522in}}{\pgfqpoint{6.331840in}{1.256356in}}{\pgfqpoint{6.324707in}{1.256356in}}%
\pgfpathcurveto{\pgfqpoint{6.317574in}{1.256356in}}{\pgfqpoint{6.310733in}{1.253522in}}{\pgfqpoint{6.305689in}{1.248479in}}%
\pgfpathcurveto{\pgfqpoint{6.300645in}{1.243435in}}{\pgfqpoint{6.297811in}{1.236593in}}{\pgfqpoint{6.297811in}{1.229460in}}%
\pgfpathcurveto{\pgfqpoint{6.297811in}{1.222328in}}{\pgfqpoint{6.300645in}{1.215486in}}{\pgfqpoint{6.305689in}{1.210442in}}%
\pgfpathcurveto{\pgfqpoint{6.310733in}{1.205399in}}{\pgfqpoint{6.317574in}{1.202565in}}{\pgfqpoint{6.324707in}{1.202565in}}%
\pgfpathclose%
\pgfusepath{stroke,fill}%
\end{pgfscope}%
\begin{pgfscope}%
\pgfpathrectangle{\pgfqpoint{4.985294in}{0.500000in}}{\pgfqpoint{1.764706in}{1.700000in}}%
\pgfusepath{clip}%
\pgfsetbuttcap%
\pgfsetroundjoin%
\definecolor{currentfill}{rgb}{0.968105,0.786346,0.667739}%
\pgfsetfillcolor{currentfill}%
\pgfsetlinewidth{0.311001pt}%
\definecolor{currentstroke}{rgb}{1.000000,1.000000,1.000000}%
\pgfsetstrokecolor{currentstroke}%
\pgfsetdash{}{0pt}%
\pgfpathmoveto{\pgfqpoint{6.314063in}{1.049267in}}%
\pgfpathcurveto{\pgfqpoint{6.321196in}{1.049267in}}{\pgfqpoint{6.328037in}{1.052101in}}{\pgfqpoint{6.333081in}{1.057145in}}%
\pgfpathcurveto{\pgfqpoint{6.338124in}{1.062188in}}{\pgfqpoint{6.340958in}{1.069030in}}{\pgfqpoint{6.340958in}{1.076163in}}%
\pgfpathcurveto{\pgfqpoint{6.340958in}{1.083296in}}{\pgfqpoint{6.338124in}{1.090137in}}{\pgfqpoint{6.333081in}{1.095181in}}%
\pgfpathcurveto{\pgfqpoint{6.328037in}{1.100225in}}{\pgfqpoint{6.321196in}{1.103059in}}{\pgfqpoint{6.314063in}{1.103059in}}%
\pgfpathcurveto{\pgfqpoint{6.306930in}{1.103059in}}{\pgfqpoint{6.300088in}{1.100225in}}{\pgfqpoint{6.295045in}{1.095181in}}%
\pgfpathcurveto{\pgfqpoint{6.290001in}{1.090137in}}{\pgfqpoint{6.287167in}{1.083296in}}{\pgfqpoint{6.287167in}{1.076163in}}%
\pgfpathcurveto{\pgfqpoint{6.287167in}{1.069030in}}{\pgfqpoint{6.290001in}{1.062188in}}{\pgfqpoint{6.295045in}{1.057145in}}%
\pgfpathcurveto{\pgfqpoint{6.300088in}{1.052101in}}{\pgfqpoint{6.306930in}{1.049267in}}{\pgfqpoint{6.314063in}{1.049267in}}%
\pgfpathclose%
\pgfusepath{stroke,fill}%
\end{pgfscope}%
\begin{pgfscope}%
\pgfpathrectangle{\pgfqpoint{4.985294in}{0.500000in}}{\pgfqpoint{1.764706in}{1.700000in}}%
\pgfusepath{clip}%
\pgfsetbuttcap%
\pgfsetroundjoin%
\definecolor{currentfill}{rgb}{0.979891,0.908948,0.848279}%
\pgfsetfillcolor{currentfill}%
\pgfsetlinewidth{0.311001pt}%
\definecolor{currentstroke}{rgb}{1.000000,1.000000,1.000000}%
\pgfsetstrokecolor{currentstroke}%
\pgfsetdash{}{0pt}%
\pgfpathmoveto{\pgfqpoint{6.290790in}{1.401181in}}%
\pgfpathcurveto{\pgfqpoint{6.297923in}{1.401181in}}{\pgfqpoint{6.304765in}{1.404015in}}{\pgfqpoint{6.309808in}{1.409059in}}%
\pgfpathcurveto{\pgfqpoint{6.314852in}{1.414102in}}{\pgfqpoint{6.317686in}{1.420944in}}{\pgfqpoint{6.317686in}{1.428077in}}%
\pgfpathcurveto{\pgfqpoint{6.317686in}{1.435210in}}{\pgfqpoint{6.314852in}{1.442051in}}{\pgfqpoint{6.309808in}{1.447095in}}%
\pgfpathcurveto{\pgfqpoint{6.304765in}{1.452139in}}{\pgfqpoint{6.297923in}{1.454973in}}{\pgfqpoint{6.290790in}{1.454973in}}%
\pgfpathcurveto{\pgfqpoint{6.283657in}{1.454973in}}{\pgfqpoint{6.276816in}{1.452139in}}{\pgfqpoint{6.271772in}{1.447095in}}%
\pgfpathcurveto{\pgfqpoint{6.266728in}{1.442051in}}{\pgfqpoint{6.263894in}{1.435210in}}{\pgfqpoint{6.263894in}{1.428077in}}%
\pgfpathcurveto{\pgfqpoint{6.263894in}{1.420944in}}{\pgfqpoint{6.266728in}{1.414102in}}{\pgfqpoint{6.271772in}{1.409059in}}%
\pgfpathcurveto{\pgfqpoint{6.276816in}{1.404015in}}{\pgfqpoint{6.283657in}{1.401181in}}{\pgfqpoint{6.290790in}{1.401181in}}%
\pgfpathclose%
\pgfusepath{stroke,fill}%
\end{pgfscope}%
\begin{pgfscope}%
\pgfpathrectangle{\pgfqpoint{4.985294in}{0.500000in}}{\pgfqpoint{1.764706in}{1.700000in}}%
\pgfusepath{clip}%
\pgfsetbuttcap%
\pgfsetroundjoin%
\definecolor{currentfill}{rgb}{0.972726,0.844889,0.754401}%
\pgfsetfillcolor{currentfill}%
\pgfsetlinewidth{0.311001pt}%
\definecolor{currentstroke}{rgb}{1.000000,1.000000,1.000000}%
\pgfsetstrokecolor{currentstroke}%
\pgfsetdash{}{0pt}%
\pgfpathmoveto{\pgfqpoint{5.507798in}{1.004848in}}%
\pgfpathcurveto{\pgfqpoint{5.514931in}{1.004848in}}{\pgfqpoint{5.521773in}{1.007682in}}{\pgfqpoint{5.526816in}{1.012726in}}%
\pgfpathcurveto{\pgfqpoint{5.531860in}{1.017770in}}{\pgfqpoint{5.534694in}{1.024611in}}{\pgfqpoint{5.534694in}{1.031744in}}%
\pgfpathcurveto{\pgfqpoint{5.534694in}{1.038877in}}{\pgfqpoint{5.531860in}{1.045718in}}{\pgfqpoint{5.526816in}{1.050762in}}%
\pgfpathcurveto{\pgfqpoint{5.521773in}{1.055806in}}{\pgfqpoint{5.514931in}{1.058640in}}{\pgfqpoint{5.507798in}{1.058640in}}%
\pgfpathcurveto{\pgfqpoint{5.500665in}{1.058640in}}{\pgfqpoint{5.493824in}{1.055806in}}{\pgfqpoint{5.488780in}{1.050762in}}%
\pgfpathcurveto{\pgfqpoint{5.483736in}{1.045718in}}{\pgfqpoint{5.480902in}{1.038877in}}{\pgfqpoint{5.480902in}{1.031744in}}%
\pgfpathcurveto{\pgfqpoint{5.480902in}{1.024611in}}{\pgfqpoint{5.483736in}{1.017770in}}{\pgfqpoint{5.488780in}{1.012726in}}%
\pgfpathcurveto{\pgfqpoint{5.493824in}{1.007682in}}{\pgfqpoint{5.500665in}{1.004848in}}{\pgfqpoint{5.507798in}{1.004848in}}%
\pgfpathclose%
\pgfusepath{stroke,fill}%
\end{pgfscope}%
\begin{pgfscope}%
\pgfpathrectangle{\pgfqpoint{4.985294in}{0.500000in}}{\pgfqpoint{1.764706in}{1.700000in}}%
\pgfusepath{clip}%
\pgfsetbuttcap%
\pgfsetroundjoin%
\definecolor{currentfill}{rgb}{0.958791,0.526283,0.368909}%
\pgfsetfillcolor{currentfill}%
\pgfsetlinewidth{0.311001pt}%
\definecolor{currentstroke}{rgb}{1.000000,1.000000,1.000000}%
\pgfsetstrokecolor{currentstroke}%
\pgfsetdash{}{0pt}%
\pgfpathmoveto{\pgfqpoint{6.118767in}{1.539785in}}%
\pgfpathcurveto{\pgfqpoint{6.125900in}{1.539785in}}{\pgfqpoint{6.132742in}{1.542619in}}{\pgfqpoint{6.137785in}{1.547663in}}%
\pgfpathcurveto{\pgfqpoint{6.142829in}{1.552707in}}{\pgfqpoint{6.145663in}{1.559548in}}{\pgfqpoint{6.145663in}{1.566681in}}%
\pgfpathcurveto{\pgfqpoint{6.145663in}{1.573814in}}{\pgfqpoint{6.142829in}{1.580656in}}{\pgfqpoint{6.137785in}{1.585699in}}%
\pgfpathcurveto{\pgfqpoint{6.132742in}{1.590743in}}{\pgfqpoint{6.125900in}{1.593577in}}{\pgfqpoint{6.118767in}{1.593577in}}%
\pgfpathcurveto{\pgfqpoint{6.111634in}{1.593577in}}{\pgfqpoint{6.104793in}{1.590743in}}{\pgfqpoint{6.099749in}{1.585699in}}%
\pgfpathcurveto{\pgfqpoint{6.094705in}{1.580656in}}{\pgfqpoint{6.091871in}{1.573814in}}{\pgfqpoint{6.091871in}{1.566681in}}%
\pgfpathcurveto{\pgfqpoint{6.091871in}{1.559548in}}{\pgfqpoint{6.094705in}{1.552707in}}{\pgfqpoint{6.099749in}{1.547663in}}%
\pgfpathcurveto{\pgfqpoint{6.104793in}{1.542619in}}{\pgfqpoint{6.111634in}{1.539785in}}{\pgfqpoint{6.118767in}{1.539785in}}%
\pgfpathclose%
\pgfusepath{stroke,fill}%
\end{pgfscope}%
\begin{pgfscope}%
\pgfpathrectangle{\pgfqpoint{4.985294in}{0.500000in}}{\pgfqpoint{1.764706in}{1.700000in}}%
\pgfusepath{clip}%
\pgfsetbuttcap%
\pgfsetroundjoin%
\definecolor{currentfill}{rgb}{0.934351,0.329284,0.247753}%
\pgfsetfillcolor{currentfill}%
\pgfsetlinewidth{0.311001pt}%
\definecolor{currentstroke}{rgb}{1.000000,1.000000,1.000000}%
\pgfsetstrokecolor{currentstroke}%
\pgfsetdash{}{0pt}%
\pgfpathmoveto{\pgfqpoint{6.414902in}{1.050282in}}%
\pgfpathcurveto{\pgfqpoint{6.422035in}{1.050282in}}{\pgfqpoint{6.428876in}{1.053116in}}{\pgfqpoint{6.433920in}{1.058159in}}%
\pgfpathcurveto{\pgfqpoint{6.438964in}{1.063203in}}{\pgfqpoint{6.441798in}{1.070045in}}{\pgfqpoint{6.441798in}{1.077177in}}%
\pgfpathcurveto{\pgfqpoint{6.441798in}{1.084310in}}{\pgfqpoint{6.438964in}{1.091152in}}{\pgfqpoint{6.433920in}{1.096196in}}%
\pgfpathcurveto{\pgfqpoint{6.428876in}{1.101239in}}{\pgfqpoint{6.422035in}{1.104073in}}{\pgfqpoint{6.414902in}{1.104073in}}%
\pgfpathcurveto{\pgfqpoint{6.407769in}{1.104073in}}{\pgfqpoint{6.400928in}{1.101239in}}{\pgfqpoint{6.395884in}{1.096196in}}%
\pgfpathcurveto{\pgfqpoint{6.390840in}{1.091152in}}{\pgfqpoint{6.388006in}{1.084310in}}{\pgfqpoint{6.388006in}{1.077177in}}%
\pgfpathcurveto{\pgfqpoint{6.388006in}{1.070045in}}{\pgfqpoint{6.390840in}{1.063203in}}{\pgfqpoint{6.395884in}{1.058159in}}%
\pgfpathcurveto{\pgfqpoint{6.400928in}{1.053116in}}{\pgfqpoint{6.407769in}{1.050282in}}{\pgfqpoint{6.414902in}{1.050282in}}%
\pgfpathclose%
\pgfusepath{stroke,fill}%
\end{pgfscope}%
\begin{pgfscope}%
\pgfpathrectangle{\pgfqpoint{4.985294in}{0.500000in}}{\pgfqpoint{1.764706in}{1.700000in}}%
\pgfusepath{clip}%
\pgfsetbuttcap%
\pgfsetroundjoin%
\definecolor{currentfill}{rgb}{0.967092,0.768560,0.642079}%
\pgfsetfillcolor{currentfill}%
\pgfsetlinewidth{0.311001pt}%
\definecolor{currentstroke}{rgb}{1.000000,1.000000,1.000000}%
\pgfsetstrokecolor{currentstroke}%
\pgfsetdash{}{0pt}%
\pgfpathmoveto{\pgfqpoint{6.316093in}{1.042702in}}%
\pgfpathcurveto{\pgfqpoint{6.323226in}{1.042702in}}{\pgfqpoint{6.330067in}{1.045536in}}{\pgfqpoint{6.335111in}{1.050580in}}%
\pgfpathcurveto{\pgfqpoint{6.340155in}{1.055623in}}{\pgfqpoint{6.342989in}{1.062465in}}{\pgfqpoint{6.342989in}{1.069598in}}%
\pgfpathcurveto{\pgfqpoint{6.342989in}{1.076731in}}{\pgfqpoint{6.340155in}{1.083572in}}{\pgfqpoint{6.335111in}{1.088616in}}%
\pgfpathcurveto{\pgfqpoint{6.330067in}{1.093660in}}{\pgfqpoint{6.323226in}{1.096494in}}{\pgfqpoint{6.316093in}{1.096494in}}%
\pgfpathcurveto{\pgfqpoint{6.308960in}{1.096494in}}{\pgfqpoint{6.302119in}{1.093660in}}{\pgfqpoint{6.297075in}{1.088616in}}%
\pgfpathcurveto{\pgfqpoint{6.292031in}{1.083572in}}{\pgfqpoint{6.289197in}{1.076731in}}{\pgfqpoint{6.289197in}{1.069598in}}%
\pgfpathcurveto{\pgfqpoint{6.289197in}{1.062465in}}{\pgfqpoint{6.292031in}{1.055623in}}{\pgfqpoint{6.297075in}{1.050580in}}%
\pgfpathcurveto{\pgfqpoint{6.302119in}{1.045536in}}{\pgfqpoint{6.308960in}{1.042702in}}{\pgfqpoint{6.316093in}{1.042702in}}%
\pgfpathclose%
\pgfusepath{stroke,fill}%
\end{pgfscope}%
\begin{pgfscope}%
\pgfpathrectangle{\pgfqpoint{4.985294in}{0.500000in}}{\pgfqpoint{1.764706in}{1.700000in}}%
\pgfusepath{clip}%
\pgfsetbuttcap%
\pgfsetroundjoin%
\definecolor{currentfill}{rgb}{0.969359,0.803954,0.693832}%
\pgfsetfillcolor{currentfill}%
\pgfsetlinewidth{0.311001pt}%
\definecolor{currentstroke}{rgb}{1.000000,1.000000,1.000000}%
\pgfsetstrokecolor{currentstroke}%
\pgfsetdash{}{0pt}%
\pgfpathmoveto{\pgfqpoint{6.226944in}{1.269337in}}%
\pgfpathcurveto{\pgfqpoint{6.234077in}{1.269337in}}{\pgfqpoint{6.240918in}{1.272171in}}{\pgfqpoint{6.245962in}{1.277215in}}%
\pgfpathcurveto{\pgfqpoint{6.251006in}{1.282259in}}{\pgfqpoint{6.253839in}{1.289100in}}{\pgfqpoint{6.253839in}{1.296233in}}%
\pgfpathcurveto{\pgfqpoint{6.253839in}{1.303366in}}{\pgfqpoint{6.251006in}{1.310208in}}{\pgfqpoint{6.245962in}{1.315251in}}%
\pgfpathcurveto{\pgfqpoint{6.240918in}{1.320295in}}{\pgfqpoint{6.234077in}{1.323129in}}{\pgfqpoint{6.226944in}{1.323129in}}%
\pgfpathcurveto{\pgfqpoint{6.219811in}{1.323129in}}{\pgfqpoint{6.212969in}{1.320295in}}{\pgfqpoint{6.207926in}{1.315251in}}%
\pgfpathcurveto{\pgfqpoint{6.202882in}{1.310208in}}{\pgfqpoint{6.200048in}{1.303366in}}{\pgfqpoint{6.200048in}{1.296233in}}%
\pgfpathcurveto{\pgfqpoint{6.200048in}{1.289100in}}{\pgfqpoint{6.202882in}{1.282259in}}{\pgfqpoint{6.207926in}{1.277215in}}%
\pgfpathcurveto{\pgfqpoint{6.212969in}{1.272171in}}{\pgfqpoint{6.219811in}{1.269337in}}{\pgfqpoint{6.226944in}{1.269337in}}%
\pgfpathclose%
\pgfusepath{stroke,fill}%
\end{pgfscope}%
\begin{pgfscope}%
\pgfpathrectangle{\pgfqpoint{4.985294in}{0.500000in}}{\pgfqpoint{1.764706in}{1.700000in}}%
\pgfusepath{clip}%
\pgfsetbuttcap%
\pgfsetroundjoin%
\definecolor{currentfill}{rgb}{0.977657,0.891500,0.822809}%
\pgfsetfillcolor{currentfill}%
\pgfsetlinewidth{0.311001pt}%
\definecolor{currentstroke}{rgb}{1.000000,1.000000,1.000000}%
\pgfsetstrokecolor{currentstroke}%
\pgfsetdash{}{0pt}%
\pgfpathmoveto{\pgfqpoint{5.456479in}{1.466347in}}%
\pgfpathcurveto{\pgfqpoint{5.463612in}{1.466347in}}{\pgfqpoint{5.470454in}{1.469181in}}{\pgfqpoint{5.475498in}{1.474225in}}%
\pgfpathcurveto{\pgfqpoint{5.480541in}{1.479269in}}{\pgfqpoint{5.483375in}{1.486110in}}{\pgfqpoint{5.483375in}{1.493243in}}%
\pgfpathcurveto{\pgfqpoint{5.483375in}{1.500376in}}{\pgfqpoint{5.480541in}{1.507218in}}{\pgfqpoint{5.475498in}{1.512261in}}%
\pgfpathcurveto{\pgfqpoint{5.470454in}{1.517305in}}{\pgfqpoint{5.463612in}{1.520139in}}{\pgfqpoint{5.456479in}{1.520139in}}%
\pgfpathcurveto{\pgfqpoint{5.449347in}{1.520139in}}{\pgfqpoint{5.442505in}{1.517305in}}{\pgfqpoint{5.437461in}{1.512261in}}%
\pgfpathcurveto{\pgfqpoint{5.432418in}{1.507218in}}{\pgfqpoint{5.429584in}{1.500376in}}{\pgfqpoint{5.429584in}{1.493243in}}%
\pgfpathcurveto{\pgfqpoint{5.429584in}{1.486110in}}{\pgfqpoint{5.432418in}{1.479269in}}{\pgfqpoint{5.437461in}{1.474225in}}%
\pgfpathcurveto{\pgfqpoint{5.442505in}{1.469181in}}{\pgfqpoint{5.449347in}{1.466347in}}{\pgfqpoint{5.456479in}{1.466347in}}%
\pgfpathclose%
\pgfusepath{stroke,fill}%
\end{pgfscope}%
\begin{pgfscope}%
\pgfpathrectangle{\pgfqpoint{4.985294in}{0.500000in}}{\pgfqpoint{1.764706in}{1.700000in}}%
\pgfusepath{clip}%
\pgfsetbuttcap%
\pgfsetroundjoin%
\definecolor{currentfill}{rgb}{0.974412,0.862387,0.780156}%
\pgfsetfillcolor{currentfill}%
\pgfsetlinewidth{0.311001pt}%
\definecolor{currentstroke}{rgb}{1.000000,1.000000,1.000000}%
\pgfsetstrokecolor{currentstroke}%
\pgfsetdash{}{0pt}%
\pgfpathmoveto{\pgfqpoint{6.263959in}{1.317809in}}%
\pgfpathcurveto{\pgfqpoint{6.271091in}{1.317809in}}{\pgfqpoint{6.277933in}{1.320643in}}{\pgfqpoint{6.282977in}{1.325686in}}%
\pgfpathcurveto{\pgfqpoint{6.288020in}{1.330730in}}{\pgfqpoint{6.290854in}{1.337572in}}{\pgfqpoint{6.290854in}{1.344704in}}%
\pgfpathcurveto{\pgfqpoint{6.290854in}{1.351837in}}{\pgfqpoint{6.288020in}{1.358679in}}{\pgfqpoint{6.282977in}{1.363723in}}%
\pgfpathcurveto{\pgfqpoint{6.277933in}{1.368766in}}{\pgfqpoint{6.271091in}{1.371600in}}{\pgfqpoint{6.263959in}{1.371600in}}%
\pgfpathcurveto{\pgfqpoint{6.256826in}{1.371600in}}{\pgfqpoint{6.249984in}{1.368766in}}{\pgfqpoint{6.244940in}{1.363723in}}%
\pgfpathcurveto{\pgfqpoint{6.239897in}{1.358679in}}{\pgfqpoint{6.237063in}{1.351837in}}{\pgfqpoint{6.237063in}{1.344704in}}%
\pgfpathcurveto{\pgfqpoint{6.237063in}{1.337572in}}{\pgfqpoint{6.239897in}{1.330730in}}{\pgfqpoint{6.244940in}{1.325686in}}%
\pgfpathcurveto{\pgfqpoint{6.249984in}{1.320643in}}{\pgfqpoint{6.256826in}{1.317809in}}{\pgfqpoint{6.263959in}{1.317809in}}%
\pgfpathclose%
\pgfusepath{stroke,fill}%
\end{pgfscope}%
\begin{pgfscope}%
\pgfpathrectangle{\pgfqpoint{4.985294in}{0.500000in}}{\pgfqpoint{1.764706in}{1.700000in}}%
\pgfusepath{clip}%
\pgfsetbuttcap%
\pgfsetroundjoin%
\definecolor{currentfill}{rgb}{0.964679,0.682838,0.530002}%
\pgfsetfillcolor{currentfill}%
\pgfsetlinewidth{0.311001pt}%
\definecolor{currentstroke}{rgb}{1.000000,1.000000,1.000000}%
\pgfsetstrokecolor{currentstroke}%
\pgfsetdash{}{0pt}%
\pgfpathmoveto{\pgfqpoint{6.116914in}{1.620023in}}%
\pgfpathcurveto{\pgfqpoint{6.124047in}{1.620023in}}{\pgfqpoint{6.130889in}{1.622857in}}{\pgfqpoint{6.135933in}{1.627900in}}%
\pgfpathcurveto{\pgfqpoint{6.140976in}{1.632944in}}{\pgfqpoint{6.143810in}{1.639786in}}{\pgfqpoint{6.143810in}{1.646918in}}%
\pgfpathcurveto{\pgfqpoint{6.143810in}{1.654051in}}{\pgfqpoint{6.140976in}{1.660893in}}{\pgfqpoint{6.135933in}{1.665937in}}%
\pgfpathcurveto{\pgfqpoint{6.130889in}{1.670980in}}{\pgfqpoint{6.124047in}{1.673814in}}{\pgfqpoint{6.116914in}{1.673814in}}%
\pgfpathcurveto{\pgfqpoint{6.109782in}{1.673814in}}{\pgfqpoint{6.102940in}{1.670980in}}{\pgfqpoint{6.097896in}{1.665937in}}%
\pgfpathcurveto{\pgfqpoint{6.092853in}{1.660893in}}{\pgfqpoint{6.090019in}{1.654051in}}{\pgfqpoint{6.090019in}{1.646918in}}%
\pgfpathcurveto{\pgfqpoint{6.090019in}{1.639786in}}{\pgfqpoint{6.092853in}{1.632944in}}{\pgfqpoint{6.097896in}{1.627900in}}%
\pgfpathcurveto{\pgfqpoint{6.102940in}{1.622857in}}{\pgfqpoint{6.109782in}{1.620023in}}{\pgfqpoint{6.116914in}{1.620023in}}%
\pgfpathclose%
\pgfusepath{stroke,fill}%
\end{pgfscope}%
\begin{pgfscope}%
\pgfpathrectangle{\pgfqpoint{4.985294in}{0.500000in}}{\pgfqpoint{1.764706in}{1.700000in}}%
\pgfusepath{clip}%
\pgfsetbuttcap%
\pgfsetroundjoin%
\definecolor{currentfill}{rgb}{0.968105,0.786346,0.667739}%
\pgfsetfillcolor{currentfill}%
\pgfsetlinewidth{0.311001pt}%
\definecolor{currentstroke}{rgb}{1.000000,1.000000,1.000000}%
\pgfsetstrokecolor{currentstroke}%
\pgfsetdash{}{0pt}%
\pgfpathmoveto{\pgfqpoint{5.516325in}{1.431086in}}%
\pgfpathcurveto{\pgfqpoint{5.523458in}{1.431086in}}{\pgfqpoint{5.530300in}{1.433920in}}{\pgfqpoint{5.535343in}{1.438964in}}%
\pgfpathcurveto{\pgfqpoint{5.540387in}{1.444008in}}{\pgfqpoint{5.543221in}{1.450849in}}{\pgfqpoint{5.543221in}{1.457982in}}%
\pgfpathcurveto{\pgfqpoint{5.543221in}{1.465115in}}{\pgfqpoint{5.540387in}{1.471956in}}{\pgfqpoint{5.535343in}{1.477000in}}%
\pgfpathcurveto{\pgfqpoint{5.530300in}{1.482044in}}{\pgfqpoint{5.523458in}{1.484878in}}{\pgfqpoint{5.516325in}{1.484878in}}%
\pgfpathcurveto{\pgfqpoint{5.509192in}{1.484878in}}{\pgfqpoint{5.502351in}{1.482044in}}{\pgfqpoint{5.497307in}{1.477000in}}%
\pgfpathcurveto{\pgfqpoint{5.492263in}{1.471956in}}{\pgfqpoint{5.489430in}{1.465115in}}{\pgfqpoint{5.489430in}{1.457982in}}%
\pgfpathcurveto{\pgfqpoint{5.489430in}{1.450849in}}{\pgfqpoint{5.492263in}{1.444008in}}{\pgfqpoint{5.497307in}{1.438964in}}%
\pgfpathcurveto{\pgfqpoint{5.502351in}{1.433920in}}{\pgfqpoint{5.509192in}{1.431086in}}{\pgfqpoint{5.516325in}{1.431086in}}%
\pgfpathclose%
\pgfusepath{stroke,fill}%
\end{pgfscope}%
\begin{pgfscope}%
\pgfpathrectangle{\pgfqpoint{4.985294in}{0.500000in}}{\pgfqpoint{1.764706in}{1.700000in}}%
\pgfusepath{clip}%
\pgfsetbuttcap%
\pgfsetroundjoin%
\definecolor{currentfill}{rgb}{0.973832,0.856556,0.771584}%
\pgfsetfillcolor{currentfill}%
\pgfsetlinewidth{0.311001pt}%
\definecolor{currentstroke}{rgb}{1.000000,1.000000,1.000000}%
\pgfsetstrokecolor{currentstroke}%
\pgfsetdash{}{0pt}%
\pgfpathmoveto{\pgfqpoint{5.475474in}{1.222961in}}%
\pgfpathcurveto{\pgfqpoint{5.482607in}{1.222961in}}{\pgfqpoint{5.489448in}{1.225795in}}{\pgfqpoint{5.494492in}{1.230839in}}%
\pgfpathcurveto{\pgfqpoint{5.499536in}{1.235883in}}{\pgfqpoint{5.502370in}{1.242724in}}{\pgfqpoint{5.502370in}{1.249857in}}%
\pgfpathcurveto{\pgfqpoint{5.502370in}{1.256990in}}{\pgfqpoint{5.499536in}{1.263832in}}{\pgfqpoint{5.494492in}{1.268875in}}%
\pgfpathcurveto{\pgfqpoint{5.489448in}{1.273919in}}{\pgfqpoint{5.482607in}{1.276753in}}{\pgfqpoint{5.475474in}{1.276753in}}%
\pgfpathcurveto{\pgfqpoint{5.468341in}{1.276753in}}{\pgfqpoint{5.461499in}{1.273919in}}{\pgfqpoint{5.456456in}{1.268875in}}%
\pgfpathcurveto{\pgfqpoint{5.451412in}{1.263832in}}{\pgfqpoint{5.448578in}{1.256990in}}{\pgfqpoint{5.448578in}{1.249857in}}%
\pgfpathcurveto{\pgfqpoint{5.448578in}{1.242724in}}{\pgfqpoint{5.451412in}{1.235883in}}{\pgfqpoint{5.456456in}{1.230839in}}%
\pgfpathcurveto{\pgfqpoint{5.461499in}{1.225795in}}{\pgfqpoint{5.468341in}{1.222961in}}{\pgfqpoint{5.475474in}{1.222961in}}%
\pgfpathclose%
\pgfusepath{stroke,fill}%
\end{pgfscope}%
\begin{pgfscope}%
\pgfpathrectangle{\pgfqpoint{4.985294in}{0.500000in}}{\pgfqpoint{1.764706in}{1.700000in}}%
\pgfusepath{clip}%
\pgfsetbuttcap%
\pgfsetroundjoin%
\definecolor{currentfill}{rgb}{0.818205,0.120806,0.299261}%
\pgfsetfillcolor{currentfill}%
\pgfsetlinewidth{0.311001pt}%
\definecolor{currentstroke}{rgb}{1.000000,1.000000,1.000000}%
\pgfsetstrokecolor{currentstroke}%
\pgfsetdash{}{0pt}%
\pgfpathmoveto{\pgfqpoint{6.471833in}{1.497512in}}%
\pgfpathcurveto{\pgfqpoint{6.478966in}{1.497512in}}{\pgfqpoint{6.485808in}{1.500346in}}{\pgfqpoint{6.490852in}{1.505389in}}%
\pgfpathcurveto{\pgfqpoint{6.495895in}{1.510433in}}{\pgfqpoint{6.498729in}{1.517275in}}{\pgfqpoint{6.498729in}{1.524408in}}%
\pgfpathcurveto{\pgfqpoint{6.498729in}{1.531540in}}{\pgfqpoint{6.495895in}{1.538382in}}{\pgfqpoint{6.490852in}{1.543426in}}%
\pgfpathcurveto{\pgfqpoint{6.485808in}{1.548469in}}{\pgfqpoint{6.478966in}{1.551303in}}{\pgfqpoint{6.471833in}{1.551303in}}%
\pgfpathcurveto{\pgfqpoint{6.464701in}{1.551303in}}{\pgfqpoint{6.457859in}{1.548469in}}{\pgfqpoint{6.452815in}{1.543426in}}%
\pgfpathcurveto{\pgfqpoint{6.447772in}{1.538382in}}{\pgfqpoint{6.444938in}{1.531540in}}{\pgfqpoint{6.444938in}{1.524408in}}%
\pgfpathcurveto{\pgfqpoint{6.444938in}{1.517275in}}{\pgfqpoint{6.447772in}{1.510433in}}{\pgfqpoint{6.452815in}{1.505389in}}%
\pgfpathcurveto{\pgfqpoint{6.457859in}{1.500346in}}{\pgfqpoint{6.464701in}{1.497512in}}{\pgfqpoint{6.471833in}{1.497512in}}%
\pgfpathclose%
\pgfusepath{stroke,fill}%
\end{pgfscope}%
\begin{pgfscope}%
\pgfpathrectangle{\pgfqpoint{4.985294in}{0.500000in}}{\pgfqpoint{1.764706in}{1.700000in}}%
\pgfusepath{clip}%
\pgfsetbuttcap%
\pgfsetroundjoin%
\definecolor{currentfill}{rgb}{0.967092,0.768560,0.642079}%
\pgfsetfillcolor{currentfill}%
\pgfsetlinewidth{0.311001pt}%
\definecolor{currentstroke}{rgb}{1.000000,1.000000,1.000000}%
\pgfsetstrokecolor{currentstroke}%
\pgfsetdash{}{0pt}%
\pgfpathmoveto{\pgfqpoint{6.396387in}{1.340795in}}%
\pgfpathcurveto{\pgfqpoint{6.403520in}{1.340795in}}{\pgfqpoint{6.410362in}{1.343629in}}{\pgfqpoint{6.415405in}{1.348672in}}%
\pgfpathcurveto{\pgfqpoint{6.420449in}{1.353716in}}{\pgfqpoint{6.423283in}{1.360558in}}{\pgfqpoint{6.423283in}{1.367691in}}%
\pgfpathcurveto{\pgfqpoint{6.423283in}{1.374823in}}{\pgfqpoint{6.420449in}{1.381665in}}{\pgfqpoint{6.415405in}{1.386709in}}%
\pgfpathcurveto{\pgfqpoint{6.410362in}{1.391752in}}{\pgfqpoint{6.403520in}{1.394586in}}{\pgfqpoint{6.396387in}{1.394586in}}%
\pgfpathcurveto{\pgfqpoint{6.389254in}{1.394586in}}{\pgfqpoint{6.382413in}{1.391752in}}{\pgfqpoint{6.377369in}{1.386709in}}%
\pgfpathcurveto{\pgfqpoint{6.372325in}{1.381665in}}{\pgfqpoint{6.369492in}{1.374823in}}{\pgfqpoint{6.369492in}{1.367691in}}%
\pgfpathcurveto{\pgfqpoint{6.369492in}{1.360558in}}{\pgfqpoint{6.372325in}{1.353716in}}{\pgfqpoint{6.377369in}{1.348672in}}%
\pgfpathcurveto{\pgfqpoint{6.382413in}{1.343629in}}{\pgfqpoint{6.389254in}{1.340795in}}{\pgfqpoint{6.396387in}{1.340795in}}%
\pgfpathclose%
\pgfusepath{stroke,fill}%
\end{pgfscope}%
\begin{pgfscope}%
\pgfpathrectangle{\pgfqpoint{4.985294in}{0.500000in}}{\pgfqpoint{1.764706in}{1.700000in}}%
\pgfusepath{clip}%
\pgfsetbuttcap%
\pgfsetroundjoin%
\definecolor{currentfill}{rgb}{0.966328,0.750560,0.616961}%
\pgfsetfillcolor{currentfill}%
\pgfsetlinewidth{0.311001pt}%
\definecolor{currentstroke}{rgb}{1.000000,1.000000,1.000000}%
\pgfsetstrokecolor{currentstroke}%
\pgfsetdash{}{0pt}%
\pgfpathmoveto{\pgfqpoint{5.592109in}{0.981275in}}%
\pgfpathcurveto{\pgfqpoint{5.599242in}{0.981275in}}{\pgfqpoint{5.606083in}{0.984109in}}{\pgfqpoint{5.611127in}{0.989153in}}%
\pgfpathcurveto{\pgfqpoint{5.616171in}{0.994196in}}{\pgfqpoint{5.619005in}{1.001038in}}{\pgfqpoint{5.619005in}{1.008171in}}%
\pgfpathcurveto{\pgfqpoint{5.619005in}{1.015304in}}{\pgfqpoint{5.616171in}{1.022145in}}{\pgfqpoint{5.611127in}{1.027189in}}%
\pgfpathcurveto{\pgfqpoint{5.606083in}{1.032233in}}{\pgfqpoint{5.599242in}{1.035066in}}{\pgfqpoint{5.592109in}{1.035066in}}%
\pgfpathcurveto{\pgfqpoint{5.584976in}{1.035066in}}{\pgfqpoint{5.578134in}{1.032233in}}{\pgfqpoint{5.573091in}{1.027189in}}%
\pgfpathcurveto{\pgfqpoint{5.568047in}{1.022145in}}{\pgfqpoint{5.565213in}{1.015304in}}{\pgfqpoint{5.565213in}{1.008171in}}%
\pgfpathcurveto{\pgfqpoint{5.565213in}{1.001038in}}{\pgfqpoint{5.568047in}{0.994196in}}{\pgfqpoint{5.573091in}{0.989153in}}%
\pgfpathcurveto{\pgfqpoint{5.578134in}{0.984109in}}{\pgfqpoint{5.584976in}{0.981275in}}{\pgfqpoint{5.592109in}{0.981275in}}%
\pgfpathclose%
\pgfusepath{stroke,fill}%
\end{pgfscope}%
\begin{pgfscope}%
\pgfpathrectangle{\pgfqpoint{4.985294in}{0.500000in}}{\pgfqpoint{1.764706in}{1.700000in}}%
\pgfusepath{clip}%
\pgfsetbuttcap%
\pgfsetroundjoin%
\definecolor{currentfill}{rgb}{0.968931,0.798091,0.685123}%
\pgfsetfillcolor{currentfill}%
\pgfsetlinewidth{0.311001pt}%
\definecolor{currentstroke}{rgb}{1.000000,1.000000,1.000000}%
\pgfsetstrokecolor{currentstroke}%
\pgfsetdash{}{0pt}%
\pgfpathmoveto{\pgfqpoint{5.498159in}{1.242169in}}%
\pgfpathcurveto{\pgfqpoint{5.505292in}{1.242169in}}{\pgfqpoint{5.512133in}{1.245002in}}{\pgfqpoint{5.517177in}{1.250046in}}%
\pgfpathcurveto{\pgfqpoint{5.522221in}{1.255090in}}{\pgfqpoint{5.525054in}{1.261931in}}{\pgfqpoint{5.525054in}{1.269064in}}%
\pgfpathcurveto{\pgfqpoint{5.525054in}{1.276197in}}{\pgfqpoint{5.522221in}{1.283039in}}{\pgfqpoint{5.517177in}{1.288082in}}%
\pgfpathcurveto{\pgfqpoint{5.512133in}{1.293126in}}{\pgfqpoint{5.505292in}{1.295960in}}{\pgfqpoint{5.498159in}{1.295960in}}%
\pgfpathcurveto{\pgfqpoint{5.491026in}{1.295960in}}{\pgfqpoint{5.484184in}{1.293126in}}{\pgfqpoint{5.479141in}{1.288082in}}%
\pgfpathcurveto{\pgfqpoint{5.474097in}{1.283039in}}{\pgfqpoint{5.471263in}{1.276197in}}{\pgfqpoint{5.471263in}{1.269064in}}%
\pgfpathcurveto{\pgfqpoint{5.471263in}{1.261931in}}{\pgfqpoint{5.474097in}{1.255090in}}{\pgfqpoint{5.479141in}{1.250046in}}%
\pgfpathcurveto{\pgfqpoint{5.484184in}{1.245002in}}{\pgfqpoint{5.491026in}{1.242169in}}{\pgfqpoint{5.498159in}{1.242169in}}%
\pgfpathclose%
\pgfusepath{stroke,fill}%
\end{pgfscope}%
\begin{pgfscope}%
\pgfpathrectangle{\pgfqpoint{4.985294in}{0.500000in}}{\pgfqpoint{1.764706in}{1.700000in}}%
\pgfusepath{clip}%
\pgfsetbuttcap%
\pgfsetroundjoin%
\definecolor{currentfill}{rgb}{0.964173,0.657587,0.500469}%
\pgfsetfillcolor{currentfill}%
\pgfsetlinewidth{0.311001pt}%
\definecolor{currentstroke}{rgb}{1.000000,1.000000,1.000000}%
\pgfsetstrokecolor{currentstroke}%
\pgfsetdash{}{0pt}%
\pgfpathmoveto{\pgfqpoint{6.204292in}{1.737640in}}%
\pgfpathcurveto{\pgfqpoint{6.211425in}{1.737640in}}{\pgfqpoint{6.218266in}{1.740474in}}{\pgfqpoint{6.223310in}{1.745518in}}%
\pgfpathcurveto{\pgfqpoint{6.228354in}{1.750562in}}{\pgfqpoint{6.231188in}{1.757403in}}{\pgfqpoint{6.231188in}{1.764536in}}%
\pgfpathcurveto{\pgfqpoint{6.231188in}{1.771669in}}{\pgfqpoint{6.228354in}{1.778511in}}{\pgfqpoint{6.223310in}{1.783554in}}%
\pgfpathcurveto{\pgfqpoint{6.218266in}{1.788598in}}{\pgfqpoint{6.211425in}{1.791432in}}{\pgfqpoint{6.204292in}{1.791432in}}%
\pgfpathcurveto{\pgfqpoint{6.197159in}{1.791432in}}{\pgfqpoint{6.190317in}{1.788598in}}{\pgfqpoint{6.185274in}{1.783554in}}%
\pgfpathcurveto{\pgfqpoint{6.180230in}{1.778511in}}{\pgfqpoint{6.177396in}{1.771669in}}{\pgfqpoint{6.177396in}{1.764536in}}%
\pgfpathcurveto{\pgfqpoint{6.177396in}{1.757403in}}{\pgfqpoint{6.180230in}{1.750562in}}{\pgfqpoint{6.185274in}{1.745518in}}%
\pgfpathcurveto{\pgfqpoint{6.190317in}{1.740474in}}{\pgfqpoint{6.197159in}{1.737640in}}{\pgfqpoint{6.204292in}{1.737640in}}%
\pgfpathclose%
\pgfusepath{stroke,fill}%
\end{pgfscope}%
\begin{pgfscope}%
\pgfpathrectangle{\pgfqpoint{4.985294in}{0.500000in}}{\pgfqpoint{1.764706in}{1.700000in}}%
\pgfusepath{clip}%
\pgfsetbuttcap%
\pgfsetroundjoin%
\definecolor{currentfill}{rgb}{0.979124,0.903132,0.839793}%
\pgfsetfillcolor{currentfill}%
\pgfsetlinewidth{0.311001pt}%
\definecolor{currentstroke}{rgb}{1.000000,1.000000,1.000000}%
\pgfsetstrokecolor{currentstroke}%
\pgfsetdash{}{0pt}%
\pgfpathmoveto{\pgfqpoint{6.289132in}{1.403852in}}%
\pgfpathcurveto{\pgfqpoint{6.296265in}{1.403852in}}{\pgfqpoint{6.303106in}{1.406686in}}{\pgfqpoint{6.308150in}{1.411730in}}%
\pgfpathcurveto{\pgfqpoint{6.313194in}{1.416773in}}{\pgfqpoint{6.316028in}{1.423615in}}{\pgfqpoint{6.316028in}{1.430748in}}%
\pgfpathcurveto{\pgfqpoint{6.316028in}{1.437881in}}{\pgfqpoint{6.313194in}{1.444722in}}{\pgfqpoint{6.308150in}{1.449766in}}%
\pgfpathcurveto{\pgfqpoint{6.303106in}{1.454810in}}{\pgfqpoint{6.296265in}{1.457644in}}{\pgfqpoint{6.289132in}{1.457644in}}%
\pgfpathcurveto{\pgfqpoint{6.281999in}{1.457644in}}{\pgfqpoint{6.275157in}{1.454810in}}{\pgfqpoint{6.270114in}{1.449766in}}%
\pgfpathcurveto{\pgfqpoint{6.265070in}{1.444722in}}{\pgfqpoint{6.262236in}{1.437881in}}{\pgfqpoint{6.262236in}{1.430748in}}%
\pgfpathcurveto{\pgfqpoint{6.262236in}{1.423615in}}{\pgfqpoint{6.265070in}{1.416773in}}{\pgfqpoint{6.270114in}{1.411730in}}%
\pgfpathcurveto{\pgfqpoint{6.275157in}{1.406686in}}{\pgfqpoint{6.281999in}{1.403852in}}{\pgfqpoint{6.289132in}{1.403852in}}%
\pgfpathclose%
\pgfusepath{stroke,fill}%
\end{pgfscope}%
\begin{pgfscope}%
\pgfpathrectangle{\pgfqpoint{4.985294in}{0.500000in}}{\pgfqpoint{1.764706in}{1.700000in}}%
\pgfusepath{clip}%
\pgfsetbuttcap%
\pgfsetroundjoin%
\definecolor{currentfill}{rgb}{0.976961,0.885681,0.814303}%
\pgfsetfillcolor{currentfill}%
\pgfsetlinewidth{0.311001pt}%
\definecolor{currentstroke}{rgb}{1.000000,1.000000,1.000000}%
\pgfsetstrokecolor{currentstroke}%
\pgfsetdash{}{0pt}%
\pgfpathmoveto{\pgfqpoint{5.400770in}{1.203796in}}%
\pgfpathcurveto{\pgfqpoint{5.407903in}{1.203796in}}{\pgfqpoint{5.414745in}{1.206630in}}{\pgfqpoint{5.419788in}{1.211674in}}%
\pgfpathcurveto{\pgfqpoint{5.424832in}{1.216717in}}{\pgfqpoint{5.427666in}{1.223559in}}{\pgfqpoint{5.427666in}{1.230692in}}%
\pgfpathcurveto{\pgfqpoint{5.427666in}{1.237825in}}{\pgfqpoint{5.424832in}{1.244666in}}{\pgfqpoint{5.419788in}{1.249710in}}%
\pgfpathcurveto{\pgfqpoint{5.414745in}{1.254754in}}{\pgfqpoint{5.407903in}{1.257588in}}{\pgfqpoint{5.400770in}{1.257588in}}%
\pgfpathcurveto{\pgfqpoint{5.393638in}{1.257588in}}{\pgfqpoint{5.386796in}{1.254754in}}{\pgfqpoint{5.381752in}{1.249710in}}%
\pgfpathcurveto{\pgfqpoint{5.376709in}{1.244666in}}{\pgfqpoint{5.373875in}{1.237825in}}{\pgfqpoint{5.373875in}{1.230692in}}%
\pgfpathcurveto{\pgfqpoint{5.373875in}{1.223559in}}{\pgfqpoint{5.376709in}{1.216717in}}{\pgfqpoint{5.381752in}{1.211674in}}%
\pgfpathcurveto{\pgfqpoint{5.386796in}{1.206630in}}{\pgfqpoint{5.393638in}{1.203796in}}{\pgfqpoint{5.400770in}{1.203796in}}%
\pgfpathclose%
\pgfusepath{stroke,fill}%
\end{pgfscope}%
\begin{pgfscope}%
\pgfpathrectangle{\pgfqpoint{4.985294in}{0.500000in}}{\pgfqpoint{1.764706in}{1.700000in}}%
\pgfusepath{clip}%
\pgfsetbuttcap%
\pgfsetroundjoin%
\definecolor{currentfill}{rgb}{0.968931,0.798091,0.685123}%
\pgfsetfillcolor{currentfill}%
\pgfsetlinewidth{0.311001pt}%
\definecolor{currentstroke}{rgb}{1.000000,1.000000,1.000000}%
\pgfsetstrokecolor{currentstroke}%
\pgfsetdash{}{0pt}%
\pgfpathmoveto{\pgfqpoint{5.346703in}{1.324287in}}%
\pgfpathcurveto{\pgfqpoint{5.353836in}{1.324287in}}{\pgfqpoint{5.360677in}{1.327121in}}{\pgfqpoint{5.365721in}{1.332164in}}%
\pgfpathcurveto{\pgfqpoint{5.370765in}{1.337208in}}{\pgfqpoint{5.373599in}{1.344050in}}{\pgfqpoint{5.373599in}{1.351182in}}%
\pgfpathcurveto{\pgfqpoint{5.373599in}{1.358315in}}{\pgfqpoint{5.370765in}{1.365157in}}{\pgfqpoint{5.365721in}{1.370201in}}%
\pgfpathcurveto{\pgfqpoint{5.360677in}{1.375244in}}{\pgfqpoint{5.353836in}{1.378078in}}{\pgfqpoint{5.346703in}{1.378078in}}%
\pgfpathcurveto{\pgfqpoint{5.339570in}{1.378078in}}{\pgfqpoint{5.332728in}{1.375244in}}{\pgfqpoint{5.327685in}{1.370201in}}%
\pgfpathcurveto{\pgfqpoint{5.322641in}{1.365157in}}{\pgfqpoint{5.319807in}{1.358315in}}{\pgfqpoint{5.319807in}{1.351182in}}%
\pgfpathcurveto{\pgfqpoint{5.319807in}{1.344050in}}{\pgfqpoint{5.322641in}{1.337208in}}{\pgfqpoint{5.327685in}{1.332164in}}%
\pgfpathcurveto{\pgfqpoint{5.332728in}{1.327121in}}{\pgfqpoint{5.339570in}{1.324287in}}{\pgfqpoint{5.346703in}{1.324287in}}%
\pgfpathclose%
\pgfusepath{stroke,fill}%
\end{pgfscope}%
\begin{pgfscope}%
\pgfpathrectangle{\pgfqpoint{4.985294in}{0.500000in}}{\pgfqpoint{1.764706in}{1.700000in}}%
\pgfusepath{clip}%
\pgfsetbuttcap%
\pgfsetroundjoin%
\definecolor{currentfill}{rgb}{0.970255,0.815666,0.711203}%
\pgfsetfillcolor{currentfill}%
\pgfsetlinewidth{0.311001pt}%
\definecolor{currentstroke}{rgb}{1.000000,1.000000,1.000000}%
\pgfsetstrokecolor{currentstroke}%
\pgfsetdash{}{0pt}%
\pgfpathmoveto{\pgfqpoint{6.237011in}{0.996396in}}%
\pgfpathcurveto{\pgfqpoint{6.244143in}{0.996396in}}{\pgfqpoint{6.250985in}{0.999230in}}{\pgfqpoint{6.256029in}{1.004274in}}%
\pgfpathcurveto{\pgfqpoint{6.261072in}{1.009318in}}{\pgfqpoint{6.263906in}{1.016159in}}{\pgfqpoint{6.263906in}{1.023292in}}%
\pgfpathcurveto{\pgfqpoint{6.263906in}{1.030425in}}{\pgfqpoint{6.261072in}{1.037267in}}{\pgfqpoint{6.256029in}{1.042310in}}%
\pgfpathcurveto{\pgfqpoint{6.250985in}{1.047354in}}{\pgfqpoint{6.244143in}{1.050188in}}{\pgfqpoint{6.237011in}{1.050188in}}%
\pgfpathcurveto{\pgfqpoint{6.229878in}{1.050188in}}{\pgfqpoint{6.223036in}{1.047354in}}{\pgfqpoint{6.217993in}{1.042310in}}%
\pgfpathcurveto{\pgfqpoint{6.212949in}{1.037267in}}{\pgfqpoint{6.210115in}{1.030425in}}{\pgfqpoint{6.210115in}{1.023292in}}%
\pgfpathcurveto{\pgfqpoint{6.210115in}{1.016159in}}{\pgfqpoint{6.212949in}{1.009318in}}{\pgfqpoint{6.217993in}{1.004274in}}%
\pgfpathcurveto{\pgfqpoint{6.223036in}{0.999230in}}{\pgfqpoint{6.229878in}{0.996396in}}{\pgfqpoint{6.237011in}{0.996396in}}%
\pgfpathclose%
\pgfusepath{stroke,fill}%
\end{pgfscope}%
\begin{pgfscope}%
\pgfpathrectangle{\pgfqpoint{4.985294in}{0.500000in}}{\pgfqpoint{1.764706in}{1.700000in}}%
\pgfusepath{clip}%
\pgfsetbuttcap%
\pgfsetroundjoin%
\definecolor{currentfill}{rgb}{0.965753,0.732351,0.592427}%
\pgfsetfillcolor{currentfill}%
\pgfsetlinewidth{0.311001pt}%
\definecolor{currentstroke}{rgb}{1.000000,1.000000,1.000000}%
\pgfsetstrokecolor{currentstroke}%
\pgfsetdash{}{0pt}%
\pgfpathmoveto{\pgfqpoint{5.541927in}{1.706121in}}%
\pgfpathcurveto{\pgfqpoint{5.549060in}{1.706121in}}{\pgfqpoint{5.555901in}{1.708954in}}{\pgfqpoint{5.560945in}{1.713998in}}%
\pgfpathcurveto{\pgfqpoint{5.565989in}{1.719042in}}{\pgfqpoint{5.568823in}{1.725883in}}{\pgfqpoint{5.568823in}{1.733016in}}%
\pgfpathcurveto{\pgfqpoint{5.568823in}{1.740149in}}{\pgfqpoint{5.565989in}{1.746991in}}{\pgfqpoint{5.560945in}{1.752034in}}%
\pgfpathcurveto{\pgfqpoint{5.555901in}{1.757078in}}{\pgfqpoint{5.549060in}{1.759912in}}{\pgfqpoint{5.541927in}{1.759912in}}%
\pgfpathcurveto{\pgfqpoint{5.534794in}{1.759912in}}{\pgfqpoint{5.527952in}{1.757078in}}{\pgfqpoint{5.522909in}{1.752034in}}%
\pgfpathcurveto{\pgfqpoint{5.517865in}{1.746991in}}{\pgfqpoint{5.515031in}{1.740149in}}{\pgfqpoint{5.515031in}{1.733016in}}%
\pgfpathcurveto{\pgfqpoint{5.515031in}{1.725883in}}{\pgfqpoint{5.517865in}{1.719042in}}{\pgfqpoint{5.522909in}{1.713998in}}%
\pgfpathcurveto{\pgfqpoint{5.527952in}{1.708954in}}{\pgfqpoint{5.534794in}{1.706121in}}{\pgfqpoint{5.541927in}{1.706121in}}%
\pgfpathclose%
\pgfusepath{stroke,fill}%
\end{pgfscope}%
\begin{pgfscope}%
\pgfpathrectangle{\pgfqpoint{4.985294in}{0.500000in}}{\pgfqpoint{1.764706in}{1.700000in}}%
\pgfusepath{clip}%
\pgfsetbuttcap%
\pgfsetroundjoin%
\definecolor{currentfill}{rgb}{0.838502,0.140251,0.287688}%
\pgfsetfillcolor{currentfill}%
\pgfsetlinewidth{0.311001pt}%
\definecolor{currentstroke}{rgb}{1.000000,1.000000,1.000000}%
\pgfsetstrokecolor{currentstroke}%
\pgfsetdash{}{0pt}%
\pgfpathmoveto{\pgfqpoint{6.096183in}{1.429979in}}%
\pgfpathcurveto{\pgfqpoint{6.103315in}{1.429979in}}{\pgfqpoint{6.110157in}{1.432813in}}{\pgfqpoint{6.115201in}{1.437857in}}%
\pgfpathcurveto{\pgfqpoint{6.120244in}{1.442901in}}{\pgfqpoint{6.123078in}{1.449742in}}{\pgfqpoint{6.123078in}{1.456875in}}%
\pgfpathcurveto{\pgfqpoint{6.123078in}{1.464008in}}{\pgfqpoint{6.120244in}{1.470850in}}{\pgfqpoint{6.115201in}{1.475893in}}%
\pgfpathcurveto{\pgfqpoint{6.110157in}{1.480937in}}{\pgfqpoint{6.103315in}{1.483771in}}{\pgfqpoint{6.096183in}{1.483771in}}%
\pgfpathcurveto{\pgfqpoint{6.089050in}{1.483771in}}{\pgfqpoint{6.082208in}{1.480937in}}{\pgfqpoint{6.077164in}{1.475893in}}%
\pgfpathcurveto{\pgfqpoint{6.072121in}{1.470850in}}{\pgfqpoint{6.069287in}{1.464008in}}{\pgfqpoint{6.069287in}{1.456875in}}%
\pgfpathcurveto{\pgfqpoint{6.069287in}{1.449742in}}{\pgfqpoint{6.072121in}{1.442901in}}{\pgfqpoint{6.077164in}{1.437857in}}%
\pgfpathcurveto{\pgfqpoint{6.082208in}{1.432813in}}{\pgfqpoint{6.089050in}{1.429979in}}{\pgfqpoint{6.096183in}{1.429979in}}%
\pgfpathclose%
\pgfusepath{stroke,fill}%
\end{pgfscope}%
\begin{pgfscope}%
\pgfpathrectangle{\pgfqpoint{4.985294in}{0.500000in}}{\pgfqpoint{1.764706in}{1.700000in}}%
\pgfusepath{clip}%
\pgfsetbuttcap%
\pgfsetroundjoin%
\definecolor{currentfill}{rgb}{0.981377,0.920617,0.865369}%
\pgfsetfillcolor{currentfill}%
\pgfsetlinewidth{0.311001pt}%
\definecolor{currentstroke}{rgb}{1.000000,1.000000,1.000000}%
\pgfsetstrokecolor{currentstroke}%
\pgfsetdash{}{0pt}%
\pgfpathmoveto{\pgfqpoint{6.322028in}{1.309261in}}%
\pgfpathcurveto{\pgfqpoint{6.329160in}{1.309261in}}{\pgfqpoint{6.336002in}{1.312095in}}{\pgfqpoint{6.341046in}{1.317138in}}%
\pgfpathcurveto{\pgfqpoint{6.346089in}{1.322182in}}{\pgfqpoint{6.348923in}{1.329024in}}{\pgfqpoint{6.348923in}{1.336157in}}%
\pgfpathcurveto{\pgfqpoint{6.348923in}{1.343289in}}{\pgfqpoint{6.346089in}{1.350131in}}{\pgfqpoint{6.341046in}{1.355175in}}%
\pgfpathcurveto{\pgfqpoint{6.336002in}{1.360218in}}{\pgfqpoint{6.329160in}{1.363052in}}{\pgfqpoint{6.322028in}{1.363052in}}%
\pgfpathcurveto{\pgfqpoint{6.314895in}{1.363052in}}{\pgfqpoint{6.308053in}{1.360218in}}{\pgfqpoint{6.303009in}{1.355175in}}%
\pgfpathcurveto{\pgfqpoint{6.297966in}{1.350131in}}{\pgfqpoint{6.295132in}{1.343289in}}{\pgfqpoint{6.295132in}{1.336157in}}%
\pgfpathcurveto{\pgfqpoint{6.295132in}{1.329024in}}{\pgfqpoint{6.297966in}{1.322182in}}{\pgfqpoint{6.303009in}{1.317138in}}%
\pgfpathcurveto{\pgfqpoint{6.308053in}{1.312095in}}{\pgfqpoint{6.314895in}{1.309261in}}{\pgfqpoint{6.322028in}{1.309261in}}%
\pgfpathclose%
\pgfusepath{stroke,fill}%
\end{pgfscope}%
\begin{pgfscope}%
\pgfpathrectangle{\pgfqpoint{4.985294in}{0.500000in}}{\pgfqpoint{1.764706in}{1.700000in}}%
\pgfusepath{clip}%
\pgfsetbuttcap%
\pgfsetroundjoin%
\definecolor{currentfill}{rgb}{0.843354,0.145567,0.284808}%
\pgfsetfillcolor{currentfill}%
\pgfsetlinewidth{0.311001pt}%
\definecolor{currentstroke}{rgb}{1.000000,1.000000,1.000000}%
\pgfsetstrokecolor{currentstroke}%
\pgfsetdash{}{0pt}%
\pgfpathmoveto{\pgfqpoint{5.762596in}{1.823830in}}%
\pgfpathcurveto{\pgfqpoint{5.769729in}{1.823830in}}{\pgfqpoint{5.776571in}{1.826664in}}{\pgfqpoint{5.781615in}{1.831708in}}%
\pgfpathcurveto{\pgfqpoint{5.786658in}{1.836752in}}{\pgfqpoint{5.789492in}{1.843593in}}{\pgfqpoint{5.789492in}{1.850726in}}%
\pgfpathcurveto{\pgfqpoint{5.789492in}{1.857859in}}{\pgfqpoint{5.786658in}{1.864701in}}{\pgfqpoint{5.781615in}{1.869744in}}%
\pgfpathcurveto{\pgfqpoint{5.776571in}{1.874788in}}{\pgfqpoint{5.769729in}{1.877622in}}{\pgfqpoint{5.762596in}{1.877622in}}%
\pgfpathcurveto{\pgfqpoint{5.755464in}{1.877622in}}{\pgfqpoint{5.748622in}{1.874788in}}{\pgfqpoint{5.743578in}{1.869744in}}%
\pgfpathcurveto{\pgfqpoint{5.738535in}{1.864701in}}{\pgfqpoint{5.735701in}{1.857859in}}{\pgfqpoint{5.735701in}{1.850726in}}%
\pgfpathcurveto{\pgfqpoint{5.735701in}{1.843593in}}{\pgfqpoint{5.738535in}{1.836752in}}{\pgfqpoint{5.743578in}{1.831708in}}%
\pgfpathcurveto{\pgfqpoint{5.748622in}{1.826664in}}{\pgfqpoint{5.755464in}{1.823830in}}{\pgfqpoint{5.762596in}{1.823830in}}%
\pgfpathclose%
\pgfusepath{stroke,fill}%
\end{pgfscope}%
\begin{pgfscope}%
\pgfpathrectangle{\pgfqpoint{4.985294in}{0.500000in}}{\pgfqpoint{1.764706in}{1.700000in}}%
\pgfusepath{clip}%
\pgfsetbuttcap%
\pgfsetroundjoin%
\definecolor{currentfill}{rgb}{0.975018,0.868213,0.788710}%
\pgfsetfillcolor{currentfill}%
\pgfsetlinewidth{0.311001pt}%
\definecolor{currentstroke}{rgb}{1.000000,1.000000,1.000000}%
\pgfsetstrokecolor{currentstroke}%
\pgfsetdash{}{0pt}%
\pgfpathmoveto{\pgfqpoint{5.442481in}{1.084383in}}%
\pgfpathcurveto{\pgfqpoint{5.449613in}{1.084383in}}{\pgfqpoint{5.456455in}{1.087217in}}{\pgfqpoint{5.461499in}{1.092261in}}%
\pgfpathcurveto{\pgfqpoint{5.466542in}{1.097305in}}{\pgfqpoint{5.469376in}{1.104146in}}{\pgfqpoint{5.469376in}{1.111279in}}%
\pgfpathcurveto{\pgfqpoint{5.469376in}{1.118412in}}{\pgfqpoint{5.466542in}{1.125254in}}{\pgfqpoint{5.461499in}{1.130297in}}%
\pgfpathcurveto{\pgfqpoint{5.456455in}{1.135341in}}{\pgfqpoint{5.449613in}{1.138175in}}{\pgfqpoint{5.442481in}{1.138175in}}%
\pgfpathcurveto{\pgfqpoint{5.435348in}{1.138175in}}{\pgfqpoint{5.428506in}{1.135341in}}{\pgfqpoint{5.423462in}{1.130297in}}%
\pgfpathcurveto{\pgfqpoint{5.418419in}{1.125254in}}{\pgfqpoint{5.415585in}{1.118412in}}{\pgfqpoint{5.415585in}{1.111279in}}%
\pgfpathcurveto{\pgfqpoint{5.415585in}{1.104146in}}{\pgfqpoint{5.418419in}{1.097305in}}{\pgfqpoint{5.423462in}{1.092261in}}%
\pgfpathcurveto{\pgfqpoint{5.428506in}{1.087217in}}{\pgfqpoint{5.435348in}{1.084383in}}{\pgfqpoint{5.442481in}{1.084383in}}%
\pgfpathclose%
\pgfusepath{stroke,fill}%
\end{pgfscope}%
\begin{pgfscope}%
\pgfpathrectangle{\pgfqpoint{4.985294in}{0.500000in}}{\pgfqpoint{1.764706in}{1.700000in}}%
\pgfusepath{clip}%
\pgfsetbuttcap%
\pgfsetroundjoin%
\definecolor{currentfill}{rgb}{0.972201,0.839051,0.745789}%
\pgfsetfillcolor{currentfill}%
\pgfsetlinewidth{0.311001pt}%
\definecolor{currentstroke}{rgb}{1.000000,1.000000,1.000000}%
\pgfsetstrokecolor{currentstroke}%
\pgfsetdash{}{0pt}%
\pgfpathmoveto{\pgfqpoint{5.383425in}{1.466178in}}%
\pgfpathcurveto{\pgfqpoint{5.390558in}{1.466178in}}{\pgfqpoint{5.397399in}{1.469012in}}{\pgfqpoint{5.402443in}{1.474055in}}%
\pgfpathcurveto{\pgfqpoint{5.407487in}{1.479099in}}{\pgfqpoint{5.410321in}{1.485941in}}{\pgfqpoint{5.410321in}{1.493073in}}%
\pgfpathcurveto{\pgfqpoint{5.410321in}{1.500206in}}{\pgfqpoint{5.407487in}{1.507048in}}{\pgfqpoint{5.402443in}{1.512092in}}%
\pgfpathcurveto{\pgfqpoint{5.397399in}{1.517135in}}{\pgfqpoint{5.390558in}{1.519969in}}{\pgfqpoint{5.383425in}{1.519969in}}%
\pgfpathcurveto{\pgfqpoint{5.376292in}{1.519969in}}{\pgfqpoint{5.369450in}{1.517135in}}{\pgfqpoint{5.364407in}{1.512092in}}%
\pgfpathcurveto{\pgfqpoint{5.359363in}{1.507048in}}{\pgfqpoint{5.356529in}{1.500206in}}{\pgfqpoint{5.356529in}{1.493073in}}%
\pgfpathcurveto{\pgfqpoint{5.356529in}{1.485941in}}{\pgfqpoint{5.359363in}{1.479099in}}{\pgfqpoint{5.364407in}{1.474055in}}%
\pgfpathcurveto{\pgfqpoint{5.369450in}{1.469012in}}{\pgfqpoint{5.376292in}{1.466178in}}{\pgfqpoint{5.383425in}{1.466178in}}%
\pgfpathclose%
\pgfusepath{stroke,fill}%
\end{pgfscope}%
\begin{pgfscope}%
\pgfpathrectangle{\pgfqpoint{4.985294in}{0.500000in}}{\pgfqpoint{1.764706in}{1.700000in}}%
\pgfusepath{clip}%
\pgfsetbuttcap%
\pgfsetroundjoin%
\definecolor{currentfill}{rgb}{0.965753,0.732351,0.592427}%
\pgfsetfillcolor{currentfill}%
\pgfsetlinewidth{0.311001pt}%
\definecolor{currentstroke}{rgb}{1.000000,1.000000,1.000000}%
\pgfsetstrokecolor{currentstroke}%
\pgfsetdash{}{0pt}%
\pgfpathmoveto{\pgfqpoint{6.254979in}{0.961420in}}%
\pgfpathcurveto{\pgfqpoint{6.262111in}{0.961420in}}{\pgfqpoint{6.268953in}{0.964254in}}{\pgfqpoint{6.273997in}{0.969298in}}%
\pgfpathcurveto{\pgfqpoint{6.279040in}{0.974342in}}{\pgfqpoint{6.281874in}{0.981183in}}{\pgfqpoint{6.281874in}{0.988316in}}%
\pgfpathcurveto{\pgfqpoint{6.281874in}{0.995449in}}{\pgfqpoint{6.279040in}{1.002291in}}{\pgfqpoint{6.273997in}{1.007334in}}%
\pgfpathcurveto{\pgfqpoint{6.268953in}{1.012378in}}{\pgfqpoint{6.262111in}{1.015212in}}{\pgfqpoint{6.254979in}{1.015212in}}%
\pgfpathcurveto{\pgfqpoint{6.247846in}{1.015212in}}{\pgfqpoint{6.241004in}{1.012378in}}{\pgfqpoint{6.235960in}{1.007334in}}%
\pgfpathcurveto{\pgfqpoint{6.230917in}{1.002291in}}{\pgfqpoint{6.228083in}{0.995449in}}{\pgfqpoint{6.228083in}{0.988316in}}%
\pgfpathcurveto{\pgfqpoint{6.228083in}{0.981183in}}{\pgfqpoint{6.230917in}{0.974342in}}{\pgfqpoint{6.235960in}{0.969298in}}%
\pgfpathcurveto{\pgfqpoint{6.241004in}{0.964254in}}{\pgfqpoint{6.247846in}{0.961420in}}{\pgfqpoint{6.254979in}{0.961420in}}%
\pgfpathclose%
\pgfusepath{stroke,fill}%
\end{pgfscope}%
\begin{pgfscope}%
\pgfpathrectangle{\pgfqpoint{4.985294in}{0.500000in}}{\pgfqpoint{1.764706in}{1.700000in}}%
\pgfusepath{clip}%
\pgfsetbuttcap%
\pgfsetroundjoin%
\definecolor{currentfill}{rgb}{0.965592,0.726236,0.584384}%
\pgfsetfillcolor{currentfill}%
\pgfsetlinewidth{0.311001pt}%
\definecolor{currentstroke}{rgb}{1.000000,1.000000,1.000000}%
\pgfsetstrokecolor{currentstroke}%
\pgfsetdash{}{0pt}%
\pgfpathmoveto{\pgfqpoint{5.578308in}{1.644269in}}%
\pgfpathcurveto{\pgfqpoint{5.585441in}{1.644269in}}{\pgfqpoint{5.592282in}{1.647103in}}{\pgfqpoint{5.597326in}{1.652147in}}%
\pgfpathcurveto{\pgfqpoint{5.602370in}{1.657190in}}{\pgfqpoint{5.605204in}{1.664032in}}{\pgfqpoint{5.605204in}{1.671165in}}%
\pgfpathcurveto{\pgfqpoint{5.605204in}{1.678298in}}{\pgfqpoint{5.602370in}{1.685139in}}{\pgfqpoint{5.597326in}{1.690183in}}%
\pgfpathcurveto{\pgfqpoint{5.592282in}{1.695227in}}{\pgfqpoint{5.585441in}{1.698061in}}{\pgfqpoint{5.578308in}{1.698061in}}%
\pgfpathcurveto{\pgfqpoint{5.571175in}{1.698061in}}{\pgfqpoint{5.564333in}{1.695227in}}{\pgfqpoint{5.559290in}{1.690183in}}%
\pgfpathcurveto{\pgfqpoint{5.554246in}{1.685139in}}{\pgfqpoint{5.551412in}{1.678298in}}{\pgfqpoint{5.551412in}{1.671165in}}%
\pgfpathcurveto{\pgfqpoint{5.551412in}{1.664032in}}{\pgfqpoint{5.554246in}{1.657190in}}{\pgfqpoint{5.559290in}{1.652147in}}%
\pgfpathcurveto{\pgfqpoint{5.564333in}{1.647103in}}{\pgfqpoint{5.571175in}{1.644269in}}{\pgfqpoint{5.578308in}{1.644269in}}%
\pgfpathclose%
\pgfusepath{stroke,fill}%
\end{pgfscope}%
\begin{pgfscope}%
\pgfpathrectangle{\pgfqpoint{4.985294in}{0.500000in}}{\pgfqpoint{1.764706in}{1.700000in}}%
\pgfusepath{clip}%
\pgfsetbuttcap%
\pgfsetroundjoin%
\definecolor{currentfill}{rgb}{0.796501,0.105066,0.310630}%
\pgfsetfillcolor{currentfill}%
\pgfsetlinewidth{0.311001pt}%
\definecolor{currentstroke}{rgb}{1.000000,1.000000,1.000000}%
\pgfsetstrokecolor{currentstroke}%
\pgfsetdash{}{0pt}%
\pgfpathmoveto{\pgfqpoint{6.056333in}{1.125609in}}%
\pgfpathcurveto{\pgfqpoint{6.063465in}{1.125609in}}{\pgfqpoint{6.070307in}{1.128443in}}{\pgfqpoint{6.075351in}{1.133487in}}%
\pgfpathcurveto{\pgfqpoint{6.080394in}{1.138531in}}{\pgfqpoint{6.083228in}{1.145372in}}{\pgfqpoint{6.083228in}{1.152505in}}%
\pgfpathcurveto{\pgfqpoint{6.083228in}{1.159638in}}{\pgfqpoint{6.080394in}{1.166480in}}{\pgfqpoint{6.075351in}{1.171523in}}%
\pgfpathcurveto{\pgfqpoint{6.070307in}{1.176567in}}{\pgfqpoint{6.063465in}{1.179401in}}{\pgfqpoint{6.056333in}{1.179401in}}%
\pgfpathcurveto{\pgfqpoint{6.049200in}{1.179401in}}{\pgfqpoint{6.042358in}{1.176567in}}{\pgfqpoint{6.037314in}{1.171523in}}%
\pgfpathcurveto{\pgfqpoint{6.032271in}{1.166480in}}{\pgfqpoint{6.029437in}{1.159638in}}{\pgfqpoint{6.029437in}{1.152505in}}%
\pgfpathcurveto{\pgfqpoint{6.029437in}{1.145372in}}{\pgfqpoint{6.032271in}{1.138531in}}{\pgfqpoint{6.037314in}{1.133487in}}%
\pgfpathcurveto{\pgfqpoint{6.042358in}{1.128443in}}{\pgfqpoint{6.049200in}{1.125609in}}{\pgfqpoint{6.056333in}{1.125609in}}%
\pgfpathclose%
\pgfusepath{stroke,fill}%
\end{pgfscope}%
\begin{pgfscope}%
\pgfpathrectangle{\pgfqpoint{4.985294in}{0.500000in}}{\pgfqpoint{1.764706in}{1.700000in}}%
\pgfusepath{clip}%
\pgfsetbuttcap%
\pgfsetroundjoin%
\definecolor{currentfill}{rgb}{0.942910,0.375495,0.263698}%
\pgfsetfillcolor{currentfill}%
\pgfsetlinewidth{0.311001pt}%
\definecolor{currentstroke}{rgb}{1.000000,1.000000,1.000000}%
\pgfsetstrokecolor{currentstroke}%
\pgfsetdash{}{0pt}%
\pgfpathmoveto{\pgfqpoint{6.445480in}{1.458355in}}%
\pgfpathcurveto{\pgfqpoint{6.452613in}{1.458355in}}{\pgfqpoint{6.459455in}{1.461189in}}{\pgfqpoint{6.464498in}{1.466233in}}%
\pgfpathcurveto{\pgfqpoint{6.469542in}{1.471276in}}{\pgfqpoint{6.472376in}{1.478118in}}{\pgfqpoint{6.472376in}{1.485251in}}%
\pgfpathcurveto{\pgfqpoint{6.472376in}{1.492384in}}{\pgfqpoint{6.469542in}{1.499225in}}{\pgfqpoint{6.464498in}{1.504269in}}%
\pgfpathcurveto{\pgfqpoint{6.459455in}{1.509313in}}{\pgfqpoint{6.452613in}{1.512146in}}{\pgfqpoint{6.445480in}{1.512146in}}%
\pgfpathcurveto{\pgfqpoint{6.438347in}{1.512146in}}{\pgfqpoint{6.431506in}{1.509313in}}{\pgfqpoint{6.426462in}{1.504269in}}%
\pgfpathcurveto{\pgfqpoint{6.421418in}{1.499225in}}{\pgfqpoint{6.418584in}{1.492384in}}{\pgfqpoint{6.418584in}{1.485251in}}%
\pgfpathcurveto{\pgfqpoint{6.418584in}{1.478118in}}{\pgfqpoint{6.421418in}{1.471276in}}{\pgfqpoint{6.426462in}{1.466233in}}%
\pgfpathcurveto{\pgfqpoint{6.431506in}{1.461189in}}{\pgfqpoint{6.438347in}{1.458355in}}{\pgfqpoint{6.445480in}{1.458355in}}%
\pgfpathclose%
\pgfusepath{stroke,fill}%
\end{pgfscope}%
\begin{pgfscope}%
\pgfpathrectangle{\pgfqpoint{4.985294in}{0.500000in}}{\pgfqpoint{1.764706in}{1.700000in}}%
\pgfusepath{clip}%
\pgfsetbuttcap%
\pgfsetroundjoin%
\definecolor{currentfill}{rgb}{0.961115,0.566634,0.405693}%
\pgfsetfillcolor{currentfill}%
\pgfsetlinewidth{0.311001pt}%
\definecolor{currentstroke}{rgb}{1.000000,1.000000,1.000000}%
\pgfsetstrokecolor{currentstroke}%
\pgfsetdash{}{0pt}%
\pgfpathmoveto{\pgfqpoint{5.306249in}{1.196220in}}%
\pgfpathcurveto{\pgfqpoint{5.313382in}{1.196220in}}{\pgfqpoint{5.320224in}{1.199054in}}{\pgfqpoint{5.325267in}{1.204097in}}%
\pgfpathcurveto{\pgfqpoint{5.330311in}{1.209141in}}{\pgfqpoint{5.333145in}{1.215983in}}{\pgfqpoint{5.333145in}{1.223115in}}%
\pgfpathcurveto{\pgfqpoint{5.333145in}{1.230248in}}{\pgfqpoint{5.330311in}{1.237090in}}{\pgfqpoint{5.325267in}{1.242134in}}%
\pgfpathcurveto{\pgfqpoint{5.320224in}{1.247177in}}{\pgfqpoint{5.313382in}{1.250011in}}{\pgfqpoint{5.306249in}{1.250011in}}%
\pgfpathcurveto{\pgfqpoint{5.299116in}{1.250011in}}{\pgfqpoint{5.292275in}{1.247177in}}{\pgfqpoint{5.287231in}{1.242134in}}%
\pgfpathcurveto{\pgfqpoint{5.282187in}{1.237090in}}{\pgfqpoint{5.279353in}{1.230248in}}{\pgfqpoint{5.279353in}{1.223115in}}%
\pgfpathcurveto{\pgfqpoint{5.279353in}{1.215983in}}{\pgfqpoint{5.282187in}{1.209141in}}{\pgfqpoint{5.287231in}{1.204097in}}%
\pgfpathcurveto{\pgfqpoint{5.292275in}{1.199054in}}{\pgfqpoint{5.299116in}{1.196220in}}{\pgfqpoint{5.306249in}{1.196220in}}%
\pgfpathclose%
\pgfusepath{stroke,fill}%
\end{pgfscope}%
\begin{pgfscope}%
\pgfpathrectangle{\pgfqpoint{4.985294in}{0.500000in}}{\pgfqpoint{1.764706in}{1.700000in}}%
\pgfusepath{clip}%
\pgfsetbuttcap%
\pgfsetroundjoin%
\definecolor{currentfill}{rgb}{0.975018,0.868213,0.788710}%
\pgfsetfillcolor{currentfill}%
\pgfsetlinewidth{0.311001pt}%
\definecolor{currentstroke}{rgb}{1.000000,1.000000,1.000000}%
\pgfsetstrokecolor{currentstroke}%
\pgfsetdash{}{0pt}%
\pgfpathmoveto{\pgfqpoint{5.479995in}{1.466976in}}%
\pgfpathcurveto{\pgfqpoint{5.487128in}{1.466976in}}{\pgfqpoint{5.493970in}{1.469810in}}{\pgfqpoint{5.499014in}{1.474853in}}%
\pgfpathcurveto{\pgfqpoint{5.504057in}{1.479897in}}{\pgfqpoint{5.506891in}{1.486739in}}{\pgfqpoint{5.506891in}{1.493871in}}%
\pgfpathcurveto{\pgfqpoint{5.506891in}{1.501004in}}{\pgfqpoint{5.504057in}{1.507846in}}{\pgfqpoint{5.499014in}{1.512890in}}%
\pgfpathcurveto{\pgfqpoint{5.493970in}{1.517933in}}{\pgfqpoint{5.487128in}{1.520767in}}{\pgfqpoint{5.479995in}{1.520767in}}%
\pgfpathcurveto{\pgfqpoint{5.472863in}{1.520767in}}{\pgfqpoint{5.466021in}{1.517933in}}{\pgfqpoint{5.460977in}{1.512890in}}%
\pgfpathcurveto{\pgfqpoint{5.455934in}{1.507846in}}{\pgfqpoint{5.453100in}{1.501004in}}{\pgfqpoint{5.453100in}{1.493871in}}%
\pgfpathcurveto{\pgfqpoint{5.453100in}{1.486739in}}{\pgfqpoint{5.455934in}{1.479897in}}{\pgfqpoint{5.460977in}{1.474853in}}%
\pgfpathcurveto{\pgfqpoint{5.466021in}{1.469810in}}{\pgfqpoint{5.472863in}{1.466976in}}{\pgfqpoint{5.479995in}{1.466976in}}%
\pgfpathclose%
\pgfusepath{stroke,fill}%
\end{pgfscope}%
\begin{pgfscope}%
\pgfpathrectangle{\pgfqpoint{4.985294in}{0.500000in}}{\pgfqpoint{1.764706in}{1.700000in}}%
\pgfusepath{clip}%
\pgfsetbuttcap%
\pgfsetroundjoin%
\definecolor{currentfill}{rgb}{0.838502,0.140251,0.287688}%
\pgfsetfillcolor{currentfill}%
\pgfsetlinewidth{0.311001pt}%
\definecolor{currentstroke}{rgb}{1.000000,1.000000,1.000000}%
\pgfsetstrokecolor{currentstroke}%
\pgfsetdash{}{0pt}%
\pgfpathmoveto{\pgfqpoint{5.912286in}{1.734134in}}%
\pgfpathcurveto{\pgfqpoint{5.919418in}{1.734134in}}{\pgfqpoint{5.926260in}{1.736968in}}{\pgfqpoint{5.931304in}{1.742012in}}%
\pgfpathcurveto{\pgfqpoint{5.936347in}{1.747056in}}{\pgfqpoint{5.939181in}{1.753897in}}{\pgfqpoint{5.939181in}{1.761030in}}%
\pgfpathcurveto{\pgfqpoint{5.939181in}{1.768163in}}{\pgfqpoint{5.936347in}{1.775005in}}{\pgfqpoint{5.931304in}{1.780048in}}%
\pgfpathcurveto{\pgfqpoint{5.926260in}{1.785092in}}{\pgfqpoint{5.919418in}{1.787926in}}{\pgfqpoint{5.912286in}{1.787926in}}%
\pgfpathcurveto{\pgfqpoint{5.905153in}{1.787926in}}{\pgfqpoint{5.898311in}{1.785092in}}{\pgfqpoint{5.893267in}{1.780048in}}%
\pgfpathcurveto{\pgfqpoint{5.888224in}{1.775005in}}{\pgfqpoint{5.885390in}{1.768163in}}{\pgfqpoint{5.885390in}{1.761030in}}%
\pgfpathcurveto{\pgfqpoint{5.885390in}{1.753897in}}{\pgfqpoint{5.888224in}{1.747056in}}{\pgfqpoint{5.893267in}{1.742012in}}%
\pgfpathcurveto{\pgfqpoint{5.898311in}{1.736968in}}{\pgfqpoint{5.905153in}{1.734134in}}{\pgfqpoint{5.912286in}{1.734134in}}%
\pgfpathclose%
\pgfusepath{stroke,fill}%
\end{pgfscope}%
\begin{pgfscope}%
\pgfpathrectangle{\pgfqpoint{4.985294in}{0.500000in}}{\pgfqpoint{1.764706in}{1.700000in}}%
\pgfusepath{clip}%
\pgfsetbuttcap%
\pgfsetroundjoin%
\definecolor{currentfill}{rgb}{0.914423,0.260289,0.243694}%
\pgfsetfillcolor{currentfill}%
\pgfsetlinewidth{0.311001pt}%
\definecolor{currentstroke}{rgb}{1.000000,1.000000,1.000000}%
\pgfsetstrokecolor{currentstroke}%
\pgfsetdash{}{0pt}%
\pgfpathmoveto{\pgfqpoint{6.088838in}{1.121157in}}%
\pgfpathcurveto{\pgfqpoint{6.095971in}{1.121157in}}{\pgfqpoint{6.102813in}{1.123991in}}{\pgfqpoint{6.107857in}{1.129034in}}%
\pgfpathcurveto{\pgfqpoint{6.112900in}{1.134078in}}{\pgfqpoint{6.115734in}{1.140920in}}{\pgfqpoint{6.115734in}{1.148053in}}%
\pgfpathcurveto{\pgfqpoint{6.115734in}{1.155185in}}{\pgfqpoint{6.112900in}{1.162027in}}{\pgfqpoint{6.107857in}{1.167071in}}%
\pgfpathcurveto{\pgfqpoint{6.102813in}{1.172114in}}{\pgfqpoint{6.095971in}{1.174948in}}{\pgfqpoint{6.088838in}{1.174948in}}%
\pgfpathcurveto{\pgfqpoint{6.081706in}{1.174948in}}{\pgfqpoint{6.074864in}{1.172114in}}{\pgfqpoint{6.069820in}{1.167071in}}%
\pgfpathcurveto{\pgfqpoint{6.064777in}{1.162027in}}{\pgfqpoint{6.061943in}{1.155185in}}{\pgfqpoint{6.061943in}{1.148053in}}%
\pgfpathcurveto{\pgfqpoint{6.061943in}{1.140920in}}{\pgfqpoint{6.064777in}{1.134078in}}{\pgfqpoint{6.069820in}{1.129034in}}%
\pgfpathcurveto{\pgfqpoint{6.074864in}{1.123991in}}{\pgfqpoint{6.081706in}{1.121157in}}{\pgfqpoint{6.088838in}{1.121157in}}%
\pgfpathclose%
\pgfusepath{stroke,fill}%
\end{pgfscope}%
\begin{pgfscope}%
\pgfpathrectangle{\pgfqpoint{4.985294in}{0.500000in}}{\pgfqpoint{1.764706in}{1.700000in}}%
\pgfusepath{clip}%
\pgfsetbuttcap%
\pgfsetroundjoin%
\definecolor{currentfill}{rgb}{0.828528,0.130141,0.293475}%
\pgfsetfillcolor{currentfill}%
\pgfsetlinewidth{0.311001pt}%
\definecolor{currentstroke}{rgb}{1.000000,1.000000,1.000000}%
\pgfsetstrokecolor{currentstroke}%
\pgfsetdash{}{0pt}%
\pgfpathmoveto{\pgfqpoint{5.661949in}{1.856063in}}%
\pgfpathcurveto{\pgfqpoint{5.669082in}{1.856063in}}{\pgfqpoint{5.675924in}{1.858897in}}{\pgfqpoint{5.680967in}{1.863940in}}%
\pgfpathcurveto{\pgfqpoint{5.686011in}{1.868984in}}{\pgfqpoint{5.688845in}{1.875826in}}{\pgfqpoint{5.688845in}{1.882958in}}%
\pgfpathcurveto{\pgfqpoint{5.688845in}{1.890091in}}{\pgfqpoint{5.686011in}{1.896933in}}{\pgfqpoint{5.680967in}{1.901977in}}%
\pgfpathcurveto{\pgfqpoint{5.675924in}{1.907020in}}{\pgfqpoint{5.669082in}{1.909854in}}{\pgfqpoint{5.661949in}{1.909854in}}%
\pgfpathcurveto{\pgfqpoint{5.654816in}{1.909854in}}{\pgfqpoint{5.647975in}{1.907020in}}{\pgfqpoint{5.642931in}{1.901977in}}%
\pgfpathcurveto{\pgfqpoint{5.637887in}{1.896933in}}{\pgfqpoint{5.635054in}{1.890091in}}{\pgfqpoint{5.635054in}{1.882958in}}%
\pgfpathcurveto{\pgfqpoint{5.635054in}{1.875826in}}{\pgfqpoint{5.637887in}{1.868984in}}{\pgfqpoint{5.642931in}{1.863940in}}%
\pgfpathcurveto{\pgfqpoint{5.647975in}{1.858897in}}{\pgfqpoint{5.654816in}{1.856063in}}{\pgfqpoint{5.661949in}{1.856063in}}%
\pgfpathclose%
\pgfusepath{stroke,fill}%
\end{pgfscope}%
\begin{pgfscope}%
\pgfpathrectangle{\pgfqpoint{4.985294in}{0.500000in}}{\pgfqpoint{1.764706in}{1.700000in}}%
\pgfusepath{clip}%
\pgfsetbuttcap%
\pgfsetroundjoin%
\definecolor{currentfill}{rgb}{0.976961,0.885681,0.814303}%
\pgfsetfillcolor{currentfill}%
\pgfsetlinewidth{0.311001pt}%
\definecolor{currentstroke}{rgb}{1.000000,1.000000,1.000000}%
\pgfsetstrokecolor{currentstroke}%
\pgfsetdash{}{0pt}%
\pgfpathmoveto{\pgfqpoint{6.253801in}{1.592285in}}%
\pgfpathcurveto{\pgfqpoint{6.260934in}{1.592285in}}{\pgfqpoint{6.267776in}{1.595118in}}{\pgfqpoint{6.272820in}{1.600162in}}%
\pgfpathcurveto{\pgfqpoint{6.277863in}{1.605206in}}{\pgfqpoint{6.280697in}{1.612047in}}{\pgfqpoint{6.280697in}{1.619180in}}%
\pgfpathcurveto{\pgfqpoint{6.280697in}{1.626313in}}{\pgfqpoint{6.277863in}{1.633155in}}{\pgfqpoint{6.272820in}{1.638198in}}%
\pgfpathcurveto{\pgfqpoint{6.267776in}{1.643242in}}{\pgfqpoint{6.260934in}{1.646076in}}{\pgfqpoint{6.253801in}{1.646076in}}%
\pgfpathcurveto{\pgfqpoint{6.246669in}{1.646076in}}{\pgfqpoint{6.239827in}{1.643242in}}{\pgfqpoint{6.234783in}{1.638198in}}%
\pgfpathcurveto{\pgfqpoint{6.229740in}{1.633155in}}{\pgfqpoint{6.226906in}{1.626313in}}{\pgfqpoint{6.226906in}{1.619180in}}%
\pgfpathcurveto{\pgfqpoint{6.226906in}{1.612047in}}{\pgfqpoint{6.229740in}{1.605206in}}{\pgfqpoint{6.234783in}{1.600162in}}%
\pgfpathcurveto{\pgfqpoint{6.239827in}{1.595118in}}{\pgfqpoint{6.246669in}{1.592285in}}{\pgfqpoint{6.253801in}{1.592285in}}%
\pgfpathclose%
\pgfusepath{stroke,fill}%
\end{pgfscope}%
\begin{pgfscope}%
\pgfpathrectangle{\pgfqpoint{4.985294in}{0.500000in}}{\pgfqpoint{1.764706in}{1.700000in}}%
\pgfusepath{clip}%
\pgfsetbuttcap%
\pgfsetroundjoin%
\definecolor{currentfill}{rgb}{0.971694,0.833208,0.737161}%
\pgfsetfillcolor{currentfill}%
\pgfsetlinewidth{0.311001pt}%
\definecolor{currentstroke}{rgb}{1.000000,1.000000,1.000000}%
\pgfsetstrokecolor{currentstroke}%
\pgfsetdash{}{0pt}%
\pgfpathmoveto{\pgfqpoint{5.517505in}{0.994478in}}%
\pgfpathcurveto{\pgfqpoint{5.524638in}{0.994478in}}{\pgfqpoint{5.531480in}{0.997312in}}{\pgfqpoint{5.536523in}{1.002356in}}%
\pgfpathcurveto{\pgfqpoint{5.541567in}{1.007399in}}{\pgfqpoint{5.544401in}{1.014241in}}{\pgfqpoint{5.544401in}{1.021374in}}%
\pgfpathcurveto{\pgfqpoint{5.544401in}{1.028507in}}{\pgfqpoint{5.541567in}{1.035348in}}{\pgfqpoint{5.536523in}{1.040392in}}%
\pgfpathcurveto{\pgfqpoint{5.531480in}{1.045436in}}{\pgfqpoint{5.524638in}{1.048269in}}{\pgfqpoint{5.517505in}{1.048269in}}%
\pgfpathcurveto{\pgfqpoint{5.510372in}{1.048269in}}{\pgfqpoint{5.503531in}{1.045436in}}{\pgfqpoint{5.498487in}{1.040392in}}%
\pgfpathcurveto{\pgfqpoint{5.493443in}{1.035348in}}{\pgfqpoint{5.490609in}{1.028507in}}{\pgfqpoint{5.490609in}{1.021374in}}%
\pgfpathcurveto{\pgfqpoint{5.490609in}{1.014241in}}{\pgfqpoint{5.493443in}{1.007399in}}{\pgfqpoint{5.498487in}{1.002356in}}%
\pgfpathcurveto{\pgfqpoint{5.503531in}{0.997312in}}{\pgfqpoint{5.510372in}{0.994478in}}{\pgfqpoint{5.517505in}{0.994478in}}%
\pgfpathclose%
\pgfusepath{stroke,fill}%
\end{pgfscope}%
\begin{pgfscope}%
\pgfpathrectangle{\pgfqpoint{4.985294in}{0.500000in}}{\pgfqpoint{1.764706in}{1.700000in}}%
\pgfusepath{clip}%
\pgfsetbuttcap%
\pgfsetroundjoin%
\definecolor{currentfill}{rgb}{0.972726,0.844889,0.754401}%
\pgfsetfillcolor{currentfill}%
\pgfsetlinewidth{0.311001pt}%
\definecolor{currentstroke}{rgb}{1.000000,1.000000,1.000000}%
\pgfsetstrokecolor{currentstroke}%
\pgfsetdash{}{0pt}%
\pgfpathmoveto{\pgfqpoint{5.487751in}{1.414897in}}%
\pgfpathcurveto{\pgfqpoint{5.494884in}{1.414897in}}{\pgfqpoint{5.501725in}{1.417730in}}{\pgfqpoint{5.506769in}{1.422774in}}%
\pgfpathcurveto{\pgfqpoint{5.511813in}{1.427818in}}{\pgfqpoint{5.514647in}{1.434659in}}{\pgfqpoint{5.514647in}{1.441792in}}%
\pgfpathcurveto{\pgfqpoint{5.514647in}{1.448925in}}{\pgfqpoint{5.511813in}{1.455767in}}{\pgfqpoint{5.506769in}{1.460810in}}%
\pgfpathcurveto{\pgfqpoint{5.501725in}{1.465854in}}{\pgfqpoint{5.494884in}{1.468688in}}{\pgfqpoint{5.487751in}{1.468688in}}%
\pgfpathcurveto{\pgfqpoint{5.480618in}{1.468688in}}{\pgfqpoint{5.473776in}{1.465854in}}{\pgfqpoint{5.468733in}{1.460810in}}%
\pgfpathcurveto{\pgfqpoint{5.463689in}{1.455767in}}{\pgfqpoint{5.460855in}{1.448925in}}{\pgfqpoint{5.460855in}{1.441792in}}%
\pgfpathcurveto{\pgfqpoint{5.460855in}{1.434659in}}{\pgfqpoint{5.463689in}{1.427818in}}{\pgfqpoint{5.468733in}{1.422774in}}%
\pgfpathcurveto{\pgfqpoint{5.473776in}{1.417730in}}{\pgfqpoint{5.480618in}{1.414897in}}{\pgfqpoint{5.487751in}{1.414897in}}%
\pgfpathclose%
\pgfusepath{stroke,fill}%
\end{pgfscope}%
\begin{pgfscope}%
\pgfpathrectangle{\pgfqpoint{4.985294in}{0.500000in}}{\pgfqpoint{1.764706in}{1.700000in}}%
\pgfusepath{clip}%
\pgfsetbuttcap%
\pgfsetroundjoin%
\definecolor{currentfill}{rgb}{0.838502,0.140251,0.287688}%
\pgfsetfillcolor{currentfill}%
\pgfsetlinewidth{0.311001pt}%
\definecolor{currentstroke}{rgb}{1.000000,1.000000,1.000000}%
\pgfsetstrokecolor{currentstroke}%
\pgfsetdash{}{0pt}%
\pgfpathmoveto{\pgfqpoint{5.973847in}{1.797639in}}%
\pgfpathcurveto{\pgfqpoint{5.980980in}{1.797639in}}{\pgfqpoint{5.987821in}{1.800473in}}{\pgfqpoint{5.992865in}{1.805516in}}%
\pgfpathcurveto{\pgfqpoint{5.997909in}{1.810560in}}{\pgfqpoint{6.000743in}{1.817402in}}{\pgfqpoint{6.000743in}{1.824535in}}%
\pgfpathcurveto{\pgfqpoint{6.000743in}{1.831667in}}{\pgfqpoint{5.997909in}{1.838509in}}{\pgfqpoint{5.992865in}{1.843553in}}%
\pgfpathcurveto{\pgfqpoint{5.987821in}{1.848596in}}{\pgfqpoint{5.980980in}{1.851430in}}{\pgfqpoint{5.973847in}{1.851430in}}%
\pgfpathcurveto{\pgfqpoint{5.966714in}{1.851430in}}{\pgfqpoint{5.959873in}{1.848596in}}{\pgfqpoint{5.954829in}{1.843553in}}%
\pgfpathcurveto{\pgfqpoint{5.949785in}{1.838509in}}{\pgfqpoint{5.946951in}{1.831667in}}{\pgfqpoint{5.946951in}{1.824535in}}%
\pgfpathcurveto{\pgfqpoint{5.946951in}{1.817402in}}{\pgfqpoint{5.949785in}{1.810560in}}{\pgfqpoint{5.954829in}{1.805516in}}%
\pgfpathcurveto{\pgfqpoint{5.959873in}{1.800473in}}{\pgfqpoint{5.966714in}{1.797639in}}{\pgfqpoint{5.973847in}{1.797639in}}%
\pgfpathclose%
\pgfusepath{stroke,fill}%
\end{pgfscope}%
\begin{pgfscope}%
\pgfpathrectangle{\pgfqpoint{4.985294in}{0.500000in}}{\pgfqpoint{1.764706in}{1.700000in}}%
\pgfusepath{clip}%
\pgfsetbuttcap%
\pgfsetroundjoin%
\definecolor{currentfill}{rgb}{0.975018,0.868213,0.788710}%
\pgfsetfillcolor{currentfill}%
\pgfsetlinewidth{0.311001pt}%
\definecolor{currentstroke}{rgb}{1.000000,1.000000,1.000000}%
\pgfsetstrokecolor{currentstroke}%
\pgfsetdash{}{0pt}%
\pgfpathmoveto{\pgfqpoint{5.482502in}{1.542079in}}%
\pgfpathcurveto{\pgfqpoint{5.489635in}{1.542079in}}{\pgfqpoint{5.496476in}{1.544912in}}{\pgfqpoint{5.501520in}{1.549956in}}%
\pgfpathcurveto{\pgfqpoint{5.506564in}{1.555000in}}{\pgfqpoint{5.509398in}{1.561841in}}{\pgfqpoint{5.509398in}{1.568974in}}%
\pgfpathcurveto{\pgfqpoint{5.509398in}{1.576107in}}{\pgfqpoint{5.506564in}{1.582949in}}{\pgfqpoint{5.501520in}{1.587992in}}%
\pgfpathcurveto{\pgfqpoint{5.496476in}{1.593036in}}{\pgfqpoint{5.489635in}{1.595870in}}{\pgfqpoint{5.482502in}{1.595870in}}%
\pgfpathcurveto{\pgfqpoint{5.475369in}{1.595870in}}{\pgfqpoint{5.468527in}{1.593036in}}{\pgfqpoint{5.463484in}{1.587992in}}%
\pgfpathcurveto{\pgfqpoint{5.458440in}{1.582949in}}{\pgfqpoint{5.455606in}{1.576107in}}{\pgfqpoint{5.455606in}{1.568974in}}%
\pgfpathcurveto{\pgfqpoint{5.455606in}{1.561841in}}{\pgfqpoint{5.458440in}{1.555000in}}{\pgfqpoint{5.463484in}{1.549956in}}%
\pgfpathcurveto{\pgfqpoint{5.468527in}{1.544912in}}{\pgfqpoint{5.475369in}{1.542079in}}{\pgfqpoint{5.482502in}{1.542079in}}%
\pgfpathclose%
\pgfusepath{stroke,fill}%
\end{pgfscope}%
\begin{pgfscope}%
\pgfpathrectangle{\pgfqpoint{4.985294in}{0.500000in}}{\pgfqpoint{1.764706in}{1.700000in}}%
\pgfusepath{clip}%
\pgfsetbuttcap%
\pgfsetroundjoin%
\definecolor{currentfill}{rgb}{0.962985,0.612625,0.451451}%
\pgfsetfillcolor{currentfill}%
\pgfsetlinewidth{0.311001pt}%
\definecolor{currentstroke}{rgb}{1.000000,1.000000,1.000000}%
\pgfsetstrokecolor{currentstroke}%
\pgfsetdash{}{0pt}%
\pgfpathmoveto{\pgfqpoint{5.651077in}{0.927820in}}%
\pgfpathcurveto{\pgfqpoint{5.658210in}{0.927820in}}{\pgfqpoint{5.665052in}{0.930653in}}{\pgfqpoint{5.670095in}{0.935697in}}%
\pgfpathcurveto{\pgfqpoint{5.675139in}{0.940741in}}{\pgfqpoint{5.677973in}{0.947582in}}{\pgfqpoint{5.677973in}{0.954715in}}%
\pgfpathcurveto{\pgfqpoint{5.677973in}{0.961848in}}{\pgfqpoint{5.675139in}{0.968690in}}{\pgfqpoint{5.670095in}{0.973733in}}%
\pgfpathcurveto{\pgfqpoint{5.665052in}{0.978777in}}{\pgfqpoint{5.658210in}{0.981611in}}{\pgfqpoint{5.651077in}{0.981611in}}%
\pgfpathcurveto{\pgfqpoint{5.643944in}{0.981611in}}{\pgfqpoint{5.637103in}{0.978777in}}{\pgfqpoint{5.632059in}{0.973733in}}%
\pgfpathcurveto{\pgfqpoint{5.627015in}{0.968690in}}{\pgfqpoint{5.624181in}{0.961848in}}{\pgfqpoint{5.624181in}{0.954715in}}%
\pgfpathcurveto{\pgfqpoint{5.624181in}{0.947582in}}{\pgfqpoint{5.627015in}{0.940741in}}{\pgfqpoint{5.632059in}{0.935697in}}%
\pgfpathcurveto{\pgfqpoint{5.637103in}{0.930653in}}{\pgfqpoint{5.643944in}{0.927820in}}{\pgfqpoint{5.651077in}{0.927820in}}%
\pgfpathclose%
\pgfusepath{stroke,fill}%
\end{pgfscope}%
\begin{pgfscope}%
\pgfpathrectangle{\pgfqpoint{4.985294in}{0.500000in}}{\pgfqpoint{1.764706in}{1.700000in}}%
\pgfusepath{clip}%
\pgfsetbuttcap%
\pgfsetroundjoin%
\definecolor{currentfill}{rgb}{0.980678,0.914765,0.856766}%
\pgfsetfillcolor{currentfill}%
\pgfsetlinewidth{0.311001pt}%
\definecolor{currentstroke}{rgb}{1.000000,1.000000,1.000000}%
\pgfsetstrokecolor{currentstroke}%
\pgfsetdash{}{0pt}%
\pgfpathmoveto{\pgfqpoint{6.299524in}{1.363669in}}%
\pgfpathcurveto{\pgfqpoint{6.306657in}{1.363669in}}{\pgfqpoint{6.313499in}{1.366503in}}{\pgfqpoint{6.318543in}{1.371546in}}%
\pgfpathcurveto{\pgfqpoint{6.323586in}{1.376590in}}{\pgfqpoint{6.326420in}{1.383432in}}{\pgfqpoint{6.326420in}{1.390565in}}%
\pgfpathcurveto{\pgfqpoint{6.326420in}{1.397697in}}{\pgfqpoint{6.323586in}{1.404539in}}{\pgfqpoint{6.318543in}{1.409583in}}%
\pgfpathcurveto{\pgfqpoint{6.313499in}{1.414626in}}{\pgfqpoint{6.306657in}{1.417460in}}{\pgfqpoint{6.299524in}{1.417460in}}%
\pgfpathcurveto{\pgfqpoint{6.292392in}{1.417460in}}{\pgfqpoint{6.285550in}{1.414626in}}{\pgfqpoint{6.280506in}{1.409583in}}%
\pgfpathcurveto{\pgfqpoint{6.275463in}{1.404539in}}{\pgfqpoint{6.272629in}{1.397697in}}{\pgfqpoint{6.272629in}{1.390565in}}%
\pgfpathcurveto{\pgfqpoint{6.272629in}{1.383432in}}{\pgfqpoint{6.275463in}{1.376590in}}{\pgfqpoint{6.280506in}{1.371546in}}%
\pgfpathcurveto{\pgfqpoint{6.285550in}{1.366503in}}{\pgfqpoint{6.292392in}{1.363669in}}{\pgfqpoint{6.299524in}{1.363669in}}%
\pgfpathclose%
\pgfusepath{stroke,fill}%
\end{pgfscope}%
\begin{pgfscope}%
\pgfpathrectangle{\pgfqpoint{4.985294in}{0.500000in}}{\pgfqpoint{1.764706in}{1.700000in}}%
\pgfusepath{clip}%
\pgfsetbuttcap%
\pgfsetroundjoin%
\definecolor{currentfill}{rgb}{0.964173,0.657587,0.500469}%
\pgfsetfillcolor{currentfill}%
\pgfsetlinewidth{0.311001pt}%
\definecolor{currentstroke}{rgb}{1.000000,1.000000,1.000000}%
\pgfsetstrokecolor{currentstroke}%
\pgfsetdash{}{0pt}%
\pgfpathmoveto{\pgfqpoint{6.187546in}{1.332222in}}%
\pgfpathcurveto{\pgfqpoint{6.194678in}{1.332222in}}{\pgfqpoint{6.201520in}{1.335056in}}{\pgfqpoint{6.206564in}{1.340100in}}%
\pgfpathcurveto{\pgfqpoint{6.211607in}{1.345144in}}{\pgfqpoint{6.214441in}{1.351985in}}{\pgfqpoint{6.214441in}{1.359118in}}%
\pgfpathcurveto{\pgfqpoint{6.214441in}{1.366251in}}{\pgfqpoint{6.211607in}{1.373093in}}{\pgfqpoint{6.206564in}{1.378136in}}%
\pgfpathcurveto{\pgfqpoint{6.201520in}{1.383180in}}{\pgfqpoint{6.194678in}{1.386014in}}{\pgfqpoint{6.187546in}{1.386014in}}%
\pgfpathcurveto{\pgfqpoint{6.180413in}{1.386014in}}{\pgfqpoint{6.173571in}{1.383180in}}{\pgfqpoint{6.168527in}{1.378136in}}%
\pgfpathcurveto{\pgfqpoint{6.163484in}{1.373093in}}{\pgfqpoint{6.160650in}{1.366251in}}{\pgfqpoint{6.160650in}{1.359118in}}%
\pgfpathcurveto{\pgfqpoint{6.160650in}{1.351985in}}{\pgfqpoint{6.163484in}{1.345144in}}{\pgfqpoint{6.168527in}{1.340100in}}%
\pgfpathcurveto{\pgfqpoint{6.173571in}{1.335056in}}{\pgfqpoint{6.180413in}{1.332222in}}{\pgfqpoint{6.187546in}{1.332222in}}%
\pgfpathclose%
\pgfusepath{stroke,fill}%
\end{pgfscope}%
\begin{pgfscope}%
\pgfpathrectangle{\pgfqpoint{4.985294in}{0.500000in}}{\pgfqpoint{1.764706in}{1.700000in}}%
\pgfusepath{clip}%
\pgfsetbuttcap%
\pgfsetroundjoin%
\definecolor{currentfill}{rgb}{0.979124,0.903132,0.839793}%
\pgfsetfillcolor{currentfill}%
\pgfsetlinewidth{0.311001pt}%
\definecolor{currentstroke}{rgb}{1.000000,1.000000,1.000000}%
\pgfsetstrokecolor{currentstroke}%
\pgfsetdash{}{0pt}%
\pgfpathmoveto{\pgfqpoint{5.422997in}{1.450219in}}%
\pgfpathcurveto{\pgfqpoint{5.430130in}{1.450219in}}{\pgfqpoint{5.436971in}{1.453052in}}{\pgfqpoint{5.442015in}{1.458096in}}%
\pgfpathcurveto{\pgfqpoint{5.447059in}{1.463140in}}{\pgfqpoint{5.449893in}{1.469981in}}{\pgfqpoint{5.449893in}{1.477114in}}%
\pgfpathcurveto{\pgfqpoint{5.449893in}{1.484247in}}{\pgfqpoint{5.447059in}{1.491089in}}{\pgfqpoint{5.442015in}{1.496132in}}%
\pgfpathcurveto{\pgfqpoint{5.436971in}{1.501176in}}{\pgfqpoint{5.430130in}{1.504010in}}{\pgfqpoint{5.422997in}{1.504010in}}%
\pgfpathcurveto{\pgfqpoint{5.415864in}{1.504010in}}{\pgfqpoint{5.409022in}{1.501176in}}{\pgfqpoint{5.403979in}{1.496132in}}%
\pgfpathcurveto{\pgfqpoint{5.398935in}{1.491089in}}{\pgfqpoint{5.396101in}{1.484247in}}{\pgfqpoint{5.396101in}{1.477114in}}%
\pgfpathcurveto{\pgfqpoint{5.396101in}{1.469981in}}{\pgfqpoint{5.398935in}{1.463140in}}{\pgfqpoint{5.403979in}{1.458096in}}%
\pgfpathcurveto{\pgfqpoint{5.409022in}{1.453052in}}{\pgfqpoint{5.415864in}{1.450219in}}{\pgfqpoint{5.422997in}{1.450219in}}%
\pgfpathclose%
\pgfusepath{stroke,fill}%
\end{pgfscope}%
\begin{pgfscope}%
\pgfpathrectangle{\pgfqpoint{4.985294in}{0.500000in}}{\pgfqpoint{1.764706in}{1.700000in}}%
\pgfusepath{clip}%
\pgfsetbuttcap%
\pgfsetroundjoin%
\definecolor{currentfill}{rgb}{0.966812,0.762584,0.633643}%
\pgfsetfillcolor{currentfill}%
\pgfsetlinewidth{0.311001pt}%
\definecolor{currentstroke}{rgb}{1.000000,1.000000,1.000000}%
\pgfsetstrokecolor{currentstroke}%
\pgfsetdash{}{0pt}%
\pgfpathmoveto{\pgfqpoint{5.359973in}{1.155646in}}%
\pgfpathcurveto{\pgfqpoint{5.367106in}{1.155646in}}{\pgfqpoint{5.373948in}{1.158480in}}{\pgfqpoint{5.378992in}{1.163523in}}%
\pgfpathcurveto{\pgfqpoint{5.384035in}{1.168567in}}{\pgfqpoint{5.386869in}{1.175409in}}{\pgfqpoint{5.386869in}{1.182541in}}%
\pgfpathcurveto{\pgfqpoint{5.386869in}{1.189674in}}{\pgfqpoint{5.384035in}{1.196516in}}{\pgfqpoint{5.378992in}{1.201559in}}%
\pgfpathcurveto{\pgfqpoint{5.373948in}{1.206603in}}{\pgfqpoint{5.367106in}{1.209437in}}{\pgfqpoint{5.359973in}{1.209437in}}%
\pgfpathcurveto{\pgfqpoint{5.352841in}{1.209437in}}{\pgfqpoint{5.345999in}{1.206603in}}{\pgfqpoint{5.340955in}{1.201559in}}%
\pgfpathcurveto{\pgfqpoint{5.335912in}{1.196516in}}{\pgfqpoint{5.333078in}{1.189674in}}{\pgfqpoint{5.333078in}{1.182541in}}%
\pgfpathcurveto{\pgfqpoint{5.333078in}{1.175409in}}{\pgfqpoint{5.335912in}{1.168567in}}{\pgfqpoint{5.340955in}{1.163523in}}%
\pgfpathcurveto{\pgfqpoint{5.345999in}{1.158480in}}{\pgfqpoint{5.352841in}{1.155646in}}{\pgfqpoint{5.359973in}{1.155646in}}%
\pgfpathclose%
\pgfusepath{stroke,fill}%
\end{pgfscope}%
\begin{pgfscope}%
\pgfpathrectangle{\pgfqpoint{4.985294in}{0.500000in}}{\pgfqpoint{1.764706in}{1.700000in}}%
\pgfusepath{clip}%
\pgfsetbuttcap%
\pgfsetroundjoin%
\definecolor{currentfill}{rgb}{0.979124,0.903132,0.839793}%
\pgfsetfillcolor{currentfill}%
\pgfsetlinewidth{0.311001pt}%
\definecolor{currentstroke}{rgb}{1.000000,1.000000,1.000000}%
\pgfsetstrokecolor{currentstroke}%
\pgfsetdash{}{0pt}%
\pgfpathmoveto{\pgfqpoint{6.285865in}{1.293397in}}%
\pgfpathcurveto{\pgfqpoint{6.292997in}{1.293397in}}{\pgfqpoint{6.299839in}{1.296231in}}{\pgfqpoint{6.304883in}{1.301274in}}%
\pgfpathcurveto{\pgfqpoint{6.309926in}{1.306318in}}{\pgfqpoint{6.312760in}{1.313160in}}{\pgfqpoint{6.312760in}{1.320292in}}%
\pgfpathcurveto{\pgfqpoint{6.312760in}{1.327425in}}{\pgfqpoint{6.309926in}{1.334267in}}{\pgfqpoint{6.304883in}{1.339311in}}%
\pgfpathcurveto{\pgfqpoint{6.299839in}{1.344354in}}{\pgfqpoint{6.292997in}{1.347188in}}{\pgfqpoint{6.285865in}{1.347188in}}%
\pgfpathcurveto{\pgfqpoint{6.278732in}{1.347188in}}{\pgfqpoint{6.271890in}{1.344354in}}{\pgfqpoint{6.266846in}{1.339311in}}%
\pgfpathcurveto{\pgfqpoint{6.261803in}{1.334267in}}{\pgfqpoint{6.258969in}{1.327425in}}{\pgfqpoint{6.258969in}{1.320292in}}%
\pgfpathcurveto{\pgfqpoint{6.258969in}{1.313160in}}{\pgfqpoint{6.261803in}{1.306318in}}{\pgfqpoint{6.266846in}{1.301274in}}%
\pgfpathcurveto{\pgfqpoint{6.271890in}{1.296231in}}{\pgfqpoint{6.278732in}{1.293397in}}{\pgfqpoint{6.285865in}{1.293397in}}%
\pgfpathclose%
\pgfusepath{stroke,fill}%
\end{pgfscope}%
\begin{pgfscope}%
\pgfpathrectangle{\pgfqpoint{4.985294in}{0.500000in}}{\pgfqpoint{1.764706in}{1.700000in}}%
\pgfusepath{clip}%
\pgfsetbuttcap%
\pgfsetroundjoin%
\definecolor{currentfill}{rgb}{0.968931,0.798091,0.685123}%
\pgfsetfillcolor{currentfill}%
\pgfsetlinewidth{0.311001pt}%
\definecolor{currentstroke}{rgb}{1.000000,1.000000,1.000000}%
\pgfsetstrokecolor{currentstroke}%
\pgfsetdash{}{0pt}%
\pgfpathmoveto{\pgfqpoint{5.407819in}{1.567592in}}%
\pgfpathcurveto{\pgfqpoint{5.414952in}{1.567592in}}{\pgfqpoint{5.421794in}{1.570426in}}{\pgfqpoint{5.426837in}{1.575469in}}%
\pgfpathcurveto{\pgfqpoint{5.431881in}{1.580513in}}{\pgfqpoint{5.434715in}{1.587355in}}{\pgfqpoint{5.434715in}{1.594488in}}%
\pgfpathcurveto{\pgfqpoint{5.434715in}{1.601620in}}{\pgfqpoint{5.431881in}{1.608462in}}{\pgfqpoint{5.426837in}{1.613506in}}%
\pgfpathcurveto{\pgfqpoint{5.421794in}{1.618549in}}{\pgfqpoint{5.414952in}{1.621383in}}{\pgfqpoint{5.407819in}{1.621383in}}%
\pgfpathcurveto{\pgfqpoint{5.400686in}{1.621383in}}{\pgfqpoint{5.393845in}{1.618549in}}{\pgfqpoint{5.388801in}{1.613506in}}%
\pgfpathcurveto{\pgfqpoint{5.383757in}{1.608462in}}{\pgfqpoint{5.380924in}{1.601620in}}{\pgfqpoint{5.380924in}{1.594488in}}%
\pgfpathcurveto{\pgfqpoint{5.380924in}{1.587355in}}{\pgfqpoint{5.383757in}{1.580513in}}{\pgfqpoint{5.388801in}{1.575469in}}%
\pgfpathcurveto{\pgfqpoint{5.393845in}{1.570426in}}{\pgfqpoint{5.400686in}{1.567592in}}{\pgfqpoint{5.407819in}{1.567592in}}%
\pgfpathclose%
\pgfusepath{stroke,fill}%
\end{pgfscope}%
\begin{pgfscope}%
\pgfpathrectangle{\pgfqpoint{4.985294in}{0.500000in}}{\pgfqpoint{1.764706in}{1.700000in}}%
\pgfusepath{clip}%
\pgfsetbuttcap%
\pgfsetroundjoin%
\definecolor{currentfill}{rgb}{0.961734,0.579886,0.418445}%
\pgfsetfillcolor{currentfill}%
\pgfsetlinewidth{0.311001pt}%
\definecolor{currentstroke}{rgb}{1.000000,1.000000,1.000000}%
\pgfsetstrokecolor{currentstroke}%
\pgfsetdash{}{0pt}%
\pgfpathmoveto{\pgfqpoint{6.407830in}{1.499563in}}%
\pgfpathcurveto{\pgfqpoint{6.414963in}{1.499563in}}{\pgfqpoint{6.421805in}{1.502397in}}{\pgfqpoint{6.426848in}{1.507441in}}%
\pgfpathcurveto{\pgfqpoint{6.431892in}{1.512484in}}{\pgfqpoint{6.434726in}{1.519326in}}{\pgfqpoint{6.434726in}{1.526459in}}%
\pgfpathcurveto{\pgfqpoint{6.434726in}{1.533592in}}{\pgfqpoint{6.431892in}{1.540433in}}{\pgfqpoint{6.426848in}{1.545477in}}%
\pgfpathcurveto{\pgfqpoint{6.421805in}{1.550521in}}{\pgfqpoint{6.414963in}{1.553355in}}{\pgfqpoint{6.407830in}{1.553355in}}%
\pgfpathcurveto{\pgfqpoint{6.400697in}{1.553355in}}{\pgfqpoint{6.393856in}{1.550521in}}{\pgfqpoint{6.388812in}{1.545477in}}%
\pgfpathcurveto{\pgfqpoint{6.383768in}{1.540433in}}{\pgfqpoint{6.380934in}{1.533592in}}{\pgfqpoint{6.380934in}{1.526459in}}%
\pgfpathcurveto{\pgfqpoint{6.380934in}{1.519326in}}{\pgfqpoint{6.383768in}{1.512484in}}{\pgfqpoint{6.388812in}{1.507441in}}%
\pgfpathcurveto{\pgfqpoint{6.393856in}{1.502397in}}{\pgfqpoint{6.400697in}{1.499563in}}{\pgfqpoint{6.407830in}{1.499563in}}%
\pgfpathclose%
\pgfusepath{stroke,fill}%
\end{pgfscope}%
\begin{pgfscope}%
\pgfpathrectangle{\pgfqpoint{4.985294in}{0.500000in}}{\pgfqpoint{1.764706in}{1.700000in}}%
\pgfusepath{clip}%
\pgfsetbuttcap%
\pgfsetroundjoin%
\definecolor{currentfill}{rgb}{0.957344,0.505732,0.351309}%
\pgfsetfillcolor{currentfill}%
\pgfsetlinewidth{0.311001pt}%
\definecolor{currentstroke}{rgb}{1.000000,1.000000,1.000000}%
\pgfsetstrokecolor{currentstroke}%
\pgfsetdash{}{0pt}%
\pgfpathmoveto{\pgfqpoint{6.137926in}{1.478902in}}%
\pgfpathcurveto{\pgfqpoint{6.145059in}{1.478902in}}{\pgfqpoint{6.151901in}{1.481736in}}{\pgfqpoint{6.156945in}{1.486780in}}%
\pgfpathcurveto{\pgfqpoint{6.161988in}{1.491824in}}{\pgfqpoint{6.164822in}{1.498665in}}{\pgfqpoint{6.164822in}{1.505798in}}%
\pgfpathcurveto{\pgfqpoint{6.164822in}{1.512931in}}{\pgfqpoint{6.161988in}{1.519772in}}{\pgfqpoint{6.156945in}{1.524816in}}%
\pgfpathcurveto{\pgfqpoint{6.151901in}{1.529860in}}{\pgfqpoint{6.145059in}{1.532694in}}{\pgfqpoint{6.137926in}{1.532694in}}%
\pgfpathcurveto{\pgfqpoint{6.130794in}{1.532694in}}{\pgfqpoint{6.123952in}{1.529860in}}{\pgfqpoint{6.118908in}{1.524816in}}%
\pgfpathcurveto{\pgfqpoint{6.113865in}{1.519772in}}{\pgfqpoint{6.111031in}{1.512931in}}{\pgfqpoint{6.111031in}{1.505798in}}%
\pgfpathcurveto{\pgfqpoint{6.111031in}{1.498665in}}{\pgfqpoint{6.113865in}{1.491824in}}{\pgfqpoint{6.118908in}{1.486780in}}%
\pgfpathcurveto{\pgfqpoint{6.123952in}{1.481736in}}{\pgfqpoint{6.130794in}{1.478902in}}{\pgfqpoint{6.137926in}{1.478902in}}%
\pgfpathclose%
\pgfusepath{stroke,fill}%
\end{pgfscope}%
\begin{pgfscope}%
\pgfpathrectangle{\pgfqpoint{4.985294in}{0.500000in}}{\pgfqpoint{1.764706in}{1.700000in}}%
\pgfusepath{clip}%
\pgfsetbuttcap%
\pgfsetroundjoin%
\definecolor{currentfill}{rgb}{0.976961,0.885681,0.814303}%
\pgfsetfillcolor{currentfill}%
\pgfsetlinewidth{0.311001pt}%
\definecolor{currentstroke}{rgb}{1.000000,1.000000,1.000000}%
\pgfsetstrokecolor{currentstroke}%
\pgfsetdash{}{0pt}%
\pgfpathmoveto{\pgfqpoint{6.354026in}{1.344642in}}%
\pgfpathcurveto{\pgfqpoint{6.361158in}{1.344642in}}{\pgfqpoint{6.368000in}{1.347476in}}{\pgfqpoint{6.373044in}{1.352520in}}%
\pgfpathcurveto{\pgfqpoint{6.378087in}{1.357563in}}{\pgfqpoint{6.380921in}{1.364405in}}{\pgfqpoint{6.380921in}{1.371538in}}%
\pgfpathcurveto{\pgfqpoint{6.380921in}{1.378670in}}{\pgfqpoint{6.378087in}{1.385512in}}{\pgfqpoint{6.373044in}{1.390556in}}%
\pgfpathcurveto{\pgfqpoint{6.368000in}{1.395599in}}{\pgfqpoint{6.361158in}{1.398433in}}{\pgfqpoint{6.354026in}{1.398433in}}%
\pgfpathcurveto{\pgfqpoint{6.346893in}{1.398433in}}{\pgfqpoint{6.340051in}{1.395599in}}{\pgfqpoint{6.335007in}{1.390556in}}%
\pgfpathcurveto{\pgfqpoint{6.329964in}{1.385512in}}{\pgfqpoint{6.327130in}{1.378670in}}{\pgfqpoint{6.327130in}{1.371538in}}%
\pgfpathcurveto{\pgfqpoint{6.327130in}{1.364405in}}{\pgfqpoint{6.329964in}{1.357563in}}{\pgfqpoint{6.335007in}{1.352520in}}%
\pgfpathcurveto{\pgfqpoint{6.340051in}{1.347476in}}{\pgfqpoint{6.346893in}{1.344642in}}{\pgfqpoint{6.354026in}{1.344642in}}%
\pgfpathclose%
\pgfusepath{stroke,fill}%
\end{pgfscope}%
\begin{pgfscope}%
\pgfpathrectangle{\pgfqpoint{4.985294in}{0.500000in}}{\pgfqpoint{1.764706in}{1.700000in}}%
\pgfusepath{clip}%
\pgfsetbuttcap%
\pgfsetroundjoin%
\definecolor{currentfill}{rgb}{0.964433,0.670254,0.515093}%
\pgfsetfillcolor{currentfill}%
\pgfsetlinewidth{0.311001pt}%
\definecolor{currentstroke}{rgb}{1.000000,1.000000,1.000000}%
\pgfsetstrokecolor{currentstroke}%
\pgfsetdash{}{0pt}%
\pgfpathmoveto{\pgfqpoint{5.550565in}{1.737511in}}%
\pgfpathcurveto{\pgfqpoint{5.557698in}{1.737511in}}{\pgfqpoint{5.564539in}{1.740345in}}{\pgfqpoint{5.569583in}{1.745388in}}%
\pgfpathcurveto{\pgfqpoint{5.574627in}{1.750432in}}{\pgfqpoint{5.577460in}{1.757274in}}{\pgfqpoint{5.577460in}{1.764406in}}%
\pgfpathcurveto{\pgfqpoint{5.577460in}{1.771539in}}{\pgfqpoint{5.574627in}{1.778381in}}{\pgfqpoint{5.569583in}{1.783425in}}%
\pgfpathcurveto{\pgfqpoint{5.564539in}{1.788468in}}{\pgfqpoint{5.557698in}{1.791302in}}{\pgfqpoint{5.550565in}{1.791302in}}%
\pgfpathcurveto{\pgfqpoint{5.543432in}{1.791302in}}{\pgfqpoint{5.536590in}{1.788468in}}{\pgfqpoint{5.531547in}{1.783425in}}%
\pgfpathcurveto{\pgfqpoint{5.526503in}{1.778381in}}{\pgfqpoint{5.523669in}{1.771539in}}{\pgfqpoint{5.523669in}{1.764406in}}%
\pgfpathcurveto{\pgfqpoint{5.523669in}{1.757274in}}{\pgfqpoint{5.526503in}{1.750432in}}{\pgfqpoint{5.531547in}{1.745388in}}%
\pgfpathcurveto{\pgfqpoint{5.536590in}{1.740345in}}{\pgfqpoint{5.543432in}{1.737511in}}{\pgfqpoint{5.550565in}{1.737511in}}%
\pgfpathclose%
\pgfusepath{stroke,fill}%
\end{pgfscope}%
\begin{pgfscope}%
\pgfpathrectangle{\pgfqpoint{4.985294in}{0.500000in}}{\pgfqpoint{1.764706in}{1.700000in}}%
\pgfusepath{clip}%
\pgfsetbuttcap%
\pgfsetroundjoin%
\definecolor{currentfill}{rgb}{0.974412,0.862387,0.780156}%
\pgfsetfillcolor{currentfill}%
\pgfsetlinewidth{0.311001pt}%
\definecolor{currentstroke}{rgb}{1.000000,1.000000,1.000000}%
\pgfsetstrokecolor{currentstroke}%
\pgfsetdash{}{0pt}%
\pgfpathmoveto{\pgfqpoint{6.211640in}{1.623495in}}%
\pgfpathcurveto{\pgfqpoint{6.218773in}{1.623495in}}{\pgfqpoint{6.225614in}{1.626329in}}{\pgfqpoint{6.230658in}{1.631373in}}%
\pgfpathcurveto{\pgfqpoint{6.235701in}{1.636416in}}{\pgfqpoint{6.238535in}{1.643258in}}{\pgfqpoint{6.238535in}{1.650391in}}%
\pgfpathcurveto{\pgfqpoint{6.238535in}{1.657524in}}{\pgfqpoint{6.235701in}{1.664365in}}{\pgfqpoint{6.230658in}{1.669409in}}%
\pgfpathcurveto{\pgfqpoint{6.225614in}{1.674453in}}{\pgfqpoint{6.218773in}{1.677287in}}{\pgfqpoint{6.211640in}{1.677287in}}%
\pgfpathcurveto{\pgfqpoint{6.204507in}{1.677287in}}{\pgfqpoint{6.197665in}{1.674453in}}{\pgfqpoint{6.192622in}{1.669409in}}%
\pgfpathcurveto{\pgfqpoint{6.187578in}{1.664365in}}{\pgfqpoint{6.184744in}{1.657524in}}{\pgfqpoint{6.184744in}{1.650391in}}%
\pgfpathcurveto{\pgfqpoint{6.184744in}{1.643258in}}{\pgfqpoint{6.187578in}{1.636416in}}{\pgfqpoint{6.192622in}{1.631373in}}%
\pgfpathcurveto{\pgfqpoint{6.197665in}{1.626329in}}{\pgfqpoint{6.204507in}{1.623495in}}{\pgfqpoint{6.211640in}{1.623495in}}%
\pgfpathclose%
\pgfusepath{stroke,fill}%
\end{pgfscope}%
\begin{pgfscope}%
\pgfpathrectangle{\pgfqpoint{4.985294in}{0.500000in}}{\pgfqpoint{1.764706in}{1.700000in}}%
\pgfusepath{clip}%
\pgfsetbuttcap%
\pgfsetroundjoin%
\definecolor{currentfill}{rgb}{0.779326,0.096348,0.318766}%
\pgfsetfillcolor{currentfill}%
\pgfsetlinewidth{0.311001pt}%
\definecolor{currentstroke}{rgb}{1.000000,1.000000,1.000000}%
\pgfsetstrokecolor{currentstroke}%
\pgfsetdash{}{0pt}%
\pgfpathmoveto{\pgfqpoint{5.922006in}{1.822965in}}%
\pgfpathcurveto{\pgfqpoint{5.929139in}{1.822965in}}{\pgfqpoint{5.935980in}{1.825799in}}{\pgfqpoint{5.941024in}{1.830843in}}%
\pgfpathcurveto{\pgfqpoint{5.946068in}{1.835886in}}{\pgfqpoint{5.948902in}{1.842728in}}{\pgfqpoint{5.948902in}{1.849861in}}%
\pgfpathcurveto{\pgfqpoint{5.948902in}{1.856994in}}{\pgfqpoint{5.946068in}{1.863835in}}{\pgfqpoint{5.941024in}{1.868879in}}%
\pgfpathcurveto{\pgfqpoint{5.935980in}{1.873923in}}{\pgfqpoint{5.929139in}{1.876756in}}{\pgfqpoint{5.922006in}{1.876756in}}%
\pgfpathcurveto{\pgfqpoint{5.914873in}{1.876756in}}{\pgfqpoint{5.908031in}{1.873923in}}{\pgfqpoint{5.902988in}{1.868879in}}%
\pgfpathcurveto{\pgfqpoint{5.897944in}{1.863835in}}{\pgfqpoint{5.895110in}{1.856994in}}{\pgfqpoint{5.895110in}{1.849861in}}%
\pgfpathcurveto{\pgfqpoint{5.895110in}{1.842728in}}{\pgfqpoint{5.897944in}{1.835886in}}{\pgfqpoint{5.902988in}{1.830843in}}%
\pgfpathcurveto{\pgfqpoint{5.908031in}{1.825799in}}{\pgfqpoint{5.914873in}{1.822965in}}{\pgfqpoint{5.922006in}{1.822965in}}%
\pgfpathclose%
\pgfusepath{stroke,fill}%
\end{pgfscope}%
\begin{pgfscope}%
\pgfpathrectangle{\pgfqpoint{4.985294in}{0.500000in}}{\pgfqpoint{1.764706in}{1.700000in}}%
\pgfusepath{clip}%
\pgfsetbuttcap%
\pgfsetroundjoin%
\definecolor{currentfill}{rgb}{0.970255,0.815666,0.711203}%
\pgfsetfillcolor{currentfill}%
\pgfsetlinewidth{0.311001pt}%
\definecolor{currentstroke}{rgb}{1.000000,1.000000,1.000000}%
\pgfsetstrokecolor{currentstroke}%
\pgfsetdash{}{0pt}%
\pgfpathmoveto{\pgfqpoint{5.502554in}{1.171575in}}%
\pgfpathcurveto{\pgfqpoint{5.509687in}{1.171575in}}{\pgfqpoint{5.516529in}{1.174409in}}{\pgfqpoint{5.521573in}{1.179453in}}%
\pgfpathcurveto{\pgfqpoint{5.526616in}{1.184496in}}{\pgfqpoint{5.529450in}{1.191338in}}{\pgfqpoint{5.529450in}{1.198471in}}%
\pgfpathcurveto{\pgfqpoint{5.529450in}{1.205604in}}{\pgfqpoint{5.526616in}{1.212445in}}{\pgfqpoint{5.521573in}{1.217489in}}%
\pgfpathcurveto{\pgfqpoint{5.516529in}{1.222533in}}{\pgfqpoint{5.509687in}{1.225366in}}{\pgfqpoint{5.502554in}{1.225366in}}%
\pgfpathcurveto{\pgfqpoint{5.495422in}{1.225366in}}{\pgfqpoint{5.488580in}{1.222533in}}{\pgfqpoint{5.483536in}{1.217489in}}%
\pgfpathcurveto{\pgfqpoint{5.478493in}{1.212445in}}{\pgfqpoint{5.475659in}{1.205604in}}{\pgfqpoint{5.475659in}{1.198471in}}%
\pgfpathcurveto{\pgfqpoint{5.475659in}{1.191338in}}{\pgfqpoint{5.478493in}{1.184496in}}{\pgfqpoint{5.483536in}{1.179453in}}%
\pgfpathcurveto{\pgfqpoint{5.488580in}{1.174409in}}{\pgfqpoint{5.495422in}{1.171575in}}{\pgfqpoint{5.502554in}{1.171575in}}%
\pgfpathclose%
\pgfusepath{stroke,fill}%
\end{pgfscope}%
\begin{pgfscope}%
\pgfpathrectangle{\pgfqpoint{4.985294in}{0.500000in}}{\pgfqpoint{1.764706in}{1.700000in}}%
\pgfusepath{clip}%
\pgfsetbuttcap%
\pgfsetroundjoin%
\definecolor{currentfill}{rgb}{0.965753,0.732351,0.592427}%
\pgfsetfillcolor{currentfill}%
\pgfsetlinewidth{0.311001pt}%
\definecolor{currentstroke}{rgb}{1.000000,1.000000,1.000000}%
\pgfsetstrokecolor{currentstroke}%
\pgfsetdash{}{0pt}%
\pgfpathmoveto{\pgfqpoint{6.163132in}{1.551313in}}%
\pgfpathcurveto{\pgfqpoint{6.170264in}{1.551313in}}{\pgfqpoint{6.177106in}{1.554147in}}{\pgfqpoint{6.182150in}{1.559191in}}%
\pgfpathcurveto{\pgfqpoint{6.187193in}{1.564235in}}{\pgfqpoint{6.190027in}{1.571076in}}{\pgfqpoint{6.190027in}{1.578209in}}%
\pgfpathcurveto{\pgfqpoint{6.190027in}{1.585342in}}{\pgfqpoint{6.187193in}{1.592184in}}{\pgfqpoint{6.182150in}{1.597227in}}%
\pgfpathcurveto{\pgfqpoint{6.177106in}{1.602271in}}{\pgfqpoint{6.170264in}{1.605105in}}{\pgfqpoint{6.163132in}{1.605105in}}%
\pgfpathcurveto{\pgfqpoint{6.155999in}{1.605105in}}{\pgfqpoint{6.149157in}{1.602271in}}{\pgfqpoint{6.144114in}{1.597227in}}%
\pgfpathcurveto{\pgfqpoint{6.139070in}{1.592184in}}{\pgfqpoint{6.136236in}{1.585342in}}{\pgfqpoint{6.136236in}{1.578209in}}%
\pgfpathcurveto{\pgfqpoint{6.136236in}{1.571076in}}{\pgfqpoint{6.139070in}{1.564235in}}{\pgfqpoint{6.144114in}{1.559191in}}%
\pgfpathcurveto{\pgfqpoint{6.149157in}{1.554147in}}{\pgfqpoint{6.155999in}{1.551313in}}{\pgfqpoint{6.163132in}{1.551313in}}%
\pgfpathclose%
\pgfusepath{stroke,fill}%
\end{pgfscope}%
\begin{pgfscope}%
\pgfpathrectangle{\pgfqpoint{4.985294in}{0.500000in}}{\pgfqpoint{1.764706in}{1.700000in}}%
\pgfusepath{clip}%
\pgfsetbuttcap%
\pgfsetroundjoin%
\definecolor{currentfill}{rgb}{0.979891,0.908948,0.848279}%
\pgfsetfillcolor{currentfill}%
\pgfsetlinewidth{0.311001pt}%
\definecolor{currentstroke}{rgb}{1.000000,1.000000,1.000000}%
\pgfsetstrokecolor{currentstroke}%
\pgfsetdash{}{0pt}%
\pgfpathmoveto{\pgfqpoint{5.399766in}{1.301237in}}%
\pgfpathcurveto{\pgfqpoint{5.406899in}{1.301237in}}{\pgfqpoint{5.413741in}{1.304071in}}{\pgfqpoint{5.418785in}{1.309114in}}%
\pgfpathcurveto{\pgfqpoint{5.423828in}{1.314158in}}{\pgfqpoint{5.426662in}{1.321000in}}{\pgfqpoint{5.426662in}{1.328132in}}%
\pgfpathcurveto{\pgfqpoint{5.426662in}{1.335265in}}{\pgfqpoint{5.423828in}{1.342107in}}{\pgfqpoint{5.418785in}{1.347151in}}%
\pgfpathcurveto{\pgfqpoint{5.413741in}{1.352194in}}{\pgfqpoint{5.406899in}{1.355028in}}{\pgfqpoint{5.399766in}{1.355028in}}%
\pgfpathcurveto{\pgfqpoint{5.392634in}{1.355028in}}{\pgfqpoint{5.385792in}{1.352194in}}{\pgfqpoint{5.380748in}{1.347151in}}%
\pgfpathcurveto{\pgfqpoint{5.375705in}{1.342107in}}{\pgfqpoint{5.372871in}{1.335265in}}{\pgfqpoint{5.372871in}{1.328132in}}%
\pgfpathcurveto{\pgfqpoint{5.372871in}{1.321000in}}{\pgfqpoint{5.375705in}{1.314158in}}{\pgfqpoint{5.380748in}{1.309114in}}%
\pgfpathcurveto{\pgfqpoint{5.385792in}{1.304071in}}{\pgfqpoint{5.392634in}{1.301237in}}{\pgfqpoint{5.399766in}{1.301237in}}%
\pgfpathclose%
\pgfusepath{stroke,fill}%
\end{pgfscope}%
\begin{pgfscope}%
\pgfpathrectangle{\pgfqpoint{4.985294in}{0.500000in}}{\pgfqpoint{1.764706in}{1.700000in}}%
\pgfusepath{clip}%
\pgfsetbuttcap%
\pgfsetroundjoin%
\definecolor{currentfill}{rgb}{0.963379,0.625574,0.465113}%
\pgfsetfillcolor{currentfill}%
\pgfsetlinewidth{0.311001pt}%
\definecolor{currentstroke}{rgb}{1.000000,1.000000,1.000000}%
\pgfsetstrokecolor{currentstroke}%
\pgfsetdash{}{0pt}%
\pgfpathmoveto{\pgfqpoint{5.645271in}{0.927517in}}%
\pgfpathcurveto{\pgfqpoint{5.652404in}{0.927517in}}{\pgfqpoint{5.659246in}{0.930351in}}{\pgfqpoint{5.664290in}{0.935395in}}%
\pgfpathcurveto{\pgfqpoint{5.669333in}{0.940438in}}{\pgfqpoint{5.672167in}{0.947280in}}{\pgfqpoint{5.672167in}{0.954413in}}%
\pgfpathcurveto{\pgfqpoint{5.672167in}{0.961546in}}{\pgfqpoint{5.669333in}{0.968387in}}{\pgfqpoint{5.664290in}{0.973431in}}%
\pgfpathcurveto{\pgfqpoint{5.659246in}{0.978475in}}{\pgfqpoint{5.652404in}{0.981309in}}{\pgfqpoint{5.645271in}{0.981309in}}%
\pgfpathcurveto{\pgfqpoint{5.638139in}{0.981309in}}{\pgfqpoint{5.631297in}{0.978475in}}{\pgfqpoint{5.626253in}{0.973431in}}%
\pgfpathcurveto{\pgfqpoint{5.621210in}{0.968387in}}{\pgfqpoint{5.618376in}{0.961546in}}{\pgfqpoint{5.618376in}{0.954413in}}%
\pgfpathcurveto{\pgfqpoint{5.618376in}{0.947280in}}{\pgfqpoint{5.621210in}{0.940438in}}{\pgfqpoint{5.626253in}{0.935395in}}%
\pgfpathcurveto{\pgfqpoint{5.631297in}{0.930351in}}{\pgfqpoint{5.638139in}{0.927517in}}{\pgfqpoint{5.645271in}{0.927517in}}%
\pgfpathclose%
\pgfusepath{stroke,fill}%
\end{pgfscope}%
\begin{pgfscope}%
\pgfpathrectangle{\pgfqpoint{4.985294in}{0.500000in}}{\pgfqpoint{1.764706in}{1.700000in}}%
\pgfusepath{clip}%
\pgfsetbuttcap%
\pgfsetroundjoin%
\definecolor{currentfill}{rgb}{0.955697,0.484891,0.334214}%
\pgfsetfillcolor{currentfill}%
\pgfsetlinewidth{0.311001pt}%
\definecolor{currentstroke}{rgb}{1.000000,1.000000,1.000000}%
\pgfsetstrokecolor{currentstroke}%
\pgfsetdash{}{0pt}%
\pgfpathmoveto{\pgfqpoint{6.158006in}{1.346344in}}%
\pgfpathcurveto{\pgfqpoint{6.165139in}{1.346344in}}{\pgfqpoint{6.171980in}{1.349177in}}{\pgfqpoint{6.177024in}{1.354221in}}%
\pgfpathcurveto{\pgfqpoint{6.182067in}{1.359265in}}{\pgfqpoint{6.184901in}{1.366106in}}{\pgfqpoint{6.184901in}{1.373239in}}%
\pgfpathcurveto{\pgfqpoint{6.184901in}{1.380372in}}{\pgfqpoint{6.182067in}{1.387214in}}{\pgfqpoint{6.177024in}{1.392257in}}%
\pgfpathcurveto{\pgfqpoint{6.171980in}{1.397301in}}{\pgfqpoint{6.165139in}{1.400135in}}{\pgfqpoint{6.158006in}{1.400135in}}%
\pgfpathcurveto{\pgfqpoint{6.150873in}{1.400135in}}{\pgfqpoint{6.144031in}{1.397301in}}{\pgfqpoint{6.138988in}{1.392257in}}%
\pgfpathcurveto{\pgfqpoint{6.133944in}{1.387214in}}{\pgfqpoint{6.131110in}{1.380372in}}{\pgfqpoint{6.131110in}{1.373239in}}%
\pgfpathcurveto{\pgfqpoint{6.131110in}{1.366106in}}{\pgfqpoint{6.133944in}{1.359265in}}{\pgfqpoint{6.138988in}{1.354221in}}%
\pgfpathcurveto{\pgfqpoint{6.144031in}{1.349177in}}{\pgfqpoint{6.150873in}{1.346344in}}{\pgfqpoint{6.158006in}{1.346344in}}%
\pgfpathclose%
\pgfusepath{stroke,fill}%
\end{pgfscope}%
\begin{pgfscope}%
\pgfpathrectangle{\pgfqpoint{4.985294in}{0.500000in}}{\pgfqpoint{1.764706in}{1.700000in}}%
\pgfusepath{clip}%
\pgfsetbuttcap%
\pgfsetroundjoin%
\definecolor{currentfill}{rgb}{0.967735,0.780441,0.659127}%
\pgfsetfillcolor{currentfill}%
\pgfsetlinewidth{0.311001pt}%
\definecolor{currentstroke}{rgb}{1.000000,1.000000,1.000000}%
\pgfsetstrokecolor{currentstroke}%
\pgfsetdash{}{0pt}%
\pgfpathmoveto{\pgfqpoint{5.420519in}{1.602200in}}%
\pgfpathcurveto{\pgfqpoint{5.427652in}{1.602200in}}{\pgfqpoint{5.434493in}{1.605034in}}{\pgfqpoint{5.439537in}{1.610077in}}%
\pgfpathcurveto{\pgfqpoint{5.444581in}{1.615121in}}{\pgfqpoint{5.447414in}{1.621963in}}{\pgfqpoint{5.447414in}{1.629095in}}%
\pgfpathcurveto{\pgfqpoint{5.447414in}{1.636228in}}{\pgfqpoint{5.444581in}{1.643070in}}{\pgfqpoint{5.439537in}{1.648114in}}%
\pgfpathcurveto{\pgfqpoint{5.434493in}{1.653157in}}{\pgfqpoint{5.427652in}{1.655991in}}{\pgfqpoint{5.420519in}{1.655991in}}%
\pgfpathcurveto{\pgfqpoint{5.413386in}{1.655991in}}{\pgfqpoint{5.406544in}{1.653157in}}{\pgfqpoint{5.401501in}{1.648114in}}%
\pgfpathcurveto{\pgfqpoint{5.396457in}{1.643070in}}{\pgfqpoint{5.393623in}{1.636228in}}{\pgfqpoint{5.393623in}{1.629095in}}%
\pgfpathcurveto{\pgfqpoint{5.393623in}{1.621963in}}{\pgfqpoint{5.396457in}{1.615121in}}{\pgfqpoint{5.401501in}{1.610077in}}%
\pgfpathcurveto{\pgfqpoint{5.406544in}{1.605034in}}{\pgfqpoint{5.413386in}{1.602200in}}{\pgfqpoint{5.420519in}{1.602200in}}%
\pgfpathclose%
\pgfusepath{stroke,fill}%
\end{pgfscope}%
\begin{pgfscope}%
\pgfpathrectangle{\pgfqpoint{4.985294in}{0.500000in}}{\pgfqpoint{1.764706in}{1.700000in}}%
\pgfusepath{clip}%
\pgfsetbuttcap%
\pgfsetroundjoin%
\definecolor{currentfill}{rgb}{0.969803,0.809811,0.702523}%
\pgfsetfillcolor{currentfill}%
\pgfsetlinewidth{0.311001pt}%
\definecolor{currentstroke}{rgb}{1.000000,1.000000,1.000000}%
\pgfsetstrokecolor{currentstroke}%
\pgfsetdash{}{0pt}%
\pgfpathmoveto{\pgfqpoint{5.479384in}{1.648158in}}%
\pgfpathcurveto{\pgfqpoint{5.486517in}{1.648158in}}{\pgfqpoint{5.493359in}{1.650992in}}{\pgfqpoint{5.498403in}{1.656036in}}%
\pgfpathcurveto{\pgfqpoint{5.503446in}{1.661079in}}{\pgfqpoint{5.506280in}{1.667921in}}{\pgfqpoint{5.506280in}{1.675054in}}%
\pgfpathcurveto{\pgfqpoint{5.506280in}{1.682187in}}{\pgfqpoint{5.503446in}{1.689028in}}{\pgfqpoint{5.498403in}{1.694072in}}%
\pgfpathcurveto{\pgfqpoint{5.493359in}{1.699116in}}{\pgfqpoint{5.486517in}{1.701950in}}{\pgfqpoint{5.479384in}{1.701950in}}%
\pgfpathcurveto{\pgfqpoint{5.472252in}{1.701950in}}{\pgfqpoint{5.465410in}{1.699116in}}{\pgfqpoint{5.460366in}{1.694072in}}%
\pgfpathcurveto{\pgfqpoint{5.455323in}{1.689028in}}{\pgfqpoint{5.452489in}{1.682187in}}{\pgfqpoint{5.452489in}{1.675054in}}%
\pgfpathcurveto{\pgfqpoint{5.452489in}{1.667921in}}{\pgfqpoint{5.455323in}{1.661079in}}{\pgfqpoint{5.460366in}{1.656036in}}%
\pgfpathcurveto{\pgfqpoint{5.465410in}{1.650992in}}{\pgfqpoint{5.472252in}{1.648158in}}{\pgfqpoint{5.479384in}{1.648158in}}%
\pgfpathclose%
\pgfusepath{stroke,fill}%
\end{pgfscope}%
\begin{pgfscope}%
\pgfpathrectangle{\pgfqpoint{4.985294in}{0.500000in}}{\pgfqpoint{1.764706in}{1.700000in}}%
\pgfusepath{clip}%
\pgfsetbuttcap%
\pgfsetroundjoin%
\definecolor{currentfill}{rgb}{0.965042,0.701564,0.552889}%
\pgfsetfillcolor{currentfill}%
\pgfsetlinewidth{0.311001pt}%
\definecolor{currentstroke}{rgb}{1.000000,1.000000,1.000000}%
\pgfsetstrokecolor{currentstroke}%
\pgfsetdash{}{0pt}%
\pgfpathmoveto{\pgfqpoint{6.242801in}{1.710888in}}%
\pgfpathcurveto{\pgfqpoint{6.249934in}{1.710888in}}{\pgfqpoint{6.256775in}{1.713722in}}{\pgfqpoint{6.261819in}{1.718765in}}%
\pgfpathcurveto{\pgfqpoint{6.266863in}{1.723809in}}{\pgfqpoint{6.269696in}{1.730651in}}{\pgfqpoint{6.269696in}{1.737783in}}%
\pgfpathcurveto{\pgfqpoint{6.269696in}{1.744916in}}{\pgfqpoint{6.266863in}{1.751758in}}{\pgfqpoint{6.261819in}{1.756802in}}%
\pgfpathcurveto{\pgfqpoint{6.256775in}{1.761845in}}{\pgfqpoint{6.249934in}{1.764679in}}{\pgfqpoint{6.242801in}{1.764679in}}%
\pgfpathcurveto{\pgfqpoint{6.235668in}{1.764679in}}{\pgfqpoint{6.228826in}{1.761845in}}{\pgfqpoint{6.223783in}{1.756802in}}%
\pgfpathcurveto{\pgfqpoint{6.218739in}{1.751758in}}{\pgfqpoint{6.215905in}{1.744916in}}{\pgfqpoint{6.215905in}{1.737783in}}%
\pgfpathcurveto{\pgfqpoint{6.215905in}{1.730651in}}{\pgfqpoint{6.218739in}{1.723809in}}{\pgfqpoint{6.223783in}{1.718765in}}%
\pgfpathcurveto{\pgfqpoint{6.228826in}{1.713722in}}{\pgfqpoint{6.235668in}{1.710888in}}{\pgfqpoint{6.242801in}{1.710888in}}%
\pgfpathclose%
\pgfusepath{stroke,fill}%
\end{pgfscope}%
\begin{pgfscope}%
\pgfpathrectangle{\pgfqpoint{4.985294in}{0.500000in}}{\pgfqpoint{1.764706in}{1.700000in}}%
\pgfusepath{clip}%
\pgfsetbuttcap%
\pgfsetroundjoin%
\definecolor{currentfill}{rgb}{0.964799,0.689101,0.537560}%
\pgfsetfillcolor{currentfill}%
\pgfsetlinewidth{0.311001pt}%
\definecolor{currentstroke}{rgb}{1.000000,1.000000,1.000000}%
\pgfsetstrokecolor{currentstroke}%
\pgfsetdash{}{0pt}%
\pgfpathmoveto{\pgfqpoint{5.563410in}{1.718748in}}%
\pgfpathcurveto{\pgfqpoint{5.570543in}{1.718748in}}{\pgfqpoint{5.577384in}{1.721582in}}{\pgfqpoint{5.582428in}{1.726626in}}%
\pgfpathcurveto{\pgfqpoint{5.587472in}{1.731669in}}{\pgfqpoint{5.590305in}{1.738511in}}{\pgfqpoint{5.590305in}{1.745644in}}%
\pgfpathcurveto{\pgfqpoint{5.590305in}{1.752777in}}{\pgfqpoint{5.587472in}{1.759618in}}{\pgfqpoint{5.582428in}{1.764662in}}%
\pgfpathcurveto{\pgfqpoint{5.577384in}{1.769706in}}{\pgfqpoint{5.570543in}{1.772540in}}{\pgfqpoint{5.563410in}{1.772540in}}%
\pgfpathcurveto{\pgfqpoint{5.556277in}{1.772540in}}{\pgfqpoint{5.549435in}{1.769706in}}{\pgfqpoint{5.544392in}{1.764662in}}%
\pgfpathcurveto{\pgfqpoint{5.539348in}{1.759618in}}{\pgfqpoint{5.536514in}{1.752777in}}{\pgfqpoint{5.536514in}{1.745644in}}%
\pgfpathcurveto{\pgfqpoint{5.536514in}{1.738511in}}{\pgfqpoint{5.539348in}{1.731669in}}{\pgfqpoint{5.544392in}{1.726626in}}%
\pgfpathcurveto{\pgfqpoint{5.549435in}{1.721582in}}{\pgfqpoint{5.556277in}{1.718748in}}{\pgfqpoint{5.563410in}{1.718748in}}%
\pgfpathclose%
\pgfusepath{stroke,fill}%
\end{pgfscope}%
\begin{pgfscope}%
\pgfpathrectangle{\pgfqpoint{4.985294in}{0.500000in}}{\pgfqpoint{1.764706in}{1.700000in}}%
\pgfusepath{clip}%
\pgfsetbuttcap%
\pgfsetroundjoin%
\definecolor{currentfill}{rgb}{0.979891,0.908948,0.848279}%
\pgfsetfillcolor{currentfill}%
\pgfsetlinewidth{0.311001pt}%
\definecolor{currentstroke}{rgb}{1.000000,1.000000,1.000000}%
\pgfsetstrokecolor{currentstroke}%
\pgfsetdash{}{0pt}%
\pgfpathmoveto{\pgfqpoint{6.314781in}{1.483569in}}%
\pgfpathcurveto{\pgfqpoint{6.321914in}{1.483569in}}{\pgfqpoint{6.328755in}{1.486402in}}{\pgfqpoint{6.333799in}{1.491446in}}%
\pgfpathcurveto{\pgfqpoint{6.338843in}{1.496490in}}{\pgfqpoint{6.341676in}{1.503331in}}{\pgfqpoint{6.341676in}{1.510464in}}%
\pgfpathcurveto{\pgfqpoint{6.341676in}{1.517597in}}{\pgfqpoint{6.338843in}{1.524439in}}{\pgfqpoint{6.333799in}{1.529482in}}%
\pgfpathcurveto{\pgfqpoint{6.328755in}{1.534526in}}{\pgfqpoint{6.321914in}{1.537360in}}{\pgfqpoint{6.314781in}{1.537360in}}%
\pgfpathcurveto{\pgfqpoint{6.307648in}{1.537360in}}{\pgfqpoint{6.300806in}{1.534526in}}{\pgfqpoint{6.295763in}{1.529482in}}%
\pgfpathcurveto{\pgfqpoint{6.290719in}{1.524439in}}{\pgfqpoint{6.287885in}{1.517597in}}{\pgfqpoint{6.287885in}{1.510464in}}%
\pgfpathcurveto{\pgfqpoint{6.287885in}{1.503331in}}{\pgfqpoint{6.290719in}{1.496490in}}{\pgfqpoint{6.295763in}{1.491446in}}%
\pgfpathcurveto{\pgfqpoint{6.300806in}{1.486402in}}{\pgfqpoint{6.307648in}{1.483569in}}{\pgfqpoint{6.314781in}{1.483569in}}%
\pgfpathclose%
\pgfusepath{stroke,fill}%
\end{pgfscope}%
\begin{pgfscope}%
\pgfpathrectangle{\pgfqpoint{4.985294in}{0.500000in}}{\pgfqpoint{1.764706in}{1.700000in}}%
\pgfusepath{clip}%
\pgfsetbuttcap%
\pgfsetroundjoin%
\definecolor{currentfill}{rgb}{0.975644,0.874038,0.797253}%
\pgfsetfillcolor{currentfill}%
\pgfsetlinewidth{0.311001pt}%
\definecolor{currentstroke}{rgb}{1.000000,1.000000,1.000000}%
\pgfsetstrokecolor{currentstroke}%
\pgfsetdash{}{0pt}%
\pgfpathmoveto{\pgfqpoint{5.448496in}{1.539838in}}%
\pgfpathcurveto{\pgfqpoint{5.455629in}{1.539838in}}{\pgfqpoint{5.462471in}{1.542672in}}{\pgfqpoint{5.467515in}{1.547716in}}%
\pgfpathcurveto{\pgfqpoint{5.472558in}{1.552759in}}{\pgfqpoint{5.475392in}{1.559601in}}{\pgfqpoint{5.475392in}{1.566734in}}%
\pgfpathcurveto{\pgfqpoint{5.475392in}{1.573867in}}{\pgfqpoint{5.472558in}{1.580708in}}{\pgfqpoint{5.467515in}{1.585752in}}%
\pgfpathcurveto{\pgfqpoint{5.462471in}{1.590796in}}{\pgfqpoint{5.455629in}{1.593629in}}{\pgfqpoint{5.448496in}{1.593629in}}%
\pgfpathcurveto{\pgfqpoint{5.441364in}{1.593629in}}{\pgfqpoint{5.434522in}{1.590796in}}{\pgfqpoint{5.429478in}{1.585752in}}%
\pgfpathcurveto{\pgfqpoint{5.424435in}{1.580708in}}{\pgfqpoint{5.421601in}{1.573867in}}{\pgfqpoint{5.421601in}{1.566734in}}%
\pgfpathcurveto{\pgfqpoint{5.421601in}{1.559601in}}{\pgfqpoint{5.424435in}{1.552759in}}{\pgfqpoint{5.429478in}{1.547716in}}%
\pgfpathcurveto{\pgfqpoint{5.434522in}{1.542672in}}{\pgfqpoint{5.441364in}{1.539838in}}{\pgfqpoint{5.448496in}{1.539838in}}%
\pgfpathclose%
\pgfusepath{stroke,fill}%
\end{pgfscope}%
\begin{pgfscope}%
\pgfpathrectangle{\pgfqpoint{4.985294in}{0.500000in}}{\pgfqpoint{1.764706in}{1.700000in}}%
\pgfusepath{clip}%
\pgfsetbuttcap%
\pgfsetroundjoin%
\definecolor{currentfill}{rgb}{0.965753,0.732351,0.592427}%
\pgfsetfillcolor{currentfill}%
\pgfsetlinewidth{0.311001pt}%
\definecolor{currentstroke}{rgb}{1.000000,1.000000,1.000000}%
\pgfsetstrokecolor{currentstroke}%
\pgfsetdash{}{0pt}%
\pgfpathmoveto{\pgfqpoint{5.534586in}{1.155681in}}%
\pgfpathcurveto{\pgfqpoint{5.541719in}{1.155681in}}{\pgfqpoint{5.548561in}{1.158515in}}{\pgfqpoint{5.553604in}{1.163558in}}%
\pgfpathcurveto{\pgfqpoint{5.558648in}{1.168602in}}{\pgfqpoint{5.561482in}{1.175444in}}{\pgfqpoint{5.561482in}{1.182577in}}%
\pgfpathcurveto{\pgfqpoint{5.561482in}{1.189709in}}{\pgfqpoint{5.558648in}{1.196551in}}{\pgfqpoint{5.553604in}{1.201595in}}%
\pgfpathcurveto{\pgfqpoint{5.548561in}{1.206638in}}{\pgfqpoint{5.541719in}{1.209472in}}{\pgfqpoint{5.534586in}{1.209472in}}%
\pgfpathcurveto{\pgfqpoint{5.527453in}{1.209472in}}{\pgfqpoint{5.520612in}{1.206638in}}{\pgfqpoint{5.515568in}{1.201595in}}%
\pgfpathcurveto{\pgfqpoint{5.510524in}{1.196551in}}{\pgfqpoint{5.507690in}{1.189709in}}{\pgfqpoint{5.507690in}{1.182577in}}%
\pgfpathcurveto{\pgfqpoint{5.507690in}{1.175444in}}{\pgfqpoint{5.510524in}{1.168602in}}{\pgfqpoint{5.515568in}{1.163558in}}%
\pgfpathcurveto{\pgfqpoint{5.520612in}{1.158515in}}{\pgfqpoint{5.527453in}{1.155681in}}{\pgfqpoint{5.534586in}{1.155681in}}%
\pgfpathclose%
\pgfusepath{stroke,fill}%
\end{pgfscope}%
\begin{pgfscope}%
\pgfpathrectangle{\pgfqpoint{4.985294in}{0.500000in}}{\pgfqpoint{1.764706in}{1.700000in}}%
\pgfusepath{clip}%
\pgfsetbuttcap%
\pgfsetroundjoin%
\definecolor{currentfill}{rgb}{0.964306,0.663930,0.507747}%
\pgfsetfillcolor{currentfill}%
\pgfsetlinewidth{0.311001pt}%
\definecolor{currentstroke}{rgb}{1.000000,1.000000,1.000000}%
\pgfsetstrokecolor{currentstroke}%
\pgfsetdash{}{0pt}%
\pgfpathmoveto{\pgfqpoint{6.107727in}{1.698895in}}%
\pgfpathcurveto{\pgfqpoint{6.114860in}{1.698895in}}{\pgfqpoint{6.121702in}{1.701729in}}{\pgfqpoint{6.126745in}{1.706773in}}%
\pgfpathcurveto{\pgfqpoint{6.131789in}{1.711816in}}{\pgfqpoint{6.134623in}{1.718658in}}{\pgfqpoint{6.134623in}{1.725791in}}%
\pgfpathcurveto{\pgfqpoint{6.134623in}{1.732924in}}{\pgfqpoint{6.131789in}{1.739765in}}{\pgfqpoint{6.126745in}{1.744809in}}%
\pgfpathcurveto{\pgfqpoint{6.121702in}{1.749853in}}{\pgfqpoint{6.114860in}{1.752687in}}{\pgfqpoint{6.107727in}{1.752687in}}%
\pgfpathcurveto{\pgfqpoint{6.100594in}{1.752687in}}{\pgfqpoint{6.093753in}{1.749853in}}{\pgfqpoint{6.088709in}{1.744809in}}%
\pgfpathcurveto{\pgfqpoint{6.083665in}{1.739765in}}{\pgfqpoint{6.080832in}{1.732924in}}{\pgfqpoint{6.080832in}{1.725791in}}%
\pgfpathcurveto{\pgfqpoint{6.080832in}{1.718658in}}{\pgfqpoint{6.083665in}{1.711816in}}{\pgfqpoint{6.088709in}{1.706773in}}%
\pgfpathcurveto{\pgfqpoint{6.093753in}{1.701729in}}{\pgfqpoint{6.100594in}{1.698895in}}{\pgfqpoint{6.107727in}{1.698895in}}%
\pgfpathclose%
\pgfusepath{stroke,fill}%
\end{pgfscope}%
\begin{pgfscope}%
\pgfpathrectangle{\pgfqpoint{4.985294in}{0.500000in}}{\pgfqpoint{1.764706in}{1.700000in}}%
\pgfusepath{clip}%
\pgfsetbuttcap%
\pgfsetroundjoin%
\definecolor{currentfill}{rgb}{0.974412,0.862387,0.780156}%
\pgfsetfillcolor{currentfill}%
\pgfsetlinewidth{0.311001pt}%
\definecolor{currentstroke}{rgb}{1.000000,1.000000,1.000000}%
\pgfsetstrokecolor{currentstroke}%
\pgfsetdash{}{0pt}%
\pgfpathmoveto{\pgfqpoint{6.255292in}{1.249897in}}%
\pgfpathcurveto{\pgfqpoint{6.262424in}{1.249897in}}{\pgfqpoint{6.269266in}{1.252731in}}{\pgfqpoint{6.274310in}{1.257775in}}%
\pgfpathcurveto{\pgfqpoint{6.279353in}{1.262818in}}{\pgfqpoint{6.282187in}{1.269660in}}{\pgfqpoint{6.282187in}{1.276793in}}%
\pgfpathcurveto{\pgfqpoint{6.282187in}{1.283926in}}{\pgfqpoint{6.279353in}{1.290767in}}{\pgfqpoint{6.274310in}{1.295811in}}%
\pgfpathcurveto{\pgfqpoint{6.269266in}{1.300855in}}{\pgfqpoint{6.262424in}{1.303689in}}{\pgfqpoint{6.255292in}{1.303689in}}%
\pgfpathcurveto{\pgfqpoint{6.248159in}{1.303689in}}{\pgfqpoint{6.241317in}{1.300855in}}{\pgfqpoint{6.236273in}{1.295811in}}%
\pgfpathcurveto{\pgfqpoint{6.231230in}{1.290767in}}{\pgfqpoint{6.228396in}{1.283926in}}{\pgfqpoint{6.228396in}{1.276793in}}%
\pgfpathcurveto{\pgfqpoint{6.228396in}{1.269660in}}{\pgfqpoint{6.231230in}{1.262818in}}{\pgfqpoint{6.236273in}{1.257775in}}%
\pgfpathcurveto{\pgfqpoint{6.241317in}{1.252731in}}{\pgfqpoint{6.248159in}{1.249897in}}{\pgfqpoint{6.255292in}{1.249897in}}%
\pgfpathclose%
\pgfusepath{stroke,fill}%
\end{pgfscope}%
\begin{pgfscope}%
\pgfpathrectangle{\pgfqpoint{4.985294in}{0.500000in}}{\pgfqpoint{1.764706in}{1.700000in}}%
\pgfusepath{clip}%
\pgfsetbuttcap%
\pgfsetroundjoin%
\definecolor{currentfill}{rgb}{0.965042,0.701564,0.552889}%
\pgfsetfillcolor{currentfill}%
\pgfsetlinewidth{0.311001pt}%
\definecolor{currentstroke}{rgb}{1.000000,1.000000,1.000000}%
\pgfsetstrokecolor{currentstroke}%
\pgfsetdash{}{0pt}%
\pgfpathmoveto{\pgfqpoint{5.592159in}{0.912136in}}%
\pgfpathcurveto{\pgfqpoint{5.599292in}{0.912136in}}{\pgfqpoint{5.606134in}{0.914970in}}{\pgfqpoint{5.611178in}{0.920014in}}%
\pgfpathcurveto{\pgfqpoint{5.616221in}{0.925058in}}{\pgfqpoint{5.619055in}{0.931899in}}{\pgfqpoint{5.619055in}{0.939032in}}%
\pgfpathcurveto{\pgfqpoint{5.619055in}{0.946165in}}{\pgfqpoint{5.616221in}{0.953007in}}{\pgfqpoint{5.611178in}{0.958050in}}%
\pgfpathcurveto{\pgfqpoint{5.606134in}{0.963094in}}{\pgfqpoint{5.599292in}{0.965928in}}{\pgfqpoint{5.592159in}{0.965928in}}%
\pgfpathcurveto{\pgfqpoint{5.585027in}{0.965928in}}{\pgfqpoint{5.578185in}{0.963094in}}{\pgfqpoint{5.573141in}{0.958050in}}%
\pgfpathcurveto{\pgfqpoint{5.568098in}{0.953007in}}{\pgfqpoint{5.565264in}{0.946165in}}{\pgfqpoint{5.565264in}{0.939032in}}%
\pgfpathcurveto{\pgfqpoint{5.565264in}{0.931899in}}{\pgfqpoint{5.568098in}{0.925058in}}{\pgfqpoint{5.573141in}{0.920014in}}%
\pgfpathcurveto{\pgfqpoint{5.578185in}{0.914970in}}{\pgfqpoint{5.585027in}{0.912136in}}{\pgfqpoint{5.592159in}{0.912136in}}%
\pgfpathclose%
\pgfusepath{stroke,fill}%
\end{pgfscope}%
\begin{pgfscope}%
\pgfpathrectangle{\pgfqpoint{4.985294in}{0.500000in}}{\pgfqpoint{1.764706in}{1.700000in}}%
\pgfusepath{clip}%
\pgfsetbuttcap%
\pgfsetroundjoin%
\definecolor{currentfill}{rgb}{0.970718,0.821518,0.719872}%
\pgfsetfillcolor{currentfill}%
\pgfsetlinewidth{0.311001pt}%
\definecolor{currentstroke}{rgb}{1.000000,1.000000,1.000000}%
\pgfsetstrokecolor{currentstroke}%
\pgfsetdash{}{0pt}%
\pgfpathmoveto{\pgfqpoint{6.297668in}{1.057078in}}%
\pgfpathcurveto{\pgfqpoint{6.304801in}{1.057078in}}{\pgfqpoint{6.311642in}{1.059912in}}{\pgfqpoint{6.316686in}{1.064955in}}%
\pgfpathcurveto{\pgfqpoint{6.321730in}{1.069999in}}{\pgfqpoint{6.324564in}{1.076841in}}{\pgfqpoint{6.324564in}{1.083974in}}%
\pgfpathcurveto{\pgfqpoint{6.324564in}{1.091106in}}{\pgfqpoint{6.321730in}{1.097948in}}{\pgfqpoint{6.316686in}{1.102992in}}%
\pgfpathcurveto{\pgfqpoint{6.311642in}{1.108035in}}{\pgfqpoint{6.304801in}{1.110869in}}{\pgfqpoint{6.297668in}{1.110869in}}%
\pgfpathcurveto{\pgfqpoint{6.290535in}{1.110869in}}{\pgfqpoint{6.283693in}{1.108035in}}{\pgfqpoint{6.278650in}{1.102992in}}%
\pgfpathcurveto{\pgfqpoint{6.273606in}{1.097948in}}{\pgfqpoint{6.270772in}{1.091106in}}{\pgfqpoint{6.270772in}{1.083974in}}%
\pgfpathcurveto{\pgfqpoint{6.270772in}{1.076841in}}{\pgfqpoint{6.273606in}{1.069999in}}{\pgfqpoint{6.278650in}{1.064955in}}%
\pgfpathcurveto{\pgfqpoint{6.283693in}{1.059912in}}{\pgfqpoint{6.290535in}{1.057078in}}{\pgfqpoint{6.297668in}{1.057078in}}%
\pgfpathclose%
\pgfusepath{stroke,fill}%
\end{pgfscope}%
\begin{pgfscope}%
\pgfpathrectangle{\pgfqpoint{4.985294in}{0.500000in}}{\pgfqpoint{1.764706in}{1.700000in}}%
\pgfusepath{clip}%
\pgfsetbuttcap%
\pgfsetroundjoin%
\definecolor{currentfill}{rgb}{0.968105,0.786346,0.667739}%
\pgfsetfillcolor{currentfill}%
\pgfsetlinewidth{0.311001pt}%
\definecolor{currentstroke}{rgb}{1.000000,1.000000,1.000000}%
\pgfsetstrokecolor{currentstroke}%
\pgfsetdash{}{0pt}%
\pgfpathmoveto{\pgfqpoint{5.388823in}{1.091370in}}%
\pgfpathcurveto{\pgfqpoint{5.395956in}{1.091370in}}{\pgfqpoint{5.402797in}{1.094204in}}{\pgfqpoint{5.407841in}{1.099247in}}%
\pgfpathcurveto{\pgfqpoint{5.412884in}{1.104291in}}{\pgfqpoint{5.415718in}{1.111133in}}{\pgfqpoint{5.415718in}{1.118266in}}%
\pgfpathcurveto{\pgfqpoint{5.415718in}{1.125398in}}{\pgfqpoint{5.412884in}{1.132240in}}{\pgfqpoint{5.407841in}{1.137284in}}%
\pgfpathcurveto{\pgfqpoint{5.402797in}{1.142327in}}{\pgfqpoint{5.395956in}{1.145161in}}{\pgfqpoint{5.388823in}{1.145161in}}%
\pgfpathcurveto{\pgfqpoint{5.381690in}{1.145161in}}{\pgfqpoint{5.374848in}{1.142327in}}{\pgfqpoint{5.369805in}{1.137284in}}%
\pgfpathcurveto{\pgfqpoint{5.364761in}{1.132240in}}{\pgfqpoint{5.361927in}{1.125398in}}{\pgfqpoint{5.361927in}{1.118266in}}%
\pgfpathcurveto{\pgfqpoint{5.361927in}{1.111133in}}{\pgfqpoint{5.364761in}{1.104291in}}{\pgfqpoint{5.369805in}{1.099247in}}%
\pgfpathcurveto{\pgfqpoint{5.374848in}{1.094204in}}{\pgfqpoint{5.381690in}{1.091370in}}{\pgfqpoint{5.388823in}{1.091370in}}%
\pgfpathclose%
\pgfusepath{stroke,fill}%
\end{pgfscope}%
\begin{pgfscope}%
\pgfpathrectangle{\pgfqpoint{4.985294in}{0.500000in}}{\pgfqpoint{1.764706in}{1.700000in}}%
\pgfusepath{clip}%
\pgfsetbuttcap%
\pgfsetroundjoin%
\definecolor{currentfill}{rgb}{0.977657,0.891500,0.822809}%
\pgfsetfillcolor{currentfill}%
\pgfsetlinewidth{0.311001pt}%
\definecolor{currentstroke}{rgb}{1.000000,1.000000,1.000000}%
\pgfsetstrokecolor{currentstroke}%
\pgfsetdash{}{0pt}%
\pgfpathmoveto{\pgfqpoint{5.445729in}{1.134566in}}%
\pgfpathcurveto{\pgfqpoint{5.452862in}{1.134566in}}{\pgfqpoint{5.459704in}{1.137399in}}{\pgfqpoint{5.464747in}{1.142443in}}%
\pgfpathcurveto{\pgfqpoint{5.469791in}{1.147487in}}{\pgfqpoint{5.472625in}{1.154328in}}{\pgfqpoint{5.472625in}{1.161461in}}%
\pgfpathcurveto{\pgfqpoint{5.472625in}{1.168594in}}{\pgfqpoint{5.469791in}{1.175436in}}{\pgfqpoint{5.464747in}{1.180479in}}%
\pgfpathcurveto{\pgfqpoint{5.459704in}{1.185523in}}{\pgfqpoint{5.452862in}{1.188357in}}{\pgfqpoint{5.445729in}{1.188357in}}%
\pgfpathcurveto{\pgfqpoint{5.438596in}{1.188357in}}{\pgfqpoint{5.431755in}{1.185523in}}{\pgfqpoint{5.426711in}{1.180479in}}%
\pgfpathcurveto{\pgfqpoint{5.421667in}{1.175436in}}{\pgfqpoint{5.418834in}{1.168594in}}{\pgfqpoint{5.418834in}{1.161461in}}%
\pgfpathcurveto{\pgfqpoint{5.418834in}{1.154328in}}{\pgfqpoint{5.421667in}{1.147487in}}{\pgfqpoint{5.426711in}{1.142443in}}%
\pgfpathcurveto{\pgfqpoint{5.431755in}{1.137399in}}{\pgfqpoint{5.438596in}{1.134566in}}{\pgfqpoint{5.445729in}{1.134566in}}%
\pgfpathclose%
\pgfusepath{stroke,fill}%
\end{pgfscope}%
\begin{pgfscope}%
\pgfpathrectangle{\pgfqpoint{4.985294in}{0.500000in}}{\pgfqpoint{1.764706in}{1.700000in}}%
\pgfusepath{clip}%
\pgfsetbuttcap%
\pgfsetroundjoin%
\definecolor{currentfill}{rgb}{0.973832,0.856556,0.771584}%
\pgfsetfillcolor{currentfill}%
\pgfsetlinewidth{0.311001pt}%
\definecolor{currentstroke}{rgb}{1.000000,1.000000,1.000000}%
\pgfsetstrokecolor{currentstroke}%
\pgfsetdash{}{0pt}%
\pgfpathmoveto{\pgfqpoint{5.473584in}{1.034632in}}%
\pgfpathcurveto{\pgfqpoint{5.480717in}{1.034632in}}{\pgfqpoint{5.487558in}{1.037466in}}{\pgfqpoint{5.492602in}{1.042510in}}%
\pgfpathcurveto{\pgfqpoint{5.497646in}{1.047553in}}{\pgfqpoint{5.500479in}{1.054395in}}{\pgfqpoint{5.500479in}{1.061528in}}%
\pgfpathcurveto{\pgfqpoint{5.500479in}{1.068661in}}{\pgfqpoint{5.497646in}{1.075502in}}{\pgfqpoint{5.492602in}{1.080546in}}%
\pgfpathcurveto{\pgfqpoint{5.487558in}{1.085590in}}{\pgfqpoint{5.480717in}{1.088424in}}{\pgfqpoint{5.473584in}{1.088424in}}%
\pgfpathcurveto{\pgfqpoint{5.466451in}{1.088424in}}{\pgfqpoint{5.459609in}{1.085590in}}{\pgfqpoint{5.454566in}{1.080546in}}%
\pgfpathcurveto{\pgfqpoint{5.449522in}{1.075502in}}{\pgfqpoint{5.446688in}{1.068661in}}{\pgfqpoint{5.446688in}{1.061528in}}%
\pgfpathcurveto{\pgfqpoint{5.446688in}{1.054395in}}{\pgfqpoint{5.449522in}{1.047553in}}{\pgfqpoint{5.454566in}{1.042510in}}%
\pgfpathcurveto{\pgfqpoint{5.459609in}{1.037466in}}{\pgfqpoint{5.466451in}{1.034632in}}{\pgfqpoint{5.473584in}{1.034632in}}%
\pgfpathclose%
\pgfusepath{stroke,fill}%
\end{pgfscope}%
\begin{pgfscope}%
\pgfpathrectangle{\pgfqpoint{4.985294in}{0.500000in}}{\pgfqpoint{1.764706in}{1.700000in}}%
\pgfusepath{clip}%
\pgfsetbuttcap%
\pgfsetroundjoin%
\definecolor{currentfill}{rgb}{0.954476,0.470822,0.323110}%
\pgfsetfillcolor{currentfill}%
\pgfsetlinewidth{0.311001pt}%
\definecolor{currentstroke}{rgb}{1.000000,1.000000,1.000000}%
\pgfsetstrokecolor{currentstroke}%
\pgfsetdash{}{0pt}%
\pgfpathmoveto{\pgfqpoint{5.524412in}{1.794219in}}%
\pgfpathcurveto{\pgfqpoint{5.531545in}{1.794219in}}{\pgfqpoint{5.538387in}{1.797053in}}{\pgfqpoint{5.543430in}{1.802096in}}%
\pgfpathcurveto{\pgfqpoint{5.548474in}{1.807140in}}{\pgfqpoint{5.551308in}{1.813982in}}{\pgfqpoint{5.551308in}{1.821115in}}%
\pgfpathcurveto{\pgfqpoint{5.551308in}{1.828247in}}{\pgfqpoint{5.548474in}{1.835089in}}{\pgfqpoint{5.543430in}{1.840133in}}%
\pgfpathcurveto{\pgfqpoint{5.538387in}{1.845176in}}{\pgfqpoint{5.531545in}{1.848010in}}{\pgfqpoint{5.524412in}{1.848010in}}%
\pgfpathcurveto{\pgfqpoint{5.517279in}{1.848010in}}{\pgfqpoint{5.510438in}{1.845176in}}{\pgfqpoint{5.505394in}{1.840133in}}%
\pgfpathcurveto{\pgfqpoint{5.500350in}{1.835089in}}{\pgfqpoint{5.497516in}{1.828247in}}{\pgfqpoint{5.497516in}{1.821115in}}%
\pgfpathcurveto{\pgfqpoint{5.497516in}{1.813982in}}{\pgfqpoint{5.500350in}{1.807140in}}{\pgfqpoint{5.505394in}{1.802096in}}%
\pgfpathcurveto{\pgfqpoint{5.510438in}{1.797053in}}{\pgfqpoint{5.517279in}{1.794219in}}{\pgfqpoint{5.524412in}{1.794219in}}%
\pgfpathclose%
\pgfusepath{stroke,fill}%
\end{pgfscope}%
\begin{pgfscope}%
\pgfpathrectangle{\pgfqpoint{4.985294in}{0.500000in}}{\pgfqpoint{1.764706in}{1.700000in}}%
\pgfusepath{clip}%
\pgfsetbuttcap%
\pgfsetroundjoin%
\definecolor{currentfill}{rgb}{0.973271,0.850724,0.762998}%
\pgfsetfillcolor{currentfill}%
\pgfsetlinewidth{0.311001pt}%
\definecolor{currentstroke}{rgb}{1.000000,1.000000,1.000000}%
\pgfsetstrokecolor{currentstroke}%
\pgfsetdash{}{0pt}%
\pgfpathmoveto{\pgfqpoint{5.447192in}{1.575947in}}%
\pgfpathcurveto{\pgfqpoint{5.454324in}{1.575947in}}{\pgfqpoint{5.461166in}{1.578781in}}{\pgfqpoint{5.466210in}{1.583825in}}%
\pgfpathcurveto{\pgfqpoint{5.471253in}{1.588868in}}{\pgfqpoint{5.474087in}{1.595710in}}{\pgfqpoint{5.474087in}{1.602843in}}%
\pgfpathcurveto{\pgfqpoint{5.474087in}{1.609976in}}{\pgfqpoint{5.471253in}{1.616817in}}{\pgfqpoint{5.466210in}{1.621861in}}%
\pgfpathcurveto{\pgfqpoint{5.461166in}{1.626905in}}{\pgfqpoint{5.454324in}{1.629739in}}{\pgfqpoint{5.447192in}{1.629739in}}%
\pgfpathcurveto{\pgfqpoint{5.440059in}{1.629739in}}{\pgfqpoint{5.433217in}{1.626905in}}{\pgfqpoint{5.428173in}{1.621861in}}%
\pgfpathcurveto{\pgfqpoint{5.423130in}{1.616817in}}{\pgfqpoint{5.420296in}{1.609976in}}{\pgfqpoint{5.420296in}{1.602843in}}%
\pgfpathcurveto{\pgfqpoint{5.420296in}{1.595710in}}{\pgfqpoint{5.423130in}{1.588868in}}{\pgfqpoint{5.428173in}{1.583825in}}%
\pgfpathcurveto{\pgfqpoint{5.433217in}{1.578781in}}{\pgfqpoint{5.440059in}{1.575947in}}{\pgfqpoint{5.447192in}{1.575947in}}%
\pgfpathclose%
\pgfusepath{stroke,fill}%
\end{pgfscope}%
\begin{pgfscope}%
\pgfpathrectangle{\pgfqpoint{4.985294in}{0.500000in}}{\pgfqpoint{1.764706in}{1.700000in}}%
\pgfusepath{clip}%
\pgfsetbuttcap%
\pgfsetroundjoin%
\definecolor{currentfill}{rgb}{0.965302,0.713942,0.568499}%
\pgfsetfillcolor{currentfill}%
\pgfsetlinewidth{0.311001pt}%
\definecolor{currentstroke}{rgb}{1.000000,1.000000,1.000000}%
\pgfsetstrokecolor{currentstroke}%
\pgfsetdash{}{0pt}%
\pgfpathmoveto{\pgfqpoint{6.356611in}{1.562442in}}%
\pgfpathcurveto{\pgfqpoint{6.363744in}{1.562442in}}{\pgfqpoint{6.370585in}{1.565276in}}{\pgfqpoint{6.375629in}{1.570319in}}%
\pgfpathcurveto{\pgfqpoint{6.380673in}{1.575363in}}{\pgfqpoint{6.383507in}{1.582205in}}{\pgfqpoint{6.383507in}{1.589338in}}%
\pgfpathcurveto{\pgfqpoint{6.383507in}{1.596470in}}{\pgfqpoint{6.380673in}{1.603312in}}{\pgfqpoint{6.375629in}{1.608356in}}%
\pgfpathcurveto{\pgfqpoint{6.370585in}{1.613399in}}{\pgfqpoint{6.363744in}{1.616233in}}{\pgfqpoint{6.356611in}{1.616233in}}%
\pgfpathcurveto{\pgfqpoint{6.349478in}{1.616233in}}{\pgfqpoint{6.342636in}{1.613399in}}{\pgfqpoint{6.337593in}{1.608356in}}%
\pgfpathcurveto{\pgfqpoint{6.332549in}{1.603312in}}{\pgfqpoint{6.329715in}{1.596470in}}{\pgfqpoint{6.329715in}{1.589338in}}%
\pgfpathcurveto{\pgfqpoint{6.329715in}{1.582205in}}{\pgfqpoint{6.332549in}{1.575363in}}{\pgfqpoint{6.337593in}{1.570319in}}%
\pgfpathcurveto{\pgfqpoint{6.342636in}{1.565276in}}{\pgfqpoint{6.349478in}{1.562442in}}{\pgfqpoint{6.356611in}{1.562442in}}%
\pgfpathclose%
\pgfusepath{stroke,fill}%
\end{pgfscope}%
\begin{pgfscope}%
\pgfpathrectangle{\pgfqpoint{4.985294in}{0.500000in}}{\pgfqpoint{1.764706in}{1.700000in}}%
\pgfusepath{clip}%
\pgfsetbuttcap%
\pgfsetroundjoin%
\definecolor{currentfill}{rgb}{0.973832,0.856556,0.771584}%
\pgfsetfillcolor{currentfill}%
\pgfsetlinewidth{0.311001pt}%
\definecolor{currentstroke}{rgb}{1.000000,1.000000,1.000000}%
\pgfsetstrokecolor{currentstroke}%
\pgfsetdash{}{0pt}%
\pgfpathmoveto{\pgfqpoint{5.494434in}{1.029359in}}%
\pgfpathcurveto{\pgfqpoint{5.501567in}{1.029359in}}{\pgfqpoint{5.508408in}{1.032193in}}{\pgfqpoint{5.513452in}{1.037237in}}%
\pgfpathcurveto{\pgfqpoint{5.518496in}{1.042281in}}{\pgfqpoint{5.521329in}{1.049122in}}{\pgfqpoint{5.521329in}{1.056255in}}%
\pgfpathcurveto{\pgfqpoint{5.521329in}{1.063388in}}{\pgfqpoint{5.518496in}{1.070230in}}{\pgfqpoint{5.513452in}{1.075273in}}%
\pgfpathcurveto{\pgfqpoint{5.508408in}{1.080317in}}{\pgfqpoint{5.501567in}{1.083151in}}{\pgfqpoint{5.494434in}{1.083151in}}%
\pgfpathcurveto{\pgfqpoint{5.487301in}{1.083151in}}{\pgfqpoint{5.480459in}{1.080317in}}{\pgfqpoint{5.475416in}{1.075273in}}%
\pgfpathcurveto{\pgfqpoint{5.470372in}{1.070230in}}{\pgfqpoint{5.467538in}{1.063388in}}{\pgfqpoint{5.467538in}{1.056255in}}%
\pgfpathcurveto{\pgfqpoint{5.467538in}{1.049122in}}{\pgfqpoint{5.470372in}{1.042281in}}{\pgfqpoint{5.475416in}{1.037237in}}%
\pgfpathcurveto{\pgfqpoint{5.480459in}{1.032193in}}{\pgfqpoint{5.487301in}{1.029359in}}{\pgfqpoint{5.494434in}{1.029359in}}%
\pgfpathclose%
\pgfusepath{stroke,fill}%
\end{pgfscope}%
\begin{pgfscope}%
\pgfpathrectangle{\pgfqpoint{4.985294in}{0.500000in}}{\pgfqpoint{1.764706in}{1.700000in}}%
\pgfusepath{clip}%
\pgfsetbuttcap%
\pgfsetroundjoin%
\definecolor{currentfill}{rgb}{0.964306,0.663930,0.507747}%
\pgfsetfillcolor{currentfill}%
\pgfsetlinewidth{0.311001pt}%
\definecolor{currentstroke}{rgb}{1.000000,1.000000,1.000000}%
\pgfsetstrokecolor{currentstroke}%
\pgfsetdash{}{0pt}%
\pgfpathmoveto{\pgfqpoint{5.587880in}{1.574071in}}%
\pgfpathcurveto{\pgfqpoint{5.595013in}{1.574071in}}{\pgfqpoint{5.601855in}{1.576905in}}{\pgfqpoint{5.606898in}{1.581948in}}%
\pgfpathcurveto{\pgfqpoint{5.611942in}{1.586992in}}{\pgfqpoint{5.614776in}{1.593834in}}{\pgfqpoint{5.614776in}{1.600966in}}%
\pgfpathcurveto{\pgfqpoint{5.614776in}{1.608099in}}{\pgfqpoint{5.611942in}{1.614941in}}{\pgfqpoint{5.606898in}{1.619985in}}%
\pgfpathcurveto{\pgfqpoint{5.601855in}{1.625028in}}{\pgfqpoint{5.595013in}{1.627862in}}{\pgfqpoint{5.587880in}{1.627862in}}%
\pgfpathcurveto{\pgfqpoint{5.580748in}{1.627862in}}{\pgfqpoint{5.573906in}{1.625028in}}{\pgfqpoint{5.568862in}{1.619985in}}%
\pgfpathcurveto{\pgfqpoint{5.563819in}{1.614941in}}{\pgfqpoint{5.560985in}{1.608099in}}{\pgfqpoint{5.560985in}{1.600966in}}%
\pgfpathcurveto{\pgfqpoint{5.560985in}{1.593834in}}{\pgfqpoint{5.563819in}{1.586992in}}{\pgfqpoint{5.568862in}{1.581948in}}%
\pgfpathcurveto{\pgfqpoint{5.573906in}{1.576905in}}{\pgfqpoint{5.580748in}{1.574071in}}{\pgfqpoint{5.587880in}{1.574071in}}%
\pgfpathclose%
\pgfusepath{stroke,fill}%
\end{pgfscope}%
\begin{pgfscope}%
\pgfpathrectangle{\pgfqpoint{4.985294in}{0.500000in}}{\pgfqpoint{1.764706in}{1.700000in}}%
\pgfusepath{clip}%
\pgfsetbuttcap%
\pgfsetroundjoin%
\definecolor{currentfill}{rgb}{0.972201,0.839051,0.745789}%
\pgfsetfillcolor{currentfill}%
\pgfsetlinewidth{0.311001pt}%
\definecolor{currentstroke}{rgb}{1.000000,1.000000,1.000000}%
\pgfsetstrokecolor{currentstroke}%
\pgfsetdash{}{0pt}%
\pgfpathmoveto{\pgfqpoint{5.524629in}{1.014818in}}%
\pgfpathcurveto{\pgfqpoint{5.531762in}{1.014818in}}{\pgfqpoint{5.538604in}{1.017652in}}{\pgfqpoint{5.543647in}{1.022695in}}%
\pgfpathcurveto{\pgfqpoint{5.548691in}{1.027739in}}{\pgfqpoint{5.551525in}{1.034581in}}{\pgfqpoint{5.551525in}{1.041713in}}%
\pgfpathcurveto{\pgfqpoint{5.551525in}{1.048846in}}{\pgfqpoint{5.548691in}{1.055688in}}{\pgfqpoint{5.543647in}{1.060732in}}%
\pgfpathcurveto{\pgfqpoint{5.538604in}{1.065775in}}{\pgfqpoint{5.531762in}{1.068609in}}{\pgfqpoint{5.524629in}{1.068609in}}%
\pgfpathcurveto{\pgfqpoint{5.517496in}{1.068609in}}{\pgfqpoint{5.510655in}{1.065775in}}{\pgfqpoint{5.505611in}{1.060732in}}%
\pgfpathcurveto{\pgfqpoint{5.500567in}{1.055688in}}{\pgfqpoint{5.497733in}{1.048846in}}{\pgfqpoint{5.497733in}{1.041713in}}%
\pgfpathcurveto{\pgfqpoint{5.497733in}{1.034581in}}{\pgfqpoint{5.500567in}{1.027739in}}{\pgfqpoint{5.505611in}{1.022695in}}%
\pgfpathcurveto{\pgfqpoint{5.510655in}{1.017652in}}{\pgfqpoint{5.517496in}{1.014818in}}{\pgfqpoint{5.524629in}{1.014818in}}%
\pgfpathclose%
\pgfusepath{stroke,fill}%
\end{pgfscope}%
\begin{pgfscope}%
\pgfpathrectangle{\pgfqpoint{4.985294in}{0.500000in}}{\pgfqpoint{1.764706in}{1.700000in}}%
\pgfusepath{clip}%
\pgfsetbuttcap%
\pgfsetroundjoin%
\definecolor{currentfill}{rgb}{0.972726,0.844889,0.754401}%
\pgfsetfillcolor{currentfill}%
\pgfsetlinewidth{0.311001pt}%
\definecolor{currentstroke}{rgb}{1.000000,1.000000,1.000000}%
\pgfsetstrokecolor{currentstroke}%
\pgfsetdash{}{0pt}%
\pgfpathmoveto{\pgfqpoint{5.368286in}{1.400167in}}%
\pgfpathcurveto{\pgfqpoint{5.375419in}{1.400167in}}{\pgfqpoint{5.382261in}{1.403001in}}{\pgfqpoint{5.387304in}{1.408045in}}%
\pgfpathcurveto{\pgfqpoint{5.392348in}{1.413089in}}{\pgfqpoint{5.395182in}{1.419930in}}{\pgfqpoint{5.395182in}{1.427063in}}%
\pgfpathcurveto{\pgfqpoint{5.395182in}{1.434196in}}{\pgfqpoint{5.392348in}{1.441037in}}{\pgfqpoint{5.387304in}{1.446081in}}%
\pgfpathcurveto{\pgfqpoint{5.382261in}{1.451125in}}{\pgfqpoint{5.375419in}{1.453959in}}{\pgfqpoint{5.368286in}{1.453959in}}%
\pgfpathcurveto{\pgfqpoint{5.361153in}{1.453959in}}{\pgfqpoint{5.354312in}{1.451125in}}{\pgfqpoint{5.349268in}{1.446081in}}%
\pgfpathcurveto{\pgfqpoint{5.344224in}{1.441037in}}{\pgfqpoint{5.341390in}{1.434196in}}{\pgfqpoint{5.341390in}{1.427063in}}%
\pgfpathcurveto{\pgfqpoint{5.341390in}{1.419930in}}{\pgfqpoint{5.344224in}{1.413089in}}{\pgfqpoint{5.349268in}{1.408045in}}%
\pgfpathcurveto{\pgfqpoint{5.354312in}{1.403001in}}{\pgfqpoint{5.361153in}{1.400167in}}{\pgfqpoint{5.368286in}{1.400167in}}%
\pgfpathclose%
\pgfusepath{stroke,fill}%
\end{pgfscope}%
\begin{pgfscope}%
\pgfpathrectangle{\pgfqpoint{4.985294in}{0.500000in}}{\pgfqpoint{1.764706in}{1.700000in}}%
\pgfusepath{clip}%
\pgfsetbuttcap%
\pgfsetroundjoin%
\definecolor{currentfill}{rgb}{0.976287,0.879862,0.805788}%
\pgfsetfillcolor{currentfill}%
\pgfsetlinewidth{0.311001pt}%
\definecolor{currentstroke}{rgb}{1.000000,1.000000,1.000000}%
\pgfsetstrokecolor{currentstroke}%
\pgfsetdash{}{0pt}%
\pgfpathmoveto{\pgfqpoint{6.307526in}{1.549235in}}%
\pgfpathcurveto{\pgfqpoint{6.314659in}{1.549235in}}{\pgfqpoint{6.321500in}{1.552069in}}{\pgfqpoint{6.326544in}{1.557112in}}%
\pgfpathcurveto{\pgfqpoint{6.331588in}{1.562156in}}{\pgfqpoint{6.334422in}{1.568998in}}{\pgfqpoint{6.334422in}{1.576130in}}%
\pgfpathcurveto{\pgfqpoint{6.334422in}{1.583263in}}{\pgfqpoint{6.331588in}{1.590105in}}{\pgfqpoint{6.326544in}{1.595149in}}%
\pgfpathcurveto{\pgfqpoint{6.321500in}{1.600192in}}{\pgfqpoint{6.314659in}{1.603026in}}{\pgfqpoint{6.307526in}{1.603026in}}%
\pgfpathcurveto{\pgfqpoint{6.300393in}{1.603026in}}{\pgfqpoint{6.293552in}{1.600192in}}{\pgfqpoint{6.288508in}{1.595149in}}%
\pgfpathcurveto{\pgfqpoint{6.283464in}{1.590105in}}{\pgfqpoint{6.280630in}{1.583263in}}{\pgfqpoint{6.280630in}{1.576130in}}%
\pgfpathcurveto{\pgfqpoint{6.280630in}{1.568998in}}{\pgfqpoint{6.283464in}{1.562156in}}{\pgfqpoint{6.288508in}{1.557112in}}%
\pgfpathcurveto{\pgfqpoint{6.293552in}{1.552069in}}{\pgfqpoint{6.300393in}{1.549235in}}{\pgfqpoint{6.307526in}{1.549235in}}%
\pgfpathclose%
\pgfusepath{stroke,fill}%
\end{pgfscope}%
\begin{pgfscope}%
\pgfpathrectangle{\pgfqpoint{4.985294in}{0.500000in}}{\pgfqpoint{1.764706in}{1.700000in}}%
\pgfusepath{clip}%
\pgfsetbuttcap%
\pgfsetroundjoin%
\definecolor{currentfill}{rgb}{0.974412,0.862387,0.780156}%
\pgfsetfillcolor{currentfill}%
\pgfsetlinewidth{0.311001pt}%
\definecolor{currentstroke}{rgb}{1.000000,1.000000,1.000000}%
\pgfsetstrokecolor{currentstroke}%
\pgfsetdash{}{0pt}%
\pgfpathmoveto{\pgfqpoint{5.478267in}{1.414626in}}%
\pgfpathcurveto{\pgfqpoint{5.485400in}{1.414626in}}{\pgfqpoint{5.492242in}{1.417460in}}{\pgfqpoint{5.497285in}{1.422504in}}%
\pgfpathcurveto{\pgfqpoint{5.502329in}{1.427547in}}{\pgfqpoint{5.505163in}{1.434389in}}{\pgfqpoint{5.505163in}{1.441522in}}%
\pgfpathcurveto{\pgfqpoint{5.505163in}{1.448655in}}{\pgfqpoint{5.502329in}{1.455496in}}{\pgfqpoint{5.497285in}{1.460540in}}%
\pgfpathcurveto{\pgfqpoint{5.492242in}{1.465584in}}{\pgfqpoint{5.485400in}{1.468418in}}{\pgfqpoint{5.478267in}{1.468418in}}%
\pgfpathcurveto{\pgfqpoint{5.471134in}{1.468418in}}{\pgfqpoint{5.464293in}{1.465584in}}{\pgfqpoint{5.459249in}{1.460540in}}%
\pgfpathcurveto{\pgfqpoint{5.454205in}{1.455496in}}{\pgfqpoint{5.451371in}{1.448655in}}{\pgfqpoint{5.451371in}{1.441522in}}%
\pgfpathcurveto{\pgfqpoint{5.451371in}{1.434389in}}{\pgfqpoint{5.454205in}{1.427547in}}{\pgfqpoint{5.459249in}{1.422504in}}%
\pgfpathcurveto{\pgfqpoint{5.464293in}{1.417460in}}{\pgfqpoint{5.471134in}{1.414626in}}{\pgfqpoint{5.478267in}{1.414626in}}%
\pgfpathclose%
\pgfusepath{stroke,fill}%
\end{pgfscope}%
\begin{pgfscope}%
\pgfpathrectangle{\pgfqpoint{4.985294in}{0.500000in}}{\pgfqpoint{1.764706in}{1.700000in}}%
\pgfusepath{clip}%
\pgfsetbuttcap%
\pgfsetroundjoin%
\definecolor{currentfill}{rgb}{0.964679,0.682838,0.530002}%
\pgfsetfillcolor{currentfill}%
\pgfsetlinewidth{0.311001pt}%
\definecolor{currentstroke}{rgb}{1.000000,1.000000,1.000000}%
\pgfsetstrokecolor{currentstroke}%
\pgfsetdash{}{0pt}%
\pgfpathmoveto{\pgfqpoint{5.450702in}{1.692900in}}%
\pgfpathcurveto{\pgfqpoint{5.457835in}{1.692900in}}{\pgfqpoint{5.464677in}{1.695734in}}{\pgfqpoint{5.469721in}{1.700778in}}%
\pgfpathcurveto{\pgfqpoint{5.474764in}{1.705821in}}{\pgfqpoint{5.477598in}{1.712663in}}{\pgfqpoint{5.477598in}{1.719796in}}%
\pgfpathcurveto{\pgfqpoint{5.477598in}{1.726929in}}{\pgfqpoint{5.474764in}{1.733770in}}{\pgfqpoint{5.469721in}{1.738814in}}%
\pgfpathcurveto{\pgfqpoint{5.464677in}{1.743858in}}{\pgfqpoint{5.457835in}{1.746691in}}{\pgfqpoint{5.450702in}{1.746691in}}%
\pgfpathcurveto{\pgfqpoint{5.443570in}{1.746691in}}{\pgfqpoint{5.436728in}{1.743858in}}{\pgfqpoint{5.431684in}{1.738814in}}%
\pgfpathcurveto{\pgfqpoint{5.426641in}{1.733770in}}{\pgfqpoint{5.423807in}{1.726929in}}{\pgfqpoint{5.423807in}{1.719796in}}%
\pgfpathcurveto{\pgfqpoint{5.423807in}{1.712663in}}{\pgfqpoint{5.426641in}{1.705821in}}{\pgfqpoint{5.431684in}{1.700778in}}%
\pgfpathcurveto{\pgfqpoint{5.436728in}{1.695734in}}{\pgfqpoint{5.443570in}{1.692900in}}{\pgfqpoint{5.450702in}{1.692900in}}%
\pgfpathclose%
\pgfusepath{stroke,fill}%
\end{pgfscope}%
\begin{pgfscope}%
\pgfpathrectangle{\pgfqpoint{4.985294in}{0.500000in}}{\pgfqpoint{1.764706in}{1.700000in}}%
\pgfusepath{clip}%
\pgfsetbuttcap%
\pgfsetroundjoin%
\definecolor{currentfill}{rgb}{0.975018,0.868213,0.788710}%
\pgfsetfillcolor{currentfill}%
\pgfsetlinewidth{0.311001pt}%
\definecolor{currentstroke}{rgb}{1.000000,1.000000,1.000000}%
\pgfsetstrokecolor{currentstroke}%
\pgfsetdash{}{0pt}%
\pgfpathmoveto{\pgfqpoint{6.260138in}{1.489271in}}%
\pgfpathcurveto{\pgfqpoint{6.267271in}{1.489271in}}{\pgfqpoint{6.274112in}{1.492105in}}{\pgfqpoint{6.279156in}{1.497148in}}%
\pgfpathcurveto{\pgfqpoint{6.284200in}{1.502192in}}{\pgfqpoint{6.287034in}{1.509034in}}{\pgfqpoint{6.287034in}{1.516166in}}%
\pgfpathcurveto{\pgfqpoint{6.287034in}{1.523299in}}{\pgfqpoint{6.284200in}{1.530141in}}{\pgfqpoint{6.279156in}{1.535185in}}%
\pgfpathcurveto{\pgfqpoint{6.274112in}{1.540228in}}{\pgfqpoint{6.267271in}{1.543062in}}{\pgfqpoint{6.260138in}{1.543062in}}%
\pgfpathcurveto{\pgfqpoint{6.253005in}{1.543062in}}{\pgfqpoint{6.246163in}{1.540228in}}{\pgfqpoint{6.241120in}{1.535185in}}%
\pgfpathcurveto{\pgfqpoint{6.236076in}{1.530141in}}{\pgfqpoint{6.233242in}{1.523299in}}{\pgfqpoint{6.233242in}{1.516166in}}%
\pgfpathcurveto{\pgfqpoint{6.233242in}{1.509034in}}{\pgfqpoint{6.236076in}{1.502192in}}{\pgfqpoint{6.241120in}{1.497148in}}%
\pgfpathcurveto{\pgfqpoint{6.246163in}{1.492105in}}{\pgfqpoint{6.253005in}{1.489271in}}{\pgfqpoint{6.260138in}{1.489271in}}%
\pgfpathclose%
\pgfusepath{stroke,fill}%
\end{pgfscope}%
\begin{pgfscope}%
\pgfpathrectangle{\pgfqpoint{4.985294in}{0.500000in}}{\pgfqpoint{1.764706in}{1.700000in}}%
\pgfusepath{clip}%
\pgfsetbuttcap%
\pgfsetroundjoin%
\definecolor{currentfill}{rgb}{0.969803,0.809811,0.702523}%
\pgfsetfillcolor{currentfill}%
\pgfsetlinewidth{0.311001pt}%
\definecolor{currentstroke}{rgb}{1.000000,1.000000,1.000000}%
\pgfsetstrokecolor{currentstroke}%
\pgfsetdash{}{0pt}%
\pgfpathmoveto{\pgfqpoint{5.431741in}{1.030185in}}%
\pgfpathcurveto{\pgfqpoint{5.438874in}{1.030185in}}{\pgfqpoint{5.445716in}{1.033019in}}{\pgfqpoint{5.450759in}{1.038063in}}%
\pgfpathcurveto{\pgfqpoint{5.455803in}{1.043107in}}{\pgfqpoint{5.458637in}{1.049948in}}{\pgfqpoint{5.458637in}{1.057081in}}%
\pgfpathcurveto{\pgfqpoint{5.458637in}{1.064214in}}{\pgfqpoint{5.455803in}{1.071056in}}{\pgfqpoint{5.450759in}{1.076099in}}%
\pgfpathcurveto{\pgfqpoint{5.445716in}{1.081143in}}{\pgfqpoint{5.438874in}{1.083977in}}{\pgfqpoint{5.431741in}{1.083977in}}%
\pgfpathcurveto{\pgfqpoint{5.424608in}{1.083977in}}{\pgfqpoint{5.417767in}{1.081143in}}{\pgfqpoint{5.412723in}{1.076099in}}%
\pgfpathcurveto{\pgfqpoint{5.407679in}{1.071056in}}{\pgfqpoint{5.404845in}{1.064214in}}{\pgfqpoint{5.404845in}{1.057081in}}%
\pgfpathcurveto{\pgfqpoint{5.404845in}{1.049948in}}{\pgfqpoint{5.407679in}{1.043107in}}{\pgfqpoint{5.412723in}{1.038063in}}%
\pgfpathcurveto{\pgfqpoint{5.417767in}{1.033019in}}{\pgfqpoint{5.424608in}{1.030185in}}{\pgfqpoint{5.431741in}{1.030185in}}%
\pgfpathclose%
\pgfusepath{stroke,fill}%
\end{pgfscope}%
\begin{pgfscope}%
\pgfpathrectangle{\pgfqpoint{4.985294in}{0.500000in}}{\pgfqpoint{1.764706in}{1.700000in}}%
\pgfusepath{clip}%
\pgfsetbuttcap%
\pgfsetroundjoin%
\definecolor{currentfill}{rgb}{0.973271,0.850724,0.762998}%
\pgfsetfillcolor{currentfill}%
\pgfsetlinewidth{0.311001pt}%
\definecolor{currentstroke}{rgb}{1.000000,1.000000,1.000000}%
\pgfsetstrokecolor{currentstroke}%
\pgfsetdash{}{0pt}%
\pgfpathmoveto{\pgfqpoint{6.328549in}{1.126760in}}%
\pgfpathcurveto{\pgfqpoint{6.335682in}{1.126760in}}{\pgfqpoint{6.342524in}{1.129594in}}{\pgfqpoint{6.347567in}{1.134638in}}%
\pgfpathcurveto{\pgfqpoint{6.352611in}{1.139682in}}{\pgfqpoint{6.355445in}{1.146523in}}{\pgfqpoint{6.355445in}{1.153656in}}%
\pgfpathcurveto{\pgfqpoint{6.355445in}{1.160789in}}{\pgfqpoint{6.352611in}{1.167631in}}{\pgfqpoint{6.347567in}{1.172674in}}%
\pgfpathcurveto{\pgfqpoint{6.342524in}{1.177718in}}{\pgfqpoint{6.335682in}{1.180552in}}{\pgfqpoint{6.328549in}{1.180552in}}%
\pgfpathcurveto{\pgfqpoint{6.321416in}{1.180552in}}{\pgfqpoint{6.314575in}{1.177718in}}{\pgfqpoint{6.309531in}{1.172674in}}%
\pgfpathcurveto{\pgfqpoint{6.304487in}{1.167631in}}{\pgfqpoint{6.301653in}{1.160789in}}{\pgfqpoint{6.301653in}{1.153656in}}%
\pgfpathcurveto{\pgfqpoint{6.301653in}{1.146523in}}{\pgfqpoint{6.304487in}{1.139682in}}{\pgfqpoint{6.309531in}{1.134638in}}%
\pgfpathcurveto{\pgfqpoint{6.314575in}{1.129594in}}{\pgfqpoint{6.321416in}{1.126760in}}{\pgfqpoint{6.328549in}{1.126760in}}%
\pgfpathclose%
\pgfusepath{stroke,fill}%
\end{pgfscope}%
\begin{pgfscope}%
\pgfpathrectangle{\pgfqpoint{4.985294in}{0.500000in}}{\pgfqpoint{1.764706in}{1.700000in}}%
\pgfusepath{clip}%
\pgfsetbuttcap%
\pgfsetroundjoin%
\definecolor{currentfill}{rgb}{0.970718,0.821518,0.719872}%
\pgfsetfillcolor{currentfill}%
\pgfsetlinewidth{0.311001pt}%
\definecolor{currentstroke}{rgb}{1.000000,1.000000,1.000000}%
\pgfsetstrokecolor{currentstroke}%
\pgfsetdash{}{0pt}%
\pgfpathmoveto{\pgfqpoint{6.366626in}{1.432373in}}%
\pgfpathcurveto{\pgfqpoint{6.373758in}{1.432373in}}{\pgfqpoint{6.380600in}{1.435207in}}{\pgfqpoint{6.385644in}{1.440251in}}%
\pgfpathcurveto{\pgfqpoint{6.390687in}{1.445294in}}{\pgfqpoint{6.393521in}{1.452136in}}{\pgfqpoint{6.393521in}{1.459269in}}%
\pgfpathcurveto{\pgfqpoint{6.393521in}{1.466402in}}{\pgfqpoint{6.390687in}{1.473243in}}{\pgfqpoint{6.385644in}{1.478287in}}%
\pgfpathcurveto{\pgfqpoint{6.380600in}{1.483331in}}{\pgfqpoint{6.373758in}{1.486165in}}{\pgfqpoint{6.366626in}{1.486165in}}%
\pgfpathcurveto{\pgfqpoint{6.359493in}{1.486165in}}{\pgfqpoint{6.352651in}{1.483331in}}{\pgfqpoint{6.347607in}{1.478287in}}%
\pgfpathcurveto{\pgfqpoint{6.342564in}{1.473243in}}{\pgfqpoint{6.339730in}{1.466402in}}{\pgfqpoint{6.339730in}{1.459269in}}%
\pgfpathcurveto{\pgfqpoint{6.339730in}{1.452136in}}{\pgfqpoint{6.342564in}{1.445294in}}{\pgfqpoint{6.347607in}{1.440251in}}%
\pgfpathcurveto{\pgfqpoint{6.352651in}{1.435207in}}{\pgfqpoint{6.359493in}{1.432373in}}{\pgfqpoint{6.366626in}{1.432373in}}%
\pgfpathclose%
\pgfusepath{stroke,fill}%
\end{pgfscope}%
\begin{pgfscope}%
\pgfpathrectangle{\pgfqpoint{4.985294in}{0.500000in}}{\pgfqpoint{1.764706in}{1.700000in}}%
\pgfusepath{clip}%
\pgfsetbuttcap%
\pgfsetroundjoin%
\definecolor{currentfill}{rgb}{0.965302,0.713942,0.568499}%
\pgfsetfillcolor{currentfill}%
\pgfsetlinewidth{0.311001pt}%
\definecolor{currentstroke}{rgb}{1.000000,1.000000,1.000000}%
\pgfsetstrokecolor{currentstroke}%
\pgfsetdash{}{0pt}%
\pgfpathmoveto{\pgfqpoint{5.441603in}{1.669888in}}%
\pgfpathcurveto{\pgfqpoint{5.448736in}{1.669888in}}{\pgfqpoint{5.455578in}{1.672722in}}{\pgfqpoint{5.460621in}{1.677765in}}%
\pgfpathcurveto{\pgfqpoint{5.465665in}{1.682809in}}{\pgfqpoint{5.468499in}{1.689651in}}{\pgfqpoint{5.468499in}{1.696783in}}%
\pgfpathcurveto{\pgfqpoint{5.468499in}{1.703916in}}{\pgfqpoint{5.465665in}{1.710758in}}{\pgfqpoint{5.460621in}{1.715802in}}%
\pgfpathcurveto{\pgfqpoint{5.455578in}{1.720845in}}{\pgfqpoint{5.448736in}{1.723679in}}{\pgfqpoint{5.441603in}{1.723679in}}%
\pgfpathcurveto{\pgfqpoint{5.434470in}{1.723679in}}{\pgfqpoint{5.427629in}{1.720845in}}{\pgfqpoint{5.422585in}{1.715802in}}%
\pgfpathcurveto{\pgfqpoint{5.417541in}{1.710758in}}{\pgfqpoint{5.414707in}{1.703916in}}{\pgfqpoint{5.414707in}{1.696783in}}%
\pgfpathcurveto{\pgfqpoint{5.414707in}{1.689651in}}{\pgfqpoint{5.417541in}{1.682809in}}{\pgfqpoint{5.422585in}{1.677765in}}%
\pgfpathcurveto{\pgfqpoint{5.427629in}{1.672722in}}{\pgfqpoint{5.434470in}{1.669888in}}{\pgfqpoint{5.441603in}{1.669888in}}%
\pgfpathclose%
\pgfusepath{stroke,fill}%
\end{pgfscope}%
\begin{pgfscope}%
\pgfpathrectangle{\pgfqpoint{4.985294in}{0.500000in}}{\pgfqpoint{1.764706in}{1.700000in}}%
\pgfusepath{clip}%
\pgfsetbuttcap%
\pgfsetroundjoin%
\definecolor{currentfill}{rgb}{0.958331,0.519463,0.362986}%
\pgfsetfillcolor{currentfill}%
\pgfsetlinewidth{0.311001pt}%
\definecolor{currentstroke}{rgb}{1.000000,1.000000,1.000000}%
\pgfsetstrokecolor{currentstroke}%
\pgfsetdash{}{0pt}%
\pgfpathmoveto{\pgfqpoint{5.632592in}{1.603949in}}%
\pgfpathcurveto{\pgfqpoint{5.639725in}{1.603949in}}{\pgfqpoint{5.646566in}{1.606783in}}{\pgfqpoint{5.651610in}{1.611827in}}%
\pgfpathcurveto{\pgfqpoint{5.656654in}{1.616870in}}{\pgfqpoint{5.659488in}{1.623712in}}{\pgfqpoint{5.659488in}{1.630845in}}%
\pgfpathcurveto{\pgfqpoint{5.659488in}{1.637978in}}{\pgfqpoint{5.656654in}{1.644819in}}{\pgfqpoint{5.651610in}{1.649863in}}%
\pgfpathcurveto{\pgfqpoint{5.646566in}{1.654907in}}{\pgfqpoint{5.639725in}{1.657740in}}{\pgfqpoint{5.632592in}{1.657740in}}%
\pgfpathcurveto{\pgfqpoint{5.625459in}{1.657740in}}{\pgfqpoint{5.618617in}{1.654907in}}{\pgfqpoint{5.613574in}{1.649863in}}%
\pgfpathcurveto{\pgfqpoint{5.608530in}{1.644819in}}{\pgfqpoint{5.605696in}{1.637978in}}{\pgfqpoint{5.605696in}{1.630845in}}%
\pgfpathcurveto{\pgfqpoint{5.605696in}{1.623712in}}{\pgfqpoint{5.608530in}{1.616870in}}{\pgfqpoint{5.613574in}{1.611827in}}%
\pgfpathcurveto{\pgfqpoint{5.618617in}{1.606783in}}{\pgfqpoint{5.625459in}{1.603949in}}{\pgfqpoint{5.632592in}{1.603949in}}%
\pgfpathclose%
\pgfusepath{stroke,fill}%
\end{pgfscope}%
\begin{pgfscope}%
\pgfpathrectangle{\pgfqpoint{4.985294in}{0.500000in}}{\pgfqpoint{1.764706in}{1.700000in}}%
\pgfusepath{clip}%
\pgfsetbuttcap%
\pgfsetroundjoin%
\definecolor{currentfill}{rgb}{0.966328,0.750560,0.616961}%
\pgfsetfillcolor{currentfill}%
\pgfsetlinewidth{0.311001pt}%
\definecolor{currentstroke}{rgb}{1.000000,1.000000,1.000000}%
\pgfsetstrokecolor{currentstroke}%
\pgfsetdash{}{0pt}%
\pgfpathmoveto{\pgfqpoint{5.526125in}{1.167857in}}%
\pgfpathcurveto{\pgfqpoint{5.533258in}{1.167857in}}{\pgfqpoint{5.540099in}{1.170690in}}{\pgfqpoint{5.545143in}{1.175734in}}%
\pgfpathcurveto{\pgfqpoint{5.550187in}{1.180778in}}{\pgfqpoint{5.553020in}{1.187619in}}{\pgfqpoint{5.553020in}{1.194752in}}%
\pgfpathcurveto{\pgfqpoint{5.553020in}{1.201885in}}{\pgfqpoint{5.550187in}{1.208727in}}{\pgfqpoint{5.545143in}{1.213770in}}%
\pgfpathcurveto{\pgfqpoint{5.540099in}{1.218814in}}{\pgfqpoint{5.533258in}{1.221648in}}{\pgfqpoint{5.526125in}{1.221648in}}%
\pgfpathcurveto{\pgfqpoint{5.518992in}{1.221648in}}{\pgfqpoint{5.512150in}{1.218814in}}{\pgfqpoint{5.507107in}{1.213770in}}%
\pgfpathcurveto{\pgfqpoint{5.502063in}{1.208727in}}{\pgfqpoint{5.499229in}{1.201885in}}{\pgfqpoint{5.499229in}{1.194752in}}%
\pgfpathcurveto{\pgfqpoint{5.499229in}{1.187619in}}{\pgfqpoint{5.502063in}{1.180778in}}{\pgfqpoint{5.507107in}{1.175734in}}%
\pgfpathcurveto{\pgfqpoint{5.512150in}{1.170690in}}{\pgfqpoint{5.518992in}{1.167857in}}{\pgfqpoint{5.526125in}{1.167857in}}%
\pgfpathclose%
\pgfusepath{stroke,fill}%
\end{pgfscope}%
\begin{pgfscope}%
\pgfpathrectangle{\pgfqpoint{4.985294in}{0.500000in}}{\pgfqpoint{1.764706in}{1.700000in}}%
\pgfusepath{clip}%
\pgfsetbuttcap%
\pgfsetroundjoin%
\definecolor{currentfill}{rgb}{0.972726,0.844889,0.754401}%
\pgfsetfillcolor{currentfill}%
\pgfsetlinewidth{0.311001pt}%
\definecolor{currentstroke}{rgb}{1.000000,1.000000,1.000000}%
\pgfsetstrokecolor{currentstroke}%
\pgfsetdash{}{0pt}%
\pgfpathmoveto{\pgfqpoint{5.494661in}{1.592438in}}%
\pgfpathcurveto{\pgfqpoint{5.501794in}{1.592438in}}{\pgfqpoint{5.508636in}{1.595272in}}{\pgfqpoint{5.513679in}{1.600315in}}%
\pgfpathcurveto{\pgfqpoint{5.518723in}{1.605359in}}{\pgfqpoint{5.521557in}{1.612201in}}{\pgfqpoint{5.521557in}{1.619334in}}%
\pgfpathcurveto{\pgfqpoint{5.521557in}{1.626466in}}{\pgfqpoint{5.518723in}{1.633308in}}{\pgfqpoint{5.513679in}{1.638352in}}%
\pgfpathcurveto{\pgfqpoint{5.508636in}{1.643395in}}{\pgfqpoint{5.501794in}{1.646229in}}{\pgfqpoint{5.494661in}{1.646229in}}%
\pgfpathcurveto{\pgfqpoint{5.487528in}{1.646229in}}{\pgfqpoint{5.480687in}{1.643395in}}{\pgfqpoint{5.475643in}{1.638352in}}%
\pgfpathcurveto{\pgfqpoint{5.470599in}{1.633308in}}{\pgfqpoint{5.467765in}{1.626466in}}{\pgfqpoint{5.467765in}{1.619334in}}%
\pgfpathcurveto{\pgfqpoint{5.467765in}{1.612201in}}{\pgfqpoint{5.470599in}{1.605359in}}{\pgfqpoint{5.475643in}{1.600315in}}%
\pgfpathcurveto{\pgfqpoint{5.480687in}{1.595272in}}{\pgfqpoint{5.487528in}{1.592438in}}{\pgfqpoint{5.494661in}{1.592438in}}%
\pgfpathclose%
\pgfusepath{stroke,fill}%
\end{pgfscope}%
\begin{pgfscope}%
\pgfpathrectangle{\pgfqpoint{4.985294in}{0.500000in}}{\pgfqpoint{1.764706in}{1.700000in}}%
\pgfusepath{clip}%
\pgfsetbuttcap%
\pgfsetroundjoin%
\definecolor{currentfill}{rgb}{0.952404,0.449449,0.307210}%
\pgfsetfillcolor{currentfill}%
\pgfsetlinewidth{0.311001pt}%
\definecolor{currentstroke}{rgb}{1.000000,1.000000,1.000000}%
\pgfsetstrokecolor{currentstroke}%
\pgfsetdash{}{0pt}%
\pgfpathmoveto{\pgfqpoint{6.307063in}{1.719426in}}%
\pgfpathcurveto{\pgfqpoint{6.314196in}{1.719426in}}{\pgfqpoint{6.321037in}{1.722260in}}{\pgfqpoint{6.326081in}{1.727304in}}%
\pgfpathcurveto{\pgfqpoint{6.331125in}{1.732347in}}{\pgfqpoint{6.333959in}{1.739189in}}{\pgfqpoint{6.333959in}{1.746322in}}%
\pgfpathcurveto{\pgfqpoint{6.333959in}{1.753455in}}{\pgfqpoint{6.331125in}{1.760296in}}{\pgfqpoint{6.326081in}{1.765340in}}%
\pgfpathcurveto{\pgfqpoint{6.321037in}{1.770384in}}{\pgfqpoint{6.314196in}{1.773218in}}{\pgfqpoint{6.307063in}{1.773218in}}%
\pgfpathcurveto{\pgfqpoint{6.299930in}{1.773218in}}{\pgfqpoint{6.293088in}{1.770384in}}{\pgfqpoint{6.288045in}{1.765340in}}%
\pgfpathcurveto{\pgfqpoint{6.283001in}{1.760296in}}{\pgfqpoint{6.280167in}{1.753455in}}{\pgfqpoint{6.280167in}{1.746322in}}%
\pgfpathcurveto{\pgfqpoint{6.280167in}{1.739189in}}{\pgfqpoint{6.283001in}{1.732347in}}{\pgfqpoint{6.288045in}{1.727304in}}%
\pgfpathcurveto{\pgfqpoint{6.293088in}{1.722260in}}{\pgfqpoint{6.299930in}{1.719426in}}{\pgfqpoint{6.307063in}{1.719426in}}%
\pgfpathclose%
\pgfusepath{stroke,fill}%
\end{pgfscope}%
\begin{pgfscope}%
\pgfpathrectangle{\pgfqpoint{4.985294in}{0.500000in}}{\pgfqpoint{1.764706in}{1.700000in}}%
\pgfusepath{clip}%
\pgfsetbuttcap%
\pgfsetroundjoin%
\definecolor{currentfill}{rgb}{0.957344,0.505732,0.351309}%
\pgfsetfillcolor{currentfill}%
\pgfsetlinewidth{0.311001pt}%
\definecolor{currentstroke}{rgb}{1.000000,1.000000,1.000000}%
\pgfsetstrokecolor{currentstroke}%
\pgfsetdash{}{0pt}%
\pgfpathmoveto{\pgfqpoint{5.645674in}{1.037148in}}%
\pgfpathcurveto{\pgfqpoint{5.652807in}{1.037148in}}{\pgfqpoint{5.659648in}{1.039982in}}{\pgfqpoint{5.664692in}{1.045026in}}%
\pgfpathcurveto{\pgfqpoint{5.669736in}{1.050070in}}{\pgfqpoint{5.672570in}{1.056911in}}{\pgfqpoint{5.672570in}{1.064044in}}%
\pgfpathcurveto{\pgfqpoint{5.672570in}{1.071177in}}{\pgfqpoint{5.669736in}{1.078019in}}{\pgfqpoint{5.664692in}{1.083062in}}%
\pgfpathcurveto{\pgfqpoint{5.659648in}{1.088106in}}{\pgfqpoint{5.652807in}{1.090940in}}{\pgfqpoint{5.645674in}{1.090940in}}%
\pgfpathcurveto{\pgfqpoint{5.638541in}{1.090940in}}{\pgfqpoint{5.631699in}{1.088106in}}{\pgfqpoint{5.626656in}{1.083062in}}%
\pgfpathcurveto{\pgfqpoint{5.621612in}{1.078019in}}{\pgfqpoint{5.618778in}{1.071177in}}{\pgfqpoint{5.618778in}{1.064044in}}%
\pgfpathcurveto{\pgfqpoint{5.618778in}{1.056911in}}{\pgfqpoint{5.621612in}{1.050070in}}{\pgfqpoint{5.626656in}{1.045026in}}%
\pgfpathcurveto{\pgfqpoint{5.631699in}{1.039982in}}{\pgfqpoint{5.638541in}{1.037148in}}{\pgfqpoint{5.645674in}{1.037148in}}%
\pgfpathclose%
\pgfusepath{stroke,fill}%
\end{pgfscope}%
\begin{pgfscope}%
\pgfpathrectangle{\pgfqpoint{4.985294in}{0.500000in}}{\pgfqpoint{1.764706in}{1.700000in}}%
\pgfusepath{clip}%
\pgfsetbuttcap%
\pgfsetroundjoin%
\definecolor{currentfill}{rgb}{0.979891,0.908948,0.848279}%
\pgfsetfillcolor{currentfill}%
\pgfsetlinewidth{0.311001pt}%
\definecolor{currentstroke}{rgb}{1.000000,1.000000,1.000000}%
\pgfsetstrokecolor{currentstroke}%
\pgfsetdash{}{0pt}%
\pgfpathmoveto{\pgfqpoint{6.325223in}{1.428131in}}%
\pgfpathcurveto{\pgfqpoint{6.332356in}{1.428131in}}{\pgfqpoint{6.339198in}{1.430965in}}{\pgfqpoint{6.344241in}{1.436009in}}%
\pgfpathcurveto{\pgfqpoint{6.349285in}{1.441052in}}{\pgfqpoint{6.352119in}{1.447894in}}{\pgfqpoint{6.352119in}{1.455027in}}%
\pgfpathcurveto{\pgfqpoint{6.352119in}{1.462160in}}{\pgfqpoint{6.349285in}{1.469001in}}{\pgfqpoint{6.344241in}{1.474045in}}%
\pgfpathcurveto{\pgfqpoint{6.339198in}{1.479089in}}{\pgfqpoint{6.332356in}{1.481923in}}{\pgfqpoint{6.325223in}{1.481923in}}%
\pgfpathcurveto{\pgfqpoint{6.318090in}{1.481923in}}{\pgfqpoint{6.311249in}{1.479089in}}{\pgfqpoint{6.306205in}{1.474045in}}%
\pgfpathcurveto{\pgfqpoint{6.301161in}{1.469001in}}{\pgfqpoint{6.298328in}{1.462160in}}{\pgfqpoint{6.298328in}{1.455027in}}%
\pgfpathcurveto{\pgfqpoint{6.298328in}{1.447894in}}{\pgfqpoint{6.301161in}{1.441052in}}{\pgfqpoint{6.306205in}{1.436009in}}%
\pgfpathcurveto{\pgfqpoint{6.311249in}{1.430965in}}{\pgfqpoint{6.318090in}{1.428131in}}{\pgfqpoint{6.325223in}{1.428131in}}%
\pgfpathclose%
\pgfusepath{stroke,fill}%
\end{pgfscope}%
\begin{pgfscope}%
\pgfpathrectangle{\pgfqpoint{4.985294in}{0.500000in}}{\pgfqpoint{1.764706in}{1.700000in}}%
\pgfusepath{clip}%
\pgfsetbuttcap%
\pgfsetroundjoin%
\definecolor{currentfill}{rgb}{0.967398,0.774513,0.650573}%
\pgfsetfillcolor{currentfill}%
\pgfsetlinewidth{0.311001pt}%
\definecolor{currentstroke}{rgb}{1.000000,1.000000,1.000000}%
\pgfsetstrokecolor{currentstroke}%
\pgfsetdash{}{0pt}%
\pgfpathmoveto{\pgfqpoint{5.557550in}{1.576797in}}%
\pgfpathcurveto{\pgfqpoint{5.564683in}{1.576797in}}{\pgfqpoint{5.571524in}{1.579631in}}{\pgfqpoint{5.576568in}{1.584675in}}%
\pgfpathcurveto{\pgfqpoint{5.581612in}{1.589718in}}{\pgfqpoint{5.584445in}{1.596560in}}{\pgfqpoint{5.584445in}{1.603693in}}%
\pgfpathcurveto{\pgfqpoint{5.584445in}{1.610826in}}{\pgfqpoint{5.581612in}{1.617667in}}{\pgfqpoint{5.576568in}{1.622711in}}%
\pgfpathcurveto{\pgfqpoint{5.571524in}{1.627755in}}{\pgfqpoint{5.564683in}{1.630588in}}{\pgfqpoint{5.557550in}{1.630588in}}%
\pgfpathcurveto{\pgfqpoint{5.550417in}{1.630588in}}{\pgfqpoint{5.543575in}{1.627755in}}{\pgfqpoint{5.538532in}{1.622711in}}%
\pgfpathcurveto{\pgfqpoint{5.533488in}{1.617667in}}{\pgfqpoint{5.530654in}{1.610826in}}{\pgfqpoint{5.530654in}{1.603693in}}%
\pgfpathcurveto{\pgfqpoint{5.530654in}{1.596560in}}{\pgfqpoint{5.533488in}{1.589718in}}{\pgfqpoint{5.538532in}{1.584675in}}%
\pgfpathcurveto{\pgfqpoint{5.543575in}{1.579631in}}{\pgfqpoint{5.550417in}{1.576797in}}{\pgfqpoint{5.557550in}{1.576797in}}%
\pgfpathclose%
\pgfusepath{stroke,fill}%
\end{pgfscope}%
\begin{pgfscope}%
\pgfpathrectangle{\pgfqpoint{4.985294in}{0.500000in}}{\pgfqpoint{1.764706in}{1.700000in}}%
\pgfusepath{clip}%
\pgfsetbuttcap%
\pgfsetroundjoin%
\definecolor{currentfill}{rgb}{0.972726,0.844889,0.754401}%
\pgfsetfillcolor{currentfill}%
\pgfsetlinewidth{0.311001pt}%
\definecolor{currentstroke}{rgb}{1.000000,1.000000,1.000000}%
\pgfsetstrokecolor{currentstroke}%
\pgfsetdash{}{0pt}%
\pgfpathmoveto{\pgfqpoint{6.299996in}{1.081322in}}%
\pgfpathcurveto{\pgfqpoint{6.307129in}{1.081322in}}{\pgfqpoint{6.313970in}{1.084156in}}{\pgfqpoint{6.319014in}{1.089200in}}%
\pgfpathcurveto{\pgfqpoint{6.324058in}{1.094244in}}{\pgfqpoint{6.326892in}{1.101085in}}{\pgfqpoint{6.326892in}{1.108218in}}%
\pgfpathcurveto{\pgfqpoint{6.326892in}{1.115351in}}{\pgfqpoint{6.324058in}{1.122193in}}{\pgfqpoint{6.319014in}{1.127236in}}%
\pgfpathcurveto{\pgfqpoint{6.313970in}{1.132280in}}{\pgfqpoint{6.307129in}{1.135114in}}{\pgfqpoint{6.299996in}{1.135114in}}%
\pgfpathcurveto{\pgfqpoint{6.292863in}{1.135114in}}{\pgfqpoint{6.286022in}{1.132280in}}{\pgfqpoint{6.280978in}{1.127236in}}%
\pgfpathcurveto{\pgfqpoint{6.275934in}{1.122193in}}{\pgfqpoint{6.273100in}{1.115351in}}{\pgfqpoint{6.273100in}{1.108218in}}%
\pgfpathcurveto{\pgfqpoint{6.273100in}{1.101085in}}{\pgfqpoint{6.275934in}{1.094244in}}{\pgfqpoint{6.280978in}{1.089200in}}%
\pgfpathcurveto{\pgfqpoint{6.286022in}{1.084156in}}{\pgfqpoint{6.292863in}{1.081322in}}{\pgfqpoint{6.299996in}{1.081322in}}%
\pgfpathclose%
\pgfusepath{stroke,fill}%
\end{pgfscope}%
\begin{pgfscope}%
\pgfpathrectangle{\pgfqpoint{4.985294in}{0.500000in}}{\pgfqpoint{1.764706in}{1.700000in}}%
\pgfusepath{clip}%
\pgfsetbuttcap%
\pgfsetroundjoin%
\definecolor{currentfill}{rgb}{0.977657,0.891500,0.822809}%
\pgfsetfillcolor{currentfill}%
\pgfsetlinewidth{0.311001pt}%
\definecolor{currentstroke}{rgb}{1.000000,1.000000,1.000000}%
\pgfsetstrokecolor{currentstroke}%
\pgfsetdash{}{0pt}%
\pgfpathmoveto{\pgfqpoint{5.453330in}{1.262614in}}%
\pgfpathcurveto{\pgfqpoint{5.460463in}{1.262614in}}{\pgfqpoint{5.467305in}{1.265448in}}{\pgfqpoint{5.472349in}{1.270491in}}%
\pgfpathcurveto{\pgfqpoint{5.477392in}{1.275535in}}{\pgfqpoint{5.480226in}{1.282377in}}{\pgfqpoint{5.480226in}{1.289510in}}%
\pgfpathcurveto{\pgfqpoint{5.480226in}{1.296642in}}{\pgfqpoint{5.477392in}{1.303484in}}{\pgfqpoint{5.472349in}{1.308528in}}%
\pgfpathcurveto{\pgfqpoint{5.467305in}{1.313571in}}{\pgfqpoint{5.460463in}{1.316405in}}{\pgfqpoint{5.453330in}{1.316405in}}%
\pgfpathcurveto{\pgfqpoint{5.446198in}{1.316405in}}{\pgfqpoint{5.439356in}{1.313571in}}{\pgfqpoint{5.434312in}{1.308528in}}%
\pgfpathcurveto{\pgfqpoint{5.429269in}{1.303484in}}{\pgfqpoint{5.426435in}{1.296642in}}{\pgfqpoint{5.426435in}{1.289510in}}%
\pgfpathcurveto{\pgfqpoint{5.426435in}{1.282377in}}{\pgfqpoint{5.429269in}{1.275535in}}{\pgfqpoint{5.434312in}{1.270491in}}%
\pgfpathcurveto{\pgfqpoint{5.439356in}{1.265448in}}{\pgfqpoint{5.446198in}{1.262614in}}{\pgfqpoint{5.453330in}{1.262614in}}%
\pgfpathclose%
\pgfusepath{stroke,fill}%
\end{pgfscope}%
\begin{pgfscope}%
\pgfpathrectangle{\pgfqpoint{4.985294in}{0.500000in}}{\pgfqpoint{1.764706in}{1.700000in}}%
\pgfusepath{clip}%
\pgfsetbuttcap%
\pgfsetroundjoin%
\definecolor{currentfill}{rgb}{0.976287,0.879862,0.805788}%
\pgfsetfillcolor{currentfill}%
\pgfsetlinewidth{0.311001pt}%
\definecolor{currentstroke}{rgb}{1.000000,1.000000,1.000000}%
\pgfsetstrokecolor{currentstroke}%
\pgfsetdash{}{0pt}%
\pgfpathmoveto{\pgfqpoint{6.351939in}{1.391739in}}%
\pgfpathcurveto{\pgfqpoint{6.359072in}{1.391739in}}{\pgfqpoint{6.365914in}{1.394573in}}{\pgfqpoint{6.370957in}{1.399616in}}%
\pgfpathcurveto{\pgfqpoint{6.376001in}{1.404660in}}{\pgfqpoint{6.378835in}{1.411502in}}{\pgfqpoint{6.378835in}{1.418634in}}%
\pgfpathcurveto{\pgfqpoint{6.378835in}{1.425767in}}{\pgfqpoint{6.376001in}{1.432609in}}{\pgfqpoint{6.370957in}{1.437653in}}%
\pgfpathcurveto{\pgfqpoint{6.365914in}{1.442696in}}{\pgfqpoint{6.359072in}{1.445530in}}{\pgfqpoint{6.351939in}{1.445530in}}%
\pgfpathcurveto{\pgfqpoint{6.344806in}{1.445530in}}{\pgfqpoint{6.337965in}{1.442696in}}{\pgfqpoint{6.332921in}{1.437653in}}%
\pgfpathcurveto{\pgfqpoint{6.327877in}{1.432609in}}{\pgfqpoint{6.325044in}{1.425767in}}{\pgfqpoint{6.325044in}{1.418634in}}%
\pgfpathcurveto{\pgfqpoint{6.325044in}{1.411502in}}{\pgfqpoint{6.327877in}{1.404660in}}{\pgfqpoint{6.332921in}{1.399616in}}%
\pgfpathcurveto{\pgfqpoint{6.337965in}{1.394573in}}{\pgfqpoint{6.344806in}{1.391739in}}{\pgfqpoint{6.351939in}{1.391739in}}%
\pgfpathclose%
\pgfusepath{stroke,fill}%
\end{pgfscope}%
\begin{pgfscope}%
\pgfpathrectangle{\pgfqpoint{4.985294in}{0.500000in}}{\pgfqpoint{1.764706in}{1.700000in}}%
\pgfusepath{clip}%
\pgfsetbuttcap%
\pgfsetroundjoin%
\definecolor{currentfill}{rgb}{0.870791,0.179821,0.267974}%
\pgfsetfillcolor{currentfill}%
\pgfsetlinewidth{0.311001pt}%
\definecolor{currentstroke}{rgb}{1.000000,1.000000,1.000000}%
\pgfsetstrokecolor{currentstroke}%
\pgfsetdash{}{0pt}%
\pgfpathmoveto{\pgfqpoint{5.243414in}{1.338186in}}%
\pgfpathcurveto{\pgfqpoint{5.250547in}{1.338186in}}{\pgfqpoint{5.257389in}{1.341020in}}{\pgfqpoint{5.262433in}{1.346063in}}%
\pgfpathcurveto{\pgfqpoint{5.267476in}{1.351107in}}{\pgfqpoint{5.270310in}{1.357949in}}{\pgfqpoint{5.270310in}{1.365081in}}%
\pgfpathcurveto{\pgfqpoint{5.270310in}{1.372214in}}{\pgfqpoint{5.267476in}{1.379056in}}{\pgfqpoint{5.262433in}{1.384100in}}%
\pgfpathcurveto{\pgfqpoint{5.257389in}{1.389143in}}{\pgfqpoint{5.250547in}{1.391977in}}{\pgfqpoint{5.243414in}{1.391977in}}%
\pgfpathcurveto{\pgfqpoint{5.236282in}{1.391977in}}{\pgfqpoint{5.229440in}{1.389143in}}{\pgfqpoint{5.224396in}{1.384100in}}%
\pgfpathcurveto{\pgfqpoint{5.219353in}{1.379056in}}{\pgfqpoint{5.216519in}{1.372214in}}{\pgfqpoint{5.216519in}{1.365081in}}%
\pgfpathcurveto{\pgfqpoint{5.216519in}{1.357949in}}{\pgfqpoint{5.219353in}{1.351107in}}{\pgfqpoint{5.224396in}{1.346063in}}%
\pgfpathcurveto{\pgfqpoint{5.229440in}{1.341020in}}{\pgfqpoint{5.236282in}{1.338186in}}{\pgfqpoint{5.243414in}{1.338186in}}%
\pgfpathclose%
\pgfusepath{stroke,fill}%
\end{pgfscope}%
\begin{pgfscope}%
\pgfpathrectangle{\pgfqpoint{4.985294in}{0.500000in}}{\pgfqpoint{1.764706in}{1.700000in}}%
\pgfusepath{clip}%
\pgfsetbuttcap%
\pgfsetroundjoin%
\definecolor{currentfill}{rgb}{0.981377,0.920617,0.865369}%
\pgfsetfillcolor{currentfill}%
\pgfsetlinewidth{0.311001pt}%
\definecolor{currentstroke}{rgb}{1.000000,1.000000,1.000000}%
\pgfsetstrokecolor{currentstroke}%
\pgfsetdash{}{0pt}%
\pgfpathmoveto{\pgfqpoint{6.303511in}{1.484196in}}%
\pgfpathcurveto{\pgfqpoint{6.310644in}{1.484196in}}{\pgfqpoint{6.317486in}{1.487030in}}{\pgfqpoint{6.322530in}{1.492073in}}%
\pgfpathcurveto{\pgfqpoint{6.327573in}{1.497117in}}{\pgfqpoint{6.330407in}{1.503959in}}{\pgfqpoint{6.330407in}{1.511092in}}%
\pgfpathcurveto{\pgfqpoint{6.330407in}{1.518224in}}{\pgfqpoint{6.327573in}{1.525066in}}{\pgfqpoint{6.322530in}{1.530110in}}%
\pgfpathcurveto{\pgfqpoint{6.317486in}{1.535153in}}{\pgfqpoint{6.310644in}{1.537987in}}{\pgfqpoint{6.303511in}{1.537987in}}%
\pgfpathcurveto{\pgfqpoint{6.296379in}{1.537987in}}{\pgfqpoint{6.289537in}{1.535153in}}{\pgfqpoint{6.284493in}{1.530110in}}%
\pgfpathcurveto{\pgfqpoint{6.279450in}{1.525066in}}{\pgfqpoint{6.276616in}{1.518224in}}{\pgfqpoint{6.276616in}{1.511092in}}%
\pgfpathcurveto{\pgfqpoint{6.276616in}{1.503959in}}{\pgfqpoint{6.279450in}{1.497117in}}{\pgfqpoint{6.284493in}{1.492073in}}%
\pgfpathcurveto{\pgfqpoint{6.289537in}{1.487030in}}{\pgfqpoint{6.296379in}{1.484196in}}{\pgfqpoint{6.303511in}{1.484196in}}%
\pgfpathclose%
\pgfusepath{stroke,fill}%
\end{pgfscope}%
\begin{pgfscope}%
\pgfpathrectangle{\pgfqpoint{4.985294in}{0.500000in}}{\pgfqpoint{1.764706in}{1.700000in}}%
\pgfusepath{clip}%
\pgfsetbuttcap%
\pgfsetroundjoin%
\definecolor{currentfill}{rgb}{0.962765,0.606121,0.444717}%
\pgfsetfillcolor{currentfill}%
\pgfsetlinewidth{0.311001pt}%
\definecolor{currentstroke}{rgb}{1.000000,1.000000,1.000000}%
\pgfsetstrokecolor{currentstroke}%
\pgfsetdash{}{0pt}%
\pgfpathmoveto{\pgfqpoint{5.544450in}{1.269128in}}%
\pgfpathcurveto{\pgfqpoint{5.551583in}{1.269128in}}{\pgfqpoint{5.558425in}{1.271962in}}{\pgfqpoint{5.563468in}{1.277006in}}%
\pgfpathcurveto{\pgfqpoint{5.568512in}{1.282049in}}{\pgfqpoint{5.571346in}{1.288891in}}{\pgfqpoint{5.571346in}{1.296024in}}%
\pgfpathcurveto{\pgfqpoint{5.571346in}{1.303157in}}{\pgfqpoint{5.568512in}{1.309998in}}{\pgfqpoint{5.563468in}{1.315042in}}%
\pgfpathcurveto{\pgfqpoint{5.558425in}{1.320086in}}{\pgfqpoint{5.551583in}{1.322920in}}{\pgfqpoint{5.544450in}{1.322920in}}%
\pgfpathcurveto{\pgfqpoint{5.537317in}{1.322920in}}{\pgfqpoint{5.530476in}{1.320086in}}{\pgfqpoint{5.525432in}{1.315042in}}%
\pgfpathcurveto{\pgfqpoint{5.520388in}{1.309998in}}{\pgfqpoint{5.517554in}{1.303157in}}{\pgfqpoint{5.517554in}{1.296024in}}%
\pgfpathcurveto{\pgfqpoint{5.517554in}{1.288891in}}{\pgfqpoint{5.520388in}{1.282049in}}{\pgfqpoint{5.525432in}{1.277006in}}%
\pgfpathcurveto{\pgfqpoint{5.530476in}{1.271962in}}{\pgfqpoint{5.537317in}{1.269128in}}{\pgfqpoint{5.544450in}{1.269128in}}%
\pgfpathclose%
\pgfusepath{stroke,fill}%
\end{pgfscope}%
\begin{pgfscope}%
\pgfpathrectangle{\pgfqpoint{4.985294in}{0.500000in}}{\pgfqpoint{1.764706in}{1.700000in}}%
\pgfusepath{clip}%
\pgfsetbuttcap%
\pgfsetroundjoin%
\definecolor{currentfill}{rgb}{0.975644,0.874038,0.797253}%
\pgfsetfillcolor{currentfill}%
\pgfsetlinewidth{0.311001pt}%
\definecolor{currentstroke}{rgb}{1.000000,1.000000,1.000000}%
\pgfsetstrokecolor{currentstroke}%
\pgfsetdash{}{0pt}%
\pgfpathmoveto{\pgfqpoint{6.272339in}{1.406277in}}%
\pgfpathcurveto{\pgfqpoint{6.279472in}{1.406277in}}{\pgfqpoint{6.286313in}{1.409111in}}{\pgfqpoint{6.291357in}{1.414154in}}%
\pgfpathcurveto{\pgfqpoint{6.296400in}{1.419198in}}{\pgfqpoint{6.299234in}{1.426040in}}{\pgfqpoint{6.299234in}{1.433173in}}%
\pgfpathcurveto{\pgfqpoint{6.299234in}{1.440305in}}{\pgfqpoint{6.296400in}{1.447147in}}{\pgfqpoint{6.291357in}{1.452191in}}%
\pgfpathcurveto{\pgfqpoint{6.286313in}{1.457234in}}{\pgfqpoint{6.279472in}{1.460068in}}{\pgfqpoint{6.272339in}{1.460068in}}%
\pgfpathcurveto{\pgfqpoint{6.265206in}{1.460068in}}{\pgfqpoint{6.258364in}{1.457234in}}{\pgfqpoint{6.253321in}{1.452191in}}%
\pgfpathcurveto{\pgfqpoint{6.248277in}{1.447147in}}{\pgfqpoint{6.245443in}{1.440305in}}{\pgfqpoint{6.245443in}{1.433173in}}%
\pgfpathcurveto{\pgfqpoint{6.245443in}{1.426040in}}{\pgfqpoint{6.248277in}{1.419198in}}{\pgfqpoint{6.253321in}{1.414154in}}%
\pgfpathcurveto{\pgfqpoint{6.258364in}{1.409111in}}{\pgfqpoint{6.265206in}{1.406277in}}{\pgfqpoint{6.272339in}{1.406277in}}%
\pgfpathclose%
\pgfusepath{stroke,fill}%
\end{pgfscope}%
\begin{pgfscope}%
\pgfpathrectangle{\pgfqpoint{4.985294in}{0.500000in}}{\pgfqpoint{1.764706in}{1.700000in}}%
\pgfusepath{clip}%
\pgfsetbuttcap%
\pgfsetroundjoin%
\definecolor{currentfill}{rgb}{0.963728,0.638439,0.479050}%
\pgfsetfillcolor{currentfill}%
\pgfsetlinewidth{0.311001pt}%
\definecolor{currentstroke}{rgb}{1.000000,1.000000,1.000000}%
\pgfsetstrokecolor{currentstroke}%
\pgfsetdash{}{0pt}%
\pgfpathmoveto{\pgfqpoint{5.604962in}{1.053128in}}%
\pgfpathcurveto{\pgfqpoint{5.612095in}{1.053128in}}{\pgfqpoint{5.618937in}{1.055962in}}{\pgfqpoint{5.623981in}{1.061006in}}%
\pgfpathcurveto{\pgfqpoint{5.629024in}{1.066049in}}{\pgfqpoint{5.631858in}{1.072891in}}{\pgfqpoint{5.631858in}{1.080024in}}%
\pgfpathcurveto{\pgfqpoint{5.631858in}{1.087157in}}{\pgfqpoint{5.629024in}{1.093998in}}{\pgfqpoint{5.623981in}{1.099042in}}%
\pgfpathcurveto{\pgfqpoint{5.618937in}{1.104085in}}{\pgfqpoint{5.612095in}{1.106919in}}{\pgfqpoint{5.604962in}{1.106919in}}%
\pgfpathcurveto{\pgfqpoint{5.597830in}{1.106919in}}{\pgfqpoint{5.590988in}{1.104085in}}{\pgfqpoint{5.585944in}{1.099042in}}%
\pgfpathcurveto{\pgfqpoint{5.580901in}{1.093998in}}{\pgfqpoint{5.578067in}{1.087157in}}{\pgfqpoint{5.578067in}{1.080024in}}%
\pgfpathcurveto{\pgfqpoint{5.578067in}{1.072891in}}{\pgfqpoint{5.580901in}{1.066049in}}{\pgfqpoint{5.585944in}{1.061006in}}%
\pgfpathcurveto{\pgfqpoint{5.590988in}{1.055962in}}{\pgfqpoint{5.597830in}{1.053128in}}{\pgfqpoint{5.604962in}{1.053128in}}%
\pgfpathclose%
\pgfusepath{stroke,fill}%
\end{pgfscope}%
\begin{pgfscope}%
\pgfpathrectangle{\pgfqpoint{4.985294in}{0.500000in}}{\pgfqpoint{1.764706in}{1.700000in}}%
\pgfusepath{clip}%
\pgfsetbuttcap%
\pgfsetroundjoin%
\definecolor{currentfill}{rgb}{0.975644,0.874038,0.797253}%
\pgfsetfillcolor{currentfill}%
\pgfsetlinewidth{0.311001pt}%
\definecolor{currentstroke}{rgb}{1.000000,1.000000,1.000000}%
\pgfsetstrokecolor{currentstroke}%
\pgfsetdash{}{0pt}%
\pgfpathmoveto{\pgfqpoint{5.475909in}{1.520705in}}%
\pgfpathcurveto{\pgfqpoint{5.483042in}{1.520705in}}{\pgfqpoint{5.489883in}{1.523539in}}{\pgfqpoint{5.494927in}{1.528582in}}%
\pgfpathcurveto{\pgfqpoint{5.499971in}{1.533626in}}{\pgfqpoint{5.502804in}{1.540468in}}{\pgfqpoint{5.502804in}{1.547600in}}%
\pgfpathcurveto{\pgfqpoint{5.502804in}{1.554733in}}{\pgfqpoint{5.499971in}{1.561575in}}{\pgfqpoint{5.494927in}{1.566619in}}%
\pgfpathcurveto{\pgfqpoint{5.489883in}{1.571662in}}{\pgfqpoint{5.483042in}{1.574496in}}{\pgfqpoint{5.475909in}{1.574496in}}%
\pgfpathcurveto{\pgfqpoint{5.468776in}{1.574496in}}{\pgfqpoint{5.461934in}{1.571662in}}{\pgfqpoint{5.456891in}{1.566619in}}%
\pgfpathcurveto{\pgfqpoint{5.451847in}{1.561575in}}{\pgfqpoint{5.449013in}{1.554733in}}{\pgfqpoint{5.449013in}{1.547600in}}%
\pgfpathcurveto{\pgfqpoint{5.449013in}{1.540468in}}{\pgfqpoint{5.451847in}{1.533626in}}{\pgfqpoint{5.456891in}{1.528582in}}%
\pgfpathcurveto{\pgfqpoint{5.461934in}{1.523539in}}{\pgfqpoint{5.468776in}{1.520705in}}{\pgfqpoint{5.475909in}{1.520705in}}%
\pgfpathclose%
\pgfusepath{stroke,fill}%
\end{pgfscope}%
\begin{pgfscope}%
\pgfpathrectangle{\pgfqpoint{4.985294in}{0.500000in}}{\pgfqpoint{1.764706in}{1.700000in}}%
\pgfusepath{clip}%
\pgfsetbuttcap%
\pgfsetroundjoin%
\definecolor{currentfill}{rgb}{0.970718,0.821518,0.719872}%
\pgfsetfillcolor{currentfill}%
\pgfsetlinewidth{0.311001pt}%
\definecolor{currentstroke}{rgb}{1.000000,1.000000,1.000000}%
\pgfsetstrokecolor{currentstroke}%
\pgfsetdash{}{0pt}%
\pgfpathmoveto{\pgfqpoint{6.243434in}{1.334327in}}%
\pgfpathcurveto{\pgfqpoint{6.250567in}{1.334327in}}{\pgfqpoint{6.257409in}{1.337161in}}{\pgfqpoint{6.262452in}{1.342205in}}%
\pgfpathcurveto{\pgfqpoint{6.267496in}{1.347249in}}{\pgfqpoint{6.270330in}{1.354090in}}{\pgfqpoint{6.270330in}{1.361223in}}%
\pgfpathcurveto{\pgfqpoint{6.270330in}{1.368356in}}{\pgfqpoint{6.267496in}{1.375197in}}{\pgfqpoint{6.262452in}{1.380241in}}%
\pgfpathcurveto{\pgfqpoint{6.257409in}{1.385285in}}{\pgfqpoint{6.250567in}{1.388119in}}{\pgfqpoint{6.243434in}{1.388119in}}%
\pgfpathcurveto{\pgfqpoint{6.236301in}{1.388119in}}{\pgfqpoint{6.229460in}{1.385285in}}{\pgfqpoint{6.224416in}{1.380241in}}%
\pgfpathcurveto{\pgfqpoint{6.219372in}{1.375197in}}{\pgfqpoint{6.216538in}{1.368356in}}{\pgfqpoint{6.216538in}{1.361223in}}%
\pgfpathcurveto{\pgfqpoint{6.216538in}{1.354090in}}{\pgfqpoint{6.219372in}{1.347249in}}{\pgfqpoint{6.224416in}{1.342205in}}%
\pgfpathcurveto{\pgfqpoint{6.229460in}{1.337161in}}{\pgfqpoint{6.236301in}{1.334327in}}{\pgfqpoint{6.243434in}{1.334327in}}%
\pgfpathclose%
\pgfusepath{stroke,fill}%
\end{pgfscope}%
\begin{pgfscope}%
\pgfpathrectangle{\pgfqpoint{4.985294in}{0.500000in}}{\pgfqpoint{1.764706in}{1.700000in}}%
\pgfusepath{clip}%
\pgfsetbuttcap%
\pgfsetroundjoin%
\definecolor{currentfill}{rgb}{0.972726,0.844889,0.754401}%
\pgfsetfillcolor{currentfill}%
\pgfsetlinewidth{0.311001pt}%
\definecolor{currentstroke}{rgb}{1.000000,1.000000,1.000000}%
\pgfsetstrokecolor{currentstroke}%
\pgfsetdash{}{0pt}%
\pgfpathmoveto{\pgfqpoint{5.454775in}{1.038171in}}%
\pgfpathcurveto{\pgfqpoint{5.461908in}{1.038171in}}{\pgfqpoint{5.468750in}{1.041005in}}{\pgfqpoint{5.473793in}{1.046049in}}%
\pgfpathcurveto{\pgfqpoint{5.478837in}{1.051092in}}{\pgfqpoint{5.481671in}{1.057934in}}{\pgfqpoint{5.481671in}{1.065067in}}%
\pgfpathcurveto{\pgfqpoint{5.481671in}{1.072200in}}{\pgfqpoint{5.478837in}{1.079041in}}{\pgfqpoint{5.473793in}{1.084085in}}%
\pgfpathcurveto{\pgfqpoint{5.468750in}{1.089129in}}{\pgfqpoint{5.461908in}{1.091963in}}{\pgfqpoint{5.454775in}{1.091963in}}%
\pgfpathcurveto{\pgfqpoint{5.447642in}{1.091963in}}{\pgfqpoint{5.440801in}{1.089129in}}{\pgfqpoint{5.435757in}{1.084085in}}%
\pgfpathcurveto{\pgfqpoint{5.430713in}{1.079041in}}{\pgfqpoint{5.427880in}{1.072200in}}{\pgfqpoint{5.427880in}{1.065067in}}%
\pgfpathcurveto{\pgfqpoint{5.427880in}{1.057934in}}{\pgfqpoint{5.430713in}{1.051092in}}{\pgfqpoint{5.435757in}{1.046049in}}%
\pgfpathcurveto{\pgfqpoint{5.440801in}{1.041005in}}{\pgfqpoint{5.447642in}{1.038171in}}{\pgfqpoint{5.454775in}{1.038171in}}%
\pgfpathclose%
\pgfusepath{stroke,fill}%
\end{pgfscope}%
\begin{pgfscope}%
\pgfpathrectangle{\pgfqpoint{4.985294in}{0.500000in}}{\pgfqpoint{1.764706in}{1.700000in}}%
\pgfusepath{clip}%
\pgfsetbuttcap%
\pgfsetroundjoin%
\definecolor{currentfill}{rgb}{0.979124,0.903132,0.839793}%
\pgfsetfillcolor{currentfill}%
\pgfsetlinewidth{0.311001pt}%
\definecolor{currentstroke}{rgb}{1.000000,1.000000,1.000000}%
\pgfsetstrokecolor{currentstroke}%
\pgfsetdash{}{0pt}%
\pgfpathmoveto{\pgfqpoint{6.285087in}{1.264014in}}%
\pgfpathcurveto{\pgfqpoint{6.292220in}{1.264014in}}{\pgfqpoint{6.299061in}{1.266848in}}{\pgfqpoint{6.304105in}{1.271891in}}%
\pgfpathcurveto{\pgfqpoint{6.309149in}{1.276935in}}{\pgfqpoint{6.311983in}{1.283777in}}{\pgfqpoint{6.311983in}{1.290910in}}%
\pgfpathcurveto{\pgfqpoint{6.311983in}{1.298042in}}{\pgfqpoint{6.309149in}{1.304884in}}{\pgfqpoint{6.304105in}{1.309928in}}%
\pgfpathcurveto{\pgfqpoint{6.299061in}{1.314971in}}{\pgfqpoint{6.292220in}{1.317805in}}{\pgfqpoint{6.285087in}{1.317805in}}%
\pgfpathcurveto{\pgfqpoint{6.277954in}{1.317805in}}{\pgfqpoint{6.271112in}{1.314971in}}{\pgfqpoint{6.266069in}{1.309928in}}%
\pgfpathcurveto{\pgfqpoint{6.261025in}{1.304884in}}{\pgfqpoint{6.258191in}{1.298042in}}{\pgfqpoint{6.258191in}{1.290910in}}%
\pgfpathcurveto{\pgfqpoint{6.258191in}{1.283777in}}{\pgfqpoint{6.261025in}{1.276935in}}{\pgfqpoint{6.266069in}{1.271891in}}%
\pgfpathcurveto{\pgfqpoint{6.271112in}{1.266848in}}{\pgfqpoint{6.277954in}{1.264014in}}{\pgfqpoint{6.285087in}{1.264014in}}%
\pgfpathclose%
\pgfusepath{stroke,fill}%
\end{pgfscope}%
\begin{pgfscope}%
\pgfpathrectangle{\pgfqpoint{4.985294in}{0.500000in}}{\pgfqpoint{1.764706in}{1.700000in}}%
\pgfusepath{clip}%
\pgfsetbuttcap%
\pgfsetroundjoin%
\definecolor{currentfill}{rgb}{0.975644,0.874038,0.797253}%
\pgfsetfillcolor{currentfill}%
\pgfsetlinewidth{0.311001pt}%
\definecolor{currentstroke}{rgb}{1.000000,1.000000,1.000000}%
\pgfsetstrokecolor{currentstroke}%
\pgfsetdash{}{0pt}%
\pgfpathmoveto{\pgfqpoint{6.336438in}{1.472023in}}%
\pgfpathcurveto{\pgfqpoint{6.343571in}{1.472023in}}{\pgfqpoint{6.350412in}{1.474857in}}{\pgfqpoint{6.355456in}{1.479901in}}%
\pgfpathcurveto{\pgfqpoint{6.360500in}{1.484944in}}{\pgfqpoint{6.363334in}{1.491786in}}{\pgfqpoint{6.363334in}{1.498919in}}%
\pgfpathcurveto{\pgfqpoint{6.363334in}{1.506051in}}{\pgfqpoint{6.360500in}{1.512893in}}{\pgfqpoint{6.355456in}{1.517937in}}%
\pgfpathcurveto{\pgfqpoint{6.350412in}{1.522980in}}{\pgfqpoint{6.343571in}{1.525814in}}{\pgfqpoint{6.336438in}{1.525814in}}%
\pgfpathcurveto{\pgfqpoint{6.329305in}{1.525814in}}{\pgfqpoint{6.322463in}{1.522980in}}{\pgfqpoint{6.317420in}{1.517937in}}%
\pgfpathcurveto{\pgfqpoint{6.312376in}{1.512893in}}{\pgfqpoint{6.309542in}{1.506051in}}{\pgfqpoint{6.309542in}{1.498919in}}%
\pgfpathcurveto{\pgfqpoint{6.309542in}{1.491786in}}{\pgfqpoint{6.312376in}{1.484944in}}{\pgfqpoint{6.317420in}{1.479901in}}%
\pgfpathcurveto{\pgfqpoint{6.322463in}{1.474857in}}{\pgfqpoint{6.329305in}{1.472023in}}{\pgfqpoint{6.336438in}{1.472023in}}%
\pgfpathclose%
\pgfusepath{stroke,fill}%
\end{pgfscope}%
\begin{pgfscope}%
\pgfpathrectangle{\pgfqpoint{4.985294in}{0.500000in}}{\pgfqpoint{1.764706in}{1.700000in}}%
\pgfusepath{clip}%
\pgfsetbuttcap%
\pgfsetroundjoin%
\definecolor{currentfill}{rgb}{0.979891,0.908948,0.848279}%
\pgfsetfillcolor{currentfill}%
\pgfsetlinewidth{0.311001pt}%
\definecolor{currentstroke}{rgb}{1.000000,1.000000,1.000000}%
\pgfsetstrokecolor{currentstroke}%
\pgfsetdash{}{0pt}%
\pgfpathmoveto{\pgfqpoint{5.433862in}{1.405819in}}%
\pgfpathcurveto{\pgfqpoint{5.440995in}{1.405819in}}{\pgfqpoint{5.447836in}{1.408653in}}{\pgfqpoint{5.452880in}{1.413696in}}%
\pgfpathcurveto{\pgfqpoint{5.457924in}{1.418740in}}{\pgfqpoint{5.460758in}{1.425582in}}{\pgfqpoint{5.460758in}{1.432714in}}%
\pgfpathcurveto{\pgfqpoint{5.460758in}{1.439847in}}{\pgfqpoint{5.457924in}{1.446689in}}{\pgfqpoint{5.452880in}{1.451732in}}%
\pgfpathcurveto{\pgfqpoint{5.447836in}{1.456776in}}{\pgfqpoint{5.440995in}{1.459610in}}{\pgfqpoint{5.433862in}{1.459610in}}%
\pgfpathcurveto{\pgfqpoint{5.426729in}{1.459610in}}{\pgfqpoint{5.419888in}{1.456776in}}{\pgfqpoint{5.414844in}{1.451732in}}%
\pgfpathcurveto{\pgfqpoint{5.409800in}{1.446689in}}{\pgfqpoint{5.406966in}{1.439847in}}{\pgfqpoint{5.406966in}{1.432714in}}%
\pgfpathcurveto{\pgfqpoint{5.406966in}{1.425582in}}{\pgfqpoint{5.409800in}{1.418740in}}{\pgfqpoint{5.414844in}{1.413696in}}%
\pgfpathcurveto{\pgfqpoint{5.419888in}{1.408653in}}{\pgfqpoint{5.426729in}{1.405819in}}{\pgfqpoint{5.433862in}{1.405819in}}%
\pgfpathclose%
\pgfusepath{stroke,fill}%
\end{pgfscope}%
\begin{pgfscope}%
\pgfpathrectangle{\pgfqpoint{4.985294in}{0.500000in}}{\pgfqpoint{1.764706in}{1.700000in}}%
\pgfusepath{clip}%
\pgfsetbuttcap%
\pgfsetroundjoin%
\definecolor{currentfill}{rgb}{0.973271,0.850724,0.762998}%
\pgfsetfillcolor{currentfill}%
\pgfsetlinewidth{0.311001pt}%
\definecolor{currentstroke}{rgb}{1.000000,1.000000,1.000000}%
\pgfsetstrokecolor{currentstroke}%
\pgfsetdash{}{0pt}%
\pgfpathmoveto{\pgfqpoint{5.511632in}{1.034739in}}%
\pgfpathcurveto{\pgfqpoint{5.518765in}{1.034739in}}{\pgfqpoint{5.525606in}{1.037572in}}{\pgfqpoint{5.530650in}{1.042616in}}%
\pgfpathcurveto{\pgfqpoint{5.535694in}{1.047660in}}{\pgfqpoint{5.538528in}{1.054501in}}{\pgfqpoint{5.538528in}{1.061634in}}%
\pgfpathcurveto{\pgfqpoint{5.538528in}{1.068767in}}{\pgfqpoint{5.535694in}{1.075609in}}{\pgfqpoint{5.530650in}{1.080652in}}%
\pgfpathcurveto{\pgfqpoint{5.525606in}{1.085696in}}{\pgfqpoint{5.518765in}{1.088530in}}{\pgfqpoint{5.511632in}{1.088530in}}%
\pgfpathcurveto{\pgfqpoint{5.504499in}{1.088530in}}{\pgfqpoint{5.497657in}{1.085696in}}{\pgfqpoint{5.492614in}{1.080652in}}%
\pgfpathcurveto{\pgfqpoint{5.487570in}{1.075609in}}{\pgfqpoint{5.484736in}{1.068767in}}{\pgfqpoint{5.484736in}{1.061634in}}%
\pgfpathcurveto{\pgfqpoint{5.484736in}{1.054501in}}{\pgfqpoint{5.487570in}{1.047660in}}{\pgfqpoint{5.492614in}{1.042616in}}%
\pgfpathcurveto{\pgfqpoint{5.497657in}{1.037572in}}{\pgfqpoint{5.504499in}{1.034739in}}{\pgfqpoint{5.511632in}{1.034739in}}%
\pgfpathclose%
\pgfusepath{stroke,fill}%
\end{pgfscope}%
\begin{pgfscope}%
\pgfpathrectangle{\pgfqpoint{4.985294in}{0.500000in}}{\pgfqpoint{1.764706in}{1.700000in}}%
\pgfusepath{clip}%
\pgfsetbuttcap%
\pgfsetroundjoin%
\definecolor{currentfill}{rgb}{0.955103,0.477872,0.328626}%
\pgfsetfillcolor{currentfill}%
\pgfsetlinewidth{0.311001pt}%
\definecolor{currentstroke}{rgb}{1.000000,1.000000,1.000000}%
\pgfsetstrokecolor{currentstroke}%
\pgfsetdash{}{0pt}%
\pgfpathmoveto{\pgfqpoint{5.293189in}{1.189829in}}%
\pgfpathcurveto{\pgfqpoint{5.300322in}{1.189829in}}{\pgfqpoint{5.307164in}{1.192663in}}{\pgfqpoint{5.312207in}{1.197707in}}%
\pgfpathcurveto{\pgfqpoint{5.317251in}{1.202751in}}{\pgfqpoint{5.320085in}{1.209592in}}{\pgfqpoint{5.320085in}{1.216725in}}%
\pgfpathcurveto{\pgfqpoint{5.320085in}{1.223858in}}{\pgfqpoint{5.317251in}{1.230700in}}{\pgfqpoint{5.312207in}{1.235743in}}%
\pgfpathcurveto{\pgfqpoint{5.307164in}{1.240787in}}{\pgfqpoint{5.300322in}{1.243621in}}{\pgfqpoint{5.293189in}{1.243621in}}%
\pgfpathcurveto{\pgfqpoint{5.286056in}{1.243621in}}{\pgfqpoint{5.279215in}{1.240787in}}{\pgfqpoint{5.274171in}{1.235743in}}%
\pgfpathcurveto{\pgfqpoint{5.269127in}{1.230700in}}{\pgfqpoint{5.266294in}{1.223858in}}{\pgfqpoint{5.266294in}{1.216725in}}%
\pgfpathcurveto{\pgfqpoint{5.266294in}{1.209592in}}{\pgfqpoint{5.269127in}{1.202751in}}{\pgfqpoint{5.274171in}{1.197707in}}%
\pgfpathcurveto{\pgfqpoint{5.279215in}{1.192663in}}{\pgfqpoint{5.286056in}{1.189829in}}{\pgfqpoint{5.293189in}{1.189829in}}%
\pgfpathclose%
\pgfusepath{stroke,fill}%
\end{pgfscope}%
\begin{pgfscope}%
\pgfpathrectangle{\pgfqpoint{4.985294in}{0.500000in}}{\pgfqpoint{1.764706in}{1.700000in}}%
\pgfusepath{clip}%
\pgfsetbuttcap%
\pgfsetroundjoin%
\definecolor{currentfill}{rgb}{0.964306,0.663930,0.507747}%
\pgfsetfillcolor{currentfill}%
\pgfsetlinewidth{0.311001pt}%
\definecolor{currentstroke}{rgb}{1.000000,1.000000,1.000000}%
\pgfsetstrokecolor{currentstroke}%
\pgfsetdash{}{0pt}%
\pgfpathmoveto{\pgfqpoint{5.347716in}{1.108933in}}%
\pgfpathcurveto{\pgfqpoint{5.354849in}{1.108933in}}{\pgfqpoint{5.361691in}{1.111767in}}{\pgfqpoint{5.366735in}{1.116810in}}%
\pgfpathcurveto{\pgfqpoint{5.371778in}{1.121854in}}{\pgfqpoint{5.374612in}{1.128696in}}{\pgfqpoint{5.374612in}{1.135828in}}%
\pgfpathcurveto{\pgfqpoint{5.374612in}{1.142961in}}{\pgfqpoint{5.371778in}{1.149803in}}{\pgfqpoint{5.366735in}{1.154847in}}%
\pgfpathcurveto{\pgfqpoint{5.361691in}{1.159890in}}{\pgfqpoint{5.354849in}{1.162724in}}{\pgfqpoint{5.347716in}{1.162724in}}%
\pgfpathcurveto{\pgfqpoint{5.340584in}{1.162724in}}{\pgfqpoint{5.333742in}{1.159890in}}{\pgfqpoint{5.328698in}{1.154847in}}%
\pgfpathcurveto{\pgfqpoint{5.323655in}{1.149803in}}{\pgfqpoint{5.320821in}{1.142961in}}{\pgfqpoint{5.320821in}{1.135828in}}%
\pgfpathcurveto{\pgfqpoint{5.320821in}{1.128696in}}{\pgfqpoint{5.323655in}{1.121854in}}{\pgfqpoint{5.328698in}{1.116810in}}%
\pgfpathcurveto{\pgfqpoint{5.333742in}{1.111767in}}{\pgfqpoint{5.340584in}{1.108933in}}{\pgfqpoint{5.347716in}{1.108933in}}%
\pgfpathclose%
\pgfusepath{stroke,fill}%
\end{pgfscope}%
\begin{pgfscope}%
\pgfpathrectangle{\pgfqpoint{4.985294in}{0.500000in}}{\pgfqpoint{1.764706in}{1.700000in}}%
\pgfusepath{clip}%
\pgfsetbuttcap%
\pgfsetroundjoin%
\definecolor{currentfill}{rgb}{0.979891,0.908948,0.848279}%
\pgfsetfillcolor{currentfill}%
\pgfsetlinewidth{0.311001pt}%
\definecolor{currentstroke}{rgb}{1.000000,1.000000,1.000000}%
\pgfsetstrokecolor{currentstroke}%
\pgfsetdash{}{0pt}%
\pgfpathmoveto{\pgfqpoint{6.286101in}{1.460392in}}%
\pgfpathcurveto{\pgfqpoint{6.293234in}{1.460392in}}{\pgfqpoint{6.300075in}{1.463226in}}{\pgfqpoint{6.305119in}{1.468270in}}%
\pgfpathcurveto{\pgfqpoint{6.310163in}{1.473314in}}{\pgfqpoint{6.312997in}{1.480155in}}{\pgfqpoint{6.312997in}{1.487288in}}%
\pgfpathcurveto{\pgfqpoint{6.312997in}{1.494421in}}{\pgfqpoint{6.310163in}{1.501263in}}{\pgfqpoint{6.305119in}{1.506306in}}%
\pgfpathcurveto{\pgfqpoint{6.300075in}{1.511350in}}{\pgfqpoint{6.293234in}{1.514184in}}{\pgfqpoint{6.286101in}{1.514184in}}%
\pgfpathcurveto{\pgfqpoint{6.278968in}{1.514184in}}{\pgfqpoint{6.272126in}{1.511350in}}{\pgfqpoint{6.267083in}{1.506306in}}%
\pgfpathcurveto{\pgfqpoint{6.262039in}{1.501263in}}{\pgfqpoint{6.259205in}{1.494421in}}{\pgfqpoint{6.259205in}{1.487288in}}%
\pgfpathcurveto{\pgfqpoint{6.259205in}{1.480155in}}{\pgfqpoint{6.262039in}{1.473314in}}{\pgfqpoint{6.267083in}{1.468270in}}%
\pgfpathcurveto{\pgfqpoint{6.272126in}{1.463226in}}{\pgfqpoint{6.278968in}{1.460392in}}{\pgfqpoint{6.286101in}{1.460392in}}%
\pgfpathclose%
\pgfusepath{stroke,fill}%
\end{pgfscope}%
\begin{pgfscope}%
\pgfpathrectangle{\pgfqpoint{4.985294in}{0.500000in}}{\pgfqpoint{1.764706in}{1.700000in}}%
\pgfusepath{clip}%
\pgfsetbuttcap%
\pgfsetroundjoin%
\definecolor{currentfill}{rgb}{0.964799,0.689101,0.537560}%
\pgfsetfillcolor{currentfill}%
\pgfsetlinewidth{0.311001pt}%
\definecolor{currentstroke}{rgb}{1.000000,1.000000,1.000000}%
\pgfsetstrokecolor{currentstroke}%
\pgfsetdash{}{0pt}%
\pgfpathmoveto{\pgfqpoint{6.393908in}{1.466156in}}%
\pgfpathcurveto{\pgfqpoint{6.401041in}{1.466156in}}{\pgfqpoint{6.407883in}{1.468990in}}{\pgfqpoint{6.412927in}{1.474034in}}%
\pgfpathcurveto{\pgfqpoint{6.417970in}{1.479077in}}{\pgfqpoint{6.420804in}{1.485919in}}{\pgfqpoint{6.420804in}{1.493052in}}%
\pgfpathcurveto{\pgfqpoint{6.420804in}{1.500185in}}{\pgfqpoint{6.417970in}{1.507026in}}{\pgfqpoint{6.412927in}{1.512070in}}%
\pgfpathcurveto{\pgfqpoint{6.407883in}{1.517113in}}{\pgfqpoint{6.401041in}{1.519947in}}{\pgfqpoint{6.393908in}{1.519947in}}%
\pgfpathcurveto{\pgfqpoint{6.386776in}{1.519947in}}{\pgfqpoint{6.379934in}{1.517113in}}{\pgfqpoint{6.374890in}{1.512070in}}%
\pgfpathcurveto{\pgfqpoint{6.369847in}{1.507026in}}{\pgfqpoint{6.367013in}{1.500185in}}{\pgfqpoint{6.367013in}{1.493052in}}%
\pgfpathcurveto{\pgfqpoint{6.367013in}{1.485919in}}{\pgfqpoint{6.369847in}{1.479077in}}{\pgfqpoint{6.374890in}{1.474034in}}%
\pgfpathcurveto{\pgfqpoint{6.379934in}{1.468990in}}{\pgfqpoint{6.386776in}{1.466156in}}{\pgfqpoint{6.393908in}{1.466156in}}%
\pgfpathclose%
\pgfusepath{stroke,fill}%
\end{pgfscope}%
\begin{pgfscope}%
\pgfpathrectangle{\pgfqpoint{4.985294in}{0.500000in}}{\pgfqpoint{1.764706in}{1.700000in}}%
\pgfusepath{clip}%
\pgfsetbuttcap%
\pgfsetroundjoin%
\definecolor{currentfill}{rgb}{0.963884,0.644842,0.486120}%
\pgfsetfillcolor{currentfill}%
\pgfsetlinewidth{0.311001pt}%
\definecolor{currentstroke}{rgb}{1.000000,1.000000,1.000000}%
\pgfsetstrokecolor{currentstroke}%
\pgfsetdash{}{0pt}%
\pgfpathmoveto{\pgfqpoint{6.183254in}{1.380067in}}%
\pgfpathcurveto{\pgfqpoint{6.190387in}{1.380067in}}{\pgfqpoint{6.197229in}{1.382901in}}{\pgfqpoint{6.202272in}{1.387944in}}%
\pgfpathcurveto{\pgfqpoint{6.207316in}{1.392988in}}{\pgfqpoint{6.210150in}{1.399830in}}{\pgfqpoint{6.210150in}{1.406962in}}%
\pgfpathcurveto{\pgfqpoint{6.210150in}{1.414095in}}{\pgfqpoint{6.207316in}{1.420937in}}{\pgfqpoint{6.202272in}{1.425980in}}%
\pgfpathcurveto{\pgfqpoint{6.197229in}{1.431024in}}{\pgfqpoint{6.190387in}{1.433858in}}{\pgfqpoint{6.183254in}{1.433858in}}%
\pgfpathcurveto{\pgfqpoint{6.176121in}{1.433858in}}{\pgfqpoint{6.169280in}{1.431024in}}{\pgfqpoint{6.164236in}{1.425980in}}%
\pgfpathcurveto{\pgfqpoint{6.159192in}{1.420937in}}{\pgfqpoint{6.156358in}{1.414095in}}{\pgfqpoint{6.156358in}{1.406962in}}%
\pgfpathcurveto{\pgfqpoint{6.156358in}{1.399830in}}{\pgfqpoint{6.159192in}{1.392988in}}{\pgfqpoint{6.164236in}{1.387944in}}%
\pgfpathcurveto{\pgfqpoint{6.169280in}{1.382901in}}{\pgfqpoint{6.176121in}{1.380067in}}{\pgfqpoint{6.183254in}{1.380067in}}%
\pgfpathclose%
\pgfusepath{stroke,fill}%
\end{pgfscope}%
\begin{pgfscope}%
\pgfpathrectangle{\pgfqpoint{4.985294in}{0.500000in}}{\pgfqpoint{1.764706in}{1.700000in}}%
\pgfusepath{clip}%
\pgfsetbuttcap%
\pgfsetroundjoin%
\definecolor{currentfill}{rgb}{0.968931,0.798091,0.685123}%
\pgfsetfillcolor{currentfill}%
\pgfsetlinewidth{0.311001pt}%
\definecolor{currentstroke}{rgb}{1.000000,1.000000,1.000000}%
\pgfsetstrokecolor{currentstroke}%
\pgfsetdash{}{0pt}%
\pgfpathmoveto{\pgfqpoint{5.360199in}{1.199341in}}%
\pgfpathcurveto{\pgfqpoint{5.367332in}{1.199341in}}{\pgfqpoint{5.374173in}{1.202175in}}{\pgfqpoint{5.379217in}{1.207219in}}%
\pgfpathcurveto{\pgfqpoint{5.384261in}{1.212262in}}{\pgfqpoint{5.387094in}{1.219104in}}{\pgfqpoint{5.387094in}{1.226237in}}%
\pgfpathcurveto{\pgfqpoint{5.387094in}{1.233370in}}{\pgfqpoint{5.384261in}{1.240211in}}{\pgfqpoint{5.379217in}{1.245255in}}%
\pgfpathcurveto{\pgfqpoint{5.374173in}{1.250299in}}{\pgfqpoint{5.367332in}{1.253133in}}{\pgfqpoint{5.360199in}{1.253133in}}%
\pgfpathcurveto{\pgfqpoint{5.353066in}{1.253133in}}{\pgfqpoint{5.346224in}{1.250299in}}{\pgfqpoint{5.341181in}{1.245255in}}%
\pgfpathcurveto{\pgfqpoint{5.336137in}{1.240211in}}{\pgfqpoint{5.333303in}{1.233370in}}{\pgfqpoint{5.333303in}{1.226237in}}%
\pgfpathcurveto{\pgfqpoint{5.333303in}{1.219104in}}{\pgfqpoint{5.336137in}{1.212262in}}{\pgfqpoint{5.341181in}{1.207219in}}%
\pgfpathcurveto{\pgfqpoint{5.346224in}{1.202175in}}{\pgfqpoint{5.353066in}{1.199341in}}{\pgfqpoint{5.360199in}{1.199341in}}%
\pgfpathclose%
\pgfusepath{stroke,fill}%
\end{pgfscope}%
\begin{pgfscope}%
\pgfpathrectangle{\pgfqpoint{4.985294in}{0.500000in}}{\pgfqpoint{1.764706in}{1.700000in}}%
\pgfusepath{clip}%
\pgfsetbuttcap%
\pgfsetroundjoin%
\definecolor{currentfill}{rgb}{0.975644,0.874038,0.797253}%
\pgfsetfillcolor{currentfill}%
\pgfsetlinewidth{0.311001pt}%
\definecolor{currentstroke}{rgb}{1.000000,1.000000,1.000000}%
\pgfsetstrokecolor{currentstroke}%
\pgfsetdash{}{0pt}%
\pgfpathmoveto{\pgfqpoint{6.261063in}{1.080610in}}%
\pgfpathcurveto{\pgfqpoint{6.268196in}{1.080610in}}{\pgfqpoint{6.275038in}{1.083444in}}{\pgfqpoint{6.280082in}{1.088488in}}%
\pgfpathcurveto{\pgfqpoint{6.285125in}{1.093531in}}{\pgfqpoint{6.287959in}{1.100373in}}{\pgfqpoint{6.287959in}{1.107506in}}%
\pgfpathcurveto{\pgfqpoint{6.287959in}{1.114638in}}{\pgfqpoint{6.285125in}{1.121480in}}{\pgfqpoint{6.280082in}{1.126524in}}%
\pgfpathcurveto{\pgfqpoint{6.275038in}{1.131567in}}{\pgfqpoint{6.268196in}{1.134401in}}{\pgfqpoint{6.261063in}{1.134401in}}%
\pgfpathcurveto{\pgfqpoint{6.253931in}{1.134401in}}{\pgfqpoint{6.247089in}{1.131567in}}{\pgfqpoint{6.242045in}{1.126524in}}%
\pgfpathcurveto{\pgfqpoint{6.237002in}{1.121480in}}{\pgfqpoint{6.234168in}{1.114638in}}{\pgfqpoint{6.234168in}{1.107506in}}%
\pgfpathcurveto{\pgfqpoint{6.234168in}{1.100373in}}{\pgfqpoint{6.237002in}{1.093531in}}{\pgfqpoint{6.242045in}{1.088488in}}%
\pgfpathcurveto{\pgfqpoint{6.247089in}{1.083444in}}{\pgfqpoint{6.253931in}{1.080610in}}{\pgfqpoint{6.261063in}{1.080610in}}%
\pgfpathclose%
\pgfusepath{stroke,fill}%
\end{pgfscope}%
\begin{pgfscope}%
\pgfpathrectangle{\pgfqpoint{4.985294in}{0.500000in}}{\pgfqpoint{1.764706in}{1.700000in}}%
\pgfusepath{clip}%
\pgfsetbuttcap%
\pgfsetroundjoin%
\definecolor{currentfill}{rgb}{0.969803,0.809811,0.702523}%
\pgfsetfillcolor{currentfill}%
\pgfsetlinewidth{0.311001pt}%
\definecolor{currentstroke}{rgb}{1.000000,1.000000,1.000000}%
\pgfsetstrokecolor{currentstroke}%
\pgfsetdash{}{0pt}%
\pgfpathmoveto{\pgfqpoint{6.380126in}{1.229155in}}%
\pgfpathcurveto{\pgfqpoint{6.387259in}{1.229155in}}{\pgfqpoint{6.394100in}{1.231989in}}{\pgfqpoint{6.399144in}{1.237033in}}%
\pgfpathcurveto{\pgfqpoint{6.404188in}{1.242076in}}{\pgfqpoint{6.407021in}{1.248918in}}{\pgfqpoint{6.407021in}{1.256051in}}%
\pgfpathcurveto{\pgfqpoint{6.407021in}{1.263184in}}{\pgfqpoint{6.404188in}{1.270025in}}{\pgfqpoint{6.399144in}{1.275069in}}%
\pgfpathcurveto{\pgfqpoint{6.394100in}{1.280113in}}{\pgfqpoint{6.387259in}{1.282946in}}{\pgfqpoint{6.380126in}{1.282946in}}%
\pgfpathcurveto{\pgfqpoint{6.372993in}{1.282946in}}{\pgfqpoint{6.366151in}{1.280113in}}{\pgfqpoint{6.361108in}{1.275069in}}%
\pgfpathcurveto{\pgfqpoint{6.356064in}{1.270025in}}{\pgfqpoint{6.353230in}{1.263184in}}{\pgfqpoint{6.353230in}{1.256051in}}%
\pgfpathcurveto{\pgfqpoint{6.353230in}{1.248918in}}{\pgfqpoint{6.356064in}{1.242076in}}{\pgfqpoint{6.361108in}{1.237033in}}%
\pgfpathcurveto{\pgfqpoint{6.366151in}{1.231989in}}{\pgfqpoint{6.372993in}{1.229155in}}{\pgfqpoint{6.380126in}{1.229155in}}%
\pgfpathclose%
\pgfusepath{stroke,fill}%
\end{pgfscope}%
\begin{pgfscope}%
\pgfpathrectangle{\pgfqpoint{4.985294in}{0.500000in}}{\pgfqpoint{1.764706in}{1.700000in}}%
\pgfusepath{clip}%
\pgfsetbuttcap%
\pgfsetroundjoin%
\definecolor{currentfill}{rgb}{0.978376,0.897317,0.831308}%
\pgfsetfillcolor{currentfill}%
\pgfsetlinewidth{0.311001pt}%
\definecolor{currentstroke}{rgb}{1.000000,1.000000,1.000000}%
\pgfsetstrokecolor{currentstroke}%
\pgfsetdash{}{0pt}%
\pgfpathmoveto{\pgfqpoint{5.431700in}{1.158595in}}%
\pgfpathcurveto{\pgfqpoint{5.438833in}{1.158595in}}{\pgfqpoint{5.445674in}{1.161429in}}{\pgfqpoint{5.450718in}{1.166472in}}%
\pgfpathcurveto{\pgfqpoint{5.455762in}{1.171516in}}{\pgfqpoint{5.458595in}{1.178358in}}{\pgfqpoint{5.458595in}{1.185491in}}%
\pgfpathcurveto{\pgfqpoint{5.458595in}{1.192623in}}{\pgfqpoint{5.455762in}{1.199465in}}{\pgfqpoint{5.450718in}{1.204509in}}%
\pgfpathcurveto{\pgfqpoint{5.445674in}{1.209552in}}{\pgfqpoint{5.438833in}{1.212386in}}{\pgfqpoint{5.431700in}{1.212386in}}%
\pgfpathcurveto{\pgfqpoint{5.424567in}{1.212386in}}{\pgfqpoint{5.417725in}{1.209552in}}{\pgfqpoint{5.412682in}{1.204509in}}%
\pgfpathcurveto{\pgfqpoint{5.407638in}{1.199465in}}{\pgfqpoint{5.404804in}{1.192623in}}{\pgfqpoint{5.404804in}{1.185491in}}%
\pgfpathcurveto{\pgfqpoint{5.404804in}{1.178358in}}{\pgfqpoint{5.407638in}{1.171516in}}{\pgfqpoint{5.412682in}{1.166472in}}%
\pgfpathcurveto{\pgfqpoint{5.417725in}{1.161429in}}{\pgfqpoint{5.424567in}{1.158595in}}{\pgfqpoint{5.431700in}{1.158595in}}%
\pgfpathclose%
\pgfusepath{stroke,fill}%
\end{pgfscope}%
\begin{pgfscope}%
\pgfpathrectangle{\pgfqpoint{4.985294in}{0.500000in}}{\pgfqpoint{1.764706in}{1.700000in}}%
\pgfusepath{clip}%
\pgfsetbuttcap%
\pgfsetroundjoin%
\definecolor{currentfill}{rgb}{0.964799,0.689101,0.537560}%
\pgfsetfillcolor{currentfill}%
\pgfsetlinewidth{0.311001pt}%
\definecolor{currentstroke}{rgb}{1.000000,1.000000,1.000000}%
\pgfsetstrokecolor{currentstroke}%
\pgfsetdash{}{0pt}%
\pgfpathmoveto{\pgfqpoint{5.567834in}{0.896816in}}%
\pgfpathcurveto{\pgfqpoint{5.574967in}{0.896816in}}{\pgfqpoint{5.581809in}{0.899650in}}{\pgfqpoint{5.586852in}{0.904694in}}%
\pgfpathcurveto{\pgfqpoint{5.591896in}{0.909737in}}{\pgfqpoint{5.594730in}{0.916579in}}{\pgfqpoint{5.594730in}{0.923712in}}%
\pgfpathcurveto{\pgfqpoint{5.594730in}{0.930845in}}{\pgfqpoint{5.591896in}{0.937686in}}{\pgfqpoint{5.586852in}{0.942730in}}%
\pgfpathcurveto{\pgfqpoint{5.581809in}{0.947774in}}{\pgfqpoint{5.574967in}{0.950608in}}{\pgfqpoint{5.567834in}{0.950608in}}%
\pgfpathcurveto{\pgfqpoint{5.560701in}{0.950608in}}{\pgfqpoint{5.553860in}{0.947774in}}{\pgfqpoint{5.548816in}{0.942730in}}%
\pgfpathcurveto{\pgfqpoint{5.543772in}{0.937686in}}{\pgfqpoint{5.540938in}{0.930845in}}{\pgfqpoint{5.540938in}{0.923712in}}%
\pgfpathcurveto{\pgfqpoint{5.540938in}{0.916579in}}{\pgfqpoint{5.543772in}{0.909737in}}{\pgfqpoint{5.548816in}{0.904694in}}%
\pgfpathcurveto{\pgfqpoint{5.553860in}{0.899650in}}{\pgfqpoint{5.560701in}{0.896816in}}{\pgfqpoint{5.567834in}{0.896816in}}%
\pgfpathclose%
\pgfusepath{stroke,fill}%
\end{pgfscope}%
\begin{pgfscope}%
\pgfpathrectangle{\pgfqpoint{4.985294in}{0.500000in}}{\pgfqpoint{1.764706in}{1.700000in}}%
\pgfusepath{clip}%
\pgfsetbuttcap%
\pgfsetroundjoin%
\definecolor{currentfill}{rgb}{0.970255,0.815666,0.711203}%
\pgfsetfillcolor{currentfill}%
\pgfsetlinewidth{0.311001pt}%
\definecolor{currentstroke}{rgb}{1.000000,1.000000,1.000000}%
\pgfsetstrokecolor{currentstroke}%
\pgfsetdash{}{0pt}%
\pgfpathmoveto{\pgfqpoint{5.468942in}{1.635989in}}%
\pgfpathcurveto{\pgfqpoint{5.476075in}{1.635989in}}{\pgfqpoint{5.482916in}{1.638823in}}{\pgfqpoint{5.487960in}{1.643867in}}%
\pgfpathcurveto{\pgfqpoint{5.493004in}{1.648911in}}{\pgfqpoint{5.495837in}{1.655752in}}{\pgfqpoint{5.495837in}{1.662885in}}%
\pgfpathcurveto{\pgfqpoint{5.495837in}{1.670018in}}{\pgfqpoint{5.493004in}{1.676860in}}{\pgfqpoint{5.487960in}{1.681903in}}%
\pgfpathcurveto{\pgfqpoint{5.482916in}{1.686947in}}{\pgfqpoint{5.476075in}{1.689781in}}{\pgfqpoint{5.468942in}{1.689781in}}%
\pgfpathcurveto{\pgfqpoint{5.461809in}{1.689781in}}{\pgfqpoint{5.454967in}{1.686947in}}{\pgfqpoint{5.449924in}{1.681903in}}%
\pgfpathcurveto{\pgfqpoint{5.444880in}{1.676860in}}{\pgfqpoint{5.442046in}{1.670018in}}{\pgfqpoint{5.442046in}{1.662885in}}%
\pgfpathcurveto{\pgfqpoint{5.442046in}{1.655752in}}{\pgfqpoint{5.444880in}{1.648911in}}{\pgfqpoint{5.449924in}{1.643867in}}%
\pgfpathcurveto{\pgfqpoint{5.454967in}{1.638823in}}{\pgfqpoint{5.461809in}{1.635989in}}{\pgfqpoint{5.468942in}{1.635989in}}%
\pgfpathclose%
\pgfusepath{stroke,fill}%
\end{pgfscope}%
\begin{pgfscope}%
\pgfpathrectangle{\pgfqpoint{4.985294in}{0.500000in}}{\pgfqpoint{1.764706in}{1.700000in}}%
\pgfusepath{clip}%
\pgfsetbuttcap%
\pgfsetroundjoin%
\definecolor{currentfill}{rgb}{0.969803,0.809811,0.702523}%
\pgfsetfillcolor{currentfill}%
\pgfsetlinewidth{0.311001pt}%
\definecolor{currentstroke}{rgb}{1.000000,1.000000,1.000000}%
\pgfsetstrokecolor{currentstroke}%
\pgfsetdash{}{0pt}%
\pgfpathmoveto{\pgfqpoint{6.233674in}{1.670248in}}%
\pgfpathcurveto{\pgfqpoint{6.240807in}{1.670248in}}{\pgfqpoint{6.247648in}{1.673082in}}{\pgfqpoint{6.252692in}{1.678126in}}%
\pgfpathcurveto{\pgfqpoint{6.257736in}{1.683169in}}{\pgfqpoint{6.260570in}{1.690011in}}{\pgfqpoint{6.260570in}{1.697144in}}%
\pgfpathcurveto{\pgfqpoint{6.260570in}{1.704277in}}{\pgfqpoint{6.257736in}{1.711118in}}{\pgfqpoint{6.252692in}{1.716162in}}%
\pgfpathcurveto{\pgfqpoint{6.247648in}{1.721206in}}{\pgfqpoint{6.240807in}{1.724039in}}{\pgfqpoint{6.233674in}{1.724039in}}%
\pgfpathcurveto{\pgfqpoint{6.226541in}{1.724039in}}{\pgfqpoint{6.219699in}{1.721206in}}{\pgfqpoint{6.214656in}{1.716162in}}%
\pgfpathcurveto{\pgfqpoint{6.209612in}{1.711118in}}{\pgfqpoint{6.206778in}{1.704277in}}{\pgfqpoint{6.206778in}{1.697144in}}%
\pgfpathcurveto{\pgfqpoint{6.206778in}{1.690011in}}{\pgfqpoint{6.209612in}{1.683169in}}{\pgfqpoint{6.214656in}{1.678126in}}%
\pgfpathcurveto{\pgfqpoint{6.219699in}{1.673082in}}{\pgfqpoint{6.226541in}{1.670248in}}{\pgfqpoint{6.233674in}{1.670248in}}%
\pgfpathclose%
\pgfusepath{stroke,fill}%
\end{pgfscope}%
\begin{pgfscope}%
\pgfpathrectangle{\pgfqpoint{4.985294in}{0.500000in}}{\pgfqpoint{1.764706in}{1.700000in}}%
\pgfusepath{clip}%
\pgfsetbuttcap%
\pgfsetroundjoin%
\definecolor{currentfill}{rgb}{0.968509,0.792226,0.676405}%
\pgfsetfillcolor{currentfill}%
\pgfsetlinewidth{0.311001pt}%
\definecolor{currentstroke}{rgb}{1.000000,1.000000,1.000000}%
\pgfsetstrokecolor{currentstroke}%
\pgfsetdash{}{0pt}%
\pgfpathmoveto{\pgfqpoint{5.567213in}{0.998658in}}%
\pgfpathcurveto{\pgfqpoint{5.574346in}{0.998658in}}{\pgfqpoint{5.581188in}{1.001492in}}{\pgfqpoint{5.586231in}{1.006536in}}%
\pgfpathcurveto{\pgfqpoint{5.591275in}{1.011579in}}{\pgfqpoint{5.594109in}{1.018421in}}{\pgfqpoint{5.594109in}{1.025554in}}%
\pgfpathcurveto{\pgfqpoint{5.594109in}{1.032686in}}{\pgfqpoint{5.591275in}{1.039528in}}{\pgfqpoint{5.586231in}{1.044572in}}%
\pgfpathcurveto{\pgfqpoint{5.581188in}{1.049615in}}{\pgfqpoint{5.574346in}{1.052449in}}{\pgfqpoint{5.567213in}{1.052449in}}%
\pgfpathcurveto{\pgfqpoint{5.560080in}{1.052449in}}{\pgfqpoint{5.553239in}{1.049615in}}{\pgfqpoint{5.548195in}{1.044572in}}%
\pgfpathcurveto{\pgfqpoint{5.543151in}{1.039528in}}{\pgfqpoint{5.540318in}{1.032686in}}{\pgfqpoint{5.540318in}{1.025554in}}%
\pgfpathcurveto{\pgfqpoint{5.540318in}{1.018421in}}{\pgfqpoint{5.543151in}{1.011579in}}{\pgfqpoint{5.548195in}{1.006536in}}%
\pgfpathcurveto{\pgfqpoint{5.553239in}{1.001492in}}{\pgfqpoint{5.560080in}{0.998658in}}{\pgfqpoint{5.567213in}{0.998658in}}%
\pgfpathclose%
\pgfusepath{stroke,fill}%
\end{pgfscope}%
\begin{pgfscope}%
\pgfpathrectangle{\pgfqpoint{4.985294in}{0.500000in}}{\pgfqpoint{1.764706in}{1.700000in}}%
\pgfusepath{clip}%
\pgfsetbuttcap%
\pgfsetroundjoin%
\definecolor{currentfill}{rgb}{0.972201,0.839051,0.745789}%
\pgfsetfillcolor{currentfill}%
\pgfsetlinewidth{0.311001pt}%
\definecolor{currentstroke}{rgb}{1.000000,1.000000,1.000000}%
\pgfsetstrokecolor{currentstroke}%
\pgfsetdash{}{0pt}%
\pgfpathmoveto{\pgfqpoint{5.501886in}{1.604987in}}%
\pgfpathcurveto{\pgfqpoint{5.509019in}{1.604987in}}{\pgfqpoint{5.515861in}{1.607821in}}{\pgfqpoint{5.520904in}{1.612864in}}%
\pgfpathcurveto{\pgfqpoint{5.525948in}{1.617908in}}{\pgfqpoint{5.528782in}{1.624750in}}{\pgfqpoint{5.528782in}{1.631882in}}%
\pgfpathcurveto{\pgfqpoint{5.528782in}{1.639015in}}{\pgfqpoint{5.525948in}{1.645857in}}{\pgfqpoint{5.520904in}{1.650901in}}%
\pgfpathcurveto{\pgfqpoint{5.515861in}{1.655944in}}{\pgfqpoint{5.509019in}{1.658778in}}{\pgfqpoint{5.501886in}{1.658778in}}%
\pgfpathcurveto{\pgfqpoint{5.494753in}{1.658778in}}{\pgfqpoint{5.487912in}{1.655944in}}{\pgfqpoint{5.482868in}{1.650901in}}%
\pgfpathcurveto{\pgfqpoint{5.477824in}{1.645857in}}{\pgfqpoint{5.474990in}{1.639015in}}{\pgfqpoint{5.474990in}{1.631882in}}%
\pgfpathcurveto{\pgfqpoint{5.474990in}{1.624750in}}{\pgfqpoint{5.477824in}{1.617908in}}{\pgfqpoint{5.482868in}{1.612864in}}%
\pgfpathcurveto{\pgfqpoint{5.487912in}{1.607821in}}{\pgfqpoint{5.494753in}{1.604987in}}{\pgfqpoint{5.501886in}{1.604987in}}%
\pgfpathclose%
\pgfusepath{stroke,fill}%
\end{pgfscope}%
\begin{pgfscope}%
\pgfpathrectangle{\pgfqpoint{4.985294in}{0.500000in}}{\pgfqpoint{1.764706in}{1.700000in}}%
\pgfusepath{clip}%
\pgfsetbuttcap%
\pgfsetroundjoin%
\definecolor{currentfill}{rgb}{0.976287,0.879862,0.805788}%
\pgfsetfillcolor{currentfill}%
\pgfsetlinewidth{0.311001pt}%
\definecolor{currentstroke}{rgb}{1.000000,1.000000,1.000000}%
\pgfsetstrokecolor{currentstroke}%
\pgfsetdash{}{0pt}%
\pgfpathmoveto{\pgfqpoint{6.267226in}{1.247800in}}%
\pgfpathcurveto{\pgfqpoint{6.274359in}{1.247800in}}{\pgfqpoint{6.281201in}{1.250634in}}{\pgfqpoint{6.286245in}{1.255678in}}%
\pgfpathcurveto{\pgfqpoint{6.291288in}{1.260722in}}{\pgfqpoint{6.294122in}{1.267563in}}{\pgfqpoint{6.294122in}{1.274696in}}%
\pgfpathcurveto{\pgfqpoint{6.294122in}{1.281829in}}{\pgfqpoint{6.291288in}{1.288671in}}{\pgfqpoint{6.286245in}{1.293714in}}%
\pgfpathcurveto{\pgfqpoint{6.281201in}{1.298758in}}{\pgfqpoint{6.274359in}{1.301592in}}{\pgfqpoint{6.267226in}{1.301592in}}%
\pgfpathcurveto{\pgfqpoint{6.260094in}{1.301592in}}{\pgfqpoint{6.253252in}{1.298758in}}{\pgfqpoint{6.248208in}{1.293714in}}%
\pgfpathcurveto{\pgfqpoint{6.243165in}{1.288671in}}{\pgfqpoint{6.240331in}{1.281829in}}{\pgfqpoint{6.240331in}{1.274696in}}%
\pgfpathcurveto{\pgfqpoint{6.240331in}{1.267563in}}{\pgfqpoint{6.243165in}{1.260722in}}{\pgfqpoint{6.248208in}{1.255678in}}%
\pgfpathcurveto{\pgfqpoint{6.253252in}{1.250634in}}{\pgfqpoint{6.260094in}{1.247800in}}{\pgfqpoint{6.267226in}{1.247800in}}%
\pgfpathclose%
\pgfusepath{stroke,fill}%
\end{pgfscope}%
\begin{pgfscope}%
\pgfpathrectangle{\pgfqpoint{4.985294in}{0.500000in}}{\pgfqpoint{1.764706in}{1.700000in}}%
\pgfusepath{clip}%
\pgfsetbuttcap%
\pgfsetroundjoin%
\definecolor{currentfill}{rgb}{0.955697,0.484891,0.334214}%
\pgfsetfillcolor{currentfill}%
\pgfsetlinewidth{0.311001pt}%
\definecolor{currentstroke}{rgb}{1.000000,1.000000,1.000000}%
\pgfsetstrokecolor{currentstroke}%
\pgfsetdash{}{0pt}%
\pgfpathmoveto{\pgfqpoint{5.654198in}{1.632510in}}%
\pgfpathcurveto{\pgfqpoint{5.661331in}{1.632510in}}{\pgfqpoint{5.668172in}{1.635344in}}{\pgfqpoint{5.673216in}{1.640388in}}%
\pgfpathcurveto{\pgfqpoint{5.678260in}{1.645431in}}{\pgfqpoint{5.681094in}{1.652273in}}{\pgfqpoint{5.681094in}{1.659406in}}%
\pgfpathcurveto{\pgfqpoint{5.681094in}{1.666539in}}{\pgfqpoint{5.678260in}{1.673380in}}{\pgfqpoint{5.673216in}{1.678424in}}%
\pgfpathcurveto{\pgfqpoint{5.668172in}{1.683468in}}{\pgfqpoint{5.661331in}{1.686302in}}{\pgfqpoint{5.654198in}{1.686302in}}%
\pgfpathcurveto{\pgfqpoint{5.647065in}{1.686302in}}{\pgfqpoint{5.640223in}{1.683468in}}{\pgfqpoint{5.635180in}{1.678424in}}%
\pgfpathcurveto{\pgfqpoint{5.630136in}{1.673380in}}{\pgfqpoint{5.627302in}{1.666539in}}{\pgfqpoint{5.627302in}{1.659406in}}%
\pgfpathcurveto{\pgfqpoint{5.627302in}{1.652273in}}{\pgfqpoint{5.630136in}{1.645431in}}{\pgfqpoint{5.635180in}{1.640388in}}%
\pgfpathcurveto{\pgfqpoint{5.640223in}{1.635344in}}{\pgfqpoint{5.647065in}{1.632510in}}{\pgfqpoint{5.654198in}{1.632510in}}%
\pgfpathclose%
\pgfusepath{stroke,fill}%
\end{pgfscope}%
\begin{pgfscope}%
\pgfpathrectangle{\pgfqpoint{4.985294in}{0.500000in}}{\pgfqpoint{1.764706in}{1.700000in}}%
\pgfusepath{clip}%
\pgfsetbuttcap%
\pgfsetroundjoin%
\definecolor{currentfill}{rgb}{0.962765,0.606121,0.444717}%
\pgfsetfillcolor{currentfill}%
\pgfsetlinewidth{0.311001pt}%
\definecolor{currentstroke}{rgb}{1.000000,1.000000,1.000000}%
\pgfsetstrokecolor{currentstroke}%
\pgfsetdash{}{0pt}%
\pgfpathmoveto{\pgfqpoint{5.441747in}{1.709898in}}%
\pgfpathcurveto{\pgfqpoint{5.448879in}{1.709898in}}{\pgfqpoint{5.455721in}{1.712732in}}{\pgfqpoint{5.460765in}{1.717775in}}%
\pgfpathcurveto{\pgfqpoint{5.465808in}{1.722819in}}{\pgfqpoint{5.468642in}{1.729661in}}{\pgfqpoint{5.468642in}{1.736794in}}%
\pgfpathcurveto{\pgfqpoint{5.468642in}{1.743926in}}{\pgfqpoint{5.465808in}{1.750768in}}{\pgfqpoint{5.460765in}{1.755812in}}%
\pgfpathcurveto{\pgfqpoint{5.455721in}{1.760855in}}{\pgfqpoint{5.448879in}{1.763689in}}{\pgfqpoint{5.441747in}{1.763689in}}%
\pgfpathcurveto{\pgfqpoint{5.434614in}{1.763689in}}{\pgfqpoint{5.427772in}{1.760855in}}{\pgfqpoint{5.422728in}{1.755812in}}%
\pgfpathcurveto{\pgfqpoint{5.417685in}{1.750768in}}{\pgfqpoint{5.414851in}{1.743926in}}{\pgfqpoint{5.414851in}{1.736794in}}%
\pgfpathcurveto{\pgfqpoint{5.414851in}{1.729661in}}{\pgfqpoint{5.417685in}{1.722819in}}{\pgfqpoint{5.422728in}{1.717775in}}%
\pgfpathcurveto{\pgfqpoint{5.427772in}{1.712732in}}{\pgfqpoint{5.434614in}{1.709898in}}{\pgfqpoint{5.441747in}{1.709898in}}%
\pgfpathclose%
\pgfusepath{stroke,fill}%
\end{pgfscope}%
\begin{pgfscope}%
\pgfpathrectangle{\pgfqpoint{4.985294in}{0.500000in}}{\pgfqpoint{1.764706in}{1.700000in}}%
\pgfusepath{clip}%
\pgfsetbuttcap%
\pgfsetroundjoin%
\definecolor{currentfill}{rgb}{0.980678,0.914765,0.856766}%
\pgfsetfillcolor{currentfill}%
\pgfsetlinewidth{0.311001pt}%
\definecolor{currentstroke}{rgb}{1.000000,1.000000,1.000000}%
\pgfsetstrokecolor{currentstroke}%
\pgfsetdash{}{0pt}%
\pgfpathmoveto{\pgfqpoint{6.294618in}{1.468841in}}%
\pgfpathcurveto{\pgfqpoint{6.301751in}{1.468841in}}{\pgfqpoint{6.308592in}{1.471675in}}{\pgfqpoint{6.313636in}{1.476719in}}%
\pgfpathcurveto{\pgfqpoint{6.318680in}{1.481762in}}{\pgfqpoint{6.321514in}{1.488604in}}{\pgfqpoint{6.321514in}{1.495737in}}%
\pgfpathcurveto{\pgfqpoint{6.321514in}{1.502870in}}{\pgfqpoint{6.318680in}{1.509711in}}{\pgfqpoint{6.313636in}{1.514755in}}%
\pgfpathcurveto{\pgfqpoint{6.308592in}{1.519799in}}{\pgfqpoint{6.301751in}{1.522633in}}{\pgfqpoint{6.294618in}{1.522633in}}%
\pgfpathcurveto{\pgfqpoint{6.287485in}{1.522633in}}{\pgfqpoint{6.280643in}{1.519799in}}{\pgfqpoint{6.275600in}{1.514755in}}%
\pgfpathcurveto{\pgfqpoint{6.270556in}{1.509711in}}{\pgfqpoint{6.267722in}{1.502870in}}{\pgfqpoint{6.267722in}{1.495737in}}%
\pgfpathcurveto{\pgfqpoint{6.267722in}{1.488604in}}{\pgfqpoint{6.270556in}{1.481762in}}{\pgfqpoint{6.275600in}{1.476719in}}%
\pgfpathcurveto{\pgfqpoint{6.280643in}{1.471675in}}{\pgfqpoint{6.287485in}{1.468841in}}{\pgfqpoint{6.294618in}{1.468841in}}%
\pgfpathclose%
\pgfusepath{stroke,fill}%
\end{pgfscope}%
\begin{pgfscope}%
\pgfpathrectangle{\pgfqpoint{4.985294in}{0.500000in}}{\pgfqpoint{1.764706in}{1.700000in}}%
\pgfusepath{clip}%
\pgfsetbuttcap%
\pgfsetroundjoin%
\definecolor{currentfill}{rgb}{0.963190,0.619109,0.458249}%
\pgfsetfillcolor{currentfill}%
\pgfsetlinewidth{0.311001pt}%
\definecolor{currentstroke}{rgb}{1.000000,1.000000,1.000000}%
\pgfsetstrokecolor{currentstroke}%
\pgfsetdash{}{0pt}%
\pgfpathmoveto{\pgfqpoint{5.540888in}{1.341132in}}%
\pgfpathcurveto{\pgfqpoint{5.548021in}{1.341132in}}{\pgfqpoint{5.554862in}{1.343966in}}{\pgfqpoint{5.559906in}{1.349010in}}%
\pgfpathcurveto{\pgfqpoint{5.564950in}{1.354053in}}{\pgfqpoint{5.567783in}{1.360895in}}{\pgfqpoint{5.567783in}{1.368028in}}%
\pgfpathcurveto{\pgfqpoint{5.567783in}{1.375161in}}{\pgfqpoint{5.564950in}{1.382002in}}{\pgfqpoint{5.559906in}{1.387046in}}%
\pgfpathcurveto{\pgfqpoint{5.554862in}{1.392090in}}{\pgfqpoint{5.548021in}{1.394923in}}{\pgfqpoint{5.540888in}{1.394923in}}%
\pgfpathcurveto{\pgfqpoint{5.533755in}{1.394923in}}{\pgfqpoint{5.526913in}{1.392090in}}{\pgfqpoint{5.521870in}{1.387046in}}%
\pgfpathcurveto{\pgfqpoint{5.516826in}{1.382002in}}{\pgfqpoint{5.513992in}{1.375161in}}{\pgfqpoint{5.513992in}{1.368028in}}%
\pgfpathcurveto{\pgfqpoint{5.513992in}{1.360895in}}{\pgfqpoint{5.516826in}{1.354053in}}{\pgfqpoint{5.521870in}{1.349010in}}%
\pgfpathcurveto{\pgfqpoint{5.526913in}{1.343966in}}{\pgfqpoint{5.533755in}{1.341132in}}{\pgfqpoint{5.540888in}{1.341132in}}%
\pgfpathclose%
\pgfusepath{stroke,fill}%
\end{pgfscope}%
\begin{pgfscope}%
\pgfpathrectangle{\pgfqpoint{4.985294in}{0.500000in}}{\pgfqpoint{1.764706in}{1.700000in}}%
\pgfusepath{clip}%
\pgfsetbuttcap%
\pgfsetroundjoin%
\definecolor{currentfill}{rgb}{0.973832,0.856556,0.771584}%
\pgfsetfillcolor{currentfill}%
\pgfsetlinewidth{0.311001pt}%
\definecolor{currentstroke}{rgb}{1.000000,1.000000,1.000000}%
\pgfsetstrokecolor{currentstroke}%
\pgfsetdash{}{0pt}%
\pgfpathmoveto{\pgfqpoint{6.242359in}{1.525275in}}%
\pgfpathcurveto{\pgfqpoint{6.249492in}{1.525275in}}{\pgfqpoint{6.256334in}{1.528109in}}{\pgfqpoint{6.261378in}{1.533152in}}%
\pgfpathcurveto{\pgfqpoint{6.266421in}{1.538196in}}{\pgfqpoint{6.269255in}{1.545038in}}{\pgfqpoint{6.269255in}{1.552171in}}%
\pgfpathcurveto{\pgfqpoint{6.269255in}{1.559303in}}{\pgfqpoint{6.266421in}{1.566145in}}{\pgfqpoint{6.261378in}{1.571189in}}%
\pgfpathcurveto{\pgfqpoint{6.256334in}{1.576232in}}{\pgfqpoint{6.249492in}{1.579066in}}{\pgfqpoint{6.242359in}{1.579066in}}%
\pgfpathcurveto{\pgfqpoint{6.235227in}{1.579066in}}{\pgfqpoint{6.228385in}{1.576232in}}{\pgfqpoint{6.223341in}{1.571189in}}%
\pgfpathcurveto{\pgfqpoint{6.218298in}{1.566145in}}{\pgfqpoint{6.215464in}{1.559303in}}{\pgfqpoint{6.215464in}{1.552171in}}%
\pgfpathcurveto{\pgfqpoint{6.215464in}{1.545038in}}{\pgfqpoint{6.218298in}{1.538196in}}{\pgfqpoint{6.223341in}{1.533152in}}%
\pgfpathcurveto{\pgfqpoint{6.228385in}{1.528109in}}{\pgfqpoint{6.235227in}{1.525275in}}{\pgfqpoint{6.242359in}{1.525275in}}%
\pgfpathclose%
\pgfusepath{stroke,fill}%
\end{pgfscope}%
\begin{pgfscope}%
\pgfpathrectangle{\pgfqpoint{4.985294in}{0.500000in}}{\pgfqpoint{1.764706in}{1.700000in}}%
\pgfusepath{clip}%
\pgfsetbuttcap%
\pgfsetroundjoin%
\definecolor{currentfill}{rgb}{0.967735,0.780441,0.659127}%
\pgfsetfillcolor{currentfill}%
\pgfsetlinewidth{0.311001pt}%
\definecolor{currentstroke}{rgb}{1.000000,1.000000,1.000000}%
\pgfsetstrokecolor{currentstroke}%
\pgfsetdash{}{0pt}%
\pgfpathmoveto{\pgfqpoint{6.354672in}{1.518021in}}%
\pgfpathcurveto{\pgfqpoint{6.361805in}{1.518021in}}{\pgfqpoint{6.368647in}{1.520855in}}{\pgfqpoint{6.373690in}{1.525898in}}%
\pgfpathcurveto{\pgfqpoint{6.378734in}{1.530942in}}{\pgfqpoint{6.381568in}{1.537784in}}{\pgfqpoint{6.381568in}{1.544916in}}%
\pgfpathcurveto{\pgfqpoint{6.381568in}{1.552049in}}{\pgfqpoint{6.378734in}{1.558891in}}{\pgfqpoint{6.373690in}{1.563935in}}%
\pgfpathcurveto{\pgfqpoint{6.368647in}{1.568978in}}{\pgfqpoint{6.361805in}{1.571812in}}{\pgfqpoint{6.354672in}{1.571812in}}%
\pgfpathcurveto{\pgfqpoint{6.347539in}{1.571812in}}{\pgfqpoint{6.340698in}{1.568978in}}{\pgfqpoint{6.335654in}{1.563935in}}%
\pgfpathcurveto{\pgfqpoint{6.330610in}{1.558891in}}{\pgfqpoint{6.327777in}{1.552049in}}{\pgfqpoint{6.327777in}{1.544916in}}%
\pgfpathcurveto{\pgfqpoint{6.327777in}{1.537784in}}{\pgfqpoint{6.330610in}{1.530942in}}{\pgfqpoint{6.335654in}{1.525898in}}%
\pgfpathcurveto{\pgfqpoint{6.340698in}{1.520855in}}{\pgfqpoint{6.347539in}{1.518021in}}{\pgfqpoint{6.354672in}{1.518021in}}%
\pgfpathclose%
\pgfusepath{stroke,fill}%
\end{pgfscope}%
\begin{pgfscope}%
\pgfpathrectangle{\pgfqpoint{4.985294in}{0.500000in}}{\pgfqpoint{1.764706in}{1.700000in}}%
\pgfusepath{clip}%
\pgfsetbuttcap%
\pgfsetroundjoin%
\definecolor{currentfill}{rgb}{0.971694,0.833208,0.737161}%
\pgfsetfillcolor{currentfill}%
\pgfsetlinewidth{0.311001pt}%
\definecolor{currentstroke}{rgb}{1.000000,1.000000,1.000000}%
\pgfsetstrokecolor{currentstroke}%
\pgfsetdash{}{0pt}%
\pgfpathmoveto{\pgfqpoint{5.480969in}{1.619997in}}%
\pgfpathcurveto{\pgfqpoint{5.488102in}{1.619997in}}{\pgfqpoint{5.494944in}{1.622830in}}{\pgfqpoint{5.499987in}{1.627874in}}%
\pgfpathcurveto{\pgfqpoint{5.505031in}{1.632918in}}{\pgfqpoint{5.507865in}{1.639759in}}{\pgfqpoint{5.507865in}{1.646892in}}%
\pgfpathcurveto{\pgfqpoint{5.507865in}{1.654025in}}{\pgfqpoint{5.505031in}{1.660867in}}{\pgfqpoint{5.499987in}{1.665910in}}%
\pgfpathcurveto{\pgfqpoint{5.494944in}{1.670954in}}{\pgfqpoint{5.488102in}{1.673788in}}{\pgfqpoint{5.480969in}{1.673788in}}%
\pgfpathcurveto{\pgfqpoint{5.473836in}{1.673788in}}{\pgfqpoint{5.466995in}{1.670954in}}{\pgfqpoint{5.461951in}{1.665910in}}%
\pgfpathcurveto{\pgfqpoint{5.456907in}{1.660867in}}{\pgfqpoint{5.454074in}{1.654025in}}{\pgfqpoint{5.454074in}{1.646892in}}%
\pgfpathcurveto{\pgfqpoint{5.454074in}{1.639759in}}{\pgfqpoint{5.456907in}{1.632918in}}{\pgfqpoint{5.461951in}{1.627874in}}%
\pgfpathcurveto{\pgfqpoint{5.466995in}{1.622830in}}{\pgfqpoint{5.473836in}{1.619997in}}{\pgfqpoint{5.480969in}{1.619997in}}%
\pgfpathclose%
\pgfusepath{stroke,fill}%
\end{pgfscope}%
\begin{pgfscope}%
\pgfpathrectangle{\pgfqpoint{4.985294in}{0.500000in}}{\pgfqpoint{1.764706in}{1.700000in}}%
\pgfusepath{clip}%
\pgfsetbuttcap%
\pgfsetroundjoin%
\definecolor{currentfill}{rgb}{0.974412,0.862387,0.780156}%
\pgfsetfillcolor{currentfill}%
\pgfsetlinewidth{0.311001pt}%
\definecolor{currentstroke}{rgb}{1.000000,1.000000,1.000000}%
\pgfsetstrokecolor{currentstroke}%
\pgfsetdash{}{0pt}%
\pgfpathmoveto{\pgfqpoint{5.469433in}{1.344829in}}%
\pgfpathcurveto{\pgfqpoint{5.476565in}{1.344829in}}{\pgfqpoint{5.483407in}{1.347663in}}{\pgfqpoint{5.488451in}{1.352707in}}%
\pgfpathcurveto{\pgfqpoint{5.493494in}{1.357750in}}{\pgfqpoint{5.496328in}{1.364592in}}{\pgfqpoint{5.496328in}{1.371725in}}%
\pgfpathcurveto{\pgfqpoint{5.496328in}{1.378858in}}{\pgfqpoint{5.493494in}{1.385699in}}{\pgfqpoint{5.488451in}{1.390743in}}%
\pgfpathcurveto{\pgfqpoint{5.483407in}{1.395787in}}{\pgfqpoint{5.476565in}{1.398621in}}{\pgfqpoint{5.469433in}{1.398621in}}%
\pgfpathcurveto{\pgfqpoint{5.462300in}{1.398621in}}{\pgfqpoint{5.455458in}{1.395787in}}{\pgfqpoint{5.450414in}{1.390743in}}%
\pgfpathcurveto{\pgfqpoint{5.445371in}{1.385699in}}{\pgfqpoint{5.442537in}{1.378858in}}{\pgfqpoint{5.442537in}{1.371725in}}%
\pgfpathcurveto{\pgfqpoint{5.442537in}{1.364592in}}{\pgfqpoint{5.445371in}{1.357750in}}{\pgfqpoint{5.450414in}{1.352707in}}%
\pgfpathcurveto{\pgfqpoint{5.455458in}{1.347663in}}{\pgfqpoint{5.462300in}{1.344829in}}{\pgfqpoint{5.469433in}{1.344829in}}%
\pgfpathclose%
\pgfusepath{stroke,fill}%
\end{pgfscope}%
\begin{pgfscope}%
\pgfpathrectangle{\pgfqpoint{4.985294in}{0.500000in}}{\pgfqpoint{1.764706in}{1.700000in}}%
\pgfusepath{clip}%
\pgfsetbuttcap%
\pgfsetroundjoin%
\definecolor{currentfill}{rgb}{0.960421,0.553286,0.393191}%
\pgfsetfillcolor{currentfill}%
\pgfsetlinewidth{0.311001pt}%
\definecolor{currentstroke}{rgb}{1.000000,1.000000,1.000000}%
\pgfsetstrokecolor{currentstroke}%
\pgfsetdash{}{0pt}%
\pgfpathmoveto{\pgfqpoint{5.506870in}{0.862432in}}%
\pgfpathcurveto{\pgfqpoint{5.514003in}{0.862432in}}{\pgfqpoint{5.520845in}{0.865266in}}{\pgfqpoint{5.525888in}{0.870309in}}%
\pgfpathcurveto{\pgfqpoint{5.530932in}{0.875353in}}{\pgfqpoint{5.533766in}{0.882195in}}{\pgfqpoint{5.533766in}{0.889328in}}%
\pgfpathcurveto{\pgfqpoint{5.533766in}{0.896460in}}{\pgfqpoint{5.530932in}{0.903302in}}{\pgfqpoint{5.525888in}{0.908346in}}%
\pgfpathcurveto{\pgfqpoint{5.520845in}{0.913389in}}{\pgfqpoint{5.514003in}{0.916223in}}{\pgfqpoint{5.506870in}{0.916223in}}%
\pgfpathcurveto{\pgfqpoint{5.499737in}{0.916223in}}{\pgfqpoint{5.492896in}{0.913389in}}{\pgfqpoint{5.487852in}{0.908346in}}%
\pgfpathcurveto{\pgfqpoint{5.482808in}{0.903302in}}{\pgfqpoint{5.479975in}{0.896460in}}{\pgfqpoint{5.479975in}{0.889328in}}%
\pgfpathcurveto{\pgfqpoint{5.479975in}{0.882195in}}{\pgfqpoint{5.482808in}{0.875353in}}{\pgfqpoint{5.487852in}{0.870309in}}%
\pgfpathcurveto{\pgfqpoint{5.492896in}{0.865266in}}{\pgfqpoint{5.499737in}{0.862432in}}{\pgfqpoint{5.506870in}{0.862432in}}%
\pgfpathclose%
\pgfusepath{stroke,fill}%
\end{pgfscope}%
\begin{pgfscope}%
\pgfpathrectangle{\pgfqpoint{4.985294in}{0.500000in}}{\pgfqpoint{1.764706in}{1.700000in}}%
\pgfusepath{clip}%
\pgfsetbuttcap%
\pgfsetroundjoin%
\definecolor{currentfill}{rgb}{0.970718,0.821518,0.719872}%
\pgfsetfillcolor{currentfill}%
\pgfsetlinewidth{0.311001pt}%
\definecolor{currentstroke}{rgb}{1.000000,1.000000,1.000000}%
\pgfsetstrokecolor{currentstroke}%
\pgfsetdash{}{0pt}%
\pgfpathmoveto{\pgfqpoint{5.518518in}{1.098499in}}%
\pgfpathcurveto{\pgfqpoint{5.525651in}{1.098499in}}{\pgfqpoint{5.532493in}{1.101333in}}{\pgfqpoint{5.537536in}{1.106377in}}%
\pgfpathcurveto{\pgfqpoint{5.542580in}{1.111420in}}{\pgfqpoint{5.545414in}{1.118262in}}{\pgfqpoint{5.545414in}{1.125395in}}%
\pgfpathcurveto{\pgfqpoint{5.545414in}{1.132528in}}{\pgfqpoint{5.542580in}{1.139369in}}{\pgfqpoint{5.537536in}{1.144413in}}%
\pgfpathcurveto{\pgfqpoint{5.532493in}{1.149457in}}{\pgfqpoint{5.525651in}{1.152290in}}{\pgfqpoint{5.518518in}{1.152290in}}%
\pgfpathcurveto{\pgfqpoint{5.511385in}{1.152290in}}{\pgfqpoint{5.504544in}{1.149457in}}{\pgfqpoint{5.499500in}{1.144413in}}%
\pgfpathcurveto{\pgfqpoint{5.494456in}{1.139369in}}{\pgfqpoint{5.491622in}{1.132528in}}{\pgfqpoint{5.491622in}{1.125395in}}%
\pgfpathcurveto{\pgfqpoint{5.491622in}{1.118262in}}{\pgfqpoint{5.494456in}{1.111420in}}{\pgfqpoint{5.499500in}{1.106377in}}%
\pgfpathcurveto{\pgfqpoint{5.504544in}{1.101333in}}{\pgfqpoint{5.511385in}{1.098499in}}{\pgfqpoint{5.518518in}{1.098499in}}%
\pgfpathclose%
\pgfusepath{stroke,fill}%
\end{pgfscope}%
\begin{pgfscope}%
\pgfpathrectangle{\pgfqpoint{4.985294in}{0.500000in}}{\pgfqpoint{1.764706in}{1.700000in}}%
\pgfusepath{clip}%
\pgfsetbuttcap%
\pgfsetroundjoin%
\definecolor{currentfill}{rgb}{0.965169,0.707764,0.560659}%
\pgfsetfillcolor{currentfill}%
\pgfsetlinewidth{0.311001pt}%
\definecolor{currentstroke}{rgb}{1.000000,1.000000,1.000000}%
\pgfsetstrokecolor{currentstroke}%
\pgfsetdash{}{0pt}%
\pgfpathmoveto{\pgfqpoint{5.348118in}{1.140990in}}%
\pgfpathcurveto{\pgfqpoint{5.355251in}{1.140990in}}{\pgfqpoint{5.362093in}{1.143824in}}{\pgfqpoint{5.367136in}{1.148868in}}%
\pgfpathcurveto{\pgfqpoint{5.372180in}{1.153911in}}{\pgfqpoint{5.375014in}{1.160753in}}{\pgfqpoint{5.375014in}{1.167886in}}%
\pgfpathcurveto{\pgfqpoint{5.375014in}{1.175019in}}{\pgfqpoint{5.372180in}{1.181860in}}{\pgfqpoint{5.367136in}{1.186904in}}%
\pgfpathcurveto{\pgfqpoint{5.362093in}{1.191948in}}{\pgfqpoint{5.355251in}{1.194782in}}{\pgfqpoint{5.348118in}{1.194782in}}%
\pgfpathcurveto{\pgfqpoint{5.340985in}{1.194782in}}{\pgfqpoint{5.334144in}{1.191948in}}{\pgfqpoint{5.329100in}{1.186904in}}%
\pgfpathcurveto{\pgfqpoint{5.324056in}{1.181860in}}{\pgfqpoint{5.321222in}{1.175019in}}{\pgfqpoint{5.321222in}{1.167886in}}%
\pgfpathcurveto{\pgfqpoint{5.321222in}{1.160753in}}{\pgfqpoint{5.324056in}{1.153911in}}{\pgfqpoint{5.329100in}{1.148868in}}%
\pgfpathcurveto{\pgfqpoint{5.334144in}{1.143824in}}{\pgfqpoint{5.340985in}{1.140990in}}{\pgfqpoint{5.348118in}{1.140990in}}%
\pgfpathclose%
\pgfusepath{stroke,fill}%
\end{pgfscope}%
\begin{pgfscope}%
\pgfpathrectangle{\pgfqpoint{4.985294in}{0.500000in}}{\pgfqpoint{1.764706in}{1.700000in}}%
\pgfusepath{clip}%
\pgfsetbuttcap%
\pgfsetroundjoin%
\definecolor{currentfill}{rgb}{0.965042,0.701564,0.552889}%
\pgfsetfillcolor{currentfill}%
\pgfsetlinewidth{0.311001pt}%
\definecolor{currentstroke}{rgb}{1.000000,1.000000,1.000000}%
\pgfsetstrokecolor{currentstroke}%
\pgfsetdash{}{0pt}%
\pgfpathmoveto{\pgfqpoint{5.498645in}{1.721600in}}%
\pgfpathcurveto{\pgfqpoint{5.505777in}{1.721600in}}{\pgfqpoint{5.512619in}{1.724434in}}{\pgfqpoint{5.517663in}{1.729478in}}%
\pgfpathcurveto{\pgfqpoint{5.522706in}{1.734521in}}{\pgfqpoint{5.525540in}{1.741363in}}{\pgfqpoint{5.525540in}{1.748496in}}%
\pgfpathcurveto{\pgfqpoint{5.525540in}{1.755629in}}{\pgfqpoint{5.522706in}{1.762470in}}{\pgfqpoint{5.517663in}{1.767514in}}%
\pgfpathcurveto{\pgfqpoint{5.512619in}{1.772558in}}{\pgfqpoint{5.505777in}{1.775391in}}{\pgfqpoint{5.498645in}{1.775391in}}%
\pgfpathcurveto{\pgfqpoint{5.491512in}{1.775391in}}{\pgfqpoint{5.484670in}{1.772558in}}{\pgfqpoint{5.479626in}{1.767514in}}%
\pgfpathcurveto{\pgfqpoint{5.474583in}{1.762470in}}{\pgfqpoint{5.471749in}{1.755629in}}{\pgfqpoint{5.471749in}{1.748496in}}%
\pgfpathcurveto{\pgfqpoint{5.471749in}{1.741363in}}{\pgfqpoint{5.474583in}{1.734521in}}{\pgfqpoint{5.479626in}{1.729478in}}%
\pgfpathcurveto{\pgfqpoint{5.484670in}{1.724434in}}{\pgfqpoint{5.491512in}{1.721600in}}{\pgfqpoint{5.498645in}{1.721600in}}%
\pgfpathclose%
\pgfusepath{stroke,fill}%
\end{pgfscope}%
\begin{pgfscope}%
\pgfpathrectangle{\pgfqpoint{4.985294in}{0.500000in}}{\pgfqpoint{1.764706in}{1.700000in}}%
\pgfusepath{clip}%
\pgfsetbuttcap%
\pgfsetroundjoin%
\definecolor{currentfill}{rgb}{0.975644,0.874038,0.797253}%
\pgfsetfillcolor{currentfill}%
\pgfsetlinewidth{0.311001pt}%
\definecolor{currentstroke}{rgb}{1.000000,1.000000,1.000000}%
\pgfsetstrokecolor{currentstroke}%
\pgfsetdash{}{0pt}%
\pgfpathmoveto{\pgfqpoint{5.464087in}{1.239987in}}%
\pgfpathcurveto{\pgfqpoint{5.471220in}{1.239987in}}{\pgfqpoint{5.478062in}{1.242821in}}{\pgfqpoint{5.483105in}{1.247865in}}%
\pgfpathcurveto{\pgfqpoint{5.488149in}{1.252909in}}{\pgfqpoint{5.490983in}{1.259750in}}{\pgfqpoint{5.490983in}{1.266883in}}%
\pgfpathcurveto{\pgfqpoint{5.490983in}{1.274016in}}{\pgfqpoint{5.488149in}{1.280857in}}{\pgfqpoint{5.483105in}{1.285901in}}%
\pgfpathcurveto{\pgfqpoint{5.478062in}{1.290945in}}{\pgfqpoint{5.471220in}{1.293779in}}{\pgfqpoint{5.464087in}{1.293779in}}%
\pgfpathcurveto{\pgfqpoint{5.456954in}{1.293779in}}{\pgfqpoint{5.450113in}{1.290945in}}{\pgfqpoint{5.445069in}{1.285901in}}%
\pgfpathcurveto{\pgfqpoint{5.440025in}{1.280857in}}{\pgfqpoint{5.437191in}{1.274016in}}{\pgfqpoint{5.437191in}{1.266883in}}%
\pgfpathcurveto{\pgfqpoint{5.437191in}{1.259750in}}{\pgfqpoint{5.440025in}{1.252909in}}{\pgfqpoint{5.445069in}{1.247865in}}%
\pgfpathcurveto{\pgfqpoint{5.450113in}{1.242821in}}{\pgfqpoint{5.456954in}{1.239987in}}{\pgfqpoint{5.464087in}{1.239987in}}%
\pgfpathclose%
\pgfusepath{stroke,fill}%
\end{pgfscope}%
\begin{pgfscope}%
\pgfpathrectangle{\pgfqpoint{4.985294in}{0.500000in}}{\pgfqpoint{1.764706in}{1.700000in}}%
\pgfusepath{clip}%
\pgfsetbuttcap%
\pgfsetroundjoin%
\definecolor{currentfill}{rgb}{0.973271,0.850724,0.762998}%
\pgfsetfillcolor{currentfill}%
\pgfsetlinewidth{0.311001pt}%
\definecolor{currentstroke}{rgb}{1.000000,1.000000,1.000000}%
\pgfsetstrokecolor{currentstroke}%
\pgfsetdash{}{0pt}%
\pgfpathmoveto{\pgfqpoint{5.444026in}{1.056416in}}%
\pgfpathcurveto{\pgfqpoint{5.451159in}{1.056416in}}{\pgfqpoint{5.458001in}{1.059250in}}{\pgfqpoint{5.463045in}{1.064294in}}%
\pgfpathcurveto{\pgfqpoint{5.468088in}{1.069337in}}{\pgfqpoint{5.470922in}{1.076179in}}{\pgfqpoint{5.470922in}{1.083312in}}%
\pgfpathcurveto{\pgfqpoint{5.470922in}{1.090445in}}{\pgfqpoint{5.468088in}{1.097286in}}{\pgfqpoint{5.463045in}{1.102330in}}%
\pgfpathcurveto{\pgfqpoint{5.458001in}{1.107374in}}{\pgfqpoint{5.451159in}{1.110207in}}{\pgfqpoint{5.444026in}{1.110207in}}%
\pgfpathcurveto{\pgfqpoint{5.436894in}{1.110207in}}{\pgfqpoint{5.430052in}{1.107374in}}{\pgfqpoint{5.425008in}{1.102330in}}%
\pgfpathcurveto{\pgfqpoint{5.419965in}{1.097286in}}{\pgfqpoint{5.417131in}{1.090445in}}{\pgfqpoint{5.417131in}{1.083312in}}%
\pgfpathcurveto{\pgfqpoint{5.417131in}{1.076179in}}{\pgfqpoint{5.419965in}{1.069337in}}{\pgfqpoint{5.425008in}{1.064294in}}%
\pgfpathcurveto{\pgfqpoint{5.430052in}{1.059250in}}{\pgfqpoint{5.436894in}{1.056416in}}{\pgfqpoint{5.444026in}{1.056416in}}%
\pgfpathclose%
\pgfusepath{stroke,fill}%
\end{pgfscope}%
\begin{pgfscope}%
\pgfpathrectangle{\pgfqpoint{4.985294in}{0.500000in}}{\pgfqpoint{1.764706in}{1.700000in}}%
\pgfusepath{clip}%
\pgfsetbuttcap%
\pgfsetroundjoin%
\definecolor{currentfill}{rgb}{0.965302,0.713942,0.568499}%
\pgfsetfillcolor{currentfill}%
\pgfsetlinewidth{0.311001pt}%
\definecolor{currentstroke}{rgb}{1.000000,1.000000,1.000000}%
\pgfsetstrokecolor{currentstroke}%
\pgfsetdash{}{0pt}%
\pgfpathmoveto{\pgfqpoint{6.198037in}{1.303241in}}%
\pgfpathcurveto{\pgfqpoint{6.205170in}{1.303241in}}{\pgfqpoint{6.212011in}{1.306075in}}{\pgfqpoint{6.217055in}{1.311118in}}%
\pgfpathcurveto{\pgfqpoint{6.222099in}{1.316162in}}{\pgfqpoint{6.224933in}{1.323004in}}{\pgfqpoint{6.224933in}{1.330137in}}%
\pgfpathcurveto{\pgfqpoint{6.224933in}{1.337269in}}{\pgfqpoint{6.222099in}{1.344111in}}{\pgfqpoint{6.217055in}{1.349155in}}%
\pgfpathcurveto{\pgfqpoint{6.212011in}{1.354198in}}{\pgfqpoint{6.205170in}{1.357032in}}{\pgfqpoint{6.198037in}{1.357032in}}%
\pgfpathcurveto{\pgfqpoint{6.190904in}{1.357032in}}{\pgfqpoint{6.184063in}{1.354198in}}{\pgfqpoint{6.179019in}{1.349155in}}%
\pgfpathcurveto{\pgfqpoint{6.173975in}{1.344111in}}{\pgfqpoint{6.171141in}{1.337269in}}{\pgfqpoint{6.171141in}{1.330137in}}%
\pgfpathcurveto{\pgfqpoint{6.171141in}{1.323004in}}{\pgfqpoint{6.173975in}{1.316162in}}{\pgfqpoint{6.179019in}{1.311118in}}%
\pgfpathcurveto{\pgfqpoint{6.184063in}{1.306075in}}{\pgfqpoint{6.190904in}{1.303241in}}{\pgfqpoint{6.198037in}{1.303241in}}%
\pgfpathclose%
\pgfusepath{stroke,fill}%
\end{pgfscope}%
\begin{pgfscope}%
\pgfpathrectangle{\pgfqpoint{4.985294in}{0.500000in}}{\pgfqpoint{1.764706in}{1.700000in}}%
\pgfusepath{clip}%
\pgfsetbuttcap%
\pgfsetroundjoin%
\definecolor{currentfill}{rgb}{0.969803,0.809811,0.702523}%
\pgfsetfillcolor{currentfill}%
\pgfsetlinewidth{0.311001pt}%
\definecolor{currentstroke}{rgb}{1.000000,1.000000,1.000000}%
\pgfsetstrokecolor{currentstroke}%
\pgfsetdash{}{0pt}%
\pgfpathmoveto{\pgfqpoint{6.209958in}{1.175170in}}%
\pgfpathcurveto{\pgfqpoint{6.217091in}{1.175170in}}{\pgfqpoint{6.223932in}{1.178004in}}{\pgfqpoint{6.228976in}{1.183047in}}%
\pgfpathcurveto{\pgfqpoint{6.234020in}{1.188091in}}{\pgfqpoint{6.236853in}{1.194933in}}{\pgfqpoint{6.236853in}{1.202065in}}%
\pgfpathcurveto{\pgfqpoint{6.236853in}{1.209198in}}{\pgfqpoint{6.234020in}{1.216040in}}{\pgfqpoint{6.228976in}{1.221083in}}%
\pgfpathcurveto{\pgfqpoint{6.223932in}{1.226127in}}{\pgfqpoint{6.217091in}{1.228961in}}{\pgfqpoint{6.209958in}{1.228961in}}%
\pgfpathcurveto{\pgfqpoint{6.202825in}{1.228961in}}{\pgfqpoint{6.195983in}{1.226127in}}{\pgfqpoint{6.190940in}{1.221083in}}%
\pgfpathcurveto{\pgfqpoint{6.185896in}{1.216040in}}{\pgfqpoint{6.183062in}{1.209198in}}{\pgfqpoint{6.183062in}{1.202065in}}%
\pgfpathcurveto{\pgfqpoint{6.183062in}{1.194933in}}{\pgfqpoint{6.185896in}{1.188091in}}{\pgfqpoint{6.190940in}{1.183047in}}%
\pgfpathcurveto{\pgfqpoint{6.195983in}{1.178004in}}{\pgfqpoint{6.202825in}{1.175170in}}{\pgfqpoint{6.209958in}{1.175170in}}%
\pgfpathclose%
\pgfusepath{stroke,fill}%
\end{pgfscope}%
\begin{pgfscope}%
\pgfpathrectangle{\pgfqpoint{4.985294in}{0.500000in}}{\pgfqpoint{1.764706in}{1.700000in}}%
\pgfusepath{clip}%
\pgfsetbuttcap%
\pgfsetroundjoin%
\definecolor{currentfill}{rgb}{0.966120,0.744512,0.608720}%
\pgfsetfillcolor{currentfill}%
\pgfsetlinewidth{0.311001pt}%
\definecolor{currentstroke}{rgb}{1.000000,1.000000,1.000000}%
\pgfsetstrokecolor{currentstroke}%
\pgfsetdash{}{0pt}%
\pgfpathmoveto{\pgfqpoint{5.597473in}{0.974133in}}%
\pgfpathcurveto{\pgfqpoint{5.604606in}{0.974133in}}{\pgfqpoint{5.611448in}{0.976967in}}{\pgfqpoint{5.616491in}{0.982011in}}%
\pgfpathcurveto{\pgfqpoint{5.621535in}{0.987055in}}{\pgfqpoint{5.624369in}{0.993896in}}{\pgfqpoint{5.624369in}{1.001029in}}%
\pgfpathcurveto{\pgfqpoint{5.624369in}{1.008162in}}{\pgfqpoint{5.621535in}{1.015004in}}{\pgfqpoint{5.616491in}{1.020047in}}%
\pgfpathcurveto{\pgfqpoint{5.611448in}{1.025091in}}{\pgfqpoint{5.604606in}{1.027925in}}{\pgfqpoint{5.597473in}{1.027925in}}%
\pgfpathcurveto{\pgfqpoint{5.590340in}{1.027925in}}{\pgfqpoint{5.583499in}{1.025091in}}{\pgfqpoint{5.578455in}{1.020047in}}%
\pgfpathcurveto{\pgfqpoint{5.573411in}{1.015004in}}{\pgfqpoint{5.570578in}{1.008162in}}{\pgfqpoint{5.570578in}{1.001029in}}%
\pgfpathcurveto{\pgfqpoint{5.570578in}{0.993896in}}{\pgfqpoint{5.573411in}{0.987055in}}{\pgfqpoint{5.578455in}{0.982011in}}%
\pgfpathcurveto{\pgfqpoint{5.583499in}{0.976967in}}{\pgfqpoint{5.590340in}{0.974133in}}{\pgfqpoint{5.597473in}{0.974133in}}%
\pgfpathclose%
\pgfusepath{stroke,fill}%
\end{pgfscope}%
\begin{pgfscope}%
\pgfpathrectangle{\pgfqpoint{4.985294in}{0.500000in}}{\pgfqpoint{1.764706in}{1.700000in}}%
\pgfusepath{clip}%
\pgfsetbuttcap%
\pgfsetroundjoin%
\definecolor{currentfill}{rgb}{0.966812,0.762584,0.633643}%
\pgfsetfillcolor{currentfill}%
\pgfsetlinewidth{0.311001pt}%
\definecolor{currentstroke}{rgb}{1.000000,1.000000,1.000000}%
\pgfsetstrokecolor{currentstroke}%
\pgfsetdash{}{0pt}%
\pgfpathmoveto{\pgfqpoint{5.514031in}{1.202247in}}%
\pgfpathcurveto{\pgfqpoint{5.521164in}{1.202247in}}{\pgfqpoint{5.528005in}{1.205081in}}{\pgfqpoint{5.533049in}{1.210124in}}%
\pgfpathcurveto{\pgfqpoint{5.538093in}{1.215168in}}{\pgfqpoint{5.540927in}{1.222010in}}{\pgfqpoint{5.540927in}{1.229143in}}%
\pgfpathcurveto{\pgfqpoint{5.540927in}{1.236275in}}{\pgfqpoint{5.538093in}{1.243117in}}{\pgfqpoint{5.533049in}{1.248161in}}%
\pgfpathcurveto{\pgfqpoint{5.528005in}{1.253204in}}{\pgfqpoint{5.521164in}{1.256038in}}{\pgfqpoint{5.514031in}{1.256038in}}%
\pgfpathcurveto{\pgfqpoint{5.506898in}{1.256038in}}{\pgfqpoint{5.500056in}{1.253204in}}{\pgfqpoint{5.495013in}{1.248161in}}%
\pgfpathcurveto{\pgfqpoint{5.489969in}{1.243117in}}{\pgfqpoint{5.487135in}{1.236275in}}{\pgfqpoint{5.487135in}{1.229143in}}%
\pgfpathcurveto{\pgfqpoint{5.487135in}{1.222010in}}{\pgfqpoint{5.489969in}{1.215168in}}{\pgfqpoint{5.495013in}{1.210124in}}%
\pgfpathcurveto{\pgfqpoint{5.500056in}{1.205081in}}{\pgfqpoint{5.506898in}{1.202247in}}{\pgfqpoint{5.514031in}{1.202247in}}%
\pgfpathclose%
\pgfusepath{stroke,fill}%
\end{pgfscope}%
\begin{pgfscope}%
\pgfpathrectangle{\pgfqpoint{4.985294in}{0.500000in}}{\pgfqpoint{1.764706in}{1.700000in}}%
\pgfusepath{clip}%
\pgfsetbuttcap%
\pgfsetroundjoin%
\definecolor{currentfill}{rgb}{0.976287,0.879862,0.805788}%
\pgfsetfillcolor{currentfill}%
\pgfsetlinewidth{0.311001pt}%
\definecolor{currentstroke}{rgb}{1.000000,1.000000,1.000000}%
\pgfsetstrokecolor{currentstroke}%
\pgfsetdash{}{0pt}%
\pgfpathmoveto{\pgfqpoint{6.276399in}{1.325424in}}%
\pgfpathcurveto{\pgfqpoint{6.283532in}{1.325424in}}{\pgfqpoint{6.290374in}{1.328258in}}{\pgfqpoint{6.295417in}{1.333302in}}%
\pgfpathcurveto{\pgfqpoint{6.300461in}{1.338345in}}{\pgfqpoint{6.303295in}{1.345187in}}{\pgfqpoint{6.303295in}{1.352320in}}%
\pgfpathcurveto{\pgfqpoint{6.303295in}{1.359453in}}{\pgfqpoint{6.300461in}{1.366294in}}{\pgfqpoint{6.295417in}{1.371338in}}%
\pgfpathcurveto{\pgfqpoint{6.290374in}{1.376382in}}{\pgfqpoint{6.283532in}{1.379216in}}{\pgfqpoint{6.276399in}{1.379216in}}%
\pgfpathcurveto{\pgfqpoint{6.269266in}{1.379216in}}{\pgfqpoint{6.262425in}{1.376382in}}{\pgfqpoint{6.257381in}{1.371338in}}%
\pgfpathcurveto{\pgfqpoint{6.252337in}{1.366294in}}{\pgfqpoint{6.249503in}{1.359453in}}{\pgfqpoint{6.249503in}{1.352320in}}%
\pgfpathcurveto{\pgfqpoint{6.249503in}{1.345187in}}{\pgfqpoint{6.252337in}{1.338345in}}{\pgfqpoint{6.257381in}{1.333302in}}%
\pgfpathcurveto{\pgfqpoint{6.262425in}{1.328258in}}{\pgfqpoint{6.269266in}{1.325424in}}{\pgfqpoint{6.276399in}{1.325424in}}%
\pgfpathclose%
\pgfusepath{stroke,fill}%
\end{pgfscope}%
\begin{pgfscope}%
\pgfpathrectangle{\pgfqpoint{4.985294in}{0.500000in}}{\pgfqpoint{1.764706in}{1.700000in}}%
\pgfusepath{clip}%
\pgfsetbuttcap%
\pgfsetroundjoin%
\definecolor{currentfill}{rgb}{0.969803,0.809811,0.702523}%
\pgfsetfillcolor{currentfill}%
\pgfsetlinewidth{0.311001pt}%
\definecolor{currentstroke}{rgb}{1.000000,1.000000,1.000000}%
\pgfsetstrokecolor{currentstroke}%
\pgfsetdash{}{0pt}%
\pgfpathmoveto{\pgfqpoint{6.259866in}{1.009534in}}%
\pgfpathcurveto{\pgfqpoint{6.266999in}{1.009534in}}{\pgfqpoint{6.273840in}{1.012368in}}{\pgfqpoint{6.278884in}{1.017411in}}%
\pgfpathcurveto{\pgfqpoint{6.283928in}{1.022455in}}{\pgfqpoint{6.286762in}{1.029297in}}{\pgfqpoint{6.286762in}{1.036429in}}%
\pgfpathcurveto{\pgfqpoint{6.286762in}{1.043562in}}{\pgfqpoint{6.283928in}{1.050404in}}{\pgfqpoint{6.278884in}{1.055447in}}%
\pgfpathcurveto{\pgfqpoint{6.273840in}{1.060491in}}{\pgfqpoint{6.266999in}{1.063325in}}{\pgfqpoint{6.259866in}{1.063325in}}%
\pgfpathcurveto{\pgfqpoint{6.252733in}{1.063325in}}{\pgfqpoint{6.245891in}{1.060491in}}{\pgfqpoint{6.240848in}{1.055447in}}%
\pgfpathcurveto{\pgfqpoint{6.235804in}{1.050404in}}{\pgfqpoint{6.232970in}{1.043562in}}{\pgfqpoint{6.232970in}{1.036429in}}%
\pgfpathcurveto{\pgfqpoint{6.232970in}{1.029297in}}{\pgfqpoint{6.235804in}{1.022455in}}{\pgfqpoint{6.240848in}{1.017411in}}%
\pgfpathcurveto{\pgfqpoint{6.245891in}{1.012368in}}{\pgfqpoint{6.252733in}{1.009534in}}{\pgfqpoint{6.259866in}{1.009534in}}%
\pgfpathclose%
\pgfusepath{stroke,fill}%
\end{pgfscope}%
\begin{pgfscope}%
\pgfpathrectangle{\pgfqpoint{4.985294in}{0.500000in}}{\pgfqpoint{1.764706in}{1.700000in}}%
\pgfusepath{clip}%
\pgfsetbuttcap%
\pgfsetroundjoin%
\definecolor{currentfill}{rgb}{0.924566,0.290534,0.242426}%
\pgfsetfillcolor{currentfill}%
\pgfsetlinewidth{0.311001pt}%
\definecolor{currentstroke}{rgb}{1.000000,1.000000,1.000000}%
\pgfsetstrokecolor{currentstroke}%
\pgfsetdash{}{0pt}%
\pgfpathmoveto{\pgfqpoint{6.119218in}{1.424566in}}%
\pgfpathcurveto{\pgfqpoint{6.126351in}{1.424566in}}{\pgfqpoint{6.133193in}{1.427400in}}{\pgfqpoint{6.138236in}{1.432444in}}%
\pgfpathcurveto{\pgfqpoint{6.143280in}{1.437487in}}{\pgfqpoint{6.146114in}{1.444329in}}{\pgfqpoint{6.146114in}{1.451462in}}%
\pgfpathcurveto{\pgfqpoint{6.146114in}{1.458595in}}{\pgfqpoint{6.143280in}{1.465436in}}{\pgfqpoint{6.138236in}{1.470480in}}%
\pgfpathcurveto{\pgfqpoint{6.133193in}{1.475524in}}{\pgfqpoint{6.126351in}{1.478357in}}{\pgfqpoint{6.119218in}{1.478357in}}%
\pgfpathcurveto{\pgfqpoint{6.112086in}{1.478357in}}{\pgfqpoint{6.105244in}{1.475524in}}{\pgfqpoint{6.100200in}{1.470480in}}%
\pgfpathcurveto{\pgfqpoint{6.095157in}{1.465436in}}{\pgfqpoint{6.092323in}{1.458595in}}{\pgfqpoint{6.092323in}{1.451462in}}%
\pgfpathcurveto{\pgfqpoint{6.092323in}{1.444329in}}{\pgfqpoint{6.095157in}{1.437487in}}{\pgfqpoint{6.100200in}{1.432444in}}%
\pgfpathcurveto{\pgfqpoint{6.105244in}{1.427400in}}{\pgfqpoint{6.112086in}{1.424566in}}{\pgfqpoint{6.119218in}{1.424566in}}%
\pgfpathclose%
\pgfusepath{stroke,fill}%
\end{pgfscope}%
\begin{pgfscope}%
\pgfpathrectangle{\pgfqpoint{4.985294in}{0.500000in}}{\pgfqpoint{1.764706in}{1.700000in}}%
\pgfusepath{clip}%
\pgfsetbuttcap%
\pgfsetroundjoin%
\definecolor{currentfill}{rgb}{0.979124,0.903132,0.839793}%
\pgfsetfillcolor{currentfill}%
\pgfsetlinewidth{0.311001pt}%
\definecolor{currentstroke}{rgb}{1.000000,1.000000,1.000000}%
\pgfsetstrokecolor{currentstroke}%
\pgfsetdash{}{0pt}%
\pgfpathmoveto{\pgfqpoint{5.395995in}{1.283215in}}%
\pgfpathcurveto{\pgfqpoint{5.403128in}{1.283215in}}{\pgfqpoint{5.409970in}{1.286049in}}{\pgfqpoint{5.415014in}{1.291092in}}%
\pgfpathcurveto{\pgfqpoint{5.420057in}{1.296136in}}{\pgfqpoint{5.422891in}{1.302978in}}{\pgfqpoint{5.422891in}{1.310110in}}%
\pgfpathcurveto{\pgfqpoint{5.422891in}{1.317243in}}{\pgfqpoint{5.420057in}{1.324085in}}{\pgfqpoint{5.415014in}{1.329129in}}%
\pgfpathcurveto{\pgfqpoint{5.409970in}{1.334172in}}{\pgfqpoint{5.403128in}{1.337006in}}{\pgfqpoint{5.395995in}{1.337006in}}%
\pgfpathcurveto{\pgfqpoint{5.388863in}{1.337006in}}{\pgfqpoint{5.382021in}{1.334172in}}{\pgfqpoint{5.376977in}{1.329129in}}%
\pgfpathcurveto{\pgfqpoint{5.371934in}{1.324085in}}{\pgfqpoint{5.369100in}{1.317243in}}{\pgfqpoint{5.369100in}{1.310110in}}%
\pgfpathcurveto{\pgfqpoint{5.369100in}{1.302978in}}{\pgfqpoint{5.371934in}{1.296136in}}{\pgfqpoint{5.376977in}{1.291092in}}%
\pgfpathcurveto{\pgfqpoint{5.382021in}{1.286049in}}{\pgfqpoint{5.388863in}{1.283215in}}{\pgfqpoint{5.395995in}{1.283215in}}%
\pgfpathclose%
\pgfusepath{stroke,fill}%
\end{pgfscope}%
\begin{pgfscope}%
\pgfpathrectangle{\pgfqpoint{4.985294in}{0.500000in}}{\pgfqpoint{1.764706in}{1.700000in}}%
\pgfusepath{clip}%
\pgfsetbuttcap%
\pgfsetroundjoin%
\definecolor{currentfill}{rgb}{0.973832,0.856556,0.771584}%
\pgfsetfillcolor{currentfill}%
\pgfsetlinewidth{0.311001pt}%
\definecolor{currentstroke}{rgb}{1.000000,1.000000,1.000000}%
\pgfsetstrokecolor{currentstroke}%
\pgfsetdash{}{0pt}%
\pgfpathmoveto{\pgfqpoint{6.205329in}{1.626373in}}%
\pgfpathcurveto{\pgfqpoint{6.212462in}{1.626373in}}{\pgfqpoint{6.219304in}{1.629207in}}{\pgfqpoint{6.224347in}{1.634251in}}%
\pgfpathcurveto{\pgfqpoint{6.229391in}{1.639294in}}{\pgfqpoint{6.232225in}{1.646136in}}{\pgfqpoint{6.232225in}{1.653269in}}%
\pgfpathcurveto{\pgfqpoint{6.232225in}{1.660402in}}{\pgfqpoint{6.229391in}{1.667243in}}{\pgfqpoint{6.224347in}{1.672287in}}%
\pgfpathcurveto{\pgfqpoint{6.219304in}{1.677331in}}{\pgfqpoint{6.212462in}{1.680164in}}{\pgfqpoint{6.205329in}{1.680164in}}%
\pgfpathcurveto{\pgfqpoint{6.198197in}{1.680164in}}{\pgfqpoint{6.191355in}{1.677331in}}{\pgfqpoint{6.186311in}{1.672287in}}%
\pgfpathcurveto{\pgfqpoint{6.181268in}{1.667243in}}{\pgfqpoint{6.178434in}{1.660402in}}{\pgfqpoint{6.178434in}{1.653269in}}%
\pgfpathcurveto{\pgfqpoint{6.178434in}{1.646136in}}{\pgfqpoint{6.181268in}{1.639294in}}{\pgfqpoint{6.186311in}{1.634251in}}%
\pgfpathcurveto{\pgfqpoint{6.191355in}{1.629207in}}{\pgfqpoint{6.198197in}{1.626373in}}{\pgfqpoint{6.205329in}{1.626373in}}%
\pgfpathclose%
\pgfusepath{stroke,fill}%
\end{pgfscope}%
\begin{pgfscope}%
\pgfpathrectangle{\pgfqpoint{4.985294in}{0.500000in}}{\pgfqpoint{1.764706in}{1.700000in}}%
\pgfusepath{clip}%
\pgfsetbuttcap%
\pgfsetroundjoin%
\definecolor{currentfill}{rgb}{0.966120,0.744512,0.608720}%
\pgfsetfillcolor{currentfill}%
\pgfsetlinewidth{0.311001pt}%
\definecolor{currentstroke}{rgb}{1.000000,1.000000,1.000000}%
\pgfsetstrokecolor{currentstroke}%
\pgfsetdash{}{0pt}%
\pgfpathmoveto{\pgfqpoint{6.141873in}{1.614025in}}%
\pgfpathcurveto{\pgfqpoint{6.149006in}{1.614025in}}{\pgfqpoint{6.155847in}{1.616859in}}{\pgfqpoint{6.160891in}{1.621903in}}%
\pgfpathcurveto{\pgfqpoint{6.165935in}{1.626947in}}{\pgfqpoint{6.168769in}{1.633788in}}{\pgfqpoint{6.168769in}{1.640921in}}%
\pgfpathcurveto{\pgfqpoint{6.168769in}{1.648054in}}{\pgfqpoint{6.165935in}{1.654896in}}{\pgfqpoint{6.160891in}{1.659939in}}%
\pgfpathcurveto{\pgfqpoint{6.155847in}{1.664983in}}{\pgfqpoint{6.149006in}{1.667817in}}{\pgfqpoint{6.141873in}{1.667817in}}%
\pgfpathcurveto{\pgfqpoint{6.134740in}{1.667817in}}{\pgfqpoint{6.127898in}{1.664983in}}{\pgfqpoint{6.122855in}{1.659939in}}%
\pgfpathcurveto{\pgfqpoint{6.117811in}{1.654896in}}{\pgfqpoint{6.114977in}{1.648054in}}{\pgfqpoint{6.114977in}{1.640921in}}%
\pgfpathcurveto{\pgfqpoint{6.114977in}{1.633788in}}{\pgfqpoint{6.117811in}{1.626947in}}{\pgfqpoint{6.122855in}{1.621903in}}%
\pgfpathcurveto{\pgfqpoint{6.127898in}{1.616859in}}{\pgfqpoint{6.134740in}{1.614025in}}{\pgfqpoint{6.141873in}{1.614025in}}%
\pgfpathclose%
\pgfusepath{stroke,fill}%
\end{pgfscope}%
\begin{pgfscope}%
\pgfpathrectangle{\pgfqpoint{4.985294in}{0.500000in}}{\pgfqpoint{1.764706in}{1.700000in}}%
\pgfusepath{clip}%
\pgfsetbuttcap%
\pgfsetroundjoin%
\definecolor{currentfill}{rgb}{0.965440,0.720101,0.576404}%
\pgfsetfillcolor{currentfill}%
\pgfsetlinewidth{0.311001pt}%
\definecolor{currentstroke}{rgb}{1.000000,1.000000,1.000000}%
\pgfsetstrokecolor{currentstroke}%
\pgfsetdash{}{0pt}%
\pgfpathmoveto{\pgfqpoint{5.550419in}{1.119344in}}%
\pgfpathcurveto{\pgfqpoint{5.557552in}{1.119344in}}{\pgfqpoint{5.564394in}{1.122178in}}{\pgfqpoint{5.569438in}{1.127222in}}%
\pgfpathcurveto{\pgfqpoint{5.574481in}{1.132265in}}{\pgfqpoint{5.577315in}{1.139107in}}{\pgfqpoint{5.577315in}{1.146240in}}%
\pgfpathcurveto{\pgfqpoint{5.577315in}{1.153373in}}{\pgfqpoint{5.574481in}{1.160214in}}{\pgfqpoint{5.569438in}{1.165258in}}%
\pgfpathcurveto{\pgfqpoint{5.564394in}{1.170302in}}{\pgfqpoint{5.557552in}{1.173136in}}{\pgfqpoint{5.550419in}{1.173136in}}%
\pgfpathcurveto{\pgfqpoint{5.543287in}{1.173136in}}{\pgfqpoint{5.536445in}{1.170302in}}{\pgfqpoint{5.531401in}{1.165258in}}%
\pgfpathcurveto{\pgfqpoint{5.526358in}{1.160214in}}{\pgfqpoint{5.523524in}{1.153373in}}{\pgfqpoint{5.523524in}{1.146240in}}%
\pgfpathcurveto{\pgfqpoint{5.523524in}{1.139107in}}{\pgfqpoint{5.526358in}{1.132265in}}{\pgfqpoint{5.531401in}{1.127222in}}%
\pgfpathcurveto{\pgfqpoint{5.536445in}{1.122178in}}{\pgfqpoint{5.543287in}{1.119344in}}{\pgfqpoint{5.550419in}{1.119344in}}%
\pgfpathclose%
\pgfusepath{stroke,fill}%
\end{pgfscope}%
\begin{pgfscope}%
\pgfpathrectangle{\pgfqpoint{4.985294in}{0.500000in}}{\pgfqpoint{1.764706in}{1.700000in}}%
\pgfusepath{clip}%
\pgfsetbuttcap%
\pgfsetroundjoin%
\definecolor{currentfill}{rgb}{0.976961,0.885681,0.814303}%
\pgfsetfillcolor{currentfill}%
\pgfsetlinewidth{0.311001pt}%
\definecolor{currentstroke}{rgb}{1.000000,1.000000,1.000000}%
\pgfsetstrokecolor{currentstroke}%
\pgfsetdash{}{0pt}%
\pgfpathmoveto{\pgfqpoint{5.462421in}{1.421390in}}%
\pgfpathcurveto{\pgfqpoint{5.469554in}{1.421390in}}{\pgfqpoint{5.476395in}{1.424224in}}{\pgfqpoint{5.481439in}{1.429268in}}%
\pgfpathcurveto{\pgfqpoint{5.486483in}{1.434312in}}{\pgfqpoint{5.489316in}{1.441153in}}{\pgfqpoint{5.489316in}{1.448286in}}%
\pgfpathcurveto{\pgfqpoint{5.489316in}{1.455419in}}{\pgfqpoint{5.486483in}{1.462261in}}{\pgfqpoint{5.481439in}{1.467304in}}%
\pgfpathcurveto{\pgfqpoint{5.476395in}{1.472348in}}{\pgfqpoint{5.469554in}{1.475182in}}{\pgfqpoint{5.462421in}{1.475182in}}%
\pgfpathcurveto{\pgfqpoint{5.455288in}{1.475182in}}{\pgfqpoint{5.448446in}{1.472348in}}{\pgfqpoint{5.443403in}{1.467304in}}%
\pgfpathcurveto{\pgfqpoint{5.438359in}{1.462261in}}{\pgfqpoint{5.435525in}{1.455419in}}{\pgfqpoint{5.435525in}{1.448286in}}%
\pgfpathcurveto{\pgfqpoint{5.435525in}{1.441153in}}{\pgfqpoint{5.438359in}{1.434312in}}{\pgfqpoint{5.443403in}{1.429268in}}%
\pgfpathcurveto{\pgfqpoint{5.448446in}{1.424224in}}{\pgfqpoint{5.455288in}{1.421390in}}{\pgfqpoint{5.462421in}{1.421390in}}%
\pgfpathclose%
\pgfusepath{stroke,fill}%
\end{pgfscope}%
\begin{pgfscope}%
\pgfpathrectangle{\pgfqpoint{4.985294in}{0.500000in}}{\pgfqpoint{1.764706in}{1.700000in}}%
\pgfusepath{clip}%
\pgfsetbuttcap%
\pgfsetroundjoin%
\definecolor{currentfill}{rgb}{0.965042,0.701564,0.552889}%
\pgfsetfillcolor{currentfill}%
\pgfsetlinewidth{0.311001pt}%
\definecolor{currentstroke}{rgb}{1.000000,1.000000,1.000000}%
\pgfsetstrokecolor{currentstroke}%
\pgfsetdash{}{0pt}%
\pgfpathmoveto{\pgfqpoint{5.585674in}{1.599159in}}%
\pgfpathcurveto{\pgfqpoint{5.592807in}{1.599159in}}{\pgfqpoint{5.599649in}{1.601993in}}{\pgfqpoint{5.604692in}{1.607036in}}%
\pgfpathcurveto{\pgfqpoint{5.609736in}{1.612080in}}{\pgfqpoint{5.612570in}{1.618922in}}{\pgfqpoint{5.612570in}{1.626054in}}%
\pgfpathcurveto{\pgfqpoint{5.612570in}{1.633187in}}{\pgfqpoint{5.609736in}{1.640029in}}{\pgfqpoint{5.604692in}{1.645073in}}%
\pgfpathcurveto{\pgfqpoint{5.599649in}{1.650116in}}{\pgfqpoint{5.592807in}{1.652950in}}{\pgfqpoint{5.585674in}{1.652950in}}%
\pgfpathcurveto{\pgfqpoint{5.578541in}{1.652950in}}{\pgfqpoint{5.571700in}{1.650116in}}{\pgfqpoint{5.566656in}{1.645073in}}%
\pgfpathcurveto{\pgfqpoint{5.561612in}{1.640029in}}{\pgfqpoint{5.558779in}{1.633187in}}{\pgfqpoint{5.558779in}{1.626054in}}%
\pgfpathcurveto{\pgfqpoint{5.558779in}{1.618922in}}{\pgfqpoint{5.561612in}{1.612080in}}{\pgfqpoint{5.566656in}{1.607036in}}%
\pgfpathcurveto{\pgfqpoint{5.571700in}{1.601993in}}{\pgfqpoint{5.578541in}{1.599159in}}{\pgfqpoint{5.585674in}{1.599159in}}%
\pgfpathclose%
\pgfusepath{stroke,fill}%
\end{pgfscope}%
\begin{pgfscope}%
\pgfpathrectangle{\pgfqpoint{4.985294in}{0.500000in}}{\pgfqpoint{1.764706in}{1.700000in}}%
\pgfusepath{clip}%
\pgfsetbuttcap%
\pgfsetroundjoin%
\definecolor{currentfill}{rgb}{0.980678,0.914765,0.856766}%
\pgfsetfillcolor{currentfill}%
\pgfsetlinewidth{0.311001pt}%
\definecolor{currentstroke}{rgb}{1.000000,1.000000,1.000000}%
\pgfsetstrokecolor{currentstroke}%
\pgfsetdash{}{0pt}%
\pgfpathmoveto{\pgfqpoint{5.406342in}{1.330383in}}%
\pgfpathcurveto{\pgfqpoint{5.413475in}{1.330383in}}{\pgfqpoint{5.420317in}{1.333217in}}{\pgfqpoint{5.425360in}{1.338261in}}%
\pgfpathcurveto{\pgfqpoint{5.430404in}{1.343304in}}{\pgfqpoint{5.433238in}{1.350146in}}{\pgfqpoint{5.433238in}{1.357279in}}%
\pgfpathcurveto{\pgfqpoint{5.433238in}{1.364412in}}{\pgfqpoint{5.430404in}{1.371253in}}{\pgfqpoint{5.425360in}{1.376297in}}%
\pgfpathcurveto{\pgfqpoint{5.420317in}{1.381341in}}{\pgfqpoint{5.413475in}{1.384174in}}{\pgfqpoint{5.406342in}{1.384174in}}%
\pgfpathcurveto{\pgfqpoint{5.399209in}{1.384174in}}{\pgfqpoint{5.392368in}{1.381341in}}{\pgfqpoint{5.387324in}{1.376297in}}%
\pgfpathcurveto{\pgfqpoint{5.382280in}{1.371253in}}{\pgfqpoint{5.379446in}{1.364412in}}{\pgfqpoint{5.379446in}{1.357279in}}%
\pgfpathcurveto{\pgfqpoint{5.379446in}{1.350146in}}{\pgfqpoint{5.382280in}{1.343304in}}{\pgfqpoint{5.387324in}{1.338261in}}%
\pgfpathcurveto{\pgfqpoint{5.392368in}{1.333217in}}{\pgfqpoint{5.399209in}{1.330383in}}{\pgfqpoint{5.406342in}{1.330383in}}%
\pgfpathclose%
\pgfusepath{stroke,fill}%
\end{pgfscope}%
\begin{pgfscope}%
\pgfpathrectangle{\pgfqpoint{4.985294in}{0.500000in}}{\pgfqpoint{1.764706in}{1.700000in}}%
\pgfusepath{clip}%
\pgfsetbuttcap%
\pgfsetroundjoin%
\definecolor{currentfill}{rgb}{0.966120,0.744512,0.608720}%
\pgfsetfillcolor{currentfill}%
\pgfsetlinewidth{0.311001pt}%
\definecolor{currentstroke}{rgb}{1.000000,1.000000,1.000000}%
\pgfsetstrokecolor{currentstroke}%
\pgfsetdash{}{0pt}%
\pgfpathmoveto{\pgfqpoint{6.282043in}{0.990537in}}%
\pgfpathcurveto{\pgfqpoint{6.289176in}{0.990537in}}{\pgfqpoint{6.296017in}{0.993371in}}{\pgfqpoint{6.301061in}{0.998414in}}%
\pgfpathcurveto{\pgfqpoint{6.306105in}{1.003458in}}{\pgfqpoint{6.308938in}{1.010300in}}{\pgfqpoint{6.308938in}{1.017433in}}%
\pgfpathcurveto{\pgfqpoint{6.308938in}{1.024565in}}{\pgfqpoint{6.306105in}{1.031407in}}{\pgfqpoint{6.301061in}{1.036451in}}%
\pgfpathcurveto{\pgfqpoint{6.296017in}{1.041494in}}{\pgfqpoint{6.289176in}{1.044328in}}{\pgfqpoint{6.282043in}{1.044328in}}%
\pgfpathcurveto{\pgfqpoint{6.274910in}{1.044328in}}{\pgfqpoint{6.268068in}{1.041494in}}{\pgfqpoint{6.263025in}{1.036451in}}%
\pgfpathcurveto{\pgfqpoint{6.257981in}{1.031407in}}{\pgfqpoint{6.255147in}{1.024565in}}{\pgfqpoint{6.255147in}{1.017433in}}%
\pgfpathcurveto{\pgfqpoint{6.255147in}{1.010300in}}{\pgfqpoint{6.257981in}{1.003458in}}{\pgfqpoint{6.263025in}{0.998414in}}%
\pgfpathcurveto{\pgfqpoint{6.268068in}{0.993371in}}{\pgfqpoint{6.274910in}{0.990537in}}{\pgfqpoint{6.282043in}{0.990537in}}%
\pgfpathclose%
\pgfusepath{stroke,fill}%
\end{pgfscope}%
\begin{pgfscope}%
\pgfpathrectangle{\pgfqpoint{4.985294in}{0.500000in}}{\pgfqpoint{1.764706in}{1.700000in}}%
\pgfusepath{clip}%
\pgfsetbuttcap%
\pgfsetroundjoin%
\definecolor{currentfill}{rgb}{0.980678,0.914765,0.856766}%
\pgfsetfillcolor{currentfill}%
\pgfsetlinewidth{0.311001pt}%
\definecolor{currentstroke}{rgb}{1.000000,1.000000,1.000000}%
\pgfsetstrokecolor{currentstroke}%
\pgfsetdash{}{0pt}%
\pgfpathmoveto{\pgfqpoint{6.310395in}{1.229204in}}%
\pgfpathcurveto{\pgfqpoint{6.317528in}{1.229204in}}{\pgfqpoint{6.324369in}{1.232038in}}{\pgfqpoint{6.329413in}{1.237082in}}%
\pgfpathcurveto{\pgfqpoint{6.334457in}{1.242126in}}{\pgfqpoint{6.337290in}{1.248967in}}{\pgfqpoint{6.337290in}{1.256100in}}%
\pgfpathcurveto{\pgfqpoint{6.337290in}{1.263233in}}{\pgfqpoint{6.334457in}{1.270075in}}{\pgfqpoint{6.329413in}{1.275118in}}%
\pgfpathcurveto{\pgfqpoint{6.324369in}{1.280162in}}{\pgfqpoint{6.317528in}{1.282996in}}{\pgfqpoint{6.310395in}{1.282996in}}%
\pgfpathcurveto{\pgfqpoint{6.303262in}{1.282996in}}{\pgfqpoint{6.296420in}{1.280162in}}{\pgfqpoint{6.291377in}{1.275118in}}%
\pgfpathcurveto{\pgfqpoint{6.286333in}{1.270075in}}{\pgfqpoint{6.283499in}{1.263233in}}{\pgfqpoint{6.283499in}{1.256100in}}%
\pgfpathcurveto{\pgfqpoint{6.283499in}{1.248967in}}{\pgfqpoint{6.286333in}{1.242126in}}{\pgfqpoint{6.291377in}{1.237082in}}%
\pgfpathcurveto{\pgfqpoint{6.296420in}{1.232038in}}{\pgfqpoint{6.303262in}{1.229204in}}{\pgfqpoint{6.310395in}{1.229204in}}%
\pgfpathclose%
\pgfusepath{stroke,fill}%
\end{pgfscope}%
\begin{pgfscope}%
\pgfpathrectangle{\pgfqpoint{4.985294in}{0.500000in}}{\pgfqpoint{1.764706in}{1.700000in}}%
\pgfusepath{clip}%
\pgfsetbuttcap%
\pgfsetroundjoin%
\definecolor{currentfill}{rgb}{0.924566,0.290534,0.242426}%
\pgfsetfillcolor{currentfill}%
\pgfsetlinewidth{0.311001pt}%
\definecolor{currentstroke}{rgb}{1.000000,1.000000,1.000000}%
\pgfsetstrokecolor{currentstroke}%
\pgfsetdash{}{0pt}%
\pgfpathmoveto{\pgfqpoint{5.698817in}{1.003893in}}%
\pgfpathcurveto{\pgfqpoint{5.705949in}{1.003893in}}{\pgfqpoint{5.712791in}{1.006727in}}{\pgfqpoint{5.717835in}{1.011771in}}%
\pgfpathcurveto{\pgfqpoint{5.722878in}{1.016815in}}{\pgfqpoint{5.725712in}{1.023656in}}{\pgfqpoint{5.725712in}{1.030789in}}%
\pgfpathcurveto{\pgfqpoint{5.725712in}{1.037922in}}{\pgfqpoint{5.722878in}{1.044764in}}{\pgfqpoint{5.717835in}{1.049807in}}%
\pgfpathcurveto{\pgfqpoint{5.712791in}{1.054851in}}{\pgfqpoint{5.705949in}{1.057685in}}{\pgfqpoint{5.698817in}{1.057685in}}%
\pgfpathcurveto{\pgfqpoint{5.691684in}{1.057685in}}{\pgfqpoint{5.684842in}{1.054851in}}{\pgfqpoint{5.679798in}{1.049807in}}%
\pgfpathcurveto{\pgfqpoint{5.674755in}{1.044764in}}{\pgfqpoint{5.671921in}{1.037922in}}{\pgfqpoint{5.671921in}{1.030789in}}%
\pgfpathcurveto{\pgfqpoint{5.671921in}{1.023656in}}{\pgfqpoint{5.674755in}{1.016815in}}{\pgfqpoint{5.679798in}{1.011771in}}%
\pgfpathcurveto{\pgfqpoint{5.684842in}{1.006727in}}{\pgfqpoint{5.691684in}{1.003893in}}{\pgfqpoint{5.698817in}{1.003893in}}%
\pgfpathclose%
\pgfusepath{stroke,fill}%
\end{pgfscope}%
\begin{pgfscope}%
\pgfpathrectangle{\pgfqpoint{4.985294in}{0.500000in}}{\pgfqpoint{1.764706in}{1.700000in}}%
\pgfusepath{clip}%
\pgfsetbuttcap%
\pgfsetroundjoin%
\definecolor{currentfill}{rgb}{0.965592,0.726236,0.584384}%
\pgfsetfillcolor{currentfill}%
\pgfsetlinewidth{0.311001pt}%
\definecolor{currentstroke}{rgb}{1.000000,1.000000,1.000000}%
\pgfsetstrokecolor{currentstroke}%
\pgfsetdash{}{0pt}%
\pgfpathmoveto{\pgfqpoint{6.159548in}{1.550533in}}%
\pgfpathcurveto{\pgfqpoint{6.166681in}{1.550533in}}{\pgfqpoint{6.173522in}{1.553367in}}{\pgfqpoint{6.178566in}{1.558411in}}%
\pgfpathcurveto{\pgfqpoint{6.183610in}{1.563454in}}{\pgfqpoint{6.186443in}{1.570296in}}{\pgfqpoint{6.186443in}{1.577429in}}%
\pgfpathcurveto{\pgfqpoint{6.186443in}{1.584562in}}{\pgfqpoint{6.183610in}{1.591403in}}{\pgfqpoint{6.178566in}{1.596447in}}%
\pgfpathcurveto{\pgfqpoint{6.173522in}{1.601491in}}{\pgfqpoint{6.166681in}{1.604324in}}{\pgfqpoint{6.159548in}{1.604324in}}%
\pgfpathcurveto{\pgfqpoint{6.152415in}{1.604324in}}{\pgfqpoint{6.145573in}{1.601491in}}{\pgfqpoint{6.140530in}{1.596447in}}%
\pgfpathcurveto{\pgfqpoint{6.135486in}{1.591403in}}{\pgfqpoint{6.132652in}{1.584562in}}{\pgfqpoint{6.132652in}{1.577429in}}%
\pgfpathcurveto{\pgfqpoint{6.132652in}{1.570296in}}{\pgfqpoint{6.135486in}{1.563454in}}{\pgfqpoint{6.140530in}{1.558411in}}%
\pgfpathcurveto{\pgfqpoint{6.145573in}{1.553367in}}{\pgfqpoint{6.152415in}{1.550533in}}{\pgfqpoint{6.159548in}{1.550533in}}%
\pgfpathclose%
\pgfusepath{stroke,fill}%
\end{pgfscope}%
\begin{pgfscope}%
\pgfpathrectangle{\pgfqpoint{4.985294in}{0.500000in}}{\pgfqpoint{1.764706in}{1.700000in}}%
\pgfusepath{clip}%
\pgfsetbuttcap%
\pgfsetroundjoin%
\definecolor{currentfill}{rgb}{0.975644,0.874038,0.797253}%
\pgfsetfillcolor{currentfill}%
\pgfsetlinewidth{0.311001pt}%
\definecolor{currentstroke}{rgb}{1.000000,1.000000,1.000000}%
\pgfsetstrokecolor{currentstroke}%
\pgfsetdash{}{0pt}%
\pgfpathmoveto{\pgfqpoint{5.477237in}{1.461212in}}%
\pgfpathcurveto{\pgfqpoint{5.484370in}{1.461212in}}{\pgfqpoint{5.491212in}{1.464046in}}{\pgfqpoint{5.496255in}{1.469090in}}%
\pgfpathcurveto{\pgfqpoint{5.501299in}{1.474133in}}{\pgfqpoint{5.504133in}{1.480975in}}{\pgfqpoint{5.504133in}{1.488108in}}%
\pgfpathcurveto{\pgfqpoint{5.504133in}{1.495241in}}{\pgfqpoint{5.501299in}{1.502082in}}{\pgfqpoint{5.496255in}{1.507126in}}%
\pgfpathcurveto{\pgfqpoint{5.491212in}{1.512170in}}{\pgfqpoint{5.484370in}{1.515004in}}{\pgfqpoint{5.477237in}{1.515004in}}%
\pgfpathcurveto{\pgfqpoint{5.470104in}{1.515004in}}{\pgfqpoint{5.463263in}{1.512170in}}{\pgfqpoint{5.458219in}{1.507126in}}%
\pgfpathcurveto{\pgfqpoint{5.453175in}{1.502082in}}{\pgfqpoint{5.450342in}{1.495241in}}{\pgfqpoint{5.450342in}{1.488108in}}%
\pgfpathcurveto{\pgfqpoint{5.450342in}{1.480975in}}{\pgfqpoint{5.453175in}{1.474133in}}{\pgfqpoint{5.458219in}{1.469090in}}%
\pgfpathcurveto{\pgfqpoint{5.463263in}{1.464046in}}{\pgfqpoint{5.470104in}{1.461212in}}{\pgfqpoint{5.477237in}{1.461212in}}%
\pgfpathclose%
\pgfusepath{stroke,fill}%
\end{pgfscope}%
\begin{pgfscope}%
\pgfpathrectangle{\pgfqpoint{4.985294in}{0.500000in}}{\pgfqpoint{1.764706in}{1.700000in}}%
\pgfusepath{clip}%
\pgfsetbuttcap%
\pgfsetroundjoin%
\definecolor{currentfill}{rgb}{0.975644,0.874038,0.797253}%
\pgfsetfillcolor{currentfill}%
\pgfsetlinewidth{0.311001pt}%
\definecolor{currentstroke}{rgb}{1.000000,1.000000,1.000000}%
\pgfsetstrokecolor{currentstroke}%
\pgfsetdash{}{0pt}%
\pgfpathmoveto{\pgfqpoint{5.474057in}{1.156828in}}%
\pgfpathcurveto{\pgfqpoint{5.481189in}{1.156828in}}{\pgfqpoint{5.488031in}{1.159661in}}{\pgfqpoint{5.493075in}{1.164705in}}%
\pgfpathcurveto{\pgfqpoint{5.498118in}{1.169749in}}{\pgfqpoint{5.500952in}{1.176590in}}{\pgfqpoint{5.500952in}{1.183723in}}%
\pgfpathcurveto{\pgfqpoint{5.500952in}{1.190856in}}{\pgfqpoint{5.498118in}{1.197698in}}{\pgfqpoint{5.493075in}{1.202741in}}%
\pgfpathcurveto{\pgfqpoint{5.488031in}{1.207785in}}{\pgfqpoint{5.481189in}{1.210619in}}{\pgfqpoint{5.474057in}{1.210619in}}%
\pgfpathcurveto{\pgfqpoint{5.466924in}{1.210619in}}{\pgfqpoint{5.460082in}{1.207785in}}{\pgfqpoint{5.455038in}{1.202741in}}%
\pgfpathcurveto{\pgfqpoint{5.449995in}{1.197698in}}{\pgfqpoint{5.447161in}{1.190856in}}{\pgfqpoint{5.447161in}{1.183723in}}%
\pgfpathcurveto{\pgfqpoint{5.447161in}{1.176590in}}{\pgfqpoint{5.449995in}{1.169749in}}{\pgfqpoint{5.455038in}{1.164705in}}%
\pgfpathcurveto{\pgfqpoint{5.460082in}{1.159661in}}{\pgfqpoint{5.466924in}{1.156828in}}{\pgfqpoint{5.474057in}{1.156828in}}%
\pgfpathclose%
\pgfusepath{stroke,fill}%
\end{pgfscope}%
\begin{pgfscope}%
\pgfpathrectangle{\pgfqpoint{4.985294in}{0.500000in}}{\pgfqpoint{1.764706in}{1.700000in}}%
\pgfusepath{clip}%
\pgfsetbuttcap%
\pgfsetroundjoin%
\definecolor{currentfill}{rgb}{0.962283,0.593046,0.431453}%
\pgfsetfillcolor{currentfill}%
\pgfsetlinewidth{0.311001pt}%
\definecolor{currentstroke}{rgb}{1.000000,1.000000,1.000000}%
\pgfsetstrokecolor{currentstroke}%
\pgfsetdash{}{0pt}%
\pgfpathmoveto{\pgfqpoint{5.328782in}{1.494704in}}%
\pgfpathcurveto{\pgfqpoint{5.335915in}{1.494704in}}{\pgfqpoint{5.342756in}{1.497537in}}{\pgfqpoint{5.347800in}{1.502581in}}%
\pgfpathcurveto{\pgfqpoint{5.352844in}{1.507625in}}{\pgfqpoint{5.355678in}{1.514466in}}{\pgfqpoint{5.355678in}{1.521599in}}%
\pgfpathcurveto{\pgfqpoint{5.355678in}{1.528732in}}{\pgfqpoint{5.352844in}{1.535574in}}{\pgfqpoint{5.347800in}{1.540617in}}%
\pgfpathcurveto{\pgfqpoint{5.342756in}{1.545661in}}{\pgfqpoint{5.335915in}{1.548495in}}{\pgfqpoint{5.328782in}{1.548495in}}%
\pgfpathcurveto{\pgfqpoint{5.321649in}{1.548495in}}{\pgfqpoint{5.314807in}{1.545661in}}{\pgfqpoint{5.309764in}{1.540617in}}%
\pgfpathcurveto{\pgfqpoint{5.304720in}{1.535574in}}{\pgfqpoint{5.301886in}{1.528732in}}{\pgfqpoint{5.301886in}{1.521599in}}%
\pgfpathcurveto{\pgfqpoint{5.301886in}{1.514466in}}{\pgfqpoint{5.304720in}{1.507625in}}{\pgfqpoint{5.309764in}{1.502581in}}%
\pgfpathcurveto{\pgfqpoint{5.314807in}{1.497537in}}{\pgfqpoint{5.321649in}{1.494704in}}{\pgfqpoint{5.328782in}{1.494704in}}%
\pgfpathclose%
\pgfusepath{stroke,fill}%
\end{pgfscope}%
\begin{pgfscope}%
\pgfpathrectangle{\pgfqpoint{4.985294in}{0.500000in}}{\pgfqpoint{1.764706in}{1.700000in}}%
\pgfusepath{clip}%
\pgfsetbuttcap%
\pgfsetroundjoin%
\definecolor{currentfill}{rgb}{0.964173,0.657587,0.500469}%
\pgfsetfillcolor{currentfill}%
\pgfsetlinewidth{0.311001pt}%
\definecolor{currentstroke}{rgb}{1.000000,1.000000,1.000000}%
\pgfsetstrokecolor{currentstroke}%
\pgfsetdash{}{0pt}%
\pgfpathmoveto{\pgfqpoint{6.382507in}{1.533471in}}%
\pgfpathcurveto{\pgfqpoint{6.389639in}{1.533471in}}{\pgfqpoint{6.396481in}{1.536305in}}{\pgfqpoint{6.401525in}{1.541349in}}%
\pgfpathcurveto{\pgfqpoint{6.406568in}{1.546392in}}{\pgfqpoint{6.409402in}{1.553234in}}{\pgfqpoint{6.409402in}{1.560367in}}%
\pgfpathcurveto{\pgfqpoint{6.409402in}{1.567500in}}{\pgfqpoint{6.406568in}{1.574341in}}{\pgfqpoint{6.401525in}{1.579385in}}%
\pgfpathcurveto{\pgfqpoint{6.396481in}{1.584429in}}{\pgfqpoint{6.389639in}{1.587262in}}{\pgfqpoint{6.382507in}{1.587262in}}%
\pgfpathcurveto{\pgfqpoint{6.375374in}{1.587262in}}{\pgfqpoint{6.368532in}{1.584429in}}{\pgfqpoint{6.363488in}{1.579385in}}%
\pgfpathcurveto{\pgfqpoint{6.358445in}{1.574341in}}{\pgfqpoint{6.355611in}{1.567500in}}{\pgfqpoint{6.355611in}{1.560367in}}%
\pgfpathcurveto{\pgfqpoint{6.355611in}{1.553234in}}{\pgfqpoint{6.358445in}{1.546392in}}{\pgfqpoint{6.363488in}{1.541349in}}%
\pgfpathcurveto{\pgfqpoint{6.368532in}{1.536305in}}{\pgfqpoint{6.375374in}{1.533471in}}{\pgfqpoint{6.382507in}{1.533471in}}%
\pgfpathclose%
\pgfusepath{stroke,fill}%
\end{pgfscope}%
\begin{pgfscope}%
\pgfpathrectangle{\pgfqpoint{4.985294in}{0.500000in}}{\pgfqpoint{1.764706in}{1.700000in}}%
\pgfusepath{clip}%
\pgfsetbuttcap%
\pgfsetroundjoin%
\definecolor{currentfill}{rgb}{0.962765,0.606121,0.444717}%
\pgfsetfillcolor{currentfill}%
\pgfsetlinewidth{0.311001pt}%
\definecolor{currentstroke}{rgb}{1.000000,1.000000,1.000000}%
\pgfsetstrokecolor{currentstroke}%
\pgfsetdash{}{0pt}%
\pgfpathmoveto{\pgfqpoint{5.513239in}{0.874053in}}%
\pgfpathcurveto{\pgfqpoint{5.520371in}{0.874053in}}{\pgfqpoint{5.527213in}{0.876887in}}{\pgfqpoint{5.532257in}{0.881931in}}%
\pgfpathcurveto{\pgfqpoint{5.537300in}{0.886975in}}{\pgfqpoint{5.540134in}{0.893816in}}{\pgfqpoint{5.540134in}{0.900949in}}%
\pgfpathcurveto{\pgfqpoint{5.540134in}{0.908082in}}{\pgfqpoint{5.537300in}{0.914924in}}{\pgfqpoint{5.532257in}{0.919967in}}%
\pgfpathcurveto{\pgfqpoint{5.527213in}{0.925011in}}{\pgfqpoint{5.520371in}{0.927845in}}{\pgfqpoint{5.513239in}{0.927845in}}%
\pgfpathcurveto{\pgfqpoint{5.506106in}{0.927845in}}{\pgfqpoint{5.499264in}{0.925011in}}{\pgfqpoint{5.494220in}{0.919967in}}%
\pgfpathcurveto{\pgfqpoint{5.489177in}{0.914924in}}{\pgfqpoint{5.486343in}{0.908082in}}{\pgfqpoint{5.486343in}{0.900949in}}%
\pgfpathcurveto{\pgfqpoint{5.486343in}{0.893816in}}{\pgfqpoint{5.489177in}{0.886975in}}{\pgfqpoint{5.494220in}{0.881931in}}%
\pgfpathcurveto{\pgfqpoint{5.499264in}{0.876887in}}{\pgfqpoint{5.506106in}{0.874053in}}{\pgfqpoint{5.513239in}{0.874053in}}%
\pgfpathclose%
\pgfusepath{stroke,fill}%
\end{pgfscope}%
\begin{pgfscope}%
\pgfpathrectangle{\pgfqpoint{4.985294in}{0.500000in}}{\pgfqpoint{1.764706in}{1.700000in}}%
\pgfusepath{clip}%
\pgfsetbuttcap%
\pgfsetroundjoin%
\definecolor{currentfill}{rgb}{0.962985,0.612625,0.451451}%
\pgfsetfillcolor{currentfill}%
\pgfsetlinewidth{0.311001pt}%
\definecolor{currentstroke}{rgb}{1.000000,1.000000,1.000000}%
\pgfsetstrokecolor{currentstroke}%
\pgfsetdash{}{0pt}%
\pgfpathmoveto{\pgfqpoint{6.153203in}{1.750951in}}%
\pgfpathcurveto{\pgfqpoint{6.160336in}{1.750951in}}{\pgfqpoint{6.167177in}{1.753785in}}{\pgfqpoint{6.172221in}{1.758828in}}%
\pgfpathcurveto{\pgfqpoint{6.177265in}{1.763872in}}{\pgfqpoint{6.180099in}{1.770714in}}{\pgfqpoint{6.180099in}{1.777847in}}%
\pgfpathcurveto{\pgfqpoint{6.180099in}{1.784979in}}{\pgfqpoint{6.177265in}{1.791821in}}{\pgfqpoint{6.172221in}{1.796865in}}%
\pgfpathcurveto{\pgfqpoint{6.167177in}{1.801908in}}{\pgfqpoint{6.160336in}{1.804742in}}{\pgfqpoint{6.153203in}{1.804742in}}%
\pgfpathcurveto{\pgfqpoint{6.146070in}{1.804742in}}{\pgfqpoint{6.139228in}{1.801908in}}{\pgfqpoint{6.134185in}{1.796865in}}%
\pgfpathcurveto{\pgfqpoint{6.129141in}{1.791821in}}{\pgfqpoint{6.126307in}{1.784979in}}{\pgfqpoint{6.126307in}{1.777847in}}%
\pgfpathcurveto{\pgfqpoint{6.126307in}{1.770714in}}{\pgfqpoint{6.129141in}{1.763872in}}{\pgfqpoint{6.134185in}{1.758828in}}%
\pgfpathcurveto{\pgfqpoint{6.139228in}{1.753785in}}{\pgfqpoint{6.146070in}{1.750951in}}{\pgfqpoint{6.153203in}{1.750951in}}%
\pgfpathclose%
\pgfusepath{stroke,fill}%
\end{pgfscope}%
\begin{pgfscope}%
\pgfpathrectangle{\pgfqpoint{4.985294in}{0.500000in}}{\pgfqpoint{1.764706in}{1.700000in}}%
\pgfusepath{clip}%
\pgfsetbuttcap%
\pgfsetroundjoin%
\definecolor{currentfill}{rgb}{0.969359,0.803954,0.693832}%
\pgfsetfillcolor{currentfill}%
\pgfsetlinewidth{0.311001pt}%
\definecolor{currentstroke}{rgb}{1.000000,1.000000,1.000000}%
\pgfsetstrokecolor{currentstroke}%
\pgfsetdash{}{0pt}%
\pgfpathmoveto{\pgfqpoint{6.195679in}{1.683871in}}%
\pgfpathcurveto{\pgfqpoint{6.202812in}{1.683871in}}{\pgfqpoint{6.209653in}{1.686705in}}{\pgfqpoint{6.214697in}{1.691749in}}%
\pgfpathcurveto{\pgfqpoint{6.219741in}{1.696792in}}{\pgfqpoint{6.222575in}{1.703634in}}{\pgfqpoint{6.222575in}{1.710767in}}%
\pgfpathcurveto{\pgfqpoint{6.222575in}{1.717900in}}{\pgfqpoint{6.219741in}{1.724741in}}{\pgfqpoint{6.214697in}{1.729785in}}%
\pgfpathcurveto{\pgfqpoint{6.209653in}{1.734829in}}{\pgfqpoint{6.202812in}{1.737663in}}{\pgfqpoint{6.195679in}{1.737663in}}%
\pgfpathcurveto{\pgfqpoint{6.188546in}{1.737663in}}{\pgfqpoint{6.181704in}{1.734829in}}{\pgfqpoint{6.176661in}{1.729785in}}%
\pgfpathcurveto{\pgfqpoint{6.171617in}{1.724741in}}{\pgfqpoint{6.168783in}{1.717900in}}{\pgfqpoint{6.168783in}{1.710767in}}%
\pgfpathcurveto{\pgfqpoint{6.168783in}{1.703634in}}{\pgfqpoint{6.171617in}{1.696792in}}{\pgfqpoint{6.176661in}{1.691749in}}%
\pgfpathcurveto{\pgfqpoint{6.181704in}{1.686705in}}{\pgfqpoint{6.188546in}{1.683871in}}{\pgfqpoint{6.195679in}{1.683871in}}%
\pgfpathclose%
\pgfusepath{stroke,fill}%
\end{pgfscope}%
\begin{pgfscope}%
\pgfpathrectangle{\pgfqpoint{4.985294in}{0.500000in}}{\pgfqpoint{1.764706in}{1.700000in}}%
\pgfusepath{clip}%
\pgfsetbuttcap%
\pgfsetroundjoin%
\definecolor{currentfill}{rgb}{0.964679,0.682838,0.530002}%
\pgfsetfillcolor{currentfill}%
\pgfsetlinewidth{0.311001pt}%
\definecolor{currentstroke}{rgb}{1.000000,1.000000,1.000000}%
\pgfsetstrokecolor{currentstroke}%
\pgfsetdash{}{0pt}%
\pgfpathmoveto{\pgfqpoint{5.528789in}{1.335958in}}%
\pgfpathcurveto{\pgfqpoint{5.535922in}{1.335958in}}{\pgfqpoint{5.542764in}{1.338792in}}{\pgfqpoint{5.547807in}{1.343836in}}%
\pgfpathcurveto{\pgfqpoint{5.552851in}{1.348879in}}{\pgfqpoint{5.555685in}{1.355721in}}{\pgfqpoint{5.555685in}{1.362854in}}%
\pgfpathcurveto{\pgfqpoint{5.555685in}{1.369987in}}{\pgfqpoint{5.552851in}{1.376828in}}{\pgfqpoint{5.547807in}{1.381872in}}%
\pgfpathcurveto{\pgfqpoint{5.542764in}{1.386916in}}{\pgfqpoint{5.535922in}{1.389750in}}{\pgfqpoint{5.528789in}{1.389750in}}%
\pgfpathcurveto{\pgfqpoint{5.521656in}{1.389750in}}{\pgfqpoint{5.514815in}{1.386916in}}{\pgfqpoint{5.509771in}{1.381872in}}%
\pgfpathcurveto{\pgfqpoint{5.504727in}{1.376828in}}{\pgfqpoint{5.501893in}{1.369987in}}{\pgfqpoint{5.501893in}{1.362854in}}%
\pgfpathcurveto{\pgfqpoint{5.501893in}{1.355721in}}{\pgfqpoint{5.504727in}{1.348879in}}{\pgfqpoint{5.509771in}{1.343836in}}%
\pgfpathcurveto{\pgfqpoint{5.514815in}{1.338792in}}{\pgfqpoint{5.521656in}{1.335958in}}{\pgfqpoint{5.528789in}{1.335958in}}%
\pgfpathclose%
\pgfusepath{stroke,fill}%
\end{pgfscope}%
\begin{pgfscope}%
\pgfpathrectangle{\pgfqpoint{4.985294in}{0.500000in}}{\pgfqpoint{1.764706in}{1.700000in}}%
\pgfusepath{clip}%
\pgfsetbuttcap%
\pgfsetroundjoin%
\definecolor{currentfill}{rgb}{0.975644,0.874038,0.797253}%
\pgfsetfillcolor{currentfill}%
\pgfsetlinewidth{0.311001pt}%
\definecolor{currentstroke}{rgb}{1.000000,1.000000,1.000000}%
\pgfsetstrokecolor{currentstroke}%
\pgfsetdash{}{0pt}%
\pgfpathmoveto{\pgfqpoint{6.349275in}{1.420637in}}%
\pgfpathcurveto{\pgfqpoint{6.356408in}{1.420637in}}{\pgfqpoint{6.363250in}{1.423471in}}{\pgfqpoint{6.368293in}{1.428515in}}%
\pgfpathcurveto{\pgfqpoint{6.373337in}{1.433558in}}{\pgfqpoint{6.376171in}{1.440400in}}{\pgfqpoint{6.376171in}{1.447533in}}%
\pgfpathcurveto{\pgfqpoint{6.376171in}{1.454665in}}{\pgfqpoint{6.373337in}{1.461507in}}{\pgfqpoint{6.368293in}{1.466551in}}%
\pgfpathcurveto{\pgfqpoint{6.363250in}{1.471594in}}{\pgfqpoint{6.356408in}{1.474428in}}{\pgfqpoint{6.349275in}{1.474428in}}%
\pgfpathcurveto{\pgfqpoint{6.342143in}{1.474428in}}{\pgfqpoint{6.335301in}{1.471594in}}{\pgfqpoint{6.330257in}{1.466551in}}%
\pgfpathcurveto{\pgfqpoint{6.325214in}{1.461507in}}{\pgfqpoint{6.322380in}{1.454665in}}{\pgfqpoint{6.322380in}{1.447533in}}%
\pgfpathcurveto{\pgfqpoint{6.322380in}{1.440400in}}{\pgfqpoint{6.325214in}{1.433558in}}{\pgfqpoint{6.330257in}{1.428515in}}%
\pgfpathcurveto{\pgfqpoint{6.335301in}{1.423471in}}{\pgfqpoint{6.342143in}{1.420637in}}{\pgfqpoint{6.349275in}{1.420637in}}%
\pgfpathclose%
\pgfusepath{stroke,fill}%
\end{pgfscope}%
\begin{pgfscope}%
\pgfpathrectangle{\pgfqpoint{4.985294in}{0.500000in}}{\pgfqpoint{1.764706in}{1.700000in}}%
\pgfusepath{clip}%
\pgfsetbuttcap%
\pgfsetroundjoin%
\definecolor{currentfill}{rgb}{0.959229,0.533075,0.374889}%
\pgfsetfillcolor{currentfill}%
\pgfsetlinewidth{0.311001pt}%
\definecolor{currentstroke}{rgb}{1.000000,1.000000,1.000000}%
\pgfsetstrokecolor{currentstroke}%
\pgfsetdash{}{0pt}%
\pgfpathmoveto{\pgfqpoint{6.416531in}{1.494321in}}%
\pgfpathcurveto{\pgfqpoint{6.423663in}{1.494321in}}{\pgfqpoint{6.430505in}{1.497155in}}{\pgfqpoint{6.435549in}{1.502198in}}%
\pgfpathcurveto{\pgfqpoint{6.440592in}{1.507242in}}{\pgfqpoint{6.443426in}{1.514084in}}{\pgfqpoint{6.443426in}{1.521217in}}%
\pgfpathcurveto{\pgfqpoint{6.443426in}{1.528349in}}{\pgfqpoint{6.440592in}{1.535191in}}{\pgfqpoint{6.435549in}{1.540235in}}%
\pgfpathcurveto{\pgfqpoint{6.430505in}{1.545278in}}{\pgfqpoint{6.423663in}{1.548112in}}{\pgfqpoint{6.416531in}{1.548112in}}%
\pgfpathcurveto{\pgfqpoint{6.409398in}{1.548112in}}{\pgfqpoint{6.402556in}{1.545278in}}{\pgfqpoint{6.397512in}{1.540235in}}%
\pgfpathcurveto{\pgfqpoint{6.392469in}{1.535191in}}{\pgfqpoint{6.389635in}{1.528349in}}{\pgfqpoint{6.389635in}{1.521217in}}%
\pgfpathcurveto{\pgfqpoint{6.389635in}{1.514084in}}{\pgfqpoint{6.392469in}{1.507242in}}{\pgfqpoint{6.397512in}{1.502198in}}%
\pgfpathcurveto{\pgfqpoint{6.402556in}{1.497155in}}{\pgfqpoint{6.409398in}{1.494321in}}{\pgfqpoint{6.416531in}{1.494321in}}%
\pgfpathclose%
\pgfusepath{stroke,fill}%
\end{pgfscope}%
\begin{pgfscope}%
\pgfpathrectangle{\pgfqpoint{4.985294in}{0.500000in}}{\pgfqpoint{1.764706in}{1.700000in}}%
\pgfusepath{clip}%
\pgfsetbuttcap%
\pgfsetroundjoin%
\definecolor{currentfill}{rgb}{0.966120,0.744512,0.608720}%
\pgfsetfillcolor{currentfill}%
\pgfsetlinewidth{0.311001pt}%
\definecolor{currentstroke}{rgb}{1.000000,1.000000,1.000000}%
\pgfsetstrokecolor{currentstroke}%
\pgfsetdash{}{0pt}%
\pgfpathmoveto{\pgfqpoint{5.571620in}{1.600180in}}%
\pgfpathcurveto{\pgfqpoint{5.578752in}{1.600180in}}{\pgfqpoint{5.585594in}{1.603014in}}{\pgfqpoint{5.590638in}{1.608058in}}%
\pgfpathcurveto{\pgfqpoint{5.595681in}{1.613102in}}{\pgfqpoint{5.598515in}{1.619943in}}{\pgfqpoint{5.598515in}{1.627076in}}%
\pgfpathcurveto{\pgfqpoint{5.598515in}{1.634209in}}{\pgfqpoint{5.595681in}{1.641050in}}{\pgfqpoint{5.590638in}{1.646094in}}%
\pgfpathcurveto{\pgfqpoint{5.585594in}{1.651138in}}{\pgfqpoint{5.578752in}{1.653972in}}{\pgfqpoint{5.571620in}{1.653972in}}%
\pgfpathcurveto{\pgfqpoint{5.564487in}{1.653972in}}{\pgfqpoint{5.557645in}{1.651138in}}{\pgfqpoint{5.552601in}{1.646094in}}%
\pgfpathcurveto{\pgfqpoint{5.547558in}{1.641050in}}{\pgfqpoint{5.544724in}{1.634209in}}{\pgfqpoint{5.544724in}{1.627076in}}%
\pgfpathcurveto{\pgfqpoint{5.544724in}{1.619943in}}{\pgfqpoint{5.547558in}{1.613102in}}{\pgfqpoint{5.552601in}{1.608058in}}%
\pgfpathcurveto{\pgfqpoint{5.557645in}{1.603014in}}{\pgfqpoint{5.564487in}{1.600180in}}{\pgfqpoint{5.571620in}{1.600180in}}%
\pgfpathclose%
\pgfusepath{stroke,fill}%
\end{pgfscope}%
\begin{pgfscope}%
\pgfpathrectangle{\pgfqpoint{4.985294in}{0.500000in}}{\pgfqpoint{1.764706in}{1.700000in}}%
\pgfusepath{clip}%
\pgfsetbuttcap%
\pgfsetroundjoin%
\definecolor{currentfill}{rgb}{0.966812,0.762584,0.633643}%
\pgfsetfillcolor{currentfill}%
\pgfsetlinewidth{0.311001pt}%
\definecolor{currentstroke}{rgb}{1.000000,1.000000,1.000000}%
\pgfsetstrokecolor{currentstroke}%
\pgfsetdash{}{0pt}%
\pgfpathmoveto{\pgfqpoint{5.470802in}{0.953445in}}%
\pgfpathcurveto{\pgfqpoint{5.477934in}{0.953445in}}{\pgfqpoint{5.484776in}{0.956278in}}{\pgfqpoint{5.489820in}{0.961322in}}%
\pgfpathcurveto{\pgfqpoint{5.494863in}{0.966366in}}{\pgfqpoint{5.497697in}{0.973207in}}{\pgfqpoint{5.497697in}{0.980340in}}%
\pgfpathcurveto{\pgfqpoint{5.497697in}{0.987473in}}{\pgfqpoint{5.494863in}{0.994315in}}{\pgfqpoint{5.489820in}{0.999358in}}%
\pgfpathcurveto{\pgfqpoint{5.484776in}{1.004402in}}{\pgfqpoint{5.477934in}{1.007236in}}{\pgfqpoint{5.470802in}{1.007236in}}%
\pgfpathcurveto{\pgfqpoint{5.463669in}{1.007236in}}{\pgfqpoint{5.456827in}{1.004402in}}{\pgfqpoint{5.451784in}{0.999358in}}%
\pgfpathcurveto{\pgfqpoint{5.446740in}{0.994315in}}{\pgfqpoint{5.443906in}{0.987473in}}{\pgfqpoint{5.443906in}{0.980340in}}%
\pgfpathcurveto{\pgfqpoint{5.443906in}{0.973207in}}{\pgfqpoint{5.446740in}{0.966366in}}{\pgfqpoint{5.451784in}{0.961322in}}%
\pgfpathcurveto{\pgfqpoint{5.456827in}{0.956278in}}{\pgfqpoint{5.463669in}{0.953445in}}{\pgfqpoint{5.470802in}{0.953445in}}%
\pgfpathclose%
\pgfusepath{stroke,fill}%
\end{pgfscope}%
\begin{pgfscope}%
\pgfpathrectangle{\pgfqpoint{4.985294in}{0.500000in}}{\pgfqpoint{1.764706in}{1.700000in}}%
\pgfusepath{clip}%
\pgfsetbuttcap%
\pgfsetroundjoin%
\definecolor{currentfill}{rgb}{0.975644,0.874038,0.797253}%
\pgfsetfillcolor{currentfill}%
\pgfsetlinewidth{0.311001pt}%
\definecolor{currentstroke}{rgb}{1.000000,1.000000,1.000000}%
\pgfsetstrokecolor{currentstroke}%
\pgfsetdash{}{0pt}%
\pgfpathmoveto{\pgfqpoint{6.265476in}{1.473108in}}%
\pgfpathcurveto{\pgfqpoint{6.272608in}{1.473108in}}{\pgfqpoint{6.279450in}{1.475942in}}{\pgfqpoint{6.284494in}{1.480985in}}%
\pgfpathcurveto{\pgfqpoint{6.289537in}{1.486029in}}{\pgfqpoint{6.292371in}{1.492871in}}{\pgfqpoint{6.292371in}{1.500003in}}%
\pgfpathcurveto{\pgfqpoint{6.292371in}{1.507136in}}{\pgfqpoint{6.289537in}{1.513978in}}{\pgfqpoint{6.284494in}{1.519022in}}%
\pgfpathcurveto{\pgfqpoint{6.279450in}{1.524065in}}{\pgfqpoint{6.272608in}{1.526899in}}{\pgfqpoint{6.265476in}{1.526899in}}%
\pgfpathcurveto{\pgfqpoint{6.258343in}{1.526899in}}{\pgfqpoint{6.251501in}{1.524065in}}{\pgfqpoint{6.246457in}{1.519022in}}%
\pgfpathcurveto{\pgfqpoint{6.241414in}{1.513978in}}{\pgfqpoint{6.238580in}{1.507136in}}{\pgfqpoint{6.238580in}{1.500003in}}%
\pgfpathcurveto{\pgfqpoint{6.238580in}{1.492871in}}{\pgfqpoint{6.241414in}{1.486029in}}{\pgfqpoint{6.246457in}{1.480985in}}%
\pgfpathcurveto{\pgfqpoint{6.251501in}{1.475942in}}{\pgfqpoint{6.258343in}{1.473108in}}{\pgfqpoint{6.265476in}{1.473108in}}%
\pgfpathclose%
\pgfusepath{stroke,fill}%
\end{pgfscope}%
\begin{pgfscope}%
\pgfpathrectangle{\pgfqpoint{4.985294in}{0.500000in}}{\pgfqpoint{1.764706in}{1.700000in}}%
\pgfusepath{clip}%
\pgfsetbuttcap%
\pgfsetroundjoin%
\definecolor{currentfill}{rgb}{0.978376,0.897317,0.831308}%
\pgfsetfillcolor{currentfill}%
\pgfsetlinewidth{0.311001pt}%
\definecolor{currentstroke}{rgb}{1.000000,1.000000,1.000000}%
\pgfsetstrokecolor{currentstroke}%
\pgfsetdash{}{0pt}%
\pgfpathmoveto{\pgfqpoint{6.336360in}{1.416217in}}%
\pgfpathcurveto{\pgfqpoint{6.343493in}{1.416217in}}{\pgfqpoint{6.350334in}{1.419051in}}{\pgfqpoint{6.355378in}{1.424094in}}%
\pgfpathcurveto{\pgfqpoint{6.360422in}{1.429138in}}{\pgfqpoint{6.363256in}{1.435980in}}{\pgfqpoint{6.363256in}{1.443112in}}%
\pgfpathcurveto{\pgfqpoint{6.363256in}{1.450245in}}{\pgfqpoint{6.360422in}{1.457087in}}{\pgfqpoint{6.355378in}{1.462131in}}%
\pgfpathcurveto{\pgfqpoint{6.350334in}{1.467174in}}{\pgfqpoint{6.343493in}{1.470008in}}{\pgfqpoint{6.336360in}{1.470008in}}%
\pgfpathcurveto{\pgfqpoint{6.329227in}{1.470008in}}{\pgfqpoint{6.322385in}{1.467174in}}{\pgfqpoint{6.317342in}{1.462131in}}%
\pgfpathcurveto{\pgfqpoint{6.312298in}{1.457087in}}{\pgfqpoint{6.309464in}{1.450245in}}{\pgfqpoint{6.309464in}{1.443112in}}%
\pgfpathcurveto{\pgfqpoint{6.309464in}{1.435980in}}{\pgfqpoint{6.312298in}{1.429138in}}{\pgfqpoint{6.317342in}{1.424094in}}%
\pgfpathcurveto{\pgfqpoint{6.322385in}{1.419051in}}{\pgfqpoint{6.329227in}{1.416217in}}{\pgfqpoint{6.336360in}{1.416217in}}%
\pgfpathclose%
\pgfusepath{stroke,fill}%
\end{pgfscope}%
\begin{pgfscope}%
\pgfpathrectangle{\pgfqpoint{4.985294in}{0.500000in}}{\pgfqpoint{1.764706in}{1.700000in}}%
\pgfusepath{clip}%
\pgfsetbuttcap%
\pgfsetroundjoin%
\definecolor{currentfill}{rgb}{0.974412,0.862387,0.780156}%
\pgfsetfillcolor{currentfill}%
\pgfsetlinewidth{0.311001pt}%
\definecolor{currentstroke}{rgb}{1.000000,1.000000,1.000000}%
\pgfsetstrokecolor{currentstroke}%
\pgfsetdash{}{0pt}%
\pgfpathmoveto{\pgfqpoint{5.493139in}{1.513955in}}%
\pgfpathcurveto{\pgfqpoint{5.500272in}{1.513955in}}{\pgfqpoint{5.507114in}{1.516788in}}{\pgfqpoint{5.512157in}{1.521832in}}%
\pgfpathcurveto{\pgfqpoint{5.517201in}{1.526876in}}{\pgfqpoint{5.520035in}{1.533717in}}{\pgfqpoint{5.520035in}{1.540850in}}%
\pgfpathcurveto{\pgfqpoint{5.520035in}{1.547983in}}{\pgfqpoint{5.517201in}{1.554825in}}{\pgfqpoint{5.512157in}{1.559868in}}%
\pgfpathcurveto{\pgfqpoint{5.507114in}{1.564912in}}{\pgfqpoint{5.500272in}{1.567746in}}{\pgfqpoint{5.493139in}{1.567746in}}%
\pgfpathcurveto{\pgfqpoint{5.486006in}{1.567746in}}{\pgfqpoint{5.479165in}{1.564912in}}{\pgfqpoint{5.474121in}{1.559868in}}%
\pgfpathcurveto{\pgfqpoint{5.469077in}{1.554825in}}{\pgfqpoint{5.466244in}{1.547983in}}{\pgfqpoint{5.466244in}{1.540850in}}%
\pgfpathcurveto{\pgfqpoint{5.466244in}{1.533717in}}{\pgfqpoint{5.469077in}{1.526876in}}{\pgfqpoint{5.474121in}{1.521832in}}%
\pgfpathcurveto{\pgfqpoint{5.479165in}{1.516788in}}{\pgfqpoint{5.486006in}{1.513955in}}{\pgfqpoint{5.493139in}{1.513955in}}%
\pgfpathclose%
\pgfusepath{stroke,fill}%
\end{pgfscope}%
\begin{pgfscope}%
\pgfpathrectangle{\pgfqpoint{4.985294in}{0.500000in}}{\pgfqpoint{1.764706in}{1.700000in}}%
\pgfusepath{clip}%
\pgfsetbuttcap%
\pgfsetroundjoin%
\definecolor{currentfill}{rgb}{0.964433,0.670254,0.515093}%
\pgfsetfillcolor{currentfill}%
\pgfsetlinewidth{0.311001pt}%
\definecolor{currentstroke}{rgb}{1.000000,1.000000,1.000000}%
\pgfsetstrokecolor{currentstroke}%
\pgfsetdash{}{0pt}%
\pgfpathmoveto{\pgfqpoint{6.358459in}{1.583327in}}%
\pgfpathcurveto{\pgfqpoint{6.365592in}{1.583327in}}{\pgfqpoint{6.372433in}{1.586161in}}{\pgfqpoint{6.377477in}{1.591205in}}%
\pgfpathcurveto{\pgfqpoint{6.382521in}{1.596248in}}{\pgfqpoint{6.385354in}{1.603090in}}{\pgfqpoint{6.385354in}{1.610223in}}%
\pgfpathcurveto{\pgfqpoint{6.385354in}{1.617356in}}{\pgfqpoint{6.382521in}{1.624197in}}{\pgfqpoint{6.377477in}{1.629241in}}%
\pgfpathcurveto{\pgfqpoint{6.372433in}{1.634285in}}{\pgfqpoint{6.365592in}{1.637118in}}{\pgfqpoint{6.358459in}{1.637118in}}%
\pgfpathcurveto{\pgfqpoint{6.351326in}{1.637118in}}{\pgfqpoint{6.344484in}{1.634285in}}{\pgfqpoint{6.339441in}{1.629241in}}%
\pgfpathcurveto{\pgfqpoint{6.334397in}{1.624197in}}{\pgfqpoint{6.331563in}{1.617356in}}{\pgfqpoint{6.331563in}{1.610223in}}%
\pgfpathcurveto{\pgfqpoint{6.331563in}{1.603090in}}{\pgfqpoint{6.334397in}{1.596248in}}{\pgfqpoint{6.339441in}{1.591205in}}%
\pgfpathcurveto{\pgfqpoint{6.344484in}{1.586161in}}{\pgfqpoint{6.351326in}{1.583327in}}{\pgfqpoint{6.358459in}{1.583327in}}%
\pgfpathclose%
\pgfusepath{stroke,fill}%
\end{pgfscope}%
\begin{pgfscope}%
\pgfpathrectangle{\pgfqpoint{4.985294in}{0.500000in}}{\pgfqpoint{1.764706in}{1.700000in}}%
\pgfusepath{clip}%
\pgfsetbuttcap%
\pgfsetroundjoin%
\definecolor{currentfill}{rgb}{0.968105,0.786346,0.667739}%
\pgfsetfillcolor{currentfill}%
\pgfsetlinewidth{0.311001pt}%
\definecolor{currentstroke}{rgb}{1.000000,1.000000,1.000000}%
\pgfsetstrokecolor{currentstroke}%
\pgfsetdash{}{0pt}%
\pgfpathmoveto{\pgfqpoint{5.525704in}{1.664115in}}%
\pgfpathcurveto{\pgfqpoint{5.532837in}{1.664115in}}{\pgfqpoint{5.539679in}{1.666948in}}{\pgfqpoint{5.544723in}{1.671992in}}%
\pgfpathcurveto{\pgfqpoint{5.549766in}{1.677036in}}{\pgfqpoint{5.552600in}{1.683877in}}{\pgfqpoint{5.552600in}{1.691010in}}%
\pgfpathcurveto{\pgfqpoint{5.552600in}{1.698143in}}{\pgfqpoint{5.549766in}{1.704985in}}{\pgfqpoint{5.544723in}{1.710028in}}%
\pgfpathcurveto{\pgfqpoint{5.539679in}{1.715072in}}{\pgfqpoint{5.532837in}{1.717906in}}{\pgfqpoint{5.525704in}{1.717906in}}%
\pgfpathcurveto{\pgfqpoint{5.518572in}{1.717906in}}{\pgfqpoint{5.511730in}{1.715072in}}{\pgfqpoint{5.506686in}{1.710028in}}%
\pgfpathcurveto{\pgfqpoint{5.501643in}{1.704985in}}{\pgfqpoint{5.498809in}{1.698143in}}{\pgfqpoint{5.498809in}{1.691010in}}%
\pgfpathcurveto{\pgfqpoint{5.498809in}{1.683877in}}{\pgfqpoint{5.501643in}{1.677036in}}{\pgfqpoint{5.506686in}{1.671992in}}%
\pgfpathcurveto{\pgfqpoint{5.511730in}{1.666948in}}{\pgfqpoint{5.518572in}{1.664115in}}{\pgfqpoint{5.525704in}{1.664115in}}%
\pgfpathclose%
\pgfusepath{stroke,fill}%
\end{pgfscope}%
\begin{pgfscope}%
\pgfpathrectangle{\pgfqpoint{4.985294in}{0.500000in}}{\pgfqpoint{1.764706in}{1.700000in}}%
\pgfusepath{clip}%
\pgfsetbuttcap%
\pgfsetroundjoin%
\definecolor{currentfill}{rgb}{0.969803,0.809811,0.702523}%
\pgfsetfillcolor{currentfill}%
\pgfsetlinewidth{0.311001pt}%
\definecolor{currentstroke}{rgb}{1.000000,1.000000,1.000000}%
\pgfsetstrokecolor{currentstroke}%
\pgfsetdash{}{0pt}%
\pgfpathmoveto{\pgfqpoint{6.186751in}{1.087801in}}%
\pgfpathcurveto{\pgfqpoint{6.193884in}{1.087801in}}{\pgfqpoint{6.200726in}{1.090635in}}{\pgfqpoint{6.205770in}{1.095678in}}%
\pgfpathcurveto{\pgfqpoint{6.210813in}{1.100722in}}{\pgfqpoint{6.213647in}{1.107564in}}{\pgfqpoint{6.213647in}{1.114696in}}%
\pgfpathcurveto{\pgfqpoint{6.213647in}{1.121829in}}{\pgfqpoint{6.210813in}{1.128671in}}{\pgfqpoint{6.205770in}{1.133715in}}%
\pgfpathcurveto{\pgfqpoint{6.200726in}{1.138758in}}{\pgfqpoint{6.193884in}{1.141592in}}{\pgfqpoint{6.186751in}{1.141592in}}%
\pgfpathcurveto{\pgfqpoint{6.179619in}{1.141592in}}{\pgfqpoint{6.172777in}{1.138758in}}{\pgfqpoint{6.167733in}{1.133715in}}%
\pgfpathcurveto{\pgfqpoint{6.162690in}{1.128671in}}{\pgfqpoint{6.159856in}{1.121829in}}{\pgfqpoint{6.159856in}{1.114696in}}%
\pgfpathcurveto{\pgfqpoint{6.159856in}{1.107564in}}{\pgfqpoint{6.162690in}{1.100722in}}{\pgfqpoint{6.167733in}{1.095678in}}%
\pgfpathcurveto{\pgfqpoint{6.172777in}{1.090635in}}{\pgfqpoint{6.179619in}{1.087801in}}{\pgfqpoint{6.186751in}{1.087801in}}%
\pgfpathclose%
\pgfusepath{stroke,fill}%
\end{pgfscope}%
\begin{pgfscope}%
\pgfpathrectangle{\pgfqpoint{4.985294in}{0.500000in}}{\pgfqpoint{1.764706in}{1.700000in}}%
\pgfusepath{clip}%
\pgfsetbuttcap%
\pgfsetroundjoin%
\definecolor{currentfill}{rgb}{0.966812,0.762584,0.633643}%
\pgfsetfillcolor{currentfill}%
\pgfsetlinewidth{0.311001pt}%
\definecolor{currentstroke}{rgb}{1.000000,1.000000,1.000000}%
\pgfsetstrokecolor{currentstroke}%
\pgfsetdash{}{0pt}%
\pgfpathmoveto{\pgfqpoint{5.539863in}{1.676056in}}%
\pgfpathcurveto{\pgfqpoint{5.546995in}{1.676056in}}{\pgfqpoint{5.553837in}{1.678890in}}{\pgfqpoint{5.558881in}{1.683934in}}%
\pgfpathcurveto{\pgfqpoint{5.563924in}{1.688977in}}{\pgfqpoint{5.566758in}{1.695819in}}{\pgfqpoint{5.566758in}{1.702952in}}%
\pgfpathcurveto{\pgfqpoint{5.566758in}{1.710085in}}{\pgfqpoint{5.563924in}{1.716926in}}{\pgfqpoint{5.558881in}{1.721970in}}%
\pgfpathcurveto{\pgfqpoint{5.553837in}{1.727014in}}{\pgfqpoint{5.546995in}{1.729847in}}{\pgfqpoint{5.539863in}{1.729847in}}%
\pgfpathcurveto{\pgfqpoint{5.532730in}{1.729847in}}{\pgfqpoint{5.525888in}{1.727014in}}{\pgfqpoint{5.520844in}{1.721970in}}%
\pgfpathcurveto{\pgfqpoint{5.515801in}{1.716926in}}{\pgfqpoint{5.512967in}{1.710085in}}{\pgfqpoint{5.512967in}{1.702952in}}%
\pgfpathcurveto{\pgfqpoint{5.512967in}{1.695819in}}{\pgfqpoint{5.515801in}{1.688977in}}{\pgfqpoint{5.520844in}{1.683934in}}%
\pgfpathcurveto{\pgfqpoint{5.525888in}{1.678890in}}{\pgfqpoint{5.532730in}{1.676056in}}{\pgfqpoint{5.539863in}{1.676056in}}%
\pgfpathclose%
\pgfusepath{stroke,fill}%
\end{pgfscope}%
\begin{pgfscope}%
\pgfpathrectangle{\pgfqpoint{4.985294in}{0.500000in}}{\pgfqpoint{1.764706in}{1.700000in}}%
\pgfusepath{clip}%
\pgfsetbuttcap%
\pgfsetroundjoin%
\definecolor{currentfill}{rgb}{0.953126,0.456614,0.312398}%
\pgfsetfillcolor{currentfill}%
\pgfsetlinewidth{0.311001pt}%
\definecolor{currentstroke}{rgb}{1.000000,1.000000,1.000000}%
\pgfsetstrokecolor{currentstroke}%
\pgfsetdash{}{0pt}%
\pgfpathmoveto{\pgfqpoint{6.177089in}{0.833864in}}%
\pgfpathcurveto{\pgfqpoint{6.184222in}{0.833864in}}{\pgfqpoint{6.191063in}{0.836698in}}{\pgfqpoint{6.196107in}{0.841742in}}%
\pgfpathcurveto{\pgfqpoint{6.201151in}{0.846786in}}{\pgfqpoint{6.203985in}{0.853627in}}{\pgfqpoint{6.203985in}{0.860760in}}%
\pgfpathcurveto{\pgfqpoint{6.203985in}{0.867893in}}{\pgfqpoint{6.201151in}{0.874735in}}{\pgfqpoint{6.196107in}{0.879778in}}%
\pgfpathcurveto{\pgfqpoint{6.191063in}{0.884822in}}{\pgfqpoint{6.184222in}{0.887656in}}{\pgfqpoint{6.177089in}{0.887656in}}%
\pgfpathcurveto{\pgfqpoint{6.169956in}{0.887656in}}{\pgfqpoint{6.163114in}{0.884822in}}{\pgfqpoint{6.158071in}{0.879778in}}%
\pgfpathcurveto{\pgfqpoint{6.153027in}{0.874735in}}{\pgfqpoint{6.150193in}{0.867893in}}{\pgfqpoint{6.150193in}{0.860760in}}%
\pgfpathcurveto{\pgfqpoint{6.150193in}{0.853627in}}{\pgfqpoint{6.153027in}{0.846786in}}{\pgfqpoint{6.158071in}{0.841742in}}%
\pgfpathcurveto{\pgfqpoint{6.163114in}{0.836698in}}{\pgfqpoint{6.169956in}{0.833864in}}{\pgfqpoint{6.177089in}{0.833864in}}%
\pgfpathclose%
\pgfusepath{stroke,fill}%
\end{pgfscope}%
\begin{pgfscope}%
\pgfpathrectangle{\pgfqpoint{4.985294in}{0.500000in}}{\pgfqpoint{1.764706in}{1.700000in}}%
\pgfusepath{clip}%
\pgfsetbuttcap%
\pgfsetroundjoin%
\definecolor{currentfill}{rgb}{0.968931,0.798091,0.685123}%
\pgfsetfillcolor{currentfill}%
\pgfsetlinewidth{0.311001pt}%
\definecolor{currentstroke}{rgb}{1.000000,1.000000,1.000000}%
\pgfsetstrokecolor{currentstroke}%
\pgfsetdash{}{0pt}%
\pgfpathmoveto{\pgfqpoint{5.457286in}{1.640228in}}%
\pgfpathcurveto{\pgfqpoint{5.464419in}{1.640228in}}{\pgfqpoint{5.471261in}{1.643062in}}{\pgfqpoint{5.476305in}{1.648106in}}%
\pgfpathcurveto{\pgfqpoint{5.481348in}{1.653149in}}{\pgfqpoint{5.484182in}{1.659991in}}{\pgfqpoint{5.484182in}{1.667124in}}%
\pgfpathcurveto{\pgfqpoint{5.484182in}{1.674257in}}{\pgfqpoint{5.481348in}{1.681098in}}{\pgfqpoint{5.476305in}{1.686142in}}%
\pgfpathcurveto{\pgfqpoint{5.471261in}{1.691186in}}{\pgfqpoint{5.464419in}{1.694020in}}{\pgfqpoint{5.457286in}{1.694020in}}%
\pgfpathcurveto{\pgfqpoint{5.450154in}{1.694020in}}{\pgfqpoint{5.443312in}{1.691186in}}{\pgfqpoint{5.438268in}{1.686142in}}%
\pgfpathcurveto{\pgfqpoint{5.433225in}{1.681098in}}{\pgfqpoint{5.430391in}{1.674257in}}{\pgfqpoint{5.430391in}{1.667124in}}%
\pgfpathcurveto{\pgfqpoint{5.430391in}{1.659991in}}{\pgfqpoint{5.433225in}{1.653149in}}{\pgfqpoint{5.438268in}{1.648106in}}%
\pgfpathcurveto{\pgfqpoint{5.443312in}{1.643062in}}{\pgfqpoint{5.450154in}{1.640228in}}{\pgfqpoint{5.457286in}{1.640228in}}%
\pgfpathclose%
\pgfusepath{stroke,fill}%
\end{pgfscope}%
\begin{pgfscope}%
\pgfpathrectangle{\pgfqpoint{4.985294in}{0.500000in}}{\pgfqpoint{1.764706in}{1.700000in}}%
\pgfusepath{clip}%
\pgfsetbuttcap%
\pgfsetroundjoin%
\definecolor{currentfill}{rgb}{0.965169,0.707764,0.560659}%
\pgfsetfillcolor{currentfill}%
\pgfsetlinewidth{0.311001pt}%
\definecolor{currentstroke}{rgb}{1.000000,1.000000,1.000000}%
\pgfsetstrokecolor{currentstroke}%
\pgfsetdash{}{0pt}%
\pgfpathmoveto{\pgfqpoint{5.323723in}{1.299244in}}%
\pgfpathcurveto{\pgfqpoint{5.330856in}{1.299244in}}{\pgfqpoint{5.337697in}{1.302078in}}{\pgfqpoint{5.342741in}{1.307122in}}%
\pgfpathcurveto{\pgfqpoint{5.347784in}{1.312166in}}{\pgfqpoint{5.350618in}{1.319007in}}{\pgfqpoint{5.350618in}{1.326140in}}%
\pgfpathcurveto{\pgfqpoint{5.350618in}{1.333273in}}{\pgfqpoint{5.347784in}{1.340115in}}{\pgfqpoint{5.342741in}{1.345158in}}%
\pgfpathcurveto{\pgfqpoint{5.337697in}{1.350202in}}{\pgfqpoint{5.330856in}{1.353036in}}{\pgfqpoint{5.323723in}{1.353036in}}%
\pgfpathcurveto{\pgfqpoint{5.316590in}{1.353036in}}{\pgfqpoint{5.309748in}{1.350202in}}{\pgfqpoint{5.304705in}{1.345158in}}%
\pgfpathcurveto{\pgfqpoint{5.299661in}{1.340115in}}{\pgfqpoint{5.296827in}{1.333273in}}{\pgfqpoint{5.296827in}{1.326140in}}%
\pgfpathcurveto{\pgfqpoint{5.296827in}{1.319007in}}{\pgfqpoint{5.299661in}{1.312166in}}{\pgfqpoint{5.304705in}{1.307122in}}%
\pgfpathcurveto{\pgfqpoint{5.309748in}{1.302078in}}{\pgfqpoint{5.316590in}{1.299244in}}{\pgfqpoint{5.323723in}{1.299244in}}%
\pgfpathclose%
\pgfusepath{stroke,fill}%
\end{pgfscope}%
\begin{pgfscope}%
\pgfpathrectangle{\pgfqpoint{4.985294in}{0.500000in}}{\pgfqpoint{1.764706in}{1.700000in}}%
\pgfusepath{clip}%
\pgfsetbuttcap%
\pgfsetroundjoin%
\definecolor{currentfill}{rgb}{0.961433,0.573272,0.412036}%
\pgfsetfillcolor{currentfill}%
\pgfsetlinewidth{0.311001pt}%
\definecolor{currentstroke}{rgb}{1.000000,1.000000,1.000000}%
\pgfsetstrokecolor{currentstroke}%
\pgfsetdash{}{0pt}%
\pgfpathmoveto{\pgfqpoint{5.659455in}{0.967652in}}%
\pgfpathcurveto{\pgfqpoint{5.666588in}{0.967652in}}{\pgfqpoint{5.673429in}{0.970486in}}{\pgfqpoint{5.678473in}{0.975530in}}%
\pgfpathcurveto{\pgfqpoint{5.683517in}{0.980573in}}{\pgfqpoint{5.686350in}{0.987415in}}{\pgfqpoint{5.686350in}{0.994548in}}%
\pgfpathcurveto{\pgfqpoint{5.686350in}{1.001681in}}{\pgfqpoint{5.683517in}{1.008522in}}{\pgfqpoint{5.678473in}{1.013566in}}%
\pgfpathcurveto{\pgfqpoint{5.673429in}{1.018610in}}{\pgfqpoint{5.666588in}{1.021443in}}{\pgfqpoint{5.659455in}{1.021443in}}%
\pgfpathcurveto{\pgfqpoint{5.652322in}{1.021443in}}{\pgfqpoint{5.645480in}{1.018610in}}{\pgfqpoint{5.640437in}{1.013566in}}%
\pgfpathcurveto{\pgfqpoint{5.635393in}{1.008522in}}{\pgfqpoint{5.632559in}{1.001681in}}{\pgfqpoint{5.632559in}{0.994548in}}%
\pgfpathcurveto{\pgfqpoint{5.632559in}{0.987415in}}{\pgfqpoint{5.635393in}{0.980573in}}{\pgfqpoint{5.640437in}{0.975530in}}%
\pgfpathcurveto{\pgfqpoint{5.645480in}{0.970486in}}{\pgfqpoint{5.652322in}{0.967652in}}{\pgfqpoint{5.659455in}{0.967652in}}%
\pgfpathclose%
\pgfusepath{stroke,fill}%
\end{pgfscope}%
\begin{pgfscope}%
\pgfpathrectangle{\pgfqpoint{4.985294in}{0.500000in}}{\pgfqpoint{1.764706in}{1.700000in}}%
\pgfusepath{clip}%
\pgfsetbuttcap%
\pgfsetroundjoin%
\definecolor{currentfill}{rgb}{0.973271,0.850724,0.762998}%
\pgfsetfillcolor{currentfill}%
\pgfsetlinewidth{0.311001pt}%
\definecolor{currentstroke}{rgb}{1.000000,1.000000,1.000000}%
\pgfsetstrokecolor{currentstroke}%
\pgfsetdash{}{0pt}%
\pgfpathmoveto{\pgfqpoint{6.190203in}{1.633008in}}%
\pgfpathcurveto{\pgfqpoint{6.197336in}{1.633008in}}{\pgfqpoint{6.204178in}{1.635842in}}{\pgfqpoint{6.209222in}{1.640886in}}%
\pgfpathcurveto{\pgfqpoint{6.214265in}{1.645930in}}{\pgfqpoint{6.217099in}{1.652771in}}{\pgfqpoint{6.217099in}{1.659904in}}%
\pgfpathcurveto{\pgfqpoint{6.217099in}{1.667037in}}{\pgfqpoint{6.214265in}{1.673879in}}{\pgfqpoint{6.209222in}{1.678922in}}%
\pgfpathcurveto{\pgfqpoint{6.204178in}{1.683966in}}{\pgfqpoint{6.197336in}{1.686800in}}{\pgfqpoint{6.190203in}{1.686800in}}%
\pgfpathcurveto{\pgfqpoint{6.183071in}{1.686800in}}{\pgfqpoint{6.176229in}{1.683966in}}{\pgfqpoint{6.171185in}{1.678922in}}%
\pgfpathcurveto{\pgfqpoint{6.166142in}{1.673879in}}{\pgfqpoint{6.163308in}{1.667037in}}{\pgfqpoint{6.163308in}{1.659904in}}%
\pgfpathcurveto{\pgfqpoint{6.163308in}{1.652771in}}{\pgfqpoint{6.166142in}{1.645930in}}{\pgfqpoint{6.171185in}{1.640886in}}%
\pgfpathcurveto{\pgfqpoint{6.176229in}{1.635842in}}{\pgfqpoint{6.183071in}{1.633008in}}{\pgfqpoint{6.190203in}{1.633008in}}%
\pgfpathclose%
\pgfusepath{stroke,fill}%
\end{pgfscope}%
\begin{pgfscope}%
\pgfpathrectangle{\pgfqpoint{4.985294in}{0.500000in}}{\pgfqpoint{1.764706in}{1.700000in}}%
\pgfusepath{clip}%
\pgfsetbuttcap%
\pgfsetroundjoin%
\definecolor{currentfill}{rgb}{0.980678,0.914765,0.856766}%
\pgfsetfillcolor{currentfill}%
\pgfsetlinewidth{0.311001pt}%
\definecolor{currentstroke}{rgb}{1.000000,1.000000,1.000000}%
\pgfsetstrokecolor{currentstroke}%
\pgfsetdash{}{0pt}%
\pgfpathmoveto{\pgfqpoint{5.424069in}{1.289750in}}%
\pgfpathcurveto{\pgfqpoint{5.431202in}{1.289750in}}{\pgfqpoint{5.438044in}{1.292584in}}{\pgfqpoint{5.443087in}{1.297628in}}%
\pgfpathcurveto{\pgfqpoint{5.448131in}{1.302671in}}{\pgfqpoint{5.450965in}{1.309513in}}{\pgfqpoint{5.450965in}{1.316646in}}%
\pgfpathcurveto{\pgfqpoint{5.450965in}{1.323779in}}{\pgfqpoint{5.448131in}{1.330620in}}{\pgfqpoint{5.443087in}{1.335664in}}%
\pgfpathcurveto{\pgfqpoint{5.438044in}{1.340708in}}{\pgfqpoint{5.431202in}{1.343542in}}{\pgfqpoint{5.424069in}{1.343542in}}%
\pgfpathcurveto{\pgfqpoint{5.416936in}{1.343542in}}{\pgfqpoint{5.410095in}{1.340708in}}{\pgfqpoint{5.405051in}{1.335664in}}%
\pgfpathcurveto{\pgfqpoint{5.400007in}{1.330620in}}{\pgfqpoint{5.397174in}{1.323779in}}{\pgfqpoint{5.397174in}{1.316646in}}%
\pgfpathcurveto{\pgfqpoint{5.397174in}{1.309513in}}{\pgfqpoint{5.400007in}{1.302671in}}{\pgfqpoint{5.405051in}{1.297628in}}%
\pgfpathcurveto{\pgfqpoint{5.410095in}{1.292584in}}{\pgfqpoint{5.416936in}{1.289750in}}{\pgfqpoint{5.424069in}{1.289750in}}%
\pgfpathclose%
\pgfusepath{stroke,fill}%
\end{pgfscope}%
\begin{pgfscope}%
\pgfpathrectangle{\pgfqpoint{4.985294in}{0.500000in}}{\pgfqpoint{1.764706in}{1.700000in}}%
\pgfusepath{clip}%
\pgfsetbuttcap%
\pgfsetroundjoin%
\definecolor{currentfill}{rgb}{0.972201,0.839051,0.745789}%
\pgfsetfillcolor{currentfill}%
\pgfsetlinewidth{0.311001pt}%
\definecolor{currentstroke}{rgb}{1.000000,1.000000,1.000000}%
\pgfsetstrokecolor{currentstroke}%
\pgfsetdash{}{0pt}%
\pgfpathmoveto{\pgfqpoint{6.212620in}{1.126099in}}%
\pgfpathcurveto{\pgfqpoint{6.219753in}{1.126099in}}{\pgfqpoint{6.226595in}{1.128933in}}{\pgfqpoint{6.231638in}{1.133977in}}%
\pgfpathcurveto{\pgfqpoint{6.236682in}{1.139020in}}{\pgfqpoint{6.239516in}{1.145862in}}{\pgfqpoint{6.239516in}{1.152995in}}%
\pgfpathcurveto{\pgfqpoint{6.239516in}{1.160127in}}{\pgfqpoint{6.236682in}{1.166969in}}{\pgfqpoint{6.231638in}{1.172013in}}%
\pgfpathcurveto{\pgfqpoint{6.226595in}{1.177056in}}{\pgfqpoint{6.219753in}{1.179890in}}{\pgfqpoint{6.212620in}{1.179890in}}%
\pgfpathcurveto{\pgfqpoint{6.205487in}{1.179890in}}{\pgfqpoint{6.198646in}{1.177056in}}{\pgfqpoint{6.193602in}{1.172013in}}%
\pgfpathcurveto{\pgfqpoint{6.188558in}{1.166969in}}{\pgfqpoint{6.185724in}{1.160127in}}{\pgfqpoint{6.185724in}{1.152995in}}%
\pgfpathcurveto{\pgfqpoint{6.185724in}{1.145862in}}{\pgfqpoint{6.188558in}{1.139020in}}{\pgfqpoint{6.193602in}{1.133977in}}%
\pgfpathcurveto{\pgfqpoint{6.198646in}{1.128933in}}{\pgfqpoint{6.205487in}{1.126099in}}{\pgfqpoint{6.212620in}{1.126099in}}%
\pgfpathclose%
\pgfusepath{stroke,fill}%
\end{pgfscope}%
\begin{pgfscope}%
\pgfpathrectangle{\pgfqpoint{4.985294in}{0.500000in}}{\pgfqpoint{1.764706in}{1.700000in}}%
\pgfusepath{clip}%
\pgfsetbuttcap%
\pgfsetroundjoin%
\definecolor{currentfill}{rgb}{0.977657,0.891500,0.822809}%
\pgfsetfillcolor{currentfill}%
\pgfsetlinewidth{0.311001pt}%
\definecolor{currentstroke}{rgb}{1.000000,1.000000,1.000000}%
\pgfsetstrokecolor{currentstroke}%
\pgfsetdash{}{0pt}%
\pgfpathmoveto{\pgfqpoint{5.384439in}{1.351328in}}%
\pgfpathcurveto{\pgfqpoint{5.391572in}{1.351328in}}{\pgfqpoint{5.398413in}{1.354162in}}{\pgfqpoint{5.403457in}{1.359206in}}%
\pgfpathcurveto{\pgfqpoint{5.408501in}{1.364249in}}{\pgfqpoint{5.411334in}{1.371091in}}{\pgfqpoint{5.411334in}{1.378224in}}%
\pgfpathcurveto{\pgfqpoint{5.411334in}{1.385357in}}{\pgfqpoint{5.408501in}{1.392198in}}{\pgfqpoint{5.403457in}{1.397242in}}%
\pgfpathcurveto{\pgfqpoint{5.398413in}{1.402286in}}{\pgfqpoint{5.391572in}{1.405119in}}{\pgfqpoint{5.384439in}{1.405119in}}%
\pgfpathcurveto{\pgfqpoint{5.377306in}{1.405119in}}{\pgfqpoint{5.370464in}{1.402286in}}{\pgfqpoint{5.365421in}{1.397242in}}%
\pgfpathcurveto{\pgfqpoint{5.360377in}{1.392198in}}{\pgfqpoint{5.357543in}{1.385357in}}{\pgfqpoint{5.357543in}{1.378224in}}%
\pgfpathcurveto{\pgfqpoint{5.357543in}{1.371091in}}{\pgfqpoint{5.360377in}{1.364249in}}{\pgfqpoint{5.365421in}{1.359206in}}%
\pgfpathcurveto{\pgfqpoint{5.370464in}{1.354162in}}{\pgfqpoint{5.377306in}{1.351328in}}{\pgfqpoint{5.384439in}{1.351328in}}%
\pgfpathclose%
\pgfusepath{stroke,fill}%
\end{pgfscope}%
\begin{pgfscope}%
\pgfpathrectangle{\pgfqpoint{4.985294in}{0.500000in}}{\pgfqpoint{1.764706in}{1.700000in}}%
\pgfusepath{clip}%
\pgfsetbuttcap%
\pgfsetroundjoin%
\definecolor{currentfill}{rgb}{0.971202,0.827364,0.728520}%
\pgfsetfillcolor{currentfill}%
\pgfsetlinewidth{0.311001pt}%
\definecolor{currentstroke}{rgb}{1.000000,1.000000,1.000000}%
\pgfsetstrokecolor{currentstroke}%
\pgfsetdash{}{0pt}%
\pgfpathmoveto{\pgfqpoint{5.525787in}{1.599317in}}%
\pgfpathcurveto{\pgfqpoint{5.532920in}{1.599317in}}{\pgfqpoint{5.539762in}{1.602151in}}{\pgfqpoint{5.544805in}{1.607195in}}%
\pgfpathcurveto{\pgfqpoint{5.549849in}{1.612238in}}{\pgfqpoint{5.552683in}{1.619080in}}{\pgfqpoint{5.552683in}{1.626213in}}%
\pgfpathcurveto{\pgfqpoint{5.552683in}{1.633346in}}{\pgfqpoint{5.549849in}{1.640187in}}{\pgfqpoint{5.544805in}{1.645231in}}%
\pgfpathcurveto{\pgfqpoint{5.539762in}{1.650274in}}{\pgfqpoint{5.532920in}{1.653108in}}{\pgfqpoint{5.525787in}{1.653108in}}%
\pgfpathcurveto{\pgfqpoint{5.518654in}{1.653108in}}{\pgfqpoint{5.511813in}{1.650274in}}{\pgfqpoint{5.506769in}{1.645231in}}%
\pgfpathcurveto{\pgfqpoint{5.501725in}{1.640187in}}{\pgfqpoint{5.498892in}{1.633346in}}{\pgfqpoint{5.498892in}{1.626213in}}%
\pgfpathcurveto{\pgfqpoint{5.498892in}{1.619080in}}{\pgfqpoint{5.501725in}{1.612238in}}{\pgfqpoint{5.506769in}{1.607195in}}%
\pgfpathcurveto{\pgfqpoint{5.511813in}{1.602151in}}{\pgfqpoint{5.518654in}{1.599317in}}{\pgfqpoint{5.525787in}{1.599317in}}%
\pgfpathclose%
\pgfusepath{stroke,fill}%
\end{pgfscope}%
\begin{pgfscope}%
\pgfpathrectangle{\pgfqpoint{4.985294in}{0.500000in}}{\pgfqpoint{1.764706in}{1.700000in}}%
\pgfusepath{clip}%
\pgfsetbuttcap%
\pgfsetroundjoin%
\definecolor{currentfill}{rgb}{0.963728,0.638439,0.479050}%
\pgfsetfillcolor{currentfill}%
\pgfsetlinewidth{0.311001pt}%
\definecolor{currentstroke}{rgb}{1.000000,1.000000,1.000000}%
\pgfsetstrokecolor{currentstroke}%
\pgfsetdash{}{0pt}%
\pgfpathmoveto{\pgfqpoint{6.335105in}{1.639051in}}%
\pgfpathcurveto{\pgfqpoint{6.342238in}{1.639051in}}{\pgfqpoint{6.349079in}{1.641885in}}{\pgfqpoint{6.354123in}{1.646928in}}%
\pgfpathcurveto{\pgfqpoint{6.359167in}{1.651972in}}{\pgfqpoint{6.362000in}{1.658814in}}{\pgfqpoint{6.362000in}{1.665946in}}%
\pgfpathcurveto{\pgfqpoint{6.362000in}{1.673079in}}{\pgfqpoint{6.359167in}{1.679921in}}{\pgfqpoint{6.354123in}{1.684965in}}%
\pgfpathcurveto{\pgfqpoint{6.349079in}{1.690008in}}{\pgfqpoint{6.342238in}{1.692842in}}{\pgfqpoint{6.335105in}{1.692842in}}%
\pgfpathcurveto{\pgfqpoint{6.327972in}{1.692842in}}{\pgfqpoint{6.321130in}{1.690008in}}{\pgfqpoint{6.316087in}{1.684965in}}%
\pgfpathcurveto{\pgfqpoint{6.311043in}{1.679921in}}{\pgfqpoint{6.308209in}{1.673079in}}{\pgfqpoint{6.308209in}{1.665946in}}%
\pgfpathcurveto{\pgfqpoint{6.308209in}{1.658814in}}{\pgfqpoint{6.311043in}{1.651972in}}{\pgfqpoint{6.316087in}{1.646928in}}%
\pgfpathcurveto{\pgfqpoint{6.321130in}{1.641885in}}{\pgfqpoint{6.327972in}{1.639051in}}{\pgfqpoint{6.335105in}{1.639051in}}%
\pgfpathclose%
\pgfusepath{stroke,fill}%
\end{pgfscope}%
\begin{pgfscope}%
\pgfpathrectangle{\pgfqpoint{4.985294in}{0.500000in}}{\pgfqpoint{1.764706in}{1.700000in}}%
\pgfusepath{clip}%
\pgfsetbuttcap%
\pgfsetroundjoin%
\definecolor{currentfill}{rgb}{0.961115,0.566634,0.405693}%
\pgfsetfillcolor{currentfill}%
\pgfsetlinewidth{0.311001pt}%
\definecolor{currentstroke}{rgb}{1.000000,1.000000,1.000000}%
\pgfsetstrokecolor{currentstroke}%
\pgfsetdash{}{0pt}%
\pgfpathmoveto{\pgfqpoint{5.402942in}{0.942456in}}%
\pgfpathcurveto{\pgfqpoint{5.410075in}{0.942456in}}{\pgfqpoint{5.416916in}{0.945290in}}{\pgfqpoint{5.421960in}{0.950334in}}%
\pgfpathcurveto{\pgfqpoint{5.427004in}{0.955378in}}{\pgfqpoint{5.429838in}{0.962219in}}{\pgfqpoint{5.429838in}{0.969352in}}%
\pgfpathcurveto{\pgfqpoint{5.429838in}{0.976485in}}{\pgfqpoint{5.427004in}{0.983327in}}{\pgfqpoint{5.421960in}{0.988370in}}%
\pgfpathcurveto{\pgfqpoint{5.416916in}{0.993414in}}{\pgfqpoint{5.410075in}{0.996248in}}{\pgfqpoint{5.402942in}{0.996248in}}%
\pgfpathcurveto{\pgfqpoint{5.395809in}{0.996248in}}{\pgfqpoint{5.388967in}{0.993414in}}{\pgfqpoint{5.383924in}{0.988370in}}%
\pgfpathcurveto{\pgfqpoint{5.378880in}{0.983327in}}{\pgfqpoint{5.376046in}{0.976485in}}{\pgfqpoint{5.376046in}{0.969352in}}%
\pgfpathcurveto{\pgfqpoint{5.376046in}{0.962219in}}{\pgfqpoint{5.378880in}{0.955378in}}{\pgfqpoint{5.383924in}{0.950334in}}%
\pgfpathcurveto{\pgfqpoint{5.388967in}{0.945290in}}{\pgfqpoint{5.395809in}{0.942456in}}{\pgfqpoint{5.402942in}{0.942456in}}%
\pgfpathclose%
\pgfusepath{stroke,fill}%
\end{pgfscope}%
\begin{pgfscope}%
\pgfpathrectangle{\pgfqpoint{4.985294in}{0.500000in}}{\pgfqpoint{1.764706in}{1.700000in}}%
\pgfusepath{clip}%
\pgfsetbuttcap%
\pgfsetroundjoin%
\definecolor{currentfill}{rgb}{0.967398,0.774513,0.650573}%
\pgfsetfillcolor{currentfill}%
\pgfsetlinewidth{0.311001pt}%
\definecolor{currentstroke}{rgb}{1.000000,1.000000,1.000000}%
\pgfsetstrokecolor{currentstroke}%
\pgfsetdash{}{0pt}%
\pgfpathmoveto{\pgfqpoint{5.527617in}{1.132941in}}%
\pgfpathcurveto{\pgfqpoint{5.534750in}{1.132941in}}{\pgfqpoint{5.541592in}{1.135775in}}{\pgfqpoint{5.546636in}{1.140819in}}%
\pgfpathcurveto{\pgfqpoint{5.551679in}{1.145862in}}{\pgfqpoint{5.554513in}{1.152704in}}{\pgfqpoint{5.554513in}{1.159837in}}%
\pgfpathcurveto{\pgfqpoint{5.554513in}{1.166970in}}{\pgfqpoint{5.551679in}{1.173811in}}{\pgfqpoint{5.546636in}{1.178855in}}%
\pgfpathcurveto{\pgfqpoint{5.541592in}{1.183899in}}{\pgfqpoint{5.534750in}{1.186733in}}{\pgfqpoint{5.527617in}{1.186733in}}%
\pgfpathcurveto{\pgfqpoint{5.520485in}{1.186733in}}{\pgfqpoint{5.513643in}{1.183899in}}{\pgfqpoint{5.508599in}{1.178855in}}%
\pgfpathcurveto{\pgfqpoint{5.503556in}{1.173811in}}{\pgfqpoint{5.500722in}{1.166970in}}{\pgfqpoint{5.500722in}{1.159837in}}%
\pgfpathcurveto{\pgfqpoint{5.500722in}{1.152704in}}{\pgfqpoint{5.503556in}{1.145862in}}{\pgfqpoint{5.508599in}{1.140819in}}%
\pgfpathcurveto{\pgfqpoint{5.513643in}{1.135775in}}{\pgfqpoint{5.520485in}{1.132941in}}{\pgfqpoint{5.527617in}{1.132941in}}%
\pgfpathclose%
\pgfusepath{stroke,fill}%
\end{pgfscope}%
\begin{pgfscope}%
\pgfpathrectangle{\pgfqpoint{4.985294in}{0.500000in}}{\pgfqpoint{1.764706in}{1.700000in}}%
\pgfusepath{clip}%
\pgfsetbuttcap%
\pgfsetroundjoin%
\definecolor{currentfill}{rgb}{0.977657,0.891500,0.822809}%
\pgfsetfillcolor{currentfill}%
\pgfsetlinewidth{0.311001pt}%
\definecolor{currentstroke}{rgb}{1.000000,1.000000,1.000000}%
\pgfsetstrokecolor{currentstroke}%
\pgfsetdash{}{0pt}%
\pgfpathmoveto{\pgfqpoint{6.269485in}{1.510282in}}%
\pgfpathcurveto{\pgfqpoint{6.276618in}{1.510282in}}{\pgfqpoint{6.283460in}{1.513116in}}{\pgfqpoint{6.288503in}{1.518160in}}%
\pgfpathcurveto{\pgfqpoint{6.293547in}{1.523204in}}{\pgfqpoint{6.296381in}{1.530045in}}{\pgfqpoint{6.296381in}{1.537178in}}%
\pgfpathcurveto{\pgfqpoint{6.296381in}{1.544311in}}{\pgfqpoint{6.293547in}{1.551153in}}{\pgfqpoint{6.288503in}{1.556196in}}%
\pgfpathcurveto{\pgfqpoint{6.283460in}{1.561240in}}{\pgfqpoint{6.276618in}{1.564074in}}{\pgfqpoint{6.269485in}{1.564074in}}%
\pgfpathcurveto{\pgfqpoint{6.262352in}{1.564074in}}{\pgfqpoint{6.255511in}{1.561240in}}{\pgfqpoint{6.250467in}{1.556196in}}%
\pgfpathcurveto{\pgfqpoint{6.245423in}{1.551153in}}{\pgfqpoint{6.242589in}{1.544311in}}{\pgfqpoint{6.242589in}{1.537178in}}%
\pgfpathcurveto{\pgfqpoint{6.242589in}{1.530045in}}{\pgfqpoint{6.245423in}{1.523204in}}{\pgfqpoint{6.250467in}{1.518160in}}%
\pgfpathcurveto{\pgfqpoint{6.255511in}{1.513116in}}{\pgfqpoint{6.262352in}{1.510282in}}{\pgfqpoint{6.269485in}{1.510282in}}%
\pgfpathclose%
\pgfusepath{stroke,fill}%
\end{pgfscope}%
\begin{pgfscope}%
\pgfpathrectangle{\pgfqpoint{4.985294in}{0.500000in}}{\pgfqpoint{1.764706in}{1.700000in}}%
\pgfusepath{clip}%
\pgfsetbuttcap%
\pgfsetroundjoin%
\definecolor{currentfill}{rgb}{0.968105,0.786346,0.667739}%
\pgfsetfillcolor{currentfill}%
\pgfsetlinewidth{0.311001pt}%
\definecolor{currentstroke}{rgb}{1.000000,1.000000,1.000000}%
\pgfsetstrokecolor{currentstroke}%
\pgfsetdash{}{0pt}%
\pgfpathmoveto{\pgfqpoint{6.245099in}{1.679136in}}%
\pgfpathcurveto{\pgfqpoint{6.252232in}{1.679136in}}{\pgfqpoint{6.259074in}{1.681970in}}{\pgfqpoint{6.264117in}{1.687013in}}%
\pgfpathcurveto{\pgfqpoint{6.269161in}{1.692057in}}{\pgfqpoint{6.271995in}{1.698899in}}{\pgfqpoint{6.271995in}{1.706031in}}%
\pgfpathcurveto{\pgfqpoint{6.271995in}{1.713164in}}{\pgfqpoint{6.269161in}{1.720006in}}{\pgfqpoint{6.264117in}{1.725050in}}%
\pgfpathcurveto{\pgfqpoint{6.259074in}{1.730093in}}{\pgfqpoint{6.252232in}{1.732927in}}{\pgfqpoint{6.245099in}{1.732927in}}%
\pgfpathcurveto{\pgfqpoint{6.237966in}{1.732927in}}{\pgfqpoint{6.231125in}{1.730093in}}{\pgfqpoint{6.226081in}{1.725050in}}%
\pgfpathcurveto{\pgfqpoint{6.221037in}{1.720006in}}{\pgfqpoint{6.218203in}{1.713164in}}{\pgfqpoint{6.218203in}{1.706031in}}%
\pgfpathcurveto{\pgfqpoint{6.218203in}{1.698899in}}{\pgfqpoint{6.221037in}{1.692057in}}{\pgfqpoint{6.226081in}{1.687013in}}%
\pgfpathcurveto{\pgfqpoint{6.231125in}{1.681970in}}{\pgfqpoint{6.237966in}{1.679136in}}{\pgfqpoint{6.245099in}{1.679136in}}%
\pgfpathclose%
\pgfusepath{stroke,fill}%
\end{pgfscope}%
\begin{pgfscope}%
\pgfpathrectangle{\pgfqpoint{4.985294in}{0.500000in}}{\pgfqpoint{1.764706in}{1.700000in}}%
\pgfusepath{clip}%
\pgfsetbuttcap%
\pgfsetroundjoin%
\definecolor{currentfill}{rgb}{0.970255,0.815666,0.711203}%
\pgfsetfillcolor{currentfill}%
\pgfsetlinewidth{0.311001pt}%
\definecolor{currentstroke}{rgb}{1.000000,1.000000,1.000000}%
\pgfsetstrokecolor{currentstroke}%
\pgfsetdash{}{0pt}%
\pgfpathmoveto{\pgfqpoint{5.396949in}{1.103613in}}%
\pgfpathcurveto{\pgfqpoint{5.404081in}{1.103613in}}{\pgfqpoint{5.410923in}{1.106447in}}{\pgfqpoint{5.415967in}{1.111491in}}%
\pgfpathcurveto{\pgfqpoint{5.421010in}{1.116535in}}{\pgfqpoint{5.423844in}{1.123376in}}{\pgfqpoint{5.423844in}{1.130509in}}%
\pgfpathcurveto{\pgfqpoint{5.423844in}{1.137642in}}{\pgfqpoint{5.421010in}{1.144484in}}{\pgfqpoint{5.415967in}{1.149527in}}%
\pgfpathcurveto{\pgfqpoint{5.410923in}{1.154571in}}{\pgfqpoint{5.404081in}{1.157405in}}{\pgfqpoint{5.396949in}{1.157405in}}%
\pgfpathcurveto{\pgfqpoint{5.389816in}{1.157405in}}{\pgfqpoint{5.382974in}{1.154571in}}{\pgfqpoint{5.377931in}{1.149527in}}%
\pgfpathcurveto{\pgfqpoint{5.372887in}{1.144484in}}{\pgfqpoint{5.370053in}{1.137642in}}{\pgfqpoint{5.370053in}{1.130509in}}%
\pgfpathcurveto{\pgfqpoint{5.370053in}{1.123376in}}{\pgfqpoint{5.372887in}{1.116535in}}{\pgfqpoint{5.377931in}{1.111491in}}%
\pgfpathcurveto{\pgfqpoint{5.382974in}{1.106447in}}{\pgfqpoint{5.389816in}{1.103613in}}{\pgfqpoint{5.396949in}{1.103613in}}%
\pgfpathclose%
\pgfusepath{stroke,fill}%
\end{pgfscope}%
\begin{pgfscope}%
\pgfpathrectangle{\pgfqpoint{4.985294in}{0.500000in}}{\pgfqpoint{1.764706in}{1.700000in}}%
\pgfusepath{clip}%
\pgfsetbuttcap%
\pgfsetroundjoin%
\definecolor{currentfill}{rgb}{0.975018,0.868213,0.788710}%
\pgfsetfillcolor{currentfill}%
\pgfsetlinewidth{0.311001pt}%
\definecolor{currentstroke}{rgb}{1.000000,1.000000,1.000000}%
\pgfsetstrokecolor{currentstroke}%
\pgfsetdash{}{0pt}%
\pgfpathmoveto{\pgfqpoint{5.482331in}{1.070528in}}%
\pgfpathcurveto{\pgfqpoint{5.489464in}{1.070528in}}{\pgfqpoint{5.496305in}{1.073362in}}{\pgfqpoint{5.501349in}{1.078405in}}%
\pgfpathcurveto{\pgfqpoint{5.506393in}{1.083449in}}{\pgfqpoint{5.509227in}{1.090291in}}{\pgfqpoint{5.509227in}{1.097423in}}%
\pgfpathcurveto{\pgfqpoint{5.509227in}{1.104556in}}{\pgfqpoint{5.506393in}{1.111398in}}{\pgfqpoint{5.501349in}{1.116442in}}%
\pgfpathcurveto{\pgfqpoint{5.496305in}{1.121485in}}{\pgfqpoint{5.489464in}{1.124319in}}{\pgfqpoint{5.482331in}{1.124319in}}%
\pgfpathcurveto{\pgfqpoint{5.475198in}{1.124319in}}{\pgfqpoint{5.468356in}{1.121485in}}{\pgfqpoint{5.463313in}{1.116442in}}%
\pgfpathcurveto{\pgfqpoint{5.458269in}{1.111398in}}{\pgfqpoint{5.455435in}{1.104556in}}{\pgfqpoint{5.455435in}{1.097423in}}%
\pgfpathcurveto{\pgfqpoint{5.455435in}{1.090291in}}{\pgfqpoint{5.458269in}{1.083449in}}{\pgfqpoint{5.463313in}{1.078405in}}%
\pgfpathcurveto{\pgfqpoint{5.468356in}{1.073362in}}{\pgfqpoint{5.475198in}{1.070528in}}{\pgfqpoint{5.482331in}{1.070528in}}%
\pgfpathclose%
\pgfusepath{stroke,fill}%
\end{pgfscope}%
\begin{pgfscope}%
\pgfpathrectangle{\pgfqpoint{4.985294in}{0.500000in}}{\pgfqpoint{1.764706in}{1.700000in}}%
\pgfusepath{clip}%
\pgfsetbuttcap%
\pgfsetroundjoin%
\definecolor{currentfill}{rgb}{0.979891,0.908948,0.848279}%
\pgfsetfillcolor{currentfill}%
\pgfsetlinewidth{0.311001pt}%
\definecolor{currentstroke}{rgb}{1.000000,1.000000,1.000000}%
\pgfsetstrokecolor{currentstroke}%
\pgfsetdash{}{0pt}%
\pgfpathmoveto{\pgfqpoint{6.334165in}{1.271437in}}%
\pgfpathcurveto{\pgfqpoint{6.341298in}{1.271437in}}{\pgfqpoint{6.348139in}{1.274271in}}{\pgfqpoint{6.353183in}{1.279315in}}%
\pgfpathcurveto{\pgfqpoint{6.358227in}{1.284358in}}{\pgfqpoint{6.361060in}{1.291200in}}{\pgfqpoint{6.361060in}{1.298333in}}%
\pgfpathcurveto{\pgfqpoint{6.361060in}{1.305466in}}{\pgfqpoint{6.358227in}{1.312307in}}{\pgfqpoint{6.353183in}{1.317351in}}%
\pgfpathcurveto{\pgfqpoint{6.348139in}{1.322395in}}{\pgfqpoint{6.341298in}{1.325229in}}{\pgfqpoint{6.334165in}{1.325229in}}%
\pgfpathcurveto{\pgfqpoint{6.327032in}{1.325229in}}{\pgfqpoint{6.320190in}{1.322395in}}{\pgfqpoint{6.315147in}{1.317351in}}%
\pgfpathcurveto{\pgfqpoint{6.310103in}{1.312307in}}{\pgfqpoint{6.307269in}{1.305466in}}{\pgfqpoint{6.307269in}{1.298333in}}%
\pgfpathcurveto{\pgfqpoint{6.307269in}{1.291200in}}{\pgfqpoint{6.310103in}{1.284358in}}{\pgfqpoint{6.315147in}{1.279315in}}%
\pgfpathcurveto{\pgfqpoint{6.320190in}{1.274271in}}{\pgfqpoint{6.327032in}{1.271437in}}{\pgfqpoint{6.334165in}{1.271437in}}%
\pgfpathclose%
\pgfusepath{stroke,fill}%
\end{pgfscope}%
\begin{pgfscope}%
\pgfpathrectangle{\pgfqpoint{4.985294in}{0.500000in}}{\pgfqpoint{1.764706in}{1.700000in}}%
\pgfusepath{clip}%
\pgfsetbuttcap%
\pgfsetroundjoin%
\definecolor{currentfill}{rgb}{0.960421,0.553286,0.393191}%
\pgfsetfillcolor{currentfill}%
\pgfsetlinewidth{0.311001pt}%
\definecolor{currentstroke}{rgb}{1.000000,1.000000,1.000000}%
\pgfsetstrokecolor{currentstroke}%
\pgfsetdash{}{0pt}%
\pgfpathmoveto{\pgfqpoint{6.112501in}{0.868566in}}%
\pgfpathcurveto{\pgfqpoint{6.119633in}{0.868566in}}{\pgfqpoint{6.126475in}{0.871399in}}{\pgfqpoint{6.131519in}{0.876443in}}%
\pgfpathcurveto{\pgfqpoint{6.136562in}{0.881487in}}{\pgfqpoint{6.139396in}{0.888328in}}{\pgfqpoint{6.139396in}{0.895461in}}%
\pgfpathcurveto{\pgfqpoint{6.139396in}{0.902594in}}{\pgfqpoint{6.136562in}{0.909436in}}{\pgfqpoint{6.131519in}{0.914479in}}%
\pgfpathcurveto{\pgfqpoint{6.126475in}{0.919523in}}{\pgfqpoint{6.119633in}{0.922357in}}{\pgfqpoint{6.112501in}{0.922357in}}%
\pgfpathcurveto{\pgfqpoint{6.105368in}{0.922357in}}{\pgfqpoint{6.098526in}{0.919523in}}{\pgfqpoint{6.093482in}{0.914479in}}%
\pgfpathcurveto{\pgfqpoint{6.088439in}{0.909436in}}{\pgfqpoint{6.085605in}{0.902594in}}{\pgfqpoint{6.085605in}{0.895461in}}%
\pgfpathcurveto{\pgfqpoint{6.085605in}{0.888328in}}{\pgfqpoint{6.088439in}{0.881487in}}{\pgfqpoint{6.093482in}{0.876443in}}%
\pgfpathcurveto{\pgfqpoint{6.098526in}{0.871399in}}{\pgfqpoint{6.105368in}{0.868566in}}{\pgfqpoint{6.112501in}{0.868566in}}%
\pgfpathclose%
\pgfusepath{stroke,fill}%
\end{pgfscope}%
\begin{pgfscope}%
\pgfpathrectangle{\pgfqpoint{4.985294in}{0.500000in}}{\pgfqpoint{1.764706in}{1.700000in}}%
\pgfusepath{clip}%
\pgfsetbuttcap%
\pgfsetroundjoin%
\definecolor{currentfill}{rgb}{0.979124,0.903132,0.839793}%
\pgfsetfillcolor{currentfill}%
\pgfsetlinewidth{0.311001pt}%
\definecolor{currentstroke}{rgb}{1.000000,1.000000,1.000000}%
\pgfsetstrokecolor{currentstroke}%
\pgfsetdash{}{0pt}%
\pgfpathmoveto{\pgfqpoint{5.443776in}{1.414344in}}%
\pgfpathcurveto{\pgfqpoint{5.450909in}{1.414344in}}{\pgfqpoint{5.457750in}{1.417178in}}{\pgfqpoint{5.462794in}{1.422222in}}%
\pgfpathcurveto{\pgfqpoint{5.467838in}{1.427265in}}{\pgfqpoint{5.470671in}{1.434107in}}{\pgfqpoint{5.470671in}{1.441240in}}%
\pgfpathcurveto{\pgfqpoint{5.470671in}{1.448373in}}{\pgfqpoint{5.467838in}{1.455214in}}{\pgfqpoint{5.462794in}{1.460258in}}%
\pgfpathcurveto{\pgfqpoint{5.457750in}{1.465302in}}{\pgfqpoint{5.450909in}{1.468136in}}{\pgfqpoint{5.443776in}{1.468136in}}%
\pgfpathcurveto{\pgfqpoint{5.436643in}{1.468136in}}{\pgfqpoint{5.429801in}{1.465302in}}{\pgfqpoint{5.424758in}{1.460258in}}%
\pgfpathcurveto{\pgfqpoint{5.419714in}{1.455214in}}{\pgfqpoint{5.416880in}{1.448373in}}{\pgfqpoint{5.416880in}{1.441240in}}%
\pgfpathcurveto{\pgfqpoint{5.416880in}{1.434107in}}{\pgfqpoint{5.419714in}{1.427265in}}{\pgfqpoint{5.424758in}{1.422222in}}%
\pgfpathcurveto{\pgfqpoint{5.429801in}{1.417178in}}{\pgfqpoint{5.436643in}{1.414344in}}{\pgfqpoint{5.443776in}{1.414344in}}%
\pgfpathclose%
\pgfusepath{stroke,fill}%
\end{pgfscope}%
\begin{pgfscope}%
\pgfpathrectangle{\pgfqpoint{4.985294in}{0.500000in}}{\pgfqpoint{1.764706in}{1.700000in}}%
\pgfusepath{clip}%
\pgfsetbuttcap%
\pgfsetroundjoin%
\definecolor{currentfill}{rgb}{0.968105,0.786346,0.667739}%
\pgfsetfillcolor{currentfill}%
\pgfsetlinewidth{0.311001pt}%
\definecolor{currentstroke}{rgb}{1.000000,1.000000,1.000000}%
\pgfsetstrokecolor{currentstroke}%
\pgfsetdash{}{0pt}%
\pgfpathmoveto{\pgfqpoint{6.348667in}{1.525137in}}%
\pgfpathcurveto{\pgfqpoint{6.355800in}{1.525137in}}{\pgfqpoint{6.362642in}{1.527971in}}{\pgfqpoint{6.367685in}{1.533015in}}%
\pgfpathcurveto{\pgfqpoint{6.372729in}{1.538059in}}{\pgfqpoint{6.375563in}{1.544900in}}{\pgfqpoint{6.375563in}{1.552033in}}%
\pgfpathcurveto{\pgfqpoint{6.375563in}{1.559166in}}{\pgfqpoint{6.372729in}{1.566008in}}{\pgfqpoint{6.367685in}{1.571051in}}%
\pgfpathcurveto{\pgfqpoint{6.362642in}{1.576095in}}{\pgfqpoint{6.355800in}{1.578929in}}{\pgfqpoint{6.348667in}{1.578929in}}%
\pgfpathcurveto{\pgfqpoint{6.341534in}{1.578929in}}{\pgfqpoint{6.334693in}{1.576095in}}{\pgfqpoint{6.329649in}{1.571051in}}%
\pgfpathcurveto{\pgfqpoint{6.324605in}{1.566008in}}{\pgfqpoint{6.321771in}{1.559166in}}{\pgfqpoint{6.321771in}{1.552033in}}%
\pgfpathcurveto{\pgfqpoint{6.321771in}{1.544900in}}{\pgfqpoint{6.324605in}{1.538059in}}{\pgfqpoint{6.329649in}{1.533015in}}%
\pgfpathcurveto{\pgfqpoint{6.334693in}{1.527971in}}{\pgfqpoint{6.341534in}{1.525137in}}{\pgfqpoint{6.348667in}{1.525137in}}%
\pgfpathclose%
\pgfusepath{stroke,fill}%
\end{pgfscope}%
\begin{pgfscope}%
\pgfpathrectangle{\pgfqpoint{4.985294in}{0.500000in}}{\pgfqpoint{1.764706in}{1.700000in}}%
\pgfusepath{clip}%
\pgfsetbuttcap%
\pgfsetroundjoin%
\definecolor{currentfill}{rgb}{0.961433,0.573272,0.412036}%
\pgfsetfillcolor{currentfill}%
\pgfsetlinewidth{0.311001pt}%
\definecolor{currentstroke}{rgb}{1.000000,1.000000,1.000000}%
\pgfsetstrokecolor{currentstroke}%
\pgfsetdash{}{0pt}%
\pgfpathmoveto{\pgfqpoint{6.127682in}{1.771965in}}%
\pgfpathcurveto{\pgfqpoint{6.134814in}{1.771965in}}{\pgfqpoint{6.141656in}{1.774799in}}{\pgfqpoint{6.146700in}{1.779843in}}%
\pgfpathcurveto{\pgfqpoint{6.151743in}{1.784886in}}{\pgfqpoint{6.154577in}{1.791728in}}{\pgfqpoint{6.154577in}{1.798861in}}%
\pgfpathcurveto{\pgfqpoint{6.154577in}{1.805994in}}{\pgfqpoint{6.151743in}{1.812835in}}{\pgfqpoint{6.146700in}{1.817879in}}%
\pgfpathcurveto{\pgfqpoint{6.141656in}{1.822923in}}{\pgfqpoint{6.134814in}{1.825756in}}{\pgfqpoint{6.127682in}{1.825756in}}%
\pgfpathcurveto{\pgfqpoint{6.120549in}{1.825756in}}{\pgfqpoint{6.113707in}{1.822923in}}{\pgfqpoint{6.108663in}{1.817879in}}%
\pgfpathcurveto{\pgfqpoint{6.103620in}{1.812835in}}{\pgfqpoint{6.100786in}{1.805994in}}{\pgfqpoint{6.100786in}{1.798861in}}%
\pgfpathcurveto{\pgfqpoint{6.100786in}{1.791728in}}{\pgfqpoint{6.103620in}{1.784886in}}{\pgfqpoint{6.108663in}{1.779843in}}%
\pgfpathcurveto{\pgfqpoint{6.113707in}{1.774799in}}{\pgfqpoint{6.120549in}{1.771965in}}{\pgfqpoint{6.127682in}{1.771965in}}%
\pgfpathclose%
\pgfusepath{stroke,fill}%
\end{pgfscope}%
\begin{pgfscope}%
\pgfpathrectangle{\pgfqpoint{4.985294in}{0.500000in}}{\pgfqpoint{1.764706in}{1.700000in}}%
\pgfusepath{clip}%
\pgfsetbuttcap%
\pgfsetroundjoin%
\definecolor{currentfill}{rgb}{0.977657,0.891500,0.822809}%
\pgfsetfillcolor{currentfill}%
\pgfsetlinewidth{0.311001pt}%
\definecolor{currentstroke}{rgb}{1.000000,1.000000,1.000000}%
\pgfsetstrokecolor{currentstroke}%
\pgfsetdash{}{0pt}%
\pgfpathmoveto{\pgfqpoint{5.396884in}{1.230553in}}%
\pgfpathcurveto{\pgfqpoint{5.404017in}{1.230553in}}{\pgfqpoint{5.410859in}{1.233387in}}{\pgfqpoint{5.415903in}{1.238431in}}%
\pgfpathcurveto{\pgfqpoint{5.420946in}{1.243474in}}{\pgfqpoint{5.423780in}{1.250316in}}{\pgfqpoint{5.423780in}{1.257449in}}%
\pgfpathcurveto{\pgfqpoint{5.423780in}{1.264582in}}{\pgfqpoint{5.420946in}{1.271423in}}{\pgfqpoint{5.415903in}{1.276467in}}%
\pgfpathcurveto{\pgfqpoint{5.410859in}{1.281511in}}{\pgfqpoint{5.404017in}{1.284345in}}{\pgfqpoint{5.396884in}{1.284345in}}%
\pgfpathcurveto{\pgfqpoint{5.389752in}{1.284345in}}{\pgfqpoint{5.382910in}{1.281511in}}{\pgfqpoint{5.377866in}{1.276467in}}%
\pgfpathcurveto{\pgfqpoint{5.372823in}{1.271423in}}{\pgfqpoint{5.369989in}{1.264582in}}{\pgfqpoint{5.369989in}{1.257449in}}%
\pgfpathcurveto{\pgfqpoint{5.369989in}{1.250316in}}{\pgfqpoint{5.372823in}{1.243474in}}{\pgfqpoint{5.377866in}{1.238431in}}%
\pgfpathcurveto{\pgfqpoint{5.382910in}{1.233387in}}{\pgfqpoint{5.389752in}{1.230553in}}{\pgfqpoint{5.396884in}{1.230553in}}%
\pgfpathclose%
\pgfusepath{stroke,fill}%
\end{pgfscope}%
\begin{pgfscope}%
\pgfpathrectangle{\pgfqpoint{4.985294in}{0.500000in}}{\pgfqpoint{1.764706in}{1.700000in}}%
\pgfusepath{clip}%
\pgfsetbuttcap%
\pgfsetroundjoin%
\definecolor{currentfill}{rgb}{0.971694,0.833208,0.737161}%
\pgfsetfillcolor{currentfill}%
\pgfsetlinewidth{0.311001pt}%
\definecolor{currentstroke}{rgb}{1.000000,1.000000,1.000000}%
\pgfsetstrokecolor{currentstroke}%
\pgfsetdash{}{0pt}%
\pgfpathmoveto{\pgfqpoint{5.401037in}{1.522756in}}%
\pgfpathcurveto{\pgfqpoint{5.408170in}{1.522756in}}{\pgfqpoint{5.415011in}{1.525590in}}{\pgfqpoint{5.420055in}{1.530633in}}%
\pgfpathcurveto{\pgfqpoint{5.425099in}{1.535677in}}{\pgfqpoint{5.427933in}{1.542519in}}{\pgfqpoint{5.427933in}{1.549651in}}%
\pgfpathcurveto{\pgfqpoint{5.427933in}{1.556784in}}{\pgfqpoint{5.425099in}{1.563626in}}{\pgfqpoint{5.420055in}{1.568670in}}%
\pgfpathcurveto{\pgfqpoint{5.415011in}{1.573713in}}{\pgfqpoint{5.408170in}{1.576547in}}{\pgfqpoint{5.401037in}{1.576547in}}%
\pgfpathcurveto{\pgfqpoint{5.393904in}{1.576547in}}{\pgfqpoint{5.387062in}{1.573713in}}{\pgfqpoint{5.382019in}{1.568670in}}%
\pgfpathcurveto{\pgfqpoint{5.376975in}{1.563626in}}{\pgfqpoint{5.374141in}{1.556784in}}{\pgfqpoint{5.374141in}{1.549651in}}%
\pgfpathcurveto{\pgfqpoint{5.374141in}{1.542519in}}{\pgfqpoint{5.376975in}{1.535677in}}{\pgfqpoint{5.382019in}{1.530633in}}%
\pgfpathcurveto{\pgfqpoint{5.387062in}{1.525590in}}{\pgfqpoint{5.393904in}{1.522756in}}{\pgfqpoint{5.401037in}{1.522756in}}%
\pgfpathclose%
\pgfusepath{stroke,fill}%
\end{pgfscope}%
\begin{pgfscope}%
\pgfpathrectangle{\pgfqpoint{4.985294in}{0.500000in}}{\pgfqpoint{1.764706in}{1.700000in}}%
\pgfusepath{clip}%
\pgfsetbuttcap%
\pgfsetroundjoin%
\definecolor{currentfill}{rgb}{0.948235,0.413004,0.283323}%
\pgfsetfillcolor{currentfill}%
\pgfsetlinewidth{0.311001pt}%
\definecolor{currentstroke}{rgb}{1.000000,1.000000,1.000000}%
\pgfsetstrokecolor{currentstroke}%
\pgfsetdash{}{0pt}%
\pgfpathmoveto{\pgfqpoint{6.113301in}{1.134486in}}%
\pgfpathcurveto{\pgfqpoint{6.120434in}{1.134486in}}{\pgfqpoint{6.127276in}{1.137320in}}{\pgfqpoint{6.132319in}{1.142364in}}%
\pgfpathcurveto{\pgfqpoint{6.137363in}{1.147408in}}{\pgfqpoint{6.140197in}{1.154249in}}{\pgfqpoint{6.140197in}{1.161382in}}%
\pgfpathcurveto{\pgfqpoint{6.140197in}{1.168515in}}{\pgfqpoint{6.137363in}{1.175357in}}{\pgfqpoint{6.132319in}{1.180400in}}%
\pgfpathcurveto{\pgfqpoint{6.127276in}{1.185444in}}{\pgfqpoint{6.120434in}{1.188278in}}{\pgfqpoint{6.113301in}{1.188278in}}%
\pgfpathcurveto{\pgfqpoint{6.106168in}{1.188278in}}{\pgfqpoint{6.099327in}{1.185444in}}{\pgfqpoint{6.094283in}{1.180400in}}%
\pgfpathcurveto{\pgfqpoint{6.089239in}{1.175357in}}{\pgfqpoint{6.086406in}{1.168515in}}{\pgfqpoint{6.086406in}{1.161382in}}%
\pgfpathcurveto{\pgfqpoint{6.086406in}{1.154249in}}{\pgfqpoint{6.089239in}{1.147408in}}{\pgfqpoint{6.094283in}{1.142364in}}%
\pgfpathcurveto{\pgfqpoint{6.099327in}{1.137320in}}{\pgfqpoint{6.106168in}{1.134486in}}{\pgfqpoint{6.113301in}{1.134486in}}%
\pgfpathclose%
\pgfusepath{stroke,fill}%
\end{pgfscope}%
\begin{pgfscope}%
\pgfpathrectangle{\pgfqpoint{4.985294in}{0.500000in}}{\pgfqpoint{1.764706in}{1.700000in}}%
\pgfusepath{clip}%
\pgfsetbuttcap%
\pgfsetroundjoin%
\definecolor{currentfill}{rgb}{0.964679,0.682838,0.530002}%
\pgfsetfillcolor{currentfill}%
\pgfsetlinewidth{0.311001pt}%
\definecolor{currentstroke}{rgb}{1.000000,1.000000,1.000000}%
\pgfsetstrokecolor{currentstroke}%
\pgfsetdash{}{0pt}%
\pgfpathmoveto{\pgfqpoint{6.138148in}{1.053871in}}%
\pgfpathcurveto{\pgfqpoint{6.145281in}{1.053871in}}{\pgfqpoint{6.152122in}{1.056705in}}{\pgfqpoint{6.157166in}{1.061748in}}%
\pgfpathcurveto{\pgfqpoint{6.162210in}{1.066792in}}{\pgfqpoint{6.165044in}{1.073634in}}{\pgfqpoint{6.165044in}{1.080767in}}%
\pgfpathcurveto{\pgfqpoint{6.165044in}{1.087899in}}{\pgfqpoint{6.162210in}{1.094741in}}{\pgfqpoint{6.157166in}{1.099785in}}%
\pgfpathcurveto{\pgfqpoint{6.152122in}{1.104828in}}{\pgfqpoint{6.145281in}{1.107662in}}{\pgfqpoint{6.138148in}{1.107662in}}%
\pgfpathcurveto{\pgfqpoint{6.131015in}{1.107662in}}{\pgfqpoint{6.124173in}{1.104828in}}{\pgfqpoint{6.119130in}{1.099785in}}%
\pgfpathcurveto{\pgfqpoint{6.114086in}{1.094741in}}{\pgfqpoint{6.111252in}{1.087899in}}{\pgfqpoint{6.111252in}{1.080767in}}%
\pgfpathcurveto{\pgfqpoint{6.111252in}{1.073634in}}{\pgfqpoint{6.114086in}{1.066792in}}{\pgfqpoint{6.119130in}{1.061748in}}%
\pgfpathcurveto{\pgfqpoint{6.124173in}{1.056705in}}{\pgfqpoint{6.131015in}{1.053871in}}{\pgfqpoint{6.138148in}{1.053871in}}%
\pgfpathclose%
\pgfusepath{stroke,fill}%
\end{pgfscope}%
\begin{pgfscope}%
\pgfpathrectangle{\pgfqpoint{4.985294in}{0.500000in}}{\pgfqpoint{1.764706in}{1.700000in}}%
\pgfusepath{clip}%
\pgfsetbuttcap%
\pgfsetroundjoin%
\definecolor{currentfill}{rgb}{0.965753,0.732351,0.592427}%
\pgfsetfillcolor{currentfill}%
\pgfsetlinewidth{0.311001pt}%
\definecolor{currentstroke}{rgb}{1.000000,1.000000,1.000000}%
\pgfsetstrokecolor{currentstroke}%
\pgfsetdash{}{0pt}%
\pgfpathmoveto{\pgfqpoint{5.589531in}{1.021775in}}%
\pgfpathcurveto{\pgfqpoint{5.596664in}{1.021775in}}{\pgfqpoint{5.603505in}{1.024608in}}{\pgfqpoint{5.608549in}{1.029652in}}%
\pgfpathcurveto{\pgfqpoint{5.613593in}{1.034696in}}{\pgfqpoint{5.616426in}{1.041537in}}{\pgfqpoint{5.616426in}{1.048670in}}%
\pgfpathcurveto{\pgfqpoint{5.616426in}{1.055803in}}{\pgfqpoint{5.613593in}{1.062645in}}{\pgfqpoint{5.608549in}{1.067688in}}%
\pgfpathcurveto{\pgfqpoint{5.603505in}{1.072732in}}{\pgfqpoint{5.596664in}{1.075566in}}{\pgfqpoint{5.589531in}{1.075566in}}%
\pgfpathcurveto{\pgfqpoint{5.582398in}{1.075566in}}{\pgfqpoint{5.575556in}{1.072732in}}{\pgfqpoint{5.570513in}{1.067688in}}%
\pgfpathcurveto{\pgfqpoint{5.565469in}{1.062645in}}{\pgfqpoint{5.562635in}{1.055803in}}{\pgfqpoint{5.562635in}{1.048670in}}%
\pgfpathcurveto{\pgfqpoint{5.562635in}{1.041537in}}{\pgfqpoint{5.565469in}{1.034696in}}{\pgfqpoint{5.570513in}{1.029652in}}%
\pgfpathcurveto{\pgfqpoint{5.575556in}{1.024608in}}{\pgfqpoint{5.582398in}{1.021775in}}{\pgfqpoint{5.589531in}{1.021775in}}%
\pgfpathclose%
\pgfusepath{stroke,fill}%
\end{pgfscope}%
\begin{pgfscope}%
\pgfpathrectangle{\pgfqpoint{4.985294in}{0.500000in}}{\pgfqpoint{1.764706in}{1.700000in}}%
\pgfusepath{clip}%
\pgfsetbuttcap%
\pgfsetroundjoin%
\definecolor{currentfill}{rgb}{0.968105,0.786346,0.667739}%
\pgfsetfillcolor{currentfill}%
\pgfsetlinewidth{0.311001pt}%
\definecolor{currentstroke}{rgb}{1.000000,1.000000,1.000000}%
\pgfsetstrokecolor{currentstroke}%
\pgfsetdash{}{0pt}%
\pgfpathmoveto{\pgfqpoint{6.213119in}{1.242165in}}%
\pgfpathcurveto{\pgfqpoint{6.220252in}{1.242165in}}{\pgfqpoint{6.227093in}{1.244999in}}{\pgfqpoint{6.232137in}{1.250042in}}%
\pgfpathcurveto{\pgfqpoint{6.237181in}{1.255086in}}{\pgfqpoint{6.240015in}{1.261928in}}{\pgfqpoint{6.240015in}{1.269061in}}%
\pgfpathcurveto{\pgfqpoint{6.240015in}{1.276193in}}{\pgfqpoint{6.237181in}{1.283035in}}{\pgfqpoint{6.232137in}{1.288079in}}%
\pgfpathcurveto{\pgfqpoint{6.227093in}{1.293122in}}{\pgfqpoint{6.220252in}{1.295956in}}{\pgfqpoint{6.213119in}{1.295956in}}%
\pgfpathcurveto{\pgfqpoint{6.205986in}{1.295956in}}{\pgfqpoint{6.199144in}{1.293122in}}{\pgfqpoint{6.194101in}{1.288079in}}%
\pgfpathcurveto{\pgfqpoint{6.189057in}{1.283035in}}{\pgfqpoint{6.186223in}{1.276193in}}{\pgfqpoint{6.186223in}{1.269061in}}%
\pgfpathcurveto{\pgfqpoint{6.186223in}{1.261928in}}{\pgfqpoint{6.189057in}{1.255086in}}{\pgfqpoint{6.194101in}{1.250042in}}%
\pgfpathcurveto{\pgfqpoint{6.199144in}{1.244999in}}{\pgfqpoint{6.205986in}{1.242165in}}{\pgfqpoint{6.213119in}{1.242165in}}%
\pgfpathclose%
\pgfusepath{stroke,fill}%
\end{pgfscope}%
\begin{pgfscope}%
\pgfpathrectangle{\pgfqpoint{4.985294in}{0.500000in}}{\pgfqpoint{1.764706in}{1.700000in}}%
\pgfusepath{clip}%
\pgfsetbuttcap%
\pgfsetroundjoin%
\definecolor{currentfill}{rgb}{0.975018,0.868213,0.788710}%
\pgfsetfillcolor{currentfill}%
\pgfsetlinewidth{0.311001pt}%
\definecolor{currentstroke}{rgb}{1.000000,1.000000,1.000000}%
\pgfsetstrokecolor{currentstroke}%
\pgfsetdash{}{0pt}%
\pgfpathmoveto{\pgfqpoint{5.474539in}{1.423664in}}%
\pgfpathcurveto{\pgfqpoint{5.481672in}{1.423664in}}{\pgfqpoint{5.488513in}{1.426498in}}{\pgfqpoint{5.493557in}{1.431542in}}%
\pgfpathcurveto{\pgfqpoint{5.498601in}{1.436585in}}{\pgfqpoint{5.501435in}{1.443427in}}{\pgfqpoint{5.501435in}{1.450560in}}%
\pgfpathcurveto{\pgfqpoint{5.501435in}{1.457693in}}{\pgfqpoint{5.498601in}{1.464534in}}{\pgfqpoint{5.493557in}{1.469578in}}%
\pgfpathcurveto{\pgfqpoint{5.488513in}{1.474622in}}{\pgfqpoint{5.481672in}{1.477455in}}{\pgfqpoint{5.474539in}{1.477455in}}%
\pgfpathcurveto{\pgfqpoint{5.467406in}{1.477455in}}{\pgfqpoint{5.460564in}{1.474622in}}{\pgfqpoint{5.455521in}{1.469578in}}%
\pgfpathcurveto{\pgfqpoint{5.450477in}{1.464534in}}{\pgfqpoint{5.447643in}{1.457693in}}{\pgfqpoint{5.447643in}{1.450560in}}%
\pgfpathcurveto{\pgfqpoint{5.447643in}{1.443427in}}{\pgfqpoint{5.450477in}{1.436585in}}{\pgfqpoint{5.455521in}{1.431542in}}%
\pgfpathcurveto{\pgfqpoint{5.460564in}{1.426498in}}{\pgfqpoint{5.467406in}{1.423664in}}{\pgfqpoint{5.474539in}{1.423664in}}%
\pgfpathclose%
\pgfusepath{stroke,fill}%
\end{pgfscope}%
\begin{pgfscope}%
\pgfpathrectangle{\pgfqpoint{4.985294in}{0.500000in}}{\pgfqpoint{1.764706in}{1.700000in}}%
\pgfusepath{clip}%
\pgfsetbuttcap%
\pgfsetroundjoin%
\definecolor{currentfill}{rgb}{0.980678,0.914765,0.856766}%
\pgfsetfillcolor{currentfill}%
\pgfsetlinewidth{0.311001pt}%
\definecolor{currentstroke}{rgb}{1.000000,1.000000,1.000000}%
\pgfsetstrokecolor{currentstroke}%
\pgfsetdash{}{0pt}%
\pgfpathmoveto{\pgfqpoint{6.299862in}{1.318039in}}%
\pgfpathcurveto{\pgfqpoint{6.306995in}{1.318039in}}{\pgfqpoint{6.313837in}{1.320873in}}{\pgfqpoint{6.318880in}{1.325916in}}%
\pgfpathcurveto{\pgfqpoint{6.323924in}{1.330960in}}{\pgfqpoint{6.326758in}{1.337802in}}{\pgfqpoint{6.326758in}{1.344934in}}%
\pgfpathcurveto{\pgfqpoint{6.326758in}{1.352067in}}{\pgfqpoint{6.323924in}{1.358909in}}{\pgfqpoint{6.318880in}{1.363953in}}%
\pgfpathcurveto{\pgfqpoint{6.313837in}{1.368996in}}{\pgfqpoint{6.306995in}{1.371830in}}{\pgfqpoint{6.299862in}{1.371830in}}%
\pgfpathcurveto{\pgfqpoint{6.292729in}{1.371830in}}{\pgfqpoint{6.285888in}{1.368996in}}{\pgfqpoint{6.280844in}{1.363953in}}%
\pgfpathcurveto{\pgfqpoint{6.275800in}{1.358909in}}{\pgfqpoint{6.272966in}{1.352067in}}{\pgfqpoint{6.272966in}{1.344934in}}%
\pgfpathcurveto{\pgfqpoint{6.272966in}{1.337802in}}{\pgfqpoint{6.275800in}{1.330960in}}{\pgfqpoint{6.280844in}{1.325916in}}%
\pgfpathcurveto{\pgfqpoint{6.285888in}{1.320873in}}{\pgfqpoint{6.292729in}{1.318039in}}{\pgfqpoint{6.299862in}{1.318039in}}%
\pgfpathclose%
\pgfusepath{stroke,fill}%
\end{pgfscope}%
\begin{pgfscope}%
\pgfpathrectangle{\pgfqpoint{4.985294in}{0.500000in}}{\pgfqpoint{1.764706in}{1.700000in}}%
\pgfusepath{clip}%
\pgfsetbuttcap%
\pgfsetroundjoin%
\definecolor{currentfill}{rgb}{0.969359,0.803954,0.693832}%
\pgfsetfillcolor{currentfill}%
\pgfsetlinewidth{0.311001pt}%
\definecolor{currentstroke}{rgb}{1.000000,1.000000,1.000000}%
\pgfsetstrokecolor{currentstroke}%
\pgfsetdash{}{0pt}%
\pgfpathmoveto{\pgfqpoint{6.213540in}{1.495135in}}%
\pgfpathcurveto{\pgfqpoint{6.220672in}{1.495135in}}{\pgfqpoint{6.227514in}{1.497969in}}{\pgfqpoint{6.232558in}{1.503013in}}%
\pgfpathcurveto{\pgfqpoint{6.237601in}{1.508057in}}{\pgfqpoint{6.240435in}{1.514898in}}{\pgfqpoint{6.240435in}{1.522031in}}%
\pgfpathcurveto{\pgfqpoint{6.240435in}{1.529164in}}{\pgfqpoint{6.237601in}{1.536006in}}{\pgfqpoint{6.232558in}{1.541049in}}%
\pgfpathcurveto{\pgfqpoint{6.227514in}{1.546093in}}{\pgfqpoint{6.220672in}{1.548927in}}{\pgfqpoint{6.213540in}{1.548927in}}%
\pgfpathcurveto{\pgfqpoint{6.206407in}{1.548927in}}{\pgfqpoint{6.199565in}{1.546093in}}{\pgfqpoint{6.194521in}{1.541049in}}%
\pgfpathcurveto{\pgfqpoint{6.189478in}{1.536006in}}{\pgfqpoint{6.186644in}{1.529164in}}{\pgfqpoint{6.186644in}{1.522031in}}%
\pgfpathcurveto{\pgfqpoint{6.186644in}{1.514898in}}{\pgfqpoint{6.189478in}{1.508057in}}{\pgfqpoint{6.194521in}{1.503013in}}%
\pgfpathcurveto{\pgfqpoint{6.199565in}{1.497969in}}{\pgfqpoint{6.206407in}{1.495135in}}{\pgfqpoint{6.213540in}{1.495135in}}%
\pgfpathclose%
\pgfusepath{stroke,fill}%
\end{pgfscope}%
\begin{pgfscope}%
\pgfpathrectangle{\pgfqpoint{4.985294in}{0.500000in}}{\pgfqpoint{1.764706in}{1.700000in}}%
\pgfusepath{clip}%
\pgfsetbuttcap%
\pgfsetroundjoin%
\definecolor{currentfill}{rgb}{0.955103,0.477872,0.328626}%
\pgfsetfillcolor{currentfill}%
\pgfsetlinewidth{0.311001pt}%
\definecolor{currentstroke}{rgb}{1.000000,1.000000,1.000000}%
\pgfsetstrokecolor{currentstroke}%
\pgfsetdash{}{0pt}%
\pgfpathmoveto{\pgfqpoint{6.402725in}{1.583919in}}%
\pgfpathcurveto{\pgfqpoint{6.409857in}{1.583919in}}{\pgfqpoint{6.416699in}{1.586753in}}{\pgfqpoint{6.421743in}{1.591797in}}%
\pgfpathcurveto{\pgfqpoint{6.426786in}{1.596841in}}{\pgfqpoint{6.429620in}{1.603682in}}{\pgfqpoint{6.429620in}{1.610815in}}%
\pgfpathcurveto{\pgfqpoint{6.429620in}{1.617948in}}{\pgfqpoint{6.426786in}{1.624790in}}{\pgfqpoint{6.421743in}{1.629833in}}%
\pgfpathcurveto{\pgfqpoint{6.416699in}{1.634877in}}{\pgfqpoint{6.409857in}{1.637711in}}{\pgfqpoint{6.402725in}{1.637711in}}%
\pgfpathcurveto{\pgfqpoint{6.395592in}{1.637711in}}{\pgfqpoint{6.388750in}{1.634877in}}{\pgfqpoint{6.383707in}{1.629833in}}%
\pgfpathcurveto{\pgfqpoint{6.378663in}{1.624790in}}{\pgfqpoint{6.375829in}{1.617948in}}{\pgfqpoint{6.375829in}{1.610815in}}%
\pgfpathcurveto{\pgfqpoint{6.375829in}{1.603682in}}{\pgfqpoint{6.378663in}{1.596841in}}{\pgfqpoint{6.383707in}{1.591797in}}%
\pgfpathcurveto{\pgfqpoint{6.388750in}{1.586753in}}{\pgfqpoint{6.395592in}{1.583919in}}{\pgfqpoint{6.402725in}{1.583919in}}%
\pgfpathclose%
\pgfusepath{stroke,fill}%
\end{pgfscope}%
\begin{pgfscope}%
\pgfpathrectangle{\pgfqpoint{4.985294in}{0.500000in}}{\pgfqpoint{1.764706in}{1.700000in}}%
\pgfusepath{clip}%
\pgfsetbuttcap%
\pgfsetroundjoin%
\definecolor{currentfill}{rgb}{0.972726,0.844889,0.754401}%
\pgfsetfillcolor{currentfill}%
\pgfsetlinewidth{0.311001pt}%
\definecolor{currentstroke}{rgb}{1.000000,1.000000,1.000000}%
\pgfsetstrokecolor{currentstroke}%
\pgfsetdash{}{0pt}%
\pgfpathmoveto{\pgfqpoint{5.438461in}{1.057764in}}%
\pgfpathcurveto{\pgfqpoint{5.445594in}{1.057764in}}{\pgfqpoint{5.452435in}{1.060598in}}{\pgfqpoint{5.457479in}{1.065642in}}%
\pgfpathcurveto{\pgfqpoint{5.462523in}{1.070685in}}{\pgfqpoint{5.465356in}{1.077527in}}{\pgfqpoint{5.465356in}{1.084660in}}%
\pgfpathcurveto{\pgfqpoint{5.465356in}{1.091793in}}{\pgfqpoint{5.462523in}{1.098634in}}{\pgfqpoint{5.457479in}{1.103678in}}%
\pgfpathcurveto{\pgfqpoint{5.452435in}{1.108722in}}{\pgfqpoint{5.445594in}{1.111556in}}{\pgfqpoint{5.438461in}{1.111556in}}%
\pgfpathcurveto{\pgfqpoint{5.431328in}{1.111556in}}{\pgfqpoint{5.424486in}{1.108722in}}{\pgfqpoint{5.419443in}{1.103678in}}%
\pgfpathcurveto{\pgfqpoint{5.414399in}{1.098634in}}{\pgfqpoint{5.411565in}{1.091793in}}{\pgfqpoint{5.411565in}{1.084660in}}%
\pgfpathcurveto{\pgfqpoint{5.411565in}{1.077527in}}{\pgfqpoint{5.414399in}{1.070685in}}{\pgfqpoint{5.419443in}{1.065642in}}%
\pgfpathcurveto{\pgfqpoint{5.424486in}{1.060598in}}{\pgfqpoint{5.431328in}{1.057764in}}{\pgfqpoint{5.438461in}{1.057764in}}%
\pgfpathclose%
\pgfusepath{stroke,fill}%
\end{pgfscope}%
\begin{pgfscope}%
\pgfpathrectangle{\pgfqpoint{4.985294in}{0.500000in}}{\pgfqpoint{1.764706in}{1.700000in}}%
\pgfusepath{clip}%
\pgfsetbuttcap%
\pgfsetroundjoin%
\definecolor{currentfill}{rgb}{0.979891,0.908948,0.848279}%
\pgfsetfillcolor{currentfill}%
\pgfsetlinewidth{0.311001pt}%
\definecolor{currentstroke}{rgb}{1.000000,1.000000,1.000000}%
\pgfsetstrokecolor{currentstroke}%
\pgfsetdash{}{0pt}%
\pgfpathmoveto{\pgfqpoint{6.291939in}{1.279348in}}%
\pgfpathcurveto{\pgfqpoint{6.299072in}{1.279348in}}{\pgfqpoint{6.305913in}{1.282181in}}{\pgfqpoint{6.310957in}{1.287225in}}%
\pgfpathcurveto{\pgfqpoint{6.316001in}{1.292269in}}{\pgfqpoint{6.318835in}{1.299110in}}{\pgfqpoint{6.318835in}{1.306243in}}%
\pgfpathcurveto{\pgfqpoint{6.318835in}{1.313376in}}{\pgfqpoint{6.316001in}{1.320218in}}{\pgfqpoint{6.310957in}{1.325261in}}%
\pgfpathcurveto{\pgfqpoint{6.305913in}{1.330305in}}{\pgfqpoint{6.299072in}{1.333139in}}{\pgfqpoint{6.291939in}{1.333139in}}%
\pgfpathcurveto{\pgfqpoint{6.284806in}{1.333139in}}{\pgfqpoint{6.277964in}{1.330305in}}{\pgfqpoint{6.272921in}{1.325261in}}%
\pgfpathcurveto{\pgfqpoint{6.267877in}{1.320218in}}{\pgfqpoint{6.265043in}{1.313376in}}{\pgfqpoint{6.265043in}{1.306243in}}%
\pgfpathcurveto{\pgfqpoint{6.265043in}{1.299110in}}{\pgfqpoint{6.267877in}{1.292269in}}{\pgfqpoint{6.272921in}{1.287225in}}%
\pgfpathcurveto{\pgfqpoint{6.277964in}{1.282181in}}{\pgfqpoint{6.284806in}{1.279348in}}{\pgfqpoint{6.291939in}{1.279348in}}%
\pgfpathclose%
\pgfusepath{stroke,fill}%
\end{pgfscope}%
\begin{pgfscope}%
\pgfpathrectangle{\pgfqpoint{4.985294in}{0.500000in}}{\pgfqpoint{1.764706in}{1.700000in}}%
\pgfusepath{clip}%
\pgfsetbuttcap%
\pgfsetroundjoin%
\definecolor{currentfill}{rgb}{0.979891,0.908948,0.848279}%
\pgfsetfillcolor{currentfill}%
\pgfsetlinewidth{0.311001pt}%
\definecolor{currentstroke}{rgb}{1.000000,1.000000,1.000000}%
\pgfsetstrokecolor{currentstroke}%
\pgfsetdash{}{0pt}%
\pgfpathmoveto{\pgfqpoint{5.427648in}{1.270486in}}%
\pgfpathcurveto{\pgfqpoint{5.434781in}{1.270486in}}{\pgfqpoint{5.441623in}{1.273320in}}{\pgfqpoint{5.446667in}{1.278364in}}%
\pgfpathcurveto{\pgfqpoint{5.451710in}{1.283408in}}{\pgfqpoint{5.454544in}{1.290249in}}{\pgfqpoint{5.454544in}{1.297382in}}%
\pgfpathcurveto{\pgfqpoint{5.454544in}{1.304515in}}{\pgfqpoint{5.451710in}{1.311357in}}{\pgfqpoint{5.446667in}{1.316400in}}%
\pgfpathcurveto{\pgfqpoint{5.441623in}{1.321444in}}{\pgfqpoint{5.434781in}{1.324278in}}{\pgfqpoint{5.427648in}{1.324278in}}%
\pgfpathcurveto{\pgfqpoint{5.420516in}{1.324278in}}{\pgfqpoint{5.413674in}{1.321444in}}{\pgfqpoint{5.408630in}{1.316400in}}%
\pgfpathcurveto{\pgfqpoint{5.403587in}{1.311357in}}{\pgfqpoint{5.400753in}{1.304515in}}{\pgfqpoint{5.400753in}{1.297382in}}%
\pgfpathcurveto{\pgfqpoint{5.400753in}{1.290249in}}{\pgfqpoint{5.403587in}{1.283408in}}{\pgfqpoint{5.408630in}{1.278364in}}%
\pgfpathcurveto{\pgfqpoint{5.413674in}{1.273320in}}{\pgfqpoint{5.420516in}{1.270486in}}{\pgfqpoint{5.427648in}{1.270486in}}%
\pgfpathclose%
\pgfusepath{stroke,fill}%
\end{pgfscope}%
\begin{pgfscope}%
\pgfpathrectangle{\pgfqpoint{4.985294in}{0.500000in}}{\pgfqpoint{1.764706in}{1.700000in}}%
\pgfusepath{clip}%
\pgfsetbuttcap%
\pgfsetroundjoin%
\definecolor{currentfill}{rgb}{0.971694,0.833208,0.737161}%
\pgfsetfillcolor{currentfill}%
\pgfsetlinewidth{0.311001pt}%
\definecolor{currentstroke}{rgb}{1.000000,1.000000,1.000000}%
\pgfsetstrokecolor{currentstroke}%
\pgfsetdash{}{0pt}%
\pgfpathmoveto{\pgfqpoint{5.387708in}{1.148807in}}%
\pgfpathcurveto{\pgfqpoint{5.394841in}{1.148807in}}{\pgfqpoint{5.401683in}{1.151641in}}{\pgfqpoint{5.406726in}{1.156684in}}%
\pgfpathcurveto{\pgfqpoint{5.411770in}{1.161728in}}{\pgfqpoint{5.414604in}{1.168570in}}{\pgfqpoint{5.414604in}{1.175702in}}%
\pgfpathcurveto{\pgfqpoint{5.414604in}{1.182835in}}{\pgfqpoint{5.411770in}{1.189677in}}{\pgfqpoint{5.406726in}{1.194721in}}%
\pgfpathcurveto{\pgfqpoint{5.401683in}{1.199764in}}{\pgfqpoint{5.394841in}{1.202598in}}{\pgfqpoint{5.387708in}{1.202598in}}%
\pgfpathcurveto{\pgfqpoint{5.380575in}{1.202598in}}{\pgfqpoint{5.373734in}{1.199764in}}{\pgfqpoint{5.368690in}{1.194721in}}%
\pgfpathcurveto{\pgfqpoint{5.363646in}{1.189677in}}{\pgfqpoint{5.360813in}{1.182835in}}{\pgfqpoint{5.360813in}{1.175702in}}%
\pgfpathcurveto{\pgfqpoint{5.360813in}{1.168570in}}{\pgfqpoint{5.363646in}{1.161728in}}{\pgfqpoint{5.368690in}{1.156684in}}%
\pgfpathcurveto{\pgfqpoint{5.373734in}{1.151641in}}{\pgfqpoint{5.380575in}{1.148807in}}{\pgfqpoint{5.387708in}{1.148807in}}%
\pgfpathclose%
\pgfusepath{stroke,fill}%
\end{pgfscope}%
\begin{pgfscope}%
\pgfpathrectangle{\pgfqpoint{4.985294in}{0.500000in}}{\pgfqpoint{1.764706in}{1.700000in}}%
\pgfusepath{clip}%
\pgfsetbuttcap%
\pgfsetroundjoin%
\definecolor{currentfill}{rgb}{0.965302,0.713942,0.568499}%
\pgfsetfillcolor{currentfill}%
\pgfsetlinewidth{0.311001pt}%
\definecolor{currentstroke}{rgb}{1.000000,1.000000,1.000000}%
\pgfsetstrokecolor{currentstroke}%
\pgfsetdash{}{0pt}%
\pgfpathmoveto{\pgfqpoint{5.487205in}{1.709049in}}%
\pgfpathcurveto{\pgfqpoint{5.494338in}{1.709049in}}{\pgfqpoint{5.501179in}{1.711883in}}{\pgfqpoint{5.506223in}{1.716927in}}%
\pgfpathcurveto{\pgfqpoint{5.511267in}{1.721971in}}{\pgfqpoint{5.514101in}{1.728812in}}{\pgfqpoint{5.514101in}{1.735945in}}%
\pgfpathcurveto{\pgfqpoint{5.514101in}{1.743078in}}{\pgfqpoint{5.511267in}{1.749920in}}{\pgfqpoint{5.506223in}{1.754963in}}%
\pgfpathcurveto{\pgfqpoint{5.501179in}{1.760007in}}{\pgfqpoint{5.494338in}{1.762841in}}{\pgfqpoint{5.487205in}{1.762841in}}%
\pgfpathcurveto{\pgfqpoint{5.480072in}{1.762841in}}{\pgfqpoint{5.473230in}{1.760007in}}{\pgfqpoint{5.468187in}{1.754963in}}%
\pgfpathcurveto{\pgfqpoint{5.463143in}{1.749920in}}{\pgfqpoint{5.460309in}{1.743078in}}{\pgfqpoint{5.460309in}{1.735945in}}%
\pgfpathcurveto{\pgfqpoint{5.460309in}{1.728812in}}{\pgfqpoint{5.463143in}{1.721971in}}{\pgfqpoint{5.468187in}{1.716927in}}%
\pgfpathcurveto{\pgfqpoint{5.473230in}{1.711883in}}{\pgfqpoint{5.480072in}{1.709049in}}{\pgfqpoint{5.487205in}{1.709049in}}%
\pgfpathclose%
\pgfusepath{stroke,fill}%
\end{pgfscope}%
\begin{pgfscope}%
\pgfpathrectangle{\pgfqpoint{4.985294in}{0.500000in}}{\pgfqpoint{1.764706in}{1.700000in}}%
\pgfusepath{clip}%
\pgfsetbuttcap%
\pgfsetroundjoin%
\definecolor{currentfill}{rgb}{0.975018,0.868213,0.788710}%
\pgfsetfillcolor{currentfill}%
\pgfsetlinewidth{0.311001pt}%
\definecolor{currentstroke}{rgb}{1.000000,1.000000,1.000000}%
\pgfsetstrokecolor{currentstroke}%
\pgfsetdash{}{0pt}%
\pgfpathmoveto{\pgfqpoint{6.251629in}{1.517438in}}%
\pgfpathcurveto{\pgfqpoint{6.258762in}{1.517438in}}{\pgfqpoint{6.265604in}{1.520272in}}{\pgfqpoint{6.270647in}{1.525315in}}%
\pgfpathcurveto{\pgfqpoint{6.275691in}{1.530359in}}{\pgfqpoint{6.278525in}{1.537201in}}{\pgfqpoint{6.278525in}{1.544334in}}%
\pgfpathcurveto{\pgfqpoint{6.278525in}{1.551466in}}{\pgfqpoint{6.275691in}{1.558308in}}{\pgfqpoint{6.270647in}{1.563352in}}%
\pgfpathcurveto{\pgfqpoint{6.265604in}{1.568395in}}{\pgfqpoint{6.258762in}{1.571229in}}{\pgfqpoint{6.251629in}{1.571229in}}%
\pgfpathcurveto{\pgfqpoint{6.244496in}{1.571229in}}{\pgfqpoint{6.237655in}{1.568395in}}{\pgfqpoint{6.232611in}{1.563352in}}%
\pgfpathcurveto{\pgfqpoint{6.227567in}{1.558308in}}{\pgfqpoint{6.224734in}{1.551466in}}{\pgfqpoint{6.224734in}{1.544334in}}%
\pgfpathcurveto{\pgfqpoint{6.224734in}{1.537201in}}{\pgfqpoint{6.227567in}{1.530359in}}{\pgfqpoint{6.232611in}{1.525315in}}%
\pgfpathcurveto{\pgfqpoint{6.237655in}{1.520272in}}{\pgfqpoint{6.244496in}{1.517438in}}{\pgfqpoint{6.251629in}{1.517438in}}%
\pgfpathclose%
\pgfusepath{stroke,fill}%
\end{pgfscope}%
\begin{pgfscope}%
\pgfpathrectangle{\pgfqpoint{4.985294in}{0.500000in}}{\pgfqpoint{1.764706in}{1.700000in}}%
\pgfusepath{clip}%
\pgfsetbuttcap%
\pgfsetroundjoin%
\definecolor{currentfill}{rgb}{0.905301,0.238545,0.247481}%
\pgfsetfillcolor{currentfill}%
\pgfsetlinewidth{0.311001pt}%
\definecolor{currentstroke}{rgb}{1.000000,1.000000,1.000000}%
\pgfsetstrokecolor{currentstroke}%
\pgfsetdash{}{0pt}%
\pgfpathmoveto{\pgfqpoint{6.379363in}{0.954240in}}%
\pgfpathcurveto{\pgfqpoint{6.386495in}{0.954240in}}{\pgfqpoint{6.393337in}{0.957074in}}{\pgfqpoint{6.398381in}{0.962117in}}%
\pgfpathcurveto{\pgfqpoint{6.403424in}{0.967161in}}{\pgfqpoint{6.406258in}{0.974003in}}{\pgfqpoint{6.406258in}{0.981136in}}%
\pgfpathcurveto{\pgfqpoint{6.406258in}{0.988268in}}{\pgfqpoint{6.403424in}{0.995110in}}{\pgfqpoint{6.398381in}{1.000154in}}%
\pgfpathcurveto{\pgfqpoint{6.393337in}{1.005197in}}{\pgfqpoint{6.386495in}{1.008031in}}{\pgfqpoint{6.379363in}{1.008031in}}%
\pgfpathcurveto{\pgfqpoint{6.372230in}{1.008031in}}{\pgfqpoint{6.365388in}{1.005197in}}{\pgfqpoint{6.360344in}{1.000154in}}%
\pgfpathcurveto{\pgfqpoint{6.355301in}{0.995110in}}{\pgfqpoint{6.352467in}{0.988268in}}{\pgfqpoint{6.352467in}{0.981136in}}%
\pgfpathcurveto{\pgfqpoint{6.352467in}{0.974003in}}{\pgfqpoint{6.355301in}{0.967161in}}{\pgfqpoint{6.360344in}{0.962117in}}%
\pgfpathcurveto{\pgfqpoint{6.365388in}{0.957074in}}{\pgfqpoint{6.372230in}{0.954240in}}{\pgfqpoint{6.379363in}{0.954240in}}%
\pgfpathclose%
\pgfusepath{stroke,fill}%
\end{pgfscope}%
\begin{pgfscope}%
\pgfpathrectangle{\pgfqpoint{4.985294in}{0.500000in}}{\pgfqpoint{1.764706in}{1.700000in}}%
\pgfusepath{clip}%
\pgfsetbuttcap%
\pgfsetroundjoin%
\definecolor{currentfill}{rgb}{0.965928,0.738443,0.600540}%
\pgfsetfillcolor{currentfill}%
\pgfsetlinewidth{0.311001pt}%
\definecolor{currentstroke}{rgb}{1.000000,1.000000,1.000000}%
\pgfsetstrokecolor{currentstroke}%
\pgfsetdash{}{0pt}%
\pgfpathmoveto{\pgfqpoint{6.180256in}{0.927858in}}%
\pgfpathcurveto{\pgfqpoint{6.187389in}{0.927858in}}{\pgfqpoint{6.194230in}{0.930692in}}{\pgfqpoint{6.199274in}{0.935736in}}%
\pgfpathcurveto{\pgfqpoint{6.204317in}{0.940779in}}{\pgfqpoint{6.207151in}{0.947621in}}{\pgfqpoint{6.207151in}{0.954754in}}%
\pgfpathcurveto{\pgfqpoint{6.207151in}{0.961887in}}{\pgfqpoint{6.204317in}{0.968728in}}{\pgfqpoint{6.199274in}{0.973772in}}%
\pgfpathcurveto{\pgfqpoint{6.194230in}{0.978816in}}{\pgfqpoint{6.187389in}{0.981650in}}{\pgfqpoint{6.180256in}{0.981650in}}%
\pgfpathcurveto{\pgfqpoint{6.173123in}{0.981650in}}{\pgfqpoint{6.166281in}{0.978816in}}{\pgfqpoint{6.161238in}{0.973772in}}%
\pgfpathcurveto{\pgfqpoint{6.156194in}{0.968728in}}{\pgfqpoint{6.153360in}{0.961887in}}{\pgfqpoint{6.153360in}{0.954754in}}%
\pgfpathcurveto{\pgfqpoint{6.153360in}{0.947621in}}{\pgfqpoint{6.156194in}{0.940779in}}{\pgfqpoint{6.161238in}{0.935736in}}%
\pgfpathcurveto{\pgfqpoint{6.166281in}{0.930692in}}{\pgfqpoint{6.173123in}{0.927858in}}{\pgfqpoint{6.180256in}{0.927858in}}%
\pgfpathclose%
\pgfusepath{stroke,fill}%
\end{pgfscope}%
\begin{pgfscope}%
\pgfpathrectangle{\pgfqpoint{4.985294in}{0.500000in}}{\pgfqpoint{1.764706in}{1.700000in}}%
\pgfusepath{clip}%
\pgfsetbuttcap%
\pgfsetroundjoin%
\definecolor{currentfill}{rgb}{0.958331,0.519463,0.362986}%
\pgfsetfillcolor{currentfill}%
\pgfsetlinewidth{0.311001pt}%
\definecolor{currentstroke}{rgb}{1.000000,1.000000,1.000000}%
\pgfsetstrokecolor{currentstroke}%
\pgfsetdash{}{0pt}%
\pgfpathmoveto{\pgfqpoint{5.290332in}{1.358448in}}%
\pgfpathcurveto{\pgfqpoint{5.297465in}{1.358448in}}{\pgfqpoint{5.304307in}{1.361282in}}{\pgfqpoint{5.309350in}{1.366326in}}%
\pgfpathcurveto{\pgfqpoint{5.314394in}{1.371369in}}{\pgfqpoint{5.317228in}{1.378211in}}{\pgfqpoint{5.317228in}{1.385344in}}%
\pgfpathcurveto{\pgfqpoint{5.317228in}{1.392477in}}{\pgfqpoint{5.314394in}{1.399318in}}{\pgfqpoint{5.309350in}{1.404362in}}%
\pgfpathcurveto{\pgfqpoint{5.304307in}{1.409406in}}{\pgfqpoint{5.297465in}{1.412240in}}{\pgfqpoint{5.290332in}{1.412240in}}%
\pgfpathcurveto{\pgfqpoint{5.283199in}{1.412240in}}{\pgfqpoint{5.276358in}{1.409406in}}{\pgfqpoint{5.271314in}{1.404362in}}%
\pgfpathcurveto{\pgfqpoint{5.266270in}{1.399318in}}{\pgfqpoint{5.263436in}{1.392477in}}{\pgfqpoint{5.263436in}{1.385344in}}%
\pgfpathcurveto{\pgfqpoint{5.263436in}{1.378211in}}{\pgfqpoint{5.266270in}{1.371369in}}{\pgfqpoint{5.271314in}{1.366326in}}%
\pgfpathcurveto{\pgfqpoint{5.276358in}{1.361282in}}{\pgfqpoint{5.283199in}{1.358448in}}{\pgfqpoint{5.290332in}{1.358448in}}%
\pgfpathclose%
\pgfusepath{stroke,fill}%
\end{pgfscope}%
\begin{pgfscope}%
\pgfpathrectangle{\pgfqpoint{4.985294in}{0.500000in}}{\pgfqpoint{1.764706in}{1.700000in}}%
\pgfusepath{clip}%
\pgfsetbuttcap%
\pgfsetroundjoin%
\definecolor{currentfill}{rgb}{0.981377,0.920617,0.865369}%
\pgfsetfillcolor{currentfill}%
\pgfsetlinewidth{0.311001pt}%
\definecolor{currentstroke}{rgb}{1.000000,1.000000,1.000000}%
\pgfsetstrokecolor{currentstroke}%
\pgfsetdash{}{0pt}%
\pgfpathmoveto{\pgfqpoint{6.304263in}{1.462937in}}%
\pgfpathcurveto{\pgfqpoint{6.311396in}{1.462937in}}{\pgfqpoint{6.318238in}{1.465771in}}{\pgfqpoint{6.323281in}{1.470815in}}%
\pgfpathcurveto{\pgfqpoint{6.328325in}{1.475858in}}{\pgfqpoint{6.331159in}{1.482700in}}{\pgfqpoint{6.331159in}{1.489833in}}%
\pgfpathcurveto{\pgfqpoint{6.331159in}{1.496966in}}{\pgfqpoint{6.328325in}{1.503807in}}{\pgfqpoint{6.323281in}{1.508851in}}%
\pgfpathcurveto{\pgfqpoint{6.318238in}{1.513895in}}{\pgfqpoint{6.311396in}{1.516729in}}{\pgfqpoint{6.304263in}{1.516729in}}%
\pgfpathcurveto{\pgfqpoint{6.297131in}{1.516729in}}{\pgfqpoint{6.290289in}{1.513895in}}{\pgfqpoint{6.285245in}{1.508851in}}%
\pgfpathcurveto{\pgfqpoint{6.280202in}{1.503807in}}{\pgfqpoint{6.277368in}{1.496966in}}{\pgfqpoint{6.277368in}{1.489833in}}%
\pgfpathcurveto{\pgfqpoint{6.277368in}{1.482700in}}{\pgfqpoint{6.280202in}{1.475858in}}{\pgfqpoint{6.285245in}{1.470815in}}%
\pgfpathcurveto{\pgfqpoint{6.290289in}{1.465771in}}{\pgfqpoint{6.297131in}{1.462937in}}{\pgfqpoint{6.304263in}{1.462937in}}%
\pgfpathclose%
\pgfusepath{stroke,fill}%
\end{pgfscope}%
\begin{pgfscope}%
\pgfpathrectangle{\pgfqpoint{4.985294in}{0.500000in}}{\pgfqpoint{1.764706in}{1.700000in}}%
\pgfusepath{clip}%
\pgfsetbuttcap%
\pgfsetroundjoin%
\definecolor{currentfill}{rgb}{0.977657,0.891500,0.822809}%
\pgfsetfillcolor{currentfill}%
\pgfsetlinewidth{0.311001pt}%
\definecolor{currentstroke}{rgb}{1.000000,1.000000,1.000000}%
\pgfsetstrokecolor{currentstroke}%
\pgfsetdash{}{0pt}%
\pgfpathmoveto{\pgfqpoint{5.433102in}{1.493239in}}%
\pgfpathcurveto{\pgfqpoint{5.440234in}{1.493239in}}{\pgfqpoint{5.447076in}{1.496073in}}{\pgfqpoint{5.452120in}{1.501117in}}%
\pgfpathcurveto{\pgfqpoint{5.457163in}{1.506161in}}{\pgfqpoint{5.459997in}{1.513002in}}{\pgfqpoint{5.459997in}{1.520135in}}%
\pgfpathcurveto{\pgfqpoint{5.459997in}{1.527268in}}{\pgfqpoint{5.457163in}{1.534110in}}{\pgfqpoint{5.452120in}{1.539153in}}%
\pgfpathcurveto{\pgfqpoint{5.447076in}{1.544197in}}{\pgfqpoint{5.440234in}{1.547031in}}{\pgfqpoint{5.433102in}{1.547031in}}%
\pgfpathcurveto{\pgfqpoint{5.425969in}{1.547031in}}{\pgfqpoint{5.419127in}{1.544197in}}{\pgfqpoint{5.414084in}{1.539153in}}%
\pgfpathcurveto{\pgfqpoint{5.409040in}{1.534110in}}{\pgfqpoint{5.406206in}{1.527268in}}{\pgfqpoint{5.406206in}{1.520135in}}%
\pgfpathcurveto{\pgfqpoint{5.406206in}{1.513002in}}{\pgfqpoint{5.409040in}{1.506161in}}{\pgfqpoint{5.414084in}{1.501117in}}%
\pgfpathcurveto{\pgfqpoint{5.419127in}{1.496073in}}{\pgfqpoint{5.425969in}{1.493239in}}{\pgfqpoint{5.433102in}{1.493239in}}%
\pgfpathclose%
\pgfusepath{stroke,fill}%
\end{pgfscope}%
\begin{pgfscope}%
\pgfpathrectangle{\pgfqpoint{4.985294in}{0.500000in}}{\pgfqpoint{1.764706in}{1.700000in}}%
\pgfusepath{clip}%
\pgfsetbuttcap%
\pgfsetroundjoin%
\definecolor{currentfill}{rgb}{0.968931,0.798091,0.685123}%
\pgfsetfillcolor{currentfill}%
\pgfsetlinewidth{0.311001pt}%
\definecolor{currentstroke}{rgb}{1.000000,1.000000,1.000000}%
\pgfsetstrokecolor{currentstroke}%
\pgfsetdash{}{0pt}%
\pgfpathmoveto{\pgfqpoint{6.284919in}{1.025435in}}%
\pgfpathcurveto{\pgfqpoint{6.292052in}{1.025435in}}{\pgfqpoint{6.298894in}{1.028269in}}{\pgfqpoint{6.303937in}{1.033313in}}%
\pgfpathcurveto{\pgfqpoint{6.308981in}{1.038356in}}{\pgfqpoint{6.311815in}{1.045198in}}{\pgfqpoint{6.311815in}{1.052331in}}%
\pgfpathcurveto{\pgfqpoint{6.311815in}{1.059464in}}{\pgfqpoint{6.308981in}{1.066305in}}{\pgfqpoint{6.303937in}{1.071349in}}%
\pgfpathcurveto{\pgfqpoint{6.298894in}{1.076393in}}{\pgfqpoint{6.292052in}{1.079226in}}{\pgfqpoint{6.284919in}{1.079226in}}%
\pgfpathcurveto{\pgfqpoint{6.277786in}{1.079226in}}{\pgfqpoint{6.270945in}{1.076393in}}{\pgfqpoint{6.265901in}{1.071349in}}%
\pgfpathcurveto{\pgfqpoint{6.260857in}{1.066305in}}{\pgfqpoint{6.258023in}{1.059464in}}{\pgfqpoint{6.258023in}{1.052331in}}%
\pgfpathcurveto{\pgfqpoint{6.258023in}{1.045198in}}{\pgfqpoint{6.260857in}{1.038356in}}{\pgfqpoint{6.265901in}{1.033313in}}%
\pgfpathcurveto{\pgfqpoint{6.270945in}{1.028269in}}{\pgfqpoint{6.277786in}{1.025435in}}{\pgfqpoint{6.284919in}{1.025435in}}%
\pgfpathclose%
\pgfusepath{stroke,fill}%
\end{pgfscope}%
\begin{pgfscope}%
\pgfpathrectangle{\pgfqpoint{4.985294in}{0.500000in}}{\pgfqpoint{1.764706in}{1.700000in}}%
\pgfusepath{clip}%
\pgfsetbuttcap%
\pgfsetroundjoin%
\definecolor{currentfill}{rgb}{0.973271,0.850724,0.762998}%
\pgfsetfillcolor{currentfill}%
\pgfsetlinewidth{0.311001pt}%
\definecolor{currentstroke}{rgb}{1.000000,1.000000,1.000000}%
\pgfsetstrokecolor{currentstroke}%
\pgfsetdash{}{0pt}%
\pgfpathmoveto{\pgfqpoint{6.323881in}{1.121250in}}%
\pgfpathcurveto{\pgfqpoint{6.331014in}{1.121250in}}{\pgfqpoint{6.337856in}{1.124084in}}{\pgfqpoint{6.342900in}{1.129128in}}%
\pgfpathcurveto{\pgfqpoint{6.347943in}{1.134171in}}{\pgfqpoint{6.350777in}{1.141013in}}{\pgfqpoint{6.350777in}{1.148146in}}%
\pgfpathcurveto{\pgfqpoint{6.350777in}{1.155279in}}{\pgfqpoint{6.347943in}{1.162120in}}{\pgfqpoint{6.342900in}{1.167164in}}%
\pgfpathcurveto{\pgfqpoint{6.337856in}{1.172208in}}{\pgfqpoint{6.331014in}{1.175042in}}{\pgfqpoint{6.323881in}{1.175042in}}%
\pgfpathcurveto{\pgfqpoint{6.316749in}{1.175042in}}{\pgfqpoint{6.309907in}{1.172208in}}{\pgfqpoint{6.304863in}{1.167164in}}%
\pgfpathcurveto{\pgfqpoint{6.299820in}{1.162120in}}{\pgfqpoint{6.296986in}{1.155279in}}{\pgfqpoint{6.296986in}{1.148146in}}%
\pgfpathcurveto{\pgfqpoint{6.296986in}{1.141013in}}{\pgfqpoint{6.299820in}{1.134171in}}{\pgfqpoint{6.304863in}{1.129128in}}%
\pgfpathcurveto{\pgfqpoint{6.309907in}{1.124084in}}{\pgfqpoint{6.316749in}{1.121250in}}{\pgfqpoint{6.323881in}{1.121250in}}%
\pgfpathclose%
\pgfusepath{stroke,fill}%
\end{pgfscope}%
\begin{pgfscope}%
\pgfpathrectangle{\pgfqpoint{4.985294in}{0.500000in}}{\pgfqpoint{1.764706in}{1.700000in}}%
\pgfusepath{clip}%
\pgfsetbuttcap%
\pgfsetroundjoin%
\definecolor{currentfill}{rgb}{0.973832,0.856556,0.771584}%
\pgfsetfillcolor{currentfill}%
\pgfsetlinewidth{0.311001pt}%
\definecolor{currentstroke}{rgb}{1.000000,1.000000,1.000000}%
\pgfsetstrokecolor{currentstroke}%
\pgfsetdash{}{0pt}%
\pgfpathmoveto{\pgfqpoint{6.304421in}{1.577877in}}%
\pgfpathcurveto{\pgfqpoint{6.311554in}{1.577877in}}{\pgfqpoint{6.318395in}{1.580711in}}{\pgfqpoint{6.323439in}{1.585755in}}%
\pgfpathcurveto{\pgfqpoint{6.328483in}{1.590799in}}{\pgfqpoint{6.331316in}{1.597640in}}{\pgfqpoint{6.331316in}{1.604773in}}%
\pgfpathcurveto{\pgfqpoint{6.331316in}{1.611906in}}{\pgfqpoint{6.328483in}{1.618748in}}{\pgfqpoint{6.323439in}{1.623791in}}%
\pgfpathcurveto{\pgfqpoint{6.318395in}{1.628835in}}{\pgfqpoint{6.311554in}{1.631669in}}{\pgfqpoint{6.304421in}{1.631669in}}%
\pgfpathcurveto{\pgfqpoint{6.297288in}{1.631669in}}{\pgfqpoint{6.290446in}{1.628835in}}{\pgfqpoint{6.285403in}{1.623791in}}%
\pgfpathcurveto{\pgfqpoint{6.280359in}{1.618748in}}{\pgfqpoint{6.277525in}{1.611906in}}{\pgfqpoint{6.277525in}{1.604773in}}%
\pgfpathcurveto{\pgfqpoint{6.277525in}{1.597640in}}{\pgfqpoint{6.280359in}{1.590799in}}{\pgfqpoint{6.285403in}{1.585755in}}%
\pgfpathcurveto{\pgfqpoint{6.290446in}{1.580711in}}{\pgfqpoint{6.297288in}{1.577877in}}{\pgfqpoint{6.304421in}{1.577877in}}%
\pgfpathclose%
\pgfusepath{stroke,fill}%
\end{pgfscope}%
\begin{pgfscope}%
\pgfpathrectangle{\pgfqpoint{4.985294in}{0.500000in}}{\pgfqpoint{1.764706in}{1.700000in}}%
\pgfusepath{clip}%
\pgfsetbuttcap%
\pgfsetroundjoin%
\definecolor{currentfill}{rgb}{0.963379,0.625574,0.465113}%
\pgfsetfillcolor{currentfill}%
\pgfsetlinewidth{0.311001pt}%
\definecolor{currentstroke}{rgb}{1.000000,1.000000,1.000000}%
\pgfsetstrokecolor{currentstroke}%
\pgfsetdash{}{0pt}%
\pgfpathmoveto{\pgfqpoint{5.547318in}{1.213632in}}%
\pgfpathcurveto{\pgfqpoint{5.554450in}{1.213632in}}{\pgfqpoint{5.561292in}{1.216466in}}{\pgfqpoint{5.566336in}{1.221510in}}%
\pgfpathcurveto{\pgfqpoint{5.571379in}{1.226554in}}{\pgfqpoint{5.574213in}{1.233395in}}{\pgfqpoint{5.574213in}{1.240528in}}%
\pgfpathcurveto{\pgfqpoint{5.574213in}{1.247661in}}{\pgfqpoint{5.571379in}{1.254503in}}{\pgfqpoint{5.566336in}{1.259546in}}%
\pgfpathcurveto{\pgfqpoint{5.561292in}{1.264590in}}{\pgfqpoint{5.554450in}{1.267424in}}{\pgfqpoint{5.547318in}{1.267424in}}%
\pgfpathcurveto{\pgfqpoint{5.540185in}{1.267424in}}{\pgfqpoint{5.533343in}{1.264590in}}{\pgfqpoint{5.528299in}{1.259546in}}%
\pgfpathcurveto{\pgfqpoint{5.523256in}{1.254503in}}{\pgfqpoint{5.520422in}{1.247661in}}{\pgfqpoint{5.520422in}{1.240528in}}%
\pgfpathcurveto{\pgfqpoint{5.520422in}{1.233395in}}{\pgfqpoint{5.523256in}{1.226554in}}{\pgfqpoint{5.528299in}{1.221510in}}%
\pgfpathcurveto{\pgfqpoint{5.533343in}{1.216466in}}{\pgfqpoint{5.540185in}{1.213632in}}{\pgfqpoint{5.547318in}{1.213632in}}%
\pgfpathclose%
\pgfusepath{stroke,fill}%
\end{pgfscope}%
\begin{pgfscope}%
\pgfpathrectangle{\pgfqpoint{4.985294in}{0.500000in}}{\pgfqpoint{1.764706in}{1.700000in}}%
\pgfusepath{clip}%
\pgfsetbuttcap%
\pgfsetroundjoin%
\definecolor{currentfill}{rgb}{0.965302,0.713942,0.568499}%
\pgfsetfillcolor{currentfill}%
\pgfsetlinewidth{0.311001pt}%
\definecolor{currentstroke}{rgb}{1.000000,1.000000,1.000000}%
\pgfsetstrokecolor{currentstroke}%
\pgfsetdash{}{0pt}%
\pgfpathmoveto{\pgfqpoint{5.407350in}{1.002087in}}%
\pgfpathcurveto{\pgfqpoint{5.414483in}{1.002087in}}{\pgfqpoint{5.421325in}{1.004920in}}{\pgfqpoint{5.426369in}{1.009964in}}%
\pgfpathcurveto{\pgfqpoint{5.431412in}{1.015008in}}{\pgfqpoint{5.434246in}{1.021849in}}{\pgfqpoint{5.434246in}{1.028982in}}%
\pgfpathcurveto{\pgfqpoint{5.434246in}{1.036115in}}{\pgfqpoint{5.431412in}{1.042957in}}{\pgfqpoint{5.426369in}{1.048000in}}%
\pgfpathcurveto{\pgfqpoint{5.421325in}{1.053044in}}{\pgfqpoint{5.414483in}{1.055878in}}{\pgfqpoint{5.407350in}{1.055878in}}%
\pgfpathcurveto{\pgfqpoint{5.400218in}{1.055878in}}{\pgfqpoint{5.393376in}{1.053044in}}{\pgfqpoint{5.388332in}{1.048000in}}%
\pgfpathcurveto{\pgfqpoint{5.383289in}{1.042957in}}{\pgfqpoint{5.380455in}{1.036115in}}{\pgfqpoint{5.380455in}{1.028982in}}%
\pgfpathcurveto{\pgfqpoint{5.380455in}{1.021849in}}{\pgfqpoint{5.383289in}{1.015008in}}{\pgfqpoint{5.388332in}{1.009964in}}%
\pgfpathcurveto{\pgfqpoint{5.393376in}{1.004920in}}{\pgfqpoint{5.400218in}{1.002087in}}{\pgfqpoint{5.407350in}{1.002087in}}%
\pgfpathclose%
\pgfusepath{stroke,fill}%
\end{pgfscope}%
\begin{pgfscope}%
\pgfpathrectangle{\pgfqpoint{4.985294in}{0.500000in}}{\pgfqpoint{1.764706in}{1.700000in}}%
\pgfusepath{clip}%
\pgfsetbuttcap%
\pgfsetroundjoin%
\definecolor{currentfill}{rgb}{0.979124,0.903132,0.839793}%
\pgfsetfillcolor{currentfill}%
\pgfsetlinewidth{0.311001pt}%
\definecolor{currentstroke}{rgb}{1.000000,1.000000,1.000000}%
\pgfsetstrokecolor{currentstroke}%
\pgfsetdash{}{0pt}%
\pgfpathmoveto{\pgfqpoint{6.284631in}{1.233317in}}%
\pgfpathcurveto{\pgfqpoint{6.291764in}{1.233317in}}{\pgfqpoint{6.298606in}{1.236151in}}{\pgfqpoint{6.303649in}{1.241194in}}%
\pgfpathcurveto{\pgfqpoint{6.308693in}{1.246238in}}{\pgfqpoint{6.311527in}{1.253080in}}{\pgfqpoint{6.311527in}{1.260212in}}%
\pgfpathcurveto{\pgfqpoint{6.311527in}{1.267345in}}{\pgfqpoint{6.308693in}{1.274187in}}{\pgfqpoint{6.303649in}{1.279231in}}%
\pgfpathcurveto{\pgfqpoint{6.298606in}{1.284274in}}{\pgfqpoint{6.291764in}{1.287108in}}{\pgfqpoint{6.284631in}{1.287108in}}%
\pgfpathcurveto{\pgfqpoint{6.277498in}{1.287108in}}{\pgfqpoint{6.270657in}{1.284274in}}{\pgfqpoint{6.265613in}{1.279231in}}%
\pgfpathcurveto{\pgfqpoint{6.260569in}{1.274187in}}{\pgfqpoint{6.257735in}{1.267345in}}{\pgfqpoint{6.257735in}{1.260212in}}%
\pgfpathcurveto{\pgfqpoint{6.257735in}{1.253080in}}{\pgfqpoint{6.260569in}{1.246238in}}{\pgfqpoint{6.265613in}{1.241194in}}%
\pgfpathcurveto{\pgfqpoint{6.270657in}{1.236151in}}{\pgfqpoint{6.277498in}{1.233317in}}{\pgfqpoint{6.284631in}{1.233317in}}%
\pgfpathclose%
\pgfusepath{stroke,fill}%
\end{pgfscope}%
\begin{pgfscope}%
\pgfpathrectangle{\pgfqpoint{4.985294in}{0.500000in}}{\pgfqpoint{1.764706in}{1.700000in}}%
\pgfusepath{clip}%
\pgfsetbuttcap%
\pgfsetroundjoin%
\definecolor{currentfill}{rgb}{0.968931,0.798091,0.685123}%
\pgfsetfillcolor{currentfill}%
\pgfsetlinewidth{0.311001pt}%
\definecolor{currentstroke}{rgb}{1.000000,1.000000,1.000000}%
\pgfsetstrokecolor{currentstroke}%
\pgfsetdash{}{0pt}%
\pgfpathmoveto{\pgfqpoint{6.183298in}{1.096010in}}%
\pgfpathcurveto{\pgfqpoint{6.190430in}{1.096010in}}{\pgfqpoint{6.197272in}{1.098844in}}{\pgfqpoint{6.202316in}{1.103887in}}%
\pgfpathcurveto{\pgfqpoint{6.207359in}{1.108931in}}{\pgfqpoint{6.210193in}{1.115773in}}{\pgfqpoint{6.210193in}{1.122905in}}%
\pgfpathcurveto{\pgfqpoint{6.210193in}{1.130038in}}{\pgfqpoint{6.207359in}{1.136880in}}{\pgfqpoint{6.202316in}{1.141924in}}%
\pgfpathcurveto{\pgfqpoint{6.197272in}{1.146967in}}{\pgfqpoint{6.190430in}{1.149801in}}{\pgfqpoint{6.183298in}{1.149801in}}%
\pgfpathcurveto{\pgfqpoint{6.176165in}{1.149801in}}{\pgfqpoint{6.169323in}{1.146967in}}{\pgfqpoint{6.164279in}{1.141924in}}%
\pgfpathcurveto{\pgfqpoint{6.159236in}{1.136880in}}{\pgfqpoint{6.156402in}{1.130038in}}{\pgfqpoint{6.156402in}{1.122905in}}%
\pgfpathcurveto{\pgfqpoint{6.156402in}{1.115773in}}{\pgfqpoint{6.159236in}{1.108931in}}{\pgfqpoint{6.164279in}{1.103887in}}%
\pgfpathcurveto{\pgfqpoint{6.169323in}{1.098844in}}{\pgfqpoint{6.176165in}{1.096010in}}{\pgfqpoint{6.183298in}{1.096010in}}%
\pgfpathclose%
\pgfusepath{stroke,fill}%
\end{pgfscope}%
\begin{pgfscope}%
\pgfpathrectangle{\pgfqpoint{4.985294in}{0.500000in}}{\pgfqpoint{1.764706in}{1.700000in}}%
\pgfusepath{clip}%
\pgfsetbuttcap%
\pgfsetroundjoin%
\definecolor{currentfill}{rgb}{0.979891,0.908948,0.848279}%
\pgfsetfillcolor{currentfill}%
\pgfsetlinewidth{0.311001pt}%
\definecolor{currentstroke}{rgb}{1.000000,1.000000,1.000000}%
\pgfsetstrokecolor{currentstroke}%
\pgfsetdash{}{0pt}%
\pgfpathmoveto{\pgfqpoint{5.428423in}{1.276757in}}%
\pgfpathcurveto{\pgfqpoint{5.435555in}{1.276757in}}{\pgfqpoint{5.442397in}{1.279591in}}{\pgfqpoint{5.447441in}{1.284635in}}%
\pgfpathcurveto{\pgfqpoint{5.452484in}{1.289679in}}{\pgfqpoint{5.455318in}{1.296520in}}{\pgfqpoint{5.455318in}{1.303653in}}%
\pgfpathcurveto{\pgfqpoint{5.455318in}{1.310786in}}{\pgfqpoint{5.452484in}{1.317627in}}{\pgfqpoint{5.447441in}{1.322671in}}%
\pgfpathcurveto{\pgfqpoint{5.442397in}{1.327715in}}{\pgfqpoint{5.435555in}{1.330549in}}{\pgfqpoint{5.428423in}{1.330549in}}%
\pgfpathcurveto{\pgfqpoint{5.421290in}{1.330549in}}{\pgfqpoint{5.414448in}{1.327715in}}{\pgfqpoint{5.409404in}{1.322671in}}%
\pgfpathcurveto{\pgfqpoint{5.404361in}{1.317627in}}{\pgfqpoint{5.401527in}{1.310786in}}{\pgfqpoint{5.401527in}{1.303653in}}%
\pgfpathcurveto{\pgfqpoint{5.401527in}{1.296520in}}{\pgfqpoint{5.404361in}{1.289679in}}{\pgfqpoint{5.409404in}{1.284635in}}%
\pgfpathcurveto{\pgfqpoint{5.414448in}{1.279591in}}{\pgfqpoint{5.421290in}{1.276757in}}{\pgfqpoint{5.428423in}{1.276757in}}%
\pgfpathclose%
\pgfusepath{stroke,fill}%
\end{pgfscope}%
\begin{pgfscope}%
\pgfpathrectangle{\pgfqpoint{4.985294in}{0.500000in}}{\pgfqpoint{1.764706in}{1.700000in}}%
\pgfusepath{clip}%
\pgfsetbuttcap%
\pgfsetroundjoin%
\definecolor{currentfill}{rgb}{0.973271,0.850724,0.762998}%
\pgfsetfillcolor{currentfill}%
\pgfsetlinewidth{0.311001pt}%
\definecolor{currentstroke}{rgb}{1.000000,1.000000,1.000000}%
\pgfsetstrokecolor{currentstroke}%
\pgfsetdash{}{0pt}%
\pgfpathmoveto{\pgfqpoint{6.265379in}{1.636990in}}%
\pgfpathcurveto{\pgfqpoint{6.272512in}{1.636990in}}{\pgfqpoint{6.279354in}{1.639824in}}{\pgfqpoint{6.284397in}{1.644868in}}%
\pgfpathcurveto{\pgfqpoint{6.289441in}{1.649912in}}{\pgfqpoint{6.292275in}{1.656753in}}{\pgfqpoint{6.292275in}{1.663886in}}%
\pgfpathcurveto{\pgfqpoint{6.292275in}{1.671019in}}{\pgfqpoint{6.289441in}{1.677860in}}{\pgfqpoint{6.284397in}{1.682904in}}%
\pgfpathcurveto{\pgfqpoint{6.279354in}{1.687948in}}{\pgfqpoint{6.272512in}{1.690782in}}{\pgfqpoint{6.265379in}{1.690782in}}%
\pgfpathcurveto{\pgfqpoint{6.258246in}{1.690782in}}{\pgfqpoint{6.251405in}{1.687948in}}{\pgfqpoint{6.246361in}{1.682904in}}%
\pgfpathcurveto{\pgfqpoint{6.241317in}{1.677860in}}{\pgfqpoint{6.238483in}{1.671019in}}{\pgfqpoint{6.238483in}{1.663886in}}%
\pgfpathcurveto{\pgfqpoint{6.238483in}{1.656753in}}{\pgfqpoint{6.241317in}{1.649912in}}{\pgfqpoint{6.246361in}{1.644868in}}%
\pgfpathcurveto{\pgfqpoint{6.251405in}{1.639824in}}{\pgfqpoint{6.258246in}{1.636990in}}{\pgfqpoint{6.265379in}{1.636990in}}%
\pgfpathclose%
\pgfusepath{stroke,fill}%
\end{pgfscope}%
\begin{pgfscope}%
\pgfpathrectangle{\pgfqpoint{4.985294in}{0.500000in}}{\pgfqpoint{1.764706in}{1.700000in}}%
\pgfusepath{clip}%
\pgfsetbuttcap%
\pgfsetroundjoin%
\definecolor{currentfill}{rgb}{0.962532,0.599594,0.438051}%
\pgfsetfillcolor{currentfill}%
\pgfsetlinewidth{0.311001pt}%
\definecolor{currentstroke}{rgb}{1.000000,1.000000,1.000000}%
\pgfsetstrokecolor{currentstroke}%
\pgfsetdash{}{0pt}%
\pgfpathmoveto{\pgfqpoint{6.312758in}{0.962668in}}%
\pgfpathcurveto{\pgfqpoint{6.319891in}{0.962668in}}{\pgfqpoint{6.326732in}{0.965502in}}{\pgfqpoint{6.331776in}{0.970546in}}%
\pgfpathcurveto{\pgfqpoint{6.336820in}{0.975589in}}{\pgfqpoint{6.339653in}{0.982431in}}{\pgfqpoint{6.339653in}{0.989564in}}%
\pgfpathcurveto{\pgfqpoint{6.339653in}{0.996697in}}{\pgfqpoint{6.336820in}{1.003538in}}{\pgfqpoint{6.331776in}{1.008582in}}%
\pgfpathcurveto{\pgfqpoint{6.326732in}{1.013626in}}{\pgfqpoint{6.319891in}{1.016459in}}{\pgfqpoint{6.312758in}{1.016459in}}%
\pgfpathcurveto{\pgfqpoint{6.305625in}{1.016459in}}{\pgfqpoint{6.298783in}{1.013626in}}{\pgfqpoint{6.293740in}{1.008582in}}%
\pgfpathcurveto{\pgfqpoint{6.288696in}{1.003538in}}{\pgfqpoint{6.285862in}{0.996697in}}{\pgfqpoint{6.285862in}{0.989564in}}%
\pgfpathcurveto{\pgfqpoint{6.285862in}{0.982431in}}{\pgfqpoint{6.288696in}{0.975589in}}{\pgfqpoint{6.293740in}{0.970546in}}%
\pgfpathcurveto{\pgfqpoint{6.298783in}{0.965502in}}{\pgfqpoint{6.305625in}{0.962668in}}{\pgfqpoint{6.312758in}{0.962668in}}%
\pgfpathclose%
\pgfusepath{stroke,fill}%
\end{pgfscope}%
\begin{pgfscope}%
\pgfpathrectangle{\pgfqpoint{4.985294in}{0.500000in}}{\pgfqpoint{1.764706in}{1.700000in}}%
\pgfusepath{clip}%
\pgfsetbuttcap%
\pgfsetroundjoin%
\definecolor{currentfill}{rgb}{0.978376,0.897317,0.831308}%
\pgfsetfillcolor{currentfill}%
\pgfsetlinewidth{0.311001pt}%
\definecolor{currentstroke}{rgb}{1.000000,1.000000,1.000000}%
\pgfsetstrokecolor{currentstroke}%
\pgfsetdash{}{0pt}%
\pgfpathmoveto{\pgfqpoint{5.399284in}{1.414796in}}%
\pgfpathcurveto{\pgfqpoint{5.406417in}{1.414796in}}{\pgfqpoint{5.413259in}{1.417630in}}{\pgfqpoint{5.418302in}{1.422674in}}%
\pgfpathcurveto{\pgfqpoint{5.423346in}{1.427718in}}{\pgfqpoint{5.426180in}{1.434559in}}{\pgfqpoint{5.426180in}{1.441692in}}%
\pgfpathcurveto{\pgfqpoint{5.426180in}{1.448825in}}{\pgfqpoint{5.423346in}{1.455667in}}{\pgfqpoint{5.418302in}{1.460710in}}%
\pgfpathcurveto{\pgfqpoint{5.413259in}{1.465754in}}{\pgfqpoint{5.406417in}{1.468588in}}{\pgfqpoint{5.399284in}{1.468588in}}%
\pgfpathcurveto{\pgfqpoint{5.392151in}{1.468588in}}{\pgfqpoint{5.385310in}{1.465754in}}{\pgfqpoint{5.380266in}{1.460710in}}%
\pgfpathcurveto{\pgfqpoint{5.375222in}{1.455667in}}{\pgfqpoint{5.372388in}{1.448825in}}{\pgfqpoint{5.372388in}{1.441692in}}%
\pgfpathcurveto{\pgfqpoint{5.372388in}{1.434559in}}{\pgfqpoint{5.375222in}{1.427718in}}{\pgfqpoint{5.380266in}{1.422674in}}%
\pgfpathcurveto{\pgfqpoint{5.385310in}{1.417630in}}{\pgfqpoint{5.392151in}{1.414796in}}{\pgfqpoint{5.399284in}{1.414796in}}%
\pgfpathclose%
\pgfusepath{stroke,fill}%
\end{pgfscope}%
\begin{pgfscope}%
\pgfpathrectangle{\pgfqpoint{4.985294in}{0.500000in}}{\pgfqpoint{1.764706in}{1.700000in}}%
\pgfusepath{clip}%
\pgfsetbuttcap%
\pgfsetroundjoin%
\definecolor{currentfill}{rgb}{0.967092,0.768560,0.642079}%
\pgfsetfillcolor{currentfill}%
\pgfsetlinewidth{0.311001pt}%
\definecolor{currentstroke}{rgb}{1.000000,1.000000,1.000000}%
\pgfsetstrokecolor{currentstroke}%
\pgfsetdash{}{0pt}%
\pgfpathmoveto{\pgfqpoint{5.420198in}{1.609217in}}%
\pgfpathcurveto{\pgfqpoint{5.427331in}{1.609217in}}{\pgfqpoint{5.434173in}{1.612051in}}{\pgfqpoint{5.439217in}{1.617095in}}%
\pgfpathcurveto{\pgfqpoint{5.444260in}{1.622139in}}{\pgfqpoint{5.447094in}{1.628980in}}{\pgfqpoint{5.447094in}{1.636113in}}%
\pgfpathcurveto{\pgfqpoint{5.447094in}{1.643246in}}{\pgfqpoint{5.444260in}{1.650087in}}{\pgfqpoint{5.439217in}{1.655131in}}%
\pgfpathcurveto{\pgfqpoint{5.434173in}{1.660175in}}{\pgfqpoint{5.427331in}{1.663009in}}{\pgfqpoint{5.420198in}{1.663009in}}%
\pgfpathcurveto{\pgfqpoint{5.413066in}{1.663009in}}{\pgfqpoint{5.406224in}{1.660175in}}{\pgfqpoint{5.401180in}{1.655131in}}%
\pgfpathcurveto{\pgfqpoint{5.396137in}{1.650087in}}{\pgfqpoint{5.393303in}{1.643246in}}{\pgfqpoint{5.393303in}{1.636113in}}%
\pgfpathcurveto{\pgfqpoint{5.393303in}{1.628980in}}{\pgfqpoint{5.396137in}{1.622139in}}{\pgfqpoint{5.401180in}{1.617095in}}%
\pgfpathcurveto{\pgfqpoint{5.406224in}{1.612051in}}{\pgfqpoint{5.413066in}{1.609217in}}{\pgfqpoint{5.420198in}{1.609217in}}%
\pgfpathclose%
\pgfusepath{stroke,fill}%
\end{pgfscope}%
\begin{pgfscope}%
\pgfpathrectangle{\pgfqpoint{4.985294in}{0.500000in}}{\pgfqpoint{1.764706in}{1.700000in}}%
\pgfusepath{clip}%
\pgfsetbuttcap%
\pgfsetroundjoin%
\definecolor{currentfill}{rgb}{0.977657,0.891500,0.822809}%
\pgfsetfillcolor{currentfill}%
\pgfsetlinewidth{0.311001pt}%
\definecolor{currentstroke}{rgb}{1.000000,1.000000,1.000000}%
\pgfsetstrokecolor{currentstroke}%
\pgfsetdash{}{0pt}%
\pgfpathmoveto{\pgfqpoint{5.431500in}{1.158325in}}%
\pgfpathcurveto{\pgfqpoint{5.438633in}{1.158325in}}{\pgfqpoint{5.445474in}{1.161159in}}{\pgfqpoint{5.450518in}{1.166203in}}%
\pgfpathcurveto{\pgfqpoint{5.455562in}{1.171246in}}{\pgfqpoint{5.458396in}{1.178088in}}{\pgfqpoint{5.458396in}{1.185221in}}%
\pgfpathcurveto{\pgfqpoint{5.458396in}{1.192354in}}{\pgfqpoint{5.455562in}{1.199195in}}{\pgfqpoint{5.450518in}{1.204239in}}%
\pgfpathcurveto{\pgfqpoint{5.445474in}{1.209283in}}{\pgfqpoint{5.438633in}{1.212116in}}{\pgfqpoint{5.431500in}{1.212116in}}%
\pgfpathcurveto{\pgfqpoint{5.424367in}{1.212116in}}{\pgfqpoint{5.417525in}{1.209283in}}{\pgfqpoint{5.412482in}{1.204239in}}%
\pgfpathcurveto{\pgfqpoint{5.407438in}{1.199195in}}{\pgfqpoint{5.404604in}{1.192354in}}{\pgfqpoint{5.404604in}{1.185221in}}%
\pgfpathcurveto{\pgfqpoint{5.404604in}{1.178088in}}{\pgfqpoint{5.407438in}{1.171246in}}{\pgfqpoint{5.412482in}{1.166203in}}%
\pgfpathcurveto{\pgfqpoint{5.417525in}{1.161159in}}{\pgfqpoint{5.424367in}{1.158325in}}{\pgfqpoint{5.431500in}{1.158325in}}%
\pgfpathclose%
\pgfusepath{stroke,fill}%
\end{pgfscope}%
\begin{pgfscope}%
\pgfpathrectangle{\pgfqpoint{4.985294in}{0.500000in}}{\pgfqpoint{1.764706in}{1.700000in}}%
\pgfusepath{clip}%
\pgfsetbuttcap%
\pgfsetroundjoin%
\definecolor{currentfill}{rgb}{0.981377,0.920617,0.865369}%
\pgfsetfillcolor{currentfill}%
\pgfsetlinewidth{0.311001pt}%
\definecolor{currentstroke}{rgb}{1.000000,1.000000,1.000000}%
\pgfsetstrokecolor{currentstroke}%
\pgfsetdash{}{0pt}%
\pgfpathmoveto{\pgfqpoint{6.307696in}{1.473757in}}%
\pgfpathcurveto{\pgfqpoint{6.314829in}{1.473757in}}{\pgfqpoint{6.321671in}{1.476591in}}{\pgfqpoint{6.326715in}{1.481635in}}%
\pgfpathcurveto{\pgfqpoint{6.331758in}{1.486678in}}{\pgfqpoint{6.334592in}{1.493520in}}{\pgfqpoint{6.334592in}{1.500653in}}%
\pgfpathcurveto{\pgfqpoint{6.334592in}{1.507786in}}{\pgfqpoint{6.331758in}{1.514627in}}{\pgfqpoint{6.326715in}{1.519671in}}%
\pgfpathcurveto{\pgfqpoint{6.321671in}{1.524715in}}{\pgfqpoint{6.314829in}{1.527549in}}{\pgfqpoint{6.307696in}{1.527549in}}%
\pgfpathcurveto{\pgfqpoint{6.300564in}{1.527549in}}{\pgfqpoint{6.293722in}{1.524715in}}{\pgfqpoint{6.288678in}{1.519671in}}%
\pgfpathcurveto{\pgfqpoint{6.283635in}{1.514627in}}{\pgfqpoint{6.280801in}{1.507786in}}{\pgfqpoint{6.280801in}{1.500653in}}%
\pgfpathcurveto{\pgfqpoint{6.280801in}{1.493520in}}{\pgfqpoint{6.283635in}{1.486678in}}{\pgfqpoint{6.288678in}{1.481635in}}%
\pgfpathcurveto{\pgfqpoint{6.293722in}{1.476591in}}{\pgfqpoint{6.300564in}{1.473757in}}{\pgfqpoint{6.307696in}{1.473757in}}%
\pgfpathclose%
\pgfusepath{stroke,fill}%
\end{pgfscope}%
\begin{pgfscope}%
\pgfpathrectangle{\pgfqpoint{4.985294in}{0.500000in}}{\pgfqpoint{1.764706in}{1.700000in}}%
\pgfusepath{clip}%
\pgfsetbuttcap%
\pgfsetroundjoin%
\definecolor{currentfill}{rgb}{0.969359,0.803954,0.693832}%
\pgfsetfillcolor{currentfill}%
\pgfsetlinewidth{0.311001pt}%
\definecolor{currentstroke}{rgb}{1.000000,1.000000,1.000000}%
\pgfsetstrokecolor{currentstroke}%
\pgfsetdash{}{0pt}%
\pgfpathmoveto{\pgfqpoint{5.533558in}{1.082897in}}%
\pgfpathcurveto{\pgfqpoint{5.540691in}{1.082897in}}{\pgfqpoint{5.547533in}{1.085731in}}{\pgfqpoint{5.552576in}{1.090775in}}%
\pgfpathcurveto{\pgfqpoint{5.557620in}{1.095818in}}{\pgfqpoint{5.560454in}{1.102660in}}{\pgfqpoint{5.560454in}{1.109793in}}%
\pgfpathcurveto{\pgfqpoint{5.560454in}{1.116926in}}{\pgfqpoint{5.557620in}{1.123767in}}{\pgfqpoint{5.552576in}{1.128811in}}%
\pgfpathcurveto{\pgfqpoint{5.547533in}{1.133855in}}{\pgfqpoint{5.540691in}{1.136689in}}{\pgfqpoint{5.533558in}{1.136689in}}%
\pgfpathcurveto{\pgfqpoint{5.526425in}{1.136689in}}{\pgfqpoint{5.519584in}{1.133855in}}{\pgfqpoint{5.514540in}{1.128811in}}%
\pgfpathcurveto{\pgfqpoint{5.509496in}{1.123767in}}{\pgfqpoint{5.506663in}{1.116926in}}{\pgfqpoint{5.506663in}{1.109793in}}%
\pgfpathcurveto{\pgfqpoint{5.506663in}{1.102660in}}{\pgfqpoint{5.509496in}{1.095818in}}{\pgfqpoint{5.514540in}{1.090775in}}%
\pgfpathcurveto{\pgfqpoint{5.519584in}{1.085731in}}{\pgfqpoint{5.526425in}{1.082897in}}{\pgfqpoint{5.533558in}{1.082897in}}%
\pgfpathclose%
\pgfusepath{stroke,fill}%
\end{pgfscope}%
\begin{pgfscope}%
\pgfpathrectangle{\pgfqpoint{4.985294in}{0.500000in}}{\pgfqpoint{1.764706in}{1.700000in}}%
\pgfusepath{clip}%
\pgfsetbuttcap%
\pgfsetroundjoin%
\definecolor{currentfill}{rgb}{0.941676,0.367866,0.260395}%
\pgfsetfillcolor{currentfill}%
\pgfsetlinewidth{0.311001pt}%
\definecolor{currentstroke}{rgb}{1.000000,1.000000,1.000000}%
\pgfsetstrokecolor{currentstroke}%
\pgfsetdash{}{0pt}%
\pgfpathmoveto{\pgfqpoint{6.412349in}{1.056310in}}%
\pgfpathcurveto{\pgfqpoint{6.419482in}{1.056310in}}{\pgfqpoint{6.426324in}{1.059144in}}{\pgfqpoint{6.431368in}{1.064188in}}%
\pgfpathcurveto{\pgfqpoint{6.436411in}{1.069231in}}{\pgfqpoint{6.439245in}{1.076073in}}{\pgfqpoint{6.439245in}{1.083206in}}%
\pgfpathcurveto{\pgfqpoint{6.439245in}{1.090339in}}{\pgfqpoint{6.436411in}{1.097180in}}{\pgfqpoint{6.431368in}{1.102224in}}%
\pgfpathcurveto{\pgfqpoint{6.426324in}{1.107268in}}{\pgfqpoint{6.419482in}{1.110101in}}{\pgfqpoint{6.412349in}{1.110101in}}%
\pgfpathcurveto{\pgfqpoint{6.405217in}{1.110101in}}{\pgfqpoint{6.398375in}{1.107268in}}{\pgfqpoint{6.393331in}{1.102224in}}%
\pgfpathcurveto{\pgfqpoint{6.388288in}{1.097180in}}{\pgfqpoint{6.385454in}{1.090339in}}{\pgfqpoint{6.385454in}{1.083206in}}%
\pgfpathcurveto{\pgfqpoint{6.385454in}{1.076073in}}{\pgfqpoint{6.388288in}{1.069231in}}{\pgfqpoint{6.393331in}{1.064188in}}%
\pgfpathcurveto{\pgfqpoint{6.398375in}{1.059144in}}{\pgfqpoint{6.405217in}{1.056310in}}{\pgfqpoint{6.412349in}{1.056310in}}%
\pgfpathclose%
\pgfusepath{stroke,fill}%
\end{pgfscope}%
\begin{pgfscope}%
\pgfpathrectangle{\pgfqpoint{4.985294in}{0.500000in}}{\pgfqpoint{1.764706in}{1.700000in}}%
\pgfusepath{clip}%
\pgfsetbuttcap%
\pgfsetroundjoin%
\definecolor{currentfill}{rgb}{0.950851,0.435000,0.297228}%
\pgfsetfillcolor{currentfill}%
\pgfsetlinewidth{0.311001pt}%
\definecolor{currentstroke}{rgb}{1.000000,1.000000,1.000000}%
\pgfsetstrokecolor{currentstroke}%
\pgfsetdash{}{0pt}%
\pgfpathmoveto{\pgfqpoint{5.323554in}{1.049757in}}%
\pgfpathcurveto{\pgfqpoint{5.330687in}{1.049757in}}{\pgfqpoint{5.337528in}{1.052591in}}{\pgfqpoint{5.342572in}{1.057635in}}%
\pgfpathcurveto{\pgfqpoint{5.347616in}{1.062678in}}{\pgfqpoint{5.350450in}{1.069520in}}{\pgfqpoint{5.350450in}{1.076653in}}%
\pgfpathcurveto{\pgfqpoint{5.350450in}{1.083786in}}{\pgfqpoint{5.347616in}{1.090627in}}{\pgfqpoint{5.342572in}{1.095671in}}%
\pgfpathcurveto{\pgfqpoint{5.337528in}{1.100715in}}{\pgfqpoint{5.330687in}{1.103549in}}{\pgfqpoint{5.323554in}{1.103549in}}%
\pgfpathcurveto{\pgfqpoint{5.316421in}{1.103549in}}{\pgfqpoint{5.309579in}{1.100715in}}{\pgfqpoint{5.304536in}{1.095671in}}%
\pgfpathcurveto{\pgfqpoint{5.299492in}{1.090627in}}{\pgfqpoint{5.296658in}{1.083786in}}{\pgfqpoint{5.296658in}{1.076653in}}%
\pgfpathcurveto{\pgfqpoint{5.296658in}{1.069520in}}{\pgfqpoint{5.299492in}{1.062678in}}{\pgfqpoint{5.304536in}{1.057635in}}%
\pgfpathcurveto{\pgfqpoint{5.309579in}{1.052591in}}{\pgfqpoint{5.316421in}{1.049757in}}{\pgfqpoint{5.323554in}{1.049757in}}%
\pgfpathclose%
\pgfusepath{stroke,fill}%
\end{pgfscope}%
\begin{pgfscope}%
\pgfpathrectangle{\pgfqpoint{4.985294in}{0.500000in}}{\pgfqpoint{1.764706in}{1.700000in}}%
\pgfusepath{clip}%
\pgfsetbuttcap%
\pgfsetroundjoin%
\definecolor{currentfill}{rgb}{0.966328,0.750560,0.616961}%
\pgfsetfillcolor{currentfill}%
\pgfsetlinewidth{0.311001pt}%
\definecolor{currentstroke}{rgb}{1.000000,1.000000,1.000000}%
\pgfsetstrokecolor{currentstroke}%
\pgfsetdash{}{0pt}%
\pgfpathmoveto{\pgfqpoint{5.518651in}{1.197653in}}%
\pgfpathcurveto{\pgfqpoint{5.525783in}{1.197653in}}{\pgfqpoint{5.532625in}{1.200487in}}{\pgfqpoint{5.537669in}{1.205530in}}%
\pgfpathcurveto{\pgfqpoint{5.542712in}{1.210574in}}{\pgfqpoint{5.545546in}{1.217416in}}{\pgfqpoint{5.545546in}{1.224548in}}%
\pgfpathcurveto{\pgfqpoint{5.545546in}{1.231681in}}{\pgfqpoint{5.542712in}{1.238523in}}{\pgfqpoint{5.537669in}{1.243567in}}%
\pgfpathcurveto{\pgfqpoint{5.532625in}{1.248610in}}{\pgfqpoint{5.525783in}{1.251444in}}{\pgfqpoint{5.518651in}{1.251444in}}%
\pgfpathcurveto{\pgfqpoint{5.511518in}{1.251444in}}{\pgfqpoint{5.504676in}{1.248610in}}{\pgfqpoint{5.499632in}{1.243567in}}%
\pgfpathcurveto{\pgfqpoint{5.494589in}{1.238523in}}{\pgfqpoint{5.491755in}{1.231681in}}{\pgfqpoint{5.491755in}{1.224548in}}%
\pgfpathcurveto{\pgfqpoint{5.491755in}{1.217416in}}{\pgfqpoint{5.494589in}{1.210574in}}{\pgfqpoint{5.499632in}{1.205530in}}%
\pgfpathcurveto{\pgfqpoint{5.504676in}{1.200487in}}{\pgfqpoint{5.511518in}{1.197653in}}{\pgfqpoint{5.518651in}{1.197653in}}%
\pgfpathclose%
\pgfusepath{stroke,fill}%
\end{pgfscope}%
\begin{pgfscope}%
\pgfpathrectangle{\pgfqpoint{4.985294in}{0.500000in}}{\pgfqpoint{1.764706in}{1.700000in}}%
\pgfusepath{clip}%
\pgfsetbuttcap%
\pgfsetroundjoin%
\definecolor{currentfill}{rgb}{0.966120,0.744512,0.608720}%
\pgfsetfillcolor{currentfill}%
\pgfsetlinewidth{0.311001pt}%
\definecolor{currentstroke}{rgb}{1.000000,1.000000,1.000000}%
\pgfsetstrokecolor{currentstroke}%
\pgfsetdash{}{0pt}%
\pgfpathmoveto{\pgfqpoint{5.557238in}{1.539886in}}%
\pgfpathcurveto{\pgfqpoint{5.564371in}{1.539886in}}{\pgfqpoint{5.571213in}{1.542720in}}{\pgfqpoint{5.576256in}{1.547763in}}%
\pgfpathcurveto{\pgfqpoint{5.581300in}{1.552807in}}{\pgfqpoint{5.584134in}{1.559649in}}{\pgfqpoint{5.584134in}{1.566781in}}%
\pgfpathcurveto{\pgfqpoint{5.584134in}{1.573914in}}{\pgfqpoint{5.581300in}{1.580756in}}{\pgfqpoint{5.576256in}{1.585800in}}%
\pgfpathcurveto{\pgfqpoint{5.571213in}{1.590843in}}{\pgfqpoint{5.564371in}{1.593677in}}{\pgfqpoint{5.557238in}{1.593677in}}%
\pgfpathcurveto{\pgfqpoint{5.550105in}{1.593677in}}{\pgfqpoint{5.543264in}{1.590843in}}{\pgfqpoint{5.538220in}{1.585800in}}%
\pgfpathcurveto{\pgfqpoint{5.533176in}{1.580756in}}{\pgfqpoint{5.530342in}{1.573914in}}{\pgfqpoint{5.530342in}{1.566781in}}%
\pgfpathcurveto{\pgfqpoint{5.530342in}{1.559649in}}{\pgfqpoint{5.533176in}{1.552807in}}{\pgfqpoint{5.538220in}{1.547763in}}%
\pgfpathcurveto{\pgfqpoint{5.543264in}{1.542720in}}{\pgfqpoint{5.550105in}{1.539886in}}{\pgfqpoint{5.557238in}{1.539886in}}%
\pgfpathclose%
\pgfusepath{stroke,fill}%
\end{pgfscope}%
\begin{pgfscope}%
\pgfpathrectangle{\pgfqpoint{4.985294in}{0.500000in}}{\pgfqpoint{1.764706in}{1.700000in}}%
\pgfusepath{clip}%
\pgfsetbuttcap%
\pgfsetroundjoin%
\definecolor{currentfill}{rgb}{0.969359,0.803954,0.693832}%
\pgfsetfillcolor{currentfill}%
\pgfsetlinewidth{0.311001pt}%
\definecolor{currentstroke}{rgb}{1.000000,1.000000,1.000000}%
\pgfsetstrokecolor{currentstroke}%
\pgfsetdash{}{0pt}%
\pgfpathmoveto{\pgfqpoint{5.349371in}{1.355204in}}%
\pgfpathcurveto{\pgfqpoint{5.356503in}{1.355204in}}{\pgfqpoint{5.363345in}{1.358038in}}{\pgfqpoint{5.368389in}{1.363082in}}%
\pgfpathcurveto{\pgfqpoint{5.373432in}{1.368126in}}{\pgfqpoint{5.376266in}{1.374967in}}{\pgfqpoint{5.376266in}{1.382100in}}%
\pgfpathcurveto{\pgfqpoint{5.376266in}{1.389233in}}{\pgfqpoint{5.373432in}{1.396075in}}{\pgfqpoint{5.368389in}{1.401118in}}%
\pgfpathcurveto{\pgfqpoint{5.363345in}{1.406162in}}{\pgfqpoint{5.356503in}{1.408996in}}{\pgfqpoint{5.349371in}{1.408996in}}%
\pgfpathcurveto{\pgfqpoint{5.342238in}{1.408996in}}{\pgfqpoint{5.335396in}{1.406162in}}{\pgfqpoint{5.330352in}{1.401118in}}%
\pgfpathcurveto{\pgfqpoint{5.325309in}{1.396075in}}{\pgfqpoint{5.322475in}{1.389233in}}{\pgfqpoint{5.322475in}{1.382100in}}%
\pgfpathcurveto{\pgfqpoint{5.322475in}{1.374967in}}{\pgfqpoint{5.325309in}{1.368126in}}{\pgfqpoint{5.330352in}{1.363082in}}%
\pgfpathcurveto{\pgfqpoint{5.335396in}{1.358038in}}{\pgfqpoint{5.342238in}{1.355204in}}{\pgfqpoint{5.349371in}{1.355204in}}%
\pgfpathclose%
\pgfusepath{stroke,fill}%
\end{pgfscope}%
\begin{pgfscope}%
\pgfpathrectangle{\pgfqpoint{4.985294in}{0.500000in}}{\pgfqpoint{1.764706in}{1.700000in}}%
\pgfusepath{clip}%
\pgfsetbuttcap%
\pgfsetroundjoin%
\definecolor{currentfill}{rgb}{0.976287,0.879862,0.805788}%
\pgfsetfillcolor{currentfill}%
\pgfsetlinewidth{0.311001pt}%
\definecolor{currentstroke}{rgb}{1.000000,1.000000,1.000000}%
\pgfsetstrokecolor{currentstroke}%
\pgfsetdash{}{0pt}%
\pgfpathmoveto{\pgfqpoint{5.424504in}{1.505320in}}%
\pgfpathcurveto{\pgfqpoint{5.431637in}{1.505320in}}{\pgfqpoint{5.438479in}{1.508154in}}{\pgfqpoint{5.443522in}{1.513198in}}%
\pgfpathcurveto{\pgfqpoint{5.448566in}{1.518242in}}{\pgfqpoint{5.451400in}{1.525083in}}{\pgfqpoint{5.451400in}{1.532216in}}%
\pgfpathcurveto{\pgfqpoint{5.451400in}{1.539349in}}{\pgfqpoint{5.448566in}{1.546190in}}{\pgfqpoint{5.443522in}{1.551234in}}%
\pgfpathcurveto{\pgfqpoint{5.438479in}{1.556278in}}{\pgfqpoint{5.431637in}{1.559112in}}{\pgfqpoint{5.424504in}{1.559112in}}%
\pgfpathcurveto{\pgfqpoint{5.417371in}{1.559112in}}{\pgfqpoint{5.410530in}{1.556278in}}{\pgfqpoint{5.405486in}{1.551234in}}%
\pgfpathcurveto{\pgfqpoint{5.400442in}{1.546190in}}{\pgfqpoint{5.397609in}{1.539349in}}{\pgfqpoint{5.397609in}{1.532216in}}%
\pgfpathcurveto{\pgfqpoint{5.397609in}{1.525083in}}{\pgfqpoint{5.400442in}{1.518242in}}{\pgfqpoint{5.405486in}{1.513198in}}%
\pgfpathcurveto{\pgfqpoint{5.410530in}{1.508154in}}{\pgfqpoint{5.417371in}{1.505320in}}{\pgfqpoint{5.424504in}{1.505320in}}%
\pgfpathclose%
\pgfusepath{stroke,fill}%
\end{pgfscope}%
\begin{pgfscope}%
\pgfpathrectangle{\pgfqpoint{4.985294in}{0.500000in}}{\pgfqpoint{1.764706in}{1.700000in}}%
\pgfusepath{clip}%
\pgfsetbuttcap%
\pgfsetroundjoin%
\definecolor{currentfill}{rgb}{0.964032,0.651225,0.493258}%
\pgfsetfillcolor{currentfill}%
\pgfsetlinewidth{0.311001pt}%
\definecolor{currentstroke}{rgb}{1.000000,1.000000,1.000000}%
\pgfsetstrokecolor{currentstroke}%
\pgfsetdash{}{0pt}%
\pgfpathmoveto{\pgfqpoint{6.421152in}{1.304494in}}%
\pgfpathcurveto{\pgfqpoint{6.428284in}{1.304494in}}{\pgfqpoint{6.435126in}{1.307328in}}{\pgfqpoint{6.440170in}{1.312371in}}%
\pgfpathcurveto{\pgfqpoint{6.445213in}{1.317415in}}{\pgfqpoint{6.448047in}{1.324257in}}{\pgfqpoint{6.448047in}{1.331390in}}%
\pgfpathcurveto{\pgfqpoint{6.448047in}{1.338522in}}{\pgfqpoint{6.445213in}{1.345364in}}{\pgfqpoint{6.440170in}{1.350408in}}%
\pgfpathcurveto{\pgfqpoint{6.435126in}{1.355451in}}{\pgfqpoint{6.428284in}{1.358285in}}{\pgfqpoint{6.421152in}{1.358285in}}%
\pgfpathcurveto{\pgfqpoint{6.414019in}{1.358285in}}{\pgfqpoint{6.407177in}{1.355451in}}{\pgfqpoint{6.402134in}{1.350408in}}%
\pgfpathcurveto{\pgfqpoint{6.397090in}{1.345364in}}{\pgfqpoint{6.394256in}{1.338522in}}{\pgfqpoint{6.394256in}{1.331390in}}%
\pgfpathcurveto{\pgfqpoint{6.394256in}{1.324257in}}{\pgfqpoint{6.397090in}{1.317415in}}{\pgfqpoint{6.402134in}{1.312371in}}%
\pgfpathcurveto{\pgfqpoint{6.407177in}{1.307328in}}{\pgfqpoint{6.414019in}{1.304494in}}{\pgfqpoint{6.421152in}{1.304494in}}%
\pgfpathclose%
\pgfusepath{stroke,fill}%
\end{pgfscope}%
\begin{pgfscope}%
\pgfpathrectangle{\pgfqpoint{4.985294in}{0.500000in}}{\pgfqpoint{1.764706in}{1.700000in}}%
\pgfusepath{clip}%
\pgfsetbuttcap%
\pgfsetroundjoin%
\definecolor{currentfill}{rgb}{0.979891,0.908948,0.848279}%
\pgfsetfillcolor{currentfill}%
\pgfsetlinewidth{0.311001pt}%
\definecolor{currentstroke}{rgb}{1.000000,1.000000,1.000000}%
\pgfsetstrokecolor{currentstroke}%
\pgfsetdash{}{0pt}%
\pgfpathmoveto{\pgfqpoint{5.409766in}{1.251967in}}%
\pgfpathcurveto{\pgfqpoint{5.416899in}{1.251967in}}{\pgfqpoint{5.423741in}{1.254801in}}{\pgfqpoint{5.428784in}{1.259845in}}%
\pgfpathcurveto{\pgfqpoint{5.433828in}{1.264888in}}{\pgfqpoint{5.436662in}{1.271730in}}{\pgfqpoint{5.436662in}{1.278863in}}%
\pgfpathcurveto{\pgfqpoint{5.436662in}{1.285996in}}{\pgfqpoint{5.433828in}{1.292837in}}{\pgfqpoint{5.428784in}{1.297881in}}%
\pgfpathcurveto{\pgfqpoint{5.423741in}{1.302925in}}{\pgfqpoint{5.416899in}{1.305758in}}{\pgfqpoint{5.409766in}{1.305758in}}%
\pgfpathcurveto{\pgfqpoint{5.402633in}{1.305758in}}{\pgfqpoint{5.395792in}{1.302925in}}{\pgfqpoint{5.390748in}{1.297881in}}%
\pgfpathcurveto{\pgfqpoint{5.385704in}{1.292837in}}{\pgfqpoint{5.382870in}{1.285996in}}{\pgfqpoint{5.382870in}{1.278863in}}%
\pgfpathcurveto{\pgfqpoint{5.382870in}{1.271730in}}{\pgfqpoint{5.385704in}{1.264888in}}{\pgfqpoint{5.390748in}{1.259845in}}%
\pgfpathcurveto{\pgfqpoint{5.395792in}{1.254801in}}{\pgfqpoint{5.402633in}{1.251967in}}{\pgfqpoint{5.409766in}{1.251967in}}%
\pgfpathclose%
\pgfusepath{stroke,fill}%
\end{pgfscope}%
\begin{pgfscope}%
\pgfpathrectangle{\pgfqpoint{4.985294in}{0.500000in}}{\pgfqpoint{1.764706in}{1.700000in}}%
\pgfusepath{clip}%
\pgfsetbuttcap%
\pgfsetroundjoin%
\definecolor{currentfill}{rgb}{0.977657,0.891500,0.822809}%
\pgfsetfillcolor{currentfill}%
\pgfsetlinewidth{0.311001pt}%
\definecolor{currentstroke}{rgb}{1.000000,1.000000,1.000000}%
\pgfsetstrokecolor{currentstroke}%
\pgfsetdash{}{0pt}%
\pgfpathmoveto{\pgfqpoint{6.353332in}{1.309970in}}%
\pgfpathcurveto{\pgfqpoint{6.360465in}{1.309970in}}{\pgfqpoint{6.367307in}{1.312803in}}{\pgfqpoint{6.372351in}{1.317847in}}%
\pgfpathcurveto{\pgfqpoint{6.377394in}{1.322891in}}{\pgfqpoint{6.380228in}{1.329732in}}{\pgfqpoint{6.380228in}{1.336865in}}%
\pgfpathcurveto{\pgfqpoint{6.380228in}{1.343998in}}{\pgfqpoint{6.377394in}{1.350840in}}{\pgfqpoint{6.372351in}{1.355883in}}%
\pgfpathcurveto{\pgfqpoint{6.367307in}{1.360927in}}{\pgfqpoint{6.360465in}{1.363761in}}{\pgfqpoint{6.353332in}{1.363761in}}%
\pgfpathcurveto{\pgfqpoint{6.346200in}{1.363761in}}{\pgfqpoint{6.339358in}{1.360927in}}{\pgfqpoint{6.334314in}{1.355883in}}%
\pgfpathcurveto{\pgfqpoint{6.329271in}{1.350840in}}{\pgfqpoint{6.326437in}{1.343998in}}{\pgfqpoint{6.326437in}{1.336865in}}%
\pgfpathcurveto{\pgfqpoint{6.326437in}{1.329732in}}{\pgfqpoint{6.329271in}{1.322891in}}{\pgfqpoint{6.334314in}{1.317847in}}%
\pgfpathcurveto{\pgfqpoint{6.339358in}{1.312803in}}{\pgfqpoint{6.346200in}{1.309970in}}{\pgfqpoint{6.353332in}{1.309970in}}%
\pgfpathclose%
\pgfusepath{stroke,fill}%
\end{pgfscope}%
\begin{pgfscope}%
\pgfpathrectangle{\pgfqpoint{4.985294in}{0.500000in}}{\pgfqpoint{1.764706in}{1.700000in}}%
\pgfusepath{clip}%
\pgfsetbuttcap%
\pgfsetroundjoin%
\definecolor{currentfill}{rgb}{0.962765,0.606121,0.444717}%
\pgfsetfillcolor{currentfill}%
\pgfsetlinewidth{0.311001pt}%
\definecolor{currentstroke}{rgb}{1.000000,1.000000,1.000000}%
\pgfsetstrokecolor{currentstroke}%
\pgfsetdash{}{0pt}%
\pgfpathmoveto{\pgfqpoint{5.616586in}{1.047006in}}%
\pgfpathcurveto{\pgfqpoint{5.623719in}{1.047006in}}{\pgfqpoint{5.630561in}{1.049840in}}{\pgfqpoint{5.635604in}{1.054884in}}%
\pgfpathcurveto{\pgfqpoint{5.640648in}{1.059928in}}{\pgfqpoint{5.643482in}{1.066769in}}{\pgfqpoint{5.643482in}{1.073902in}}%
\pgfpathcurveto{\pgfqpoint{5.643482in}{1.081035in}}{\pgfqpoint{5.640648in}{1.087877in}}{\pgfqpoint{5.635604in}{1.092920in}}%
\pgfpathcurveto{\pgfqpoint{5.630561in}{1.097964in}}{\pgfqpoint{5.623719in}{1.100798in}}{\pgfqpoint{5.616586in}{1.100798in}}%
\pgfpathcurveto{\pgfqpoint{5.609453in}{1.100798in}}{\pgfqpoint{5.602612in}{1.097964in}}{\pgfqpoint{5.597568in}{1.092920in}}%
\pgfpathcurveto{\pgfqpoint{5.592524in}{1.087877in}}{\pgfqpoint{5.589690in}{1.081035in}}{\pgfqpoint{5.589690in}{1.073902in}}%
\pgfpathcurveto{\pgfqpoint{5.589690in}{1.066769in}}{\pgfqpoint{5.592524in}{1.059928in}}{\pgfqpoint{5.597568in}{1.054884in}}%
\pgfpathcurveto{\pgfqpoint{5.602612in}{1.049840in}}{\pgfqpoint{5.609453in}{1.047006in}}{\pgfqpoint{5.616586in}{1.047006in}}%
\pgfpathclose%
\pgfusepath{stroke,fill}%
\end{pgfscope}%
\begin{pgfscope}%
\pgfpathrectangle{\pgfqpoint{4.985294in}{0.500000in}}{\pgfqpoint{1.764706in}{1.700000in}}%
\pgfusepath{clip}%
\pgfsetbuttcap%
\pgfsetroundjoin%
\definecolor{currentfill}{rgb}{0.968931,0.798091,0.685123}%
\pgfsetfillcolor{currentfill}%
\pgfsetlinewidth{0.311001pt}%
\definecolor{currentstroke}{rgb}{1.000000,1.000000,1.000000}%
\pgfsetstrokecolor{currentstroke}%
\pgfsetdash{}{0pt}%
\pgfpathmoveto{\pgfqpoint{5.379352in}{1.129085in}}%
\pgfpathcurveto{\pgfqpoint{5.386484in}{1.129085in}}{\pgfqpoint{5.393326in}{1.131919in}}{\pgfqpoint{5.398370in}{1.136963in}}%
\pgfpathcurveto{\pgfqpoint{5.403413in}{1.142006in}}{\pgfqpoint{5.406247in}{1.148848in}}{\pgfqpoint{5.406247in}{1.155981in}}%
\pgfpathcurveto{\pgfqpoint{5.406247in}{1.163114in}}{\pgfqpoint{5.403413in}{1.169955in}}{\pgfqpoint{5.398370in}{1.174999in}}%
\pgfpathcurveto{\pgfqpoint{5.393326in}{1.180043in}}{\pgfqpoint{5.386484in}{1.182877in}}{\pgfqpoint{5.379352in}{1.182877in}}%
\pgfpathcurveto{\pgfqpoint{5.372219in}{1.182877in}}{\pgfqpoint{5.365377in}{1.180043in}}{\pgfqpoint{5.360333in}{1.174999in}}%
\pgfpathcurveto{\pgfqpoint{5.355290in}{1.169955in}}{\pgfqpoint{5.352456in}{1.163114in}}{\pgfqpoint{5.352456in}{1.155981in}}%
\pgfpathcurveto{\pgfqpoint{5.352456in}{1.148848in}}{\pgfqpoint{5.355290in}{1.142006in}}{\pgfqpoint{5.360333in}{1.136963in}}%
\pgfpathcurveto{\pgfqpoint{5.365377in}{1.131919in}}{\pgfqpoint{5.372219in}{1.129085in}}{\pgfqpoint{5.379352in}{1.129085in}}%
\pgfpathclose%
\pgfusepath{stroke,fill}%
\end{pgfscope}%
\begin{pgfscope}%
\pgfpathrectangle{\pgfqpoint{4.985294in}{0.500000in}}{\pgfqpoint{1.764706in}{1.700000in}}%
\pgfusepath{clip}%
\pgfsetbuttcap%
\pgfsetroundjoin%
\definecolor{currentfill}{rgb}{0.958791,0.526283,0.368909}%
\pgfsetfillcolor{currentfill}%
\pgfsetlinewidth{0.311001pt}%
\definecolor{currentstroke}{rgb}{1.000000,1.000000,1.000000}%
\pgfsetstrokecolor{currentstroke}%
\pgfsetdash{}{0pt}%
\pgfpathmoveto{\pgfqpoint{6.438934in}{1.245297in}}%
\pgfpathcurveto{\pgfqpoint{6.446067in}{1.245297in}}{\pgfqpoint{6.452908in}{1.248131in}}{\pgfqpoint{6.457952in}{1.253175in}}%
\pgfpathcurveto{\pgfqpoint{6.462996in}{1.258218in}}{\pgfqpoint{6.465830in}{1.265060in}}{\pgfqpoint{6.465830in}{1.272193in}}%
\pgfpathcurveto{\pgfqpoint{6.465830in}{1.279326in}}{\pgfqpoint{6.462996in}{1.286167in}}{\pgfqpoint{6.457952in}{1.291211in}}%
\pgfpathcurveto{\pgfqpoint{6.452908in}{1.296255in}}{\pgfqpoint{6.446067in}{1.299089in}}{\pgfqpoint{6.438934in}{1.299089in}}%
\pgfpathcurveto{\pgfqpoint{6.431801in}{1.299089in}}{\pgfqpoint{6.424960in}{1.296255in}}{\pgfqpoint{6.419916in}{1.291211in}}%
\pgfpathcurveto{\pgfqpoint{6.414872in}{1.286167in}}{\pgfqpoint{6.412038in}{1.279326in}}{\pgfqpoint{6.412038in}{1.272193in}}%
\pgfpathcurveto{\pgfqpoint{6.412038in}{1.265060in}}{\pgfqpoint{6.414872in}{1.258218in}}{\pgfqpoint{6.419916in}{1.253175in}}%
\pgfpathcurveto{\pgfqpoint{6.424960in}{1.248131in}}{\pgfqpoint{6.431801in}{1.245297in}}{\pgfqpoint{6.438934in}{1.245297in}}%
\pgfpathclose%
\pgfusepath{stroke,fill}%
\end{pgfscope}%
\begin{pgfscope}%
\pgfpathrectangle{\pgfqpoint{4.985294in}{0.500000in}}{\pgfqpoint{1.764706in}{1.700000in}}%
\pgfusepath{clip}%
\pgfsetbuttcap%
\pgfsetroundjoin%
\definecolor{currentfill}{rgb}{0.975018,0.868213,0.788710}%
\pgfsetfillcolor{currentfill}%
\pgfsetlinewidth{0.311001pt}%
\definecolor{currentstroke}{rgb}{1.000000,1.000000,1.000000}%
\pgfsetstrokecolor{currentstroke}%
\pgfsetdash{}{0pt}%
\pgfpathmoveto{\pgfqpoint{6.262817in}{1.276075in}}%
\pgfpathcurveto{\pgfqpoint{6.269950in}{1.276075in}}{\pgfqpoint{6.276792in}{1.278909in}}{\pgfqpoint{6.281835in}{1.283953in}}%
\pgfpathcurveto{\pgfqpoint{6.286879in}{1.288997in}}{\pgfqpoint{6.289713in}{1.295838in}}{\pgfqpoint{6.289713in}{1.302971in}}%
\pgfpathcurveto{\pgfqpoint{6.289713in}{1.310104in}}{\pgfqpoint{6.286879in}{1.316945in}}{\pgfqpoint{6.281835in}{1.321989in}}%
\pgfpathcurveto{\pgfqpoint{6.276792in}{1.327033in}}{\pgfqpoint{6.269950in}{1.329867in}}{\pgfqpoint{6.262817in}{1.329867in}}%
\pgfpathcurveto{\pgfqpoint{6.255684in}{1.329867in}}{\pgfqpoint{6.248843in}{1.327033in}}{\pgfqpoint{6.243799in}{1.321989in}}%
\pgfpathcurveto{\pgfqpoint{6.238755in}{1.316945in}}{\pgfqpoint{6.235922in}{1.310104in}}{\pgfqpoint{6.235922in}{1.302971in}}%
\pgfpathcurveto{\pgfqpoint{6.235922in}{1.295838in}}{\pgfqpoint{6.238755in}{1.288997in}}{\pgfqpoint{6.243799in}{1.283953in}}%
\pgfpathcurveto{\pgfqpoint{6.248843in}{1.278909in}}{\pgfqpoint{6.255684in}{1.276075in}}{\pgfqpoint{6.262817in}{1.276075in}}%
\pgfpathclose%
\pgfusepath{stroke,fill}%
\end{pgfscope}%
\begin{pgfscope}%
\pgfpathrectangle{\pgfqpoint{4.985294in}{0.500000in}}{\pgfqpoint{1.764706in}{1.700000in}}%
\pgfusepath{clip}%
\pgfsetbuttcap%
\pgfsetroundjoin%
\definecolor{currentfill}{rgb}{0.954476,0.470822,0.323110}%
\pgfsetfillcolor{currentfill}%
\pgfsetlinewidth{0.311001pt}%
\definecolor{currentstroke}{rgb}{1.000000,1.000000,1.000000}%
\pgfsetstrokecolor{currentstroke}%
\pgfsetdash{}{0pt}%
\pgfpathmoveto{\pgfqpoint{6.054725in}{1.734462in}}%
\pgfpathcurveto{\pgfqpoint{6.061858in}{1.734462in}}{\pgfqpoint{6.068699in}{1.737296in}}{\pgfqpoint{6.073743in}{1.742339in}}%
\pgfpathcurveto{\pgfqpoint{6.078787in}{1.747383in}}{\pgfqpoint{6.081621in}{1.754225in}}{\pgfqpoint{6.081621in}{1.761357in}}%
\pgfpathcurveto{\pgfqpoint{6.081621in}{1.768490in}}{\pgfqpoint{6.078787in}{1.775332in}}{\pgfqpoint{6.073743in}{1.780376in}}%
\pgfpathcurveto{\pgfqpoint{6.068699in}{1.785419in}}{\pgfqpoint{6.061858in}{1.788253in}}{\pgfqpoint{6.054725in}{1.788253in}}%
\pgfpathcurveto{\pgfqpoint{6.047592in}{1.788253in}}{\pgfqpoint{6.040750in}{1.785419in}}{\pgfqpoint{6.035707in}{1.780376in}}%
\pgfpathcurveto{\pgfqpoint{6.030663in}{1.775332in}}{\pgfqpoint{6.027829in}{1.768490in}}{\pgfqpoint{6.027829in}{1.761357in}}%
\pgfpathcurveto{\pgfqpoint{6.027829in}{1.754225in}}{\pgfqpoint{6.030663in}{1.747383in}}{\pgfqpoint{6.035707in}{1.742339in}}%
\pgfpathcurveto{\pgfqpoint{6.040750in}{1.737296in}}{\pgfqpoint{6.047592in}{1.734462in}}{\pgfqpoint{6.054725in}{1.734462in}}%
\pgfpathclose%
\pgfusepath{stroke,fill}%
\end{pgfscope}%
\begin{pgfscope}%
\pgfpathrectangle{\pgfqpoint{4.985294in}{0.500000in}}{\pgfqpoint{1.764706in}{1.700000in}}%
\pgfusepath{clip}%
\pgfsetbuttcap%
\pgfsetroundjoin%
\definecolor{currentfill}{rgb}{0.968105,0.786346,0.667739}%
\pgfsetfillcolor{currentfill}%
\pgfsetlinewidth{0.311001pt}%
\definecolor{currentstroke}{rgb}{1.000000,1.000000,1.000000}%
\pgfsetstrokecolor{currentstroke}%
\pgfsetdash{}{0pt}%
\pgfpathmoveto{\pgfqpoint{5.537076in}{0.945709in}}%
\pgfpathcurveto{\pgfqpoint{5.544209in}{0.945709in}}{\pgfqpoint{5.551051in}{0.948542in}}{\pgfqpoint{5.556094in}{0.953586in}}%
\pgfpathcurveto{\pgfqpoint{5.561138in}{0.958630in}}{\pgfqpoint{5.563972in}{0.965471in}}{\pgfqpoint{5.563972in}{0.972604in}}%
\pgfpathcurveto{\pgfqpoint{5.563972in}{0.979737in}}{\pgfqpoint{5.561138in}{0.986579in}}{\pgfqpoint{5.556094in}{0.991622in}}%
\pgfpathcurveto{\pgfqpoint{5.551051in}{0.996666in}}{\pgfqpoint{5.544209in}{0.999500in}}{\pgfqpoint{5.537076in}{0.999500in}}%
\pgfpathcurveto{\pgfqpoint{5.529943in}{0.999500in}}{\pgfqpoint{5.523102in}{0.996666in}}{\pgfqpoint{5.518058in}{0.991622in}}%
\pgfpathcurveto{\pgfqpoint{5.513014in}{0.986579in}}{\pgfqpoint{5.510180in}{0.979737in}}{\pgfqpoint{5.510180in}{0.972604in}}%
\pgfpathcurveto{\pgfqpoint{5.510180in}{0.965471in}}{\pgfqpoint{5.513014in}{0.958630in}}{\pgfqpoint{5.518058in}{0.953586in}}%
\pgfpathcurveto{\pgfqpoint{5.523102in}{0.948542in}}{\pgfqpoint{5.529943in}{0.945709in}}{\pgfqpoint{5.537076in}{0.945709in}}%
\pgfpathclose%
\pgfusepath{stroke,fill}%
\end{pgfscope}%
\begin{pgfscope}%
\pgfpathrectangle{\pgfqpoint{4.985294in}{0.500000in}}{\pgfqpoint{1.764706in}{1.700000in}}%
\pgfusepath{clip}%
\pgfsetbuttcap%
\pgfsetroundjoin%
\definecolor{currentfill}{rgb}{0.971202,0.827364,0.728520}%
\pgfsetfillcolor{currentfill}%
\pgfsetlinewidth{0.311001pt}%
\definecolor{currentstroke}{rgb}{1.000000,1.000000,1.000000}%
\pgfsetstrokecolor{currentstroke}%
\pgfsetdash{}{0pt}%
\pgfpathmoveto{\pgfqpoint{5.426655in}{1.056569in}}%
\pgfpathcurveto{\pgfqpoint{5.433787in}{1.056569in}}{\pgfqpoint{5.440629in}{1.059403in}}{\pgfqpoint{5.445673in}{1.064447in}}%
\pgfpathcurveto{\pgfqpoint{5.450716in}{1.069491in}}{\pgfqpoint{5.453550in}{1.076332in}}{\pgfqpoint{5.453550in}{1.083465in}}%
\pgfpathcurveto{\pgfqpoint{5.453550in}{1.090598in}}{\pgfqpoint{5.450716in}{1.097439in}}{\pgfqpoint{5.445673in}{1.102483in}}%
\pgfpathcurveto{\pgfqpoint{5.440629in}{1.107527in}}{\pgfqpoint{5.433787in}{1.110361in}}{\pgfqpoint{5.426655in}{1.110361in}}%
\pgfpathcurveto{\pgfqpoint{5.419522in}{1.110361in}}{\pgfqpoint{5.412680in}{1.107527in}}{\pgfqpoint{5.407636in}{1.102483in}}%
\pgfpathcurveto{\pgfqpoint{5.402593in}{1.097439in}}{\pgfqpoint{5.399759in}{1.090598in}}{\pgfqpoint{5.399759in}{1.083465in}}%
\pgfpathcurveto{\pgfqpoint{5.399759in}{1.076332in}}{\pgfqpoint{5.402593in}{1.069491in}}{\pgfqpoint{5.407636in}{1.064447in}}%
\pgfpathcurveto{\pgfqpoint{5.412680in}{1.059403in}}{\pgfqpoint{5.419522in}{1.056569in}}{\pgfqpoint{5.426655in}{1.056569in}}%
\pgfpathclose%
\pgfusepath{stroke,fill}%
\end{pgfscope}%
\begin{pgfscope}%
\pgfpathrectangle{\pgfqpoint{4.985294in}{0.500000in}}{\pgfqpoint{1.764706in}{1.700000in}}%
\pgfusepath{clip}%
\pgfsetbuttcap%
\pgfsetroundjoin%
\definecolor{currentfill}{rgb}{0.924566,0.290534,0.242426}%
\pgfsetfillcolor{currentfill}%
\pgfsetlinewidth{0.311001pt}%
\definecolor{currentstroke}{rgb}{1.000000,1.000000,1.000000}%
\pgfsetstrokecolor{currentstroke}%
\pgfsetdash{}{0pt}%
\pgfpathmoveto{\pgfqpoint{5.640932in}{1.115449in}}%
\pgfpathcurveto{\pgfqpoint{5.648065in}{1.115449in}}{\pgfqpoint{5.654906in}{1.118283in}}{\pgfqpoint{5.659950in}{1.123327in}}%
\pgfpathcurveto{\pgfqpoint{5.664994in}{1.128370in}}{\pgfqpoint{5.667828in}{1.135212in}}{\pgfqpoint{5.667828in}{1.142345in}}%
\pgfpathcurveto{\pgfqpoint{5.667828in}{1.149478in}}{\pgfqpoint{5.664994in}{1.156319in}}{\pgfqpoint{5.659950in}{1.161363in}}%
\pgfpathcurveto{\pgfqpoint{5.654906in}{1.166407in}}{\pgfqpoint{5.648065in}{1.169240in}}{\pgfqpoint{5.640932in}{1.169240in}}%
\pgfpathcurveto{\pgfqpoint{5.633799in}{1.169240in}}{\pgfqpoint{5.626957in}{1.166407in}}{\pgfqpoint{5.621914in}{1.161363in}}%
\pgfpathcurveto{\pgfqpoint{5.616870in}{1.156319in}}{\pgfqpoint{5.614036in}{1.149478in}}{\pgfqpoint{5.614036in}{1.142345in}}%
\pgfpathcurveto{\pgfqpoint{5.614036in}{1.135212in}}{\pgfqpoint{5.616870in}{1.128370in}}{\pgfqpoint{5.621914in}{1.123327in}}%
\pgfpathcurveto{\pgfqpoint{5.626957in}{1.118283in}}{\pgfqpoint{5.633799in}{1.115449in}}{\pgfqpoint{5.640932in}{1.115449in}}%
\pgfpathclose%
\pgfusepath{stroke,fill}%
\end{pgfscope}%
\begin{pgfscope}%
\pgfpathrectangle{\pgfqpoint{4.985294in}{0.500000in}}{\pgfqpoint{1.764706in}{1.700000in}}%
\pgfusepath{clip}%
\pgfsetbuttcap%
\pgfsetroundjoin%
\definecolor{currentfill}{rgb}{0.964920,0.695342,0.545192}%
\pgfsetfillcolor{currentfill}%
\pgfsetlinewidth{0.311001pt}%
\definecolor{currentstroke}{rgb}{1.000000,1.000000,1.000000}%
\pgfsetstrokecolor{currentstroke}%
\pgfsetdash{}{0pt}%
\pgfpathmoveto{\pgfqpoint{6.335016in}{1.028820in}}%
\pgfpathcurveto{\pgfqpoint{6.342149in}{1.028820in}}{\pgfqpoint{6.348991in}{1.031653in}}{\pgfqpoint{6.354034in}{1.036697in}}%
\pgfpathcurveto{\pgfqpoint{6.359078in}{1.041741in}}{\pgfqpoint{6.361912in}{1.048582in}}{\pgfqpoint{6.361912in}{1.055715in}}%
\pgfpathcurveto{\pgfqpoint{6.361912in}{1.062848in}}{\pgfqpoint{6.359078in}{1.069690in}}{\pgfqpoint{6.354034in}{1.074733in}}%
\pgfpathcurveto{\pgfqpoint{6.348991in}{1.079777in}}{\pgfqpoint{6.342149in}{1.082611in}}{\pgfqpoint{6.335016in}{1.082611in}}%
\pgfpathcurveto{\pgfqpoint{6.327884in}{1.082611in}}{\pgfqpoint{6.321042in}{1.079777in}}{\pgfqpoint{6.315998in}{1.074733in}}%
\pgfpathcurveto{\pgfqpoint{6.310955in}{1.069690in}}{\pgfqpoint{6.308121in}{1.062848in}}{\pgfqpoint{6.308121in}{1.055715in}}%
\pgfpathcurveto{\pgfqpoint{6.308121in}{1.048582in}}{\pgfqpoint{6.310955in}{1.041741in}}{\pgfqpoint{6.315998in}{1.036697in}}%
\pgfpathcurveto{\pgfqpoint{6.321042in}{1.031653in}}{\pgfqpoint{6.327884in}{1.028820in}}{\pgfqpoint{6.335016in}{1.028820in}}%
\pgfpathclose%
\pgfusepath{stroke,fill}%
\end{pgfscope}%
\begin{pgfscope}%
\pgfpathrectangle{\pgfqpoint{4.985294in}{0.500000in}}{\pgfqpoint{1.764706in}{1.700000in}}%
\pgfusepath{clip}%
\pgfsetbuttcap%
\pgfsetroundjoin%
\definecolor{currentfill}{rgb}{0.959645,0.539840,0.380928}%
\pgfsetfillcolor{currentfill}%
\pgfsetlinewidth{0.311001pt}%
\definecolor{currentstroke}{rgb}{1.000000,1.000000,1.000000}%
\pgfsetstrokecolor{currentstroke}%
\pgfsetdash{}{0pt}%
\pgfpathmoveto{\pgfqpoint{6.438182in}{1.333965in}}%
\pgfpathcurveto{\pgfqpoint{6.445315in}{1.333965in}}{\pgfqpoint{6.452157in}{1.336799in}}{\pgfqpoint{6.457201in}{1.341843in}}%
\pgfpathcurveto{\pgfqpoint{6.462244in}{1.346886in}}{\pgfqpoint{6.465078in}{1.353728in}}{\pgfqpoint{6.465078in}{1.360861in}}%
\pgfpathcurveto{\pgfqpoint{6.465078in}{1.367994in}}{\pgfqpoint{6.462244in}{1.374835in}}{\pgfqpoint{6.457201in}{1.379879in}}%
\pgfpathcurveto{\pgfqpoint{6.452157in}{1.384923in}}{\pgfqpoint{6.445315in}{1.387757in}}{\pgfqpoint{6.438182in}{1.387757in}}%
\pgfpathcurveto{\pgfqpoint{6.431050in}{1.387757in}}{\pgfqpoint{6.424208in}{1.384923in}}{\pgfqpoint{6.419164in}{1.379879in}}%
\pgfpathcurveto{\pgfqpoint{6.414121in}{1.374835in}}{\pgfqpoint{6.411287in}{1.367994in}}{\pgfqpoint{6.411287in}{1.360861in}}%
\pgfpathcurveto{\pgfqpoint{6.411287in}{1.353728in}}{\pgfqpoint{6.414121in}{1.346886in}}{\pgfqpoint{6.419164in}{1.341843in}}%
\pgfpathcurveto{\pgfqpoint{6.424208in}{1.336799in}}{\pgfqpoint{6.431050in}{1.333965in}}{\pgfqpoint{6.438182in}{1.333965in}}%
\pgfpathclose%
\pgfusepath{stroke,fill}%
\end{pgfscope}%
\begin{pgfscope}%
\pgfpathrectangle{\pgfqpoint{4.985294in}{0.500000in}}{\pgfqpoint{1.764706in}{1.700000in}}%
\pgfusepath{clip}%
\pgfsetbuttcap%
\pgfsetroundjoin%
\definecolor{currentfill}{rgb}{0.971694,0.833208,0.737161}%
\pgfsetfillcolor{currentfill}%
\pgfsetlinewidth{0.311001pt}%
\definecolor{currentstroke}{rgb}{1.000000,1.000000,1.000000}%
\pgfsetstrokecolor{currentstroke}%
\pgfsetdash{}{0pt}%
\pgfpathmoveto{\pgfqpoint{6.225853in}{1.516426in}}%
\pgfpathcurveto{\pgfqpoint{6.232986in}{1.516426in}}{\pgfqpoint{6.239827in}{1.519259in}}{\pgfqpoint{6.244871in}{1.524303in}}%
\pgfpathcurveto{\pgfqpoint{6.249915in}{1.529347in}}{\pgfqpoint{6.252748in}{1.536188in}}{\pgfqpoint{6.252748in}{1.543321in}}%
\pgfpathcurveto{\pgfqpoint{6.252748in}{1.550454in}}{\pgfqpoint{6.249915in}{1.557296in}}{\pgfqpoint{6.244871in}{1.562339in}}%
\pgfpathcurveto{\pgfqpoint{6.239827in}{1.567383in}}{\pgfqpoint{6.232986in}{1.570217in}}{\pgfqpoint{6.225853in}{1.570217in}}%
\pgfpathcurveto{\pgfqpoint{6.218720in}{1.570217in}}{\pgfqpoint{6.211878in}{1.567383in}}{\pgfqpoint{6.206835in}{1.562339in}}%
\pgfpathcurveto{\pgfqpoint{6.201791in}{1.557296in}}{\pgfqpoint{6.198957in}{1.550454in}}{\pgfqpoint{6.198957in}{1.543321in}}%
\pgfpathcurveto{\pgfqpoint{6.198957in}{1.536188in}}{\pgfqpoint{6.201791in}{1.529347in}}{\pgfqpoint{6.206835in}{1.524303in}}%
\pgfpathcurveto{\pgfqpoint{6.211878in}{1.519259in}}{\pgfqpoint{6.218720in}{1.516426in}}{\pgfqpoint{6.225853in}{1.516426in}}%
\pgfpathclose%
\pgfusepath{stroke,fill}%
\end{pgfscope}%
\begin{pgfscope}%
\pgfpathrectangle{\pgfqpoint{4.985294in}{0.500000in}}{\pgfqpoint{1.764706in}{1.700000in}}%
\pgfusepath{clip}%
\pgfsetbuttcap%
\pgfsetroundjoin%
\definecolor{currentfill}{rgb}{0.964173,0.657587,0.500469}%
\pgfsetfillcolor{currentfill}%
\pgfsetlinewidth{0.311001pt}%
\definecolor{currentstroke}{rgb}{1.000000,1.000000,1.000000}%
\pgfsetstrokecolor{currentstroke}%
\pgfsetdash{}{0pt}%
\pgfpathmoveto{\pgfqpoint{6.119802in}{1.010590in}}%
\pgfpathcurveto{\pgfqpoint{6.126934in}{1.010590in}}{\pgfqpoint{6.133776in}{1.013424in}}{\pgfqpoint{6.138820in}{1.018468in}}%
\pgfpathcurveto{\pgfqpoint{6.143863in}{1.023511in}}{\pgfqpoint{6.146697in}{1.030353in}}{\pgfqpoint{6.146697in}{1.037486in}}%
\pgfpathcurveto{\pgfqpoint{6.146697in}{1.044619in}}{\pgfqpoint{6.143863in}{1.051460in}}{\pgfqpoint{6.138820in}{1.056504in}}%
\pgfpathcurveto{\pgfqpoint{6.133776in}{1.061548in}}{\pgfqpoint{6.126934in}{1.064382in}}{\pgfqpoint{6.119802in}{1.064382in}}%
\pgfpathcurveto{\pgfqpoint{6.112669in}{1.064382in}}{\pgfqpoint{6.105827in}{1.061548in}}{\pgfqpoint{6.100783in}{1.056504in}}%
\pgfpathcurveto{\pgfqpoint{6.095740in}{1.051460in}}{\pgfqpoint{6.092906in}{1.044619in}}{\pgfqpoint{6.092906in}{1.037486in}}%
\pgfpathcurveto{\pgfqpoint{6.092906in}{1.030353in}}{\pgfqpoint{6.095740in}{1.023511in}}{\pgfqpoint{6.100783in}{1.018468in}}%
\pgfpathcurveto{\pgfqpoint{6.105827in}{1.013424in}}{\pgfqpoint{6.112669in}{1.010590in}}{\pgfqpoint{6.119802in}{1.010590in}}%
\pgfpathclose%
\pgfusepath{stroke,fill}%
\end{pgfscope}%
\begin{pgfscope}%
\pgfpathrectangle{\pgfqpoint{4.985294in}{0.500000in}}{\pgfqpoint{1.764706in}{1.700000in}}%
\pgfusepath{clip}%
\pgfsetbuttcap%
\pgfsetroundjoin%
\definecolor{currentfill}{rgb}{0.962283,0.593046,0.431453}%
\pgfsetfillcolor{currentfill}%
\pgfsetlinewidth{0.311001pt}%
\definecolor{currentstroke}{rgb}{1.000000,1.000000,1.000000}%
\pgfsetstrokecolor{currentstroke}%
\pgfsetdash{}{0pt}%
\pgfpathmoveto{\pgfqpoint{6.346938in}{1.004890in}}%
\pgfpathcurveto{\pgfqpoint{6.354071in}{1.004890in}}{\pgfqpoint{6.360912in}{1.007723in}}{\pgfqpoint{6.365956in}{1.012767in}}%
\pgfpathcurveto{\pgfqpoint{6.371000in}{1.017811in}}{\pgfqpoint{6.373834in}{1.024652in}}{\pgfqpoint{6.373834in}{1.031785in}}%
\pgfpathcurveto{\pgfqpoint{6.373834in}{1.038918in}}{\pgfqpoint{6.371000in}{1.045760in}}{\pgfqpoint{6.365956in}{1.050803in}}%
\pgfpathcurveto{\pgfqpoint{6.360912in}{1.055847in}}{\pgfqpoint{6.354071in}{1.058681in}}{\pgfqpoint{6.346938in}{1.058681in}}%
\pgfpathcurveto{\pgfqpoint{6.339805in}{1.058681in}}{\pgfqpoint{6.332963in}{1.055847in}}{\pgfqpoint{6.327920in}{1.050803in}}%
\pgfpathcurveto{\pgfqpoint{6.322876in}{1.045760in}}{\pgfqpoint{6.320042in}{1.038918in}}{\pgfqpoint{6.320042in}{1.031785in}}%
\pgfpathcurveto{\pgfqpoint{6.320042in}{1.024652in}}{\pgfqpoint{6.322876in}{1.017811in}}{\pgfqpoint{6.327920in}{1.012767in}}%
\pgfpathcurveto{\pgfqpoint{6.332963in}{1.007723in}}{\pgfqpoint{6.339805in}{1.004890in}}{\pgfqpoint{6.346938in}{1.004890in}}%
\pgfpathclose%
\pgfusepath{stroke,fill}%
\end{pgfscope}%
\begin{pgfscope}%
\pgfpathrectangle{\pgfqpoint{4.985294in}{0.500000in}}{\pgfqpoint{1.764706in}{1.700000in}}%
\pgfusepath{clip}%
\pgfsetbuttcap%
\pgfsetroundjoin%
\definecolor{currentfill}{rgb}{0.964799,0.689101,0.537560}%
\pgfsetfillcolor{currentfill}%
\pgfsetlinewidth{0.311001pt}%
\definecolor{currentstroke}{rgb}{1.000000,1.000000,1.000000}%
\pgfsetstrokecolor{currentstroke}%
\pgfsetdash{}{0pt}%
\pgfpathmoveto{\pgfqpoint{6.345577in}{1.599717in}}%
\pgfpathcurveto{\pgfqpoint{6.352710in}{1.599717in}}{\pgfqpoint{6.359552in}{1.602551in}}{\pgfqpoint{6.364596in}{1.607594in}}%
\pgfpathcurveto{\pgfqpoint{6.369639in}{1.612638in}}{\pgfqpoint{6.372473in}{1.619480in}}{\pgfqpoint{6.372473in}{1.626613in}}%
\pgfpathcurveto{\pgfqpoint{6.372473in}{1.633745in}}{\pgfqpoint{6.369639in}{1.640587in}}{\pgfqpoint{6.364596in}{1.645631in}}%
\pgfpathcurveto{\pgfqpoint{6.359552in}{1.650674in}}{\pgfqpoint{6.352710in}{1.653508in}}{\pgfqpoint{6.345577in}{1.653508in}}%
\pgfpathcurveto{\pgfqpoint{6.338445in}{1.653508in}}{\pgfqpoint{6.331603in}{1.650674in}}{\pgfqpoint{6.326559in}{1.645631in}}%
\pgfpathcurveto{\pgfqpoint{6.321516in}{1.640587in}}{\pgfqpoint{6.318682in}{1.633745in}}{\pgfqpoint{6.318682in}{1.626613in}}%
\pgfpathcurveto{\pgfqpoint{6.318682in}{1.619480in}}{\pgfqpoint{6.321516in}{1.612638in}}{\pgfqpoint{6.326559in}{1.607594in}}%
\pgfpathcurveto{\pgfqpoint{6.331603in}{1.602551in}}{\pgfqpoint{6.338445in}{1.599717in}}{\pgfqpoint{6.345577in}{1.599717in}}%
\pgfpathclose%
\pgfusepath{stroke,fill}%
\end{pgfscope}%
\begin{pgfscope}%
\pgfpathrectangle{\pgfqpoint{4.985294in}{0.500000in}}{\pgfqpoint{1.764706in}{1.700000in}}%
\pgfusepath{clip}%
\pgfsetbuttcap%
\pgfsetroundjoin%
\definecolor{currentfill}{rgb}{0.965440,0.720101,0.576404}%
\pgfsetfillcolor{currentfill}%
\pgfsetlinewidth{0.311001pt}%
\definecolor{currentstroke}{rgb}{1.000000,1.000000,1.000000}%
\pgfsetstrokecolor{currentstroke}%
\pgfsetdash{}{0pt}%
\pgfpathmoveto{\pgfqpoint{6.132427in}{1.682604in}}%
\pgfpathcurveto{\pgfqpoint{6.139560in}{1.682604in}}{\pgfqpoint{6.146402in}{1.685438in}}{\pgfqpoint{6.151446in}{1.690482in}}%
\pgfpathcurveto{\pgfqpoint{6.156489in}{1.695526in}}{\pgfqpoint{6.159323in}{1.702367in}}{\pgfqpoint{6.159323in}{1.709500in}}%
\pgfpathcurveto{\pgfqpoint{6.159323in}{1.716633in}}{\pgfqpoint{6.156489in}{1.723474in}}{\pgfqpoint{6.151446in}{1.728518in}}%
\pgfpathcurveto{\pgfqpoint{6.146402in}{1.733562in}}{\pgfqpoint{6.139560in}{1.736396in}}{\pgfqpoint{6.132427in}{1.736396in}}%
\pgfpathcurveto{\pgfqpoint{6.125295in}{1.736396in}}{\pgfqpoint{6.118453in}{1.733562in}}{\pgfqpoint{6.113409in}{1.728518in}}%
\pgfpathcurveto{\pgfqpoint{6.108366in}{1.723474in}}{\pgfqpoint{6.105532in}{1.716633in}}{\pgfqpoint{6.105532in}{1.709500in}}%
\pgfpathcurveto{\pgfqpoint{6.105532in}{1.702367in}}{\pgfqpoint{6.108366in}{1.695526in}}{\pgfqpoint{6.113409in}{1.690482in}}%
\pgfpathcurveto{\pgfqpoint{6.118453in}{1.685438in}}{\pgfqpoint{6.125295in}{1.682604in}}{\pgfqpoint{6.132427in}{1.682604in}}%
\pgfpathclose%
\pgfusepath{stroke,fill}%
\end{pgfscope}%
\begin{pgfscope}%
\pgfpathrectangle{\pgfqpoint{4.985294in}{0.500000in}}{\pgfqpoint{1.764706in}{1.700000in}}%
\pgfusepath{clip}%
\pgfsetbuttcap%
\pgfsetroundjoin%
\definecolor{currentfill}{rgb}{0.969803,0.809811,0.702523}%
\pgfsetfillcolor{currentfill}%
\pgfsetlinewidth{0.311001pt}%
\definecolor{currentstroke}{rgb}{1.000000,1.000000,1.000000}%
\pgfsetstrokecolor{currentstroke}%
\pgfsetdash{}{0pt}%
\pgfpathmoveto{\pgfqpoint{5.352711in}{1.258156in}}%
\pgfpathcurveto{\pgfqpoint{5.359844in}{1.258156in}}{\pgfqpoint{5.366686in}{1.260990in}}{\pgfqpoint{5.371729in}{1.266034in}}%
\pgfpathcurveto{\pgfqpoint{5.376773in}{1.271078in}}{\pgfqpoint{5.379607in}{1.277919in}}{\pgfqpoint{5.379607in}{1.285052in}}%
\pgfpathcurveto{\pgfqpoint{5.379607in}{1.292185in}}{\pgfqpoint{5.376773in}{1.299027in}}{\pgfqpoint{5.371729in}{1.304070in}}%
\pgfpathcurveto{\pgfqpoint{5.366686in}{1.309114in}}{\pgfqpoint{5.359844in}{1.311948in}}{\pgfqpoint{5.352711in}{1.311948in}}%
\pgfpathcurveto{\pgfqpoint{5.345578in}{1.311948in}}{\pgfqpoint{5.338737in}{1.309114in}}{\pgfqpoint{5.333693in}{1.304070in}}%
\pgfpathcurveto{\pgfqpoint{5.328649in}{1.299027in}}{\pgfqpoint{5.325815in}{1.292185in}}{\pgfqpoint{5.325815in}{1.285052in}}%
\pgfpathcurveto{\pgfqpoint{5.325815in}{1.277919in}}{\pgfqpoint{5.328649in}{1.271078in}}{\pgfqpoint{5.333693in}{1.266034in}}%
\pgfpathcurveto{\pgfqpoint{5.338737in}{1.260990in}}{\pgfqpoint{5.345578in}{1.258156in}}{\pgfqpoint{5.352711in}{1.258156in}}%
\pgfpathclose%
\pgfusepath{stroke,fill}%
\end{pgfscope}%
\begin{pgfscope}%
\pgfpathrectangle{\pgfqpoint{4.985294in}{0.500000in}}{\pgfqpoint{1.764706in}{1.700000in}}%
\pgfusepath{clip}%
\pgfsetbuttcap%
\pgfsetroundjoin%
\definecolor{currentfill}{rgb}{0.947270,0.405591,0.279023}%
\pgfsetfillcolor{currentfill}%
\pgfsetlinewidth{0.311001pt}%
\definecolor{currentstroke}{rgb}{1.000000,1.000000,1.000000}%
\pgfsetstrokecolor{currentstroke}%
\pgfsetdash{}{0pt}%
\pgfpathmoveto{\pgfqpoint{6.454795in}{1.349071in}}%
\pgfpathcurveto{\pgfqpoint{6.461928in}{1.349071in}}{\pgfqpoint{6.468770in}{1.351905in}}{\pgfqpoint{6.473813in}{1.356948in}}%
\pgfpathcurveto{\pgfqpoint{6.478857in}{1.361992in}}{\pgfqpoint{6.481691in}{1.368834in}}{\pgfqpoint{6.481691in}{1.375966in}}%
\pgfpathcurveto{\pgfqpoint{6.481691in}{1.383099in}}{\pgfqpoint{6.478857in}{1.389941in}}{\pgfqpoint{6.473813in}{1.394985in}}%
\pgfpathcurveto{\pgfqpoint{6.468770in}{1.400028in}}{\pgfqpoint{6.461928in}{1.402862in}}{\pgfqpoint{6.454795in}{1.402862in}}%
\pgfpathcurveto{\pgfqpoint{6.447662in}{1.402862in}}{\pgfqpoint{6.440821in}{1.400028in}}{\pgfqpoint{6.435777in}{1.394985in}}%
\pgfpathcurveto{\pgfqpoint{6.430733in}{1.389941in}}{\pgfqpoint{6.427899in}{1.383099in}}{\pgfqpoint{6.427899in}{1.375966in}}%
\pgfpathcurveto{\pgfqpoint{6.427899in}{1.368834in}}{\pgfqpoint{6.430733in}{1.361992in}}{\pgfqpoint{6.435777in}{1.356948in}}%
\pgfpathcurveto{\pgfqpoint{6.440821in}{1.351905in}}{\pgfqpoint{6.447662in}{1.349071in}}{\pgfqpoint{6.454795in}{1.349071in}}%
\pgfpathclose%
\pgfusepath{stroke,fill}%
\end{pgfscope}%
\begin{pgfscope}%
\pgfpathrectangle{\pgfqpoint{4.985294in}{0.500000in}}{\pgfqpoint{1.764706in}{1.700000in}}%
\pgfusepath{clip}%
\pgfsetbuttcap%
\pgfsetroundjoin%
\definecolor{currentfill}{rgb}{0.970718,0.821518,0.719872}%
\pgfsetfillcolor{currentfill}%
\pgfsetlinewidth{0.311001pt}%
\definecolor{currentstroke}{rgb}{1.000000,1.000000,1.000000}%
\pgfsetstrokecolor{currentstroke}%
\pgfsetdash{}{0pt}%
\pgfpathmoveto{\pgfqpoint{6.236663in}{1.447171in}}%
\pgfpathcurveto{\pgfqpoint{6.243796in}{1.447171in}}{\pgfqpoint{6.250637in}{1.450005in}}{\pgfqpoint{6.255681in}{1.455049in}}%
\pgfpathcurveto{\pgfqpoint{6.260725in}{1.460092in}}{\pgfqpoint{6.263559in}{1.466934in}}{\pgfqpoint{6.263559in}{1.474067in}}%
\pgfpathcurveto{\pgfqpoint{6.263559in}{1.481200in}}{\pgfqpoint{6.260725in}{1.488041in}}{\pgfqpoint{6.255681in}{1.493085in}}%
\pgfpathcurveto{\pgfqpoint{6.250637in}{1.498129in}}{\pgfqpoint{6.243796in}{1.500962in}}{\pgfqpoint{6.236663in}{1.500962in}}%
\pgfpathcurveto{\pgfqpoint{6.229530in}{1.500962in}}{\pgfqpoint{6.222688in}{1.498129in}}{\pgfqpoint{6.217645in}{1.493085in}}%
\pgfpathcurveto{\pgfqpoint{6.212601in}{1.488041in}}{\pgfqpoint{6.209767in}{1.481200in}}{\pgfqpoint{6.209767in}{1.474067in}}%
\pgfpathcurveto{\pgfqpoint{6.209767in}{1.466934in}}{\pgfqpoint{6.212601in}{1.460092in}}{\pgfqpoint{6.217645in}{1.455049in}}%
\pgfpathcurveto{\pgfqpoint{6.222688in}{1.450005in}}{\pgfqpoint{6.229530in}{1.447171in}}{\pgfqpoint{6.236663in}{1.447171in}}%
\pgfpathclose%
\pgfusepath{stroke,fill}%
\end{pgfscope}%
\begin{pgfscope}%
\pgfpathrectangle{\pgfqpoint{4.985294in}{0.500000in}}{\pgfqpoint{1.764706in}{1.700000in}}%
\pgfusepath{clip}%
\pgfsetbuttcap%
\pgfsetroundjoin%
\definecolor{currentfill}{rgb}{0.949145,0.420383,0.287810}%
\pgfsetfillcolor{currentfill}%
\pgfsetlinewidth{0.311001pt}%
\definecolor{currentstroke}{rgb}{1.000000,1.000000,1.000000}%
\pgfsetstrokecolor{currentstroke}%
\pgfsetdash{}{0pt}%
\pgfpathmoveto{\pgfqpoint{6.144418in}{1.285883in}}%
\pgfpathcurveto{\pgfqpoint{6.151551in}{1.285883in}}{\pgfqpoint{6.158392in}{1.288717in}}{\pgfqpoint{6.163436in}{1.293760in}}%
\pgfpathcurveto{\pgfqpoint{6.168480in}{1.298804in}}{\pgfqpoint{6.171314in}{1.305646in}}{\pgfqpoint{6.171314in}{1.312778in}}%
\pgfpathcurveto{\pgfqpoint{6.171314in}{1.319911in}}{\pgfqpoint{6.168480in}{1.326753in}}{\pgfqpoint{6.163436in}{1.331796in}}%
\pgfpathcurveto{\pgfqpoint{6.158392in}{1.336840in}}{\pgfqpoint{6.151551in}{1.339674in}}{\pgfqpoint{6.144418in}{1.339674in}}%
\pgfpathcurveto{\pgfqpoint{6.137285in}{1.339674in}}{\pgfqpoint{6.130444in}{1.336840in}}{\pgfqpoint{6.125400in}{1.331796in}}%
\pgfpathcurveto{\pgfqpoint{6.120356in}{1.326753in}}{\pgfqpoint{6.117522in}{1.319911in}}{\pgfqpoint{6.117522in}{1.312778in}}%
\pgfpathcurveto{\pgfqpoint{6.117522in}{1.305646in}}{\pgfqpoint{6.120356in}{1.298804in}}{\pgfqpoint{6.125400in}{1.293760in}}%
\pgfpathcurveto{\pgfqpoint{6.130444in}{1.288717in}}{\pgfqpoint{6.137285in}{1.285883in}}{\pgfqpoint{6.144418in}{1.285883in}}%
\pgfpathclose%
\pgfusepath{stroke,fill}%
\end{pgfscope}%
\begin{pgfscope}%
\pgfpathrectangle{\pgfqpoint{4.985294in}{0.500000in}}{\pgfqpoint{1.764706in}{1.700000in}}%
\pgfusepath{clip}%
\pgfsetbuttcap%
\pgfsetroundjoin%
\definecolor{currentfill}{rgb}{0.962283,0.593046,0.431453}%
\pgfsetfillcolor{currentfill}%
\pgfsetlinewidth{0.311001pt}%
\definecolor{currentstroke}{rgb}{1.000000,1.000000,1.000000}%
\pgfsetstrokecolor{currentstroke}%
\pgfsetdash{}{0pt}%
\pgfpathmoveto{\pgfqpoint{6.138456in}{1.120363in}}%
\pgfpathcurveto{\pgfqpoint{6.145589in}{1.120363in}}{\pgfqpoint{6.152430in}{1.123197in}}{\pgfqpoint{6.157474in}{1.128241in}}%
\pgfpathcurveto{\pgfqpoint{6.162518in}{1.133285in}}{\pgfqpoint{6.165352in}{1.140126in}}{\pgfqpoint{6.165352in}{1.147259in}}%
\pgfpathcurveto{\pgfqpoint{6.165352in}{1.154392in}}{\pgfqpoint{6.162518in}{1.161234in}}{\pgfqpoint{6.157474in}{1.166277in}}%
\pgfpathcurveto{\pgfqpoint{6.152430in}{1.171321in}}{\pgfqpoint{6.145589in}{1.174155in}}{\pgfqpoint{6.138456in}{1.174155in}}%
\pgfpathcurveto{\pgfqpoint{6.131323in}{1.174155in}}{\pgfqpoint{6.124481in}{1.171321in}}{\pgfqpoint{6.119438in}{1.166277in}}%
\pgfpathcurveto{\pgfqpoint{6.114394in}{1.161234in}}{\pgfqpoint{6.111560in}{1.154392in}}{\pgfqpoint{6.111560in}{1.147259in}}%
\pgfpathcurveto{\pgfqpoint{6.111560in}{1.140126in}}{\pgfqpoint{6.114394in}{1.133285in}}{\pgfqpoint{6.119438in}{1.128241in}}%
\pgfpathcurveto{\pgfqpoint{6.124481in}{1.123197in}}{\pgfqpoint{6.131323in}{1.120363in}}{\pgfqpoint{6.138456in}{1.120363in}}%
\pgfpathclose%
\pgfusepath{stroke,fill}%
\end{pgfscope}%
\begin{pgfscope}%
\pgfpathrectangle{\pgfqpoint{4.985294in}{0.500000in}}{\pgfqpoint{1.764706in}{1.700000in}}%
\pgfusepath{clip}%
\pgfsetbuttcap%
\pgfsetroundjoin%
\definecolor{currentfill}{rgb}{0.973832,0.856556,0.771584}%
\pgfsetfillcolor{currentfill}%
\pgfsetlinewidth{0.311001pt}%
\definecolor{currentstroke}{rgb}{1.000000,1.000000,1.000000}%
\pgfsetstrokecolor{currentstroke}%
\pgfsetdash{}{0pt}%
\pgfpathmoveto{\pgfqpoint{5.472315in}{1.345716in}}%
\pgfpathcurveto{\pgfqpoint{5.479448in}{1.345716in}}{\pgfqpoint{5.486290in}{1.348550in}}{\pgfqpoint{5.491333in}{1.353594in}}%
\pgfpathcurveto{\pgfqpoint{5.496377in}{1.358638in}}{\pgfqpoint{5.499211in}{1.365479in}}{\pgfqpoint{5.499211in}{1.372612in}}%
\pgfpathcurveto{\pgfqpoint{5.499211in}{1.379745in}}{\pgfqpoint{5.496377in}{1.386587in}}{\pgfqpoint{5.491333in}{1.391630in}}%
\pgfpathcurveto{\pgfqpoint{5.486290in}{1.396674in}}{\pgfqpoint{5.479448in}{1.399508in}}{\pgfqpoint{5.472315in}{1.399508in}}%
\pgfpathcurveto{\pgfqpoint{5.465182in}{1.399508in}}{\pgfqpoint{5.458341in}{1.396674in}}{\pgfqpoint{5.453297in}{1.391630in}}%
\pgfpathcurveto{\pgfqpoint{5.448253in}{1.386587in}}{\pgfqpoint{5.445420in}{1.379745in}}{\pgfqpoint{5.445420in}{1.372612in}}%
\pgfpathcurveto{\pgfqpoint{5.445420in}{1.365479in}}{\pgfqpoint{5.448253in}{1.358638in}}{\pgfqpoint{5.453297in}{1.353594in}}%
\pgfpathcurveto{\pgfqpoint{5.458341in}{1.348550in}}{\pgfqpoint{5.465182in}{1.345716in}}{\pgfqpoint{5.472315in}{1.345716in}}%
\pgfpathclose%
\pgfusepath{stroke,fill}%
\end{pgfscope}%
\begin{pgfscope}%
\pgfpathrectangle{\pgfqpoint{4.985294in}{0.500000in}}{\pgfqpoint{1.764706in}{1.700000in}}%
\pgfusepath{clip}%
\pgfsetbuttcap%
\pgfsetroundjoin%
\definecolor{currentfill}{rgb}{0.965928,0.738443,0.600540}%
\pgfsetfillcolor{currentfill}%
\pgfsetlinewidth{0.311001pt}%
\definecolor{currentstroke}{rgb}{1.000000,1.000000,1.000000}%
\pgfsetstrokecolor{currentstroke}%
\pgfsetdash{}{0pt}%
\pgfpathmoveto{\pgfqpoint{6.190126in}{1.201242in}}%
\pgfpathcurveto{\pgfqpoint{6.197259in}{1.201242in}}{\pgfqpoint{6.204101in}{1.204075in}}{\pgfqpoint{6.209145in}{1.209119in}}%
\pgfpathcurveto{\pgfqpoint{6.214188in}{1.214163in}}{\pgfqpoint{6.217022in}{1.221004in}}{\pgfqpoint{6.217022in}{1.228137in}}%
\pgfpathcurveto{\pgfqpoint{6.217022in}{1.235270in}}{\pgfqpoint{6.214188in}{1.242112in}}{\pgfqpoint{6.209145in}{1.247155in}}%
\pgfpathcurveto{\pgfqpoint{6.204101in}{1.252199in}}{\pgfqpoint{6.197259in}{1.255033in}}{\pgfqpoint{6.190126in}{1.255033in}}%
\pgfpathcurveto{\pgfqpoint{6.182994in}{1.255033in}}{\pgfqpoint{6.176152in}{1.252199in}}{\pgfqpoint{6.171108in}{1.247155in}}%
\pgfpathcurveto{\pgfqpoint{6.166065in}{1.242112in}}{\pgfqpoint{6.163231in}{1.235270in}}{\pgfqpoint{6.163231in}{1.228137in}}%
\pgfpathcurveto{\pgfqpoint{6.163231in}{1.221004in}}{\pgfqpoint{6.166065in}{1.214163in}}{\pgfqpoint{6.171108in}{1.209119in}}%
\pgfpathcurveto{\pgfqpoint{6.176152in}{1.204075in}}{\pgfqpoint{6.182994in}{1.201242in}}{\pgfqpoint{6.190126in}{1.201242in}}%
\pgfpathclose%
\pgfusepath{stroke,fill}%
\end{pgfscope}%
\begin{pgfscope}%
\pgfpathrectangle{\pgfqpoint{4.985294in}{0.500000in}}{\pgfqpoint{1.764706in}{1.700000in}}%
\pgfusepath{clip}%
\pgfsetbuttcap%
\pgfsetroundjoin%
\definecolor{currentfill}{rgb}{0.970255,0.815666,0.711203}%
\pgfsetfillcolor{currentfill}%
\pgfsetlinewidth{0.311001pt}%
\definecolor{currentstroke}{rgb}{1.000000,1.000000,1.000000}%
\pgfsetstrokecolor{currentstroke}%
\pgfsetdash{}{0pt}%
\pgfpathmoveto{\pgfqpoint{6.205819in}{1.528097in}}%
\pgfpathcurveto{\pgfqpoint{6.212952in}{1.528097in}}{\pgfqpoint{6.219793in}{1.530931in}}{\pgfqpoint{6.224837in}{1.535975in}}%
\pgfpathcurveto{\pgfqpoint{6.229881in}{1.541019in}}{\pgfqpoint{6.232715in}{1.547860in}}{\pgfqpoint{6.232715in}{1.554993in}}%
\pgfpathcurveto{\pgfqpoint{6.232715in}{1.562126in}}{\pgfqpoint{6.229881in}{1.568968in}}{\pgfqpoint{6.224837in}{1.574011in}}%
\pgfpathcurveto{\pgfqpoint{6.219793in}{1.579055in}}{\pgfqpoint{6.212952in}{1.581889in}}{\pgfqpoint{6.205819in}{1.581889in}}%
\pgfpathcurveto{\pgfqpoint{6.198686in}{1.581889in}}{\pgfqpoint{6.191845in}{1.579055in}}{\pgfqpoint{6.186801in}{1.574011in}}%
\pgfpathcurveto{\pgfqpoint{6.181757in}{1.568968in}}{\pgfqpoint{6.178923in}{1.562126in}}{\pgfqpoint{6.178923in}{1.554993in}}%
\pgfpathcurveto{\pgfqpoint{6.178923in}{1.547860in}}{\pgfqpoint{6.181757in}{1.541019in}}{\pgfqpoint{6.186801in}{1.535975in}}%
\pgfpathcurveto{\pgfqpoint{6.191845in}{1.530931in}}{\pgfqpoint{6.198686in}{1.528097in}}{\pgfqpoint{6.205819in}{1.528097in}}%
\pgfpathclose%
\pgfusepath{stroke,fill}%
\end{pgfscope}%
\begin{pgfscope}%
\pgfpathrectangle{\pgfqpoint{4.985294in}{0.500000in}}{\pgfqpoint{1.764706in}{1.700000in}}%
\pgfusepath{clip}%
\pgfsetbuttcap%
\pgfsetroundjoin%
\definecolor{currentfill}{rgb}{0.974412,0.862387,0.780156}%
\pgfsetfillcolor{currentfill}%
\pgfsetlinewidth{0.311001pt}%
\definecolor{currentstroke}{rgb}{1.000000,1.000000,1.000000}%
\pgfsetstrokecolor{currentstroke}%
\pgfsetdash{}{0pt}%
\pgfpathmoveto{\pgfqpoint{6.263309in}{1.304255in}}%
\pgfpathcurveto{\pgfqpoint{6.270442in}{1.304255in}}{\pgfqpoint{6.277283in}{1.307089in}}{\pgfqpoint{6.282327in}{1.312132in}}%
\pgfpathcurveto{\pgfqpoint{6.287371in}{1.317176in}}{\pgfqpoint{6.290205in}{1.324018in}}{\pgfqpoint{6.290205in}{1.331151in}}%
\pgfpathcurveto{\pgfqpoint{6.290205in}{1.338283in}}{\pgfqpoint{6.287371in}{1.345125in}}{\pgfqpoint{6.282327in}{1.350169in}}%
\pgfpathcurveto{\pgfqpoint{6.277283in}{1.355212in}}{\pgfqpoint{6.270442in}{1.358046in}}{\pgfqpoint{6.263309in}{1.358046in}}%
\pgfpathcurveto{\pgfqpoint{6.256176in}{1.358046in}}{\pgfqpoint{6.249335in}{1.355212in}}{\pgfqpoint{6.244291in}{1.350169in}}%
\pgfpathcurveto{\pgfqpoint{6.239247in}{1.345125in}}{\pgfqpoint{6.236413in}{1.338283in}}{\pgfqpoint{6.236413in}{1.331151in}}%
\pgfpathcurveto{\pgfqpoint{6.236413in}{1.324018in}}{\pgfqpoint{6.239247in}{1.317176in}}{\pgfqpoint{6.244291in}{1.312132in}}%
\pgfpathcurveto{\pgfqpoint{6.249335in}{1.307089in}}{\pgfqpoint{6.256176in}{1.304255in}}{\pgfqpoint{6.263309in}{1.304255in}}%
\pgfpathclose%
\pgfusepath{stroke,fill}%
\end{pgfscope}%
\begin{pgfscope}%
\pgfpathrectangle{\pgfqpoint{4.985294in}{0.500000in}}{\pgfqpoint{1.764706in}{1.700000in}}%
\pgfusepath{clip}%
\pgfsetbuttcap%
\pgfsetroundjoin%
\definecolor{currentfill}{rgb}{0.973832,0.856556,0.771584}%
\pgfsetfillcolor{currentfill}%
\pgfsetlinewidth{0.311001pt}%
\definecolor{currentstroke}{rgb}{1.000000,1.000000,1.000000}%
\pgfsetstrokecolor{currentstroke}%
\pgfsetdash{}{0pt}%
\pgfpathmoveto{\pgfqpoint{5.433197in}{1.551679in}}%
\pgfpathcurveto{\pgfqpoint{5.440330in}{1.551679in}}{\pgfqpoint{5.447172in}{1.554513in}}{\pgfqpoint{5.452215in}{1.559557in}}%
\pgfpathcurveto{\pgfqpoint{5.457259in}{1.564601in}}{\pgfqpoint{5.460093in}{1.571442in}}{\pgfqpoint{5.460093in}{1.578575in}}%
\pgfpathcurveto{\pgfqpoint{5.460093in}{1.585708in}}{\pgfqpoint{5.457259in}{1.592550in}}{\pgfqpoint{5.452215in}{1.597593in}}%
\pgfpathcurveto{\pgfqpoint{5.447172in}{1.602637in}}{\pgfqpoint{5.440330in}{1.605471in}}{\pgfqpoint{5.433197in}{1.605471in}}%
\pgfpathcurveto{\pgfqpoint{5.426064in}{1.605471in}}{\pgfqpoint{5.419223in}{1.602637in}}{\pgfqpoint{5.414179in}{1.597593in}}%
\pgfpathcurveto{\pgfqpoint{5.409135in}{1.592550in}}{\pgfqpoint{5.406301in}{1.585708in}}{\pgfqpoint{5.406301in}{1.578575in}}%
\pgfpathcurveto{\pgfqpoint{5.406301in}{1.571442in}}{\pgfqpoint{5.409135in}{1.564601in}}{\pgfqpoint{5.414179in}{1.559557in}}%
\pgfpathcurveto{\pgfqpoint{5.419223in}{1.554513in}}{\pgfqpoint{5.426064in}{1.551679in}}{\pgfqpoint{5.433197in}{1.551679in}}%
\pgfpathclose%
\pgfusepath{stroke,fill}%
\end{pgfscope}%
\begin{pgfscope}%
\pgfpathrectangle{\pgfqpoint{4.985294in}{0.500000in}}{\pgfqpoint{1.764706in}{1.700000in}}%
\pgfusepath{clip}%
\pgfsetbuttcap%
\pgfsetroundjoin%
\definecolor{currentfill}{rgb}{0.955697,0.484891,0.334214}%
\pgfsetfillcolor{currentfill}%
\pgfsetlinewidth{0.311001pt}%
\definecolor{currentstroke}{rgb}{1.000000,1.000000,1.000000}%
\pgfsetstrokecolor{currentstroke}%
\pgfsetdash{}{0pt}%
\pgfpathmoveto{\pgfqpoint{5.292734in}{1.412197in}}%
\pgfpathcurveto{\pgfqpoint{5.299866in}{1.412197in}}{\pgfqpoint{5.306708in}{1.415031in}}{\pgfqpoint{5.311752in}{1.420075in}}%
\pgfpathcurveto{\pgfqpoint{5.316795in}{1.425119in}}{\pgfqpoint{5.319629in}{1.431960in}}{\pgfqpoint{5.319629in}{1.439093in}}%
\pgfpathcurveto{\pgfqpoint{5.319629in}{1.446226in}}{\pgfqpoint{5.316795in}{1.453067in}}{\pgfqpoint{5.311752in}{1.458111in}}%
\pgfpathcurveto{\pgfqpoint{5.306708in}{1.463155in}}{\pgfqpoint{5.299866in}{1.465989in}}{\pgfqpoint{5.292734in}{1.465989in}}%
\pgfpathcurveto{\pgfqpoint{5.285601in}{1.465989in}}{\pgfqpoint{5.278759in}{1.463155in}}{\pgfqpoint{5.273716in}{1.458111in}}%
\pgfpathcurveto{\pgfqpoint{5.268672in}{1.453067in}}{\pgfqpoint{5.265838in}{1.446226in}}{\pgfqpoint{5.265838in}{1.439093in}}%
\pgfpathcurveto{\pgfqpoint{5.265838in}{1.431960in}}{\pgfqpoint{5.268672in}{1.425119in}}{\pgfqpoint{5.273716in}{1.420075in}}%
\pgfpathcurveto{\pgfqpoint{5.278759in}{1.415031in}}{\pgfqpoint{5.285601in}{1.412197in}}{\pgfqpoint{5.292734in}{1.412197in}}%
\pgfpathclose%
\pgfusepath{stroke,fill}%
\end{pgfscope}%
\begin{pgfscope}%
\pgfpathrectangle{\pgfqpoint{4.985294in}{0.500000in}}{\pgfqpoint{1.764706in}{1.700000in}}%
\pgfusepath{clip}%
\pgfsetbuttcap%
\pgfsetroundjoin%
\definecolor{currentfill}{rgb}{0.969803,0.809811,0.702523}%
\pgfsetfillcolor{currentfill}%
\pgfsetlinewidth{0.311001pt}%
\definecolor{currentstroke}{rgb}{1.000000,1.000000,1.000000}%
\pgfsetstrokecolor{currentstroke}%
\pgfsetdash{}{0pt}%
\pgfpathmoveto{\pgfqpoint{5.423850in}{1.585567in}}%
\pgfpathcurveto{\pgfqpoint{5.430983in}{1.585567in}}{\pgfqpoint{5.437825in}{1.588401in}}{\pgfqpoint{5.442868in}{1.593445in}}%
\pgfpathcurveto{\pgfqpoint{5.447912in}{1.598488in}}{\pgfqpoint{5.450746in}{1.605330in}}{\pgfqpoint{5.450746in}{1.612463in}}%
\pgfpathcurveto{\pgfqpoint{5.450746in}{1.619596in}}{\pgfqpoint{5.447912in}{1.626437in}}{\pgfqpoint{5.442868in}{1.631481in}}%
\pgfpathcurveto{\pgfqpoint{5.437825in}{1.636525in}}{\pgfqpoint{5.430983in}{1.639358in}}{\pgfqpoint{5.423850in}{1.639358in}}%
\pgfpathcurveto{\pgfqpoint{5.416718in}{1.639358in}}{\pgfqpoint{5.409876in}{1.636525in}}{\pgfqpoint{5.404832in}{1.631481in}}%
\pgfpathcurveto{\pgfqpoint{5.399789in}{1.626437in}}{\pgfqpoint{5.396955in}{1.619596in}}{\pgfqpoint{5.396955in}{1.612463in}}%
\pgfpathcurveto{\pgfqpoint{5.396955in}{1.605330in}}{\pgfqpoint{5.399789in}{1.598488in}}{\pgfqpoint{5.404832in}{1.593445in}}%
\pgfpathcurveto{\pgfqpoint{5.409876in}{1.588401in}}{\pgfqpoint{5.416718in}{1.585567in}}{\pgfqpoint{5.423850in}{1.585567in}}%
\pgfpathclose%
\pgfusepath{stroke,fill}%
\end{pgfscope}%
\begin{pgfscope}%
\pgfpathrectangle{\pgfqpoint{4.985294in}{0.500000in}}{\pgfqpoint{1.764706in}{1.700000in}}%
\pgfusepath{clip}%
\pgfsetbuttcap%
\pgfsetroundjoin%
\definecolor{currentfill}{rgb}{0.971202,0.827364,0.728520}%
\pgfsetfillcolor{currentfill}%
\pgfsetlinewidth{0.311001pt}%
\definecolor{currentstroke}{rgb}{1.000000,1.000000,1.000000}%
\pgfsetstrokecolor{currentstroke}%
\pgfsetdash{}{0pt}%
\pgfpathmoveto{\pgfqpoint{5.402993in}{1.531822in}}%
\pgfpathcurveto{\pgfqpoint{5.410126in}{1.531822in}}{\pgfqpoint{5.416967in}{1.534655in}}{\pgfqpoint{5.422011in}{1.539699in}}%
\pgfpathcurveto{\pgfqpoint{5.427055in}{1.544743in}}{\pgfqpoint{5.429889in}{1.551584in}}{\pgfqpoint{5.429889in}{1.558717in}}%
\pgfpathcurveto{\pgfqpoint{5.429889in}{1.565850in}}{\pgfqpoint{5.427055in}{1.572692in}}{\pgfqpoint{5.422011in}{1.577735in}}%
\pgfpathcurveto{\pgfqpoint{5.416967in}{1.582779in}}{\pgfqpoint{5.410126in}{1.585613in}}{\pgfqpoint{5.402993in}{1.585613in}}%
\pgfpathcurveto{\pgfqpoint{5.395860in}{1.585613in}}{\pgfqpoint{5.389018in}{1.582779in}}{\pgfqpoint{5.383975in}{1.577735in}}%
\pgfpathcurveto{\pgfqpoint{5.378931in}{1.572692in}}{\pgfqpoint{5.376097in}{1.565850in}}{\pgfqpoint{5.376097in}{1.558717in}}%
\pgfpathcurveto{\pgfqpoint{5.376097in}{1.551584in}}{\pgfqpoint{5.378931in}{1.544743in}}{\pgfqpoint{5.383975in}{1.539699in}}%
\pgfpathcurveto{\pgfqpoint{5.389018in}{1.534655in}}{\pgfqpoint{5.395860in}{1.531822in}}{\pgfqpoint{5.402993in}{1.531822in}}%
\pgfpathclose%
\pgfusepath{stroke,fill}%
\end{pgfscope}%
\begin{pgfscope}%
\pgfpathrectangle{\pgfqpoint{4.985294in}{0.500000in}}{\pgfqpoint{1.764706in}{1.700000in}}%
\pgfusepath{clip}%
\pgfsetbuttcap%
\pgfsetroundjoin%
\definecolor{currentfill}{rgb}{0.975644,0.874038,0.797253}%
\pgfsetfillcolor{currentfill}%
\pgfsetlinewidth{0.311001pt}%
\definecolor{currentstroke}{rgb}{1.000000,1.000000,1.000000}%
\pgfsetstrokecolor{currentstroke}%
\pgfsetdash{}{0pt}%
\pgfpathmoveto{\pgfqpoint{5.468940in}{1.180997in}}%
\pgfpathcurveto{\pgfqpoint{5.476073in}{1.180997in}}{\pgfqpoint{5.482914in}{1.183831in}}{\pgfqpoint{5.487958in}{1.188874in}}%
\pgfpathcurveto{\pgfqpoint{5.493002in}{1.193918in}}{\pgfqpoint{5.495835in}{1.200760in}}{\pgfqpoint{5.495835in}{1.207893in}}%
\pgfpathcurveto{\pgfqpoint{5.495835in}{1.215025in}}{\pgfqpoint{5.493002in}{1.221867in}}{\pgfqpoint{5.487958in}{1.226911in}}%
\pgfpathcurveto{\pgfqpoint{5.482914in}{1.231954in}}{\pgfqpoint{5.476073in}{1.234788in}}{\pgfqpoint{5.468940in}{1.234788in}}%
\pgfpathcurveto{\pgfqpoint{5.461807in}{1.234788in}}{\pgfqpoint{5.454965in}{1.231954in}}{\pgfqpoint{5.449922in}{1.226911in}}%
\pgfpathcurveto{\pgfqpoint{5.444878in}{1.221867in}}{\pgfqpoint{5.442044in}{1.215025in}}{\pgfqpoint{5.442044in}{1.207893in}}%
\pgfpathcurveto{\pgfqpoint{5.442044in}{1.200760in}}{\pgfqpoint{5.444878in}{1.193918in}}{\pgfqpoint{5.449922in}{1.188874in}}%
\pgfpathcurveto{\pgfqpoint{5.454965in}{1.183831in}}{\pgfqpoint{5.461807in}{1.180997in}}{\pgfqpoint{5.468940in}{1.180997in}}%
\pgfpathclose%
\pgfusepath{stroke,fill}%
\end{pgfscope}%
\begin{pgfscope}%
\pgfpathrectangle{\pgfqpoint{4.985294in}{0.500000in}}{\pgfqpoint{1.764706in}{1.700000in}}%
\pgfusepath{clip}%
\pgfsetbuttcap%
\pgfsetroundjoin%
\definecolor{currentfill}{rgb}{0.970255,0.815666,0.711203}%
\pgfsetfillcolor{currentfill}%
\pgfsetlinewidth{0.311001pt}%
\definecolor{currentstroke}{rgb}{1.000000,1.000000,1.000000}%
\pgfsetstrokecolor{currentstroke}%
\pgfsetdash{}{0pt}%
\pgfpathmoveto{\pgfqpoint{5.410092in}{1.070944in}}%
\pgfpathcurveto{\pgfqpoint{5.417225in}{1.070944in}}{\pgfqpoint{5.424067in}{1.073778in}}{\pgfqpoint{5.429110in}{1.078822in}}%
\pgfpathcurveto{\pgfqpoint{5.434154in}{1.083866in}}{\pgfqpoint{5.436988in}{1.090707in}}{\pgfqpoint{5.436988in}{1.097840in}}%
\pgfpathcurveto{\pgfqpoint{5.436988in}{1.104973in}}{\pgfqpoint{5.434154in}{1.111815in}}{\pgfqpoint{5.429110in}{1.116858in}}%
\pgfpathcurveto{\pgfqpoint{5.424067in}{1.121902in}}{\pgfqpoint{5.417225in}{1.124736in}}{\pgfqpoint{5.410092in}{1.124736in}}%
\pgfpathcurveto{\pgfqpoint{5.402960in}{1.124736in}}{\pgfqpoint{5.396118in}{1.121902in}}{\pgfqpoint{5.391074in}{1.116858in}}%
\pgfpathcurveto{\pgfqpoint{5.386031in}{1.111815in}}{\pgfqpoint{5.383197in}{1.104973in}}{\pgfqpoint{5.383197in}{1.097840in}}%
\pgfpathcurveto{\pgfqpoint{5.383197in}{1.090707in}}{\pgfqpoint{5.386031in}{1.083866in}}{\pgfqpoint{5.391074in}{1.078822in}}%
\pgfpathcurveto{\pgfqpoint{5.396118in}{1.073778in}}{\pgfqpoint{5.402960in}{1.070944in}}{\pgfqpoint{5.410092in}{1.070944in}}%
\pgfpathclose%
\pgfusepath{stroke,fill}%
\end{pgfscope}%
\begin{pgfscope}%
\pgfpathrectangle{\pgfqpoint{4.985294in}{0.500000in}}{\pgfqpoint{1.764706in}{1.700000in}}%
\pgfusepath{clip}%
\pgfsetbuttcap%
\pgfsetroundjoin%
\definecolor{currentfill}{rgb}{0.973832,0.856556,0.771584}%
\pgfsetfillcolor{currentfill}%
\pgfsetlinewidth{0.311001pt}%
\definecolor{currentstroke}{rgb}{1.000000,1.000000,1.000000}%
\pgfsetstrokecolor{currentstroke}%
\pgfsetdash{}{0pt}%
\pgfpathmoveto{\pgfqpoint{5.472905in}{1.349924in}}%
\pgfpathcurveto{\pgfqpoint{5.480038in}{1.349924in}}{\pgfqpoint{5.486880in}{1.352757in}}{\pgfqpoint{5.491923in}{1.357801in}}%
\pgfpathcurveto{\pgfqpoint{5.496967in}{1.362845in}}{\pgfqpoint{5.499801in}{1.369686in}}{\pgfqpoint{5.499801in}{1.376819in}}%
\pgfpathcurveto{\pgfqpoint{5.499801in}{1.383952in}}{\pgfqpoint{5.496967in}{1.390794in}}{\pgfqpoint{5.491923in}{1.395837in}}%
\pgfpathcurveto{\pgfqpoint{5.486880in}{1.400881in}}{\pgfqpoint{5.480038in}{1.403715in}}{\pgfqpoint{5.472905in}{1.403715in}}%
\pgfpathcurveto{\pgfqpoint{5.465772in}{1.403715in}}{\pgfqpoint{5.458931in}{1.400881in}}{\pgfqpoint{5.453887in}{1.395837in}}%
\pgfpathcurveto{\pgfqpoint{5.448843in}{1.390794in}}{\pgfqpoint{5.446009in}{1.383952in}}{\pgfqpoint{5.446009in}{1.376819in}}%
\pgfpathcurveto{\pgfqpoint{5.446009in}{1.369686in}}{\pgfqpoint{5.448843in}{1.362845in}}{\pgfqpoint{5.453887in}{1.357801in}}%
\pgfpathcurveto{\pgfqpoint{5.458931in}{1.352757in}}{\pgfqpoint{5.465772in}{1.349924in}}{\pgfqpoint{5.472905in}{1.349924in}}%
\pgfpathclose%
\pgfusepath{stroke,fill}%
\end{pgfscope}%
\begin{pgfscope}%
\pgfpathrectangle{\pgfqpoint{4.985294in}{0.500000in}}{\pgfqpoint{1.764706in}{1.700000in}}%
\pgfusepath{clip}%
\pgfsetbuttcap%
\pgfsetroundjoin%
\definecolor{currentfill}{rgb}{0.968931,0.798091,0.685123}%
\pgfsetfillcolor{currentfill}%
\pgfsetlinewidth{0.311001pt}%
\definecolor{currentstroke}{rgb}{1.000000,1.000000,1.000000}%
\pgfsetstrokecolor{currentstroke}%
\pgfsetdash{}{0pt}%
\pgfpathmoveto{\pgfqpoint{6.207638in}{1.499033in}}%
\pgfpathcurveto{\pgfqpoint{6.214771in}{1.499033in}}{\pgfqpoint{6.221612in}{1.501867in}}{\pgfqpoint{6.226656in}{1.506911in}}%
\pgfpathcurveto{\pgfqpoint{6.231700in}{1.511954in}}{\pgfqpoint{6.234534in}{1.518796in}}{\pgfqpoint{6.234534in}{1.525929in}}%
\pgfpathcurveto{\pgfqpoint{6.234534in}{1.533062in}}{\pgfqpoint{6.231700in}{1.539903in}}{\pgfqpoint{6.226656in}{1.544947in}}%
\pgfpathcurveto{\pgfqpoint{6.221612in}{1.549991in}}{\pgfqpoint{6.214771in}{1.552824in}}{\pgfqpoint{6.207638in}{1.552824in}}%
\pgfpathcurveto{\pgfqpoint{6.200505in}{1.552824in}}{\pgfqpoint{6.193663in}{1.549991in}}{\pgfqpoint{6.188620in}{1.544947in}}%
\pgfpathcurveto{\pgfqpoint{6.183576in}{1.539903in}}{\pgfqpoint{6.180742in}{1.533062in}}{\pgfqpoint{6.180742in}{1.525929in}}%
\pgfpathcurveto{\pgfqpoint{6.180742in}{1.518796in}}{\pgfqpoint{6.183576in}{1.511954in}}{\pgfqpoint{6.188620in}{1.506911in}}%
\pgfpathcurveto{\pgfqpoint{6.193663in}{1.501867in}}{\pgfqpoint{6.200505in}{1.499033in}}{\pgfqpoint{6.207638in}{1.499033in}}%
\pgfpathclose%
\pgfusepath{stroke,fill}%
\end{pgfscope}%
\begin{pgfscope}%
\pgfpathrectangle{\pgfqpoint{4.985294in}{0.500000in}}{\pgfqpoint{1.764706in}{1.700000in}}%
\pgfusepath{clip}%
\pgfsetbuttcap%
\pgfsetroundjoin%
\definecolor{currentfill}{rgb}{0.973832,0.856556,0.771584}%
\pgfsetfillcolor{currentfill}%
\pgfsetlinewidth{0.311001pt}%
\definecolor{currentstroke}{rgb}{1.000000,1.000000,1.000000}%
\pgfsetstrokecolor{currentstroke}%
\pgfsetdash{}{0pt}%
\pgfpathmoveto{\pgfqpoint{6.364224in}{1.232774in}}%
\pgfpathcurveto{\pgfqpoint{6.371357in}{1.232774in}}{\pgfqpoint{6.378199in}{1.235608in}}{\pgfqpoint{6.383242in}{1.240652in}}%
\pgfpathcurveto{\pgfqpoint{6.388286in}{1.245695in}}{\pgfqpoint{6.391120in}{1.252537in}}{\pgfqpoint{6.391120in}{1.259670in}}%
\pgfpathcurveto{\pgfqpoint{6.391120in}{1.266803in}}{\pgfqpoint{6.388286in}{1.273644in}}{\pgfqpoint{6.383242in}{1.278688in}}%
\pgfpathcurveto{\pgfqpoint{6.378199in}{1.283732in}}{\pgfqpoint{6.371357in}{1.286565in}}{\pgfqpoint{6.364224in}{1.286565in}}%
\pgfpathcurveto{\pgfqpoint{6.357091in}{1.286565in}}{\pgfqpoint{6.350250in}{1.283732in}}{\pgfqpoint{6.345206in}{1.278688in}}%
\pgfpathcurveto{\pgfqpoint{6.340162in}{1.273644in}}{\pgfqpoint{6.337329in}{1.266803in}}{\pgfqpoint{6.337329in}{1.259670in}}%
\pgfpathcurveto{\pgfqpoint{6.337329in}{1.252537in}}{\pgfqpoint{6.340162in}{1.245695in}}{\pgfqpoint{6.345206in}{1.240652in}}%
\pgfpathcurveto{\pgfqpoint{6.350250in}{1.235608in}}{\pgfqpoint{6.357091in}{1.232774in}}{\pgfqpoint{6.364224in}{1.232774in}}%
\pgfpathclose%
\pgfusepath{stroke,fill}%
\end{pgfscope}%
\begin{pgfscope}%
\pgfpathrectangle{\pgfqpoint{4.985294in}{0.500000in}}{\pgfqpoint{1.764706in}{1.700000in}}%
\pgfusepath{clip}%
\pgfsetbuttcap%
\pgfsetroundjoin%
\definecolor{currentfill}{rgb}{0.852817,0.156578,0.279098}%
\pgfsetfillcolor{currentfill}%
\pgfsetlinewidth{0.311001pt}%
\definecolor{currentstroke}{rgb}{1.000000,1.000000,1.000000}%
\pgfsetstrokecolor{currentstroke}%
\pgfsetdash{}{0pt}%
\pgfpathmoveto{\pgfqpoint{5.859838in}{1.779328in}}%
\pgfpathcurveto{\pgfqpoint{5.866970in}{1.779328in}}{\pgfqpoint{5.873812in}{1.782161in}}{\pgfqpoint{5.878856in}{1.787205in}}%
\pgfpathcurveto{\pgfqpoint{5.883899in}{1.792249in}}{\pgfqpoint{5.886733in}{1.799090in}}{\pgfqpoint{5.886733in}{1.806223in}}%
\pgfpathcurveto{\pgfqpoint{5.886733in}{1.813356in}}{\pgfqpoint{5.883899in}{1.820198in}}{\pgfqpoint{5.878856in}{1.825241in}}%
\pgfpathcurveto{\pgfqpoint{5.873812in}{1.830285in}}{\pgfqpoint{5.866970in}{1.833119in}}{\pgfqpoint{5.859838in}{1.833119in}}%
\pgfpathcurveto{\pgfqpoint{5.852705in}{1.833119in}}{\pgfqpoint{5.845863in}{1.830285in}}{\pgfqpoint{5.840819in}{1.825241in}}%
\pgfpathcurveto{\pgfqpoint{5.835776in}{1.820198in}}{\pgfqpoint{5.832942in}{1.813356in}}{\pgfqpoint{5.832942in}{1.806223in}}%
\pgfpathcurveto{\pgfqpoint{5.832942in}{1.799090in}}{\pgfqpoint{5.835776in}{1.792249in}}{\pgfqpoint{5.840819in}{1.787205in}}%
\pgfpathcurveto{\pgfqpoint{5.845863in}{1.782161in}}{\pgfqpoint{5.852705in}{1.779328in}}{\pgfqpoint{5.859838in}{1.779328in}}%
\pgfpathclose%
\pgfusepath{stroke,fill}%
\end{pgfscope}%
\begin{pgfscope}%
\pgfpathrectangle{\pgfqpoint{4.985294in}{0.500000in}}{\pgfqpoint{1.764706in}{1.700000in}}%
\pgfusepath{clip}%
\pgfsetbuttcap%
\pgfsetroundjoin%
\definecolor{currentfill}{rgb}{0.973271,0.850724,0.762998}%
\pgfsetfillcolor{currentfill}%
\pgfsetlinewidth{0.311001pt}%
\definecolor{currentstroke}{rgb}{1.000000,1.000000,1.000000}%
\pgfsetstrokecolor{currentstroke}%
\pgfsetdash{}{0pt}%
\pgfpathmoveto{\pgfqpoint{5.486415in}{1.169275in}}%
\pgfpathcurveto{\pgfqpoint{5.493548in}{1.169275in}}{\pgfqpoint{5.500389in}{1.172109in}}{\pgfqpoint{5.505433in}{1.177153in}}%
\pgfpathcurveto{\pgfqpoint{5.510477in}{1.182197in}}{\pgfqpoint{5.513311in}{1.189038in}}{\pgfqpoint{5.513311in}{1.196171in}}%
\pgfpathcurveto{\pgfqpoint{5.513311in}{1.203304in}}{\pgfqpoint{5.510477in}{1.210146in}}{\pgfqpoint{5.505433in}{1.215189in}}%
\pgfpathcurveto{\pgfqpoint{5.500389in}{1.220233in}}{\pgfqpoint{5.493548in}{1.223067in}}{\pgfqpoint{5.486415in}{1.223067in}}%
\pgfpathcurveto{\pgfqpoint{5.479282in}{1.223067in}}{\pgfqpoint{5.472440in}{1.220233in}}{\pgfqpoint{5.467397in}{1.215189in}}%
\pgfpathcurveto{\pgfqpoint{5.462353in}{1.210146in}}{\pgfqpoint{5.459519in}{1.203304in}}{\pgfqpoint{5.459519in}{1.196171in}}%
\pgfpathcurveto{\pgfqpoint{5.459519in}{1.189038in}}{\pgfqpoint{5.462353in}{1.182197in}}{\pgfqpoint{5.467397in}{1.177153in}}%
\pgfpathcurveto{\pgfqpoint{5.472440in}{1.172109in}}{\pgfqpoint{5.479282in}{1.169275in}}{\pgfqpoint{5.486415in}{1.169275in}}%
\pgfpathclose%
\pgfusepath{stroke,fill}%
\end{pgfscope}%
\begin{pgfscope}%
\pgfpathrectangle{\pgfqpoint{4.985294in}{0.500000in}}{\pgfqpoint{1.764706in}{1.700000in}}%
\pgfusepath{clip}%
\pgfsetbuttcap%
\pgfsetroundjoin%
\definecolor{currentfill}{rgb}{0.975644,0.874038,0.797253}%
\pgfsetfillcolor{currentfill}%
\pgfsetlinewidth{0.311001pt}%
\definecolor{currentstroke}{rgb}{1.000000,1.000000,1.000000}%
\pgfsetstrokecolor{currentstroke}%
\pgfsetdash{}{0pt}%
\pgfpathmoveto{\pgfqpoint{6.300152in}{1.112229in}}%
\pgfpathcurveto{\pgfqpoint{6.307284in}{1.112229in}}{\pgfqpoint{6.314126in}{1.115063in}}{\pgfqpoint{6.319170in}{1.120107in}}%
\pgfpathcurveto{\pgfqpoint{6.324213in}{1.125150in}}{\pgfqpoint{6.327047in}{1.131992in}}{\pgfqpoint{6.327047in}{1.139125in}}%
\pgfpathcurveto{\pgfqpoint{6.327047in}{1.146258in}}{\pgfqpoint{6.324213in}{1.153099in}}{\pgfqpoint{6.319170in}{1.158143in}}%
\pgfpathcurveto{\pgfqpoint{6.314126in}{1.163187in}}{\pgfqpoint{6.307284in}{1.166021in}}{\pgfqpoint{6.300152in}{1.166021in}}%
\pgfpathcurveto{\pgfqpoint{6.293019in}{1.166021in}}{\pgfqpoint{6.286177in}{1.163187in}}{\pgfqpoint{6.281134in}{1.158143in}}%
\pgfpathcurveto{\pgfqpoint{6.276090in}{1.153099in}}{\pgfqpoint{6.273256in}{1.146258in}}{\pgfqpoint{6.273256in}{1.139125in}}%
\pgfpathcurveto{\pgfqpoint{6.273256in}{1.131992in}}{\pgfqpoint{6.276090in}{1.125150in}}{\pgfqpoint{6.281134in}{1.120107in}}%
\pgfpathcurveto{\pgfqpoint{6.286177in}{1.115063in}}{\pgfqpoint{6.293019in}{1.112229in}}{\pgfqpoint{6.300152in}{1.112229in}}%
\pgfpathclose%
\pgfusepath{stroke,fill}%
\end{pgfscope}%
\begin{pgfscope}%
\pgfpathrectangle{\pgfqpoint{4.985294in}{0.500000in}}{\pgfqpoint{1.764706in}{1.700000in}}%
\pgfusepath{clip}%
\pgfsetbuttcap%
\pgfsetroundjoin%
\definecolor{currentfill}{rgb}{0.972726,0.844889,0.754401}%
\pgfsetfillcolor{currentfill}%
\pgfsetlinewidth{0.311001pt}%
\definecolor{currentstroke}{rgb}{1.000000,1.000000,1.000000}%
\pgfsetstrokecolor{currentstroke}%
\pgfsetdash{}{0pt}%
\pgfpathmoveto{\pgfqpoint{5.488736in}{1.422540in}}%
\pgfpathcurveto{\pgfqpoint{5.495868in}{1.422540in}}{\pgfqpoint{5.502710in}{1.425374in}}{\pgfqpoint{5.507754in}{1.430418in}}%
\pgfpathcurveto{\pgfqpoint{5.512797in}{1.435462in}}{\pgfqpoint{5.515631in}{1.442303in}}{\pgfqpoint{5.515631in}{1.449436in}}%
\pgfpathcurveto{\pgfqpoint{5.515631in}{1.456569in}}{\pgfqpoint{5.512797in}{1.463411in}}{\pgfqpoint{5.507754in}{1.468454in}}%
\pgfpathcurveto{\pgfqpoint{5.502710in}{1.473498in}}{\pgfqpoint{5.495868in}{1.476332in}}{\pgfqpoint{5.488736in}{1.476332in}}%
\pgfpathcurveto{\pgfqpoint{5.481603in}{1.476332in}}{\pgfqpoint{5.474761in}{1.473498in}}{\pgfqpoint{5.469717in}{1.468454in}}%
\pgfpathcurveto{\pgfqpoint{5.464674in}{1.463411in}}{\pgfqpoint{5.461840in}{1.456569in}}{\pgfqpoint{5.461840in}{1.449436in}}%
\pgfpathcurveto{\pgfqpoint{5.461840in}{1.442303in}}{\pgfqpoint{5.464674in}{1.435462in}}{\pgfqpoint{5.469717in}{1.430418in}}%
\pgfpathcurveto{\pgfqpoint{5.474761in}{1.425374in}}{\pgfqpoint{5.481603in}{1.422540in}}{\pgfqpoint{5.488736in}{1.422540in}}%
\pgfpathclose%
\pgfusepath{stroke,fill}%
\end{pgfscope}%
\begin{pgfscope}%
\pgfpathrectangle{\pgfqpoint{4.985294in}{0.500000in}}{\pgfqpoint{1.764706in}{1.700000in}}%
\pgfusepath{clip}%
\pgfsetbuttcap%
\pgfsetroundjoin%
\definecolor{currentfill}{rgb}{0.964032,0.651225,0.493258}%
\pgfsetfillcolor{currentfill}%
\pgfsetlinewidth{0.311001pt}%
\definecolor{currentstroke}{rgb}{1.000000,1.000000,1.000000}%
\pgfsetstrokecolor{currentstroke}%
\pgfsetdash{}{0pt}%
\pgfpathmoveto{\pgfqpoint{5.361444in}{1.556103in}}%
\pgfpathcurveto{\pgfqpoint{5.368577in}{1.556103in}}{\pgfqpoint{5.375419in}{1.558937in}}{\pgfqpoint{5.380462in}{1.563981in}}%
\pgfpathcurveto{\pgfqpoint{5.385506in}{1.569024in}}{\pgfqpoint{5.388340in}{1.575866in}}{\pgfqpoint{5.388340in}{1.582999in}}%
\pgfpathcurveto{\pgfqpoint{5.388340in}{1.590132in}}{\pgfqpoint{5.385506in}{1.596973in}}{\pgfqpoint{5.380462in}{1.602017in}}%
\pgfpathcurveto{\pgfqpoint{5.375419in}{1.607061in}}{\pgfqpoint{5.368577in}{1.609895in}}{\pgfqpoint{5.361444in}{1.609895in}}%
\pgfpathcurveto{\pgfqpoint{5.354311in}{1.609895in}}{\pgfqpoint{5.347470in}{1.607061in}}{\pgfqpoint{5.342426in}{1.602017in}}%
\pgfpathcurveto{\pgfqpoint{5.337382in}{1.596973in}}{\pgfqpoint{5.334548in}{1.590132in}}{\pgfqpoint{5.334548in}{1.582999in}}%
\pgfpathcurveto{\pgfqpoint{5.334548in}{1.575866in}}{\pgfqpoint{5.337382in}{1.569024in}}{\pgfqpoint{5.342426in}{1.563981in}}%
\pgfpathcurveto{\pgfqpoint{5.347470in}{1.558937in}}{\pgfqpoint{5.354311in}{1.556103in}}{\pgfqpoint{5.361444in}{1.556103in}}%
\pgfpathclose%
\pgfusepath{stroke,fill}%
\end{pgfscope}%
\begin{pgfscope}%
\pgfpathrectangle{\pgfqpoint{4.985294in}{0.500000in}}{\pgfqpoint{1.764706in}{1.700000in}}%
\pgfusepath{clip}%
\pgfsetbuttcap%
\pgfsetroundjoin%
\definecolor{currentfill}{rgb}{0.965302,0.713942,0.568499}%
\pgfsetfillcolor{currentfill}%
\pgfsetlinewidth{0.311001pt}%
\definecolor{currentstroke}{rgb}{1.000000,1.000000,1.000000}%
\pgfsetstrokecolor{currentstroke}%
\pgfsetdash{}{0pt}%
\pgfpathmoveto{\pgfqpoint{5.551585in}{1.492722in}}%
\pgfpathcurveto{\pgfqpoint{5.558718in}{1.492722in}}{\pgfqpoint{5.565559in}{1.495556in}}{\pgfqpoint{5.570603in}{1.500599in}}%
\pgfpathcurveto{\pgfqpoint{5.575647in}{1.505643in}}{\pgfqpoint{5.578481in}{1.512485in}}{\pgfqpoint{5.578481in}{1.519617in}}%
\pgfpathcurveto{\pgfqpoint{5.578481in}{1.526750in}}{\pgfqpoint{5.575647in}{1.533592in}}{\pgfqpoint{5.570603in}{1.538636in}}%
\pgfpathcurveto{\pgfqpoint{5.565559in}{1.543679in}}{\pgfqpoint{5.558718in}{1.546513in}}{\pgfqpoint{5.551585in}{1.546513in}}%
\pgfpathcurveto{\pgfqpoint{5.544452in}{1.546513in}}{\pgfqpoint{5.537610in}{1.543679in}}{\pgfqpoint{5.532567in}{1.538636in}}%
\pgfpathcurveto{\pgfqpoint{5.527523in}{1.533592in}}{\pgfqpoint{5.524689in}{1.526750in}}{\pgfqpoint{5.524689in}{1.519617in}}%
\pgfpathcurveto{\pgfqpoint{5.524689in}{1.512485in}}{\pgfqpoint{5.527523in}{1.505643in}}{\pgfqpoint{5.532567in}{1.500599in}}%
\pgfpathcurveto{\pgfqpoint{5.537610in}{1.495556in}}{\pgfqpoint{5.544452in}{1.492722in}}{\pgfqpoint{5.551585in}{1.492722in}}%
\pgfpathclose%
\pgfusepath{stroke,fill}%
\end{pgfscope}%
\begin{pgfscope}%
\pgfpathrectangle{\pgfqpoint{4.985294in}{0.500000in}}{\pgfqpoint{1.764706in}{1.700000in}}%
\pgfusepath{clip}%
\pgfsetbuttcap%
\pgfsetroundjoin%
\definecolor{currentfill}{rgb}{0.966560,0.756582,0.625273}%
\pgfsetfillcolor{currentfill}%
\pgfsetlinewidth{0.311001pt}%
\definecolor{currentstroke}{rgb}{1.000000,1.000000,1.000000}%
\pgfsetstrokecolor{currentstroke}%
\pgfsetdash{}{0pt}%
\pgfpathmoveto{\pgfqpoint{6.151572in}{1.601598in}}%
\pgfpathcurveto{\pgfqpoint{6.158705in}{1.601598in}}{\pgfqpoint{6.165547in}{1.604432in}}{\pgfqpoint{6.170590in}{1.609475in}}%
\pgfpathcurveto{\pgfqpoint{6.175634in}{1.614519in}}{\pgfqpoint{6.178468in}{1.621361in}}{\pgfqpoint{6.178468in}{1.628493in}}%
\pgfpathcurveto{\pgfqpoint{6.178468in}{1.635626in}}{\pgfqpoint{6.175634in}{1.642468in}}{\pgfqpoint{6.170590in}{1.647512in}}%
\pgfpathcurveto{\pgfqpoint{6.165547in}{1.652555in}}{\pgfqpoint{6.158705in}{1.655389in}}{\pgfqpoint{6.151572in}{1.655389in}}%
\pgfpathcurveto{\pgfqpoint{6.144439in}{1.655389in}}{\pgfqpoint{6.137598in}{1.652555in}}{\pgfqpoint{6.132554in}{1.647512in}}%
\pgfpathcurveto{\pgfqpoint{6.127510in}{1.642468in}}{\pgfqpoint{6.124677in}{1.635626in}}{\pgfqpoint{6.124677in}{1.628493in}}%
\pgfpathcurveto{\pgfqpoint{6.124677in}{1.621361in}}{\pgfqpoint{6.127510in}{1.614519in}}{\pgfqpoint{6.132554in}{1.609475in}}%
\pgfpathcurveto{\pgfqpoint{6.137598in}{1.604432in}}{\pgfqpoint{6.144439in}{1.601598in}}{\pgfqpoint{6.151572in}{1.601598in}}%
\pgfpathclose%
\pgfusepath{stroke,fill}%
\end{pgfscope}%
\begin{pgfscope}%
\pgfpathrectangle{\pgfqpoint{4.985294in}{0.500000in}}{\pgfqpoint{1.764706in}{1.700000in}}%
\pgfusepath{clip}%
\pgfsetbuttcap%
\pgfsetroundjoin%
\definecolor{currentfill}{rgb}{0.968509,0.792226,0.676405}%
\pgfsetfillcolor{currentfill}%
\pgfsetlinewidth{0.311001pt}%
\definecolor{currentstroke}{rgb}{1.000000,1.000000,1.000000}%
\pgfsetstrokecolor{currentstroke}%
\pgfsetdash{}{0pt}%
\pgfpathmoveto{\pgfqpoint{5.499721in}{1.339995in}}%
\pgfpathcurveto{\pgfqpoint{5.506854in}{1.339995in}}{\pgfqpoint{5.513696in}{1.342829in}}{\pgfqpoint{5.518739in}{1.347873in}}%
\pgfpathcurveto{\pgfqpoint{5.523783in}{1.352916in}}{\pgfqpoint{5.526617in}{1.359758in}}{\pgfqpoint{5.526617in}{1.366891in}}%
\pgfpathcurveto{\pgfqpoint{5.526617in}{1.374024in}}{\pgfqpoint{5.523783in}{1.380865in}}{\pgfqpoint{5.518739in}{1.385909in}}%
\pgfpathcurveto{\pgfqpoint{5.513696in}{1.390953in}}{\pgfqpoint{5.506854in}{1.393787in}}{\pgfqpoint{5.499721in}{1.393787in}}%
\pgfpathcurveto{\pgfqpoint{5.492588in}{1.393787in}}{\pgfqpoint{5.485747in}{1.390953in}}{\pgfqpoint{5.480703in}{1.385909in}}%
\pgfpathcurveto{\pgfqpoint{5.475659in}{1.380865in}}{\pgfqpoint{5.472826in}{1.374024in}}{\pgfqpoint{5.472826in}{1.366891in}}%
\pgfpathcurveto{\pgfqpoint{5.472826in}{1.359758in}}{\pgfqpoint{5.475659in}{1.352916in}}{\pgfqpoint{5.480703in}{1.347873in}}%
\pgfpathcurveto{\pgfqpoint{5.485747in}{1.342829in}}{\pgfqpoint{5.492588in}{1.339995in}}{\pgfqpoint{5.499721in}{1.339995in}}%
\pgfpathclose%
\pgfusepath{stroke,fill}%
\end{pgfscope}%
\begin{pgfscope}%
\pgfpathrectangle{\pgfqpoint{4.985294in}{0.500000in}}{\pgfqpoint{1.764706in}{1.700000in}}%
\pgfusepath{clip}%
\pgfsetbuttcap%
\pgfsetroundjoin%
\definecolor{currentfill}{rgb}{0.974412,0.862387,0.780156}%
\pgfsetfillcolor{currentfill}%
\pgfsetlinewidth{0.311001pt}%
\definecolor{currentstroke}{rgb}{1.000000,1.000000,1.000000}%
\pgfsetstrokecolor{currentstroke}%
\pgfsetdash{}{0pt}%
\pgfpathmoveto{\pgfqpoint{5.479266in}{1.164400in}}%
\pgfpathcurveto{\pgfqpoint{5.486399in}{1.164400in}}{\pgfqpoint{5.493240in}{1.167234in}}{\pgfqpoint{5.498284in}{1.172277in}}%
\pgfpathcurveto{\pgfqpoint{5.503328in}{1.177321in}}{\pgfqpoint{5.506162in}{1.184163in}}{\pgfqpoint{5.506162in}{1.191296in}}%
\pgfpathcurveto{\pgfqpoint{5.506162in}{1.198428in}}{\pgfqpoint{5.503328in}{1.205270in}}{\pgfqpoint{5.498284in}{1.210314in}}%
\pgfpathcurveto{\pgfqpoint{5.493240in}{1.215357in}}{\pgfqpoint{5.486399in}{1.218191in}}{\pgfqpoint{5.479266in}{1.218191in}}%
\pgfpathcurveto{\pgfqpoint{5.472133in}{1.218191in}}{\pgfqpoint{5.465292in}{1.215357in}}{\pgfqpoint{5.460248in}{1.210314in}}%
\pgfpathcurveto{\pgfqpoint{5.455204in}{1.205270in}}{\pgfqpoint{5.452370in}{1.198428in}}{\pgfqpoint{5.452370in}{1.191296in}}%
\pgfpathcurveto{\pgfqpoint{5.452370in}{1.184163in}}{\pgfqpoint{5.455204in}{1.177321in}}{\pgfqpoint{5.460248in}{1.172277in}}%
\pgfpathcurveto{\pgfqpoint{5.465292in}{1.167234in}}{\pgfqpoint{5.472133in}{1.164400in}}{\pgfqpoint{5.479266in}{1.164400in}}%
\pgfpathclose%
\pgfusepath{stroke,fill}%
\end{pgfscope}%
\begin{pgfscope}%
\pgfpathrectangle{\pgfqpoint{4.985294in}{0.500000in}}{\pgfqpoint{1.764706in}{1.700000in}}%
\pgfusepath{clip}%
\pgfsetbuttcap%
\pgfsetroundjoin%
\definecolor{currentfill}{rgb}{0.967092,0.768560,0.642079}%
\pgfsetfillcolor{currentfill}%
\pgfsetlinewidth{0.311001pt}%
\definecolor{currentstroke}{rgb}{1.000000,1.000000,1.000000}%
\pgfsetstrokecolor{currentstroke}%
\pgfsetdash{}{0pt}%
\pgfpathmoveto{\pgfqpoint{5.511570in}{0.935507in}}%
\pgfpathcurveto{\pgfqpoint{5.518703in}{0.935507in}}{\pgfqpoint{5.525544in}{0.938341in}}{\pgfqpoint{5.530588in}{0.943384in}}%
\pgfpathcurveto{\pgfqpoint{5.535632in}{0.948428in}}{\pgfqpoint{5.538466in}{0.955270in}}{\pgfqpoint{5.538466in}{0.962402in}}%
\pgfpathcurveto{\pgfqpoint{5.538466in}{0.969535in}}{\pgfqpoint{5.535632in}{0.976377in}}{\pgfqpoint{5.530588in}{0.981421in}}%
\pgfpathcurveto{\pgfqpoint{5.525544in}{0.986464in}}{\pgfqpoint{5.518703in}{0.989298in}}{\pgfqpoint{5.511570in}{0.989298in}}%
\pgfpathcurveto{\pgfqpoint{5.504437in}{0.989298in}}{\pgfqpoint{5.497595in}{0.986464in}}{\pgfqpoint{5.492552in}{0.981421in}}%
\pgfpathcurveto{\pgfqpoint{5.487508in}{0.976377in}}{\pgfqpoint{5.484674in}{0.969535in}}{\pgfqpoint{5.484674in}{0.962402in}}%
\pgfpathcurveto{\pgfqpoint{5.484674in}{0.955270in}}{\pgfqpoint{5.487508in}{0.948428in}}{\pgfqpoint{5.492552in}{0.943384in}}%
\pgfpathcurveto{\pgfqpoint{5.497595in}{0.938341in}}{\pgfqpoint{5.504437in}{0.935507in}}{\pgfqpoint{5.511570in}{0.935507in}}%
\pgfpathclose%
\pgfusepath{stroke,fill}%
\end{pgfscope}%
\begin{pgfscope}%
\pgfpathrectangle{\pgfqpoint{4.985294in}{0.500000in}}{\pgfqpoint{1.764706in}{1.700000in}}%
\pgfusepath{clip}%
\pgfsetbuttcap%
\pgfsetroundjoin%
\definecolor{currentfill}{rgb}{0.972201,0.839051,0.745789}%
\pgfsetfillcolor{currentfill}%
\pgfsetlinewidth{0.311001pt}%
\definecolor{currentstroke}{rgb}{1.000000,1.000000,1.000000}%
\pgfsetstrokecolor{currentstroke}%
\pgfsetdash{}{0pt}%
\pgfpathmoveto{\pgfqpoint{6.244263in}{1.255375in}}%
\pgfpathcurveto{\pgfqpoint{6.251396in}{1.255375in}}{\pgfqpoint{6.258238in}{1.258209in}}{\pgfqpoint{6.263281in}{1.263253in}}%
\pgfpathcurveto{\pgfqpoint{6.268325in}{1.268297in}}{\pgfqpoint{6.271159in}{1.275138in}}{\pgfqpoint{6.271159in}{1.282271in}}%
\pgfpathcurveto{\pgfqpoint{6.271159in}{1.289404in}}{\pgfqpoint{6.268325in}{1.296246in}}{\pgfqpoint{6.263281in}{1.301289in}}%
\pgfpathcurveto{\pgfqpoint{6.258238in}{1.306333in}}{\pgfqpoint{6.251396in}{1.309167in}}{\pgfqpoint{6.244263in}{1.309167in}}%
\pgfpathcurveto{\pgfqpoint{6.237130in}{1.309167in}}{\pgfqpoint{6.230289in}{1.306333in}}{\pgfqpoint{6.225245in}{1.301289in}}%
\pgfpathcurveto{\pgfqpoint{6.220201in}{1.296246in}}{\pgfqpoint{6.217368in}{1.289404in}}{\pgfqpoint{6.217368in}{1.282271in}}%
\pgfpathcurveto{\pgfqpoint{6.217368in}{1.275138in}}{\pgfqpoint{6.220201in}{1.268297in}}{\pgfqpoint{6.225245in}{1.263253in}}%
\pgfpathcurveto{\pgfqpoint{6.230289in}{1.258209in}}{\pgfqpoint{6.237130in}{1.255375in}}{\pgfqpoint{6.244263in}{1.255375in}}%
\pgfpathclose%
\pgfusepath{stroke,fill}%
\end{pgfscope}%
\begin{pgfscope}%
\pgfpathrectangle{\pgfqpoint{4.985294in}{0.500000in}}{\pgfqpoint{1.764706in}{1.700000in}}%
\pgfusepath{clip}%
\pgfsetbuttcap%
\pgfsetroundjoin%
\definecolor{currentfill}{rgb}{0.962283,0.593046,0.431453}%
\pgfsetfillcolor{currentfill}%
\pgfsetlinewidth{0.311001pt}%
\definecolor{currentstroke}{rgb}{1.000000,1.000000,1.000000}%
\pgfsetstrokecolor{currentstroke}%
\pgfsetdash{}{0pt}%
\pgfpathmoveto{\pgfqpoint{6.392188in}{1.548799in}}%
\pgfpathcurveto{\pgfqpoint{6.399321in}{1.548799in}}{\pgfqpoint{6.406163in}{1.551633in}}{\pgfqpoint{6.411207in}{1.556676in}}%
\pgfpathcurveto{\pgfqpoint{6.416250in}{1.561720in}}{\pgfqpoint{6.419084in}{1.568562in}}{\pgfqpoint{6.419084in}{1.575694in}}%
\pgfpathcurveto{\pgfqpoint{6.419084in}{1.582827in}}{\pgfqpoint{6.416250in}{1.589669in}}{\pgfqpoint{6.411207in}{1.594713in}}%
\pgfpathcurveto{\pgfqpoint{6.406163in}{1.599756in}}{\pgfqpoint{6.399321in}{1.602590in}}{\pgfqpoint{6.392188in}{1.602590in}}%
\pgfpathcurveto{\pgfqpoint{6.385056in}{1.602590in}}{\pgfqpoint{6.378214in}{1.599756in}}{\pgfqpoint{6.373170in}{1.594713in}}%
\pgfpathcurveto{\pgfqpoint{6.368127in}{1.589669in}}{\pgfqpoint{6.365293in}{1.582827in}}{\pgfqpoint{6.365293in}{1.575694in}}%
\pgfpathcurveto{\pgfqpoint{6.365293in}{1.568562in}}{\pgfqpoint{6.368127in}{1.561720in}}{\pgfqpoint{6.373170in}{1.556676in}}%
\pgfpathcurveto{\pgfqpoint{6.378214in}{1.551633in}}{\pgfqpoint{6.385056in}{1.548799in}}{\pgfqpoint{6.392188in}{1.548799in}}%
\pgfpathclose%
\pgfusepath{stroke,fill}%
\end{pgfscope}%
\begin{pgfscope}%
\pgfpathrectangle{\pgfqpoint{4.985294in}{0.500000in}}{\pgfqpoint{1.764706in}{1.700000in}}%
\pgfusepath{clip}%
\pgfsetbuttcap%
\pgfsetroundjoin%
\definecolor{currentfill}{rgb}{0.962532,0.599594,0.438051}%
\pgfsetfillcolor{currentfill}%
\pgfsetlinewidth{0.311001pt}%
\definecolor{currentstroke}{rgb}{1.000000,1.000000,1.000000}%
\pgfsetstrokecolor{currentstroke}%
\pgfsetdash{}{0pt}%
\pgfpathmoveto{\pgfqpoint{6.141798in}{1.764824in}}%
\pgfpathcurveto{\pgfqpoint{6.148930in}{1.764824in}}{\pgfqpoint{6.155772in}{1.767658in}}{\pgfqpoint{6.160816in}{1.772702in}}%
\pgfpathcurveto{\pgfqpoint{6.165859in}{1.777746in}}{\pgfqpoint{6.168693in}{1.784587in}}{\pgfqpoint{6.168693in}{1.791720in}}%
\pgfpathcurveto{\pgfqpoint{6.168693in}{1.798853in}}{\pgfqpoint{6.165859in}{1.805694in}}{\pgfqpoint{6.160816in}{1.810738in}}%
\pgfpathcurveto{\pgfqpoint{6.155772in}{1.815782in}}{\pgfqpoint{6.148930in}{1.818616in}}{\pgfqpoint{6.141798in}{1.818616in}}%
\pgfpathcurveto{\pgfqpoint{6.134665in}{1.818616in}}{\pgfqpoint{6.127823in}{1.815782in}}{\pgfqpoint{6.122779in}{1.810738in}}%
\pgfpathcurveto{\pgfqpoint{6.117736in}{1.805694in}}{\pgfqpoint{6.114902in}{1.798853in}}{\pgfqpoint{6.114902in}{1.791720in}}%
\pgfpathcurveto{\pgfqpoint{6.114902in}{1.784587in}}{\pgfqpoint{6.117736in}{1.777746in}}{\pgfqpoint{6.122779in}{1.772702in}}%
\pgfpathcurveto{\pgfqpoint{6.127823in}{1.767658in}}{\pgfqpoint{6.134665in}{1.764824in}}{\pgfqpoint{6.141798in}{1.764824in}}%
\pgfpathclose%
\pgfusepath{stroke,fill}%
\end{pgfscope}%
\begin{pgfscope}%
\pgfpathrectangle{\pgfqpoint{4.985294in}{0.500000in}}{\pgfqpoint{1.764706in}{1.700000in}}%
\pgfusepath{clip}%
\pgfsetbuttcap%
\pgfsetroundjoin%
\definecolor{currentfill}{rgb}{0.980678,0.914765,0.856766}%
\pgfsetfillcolor{currentfill}%
\pgfsetlinewidth{0.311001pt}%
\definecolor{currentstroke}{rgb}{1.000000,1.000000,1.000000}%
\pgfsetstrokecolor{currentstroke}%
\pgfsetdash{}{0pt}%
\pgfpathmoveto{\pgfqpoint{6.302093in}{1.506221in}}%
\pgfpathcurveto{\pgfqpoint{6.309226in}{1.506221in}}{\pgfqpoint{6.316068in}{1.509055in}}{\pgfqpoint{6.321111in}{1.514099in}}%
\pgfpathcurveto{\pgfqpoint{6.326155in}{1.519142in}}{\pgfqpoint{6.328989in}{1.525984in}}{\pgfqpoint{6.328989in}{1.533117in}}%
\pgfpathcurveto{\pgfqpoint{6.328989in}{1.540250in}}{\pgfqpoint{6.326155in}{1.547091in}}{\pgfqpoint{6.321111in}{1.552135in}}%
\pgfpathcurveto{\pgfqpoint{6.316068in}{1.557179in}}{\pgfqpoint{6.309226in}{1.560013in}}{\pgfqpoint{6.302093in}{1.560013in}}%
\pgfpathcurveto{\pgfqpoint{6.294961in}{1.560013in}}{\pgfqpoint{6.288119in}{1.557179in}}{\pgfqpoint{6.283075in}{1.552135in}}%
\pgfpathcurveto{\pgfqpoint{6.278032in}{1.547091in}}{\pgfqpoint{6.275198in}{1.540250in}}{\pgfqpoint{6.275198in}{1.533117in}}%
\pgfpathcurveto{\pgfqpoint{6.275198in}{1.525984in}}{\pgfqpoint{6.278032in}{1.519142in}}{\pgfqpoint{6.283075in}{1.514099in}}%
\pgfpathcurveto{\pgfqpoint{6.288119in}{1.509055in}}{\pgfqpoint{6.294961in}{1.506221in}}{\pgfqpoint{6.302093in}{1.506221in}}%
\pgfpathclose%
\pgfusepath{stroke,fill}%
\end{pgfscope}%
\begin{pgfscope}%
\pgfpathrectangle{\pgfqpoint{4.985294in}{0.500000in}}{\pgfqpoint{1.764706in}{1.700000in}}%
\pgfusepath{clip}%
\pgfsetbuttcap%
\pgfsetroundjoin%
\definecolor{currentfill}{rgb}{0.978376,0.897317,0.831308}%
\pgfsetfillcolor{currentfill}%
\pgfsetlinewidth{0.311001pt}%
\definecolor{currentstroke}{rgb}{1.000000,1.000000,1.000000}%
\pgfsetstrokecolor{currentstroke}%
\pgfsetdash{}{0pt}%
\pgfpathmoveto{\pgfqpoint{6.329894in}{1.450162in}}%
\pgfpathcurveto{\pgfqpoint{6.337027in}{1.450162in}}{\pgfqpoint{6.343869in}{1.452996in}}{\pgfqpoint{6.348913in}{1.458040in}}%
\pgfpathcurveto{\pgfqpoint{6.353956in}{1.463084in}}{\pgfqpoint{6.356790in}{1.469925in}}{\pgfqpoint{6.356790in}{1.477058in}}%
\pgfpathcurveto{\pgfqpoint{6.356790in}{1.484191in}}{\pgfqpoint{6.353956in}{1.491033in}}{\pgfqpoint{6.348913in}{1.496076in}}%
\pgfpathcurveto{\pgfqpoint{6.343869in}{1.501120in}}{\pgfqpoint{6.337027in}{1.503954in}}{\pgfqpoint{6.329894in}{1.503954in}}%
\pgfpathcurveto{\pgfqpoint{6.322762in}{1.503954in}}{\pgfqpoint{6.315920in}{1.501120in}}{\pgfqpoint{6.310876in}{1.496076in}}%
\pgfpathcurveto{\pgfqpoint{6.305833in}{1.491033in}}{\pgfqpoint{6.302999in}{1.484191in}}{\pgfqpoint{6.302999in}{1.477058in}}%
\pgfpathcurveto{\pgfqpoint{6.302999in}{1.469925in}}{\pgfqpoint{6.305833in}{1.463084in}}{\pgfqpoint{6.310876in}{1.458040in}}%
\pgfpathcurveto{\pgfqpoint{6.315920in}{1.452996in}}{\pgfqpoint{6.322762in}{1.450162in}}{\pgfqpoint{6.329894in}{1.450162in}}%
\pgfpathclose%
\pgfusepath{stroke,fill}%
\end{pgfscope}%
\begin{pgfscope}%
\pgfpathrectangle{\pgfqpoint{4.985294in}{0.500000in}}{\pgfqpoint{1.764706in}{1.700000in}}%
\pgfusepath{clip}%
\pgfsetbuttcap%
\pgfsetroundjoin%
\definecolor{currentfill}{rgb}{0.941676,0.367866,0.260395}%
\pgfsetfillcolor{currentfill}%
\pgfsetlinewidth{0.311001pt}%
\definecolor{currentstroke}{rgb}{1.000000,1.000000,1.000000}%
\pgfsetstrokecolor{currentstroke}%
\pgfsetdash{}{0pt}%
\pgfpathmoveto{\pgfqpoint{6.461669in}{1.270907in}}%
\pgfpathcurveto{\pgfqpoint{6.468802in}{1.270907in}}{\pgfqpoint{6.475644in}{1.273741in}}{\pgfqpoint{6.480687in}{1.278785in}}%
\pgfpathcurveto{\pgfqpoint{6.485731in}{1.283828in}}{\pgfqpoint{6.488565in}{1.290670in}}{\pgfqpoint{6.488565in}{1.297803in}}%
\pgfpathcurveto{\pgfqpoint{6.488565in}{1.304936in}}{\pgfqpoint{6.485731in}{1.311777in}}{\pgfqpoint{6.480687in}{1.316821in}}%
\pgfpathcurveto{\pgfqpoint{6.475644in}{1.321865in}}{\pgfqpoint{6.468802in}{1.324698in}}{\pgfqpoint{6.461669in}{1.324698in}}%
\pgfpathcurveto{\pgfqpoint{6.454536in}{1.324698in}}{\pgfqpoint{6.447695in}{1.321865in}}{\pgfqpoint{6.442651in}{1.316821in}}%
\pgfpathcurveto{\pgfqpoint{6.437607in}{1.311777in}}{\pgfqpoint{6.434773in}{1.304936in}}{\pgfqpoint{6.434773in}{1.297803in}}%
\pgfpathcurveto{\pgfqpoint{6.434773in}{1.290670in}}{\pgfqpoint{6.437607in}{1.283828in}}{\pgfqpoint{6.442651in}{1.278785in}}%
\pgfpathcurveto{\pgfqpoint{6.447695in}{1.273741in}}{\pgfqpoint{6.454536in}{1.270907in}}{\pgfqpoint{6.461669in}{1.270907in}}%
\pgfpathclose%
\pgfusepath{stroke,fill}%
\end{pgfscope}%
\begin{pgfscope}%
\pgfpathrectangle{\pgfqpoint{4.985294in}{0.500000in}}{\pgfqpoint{1.764706in}{1.700000in}}%
\pgfusepath{clip}%
\pgfsetbuttcap%
\pgfsetroundjoin%
\definecolor{currentfill}{rgb}{0.979124,0.903132,0.839793}%
\pgfsetfillcolor{currentfill}%
\pgfsetlinewidth{0.311001pt}%
\definecolor{currentstroke}{rgb}{1.000000,1.000000,1.000000}%
\pgfsetstrokecolor{currentstroke}%
\pgfsetdash{}{0pt}%
\pgfpathmoveto{\pgfqpoint{5.439807in}{1.310889in}}%
\pgfpathcurveto{\pgfqpoint{5.446940in}{1.310889in}}{\pgfqpoint{5.453782in}{1.313723in}}{\pgfqpoint{5.458825in}{1.318767in}}%
\pgfpathcurveto{\pgfqpoint{5.463869in}{1.323811in}}{\pgfqpoint{5.466703in}{1.330652in}}{\pgfqpoint{5.466703in}{1.337785in}}%
\pgfpathcurveto{\pgfqpoint{5.466703in}{1.344918in}}{\pgfqpoint{5.463869in}{1.351760in}}{\pgfqpoint{5.458825in}{1.356803in}}%
\pgfpathcurveto{\pgfqpoint{5.453782in}{1.361847in}}{\pgfqpoint{5.446940in}{1.364681in}}{\pgfqpoint{5.439807in}{1.364681in}}%
\pgfpathcurveto{\pgfqpoint{5.432674in}{1.364681in}}{\pgfqpoint{5.425833in}{1.361847in}}{\pgfqpoint{5.420789in}{1.356803in}}%
\pgfpathcurveto{\pgfqpoint{5.415745in}{1.351760in}}{\pgfqpoint{5.412912in}{1.344918in}}{\pgfqpoint{5.412912in}{1.337785in}}%
\pgfpathcurveto{\pgfqpoint{5.412912in}{1.330652in}}{\pgfqpoint{5.415745in}{1.323811in}}{\pgfqpoint{5.420789in}{1.318767in}}%
\pgfpathcurveto{\pgfqpoint{5.425833in}{1.313723in}}{\pgfqpoint{5.432674in}{1.310889in}}{\pgfqpoint{5.439807in}{1.310889in}}%
\pgfpathclose%
\pgfusepath{stroke,fill}%
\end{pgfscope}%
\begin{pgfscope}%
\pgfpathrectangle{\pgfqpoint{4.985294in}{0.500000in}}{\pgfqpoint{1.764706in}{1.700000in}}%
\pgfusepath{clip}%
\pgfsetbuttcap%
\pgfsetroundjoin%
\definecolor{currentfill}{rgb}{0.946260,0.398132,0.274897}%
\pgfsetfillcolor{currentfill}%
\pgfsetlinewidth{0.311001pt}%
\definecolor{currentstroke}{rgb}{1.000000,1.000000,1.000000}%
\pgfsetstrokecolor{currentstroke}%
\pgfsetdash{}{0pt}%
\pgfpathmoveto{\pgfqpoint{5.574734in}{1.327881in}}%
\pgfpathcurveto{\pgfqpoint{5.581867in}{1.327881in}}{\pgfqpoint{5.588708in}{1.330715in}}{\pgfqpoint{5.593752in}{1.335759in}}%
\pgfpathcurveto{\pgfqpoint{5.598796in}{1.340803in}}{\pgfqpoint{5.601630in}{1.347644in}}{\pgfqpoint{5.601630in}{1.354777in}}%
\pgfpathcurveto{\pgfqpoint{5.601630in}{1.361910in}}{\pgfqpoint{5.598796in}{1.368752in}}{\pgfqpoint{5.593752in}{1.373795in}}%
\pgfpathcurveto{\pgfqpoint{5.588708in}{1.378839in}}{\pgfqpoint{5.581867in}{1.381673in}}{\pgfqpoint{5.574734in}{1.381673in}}%
\pgfpathcurveto{\pgfqpoint{5.567601in}{1.381673in}}{\pgfqpoint{5.560759in}{1.378839in}}{\pgfqpoint{5.555716in}{1.373795in}}%
\pgfpathcurveto{\pgfqpoint{5.550672in}{1.368752in}}{\pgfqpoint{5.547838in}{1.361910in}}{\pgfqpoint{5.547838in}{1.354777in}}%
\pgfpathcurveto{\pgfqpoint{5.547838in}{1.347644in}}{\pgfqpoint{5.550672in}{1.340803in}}{\pgfqpoint{5.555716in}{1.335759in}}%
\pgfpathcurveto{\pgfqpoint{5.560759in}{1.330715in}}{\pgfqpoint{5.567601in}{1.327881in}}{\pgfqpoint{5.574734in}{1.327881in}}%
\pgfpathclose%
\pgfusepath{stroke,fill}%
\end{pgfscope}%
\begin{pgfscope}%
\pgfpathrectangle{\pgfqpoint{4.985294in}{0.500000in}}{\pgfqpoint{1.764706in}{1.700000in}}%
\pgfusepath{clip}%
\pgfsetbuttcap%
\pgfsetroundjoin%
\definecolor{currentfill}{rgb}{0.972726,0.844889,0.754401}%
\pgfsetfillcolor{currentfill}%
\pgfsetlinewidth{0.311001pt}%
\definecolor{currentstroke}{rgb}{1.000000,1.000000,1.000000}%
\pgfsetstrokecolor{currentstroke}%
\pgfsetdash{}{0pt}%
\pgfpathmoveto{\pgfqpoint{6.263970in}{1.640293in}}%
\pgfpathcurveto{\pgfqpoint{6.271103in}{1.640293in}}{\pgfqpoint{6.277944in}{1.643127in}}{\pgfqpoint{6.282988in}{1.648170in}}%
\pgfpathcurveto{\pgfqpoint{6.288031in}{1.653214in}}{\pgfqpoint{6.290865in}{1.660056in}}{\pgfqpoint{6.290865in}{1.667189in}}%
\pgfpathcurveto{\pgfqpoint{6.290865in}{1.674321in}}{\pgfqpoint{6.288031in}{1.681163in}}{\pgfqpoint{6.282988in}{1.686207in}}%
\pgfpathcurveto{\pgfqpoint{6.277944in}{1.691250in}}{\pgfqpoint{6.271103in}{1.694084in}}{\pgfqpoint{6.263970in}{1.694084in}}%
\pgfpathcurveto{\pgfqpoint{6.256837in}{1.694084in}}{\pgfqpoint{6.249995in}{1.691250in}}{\pgfqpoint{6.244952in}{1.686207in}}%
\pgfpathcurveto{\pgfqpoint{6.239908in}{1.681163in}}{\pgfqpoint{6.237074in}{1.674321in}}{\pgfqpoint{6.237074in}{1.667189in}}%
\pgfpathcurveto{\pgfqpoint{6.237074in}{1.660056in}}{\pgfqpoint{6.239908in}{1.653214in}}{\pgfqpoint{6.244952in}{1.648170in}}%
\pgfpathcurveto{\pgfqpoint{6.249995in}{1.643127in}}{\pgfqpoint{6.256837in}{1.640293in}}{\pgfqpoint{6.263970in}{1.640293in}}%
\pgfpathclose%
\pgfusepath{stroke,fill}%
\end{pgfscope}%
\begin{pgfscope}%
\pgfpathrectangle{\pgfqpoint{4.985294in}{0.500000in}}{\pgfqpoint{1.764706in}{1.700000in}}%
\pgfusepath{clip}%
\pgfsetbuttcap%
\pgfsetroundjoin%
\definecolor{currentfill}{rgb}{0.972201,0.839051,0.745789}%
\pgfsetfillcolor{currentfill}%
\pgfsetlinewidth{0.311001pt}%
\definecolor{currentstroke}{rgb}{1.000000,1.000000,1.000000}%
\pgfsetstrokecolor{currentstroke}%
\pgfsetdash{}{0pt}%
\pgfpathmoveto{\pgfqpoint{5.418832in}{1.553201in}}%
\pgfpathcurveto{\pgfqpoint{5.425965in}{1.553201in}}{\pgfqpoint{5.432807in}{1.556035in}}{\pgfqpoint{5.437850in}{1.561079in}}%
\pgfpathcurveto{\pgfqpoint{5.442894in}{1.566122in}}{\pgfqpoint{5.445728in}{1.572964in}}{\pgfqpoint{5.445728in}{1.580097in}}%
\pgfpathcurveto{\pgfqpoint{5.445728in}{1.587230in}}{\pgfqpoint{5.442894in}{1.594071in}}{\pgfqpoint{5.437850in}{1.599115in}}%
\pgfpathcurveto{\pgfqpoint{5.432807in}{1.604159in}}{\pgfqpoint{5.425965in}{1.606993in}}{\pgfqpoint{5.418832in}{1.606993in}}%
\pgfpathcurveto{\pgfqpoint{5.411699in}{1.606993in}}{\pgfqpoint{5.404858in}{1.604159in}}{\pgfqpoint{5.399814in}{1.599115in}}%
\pgfpathcurveto{\pgfqpoint{5.394770in}{1.594071in}}{\pgfqpoint{5.391937in}{1.587230in}}{\pgfqpoint{5.391937in}{1.580097in}}%
\pgfpathcurveto{\pgfqpoint{5.391937in}{1.572964in}}{\pgfqpoint{5.394770in}{1.566122in}}{\pgfqpoint{5.399814in}{1.561079in}}%
\pgfpathcurveto{\pgfqpoint{5.404858in}{1.556035in}}{\pgfqpoint{5.411699in}{1.553201in}}{\pgfqpoint{5.418832in}{1.553201in}}%
\pgfpathclose%
\pgfusepath{stroke,fill}%
\end{pgfscope}%
\begin{pgfscope}%
\pgfpathrectangle{\pgfqpoint{4.985294in}{0.500000in}}{\pgfqpoint{1.764706in}{1.700000in}}%
\pgfusepath{clip}%
\pgfsetbuttcap%
\pgfsetroundjoin%
\definecolor{currentfill}{rgb}{0.964799,0.689101,0.537560}%
\pgfsetfillcolor{currentfill}%
\pgfsetlinewidth{0.311001pt}%
\definecolor{currentstroke}{rgb}{1.000000,1.000000,1.000000}%
\pgfsetstrokecolor{currentstroke}%
\pgfsetdash{}{0pt}%
\pgfpathmoveto{\pgfqpoint{6.156363in}{0.904397in}}%
\pgfpathcurveto{\pgfqpoint{6.163496in}{0.904397in}}{\pgfqpoint{6.170338in}{0.907231in}}{\pgfqpoint{6.175382in}{0.912275in}}%
\pgfpathcurveto{\pgfqpoint{6.180425in}{0.917318in}}{\pgfqpoint{6.183259in}{0.924160in}}{\pgfqpoint{6.183259in}{0.931293in}}%
\pgfpathcurveto{\pgfqpoint{6.183259in}{0.938426in}}{\pgfqpoint{6.180425in}{0.945267in}}{\pgfqpoint{6.175382in}{0.950311in}}%
\pgfpathcurveto{\pgfqpoint{6.170338in}{0.955355in}}{\pgfqpoint{6.163496in}{0.958189in}}{\pgfqpoint{6.156363in}{0.958189in}}%
\pgfpathcurveto{\pgfqpoint{6.149231in}{0.958189in}}{\pgfqpoint{6.142389in}{0.955355in}}{\pgfqpoint{6.137345in}{0.950311in}}%
\pgfpathcurveto{\pgfqpoint{6.132302in}{0.945267in}}{\pgfqpoint{6.129468in}{0.938426in}}{\pgfqpoint{6.129468in}{0.931293in}}%
\pgfpathcurveto{\pgfqpoint{6.129468in}{0.924160in}}{\pgfqpoint{6.132302in}{0.917318in}}{\pgfqpoint{6.137345in}{0.912275in}}%
\pgfpathcurveto{\pgfqpoint{6.142389in}{0.907231in}}{\pgfqpoint{6.149231in}{0.904397in}}{\pgfqpoint{6.156363in}{0.904397in}}%
\pgfpathclose%
\pgfusepath{stroke,fill}%
\end{pgfscope}%
\begin{pgfscope}%
\pgfpathrectangle{\pgfqpoint{4.985294in}{0.500000in}}{\pgfqpoint{1.764706in}{1.700000in}}%
\pgfusepath{clip}%
\pgfsetbuttcap%
\pgfsetroundjoin%
\definecolor{currentfill}{rgb}{0.959229,0.533075,0.374889}%
\pgfsetfillcolor{currentfill}%
\pgfsetlinewidth{0.311001pt}%
\definecolor{currentstroke}{rgb}{1.000000,1.000000,1.000000}%
\pgfsetstrokecolor{currentstroke}%
\pgfsetdash{}{0pt}%
\pgfpathmoveto{\pgfqpoint{6.131433in}{1.786489in}}%
\pgfpathcurveto{\pgfqpoint{6.138566in}{1.786489in}}{\pgfqpoint{6.145408in}{1.789323in}}{\pgfqpoint{6.150451in}{1.794367in}}%
\pgfpathcurveto{\pgfqpoint{6.155495in}{1.799410in}}{\pgfqpoint{6.158329in}{1.806252in}}{\pgfqpoint{6.158329in}{1.813385in}}%
\pgfpathcurveto{\pgfqpoint{6.158329in}{1.820518in}}{\pgfqpoint{6.155495in}{1.827359in}}{\pgfqpoint{6.150451in}{1.832403in}}%
\pgfpathcurveto{\pgfqpoint{6.145408in}{1.837447in}}{\pgfqpoint{6.138566in}{1.840280in}}{\pgfqpoint{6.131433in}{1.840280in}}%
\pgfpathcurveto{\pgfqpoint{6.124300in}{1.840280in}}{\pgfqpoint{6.117459in}{1.837447in}}{\pgfqpoint{6.112415in}{1.832403in}}%
\pgfpathcurveto{\pgfqpoint{6.107371in}{1.827359in}}{\pgfqpoint{6.104537in}{1.820518in}}{\pgfqpoint{6.104537in}{1.813385in}}%
\pgfpathcurveto{\pgfqpoint{6.104537in}{1.806252in}}{\pgfqpoint{6.107371in}{1.799410in}}{\pgfqpoint{6.112415in}{1.794367in}}%
\pgfpathcurveto{\pgfqpoint{6.117459in}{1.789323in}}{\pgfqpoint{6.124300in}{1.786489in}}{\pgfqpoint{6.131433in}{1.786489in}}%
\pgfpathclose%
\pgfusepath{stroke,fill}%
\end{pgfscope}%
\begin{pgfscope}%
\pgfpathrectangle{\pgfqpoint{4.985294in}{0.500000in}}{\pgfqpoint{1.764706in}{1.700000in}}%
\pgfusepath{clip}%
\pgfsetbuttcap%
\pgfsetroundjoin%
\definecolor{currentfill}{rgb}{0.942910,0.375495,0.263698}%
\pgfsetfillcolor{currentfill}%
\pgfsetlinewidth{0.311001pt}%
\definecolor{currentstroke}{rgb}{1.000000,1.000000,1.000000}%
\pgfsetstrokecolor{currentstroke}%
\pgfsetdash{}{0pt}%
\pgfpathmoveto{\pgfqpoint{5.587847in}{1.224362in}}%
\pgfpathcurveto{\pgfqpoint{5.594979in}{1.224362in}}{\pgfqpoint{5.601821in}{1.227196in}}{\pgfqpoint{5.606865in}{1.232239in}}%
\pgfpathcurveto{\pgfqpoint{5.611908in}{1.237283in}}{\pgfqpoint{5.614742in}{1.244125in}}{\pgfqpoint{5.614742in}{1.251258in}}%
\pgfpathcurveto{\pgfqpoint{5.614742in}{1.258390in}}{\pgfqpoint{5.611908in}{1.265232in}}{\pgfqpoint{5.606865in}{1.270276in}}%
\pgfpathcurveto{\pgfqpoint{5.601821in}{1.275319in}}{\pgfqpoint{5.594979in}{1.278153in}}{\pgfqpoint{5.587847in}{1.278153in}}%
\pgfpathcurveto{\pgfqpoint{5.580714in}{1.278153in}}{\pgfqpoint{5.573872in}{1.275319in}}{\pgfqpoint{5.568828in}{1.270276in}}%
\pgfpathcurveto{\pgfqpoint{5.563785in}{1.265232in}}{\pgfqpoint{5.560951in}{1.258390in}}{\pgfqpoint{5.560951in}{1.251258in}}%
\pgfpathcurveto{\pgfqpoint{5.560951in}{1.244125in}}{\pgfqpoint{5.563785in}{1.237283in}}{\pgfqpoint{5.568828in}{1.232239in}}%
\pgfpathcurveto{\pgfqpoint{5.573872in}{1.227196in}}{\pgfqpoint{5.580714in}{1.224362in}}{\pgfqpoint{5.587847in}{1.224362in}}%
\pgfpathclose%
\pgfusepath{stroke,fill}%
\end{pgfscope}%
\begin{pgfscope}%
\pgfpathrectangle{\pgfqpoint{4.985294in}{0.500000in}}{\pgfqpoint{1.764706in}{1.700000in}}%
\pgfusepath{clip}%
\pgfsetbuttcap%
\pgfsetroundjoin%
\definecolor{currentfill}{rgb}{0.977657,0.891500,0.822809}%
\pgfsetfillcolor{currentfill}%
\pgfsetlinewidth{0.311001pt}%
\definecolor{currentstroke}{rgb}{1.000000,1.000000,1.000000}%
\pgfsetstrokecolor{currentstroke}%
\pgfsetdash{}{0pt}%
\pgfpathmoveto{\pgfqpoint{5.453306in}{1.156883in}}%
\pgfpathcurveto{\pgfqpoint{5.460439in}{1.156883in}}{\pgfqpoint{5.467280in}{1.159717in}}{\pgfqpoint{5.472324in}{1.164760in}}%
\pgfpathcurveto{\pgfqpoint{5.477368in}{1.169804in}}{\pgfqpoint{5.480201in}{1.176646in}}{\pgfqpoint{5.480201in}{1.183779in}}%
\pgfpathcurveto{\pgfqpoint{5.480201in}{1.190911in}}{\pgfqpoint{5.477368in}{1.197753in}}{\pgfqpoint{5.472324in}{1.202797in}}%
\pgfpathcurveto{\pgfqpoint{5.467280in}{1.207840in}}{\pgfqpoint{5.460439in}{1.210674in}}{\pgfqpoint{5.453306in}{1.210674in}}%
\pgfpathcurveto{\pgfqpoint{5.446173in}{1.210674in}}{\pgfqpoint{5.439331in}{1.207840in}}{\pgfqpoint{5.434288in}{1.202797in}}%
\pgfpathcurveto{\pgfqpoint{5.429244in}{1.197753in}}{\pgfqpoint{5.426410in}{1.190911in}}{\pgfqpoint{5.426410in}{1.183779in}}%
\pgfpathcurveto{\pgfqpoint{5.426410in}{1.176646in}}{\pgfqpoint{5.429244in}{1.169804in}}{\pgfqpoint{5.434288in}{1.164760in}}%
\pgfpathcurveto{\pgfqpoint{5.439331in}{1.159717in}}{\pgfqpoint{5.446173in}{1.156883in}}{\pgfqpoint{5.453306in}{1.156883in}}%
\pgfpathclose%
\pgfusepath{stroke,fill}%
\end{pgfscope}%
\begin{pgfscope}%
\pgfpathrectangle{\pgfqpoint{4.985294in}{0.500000in}}{\pgfqpoint{1.764706in}{1.700000in}}%
\pgfusepath{clip}%
\pgfsetbuttcap%
\pgfsetroundjoin%
\definecolor{currentfill}{rgb}{0.967398,0.774513,0.650573}%
\pgfsetfillcolor{currentfill}%
\pgfsetlinewidth{0.311001pt}%
\definecolor{currentstroke}{rgb}{1.000000,1.000000,1.000000}%
\pgfsetstrokecolor{currentstroke}%
\pgfsetdash{}{0pt}%
\pgfpathmoveto{\pgfqpoint{5.406618in}{1.041850in}}%
\pgfpathcurveto{\pgfqpoint{5.413751in}{1.041850in}}{\pgfqpoint{5.420593in}{1.044684in}}{\pgfqpoint{5.425637in}{1.049728in}}%
\pgfpathcurveto{\pgfqpoint{5.430680in}{1.054771in}}{\pgfqpoint{5.433514in}{1.061613in}}{\pgfqpoint{5.433514in}{1.068746in}}%
\pgfpathcurveto{\pgfqpoint{5.433514in}{1.075879in}}{\pgfqpoint{5.430680in}{1.082720in}}{\pgfqpoint{5.425637in}{1.087764in}}%
\pgfpathcurveto{\pgfqpoint{5.420593in}{1.092808in}}{\pgfqpoint{5.413751in}{1.095642in}}{\pgfqpoint{5.406618in}{1.095642in}}%
\pgfpathcurveto{\pgfqpoint{5.399486in}{1.095642in}}{\pgfqpoint{5.392644in}{1.092808in}}{\pgfqpoint{5.387600in}{1.087764in}}%
\pgfpathcurveto{\pgfqpoint{5.382557in}{1.082720in}}{\pgfqpoint{5.379723in}{1.075879in}}{\pgfqpoint{5.379723in}{1.068746in}}%
\pgfpathcurveto{\pgfqpoint{5.379723in}{1.061613in}}{\pgfqpoint{5.382557in}{1.054771in}}{\pgfqpoint{5.387600in}{1.049728in}}%
\pgfpathcurveto{\pgfqpoint{5.392644in}{1.044684in}}{\pgfqpoint{5.399486in}{1.041850in}}{\pgfqpoint{5.406618in}{1.041850in}}%
\pgfpathclose%
\pgfusepath{stroke,fill}%
\end{pgfscope}%
\begin{pgfscope}%
\pgfpathrectangle{\pgfqpoint{4.985294in}{0.500000in}}{\pgfqpoint{1.764706in}{1.700000in}}%
\pgfusepath{clip}%
\pgfsetbuttcap%
\pgfsetroundjoin%
\definecolor{currentfill}{rgb}{0.979124,0.903132,0.839793}%
\pgfsetfillcolor{currentfill}%
\pgfsetlinewidth{0.311001pt}%
\definecolor{currentstroke}{rgb}{1.000000,1.000000,1.000000}%
\pgfsetstrokecolor{currentstroke}%
\pgfsetdash{}{0pt}%
\pgfpathmoveto{\pgfqpoint{6.290060in}{1.388620in}}%
\pgfpathcurveto{\pgfqpoint{6.297192in}{1.388620in}}{\pgfqpoint{6.304034in}{1.391454in}}{\pgfqpoint{6.309078in}{1.396498in}}%
\pgfpathcurveto{\pgfqpoint{6.314121in}{1.401541in}}{\pgfqpoint{6.316955in}{1.408383in}}{\pgfqpoint{6.316955in}{1.415516in}}%
\pgfpathcurveto{\pgfqpoint{6.316955in}{1.422649in}}{\pgfqpoint{6.314121in}{1.429490in}}{\pgfqpoint{6.309078in}{1.434534in}}%
\pgfpathcurveto{\pgfqpoint{6.304034in}{1.439578in}}{\pgfqpoint{6.297192in}{1.442411in}}{\pgfqpoint{6.290060in}{1.442411in}}%
\pgfpathcurveto{\pgfqpoint{6.282927in}{1.442411in}}{\pgfqpoint{6.276085in}{1.439578in}}{\pgfqpoint{6.271041in}{1.434534in}}%
\pgfpathcurveto{\pgfqpoint{6.265998in}{1.429490in}}{\pgfqpoint{6.263164in}{1.422649in}}{\pgfqpoint{6.263164in}{1.415516in}}%
\pgfpathcurveto{\pgfqpoint{6.263164in}{1.408383in}}{\pgfqpoint{6.265998in}{1.401541in}}{\pgfqpoint{6.271041in}{1.396498in}}%
\pgfpathcurveto{\pgfqpoint{6.276085in}{1.391454in}}{\pgfqpoint{6.282927in}{1.388620in}}{\pgfqpoint{6.290060in}{1.388620in}}%
\pgfpathclose%
\pgfusepath{stroke,fill}%
\end{pgfscope}%
\begin{pgfscope}%
\pgfpathrectangle{\pgfqpoint{4.985294in}{0.500000in}}{\pgfqpoint{1.764706in}{1.700000in}}%
\pgfusepath{clip}%
\pgfsetbuttcap%
\pgfsetroundjoin%
\definecolor{currentfill}{rgb}{0.967735,0.780441,0.659127}%
\pgfsetfillcolor{currentfill}%
\pgfsetlinewidth{0.311001pt}%
\definecolor{currentstroke}{rgb}{1.000000,1.000000,1.000000}%
\pgfsetstrokecolor{currentstroke}%
\pgfsetdash{}{0pt}%
\pgfpathmoveto{\pgfqpoint{5.502778in}{1.677379in}}%
\pgfpathcurveto{\pgfqpoint{5.509911in}{1.677379in}}{\pgfqpoint{5.516753in}{1.680213in}}{\pgfqpoint{5.521797in}{1.685257in}}%
\pgfpathcurveto{\pgfqpoint{5.526840in}{1.690301in}}{\pgfqpoint{5.529674in}{1.697142in}}{\pgfqpoint{5.529674in}{1.704275in}}%
\pgfpathcurveto{\pgfqpoint{5.529674in}{1.711408in}}{\pgfqpoint{5.526840in}{1.718250in}}{\pgfqpoint{5.521797in}{1.723293in}}%
\pgfpathcurveto{\pgfqpoint{5.516753in}{1.728337in}}{\pgfqpoint{5.509911in}{1.731171in}}{\pgfqpoint{5.502778in}{1.731171in}}%
\pgfpathcurveto{\pgfqpoint{5.495646in}{1.731171in}}{\pgfqpoint{5.488804in}{1.728337in}}{\pgfqpoint{5.483760in}{1.723293in}}%
\pgfpathcurveto{\pgfqpoint{5.478717in}{1.718250in}}{\pgfqpoint{5.475883in}{1.711408in}}{\pgfqpoint{5.475883in}{1.704275in}}%
\pgfpathcurveto{\pgfqpoint{5.475883in}{1.697142in}}{\pgfqpoint{5.478717in}{1.690301in}}{\pgfqpoint{5.483760in}{1.685257in}}%
\pgfpathcurveto{\pgfqpoint{5.488804in}{1.680213in}}{\pgfqpoint{5.495646in}{1.677379in}}{\pgfqpoint{5.502778in}{1.677379in}}%
\pgfpathclose%
\pgfusepath{stroke,fill}%
\end{pgfscope}%
\begin{pgfscope}%
\pgfpathrectangle{\pgfqpoint{4.985294in}{0.500000in}}{\pgfqpoint{1.764706in}{1.700000in}}%
\pgfusepath{clip}%
\pgfsetbuttcap%
\pgfsetroundjoin%
\definecolor{currentfill}{rgb}{0.968509,0.792226,0.676405}%
\pgfsetfillcolor{currentfill}%
\pgfsetlinewidth{0.311001pt}%
\definecolor{currentstroke}{rgb}{1.000000,1.000000,1.000000}%
\pgfsetstrokecolor{currentstroke}%
\pgfsetdash{}{0pt}%
\pgfpathmoveto{\pgfqpoint{6.355042in}{1.123249in}}%
\pgfpathcurveto{\pgfqpoint{6.362175in}{1.123249in}}{\pgfqpoint{6.369016in}{1.126083in}}{\pgfqpoint{6.374060in}{1.131127in}}%
\pgfpathcurveto{\pgfqpoint{6.379104in}{1.136170in}}{\pgfqpoint{6.381938in}{1.143012in}}{\pgfqpoint{6.381938in}{1.150145in}}%
\pgfpathcurveto{\pgfqpoint{6.381938in}{1.157278in}}{\pgfqpoint{6.379104in}{1.164119in}}{\pgfqpoint{6.374060in}{1.169163in}}%
\pgfpathcurveto{\pgfqpoint{6.369016in}{1.174207in}}{\pgfqpoint{6.362175in}{1.177041in}}{\pgfqpoint{6.355042in}{1.177041in}}%
\pgfpathcurveto{\pgfqpoint{6.347909in}{1.177041in}}{\pgfqpoint{6.341067in}{1.174207in}}{\pgfqpoint{6.336024in}{1.169163in}}%
\pgfpathcurveto{\pgfqpoint{6.330980in}{1.164119in}}{\pgfqpoint{6.328146in}{1.157278in}}{\pgfqpoint{6.328146in}{1.150145in}}%
\pgfpathcurveto{\pgfqpoint{6.328146in}{1.143012in}}{\pgfqpoint{6.330980in}{1.136170in}}{\pgfqpoint{6.336024in}{1.131127in}}%
\pgfpathcurveto{\pgfqpoint{6.341067in}{1.126083in}}{\pgfqpoint{6.347909in}{1.123249in}}{\pgfqpoint{6.355042in}{1.123249in}}%
\pgfpathclose%
\pgfusepath{stroke,fill}%
\end{pgfscope}%
\begin{pgfscope}%
\pgfpathrectangle{\pgfqpoint{4.985294in}{0.500000in}}{\pgfqpoint{1.764706in}{1.700000in}}%
\pgfusepath{clip}%
\pgfsetbuttcap%
\pgfsetroundjoin%
\definecolor{currentfill}{rgb}{0.975644,0.874038,0.797253}%
\pgfsetfillcolor{currentfill}%
\pgfsetlinewidth{0.311001pt}%
\definecolor{currentstroke}{rgb}{1.000000,1.000000,1.000000}%
\pgfsetstrokecolor{currentstroke}%
\pgfsetdash{}{0pt}%
\pgfpathmoveto{\pgfqpoint{5.434729in}{1.528006in}}%
\pgfpathcurveto{\pgfqpoint{5.441862in}{1.528006in}}{\pgfqpoint{5.448704in}{1.530840in}}{\pgfqpoint{5.453747in}{1.535884in}}%
\pgfpathcurveto{\pgfqpoint{5.458791in}{1.540927in}}{\pgfqpoint{5.461625in}{1.547769in}}{\pgfqpoint{5.461625in}{1.554902in}}%
\pgfpathcurveto{\pgfqpoint{5.461625in}{1.562035in}}{\pgfqpoint{5.458791in}{1.568876in}}{\pgfqpoint{5.453747in}{1.573920in}}%
\pgfpathcurveto{\pgfqpoint{5.448704in}{1.578964in}}{\pgfqpoint{5.441862in}{1.581797in}}{\pgfqpoint{5.434729in}{1.581797in}}%
\pgfpathcurveto{\pgfqpoint{5.427596in}{1.581797in}}{\pgfqpoint{5.420755in}{1.578964in}}{\pgfqpoint{5.415711in}{1.573920in}}%
\pgfpathcurveto{\pgfqpoint{5.410667in}{1.568876in}}{\pgfqpoint{5.407833in}{1.562035in}}{\pgfqpoint{5.407833in}{1.554902in}}%
\pgfpathcurveto{\pgfqpoint{5.407833in}{1.547769in}}{\pgfqpoint{5.410667in}{1.540927in}}{\pgfqpoint{5.415711in}{1.535884in}}%
\pgfpathcurveto{\pgfqpoint{5.420755in}{1.530840in}}{\pgfqpoint{5.427596in}{1.528006in}}{\pgfqpoint{5.434729in}{1.528006in}}%
\pgfpathclose%
\pgfusepath{stroke,fill}%
\end{pgfscope}%
\begin{pgfscope}%
\pgfpathrectangle{\pgfqpoint{4.985294in}{0.500000in}}{\pgfqpoint{1.764706in}{1.700000in}}%
\pgfusepath{clip}%
\pgfsetbuttcap%
\pgfsetroundjoin%
\definecolor{currentfill}{rgb}{0.965440,0.720101,0.576404}%
\pgfsetfillcolor{currentfill}%
\pgfsetlinewidth{0.311001pt}%
\definecolor{currentstroke}{rgb}{1.000000,1.000000,1.000000}%
\pgfsetstrokecolor{currentstroke}%
\pgfsetdash{}{0pt}%
\pgfpathmoveto{\pgfqpoint{5.580514in}{1.652994in}}%
\pgfpathcurveto{\pgfqpoint{5.587647in}{1.652994in}}{\pgfqpoint{5.594489in}{1.655828in}}{\pgfqpoint{5.599532in}{1.660872in}}%
\pgfpathcurveto{\pgfqpoint{5.604576in}{1.665916in}}{\pgfqpoint{5.607410in}{1.672757in}}{\pgfqpoint{5.607410in}{1.679890in}}%
\pgfpathcurveto{\pgfqpoint{5.607410in}{1.687023in}}{\pgfqpoint{5.604576in}{1.693865in}}{\pgfqpoint{5.599532in}{1.698908in}}%
\pgfpathcurveto{\pgfqpoint{5.594489in}{1.703952in}}{\pgfqpoint{5.587647in}{1.706786in}}{\pgfqpoint{5.580514in}{1.706786in}}%
\pgfpathcurveto{\pgfqpoint{5.573381in}{1.706786in}}{\pgfqpoint{5.566540in}{1.703952in}}{\pgfqpoint{5.561496in}{1.698908in}}%
\pgfpathcurveto{\pgfqpoint{5.556452in}{1.693865in}}{\pgfqpoint{5.553618in}{1.687023in}}{\pgfqpoint{5.553618in}{1.679890in}}%
\pgfpathcurveto{\pgfqpoint{5.553618in}{1.672757in}}{\pgfqpoint{5.556452in}{1.665916in}}{\pgfqpoint{5.561496in}{1.660872in}}%
\pgfpathcurveto{\pgfqpoint{5.566540in}{1.655828in}}{\pgfqpoint{5.573381in}{1.652994in}}{\pgfqpoint{5.580514in}{1.652994in}}%
\pgfpathclose%
\pgfusepath{stroke,fill}%
\end{pgfscope}%
\begin{pgfscope}%
\pgfpathrectangle{\pgfqpoint{4.985294in}{0.500000in}}{\pgfqpoint{1.764706in}{1.700000in}}%
\pgfusepath{clip}%
\pgfsetbuttcap%
\pgfsetroundjoin%
\definecolor{currentfill}{rgb}{0.973832,0.856556,0.771584}%
\pgfsetfillcolor{currentfill}%
\pgfsetlinewidth{0.311001pt}%
\definecolor{currentstroke}{rgb}{1.000000,1.000000,1.000000}%
\pgfsetstrokecolor{currentstroke}%
\pgfsetdash{}{0pt}%
\pgfpathmoveto{\pgfqpoint{5.485292in}{1.440630in}}%
\pgfpathcurveto{\pgfqpoint{5.492424in}{1.440630in}}{\pgfqpoint{5.499266in}{1.443463in}}{\pgfqpoint{5.504310in}{1.448507in}}%
\pgfpathcurveto{\pgfqpoint{5.509353in}{1.453551in}}{\pgfqpoint{5.512187in}{1.460392in}}{\pgfqpoint{5.512187in}{1.467525in}}%
\pgfpathcurveto{\pgfqpoint{5.512187in}{1.474658in}}{\pgfqpoint{5.509353in}{1.481500in}}{\pgfqpoint{5.504310in}{1.486543in}}%
\pgfpathcurveto{\pgfqpoint{5.499266in}{1.491587in}}{\pgfqpoint{5.492424in}{1.494421in}}{\pgfqpoint{5.485292in}{1.494421in}}%
\pgfpathcurveto{\pgfqpoint{5.478159in}{1.494421in}}{\pgfqpoint{5.471317in}{1.491587in}}{\pgfqpoint{5.466273in}{1.486543in}}%
\pgfpathcurveto{\pgfqpoint{5.461230in}{1.481500in}}{\pgfqpoint{5.458396in}{1.474658in}}{\pgfqpoint{5.458396in}{1.467525in}}%
\pgfpathcurveto{\pgfqpoint{5.458396in}{1.460392in}}{\pgfqpoint{5.461230in}{1.453551in}}{\pgfqpoint{5.466273in}{1.448507in}}%
\pgfpathcurveto{\pgfqpoint{5.471317in}{1.443463in}}{\pgfqpoint{5.478159in}{1.440630in}}{\pgfqpoint{5.485292in}{1.440630in}}%
\pgfpathclose%
\pgfusepath{stroke,fill}%
\end{pgfscope}%
\begin{pgfscope}%
\pgfpathrectangle{\pgfqpoint{4.985294in}{0.500000in}}{\pgfqpoint{1.764706in}{1.700000in}}%
\pgfusepath{clip}%
\pgfsetbuttcap%
\pgfsetroundjoin%
\definecolor{currentfill}{rgb}{0.975644,0.874038,0.797253}%
\pgfsetfillcolor{currentfill}%
\pgfsetlinewidth{0.311001pt}%
\definecolor{currentstroke}{rgb}{1.000000,1.000000,1.000000}%
\pgfsetstrokecolor{currentstroke}%
\pgfsetdash{}{0pt}%
\pgfpathmoveto{\pgfqpoint{5.443568in}{1.090586in}}%
\pgfpathcurveto{\pgfqpoint{5.450700in}{1.090586in}}{\pgfqpoint{5.457542in}{1.093420in}}{\pgfqpoint{5.462586in}{1.098463in}}%
\pgfpathcurveto{\pgfqpoint{5.467629in}{1.103507in}}{\pgfqpoint{5.470463in}{1.110349in}}{\pgfqpoint{5.470463in}{1.117481in}}%
\pgfpathcurveto{\pgfqpoint{5.470463in}{1.124614in}}{\pgfqpoint{5.467629in}{1.131456in}}{\pgfqpoint{5.462586in}{1.136500in}}%
\pgfpathcurveto{\pgfqpoint{5.457542in}{1.141543in}}{\pgfqpoint{5.450700in}{1.144377in}}{\pgfqpoint{5.443568in}{1.144377in}}%
\pgfpathcurveto{\pgfqpoint{5.436435in}{1.144377in}}{\pgfqpoint{5.429593in}{1.141543in}}{\pgfqpoint{5.424549in}{1.136500in}}%
\pgfpathcurveto{\pgfqpoint{5.419506in}{1.131456in}}{\pgfqpoint{5.416672in}{1.124614in}}{\pgfqpoint{5.416672in}{1.117481in}}%
\pgfpathcurveto{\pgfqpoint{5.416672in}{1.110349in}}{\pgfqpoint{5.419506in}{1.103507in}}{\pgfqpoint{5.424549in}{1.098463in}}%
\pgfpathcurveto{\pgfqpoint{5.429593in}{1.093420in}}{\pgfqpoint{5.436435in}{1.090586in}}{\pgfqpoint{5.443568in}{1.090586in}}%
\pgfpathclose%
\pgfusepath{stroke,fill}%
\end{pgfscope}%
\begin{pgfscope}%
\pgfpathrectangle{\pgfqpoint{4.985294in}{0.500000in}}{\pgfqpoint{1.764706in}{1.700000in}}%
\pgfusepath{clip}%
\pgfsetbuttcap%
\pgfsetroundjoin%
\definecolor{currentfill}{rgb}{0.962283,0.593046,0.431453}%
\pgfsetfillcolor{currentfill}%
\pgfsetlinewidth{0.311001pt}%
\definecolor{currentstroke}{rgb}{1.000000,1.000000,1.000000}%
\pgfsetstrokecolor{currentstroke}%
\pgfsetdash{}{0pt}%
\pgfpathmoveto{\pgfqpoint{6.084836in}{1.698191in}}%
\pgfpathcurveto{\pgfqpoint{6.091969in}{1.698191in}}{\pgfqpoint{6.098811in}{1.701024in}}{\pgfqpoint{6.103854in}{1.706068in}}%
\pgfpathcurveto{\pgfqpoint{6.108898in}{1.711112in}}{\pgfqpoint{6.111732in}{1.717953in}}{\pgfqpoint{6.111732in}{1.725086in}}%
\pgfpathcurveto{\pgfqpoint{6.111732in}{1.732219in}}{\pgfqpoint{6.108898in}{1.739061in}}{\pgfqpoint{6.103854in}{1.744104in}}%
\pgfpathcurveto{\pgfqpoint{6.098811in}{1.749148in}}{\pgfqpoint{6.091969in}{1.751982in}}{\pgfqpoint{6.084836in}{1.751982in}}%
\pgfpathcurveto{\pgfqpoint{6.077704in}{1.751982in}}{\pgfqpoint{6.070862in}{1.749148in}}{\pgfqpoint{6.065818in}{1.744104in}}%
\pgfpathcurveto{\pgfqpoint{6.060775in}{1.739061in}}{\pgfqpoint{6.057941in}{1.732219in}}{\pgfqpoint{6.057941in}{1.725086in}}%
\pgfpathcurveto{\pgfqpoint{6.057941in}{1.717953in}}{\pgfqpoint{6.060775in}{1.711112in}}{\pgfqpoint{6.065818in}{1.706068in}}%
\pgfpathcurveto{\pgfqpoint{6.070862in}{1.701024in}}{\pgfqpoint{6.077704in}{1.698191in}}{\pgfqpoint{6.084836in}{1.698191in}}%
\pgfpathclose%
\pgfusepath{stroke,fill}%
\end{pgfscope}%
\begin{pgfscope}%
\pgfpathrectangle{\pgfqpoint{4.985294in}{0.500000in}}{\pgfqpoint{1.764706in}{1.700000in}}%
\pgfusepath{clip}%
\pgfsetbuttcap%
\pgfsetroundjoin%
\definecolor{currentfill}{rgb}{0.934351,0.329284,0.247753}%
\pgfsetfillcolor{currentfill}%
\pgfsetlinewidth{0.311001pt}%
\definecolor{currentstroke}{rgb}{1.000000,1.000000,1.000000}%
\pgfsetstrokecolor{currentstroke}%
\pgfsetdash{}{0pt}%
\pgfpathmoveto{\pgfqpoint{5.473922in}{1.794706in}}%
\pgfpathcurveto{\pgfqpoint{5.481055in}{1.794706in}}{\pgfqpoint{5.487897in}{1.797540in}}{\pgfqpoint{5.492940in}{1.802584in}}%
\pgfpathcurveto{\pgfqpoint{5.497984in}{1.807628in}}{\pgfqpoint{5.500818in}{1.814469in}}{\pgfqpoint{5.500818in}{1.821602in}}%
\pgfpathcurveto{\pgfqpoint{5.500818in}{1.828735in}}{\pgfqpoint{5.497984in}{1.835577in}}{\pgfqpoint{5.492940in}{1.840620in}}%
\pgfpathcurveto{\pgfqpoint{5.487897in}{1.845664in}}{\pgfqpoint{5.481055in}{1.848498in}}{\pgfqpoint{5.473922in}{1.848498in}}%
\pgfpathcurveto{\pgfqpoint{5.466789in}{1.848498in}}{\pgfqpoint{5.459948in}{1.845664in}}{\pgfqpoint{5.454904in}{1.840620in}}%
\pgfpathcurveto{\pgfqpoint{5.449860in}{1.835577in}}{\pgfqpoint{5.447026in}{1.828735in}}{\pgfqpoint{5.447026in}{1.821602in}}%
\pgfpathcurveto{\pgfqpoint{5.447026in}{1.814469in}}{\pgfqpoint{5.449860in}{1.807628in}}{\pgfqpoint{5.454904in}{1.802584in}}%
\pgfpathcurveto{\pgfqpoint{5.459948in}{1.797540in}}{\pgfqpoint{5.466789in}{1.794706in}}{\pgfqpoint{5.473922in}{1.794706in}}%
\pgfpathclose%
\pgfusepath{stroke,fill}%
\end{pgfscope}%
\begin{pgfscope}%
\pgfpathrectangle{\pgfqpoint{4.985294in}{0.500000in}}{\pgfqpoint{1.764706in}{1.700000in}}%
\pgfusepath{clip}%
\pgfsetbuttcap%
\pgfsetroundjoin%
\definecolor{currentfill}{rgb}{0.977657,0.891500,0.822809}%
\pgfsetfillcolor{currentfill}%
\pgfsetlinewidth{0.311001pt}%
\definecolor{currentstroke}{rgb}{1.000000,1.000000,1.000000}%
\pgfsetstrokecolor{currentstroke}%
\pgfsetdash{}{0pt}%
\pgfpathmoveto{\pgfqpoint{6.310450in}{1.529328in}}%
\pgfpathcurveto{\pgfqpoint{6.317583in}{1.529328in}}{\pgfqpoint{6.324424in}{1.532162in}}{\pgfqpoint{6.329468in}{1.537206in}}%
\pgfpathcurveto{\pgfqpoint{6.334512in}{1.542249in}}{\pgfqpoint{6.337346in}{1.549091in}}{\pgfqpoint{6.337346in}{1.556224in}}%
\pgfpathcurveto{\pgfqpoint{6.337346in}{1.563357in}}{\pgfqpoint{6.334512in}{1.570198in}}{\pgfqpoint{6.329468in}{1.575242in}}%
\pgfpathcurveto{\pgfqpoint{6.324424in}{1.580286in}}{\pgfqpoint{6.317583in}{1.583120in}}{\pgfqpoint{6.310450in}{1.583120in}}%
\pgfpathcurveto{\pgfqpoint{6.303317in}{1.583120in}}{\pgfqpoint{6.296475in}{1.580286in}}{\pgfqpoint{6.291432in}{1.575242in}}%
\pgfpathcurveto{\pgfqpoint{6.286388in}{1.570198in}}{\pgfqpoint{6.283554in}{1.563357in}}{\pgfqpoint{6.283554in}{1.556224in}}%
\pgfpathcurveto{\pgfqpoint{6.283554in}{1.549091in}}{\pgfqpoint{6.286388in}{1.542249in}}{\pgfqpoint{6.291432in}{1.537206in}}%
\pgfpathcurveto{\pgfqpoint{6.296475in}{1.532162in}}{\pgfqpoint{6.303317in}{1.529328in}}{\pgfqpoint{6.310450in}{1.529328in}}%
\pgfpathclose%
\pgfusepath{stroke,fill}%
\end{pgfscope}%
\begin{pgfscope}%
\pgfpathrectangle{\pgfqpoint{4.985294in}{0.500000in}}{\pgfqpoint{1.764706in}{1.700000in}}%
\pgfusepath{clip}%
\pgfsetbuttcap%
\pgfsetroundjoin%
\definecolor{currentfill}{rgb}{0.978376,0.897317,0.831308}%
\pgfsetfillcolor{currentfill}%
\pgfsetlinewidth{0.311001pt}%
\definecolor{currentstroke}{rgb}{1.000000,1.000000,1.000000}%
\pgfsetstrokecolor{currentstroke}%
\pgfsetdash{}{0pt}%
\pgfpathmoveto{\pgfqpoint{6.272053in}{1.571086in}}%
\pgfpathcurveto{\pgfqpoint{6.279186in}{1.571086in}}{\pgfqpoint{6.286027in}{1.573919in}}{\pgfqpoint{6.291071in}{1.578963in}}%
\pgfpathcurveto{\pgfqpoint{6.296115in}{1.584007in}}{\pgfqpoint{6.298949in}{1.590848in}}{\pgfqpoint{6.298949in}{1.597981in}}%
\pgfpathcurveto{\pgfqpoint{6.298949in}{1.605114in}}{\pgfqpoint{6.296115in}{1.611956in}}{\pgfqpoint{6.291071in}{1.616999in}}%
\pgfpathcurveto{\pgfqpoint{6.286027in}{1.622043in}}{\pgfqpoint{6.279186in}{1.624877in}}{\pgfqpoint{6.272053in}{1.624877in}}%
\pgfpathcurveto{\pgfqpoint{6.264920in}{1.624877in}}{\pgfqpoint{6.258078in}{1.622043in}}{\pgfqpoint{6.253035in}{1.616999in}}%
\pgfpathcurveto{\pgfqpoint{6.247991in}{1.611956in}}{\pgfqpoint{6.245157in}{1.605114in}}{\pgfqpoint{6.245157in}{1.597981in}}%
\pgfpathcurveto{\pgfqpoint{6.245157in}{1.590848in}}{\pgfqpoint{6.247991in}{1.584007in}}{\pgfqpoint{6.253035in}{1.578963in}}%
\pgfpathcurveto{\pgfqpoint{6.258078in}{1.573919in}}{\pgfqpoint{6.264920in}{1.571086in}}{\pgfqpoint{6.272053in}{1.571086in}}%
\pgfpathclose%
\pgfusepath{stroke,fill}%
\end{pgfscope}%
\begin{pgfscope}%
\pgfpathrectangle{\pgfqpoint{4.985294in}{0.500000in}}{\pgfqpoint{1.764706in}{1.700000in}}%
\pgfusepath{clip}%
\pgfsetbuttcap%
\pgfsetroundjoin%
\definecolor{currentfill}{rgb}{0.959645,0.539840,0.380928}%
\pgfsetfillcolor{currentfill}%
\pgfsetlinewidth{0.311001pt}%
\definecolor{currentstroke}{rgb}{1.000000,1.000000,1.000000}%
\pgfsetstrokecolor{currentstroke}%
\pgfsetdash{}{0pt}%
\pgfpathmoveto{\pgfqpoint{5.331358in}{1.078871in}}%
\pgfpathcurveto{\pgfqpoint{5.338491in}{1.078871in}}{\pgfqpoint{5.345332in}{1.081705in}}{\pgfqpoint{5.350376in}{1.086748in}}%
\pgfpathcurveto{\pgfqpoint{5.355420in}{1.091792in}}{\pgfqpoint{5.358254in}{1.098634in}}{\pgfqpoint{5.358254in}{1.105766in}}%
\pgfpathcurveto{\pgfqpoint{5.358254in}{1.112899in}}{\pgfqpoint{5.355420in}{1.119741in}}{\pgfqpoint{5.350376in}{1.124785in}}%
\pgfpathcurveto{\pgfqpoint{5.345332in}{1.129828in}}{\pgfqpoint{5.338491in}{1.132662in}}{\pgfqpoint{5.331358in}{1.132662in}}%
\pgfpathcurveto{\pgfqpoint{5.324225in}{1.132662in}}{\pgfqpoint{5.317383in}{1.129828in}}{\pgfqpoint{5.312340in}{1.124785in}}%
\pgfpathcurveto{\pgfqpoint{5.307296in}{1.119741in}}{\pgfqpoint{5.304462in}{1.112899in}}{\pgfqpoint{5.304462in}{1.105766in}}%
\pgfpathcurveto{\pgfqpoint{5.304462in}{1.098634in}}{\pgfqpoint{5.307296in}{1.091792in}}{\pgfqpoint{5.312340in}{1.086748in}}%
\pgfpathcurveto{\pgfqpoint{5.317383in}{1.081705in}}{\pgfqpoint{5.324225in}{1.078871in}}{\pgfqpoint{5.331358in}{1.078871in}}%
\pgfpathclose%
\pgfusepath{stroke,fill}%
\end{pgfscope}%
\begin{pgfscope}%
\pgfpathrectangle{\pgfqpoint{4.985294in}{0.500000in}}{\pgfqpoint{1.764706in}{1.700000in}}%
\pgfusepath{clip}%
\pgfsetbuttcap%
\pgfsetroundjoin%
\definecolor{currentfill}{rgb}{0.979891,0.908948,0.848279}%
\pgfsetfillcolor{currentfill}%
\pgfsetlinewidth{0.311001pt}%
\definecolor{currentstroke}{rgb}{1.000000,1.000000,1.000000}%
\pgfsetstrokecolor{currentstroke}%
\pgfsetdash{}{0pt}%
\pgfpathmoveto{\pgfqpoint{5.405495in}{1.393142in}}%
\pgfpathcurveto{\pgfqpoint{5.412628in}{1.393142in}}{\pgfqpoint{5.419470in}{1.395976in}}{\pgfqpoint{5.424513in}{1.401020in}}%
\pgfpathcurveto{\pgfqpoint{5.429557in}{1.406063in}}{\pgfqpoint{5.432391in}{1.412905in}}{\pgfqpoint{5.432391in}{1.420038in}}%
\pgfpathcurveto{\pgfqpoint{5.432391in}{1.427171in}}{\pgfqpoint{5.429557in}{1.434012in}}{\pgfqpoint{5.424513in}{1.439056in}}%
\pgfpathcurveto{\pgfqpoint{5.419470in}{1.444100in}}{\pgfqpoint{5.412628in}{1.446933in}}{\pgfqpoint{5.405495in}{1.446933in}}%
\pgfpathcurveto{\pgfqpoint{5.398362in}{1.446933in}}{\pgfqpoint{5.391521in}{1.444100in}}{\pgfqpoint{5.386477in}{1.439056in}}%
\pgfpathcurveto{\pgfqpoint{5.381433in}{1.434012in}}{\pgfqpoint{5.378599in}{1.427171in}}{\pgfqpoint{5.378599in}{1.420038in}}%
\pgfpathcurveto{\pgfqpoint{5.378599in}{1.412905in}}{\pgfqpoint{5.381433in}{1.406063in}}{\pgfqpoint{5.386477in}{1.401020in}}%
\pgfpathcurveto{\pgfqpoint{5.391521in}{1.395976in}}{\pgfqpoint{5.398362in}{1.393142in}}{\pgfqpoint{5.405495in}{1.393142in}}%
\pgfpathclose%
\pgfusepath{stroke,fill}%
\end{pgfscope}%
\begin{pgfscope}%
\pgfpathrectangle{\pgfqpoint{4.985294in}{0.500000in}}{\pgfqpoint{1.764706in}{1.700000in}}%
\pgfusepath{clip}%
\pgfsetbuttcap%
\pgfsetroundjoin%
\definecolor{currentfill}{rgb}{0.937528,0.344792,0.251999}%
\pgfsetfillcolor{currentfill}%
\pgfsetlinewidth{0.311001pt}%
\definecolor{currentstroke}{rgb}{1.000000,1.000000,1.000000}%
\pgfsetstrokecolor{currentstroke}%
\pgfsetdash{}{0pt}%
\pgfpathmoveto{\pgfqpoint{5.588822in}{1.243008in}}%
\pgfpathcurveto{\pgfqpoint{5.595955in}{1.243008in}}{\pgfqpoint{5.602796in}{1.245842in}}{\pgfqpoint{5.607840in}{1.250886in}}%
\pgfpathcurveto{\pgfqpoint{5.612884in}{1.255930in}}{\pgfqpoint{5.615718in}{1.262771in}}{\pgfqpoint{5.615718in}{1.269904in}}%
\pgfpathcurveto{\pgfqpoint{5.615718in}{1.277037in}}{\pgfqpoint{5.612884in}{1.283879in}}{\pgfqpoint{5.607840in}{1.288922in}}%
\pgfpathcurveto{\pgfqpoint{5.602796in}{1.293966in}}{\pgfqpoint{5.595955in}{1.296800in}}{\pgfqpoint{5.588822in}{1.296800in}}%
\pgfpathcurveto{\pgfqpoint{5.581689in}{1.296800in}}{\pgfqpoint{5.574847in}{1.293966in}}{\pgfqpoint{5.569804in}{1.288922in}}%
\pgfpathcurveto{\pgfqpoint{5.564760in}{1.283879in}}{\pgfqpoint{5.561926in}{1.277037in}}{\pgfqpoint{5.561926in}{1.269904in}}%
\pgfpathcurveto{\pgfqpoint{5.561926in}{1.262771in}}{\pgfqpoint{5.564760in}{1.255930in}}{\pgfqpoint{5.569804in}{1.250886in}}%
\pgfpathcurveto{\pgfqpoint{5.574847in}{1.245842in}}{\pgfqpoint{5.581689in}{1.243008in}}{\pgfqpoint{5.588822in}{1.243008in}}%
\pgfpathclose%
\pgfusepath{stroke,fill}%
\end{pgfscope}%
\begin{pgfscope}%
\pgfpathrectangle{\pgfqpoint{4.985294in}{0.500000in}}{\pgfqpoint{1.764706in}{1.700000in}}%
\pgfusepath{clip}%
\pgfsetbuttcap%
\pgfsetroundjoin%
\definecolor{currentfill}{rgb}{0.976961,0.885681,0.814303}%
\pgfsetfillcolor{currentfill}%
\pgfsetlinewidth{0.311001pt}%
\definecolor{currentstroke}{rgb}{1.000000,1.000000,1.000000}%
\pgfsetstrokecolor{currentstroke}%
\pgfsetdash{}{0pt}%
\pgfpathmoveto{\pgfqpoint{5.449972in}{1.510542in}}%
\pgfpathcurveto{\pgfqpoint{5.457105in}{1.510542in}}{\pgfqpoint{5.463947in}{1.513376in}}{\pgfqpoint{5.468991in}{1.518420in}}%
\pgfpathcurveto{\pgfqpoint{5.474034in}{1.523463in}}{\pgfqpoint{5.476868in}{1.530305in}}{\pgfqpoint{5.476868in}{1.537438in}}%
\pgfpathcurveto{\pgfqpoint{5.476868in}{1.544571in}}{\pgfqpoint{5.474034in}{1.551412in}}{\pgfqpoint{5.468991in}{1.556456in}}%
\pgfpathcurveto{\pgfqpoint{5.463947in}{1.561500in}}{\pgfqpoint{5.457105in}{1.564334in}}{\pgfqpoint{5.449972in}{1.564334in}}%
\pgfpathcurveto{\pgfqpoint{5.442840in}{1.564334in}}{\pgfqpoint{5.435998in}{1.561500in}}{\pgfqpoint{5.430954in}{1.556456in}}%
\pgfpathcurveto{\pgfqpoint{5.425911in}{1.551412in}}{\pgfqpoint{5.423077in}{1.544571in}}{\pgfqpoint{5.423077in}{1.537438in}}%
\pgfpathcurveto{\pgfqpoint{5.423077in}{1.530305in}}{\pgfqpoint{5.425911in}{1.523463in}}{\pgfqpoint{5.430954in}{1.518420in}}%
\pgfpathcurveto{\pgfqpoint{5.435998in}{1.513376in}}{\pgfqpoint{5.442840in}{1.510542in}}{\pgfqpoint{5.449972in}{1.510542in}}%
\pgfpathclose%
\pgfusepath{stroke,fill}%
\end{pgfscope}%
\begin{pgfscope}%
\pgfpathrectangle{\pgfqpoint{4.985294in}{0.500000in}}{\pgfqpoint{1.764706in}{1.700000in}}%
\pgfusepath{clip}%
\pgfsetbuttcap%
\pgfsetroundjoin%
\definecolor{currentfill}{rgb}{0.908486,0.245685,0.245983}%
\pgfsetfillcolor{currentfill}%
\pgfsetlinewidth{0.311001pt}%
\definecolor{currentstroke}{rgb}{1.000000,1.000000,1.000000}%
\pgfsetstrokecolor{currentstroke}%
\pgfsetdash{}{0pt}%
\pgfpathmoveto{\pgfqpoint{6.400637in}{1.684319in}}%
\pgfpathcurveto{\pgfqpoint{6.407770in}{1.684319in}}{\pgfqpoint{6.414611in}{1.687152in}}{\pgfqpoint{6.419655in}{1.692196in}}%
\pgfpathcurveto{\pgfqpoint{6.424699in}{1.697240in}}{\pgfqpoint{6.427532in}{1.704081in}}{\pgfqpoint{6.427532in}{1.711214in}}%
\pgfpathcurveto{\pgfqpoint{6.427532in}{1.718347in}}{\pgfqpoint{6.424699in}{1.725189in}}{\pgfqpoint{6.419655in}{1.730232in}}%
\pgfpathcurveto{\pgfqpoint{6.414611in}{1.735276in}}{\pgfqpoint{6.407770in}{1.738110in}}{\pgfqpoint{6.400637in}{1.738110in}}%
\pgfpathcurveto{\pgfqpoint{6.393504in}{1.738110in}}{\pgfqpoint{6.386662in}{1.735276in}}{\pgfqpoint{6.381619in}{1.730232in}}%
\pgfpathcurveto{\pgfqpoint{6.376575in}{1.725189in}}{\pgfqpoint{6.373741in}{1.718347in}}{\pgfqpoint{6.373741in}{1.711214in}}%
\pgfpathcurveto{\pgfqpoint{6.373741in}{1.704081in}}{\pgfqpoint{6.376575in}{1.697240in}}{\pgfqpoint{6.381619in}{1.692196in}}%
\pgfpathcurveto{\pgfqpoint{6.386662in}{1.687152in}}{\pgfqpoint{6.393504in}{1.684319in}}{\pgfqpoint{6.400637in}{1.684319in}}%
\pgfpathclose%
\pgfusepath{stroke,fill}%
\end{pgfscope}%
\begin{pgfscope}%
\pgfpathrectangle{\pgfqpoint{4.985294in}{0.500000in}}{\pgfqpoint{1.764706in}{1.700000in}}%
\pgfusepath{clip}%
\pgfsetbuttcap%
\pgfsetroundjoin%
\definecolor{currentfill}{rgb}{0.973271,0.850724,0.762998}%
\pgfsetfillcolor{currentfill}%
\pgfsetlinewidth{0.311001pt}%
\definecolor{currentstroke}{rgb}{1.000000,1.000000,1.000000}%
\pgfsetstrokecolor{currentstroke}%
\pgfsetdash{}{0pt}%
\pgfpathmoveto{\pgfqpoint{6.305512in}{1.582406in}}%
\pgfpathcurveto{\pgfqpoint{6.312644in}{1.582406in}}{\pgfqpoint{6.319486in}{1.585240in}}{\pgfqpoint{6.324530in}{1.590284in}}%
\pgfpathcurveto{\pgfqpoint{6.329573in}{1.595327in}}{\pgfqpoint{6.332407in}{1.602169in}}{\pgfqpoint{6.332407in}{1.609302in}}%
\pgfpathcurveto{\pgfqpoint{6.332407in}{1.616435in}}{\pgfqpoint{6.329573in}{1.623276in}}{\pgfqpoint{6.324530in}{1.628320in}}%
\pgfpathcurveto{\pgfqpoint{6.319486in}{1.633364in}}{\pgfqpoint{6.312644in}{1.636198in}}{\pgfqpoint{6.305512in}{1.636198in}}%
\pgfpathcurveto{\pgfqpoint{6.298379in}{1.636198in}}{\pgfqpoint{6.291537in}{1.633364in}}{\pgfqpoint{6.286493in}{1.628320in}}%
\pgfpathcurveto{\pgfqpoint{6.281450in}{1.623276in}}{\pgfqpoint{6.278616in}{1.616435in}}{\pgfqpoint{6.278616in}{1.609302in}}%
\pgfpathcurveto{\pgfqpoint{6.278616in}{1.602169in}}{\pgfqpoint{6.281450in}{1.595327in}}{\pgfqpoint{6.286493in}{1.590284in}}%
\pgfpathcurveto{\pgfqpoint{6.291537in}{1.585240in}}{\pgfqpoint{6.298379in}{1.582406in}}{\pgfqpoint{6.305512in}{1.582406in}}%
\pgfpathclose%
\pgfusepath{stroke,fill}%
\end{pgfscope}%
\begin{pgfscope}%
\pgfpathrectangle{\pgfqpoint{4.985294in}{0.500000in}}{\pgfqpoint{1.764706in}{1.700000in}}%
\pgfusepath{clip}%
\pgfsetbuttcap%
\pgfsetroundjoin%
\definecolor{currentfill}{rgb}{0.972201,0.839051,0.745789}%
\pgfsetfillcolor{currentfill}%
\pgfsetlinewidth{0.311001pt}%
\definecolor{currentstroke}{rgb}{1.000000,1.000000,1.000000}%
\pgfsetstrokecolor{currentstroke}%
\pgfsetdash{}{0pt}%
\pgfpathmoveto{\pgfqpoint{5.359223in}{1.300707in}}%
\pgfpathcurveto{\pgfqpoint{5.366356in}{1.300707in}}{\pgfqpoint{5.373198in}{1.303541in}}{\pgfqpoint{5.378241in}{1.308585in}}%
\pgfpathcurveto{\pgfqpoint{5.383285in}{1.313628in}}{\pgfqpoint{5.386119in}{1.320470in}}{\pgfqpoint{5.386119in}{1.327603in}}%
\pgfpathcurveto{\pgfqpoint{5.386119in}{1.334736in}}{\pgfqpoint{5.383285in}{1.341577in}}{\pgfqpoint{5.378241in}{1.346621in}}%
\pgfpathcurveto{\pgfqpoint{5.373198in}{1.351665in}}{\pgfqpoint{5.366356in}{1.354498in}}{\pgfqpoint{5.359223in}{1.354498in}}%
\pgfpathcurveto{\pgfqpoint{5.352090in}{1.354498in}}{\pgfqpoint{5.345249in}{1.351665in}}{\pgfqpoint{5.340205in}{1.346621in}}%
\pgfpathcurveto{\pgfqpoint{5.335161in}{1.341577in}}{\pgfqpoint{5.332327in}{1.334736in}}{\pgfqpoint{5.332327in}{1.327603in}}%
\pgfpathcurveto{\pgfqpoint{5.332327in}{1.320470in}}{\pgfqpoint{5.335161in}{1.313628in}}{\pgfqpoint{5.340205in}{1.308585in}}%
\pgfpathcurveto{\pgfqpoint{5.345249in}{1.303541in}}{\pgfqpoint{5.352090in}{1.300707in}}{\pgfqpoint{5.359223in}{1.300707in}}%
\pgfpathclose%
\pgfusepath{stroke,fill}%
\end{pgfscope}%
\begin{pgfscope}%
\pgfpathrectangle{\pgfqpoint{4.985294in}{0.500000in}}{\pgfqpoint{1.764706in}{1.700000in}}%
\pgfusepath{clip}%
\pgfsetbuttcap%
\pgfsetroundjoin%
\definecolor{currentfill}{rgb}{0.963884,0.644842,0.486120}%
\pgfsetfillcolor{currentfill}%
\pgfsetlinewidth{0.311001pt}%
\definecolor{currentstroke}{rgb}{1.000000,1.000000,1.000000}%
\pgfsetstrokecolor{currentstroke}%
\pgfsetdash{}{0pt}%
\pgfpathmoveto{\pgfqpoint{5.326666in}{1.173292in}}%
\pgfpathcurveto{\pgfqpoint{5.333799in}{1.173292in}}{\pgfqpoint{5.340641in}{1.176126in}}{\pgfqpoint{5.345685in}{1.181170in}}%
\pgfpathcurveto{\pgfqpoint{5.350728in}{1.186213in}}{\pgfqpoint{5.353562in}{1.193055in}}{\pgfqpoint{5.353562in}{1.200188in}}%
\pgfpathcurveto{\pgfqpoint{5.353562in}{1.207321in}}{\pgfqpoint{5.350728in}{1.214162in}}{\pgfqpoint{5.345685in}{1.219206in}}%
\pgfpathcurveto{\pgfqpoint{5.340641in}{1.224250in}}{\pgfqpoint{5.333799in}{1.227084in}}{\pgfqpoint{5.326666in}{1.227084in}}%
\pgfpathcurveto{\pgfqpoint{5.319534in}{1.227084in}}{\pgfqpoint{5.312692in}{1.224250in}}{\pgfqpoint{5.307648in}{1.219206in}}%
\pgfpathcurveto{\pgfqpoint{5.302605in}{1.214162in}}{\pgfqpoint{5.299771in}{1.207321in}}{\pgfqpoint{5.299771in}{1.200188in}}%
\pgfpathcurveto{\pgfqpoint{5.299771in}{1.193055in}}{\pgfqpoint{5.302605in}{1.186213in}}{\pgfqpoint{5.307648in}{1.181170in}}%
\pgfpathcurveto{\pgfqpoint{5.312692in}{1.176126in}}{\pgfqpoint{5.319534in}{1.173292in}}{\pgfqpoint{5.326666in}{1.173292in}}%
\pgfpathclose%
\pgfusepath{stroke,fill}%
\end{pgfscope}%
\begin{pgfscope}%
\pgfpathrectangle{\pgfqpoint{4.985294in}{0.500000in}}{\pgfqpoint{1.764706in}{1.700000in}}%
\pgfusepath{clip}%
\pgfsetbuttcap%
\pgfsetroundjoin%
\definecolor{currentfill}{rgb}{0.965753,0.732351,0.592427}%
\pgfsetfillcolor{currentfill}%
\pgfsetlinewidth{0.311001pt}%
\definecolor{currentstroke}{rgb}{1.000000,1.000000,1.000000}%
\pgfsetstrokecolor{currentstroke}%
\pgfsetdash{}{0pt}%
\pgfpathmoveto{\pgfqpoint{6.202406in}{1.414745in}}%
\pgfpathcurveto{\pgfqpoint{6.209539in}{1.414745in}}{\pgfqpoint{6.216380in}{1.417579in}}{\pgfqpoint{6.221424in}{1.422622in}}%
\pgfpathcurveto{\pgfqpoint{6.226468in}{1.427666in}}{\pgfqpoint{6.229302in}{1.434508in}}{\pgfqpoint{6.229302in}{1.441640in}}%
\pgfpathcurveto{\pgfqpoint{6.229302in}{1.448773in}}{\pgfqpoint{6.226468in}{1.455615in}}{\pgfqpoint{6.221424in}{1.460658in}}%
\pgfpathcurveto{\pgfqpoint{6.216380in}{1.465702in}}{\pgfqpoint{6.209539in}{1.468536in}}{\pgfqpoint{6.202406in}{1.468536in}}%
\pgfpathcurveto{\pgfqpoint{6.195273in}{1.468536in}}{\pgfqpoint{6.188432in}{1.465702in}}{\pgfqpoint{6.183388in}{1.460658in}}%
\pgfpathcurveto{\pgfqpoint{6.178344in}{1.455615in}}{\pgfqpoint{6.175510in}{1.448773in}}{\pgfqpoint{6.175510in}{1.441640in}}%
\pgfpathcurveto{\pgfqpoint{6.175510in}{1.434508in}}{\pgfqpoint{6.178344in}{1.427666in}}{\pgfqpoint{6.183388in}{1.422622in}}%
\pgfpathcurveto{\pgfqpoint{6.188432in}{1.417579in}}{\pgfqpoint{6.195273in}{1.414745in}}{\pgfqpoint{6.202406in}{1.414745in}}%
\pgfpathclose%
\pgfusepath{stroke,fill}%
\end{pgfscope}%
\begin{pgfscope}%
\pgfpathrectangle{\pgfqpoint{4.985294in}{0.500000in}}{\pgfqpoint{1.764706in}{1.700000in}}%
\pgfusepath{clip}%
\pgfsetbuttcap%
\pgfsetroundjoin%
\definecolor{currentfill}{rgb}{0.978376,0.897317,0.831308}%
\pgfsetfillcolor{currentfill}%
\pgfsetlinewidth{0.311001pt}%
\definecolor{currentstroke}{rgb}{1.000000,1.000000,1.000000}%
\pgfsetstrokecolor{currentstroke}%
\pgfsetdash{}{0pt}%
\pgfpathmoveto{\pgfqpoint{5.421189in}{1.185948in}}%
\pgfpathcurveto{\pgfqpoint{5.428322in}{1.185948in}}{\pgfqpoint{5.435164in}{1.188782in}}{\pgfqpoint{5.440207in}{1.193826in}}%
\pgfpathcurveto{\pgfqpoint{5.445251in}{1.198869in}}{\pgfqpoint{5.448085in}{1.205711in}}{\pgfqpoint{5.448085in}{1.212844in}}%
\pgfpathcurveto{\pgfqpoint{5.448085in}{1.219977in}}{\pgfqpoint{5.445251in}{1.226818in}}{\pgfqpoint{5.440207in}{1.231862in}}%
\pgfpathcurveto{\pgfqpoint{5.435164in}{1.236906in}}{\pgfqpoint{5.428322in}{1.239739in}}{\pgfqpoint{5.421189in}{1.239739in}}%
\pgfpathcurveto{\pgfqpoint{5.414056in}{1.239739in}}{\pgfqpoint{5.407215in}{1.236906in}}{\pgfqpoint{5.402171in}{1.231862in}}%
\pgfpathcurveto{\pgfqpoint{5.397128in}{1.226818in}}{\pgfqpoint{5.394294in}{1.219977in}}{\pgfqpoint{5.394294in}{1.212844in}}%
\pgfpathcurveto{\pgfqpoint{5.394294in}{1.205711in}}{\pgfqpoint{5.397128in}{1.198869in}}{\pgfqpoint{5.402171in}{1.193826in}}%
\pgfpathcurveto{\pgfqpoint{5.407215in}{1.188782in}}{\pgfqpoint{5.414056in}{1.185948in}}{\pgfqpoint{5.421189in}{1.185948in}}%
\pgfpathclose%
\pgfusepath{stroke,fill}%
\end{pgfscope}%
\begin{pgfscope}%
\pgfpathrectangle{\pgfqpoint{4.985294in}{0.500000in}}{\pgfqpoint{1.764706in}{1.700000in}}%
\pgfusepath{clip}%
\pgfsetbuttcap%
\pgfsetroundjoin%
\definecolor{currentfill}{rgb}{0.971202,0.827364,0.728520}%
\pgfsetfillcolor{currentfill}%
\pgfsetlinewidth{0.311001pt}%
\definecolor{currentstroke}{rgb}{1.000000,1.000000,1.000000}%
\pgfsetstrokecolor{currentstroke}%
\pgfsetdash{}{0pt}%
\pgfpathmoveto{\pgfqpoint{6.203443in}{1.552930in}}%
\pgfpathcurveto{\pgfqpoint{6.210575in}{1.552930in}}{\pgfqpoint{6.217417in}{1.555764in}}{\pgfqpoint{6.222461in}{1.560808in}}%
\pgfpathcurveto{\pgfqpoint{6.227504in}{1.565852in}}{\pgfqpoint{6.230338in}{1.572693in}}{\pgfqpoint{6.230338in}{1.579826in}}%
\pgfpathcurveto{\pgfqpoint{6.230338in}{1.586959in}}{\pgfqpoint{6.227504in}{1.593801in}}{\pgfqpoint{6.222461in}{1.598844in}}%
\pgfpathcurveto{\pgfqpoint{6.217417in}{1.603888in}}{\pgfqpoint{6.210575in}{1.606722in}}{\pgfqpoint{6.203443in}{1.606722in}}%
\pgfpathcurveto{\pgfqpoint{6.196310in}{1.606722in}}{\pgfqpoint{6.189468in}{1.603888in}}{\pgfqpoint{6.184425in}{1.598844in}}%
\pgfpathcurveto{\pgfqpoint{6.179381in}{1.593801in}}{\pgfqpoint{6.176547in}{1.586959in}}{\pgfqpoint{6.176547in}{1.579826in}}%
\pgfpathcurveto{\pgfqpoint{6.176547in}{1.572693in}}{\pgfqpoint{6.179381in}{1.565852in}}{\pgfqpoint{6.184425in}{1.560808in}}%
\pgfpathcurveto{\pgfqpoint{6.189468in}{1.555764in}}{\pgfqpoint{6.196310in}{1.552930in}}{\pgfqpoint{6.203443in}{1.552930in}}%
\pgfpathclose%
\pgfusepath{stroke,fill}%
\end{pgfscope}%
\begin{pgfscope}%
\pgfpathrectangle{\pgfqpoint{4.985294in}{0.500000in}}{\pgfqpoint{1.764706in}{1.700000in}}%
\pgfusepath{clip}%
\pgfsetbuttcap%
\pgfsetroundjoin%
\definecolor{currentfill}{rgb}{0.550643,0.114208,0.357379}%
\pgfsetfillcolor{currentfill}%
\pgfsetlinewidth{0.311001pt}%
\definecolor{currentstroke}{rgb}{1.000000,1.000000,1.000000}%
\pgfsetstrokecolor{currentstroke}%
\pgfsetdash{}{0pt}%
\pgfpathmoveto{\pgfqpoint{5.949719in}{0.877492in}}%
\pgfpathcurveto{\pgfqpoint{5.956852in}{0.877492in}}{\pgfqpoint{5.963693in}{0.880326in}}{\pgfqpoint{5.968737in}{0.885369in}}%
\pgfpathcurveto{\pgfqpoint{5.973781in}{0.890413in}}{\pgfqpoint{5.976614in}{0.897255in}}{\pgfqpoint{5.976614in}{0.904387in}}%
\pgfpathcurveto{\pgfqpoint{5.976614in}{0.911520in}}{\pgfqpoint{5.973781in}{0.918362in}}{\pgfqpoint{5.968737in}{0.923406in}}%
\pgfpathcurveto{\pgfqpoint{5.963693in}{0.928449in}}{\pgfqpoint{5.956852in}{0.931283in}}{\pgfqpoint{5.949719in}{0.931283in}}%
\pgfpathcurveto{\pgfqpoint{5.942586in}{0.931283in}}{\pgfqpoint{5.935744in}{0.928449in}}{\pgfqpoint{5.930701in}{0.923406in}}%
\pgfpathcurveto{\pgfqpoint{5.925657in}{0.918362in}}{\pgfqpoint{5.922823in}{0.911520in}}{\pgfqpoint{5.922823in}{0.904387in}}%
\pgfpathcurveto{\pgfqpoint{5.922823in}{0.897255in}}{\pgfqpoint{5.925657in}{0.890413in}}{\pgfqpoint{5.930701in}{0.885369in}}%
\pgfpathcurveto{\pgfqpoint{5.935744in}{0.880326in}}{\pgfqpoint{5.942586in}{0.877492in}}{\pgfqpoint{5.949719in}{0.877492in}}%
\pgfpathclose%
\pgfusepath{stroke,fill}%
\end{pgfscope}%
\begin{pgfscope}%
\pgfpathrectangle{\pgfqpoint{4.985294in}{0.500000in}}{\pgfqpoint{1.764706in}{1.700000in}}%
\pgfusepath{clip}%
\pgfsetbuttcap%
\pgfsetroundjoin%
\definecolor{currentfill}{rgb}{0.975644,0.874038,0.797253}%
\pgfsetfillcolor{currentfill}%
\pgfsetlinewidth{0.311001pt}%
\definecolor{currentstroke}{rgb}{1.000000,1.000000,1.000000}%
\pgfsetstrokecolor{currentstroke}%
\pgfsetdash{}{0pt}%
\pgfpathmoveto{\pgfqpoint{5.458226in}{1.547007in}}%
\pgfpathcurveto{\pgfqpoint{5.465359in}{1.547007in}}{\pgfqpoint{5.472200in}{1.549841in}}{\pgfqpoint{5.477244in}{1.554885in}}%
\pgfpathcurveto{\pgfqpoint{5.482288in}{1.559929in}}{\pgfqpoint{5.485122in}{1.566770in}}{\pgfqpoint{5.485122in}{1.573903in}}%
\pgfpathcurveto{\pgfqpoint{5.485122in}{1.581036in}}{\pgfqpoint{5.482288in}{1.587877in}}{\pgfqpoint{5.477244in}{1.592921in}}%
\pgfpathcurveto{\pgfqpoint{5.472200in}{1.597965in}}{\pgfqpoint{5.465359in}{1.600799in}}{\pgfqpoint{5.458226in}{1.600799in}}%
\pgfpathcurveto{\pgfqpoint{5.451093in}{1.600799in}}{\pgfqpoint{5.444251in}{1.597965in}}{\pgfqpoint{5.439208in}{1.592921in}}%
\pgfpathcurveto{\pgfqpoint{5.434164in}{1.587877in}}{\pgfqpoint{5.431330in}{1.581036in}}{\pgfqpoint{5.431330in}{1.573903in}}%
\pgfpathcurveto{\pgfqpoint{5.431330in}{1.566770in}}{\pgfqpoint{5.434164in}{1.559929in}}{\pgfqpoint{5.439208in}{1.554885in}}%
\pgfpathcurveto{\pgfqpoint{5.444251in}{1.549841in}}{\pgfqpoint{5.451093in}{1.547007in}}{\pgfqpoint{5.458226in}{1.547007in}}%
\pgfpathclose%
\pgfusepath{stroke,fill}%
\end{pgfscope}%
\begin{pgfscope}%
\pgfpathrectangle{\pgfqpoint{4.985294in}{0.500000in}}{\pgfqpoint{1.764706in}{1.700000in}}%
\pgfusepath{clip}%
\pgfsetbuttcap%
\pgfsetroundjoin%
\definecolor{currentfill}{rgb}{0.960421,0.553286,0.393191}%
\pgfsetfillcolor{currentfill}%
\pgfsetlinewidth{0.311001pt}%
\definecolor{currentstroke}{rgb}{1.000000,1.000000,1.000000}%
\pgfsetstrokecolor{currentstroke}%
\pgfsetdash{}{0pt}%
\pgfpathmoveto{\pgfqpoint{6.113747in}{1.071955in}}%
\pgfpathcurveto{\pgfqpoint{6.120880in}{1.071955in}}{\pgfqpoint{6.127721in}{1.074789in}}{\pgfqpoint{6.132765in}{1.079833in}}%
\pgfpathcurveto{\pgfqpoint{6.137808in}{1.084876in}}{\pgfqpoint{6.140642in}{1.091718in}}{\pgfqpoint{6.140642in}{1.098851in}}%
\pgfpathcurveto{\pgfqpoint{6.140642in}{1.105984in}}{\pgfqpoint{6.137808in}{1.112825in}}{\pgfqpoint{6.132765in}{1.117869in}}%
\pgfpathcurveto{\pgfqpoint{6.127721in}{1.122913in}}{\pgfqpoint{6.120880in}{1.125747in}}{\pgfqpoint{6.113747in}{1.125747in}}%
\pgfpathcurveto{\pgfqpoint{6.106614in}{1.125747in}}{\pgfqpoint{6.099772in}{1.122913in}}{\pgfqpoint{6.094729in}{1.117869in}}%
\pgfpathcurveto{\pgfqpoint{6.089685in}{1.112825in}}{\pgfqpoint{6.086851in}{1.105984in}}{\pgfqpoint{6.086851in}{1.098851in}}%
\pgfpathcurveto{\pgfqpoint{6.086851in}{1.091718in}}{\pgfqpoint{6.089685in}{1.084876in}}{\pgfqpoint{6.094729in}{1.079833in}}%
\pgfpathcurveto{\pgfqpoint{6.099772in}{1.074789in}}{\pgfqpoint{6.106614in}{1.071955in}}{\pgfqpoint{6.113747in}{1.071955in}}%
\pgfpathclose%
\pgfusepath{stroke,fill}%
\end{pgfscope}%
\begin{pgfscope}%
\pgfpathrectangle{\pgfqpoint{4.985294in}{0.500000in}}{\pgfqpoint{1.764706in}{1.700000in}}%
\pgfusepath{clip}%
\pgfsetbuttcap%
\pgfsetroundjoin%
\definecolor{currentfill}{rgb}{0.964920,0.695342,0.545192}%
\pgfsetfillcolor{currentfill}%
\pgfsetlinewidth{0.311001pt}%
\definecolor{currentstroke}{rgb}{1.000000,1.000000,1.000000}%
\pgfsetstrokecolor{currentstroke}%
\pgfsetdash{}{0pt}%
\pgfpathmoveto{\pgfqpoint{6.398303in}{1.440018in}}%
\pgfpathcurveto{\pgfqpoint{6.405436in}{1.440018in}}{\pgfqpoint{6.412278in}{1.442851in}}{\pgfqpoint{6.417321in}{1.447895in}}%
\pgfpathcurveto{\pgfqpoint{6.422365in}{1.452939in}}{\pgfqpoint{6.425199in}{1.459780in}}{\pgfqpoint{6.425199in}{1.466913in}}%
\pgfpathcurveto{\pgfqpoint{6.425199in}{1.474046in}}{\pgfqpoint{6.422365in}{1.480888in}}{\pgfqpoint{6.417321in}{1.485931in}}%
\pgfpathcurveto{\pgfqpoint{6.412278in}{1.490975in}}{\pgfqpoint{6.405436in}{1.493809in}}{\pgfqpoint{6.398303in}{1.493809in}}%
\pgfpathcurveto{\pgfqpoint{6.391170in}{1.493809in}}{\pgfqpoint{6.384329in}{1.490975in}}{\pgfqpoint{6.379285in}{1.485931in}}%
\pgfpathcurveto{\pgfqpoint{6.374241in}{1.480888in}}{\pgfqpoint{6.371407in}{1.474046in}}{\pgfqpoint{6.371407in}{1.466913in}}%
\pgfpathcurveto{\pgfqpoint{6.371407in}{1.459780in}}{\pgfqpoint{6.374241in}{1.452939in}}{\pgfqpoint{6.379285in}{1.447895in}}%
\pgfpathcurveto{\pgfqpoint{6.384329in}{1.442851in}}{\pgfqpoint{6.391170in}{1.440018in}}{\pgfqpoint{6.398303in}{1.440018in}}%
\pgfpathclose%
\pgfusepath{stroke,fill}%
\end{pgfscope}%
\begin{pgfscope}%
\pgfpathrectangle{\pgfqpoint{4.985294in}{0.500000in}}{\pgfqpoint{1.764706in}{1.700000in}}%
\pgfusepath{clip}%
\pgfsetbuttcap%
\pgfsetroundjoin%
\definecolor{currentfill}{rgb}{0.975018,0.868213,0.788710}%
\pgfsetfillcolor{currentfill}%
\pgfsetlinewidth{0.311001pt}%
\definecolor{currentstroke}{rgb}{1.000000,1.000000,1.000000}%
\pgfsetstrokecolor{currentstroke}%
\pgfsetdash{}{0pt}%
\pgfpathmoveto{\pgfqpoint{5.373150in}{1.320378in}}%
\pgfpathcurveto{\pgfqpoint{5.380283in}{1.320378in}}{\pgfqpoint{5.387124in}{1.323212in}}{\pgfqpoint{5.392168in}{1.328256in}}%
\pgfpathcurveto{\pgfqpoint{5.397212in}{1.333299in}}{\pgfqpoint{5.400046in}{1.340141in}}{\pgfqpoint{5.400046in}{1.347274in}}%
\pgfpathcurveto{\pgfqpoint{5.400046in}{1.354406in}}{\pgfqpoint{5.397212in}{1.361248in}}{\pgfqpoint{5.392168in}{1.366292in}}%
\pgfpathcurveto{\pgfqpoint{5.387124in}{1.371335in}}{\pgfqpoint{5.380283in}{1.374169in}}{\pgfqpoint{5.373150in}{1.374169in}}%
\pgfpathcurveto{\pgfqpoint{5.366017in}{1.374169in}}{\pgfqpoint{5.359175in}{1.371335in}}{\pgfqpoint{5.354132in}{1.366292in}}%
\pgfpathcurveto{\pgfqpoint{5.349088in}{1.361248in}}{\pgfqpoint{5.346254in}{1.354406in}}{\pgfqpoint{5.346254in}{1.347274in}}%
\pgfpathcurveto{\pgfqpoint{5.346254in}{1.340141in}}{\pgfqpoint{5.349088in}{1.333299in}}{\pgfqpoint{5.354132in}{1.328256in}}%
\pgfpathcurveto{\pgfqpoint{5.359175in}{1.323212in}}{\pgfqpoint{5.366017in}{1.320378in}}{\pgfqpoint{5.373150in}{1.320378in}}%
\pgfpathclose%
\pgfusepath{stroke,fill}%
\end{pgfscope}%
\begin{pgfscope}%
\pgfpathrectangle{\pgfqpoint{4.985294in}{0.500000in}}{\pgfqpoint{1.764706in}{1.700000in}}%
\pgfusepath{clip}%
\pgfsetbuttcap%
\pgfsetroundjoin%
\definecolor{currentfill}{rgb}{0.977657,0.891500,0.822809}%
\pgfsetfillcolor{currentfill}%
\pgfsetlinewidth{0.311001pt}%
\definecolor{currentstroke}{rgb}{1.000000,1.000000,1.000000}%
\pgfsetstrokecolor{currentstroke}%
\pgfsetdash{}{0pt}%
\pgfpathmoveto{\pgfqpoint{5.447530in}{1.475774in}}%
\pgfpathcurveto{\pgfqpoint{5.454663in}{1.475774in}}{\pgfqpoint{5.461505in}{1.478608in}}{\pgfqpoint{5.466549in}{1.483652in}}%
\pgfpathcurveto{\pgfqpoint{5.471592in}{1.488696in}}{\pgfqpoint{5.474426in}{1.495537in}}{\pgfqpoint{5.474426in}{1.502670in}}%
\pgfpathcurveto{\pgfqpoint{5.474426in}{1.509803in}}{\pgfqpoint{5.471592in}{1.516645in}}{\pgfqpoint{5.466549in}{1.521688in}}%
\pgfpathcurveto{\pgfqpoint{5.461505in}{1.526732in}}{\pgfqpoint{5.454663in}{1.529566in}}{\pgfqpoint{5.447530in}{1.529566in}}%
\pgfpathcurveto{\pgfqpoint{5.440398in}{1.529566in}}{\pgfqpoint{5.433556in}{1.526732in}}{\pgfqpoint{5.428512in}{1.521688in}}%
\pgfpathcurveto{\pgfqpoint{5.423469in}{1.516645in}}{\pgfqpoint{5.420635in}{1.509803in}}{\pgfqpoint{5.420635in}{1.502670in}}%
\pgfpathcurveto{\pgfqpoint{5.420635in}{1.495537in}}{\pgfqpoint{5.423469in}{1.488696in}}{\pgfqpoint{5.428512in}{1.483652in}}%
\pgfpathcurveto{\pgfqpoint{5.433556in}{1.478608in}}{\pgfqpoint{5.440398in}{1.475774in}}{\pgfqpoint{5.447530in}{1.475774in}}%
\pgfpathclose%
\pgfusepath{stroke,fill}%
\end{pgfscope}%
\begin{pgfscope}%
\pgfpathrectangle{\pgfqpoint{4.985294in}{0.500000in}}{\pgfqpoint{1.764706in}{1.700000in}}%
\pgfusepath{clip}%
\pgfsetbuttcap%
\pgfsetroundjoin%
\definecolor{currentfill}{rgb}{0.940366,0.360209,0.257347}%
\pgfsetfillcolor{currentfill}%
\pgfsetlinewidth{0.311001pt}%
\definecolor{currentstroke}{rgb}{1.000000,1.000000,1.000000}%
\pgfsetstrokecolor{currentstroke}%
\pgfsetdash{}{0pt}%
\pgfpathmoveto{\pgfqpoint{5.569133in}{1.821031in}}%
\pgfpathcurveto{\pgfqpoint{5.576266in}{1.821031in}}{\pgfqpoint{5.583107in}{1.823865in}}{\pgfqpoint{5.588151in}{1.828909in}}%
\pgfpathcurveto{\pgfqpoint{5.593194in}{1.833952in}}{\pgfqpoint{5.596028in}{1.840794in}}{\pgfqpoint{5.596028in}{1.847927in}}%
\pgfpathcurveto{\pgfqpoint{5.596028in}{1.855059in}}{\pgfqpoint{5.593194in}{1.861901in}}{\pgfqpoint{5.588151in}{1.866945in}}%
\pgfpathcurveto{\pgfqpoint{5.583107in}{1.871988in}}{\pgfqpoint{5.576266in}{1.874822in}}{\pgfqpoint{5.569133in}{1.874822in}}%
\pgfpathcurveto{\pgfqpoint{5.562000in}{1.874822in}}{\pgfqpoint{5.555158in}{1.871988in}}{\pgfqpoint{5.550115in}{1.866945in}}%
\pgfpathcurveto{\pgfqpoint{5.545071in}{1.861901in}}{\pgfqpoint{5.542237in}{1.855059in}}{\pgfqpoint{5.542237in}{1.847927in}}%
\pgfpathcurveto{\pgfqpoint{5.542237in}{1.840794in}}{\pgfqpoint{5.545071in}{1.833952in}}{\pgfqpoint{5.550115in}{1.828909in}}%
\pgfpathcurveto{\pgfqpoint{5.555158in}{1.823865in}}{\pgfqpoint{5.562000in}{1.821031in}}{\pgfqpoint{5.569133in}{1.821031in}}%
\pgfpathclose%
\pgfusepath{stroke,fill}%
\end{pgfscope}%
\begin{pgfscope}%
\pgfpathrectangle{\pgfqpoint{4.985294in}{0.500000in}}{\pgfqpoint{1.764706in}{1.700000in}}%
\pgfusepath{clip}%
\pgfsetbuttcap%
\pgfsetroundjoin%
\definecolor{currentfill}{rgb}{0.972201,0.839051,0.745789}%
\pgfsetfillcolor{currentfill}%
\pgfsetlinewidth{0.311001pt}%
\definecolor{currentstroke}{rgb}{1.000000,1.000000,1.000000}%
\pgfsetstrokecolor{currentstroke}%
\pgfsetdash{}{0pt}%
\pgfpathmoveto{\pgfqpoint{5.374162in}{1.437717in}}%
\pgfpathcurveto{\pgfqpoint{5.381295in}{1.437717in}}{\pgfqpoint{5.388137in}{1.440551in}}{\pgfqpoint{5.393181in}{1.445595in}}%
\pgfpathcurveto{\pgfqpoint{5.398224in}{1.450638in}}{\pgfqpoint{5.401058in}{1.457480in}}{\pgfqpoint{5.401058in}{1.464613in}}%
\pgfpathcurveto{\pgfqpoint{5.401058in}{1.471746in}}{\pgfqpoint{5.398224in}{1.478587in}}{\pgfqpoint{5.393181in}{1.483631in}}%
\pgfpathcurveto{\pgfqpoint{5.388137in}{1.488675in}}{\pgfqpoint{5.381295in}{1.491508in}}{\pgfqpoint{5.374162in}{1.491508in}}%
\pgfpathcurveto{\pgfqpoint{5.367030in}{1.491508in}}{\pgfqpoint{5.360188in}{1.488675in}}{\pgfqpoint{5.355144in}{1.483631in}}%
\pgfpathcurveto{\pgfqpoint{5.350101in}{1.478587in}}{\pgfqpoint{5.347267in}{1.471746in}}{\pgfqpoint{5.347267in}{1.464613in}}%
\pgfpathcurveto{\pgfqpoint{5.347267in}{1.457480in}}{\pgfqpoint{5.350101in}{1.450638in}}{\pgfqpoint{5.355144in}{1.445595in}}%
\pgfpathcurveto{\pgfqpoint{5.360188in}{1.440551in}}{\pgfqpoint{5.367030in}{1.437717in}}{\pgfqpoint{5.374162in}{1.437717in}}%
\pgfpathclose%
\pgfusepath{stroke,fill}%
\end{pgfscope}%
\begin{pgfscope}%
\pgfpathrectangle{\pgfqpoint{4.985294in}{0.500000in}}{\pgfqpoint{1.764706in}{1.700000in}}%
\pgfusepath{clip}%
\pgfsetbuttcap%
\pgfsetroundjoin%
\definecolor{currentfill}{rgb}{0.979124,0.903132,0.839793}%
\pgfsetfillcolor{currentfill}%
\pgfsetlinewidth{0.311001pt}%
\definecolor{currentstroke}{rgb}{1.000000,1.000000,1.000000}%
\pgfsetstrokecolor{currentstroke}%
\pgfsetdash{}{0pt}%
\pgfpathmoveto{\pgfqpoint{6.321003in}{1.201334in}}%
\pgfpathcurveto{\pgfqpoint{6.328136in}{1.201334in}}{\pgfqpoint{6.334978in}{1.204168in}}{\pgfqpoint{6.340021in}{1.209212in}}%
\pgfpathcurveto{\pgfqpoint{6.345065in}{1.214255in}}{\pgfqpoint{6.347899in}{1.221097in}}{\pgfqpoint{6.347899in}{1.228230in}}%
\pgfpathcurveto{\pgfqpoint{6.347899in}{1.235363in}}{\pgfqpoint{6.345065in}{1.242204in}}{\pgfqpoint{6.340021in}{1.247248in}}%
\pgfpathcurveto{\pgfqpoint{6.334978in}{1.252292in}}{\pgfqpoint{6.328136in}{1.255126in}}{\pgfqpoint{6.321003in}{1.255126in}}%
\pgfpathcurveto{\pgfqpoint{6.313870in}{1.255126in}}{\pgfqpoint{6.307029in}{1.252292in}}{\pgfqpoint{6.301985in}{1.247248in}}%
\pgfpathcurveto{\pgfqpoint{6.296941in}{1.242204in}}{\pgfqpoint{6.294108in}{1.235363in}}{\pgfqpoint{6.294108in}{1.228230in}}%
\pgfpathcurveto{\pgfqpoint{6.294108in}{1.221097in}}{\pgfqpoint{6.296941in}{1.214255in}}{\pgfqpoint{6.301985in}{1.209212in}}%
\pgfpathcurveto{\pgfqpoint{6.307029in}{1.204168in}}{\pgfqpoint{6.313870in}{1.201334in}}{\pgfqpoint{6.321003in}{1.201334in}}%
\pgfpathclose%
\pgfusepath{stroke,fill}%
\end{pgfscope}%
\begin{pgfscope}%
\pgfpathrectangle{\pgfqpoint{4.985294in}{0.500000in}}{\pgfqpoint{1.764706in}{1.700000in}}%
\pgfusepath{clip}%
\pgfsetbuttcap%
\pgfsetroundjoin%
\definecolor{currentfill}{rgb}{0.938993,0.352507,0.254528}%
\pgfsetfillcolor{currentfill}%
\pgfsetlinewidth{0.311001pt}%
\definecolor{currentstroke}{rgb}{1.000000,1.000000,1.000000}%
\pgfsetstrokecolor{currentstroke}%
\pgfsetdash{}{0pt}%
\pgfpathmoveto{\pgfqpoint{5.688361in}{0.861641in}}%
\pgfpathcurveto{\pgfqpoint{5.695494in}{0.861641in}}{\pgfqpoint{5.702336in}{0.864475in}}{\pgfqpoint{5.707380in}{0.869519in}}%
\pgfpathcurveto{\pgfqpoint{5.712423in}{0.874563in}}{\pgfqpoint{5.715257in}{0.881404in}}{\pgfqpoint{5.715257in}{0.888537in}}%
\pgfpathcurveto{\pgfqpoint{5.715257in}{0.895670in}}{\pgfqpoint{5.712423in}{0.902512in}}{\pgfqpoint{5.707380in}{0.907555in}}%
\pgfpathcurveto{\pgfqpoint{5.702336in}{0.912599in}}{\pgfqpoint{5.695494in}{0.915433in}}{\pgfqpoint{5.688361in}{0.915433in}}%
\pgfpathcurveto{\pgfqpoint{5.681229in}{0.915433in}}{\pgfqpoint{5.674387in}{0.912599in}}{\pgfqpoint{5.669343in}{0.907555in}}%
\pgfpathcurveto{\pgfqpoint{5.664300in}{0.902512in}}{\pgfqpoint{5.661466in}{0.895670in}}{\pgfqpoint{5.661466in}{0.888537in}}%
\pgfpathcurveto{\pgfqpoint{5.661466in}{0.881404in}}{\pgfqpoint{5.664300in}{0.874563in}}{\pgfqpoint{5.669343in}{0.869519in}}%
\pgfpathcurveto{\pgfqpoint{5.674387in}{0.864475in}}{\pgfqpoint{5.681229in}{0.861641in}}{\pgfqpoint{5.688361in}{0.861641in}}%
\pgfpathclose%
\pgfusepath{stroke,fill}%
\end{pgfscope}%
\begin{pgfscope}%
\pgfpathrectangle{\pgfqpoint{4.985294in}{0.500000in}}{\pgfqpoint{1.764706in}{1.700000in}}%
\pgfusepath{clip}%
\pgfsetbuttcap%
\pgfsetroundjoin%
\definecolor{currentfill}{rgb}{0.796501,0.105066,0.310630}%
\pgfsetfillcolor{currentfill}%
\pgfsetlinewidth{0.311001pt}%
\definecolor{currentstroke}{rgb}{1.000000,1.000000,1.000000}%
\pgfsetstrokecolor{currentstroke}%
\pgfsetdash{}{0pt}%
\pgfpathmoveto{\pgfqpoint{5.502368in}{1.846642in}}%
\pgfpathcurveto{\pgfqpoint{5.509501in}{1.846642in}}{\pgfqpoint{5.516343in}{1.849476in}}{\pgfqpoint{5.521387in}{1.854520in}}%
\pgfpathcurveto{\pgfqpoint{5.526430in}{1.859564in}}{\pgfqpoint{5.529264in}{1.866405in}}{\pgfqpoint{5.529264in}{1.873538in}}%
\pgfpathcurveto{\pgfqpoint{5.529264in}{1.880671in}}{\pgfqpoint{5.526430in}{1.887513in}}{\pgfqpoint{5.521387in}{1.892556in}}%
\pgfpathcurveto{\pgfqpoint{5.516343in}{1.897600in}}{\pgfqpoint{5.509501in}{1.900434in}}{\pgfqpoint{5.502368in}{1.900434in}}%
\pgfpathcurveto{\pgfqpoint{5.495236in}{1.900434in}}{\pgfqpoint{5.488394in}{1.897600in}}{\pgfqpoint{5.483350in}{1.892556in}}%
\pgfpathcurveto{\pgfqpoint{5.478307in}{1.887513in}}{\pgfqpoint{5.475473in}{1.880671in}}{\pgfqpoint{5.475473in}{1.873538in}}%
\pgfpathcurveto{\pgfqpoint{5.475473in}{1.866405in}}{\pgfqpoint{5.478307in}{1.859564in}}{\pgfqpoint{5.483350in}{1.854520in}}%
\pgfpathcurveto{\pgfqpoint{5.488394in}{1.849476in}}{\pgfqpoint{5.495236in}{1.846642in}}{\pgfqpoint{5.502368in}{1.846642in}}%
\pgfpathclose%
\pgfusepath{stroke,fill}%
\end{pgfscope}%
\begin{pgfscope}%
\pgfpathrectangle{\pgfqpoint{4.985294in}{0.500000in}}{\pgfqpoint{1.764706in}{1.700000in}}%
\pgfusepath{clip}%
\pgfsetbuttcap%
\pgfsetroundjoin%
\definecolor{currentfill}{rgb}{0.966328,0.750560,0.616961}%
\pgfsetfillcolor{currentfill}%
\pgfsetlinewidth{0.311001pt}%
\definecolor{currentstroke}{rgb}{1.000000,1.000000,1.000000}%
\pgfsetstrokecolor{currentstroke}%
\pgfsetdash{}{0pt}%
\pgfpathmoveto{\pgfqpoint{5.516652in}{1.216287in}}%
\pgfpathcurveto{\pgfqpoint{5.523785in}{1.216287in}}{\pgfqpoint{5.530626in}{1.219121in}}{\pgfqpoint{5.535670in}{1.224165in}}%
\pgfpathcurveto{\pgfqpoint{5.540714in}{1.229208in}}{\pgfqpoint{5.543548in}{1.236050in}}{\pgfqpoint{5.543548in}{1.243183in}}%
\pgfpathcurveto{\pgfqpoint{5.543548in}{1.250316in}}{\pgfqpoint{5.540714in}{1.257157in}}{\pgfqpoint{5.535670in}{1.262201in}}%
\pgfpathcurveto{\pgfqpoint{5.530626in}{1.267245in}}{\pgfqpoint{5.523785in}{1.270079in}}{\pgfqpoint{5.516652in}{1.270079in}}%
\pgfpathcurveto{\pgfqpoint{5.509519in}{1.270079in}}{\pgfqpoint{5.502677in}{1.267245in}}{\pgfqpoint{5.497634in}{1.262201in}}%
\pgfpathcurveto{\pgfqpoint{5.492590in}{1.257157in}}{\pgfqpoint{5.489756in}{1.250316in}}{\pgfqpoint{5.489756in}{1.243183in}}%
\pgfpathcurveto{\pgfqpoint{5.489756in}{1.236050in}}{\pgfqpoint{5.492590in}{1.229208in}}{\pgfqpoint{5.497634in}{1.224165in}}%
\pgfpathcurveto{\pgfqpoint{5.502677in}{1.219121in}}{\pgfqpoint{5.509519in}{1.216287in}}{\pgfqpoint{5.516652in}{1.216287in}}%
\pgfpathclose%
\pgfusepath{stroke,fill}%
\end{pgfscope}%
\begin{pgfscope}%
\pgfpathrectangle{\pgfqpoint{4.985294in}{0.500000in}}{\pgfqpoint{1.764706in}{1.700000in}}%
\pgfusepath{clip}%
\pgfsetbuttcap%
\pgfsetroundjoin%
\definecolor{currentfill}{rgb}{0.964799,0.689101,0.537560}%
\pgfsetfillcolor{currentfill}%
\pgfsetlinewidth{0.311001pt}%
\definecolor{currentstroke}{rgb}{1.000000,1.000000,1.000000}%
\pgfsetstrokecolor{currentstroke}%
\pgfsetdash{}{0pt}%
\pgfpathmoveto{\pgfqpoint{6.157834in}{1.113220in}}%
\pgfpathcurveto{\pgfqpoint{6.164967in}{1.113220in}}{\pgfqpoint{6.171809in}{1.116054in}}{\pgfqpoint{6.176853in}{1.121097in}}%
\pgfpathcurveto{\pgfqpoint{6.181896in}{1.126141in}}{\pgfqpoint{6.184730in}{1.132983in}}{\pgfqpoint{6.184730in}{1.140115in}}%
\pgfpathcurveto{\pgfqpoint{6.184730in}{1.147248in}}{\pgfqpoint{6.181896in}{1.154090in}}{\pgfqpoint{6.176853in}{1.159134in}}%
\pgfpathcurveto{\pgfqpoint{6.171809in}{1.164177in}}{\pgfqpoint{6.164967in}{1.167011in}}{\pgfqpoint{6.157834in}{1.167011in}}%
\pgfpathcurveto{\pgfqpoint{6.150702in}{1.167011in}}{\pgfqpoint{6.143860in}{1.164177in}}{\pgfqpoint{6.138816in}{1.159134in}}%
\pgfpathcurveto{\pgfqpoint{6.133773in}{1.154090in}}{\pgfqpoint{6.130939in}{1.147248in}}{\pgfqpoint{6.130939in}{1.140115in}}%
\pgfpathcurveto{\pgfqpoint{6.130939in}{1.132983in}}{\pgfqpoint{6.133773in}{1.126141in}}{\pgfqpoint{6.138816in}{1.121097in}}%
\pgfpathcurveto{\pgfqpoint{6.143860in}{1.116054in}}{\pgfqpoint{6.150702in}{1.113220in}}{\pgfqpoint{6.157834in}{1.113220in}}%
\pgfpathclose%
\pgfusepath{stroke,fill}%
\end{pgfscope}%
\begin{pgfscope}%
\pgfpathrectangle{\pgfqpoint{4.985294in}{0.500000in}}{\pgfqpoint{1.764706in}{1.700000in}}%
\pgfusepath{clip}%
\pgfsetbuttcap%
\pgfsetroundjoin%
\definecolor{currentfill}{rgb}{0.966120,0.744512,0.608720}%
\pgfsetfillcolor{currentfill}%
\pgfsetlinewidth{0.311001pt}%
\definecolor{currentstroke}{rgb}{1.000000,1.000000,1.000000}%
\pgfsetstrokecolor{currentstroke}%
\pgfsetdash{}{0pt}%
\pgfpathmoveto{\pgfqpoint{5.524606in}{1.183088in}}%
\pgfpathcurveto{\pgfqpoint{5.531739in}{1.183088in}}{\pgfqpoint{5.538580in}{1.185922in}}{\pgfqpoint{5.543624in}{1.190966in}}%
\pgfpathcurveto{\pgfqpoint{5.548668in}{1.196009in}}{\pgfqpoint{5.551501in}{1.202851in}}{\pgfqpoint{5.551501in}{1.209984in}}%
\pgfpathcurveto{\pgfqpoint{5.551501in}{1.217117in}}{\pgfqpoint{5.548668in}{1.223958in}}{\pgfqpoint{5.543624in}{1.229002in}}%
\pgfpathcurveto{\pgfqpoint{5.538580in}{1.234046in}}{\pgfqpoint{5.531739in}{1.236880in}}{\pgfqpoint{5.524606in}{1.236880in}}%
\pgfpathcurveto{\pgfqpoint{5.517473in}{1.236880in}}{\pgfqpoint{5.510631in}{1.234046in}}{\pgfqpoint{5.505588in}{1.229002in}}%
\pgfpathcurveto{\pgfqpoint{5.500544in}{1.223958in}}{\pgfqpoint{5.497710in}{1.217117in}}{\pgfqpoint{5.497710in}{1.209984in}}%
\pgfpathcurveto{\pgfqpoint{5.497710in}{1.202851in}}{\pgfqpoint{5.500544in}{1.196009in}}{\pgfqpoint{5.505588in}{1.190966in}}%
\pgfpathcurveto{\pgfqpoint{5.510631in}{1.185922in}}{\pgfqpoint{5.517473in}{1.183088in}}{\pgfqpoint{5.524606in}{1.183088in}}%
\pgfpathclose%
\pgfusepath{stroke,fill}%
\end{pgfscope}%
\begin{pgfscope}%
\pgfpathrectangle{\pgfqpoint{4.985294in}{0.500000in}}{\pgfqpoint{1.764706in}{1.700000in}}%
\pgfusepath{clip}%
\pgfsetbuttcap%
\pgfsetroundjoin%
\definecolor{currentfill}{rgb}{0.898503,0.224633,0.251087}%
\pgfsetfillcolor{currentfill}%
\pgfsetlinewidth{0.311001pt}%
\definecolor{currentstroke}{rgb}{1.000000,1.000000,1.000000}%
\pgfsetstrokecolor{currentstroke}%
\pgfsetdash{}{0pt}%
\pgfpathmoveto{\pgfqpoint{6.321930in}{0.881797in}}%
\pgfpathcurveto{\pgfqpoint{6.329062in}{0.881797in}}{\pgfqpoint{6.335904in}{0.884631in}}{\pgfqpoint{6.340948in}{0.889674in}}%
\pgfpathcurveto{\pgfqpoint{6.345991in}{0.894718in}}{\pgfqpoint{6.348825in}{0.901560in}}{\pgfqpoint{6.348825in}{0.908692in}}%
\pgfpathcurveto{\pgfqpoint{6.348825in}{0.915825in}}{\pgfqpoint{6.345991in}{0.922667in}}{\pgfqpoint{6.340948in}{0.927711in}}%
\pgfpathcurveto{\pgfqpoint{6.335904in}{0.932754in}}{\pgfqpoint{6.329062in}{0.935588in}}{\pgfqpoint{6.321930in}{0.935588in}}%
\pgfpathcurveto{\pgfqpoint{6.314797in}{0.935588in}}{\pgfqpoint{6.307955in}{0.932754in}}{\pgfqpoint{6.302911in}{0.927711in}}%
\pgfpathcurveto{\pgfqpoint{6.297868in}{0.922667in}}{\pgfqpoint{6.295034in}{0.915825in}}{\pgfqpoint{6.295034in}{0.908692in}}%
\pgfpathcurveto{\pgfqpoint{6.295034in}{0.901560in}}{\pgfqpoint{6.297868in}{0.894718in}}{\pgfqpoint{6.302911in}{0.889674in}}%
\pgfpathcurveto{\pgfqpoint{6.307955in}{0.884631in}}{\pgfqpoint{6.314797in}{0.881797in}}{\pgfqpoint{6.321930in}{0.881797in}}%
\pgfpathclose%
\pgfusepath{stroke,fill}%
\end{pgfscope}%
\begin{pgfscope}%
\pgfpathrectangle{\pgfqpoint{4.985294in}{0.500000in}}{\pgfqpoint{1.764706in}{1.700000in}}%
\pgfusepath{clip}%
\pgfsetbuttcap%
\pgfsetroundjoin%
\definecolor{currentfill}{rgb}{0.972201,0.839051,0.745789}%
\pgfsetfillcolor{currentfill}%
\pgfsetlinewidth{0.311001pt}%
\definecolor{currentstroke}{rgb}{1.000000,1.000000,1.000000}%
\pgfsetstrokecolor{currentstroke}%
\pgfsetdash{}{0pt}%
\pgfpathmoveto{\pgfqpoint{5.472815in}{1.609262in}}%
\pgfpathcurveto{\pgfqpoint{5.479948in}{1.609262in}}{\pgfqpoint{5.486789in}{1.612096in}}{\pgfqpoint{5.491833in}{1.617140in}}%
\pgfpathcurveto{\pgfqpoint{5.496877in}{1.622183in}}{\pgfqpoint{5.499711in}{1.629025in}}{\pgfqpoint{5.499711in}{1.636158in}}%
\pgfpathcurveto{\pgfqpoint{5.499711in}{1.643290in}}{\pgfqpoint{5.496877in}{1.650132in}}{\pgfqpoint{5.491833in}{1.655176in}}%
\pgfpathcurveto{\pgfqpoint{5.486789in}{1.660219in}}{\pgfqpoint{5.479948in}{1.663053in}}{\pgfqpoint{5.472815in}{1.663053in}}%
\pgfpathcurveto{\pgfqpoint{5.465682in}{1.663053in}}{\pgfqpoint{5.458840in}{1.660219in}}{\pgfqpoint{5.453797in}{1.655176in}}%
\pgfpathcurveto{\pgfqpoint{5.448753in}{1.650132in}}{\pgfqpoint{5.445919in}{1.643290in}}{\pgfqpoint{5.445919in}{1.636158in}}%
\pgfpathcurveto{\pgfqpoint{5.445919in}{1.629025in}}{\pgfqpoint{5.448753in}{1.622183in}}{\pgfqpoint{5.453797in}{1.617140in}}%
\pgfpathcurveto{\pgfqpoint{5.458840in}{1.612096in}}{\pgfqpoint{5.465682in}{1.609262in}}{\pgfqpoint{5.472815in}{1.609262in}}%
\pgfpathclose%
\pgfusepath{stroke,fill}%
\end{pgfscope}%
\begin{pgfscope}%
\pgfpathrectangle{\pgfqpoint{4.985294in}{0.500000in}}{\pgfqpoint{1.764706in}{1.700000in}}%
\pgfusepath{clip}%
\pgfsetbuttcap%
\pgfsetroundjoin%
\definecolor{currentfill}{rgb}{0.975018,0.868213,0.788710}%
\pgfsetfillcolor{currentfill}%
\pgfsetlinewidth{0.311001pt}%
\definecolor{currentstroke}{rgb}{1.000000,1.000000,1.000000}%
\pgfsetstrokecolor{currentstroke}%
\pgfsetdash{}{0pt}%
\pgfpathmoveto{\pgfqpoint{5.468870in}{1.065375in}}%
\pgfpathcurveto{\pgfqpoint{5.476003in}{1.065375in}}{\pgfqpoint{5.482845in}{1.068209in}}{\pgfqpoint{5.487888in}{1.073253in}}%
\pgfpathcurveto{\pgfqpoint{5.492932in}{1.078296in}}{\pgfqpoint{5.495766in}{1.085138in}}{\pgfqpoint{5.495766in}{1.092271in}}%
\pgfpathcurveto{\pgfqpoint{5.495766in}{1.099404in}}{\pgfqpoint{5.492932in}{1.106245in}}{\pgfqpoint{5.487888in}{1.111289in}}%
\pgfpathcurveto{\pgfqpoint{5.482845in}{1.116333in}}{\pgfqpoint{5.476003in}{1.119166in}}{\pgfqpoint{5.468870in}{1.119166in}}%
\pgfpathcurveto{\pgfqpoint{5.461737in}{1.119166in}}{\pgfqpoint{5.454896in}{1.116333in}}{\pgfqpoint{5.449852in}{1.111289in}}%
\pgfpathcurveto{\pgfqpoint{5.444808in}{1.106245in}}{\pgfqpoint{5.441974in}{1.099404in}}{\pgfqpoint{5.441974in}{1.092271in}}%
\pgfpathcurveto{\pgfqpoint{5.441974in}{1.085138in}}{\pgfqpoint{5.444808in}{1.078296in}}{\pgfqpoint{5.449852in}{1.073253in}}%
\pgfpathcurveto{\pgfqpoint{5.454896in}{1.068209in}}{\pgfqpoint{5.461737in}{1.065375in}}{\pgfqpoint{5.468870in}{1.065375in}}%
\pgfpathclose%
\pgfusepath{stroke,fill}%
\end{pgfscope}%
\begin{pgfscope}%
\pgfpathrectangle{\pgfqpoint{4.985294in}{0.500000in}}{\pgfqpoint{1.764706in}{1.700000in}}%
\pgfusepath{clip}%
\pgfsetbuttcap%
\pgfsetroundjoin%
\definecolor{currentfill}{rgb}{0.972726,0.844889,0.754401}%
\pgfsetfillcolor{currentfill}%
\pgfsetlinewidth{0.311001pt}%
\definecolor{currentstroke}{rgb}{1.000000,1.000000,1.000000}%
\pgfsetstrokecolor{currentstroke}%
\pgfsetdash{}{0pt}%
\pgfpathmoveto{\pgfqpoint{5.512475in}{1.570501in}}%
\pgfpathcurveto{\pgfqpoint{5.519608in}{1.570501in}}{\pgfqpoint{5.526449in}{1.573335in}}{\pgfqpoint{5.531493in}{1.578379in}}%
\pgfpathcurveto{\pgfqpoint{5.536537in}{1.583423in}}{\pgfqpoint{5.539371in}{1.590264in}}{\pgfqpoint{5.539371in}{1.597397in}}%
\pgfpathcurveto{\pgfqpoint{5.539371in}{1.604530in}}{\pgfqpoint{5.536537in}{1.611371in}}{\pgfqpoint{5.531493in}{1.616415in}}%
\pgfpathcurveto{\pgfqpoint{5.526449in}{1.621459in}}{\pgfqpoint{5.519608in}{1.624293in}}{\pgfqpoint{5.512475in}{1.624293in}}%
\pgfpathcurveto{\pgfqpoint{5.505342in}{1.624293in}}{\pgfqpoint{5.498501in}{1.621459in}}{\pgfqpoint{5.493457in}{1.616415in}}%
\pgfpathcurveto{\pgfqpoint{5.488413in}{1.611371in}}{\pgfqpoint{5.485579in}{1.604530in}}{\pgfqpoint{5.485579in}{1.597397in}}%
\pgfpathcurveto{\pgfqpoint{5.485579in}{1.590264in}}{\pgfqpoint{5.488413in}{1.583423in}}{\pgfqpoint{5.493457in}{1.578379in}}%
\pgfpathcurveto{\pgfqpoint{5.498501in}{1.573335in}}{\pgfqpoint{5.505342in}{1.570501in}}{\pgfqpoint{5.512475in}{1.570501in}}%
\pgfpathclose%
\pgfusepath{stroke,fill}%
\end{pgfscope}%
\begin{pgfscope}%
\pgfpathrectangle{\pgfqpoint{4.985294in}{0.500000in}}{\pgfqpoint{1.764706in}{1.700000in}}%
\pgfusepath{clip}%
\pgfsetbuttcap%
\pgfsetroundjoin%
\definecolor{currentfill}{rgb}{0.959229,0.533075,0.374889}%
\pgfsetfillcolor{currentfill}%
\pgfsetlinewidth{0.311001pt}%
\definecolor{currentstroke}{rgb}{1.000000,1.000000,1.000000}%
\pgfsetstrokecolor{currentstroke}%
\pgfsetdash{}{0pt}%
\pgfpathmoveto{\pgfqpoint{5.319094in}{1.117065in}}%
\pgfpathcurveto{\pgfqpoint{5.326227in}{1.117065in}}{\pgfqpoint{5.333069in}{1.119898in}}{\pgfqpoint{5.338112in}{1.124942in}}%
\pgfpathcurveto{\pgfqpoint{5.343156in}{1.129986in}}{\pgfqpoint{5.345990in}{1.136827in}}{\pgfqpoint{5.345990in}{1.143960in}}%
\pgfpathcurveto{\pgfqpoint{5.345990in}{1.151093in}}{\pgfqpoint{5.343156in}{1.157935in}}{\pgfqpoint{5.338112in}{1.162978in}}%
\pgfpathcurveto{\pgfqpoint{5.333069in}{1.168022in}}{\pgfqpoint{5.326227in}{1.170856in}}{\pgfqpoint{5.319094in}{1.170856in}}%
\pgfpathcurveto{\pgfqpoint{5.311962in}{1.170856in}}{\pgfqpoint{5.305120in}{1.168022in}}{\pgfqpoint{5.300076in}{1.162978in}}%
\pgfpathcurveto{\pgfqpoint{5.295033in}{1.157935in}}{\pgfqpoint{5.292199in}{1.151093in}}{\pgfqpoint{5.292199in}{1.143960in}}%
\pgfpathcurveto{\pgfqpoint{5.292199in}{1.136827in}}{\pgfqpoint{5.295033in}{1.129986in}}{\pgfqpoint{5.300076in}{1.124942in}}%
\pgfpathcurveto{\pgfqpoint{5.305120in}{1.119898in}}{\pgfqpoint{5.311962in}{1.117065in}}{\pgfqpoint{5.319094in}{1.117065in}}%
\pgfpathclose%
\pgfusepath{stroke,fill}%
\end{pgfscope}%
\begin{pgfscope}%
\pgfpathrectangle{\pgfqpoint{4.985294in}{0.500000in}}{\pgfqpoint{1.764706in}{1.700000in}}%
\pgfusepath{clip}%
\pgfsetbuttcap%
\pgfsetroundjoin%
\definecolor{currentfill}{rgb}{0.979124,0.903132,0.839793}%
\pgfsetfillcolor{currentfill}%
\pgfsetlinewidth{0.311001pt}%
\definecolor{currentstroke}{rgb}{1.000000,1.000000,1.000000}%
\pgfsetstrokecolor{currentstroke}%
\pgfsetdash{}{0pt}%
\pgfpathmoveto{\pgfqpoint{6.287544in}{1.429559in}}%
\pgfpathcurveto{\pgfqpoint{6.294677in}{1.429559in}}{\pgfqpoint{6.301518in}{1.432393in}}{\pgfqpoint{6.306562in}{1.437437in}}%
\pgfpathcurveto{\pgfqpoint{6.311606in}{1.442481in}}{\pgfqpoint{6.314440in}{1.449322in}}{\pgfqpoint{6.314440in}{1.456455in}}%
\pgfpathcurveto{\pgfqpoint{6.314440in}{1.463588in}}{\pgfqpoint{6.311606in}{1.470430in}}{\pgfqpoint{6.306562in}{1.475473in}}%
\pgfpathcurveto{\pgfqpoint{6.301518in}{1.480517in}}{\pgfqpoint{6.294677in}{1.483351in}}{\pgfqpoint{6.287544in}{1.483351in}}%
\pgfpathcurveto{\pgfqpoint{6.280411in}{1.483351in}}{\pgfqpoint{6.273569in}{1.480517in}}{\pgfqpoint{6.268526in}{1.475473in}}%
\pgfpathcurveto{\pgfqpoint{6.263482in}{1.470430in}}{\pgfqpoint{6.260648in}{1.463588in}}{\pgfqpoint{6.260648in}{1.456455in}}%
\pgfpathcurveto{\pgfqpoint{6.260648in}{1.449322in}}{\pgfqpoint{6.263482in}{1.442481in}}{\pgfqpoint{6.268526in}{1.437437in}}%
\pgfpathcurveto{\pgfqpoint{6.273569in}{1.432393in}}{\pgfqpoint{6.280411in}{1.429559in}}{\pgfqpoint{6.287544in}{1.429559in}}%
\pgfpathclose%
\pgfusepath{stroke,fill}%
\end{pgfscope}%
\begin{pgfscope}%
\pgfpathrectangle{\pgfqpoint{4.985294in}{0.500000in}}{\pgfqpoint{1.764706in}{1.700000in}}%
\pgfusepath{clip}%
\pgfsetbuttcap%
\pgfsetroundjoin%
\definecolor{currentfill}{rgb}{0.967735,0.780441,0.659127}%
\pgfsetfillcolor{currentfill}%
\pgfsetlinewidth{0.311001pt}%
\definecolor{currentstroke}{rgb}{1.000000,1.000000,1.000000}%
\pgfsetstrokecolor{currentstroke}%
\pgfsetdash{}{0pt}%
\pgfpathmoveto{\pgfqpoint{6.233112in}{1.682881in}}%
\pgfpathcurveto{\pgfqpoint{6.240245in}{1.682881in}}{\pgfqpoint{6.247087in}{1.685715in}}{\pgfqpoint{6.252130in}{1.690759in}}%
\pgfpathcurveto{\pgfqpoint{6.257174in}{1.695802in}}{\pgfqpoint{6.260008in}{1.702644in}}{\pgfqpoint{6.260008in}{1.709777in}}%
\pgfpathcurveto{\pgfqpoint{6.260008in}{1.716910in}}{\pgfqpoint{6.257174in}{1.723751in}}{\pgfqpoint{6.252130in}{1.728795in}}%
\pgfpathcurveto{\pgfqpoint{6.247087in}{1.733839in}}{\pgfqpoint{6.240245in}{1.736673in}}{\pgfqpoint{6.233112in}{1.736673in}}%
\pgfpathcurveto{\pgfqpoint{6.225979in}{1.736673in}}{\pgfqpoint{6.219138in}{1.733839in}}{\pgfqpoint{6.214094in}{1.728795in}}%
\pgfpathcurveto{\pgfqpoint{6.209050in}{1.723751in}}{\pgfqpoint{6.206216in}{1.716910in}}{\pgfqpoint{6.206216in}{1.709777in}}%
\pgfpathcurveto{\pgfqpoint{6.206216in}{1.702644in}}{\pgfqpoint{6.209050in}{1.695802in}}{\pgfqpoint{6.214094in}{1.690759in}}%
\pgfpathcurveto{\pgfqpoint{6.219138in}{1.685715in}}{\pgfqpoint{6.225979in}{1.682881in}}{\pgfqpoint{6.233112in}{1.682881in}}%
\pgfpathclose%
\pgfusepath{stroke,fill}%
\end{pgfscope}%
\begin{pgfscope}%
\pgfpathrectangle{\pgfqpoint{4.985294in}{0.500000in}}{\pgfqpoint{1.764706in}{1.700000in}}%
\pgfusepath{clip}%
\pgfsetbuttcap%
\pgfsetroundjoin%
\definecolor{currentfill}{rgb}{0.961433,0.573272,0.412036}%
\pgfsetfillcolor{currentfill}%
\pgfsetlinewidth{0.311001pt}%
\definecolor{currentstroke}{rgb}{1.000000,1.000000,1.000000}%
\pgfsetstrokecolor{currentstroke}%
\pgfsetdash{}{0pt}%
\pgfpathmoveto{\pgfqpoint{6.297246in}{0.937645in}}%
\pgfpathcurveto{\pgfqpoint{6.304379in}{0.937645in}}{\pgfqpoint{6.311221in}{0.940479in}}{\pgfqpoint{6.316264in}{0.945522in}}%
\pgfpathcurveto{\pgfqpoint{6.321308in}{0.950566in}}{\pgfqpoint{6.324142in}{0.957408in}}{\pgfqpoint{6.324142in}{0.964540in}}%
\pgfpathcurveto{\pgfqpoint{6.324142in}{0.971673in}}{\pgfqpoint{6.321308in}{0.978515in}}{\pgfqpoint{6.316264in}{0.983559in}}%
\pgfpathcurveto{\pgfqpoint{6.311221in}{0.988602in}}{\pgfqpoint{6.304379in}{0.991436in}}{\pgfqpoint{6.297246in}{0.991436in}}%
\pgfpathcurveto{\pgfqpoint{6.290113in}{0.991436in}}{\pgfqpoint{6.283272in}{0.988602in}}{\pgfqpoint{6.278228in}{0.983559in}}%
\pgfpathcurveto{\pgfqpoint{6.273184in}{0.978515in}}{\pgfqpoint{6.270351in}{0.971673in}}{\pgfqpoint{6.270351in}{0.964540in}}%
\pgfpathcurveto{\pgfqpoint{6.270351in}{0.957408in}}{\pgfqpoint{6.273184in}{0.950566in}}{\pgfqpoint{6.278228in}{0.945522in}}%
\pgfpathcurveto{\pgfqpoint{6.283272in}{0.940479in}}{\pgfqpoint{6.290113in}{0.937645in}}{\pgfqpoint{6.297246in}{0.937645in}}%
\pgfpathclose%
\pgfusepath{stroke,fill}%
\end{pgfscope}%
\begin{pgfscope}%
\pgfpathrectangle{\pgfqpoint{4.985294in}{0.500000in}}{\pgfqpoint{1.764706in}{1.700000in}}%
\pgfusepath{clip}%
\pgfsetbuttcap%
\pgfsetroundjoin%
\definecolor{currentfill}{rgb}{0.973271,0.850724,0.762998}%
\pgfsetfillcolor{currentfill}%
\pgfsetlinewidth{0.311001pt}%
\definecolor{currentstroke}{rgb}{1.000000,1.000000,1.000000}%
\pgfsetstrokecolor{currentstroke}%
\pgfsetdash{}{0pt}%
\pgfpathmoveto{\pgfqpoint{6.258690in}{1.424683in}}%
\pgfpathcurveto{\pgfqpoint{6.265823in}{1.424683in}}{\pgfqpoint{6.272665in}{1.427517in}}{\pgfqpoint{6.277708in}{1.432561in}}%
\pgfpathcurveto{\pgfqpoint{6.282752in}{1.437605in}}{\pgfqpoint{6.285586in}{1.444446in}}{\pgfqpoint{6.285586in}{1.451579in}}%
\pgfpathcurveto{\pgfqpoint{6.285586in}{1.458712in}}{\pgfqpoint{6.282752in}{1.465554in}}{\pgfqpoint{6.277708in}{1.470597in}}%
\pgfpathcurveto{\pgfqpoint{6.272665in}{1.475641in}}{\pgfqpoint{6.265823in}{1.478475in}}{\pgfqpoint{6.258690in}{1.478475in}}%
\pgfpathcurveto{\pgfqpoint{6.251557in}{1.478475in}}{\pgfqpoint{6.244716in}{1.475641in}}{\pgfqpoint{6.239672in}{1.470597in}}%
\pgfpathcurveto{\pgfqpoint{6.234628in}{1.465554in}}{\pgfqpoint{6.231794in}{1.458712in}}{\pgfqpoint{6.231794in}{1.451579in}}%
\pgfpathcurveto{\pgfqpoint{6.231794in}{1.444446in}}{\pgfqpoint{6.234628in}{1.437605in}}{\pgfqpoint{6.239672in}{1.432561in}}%
\pgfpathcurveto{\pgfqpoint{6.244716in}{1.427517in}}{\pgfqpoint{6.251557in}{1.424683in}}{\pgfqpoint{6.258690in}{1.424683in}}%
\pgfpathclose%
\pgfusepath{stroke,fill}%
\end{pgfscope}%
\begin{pgfscope}%
\pgfpathrectangle{\pgfqpoint{4.985294in}{0.500000in}}{\pgfqpoint{1.764706in}{1.700000in}}%
\pgfusepath{clip}%
\pgfsetbuttcap%
\pgfsetroundjoin%
\definecolor{currentfill}{rgb}{0.950017,0.427714,0.292447}%
\pgfsetfillcolor{currentfill}%
\pgfsetlinewidth{0.311001pt}%
\definecolor{currentstroke}{rgb}{1.000000,1.000000,1.000000}%
\pgfsetstrokecolor{currentstroke}%
\pgfsetdash{}{0pt}%
\pgfpathmoveto{\pgfqpoint{5.325370in}{1.560202in}}%
\pgfpathcurveto{\pgfqpoint{5.332503in}{1.560202in}}{\pgfqpoint{5.339344in}{1.563036in}}{\pgfqpoint{5.344388in}{1.568079in}}%
\pgfpathcurveto{\pgfqpoint{5.349432in}{1.573123in}}{\pgfqpoint{5.352266in}{1.579965in}}{\pgfqpoint{5.352266in}{1.587097in}}%
\pgfpathcurveto{\pgfqpoint{5.352266in}{1.594230in}}{\pgfqpoint{5.349432in}{1.601072in}}{\pgfqpoint{5.344388in}{1.606116in}}%
\pgfpathcurveto{\pgfqpoint{5.339344in}{1.611159in}}{\pgfqpoint{5.332503in}{1.613993in}}{\pgfqpoint{5.325370in}{1.613993in}}%
\pgfpathcurveto{\pgfqpoint{5.318237in}{1.613993in}}{\pgfqpoint{5.311395in}{1.611159in}}{\pgfqpoint{5.306352in}{1.606116in}}%
\pgfpathcurveto{\pgfqpoint{5.301308in}{1.601072in}}{\pgfqpoint{5.298474in}{1.594230in}}{\pgfqpoint{5.298474in}{1.587097in}}%
\pgfpathcurveto{\pgfqpoint{5.298474in}{1.579965in}}{\pgfqpoint{5.301308in}{1.573123in}}{\pgfqpoint{5.306352in}{1.568079in}}%
\pgfpathcurveto{\pgfqpoint{5.311395in}{1.563036in}}{\pgfqpoint{5.318237in}{1.560202in}}{\pgfqpoint{5.325370in}{1.560202in}}%
\pgfpathclose%
\pgfusepath{stroke,fill}%
\end{pgfscope}%
\begin{pgfscope}%
\pgfpathrectangle{\pgfqpoint{4.985294in}{0.500000in}}{\pgfqpoint{1.764706in}{1.700000in}}%
\pgfusepath{clip}%
\pgfsetbuttcap%
\pgfsetroundjoin%
\definecolor{currentfill}{rgb}{0.979891,0.908948,0.848279}%
\pgfsetfillcolor{currentfill}%
\pgfsetlinewidth{0.311001pt}%
\definecolor{currentstroke}{rgb}{1.000000,1.000000,1.000000}%
\pgfsetstrokecolor{currentstroke}%
\pgfsetdash{}{0pt}%
\pgfpathmoveto{\pgfqpoint{5.419182in}{1.257478in}}%
\pgfpathcurveto{\pgfqpoint{5.426315in}{1.257478in}}{\pgfqpoint{5.433157in}{1.260311in}}{\pgfqpoint{5.438200in}{1.265355in}}%
\pgfpathcurveto{\pgfqpoint{5.443244in}{1.270399in}}{\pgfqpoint{5.446078in}{1.277240in}}{\pgfqpoint{5.446078in}{1.284373in}}%
\pgfpathcurveto{\pgfqpoint{5.446078in}{1.291506in}}{\pgfqpoint{5.443244in}{1.298348in}}{\pgfqpoint{5.438200in}{1.303391in}}%
\pgfpathcurveto{\pgfqpoint{5.433157in}{1.308435in}}{\pgfqpoint{5.426315in}{1.311269in}}{\pgfqpoint{5.419182in}{1.311269in}}%
\pgfpathcurveto{\pgfqpoint{5.412049in}{1.311269in}}{\pgfqpoint{5.405208in}{1.308435in}}{\pgfqpoint{5.400164in}{1.303391in}}%
\pgfpathcurveto{\pgfqpoint{5.395120in}{1.298348in}}{\pgfqpoint{5.392287in}{1.291506in}}{\pgfqpoint{5.392287in}{1.284373in}}%
\pgfpathcurveto{\pgfqpoint{5.392287in}{1.277240in}}{\pgfqpoint{5.395120in}{1.270399in}}{\pgfqpoint{5.400164in}{1.265355in}}%
\pgfpathcurveto{\pgfqpoint{5.405208in}{1.260311in}}{\pgfqpoint{5.412049in}{1.257478in}}{\pgfqpoint{5.419182in}{1.257478in}}%
\pgfpathclose%
\pgfusepath{stroke,fill}%
\end{pgfscope}%
\begin{pgfscope}%
\pgfpathrectangle{\pgfqpoint{4.985294in}{0.500000in}}{\pgfqpoint{1.764706in}{1.700000in}}%
\pgfusepath{clip}%
\pgfsetbuttcap%
\pgfsetroundjoin%
\definecolor{currentfill}{rgb}{0.974412,0.862387,0.780156}%
\pgfsetfillcolor{currentfill}%
\pgfsetlinewidth{0.311001pt}%
\definecolor{currentstroke}{rgb}{1.000000,1.000000,1.000000}%
\pgfsetstrokecolor{currentstroke}%
\pgfsetdash{}{0pt}%
\pgfpathmoveto{\pgfqpoint{6.235881in}{1.549933in}}%
\pgfpathcurveto{\pgfqpoint{6.243014in}{1.549933in}}{\pgfqpoint{6.249856in}{1.552767in}}{\pgfqpoint{6.254899in}{1.557810in}}%
\pgfpathcurveto{\pgfqpoint{6.259943in}{1.562854in}}{\pgfqpoint{6.262777in}{1.569696in}}{\pgfqpoint{6.262777in}{1.576828in}}%
\pgfpathcurveto{\pgfqpoint{6.262777in}{1.583961in}}{\pgfqpoint{6.259943in}{1.590803in}}{\pgfqpoint{6.254899in}{1.595847in}}%
\pgfpathcurveto{\pgfqpoint{6.249856in}{1.600890in}}{\pgfqpoint{6.243014in}{1.603724in}}{\pgfqpoint{6.235881in}{1.603724in}}%
\pgfpathcurveto{\pgfqpoint{6.228748in}{1.603724in}}{\pgfqpoint{6.221907in}{1.600890in}}{\pgfqpoint{6.216863in}{1.595847in}}%
\pgfpathcurveto{\pgfqpoint{6.211819in}{1.590803in}}{\pgfqpoint{6.208985in}{1.583961in}}{\pgfqpoint{6.208985in}{1.576828in}}%
\pgfpathcurveto{\pgfqpoint{6.208985in}{1.569696in}}{\pgfqpoint{6.211819in}{1.562854in}}{\pgfqpoint{6.216863in}{1.557810in}}%
\pgfpathcurveto{\pgfqpoint{6.221907in}{1.552767in}}{\pgfqpoint{6.228748in}{1.549933in}}{\pgfqpoint{6.235881in}{1.549933in}}%
\pgfpathclose%
\pgfusepath{stroke,fill}%
\end{pgfscope}%
\begin{pgfscope}%
\pgfpathrectangle{\pgfqpoint{4.985294in}{0.500000in}}{\pgfqpoint{1.764706in}{1.700000in}}%
\pgfusepath{clip}%
\pgfsetbuttcap%
\pgfsetroundjoin%
\definecolor{currentfill}{rgb}{0.818205,0.120806,0.299261}%
\pgfsetfillcolor{currentfill}%
\pgfsetlinewidth{0.311001pt}%
\definecolor{currentstroke}{rgb}{1.000000,1.000000,1.000000}%
\pgfsetstrokecolor{currentstroke}%
\pgfsetdash{}{0pt}%
\pgfpathmoveto{\pgfqpoint{5.610041in}{1.864441in}}%
\pgfpathcurveto{\pgfqpoint{5.617174in}{1.864441in}}{\pgfqpoint{5.624015in}{1.867274in}}{\pgfqpoint{5.629059in}{1.872318in}}%
\pgfpathcurveto{\pgfqpoint{5.634103in}{1.877362in}}{\pgfqpoint{5.636937in}{1.884203in}}{\pgfqpoint{5.636937in}{1.891336in}}%
\pgfpathcurveto{\pgfqpoint{5.636937in}{1.898469in}}{\pgfqpoint{5.634103in}{1.905311in}}{\pgfqpoint{5.629059in}{1.910354in}}%
\pgfpathcurveto{\pgfqpoint{5.624015in}{1.915398in}}{\pgfqpoint{5.617174in}{1.918232in}}{\pgfqpoint{5.610041in}{1.918232in}}%
\pgfpathcurveto{\pgfqpoint{5.602908in}{1.918232in}}{\pgfqpoint{5.596066in}{1.915398in}}{\pgfqpoint{5.591023in}{1.910354in}}%
\pgfpathcurveto{\pgfqpoint{5.585979in}{1.905311in}}{\pgfqpoint{5.583145in}{1.898469in}}{\pgfqpoint{5.583145in}{1.891336in}}%
\pgfpathcurveto{\pgfqpoint{5.583145in}{1.884203in}}{\pgfqpoint{5.585979in}{1.877362in}}{\pgfqpoint{5.591023in}{1.872318in}}%
\pgfpathcurveto{\pgfqpoint{5.596066in}{1.867274in}}{\pgfqpoint{5.602908in}{1.864441in}}{\pgfqpoint{5.610041in}{1.864441in}}%
\pgfpathclose%
\pgfusepath{stroke,fill}%
\end{pgfscope}%
\begin{pgfscope}%
\pgfpathrectangle{\pgfqpoint{4.985294in}{0.500000in}}{\pgfqpoint{1.764706in}{1.700000in}}%
\pgfusepath{clip}%
\pgfsetbuttcap%
\pgfsetroundjoin%
\definecolor{currentfill}{rgb}{0.971694,0.833208,0.737161}%
\pgfsetfillcolor{currentfill}%
\pgfsetlinewidth{0.311001pt}%
\definecolor{currentstroke}{rgb}{1.000000,1.000000,1.000000}%
\pgfsetstrokecolor{currentstroke}%
\pgfsetdash{}{0pt}%
\pgfpathmoveto{\pgfqpoint{5.506570in}{1.617980in}}%
\pgfpathcurveto{\pgfqpoint{5.513703in}{1.617980in}}{\pgfqpoint{5.520545in}{1.620814in}}{\pgfqpoint{5.525589in}{1.625857in}}%
\pgfpathcurveto{\pgfqpoint{5.530632in}{1.630901in}}{\pgfqpoint{5.533466in}{1.637743in}}{\pgfqpoint{5.533466in}{1.644875in}}%
\pgfpathcurveto{\pgfqpoint{5.533466in}{1.652008in}}{\pgfqpoint{5.530632in}{1.658850in}}{\pgfqpoint{5.525589in}{1.663894in}}%
\pgfpathcurveto{\pgfqpoint{5.520545in}{1.668937in}}{\pgfqpoint{5.513703in}{1.671771in}}{\pgfqpoint{5.506570in}{1.671771in}}%
\pgfpathcurveto{\pgfqpoint{5.499438in}{1.671771in}}{\pgfqpoint{5.492596in}{1.668937in}}{\pgfqpoint{5.487552in}{1.663894in}}%
\pgfpathcurveto{\pgfqpoint{5.482509in}{1.658850in}}{\pgfqpoint{5.479675in}{1.652008in}}{\pgfqpoint{5.479675in}{1.644875in}}%
\pgfpathcurveto{\pgfqpoint{5.479675in}{1.637743in}}{\pgfqpoint{5.482509in}{1.630901in}}{\pgfqpoint{5.487552in}{1.625857in}}%
\pgfpathcurveto{\pgfqpoint{5.492596in}{1.620814in}}{\pgfqpoint{5.499438in}{1.617980in}}{\pgfqpoint{5.506570in}{1.617980in}}%
\pgfpathclose%
\pgfusepath{stroke,fill}%
\end{pgfscope}%
\begin{pgfscope}%
\pgfpathrectangle{\pgfqpoint{4.985294in}{0.500000in}}{\pgfqpoint{1.764706in}{1.700000in}}%
\pgfusepath{clip}%
\pgfsetbuttcap%
\pgfsetroundjoin%
\definecolor{currentfill}{rgb}{0.979891,0.908948,0.848279}%
\pgfsetfillcolor{currentfill}%
\pgfsetlinewidth{0.311001pt}%
\definecolor{currentstroke}{rgb}{1.000000,1.000000,1.000000}%
\pgfsetstrokecolor{currentstroke}%
\pgfsetdash{}{0pt}%
\pgfpathmoveto{\pgfqpoint{6.323673in}{1.446627in}}%
\pgfpathcurveto{\pgfqpoint{6.330806in}{1.446627in}}{\pgfqpoint{6.337647in}{1.449461in}}{\pgfqpoint{6.342691in}{1.454505in}}%
\pgfpathcurveto{\pgfqpoint{6.347735in}{1.459549in}}{\pgfqpoint{6.350568in}{1.466390in}}{\pgfqpoint{6.350568in}{1.473523in}}%
\pgfpathcurveto{\pgfqpoint{6.350568in}{1.480656in}}{\pgfqpoint{6.347735in}{1.487498in}}{\pgfqpoint{6.342691in}{1.492541in}}%
\pgfpathcurveto{\pgfqpoint{6.337647in}{1.497585in}}{\pgfqpoint{6.330806in}{1.500419in}}{\pgfqpoint{6.323673in}{1.500419in}}%
\pgfpathcurveto{\pgfqpoint{6.316540in}{1.500419in}}{\pgfqpoint{6.309698in}{1.497585in}}{\pgfqpoint{6.304655in}{1.492541in}}%
\pgfpathcurveto{\pgfqpoint{6.299611in}{1.487498in}}{\pgfqpoint{6.296777in}{1.480656in}}{\pgfqpoint{6.296777in}{1.473523in}}%
\pgfpathcurveto{\pgfqpoint{6.296777in}{1.466390in}}{\pgfqpoint{6.299611in}{1.459549in}}{\pgfqpoint{6.304655in}{1.454505in}}%
\pgfpathcurveto{\pgfqpoint{6.309698in}{1.449461in}}{\pgfqpoint{6.316540in}{1.446627in}}{\pgfqpoint{6.323673in}{1.446627in}}%
\pgfpathclose%
\pgfusepath{stroke,fill}%
\end{pgfscope}%
\begin{pgfscope}%
\pgfpathrectangle{\pgfqpoint{4.985294in}{0.500000in}}{\pgfqpoint{1.764706in}{1.700000in}}%
\pgfusepath{clip}%
\pgfsetbuttcap%
\pgfsetroundjoin%
\definecolor{currentfill}{rgb}{0.969803,0.809811,0.702523}%
\pgfsetfillcolor{currentfill}%
\pgfsetlinewidth{0.311001pt}%
\definecolor{currentstroke}{rgb}{1.000000,1.000000,1.000000}%
\pgfsetstrokecolor{currentstroke}%
\pgfsetdash{}{0pt}%
\pgfpathmoveto{\pgfqpoint{6.192956in}{1.110402in}}%
\pgfpathcurveto{\pgfqpoint{6.200089in}{1.110402in}}{\pgfqpoint{6.206931in}{1.113236in}}{\pgfqpoint{6.211974in}{1.118279in}}%
\pgfpathcurveto{\pgfqpoint{6.217018in}{1.123323in}}{\pgfqpoint{6.219852in}{1.130164in}}{\pgfqpoint{6.219852in}{1.137297in}}%
\pgfpathcurveto{\pgfqpoint{6.219852in}{1.144430in}}{\pgfqpoint{6.217018in}{1.151272in}}{\pgfqpoint{6.211974in}{1.156315in}}%
\pgfpathcurveto{\pgfqpoint{6.206931in}{1.161359in}}{\pgfqpoint{6.200089in}{1.164193in}}{\pgfqpoint{6.192956in}{1.164193in}}%
\pgfpathcurveto{\pgfqpoint{6.185823in}{1.164193in}}{\pgfqpoint{6.178982in}{1.161359in}}{\pgfqpoint{6.173938in}{1.156315in}}%
\pgfpathcurveto{\pgfqpoint{6.168894in}{1.151272in}}{\pgfqpoint{6.166060in}{1.144430in}}{\pgfqpoint{6.166060in}{1.137297in}}%
\pgfpathcurveto{\pgfqpoint{6.166060in}{1.130164in}}{\pgfqpoint{6.168894in}{1.123323in}}{\pgfqpoint{6.173938in}{1.118279in}}%
\pgfpathcurveto{\pgfqpoint{6.178982in}{1.113236in}}{\pgfqpoint{6.185823in}{1.110402in}}{\pgfqpoint{6.192956in}{1.110402in}}%
\pgfpathclose%
\pgfusepath{stroke,fill}%
\end{pgfscope}%
\begin{pgfscope}%
\pgfpathrectangle{\pgfqpoint{4.985294in}{0.500000in}}{\pgfqpoint{1.764706in}{1.700000in}}%
\pgfusepath{clip}%
\pgfsetbuttcap%
\pgfsetroundjoin%
\definecolor{currentfill}{rgb}{0.977657,0.891500,0.822809}%
\pgfsetfillcolor{currentfill}%
\pgfsetlinewidth{0.311001pt}%
\definecolor{currentstroke}{rgb}{1.000000,1.000000,1.000000}%
\pgfsetstrokecolor{currentstroke}%
\pgfsetdash{}{0pt}%
\pgfpathmoveto{\pgfqpoint{5.451558in}{1.444602in}}%
\pgfpathcurveto{\pgfqpoint{5.458691in}{1.444602in}}{\pgfqpoint{5.465532in}{1.447436in}}{\pgfqpoint{5.470576in}{1.452480in}}%
\pgfpathcurveto{\pgfqpoint{5.475620in}{1.457524in}}{\pgfqpoint{5.478454in}{1.464365in}}{\pgfqpoint{5.478454in}{1.471498in}}%
\pgfpathcurveto{\pgfqpoint{5.478454in}{1.478631in}}{\pgfqpoint{5.475620in}{1.485473in}}{\pgfqpoint{5.470576in}{1.490516in}}%
\pgfpathcurveto{\pgfqpoint{5.465532in}{1.495560in}}{\pgfqpoint{5.458691in}{1.498394in}}{\pgfqpoint{5.451558in}{1.498394in}}%
\pgfpathcurveto{\pgfqpoint{5.444425in}{1.498394in}}{\pgfqpoint{5.437583in}{1.495560in}}{\pgfqpoint{5.432540in}{1.490516in}}%
\pgfpathcurveto{\pgfqpoint{5.427496in}{1.485473in}}{\pgfqpoint{5.424662in}{1.478631in}}{\pgfqpoint{5.424662in}{1.471498in}}%
\pgfpathcurveto{\pgfqpoint{5.424662in}{1.464365in}}{\pgfqpoint{5.427496in}{1.457524in}}{\pgfqpoint{5.432540in}{1.452480in}}%
\pgfpathcurveto{\pgfqpoint{5.437583in}{1.447436in}}{\pgfqpoint{5.444425in}{1.444602in}}{\pgfqpoint{5.451558in}{1.444602in}}%
\pgfpathclose%
\pgfusepath{stroke,fill}%
\end{pgfscope}%
\begin{pgfscope}%
\pgfpathrectangle{\pgfqpoint{4.985294in}{0.500000in}}{\pgfqpoint{1.764706in}{1.700000in}}%
\pgfusepath{clip}%
\pgfsetbuttcap%
\pgfsetroundjoin%
\definecolor{currentfill}{rgb}{0.965753,0.732351,0.592427}%
\pgfsetfillcolor{currentfill}%
\pgfsetlinewidth{0.311001pt}%
\definecolor{currentstroke}{rgb}{1.000000,1.000000,1.000000}%
\pgfsetstrokecolor{currentstroke}%
\pgfsetdash{}{0pt}%
\pgfpathmoveto{\pgfqpoint{6.335867in}{1.592208in}}%
\pgfpathcurveto{\pgfqpoint{6.343000in}{1.592208in}}{\pgfqpoint{6.349841in}{1.595041in}}{\pgfqpoint{6.354885in}{1.600085in}}%
\pgfpathcurveto{\pgfqpoint{6.359929in}{1.605129in}}{\pgfqpoint{6.362762in}{1.611970in}}{\pgfqpoint{6.362762in}{1.619103in}}%
\pgfpathcurveto{\pgfqpoint{6.362762in}{1.626236in}}{\pgfqpoint{6.359929in}{1.633078in}}{\pgfqpoint{6.354885in}{1.638121in}}%
\pgfpathcurveto{\pgfqpoint{6.349841in}{1.643165in}}{\pgfqpoint{6.343000in}{1.645999in}}{\pgfqpoint{6.335867in}{1.645999in}}%
\pgfpathcurveto{\pgfqpoint{6.328734in}{1.645999in}}{\pgfqpoint{6.321892in}{1.643165in}}{\pgfqpoint{6.316849in}{1.638121in}}%
\pgfpathcurveto{\pgfqpoint{6.311805in}{1.633078in}}{\pgfqpoint{6.308971in}{1.626236in}}{\pgfqpoint{6.308971in}{1.619103in}}%
\pgfpathcurveto{\pgfqpoint{6.308971in}{1.611970in}}{\pgfqpoint{6.311805in}{1.605129in}}{\pgfqpoint{6.316849in}{1.600085in}}%
\pgfpathcurveto{\pgfqpoint{6.321892in}{1.595041in}}{\pgfqpoint{6.328734in}{1.592208in}}{\pgfqpoint{6.335867in}{1.592208in}}%
\pgfpathclose%
\pgfusepath{stroke,fill}%
\end{pgfscope}%
\begin{pgfscope}%
\pgfpathrectangle{\pgfqpoint{4.985294in}{0.500000in}}{\pgfqpoint{1.764706in}{1.700000in}}%
\pgfusepath{clip}%
\pgfsetbuttcap%
\pgfsetroundjoin%
\definecolor{currentfill}{rgb}{0.980678,0.914765,0.856766}%
\pgfsetfillcolor{currentfill}%
\pgfsetlinewidth{0.311001pt}%
\definecolor{currentstroke}{rgb}{1.000000,1.000000,1.000000}%
\pgfsetstrokecolor{currentstroke}%
\pgfsetdash{}{0pt}%
\pgfpathmoveto{\pgfqpoint{5.422414in}{1.280189in}}%
\pgfpathcurveto{\pgfqpoint{5.429547in}{1.280189in}}{\pgfqpoint{5.436389in}{1.283023in}}{\pgfqpoint{5.441432in}{1.288067in}}%
\pgfpathcurveto{\pgfqpoint{5.446476in}{1.293111in}}{\pgfqpoint{5.449310in}{1.299952in}}{\pgfqpoint{5.449310in}{1.307085in}}%
\pgfpathcurveto{\pgfqpoint{5.449310in}{1.314218in}}{\pgfqpoint{5.446476in}{1.321060in}}{\pgfqpoint{5.441432in}{1.326103in}}%
\pgfpathcurveto{\pgfqpoint{5.436389in}{1.331147in}}{\pgfqpoint{5.429547in}{1.333981in}}{\pgfqpoint{5.422414in}{1.333981in}}%
\pgfpathcurveto{\pgfqpoint{5.415281in}{1.333981in}}{\pgfqpoint{5.408440in}{1.331147in}}{\pgfqpoint{5.403396in}{1.326103in}}%
\pgfpathcurveto{\pgfqpoint{5.398352in}{1.321060in}}{\pgfqpoint{5.395518in}{1.314218in}}{\pgfqpoint{5.395518in}{1.307085in}}%
\pgfpathcurveto{\pgfqpoint{5.395518in}{1.299952in}}{\pgfqpoint{5.398352in}{1.293111in}}{\pgfqpoint{5.403396in}{1.288067in}}%
\pgfpathcurveto{\pgfqpoint{5.408440in}{1.283023in}}{\pgfqpoint{5.415281in}{1.280189in}}{\pgfqpoint{5.422414in}{1.280189in}}%
\pgfpathclose%
\pgfusepath{stroke,fill}%
\end{pgfscope}%
\begin{pgfscope}%
\pgfpathrectangle{\pgfqpoint{4.985294in}{0.500000in}}{\pgfqpoint{1.764706in}{1.700000in}}%
\pgfusepath{clip}%
\pgfsetbuttcap%
\pgfsetroundjoin%
\definecolor{currentfill}{rgb}{0.968931,0.798091,0.685123}%
\pgfsetfillcolor{currentfill}%
\pgfsetlinewidth{0.311001pt}%
\definecolor{currentstroke}{rgb}{1.000000,1.000000,1.000000}%
\pgfsetstrokecolor{currentstroke}%
\pgfsetdash{}{0pt}%
\pgfpathmoveto{\pgfqpoint{6.376910in}{1.188771in}}%
\pgfpathcurveto{\pgfqpoint{6.384043in}{1.188771in}}{\pgfqpoint{6.390885in}{1.191605in}}{\pgfqpoint{6.395928in}{1.196648in}}%
\pgfpathcurveto{\pgfqpoint{6.400972in}{1.201692in}}{\pgfqpoint{6.403806in}{1.208534in}}{\pgfqpoint{6.403806in}{1.215667in}}%
\pgfpathcurveto{\pgfqpoint{6.403806in}{1.222799in}}{\pgfqpoint{6.400972in}{1.229641in}}{\pgfqpoint{6.395928in}{1.234685in}}%
\pgfpathcurveto{\pgfqpoint{6.390885in}{1.239728in}}{\pgfqpoint{6.384043in}{1.242562in}}{\pgfqpoint{6.376910in}{1.242562in}}%
\pgfpathcurveto{\pgfqpoint{6.369777in}{1.242562in}}{\pgfqpoint{6.362936in}{1.239728in}}{\pgfqpoint{6.357892in}{1.234685in}}%
\pgfpathcurveto{\pgfqpoint{6.352848in}{1.229641in}}{\pgfqpoint{6.350015in}{1.222799in}}{\pgfqpoint{6.350015in}{1.215667in}}%
\pgfpathcurveto{\pgfqpoint{6.350015in}{1.208534in}}{\pgfqpoint{6.352848in}{1.201692in}}{\pgfqpoint{6.357892in}{1.196648in}}%
\pgfpathcurveto{\pgfqpoint{6.362936in}{1.191605in}}{\pgfqpoint{6.369777in}{1.188771in}}{\pgfqpoint{6.376910in}{1.188771in}}%
\pgfpathclose%
\pgfusepath{stroke,fill}%
\end{pgfscope}%
\begin{pgfscope}%
\pgfpathrectangle{\pgfqpoint{4.985294in}{0.500000in}}{\pgfqpoint{1.764706in}{1.700000in}}%
\pgfusepath{clip}%
\pgfsetbuttcap%
\pgfsetroundjoin%
\definecolor{currentfill}{rgb}{0.975018,0.868213,0.788710}%
\pgfsetfillcolor{currentfill}%
\pgfsetlinewidth{0.311001pt}%
\definecolor{currentstroke}{rgb}{1.000000,1.000000,1.000000}%
\pgfsetstrokecolor{currentstroke}%
\pgfsetdash{}{0pt}%
\pgfpathmoveto{\pgfqpoint{6.231273in}{1.602317in}}%
\pgfpathcurveto{\pgfqpoint{6.238406in}{1.602317in}}{\pgfqpoint{6.245248in}{1.605151in}}{\pgfqpoint{6.250292in}{1.610195in}}%
\pgfpathcurveto{\pgfqpoint{6.255335in}{1.615239in}}{\pgfqpoint{6.258169in}{1.622080in}}{\pgfqpoint{6.258169in}{1.629213in}}%
\pgfpathcurveto{\pgfqpoint{6.258169in}{1.636346in}}{\pgfqpoint{6.255335in}{1.643188in}}{\pgfqpoint{6.250292in}{1.648231in}}%
\pgfpathcurveto{\pgfqpoint{6.245248in}{1.653275in}}{\pgfqpoint{6.238406in}{1.656109in}}{\pgfqpoint{6.231273in}{1.656109in}}%
\pgfpathcurveto{\pgfqpoint{6.224141in}{1.656109in}}{\pgfqpoint{6.217299in}{1.653275in}}{\pgfqpoint{6.212255in}{1.648231in}}%
\pgfpathcurveto{\pgfqpoint{6.207212in}{1.643188in}}{\pgfqpoint{6.204378in}{1.636346in}}{\pgfqpoint{6.204378in}{1.629213in}}%
\pgfpathcurveto{\pgfqpoint{6.204378in}{1.622080in}}{\pgfqpoint{6.207212in}{1.615239in}}{\pgfqpoint{6.212255in}{1.610195in}}%
\pgfpathcurveto{\pgfqpoint{6.217299in}{1.605151in}}{\pgfqpoint{6.224141in}{1.602317in}}{\pgfqpoint{6.231273in}{1.602317in}}%
\pgfpathclose%
\pgfusepath{stroke,fill}%
\end{pgfscope}%
\begin{pgfscope}%
\pgfpathrectangle{\pgfqpoint{4.985294in}{0.500000in}}{\pgfqpoint{1.764706in}{1.700000in}}%
\pgfusepath{clip}%
\pgfsetbuttcap%
\pgfsetroundjoin%
\definecolor{currentfill}{rgb}{0.965928,0.738443,0.600540}%
\pgfsetfillcolor{currentfill}%
\pgfsetlinewidth{0.311001pt}%
\definecolor{currentstroke}{rgb}{1.000000,1.000000,1.000000}%
\pgfsetstrokecolor{currentstroke}%
\pgfsetdash{}{0pt}%
\pgfpathmoveto{\pgfqpoint{6.206594in}{1.365906in}}%
\pgfpathcurveto{\pgfqpoint{6.213726in}{1.365906in}}{\pgfqpoint{6.220568in}{1.368740in}}{\pgfqpoint{6.225612in}{1.373784in}}%
\pgfpathcurveto{\pgfqpoint{6.230655in}{1.378827in}}{\pgfqpoint{6.233489in}{1.385669in}}{\pgfqpoint{6.233489in}{1.392802in}}%
\pgfpathcurveto{\pgfqpoint{6.233489in}{1.399935in}}{\pgfqpoint{6.230655in}{1.406776in}}{\pgfqpoint{6.225612in}{1.411820in}}%
\pgfpathcurveto{\pgfqpoint{6.220568in}{1.416864in}}{\pgfqpoint{6.213726in}{1.419698in}}{\pgfqpoint{6.206594in}{1.419698in}}%
\pgfpathcurveto{\pgfqpoint{6.199461in}{1.419698in}}{\pgfqpoint{6.192619in}{1.416864in}}{\pgfqpoint{6.187576in}{1.411820in}}%
\pgfpathcurveto{\pgfqpoint{6.182532in}{1.406776in}}{\pgfqpoint{6.179698in}{1.399935in}}{\pgfqpoint{6.179698in}{1.392802in}}%
\pgfpathcurveto{\pgfqpoint{6.179698in}{1.385669in}}{\pgfqpoint{6.182532in}{1.378827in}}{\pgfqpoint{6.187576in}{1.373784in}}%
\pgfpathcurveto{\pgfqpoint{6.192619in}{1.368740in}}{\pgfqpoint{6.199461in}{1.365906in}}{\pgfqpoint{6.206594in}{1.365906in}}%
\pgfpathclose%
\pgfusepath{stroke,fill}%
\end{pgfscope}%
\begin{pgfscope}%
\pgfpathrectangle{\pgfqpoint{4.985294in}{0.500000in}}{\pgfqpoint{1.764706in}{1.700000in}}%
\pgfusepath{clip}%
\pgfsetbuttcap%
\pgfsetroundjoin%
\definecolor{currentfill}{rgb}{0.976287,0.879862,0.805788}%
\pgfsetfillcolor{currentfill}%
\pgfsetlinewidth{0.311001pt}%
\definecolor{currentstroke}{rgb}{1.000000,1.000000,1.000000}%
\pgfsetstrokecolor{currentstroke}%
\pgfsetdash{}{0pt}%
\pgfpathmoveto{\pgfqpoint{6.277268in}{1.603863in}}%
\pgfpathcurveto{\pgfqpoint{6.284401in}{1.603863in}}{\pgfqpoint{6.291243in}{1.606697in}}{\pgfqpoint{6.296287in}{1.611740in}}%
\pgfpathcurveto{\pgfqpoint{6.301330in}{1.616784in}}{\pgfqpoint{6.304164in}{1.623626in}}{\pgfqpoint{6.304164in}{1.630758in}}%
\pgfpathcurveto{\pgfqpoint{6.304164in}{1.637891in}}{\pgfqpoint{6.301330in}{1.644733in}}{\pgfqpoint{6.296287in}{1.649777in}}%
\pgfpathcurveto{\pgfqpoint{6.291243in}{1.654820in}}{\pgfqpoint{6.284401in}{1.657654in}}{\pgfqpoint{6.277268in}{1.657654in}}%
\pgfpathcurveto{\pgfqpoint{6.270136in}{1.657654in}}{\pgfqpoint{6.263294in}{1.654820in}}{\pgfqpoint{6.258250in}{1.649777in}}%
\pgfpathcurveto{\pgfqpoint{6.253207in}{1.644733in}}{\pgfqpoint{6.250373in}{1.637891in}}{\pgfqpoint{6.250373in}{1.630758in}}%
\pgfpathcurveto{\pgfqpoint{6.250373in}{1.623626in}}{\pgfqpoint{6.253207in}{1.616784in}}{\pgfqpoint{6.258250in}{1.611740in}}%
\pgfpathcurveto{\pgfqpoint{6.263294in}{1.606697in}}{\pgfqpoint{6.270136in}{1.603863in}}{\pgfqpoint{6.277268in}{1.603863in}}%
\pgfpathclose%
\pgfusepath{stroke,fill}%
\end{pgfscope}%
\begin{pgfscope}%
\pgfpathrectangle{\pgfqpoint{4.985294in}{0.500000in}}{\pgfqpoint{1.764706in}{1.700000in}}%
\pgfusepath{clip}%
\pgfsetbuttcap%
\pgfsetroundjoin%
\definecolor{currentfill}{rgb}{0.972201,0.839051,0.745789}%
\pgfsetfillcolor{currentfill}%
\pgfsetlinewidth{0.311001pt}%
\definecolor{currentstroke}{rgb}{1.000000,1.000000,1.000000}%
\pgfsetstrokecolor{currentstroke}%
\pgfsetdash{}{0pt}%
\pgfpathmoveto{\pgfqpoint{6.203056in}{1.057898in}}%
\pgfpathcurveto{\pgfqpoint{6.210188in}{1.057898in}}{\pgfqpoint{6.217030in}{1.060732in}}{\pgfqpoint{6.222074in}{1.065776in}}%
\pgfpathcurveto{\pgfqpoint{6.227117in}{1.070820in}}{\pgfqpoint{6.229951in}{1.077661in}}{\pgfqpoint{6.229951in}{1.084794in}}%
\pgfpathcurveto{\pgfqpoint{6.229951in}{1.091927in}}{\pgfqpoint{6.227117in}{1.098769in}}{\pgfqpoint{6.222074in}{1.103812in}}%
\pgfpathcurveto{\pgfqpoint{6.217030in}{1.108856in}}{\pgfqpoint{6.210188in}{1.111690in}}{\pgfqpoint{6.203056in}{1.111690in}}%
\pgfpathcurveto{\pgfqpoint{6.195923in}{1.111690in}}{\pgfqpoint{6.189081in}{1.108856in}}{\pgfqpoint{6.184037in}{1.103812in}}%
\pgfpathcurveto{\pgfqpoint{6.178994in}{1.098769in}}{\pgfqpoint{6.176160in}{1.091927in}}{\pgfqpoint{6.176160in}{1.084794in}}%
\pgfpathcurveto{\pgfqpoint{6.176160in}{1.077661in}}{\pgfqpoint{6.178994in}{1.070820in}}{\pgfqpoint{6.184037in}{1.065776in}}%
\pgfpathcurveto{\pgfqpoint{6.189081in}{1.060732in}}{\pgfqpoint{6.195923in}{1.057898in}}{\pgfqpoint{6.203056in}{1.057898in}}%
\pgfpathclose%
\pgfusepath{stroke,fill}%
\end{pgfscope}%
\begin{pgfscope}%
\pgfpathrectangle{\pgfqpoint{4.985294in}{0.500000in}}{\pgfqpoint{1.764706in}{1.700000in}}%
\pgfusepath{clip}%
\pgfsetbuttcap%
\pgfsetroundjoin%
\definecolor{currentfill}{rgb}{0.974412,0.862387,0.780156}%
\pgfsetfillcolor{currentfill}%
\pgfsetlinewidth{0.311001pt}%
\definecolor{currentstroke}{rgb}{1.000000,1.000000,1.000000}%
\pgfsetstrokecolor{currentstroke}%
\pgfsetdash{}{0pt}%
\pgfpathmoveto{\pgfqpoint{6.311267in}{1.110224in}}%
\pgfpathcurveto{\pgfqpoint{6.318400in}{1.110224in}}{\pgfqpoint{6.325242in}{1.113058in}}{\pgfqpoint{6.330286in}{1.118102in}}%
\pgfpathcurveto{\pgfqpoint{6.335329in}{1.123145in}}{\pgfqpoint{6.338163in}{1.129987in}}{\pgfqpoint{6.338163in}{1.137120in}}%
\pgfpathcurveto{\pgfqpoint{6.338163in}{1.144252in}}{\pgfqpoint{6.335329in}{1.151094in}}{\pgfqpoint{6.330286in}{1.156138in}}%
\pgfpathcurveto{\pgfqpoint{6.325242in}{1.161181in}}{\pgfqpoint{6.318400in}{1.164015in}}{\pgfqpoint{6.311267in}{1.164015in}}%
\pgfpathcurveto{\pgfqpoint{6.304135in}{1.164015in}}{\pgfqpoint{6.297293in}{1.161181in}}{\pgfqpoint{6.292249in}{1.156138in}}%
\pgfpathcurveto{\pgfqpoint{6.287206in}{1.151094in}}{\pgfqpoint{6.284372in}{1.144252in}}{\pgfqpoint{6.284372in}{1.137120in}}%
\pgfpathcurveto{\pgfqpoint{6.284372in}{1.129987in}}{\pgfqpoint{6.287206in}{1.123145in}}{\pgfqpoint{6.292249in}{1.118102in}}%
\pgfpathcurveto{\pgfqpoint{6.297293in}{1.113058in}}{\pgfqpoint{6.304135in}{1.110224in}}{\pgfqpoint{6.311267in}{1.110224in}}%
\pgfpathclose%
\pgfusepath{stroke,fill}%
\end{pgfscope}%
\begin{pgfscope}%
\pgfpathrectangle{\pgfqpoint{4.985294in}{0.500000in}}{\pgfqpoint{1.764706in}{1.700000in}}%
\pgfusepath{clip}%
\pgfsetbuttcap%
\pgfsetroundjoin%
\definecolor{currentfill}{rgb}{0.960043,0.546576,0.387029}%
\pgfsetfillcolor{currentfill}%
\pgfsetlinewidth{0.311001pt}%
\definecolor{currentstroke}{rgb}{1.000000,1.000000,1.000000}%
\pgfsetstrokecolor{currentstroke}%
\pgfsetdash{}{0pt}%
\pgfpathmoveto{\pgfqpoint{5.603259in}{1.763072in}}%
\pgfpathcurveto{\pgfqpoint{5.610392in}{1.763072in}}{\pgfqpoint{5.617234in}{1.765906in}}{\pgfqpoint{5.622277in}{1.770950in}}%
\pgfpathcurveto{\pgfqpoint{5.627321in}{1.775994in}}{\pgfqpoint{5.630155in}{1.782835in}}{\pgfqpoint{5.630155in}{1.789968in}}%
\pgfpathcurveto{\pgfqpoint{5.630155in}{1.797101in}}{\pgfqpoint{5.627321in}{1.803943in}}{\pgfqpoint{5.622277in}{1.808986in}}%
\pgfpathcurveto{\pgfqpoint{5.617234in}{1.814030in}}{\pgfqpoint{5.610392in}{1.816864in}}{\pgfqpoint{5.603259in}{1.816864in}}%
\pgfpathcurveto{\pgfqpoint{5.596126in}{1.816864in}}{\pgfqpoint{5.589285in}{1.814030in}}{\pgfqpoint{5.584241in}{1.808986in}}%
\pgfpathcurveto{\pgfqpoint{5.579197in}{1.803943in}}{\pgfqpoint{5.576364in}{1.797101in}}{\pgfqpoint{5.576364in}{1.789968in}}%
\pgfpathcurveto{\pgfqpoint{5.576364in}{1.782835in}}{\pgfqpoint{5.579197in}{1.775994in}}{\pgfqpoint{5.584241in}{1.770950in}}%
\pgfpathcurveto{\pgfqpoint{5.589285in}{1.765906in}}{\pgfqpoint{5.596126in}{1.763072in}}{\pgfqpoint{5.603259in}{1.763072in}}%
\pgfpathclose%
\pgfusepath{stroke,fill}%
\end{pgfscope}%
\begin{pgfscope}%
\pgfpathrectangle{\pgfqpoint{4.985294in}{0.500000in}}{\pgfqpoint{1.764706in}{1.700000in}}%
\pgfusepath{clip}%
\pgfsetbuttcap%
\pgfsetroundjoin%
\definecolor{currentfill}{rgb}{0.971694,0.833208,0.737161}%
\pgfsetfillcolor{currentfill}%
\pgfsetlinewidth{0.311001pt}%
\definecolor{currentstroke}{rgb}{1.000000,1.000000,1.000000}%
\pgfsetstrokecolor{currentstroke}%
\pgfsetdash{}{0pt}%
\pgfpathmoveto{\pgfqpoint{6.352033in}{1.151797in}}%
\pgfpathcurveto{\pgfqpoint{6.359166in}{1.151797in}}{\pgfqpoint{6.366007in}{1.154631in}}{\pgfqpoint{6.371051in}{1.159675in}}%
\pgfpathcurveto{\pgfqpoint{6.376095in}{1.164718in}}{\pgfqpoint{6.378929in}{1.171560in}}{\pgfqpoint{6.378929in}{1.178693in}}%
\pgfpathcurveto{\pgfqpoint{6.378929in}{1.185826in}}{\pgfqpoint{6.376095in}{1.192667in}}{\pgfqpoint{6.371051in}{1.197711in}}%
\pgfpathcurveto{\pgfqpoint{6.366007in}{1.202755in}}{\pgfqpoint{6.359166in}{1.205589in}}{\pgfqpoint{6.352033in}{1.205589in}}%
\pgfpathcurveto{\pgfqpoint{6.344900in}{1.205589in}}{\pgfqpoint{6.338058in}{1.202755in}}{\pgfqpoint{6.333015in}{1.197711in}}%
\pgfpathcurveto{\pgfqpoint{6.327971in}{1.192667in}}{\pgfqpoint{6.325137in}{1.185826in}}{\pgfqpoint{6.325137in}{1.178693in}}%
\pgfpathcurveto{\pgfqpoint{6.325137in}{1.171560in}}{\pgfqpoint{6.327971in}{1.164718in}}{\pgfqpoint{6.333015in}{1.159675in}}%
\pgfpathcurveto{\pgfqpoint{6.338058in}{1.154631in}}{\pgfqpoint{6.344900in}{1.151797in}}{\pgfqpoint{6.352033in}{1.151797in}}%
\pgfpathclose%
\pgfusepath{stroke,fill}%
\end{pgfscope}%
\begin{pgfscope}%
\pgfpathrectangle{\pgfqpoint{4.985294in}{0.500000in}}{\pgfqpoint{1.764706in}{1.700000in}}%
\pgfusepath{clip}%
\pgfsetbuttcap%
\pgfsetroundjoin%
\definecolor{currentfill}{rgb}{0.941676,0.367866,0.260395}%
\pgfsetfillcolor{currentfill}%
\pgfsetlinewidth{0.311001pt}%
\definecolor{currentstroke}{rgb}{1.000000,1.000000,1.000000}%
\pgfsetstrokecolor{currentstroke}%
\pgfsetdash{}{0pt}%
\pgfpathmoveto{\pgfqpoint{6.379480in}{1.678910in}}%
\pgfpathcurveto{\pgfqpoint{6.386613in}{1.678910in}}{\pgfqpoint{6.393455in}{1.681744in}}{\pgfqpoint{6.398498in}{1.686787in}}%
\pgfpathcurveto{\pgfqpoint{6.403542in}{1.691831in}}{\pgfqpoint{6.406376in}{1.698673in}}{\pgfqpoint{6.406376in}{1.705805in}}%
\pgfpathcurveto{\pgfqpoint{6.406376in}{1.712938in}}{\pgfqpoint{6.403542in}{1.719780in}}{\pgfqpoint{6.398498in}{1.724824in}}%
\pgfpathcurveto{\pgfqpoint{6.393455in}{1.729867in}}{\pgfqpoint{6.386613in}{1.732701in}}{\pgfqpoint{6.379480in}{1.732701in}}%
\pgfpathcurveto{\pgfqpoint{6.372347in}{1.732701in}}{\pgfqpoint{6.365506in}{1.729867in}}{\pgfqpoint{6.360462in}{1.724824in}}%
\pgfpathcurveto{\pgfqpoint{6.355418in}{1.719780in}}{\pgfqpoint{6.352584in}{1.712938in}}{\pgfqpoint{6.352584in}{1.705805in}}%
\pgfpathcurveto{\pgfqpoint{6.352584in}{1.698673in}}{\pgfqpoint{6.355418in}{1.691831in}}{\pgfqpoint{6.360462in}{1.686787in}}%
\pgfpathcurveto{\pgfqpoint{6.365506in}{1.681744in}}{\pgfqpoint{6.372347in}{1.678910in}}{\pgfqpoint{6.379480in}{1.678910in}}%
\pgfpathclose%
\pgfusepath{stroke,fill}%
\end{pgfscope}%
\begin{pgfscope}%
\pgfpathrectangle{\pgfqpoint{4.985294in}{0.500000in}}{\pgfqpoint{1.764706in}{1.700000in}}%
\pgfusepath{clip}%
\pgfsetbuttcap%
\pgfsetroundjoin%
\definecolor{currentfill}{rgb}{0.911533,0.252926,0.244703}%
\pgfsetfillcolor{currentfill}%
\pgfsetlinewidth{0.311001pt}%
\definecolor{currentstroke}{rgb}{1.000000,1.000000,1.000000}%
\pgfsetstrokecolor{currentstroke}%
\pgfsetdash{}{0pt}%
\pgfpathmoveto{\pgfqpoint{6.401496in}{1.679290in}}%
\pgfpathcurveto{\pgfqpoint{6.408629in}{1.679290in}}{\pgfqpoint{6.415471in}{1.682124in}}{\pgfqpoint{6.420515in}{1.687168in}}%
\pgfpathcurveto{\pgfqpoint{6.425558in}{1.692211in}}{\pgfqpoint{6.428392in}{1.699053in}}{\pgfqpoint{6.428392in}{1.706186in}}%
\pgfpathcurveto{\pgfqpoint{6.428392in}{1.713318in}}{\pgfqpoint{6.425558in}{1.720160in}}{\pgfqpoint{6.420515in}{1.725204in}}%
\pgfpathcurveto{\pgfqpoint{6.415471in}{1.730247in}}{\pgfqpoint{6.408629in}{1.733081in}}{\pgfqpoint{6.401496in}{1.733081in}}%
\pgfpathcurveto{\pgfqpoint{6.394364in}{1.733081in}}{\pgfqpoint{6.387522in}{1.730247in}}{\pgfqpoint{6.382478in}{1.725204in}}%
\pgfpathcurveto{\pgfqpoint{6.377435in}{1.720160in}}{\pgfqpoint{6.374601in}{1.713318in}}{\pgfqpoint{6.374601in}{1.706186in}}%
\pgfpathcurveto{\pgfqpoint{6.374601in}{1.699053in}}{\pgfqpoint{6.377435in}{1.692211in}}{\pgfqpoint{6.382478in}{1.687168in}}%
\pgfpathcurveto{\pgfqpoint{6.387522in}{1.682124in}}{\pgfqpoint{6.394364in}{1.679290in}}{\pgfqpoint{6.401496in}{1.679290in}}%
\pgfpathclose%
\pgfusepath{stroke,fill}%
\end{pgfscope}%
\begin{pgfscope}%
\pgfpathrectangle{\pgfqpoint{4.985294in}{0.500000in}}{\pgfqpoint{1.764706in}{1.700000in}}%
\pgfusepath{clip}%
\pgfsetbuttcap%
\pgfsetroundjoin%
\definecolor{currentfill}{rgb}{0.018319,0.022977,0.107385}%
\pgfsetfillcolor{currentfill}%
\pgfsetlinewidth{0.311001pt}%
\definecolor{currentstroke}{rgb}{1.000000,1.000000,1.000000}%
\pgfsetstrokecolor{currentstroke}%
\pgfsetdash{}{0pt}%
\pgfpathmoveto{\pgfqpoint{5.612272in}{0.609349in}}%
\pgfpathcurveto{\pgfqpoint{5.619405in}{0.609349in}}{\pgfqpoint{5.626246in}{0.612183in}}{\pgfqpoint{5.631290in}{0.617226in}}%
\pgfpathcurveto{\pgfqpoint{5.636334in}{0.622270in}}{\pgfqpoint{5.639168in}{0.629112in}}{\pgfqpoint{5.639168in}{0.636244in}}%
\pgfpathcurveto{\pgfqpoint{5.639168in}{0.643377in}}{\pgfqpoint{5.636334in}{0.650219in}}{\pgfqpoint{5.631290in}{0.655263in}}%
\pgfpathcurveto{\pgfqpoint{5.626246in}{0.660306in}}{\pgfqpoint{5.619405in}{0.663140in}}{\pgfqpoint{5.612272in}{0.663140in}}%
\pgfpathcurveto{\pgfqpoint{5.605139in}{0.663140in}}{\pgfqpoint{5.598297in}{0.660306in}}{\pgfqpoint{5.593254in}{0.655263in}}%
\pgfpathcurveto{\pgfqpoint{5.588210in}{0.650219in}}{\pgfqpoint{5.585376in}{0.643377in}}{\pgfqpoint{5.585376in}{0.636244in}}%
\pgfpathcurveto{\pgfqpoint{5.585376in}{0.629112in}}{\pgfqpoint{5.588210in}{0.622270in}}{\pgfqpoint{5.593254in}{0.617226in}}%
\pgfpathcurveto{\pgfqpoint{5.598297in}{0.612183in}}{\pgfqpoint{5.605139in}{0.609349in}}{\pgfqpoint{5.612272in}{0.609349in}}%
\pgfpathclose%
\pgfusepath{stroke,fill}%
\end{pgfscope}%
\begin{pgfscope}%
\pgfpathrectangle{\pgfqpoint{4.985294in}{0.500000in}}{\pgfqpoint{1.764706in}{1.700000in}}%
\pgfusepath{clip}%
\pgfsetbuttcap%
\pgfsetroundjoin%
\definecolor{currentfill}{rgb}{0.964920,0.695342,0.545192}%
\pgfsetfillcolor{currentfill}%
\pgfsetlinewidth{0.311001pt}%
\definecolor{currentstroke}{rgb}{1.000000,1.000000,1.000000}%
\pgfsetstrokecolor{currentstroke}%
\pgfsetdash{}{0pt}%
\pgfpathmoveto{\pgfqpoint{5.604069in}{1.019714in}}%
\pgfpathcurveto{\pgfqpoint{5.611202in}{1.019714in}}{\pgfqpoint{5.618044in}{1.022547in}}{\pgfqpoint{5.623087in}{1.027591in}}%
\pgfpathcurveto{\pgfqpoint{5.628131in}{1.032635in}}{\pgfqpoint{5.630965in}{1.039476in}}{\pgfqpoint{5.630965in}{1.046609in}}%
\pgfpathcurveto{\pgfqpoint{5.630965in}{1.053742in}}{\pgfqpoint{5.628131in}{1.060584in}}{\pgfqpoint{5.623087in}{1.065627in}}%
\pgfpathcurveto{\pgfqpoint{5.618044in}{1.070671in}}{\pgfqpoint{5.611202in}{1.073505in}}{\pgfqpoint{5.604069in}{1.073505in}}%
\pgfpathcurveto{\pgfqpoint{5.596936in}{1.073505in}}{\pgfqpoint{5.590095in}{1.070671in}}{\pgfqpoint{5.585051in}{1.065627in}}%
\pgfpathcurveto{\pgfqpoint{5.580007in}{1.060584in}}{\pgfqpoint{5.577174in}{1.053742in}}{\pgfqpoint{5.577174in}{1.046609in}}%
\pgfpathcurveto{\pgfqpoint{5.577174in}{1.039476in}}{\pgfqpoint{5.580007in}{1.032635in}}{\pgfqpoint{5.585051in}{1.027591in}}%
\pgfpathcurveto{\pgfqpoint{5.590095in}{1.022547in}}{\pgfqpoint{5.596936in}{1.019714in}}{\pgfqpoint{5.604069in}{1.019714in}}%
\pgfpathclose%
\pgfusepath{stroke,fill}%
\end{pgfscope}%
\begin{pgfscope}%
\pgfpathrectangle{\pgfqpoint{4.985294in}{0.500000in}}{\pgfqpoint{1.764706in}{1.700000in}}%
\pgfusepath{clip}%
\pgfsetbuttcap%
\pgfsetroundjoin%
\definecolor{currentfill}{rgb}{0.981377,0.920617,0.865369}%
\pgfsetfillcolor{currentfill}%
\pgfsetlinewidth{0.311001pt}%
\definecolor{currentstroke}{rgb}{1.000000,1.000000,1.000000}%
\pgfsetstrokecolor{currentstroke}%
\pgfsetdash{}{0pt}%
\pgfpathmoveto{\pgfqpoint{6.323015in}{1.337147in}}%
\pgfpathcurveto{\pgfqpoint{6.330148in}{1.337147in}}{\pgfqpoint{6.336990in}{1.339981in}}{\pgfqpoint{6.342034in}{1.345024in}}%
\pgfpathcurveto{\pgfqpoint{6.347077in}{1.350068in}}{\pgfqpoint{6.349911in}{1.356910in}}{\pgfqpoint{6.349911in}{1.364043in}}%
\pgfpathcurveto{\pgfqpoint{6.349911in}{1.371175in}}{\pgfqpoint{6.347077in}{1.378017in}}{\pgfqpoint{6.342034in}{1.383061in}}%
\pgfpathcurveto{\pgfqpoint{6.336990in}{1.388104in}}{\pgfqpoint{6.330148in}{1.390938in}}{\pgfqpoint{6.323015in}{1.390938in}}%
\pgfpathcurveto{\pgfqpoint{6.315883in}{1.390938in}}{\pgfqpoint{6.309041in}{1.388104in}}{\pgfqpoint{6.303997in}{1.383061in}}%
\pgfpathcurveto{\pgfqpoint{6.298954in}{1.378017in}}{\pgfqpoint{6.296120in}{1.371175in}}{\pgfqpoint{6.296120in}{1.364043in}}%
\pgfpathcurveto{\pgfqpoint{6.296120in}{1.356910in}}{\pgfqpoint{6.298954in}{1.350068in}}{\pgfqpoint{6.303997in}{1.345024in}}%
\pgfpathcurveto{\pgfqpoint{6.309041in}{1.339981in}}{\pgfqpoint{6.315883in}{1.337147in}}{\pgfqpoint{6.323015in}{1.337147in}}%
\pgfpathclose%
\pgfusepath{stroke,fill}%
\end{pgfscope}%
\begin{pgfscope}%
\pgfpathrectangle{\pgfqpoint{4.985294in}{0.500000in}}{\pgfqpoint{1.764706in}{1.700000in}}%
\pgfusepath{clip}%
\pgfsetbuttcap%
\pgfsetroundjoin%
\definecolor{currentfill}{rgb}{0.922239,0.282873,0.242296}%
\pgfsetfillcolor{currentfill}%
\pgfsetlinewidth{0.311001pt}%
\definecolor{currentstroke}{rgb}{1.000000,1.000000,1.000000}%
\pgfsetstrokecolor{currentstroke}%
\pgfsetdash{}{0pt}%
\pgfpathmoveto{\pgfqpoint{5.537139in}{1.828424in}}%
\pgfpathcurveto{\pgfqpoint{5.544272in}{1.828424in}}{\pgfqpoint{5.551113in}{1.831258in}}{\pgfqpoint{5.556157in}{1.836302in}}%
\pgfpathcurveto{\pgfqpoint{5.561201in}{1.841346in}}{\pgfqpoint{5.564035in}{1.848187in}}{\pgfqpoint{5.564035in}{1.855320in}}%
\pgfpathcurveto{\pgfqpoint{5.564035in}{1.862453in}}{\pgfqpoint{5.561201in}{1.869295in}}{\pgfqpoint{5.556157in}{1.874338in}}%
\pgfpathcurveto{\pgfqpoint{5.551113in}{1.879382in}}{\pgfqpoint{5.544272in}{1.882216in}}{\pgfqpoint{5.537139in}{1.882216in}}%
\pgfpathcurveto{\pgfqpoint{5.530006in}{1.882216in}}{\pgfqpoint{5.523164in}{1.879382in}}{\pgfqpoint{5.518121in}{1.874338in}}%
\pgfpathcurveto{\pgfqpoint{5.513077in}{1.869295in}}{\pgfqpoint{5.510243in}{1.862453in}}{\pgfqpoint{5.510243in}{1.855320in}}%
\pgfpathcurveto{\pgfqpoint{5.510243in}{1.848187in}}{\pgfqpoint{5.513077in}{1.841346in}}{\pgfqpoint{5.518121in}{1.836302in}}%
\pgfpathcurveto{\pgfqpoint{5.523164in}{1.831258in}}{\pgfqpoint{5.530006in}{1.828424in}}{\pgfqpoint{5.537139in}{1.828424in}}%
\pgfpathclose%
\pgfusepath{stroke,fill}%
\end{pgfscope}%
\begin{pgfscope}%
\pgfpathrectangle{\pgfqpoint{4.985294in}{0.500000in}}{\pgfqpoint{1.764706in}{1.700000in}}%
\pgfusepath{clip}%
\pgfsetbuttcap%
\pgfsetroundjoin%
\definecolor{currentfill}{rgb}{0.965753,0.732351,0.592427}%
\pgfsetfillcolor{currentfill}%
\pgfsetlinewidth{0.311001pt}%
\definecolor{currentstroke}{rgb}{1.000000,1.000000,1.000000}%
\pgfsetstrokecolor{currentstroke}%
\pgfsetdash{}{0pt}%
\pgfpathmoveto{\pgfqpoint{5.583680in}{1.034765in}}%
\pgfpathcurveto{\pgfqpoint{5.590813in}{1.034765in}}{\pgfqpoint{5.597655in}{1.037598in}}{\pgfqpoint{5.602698in}{1.042642in}}%
\pgfpathcurveto{\pgfqpoint{5.607742in}{1.047686in}}{\pgfqpoint{5.610576in}{1.054527in}}{\pgfqpoint{5.610576in}{1.061660in}}%
\pgfpathcurveto{\pgfqpoint{5.610576in}{1.068793in}}{\pgfqpoint{5.607742in}{1.075635in}}{\pgfqpoint{5.602698in}{1.080678in}}%
\pgfpathcurveto{\pgfqpoint{5.597655in}{1.085722in}}{\pgfqpoint{5.590813in}{1.088556in}}{\pgfqpoint{5.583680in}{1.088556in}}%
\pgfpathcurveto{\pgfqpoint{5.576547in}{1.088556in}}{\pgfqpoint{5.569706in}{1.085722in}}{\pgfqpoint{5.564662in}{1.080678in}}%
\pgfpathcurveto{\pgfqpoint{5.559618in}{1.075635in}}{\pgfqpoint{5.556785in}{1.068793in}}{\pgfqpoint{5.556785in}{1.061660in}}%
\pgfpathcurveto{\pgfqpoint{5.556785in}{1.054527in}}{\pgfqpoint{5.559618in}{1.047686in}}{\pgfqpoint{5.564662in}{1.042642in}}%
\pgfpathcurveto{\pgfqpoint{5.569706in}{1.037598in}}{\pgfqpoint{5.576547in}{1.034765in}}{\pgfqpoint{5.583680in}{1.034765in}}%
\pgfpathclose%
\pgfusepath{stroke,fill}%
\end{pgfscope}%
\begin{pgfscope}%
\pgfpathrectangle{\pgfqpoint{4.985294in}{0.500000in}}{\pgfqpoint{1.764706in}{1.700000in}}%
\pgfusepath{clip}%
\pgfsetbuttcap%
\pgfsetroundjoin%
\definecolor{currentfill}{rgb}{0.963190,0.619109,0.458249}%
\pgfsetfillcolor{currentfill}%
\pgfsetlinewidth{0.311001pt}%
\definecolor{currentstroke}{rgb}{1.000000,1.000000,1.000000}%
\pgfsetstrokecolor{currentstroke}%
\pgfsetdash{}{0pt}%
\pgfpathmoveto{\pgfqpoint{6.409175in}{1.160139in}}%
\pgfpathcurveto{\pgfqpoint{6.416307in}{1.160139in}}{\pgfqpoint{6.423149in}{1.162973in}}{\pgfqpoint{6.428193in}{1.168017in}}%
\pgfpathcurveto{\pgfqpoint{6.433236in}{1.173061in}}{\pgfqpoint{6.436070in}{1.179902in}}{\pgfqpoint{6.436070in}{1.187035in}}%
\pgfpathcurveto{\pgfqpoint{6.436070in}{1.194168in}}{\pgfqpoint{6.433236in}{1.201010in}}{\pgfqpoint{6.428193in}{1.206053in}}%
\pgfpathcurveto{\pgfqpoint{6.423149in}{1.211097in}}{\pgfqpoint{6.416307in}{1.213931in}}{\pgfqpoint{6.409175in}{1.213931in}}%
\pgfpathcurveto{\pgfqpoint{6.402042in}{1.213931in}}{\pgfqpoint{6.395200in}{1.211097in}}{\pgfqpoint{6.390156in}{1.206053in}}%
\pgfpathcurveto{\pgfqpoint{6.385113in}{1.201010in}}{\pgfqpoint{6.382279in}{1.194168in}}{\pgfqpoint{6.382279in}{1.187035in}}%
\pgfpathcurveto{\pgfqpoint{6.382279in}{1.179902in}}{\pgfqpoint{6.385113in}{1.173061in}}{\pgfqpoint{6.390156in}{1.168017in}}%
\pgfpathcurveto{\pgfqpoint{6.395200in}{1.162973in}}{\pgfqpoint{6.402042in}{1.160139in}}{\pgfqpoint{6.409175in}{1.160139in}}%
\pgfpathclose%
\pgfusepath{stroke,fill}%
\end{pgfscope}%
\begin{pgfscope}%
\pgfpathrectangle{\pgfqpoint{4.985294in}{0.500000in}}{\pgfqpoint{1.764706in}{1.700000in}}%
\pgfusepath{clip}%
\pgfsetbuttcap%
\pgfsetroundjoin%
\definecolor{currentfill}{rgb}{0.973832,0.856556,0.771584}%
\pgfsetfillcolor{currentfill}%
\pgfsetlinewidth{0.311001pt}%
\definecolor{currentstroke}{rgb}{1.000000,1.000000,1.000000}%
\pgfsetstrokecolor{currentstroke}%
\pgfsetdash{}{0pt}%
\pgfpathmoveto{\pgfqpoint{6.251085in}{1.235284in}}%
\pgfpathcurveto{\pgfqpoint{6.258218in}{1.235284in}}{\pgfqpoint{6.265059in}{1.238117in}}{\pgfqpoint{6.270103in}{1.243161in}}%
\pgfpathcurveto{\pgfqpoint{6.275147in}{1.248205in}}{\pgfqpoint{6.277981in}{1.255046in}}{\pgfqpoint{6.277981in}{1.262179in}}%
\pgfpathcurveto{\pgfqpoint{6.277981in}{1.269312in}}{\pgfqpoint{6.275147in}{1.276154in}}{\pgfqpoint{6.270103in}{1.281197in}}%
\pgfpathcurveto{\pgfqpoint{6.265059in}{1.286241in}}{\pgfqpoint{6.258218in}{1.289075in}}{\pgfqpoint{6.251085in}{1.289075in}}%
\pgfpathcurveto{\pgfqpoint{6.243952in}{1.289075in}}{\pgfqpoint{6.237111in}{1.286241in}}{\pgfqpoint{6.232067in}{1.281197in}}%
\pgfpathcurveto{\pgfqpoint{6.227023in}{1.276154in}}{\pgfqpoint{6.224189in}{1.269312in}}{\pgfqpoint{6.224189in}{1.262179in}}%
\pgfpathcurveto{\pgfqpoint{6.224189in}{1.255046in}}{\pgfqpoint{6.227023in}{1.248205in}}{\pgfqpoint{6.232067in}{1.243161in}}%
\pgfpathcurveto{\pgfqpoint{6.237111in}{1.238117in}}{\pgfqpoint{6.243952in}{1.235284in}}{\pgfqpoint{6.251085in}{1.235284in}}%
\pgfpathclose%
\pgfusepath{stroke,fill}%
\end{pgfscope}%
\begin{pgfscope}%
\pgfpathrectangle{\pgfqpoint{4.985294in}{0.500000in}}{\pgfqpoint{1.764706in}{1.700000in}}%
\pgfusepath{clip}%
\pgfsetbuttcap%
\pgfsetroundjoin%
\definecolor{currentfill}{rgb}{0.965753,0.732351,0.592427}%
\pgfsetfillcolor{currentfill}%
\pgfsetlinewidth{0.311001pt}%
\definecolor{currentstroke}{rgb}{1.000000,1.000000,1.000000}%
\pgfsetstrokecolor{currentstroke}%
\pgfsetdash{}{0pt}%
\pgfpathmoveto{\pgfqpoint{6.203202in}{1.395787in}}%
\pgfpathcurveto{\pgfqpoint{6.210334in}{1.395787in}}{\pgfqpoint{6.217176in}{1.398621in}}{\pgfqpoint{6.222220in}{1.403665in}}%
\pgfpathcurveto{\pgfqpoint{6.227263in}{1.408708in}}{\pgfqpoint{6.230097in}{1.415550in}}{\pgfqpoint{6.230097in}{1.422683in}}%
\pgfpathcurveto{\pgfqpoint{6.230097in}{1.429816in}}{\pgfqpoint{6.227263in}{1.436657in}}{\pgfqpoint{6.222220in}{1.441701in}}%
\pgfpathcurveto{\pgfqpoint{6.217176in}{1.446745in}}{\pgfqpoint{6.210334in}{1.449578in}}{\pgfqpoint{6.203202in}{1.449578in}}%
\pgfpathcurveto{\pgfqpoint{6.196069in}{1.449578in}}{\pgfqpoint{6.189227in}{1.446745in}}{\pgfqpoint{6.184183in}{1.441701in}}%
\pgfpathcurveto{\pgfqpoint{6.179140in}{1.436657in}}{\pgfqpoint{6.176306in}{1.429816in}}{\pgfqpoint{6.176306in}{1.422683in}}%
\pgfpathcurveto{\pgfqpoint{6.176306in}{1.415550in}}{\pgfqpoint{6.179140in}{1.408708in}}{\pgfqpoint{6.184183in}{1.403665in}}%
\pgfpathcurveto{\pgfqpoint{6.189227in}{1.398621in}}{\pgfqpoint{6.196069in}{1.395787in}}{\pgfqpoint{6.203202in}{1.395787in}}%
\pgfpathclose%
\pgfusepath{stroke,fill}%
\end{pgfscope}%
\begin{pgfscope}%
\pgfpathrectangle{\pgfqpoint{4.985294in}{0.500000in}}{\pgfqpoint{1.764706in}{1.700000in}}%
\pgfusepath{clip}%
\pgfsetbuttcap%
\pgfsetroundjoin%
\definecolor{currentfill}{rgb}{0.979891,0.908948,0.848279}%
\pgfsetfillcolor{currentfill}%
\pgfsetlinewidth{0.311001pt}%
\definecolor{currentstroke}{rgb}{1.000000,1.000000,1.000000}%
\pgfsetstrokecolor{currentstroke}%
\pgfsetdash{}{0pt}%
\pgfpathmoveto{\pgfqpoint{5.407379in}{1.371429in}}%
\pgfpathcurveto{\pgfqpoint{5.414512in}{1.371429in}}{\pgfqpoint{5.421354in}{1.374263in}}{\pgfqpoint{5.426397in}{1.379306in}}%
\pgfpathcurveto{\pgfqpoint{5.431441in}{1.384350in}}{\pgfqpoint{5.434275in}{1.391192in}}{\pgfqpoint{5.434275in}{1.398324in}}%
\pgfpathcurveto{\pgfqpoint{5.434275in}{1.405457in}}{\pgfqpoint{5.431441in}{1.412299in}}{\pgfqpoint{5.426397in}{1.417343in}}%
\pgfpathcurveto{\pgfqpoint{5.421354in}{1.422386in}}{\pgfqpoint{5.414512in}{1.425220in}}{\pgfqpoint{5.407379in}{1.425220in}}%
\pgfpathcurveto{\pgfqpoint{5.400246in}{1.425220in}}{\pgfqpoint{5.393405in}{1.422386in}}{\pgfqpoint{5.388361in}{1.417343in}}%
\pgfpathcurveto{\pgfqpoint{5.383317in}{1.412299in}}{\pgfqpoint{5.380483in}{1.405457in}}{\pgfqpoint{5.380483in}{1.398324in}}%
\pgfpathcurveto{\pgfqpoint{5.380483in}{1.391192in}}{\pgfqpoint{5.383317in}{1.384350in}}{\pgfqpoint{5.388361in}{1.379306in}}%
\pgfpathcurveto{\pgfqpoint{5.393405in}{1.374263in}}{\pgfqpoint{5.400246in}{1.371429in}}{\pgfqpoint{5.407379in}{1.371429in}}%
\pgfpathclose%
\pgfusepath{stroke,fill}%
\end{pgfscope}%
\begin{pgfscope}%
\pgfpathrectangle{\pgfqpoint{4.985294in}{0.500000in}}{\pgfqpoint{1.764706in}{1.700000in}}%
\pgfusepath{clip}%
\pgfsetbuttcap%
\pgfsetroundjoin%
\definecolor{currentfill}{rgb}{0.965928,0.738443,0.600540}%
\pgfsetfillcolor{currentfill}%
\pgfsetlinewidth{0.311001pt}%
\definecolor{currentstroke}{rgb}{1.000000,1.000000,1.000000}%
\pgfsetstrokecolor{currentstroke}%
\pgfsetdash{}{0pt}%
\pgfpathmoveto{\pgfqpoint{6.168304in}{1.541686in}}%
\pgfpathcurveto{\pgfqpoint{6.175437in}{1.541686in}}{\pgfqpoint{6.182278in}{1.544520in}}{\pgfqpoint{6.187322in}{1.549563in}}%
\pgfpathcurveto{\pgfqpoint{6.192366in}{1.554607in}}{\pgfqpoint{6.195200in}{1.561449in}}{\pgfqpoint{6.195200in}{1.568582in}}%
\pgfpathcurveto{\pgfqpoint{6.195200in}{1.575714in}}{\pgfqpoint{6.192366in}{1.582556in}}{\pgfqpoint{6.187322in}{1.587600in}}%
\pgfpathcurveto{\pgfqpoint{6.182278in}{1.592643in}}{\pgfqpoint{6.175437in}{1.595477in}}{\pgfqpoint{6.168304in}{1.595477in}}%
\pgfpathcurveto{\pgfqpoint{6.161171in}{1.595477in}}{\pgfqpoint{6.154330in}{1.592643in}}{\pgfqpoint{6.149286in}{1.587600in}}%
\pgfpathcurveto{\pgfqpoint{6.144242in}{1.582556in}}{\pgfqpoint{6.141408in}{1.575714in}}{\pgfqpoint{6.141408in}{1.568582in}}%
\pgfpathcurveto{\pgfqpoint{6.141408in}{1.561449in}}{\pgfqpoint{6.144242in}{1.554607in}}{\pgfqpoint{6.149286in}{1.549563in}}%
\pgfpathcurveto{\pgfqpoint{6.154330in}{1.544520in}}{\pgfqpoint{6.161171in}{1.541686in}}{\pgfqpoint{6.168304in}{1.541686in}}%
\pgfpathclose%
\pgfusepath{stroke,fill}%
\end{pgfscope}%
\begin{pgfscope}%
\pgfpathrectangle{\pgfqpoint{4.985294in}{0.500000in}}{\pgfqpoint{1.764706in}{1.700000in}}%
\pgfusepath{clip}%
\pgfsetbuttcap%
\pgfsetroundjoin%
\definecolor{currentfill}{rgb}{0.979891,0.908948,0.848279}%
\pgfsetfillcolor{currentfill}%
\pgfsetlinewidth{0.311001pt}%
\definecolor{currentstroke}{rgb}{1.000000,1.000000,1.000000}%
\pgfsetstrokecolor{currentstroke}%
\pgfsetdash{}{0pt}%
\pgfpathmoveto{\pgfqpoint{6.299025in}{1.531893in}}%
\pgfpathcurveto{\pgfqpoint{6.306158in}{1.531893in}}{\pgfqpoint{6.313000in}{1.534726in}}{\pgfqpoint{6.318044in}{1.539770in}}%
\pgfpathcurveto{\pgfqpoint{6.323087in}{1.544814in}}{\pgfqpoint{6.325921in}{1.551655in}}{\pgfqpoint{6.325921in}{1.558788in}}%
\pgfpathcurveto{\pgfqpoint{6.325921in}{1.565921in}}{\pgfqpoint{6.323087in}{1.572763in}}{\pgfqpoint{6.318044in}{1.577806in}}%
\pgfpathcurveto{\pgfqpoint{6.313000in}{1.582850in}}{\pgfqpoint{6.306158in}{1.585684in}}{\pgfqpoint{6.299025in}{1.585684in}}%
\pgfpathcurveto{\pgfqpoint{6.291893in}{1.585684in}}{\pgfqpoint{6.285051in}{1.582850in}}{\pgfqpoint{6.280007in}{1.577806in}}%
\pgfpathcurveto{\pgfqpoint{6.274964in}{1.572763in}}{\pgfqpoint{6.272130in}{1.565921in}}{\pgfqpoint{6.272130in}{1.558788in}}%
\pgfpathcurveto{\pgfqpoint{6.272130in}{1.551655in}}{\pgfqpoint{6.274964in}{1.544814in}}{\pgfqpoint{6.280007in}{1.539770in}}%
\pgfpathcurveto{\pgfqpoint{6.285051in}{1.534726in}}{\pgfqpoint{6.291893in}{1.531893in}}{\pgfqpoint{6.299025in}{1.531893in}}%
\pgfpathclose%
\pgfusepath{stroke,fill}%
\end{pgfscope}%
\begin{pgfscope}%
\pgfpathrectangle{\pgfqpoint{4.985294in}{0.500000in}}{\pgfqpoint{1.764706in}{1.700000in}}%
\pgfusepath{clip}%
\pgfsetbuttcap%
\pgfsetroundjoin%
\definecolor{currentfill}{rgb}{0.965302,0.713942,0.568499}%
\pgfsetfillcolor{currentfill}%
\pgfsetlinewidth{0.311001pt}%
\definecolor{currentstroke}{rgb}{1.000000,1.000000,1.000000}%
\pgfsetstrokecolor{currentstroke}%
\pgfsetdash{}{0pt}%
\pgfpathmoveto{\pgfqpoint{6.139295in}{1.581205in}}%
\pgfpathcurveto{\pgfqpoint{6.146428in}{1.581205in}}{\pgfqpoint{6.153270in}{1.584039in}}{\pgfqpoint{6.158313in}{1.589083in}}%
\pgfpathcurveto{\pgfqpoint{6.163357in}{1.594127in}}{\pgfqpoint{6.166191in}{1.600968in}}{\pgfqpoint{6.166191in}{1.608101in}}%
\pgfpathcurveto{\pgfqpoint{6.166191in}{1.615234in}}{\pgfqpoint{6.163357in}{1.622076in}}{\pgfqpoint{6.158313in}{1.627119in}}%
\pgfpathcurveto{\pgfqpoint{6.153270in}{1.632163in}}{\pgfqpoint{6.146428in}{1.634997in}}{\pgfqpoint{6.139295in}{1.634997in}}%
\pgfpathcurveto{\pgfqpoint{6.132162in}{1.634997in}}{\pgfqpoint{6.125321in}{1.632163in}}{\pgfqpoint{6.120277in}{1.627119in}}%
\pgfpathcurveto{\pgfqpoint{6.115233in}{1.622076in}}{\pgfqpoint{6.112399in}{1.615234in}}{\pgfqpoint{6.112399in}{1.608101in}}%
\pgfpathcurveto{\pgfqpoint{6.112399in}{1.600968in}}{\pgfqpoint{6.115233in}{1.594127in}}{\pgfqpoint{6.120277in}{1.589083in}}%
\pgfpathcurveto{\pgfqpoint{6.125321in}{1.584039in}}{\pgfqpoint{6.132162in}{1.581205in}}{\pgfqpoint{6.139295in}{1.581205in}}%
\pgfpathclose%
\pgfusepath{stroke,fill}%
\end{pgfscope}%
\begin{pgfscope}%
\pgfpathrectangle{\pgfqpoint{4.985294in}{0.500000in}}{\pgfqpoint{1.764706in}{1.700000in}}%
\pgfusepath{clip}%
\pgfsetbuttcap%
\pgfsetroundjoin%
\definecolor{currentfill}{rgb}{0.966120,0.744512,0.608720}%
\pgfsetfillcolor{currentfill}%
\pgfsetlinewidth{0.311001pt}%
\definecolor{currentstroke}{rgb}{1.000000,1.000000,1.000000}%
\pgfsetstrokecolor{currentstroke}%
\pgfsetdash{}{0pt}%
\pgfpathmoveto{\pgfqpoint{5.521094in}{1.195379in}}%
\pgfpathcurveto{\pgfqpoint{5.528227in}{1.195379in}}{\pgfqpoint{5.535068in}{1.198213in}}{\pgfqpoint{5.540112in}{1.203256in}}%
\pgfpathcurveto{\pgfqpoint{5.545156in}{1.208300in}}{\pgfqpoint{5.547990in}{1.215142in}}{\pgfqpoint{5.547990in}{1.222274in}}%
\pgfpathcurveto{\pgfqpoint{5.547990in}{1.229407in}}{\pgfqpoint{5.545156in}{1.236249in}}{\pgfqpoint{5.540112in}{1.241293in}}%
\pgfpathcurveto{\pgfqpoint{5.535068in}{1.246336in}}{\pgfqpoint{5.528227in}{1.249170in}}{\pgfqpoint{5.521094in}{1.249170in}}%
\pgfpathcurveto{\pgfqpoint{5.513961in}{1.249170in}}{\pgfqpoint{5.507119in}{1.246336in}}{\pgfqpoint{5.502076in}{1.241293in}}%
\pgfpathcurveto{\pgfqpoint{5.497032in}{1.236249in}}{\pgfqpoint{5.494198in}{1.229407in}}{\pgfqpoint{5.494198in}{1.222274in}}%
\pgfpathcurveto{\pgfqpoint{5.494198in}{1.215142in}}{\pgfqpoint{5.497032in}{1.208300in}}{\pgfqpoint{5.502076in}{1.203256in}}%
\pgfpathcurveto{\pgfqpoint{5.507119in}{1.198213in}}{\pgfqpoint{5.513961in}{1.195379in}}{\pgfqpoint{5.521094in}{1.195379in}}%
\pgfpathclose%
\pgfusepath{stroke,fill}%
\end{pgfscope}%
\begin{pgfscope}%
\pgfpathrectangle{\pgfqpoint{4.985294in}{0.500000in}}{\pgfqpoint{1.764706in}{1.700000in}}%
\pgfusepath{clip}%
\pgfsetbuttcap%
\pgfsetroundjoin%
\definecolor{currentfill}{rgb}{0.976961,0.885681,0.814303}%
\pgfsetfillcolor{currentfill}%
\pgfsetlinewidth{0.311001pt}%
\definecolor{currentstroke}{rgb}{1.000000,1.000000,1.000000}%
\pgfsetstrokecolor{currentstroke}%
\pgfsetdash{}{0pt}%
\pgfpathmoveto{\pgfqpoint{5.435847in}{1.130568in}}%
\pgfpathcurveto{\pgfqpoint{5.442980in}{1.130568in}}{\pgfqpoint{5.449822in}{1.133402in}}{\pgfqpoint{5.454865in}{1.138445in}}%
\pgfpathcurveto{\pgfqpoint{5.459909in}{1.143489in}}{\pgfqpoint{5.462743in}{1.150330in}}{\pgfqpoint{5.462743in}{1.157463in}}%
\pgfpathcurveto{\pgfqpoint{5.462743in}{1.164596in}}{\pgfqpoint{5.459909in}{1.171438in}}{\pgfqpoint{5.454865in}{1.176481in}}%
\pgfpathcurveto{\pgfqpoint{5.449822in}{1.181525in}}{\pgfqpoint{5.442980in}{1.184359in}}{\pgfqpoint{5.435847in}{1.184359in}}%
\pgfpathcurveto{\pgfqpoint{5.428714in}{1.184359in}}{\pgfqpoint{5.421873in}{1.181525in}}{\pgfqpoint{5.416829in}{1.176481in}}%
\pgfpathcurveto{\pgfqpoint{5.411785in}{1.171438in}}{\pgfqpoint{5.408952in}{1.164596in}}{\pgfqpoint{5.408952in}{1.157463in}}%
\pgfpathcurveto{\pgfqpoint{5.408952in}{1.150330in}}{\pgfqpoint{5.411785in}{1.143489in}}{\pgfqpoint{5.416829in}{1.138445in}}%
\pgfpathcurveto{\pgfqpoint{5.421873in}{1.133402in}}{\pgfqpoint{5.428714in}{1.130568in}}{\pgfqpoint{5.435847in}{1.130568in}}%
\pgfpathclose%
\pgfusepath{stroke,fill}%
\end{pgfscope}%
\begin{pgfscope}%
\pgfpathrectangle{\pgfqpoint{4.985294in}{0.500000in}}{\pgfqpoint{1.764706in}{1.700000in}}%
\pgfusepath{clip}%
\pgfsetbuttcap%
\pgfsetroundjoin%
\definecolor{currentfill}{rgb}{0.972726,0.844889,0.754401}%
\pgfsetfillcolor{currentfill}%
\pgfsetlinewidth{0.311001pt}%
\definecolor{currentstroke}{rgb}{1.000000,1.000000,1.000000}%
\pgfsetstrokecolor{currentstroke}%
\pgfsetdash{}{0pt}%
\pgfpathmoveto{\pgfqpoint{6.358074in}{1.182324in}}%
\pgfpathcurveto{\pgfqpoint{6.365207in}{1.182324in}}{\pgfqpoint{6.372049in}{1.185158in}}{\pgfqpoint{6.377093in}{1.190201in}}%
\pgfpathcurveto{\pgfqpoint{6.382136in}{1.195245in}}{\pgfqpoint{6.384970in}{1.202087in}}{\pgfqpoint{6.384970in}{1.209219in}}%
\pgfpathcurveto{\pgfqpoint{6.384970in}{1.216352in}}{\pgfqpoint{6.382136in}{1.223194in}}{\pgfqpoint{6.377093in}{1.228237in}}%
\pgfpathcurveto{\pgfqpoint{6.372049in}{1.233281in}}{\pgfqpoint{6.365207in}{1.236115in}}{\pgfqpoint{6.358074in}{1.236115in}}%
\pgfpathcurveto{\pgfqpoint{6.350942in}{1.236115in}}{\pgfqpoint{6.344100in}{1.233281in}}{\pgfqpoint{6.339056in}{1.228237in}}%
\pgfpathcurveto{\pgfqpoint{6.334013in}{1.223194in}}{\pgfqpoint{6.331179in}{1.216352in}}{\pgfqpoint{6.331179in}{1.209219in}}%
\pgfpathcurveto{\pgfqpoint{6.331179in}{1.202087in}}{\pgfqpoint{6.334013in}{1.195245in}}{\pgfqpoint{6.339056in}{1.190201in}}%
\pgfpathcurveto{\pgfqpoint{6.344100in}{1.185158in}}{\pgfqpoint{6.350942in}{1.182324in}}{\pgfqpoint{6.358074in}{1.182324in}}%
\pgfpathclose%
\pgfusepath{stroke,fill}%
\end{pgfscope}%
\begin{pgfscope}%
\pgfpathrectangle{\pgfqpoint{4.985294in}{0.500000in}}{\pgfqpoint{1.764706in}{1.700000in}}%
\pgfusepath{clip}%
\pgfsetbuttcap%
\pgfsetroundjoin%
\definecolor{currentfill}{rgb}{0.964799,0.689101,0.537560}%
\pgfsetfillcolor{currentfill}%
\pgfsetlinewidth{0.311001pt}%
\definecolor{currentstroke}{rgb}{1.000000,1.000000,1.000000}%
\pgfsetstrokecolor{currentstroke}%
\pgfsetdash{}{0pt}%
\pgfpathmoveto{\pgfqpoint{5.571042in}{1.537812in}}%
\pgfpathcurveto{\pgfqpoint{5.578175in}{1.537812in}}{\pgfqpoint{5.585016in}{1.540646in}}{\pgfqpoint{5.590060in}{1.545690in}}%
\pgfpathcurveto{\pgfqpoint{5.595104in}{1.550734in}}{\pgfqpoint{5.597937in}{1.557575in}}{\pgfqpoint{5.597937in}{1.564708in}}%
\pgfpathcurveto{\pgfqpoint{5.597937in}{1.571841in}}{\pgfqpoint{5.595104in}{1.578682in}}{\pgfqpoint{5.590060in}{1.583726in}}%
\pgfpathcurveto{\pgfqpoint{5.585016in}{1.588770in}}{\pgfqpoint{5.578175in}{1.591604in}}{\pgfqpoint{5.571042in}{1.591604in}}%
\pgfpathcurveto{\pgfqpoint{5.563909in}{1.591604in}}{\pgfqpoint{5.557067in}{1.588770in}}{\pgfqpoint{5.552024in}{1.583726in}}%
\pgfpathcurveto{\pgfqpoint{5.546980in}{1.578682in}}{\pgfqpoint{5.544146in}{1.571841in}}{\pgfqpoint{5.544146in}{1.564708in}}%
\pgfpathcurveto{\pgfqpoint{5.544146in}{1.557575in}}{\pgfqpoint{5.546980in}{1.550734in}}{\pgfqpoint{5.552024in}{1.545690in}}%
\pgfpathcurveto{\pgfqpoint{5.557067in}{1.540646in}}{\pgfqpoint{5.563909in}{1.537812in}}{\pgfqpoint{5.571042in}{1.537812in}}%
\pgfpathclose%
\pgfusepath{stroke,fill}%
\end{pgfscope}%
\begin{pgfscope}%
\pgfpathrectangle{\pgfqpoint{4.985294in}{0.500000in}}{\pgfqpoint{1.764706in}{1.700000in}}%
\pgfusepath{clip}%
\pgfsetbuttcap%
\pgfsetroundjoin%
\definecolor{currentfill}{rgb}{0.961433,0.573272,0.412036}%
\pgfsetfillcolor{currentfill}%
\pgfsetlinewidth{0.311001pt}%
\definecolor{currentstroke}{rgb}{1.000000,1.000000,1.000000}%
\pgfsetstrokecolor{currentstroke}%
\pgfsetdash{}{0pt}%
\pgfpathmoveto{\pgfqpoint{6.233081in}{1.750979in}}%
\pgfpathcurveto{\pgfqpoint{6.240213in}{1.750979in}}{\pgfqpoint{6.247055in}{1.753813in}}{\pgfqpoint{6.252099in}{1.758857in}}%
\pgfpathcurveto{\pgfqpoint{6.257142in}{1.763900in}}{\pgfqpoint{6.259976in}{1.770742in}}{\pgfqpoint{6.259976in}{1.777875in}}%
\pgfpathcurveto{\pgfqpoint{6.259976in}{1.785008in}}{\pgfqpoint{6.257142in}{1.791849in}}{\pgfqpoint{6.252099in}{1.796893in}}%
\pgfpathcurveto{\pgfqpoint{6.247055in}{1.801937in}}{\pgfqpoint{6.240213in}{1.804770in}}{\pgfqpoint{6.233081in}{1.804770in}}%
\pgfpathcurveto{\pgfqpoint{6.225948in}{1.804770in}}{\pgfqpoint{6.219106in}{1.801937in}}{\pgfqpoint{6.214062in}{1.796893in}}%
\pgfpathcurveto{\pgfqpoint{6.209019in}{1.791849in}}{\pgfqpoint{6.206185in}{1.785008in}}{\pgfqpoint{6.206185in}{1.777875in}}%
\pgfpathcurveto{\pgfqpoint{6.206185in}{1.770742in}}{\pgfqpoint{6.209019in}{1.763900in}}{\pgfqpoint{6.214062in}{1.758857in}}%
\pgfpathcurveto{\pgfqpoint{6.219106in}{1.753813in}}{\pgfqpoint{6.225948in}{1.750979in}}{\pgfqpoint{6.233081in}{1.750979in}}%
\pgfpathclose%
\pgfusepath{stroke,fill}%
\end{pgfscope}%
\begin{pgfscope}%
\pgfpathrectangle{\pgfqpoint{4.985294in}{0.500000in}}{\pgfqpoint{1.764706in}{1.700000in}}%
\pgfusepath{clip}%
\pgfsetbuttcap%
\pgfsetroundjoin%
\definecolor{currentfill}{rgb}{0.976961,0.885681,0.814303}%
\pgfsetfillcolor{currentfill}%
\pgfsetlinewidth{0.311001pt}%
\definecolor{currentstroke}{rgb}{1.000000,1.000000,1.000000}%
\pgfsetstrokecolor{currentstroke}%
\pgfsetdash{}{0pt}%
\pgfpathmoveto{\pgfqpoint{6.280612in}{1.373901in}}%
\pgfpathcurveto{\pgfqpoint{6.287745in}{1.373901in}}{\pgfqpoint{6.294586in}{1.376734in}}{\pgfqpoint{6.299630in}{1.381778in}}%
\pgfpathcurveto{\pgfqpoint{6.304674in}{1.386822in}}{\pgfqpoint{6.307508in}{1.393663in}}{\pgfqpoint{6.307508in}{1.400796in}}%
\pgfpathcurveto{\pgfqpoint{6.307508in}{1.407929in}}{\pgfqpoint{6.304674in}{1.414771in}}{\pgfqpoint{6.299630in}{1.419814in}}%
\pgfpathcurveto{\pgfqpoint{6.294586in}{1.424858in}}{\pgfqpoint{6.287745in}{1.427692in}}{\pgfqpoint{6.280612in}{1.427692in}}%
\pgfpathcurveto{\pgfqpoint{6.273479in}{1.427692in}}{\pgfqpoint{6.266638in}{1.424858in}}{\pgfqpoint{6.261594in}{1.419814in}}%
\pgfpathcurveto{\pgfqpoint{6.256550in}{1.414771in}}{\pgfqpoint{6.253716in}{1.407929in}}{\pgfqpoint{6.253716in}{1.400796in}}%
\pgfpathcurveto{\pgfqpoint{6.253716in}{1.393663in}}{\pgfqpoint{6.256550in}{1.386822in}}{\pgfqpoint{6.261594in}{1.381778in}}%
\pgfpathcurveto{\pgfqpoint{6.266638in}{1.376734in}}{\pgfqpoint{6.273479in}{1.373901in}}{\pgfqpoint{6.280612in}{1.373901in}}%
\pgfpathclose%
\pgfusepath{stroke,fill}%
\end{pgfscope}%
\begin{pgfscope}%
\pgfpathrectangle{\pgfqpoint{4.985294in}{0.500000in}}{\pgfqpoint{1.764706in}{1.700000in}}%
\pgfusepath{clip}%
\pgfsetbuttcap%
\pgfsetroundjoin%
\definecolor{currentfill}{rgb}{0.963190,0.619109,0.458249}%
\pgfsetfillcolor{currentfill}%
\pgfsetlinewidth{0.311001pt}%
\definecolor{currentstroke}{rgb}{1.000000,1.000000,1.000000}%
\pgfsetstrokecolor{currentstroke}%
\pgfsetdash{}{0pt}%
\pgfpathmoveto{\pgfqpoint{5.568931in}{1.752969in}}%
\pgfpathcurveto{\pgfqpoint{5.576064in}{1.752969in}}{\pgfqpoint{5.582905in}{1.755802in}}{\pgfqpoint{5.587949in}{1.760846in}}%
\pgfpathcurveto{\pgfqpoint{5.592993in}{1.765890in}}{\pgfqpoint{5.595827in}{1.772731in}}{\pgfqpoint{5.595827in}{1.779864in}}%
\pgfpathcurveto{\pgfqpoint{5.595827in}{1.786997in}}{\pgfqpoint{5.592993in}{1.793839in}}{\pgfqpoint{5.587949in}{1.798882in}}%
\pgfpathcurveto{\pgfqpoint{5.582905in}{1.803926in}}{\pgfqpoint{5.576064in}{1.806760in}}{\pgfqpoint{5.568931in}{1.806760in}}%
\pgfpathcurveto{\pgfqpoint{5.561798in}{1.806760in}}{\pgfqpoint{5.554956in}{1.803926in}}{\pgfqpoint{5.549913in}{1.798882in}}%
\pgfpathcurveto{\pgfqpoint{5.544869in}{1.793839in}}{\pgfqpoint{5.542035in}{1.786997in}}{\pgfqpoint{5.542035in}{1.779864in}}%
\pgfpathcurveto{\pgfqpoint{5.542035in}{1.772731in}}{\pgfqpoint{5.544869in}{1.765890in}}{\pgfqpoint{5.549913in}{1.760846in}}%
\pgfpathcurveto{\pgfqpoint{5.554956in}{1.755802in}}{\pgfqpoint{5.561798in}{1.752969in}}{\pgfqpoint{5.568931in}{1.752969in}}%
\pgfpathclose%
\pgfusepath{stroke,fill}%
\end{pgfscope}%
\begin{pgfscope}%
\pgfpathrectangle{\pgfqpoint{4.985294in}{0.500000in}}{\pgfqpoint{1.764706in}{1.700000in}}%
\pgfusepath{clip}%
\pgfsetbuttcap%
\pgfsetroundjoin%
\definecolor{currentfill}{rgb}{0.967398,0.774513,0.650573}%
\pgfsetfillcolor{currentfill}%
\pgfsetlinewidth{0.311001pt}%
\definecolor{currentstroke}{rgb}{1.000000,1.000000,1.000000}%
\pgfsetstrokecolor{currentstroke}%
\pgfsetdash{}{0pt}%
\pgfpathmoveto{\pgfqpoint{5.371306in}{1.499838in}}%
\pgfpathcurveto{\pgfqpoint{5.378439in}{1.499838in}}{\pgfqpoint{5.385280in}{1.502672in}}{\pgfqpoint{5.390324in}{1.507715in}}%
\pgfpathcurveto{\pgfqpoint{5.395368in}{1.512759in}}{\pgfqpoint{5.398202in}{1.519601in}}{\pgfqpoint{5.398202in}{1.526734in}}%
\pgfpathcurveto{\pgfqpoint{5.398202in}{1.533866in}}{\pgfqpoint{5.395368in}{1.540708in}}{\pgfqpoint{5.390324in}{1.545752in}}%
\pgfpathcurveto{\pgfqpoint{5.385280in}{1.550795in}}{\pgfqpoint{5.378439in}{1.553629in}}{\pgfqpoint{5.371306in}{1.553629in}}%
\pgfpathcurveto{\pgfqpoint{5.364173in}{1.553629in}}{\pgfqpoint{5.357332in}{1.550795in}}{\pgfqpoint{5.352288in}{1.545752in}}%
\pgfpathcurveto{\pgfqpoint{5.347244in}{1.540708in}}{\pgfqpoint{5.344410in}{1.533866in}}{\pgfqpoint{5.344410in}{1.526734in}}%
\pgfpathcurveto{\pgfqpoint{5.344410in}{1.519601in}}{\pgfqpoint{5.347244in}{1.512759in}}{\pgfqpoint{5.352288in}{1.507715in}}%
\pgfpathcurveto{\pgfqpoint{5.357332in}{1.502672in}}{\pgfqpoint{5.364173in}{1.499838in}}{\pgfqpoint{5.371306in}{1.499838in}}%
\pgfpathclose%
\pgfusepath{stroke,fill}%
\end{pgfscope}%
\begin{pgfscope}%
\pgfpathrectangle{\pgfqpoint{4.985294in}{0.500000in}}{\pgfqpoint{1.764706in}{1.700000in}}%
\pgfusepath{clip}%
\pgfsetbuttcap%
\pgfsetroundjoin%
\definecolor{currentfill}{rgb}{0.964679,0.682838,0.530002}%
\pgfsetfillcolor{currentfill}%
\pgfsetlinewidth{0.311001pt}%
\definecolor{currentstroke}{rgb}{1.000000,1.000000,1.000000}%
\pgfsetstrokecolor{currentstroke}%
\pgfsetdash{}{0pt}%
\pgfpathmoveto{\pgfqpoint{5.616624in}{0.920314in}}%
\pgfpathcurveto{\pgfqpoint{5.623757in}{0.920314in}}{\pgfqpoint{5.630599in}{0.923148in}}{\pgfqpoint{5.635642in}{0.928191in}}%
\pgfpathcurveto{\pgfqpoint{5.640686in}{0.933235in}}{\pgfqpoint{5.643520in}{0.940077in}}{\pgfqpoint{5.643520in}{0.947210in}}%
\pgfpathcurveto{\pgfqpoint{5.643520in}{0.954342in}}{\pgfqpoint{5.640686in}{0.961184in}}{\pgfqpoint{5.635642in}{0.966228in}}%
\pgfpathcurveto{\pgfqpoint{5.630599in}{0.971271in}}{\pgfqpoint{5.623757in}{0.974105in}}{\pgfqpoint{5.616624in}{0.974105in}}%
\pgfpathcurveto{\pgfqpoint{5.609491in}{0.974105in}}{\pgfqpoint{5.602650in}{0.971271in}}{\pgfqpoint{5.597606in}{0.966228in}}%
\pgfpathcurveto{\pgfqpoint{5.592562in}{0.961184in}}{\pgfqpoint{5.589728in}{0.954342in}}{\pgfqpoint{5.589728in}{0.947210in}}%
\pgfpathcurveto{\pgfqpoint{5.589728in}{0.940077in}}{\pgfqpoint{5.592562in}{0.933235in}}{\pgfqpoint{5.597606in}{0.928191in}}%
\pgfpathcurveto{\pgfqpoint{5.602650in}{0.923148in}}{\pgfqpoint{5.609491in}{0.920314in}}{\pgfqpoint{5.616624in}{0.920314in}}%
\pgfpathclose%
\pgfusepath{stroke,fill}%
\end{pgfscope}%
\begin{pgfscope}%
\pgfpathrectangle{\pgfqpoint{4.985294in}{0.500000in}}{\pgfqpoint{1.764706in}{1.700000in}}%
\pgfusepath{clip}%
\pgfsetbuttcap%
\pgfsetroundjoin%
\definecolor{currentfill}{rgb}{0.807528,0.112400,0.304997}%
\pgfsetfillcolor{currentfill}%
\pgfsetlinewidth{0.311001pt}%
\definecolor{currentstroke}{rgb}{1.000000,1.000000,1.000000}%
\pgfsetstrokecolor{currentstroke}%
\pgfsetdash{}{0pt}%
\pgfpathmoveto{\pgfqpoint{6.260121in}{1.842473in}}%
\pgfpathcurveto{\pgfqpoint{6.267253in}{1.842473in}}{\pgfqpoint{6.274095in}{1.845306in}}{\pgfqpoint{6.279139in}{1.850350in}}%
\pgfpathcurveto{\pgfqpoint{6.284182in}{1.855394in}}{\pgfqpoint{6.287016in}{1.862235in}}{\pgfqpoint{6.287016in}{1.869368in}}%
\pgfpathcurveto{\pgfqpoint{6.287016in}{1.876501in}}{\pgfqpoint{6.284182in}{1.883343in}}{\pgfqpoint{6.279139in}{1.888386in}}%
\pgfpathcurveto{\pgfqpoint{6.274095in}{1.893430in}}{\pgfqpoint{6.267253in}{1.896264in}}{\pgfqpoint{6.260121in}{1.896264in}}%
\pgfpathcurveto{\pgfqpoint{6.252988in}{1.896264in}}{\pgfqpoint{6.246146in}{1.893430in}}{\pgfqpoint{6.241102in}{1.888386in}}%
\pgfpathcurveto{\pgfqpoint{6.236059in}{1.883343in}}{\pgfqpoint{6.233225in}{1.876501in}}{\pgfqpoint{6.233225in}{1.869368in}}%
\pgfpathcurveto{\pgfqpoint{6.233225in}{1.862235in}}{\pgfqpoint{6.236059in}{1.855394in}}{\pgfqpoint{6.241102in}{1.850350in}}%
\pgfpathcurveto{\pgfqpoint{6.246146in}{1.845306in}}{\pgfqpoint{6.252988in}{1.842473in}}{\pgfqpoint{6.260121in}{1.842473in}}%
\pgfpathclose%
\pgfusepath{stroke,fill}%
\end{pgfscope}%
\begin{pgfscope}%
\pgfpathrectangle{\pgfqpoint{4.985294in}{0.500000in}}{\pgfqpoint{1.764706in}{1.700000in}}%
\pgfusepath{clip}%
\pgfsetbuttcap%
\pgfsetroundjoin%
\definecolor{currentfill}{rgb}{0.975018,0.868213,0.788710}%
\pgfsetfillcolor{currentfill}%
\pgfsetlinewidth{0.311001pt}%
\definecolor{currentstroke}{rgb}{1.000000,1.000000,1.000000}%
\pgfsetstrokecolor{currentstroke}%
\pgfsetdash{}{0pt}%
\pgfpathmoveto{\pgfqpoint{5.469528in}{1.236605in}}%
\pgfpathcurveto{\pgfqpoint{5.476661in}{1.236605in}}{\pgfqpoint{5.483503in}{1.239439in}}{\pgfqpoint{5.488547in}{1.244482in}}%
\pgfpathcurveto{\pgfqpoint{5.493590in}{1.249526in}}{\pgfqpoint{5.496424in}{1.256368in}}{\pgfqpoint{5.496424in}{1.263501in}}%
\pgfpathcurveto{\pgfqpoint{5.496424in}{1.270633in}}{\pgfqpoint{5.493590in}{1.277475in}}{\pgfqpoint{5.488547in}{1.282519in}}%
\pgfpathcurveto{\pgfqpoint{5.483503in}{1.287562in}}{\pgfqpoint{5.476661in}{1.290396in}}{\pgfqpoint{5.469528in}{1.290396in}}%
\pgfpathcurveto{\pgfqpoint{5.462396in}{1.290396in}}{\pgfqpoint{5.455554in}{1.287562in}}{\pgfqpoint{5.450510in}{1.282519in}}%
\pgfpathcurveto{\pgfqpoint{5.445467in}{1.277475in}}{\pgfqpoint{5.442633in}{1.270633in}}{\pgfqpoint{5.442633in}{1.263501in}}%
\pgfpathcurveto{\pgfqpoint{5.442633in}{1.256368in}}{\pgfqpoint{5.445467in}{1.249526in}}{\pgfqpoint{5.450510in}{1.244482in}}%
\pgfpathcurveto{\pgfqpoint{5.455554in}{1.239439in}}{\pgfqpoint{5.462396in}{1.236605in}}{\pgfqpoint{5.469528in}{1.236605in}}%
\pgfpathclose%
\pgfusepath{stroke,fill}%
\end{pgfscope}%
\begin{pgfscope}%
\pgfpathrectangle{\pgfqpoint{4.985294in}{0.500000in}}{\pgfqpoint{1.764706in}{1.700000in}}%
\pgfusepath{clip}%
\pgfsetbuttcap%
\pgfsetroundjoin%
\definecolor{currentfill}{rgb}{0.949145,0.420383,0.287810}%
\pgfsetfillcolor{currentfill}%
\pgfsetlinewidth{0.311001pt}%
\definecolor{currentstroke}{rgb}{1.000000,1.000000,1.000000}%
\pgfsetstrokecolor{currentstroke}%
\pgfsetdash{}{0pt}%
\pgfpathmoveto{\pgfqpoint{5.291639in}{1.152988in}}%
\pgfpathcurveto{\pgfqpoint{5.298772in}{1.152988in}}{\pgfqpoint{5.305614in}{1.155822in}}{\pgfqpoint{5.310657in}{1.160866in}}%
\pgfpathcurveto{\pgfqpoint{5.315701in}{1.165909in}}{\pgfqpoint{5.318535in}{1.172751in}}{\pgfqpoint{5.318535in}{1.179884in}}%
\pgfpathcurveto{\pgfqpoint{5.318535in}{1.187017in}}{\pgfqpoint{5.315701in}{1.193858in}}{\pgfqpoint{5.310657in}{1.198902in}}%
\pgfpathcurveto{\pgfqpoint{5.305614in}{1.203946in}}{\pgfqpoint{5.298772in}{1.206780in}}{\pgfqpoint{5.291639in}{1.206780in}}%
\pgfpathcurveto{\pgfqpoint{5.284506in}{1.206780in}}{\pgfqpoint{5.277665in}{1.203946in}}{\pgfqpoint{5.272621in}{1.198902in}}%
\pgfpathcurveto{\pgfqpoint{5.267577in}{1.193858in}}{\pgfqpoint{5.264744in}{1.187017in}}{\pgfqpoint{5.264744in}{1.179884in}}%
\pgfpathcurveto{\pgfqpoint{5.264744in}{1.172751in}}{\pgfqpoint{5.267577in}{1.165909in}}{\pgfqpoint{5.272621in}{1.160866in}}%
\pgfpathcurveto{\pgfqpoint{5.277665in}{1.155822in}}{\pgfqpoint{5.284506in}{1.152988in}}{\pgfqpoint{5.291639in}{1.152988in}}%
\pgfpathclose%
\pgfusepath{stroke,fill}%
\end{pgfscope}%
\begin{pgfscope}%
\pgfpathrectangle{\pgfqpoint{4.985294in}{0.500000in}}{\pgfqpoint{1.764706in}{1.700000in}}%
\pgfusepath{clip}%
\pgfsetbuttcap%
\pgfsetroundjoin%
\definecolor{currentfill}{rgb}{0.973832,0.856556,0.771584}%
\pgfsetfillcolor{currentfill}%
\pgfsetlinewidth{0.311001pt}%
\definecolor{currentstroke}{rgb}{1.000000,1.000000,1.000000}%
\pgfsetstrokecolor{currentstroke}%
\pgfsetdash{}{0pt}%
\pgfpathmoveto{\pgfqpoint{6.319337in}{1.119877in}}%
\pgfpathcurveto{\pgfqpoint{6.326470in}{1.119877in}}{\pgfqpoint{6.333311in}{1.122711in}}{\pgfqpoint{6.338355in}{1.127755in}}%
\pgfpathcurveto{\pgfqpoint{6.343399in}{1.132799in}}{\pgfqpoint{6.346233in}{1.139640in}}{\pgfqpoint{6.346233in}{1.146773in}}%
\pgfpathcurveto{\pgfqpoint{6.346233in}{1.153906in}}{\pgfqpoint{6.343399in}{1.160748in}}{\pgfqpoint{6.338355in}{1.165791in}}%
\pgfpathcurveto{\pgfqpoint{6.333311in}{1.170835in}}{\pgfqpoint{6.326470in}{1.173669in}}{\pgfqpoint{6.319337in}{1.173669in}}%
\pgfpathcurveto{\pgfqpoint{6.312204in}{1.173669in}}{\pgfqpoint{6.305362in}{1.170835in}}{\pgfqpoint{6.300319in}{1.165791in}}%
\pgfpathcurveto{\pgfqpoint{6.295275in}{1.160748in}}{\pgfqpoint{6.292441in}{1.153906in}}{\pgfqpoint{6.292441in}{1.146773in}}%
\pgfpathcurveto{\pgfqpoint{6.292441in}{1.139640in}}{\pgfqpoint{6.295275in}{1.132799in}}{\pgfqpoint{6.300319in}{1.127755in}}%
\pgfpathcurveto{\pgfqpoint{6.305362in}{1.122711in}}{\pgfqpoint{6.312204in}{1.119877in}}{\pgfqpoint{6.319337in}{1.119877in}}%
\pgfpathclose%
\pgfusepath{stroke,fill}%
\end{pgfscope}%
\begin{pgfscope}%
\pgfpathrectangle{\pgfqpoint{4.985294in}{0.500000in}}{\pgfqpoint{1.764706in}{1.700000in}}%
\pgfusepath{clip}%
\pgfsetbuttcap%
\pgfsetroundjoin%
\definecolor{currentfill}{rgb}{0.961115,0.566634,0.405693}%
\pgfsetfillcolor{currentfill}%
\pgfsetlinewidth{0.311001pt}%
\definecolor{currentstroke}{rgb}{1.000000,1.000000,1.000000}%
\pgfsetstrokecolor{currentstroke}%
\pgfsetdash{}{0pt}%
\pgfpathmoveto{\pgfqpoint{5.603859in}{1.090919in}}%
\pgfpathcurveto{\pgfqpoint{5.610991in}{1.090919in}}{\pgfqpoint{5.617833in}{1.093753in}}{\pgfqpoint{5.622877in}{1.098796in}}%
\pgfpathcurveto{\pgfqpoint{5.627920in}{1.103840in}}{\pgfqpoint{5.630754in}{1.110681in}}{\pgfqpoint{5.630754in}{1.117814in}}%
\pgfpathcurveto{\pgfqpoint{5.630754in}{1.124947in}}{\pgfqpoint{5.627920in}{1.131789in}}{\pgfqpoint{5.622877in}{1.136832in}}%
\pgfpathcurveto{\pgfqpoint{5.617833in}{1.141876in}}{\pgfqpoint{5.610991in}{1.144710in}}{\pgfqpoint{5.603859in}{1.144710in}}%
\pgfpathcurveto{\pgfqpoint{5.596726in}{1.144710in}}{\pgfqpoint{5.589884in}{1.141876in}}{\pgfqpoint{5.584840in}{1.136832in}}%
\pgfpathcurveto{\pgfqpoint{5.579797in}{1.131789in}}{\pgfqpoint{5.576963in}{1.124947in}}{\pgfqpoint{5.576963in}{1.117814in}}%
\pgfpathcurveto{\pgfqpoint{5.576963in}{1.110681in}}{\pgfqpoint{5.579797in}{1.103840in}}{\pgfqpoint{5.584840in}{1.098796in}}%
\pgfpathcurveto{\pgfqpoint{5.589884in}{1.093753in}}{\pgfqpoint{5.596726in}{1.090919in}}{\pgfqpoint{5.603859in}{1.090919in}}%
\pgfpathclose%
\pgfusepath{stroke,fill}%
\end{pgfscope}%
\begin{pgfscope}%
\pgfpathrectangle{\pgfqpoint{4.985294in}{0.500000in}}{\pgfqpoint{1.764706in}{1.700000in}}%
\pgfusepath{clip}%
\pgfsetbuttcap%
\pgfsetroundjoin%
\definecolor{currentfill}{rgb}{0.966120,0.744512,0.608720}%
\pgfsetfillcolor{currentfill}%
\pgfsetlinewidth{0.311001pt}%
\definecolor{currentstroke}{rgb}{1.000000,1.000000,1.000000}%
\pgfsetstrokecolor{currentstroke}%
\pgfsetdash{}{0pt}%
\pgfpathmoveto{\pgfqpoint{5.549928in}{1.100346in}}%
\pgfpathcurveto{\pgfqpoint{5.557061in}{1.100346in}}{\pgfqpoint{5.563902in}{1.103180in}}{\pgfqpoint{5.568946in}{1.108224in}}%
\pgfpathcurveto{\pgfqpoint{5.573990in}{1.113268in}}{\pgfqpoint{5.576823in}{1.120109in}}{\pgfqpoint{5.576823in}{1.127242in}}%
\pgfpathcurveto{\pgfqpoint{5.576823in}{1.134375in}}{\pgfqpoint{5.573990in}{1.141217in}}{\pgfqpoint{5.568946in}{1.146260in}}%
\pgfpathcurveto{\pgfqpoint{5.563902in}{1.151304in}}{\pgfqpoint{5.557061in}{1.154138in}}{\pgfqpoint{5.549928in}{1.154138in}}%
\pgfpathcurveto{\pgfqpoint{5.542795in}{1.154138in}}{\pgfqpoint{5.535953in}{1.151304in}}{\pgfqpoint{5.530910in}{1.146260in}}%
\pgfpathcurveto{\pgfqpoint{5.525866in}{1.141217in}}{\pgfqpoint{5.523032in}{1.134375in}}{\pgfqpoint{5.523032in}{1.127242in}}%
\pgfpathcurveto{\pgfqpoint{5.523032in}{1.120109in}}{\pgfqpoint{5.525866in}{1.113268in}}{\pgfqpoint{5.530910in}{1.108224in}}%
\pgfpathcurveto{\pgfqpoint{5.535953in}{1.103180in}}{\pgfqpoint{5.542795in}{1.100346in}}{\pgfqpoint{5.549928in}{1.100346in}}%
\pgfpathclose%
\pgfusepath{stroke,fill}%
\end{pgfscope}%
\begin{pgfscope}%
\pgfpathrectangle{\pgfqpoint{4.985294in}{0.500000in}}{\pgfqpoint{1.764706in}{1.700000in}}%
\pgfusepath{clip}%
\pgfsetbuttcap%
\pgfsetroundjoin%
\definecolor{currentfill}{rgb}{0.962985,0.612625,0.451451}%
\pgfsetfillcolor{currentfill}%
\pgfsetlinewidth{0.311001pt}%
\definecolor{currentstroke}{rgb}{1.000000,1.000000,1.000000}%
\pgfsetstrokecolor{currentstroke}%
\pgfsetdash{}{0pt}%
\pgfpathmoveto{\pgfqpoint{6.111063in}{1.026831in}}%
\pgfpathcurveto{\pgfqpoint{6.118196in}{1.026831in}}{\pgfqpoint{6.125037in}{1.029665in}}{\pgfqpoint{6.130081in}{1.034708in}}%
\pgfpathcurveto{\pgfqpoint{6.135125in}{1.039752in}}{\pgfqpoint{6.137959in}{1.046594in}}{\pgfqpoint{6.137959in}{1.053727in}}%
\pgfpathcurveto{\pgfqpoint{6.137959in}{1.060859in}}{\pgfqpoint{6.135125in}{1.067701in}}{\pgfqpoint{6.130081in}{1.072745in}}%
\pgfpathcurveto{\pgfqpoint{6.125037in}{1.077788in}}{\pgfqpoint{6.118196in}{1.080622in}}{\pgfqpoint{6.111063in}{1.080622in}}%
\pgfpathcurveto{\pgfqpoint{6.103930in}{1.080622in}}{\pgfqpoint{6.097089in}{1.077788in}}{\pgfqpoint{6.092045in}{1.072745in}}%
\pgfpathcurveto{\pgfqpoint{6.087001in}{1.067701in}}{\pgfqpoint{6.084167in}{1.060859in}}{\pgfqpoint{6.084167in}{1.053727in}}%
\pgfpathcurveto{\pgfqpoint{6.084167in}{1.046594in}}{\pgfqpoint{6.087001in}{1.039752in}}{\pgfqpoint{6.092045in}{1.034708in}}%
\pgfpathcurveto{\pgfqpoint{6.097089in}{1.029665in}}{\pgfqpoint{6.103930in}{1.026831in}}{\pgfqpoint{6.111063in}{1.026831in}}%
\pgfpathclose%
\pgfusepath{stroke,fill}%
\end{pgfscope}%
\begin{pgfscope}%
\pgfpathrectangle{\pgfqpoint{4.985294in}{0.500000in}}{\pgfqpoint{1.764706in}{1.700000in}}%
\pgfusepath{clip}%
\pgfsetbuttcap%
\pgfsetroundjoin%
\definecolor{currentfill}{rgb}{0.976287,0.879862,0.805788}%
\pgfsetfillcolor{currentfill}%
\pgfsetlinewidth{0.311001pt}%
\definecolor{currentstroke}{rgb}{1.000000,1.000000,1.000000}%
\pgfsetstrokecolor{currentstroke}%
\pgfsetdash{}{0pt}%
\pgfpathmoveto{\pgfqpoint{5.458180in}{1.532953in}}%
\pgfpathcurveto{\pgfqpoint{5.465313in}{1.532953in}}{\pgfqpoint{5.472155in}{1.535787in}}{\pgfqpoint{5.477198in}{1.540830in}}%
\pgfpathcurveto{\pgfqpoint{5.482242in}{1.545874in}}{\pgfqpoint{5.485076in}{1.552716in}}{\pgfqpoint{5.485076in}{1.559848in}}%
\pgfpathcurveto{\pgfqpoint{5.485076in}{1.566981in}}{\pgfqpoint{5.482242in}{1.573823in}}{\pgfqpoint{5.477198in}{1.578867in}}%
\pgfpathcurveto{\pgfqpoint{5.472155in}{1.583910in}}{\pgfqpoint{5.465313in}{1.586744in}}{\pgfqpoint{5.458180in}{1.586744in}}%
\pgfpathcurveto{\pgfqpoint{5.451047in}{1.586744in}}{\pgfqpoint{5.444206in}{1.583910in}}{\pgfqpoint{5.439162in}{1.578867in}}%
\pgfpathcurveto{\pgfqpoint{5.434118in}{1.573823in}}{\pgfqpoint{5.431284in}{1.566981in}}{\pgfqpoint{5.431284in}{1.559848in}}%
\pgfpathcurveto{\pgfqpoint{5.431284in}{1.552716in}}{\pgfqpoint{5.434118in}{1.545874in}}{\pgfqpoint{5.439162in}{1.540830in}}%
\pgfpathcurveto{\pgfqpoint{5.444206in}{1.535787in}}{\pgfqpoint{5.451047in}{1.532953in}}{\pgfqpoint{5.458180in}{1.532953in}}%
\pgfpathclose%
\pgfusepath{stroke,fill}%
\end{pgfscope}%
\begin{pgfscope}%
\pgfpathrectangle{\pgfqpoint{4.985294in}{0.500000in}}{\pgfqpoint{1.764706in}{1.700000in}}%
\pgfusepath{clip}%
\pgfsetbuttcap%
\pgfsetroundjoin%
\definecolor{currentfill}{rgb}{0.969359,0.803954,0.693832}%
\pgfsetfillcolor{currentfill}%
\pgfsetlinewidth{0.311001pt}%
\definecolor{currentstroke}{rgb}{1.000000,1.000000,1.000000}%
\pgfsetstrokecolor{currentstroke}%
\pgfsetdash{}{0pt}%
\pgfpathmoveto{\pgfqpoint{5.351023in}{1.262862in}}%
\pgfpathcurveto{\pgfqpoint{5.358156in}{1.262862in}}{\pgfqpoint{5.364998in}{1.265696in}}{\pgfqpoint{5.370041in}{1.270740in}}%
\pgfpathcurveto{\pgfqpoint{5.375085in}{1.275783in}}{\pgfqpoint{5.377919in}{1.282625in}}{\pgfqpoint{5.377919in}{1.289758in}}%
\pgfpathcurveto{\pgfqpoint{5.377919in}{1.296891in}}{\pgfqpoint{5.375085in}{1.303732in}}{\pgfqpoint{5.370041in}{1.308776in}}%
\pgfpathcurveto{\pgfqpoint{5.364998in}{1.313820in}}{\pgfqpoint{5.358156in}{1.316654in}}{\pgfqpoint{5.351023in}{1.316654in}}%
\pgfpathcurveto{\pgfqpoint{5.343890in}{1.316654in}}{\pgfqpoint{5.337049in}{1.313820in}}{\pgfqpoint{5.332005in}{1.308776in}}%
\pgfpathcurveto{\pgfqpoint{5.326961in}{1.303732in}}{\pgfqpoint{5.324128in}{1.296891in}}{\pgfqpoint{5.324128in}{1.289758in}}%
\pgfpathcurveto{\pgfqpoint{5.324128in}{1.282625in}}{\pgfqpoint{5.326961in}{1.275783in}}{\pgfqpoint{5.332005in}{1.270740in}}%
\pgfpathcurveto{\pgfqpoint{5.337049in}{1.265696in}}{\pgfqpoint{5.343890in}{1.262862in}}{\pgfqpoint{5.351023in}{1.262862in}}%
\pgfpathclose%
\pgfusepath{stroke,fill}%
\end{pgfscope}%
\begin{pgfscope}%
\pgfpathrectangle{\pgfqpoint{4.985294in}{0.500000in}}{\pgfqpoint{1.764706in}{1.700000in}}%
\pgfusepath{clip}%
\pgfsetbuttcap%
\pgfsetroundjoin%
\definecolor{currentfill}{rgb}{0.887314,0.204699,0.257695}%
\pgfsetfillcolor{currentfill}%
\pgfsetlinewidth{0.311001pt}%
\definecolor{currentstroke}{rgb}{1.000000,1.000000,1.000000}%
\pgfsetstrokecolor{currentstroke}%
\pgfsetdash{}{0pt}%
\pgfpathmoveto{\pgfqpoint{5.247176in}{1.280337in}}%
\pgfpathcurveto{\pgfqpoint{5.254309in}{1.280337in}}{\pgfqpoint{5.261150in}{1.283171in}}{\pgfqpoint{5.266194in}{1.288214in}}%
\pgfpathcurveto{\pgfqpoint{5.271238in}{1.293258in}}{\pgfqpoint{5.274071in}{1.300100in}}{\pgfqpoint{5.274071in}{1.307233in}}%
\pgfpathcurveto{\pgfqpoint{5.274071in}{1.314365in}}{\pgfqpoint{5.271238in}{1.321207in}}{\pgfqpoint{5.266194in}{1.326251in}}%
\pgfpathcurveto{\pgfqpoint{5.261150in}{1.331294in}}{\pgfqpoint{5.254309in}{1.334128in}}{\pgfqpoint{5.247176in}{1.334128in}}%
\pgfpathcurveto{\pgfqpoint{5.240043in}{1.334128in}}{\pgfqpoint{5.233201in}{1.331294in}}{\pgfqpoint{5.228158in}{1.326251in}}%
\pgfpathcurveto{\pgfqpoint{5.223114in}{1.321207in}}{\pgfqpoint{5.220280in}{1.314365in}}{\pgfqpoint{5.220280in}{1.307233in}}%
\pgfpathcurveto{\pgfqpoint{5.220280in}{1.300100in}}{\pgfqpoint{5.223114in}{1.293258in}}{\pgfqpoint{5.228158in}{1.288214in}}%
\pgfpathcurveto{\pgfqpoint{5.233201in}{1.283171in}}{\pgfqpoint{5.240043in}{1.280337in}}{\pgfqpoint{5.247176in}{1.280337in}}%
\pgfpathclose%
\pgfusepath{stroke,fill}%
\end{pgfscope}%
\begin{pgfscope}%
\pgfpathrectangle{\pgfqpoint{4.985294in}{0.500000in}}{\pgfqpoint{1.764706in}{1.700000in}}%
\pgfusepath{clip}%
\pgfsetbuttcap%
\pgfsetroundjoin%
\definecolor{currentfill}{rgb}{0.870791,0.179821,0.267974}%
\pgfsetfillcolor{currentfill}%
\pgfsetlinewidth{0.311001pt}%
\definecolor{currentstroke}{rgb}{1.000000,1.000000,1.000000}%
\pgfsetstrokecolor{currentstroke}%
\pgfsetdash{}{0pt}%
\pgfpathmoveto{\pgfqpoint{5.329648in}{0.941074in}}%
\pgfpathcurveto{\pgfqpoint{5.336781in}{0.941074in}}{\pgfqpoint{5.343622in}{0.943908in}}{\pgfqpoint{5.348666in}{0.948952in}}%
\pgfpathcurveto{\pgfqpoint{5.353710in}{0.953996in}}{\pgfqpoint{5.356543in}{0.960837in}}{\pgfqpoint{5.356543in}{0.967970in}}%
\pgfpathcurveto{\pgfqpoint{5.356543in}{0.975103in}}{\pgfqpoint{5.353710in}{0.981945in}}{\pgfqpoint{5.348666in}{0.986988in}}%
\pgfpathcurveto{\pgfqpoint{5.343622in}{0.992032in}}{\pgfqpoint{5.336781in}{0.994866in}}{\pgfqpoint{5.329648in}{0.994866in}}%
\pgfpathcurveto{\pgfqpoint{5.322515in}{0.994866in}}{\pgfqpoint{5.315673in}{0.992032in}}{\pgfqpoint{5.310630in}{0.986988in}}%
\pgfpathcurveto{\pgfqpoint{5.305586in}{0.981945in}}{\pgfqpoint{5.302752in}{0.975103in}}{\pgfqpoint{5.302752in}{0.967970in}}%
\pgfpathcurveto{\pgfqpoint{5.302752in}{0.960837in}}{\pgfqpoint{5.305586in}{0.953996in}}{\pgfqpoint{5.310630in}{0.948952in}}%
\pgfpathcurveto{\pgfqpoint{5.315673in}{0.943908in}}{\pgfqpoint{5.322515in}{0.941074in}}{\pgfqpoint{5.329648in}{0.941074in}}%
\pgfpathclose%
\pgfusepath{stroke,fill}%
\end{pgfscope}%
\begin{pgfscope}%
\pgfpathrectangle{\pgfqpoint{4.985294in}{0.500000in}}{\pgfqpoint{1.764706in}{1.700000in}}%
\pgfusepath{clip}%
\pgfsetbuttcap%
\pgfsetroundjoin%
\definecolor{currentfill}{rgb}{0.977657,0.891500,0.822809}%
\pgfsetfillcolor{currentfill}%
\pgfsetlinewidth{0.311001pt}%
\definecolor{currentstroke}{rgb}{1.000000,1.000000,1.000000}%
\pgfsetstrokecolor{currentstroke}%
\pgfsetdash{}{0pt}%
\pgfpathmoveto{\pgfqpoint{5.417940in}{1.475370in}}%
\pgfpathcurveto{\pgfqpoint{5.425073in}{1.475370in}}{\pgfqpoint{5.431915in}{1.478204in}}{\pgfqpoint{5.436958in}{1.483247in}}%
\pgfpathcurveto{\pgfqpoint{5.442002in}{1.488291in}}{\pgfqpoint{5.444836in}{1.495133in}}{\pgfqpoint{5.444836in}{1.502265in}}%
\pgfpathcurveto{\pgfqpoint{5.444836in}{1.509398in}}{\pgfqpoint{5.442002in}{1.516240in}}{\pgfqpoint{5.436958in}{1.521284in}}%
\pgfpathcurveto{\pgfqpoint{5.431915in}{1.526327in}}{\pgfqpoint{5.425073in}{1.529161in}}{\pgfqpoint{5.417940in}{1.529161in}}%
\pgfpathcurveto{\pgfqpoint{5.410807in}{1.529161in}}{\pgfqpoint{5.403966in}{1.526327in}}{\pgfqpoint{5.398922in}{1.521284in}}%
\pgfpathcurveto{\pgfqpoint{5.393878in}{1.516240in}}{\pgfqpoint{5.391045in}{1.509398in}}{\pgfqpoint{5.391045in}{1.502265in}}%
\pgfpathcurveto{\pgfqpoint{5.391045in}{1.495133in}}{\pgfqpoint{5.393878in}{1.488291in}}{\pgfqpoint{5.398922in}{1.483247in}}%
\pgfpathcurveto{\pgfqpoint{5.403966in}{1.478204in}}{\pgfqpoint{5.410807in}{1.475370in}}{\pgfqpoint{5.417940in}{1.475370in}}%
\pgfpathclose%
\pgfusepath{stroke,fill}%
\end{pgfscope}%
\begin{pgfscope}%
\pgfpathrectangle{\pgfqpoint{4.985294in}{0.500000in}}{\pgfqpoint{1.764706in}{1.700000in}}%
\pgfusepath{clip}%
\pgfsetbuttcap%
\pgfsetroundjoin%
\definecolor{currentfill}{rgb}{0.963728,0.638439,0.479050}%
\pgfsetfillcolor{currentfill}%
\pgfsetlinewidth{0.311001pt}%
\definecolor{currentstroke}{rgb}{1.000000,1.000000,1.000000}%
\pgfsetstrokecolor{currentstroke}%
\pgfsetdash{}{0pt}%
\pgfpathmoveto{\pgfqpoint{5.567804in}{1.497103in}}%
\pgfpathcurveto{\pgfqpoint{5.574937in}{1.497103in}}{\pgfqpoint{5.581778in}{1.499936in}}{\pgfqpoint{5.586822in}{1.504980in}}%
\pgfpathcurveto{\pgfqpoint{5.591866in}{1.510024in}}{\pgfqpoint{5.594699in}{1.516865in}}{\pgfqpoint{5.594699in}{1.523998in}}%
\pgfpathcurveto{\pgfqpoint{5.594699in}{1.531131in}}{\pgfqpoint{5.591866in}{1.537973in}}{\pgfqpoint{5.586822in}{1.543016in}}%
\pgfpathcurveto{\pgfqpoint{5.581778in}{1.548060in}}{\pgfqpoint{5.574937in}{1.550894in}}{\pgfqpoint{5.567804in}{1.550894in}}%
\pgfpathcurveto{\pgfqpoint{5.560671in}{1.550894in}}{\pgfqpoint{5.553829in}{1.548060in}}{\pgfqpoint{5.548786in}{1.543016in}}%
\pgfpathcurveto{\pgfqpoint{5.543742in}{1.537973in}}{\pgfqpoint{5.540908in}{1.531131in}}{\pgfqpoint{5.540908in}{1.523998in}}%
\pgfpathcurveto{\pgfqpoint{5.540908in}{1.516865in}}{\pgfqpoint{5.543742in}{1.510024in}}{\pgfqpoint{5.548786in}{1.504980in}}%
\pgfpathcurveto{\pgfqpoint{5.553829in}{1.499936in}}{\pgfqpoint{5.560671in}{1.497103in}}{\pgfqpoint{5.567804in}{1.497103in}}%
\pgfpathclose%
\pgfusepath{stroke,fill}%
\end{pgfscope}%
\begin{pgfscope}%
\pgfpathrectangle{\pgfqpoint{4.985294in}{0.500000in}}{\pgfqpoint{1.764706in}{1.700000in}}%
\pgfusepath{clip}%
\pgfsetbuttcap%
\pgfsetroundjoin%
\definecolor{currentfill}{rgb}{0.857426,0.162258,0.276275}%
\pgfsetfillcolor{currentfill}%
\pgfsetlinewidth{0.311001pt}%
\definecolor{currentstroke}{rgb}{1.000000,1.000000,1.000000}%
\pgfsetstrokecolor{currentstroke}%
\pgfsetdash{}{0pt}%
\pgfpathmoveto{\pgfqpoint{5.736205in}{1.647760in}}%
\pgfpathcurveto{\pgfqpoint{5.743338in}{1.647760in}}{\pgfqpoint{5.750179in}{1.650594in}}{\pgfqpoint{5.755223in}{1.655638in}}%
\pgfpathcurveto{\pgfqpoint{5.760267in}{1.660682in}}{\pgfqpoint{5.763100in}{1.667523in}}{\pgfqpoint{5.763100in}{1.674656in}}%
\pgfpathcurveto{\pgfqpoint{5.763100in}{1.681789in}}{\pgfqpoint{5.760267in}{1.688631in}}{\pgfqpoint{5.755223in}{1.693674in}}%
\pgfpathcurveto{\pgfqpoint{5.750179in}{1.698718in}}{\pgfqpoint{5.743338in}{1.701552in}}{\pgfqpoint{5.736205in}{1.701552in}}%
\pgfpathcurveto{\pgfqpoint{5.729072in}{1.701552in}}{\pgfqpoint{5.722230in}{1.698718in}}{\pgfqpoint{5.717187in}{1.693674in}}%
\pgfpathcurveto{\pgfqpoint{5.712143in}{1.688631in}}{\pgfqpoint{5.709309in}{1.681789in}}{\pgfqpoint{5.709309in}{1.674656in}}%
\pgfpathcurveto{\pgfqpoint{5.709309in}{1.667523in}}{\pgfqpoint{5.712143in}{1.660682in}}{\pgfqpoint{5.717187in}{1.655638in}}%
\pgfpathcurveto{\pgfqpoint{5.722230in}{1.650594in}}{\pgfqpoint{5.729072in}{1.647760in}}{\pgfqpoint{5.736205in}{1.647760in}}%
\pgfpathclose%
\pgfusepath{stroke,fill}%
\end{pgfscope}%
\begin{pgfscope}%
\pgfpathrectangle{\pgfqpoint{4.985294in}{0.500000in}}{\pgfqpoint{1.764706in}{1.700000in}}%
\pgfusepath{clip}%
\pgfsetbuttcap%
\pgfsetroundjoin%
\definecolor{currentfill}{rgb}{0.974412,0.862387,0.780156}%
\pgfsetfillcolor{currentfill}%
\pgfsetlinewidth{0.311001pt}%
\definecolor{currentstroke}{rgb}{1.000000,1.000000,1.000000}%
\pgfsetstrokecolor{currentstroke}%
\pgfsetdash{}{0pt}%
\pgfpathmoveto{\pgfqpoint{6.249805in}{1.220242in}}%
\pgfpathcurveto{\pgfqpoint{6.256937in}{1.220242in}}{\pgfqpoint{6.263779in}{1.223076in}}{\pgfqpoint{6.268823in}{1.228120in}}%
\pgfpathcurveto{\pgfqpoint{6.273866in}{1.233163in}}{\pgfqpoint{6.276700in}{1.240005in}}{\pgfqpoint{6.276700in}{1.247138in}}%
\pgfpathcurveto{\pgfqpoint{6.276700in}{1.254271in}}{\pgfqpoint{6.273866in}{1.261112in}}{\pgfqpoint{6.268823in}{1.266156in}}%
\pgfpathcurveto{\pgfqpoint{6.263779in}{1.271200in}}{\pgfqpoint{6.256937in}{1.274034in}}{\pgfqpoint{6.249805in}{1.274034in}}%
\pgfpathcurveto{\pgfqpoint{6.242672in}{1.274034in}}{\pgfqpoint{6.235830in}{1.271200in}}{\pgfqpoint{6.230786in}{1.266156in}}%
\pgfpathcurveto{\pgfqpoint{6.225743in}{1.261112in}}{\pgfqpoint{6.222909in}{1.254271in}}{\pgfqpoint{6.222909in}{1.247138in}}%
\pgfpathcurveto{\pgfqpoint{6.222909in}{1.240005in}}{\pgfqpoint{6.225743in}{1.233163in}}{\pgfqpoint{6.230786in}{1.228120in}}%
\pgfpathcurveto{\pgfqpoint{6.235830in}{1.223076in}}{\pgfqpoint{6.242672in}{1.220242in}}{\pgfqpoint{6.249805in}{1.220242in}}%
\pgfpathclose%
\pgfusepath{stroke,fill}%
\end{pgfscope}%
\begin{pgfscope}%
\pgfpathrectangle{\pgfqpoint{4.985294in}{0.500000in}}{\pgfqpoint{1.764706in}{1.700000in}}%
\pgfusepath{clip}%
\pgfsetbuttcap%
\pgfsetroundjoin%
\definecolor{currentfill}{rgb}{0.979124,0.903132,0.839793}%
\pgfsetfillcolor{currentfill}%
\pgfsetlinewidth{0.311001pt}%
\definecolor{currentstroke}{rgb}{1.000000,1.000000,1.000000}%
\pgfsetstrokecolor{currentstroke}%
\pgfsetdash{}{0pt}%
\pgfpathmoveto{\pgfqpoint{5.406864in}{1.405437in}}%
\pgfpathcurveto{\pgfqpoint{5.413997in}{1.405437in}}{\pgfqpoint{5.420838in}{1.408271in}}{\pgfqpoint{5.425882in}{1.413314in}}%
\pgfpathcurveto{\pgfqpoint{5.430926in}{1.418358in}}{\pgfqpoint{5.433760in}{1.425200in}}{\pgfqpoint{5.433760in}{1.432333in}}%
\pgfpathcurveto{\pgfqpoint{5.433760in}{1.439465in}}{\pgfqpoint{5.430926in}{1.446307in}}{\pgfqpoint{5.425882in}{1.451351in}}%
\pgfpathcurveto{\pgfqpoint{5.420838in}{1.456394in}}{\pgfqpoint{5.413997in}{1.459228in}}{\pgfqpoint{5.406864in}{1.459228in}}%
\pgfpathcurveto{\pgfqpoint{5.399731in}{1.459228in}}{\pgfqpoint{5.392890in}{1.456394in}}{\pgfqpoint{5.387846in}{1.451351in}}%
\pgfpathcurveto{\pgfqpoint{5.382802in}{1.446307in}}{\pgfqpoint{5.379968in}{1.439465in}}{\pgfqpoint{5.379968in}{1.432333in}}%
\pgfpathcurveto{\pgfqpoint{5.379968in}{1.425200in}}{\pgfqpoint{5.382802in}{1.418358in}}{\pgfqpoint{5.387846in}{1.413314in}}%
\pgfpathcurveto{\pgfqpoint{5.392890in}{1.408271in}}{\pgfqpoint{5.399731in}{1.405437in}}{\pgfqpoint{5.406864in}{1.405437in}}%
\pgfpathclose%
\pgfusepath{stroke,fill}%
\end{pgfscope}%
\begin{pgfscope}%
\pgfpathrectangle{\pgfqpoint{4.985294in}{0.500000in}}{\pgfqpoint{1.764706in}{1.700000in}}%
\pgfusepath{clip}%
\pgfsetbuttcap%
\pgfsetroundjoin%
\definecolor{currentfill}{rgb}{0.095724,0.060501,0.162005}%
\pgfsetfillcolor{currentfill}%
\pgfsetlinewidth{0.311001pt}%
\definecolor{currentstroke}{rgb}{1.000000,1.000000,1.000000}%
\pgfsetstrokecolor{currentstroke}%
\pgfsetdash{}{0pt}%
\pgfpathmoveto{\pgfqpoint{5.778726in}{1.063839in}}%
\pgfpathcurveto{\pgfqpoint{5.785859in}{1.063839in}}{\pgfqpoint{5.792701in}{1.066673in}}{\pgfqpoint{5.797744in}{1.071717in}}%
\pgfpathcurveto{\pgfqpoint{5.802788in}{1.076760in}}{\pgfqpoint{5.805622in}{1.083602in}}{\pgfqpoint{5.805622in}{1.090735in}}%
\pgfpathcurveto{\pgfqpoint{5.805622in}{1.097868in}}{\pgfqpoint{5.802788in}{1.104709in}}{\pgfqpoint{5.797744in}{1.109753in}}%
\pgfpathcurveto{\pgfqpoint{5.792701in}{1.114797in}}{\pgfqpoint{5.785859in}{1.117630in}}{\pgfqpoint{5.778726in}{1.117630in}}%
\pgfpathcurveto{\pgfqpoint{5.771593in}{1.117630in}}{\pgfqpoint{5.764752in}{1.114797in}}{\pgfqpoint{5.759708in}{1.109753in}}%
\pgfpathcurveto{\pgfqpoint{5.754664in}{1.104709in}}{\pgfqpoint{5.751830in}{1.097868in}}{\pgfqpoint{5.751830in}{1.090735in}}%
\pgfpathcurveto{\pgfqpoint{5.751830in}{1.083602in}}{\pgfqpoint{5.754664in}{1.076760in}}{\pgfqpoint{5.759708in}{1.071717in}}%
\pgfpathcurveto{\pgfqpoint{5.764752in}{1.066673in}}{\pgfqpoint{5.771593in}{1.063839in}}{\pgfqpoint{5.778726in}{1.063839in}}%
\pgfpathclose%
\pgfusepath{stroke,fill}%
\end{pgfscope}%
\begin{pgfscope}%
\pgfpathrectangle{\pgfqpoint{4.985294in}{0.500000in}}{\pgfqpoint{1.764706in}{1.700000in}}%
\pgfusepath{clip}%
\pgfsetbuttcap%
\pgfsetroundjoin%
\definecolor{currentfill}{rgb}{0.938993,0.352507,0.254528}%
\pgfsetfillcolor{currentfill}%
\pgfsetlinewidth{0.311001pt}%
\definecolor{currentstroke}{rgb}{1.000000,1.000000,1.000000}%
\pgfsetstrokecolor{currentstroke}%
\pgfsetdash{}{0pt}%
\pgfpathmoveto{\pgfqpoint{5.488744in}{1.799366in}}%
\pgfpathcurveto{\pgfqpoint{5.495877in}{1.799366in}}{\pgfqpoint{5.502719in}{1.802200in}}{\pgfqpoint{5.507762in}{1.807243in}}%
\pgfpathcurveto{\pgfqpoint{5.512806in}{1.812287in}}{\pgfqpoint{5.515640in}{1.819128in}}{\pgfqpoint{5.515640in}{1.826261in}}%
\pgfpathcurveto{\pgfqpoint{5.515640in}{1.833394in}}{\pgfqpoint{5.512806in}{1.840236in}}{\pgfqpoint{5.507762in}{1.845279in}}%
\pgfpathcurveto{\pgfqpoint{5.502719in}{1.850323in}}{\pgfqpoint{5.495877in}{1.853157in}}{\pgfqpoint{5.488744in}{1.853157in}}%
\pgfpathcurveto{\pgfqpoint{5.481611in}{1.853157in}}{\pgfqpoint{5.474770in}{1.850323in}}{\pgfqpoint{5.469726in}{1.845279in}}%
\pgfpathcurveto{\pgfqpoint{5.464682in}{1.840236in}}{\pgfqpoint{5.461848in}{1.833394in}}{\pgfqpoint{5.461848in}{1.826261in}}%
\pgfpathcurveto{\pgfqpoint{5.461848in}{1.819128in}}{\pgfqpoint{5.464682in}{1.812287in}}{\pgfqpoint{5.469726in}{1.807243in}}%
\pgfpathcurveto{\pgfqpoint{5.474770in}{1.802200in}}{\pgfqpoint{5.481611in}{1.799366in}}{\pgfqpoint{5.488744in}{1.799366in}}%
\pgfpathclose%
\pgfusepath{stroke,fill}%
\end{pgfscope}%
\begin{pgfscope}%
\pgfpathrectangle{\pgfqpoint{4.985294in}{0.500000in}}{\pgfqpoint{1.764706in}{1.700000in}}%
\pgfusepath{clip}%
\pgfsetbuttcap%
\pgfsetroundjoin%
\definecolor{currentfill}{rgb}{0.950017,0.427714,0.292447}%
\pgfsetfillcolor{currentfill}%
\pgfsetlinewidth{0.311001pt}%
\definecolor{currentstroke}{rgb}{1.000000,1.000000,1.000000}%
\pgfsetstrokecolor{currentstroke}%
\pgfsetdash{}{0pt}%
\pgfpathmoveto{\pgfqpoint{6.203933in}{1.794108in}}%
\pgfpathcurveto{\pgfqpoint{6.211066in}{1.794108in}}{\pgfqpoint{6.217908in}{1.796942in}}{\pgfqpoint{6.222951in}{1.801985in}}%
\pgfpathcurveto{\pgfqpoint{6.227995in}{1.807029in}}{\pgfqpoint{6.230829in}{1.813871in}}{\pgfqpoint{6.230829in}{1.821003in}}%
\pgfpathcurveto{\pgfqpoint{6.230829in}{1.828136in}}{\pgfqpoint{6.227995in}{1.834978in}}{\pgfqpoint{6.222951in}{1.840022in}}%
\pgfpathcurveto{\pgfqpoint{6.217908in}{1.845065in}}{\pgfqpoint{6.211066in}{1.847899in}}{\pgfqpoint{6.203933in}{1.847899in}}%
\pgfpathcurveto{\pgfqpoint{6.196800in}{1.847899in}}{\pgfqpoint{6.189959in}{1.845065in}}{\pgfqpoint{6.184915in}{1.840022in}}%
\pgfpathcurveto{\pgfqpoint{6.179872in}{1.834978in}}{\pgfqpoint{6.177038in}{1.828136in}}{\pgfqpoint{6.177038in}{1.821003in}}%
\pgfpathcurveto{\pgfqpoint{6.177038in}{1.813871in}}{\pgfqpoint{6.179872in}{1.807029in}}{\pgfqpoint{6.184915in}{1.801985in}}%
\pgfpathcurveto{\pgfqpoint{6.189959in}{1.796942in}}{\pgfqpoint{6.196800in}{1.794108in}}{\pgfqpoint{6.203933in}{1.794108in}}%
\pgfpathclose%
\pgfusepath{stroke,fill}%
\end{pgfscope}%
\begin{pgfscope}%
\pgfpathrectangle{\pgfqpoint{4.985294in}{0.500000in}}{\pgfqpoint{1.764706in}{1.700000in}}%
\pgfusepath{clip}%
\pgfsetbuttcap%
\pgfsetroundjoin%
\definecolor{currentfill}{rgb}{0.979891,0.908948,0.848279}%
\pgfsetfillcolor{currentfill}%
\pgfsetlinewidth{0.311001pt}%
\definecolor{currentstroke}{rgb}{1.000000,1.000000,1.000000}%
\pgfsetstrokecolor{currentstroke}%
\pgfsetdash{}{0pt}%
\pgfpathmoveto{\pgfqpoint{5.435037in}{1.336549in}}%
\pgfpathcurveto{\pgfqpoint{5.442170in}{1.336549in}}{\pgfqpoint{5.449011in}{1.339383in}}{\pgfqpoint{5.454055in}{1.344427in}}%
\pgfpathcurveto{\pgfqpoint{5.459099in}{1.349471in}}{\pgfqpoint{5.461933in}{1.356312in}}{\pgfqpoint{5.461933in}{1.363445in}}%
\pgfpathcurveto{\pgfqpoint{5.461933in}{1.370578in}}{\pgfqpoint{5.459099in}{1.377420in}}{\pgfqpoint{5.454055in}{1.382463in}}%
\pgfpathcurveto{\pgfqpoint{5.449011in}{1.387507in}}{\pgfqpoint{5.442170in}{1.390341in}}{\pgfqpoint{5.435037in}{1.390341in}}%
\pgfpathcurveto{\pgfqpoint{5.427904in}{1.390341in}}{\pgfqpoint{5.421062in}{1.387507in}}{\pgfqpoint{5.416019in}{1.382463in}}%
\pgfpathcurveto{\pgfqpoint{5.410975in}{1.377420in}}{\pgfqpoint{5.408141in}{1.370578in}}{\pgfqpoint{5.408141in}{1.363445in}}%
\pgfpathcurveto{\pgfqpoint{5.408141in}{1.356312in}}{\pgfqpoint{5.410975in}{1.349471in}}{\pgfqpoint{5.416019in}{1.344427in}}%
\pgfpathcurveto{\pgfqpoint{5.421062in}{1.339383in}}{\pgfqpoint{5.427904in}{1.336549in}}{\pgfqpoint{5.435037in}{1.336549in}}%
\pgfpathclose%
\pgfusepath{stroke,fill}%
\end{pgfscope}%
\begin{pgfscope}%
\pgfpathrectangle{\pgfqpoint{4.985294in}{0.500000in}}{\pgfqpoint{1.764706in}{1.700000in}}%
\pgfusepath{clip}%
\pgfsetbuttcap%
\pgfsetroundjoin%
\definecolor{currentfill}{rgb}{0.965042,0.701564,0.552889}%
\pgfsetfillcolor{currentfill}%
\pgfsetlinewidth{0.311001pt}%
\definecolor{currentstroke}{rgb}{1.000000,1.000000,1.000000}%
\pgfsetstrokecolor{currentstroke}%
\pgfsetdash{}{0pt}%
\pgfpathmoveto{\pgfqpoint{6.123812in}{1.608981in}}%
\pgfpathcurveto{\pgfqpoint{6.130945in}{1.608981in}}{\pgfqpoint{6.137786in}{1.611815in}}{\pgfqpoint{6.142830in}{1.616859in}}%
\pgfpathcurveto{\pgfqpoint{6.147874in}{1.621903in}}{\pgfqpoint{6.150708in}{1.628744in}}{\pgfqpoint{6.150708in}{1.635877in}}%
\pgfpathcurveto{\pgfqpoint{6.150708in}{1.643010in}}{\pgfqpoint{6.147874in}{1.649852in}}{\pgfqpoint{6.142830in}{1.654895in}}%
\pgfpathcurveto{\pgfqpoint{6.137786in}{1.659939in}}{\pgfqpoint{6.130945in}{1.662773in}}{\pgfqpoint{6.123812in}{1.662773in}}%
\pgfpathcurveto{\pgfqpoint{6.116679in}{1.662773in}}{\pgfqpoint{6.109837in}{1.659939in}}{\pgfqpoint{6.104794in}{1.654895in}}%
\pgfpathcurveto{\pgfqpoint{6.099750in}{1.649852in}}{\pgfqpoint{6.096916in}{1.643010in}}{\pgfqpoint{6.096916in}{1.635877in}}%
\pgfpathcurveto{\pgfqpoint{6.096916in}{1.628744in}}{\pgfqpoint{6.099750in}{1.621903in}}{\pgfqpoint{6.104794in}{1.616859in}}%
\pgfpathcurveto{\pgfqpoint{6.109837in}{1.611815in}}{\pgfqpoint{6.116679in}{1.608981in}}{\pgfqpoint{6.123812in}{1.608981in}}%
\pgfpathclose%
\pgfusepath{stroke,fill}%
\end{pgfscope}%
\begin{pgfscope}%
\pgfpathrectangle{\pgfqpoint{4.985294in}{0.500000in}}{\pgfqpoint{1.764706in}{1.700000in}}%
\pgfusepath{clip}%
\pgfsetbuttcap%
\pgfsetroundjoin%
\definecolor{currentfill}{rgb}{0.973271,0.850724,0.762998}%
\pgfsetfillcolor{currentfill}%
\pgfsetlinewidth{0.311001pt}%
\definecolor{currentstroke}{rgb}{1.000000,1.000000,1.000000}%
\pgfsetstrokecolor{currentstroke}%
\pgfsetdash{}{0pt}%
\pgfpathmoveto{\pgfqpoint{5.365487in}{1.348693in}}%
\pgfpathcurveto{\pgfqpoint{5.372619in}{1.348693in}}{\pgfqpoint{5.379461in}{1.351527in}}{\pgfqpoint{5.384505in}{1.356571in}}%
\pgfpathcurveto{\pgfqpoint{5.389548in}{1.361615in}}{\pgfqpoint{5.392382in}{1.368456in}}{\pgfqpoint{5.392382in}{1.375589in}}%
\pgfpathcurveto{\pgfqpoint{5.392382in}{1.382722in}}{\pgfqpoint{5.389548in}{1.389564in}}{\pgfqpoint{5.384505in}{1.394607in}}%
\pgfpathcurveto{\pgfqpoint{5.379461in}{1.399651in}}{\pgfqpoint{5.372619in}{1.402485in}}{\pgfqpoint{5.365487in}{1.402485in}}%
\pgfpathcurveto{\pgfqpoint{5.358354in}{1.402485in}}{\pgfqpoint{5.351512in}{1.399651in}}{\pgfqpoint{5.346468in}{1.394607in}}%
\pgfpathcurveto{\pgfqpoint{5.341425in}{1.389564in}}{\pgfqpoint{5.338591in}{1.382722in}}{\pgfqpoint{5.338591in}{1.375589in}}%
\pgfpathcurveto{\pgfqpoint{5.338591in}{1.368456in}}{\pgfqpoint{5.341425in}{1.361615in}}{\pgfqpoint{5.346468in}{1.356571in}}%
\pgfpathcurveto{\pgfqpoint{5.351512in}{1.351527in}}{\pgfqpoint{5.358354in}{1.348693in}}{\pgfqpoint{5.365487in}{1.348693in}}%
\pgfpathclose%
\pgfusepath{stroke,fill}%
\end{pgfscope}%
\begin{pgfscope}%
\pgfpathrectangle{\pgfqpoint{4.985294in}{0.500000in}}{\pgfqpoint{1.764706in}{1.700000in}}%
\pgfusepath{clip}%
\pgfsetbuttcap%
\pgfsetroundjoin%
\definecolor{currentfill}{rgb}{0.966328,0.750560,0.616961}%
\pgfsetfillcolor{currentfill}%
\pgfsetlinewidth{0.311001pt}%
\definecolor{currentstroke}{rgb}{1.000000,1.000000,1.000000}%
\pgfsetstrokecolor{currentstroke}%
\pgfsetdash{}{0pt}%
\pgfpathmoveto{\pgfqpoint{6.225802in}{0.949354in}}%
\pgfpathcurveto{\pgfqpoint{6.232935in}{0.949354in}}{\pgfqpoint{6.239776in}{0.952188in}}{\pgfqpoint{6.244820in}{0.957231in}}%
\pgfpathcurveto{\pgfqpoint{6.249864in}{0.962275in}}{\pgfqpoint{6.252697in}{0.969117in}}{\pgfqpoint{6.252697in}{0.976250in}}%
\pgfpathcurveto{\pgfqpoint{6.252697in}{0.983382in}}{\pgfqpoint{6.249864in}{0.990224in}}{\pgfqpoint{6.244820in}{0.995268in}}%
\pgfpathcurveto{\pgfqpoint{6.239776in}{1.000311in}}{\pgfqpoint{6.232935in}{1.003145in}}{\pgfqpoint{6.225802in}{1.003145in}}%
\pgfpathcurveto{\pgfqpoint{6.218669in}{1.003145in}}{\pgfqpoint{6.211827in}{1.000311in}}{\pgfqpoint{6.206784in}{0.995268in}}%
\pgfpathcurveto{\pgfqpoint{6.201740in}{0.990224in}}{\pgfqpoint{6.198906in}{0.983382in}}{\pgfqpoint{6.198906in}{0.976250in}}%
\pgfpathcurveto{\pgfqpoint{6.198906in}{0.969117in}}{\pgfqpoint{6.201740in}{0.962275in}}{\pgfqpoint{6.206784in}{0.957231in}}%
\pgfpathcurveto{\pgfqpoint{6.211827in}{0.952188in}}{\pgfqpoint{6.218669in}{0.949354in}}{\pgfqpoint{6.225802in}{0.949354in}}%
\pgfpathclose%
\pgfusepath{stroke,fill}%
\end{pgfscope}%
\begin{pgfscope}%
\pgfpathrectangle{\pgfqpoint{4.985294in}{0.500000in}}{\pgfqpoint{1.764706in}{1.700000in}}%
\pgfusepath{clip}%
\pgfsetbuttcap%
\pgfsetroundjoin%
\definecolor{currentfill}{rgb}{0.967735,0.780441,0.659127}%
\pgfsetfillcolor{currentfill}%
\pgfsetlinewidth{0.311001pt}%
\definecolor{currentstroke}{rgb}{1.000000,1.000000,1.000000}%
\pgfsetstrokecolor{currentstroke}%
\pgfsetdash{}{0pt}%
\pgfpathmoveto{\pgfqpoint{6.169816in}{1.584183in}}%
\pgfpathcurveto{\pgfqpoint{6.176949in}{1.584183in}}{\pgfqpoint{6.183790in}{1.587017in}}{\pgfqpoint{6.188834in}{1.592060in}}%
\pgfpathcurveto{\pgfqpoint{6.193878in}{1.597104in}}{\pgfqpoint{6.196712in}{1.603946in}}{\pgfqpoint{6.196712in}{1.611078in}}%
\pgfpathcurveto{\pgfqpoint{6.196712in}{1.618211in}}{\pgfqpoint{6.193878in}{1.625053in}}{\pgfqpoint{6.188834in}{1.630097in}}%
\pgfpathcurveto{\pgfqpoint{6.183790in}{1.635140in}}{\pgfqpoint{6.176949in}{1.637974in}}{\pgfqpoint{6.169816in}{1.637974in}}%
\pgfpathcurveto{\pgfqpoint{6.162683in}{1.637974in}}{\pgfqpoint{6.155842in}{1.635140in}}{\pgfqpoint{6.150798in}{1.630097in}}%
\pgfpathcurveto{\pgfqpoint{6.145754in}{1.625053in}}{\pgfqpoint{6.142920in}{1.618211in}}{\pgfqpoint{6.142920in}{1.611078in}}%
\pgfpathcurveto{\pgfqpoint{6.142920in}{1.603946in}}{\pgfqpoint{6.145754in}{1.597104in}}{\pgfqpoint{6.150798in}{1.592060in}}%
\pgfpathcurveto{\pgfqpoint{6.155842in}{1.587017in}}{\pgfqpoint{6.162683in}{1.584183in}}{\pgfqpoint{6.169816in}{1.584183in}}%
\pgfpathclose%
\pgfusepath{stroke,fill}%
\end{pgfscope}%
\begin{pgfscope}%
\pgfpathrectangle{\pgfqpoint{4.985294in}{0.500000in}}{\pgfqpoint{1.764706in}{1.700000in}}%
\pgfusepath{clip}%
\pgfsetbuttcap%
\pgfsetroundjoin%
\definecolor{currentfill}{rgb}{0.969803,0.809811,0.702523}%
\pgfsetfillcolor{currentfill}%
\pgfsetlinewidth{0.311001pt}%
\definecolor{currentstroke}{rgb}{1.000000,1.000000,1.000000}%
\pgfsetstrokecolor{currentstroke}%
\pgfsetdash{}{0pt}%
\pgfpathmoveto{\pgfqpoint{6.370479in}{1.182526in}}%
\pgfpathcurveto{\pgfqpoint{6.377612in}{1.182526in}}{\pgfqpoint{6.384453in}{1.185360in}}{\pgfqpoint{6.389497in}{1.190404in}}%
\pgfpathcurveto{\pgfqpoint{6.394541in}{1.195448in}}{\pgfqpoint{6.397375in}{1.202289in}}{\pgfqpoint{6.397375in}{1.209422in}}%
\pgfpathcurveto{\pgfqpoint{6.397375in}{1.216555in}}{\pgfqpoint{6.394541in}{1.223396in}}{\pgfqpoint{6.389497in}{1.228440in}}%
\pgfpathcurveto{\pgfqpoint{6.384453in}{1.233484in}}{\pgfqpoint{6.377612in}{1.236318in}}{\pgfqpoint{6.370479in}{1.236318in}}%
\pgfpathcurveto{\pgfqpoint{6.363346in}{1.236318in}}{\pgfqpoint{6.356504in}{1.233484in}}{\pgfqpoint{6.351461in}{1.228440in}}%
\pgfpathcurveto{\pgfqpoint{6.346417in}{1.223396in}}{\pgfqpoint{6.343583in}{1.216555in}}{\pgfqpoint{6.343583in}{1.209422in}}%
\pgfpathcurveto{\pgfqpoint{6.343583in}{1.202289in}}{\pgfqpoint{6.346417in}{1.195448in}}{\pgfqpoint{6.351461in}{1.190404in}}%
\pgfpathcurveto{\pgfqpoint{6.356504in}{1.185360in}}{\pgfqpoint{6.363346in}{1.182526in}}{\pgfqpoint{6.370479in}{1.182526in}}%
\pgfpathclose%
\pgfusepath{stroke,fill}%
\end{pgfscope}%
\begin{pgfscope}%
\pgfpathrectangle{\pgfqpoint{4.985294in}{0.500000in}}{\pgfqpoint{1.764706in}{1.700000in}}%
\pgfusepath{clip}%
\pgfsetbuttcap%
\pgfsetroundjoin%
\definecolor{currentfill}{rgb}{0.959229,0.533075,0.374889}%
\pgfsetfillcolor{currentfill}%
\pgfsetlinewidth{0.311001pt}%
\definecolor{currentstroke}{rgb}{1.000000,1.000000,1.000000}%
\pgfsetstrokecolor{currentstroke}%
\pgfsetdash{}{0pt}%
\pgfpathmoveto{\pgfqpoint{6.438198in}{1.346844in}}%
\pgfpathcurveto{\pgfqpoint{6.445331in}{1.346844in}}{\pgfqpoint{6.452173in}{1.349678in}}{\pgfqpoint{6.457216in}{1.354722in}}%
\pgfpathcurveto{\pgfqpoint{6.462260in}{1.359765in}}{\pgfqpoint{6.465094in}{1.366607in}}{\pgfqpoint{6.465094in}{1.373740in}}%
\pgfpathcurveto{\pgfqpoint{6.465094in}{1.380873in}}{\pgfqpoint{6.462260in}{1.387714in}}{\pgfqpoint{6.457216in}{1.392758in}}%
\pgfpathcurveto{\pgfqpoint{6.452173in}{1.397801in}}{\pgfqpoint{6.445331in}{1.400635in}}{\pgfqpoint{6.438198in}{1.400635in}}%
\pgfpathcurveto{\pgfqpoint{6.431065in}{1.400635in}}{\pgfqpoint{6.424224in}{1.397801in}}{\pgfqpoint{6.419180in}{1.392758in}}%
\pgfpathcurveto{\pgfqpoint{6.414136in}{1.387714in}}{\pgfqpoint{6.411302in}{1.380873in}}{\pgfqpoint{6.411302in}{1.373740in}}%
\pgfpathcurveto{\pgfqpoint{6.411302in}{1.366607in}}{\pgfqpoint{6.414136in}{1.359765in}}{\pgfqpoint{6.419180in}{1.354722in}}%
\pgfpathcurveto{\pgfqpoint{6.424224in}{1.349678in}}{\pgfqpoint{6.431065in}{1.346844in}}{\pgfqpoint{6.438198in}{1.346844in}}%
\pgfpathclose%
\pgfusepath{stroke,fill}%
\end{pgfscope}%
\begin{pgfscope}%
\pgfpathrectangle{\pgfqpoint{4.985294in}{0.500000in}}{\pgfqpoint{1.764706in}{1.700000in}}%
\pgfusepath{clip}%
\pgfsetbuttcap%
\pgfsetroundjoin%
\definecolor{currentfill}{rgb}{0.964799,0.689101,0.537560}%
\pgfsetfillcolor{currentfill}%
\pgfsetlinewidth{0.311001pt}%
\definecolor{currentstroke}{rgb}{1.000000,1.000000,1.000000}%
\pgfsetstrokecolor{currentstroke}%
\pgfsetdash{}{0pt}%
\pgfpathmoveto{\pgfqpoint{6.148998in}{1.085106in}}%
\pgfpathcurveto{\pgfqpoint{6.156131in}{1.085106in}}{\pgfqpoint{6.162972in}{1.087939in}}{\pgfqpoint{6.168016in}{1.092983in}}%
\pgfpathcurveto{\pgfqpoint{6.173060in}{1.098027in}}{\pgfqpoint{6.175894in}{1.104868in}}{\pgfqpoint{6.175894in}{1.112001in}}%
\pgfpathcurveto{\pgfqpoint{6.175894in}{1.119134in}}{\pgfqpoint{6.173060in}{1.125976in}}{\pgfqpoint{6.168016in}{1.131019in}}%
\pgfpathcurveto{\pgfqpoint{6.162972in}{1.136063in}}{\pgfqpoint{6.156131in}{1.138897in}}{\pgfqpoint{6.148998in}{1.138897in}}%
\pgfpathcurveto{\pgfqpoint{6.141865in}{1.138897in}}{\pgfqpoint{6.135023in}{1.136063in}}{\pgfqpoint{6.129980in}{1.131019in}}%
\pgfpathcurveto{\pgfqpoint{6.124936in}{1.125976in}}{\pgfqpoint{6.122102in}{1.119134in}}{\pgfqpoint{6.122102in}{1.112001in}}%
\pgfpathcurveto{\pgfqpoint{6.122102in}{1.104868in}}{\pgfqpoint{6.124936in}{1.098027in}}{\pgfqpoint{6.129980in}{1.092983in}}%
\pgfpathcurveto{\pgfqpoint{6.135023in}{1.087939in}}{\pgfqpoint{6.141865in}{1.085106in}}{\pgfqpoint{6.148998in}{1.085106in}}%
\pgfpathclose%
\pgfusepath{stroke,fill}%
\end{pgfscope}%
\begin{pgfscope}%
\pgfpathrectangle{\pgfqpoint{4.985294in}{0.500000in}}{\pgfqpoint{1.764706in}{1.700000in}}%
\pgfusepath{clip}%
\pgfsetbuttcap%
\pgfsetroundjoin%
\definecolor{currentfill}{rgb}{0.972726,0.844889,0.754401}%
\pgfsetfillcolor{currentfill}%
\pgfsetlinewidth{0.311001pt}%
\definecolor{currentstroke}{rgb}{1.000000,1.000000,1.000000}%
\pgfsetstrokecolor{currentstroke}%
\pgfsetdash{}{0pt}%
\pgfpathmoveto{\pgfqpoint{6.252148in}{1.293634in}}%
\pgfpathcurveto{\pgfqpoint{6.259281in}{1.293634in}}{\pgfqpoint{6.266123in}{1.296468in}}{\pgfqpoint{6.271167in}{1.301511in}}%
\pgfpathcurveto{\pgfqpoint{6.276210in}{1.306555in}}{\pgfqpoint{6.279044in}{1.313397in}}{\pgfqpoint{6.279044in}{1.320529in}}%
\pgfpathcurveto{\pgfqpoint{6.279044in}{1.327662in}}{\pgfqpoint{6.276210in}{1.334504in}}{\pgfqpoint{6.271167in}{1.339548in}}%
\pgfpathcurveto{\pgfqpoint{6.266123in}{1.344591in}}{\pgfqpoint{6.259281in}{1.347425in}}{\pgfqpoint{6.252148in}{1.347425in}}%
\pgfpathcurveto{\pgfqpoint{6.245016in}{1.347425in}}{\pgfqpoint{6.238174in}{1.344591in}}{\pgfqpoint{6.233130in}{1.339548in}}%
\pgfpathcurveto{\pgfqpoint{6.228087in}{1.334504in}}{\pgfqpoint{6.225253in}{1.327662in}}{\pgfqpoint{6.225253in}{1.320529in}}%
\pgfpathcurveto{\pgfqpoint{6.225253in}{1.313397in}}{\pgfqpoint{6.228087in}{1.306555in}}{\pgfqpoint{6.233130in}{1.301511in}}%
\pgfpathcurveto{\pgfqpoint{6.238174in}{1.296468in}}{\pgfqpoint{6.245016in}{1.293634in}}{\pgfqpoint{6.252148in}{1.293634in}}%
\pgfpathclose%
\pgfusepath{stroke,fill}%
\end{pgfscope}%
\begin{pgfscope}%
\pgfpathrectangle{\pgfqpoint{4.985294in}{0.500000in}}{\pgfqpoint{1.764706in}{1.700000in}}%
\pgfusepath{clip}%
\pgfsetbuttcap%
\pgfsetroundjoin%
\definecolor{currentfill}{rgb}{0.968931,0.798091,0.685123}%
\pgfsetfillcolor{currentfill}%
\pgfsetlinewidth{0.311001pt}%
\definecolor{currentstroke}{rgb}{1.000000,1.000000,1.000000}%
\pgfsetstrokecolor{currentstroke}%
\pgfsetdash{}{0pt}%
\pgfpathmoveto{\pgfqpoint{5.470246in}{0.976279in}}%
\pgfpathcurveto{\pgfqpoint{5.477379in}{0.976279in}}{\pgfqpoint{5.484220in}{0.979113in}}{\pgfqpoint{5.489264in}{0.984156in}}%
\pgfpathcurveto{\pgfqpoint{5.494308in}{0.989200in}}{\pgfqpoint{5.497141in}{0.996041in}}{\pgfqpoint{5.497141in}{1.003174in}}%
\pgfpathcurveto{\pgfqpoint{5.497141in}{1.010307in}}{\pgfqpoint{5.494308in}{1.017149in}}{\pgfqpoint{5.489264in}{1.022192in}}%
\pgfpathcurveto{\pgfqpoint{5.484220in}{1.027236in}}{\pgfqpoint{5.477379in}{1.030070in}}{\pgfqpoint{5.470246in}{1.030070in}}%
\pgfpathcurveto{\pgfqpoint{5.463113in}{1.030070in}}{\pgfqpoint{5.456271in}{1.027236in}}{\pgfqpoint{5.451228in}{1.022192in}}%
\pgfpathcurveto{\pgfqpoint{5.446184in}{1.017149in}}{\pgfqpoint{5.443350in}{1.010307in}}{\pgfqpoint{5.443350in}{1.003174in}}%
\pgfpathcurveto{\pgfqpoint{5.443350in}{0.996041in}}{\pgfqpoint{5.446184in}{0.989200in}}{\pgfqpoint{5.451228in}{0.984156in}}%
\pgfpathcurveto{\pgfqpoint{5.456271in}{0.979113in}}{\pgfqpoint{5.463113in}{0.976279in}}{\pgfqpoint{5.470246in}{0.976279in}}%
\pgfpathclose%
\pgfusepath{stroke,fill}%
\end{pgfscope}%
\begin{pgfscope}%
\pgfpathrectangle{\pgfqpoint{4.985294in}{0.500000in}}{\pgfqpoint{1.764706in}{1.700000in}}%
\pgfusepath{clip}%
\pgfsetbuttcap%
\pgfsetroundjoin%
\definecolor{currentfill}{rgb}{0.979891,0.908948,0.848279}%
\pgfsetfillcolor{currentfill}%
\pgfsetlinewidth{0.311001pt}%
\definecolor{currentstroke}{rgb}{1.000000,1.000000,1.000000}%
\pgfsetstrokecolor{currentstroke}%
\pgfsetdash{}{0pt}%
\pgfpathmoveto{\pgfqpoint{6.294718in}{1.302453in}}%
\pgfpathcurveto{\pgfqpoint{6.301851in}{1.302453in}}{\pgfqpoint{6.308692in}{1.305287in}}{\pgfqpoint{6.313736in}{1.310331in}}%
\pgfpathcurveto{\pgfqpoint{6.318780in}{1.315374in}}{\pgfqpoint{6.321614in}{1.322216in}}{\pgfqpoint{6.321614in}{1.329349in}}%
\pgfpathcurveto{\pgfqpoint{6.321614in}{1.336482in}}{\pgfqpoint{6.318780in}{1.343323in}}{\pgfqpoint{6.313736in}{1.348367in}}%
\pgfpathcurveto{\pgfqpoint{6.308692in}{1.353411in}}{\pgfqpoint{6.301851in}{1.356245in}}{\pgfqpoint{6.294718in}{1.356245in}}%
\pgfpathcurveto{\pgfqpoint{6.287585in}{1.356245in}}{\pgfqpoint{6.280743in}{1.353411in}}{\pgfqpoint{6.275700in}{1.348367in}}%
\pgfpathcurveto{\pgfqpoint{6.270656in}{1.343323in}}{\pgfqpoint{6.267822in}{1.336482in}}{\pgfqpoint{6.267822in}{1.329349in}}%
\pgfpathcurveto{\pgfqpoint{6.267822in}{1.322216in}}{\pgfqpoint{6.270656in}{1.315374in}}{\pgfqpoint{6.275700in}{1.310331in}}%
\pgfpathcurveto{\pgfqpoint{6.280743in}{1.305287in}}{\pgfqpoint{6.287585in}{1.302453in}}{\pgfqpoint{6.294718in}{1.302453in}}%
\pgfpathclose%
\pgfusepath{stroke,fill}%
\end{pgfscope}%
\begin{pgfscope}%
\pgfpathrectangle{\pgfqpoint{4.985294in}{0.500000in}}{\pgfqpoint{1.764706in}{1.700000in}}%
\pgfusepath{clip}%
\pgfsetbuttcap%
\pgfsetroundjoin%
\definecolor{currentfill}{rgb}{0.975644,0.874038,0.797253}%
\pgfsetfillcolor{currentfill}%
\pgfsetlinewidth{0.311001pt}%
\definecolor{currentstroke}{rgb}{1.000000,1.000000,1.000000}%
\pgfsetstrokecolor{currentstroke}%
\pgfsetdash{}{0pt}%
\pgfpathmoveto{\pgfqpoint{6.235346in}{1.108904in}}%
\pgfpathcurveto{\pgfqpoint{6.242479in}{1.108904in}}{\pgfqpoint{6.249320in}{1.111737in}}{\pgfqpoint{6.254364in}{1.116781in}}%
\pgfpathcurveto{\pgfqpoint{6.259408in}{1.121825in}}{\pgfqpoint{6.262242in}{1.128666in}}{\pgfqpoint{6.262242in}{1.135799in}}%
\pgfpathcurveto{\pgfqpoint{6.262242in}{1.142932in}}{\pgfqpoint{6.259408in}{1.149774in}}{\pgfqpoint{6.254364in}{1.154817in}}%
\pgfpathcurveto{\pgfqpoint{6.249320in}{1.159861in}}{\pgfqpoint{6.242479in}{1.162695in}}{\pgfqpoint{6.235346in}{1.162695in}}%
\pgfpathcurveto{\pgfqpoint{6.228213in}{1.162695in}}{\pgfqpoint{6.221371in}{1.159861in}}{\pgfqpoint{6.216328in}{1.154817in}}%
\pgfpathcurveto{\pgfqpoint{6.211284in}{1.149774in}}{\pgfqpoint{6.208450in}{1.142932in}}{\pgfqpoint{6.208450in}{1.135799in}}%
\pgfpathcurveto{\pgfqpoint{6.208450in}{1.128666in}}{\pgfqpoint{6.211284in}{1.121825in}}{\pgfqpoint{6.216328in}{1.116781in}}%
\pgfpathcurveto{\pgfqpoint{6.221371in}{1.111737in}}{\pgfqpoint{6.228213in}{1.108904in}}{\pgfqpoint{6.235346in}{1.108904in}}%
\pgfpathclose%
\pgfusepath{stroke,fill}%
\end{pgfscope}%
\begin{pgfscope}%
\pgfpathrectangle{\pgfqpoint{4.985294in}{0.500000in}}{\pgfqpoint{1.764706in}{1.700000in}}%
\pgfusepath{clip}%
\pgfsetbuttcap%
\pgfsetroundjoin%
\definecolor{currentfill}{rgb}{0.970718,0.821518,0.719872}%
\pgfsetfillcolor{currentfill}%
\pgfsetlinewidth{0.311001pt}%
\definecolor{currentstroke}{rgb}{1.000000,1.000000,1.000000}%
\pgfsetstrokecolor{currentstroke}%
\pgfsetdash{}{0pt}%
\pgfpathmoveto{\pgfqpoint{5.492521in}{1.210545in}}%
\pgfpathcurveto{\pgfqpoint{5.499654in}{1.210545in}}{\pgfqpoint{5.506495in}{1.213378in}}{\pgfqpoint{5.511539in}{1.218422in}}%
\pgfpathcurveto{\pgfqpoint{5.516583in}{1.223466in}}{\pgfqpoint{5.519417in}{1.230307in}}{\pgfqpoint{5.519417in}{1.237440in}}%
\pgfpathcurveto{\pgfqpoint{5.519417in}{1.244573in}}{\pgfqpoint{5.516583in}{1.251415in}}{\pgfqpoint{5.511539in}{1.256458in}}%
\pgfpathcurveto{\pgfqpoint{5.506495in}{1.261502in}}{\pgfqpoint{5.499654in}{1.264336in}}{\pgfqpoint{5.492521in}{1.264336in}}%
\pgfpathcurveto{\pgfqpoint{5.485388in}{1.264336in}}{\pgfqpoint{5.478546in}{1.261502in}}{\pgfqpoint{5.473503in}{1.256458in}}%
\pgfpathcurveto{\pgfqpoint{5.468459in}{1.251415in}}{\pgfqpoint{5.465625in}{1.244573in}}{\pgfqpoint{5.465625in}{1.237440in}}%
\pgfpathcurveto{\pgfqpoint{5.465625in}{1.230307in}}{\pgfqpoint{5.468459in}{1.223466in}}{\pgfqpoint{5.473503in}{1.218422in}}%
\pgfpathcurveto{\pgfqpoint{5.478546in}{1.213378in}}{\pgfqpoint{5.485388in}{1.210545in}}{\pgfqpoint{5.492521in}{1.210545in}}%
\pgfpathclose%
\pgfusepath{stroke,fill}%
\end{pgfscope}%
\begin{pgfscope}%
\pgfpathrectangle{\pgfqpoint{4.985294in}{0.500000in}}{\pgfqpoint{1.764706in}{1.700000in}}%
\pgfusepath{clip}%
\pgfsetbuttcap%
\pgfsetroundjoin%
\definecolor{currentfill}{rgb}{0.974412,0.862387,0.780156}%
\pgfsetfillcolor{currentfill}%
\pgfsetlinewidth{0.311001pt}%
\definecolor{currentstroke}{rgb}{1.000000,1.000000,1.000000}%
\pgfsetstrokecolor{currentstroke}%
\pgfsetdash{}{0pt}%
\pgfpathmoveto{\pgfqpoint{6.247560in}{1.521160in}}%
\pgfpathcurveto{\pgfqpoint{6.254693in}{1.521160in}}{\pgfqpoint{6.261535in}{1.523994in}}{\pgfqpoint{6.266579in}{1.529038in}}%
\pgfpathcurveto{\pgfqpoint{6.271622in}{1.534081in}}{\pgfqpoint{6.274456in}{1.540923in}}{\pgfqpoint{6.274456in}{1.548056in}}%
\pgfpathcurveto{\pgfqpoint{6.274456in}{1.555188in}}{\pgfqpoint{6.271622in}{1.562030in}}{\pgfqpoint{6.266579in}{1.567074in}}%
\pgfpathcurveto{\pgfqpoint{6.261535in}{1.572117in}}{\pgfqpoint{6.254693in}{1.574951in}}{\pgfqpoint{6.247560in}{1.574951in}}%
\pgfpathcurveto{\pgfqpoint{6.240428in}{1.574951in}}{\pgfqpoint{6.233586in}{1.572117in}}{\pgfqpoint{6.228542in}{1.567074in}}%
\pgfpathcurveto{\pgfqpoint{6.223499in}{1.562030in}}{\pgfqpoint{6.220665in}{1.555188in}}{\pgfqpoint{6.220665in}{1.548056in}}%
\pgfpathcurveto{\pgfqpoint{6.220665in}{1.540923in}}{\pgfqpoint{6.223499in}{1.534081in}}{\pgfqpoint{6.228542in}{1.529038in}}%
\pgfpathcurveto{\pgfqpoint{6.233586in}{1.523994in}}{\pgfqpoint{6.240428in}{1.521160in}}{\pgfqpoint{6.247560in}{1.521160in}}%
\pgfpathclose%
\pgfusepath{stroke,fill}%
\end{pgfscope}%
\begin{pgfscope}%
\pgfpathrectangle{\pgfqpoint{4.985294in}{0.500000in}}{\pgfqpoint{1.764706in}{1.700000in}}%
\pgfusepath{clip}%
\pgfsetbuttcap%
\pgfsetroundjoin%
\definecolor{currentfill}{rgb}{0.971202,0.827364,0.728520}%
\pgfsetfillcolor{currentfill}%
\pgfsetlinewidth{0.311001pt}%
\definecolor{currentstroke}{rgb}{1.000000,1.000000,1.000000}%
\pgfsetstrokecolor{currentstroke}%
\pgfsetdash{}{0pt}%
\pgfpathmoveto{\pgfqpoint{6.187543in}{1.665731in}}%
\pgfpathcurveto{\pgfqpoint{6.194675in}{1.665731in}}{\pgfqpoint{6.201517in}{1.668565in}}{\pgfqpoint{6.206561in}{1.673609in}}%
\pgfpathcurveto{\pgfqpoint{6.211604in}{1.678652in}}{\pgfqpoint{6.214438in}{1.685494in}}{\pgfqpoint{6.214438in}{1.692627in}}%
\pgfpathcurveto{\pgfqpoint{6.214438in}{1.699759in}}{\pgfqpoint{6.211604in}{1.706601in}}{\pgfqpoint{6.206561in}{1.711645in}}%
\pgfpathcurveto{\pgfqpoint{6.201517in}{1.716688in}}{\pgfqpoint{6.194675in}{1.719522in}}{\pgfqpoint{6.187543in}{1.719522in}}%
\pgfpathcurveto{\pgfqpoint{6.180410in}{1.719522in}}{\pgfqpoint{6.173568in}{1.716688in}}{\pgfqpoint{6.168524in}{1.711645in}}%
\pgfpathcurveto{\pgfqpoint{6.163481in}{1.706601in}}{\pgfqpoint{6.160647in}{1.699759in}}{\pgfqpoint{6.160647in}{1.692627in}}%
\pgfpathcurveto{\pgfqpoint{6.160647in}{1.685494in}}{\pgfqpoint{6.163481in}{1.678652in}}{\pgfqpoint{6.168524in}{1.673609in}}%
\pgfpathcurveto{\pgfqpoint{6.173568in}{1.668565in}}{\pgfqpoint{6.180410in}{1.665731in}}{\pgfqpoint{6.187543in}{1.665731in}}%
\pgfpathclose%
\pgfusepath{stroke,fill}%
\end{pgfscope}%
\begin{pgfscope}%
\pgfpathrectangle{\pgfqpoint{4.985294in}{0.500000in}}{\pgfqpoint{1.764706in}{1.700000in}}%
\pgfusepath{clip}%
\pgfsetbuttcap%
\pgfsetroundjoin%
\definecolor{currentfill}{rgb}{0.973832,0.856556,0.771584}%
\pgfsetfillcolor{currentfill}%
\pgfsetlinewidth{0.311001pt}%
\definecolor{currentstroke}{rgb}{1.000000,1.000000,1.000000}%
\pgfsetstrokecolor{currentstroke}%
\pgfsetdash{}{0pt}%
\pgfpathmoveto{\pgfqpoint{6.350642in}{1.182826in}}%
\pgfpathcurveto{\pgfqpoint{6.357775in}{1.182826in}}{\pgfqpoint{6.364616in}{1.185660in}}{\pgfqpoint{6.369660in}{1.190703in}}%
\pgfpathcurveto{\pgfqpoint{6.374704in}{1.195747in}}{\pgfqpoint{6.377537in}{1.202589in}}{\pgfqpoint{6.377537in}{1.209722in}}%
\pgfpathcurveto{\pgfqpoint{6.377537in}{1.216854in}}{\pgfqpoint{6.374704in}{1.223696in}}{\pgfqpoint{6.369660in}{1.228740in}}%
\pgfpathcurveto{\pgfqpoint{6.364616in}{1.233783in}}{\pgfqpoint{6.357775in}{1.236617in}}{\pgfqpoint{6.350642in}{1.236617in}}%
\pgfpathcurveto{\pgfqpoint{6.343509in}{1.236617in}}{\pgfqpoint{6.336667in}{1.233783in}}{\pgfqpoint{6.331624in}{1.228740in}}%
\pgfpathcurveto{\pgfqpoint{6.326580in}{1.223696in}}{\pgfqpoint{6.323746in}{1.216854in}}{\pgfqpoint{6.323746in}{1.209722in}}%
\pgfpathcurveto{\pgfqpoint{6.323746in}{1.202589in}}{\pgfqpoint{6.326580in}{1.195747in}}{\pgfqpoint{6.331624in}{1.190703in}}%
\pgfpathcurveto{\pgfqpoint{6.336667in}{1.185660in}}{\pgfqpoint{6.343509in}{1.182826in}}{\pgfqpoint{6.350642in}{1.182826in}}%
\pgfpathclose%
\pgfusepath{stroke,fill}%
\end{pgfscope}%
\begin{pgfscope}%
\pgfpathrectangle{\pgfqpoint{4.985294in}{0.500000in}}{\pgfqpoint{1.764706in}{1.700000in}}%
\pgfusepath{clip}%
\pgfsetbuttcap%
\pgfsetroundjoin%
\definecolor{currentfill}{rgb}{0.963379,0.625574,0.465113}%
\pgfsetfillcolor{currentfill}%
\pgfsetlinewidth{0.311001pt}%
\definecolor{currentstroke}{rgb}{1.000000,1.000000,1.000000}%
\pgfsetstrokecolor{currentstroke}%
\pgfsetdash{}{0pt}%
\pgfpathmoveto{\pgfqpoint{6.181992in}{1.327428in}}%
\pgfpathcurveto{\pgfqpoint{6.189125in}{1.327428in}}{\pgfqpoint{6.195966in}{1.330261in}}{\pgfqpoint{6.201010in}{1.335305in}}%
\pgfpathcurveto{\pgfqpoint{6.206053in}{1.340349in}}{\pgfqpoint{6.208887in}{1.347190in}}{\pgfqpoint{6.208887in}{1.354323in}}%
\pgfpathcurveto{\pgfqpoint{6.208887in}{1.361456in}}{\pgfqpoint{6.206053in}{1.368298in}}{\pgfqpoint{6.201010in}{1.373341in}}%
\pgfpathcurveto{\pgfqpoint{6.195966in}{1.378385in}}{\pgfqpoint{6.189125in}{1.381219in}}{\pgfqpoint{6.181992in}{1.381219in}}%
\pgfpathcurveto{\pgfqpoint{6.174859in}{1.381219in}}{\pgfqpoint{6.168017in}{1.378385in}}{\pgfqpoint{6.162974in}{1.373341in}}%
\pgfpathcurveto{\pgfqpoint{6.157930in}{1.368298in}}{\pgfqpoint{6.155096in}{1.361456in}}{\pgfqpoint{6.155096in}{1.354323in}}%
\pgfpathcurveto{\pgfqpoint{6.155096in}{1.347190in}}{\pgfqpoint{6.157930in}{1.340349in}}{\pgfqpoint{6.162974in}{1.335305in}}%
\pgfpathcurveto{\pgfqpoint{6.168017in}{1.330261in}}{\pgfqpoint{6.174859in}{1.327428in}}{\pgfqpoint{6.181992in}{1.327428in}}%
\pgfpathclose%
\pgfusepath{stroke,fill}%
\end{pgfscope}%
\begin{pgfscope}%
\pgfpathrectangle{\pgfqpoint{4.985294in}{0.500000in}}{\pgfqpoint{1.764706in}{1.700000in}}%
\pgfusepath{clip}%
\pgfsetbuttcap%
\pgfsetroundjoin%
\definecolor{currentfill}{rgb}{0.967735,0.780441,0.659127}%
\pgfsetfillcolor{currentfill}%
\pgfsetlinewidth{0.311001pt}%
\definecolor{currentstroke}{rgb}{1.000000,1.000000,1.000000}%
\pgfsetstrokecolor{currentstroke}%
\pgfsetdash{}{0pt}%
\pgfpathmoveto{\pgfqpoint{6.166498in}{0.981928in}}%
\pgfpathcurveto{\pgfqpoint{6.173631in}{0.981928in}}{\pgfqpoint{6.180473in}{0.984762in}}{\pgfqpoint{6.185517in}{0.989806in}}%
\pgfpathcurveto{\pgfqpoint{6.190560in}{0.994850in}}{\pgfqpoint{6.193394in}{1.001691in}}{\pgfqpoint{6.193394in}{1.008824in}}%
\pgfpathcurveto{\pgfqpoint{6.193394in}{1.015957in}}{\pgfqpoint{6.190560in}{1.022798in}}{\pgfqpoint{6.185517in}{1.027842in}}%
\pgfpathcurveto{\pgfqpoint{6.180473in}{1.032886in}}{\pgfqpoint{6.173631in}{1.035720in}}{\pgfqpoint{6.166498in}{1.035720in}}%
\pgfpathcurveto{\pgfqpoint{6.159366in}{1.035720in}}{\pgfqpoint{6.152524in}{1.032886in}}{\pgfqpoint{6.147480in}{1.027842in}}%
\pgfpathcurveto{\pgfqpoint{6.142437in}{1.022798in}}{\pgfqpoint{6.139603in}{1.015957in}}{\pgfqpoint{6.139603in}{1.008824in}}%
\pgfpathcurveto{\pgfqpoint{6.139603in}{1.001691in}}{\pgfqpoint{6.142437in}{0.994850in}}{\pgfqpoint{6.147480in}{0.989806in}}%
\pgfpathcurveto{\pgfqpoint{6.152524in}{0.984762in}}{\pgfqpoint{6.159366in}{0.981928in}}{\pgfqpoint{6.166498in}{0.981928in}}%
\pgfpathclose%
\pgfusepath{stroke,fill}%
\end{pgfscope}%
\begin{pgfscope}%
\pgfpathrectangle{\pgfqpoint{4.985294in}{0.500000in}}{\pgfqpoint{1.764706in}{1.700000in}}%
\pgfusepath{clip}%
\pgfsetbuttcap%
\pgfsetroundjoin%
\definecolor{currentfill}{rgb}{0.730358,0.086862,0.337485}%
\pgfsetfillcolor{currentfill}%
\pgfsetlinewidth{0.311001pt}%
\definecolor{currentstroke}{rgb}{1.000000,1.000000,1.000000}%
\pgfsetstrokecolor{currentstroke}%
\pgfsetdash{}{0pt}%
\pgfpathmoveto{\pgfqpoint{5.226004in}{1.395190in}}%
\pgfpathcurveto{\pgfqpoint{5.233137in}{1.395190in}}{\pgfqpoint{5.239979in}{1.398023in}}{\pgfqpoint{5.245023in}{1.403067in}}%
\pgfpathcurveto{\pgfqpoint{5.250066in}{1.408111in}}{\pgfqpoint{5.252900in}{1.414952in}}{\pgfqpoint{5.252900in}{1.422085in}}%
\pgfpathcurveto{\pgfqpoint{5.252900in}{1.429218in}}{\pgfqpoint{5.250066in}{1.436060in}}{\pgfqpoint{5.245023in}{1.441103in}}%
\pgfpathcurveto{\pgfqpoint{5.239979in}{1.446147in}}{\pgfqpoint{5.233137in}{1.448981in}}{\pgfqpoint{5.226004in}{1.448981in}}%
\pgfpathcurveto{\pgfqpoint{5.218872in}{1.448981in}}{\pgfqpoint{5.212030in}{1.446147in}}{\pgfqpoint{5.206986in}{1.441103in}}%
\pgfpathcurveto{\pgfqpoint{5.201943in}{1.436060in}}{\pgfqpoint{5.199109in}{1.429218in}}{\pgfqpoint{5.199109in}{1.422085in}}%
\pgfpathcurveto{\pgfqpoint{5.199109in}{1.414952in}}{\pgfqpoint{5.201943in}{1.408111in}}{\pgfqpoint{5.206986in}{1.403067in}}%
\pgfpathcurveto{\pgfqpoint{5.212030in}{1.398023in}}{\pgfqpoint{5.218872in}{1.395190in}}{\pgfqpoint{5.226004in}{1.395190in}}%
\pgfpathclose%
\pgfusepath{stroke,fill}%
\end{pgfscope}%
\begin{pgfscope}%
\pgfpathrectangle{\pgfqpoint{4.985294in}{0.500000in}}{\pgfqpoint{1.764706in}{1.700000in}}%
\pgfusepath{clip}%
\pgfsetbuttcap%
\pgfsetroundjoin%
\definecolor{currentfill}{rgb}{0.972201,0.839051,0.745789}%
\pgfsetfillcolor{currentfill}%
\pgfsetlinewidth{0.311001pt}%
\definecolor{currentstroke}{rgb}{1.000000,1.000000,1.000000}%
\pgfsetstrokecolor{currentstroke}%
\pgfsetdash{}{0pt}%
\pgfpathmoveto{\pgfqpoint{6.214940in}{1.542787in}}%
\pgfpathcurveto{\pgfqpoint{6.222073in}{1.542787in}}{\pgfqpoint{6.228914in}{1.545621in}}{\pgfqpoint{6.233958in}{1.550665in}}%
\pgfpathcurveto{\pgfqpoint{6.239002in}{1.555709in}}{\pgfqpoint{6.241836in}{1.562550in}}{\pgfqpoint{6.241836in}{1.569683in}}%
\pgfpathcurveto{\pgfqpoint{6.241836in}{1.576816in}}{\pgfqpoint{6.239002in}{1.583658in}}{\pgfqpoint{6.233958in}{1.588701in}}%
\pgfpathcurveto{\pgfqpoint{6.228914in}{1.593745in}}{\pgfqpoint{6.222073in}{1.596579in}}{\pgfqpoint{6.214940in}{1.596579in}}%
\pgfpathcurveto{\pgfqpoint{6.207807in}{1.596579in}}{\pgfqpoint{6.200965in}{1.593745in}}{\pgfqpoint{6.195922in}{1.588701in}}%
\pgfpathcurveto{\pgfqpoint{6.190878in}{1.583658in}}{\pgfqpoint{6.188044in}{1.576816in}}{\pgfqpoint{6.188044in}{1.569683in}}%
\pgfpathcurveto{\pgfqpoint{6.188044in}{1.562550in}}{\pgfqpoint{6.190878in}{1.555709in}}{\pgfqpoint{6.195922in}{1.550665in}}%
\pgfpathcurveto{\pgfqpoint{6.200965in}{1.545621in}}{\pgfqpoint{6.207807in}{1.542787in}}{\pgfqpoint{6.214940in}{1.542787in}}%
\pgfpathclose%
\pgfusepath{stroke,fill}%
\end{pgfscope}%
\begin{pgfscope}%
\pgfpathrectangle{\pgfqpoint{4.985294in}{0.500000in}}{\pgfqpoint{1.764706in}{1.700000in}}%
\pgfusepath{clip}%
\pgfsetbuttcap%
\pgfsetroundjoin%
\definecolor{currentfill}{rgb}{0.964679,0.682838,0.530002}%
\pgfsetfillcolor{currentfill}%
\pgfsetlinewidth{0.311001pt}%
\definecolor{currentstroke}{rgb}{1.000000,1.000000,1.000000}%
\pgfsetstrokecolor{currentstroke}%
\pgfsetdash{}{0pt}%
\pgfpathmoveto{\pgfqpoint{6.413727in}{1.254417in}}%
\pgfpathcurveto{\pgfqpoint{6.420860in}{1.254417in}}{\pgfqpoint{6.427701in}{1.257251in}}{\pgfqpoint{6.432745in}{1.262295in}}%
\pgfpathcurveto{\pgfqpoint{6.437789in}{1.267339in}}{\pgfqpoint{6.440623in}{1.274180in}}{\pgfqpoint{6.440623in}{1.281313in}}%
\pgfpathcurveto{\pgfqpoint{6.440623in}{1.288446in}}{\pgfqpoint{6.437789in}{1.295288in}}{\pgfqpoint{6.432745in}{1.300331in}}%
\pgfpathcurveto{\pgfqpoint{6.427701in}{1.305375in}}{\pgfqpoint{6.420860in}{1.308209in}}{\pgfqpoint{6.413727in}{1.308209in}}%
\pgfpathcurveto{\pgfqpoint{6.406594in}{1.308209in}}{\pgfqpoint{6.399753in}{1.305375in}}{\pgfqpoint{6.394709in}{1.300331in}}%
\pgfpathcurveto{\pgfqpoint{6.389665in}{1.295288in}}{\pgfqpoint{6.386831in}{1.288446in}}{\pgfqpoint{6.386831in}{1.281313in}}%
\pgfpathcurveto{\pgfqpoint{6.386831in}{1.274180in}}{\pgfqpoint{6.389665in}{1.267339in}}{\pgfqpoint{6.394709in}{1.262295in}}%
\pgfpathcurveto{\pgfqpoint{6.399753in}{1.257251in}}{\pgfqpoint{6.406594in}{1.254417in}}{\pgfqpoint{6.413727in}{1.254417in}}%
\pgfpathclose%
\pgfusepath{stroke,fill}%
\end{pgfscope}%
\begin{pgfscope}%
\pgfpathrectangle{\pgfqpoint{4.985294in}{0.500000in}}{\pgfqpoint{1.764706in}{1.700000in}}%
\pgfusepath{clip}%
\pgfsetbuttcap%
\pgfsetroundjoin%
\definecolor{currentfill}{rgb}{0.974412,0.862387,0.780156}%
\pgfsetfillcolor{currentfill}%
\pgfsetlinewidth{0.311001pt}%
\definecolor{currentstroke}{rgb}{1.000000,1.000000,1.000000}%
\pgfsetstrokecolor{currentstroke}%
\pgfsetdash{}{0pt}%
\pgfpathmoveto{\pgfqpoint{6.265648in}{1.342118in}}%
\pgfpathcurveto{\pgfqpoint{6.272781in}{1.342118in}}{\pgfqpoint{6.279622in}{1.344952in}}{\pgfqpoint{6.284666in}{1.349996in}}%
\pgfpathcurveto{\pgfqpoint{6.289710in}{1.355039in}}{\pgfqpoint{6.292544in}{1.361881in}}{\pgfqpoint{6.292544in}{1.369014in}}%
\pgfpathcurveto{\pgfqpoint{6.292544in}{1.376147in}}{\pgfqpoint{6.289710in}{1.382988in}}{\pgfqpoint{6.284666in}{1.388032in}}%
\pgfpathcurveto{\pgfqpoint{6.279622in}{1.393076in}}{\pgfqpoint{6.272781in}{1.395910in}}{\pgfqpoint{6.265648in}{1.395910in}}%
\pgfpathcurveto{\pgfqpoint{6.258515in}{1.395910in}}{\pgfqpoint{6.251673in}{1.393076in}}{\pgfqpoint{6.246630in}{1.388032in}}%
\pgfpathcurveto{\pgfqpoint{6.241586in}{1.382988in}}{\pgfqpoint{6.238752in}{1.376147in}}{\pgfqpoint{6.238752in}{1.369014in}}%
\pgfpathcurveto{\pgfqpoint{6.238752in}{1.361881in}}{\pgfqpoint{6.241586in}{1.355039in}}{\pgfqpoint{6.246630in}{1.349996in}}%
\pgfpathcurveto{\pgfqpoint{6.251673in}{1.344952in}}{\pgfqpoint{6.258515in}{1.342118in}}{\pgfqpoint{6.265648in}{1.342118in}}%
\pgfpathclose%
\pgfusepath{stroke,fill}%
\end{pgfscope}%
\begin{pgfscope}%
\pgfpathrectangle{\pgfqpoint{4.985294in}{0.500000in}}{\pgfqpoint{1.764706in}{1.700000in}}%
\pgfusepath{clip}%
\pgfsetbuttcap%
\pgfsetroundjoin%
\definecolor{currentfill}{rgb}{0.971694,0.833208,0.737161}%
\pgfsetfillcolor{currentfill}%
\pgfsetlinewidth{0.311001pt}%
\definecolor{currentstroke}{rgb}{1.000000,1.000000,1.000000}%
\pgfsetstrokecolor{currentstroke}%
\pgfsetdash{}{0pt}%
\pgfpathmoveto{\pgfqpoint{5.478652in}{1.004260in}}%
\pgfpathcurveto{\pgfqpoint{5.485784in}{1.004260in}}{\pgfqpoint{5.492626in}{1.007094in}}{\pgfqpoint{5.497670in}{1.012138in}}%
\pgfpathcurveto{\pgfqpoint{5.502713in}{1.017181in}}{\pgfqpoint{5.505547in}{1.024023in}}{\pgfqpoint{5.505547in}{1.031156in}}%
\pgfpathcurveto{\pgfqpoint{5.505547in}{1.038289in}}{\pgfqpoint{5.502713in}{1.045130in}}{\pgfqpoint{5.497670in}{1.050174in}}%
\pgfpathcurveto{\pgfqpoint{5.492626in}{1.055218in}}{\pgfqpoint{5.485784in}{1.058052in}}{\pgfqpoint{5.478652in}{1.058052in}}%
\pgfpathcurveto{\pgfqpoint{5.471519in}{1.058052in}}{\pgfqpoint{5.464677in}{1.055218in}}{\pgfqpoint{5.459634in}{1.050174in}}%
\pgfpathcurveto{\pgfqpoint{5.454590in}{1.045130in}}{\pgfqpoint{5.451756in}{1.038289in}}{\pgfqpoint{5.451756in}{1.031156in}}%
\pgfpathcurveto{\pgfqpoint{5.451756in}{1.024023in}}{\pgfqpoint{5.454590in}{1.017181in}}{\pgfqpoint{5.459634in}{1.012138in}}%
\pgfpathcurveto{\pgfqpoint{5.464677in}{1.007094in}}{\pgfqpoint{5.471519in}{1.004260in}}{\pgfqpoint{5.478652in}{1.004260in}}%
\pgfpathclose%
\pgfusepath{stroke,fill}%
\end{pgfscope}%
\begin{pgfscope}%
\pgfpathrectangle{\pgfqpoint{4.985294in}{0.500000in}}{\pgfqpoint{1.764706in}{1.700000in}}%
\pgfusepath{clip}%
\pgfsetbuttcap%
\pgfsetroundjoin%
\definecolor{currentfill}{rgb}{0.966560,0.756582,0.625273}%
\pgfsetfillcolor{currentfill}%
\pgfsetlinewidth{0.311001pt}%
\definecolor{currentstroke}{rgb}{1.000000,1.000000,1.000000}%
\pgfsetstrokecolor{currentstroke}%
\pgfsetdash{}{0pt}%
\pgfpathmoveto{\pgfqpoint{6.156837in}{0.959362in}}%
\pgfpathcurveto{\pgfqpoint{6.163970in}{0.959362in}}{\pgfqpoint{6.170812in}{0.962196in}}{\pgfqpoint{6.175855in}{0.967239in}}%
\pgfpathcurveto{\pgfqpoint{6.180899in}{0.972283in}}{\pgfqpoint{6.183733in}{0.979125in}}{\pgfqpoint{6.183733in}{0.986258in}}%
\pgfpathcurveto{\pgfqpoint{6.183733in}{0.993390in}}{\pgfqpoint{6.180899in}{1.000232in}}{\pgfqpoint{6.175855in}{1.005276in}}%
\pgfpathcurveto{\pgfqpoint{6.170812in}{1.010319in}}{\pgfqpoint{6.163970in}{1.013153in}}{\pgfqpoint{6.156837in}{1.013153in}}%
\pgfpathcurveto{\pgfqpoint{6.149704in}{1.013153in}}{\pgfqpoint{6.142863in}{1.010319in}}{\pgfqpoint{6.137819in}{1.005276in}}%
\pgfpathcurveto{\pgfqpoint{6.132775in}{1.000232in}}{\pgfqpoint{6.129942in}{0.993390in}}{\pgfqpoint{6.129942in}{0.986258in}}%
\pgfpathcurveto{\pgfqpoint{6.129942in}{0.979125in}}{\pgfqpoint{6.132775in}{0.972283in}}{\pgfqpoint{6.137819in}{0.967239in}}%
\pgfpathcurveto{\pgfqpoint{6.142863in}{0.962196in}}{\pgfqpoint{6.149704in}{0.959362in}}{\pgfqpoint{6.156837in}{0.959362in}}%
\pgfpathclose%
\pgfusepath{stroke,fill}%
\end{pgfscope}%
\begin{pgfscope}%
\pgfpathrectangle{\pgfqpoint{4.985294in}{0.500000in}}{\pgfqpoint{1.764706in}{1.700000in}}%
\pgfusepath{clip}%
\pgfsetbuttcap%
\pgfsetroundjoin%
\definecolor{currentfill}{rgb}{0.973832,0.856556,0.771584}%
\pgfsetfillcolor{currentfill}%
\pgfsetlinewidth{0.311001pt}%
\definecolor{currentstroke}{rgb}{1.000000,1.000000,1.000000}%
\pgfsetstrokecolor{currentstroke}%
\pgfsetdash{}{0pt}%
\pgfpathmoveto{\pgfqpoint{6.284174in}{1.612045in}}%
\pgfpathcurveto{\pgfqpoint{6.291307in}{1.612045in}}{\pgfqpoint{6.298149in}{1.614879in}}{\pgfqpoint{6.303192in}{1.619923in}}%
\pgfpathcurveto{\pgfqpoint{6.308236in}{1.624966in}}{\pgfqpoint{6.311070in}{1.631808in}}{\pgfqpoint{6.311070in}{1.638941in}}%
\pgfpathcurveto{\pgfqpoint{6.311070in}{1.646074in}}{\pgfqpoint{6.308236in}{1.652915in}}{\pgfqpoint{6.303192in}{1.657959in}}%
\pgfpathcurveto{\pgfqpoint{6.298149in}{1.663003in}}{\pgfqpoint{6.291307in}{1.665836in}}{\pgfqpoint{6.284174in}{1.665836in}}%
\pgfpathcurveto{\pgfqpoint{6.277041in}{1.665836in}}{\pgfqpoint{6.270200in}{1.663003in}}{\pgfqpoint{6.265156in}{1.657959in}}%
\pgfpathcurveto{\pgfqpoint{6.260112in}{1.652915in}}{\pgfqpoint{6.257278in}{1.646074in}}{\pgfqpoint{6.257278in}{1.638941in}}%
\pgfpathcurveto{\pgfqpoint{6.257278in}{1.631808in}}{\pgfqpoint{6.260112in}{1.624966in}}{\pgfqpoint{6.265156in}{1.619923in}}%
\pgfpathcurveto{\pgfqpoint{6.270200in}{1.614879in}}{\pgfqpoint{6.277041in}{1.612045in}}{\pgfqpoint{6.284174in}{1.612045in}}%
\pgfpathclose%
\pgfusepath{stroke,fill}%
\end{pgfscope}%
\begin{pgfscope}%
\pgfpathrectangle{\pgfqpoint{4.985294in}{0.500000in}}{\pgfqpoint{1.764706in}{1.700000in}}%
\pgfusepath{clip}%
\pgfsetbuttcap%
\pgfsetroundjoin%
\definecolor{currentfill}{rgb}{0.977657,0.891500,0.822809}%
\pgfsetfillcolor{currentfill}%
\pgfsetlinewidth{0.311001pt}%
\definecolor{currentstroke}{rgb}{1.000000,1.000000,1.000000}%
\pgfsetstrokecolor{currentstroke}%
\pgfsetdash{}{0pt}%
\pgfpathmoveto{\pgfqpoint{5.430942in}{1.489478in}}%
\pgfpathcurveto{\pgfqpoint{5.438075in}{1.489478in}}{\pgfqpoint{5.444917in}{1.492312in}}{\pgfqpoint{5.449960in}{1.497356in}}%
\pgfpathcurveto{\pgfqpoint{5.455004in}{1.502400in}}{\pgfqpoint{5.457838in}{1.509241in}}{\pgfqpoint{5.457838in}{1.516374in}}%
\pgfpathcurveto{\pgfqpoint{5.457838in}{1.523507in}}{\pgfqpoint{5.455004in}{1.530349in}}{\pgfqpoint{5.449960in}{1.535392in}}%
\pgfpathcurveto{\pgfqpoint{5.444917in}{1.540436in}}{\pgfqpoint{5.438075in}{1.543270in}}{\pgfqpoint{5.430942in}{1.543270in}}%
\pgfpathcurveto{\pgfqpoint{5.423809in}{1.543270in}}{\pgfqpoint{5.416968in}{1.540436in}}{\pgfqpoint{5.411924in}{1.535392in}}%
\pgfpathcurveto{\pgfqpoint{5.406880in}{1.530349in}}{\pgfqpoint{5.404046in}{1.523507in}}{\pgfqpoint{5.404046in}{1.516374in}}%
\pgfpathcurveto{\pgfqpoint{5.404046in}{1.509241in}}{\pgfqpoint{5.406880in}{1.502400in}}{\pgfqpoint{5.411924in}{1.497356in}}%
\pgfpathcurveto{\pgfqpoint{5.416968in}{1.492312in}}{\pgfqpoint{5.423809in}{1.489478in}}{\pgfqpoint{5.430942in}{1.489478in}}%
\pgfpathclose%
\pgfusepath{stroke,fill}%
\end{pgfscope}%
\begin{pgfscope}%
\pgfpathrectangle{\pgfqpoint{4.985294in}{0.500000in}}{\pgfqpoint{1.764706in}{1.700000in}}%
\pgfusepath{clip}%
\pgfsetbuttcap%
\pgfsetroundjoin%
\definecolor{currentfill}{rgb}{0.971694,0.833208,0.737161}%
\pgfsetfillcolor{currentfill}%
\pgfsetlinewidth{0.311001pt}%
\definecolor{currentstroke}{rgb}{1.000000,1.000000,1.000000}%
\pgfsetstrokecolor{currentstroke}%
\pgfsetdash{}{0pt}%
\pgfpathmoveto{\pgfqpoint{5.515353in}{1.506489in}}%
\pgfpathcurveto{\pgfqpoint{5.522486in}{1.506489in}}{\pgfqpoint{5.529327in}{1.509323in}}{\pgfqpoint{5.534371in}{1.514367in}}%
\pgfpathcurveto{\pgfqpoint{5.539415in}{1.519410in}}{\pgfqpoint{5.542248in}{1.526252in}}{\pgfqpoint{5.542248in}{1.533385in}}%
\pgfpathcurveto{\pgfqpoint{5.542248in}{1.540518in}}{\pgfqpoint{5.539415in}{1.547359in}}{\pgfqpoint{5.534371in}{1.552403in}}%
\pgfpathcurveto{\pgfqpoint{5.529327in}{1.557447in}}{\pgfqpoint{5.522486in}{1.560281in}}{\pgfqpoint{5.515353in}{1.560281in}}%
\pgfpathcurveto{\pgfqpoint{5.508220in}{1.560281in}}{\pgfqpoint{5.501378in}{1.557447in}}{\pgfqpoint{5.496335in}{1.552403in}}%
\pgfpathcurveto{\pgfqpoint{5.491291in}{1.547359in}}{\pgfqpoint{5.488457in}{1.540518in}}{\pgfqpoint{5.488457in}{1.533385in}}%
\pgfpathcurveto{\pgfqpoint{5.488457in}{1.526252in}}{\pgfqpoint{5.491291in}{1.519410in}}{\pgfqpoint{5.496335in}{1.514367in}}%
\pgfpathcurveto{\pgfqpoint{5.501378in}{1.509323in}}{\pgfqpoint{5.508220in}{1.506489in}}{\pgfqpoint{5.515353in}{1.506489in}}%
\pgfpathclose%
\pgfusepath{stroke,fill}%
\end{pgfscope}%
\begin{pgfscope}%
\pgfpathrectangle{\pgfqpoint{4.985294in}{0.500000in}}{\pgfqpoint{1.764706in}{1.700000in}}%
\pgfusepath{clip}%
\pgfsetbuttcap%
\pgfsetroundjoin%
\definecolor{currentfill}{rgb}{0.972201,0.839051,0.745789}%
\pgfsetfillcolor{currentfill}%
\pgfsetlinewidth{0.311001pt}%
\definecolor{currentstroke}{rgb}{1.000000,1.000000,1.000000}%
\pgfsetstrokecolor{currentstroke}%
\pgfsetdash{}{0pt}%
\pgfpathmoveto{\pgfqpoint{5.518864in}{1.576282in}}%
\pgfpathcurveto{\pgfqpoint{5.525997in}{1.576282in}}{\pgfqpoint{5.532838in}{1.579116in}}{\pgfqpoint{5.537882in}{1.584160in}}%
\pgfpathcurveto{\pgfqpoint{5.542926in}{1.589204in}}{\pgfqpoint{5.545760in}{1.596045in}}{\pgfqpoint{5.545760in}{1.603178in}}%
\pgfpathcurveto{\pgfqpoint{5.545760in}{1.610311in}}{\pgfqpoint{5.542926in}{1.617153in}}{\pgfqpoint{5.537882in}{1.622196in}}%
\pgfpathcurveto{\pgfqpoint{5.532838in}{1.627240in}}{\pgfqpoint{5.525997in}{1.630074in}}{\pgfqpoint{5.518864in}{1.630074in}}%
\pgfpathcurveto{\pgfqpoint{5.511731in}{1.630074in}}{\pgfqpoint{5.504889in}{1.627240in}}{\pgfqpoint{5.499846in}{1.622196in}}%
\pgfpathcurveto{\pgfqpoint{5.494802in}{1.617153in}}{\pgfqpoint{5.491968in}{1.610311in}}{\pgfqpoint{5.491968in}{1.603178in}}%
\pgfpathcurveto{\pgfqpoint{5.491968in}{1.596045in}}{\pgfqpoint{5.494802in}{1.589204in}}{\pgfqpoint{5.499846in}{1.584160in}}%
\pgfpathcurveto{\pgfqpoint{5.504889in}{1.579116in}}{\pgfqpoint{5.511731in}{1.576282in}}{\pgfqpoint{5.518864in}{1.576282in}}%
\pgfpathclose%
\pgfusepath{stroke,fill}%
\end{pgfscope}%
\begin{pgfscope}%
\pgfpathrectangle{\pgfqpoint{4.985294in}{0.500000in}}{\pgfqpoint{1.764706in}{1.700000in}}%
\pgfusepath{clip}%
\pgfsetbuttcap%
\pgfsetroundjoin%
\definecolor{currentfill}{rgb}{0.971202,0.827364,0.728520}%
\pgfsetfillcolor{currentfill}%
\pgfsetlinewidth{0.311001pt}%
\definecolor{currentstroke}{rgb}{1.000000,1.000000,1.000000}%
\pgfsetstrokecolor{currentstroke}%
\pgfsetdash{}{0pt}%
\pgfpathmoveto{\pgfqpoint{5.356393in}{1.343419in}}%
\pgfpathcurveto{\pgfqpoint{5.363526in}{1.343419in}}{\pgfqpoint{5.370367in}{1.346253in}}{\pgfqpoint{5.375411in}{1.351297in}}%
\pgfpathcurveto{\pgfqpoint{5.380455in}{1.356341in}}{\pgfqpoint{5.383288in}{1.363182in}}{\pgfqpoint{5.383288in}{1.370315in}}%
\pgfpathcurveto{\pgfqpoint{5.383288in}{1.377448in}}{\pgfqpoint{5.380455in}{1.384290in}}{\pgfqpoint{5.375411in}{1.389333in}}%
\pgfpathcurveto{\pgfqpoint{5.370367in}{1.394377in}}{\pgfqpoint{5.363526in}{1.397211in}}{\pgfqpoint{5.356393in}{1.397211in}}%
\pgfpathcurveto{\pgfqpoint{5.349260in}{1.397211in}}{\pgfqpoint{5.342418in}{1.394377in}}{\pgfqpoint{5.337375in}{1.389333in}}%
\pgfpathcurveto{\pgfqpoint{5.332331in}{1.384290in}}{\pgfqpoint{5.329497in}{1.377448in}}{\pgfqpoint{5.329497in}{1.370315in}}%
\pgfpathcurveto{\pgfqpoint{5.329497in}{1.363182in}}{\pgfqpoint{5.332331in}{1.356341in}}{\pgfqpoint{5.337375in}{1.351297in}}%
\pgfpathcurveto{\pgfqpoint{5.342418in}{1.346253in}}{\pgfqpoint{5.349260in}{1.343419in}}{\pgfqpoint{5.356393in}{1.343419in}}%
\pgfpathclose%
\pgfusepath{stroke,fill}%
\end{pgfscope}%
\begin{pgfscope}%
\pgfpathrectangle{\pgfqpoint{4.985294in}{0.500000in}}{\pgfqpoint{1.764706in}{1.700000in}}%
\pgfusepath{clip}%
\pgfsetbuttcap%
\pgfsetroundjoin%
\definecolor{currentfill}{rgb}{0.966328,0.750560,0.616961}%
\pgfsetfillcolor{currentfill}%
\pgfsetlinewidth{0.311001pt}%
\definecolor{currentstroke}{rgb}{1.000000,1.000000,1.000000}%
\pgfsetstrokecolor{currentstroke}%
\pgfsetdash{}{0pt}%
\pgfpathmoveto{\pgfqpoint{6.149615in}{1.606756in}}%
\pgfpathcurveto{\pgfqpoint{6.156748in}{1.606756in}}{\pgfqpoint{6.163590in}{1.609590in}}{\pgfqpoint{6.168634in}{1.614634in}}%
\pgfpathcurveto{\pgfqpoint{6.173677in}{1.619677in}}{\pgfqpoint{6.176511in}{1.626519in}}{\pgfqpoint{6.176511in}{1.633652in}}%
\pgfpathcurveto{\pgfqpoint{6.176511in}{1.640785in}}{\pgfqpoint{6.173677in}{1.647626in}}{\pgfqpoint{6.168634in}{1.652670in}}%
\pgfpathcurveto{\pgfqpoint{6.163590in}{1.657714in}}{\pgfqpoint{6.156748in}{1.660548in}}{\pgfqpoint{6.149615in}{1.660548in}}%
\pgfpathcurveto{\pgfqpoint{6.142483in}{1.660548in}}{\pgfqpoint{6.135641in}{1.657714in}}{\pgfqpoint{6.130597in}{1.652670in}}%
\pgfpathcurveto{\pgfqpoint{6.125554in}{1.647626in}}{\pgfqpoint{6.122720in}{1.640785in}}{\pgfqpoint{6.122720in}{1.633652in}}%
\pgfpathcurveto{\pgfqpoint{6.122720in}{1.626519in}}{\pgfqpoint{6.125554in}{1.619677in}}{\pgfqpoint{6.130597in}{1.614634in}}%
\pgfpathcurveto{\pgfqpoint{6.135641in}{1.609590in}}{\pgfqpoint{6.142483in}{1.606756in}}{\pgfqpoint{6.149615in}{1.606756in}}%
\pgfpathclose%
\pgfusepath{stroke,fill}%
\end{pgfscope}%
\begin{pgfscope}%
\pgfpathrectangle{\pgfqpoint{4.985294in}{0.500000in}}{\pgfqpoint{1.764706in}{1.700000in}}%
\pgfusepath{clip}%
\pgfsetbuttcap%
\pgfsetroundjoin%
\definecolor{currentfill}{rgb}{0.828528,0.130141,0.293475}%
\pgfsetfillcolor{currentfill}%
\pgfsetlinewidth{0.311001pt}%
\definecolor{currentstroke}{rgb}{1.000000,1.000000,1.000000}%
\pgfsetstrokecolor{currentstroke}%
\pgfsetdash{}{0pt}%
\pgfpathmoveto{\pgfqpoint{5.418228in}{1.789525in}}%
\pgfpathcurveto{\pgfqpoint{5.425361in}{1.789525in}}{\pgfqpoint{5.432203in}{1.792358in}}{\pgfqpoint{5.437246in}{1.797402in}}%
\pgfpathcurveto{\pgfqpoint{5.442290in}{1.802446in}}{\pgfqpoint{5.445124in}{1.809287in}}{\pgfqpoint{5.445124in}{1.816420in}}%
\pgfpathcurveto{\pgfqpoint{5.445124in}{1.823553in}}{\pgfqpoint{5.442290in}{1.830395in}}{\pgfqpoint{5.437246in}{1.835438in}}%
\pgfpathcurveto{\pgfqpoint{5.432203in}{1.840482in}}{\pgfqpoint{5.425361in}{1.843316in}}{\pgfqpoint{5.418228in}{1.843316in}}%
\pgfpathcurveto{\pgfqpoint{5.411095in}{1.843316in}}{\pgfqpoint{5.404254in}{1.840482in}}{\pgfqpoint{5.399210in}{1.835438in}}%
\pgfpathcurveto{\pgfqpoint{5.394166in}{1.830395in}}{\pgfqpoint{5.391332in}{1.823553in}}{\pgfqpoint{5.391332in}{1.816420in}}%
\pgfpathcurveto{\pgfqpoint{5.391332in}{1.809287in}}{\pgfqpoint{5.394166in}{1.802446in}}{\pgfqpoint{5.399210in}{1.797402in}}%
\pgfpathcurveto{\pgfqpoint{5.404254in}{1.792358in}}{\pgfqpoint{5.411095in}{1.789525in}}{\pgfqpoint{5.418228in}{1.789525in}}%
\pgfpathclose%
\pgfusepath{stroke,fill}%
\end{pgfscope}%
\begin{pgfscope}%
\pgfpathrectangle{\pgfqpoint{4.985294in}{0.500000in}}{\pgfqpoint{1.764706in}{1.700000in}}%
\pgfusepath{clip}%
\pgfsetbuttcap%
\pgfsetroundjoin%
\definecolor{currentfill}{rgb}{0.979124,0.903132,0.839793}%
\pgfsetfillcolor{currentfill}%
\pgfsetlinewidth{0.311001pt}%
\definecolor{currentstroke}{rgb}{1.000000,1.000000,1.000000}%
\pgfsetstrokecolor{currentstroke}%
\pgfsetdash{}{0pt}%
\pgfpathmoveto{\pgfqpoint{6.318430in}{1.188731in}}%
\pgfpathcurveto{\pgfqpoint{6.325563in}{1.188731in}}{\pgfqpoint{6.332404in}{1.191565in}}{\pgfqpoint{6.337448in}{1.196609in}}%
\pgfpathcurveto{\pgfqpoint{6.342492in}{1.201652in}}{\pgfqpoint{6.345326in}{1.208494in}}{\pgfqpoint{6.345326in}{1.215627in}}%
\pgfpathcurveto{\pgfqpoint{6.345326in}{1.222760in}}{\pgfqpoint{6.342492in}{1.229601in}}{\pgfqpoint{6.337448in}{1.234645in}}%
\pgfpathcurveto{\pgfqpoint{6.332404in}{1.239689in}}{\pgfqpoint{6.325563in}{1.242522in}}{\pgfqpoint{6.318430in}{1.242522in}}%
\pgfpathcurveto{\pgfqpoint{6.311297in}{1.242522in}}{\pgfqpoint{6.304455in}{1.239689in}}{\pgfqpoint{6.299412in}{1.234645in}}%
\pgfpathcurveto{\pgfqpoint{6.294368in}{1.229601in}}{\pgfqpoint{6.291534in}{1.222760in}}{\pgfqpoint{6.291534in}{1.215627in}}%
\pgfpathcurveto{\pgfqpoint{6.291534in}{1.208494in}}{\pgfqpoint{6.294368in}{1.201652in}}{\pgfqpoint{6.299412in}{1.196609in}}%
\pgfpathcurveto{\pgfqpoint{6.304455in}{1.191565in}}{\pgfqpoint{6.311297in}{1.188731in}}{\pgfqpoint{6.318430in}{1.188731in}}%
\pgfpathclose%
\pgfusepath{stroke,fill}%
\end{pgfscope}%
\begin{pgfscope}%
\pgfpathrectangle{\pgfqpoint{4.985294in}{0.500000in}}{\pgfqpoint{1.764706in}{1.700000in}}%
\pgfusepath{clip}%
\pgfsetbuttcap%
\pgfsetroundjoin%
\definecolor{currentfill}{rgb}{0.852817,0.156578,0.279098}%
\pgfsetfillcolor{currentfill}%
\pgfsetlinewidth{0.311001pt}%
\definecolor{currentstroke}{rgb}{1.000000,1.000000,1.000000}%
\pgfsetstrokecolor{currentstroke}%
\pgfsetdash{}{0pt}%
\pgfpathmoveto{\pgfqpoint{5.805465in}{1.712792in}}%
\pgfpathcurveto{\pgfqpoint{5.812598in}{1.712792in}}{\pgfqpoint{5.819440in}{1.715626in}}{\pgfqpoint{5.824483in}{1.720670in}}%
\pgfpathcurveto{\pgfqpoint{5.829527in}{1.725714in}}{\pgfqpoint{5.832361in}{1.732555in}}{\pgfqpoint{5.832361in}{1.739688in}}%
\pgfpathcurveto{\pgfqpoint{5.832361in}{1.746821in}}{\pgfqpoint{5.829527in}{1.753663in}}{\pgfqpoint{5.824483in}{1.758706in}}%
\pgfpathcurveto{\pgfqpoint{5.819440in}{1.763750in}}{\pgfqpoint{5.812598in}{1.766584in}}{\pgfqpoint{5.805465in}{1.766584in}}%
\pgfpathcurveto{\pgfqpoint{5.798332in}{1.766584in}}{\pgfqpoint{5.791491in}{1.763750in}}{\pgfqpoint{5.786447in}{1.758706in}}%
\pgfpathcurveto{\pgfqpoint{5.781403in}{1.753663in}}{\pgfqpoint{5.778570in}{1.746821in}}{\pgfqpoint{5.778570in}{1.739688in}}%
\pgfpathcurveto{\pgfqpoint{5.778570in}{1.732555in}}{\pgfqpoint{5.781403in}{1.725714in}}{\pgfqpoint{5.786447in}{1.720670in}}%
\pgfpathcurveto{\pgfqpoint{5.791491in}{1.715626in}}{\pgfqpoint{5.798332in}{1.712792in}}{\pgfqpoint{5.805465in}{1.712792in}}%
\pgfpathclose%
\pgfusepath{stroke,fill}%
\end{pgfscope}%
\begin{pgfscope}%
\pgfpathrectangle{\pgfqpoint{4.985294in}{0.500000in}}{\pgfqpoint{1.764706in}{1.700000in}}%
\pgfusepath{clip}%
\pgfsetbuttcap%
\pgfsetroundjoin%
\definecolor{currentfill}{rgb}{0.945204,0.390623,0.270949}%
\pgfsetfillcolor{currentfill}%
\pgfsetlinewidth{0.311001pt}%
\definecolor{currentstroke}{rgb}{1.000000,1.000000,1.000000}%
\pgfsetstrokecolor{currentstroke}%
\pgfsetdash{}{0pt}%
\pgfpathmoveto{\pgfqpoint{5.708017in}{1.732652in}}%
\pgfpathcurveto{\pgfqpoint{5.715150in}{1.732652in}}{\pgfqpoint{5.721991in}{1.735486in}}{\pgfqpoint{5.727035in}{1.740530in}}%
\pgfpathcurveto{\pgfqpoint{5.732079in}{1.745574in}}{\pgfqpoint{5.734913in}{1.752415in}}{\pgfqpoint{5.734913in}{1.759548in}}%
\pgfpathcurveto{\pgfqpoint{5.734913in}{1.766681in}}{\pgfqpoint{5.732079in}{1.773523in}}{\pgfqpoint{5.727035in}{1.778566in}}%
\pgfpathcurveto{\pgfqpoint{5.721991in}{1.783610in}}{\pgfqpoint{5.715150in}{1.786444in}}{\pgfqpoint{5.708017in}{1.786444in}}%
\pgfpathcurveto{\pgfqpoint{5.700884in}{1.786444in}}{\pgfqpoint{5.694042in}{1.783610in}}{\pgfqpoint{5.688999in}{1.778566in}}%
\pgfpathcurveto{\pgfqpoint{5.683955in}{1.773523in}}{\pgfqpoint{5.681121in}{1.766681in}}{\pgfqpoint{5.681121in}{1.759548in}}%
\pgfpathcurveto{\pgfqpoint{5.681121in}{1.752415in}}{\pgfqpoint{5.683955in}{1.745574in}}{\pgfqpoint{5.688999in}{1.740530in}}%
\pgfpathcurveto{\pgfqpoint{5.694042in}{1.735486in}}{\pgfqpoint{5.700884in}{1.732652in}}{\pgfqpoint{5.708017in}{1.732652in}}%
\pgfpathclose%
\pgfusepath{stroke,fill}%
\end{pgfscope}%
\begin{pgfscope}%
\pgfpathrectangle{\pgfqpoint{4.985294in}{0.500000in}}{\pgfqpoint{1.764706in}{1.700000in}}%
\pgfusepath{clip}%
\pgfsetbuttcap%
\pgfsetroundjoin%
\definecolor{currentfill}{rgb}{0.965753,0.732351,0.592427}%
\pgfsetfillcolor{currentfill}%
\pgfsetlinewidth{0.311001pt}%
\definecolor{currentstroke}{rgb}{1.000000,1.000000,1.000000}%
\pgfsetstrokecolor{currentstroke}%
\pgfsetdash{}{0pt}%
\pgfpathmoveto{\pgfqpoint{6.156634in}{0.931815in}}%
\pgfpathcurveto{\pgfqpoint{6.163767in}{0.931815in}}{\pgfqpoint{6.170608in}{0.934649in}}{\pgfqpoint{6.175652in}{0.939692in}}%
\pgfpathcurveto{\pgfqpoint{6.180696in}{0.944736in}}{\pgfqpoint{6.183530in}{0.951578in}}{\pgfqpoint{6.183530in}{0.958711in}}%
\pgfpathcurveto{\pgfqpoint{6.183530in}{0.965843in}}{\pgfqpoint{6.180696in}{0.972685in}}{\pgfqpoint{6.175652in}{0.977729in}}%
\pgfpathcurveto{\pgfqpoint{6.170608in}{0.982772in}}{\pgfqpoint{6.163767in}{0.985606in}}{\pgfqpoint{6.156634in}{0.985606in}}%
\pgfpathcurveto{\pgfqpoint{6.149501in}{0.985606in}}{\pgfqpoint{6.142659in}{0.982772in}}{\pgfqpoint{6.137616in}{0.977729in}}%
\pgfpathcurveto{\pgfqpoint{6.132572in}{0.972685in}}{\pgfqpoint{6.129738in}{0.965843in}}{\pgfqpoint{6.129738in}{0.958711in}}%
\pgfpathcurveto{\pgfqpoint{6.129738in}{0.951578in}}{\pgfqpoint{6.132572in}{0.944736in}}{\pgfqpoint{6.137616in}{0.939692in}}%
\pgfpathcurveto{\pgfqpoint{6.142659in}{0.934649in}}{\pgfqpoint{6.149501in}{0.931815in}}{\pgfqpoint{6.156634in}{0.931815in}}%
\pgfpathclose%
\pgfusepath{stroke,fill}%
\end{pgfscope}%
\begin{pgfscope}%
\pgfpathrectangle{\pgfqpoint{4.985294in}{0.500000in}}{\pgfqpoint{1.764706in}{1.700000in}}%
\pgfusepath{clip}%
\pgfsetbuttcap%
\pgfsetroundjoin%
\definecolor{currentfill}{rgb}{0.975644,0.874038,0.797253}%
\pgfsetfillcolor{currentfill}%
\pgfsetlinewidth{0.311001pt}%
\definecolor{currentstroke}{rgb}{1.000000,1.000000,1.000000}%
\pgfsetstrokecolor{currentstroke}%
\pgfsetdash{}{0pt}%
\pgfpathmoveto{\pgfqpoint{5.375981in}{1.322316in}}%
\pgfpathcurveto{\pgfqpoint{5.383114in}{1.322316in}}{\pgfqpoint{5.389955in}{1.325150in}}{\pgfqpoint{5.394999in}{1.330194in}}%
\pgfpathcurveto{\pgfqpoint{5.400043in}{1.335238in}}{\pgfqpoint{5.402877in}{1.342079in}}{\pgfqpoint{5.402877in}{1.349212in}}%
\pgfpathcurveto{\pgfqpoint{5.402877in}{1.356345in}}{\pgfqpoint{5.400043in}{1.363186in}}{\pgfqpoint{5.394999in}{1.368230in}}%
\pgfpathcurveto{\pgfqpoint{5.389955in}{1.373274in}}{\pgfqpoint{5.383114in}{1.376108in}}{\pgfqpoint{5.375981in}{1.376108in}}%
\pgfpathcurveto{\pgfqpoint{5.368848in}{1.376108in}}{\pgfqpoint{5.362007in}{1.373274in}}{\pgfqpoint{5.356963in}{1.368230in}}%
\pgfpathcurveto{\pgfqpoint{5.351919in}{1.363186in}}{\pgfqpoint{5.349085in}{1.356345in}}{\pgfqpoint{5.349085in}{1.349212in}}%
\pgfpathcurveto{\pgfqpoint{5.349085in}{1.342079in}}{\pgfqpoint{5.351919in}{1.335238in}}{\pgfqpoint{5.356963in}{1.330194in}}%
\pgfpathcurveto{\pgfqpoint{5.362007in}{1.325150in}}{\pgfqpoint{5.368848in}{1.322316in}}{\pgfqpoint{5.375981in}{1.322316in}}%
\pgfpathclose%
\pgfusepath{stroke,fill}%
\end{pgfscope}%
\begin{pgfscope}%
\pgfpathrectangle{\pgfqpoint{4.985294in}{0.500000in}}{\pgfqpoint{1.764706in}{1.700000in}}%
\pgfusepath{clip}%
\pgfsetbuttcap%
\pgfsetroundjoin%
\definecolor{currentfill}{rgb}{0.891169,0.211218,0.255359}%
\pgfsetfillcolor{currentfill}%
\pgfsetlinewidth{0.311001pt}%
\definecolor{currentstroke}{rgb}{1.000000,1.000000,1.000000}%
\pgfsetstrokecolor{currentstroke}%
\pgfsetdash{}{0pt}%
\pgfpathmoveto{\pgfqpoint{5.721898in}{0.894902in}}%
\pgfpathcurveto{\pgfqpoint{5.729031in}{0.894902in}}{\pgfqpoint{5.735872in}{0.897736in}}{\pgfqpoint{5.740916in}{0.902779in}}%
\pgfpathcurveto{\pgfqpoint{5.745960in}{0.907823in}}{\pgfqpoint{5.748794in}{0.914665in}}{\pgfqpoint{5.748794in}{0.921797in}}%
\pgfpathcurveto{\pgfqpoint{5.748794in}{0.928930in}}{\pgfqpoint{5.745960in}{0.935772in}}{\pgfqpoint{5.740916in}{0.940816in}}%
\pgfpathcurveto{\pgfqpoint{5.735872in}{0.945859in}}{\pgfqpoint{5.729031in}{0.948693in}}{\pgfqpoint{5.721898in}{0.948693in}}%
\pgfpathcurveto{\pgfqpoint{5.714765in}{0.948693in}}{\pgfqpoint{5.707924in}{0.945859in}}{\pgfqpoint{5.702880in}{0.940816in}}%
\pgfpathcurveto{\pgfqpoint{5.697836in}{0.935772in}}{\pgfqpoint{5.695002in}{0.928930in}}{\pgfqpoint{5.695002in}{0.921797in}}%
\pgfpathcurveto{\pgfqpoint{5.695002in}{0.914665in}}{\pgfqpoint{5.697836in}{0.907823in}}{\pgfqpoint{5.702880in}{0.902779in}}%
\pgfpathcurveto{\pgfqpoint{5.707924in}{0.897736in}}{\pgfqpoint{5.714765in}{0.894902in}}{\pgfqpoint{5.721898in}{0.894902in}}%
\pgfpathclose%
\pgfusepath{stroke,fill}%
\end{pgfscope}%
\begin{pgfscope}%
\pgfpathrectangle{\pgfqpoint{4.985294in}{0.500000in}}{\pgfqpoint{1.764706in}{1.700000in}}%
\pgfusepath{clip}%
\pgfsetbuttcap%
\pgfsetroundjoin%
\definecolor{currentfill}{rgb}{0.978376,0.897317,0.831308}%
\pgfsetfillcolor{currentfill}%
\pgfsetlinewidth{0.311001pt}%
\definecolor{currentstroke}{rgb}{1.000000,1.000000,1.000000}%
\pgfsetstrokecolor{currentstroke}%
\pgfsetdash{}{0pt}%
\pgfpathmoveto{\pgfqpoint{6.287191in}{1.354211in}}%
\pgfpathcurveto{\pgfqpoint{6.294324in}{1.354211in}}{\pgfqpoint{6.301166in}{1.357044in}}{\pgfqpoint{6.306210in}{1.362088in}}%
\pgfpathcurveto{\pgfqpoint{6.311253in}{1.367132in}}{\pgfqpoint{6.314087in}{1.373973in}}{\pgfqpoint{6.314087in}{1.381106in}}%
\pgfpathcurveto{\pgfqpoint{6.314087in}{1.388239in}}{\pgfqpoint{6.311253in}{1.395081in}}{\pgfqpoint{6.306210in}{1.400124in}}%
\pgfpathcurveto{\pgfqpoint{6.301166in}{1.405168in}}{\pgfqpoint{6.294324in}{1.408002in}}{\pgfqpoint{6.287191in}{1.408002in}}%
\pgfpathcurveto{\pgfqpoint{6.280059in}{1.408002in}}{\pgfqpoint{6.273217in}{1.405168in}}{\pgfqpoint{6.268173in}{1.400124in}}%
\pgfpathcurveto{\pgfqpoint{6.263130in}{1.395081in}}{\pgfqpoint{6.260296in}{1.388239in}}{\pgfqpoint{6.260296in}{1.381106in}}%
\pgfpathcurveto{\pgfqpoint{6.260296in}{1.373973in}}{\pgfqpoint{6.263130in}{1.367132in}}{\pgfqpoint{6.268173in}{1.362088in}}%
\pgfpathcurveto{\pgfqpoint{6.273217in}{1.357044in}}{\pgfqpoint{6.280059in}{1.354211in}}{\pgfqpoint{6.287191in}{1.354211in}}%
\pgfpathclose%
\pgfusepath{stroke,fill}%
\end{pgfscope}%
\begin{pgfscope}%
\pgfpathrectangle{\pgfqpoint{4.985294in}{0.500000in}}{\pgfqpoint{1.764706in}{1.700000in}}%
\pgfusepath{clip}%
\pgfsetbuttcap%
\pgfsetroundjoin%
\definecolor{currentfill}{rgb}{0.972726,0.844889,0.754401}%
\pgfsetfillcolor{currentfill}%
\pgfsetlinewidth{0.311001pt}%
\definecolor{currentstroke}{rgb}{1.000000,1.000000,1.000000}%
\pgfsetstrokecolor{currentstroke}%
\pgfsetdash{}{0pt}%
\pgfpathmoveto{\pgfqpoint{5.405301in}{1.119483in}}%
\pgfpathcurveto{\pgfqpoint{5.412434in}{1.119483in}}{\pgfqpoint{5.419276in}{1.122317in}}{\pgfqpoint{5.424319in}{1.127360in}}%
\pgfpathcurveto{\pgfqpoint{5.429363in}{1.132404in}}{\pgfqpoint{5.432197in}{1.139246in}}{\pgfqpoint{5.432197in}{1.146378in}}%
\pgfpathcurveto{\pgfqpoint{5.432197in}{1.153511in}}{\pgfqpoint{5.429363in}{1.160353in}}{\pgfqpoint{5.424319in}{1.165397in}}%
\pgfpathcurveto{\pgfqpoint{5.419276in}{1.170440in}}{\pgfqpoint{5.412434in}{1.173274in}}{\pgfqpoint{5.405301in}{1.173274in}}%
\pgfpathcurveto{\pgfqpoint{5.398168in}{1.173274in}}{\pgfqpoint{5.391327in}{1.170440in}}{\pgfqpoint{5.386283in}{1.165397in}}%
\pgfpathcurveto{\pgfqpoint{5.381239in}{1.160353in}}{\pgfqpoint{5.378406in}{1.153511in}}{\pgfqpoint{5.378406in}{1.146378in}}%
\pgfpathcurveto{\pgfqpoint{5.378406in}{1.139246in}}{\pgfqpoint{5.381239in}{1.132404in}}{\pgfqpoint{5.386283in}{1.127360in}}%
\pgfpathcurveto{\pgfqpoint{5.391327in}{1.122317in}}{\pgfqpoint{5.398168in}{1.119483in}}{\pgfqpoint{5.405301in}{1.119483in}}%
\pgfpathclose%
\pgfusepath{stroke,fill}%
\end{pgfscope}%
\begin{pgfscope}%
\pgfpathrectangle{\pgfqpoint{4.985294in}{0.500000in}}{\pgfqpoint{1.764706in}{1.700000in}}%
\pgfusepath{clip}%
\pgfsetbuttcap%
\pgfsetroundjoin%
\definecolor{currentfill}{rgb}{0.964920,0.695342,0.545192}%
\pgfsetfillcolor{currentfill}%
\pgfsetlinewidth{0.311001pt}%
\definecolor{currentstroke}{rgb}{1.000000,1.000000,1.000000}%
\pgfsetstrokecolor{currentstroke}%
\pgfsetdash{}{0pt}%
\pgfpathmoveto{\pgfqpoint{6.138532in}{0.925052in}}%
\pgfpathcurveto{\pgfqpoint{6.145665in}{0.925052in}}{\pgfqpoint{6.152507in}{0.927886in}}{\pgfqpoint{6.157551in}{0.932930in}}%
\pgfpathcurveto{\pgfqpoint{6.162594in}{0.937974in}}{\pgfqpoint{6.165428in}{0.944815in}}{\pgfqpoint{6.165428in}{0.951948in}}%
\pgfpathcurveto{\pgfqpoint{6.165428in}{0.959081in}}{\pgfqpoint{6.162594in}{0.965923in}}{\pgfqpoint{6.157551in}{0.970966in}}%
\pgfpathcurveto{\pgfqpoint{6.152507in}{0.976010in}}{\pgfqpoint{6.145665in}{0.978844in}}{\pgfqpoint{6.138532in}{0.978844in}}%
\pgfpathcurveto{\pgfqpoint{6.131400in}{0.978844in}}{\pgfqpoint{6.124558in}{0.976010in}}{\pgfqpoint{6.119514in}{0.970966in}}%
\pgfpathcurveto{\pgfqpoint{6.114471in}{0.965923in}}{\pgfqpoint{6.111637in}{0.959081in}}{\pgfqpoint{6.111637in}{0.951948in}}%
\pgfpathcurveto{\pgfqpoint{6.111637in}{0.944815in}}{\pgfqpoint{6.114471in}{0.937974in}}{\pgfqpoint{6.119514in}{0.932930in}}%
\pgfpathcurveto{\pgfqpoint{6.124558in}{0.927886in}}{\pgfqpoint{6.131400in}{0.925052in}}{\pgfqpoint{6.138532in}{0.925052in}}%
\pgfpathclose%
\pgfusepath{stroke,fill}%
\end{pgfscope}%
\begin{pgfscope}%
\pgfpathrectangle{\pgfqpoint{4.985294in}{0.500000in}}{\pgfqpoint{1.764706in}{1.700000in}}%
\pgfusepath{clip}%
\pgfsetbuttcap%
\pgfsetroundjoin%
\definecolor{currentfill}{rgb}{0.965753,0.732351,0.592427}%
\pgfsetfillcolor{currentfill}%
\pgfsetlinewidth{0.311001pt}%
\definecolor{currentstroke}{rgb}{1.000000,1.000000,1.000000}%
\pgfsetstrokecolor{currentstroke}%
\pgfsetdash{}{0pt}%
\pgfpathmoveto{\pgfqpoint{5.603311in}{0.960925in}}%
\pgfpathcurveto{\pgfqpoint{5.610443in}{0.960925in}}{\pgfqpoint{5.617285in}{0.963759in}}{\pgfqpoint{5.622329in}{0.968802in}}%
\pgfpathcurveto{\pgfqpoint{5.627372in}{0.973846in}}{\pgfqpoint{5.630206in}{0.980688in}}{\pgfqpoint{5.630206in}{0.987820in}}%
\pgfpathcurveto{\pgfqpoint{5.630206in}{0.994953in}}{\pgfqpoint{5.627372in}{1.001795in}}{\pgfqpoint{5.622329in}{1.006839in}}%
\pgfpathcurveto{\pgfqpoint{5.617285in}{1.011882in}}{\pgfqpoint{5.610443in}{1.014716in}}{\pgfqpoint{5.603311in}{1.014716in}}%
\pgfpathcurveto{\pgfqpoint{5.596178in}{1.014716in}}{\pgfqpoint{5.589336in}{1.011882in}}{\pgfqpoint{5.584293in}{1.006839in}}%
\pgfpathcurveto{\pgfqpoint{5.579249in}{1.001795in}}{\pgfqpoint{5.576415in}{0.994953in}}{\pgfqpoint{5.576415in}{0.987820in}}%
\pgfpathcurveto{\pgfqpoint{5.576415in}{0.980688in}}{\pgfqpoint{5.579249in}{0.973846in}}{\pgfqpoint{5.584293in}{0.968802in}}%
\pgfpathcurveto{\pgfqpoint{5.589336in}{0.963759in}}{\pgfqpoint{5.596178in}{0.960925in}}{\pgfqpoint{5.603311in}{0.960925in}}%
\pgfpathclose%
\pgfusepath{stroke,fill}%
\end{pgfscope}%
\begin{pgfscope}%
\pgfpathrectangle{\pgfqpoint{4.985294in}{0.500000in}}{\pgfqpoint{1.764706in}{1.700000in}}%
\pgfusepath{clip}%
\pgfsetbuttcap%
\pgfsetroundjoin%
\definecolor{currentfill}{rgb}{0.730358,0.086862,0.337485}%
\pgfsetfillcolor{currentfill}%
\pgfsetlinewidth{0.311001pt}%
\definecolor{currentstroke}{rgb}{1.000000,1.000000,1.000000}%
\pgfsetstrokecolor{currentstroke}%
\pgfsetdash{}{0pt}%
\pgfpathmoveto{\pgfqpoint{5.253089in}{1.079009in}}%
\pgfpathcurveto{\pgfqpoint{5.260222in}{1.079009in}}{\pgfqpoint{5.267063in}{1.081842in}}{\pgfqpoint{5.272107in}{1.086886in}}%
\pgfpathcurveto{\pgfqpoint{5.277151in}{1.091930in}}{\pgfqpoint{5.279985in}{1.098771in}}{\pgfqpoint{5.279985in}{1.105904in}}%
\pgfpathcurveto{\pgfqpoint{5.279985in}{1.113037in}}{\pgfqpoint{5.277151in}{1.119879in}}{\pgfqpoint{5.272107in}{1.124922in}}%
\pgfpathcurveto{\pgfqpoint{5.267063in}{1.129966in}}{\pgfqpoint{5.260222in}{1.132800in}}{\pgfqpoint{5.253089in}{1.132800in}}%
\pgfpathcurveto{\pgfqpoint{5.245956in}{1.132800in}}{\pgfqpoint{5.239114in}{1.129966in}}{\pgfqpoint{5.234071in}{1.124922in}}%
\pgfpathcurveto{\pgfqpoint{5.229027in}{1.119879in}}{\pgfqpoint{5.226193in}{1.113037in}}{\pgfqpoint{5.226193in}{1.105904in}}%
\pgfpathcurveto{\pgfqpoint{5.226193in}{1.098771in}}{\pgfqpoint{5.229027in}{1.091930in}}{\pgfqpoint{5.234071in}{1.086886in}}%
\pgfpathcurveto{\pgfqpoint{5.239114in}{1.081842in}}{\pgfqpoint{5.245956in}{1.079009in}}{\pgfqpoint{5.253089in}{1.079009in}}%
\pgfpathclose%
\pgfusepath{stroke,fill}%
\end{pgfscope}%
\begin{pgfscope}%
\pgfpathrectangle{\pgfqpoint{4.985294in}{0.500000in}}{\pgfqpoint{1.764706in}{1.700000in}}%
\pgfusepath{clip}%
\pgfsetbuttcap%
\pgfsetroundjoin%
\definecolor{currentfill}{rgb}{0.972726,0.844889,0.754401}%
\pgfsetfillcolor{currentfill}%
\pgfsetlinewidth{0.311001pt}%
\definecolor{currentstroke}{rgb}{1.000000,1.000000,1.000000}%
\pgfsetstrokecolor{currentstroke}%
\pgfsetdash{}{0pt}%
\pgfpathmoveto{\pgfqpoint{5.471744in}{1.017878in}}%
\pgfpathcurveto{\pgfqpoint{5.478877in}{1.017878in}}{\pgfqpoint{5.485719in}{1.020712in}}{\pgfqpoint{5.490763in}{1.025756in}}%
\pgfpathcurveto{\pgfqpoint{5.495806in}{1.030800in}}{\pgfqpoint{5.498640in}{1.037641in}}{\pgfqpoint{5.498640in}{1.044774in}}%
\pgfpathcurveto{\pgfqpoint{5.498640in}{1.051907in}}{\pgfqpoint{5.495806in}{1.058749in}}{\pgfqpoint{5.490763in}{1.063792in}}%
\pgfpathcurveto{\pgfqpoint{5.485719in}{1.068836in}}{\pgfqpoint{5.478877in}{1.071670in}}{\pgfqpoint{5.471744in}{1.071670in}}%
\pgfpathcurveto{\pgfqpoint{5.464612in}{1.071670in}}{\pgfqpoint{5.457770in}{1.068836in}}{\pgfqpoint{5.452726in}{1.063792in}}%
\pgfpathcurveto{\pgfqpoint{5.447683in}{1.058749in}}{\pgfqpoint{5.444849in}{1.051907in}}{\pgfqpoint{5.444849in}{1.044774in}}%
\pgfpathcurveto{\pgfqpoint{5.444849in}{1.037641in}}{\pgfqpoint{5.447683in}{1.030800in}}{\pgfqpoint{5.452726in}{1.025756in}}%
\pgfpathcurveto{\pgfqpoint{5.457770in}{1.020712in}}{\pgfqpoint{5.464612in}{1.017878in}}{\pgfqpoint{5.471744in}{1.017878in}}%
\pgfpathclose%
\pgfusepath{stroke,fill}%
\end{pgfscope}%
\begin{pgfscope}%
\pgfpathrectangle{\pgfqpoint{4.985294in}{0.500000in}}{\pgfqpoint{1.764706in}{1.700000in}}%
\pgfusepath{clip}%
\pgfsetbuttcap%
\pgfsetroundjoin%
\definecolor{currentfill}{rgb}{0.980678,0.914765,0.856766}%
\pgfsetfillcolor{currentfill}%
\pgfsetlinewidth{0.311001pt}%
\definecolor{currentstroke}{rgb}{1.000000,1.000000,1.000000}%
\pgfsetstrokecolor{currentstroke}%
\pgfsetdash{}{0pt}%
\pgfpathmoveto{\pgfqpoint{5.418257in}{1.271782in}}%
\pgfpathcurveto{\pgfqpoint{5.425390in}{1.271782in}}{\pgfqpoint{5.432232in}{1.274616in}}{\pgfqpoint{5.437276in}{1.279659in}}%
\pgfpathcurveto{\pgfqpoint{5.442319in}{1.284703in}}{\pgfqpoint{5.445153in}{1.291545in}}{\pgfqpoint{5.445153in}{1.298677in}}%
\pgfpathcurveto{\pgfqpoint{5.445153in}{1.305810in}}{\pgfqpoint{5.442319in}{1.312652in}}{\pgfqpoint{5.437276in}{1.317696in}}%
\pgfpathcurveto{\pgfqpoint{5.432232in}{1.322739in}}{\pgfqpoint{5.425390in}{1.325573in}}{\pgfqpoint{5.418257in}{1.325573in}}%
\pgfpathcurveto{\pgfqpoint{5.411125in}{1.325573in}}{\pgfqpoint{5.404283in}{1.322739in}}{\pgfqpoint{5.399239in}{1.317696in}}%
\pgfpathcurveto{\pgfqpoint{5.394196in}{1.312652in}}{\pgfqpoint{5.391362in}{1.305810in}}{\pgfqpoint{5.391362in}{1.298677in}}%
\pgfpathcurveto{\pgfqpoint{5.391362in}{1.291545in}}{\pgfqpoint{5.394196in}{1.284703in}}{\pgfqpoint{5.399239in}{1.279659in}}%
\pgfpathcurveto{\pgfqpoint{5.404283in}{1.274616in}}{\pgfqpoint{5.411125in}{1.271782in}}{\pgfqpoint{5.418257in}{1.271782in}}%
\pgfpathclose%
\pgfusepath{stroke,fill}%
\end{pgfscope}%
\begin{pgfscope}%
\pgfpathrectangle{\pgfqpoint{4.985294in}{0.500000in}}{\pgfqpoint{1.764706in}{1.700000in}}%
\pgfusepath{clip}%
\pgfsetbuttcap%
\pgfsetroundjoin%
\definecolor{currentfill}{rgb}{0.980678,0.914765,0.856766}%
\pgfsetfillcolor{currentfill}%
\pgfsetlinewidth{0.311001pt}%
\definecolor{currentstroke}{rgb}{1.000000,1.000000,1.000000}%
\pgfsetstrokecolor{currentstroke}%
\pgfsetdash{}{0pt}%
\pgfpathmoveto{\pgfqpoint{6.325218in}{1.261826in}}%
\pgfpathcurveto{\pgfqpoint{6.332351in}{1.261826in}}{\pgfqpoint{6.339192in}{1.264660in}}{\pgfqpoint{6.344236in}{1.269703in}}%
\pgfpathcurveto{\pgfqpoint{6.349280in}{1.274747in}}{\pgfqpoint{6.352114in}{1.281589in}}{\pgfqpoint{6.352114in}{1.288721in}}%
\pgfpathcurveto{\pgfqpoint{6.352114in}{1.295854in}}{\pgfqpoint{6.349280in}{1.302696in}}{\pgfqpoint{6.344236in}{1.307740in}}%
\pgfpathcurveto{\pgfqpoint{6.339192in}{1.312783in}}{\pgfqpoint{6.332351in}{1.315617in}}{\pgfqpoint{6.325218in}{1.315617in}}%
\pgfpathcurveto{\pgfqpoint{6.318085in}{1.315617in}}{\pgfqpoint{6.311243in}{1.312783in}}{\pgfqpoint{6.306200in}{1.307740in}}%
\pgfpathcurveto{\pgfqpoint{6.301156in}{1.302696in}}{\pgfqpoint{6.298322in}{1.295854in}}{\pgfqpoint{6.298322in}{1.288721in}}%
\pgfpathcurveto{\pgfqpoint{6.298322in}{1.281589in}}{\pgfqpoint{6.301156in}{1.274747in}}{\pgfqpoint{6.306200in}{1.269703in}}%
\pgfpathcurveto{\pgfqpoint{6.311243in}{1.264660in}}{\pgfqpoint{6.318085in}{1.261826in}}{\pgfqpoint{6.325218in}{1.261826in}}%
\pgfpathclose%
\pgfusepath{stroke,fill}%
\end{pgfscope}%
\begin{pgfscope}%
\pgfpathrectangle{\pgfqpoint{4.985294in}{0.500000in}}{\pgfqpoint{1.764706in}{1.700000in}}%
\pgfusepath{clip}%
\pgfsetbuttcap%
\pgfsetroundjoin%
\definecolor{currentfill}{rgb}{0.978376,0.897317,0.831308}%
\pgfsetfillcolor{currentfill}%
\pgfsetlinewidth{0.311001pt}%
\definecolor{currentstroke}{rgb}{1.000000,1.000000,1.000000}%
\pgfsetstrokecolor{currentstroke}%
\pgfsetdash{}{0pt}%
\pgfpathmoveto{\pgfqpoint{5.446431in}{1.304952in}}%
\pgfpathcurveto{\pgfqpoint{5.453564in}{1.304952in}}{\pgfqpoint{5.460406in}{1.307786in}}{\pgfqpoint{5.465449in}{1.312829in}}%
\pgfpathcurveto{\pgfqpoint{5.470493in}{1.317873in}}{\pgfqpoint{5.473327in}{1.324715in}}{\pgfqpoint{5.473327in}{1.331847in}}%
\pgfpathcurveto{\pgfqpoint{5.473327in}{1.338980in}}{\pgfqpoint{5.470493in}{1.345822in}}{\pgfqpoint{5.465449in}{1.350866in}}%
\pgfpathcurveto{\pgfqpoint{5.460406in}{1.355909in}}{\pgfqpoint{5.453564in}{1.358743in}}{\pgfqpoint{5.446431in}{1.358743in}}%
\pgfpathcurveto{\pgfqpoint{5.439298in}{1.358743in}}{\pgfqpoint{5.432457in}{1.355909in}}{\pgfqpoint{5.427413in}{1.350866in}}%
\pgfpathcurveto{\pgfqpoint{5.422369in}{1.345822in}}{\pgfqpoint{5.419536in}{1.338980in}}{\pgfqpoint{5.419536in}{1.331847in}}%
\pgfpathcurveto{\pgfqpoint{5.419536in}{1.324715in}}{\pgfqpoint{5.422369in}{1.317873in}}{\pgfqpoint{5.427413in}{1.312829in}}%
\pgfpathcurveto{\pgfqpoint{5.432457in}{1.307786in}}{\pgfqpoint{5.439298in}{1.304952in}}{\pgfqpoint{5.446431in}{1.304952in}}%
\pgfpathclose%
\pgfusepath{stroke,fill}%
\end{pgfscope}%
\begin{pgfscope}%
\pgfpathrectangle{\pgfqpoint{4.985294in}{0.500000in}}{\pgfqpoint{1.764706in}{1.700000in}}%
\pgfusepath{clip}%
\pgfsetbuttcap%
\pgfsetroundjoin%
\definecolor{currentfill}{rgb}{0.966120,0.744512,0.608720}%
\pgfsetfillcolor{currentfill}%
\pgfsetlinewidth{0.311001pt}%
\definecolor{currentstroke}{rgb}{1.000000,1.000000,1.000000}%
\pgfsetstrokecolor{currentstroke}%
\pgfsetdash{}{0pt}%
\pgfpathmoveto{\pgfqpoint{6.393971in}{1.205107in}}%
\pgfpathcurveto{\pgfqpoint{6.401104in}{1.205107in}}{\pgfqpoint{6.407946in}{1.207941in}}{\pgfqpoint{6.412990in}{1.212984in}}%
\pgfpathcurveto{\pgfqpoint{6.418033in}{1.218028in}}{\pgfqpoint{6.420867in}{1.224870in}}{\pgfqpoint{6.420867in}{1.232002in}}%
\pgfpathcurveto{\pgfqpoint{6.420867in}{1.239135in}}{\pgfqpoint{6.418033in}{1.245977in}}{\pgfqpoint{6.412990in}{1.251021in}}%
\pgfpathcurveto{\pgfqpoint{6.407946in}{1.256064in}}{\pgfqpoint{6.401104in}{1.258898in}}{\pgfqpoint{6.393971in}{1.258898in}}%
\pgfpathcurveto{\pgfqpoint{6.386839in}{1.258898in}}{\pgfqpoint{6.379997in}{1.256064in}}{\pgfqpoint{6.374953in}{1.251021in}}%
\pgfpathcurveto{\pgfqpoint{6.369910in}{1.245977in}}{\pgfqpoint{6.367076in}{1.239135in}}{\pgfqpoint{6.367076in}{1.232002in}}%
\pgfpathcurveto{\pgfqpoint{6.367076in}{1.224870in}}{\pgfqpoint{6.369910in}{1.218028in}}{\pgfqpoint{6.374953in}{1.212984in}}%
\pgfpathcurveto{\pgfqpoint{6.379997in}{1.207941in}}{\pgfqpoint{6.386839in}{1.205107in}}{\pgfqpoint{6.393971in}{1.205107in}}%
\pgfpathclose%
\pgfusepath{stroke,fill}%
\end{pgfscope}%
\begin{pgfscope}%
\pgfpathrectangle{\pgfqpoint{4.985294in}{0.500000in}}{\pgfqpoint{1.764706in}{1.700000in}}%
\pgfusepath{clip}%
\pgfsetbuttcap%
\pgfsetroundjoin%
\definecolor{currentfill}{rgb}{0.975018,0.868213,0.788710}%
\pgfsetfillcolor{currentfill}%
\pgfsetlinewidth{0.311001pt}%
\definecolor{currentstroke}{rgb}{1.000000,1.000000,1.000000}%
\pgfsetstrokecolor{currentstroke}%
\pgfsetdash{}{0pt}%
\pgfpathmoveto{\pgfqpoint{6.243701in}{1.166462in}}%
\pgfpathcurveto{\pgfqpoint{6.250834in}{1.166462in}}{\pgfqpoint{6.257675in}{1.169296in}}{\pgfqpoint{6.262719in}{1.174339in}}%
\pgfpathcurveto{\pgfqpoint{6.267763in}{1.179383in}}{\pgfqpoint{6.270597in}{1.186225in}}{\pgfqpoint{6.270597in}{1.193357in}}%
\pgfpathcurveto{\pgfqpoint{6.270597in}{1.200490in}}{\pgfqpoint{6.267763in}{1.207332in}}{\pgfqpoint{6.262719in}{1.212376in}}%
\pgfpathcurveto{\pgfqpoint{6.257675in}{1.217419in}}{\pgfqpoint{6.250834in}{1.220253in}}{\pgfqpoint{6.243701in}{1.220253in}}%
\pgfpathcurveto{\pgfqpoint{6.236568in}{1.220253in}}{\pgfqpoint{6.229726in}{1.217419in}}{\pgfqpoint{6.224683in}{1.212376in}}%
\pgfpathcurveto{\pgfqpoint{6.219639in}{1.207332in}}{\pgfqpoint{6.216805in}{1.200490in}}{\pgfqpoint{6.216805in}{1.193357in}}%
\pgfpathcurveto{\pgfqpoint{6.216805in}{1.186225in}}{\pgfqpoint{6.219639in}{1.179383in}}{\pgfqpoint{6.224683in}{1.174339in}}%
\pgfpathcurveto{\pgfqpoint{6.229726in}{1.169296in}}{\pgfqpoint{6.236568in}{1.166462in}}{\pgfqpoint{6.243701in}{1.166462in}}%
\pgfpathclose%
\pgfusepath{stroke,fill}%
\end{pgfscope}%
\begin{pgfscope}%
\pgfpathrectangle{\pgfqpoint{4.985294in}{0.500000in}}{\pgfqpoint{1.764706in}{1.700000in}}%
\pgfusepath{clip}%
\pgfsetbuttcap%
\pgfsetroundjoin%
\definecolor{currentfill}{rgb}{0.961734,0.579886,0.418445}%
\pgfsetfillcolor{currentfill}%
\pgfsetlinewidth{0.311001pt}%
\definecolor{currentstroke}{rgb}{1.000000,1.000000,1.000000}%
\pgfsetstrokecolor{currentstroke}%
\pgfsetdash{}{0pt}%
\pgfpathmoveto{\pgfqpoint{5.616762in}{1.730279in}}%
\pgfpathcurveto{\pgfqpoint{5.623895in}{1.730279in}}{\pgfqpoint{5.630737in}{1.733113in}}{\pgfqpoint{5.635780in}{1.738156in}}%
\pgfpathcurveto{\pgfqpoint{5.640824in}{1.743200in}}{\pgfqpoint{5.643658in}{1.750042in}}{\pgfqpoint{5.643658in}{1.757174in}}%
\pgfpathcurveto{\pgfqpoint{5.643658in}{1.764307in}}{\pgfqpoint{5.640824in}{1.771149in}}{\pgfqpoint{5.635780in}{1.776193in}}%
\pgfpathcurveto{\pgfqpoint{5.630737in}{1.781236in}}{\pgfqpoint{5.623895in}{1.784070in}}{\pgfqpoint{5.616762in}{1.784070in}}%
\pgfpathcurveto{\pgfqpoint{5.609629in}{1.784070in}}{\pgfqpoint{5.602788in}{1.781236in}}{\pgfqpoint{5.597744in}{1.776193in}}%
\pgfpathcurveto{\pgfqpoint{5.592700in}{1.771149in}}{\pgfqpoint{5.589866in}{1.764307in}}{\pgfqpoint{5.589866in}{1.757174in}}%
\pgfpathcurveto{\pgfqpoint{5.589866in}{1.750042in}}{\pgfqpoint{5.592700in}{1.743200in}}{\pgfqpoint{5.597744in}{1.738156in}}%
\pgfpathcurveto{\pgfqpoint{5.602788in}{1.733113in}}{\pgfqpoint{5.609629in}{1.730279in}}{\pgfqpoint{5.616762in}{1.730279in}}%
\pgfpathclose%
\pgfusepath{stroke,fill}%
\end{pgfscope}%
\begin{pgfscope}%
\pgfpathrectangle{\pgfqpoint{4.985294in}{0.500000in}}{\pgfqpoint{1.764706in}{1.700000in}}%
\pgfusepath{clip}%
\pgfsetbuttcap%
\pgfsetroundjoin%
\definecolor{currentfill}{rgb}{0.973271,0.850724,0.762998}%
\pgfsetfillcolor{currentfill}%
\pgfsetlinewidth{0.311001pt}%
\definecolor{currentstroke}{rgb}{1.000000,1.000000,1.000000}%
\pgfsetstrokecolor{currentstroke}%
\pgfsetdash{}{0pt}%
\pgfpathmoveto{\pgfqpoint{5.466789in}{1.593282in}}%
\pgfpathcurveto{\pgfqpoint{5.473922in}{1.593282in}}{\pgfqpoint{5.480764in}{1.596116in}}{\pgfqpoint{5.485807in}{1.601160in}}%
\pgfpathcurveto{\pgfqpoint{5.490851in}{1.606204in}}{\pgfqpoint{5.493685in}{1.613045in}}{\pgfqpoint{5.493685in}{1.620178in}}%
\pgfpathcurveto{\pgfqpoint{5.493685in}{1.627311in}}{\pgfqpoint{5.490851in}{1.634153in}}{\pgfqpoint{5.485807in}{1.639196in}}%
\pgfpathcurveto{\pgfqpoint{5.480764in}{1.644240in}}{\pgfqpoint{5.473922in}{1.647074in}}{\pgfqpoint{5.466789in}{1.647074in}}%
\pgfpathcurveto{\pgfqpoint{5.459656in}{1.647074in}}{\pgfqpoint{5.452815in}{1.644240in}}{\pgfqpoint{5.447771in}{1.639196in}}%
\pgfpathcurveto{\pgfqpoint{5.442727in}{1.634153in}}{\pgfqpoint{5.439894in}{1.627311in}}{\pgfqpoint{5.439894in}{1.620178in}}%
\pgfpathcurveto{\pgfqpoint{5.439894in}{1.613045in}}{\pgfqpoint{5.442727in}{1.606204in}}{\pgfqpoint{5.447771in}{1.601160in}}%
\pgfpathcurveto{\pgfqpoint{5.452815in}{1.596116in}}{\pgfqpoint{5.459656in}{1.593282in}}{\pgfqpoint{5.466789in}{1.593282in}}%
\pgfpathclose%
\pgfusepath{stroke,fill}%
\end{pgfscope}%
\begin{pgfscope}%
\pgfpathrectangle{\pgfqpoint{4.985294in}{0.500000in}}{\pgfqpoint{1.764706in}{1.700000in}}%
\pgfusepath{clip}%
\pgfsetbuttcap%
\pgfsetroundjoin%
\definecolor{currentfill}{rgb}{0.975018,0.868213,0.788710}%
\pgfsetfillcolor{currentfill}%
\pgfsetlinewidth{0.311001pt}%
\definecolor{currentstroke}{rgb}{1.000000,1.000000,1.000000}%
\pgfsetstrokecolor{currentstroke}%
\pgfsetdash{}{0pt}%
\pgfpathmoveto{\pgfqpoint{5.417974in}{1.121779in}}%
\pgfpathcurveto{\pgfqpoint{5.425106in}{1.121779in}}{\pgfqpoint{5.431948in}{1.124613in}}{\pgfqpoint{5.436992in}{1.129656in}}%
\pgfpathcurveto{\pgfqpoint{5.442035in}{1.134700in}}{\pgfqpoint{5.444869in}{1.141542in}}{\pgfqpoint{5.444869in}{1.148674in}}%
\pgfpathcurveto{\pgfqpoint{5.444869in}{1.155807in}}{\pgfqpoint{5.442035in}{1.162649in}}{\pgfqpoint{5.436992in}{1.167692in}}%
\pgfpathcurveto{\pgfqpoint{5.431948in}{1.172736in}}{\pgfqpoint{5.425106in}{1.175570in}}{\pgfqpoint{5.417974in}{1.175570in}}%
\pgfpathcurveto{\pgfqpoint{5.410841in}{1.175570in}}{\pgfqpoint{5.403999in}{1.172736in}}{\pgfqpoint{5.398956in}{1.167692in}}%
\pgfpathcurveto{\pgfqpoint{5.393912in}{1.162649in}}{\pgfqpoint{5.391078in}{1.155807in}}{\pgfqpoint{5.391078in}{1.148674in}}%
\pgfpathcurveto{\pgfqpoint{5.391078in}{1.141542in}}{\pgfqpoint{5.393912in}{1.134700in}}{\pgfqpoint{5.398956in}{1.129656in}}%
\pgfpathcurveto{\pgfqpoint{5.403999in}{1.124613in}}{\pgfqpoint{5.410841in}{1.121779in}}{\pgfqpoint{5.417974in}{1.121779in}}%
\pgfpathclose%
\pgfusepath{stroke,fill}%
\end{pgfscope}%
\begin{pgfscope}%
\pgfpathrectangle{\pgfqpoint{4.985294in}{0.500000in}}{\pgfqpoint{1.764706in}{1.700000in}}%
\pgfusepath{clip}%
\pgfsetbuttcap%
\pgfsetroundjoin%
\definecolor{currentfill}{rgb}{0.976961,0.885681,0.814303}%
\pgfsetfillcolor{currentfill}%
\pgfsetlinewidth{0.311001pt}%
\definecolor{currentstroke}{rgb}{1.000000,1.000000,1.000000}%
\pgfsetstrokecolor{currentstroke}%
\pgfsetdash{}{0pt}%
\pgfpathmoveto{\pgfqpoint{5.396928in}{1.213768in}}%
\pgfpathcurveto{\pgfqpoint{5.404061in}{1.213768in}}{\pgfqpoint{5.410902in}{1.216602in}}{\pgfqpoint{5.415946in}{1.221646in}}%
\pgfpathcurveto{\pgfqpoint{5.420990in}{1.226690in}}{\pgfqpoint{5.423824in}{1.233531in}}{\pgfqpoint{5.423824in}{1.240664in}}%
\pgfpathcurveto{\pgfqpoint{5.423824in}{1.247797in}}{\pgfqpoint{5.420990in}{1.254639in}}{\pgfqpoint{5.415946in}{1.259682in}}%
\pgfpathcurveto{\pgfqpoint{5.410902in}{1.264726in}}{\pgfqpoint{5.404061in}{1.267560in}}{\pgfqpoint{5.396928in}{1.267560in}}%
\pgfpathcurveto{\pgfqpoint{5.389795in}{1.267560in}}{\pgfqpoint{5.382953in}{1.264726in}}{\pgfqpoint{5.377910in}{1.259682in}}%
\pgfpathcurveto{\pgfqpoint{5.372866in}{1.254639in}}{\pgfqpoint{5.370032in}{1.247797in}}{\pgfqpoint{5.370032in}{1.240664in}}%
\pgfpathcurveto{\pgfqpoint{5.370032in}{1.233531in}}{\pgfqpoint{5.372866in}{1.226690in}}{\pgfqpoint{5.377910in}{1.221646in}}%
\pgfpathcurveto{\pgfqpoint{5.382953in}{1.216602in}}{\pgfqpoint{5.389795in}{1.213768in}}{\pgfqpoint{5.396928in}{1.213768in}}%
\pgfpathclose%
\pgfusepath{stroke,fill}%
\end{pgfscope}%
\begin{pgfscope}%
\pgfpathrectangle{\pgfqpoint{4.985294in}{0.500000in}}{\pgfqpoint{1.764706in}{1.700000in}}%
\pgfusepath{clip}%
\pgfsetbuttcap%
\pgfsetroundjoin%
\definecolor{currentfill}{rgb}{0.963559,0.632016,0.472047}%
\pgfsetfillcolor{currentfill}%
\pgfsetlinewidth{0.311001pt}%
\definecolor{currentstroke}{rgb}{1.000000,1.000000,1.000000}%
\pgfsetstrokecolor{currentstroke}%
\pgfsetdash{}{0pt}%
\pgfpathmoveto{\pgfqpoint{5.574236in}{1.744969in}}%
\pgfpathcurveto{\pgfqpoint{5.581369in}{1.744969in}}{\pgfqpoint{5.588210in}{1.747803in}}{\pgfqpoint{5.593254in}{1.752846in}}%
\pgfpathcurveto{\pgfqpoint{5.598298in}{1.757890in}}{\pgfqpoint{5.601132in}{1.764732in}}{\pgfqpoint{5.601132in}{1.771865in}}%
\pgfpathcurveto{\pgfqpoint{5.601132in}{1.778997in}}{\pgfqpoint{5.598298in}{1.785839in}}{\pgfqpoint{5.593254in}{1.790883in}}%
\pgfpathcurveto{\pgfqpoint{5.588210in}{1.795926in}}{\pgfqpoint{5.581369in}{1.798760in}}{\pgfqpoint{5.574236in}{1.798760in}}%
\pgfpathcurveto{\pgfqpoint{5.567103in}{1.798760in}}{\pgfqpoint{5.560261in}{1.795926in}}{\pgfqpoint{5.555218in}{1.790883in}}%
\pgfpathcurveto{\pgfqpoint{5.550174in}{1.785839in}}{\pgfqpoint{5.547340in}{1.778997in}}{\pgfqpoint{5.547340in}{1.771865in}}%
\pgfpathcurveto{\pgfqpoint{5.547340in}{1.764732in}}{\pgfqpoint{5.550174in}{1.757890in}}{\pgfqpoint{5.555218in}{1.752846in}}%
\pgfpathcurveto{\pgfqpoint{5.560261in}{1.747803in}}{\pgfqpoint{5.567103in}{1.744969in}}{\pgfqpoint{5.574236in}{1.744969in}}%
\pgfpathclose%
\pgfusepath{stroke,fill}%
\end{pgfscope}%
\begin{pgfscope}%
\pgfpathrectangle{\pgfqpoint{4.985294in}{0.500000in}}{\pgfqpoint{1.764706in}{1.700000in}}%
\pgfusepath{clip}%
\pgfsetbuttcap%
\pgfsetroundjoin%
\definecolor{currentfill}{rgb}{0.919781,0.275262,0.242460}%
\pgfsetfillcolor{currentfill}%
\pgfsetlinewidth{0.311001pt}%
\definecolor{currentstroke}{rgb}{1.000000,1.000000,1.000000}%
\pgfsetstrokecolor{currentstroke}%
\pgfsetdash{}{0pt}%
\pgfpathmoveto{\pgfqpoint{5.256063in}{1.287323in}}%
\pgfpathcurveto{\pgfqpoint{5.263196in}{1.287323in}}{\pgfqpoint{5.270037in}{1.290157in}}{\pgfqpoint{5.275081in}{1.295201in}}%
\pgfpathcurveto{\pgfqpoint{5.280125in}{1.300244in}}{\pgfqpoint{5.282959in}{1.307086in}}{\pgfqpoint{5.282959in}{1.314219in}}%
\pgfpathcurveto{\pgfqpoint{5.282959in}{1.321352in}}{\pgfqpoint{5.280125in}{1.328193in}}{\pgfqpoint{5.275081in}{1.333237in}}%
\pgfpathcurveto{\pgfqpoint{5.270037in}{1.338281in}}{\pgfqpoint{5.263196in}{1.341114in}}{\pgfqpoint{5.256063in}{1.341114in}}%
\pgfpathcurveto{\pgfqpoint{5.248930in}{1.341114in}}{\pgfqpoint{5.242088in}{1.338281in}}{\pgfqpoint{5.237045in}{1.333237in}}%
\pgfpathcurveto{\pgfqpoint{5.232001in}{1.328193in}}{\pgfqpoint{5.229167in}{1.321352in}}{\pgfqpoint{5.229167in}{1.314219in}}%
\pgfpathcurveto{\pgfqpoint{5.229167in}{1.307086in}}{\pgfqpoint{5.232001in}{1.300244in}}{\pgfqpoint{5.237045in}{1.295201in}}%
\pgfpathcurveto{\pgfqpoint{5.242088in}{1.290157in}}{\pgfqpoint{5.248930in}{1.287323in}}{\pgfqpoint{5.256063in}{1.287323in}}%
\pgfpathclose%
\pgfusepath{stroke,fill}%
\end{pgfscope}%
\begin{pgfscope}%
\pgfpathrectangle{\pgfqpoint{4.985294in}{0.500000in}}{\pgfqpoint{1.764706in}{1.700000in}}%
\pgfusepath{clip}%
\pgfsetbuttcap%
\pgfsetroundjoin%
\definecolor{currentfill}{rgb}{0.973832,0.856556,0.771584}%
\pgfsetfillcolor{currentfill}%
\pgfsetlinewidth{0.311001pt}%
\definecolor{currentstroke}{rgb}{1.000000,1.000000,1.000000}%
\pgfsetstrokecolor{currentstroke}%
\pgfsetdash{}{0pt}%
\pgfpathmoveto{\pgfqpoint{5.438034in}{1.559621in}}%
\pgfpathcurveto{\pgfqpoint{5.445167in}{1.559621in}}{\pgfqpoint{5.452009in}{1.562455in}}{\pgfqpoint{5.457052in}{1.567498in}}%
\pgfpathcurveto{\pgfqpoint{5.462096in}{1.572542in}}{\pgfqpoint{5.464930in}{1.579384in}}{\pgfqpoint{5.464930in}{1.586517in}}%
\pgfpathcurveto{\pgfqpoint{5.464930in}{1.593649in}}{\pgfqpoint{5.462096in}{1.600491in}}{\pgfqpoint{5.457052in}{1.605535in}}%
\pgfpathcurveto{\pgfqpoint{5.452009in}{1.610578in}}{\pgfqpoint{5.445167in}{1.613412in}}{\pgfqpoint{5.438034in}{1.613412in}}%
\pgfpathcurveto{\pgfqpoint{5.430901in}{1.613412in}}{\pgfqpoint{5.424060in}{1.610578in}}{\pgfqpoint{5.419016in}{1.605535in}}%
\pgfpathcurveto{\pgfqpoint{5.413972in}{1.600491in}}{\pgfqpoint{5.411139in}{1.593649in}}{\pgfqpoint{5.411139in}{1.586517in}}%
\pgfpathcurveto{\pgfqpoint{5.411139in}{1.579384in}}{\pgfqpoint{5.413972in}{1.572542in}}{\pgfqpoint{5.419016in}{1.567498in}}%
\pgfpathcurveto{\pgfqpoint{5.424060in}{1.562455in}}{\pgfqpoint{5.430901in}{1.559621in}}{\pgfqpoint{5.438034in}{1.559621in}}%
\pgfpathclose%
\pgfusepath{stroke,fill}%
\end{pgfscope}%
\begin{pgfscope}%
\pgfpathrectangle{\pgfqpoint{4.985294in}{0.500000in}}{\pgfqpoint{1.764706in}{1.700000in}}%
\pgfusepath{clip}%
\pgfsetbuttcap%
\pgfsetroundjoin%
\definecolor{currentfill}{rgb}{0.973271,0.850724,0.762998}%
\pgfsetfillcolor{currentfill}%
\pgfsetlinewidth{0.311001pt}%
\definecolor{currentstroke}{rgb}{1.000000,1.000000,1.000000}%
\pgfsetstrokecolor{currentstroke}%
\pgfsetdash{}{0pt}%
\pgfpathmoveto{\pgfqpoint{6.354920in}{1.180311in}}%
\pgfpathcurveto{\pgfqpoint{6.362053in}{1.180311in}}{\pgfqpoint{6.368894in}{1.183145in}}{\pgfqpoint{6.373938in}{1.188188in}}%
\pgfpathcurveto{\pgfqpoint{6.378982in}{1.193232in}}{\pgfqpoint{6.381816in}{1.200074in}}{\pgfqpoint{6.381816in}{1.207206in}}%
\pgfpathcurveto{\pgfqpoint{6.381816in}{1.214339in}}{\pgfqpoint{6.378982in}{1.221181in}}{\pgfqpoint{6.373938in}{1.226224in}}%
\pgfpathcurveto{\pgfqpoint{6.368894in}{1.231268in}}{\pgfqpoint{6.362053in}{1.234102in}}{\pgfqpoint{6.354920in}{1.234102in}}%
\pgfpathcurveto{\pgfqpoint{6.347787in}{1.234102in}}{\pgfqpoint{6.340945in}{1.231268in}}{\pgfqpoint{6.335902in}{1.226224in}}%
\pgfpathcurveto{\pgfqpoint{6.330858in}{1.221181in}}{\pgfqpoint{6.328024in}{1.214339in}}{\pgfqpoint{6.328024in}{1.207206in}}%
\pgfpathcurveto{\pgfqpoint{6.328024in}{1.200074in}}{\pgfqpoint{6.330858in}{1.193232in}}{\pgfqpoint{6.335902in}{1.188188in}}%
\pgfpathcurveto{\pgfqpoint{6.340945in}{1.183145in}}{\pgfqpoint{6.347787in}{1.180311in}}{\pgfqpoint{6.354920in}{1.180311in}}%
\pgfpathclose%
\pgfusepath{stroke,fill}%
\end{pgfscope}%
\begin{pgfscope}%
\pgfpathrectangle{\pgfqpoint{4.985294in}{0.500000in}}{\pgfqpoint{1.764706in}{1.700000in}}%
\pgfusepath{clip}%
\pgfsetbuttcap%
\pgfsetroundjoin%
\definecolor{currentfill}{rgb}{0.965302,0.713942,0.568499}%
\pgfsetfillcolor{currentfill}%
\pgfsetlinewidth{0.311001pt}%
\definecolor{currentstroke}{rgb}{1.000000,1.000000,1.000000}%
\pgfsetstrokecolor{currentstroke}%
\pgfsetdash{}{0pt}%
\pgfpathmoveto{\pgfqpoint{6.269386in}{1.690822in}}%
\pgfpathcurveto{\pgfqpoint{6.276519in}{1.690822in}}{\pgfqpoint{6.283361in}{1.693656in}}{\pgfqpoint{6.288405in}{1.698700in}}%
\pgfpathcurveto{\pgfqpoint{6.293448in}{1.703743in}}{\pgfqpoint{6.296282in}{1.710585in}}{\pgfqpoint{6.296282in}{1.717718in}}%
\pgfpathcurveto{\pgfqpoint{6.296282in}{1.724851in}}{\pgfqpoint{6.293448in}{1.731692in}}{\pgfqpoint{6.288405in}{1.736736in}}%
\pgfpathcurveto{\pgfqpoint{6.283361in}{1.741780in}}{\pgfqpoint{6.276519in}{1.744613in}}{\pgfqpoint{6.269386in}{1.744613in}}%
\pgfpathcurveto{\pgfqpoint{6.262254in}{1.744613in}}{\pgfqpoint{6.255412in}{1.741780in}}{\pgfqpoint{6.250368in}{1.736736in}}%
\pgfpathcurveto{\pgfqpoint{6.245325in}{1.731692in}}{\pgfqpoint{6.242491in}{1.724851in}}{\pgfqpoint{6.242491in}{1.717718in}}%
\pgfpathcurveto{\pgfqpoint{6.242491in}{1.710585in}}{\pgfqpoint{6.245325in}{1.703743in}}{\pgfqpoint{6.250368in}{1.698700in}}%
\pgfpathcurveto{\pgfqpoint{6.255412in}{1.693656in}}{\pgfqpoint{6.262254in}{1.690822in}}{\pgfqpoint{6.269386in}{1.690822in}}%
\pgfpathclose%
\pgfusepath{stroke,fill}%
\end{pgfscope}%
\begin{pgfscope}%
\pgfpathrectangle{\pgfqpoint{4.985294in}{0.500000in}}{\pgfqpoint{1.764706in}{1.700000in}}%
\pgfusepath{clip}%
\pgfsetbuttcap%
\pgfsetroundjoin%
\definecolor{currentfill}{rgb}{0.965169,0.707764,0.560659}%
\pgfsetfillcolor{currentfill}%
\pgfsetlinewidth{0.311001pt}%
\definecolor{currentstroke}{rgb}{1.000000,1.000000,1.000000}%
\pgfsetstrokecolor{currentstroke}%
\pgfsetdash{}{0pt}%
\pgfpathmoveto{\pgfqpoint{5.450283in}{0.943187in}}%
\pgfpathcurveto{\pgfqpoint{5.457416in}{0.943187in}}{\pgfqpoint{5.464257in}{0.946021in}}{\pgfqpoint{5.469301in}{0.951064in}}%
\pgfpathcurveto{\pgfqpoint{5.474345in}{0.956108in}}{\pgfqpoint{5.477179in}{0.962950in}}{\pgfqpoint{5.477179in}{0.970082in}}%
\pgfpathcurveto{\pgfqpoint{5.477179in}{0.977215in}}{\pgfqpoint{5.474345in}{0.984057in}}{\pgfqpoint{5.469301in}{0.989101in}}%
\pgfpathcurveto{\pgfqpoint{5.464257in}{0.994144in}}{\pgfqpoint{5.457416in}{0.996978in}}{\pgfqpoint{5.450283in}{0.996978in}}%
\pgfpathcurveto{\pgfqpoint{5.443150in}{0.996978in}}{\pgfqpoint{5.436308in}{0.994144in}}{\pgfqpoint{5.431265in}{0.989101in}}%
\pgfpathcurveto{\pgfqpoint{5.426221in}{0.984057in}}{\pgfqpoint{5.423387in}{0.977215in}}{\pgfqpoint{5.423387in}{0.970082in}}%
\pgfpathcurveto{\pgfqpoint{5.423387in}{0.962950in}}{\pgfqpoint{5.426221in}{0.956108in}}{\pgfqpoint{5.431265in}{0.951064in}}%
\pgfpathcurveto{\pgfqpoint{5.436308in}{0.946021in}}{\pgfqpoint{5.443150in}{0.943187in}}{\pgfqpoint{5.450283in}{0.943187in}}%
\pgfpathclose%
\pgfusepath{stroke,fill}%
\end{pgfscope}%
\begin{pgfscope}%
\pgfpathrectangle{\pgfqpoint{4.985294in}{0.500000in}}{\pgfqpoint{1.764706in}{1.700000in}}%
\pgfusepath{clip}%
\pgfsetbuttcap%
\pgfsetroundjoin%
\definecolor{currentfill}{rgb}{0.584229,0.109227,0.358485}%
\pgfsetfillcolor{currentfill}%
\pgfsetlinewidth{0.311001pt}%
\definecolor{currentstroke}{rgb}{1.000000,1.000000,1.000000}%
\pgfsetstrokecolor{currentstroke}%
\pgfsetdash{}{0pt}%
\pgfpathmoveto{\pgfqpoint{6.058169in}{1.435390in}}%
\pgfpathcurveto{\pgfqpoint{6.065302in}{1.435390in}}{\pgfqpoint{6.072143in}{1.438224in}}{\pgfqpoint{6.077187in}{1.443268in}}%
\pgfpathcurveto{\pgfqpoint{6.082231in}{1.448311in}}{\pgfqpoint{6.085064in}{1.455153in}}{\pgfqpoint{6.085064in}{1.462286in}}%
\pgfpathcurveto{\pgfqpoint{6.085064in}{1.469418in}}{\pgfqpoint{6.082231in}{1.476260in}}{\pgfqpoint{6.077187in}{1.481304in}}%
\pgfpathcurveto{\pgfqpoint{6.072143in}{1.486347in}}{\pgfqpoint{6.065302in}{1.489181in}}{\pgfqpoint{6.058169in}{1.489181in}}%
\pgfpathcurveto{\pgfqpoint{6.051036in}{1.489181in}}{\pgfqpoint{6.044194in}{1.486347in}}{\pgfqpoint{6.039151in}{1.481304in}}%
\pgfpathcurveto{\pgfqpoint{6.034107in}{1.476260in}}{\pgfqpoint{6.031273in}{1.469418in}}{\pgfqpoint{6.031273in}{1.462286in}}%
\pgfpathcurveto{\pgfqpoint{6.031273in}{1.455153in}}{\pgfqpoint{6.034107in}{1.448311in}}{\pgfqpoint{6.039151in}{1.443268in}}%
\pgfpathcurveto{\pgfqpoint{6.044194in}{1.438224in}}{\pgfqpoint{6.051036in}{1.435390in}}{\pgfqpoint{6.058169in}{1.435390in}}%
\pgfpathclose%
\pgfusepath{stroke,fill}%
\end{pgfscope}%
\begin{pgfscope}%
\pgfpathrectangle{\pgfqpoint{4.985294in}{0.500000in}}{\pgfqpoint{1.764706in}{1.700000in}}%
\pgfusepath{clip}%
\pgfsetbuttcap%
\pgfsetroundjoin%
\definecolor{currentfill}{rgb}{0.963190,0.619109,0.458249}%
\pgfsetfillcolor{currentfill}%
\pgfsetlinewidth{0.311001pt}%
\definecolor{currentstroke}{rgb}{1.000000,1.000000,1.000000}%
\pgfsetstrokecolor{currentstroke}%
\pgfsetdash{}{0pt}%
\pgfpathmoveto{\pgfqpoint{5.612628in}{1.049457in}}%
\pgfpathcurveto{\pgfqpoint{5.619761in}{1.049457in}}{\pgfqpoint{5.626602in}{1.052291in}}{\pgfqpoint{5.631646in}{1.057335in}}%
\pgfpathcurveto{\pgfqpoint{5.636690in}{1.062378in}}{\pgfqpoint{5.639524in}{1.069220in}}{\pgfqpoint{5.639524in}{1.076353in}}%
\pgfpathcurveto{\pgfqpoint{5.639524in}{1.083486in}}{\pgfqpoint{5.636690in}{1.090327in}}{\pgfqpoint{5.631646in}{1.095371in}}%
\pgfpathcurveto{\pgfqpoint{5.626602in}{1.100415in}}{\pgfqpoint{5.619761in}{1.103249in}}{\pgfqpoint{5.612628in}{1.103249in}}%
\pgfpathcurveto{\pgfqpoint{5.605495in}{1.103249in}}{\pgfqpoint{5.598653in}{1.100415in}}{\pgfqpoint{5.593610in}{1.095371in}}%
\pgfpathcurveto{\pgfqpoint{5.588566in}{1.090327in}}{\pgfqpoint{5.585732in}{1.083486in}}{\pgfqpoint{5.585732in}{1.076353in}}%
\pgfpathcurveto{\pgfqpoint{5.585732in}{1.069220in}}{\pgfqpoint{5.588566in}{1.062378in}}{\pgfqpoint{5.593610in}{1.057335in}}%
\pgfpathcurveto{\pgfqpoint{5.598653in}{1.052291in}}{\pgfqpoint{5.605495in}{1.049457in}}{\pgfqpoint{5.612628in}{1.049457in}}%
\pgfpathclose%
\pgfusepath{stroke,fill}%
\end{pgfscope}%
\begin{pgfscope}%
\pgfpathrectangle{\pgfqpoint{4.985294in}{0.500000in}}{\pgfqpoint{1.764706in}{1.700000in}}%
\pgfusepath{clip}%
\pgfsetbuttcap%
\pgfsetroundjoin%
\definecolor{currentfill}{rgb}{0.964433,0.670254,0.515093}%
\pgfsetfillcolor{currentfill}%
\pgfsetlinewidth{0.311001pt}%
\definecolor{currentstroke}{rgb}{1.000000,1.000000,1.000000}%
\pgfsetstrokecolor{currentstroke}%
\pgfsetdash{}{0pt}%
\pgfpathmoveto{\pgfqpoint{6.155473in}{0.896587in}}%
\pgfpathcurveto{\pgfqpoint{6.162605in}{0.896587in}}{\pgfqpoint{6.169447in}{0.899421in}}{\pgfqpoint{6.174491in}{0.904464in}}%
\pgfpathcurveto{\pgfqpoint{6.179534in}{0.909508in}}{\pgfqpoint{6.182368in}{0.916350in}}{\pgfqpoint{6.182368in}{0.923483in}}%
\pgfpathcurveto{\pgfqpoint{6.182368in}{0.930615in}}{\pgfqpoint{6.179534in}{0.937457in}}{\pgfqpoint{6.174491in}{0.942501in}}%
\pgfpathcurveto{\pgfqpoint{6.169447in}{0.947544in}}{\pgfqpoint{6.162605in}{0.950378in}}{\pgfqpoint{6.155473in}{0.950378in}}%
\pgfpathcurveto{\pgfqpoint{6.148340in}{0.950378in}}{\pgfqpoint{6.141498in}{0.947544in}}{\pgfqpoint{6.136454in}{0.942501in}}%
\pgfpathcurveto{\pgfqpoint{6.131411in}{0.937457in}}{\pgfqpoint{6.128577in}{0.930615in}}{\pgfqpoint{6.128577in}{0.923483in}}%
\pgfpathcurveto{\pgfqpoint{6.128577in}{0.916350in}}{\pgfqpoint{6.131411in}{0.909508in}}{\pgfqpoint{6.136454in}{0.904464in}}%
\pgfpathcurveto{\pgfqpoint{6.141498in}{0.899421in}}{\pgfqpoint{6.148340in}{0.896587in}}{\pgfqpoint{6.155473in}{0.896587in}}%
\pgfpathclose%
\pgfusepath{stroke,fill}%
\end{pgfscope}%
\begin{pgfscope}%
\pgfpathrectangle{\pgfqpoint{4.985294in}{0.500000in}}{\pgfqpoint{1.764706in}{1.700000in}}%
\pgfusepath{clip}%
\pgfsetbuttcap%
\pgfsetroundjoin%
\definecolor{currentfill}{rgb}{0.962532,0.599594,0.438051}%
\pgfsetfillcolor{currentfill}%
\pgfsetlinewidth{0.311001pt}%
\definecolor{currentstroke}{rgb}{1.000000,1.000000,1.000000}%
\pgfsetstrokecolor{currentstroke}%
\pgfsetdash{}{0pt}%
\pgfpathmoveto{\pgfqpoint{5.389983in}{0.973594in}}%
\pgfpathcurveto{\pgfqpoint{5.397116in}{0.973594in}}{\pgfqpoint{5.403958in}{0.976427in}}{\pgfqpoint{5.409001in}{0.981471in}}%
\pgfpathcurveto{\pgfqpoint{5.414045in}{0.986515in}}{\pgfqpoint{5.416879in}{0.993356in}}{\pgfqpoint{5.416879in}{1.000489in}}%
\pgfpathcurveto{\pgfqpoint{5.416879in}{1.007622in}}{\pgfqpoint{5.414045in}{1.014464in}}{\pgfqpoint{5.409001in}{1.019507in}}%
\pgfpathcurveto{\pgfqpoint{5.403958in}{1.024551in}}{\pgfqpoint{5.397116in}{1.027385in}}{\pgfqpoint{5.389983in}{1.027385in}}%
\pgfpathcurveto{\pgfqpoint{5.382850in}{1.027385in}}{\pgfqpoint{5.376009in}{1.024551in}}{\pgfqpoint{5.370965in}{1.019507in}}%
\pgfpathcurveto{\pgfqpoint{5.365921in}{1.014464in}}{\pgfqpoint{5.363087in}{1.007622in}}{\pgfqpoint{5.363087in}{1.000489in}}%
\pgfpathcurveto{\pgfqpoint{5.363087in}{0.993356in}}{\pgfqpoint{5.365921in}{0.986515in}}{\pgfqpoint{5.370965in}{0.981471in}}%
\pgfpathcurveto{\pgfqpoint{5.376009in}{0.976427in}}{\pgfqpoint{5.382850in}{0.973594in}}{\pgfqpoint{5.389983in}{0.973594in}}%
\pgfpathclose%
\pgfusepath{stroke,fill}%
\end{pgfscope}%
\begin{pgfscope}%
\pgfpathrectangle{\pgfqpoint{4.985294in}{0.500000in}}{\pgfqpoint{1.764706in}{1.700000in}}%
\pgfusepath{clip}%
\pgfsetbuttcap%
\pgfsetroundjoin%
\definecolor{currentfill}{rgb}{0.944085,0.383081,0.267220}%
\pgfsetfillcolor{currentfill}%
\pgfsetlinewidth{0.311001pt}%
\definecolor{currentstroke}{rgb}{1.000000,1.000000,1.000000}%
\pgfsetstrokecolor{currentstroke}%
\pgfsetdash{}{0pt}%
\pgfpathmoveto{\pgfqpoint{6.311421in}{0.903564in}}%
\pgfpathcurveto{\pgfqpoint{6.318554in}{0.903564in}}{\pgfqpoint{6.325396in}{0.906398in}}{\pgfqpoint{6.330439in}{0.911442in}}%
\pgfpathcurveto{\pgfqpoint{6.335483in}{0.916486in}}{\pgfqpoint{6.338317in}{0.923327in}}{\pgfqpoint{6.338317in}{0.930460in}}%
\pgfpathcurveto{\pgfqpoint{6.338317in}{0.937593in}}{\pgfqpoint{6.335483in}{0.944435in}}{\pgfqpoint{6.330439in}{0.949478in}}%
\pgfpathcurveto{\pgfqpoint{6.325396in}{0.954522in}}{\pgfqpoint{6.318554in}{0.957356in}}{\pgfqpoint{6.311421in}{0.957356in}}%
\pgfpathcurveto{\pgfqpoint{6.304288in}{0.957356in}}{\pgfqpoint{6.297447in}{0.954522in}}{\pgfqpoint{6.292403in}{0.949478in}}%
\pgfpathcurveto{\pgfqpoint{6.287359in}{0.944435in}}{\pgfqpoint{6.284525in}{0.937593in}}{\pgfqpoint{6.284525in}{0.930460in}}%
\pgfpathcurveto{\pgfqpoint{6.284525in}{0.923327in}}{\pgfqpoint{6.287359in}{0.916486in}}{\pgfqpoint{6.292403in}{0.911442in}}%
\pgfpathcurveto{\pgfqpoint{6.297447in}{0.906398in}}{\pgfqpoint{6.304288in}{0.903564in}}{\pgfqpoint{6.311421in}{0.903564in}}%
\pgfpathclose%
\pgfusepath{stroke,fill}%
\end{pgfscope}%
\begin{pgfscope}%
\pgfpathrectangle{\pgfqpoint{4.985294in}{0.500000in}}{\pgfqpoint{1.764706in}{1.700000in}}%
\pgfusepath{clip}%
\pgfsetbuttcap%
\pgfsetroundjoin%
\definecolor{currentfill}{rgb}{0.968931,0.798091,0.685123}%
\pgfsetfillcolor{currentfill}%
\pgfsetlinewidth{0.311001pt}%
\definecolor{currentstroke}{rgb}{1.000000,1.000000,1.000000}%
\pgfsetstrokecolor{currentstroke}%
\pgfsetdash{}{0pt}%
\pgfpathmoveto{\pgfqpoint{6.175786in}{0.993959in}}%
\pgfpathcurveto{\pgfqpoint{6.182919in}{0.993959in}}{\pgfqpoint{6.189761in}{0.996793in}}{\pgfqpoint{6.194804in}{1.001836in}}%
\pgfpathcurveto{\pgfqpoint{6.199848in}{1.006880in}}{\pgfqpoint{6.202682in}{1.013722in}}{\pgfqpoint{6.202682in}{1.020854in}}%
\pgfpathcurveto{\pgfqpoint{6.202682in}{1.027987in}}{\pgfqpoint{6.199848in}{1.034829in}}{\pgfqpoint{6.194804in}{1.039873in}}%
\pgfpathcurveto{\pgfqpoint{6.189761in}{1.044916in}}{\pgfqpoint{6.182919in}{1.047750in}}{\pgfqpoint{6.175786in}{1.047750in}}%
\pgfpathcurveto{\pgfqpoint{6.168653in}{1.047750in}}{\pgfqpoint{6.161812in}{1.044916in}}{\pgfqpoint{6.156768in}{1.039873in}}%
\pgfpathcurveto{\pgfqpoint{6.151724in}{1.034829in}}{\pgfqpoint{6.148891in}{1.027987in}}{\pgfqpoint{6.148891in}{1.020854in}}%
\pgfpathcurveto{\pgfqpoint{6.148891in}{1.013722in}}{\pgfqpoint{6.151724in}{1.006880in}}{\pgfqpoint{6.156768in}{1.001836in}}%
\pgfpathcurveto{\pgfqpoint{6.161812in}{0.996793in}}{\pgfqpoint{6.168653in}{0.993959in}}{\pgfqpoint{6.175786in}{0.993959in}}%
\pgfpathclose%
\pgfusepath{stroke,fill}%
\end{pgfscope}%
\begin{pgfscope}%
\pgfpathrectangle{\pgfqpoint{4.985294in}{0.500000in}}{\pgfqpoint{1.764706in}{1.700000in}}%
\pgfusepath{clip}%
\pgfsetbuttcap%
\pgfsetroundjoin%
\definecolor{currentfill}{rgb}{0.970718,0.821518,0.719872}%
\pgfsetfillcolor{currentfill}%
\pgfsetlinewidth{0.311001pt}%
\definecolor{currentstroke}{rgb}{1.000000,1.000000,1.000000}%
\pgfsetstrokecolor{currentstroke}%
\pgfsetdash{}{0pt}%
\pgfpathmoveto{\pgfqpoint{5.524044in}{1.509697in}}%
\pgfpathcurveto{\pgfqpoint{5.531176in}{1.509697in}}{\pgfqpoint{5.538018in}{1.512530in}}{\pgfqpoint{5.543062in}{1.517574in}}%
\pgfpathcurveto{\pgfqpoint{5.548105in}{1.522618in}}{\pgfqpoint{5.550939in}{1.529459in}}{\pgfqpoint{5.550939in}{1.536592in}}%
\pgfpathcurveto{\pgfqpoint{5.550939in}{1.543725in}}{\pgfqpoint{5.548105in}{1.550567in}}{\pgfqpoint{5.543062in}{1.555610in}}%
\pgfpathcurveto{\pgfqpoint{5.538018in}{1.560654in}}{\pgfqpoint{5.531176in}{1.563488in}}{\pgfqpoint{5.524044in}{1.563488in}}%
\pgfpathcurveto{\pgfqpoint{5.516911in}{1.563488in}}{\pgfqpoint{5.510069in}{1.560654in}}{\pgfqpoint{5.505025in}{1.555610in}}%
\pgfpathcurveto{\pgfqpoint{5.499982in}{1.550567in}}{\pgfqpoint{5.497148in}{1.543725in}}{\pgfqpoint{5.497148in}{1.536592in}}%
\pgfpathcurveto{\pgfqpoint{5.497148in}{1.529459in}}{\pgfqpoint{5.499982in}{1.522618in}}{\pgfqpoint{5.505025in}{1.517574in}}%
\pgfpathcurveto{\pgfqpoint{5.510069in}{1.512530in}}{\pgfqpoint{5.516911in}{1.509697in}}{\pgfqpoint{5.524044in}{1.509697in}}%
\pgfpathclose%
\pgfusepath{stroke,fill}%
\end{pgfscope}%
\begin{pgfscope}%
\pgfpathrectangle{\pgfqpoint{4.985294in}{0.500000in}}{\pgfqpoint{1.764706in}{1.700000in}}%
\pgfusepath{clip}%
\pgfsetbuttcap%
\pgfsetroundjoin%
\definecolor{currentfill}{rgb}{0.968509,0.792226,0.676405}%
\pgfsetfillcolor{currentfill}%
\pgfsetlinewidth{0.311001pt}%
\definecolor{currentstroke}{rgb}{1.000000,1.000000,1.000000}%
\pgfsetstrokecolor{currentstroke}%
\pgfsetdash{}{0pt}%
\pgfpathmoveto{\pgfqpoint{5.521475in}{1.457314in}}%
\pgfpathcurveto{\pgfqpoint{5.528608in}{1.457314in}}{\pgfqpoint{5.535450in}{1.460148in}}{\pgfqpoint{5.540493in}{1.465191in}}%
\pgfpathcurveto{\pgfqpoint{5.545537in}{1.470235in}}{\pgfqpoint{5.548371in}{1.477077in}}{\pgfqpoint{5.548371in}{1.484210in}}%
\pgfpathcurveto{\pgfqpoint{5.548371in}{1.491342in}}{\pgfqpoint{5.545537in}{1.498184in}}{\pgfqpoint{5.540493in}{1.503228in}}%
\pgfpathcurveto{\pgfqpoint{5.535450in}{1.508271in}}{\pgfqpoint{5.528608in}{1.511105in}}{\pgfqpoint{5.521475in}{1.511105in}}%
\pgfpathcurveto{\pgfqpoint{5.514342in}{1.511105in}}{\pgfqpoint{5.507501in}{1.508271in}}{\pgfqpoint{5.502457in}{1.503228in}}%
\pgfpathcurveto{\pgfqpoint{5.497413in}{1.498184in}}{\pgfqpoint{5.494580in}{1.491342in}}{\pgfqpoint{5.494580in}{1.484210in}}%
\pgfpathcurveto{\pgfqpoint{5.494580in}{1.477077in}}{\pgfqpoint{5.497413in}{1.470235in}}{\pgfqpoint{5.502457in}{1.465191in}}%
\pgfpathcurveto{\pgfqpoint{5.507501in}{1.460148in}}{\pgfqpoint{5.514342in}{1.457314in}}{\pgfqpoint{5.521475in}{1.457314in}}%
\pgfpathclose%
\pgfusepath{stroke,fill}%
\end{pgfscope}%
\begin{pgfscope}%
\pgfpathrectangle{\pgfqpoint{4.985294in}{0.500000in}}{\pgfqpoint{1.764706in}{1.700000in}}%
\pgfusepath{clip}%
\pgfsetbuttcap%
\pgfsetroundjoin%
\definecolor{currentfill}{rgb}{0.965302,0.713942,0.568499}%
\pgfsetfillcolor{currentfill}%
\pgfsetlinewidth{0.311001pt}%
\definecolor{currentstroke}{rgb}{1.000000,1.000000,1.000000}%
\pgfsetstrokecolor{currentstroke}%
\pgfsetdash{}{0pt}%
\pgfpathmoveto{\pgfqpoint{5.404578in}{1.006875in}}%
\pgfpathcurveto{\pgfqpoint{5.411711in}{1.006875in}}{\pgfqpoint{5.418553in}{1.009709in}}{\pgfqpoint{5.423596in}{1.014753in}}%
\pgfpathcurveto{\pgfqpoint{5.428640in}{1.019796in}}{\pgfqpoint{5.431474in}{1.026638in}}{\pgfqpoint{5.431474in}{1.033771in}}%
\pgfpathcurveto{\pgfqpoint{5.431474in}{1.040904in}}{\pgfqpoint{5.428640in}{1.047745in}}{\pgfqpoint{5.423596in}{1.052789in}}%
\pgfpathcurveto{\pgfqpoint{5.418553in}{1.057833in}}{\pgfqpoint{5.411711in}{1.060667in}}{\pgfqpoint{5.404578in}{1.060667in}}%
\pgfpathcurveto{\pgfqpoint{5.397445in}{1.060667in}}{\pgfqpoint{5.390604in}{1.057833in}}{\pgfqpoint{5.385560in}{1.052789in}}%
\pgfpathcurveto{\pgfqpoint{5.380516in}{1.047745in}}{\pgfqpoint{5.377682in}{1.040904in}}{\pgfqpoint{5.377682in}{1.033771in}}%
\pgfpathcurveto{\pgfqpoint{5.377682in}{1.026638in}}{\pgfqpoint{5.380516in}{1.019796in}}{\pgfqpoint{5.385560in}{1.014753in}}%
\pgfpathcurveto{\pgfqpoint{5.390604in}{1.009709in}}{\pgfqpoint{5.397445in}{1.006875in}}{\pgfqpoint{5.404578in}{1.006875in}}%
\pgfpathclose%
\pgfusepath{stroke,fill}%
\end{pgfscope}%
\begin{pgfscope}%
\pgfpathrectangle{\pgfqpoint{4.985294in}{0.500000in}}{\pgfqpoint{1.764706in}{1.700000in}}%
\pgfusepath{clip}%
\pgfsetbuttcap%
\pgfsetroundjoin%
\definecolor{currentfill}{rgb}{0.970255,0.815666,0.711203}%
\pgfsetfillcolor{currentfill}%
\pgfsetlinewidth{0.311001pt}%
\definecolor{currentstroke}{rgb}{1.000000,1.000000,1.000000}%
\pgfsetstrokecolor{currentstroke}%
\pgfsetdash{}{0pt}%
\pgfpathmoveto{\pgfqpoint{5.533709in}{1.601978in}}%
\pgfpathcurveto{\pgfqpoint{5.540841in}{1.601978in}}{\pgfqpoint{5.547683in}{1.604812in}}{\pgfqpoint{5.552727in}{1.609856in}}%
\pgfpathcurveto{\pgfqpoint{5.557770in}{1.614900in}}{\pgfqpoint{5.560604in}{1.621741in}}{\pgfqpoint{5.560604in}{1.628874in}}%
\pgfpathcurveto{\pgfqpoint{5.560604in}{1.636007in}}{\pgfqpoint{5.557770in}{1.642849in}}{\pgfqpoint{5.552727in}{1.647892in}}%
\pgfpathcurveto{\pgfqpoint{5.547683in}{1.652936in}}{\pgfqpoint{5.540841in}{1.655770in}}{\pgfqpoint{5.533709in}{1.655770in}}%
\pgfpathcurveto{\pgfqpoint{5.526576in}{1.655770in}}{\pgfqpoint{5.519734in}{1.652936in}}{\pgfqpoint{5.514690in}{1.647892in}}%
\pgfpathcurveto{\pgfqpoint{5.509647in}{1.642849in}}{\pgfqpoint{5.506813in}{1.636007in}}{\pgfqpoint{5.506813in}{1.628874in}}%
\pgfpathcurveto{\pgfqpoint{5.506813in}{1.621741in}}{\pgfqpoint{5.509647in}{1.614900in}}{\pgfqpoint{5.514690in}{1.609856in}}%
\pgfpathcurveto{\pgfqpoint{5.519734in}{1.604812in}}{\pgfqpoint{5.526576in}{1.601978in}}{\pgfqpoint{5.533709in}{1.601978in}}%
\pgfpathclose%
\pgfusepath{stroke,fill}%
\end{pgfscope}%
\begin{pgfscope}%
\pgfpathrectangle{\pgfqpoint{4.985294in}{0.500000in}}{\pgfqpoint{1.764706in}{1.700000in}}%
\pgfusepath{clip}%
\pgfsetbuttcap%
\pgfsetroundjoin%
\definecolor{currentfill}{rgb}{0.965302,0.713942,0.568499}%
\pgfsetfillcolor{currentfill}%
\pgfsetlinewidth{0.311001pt}%
\definecolor{currentstroke}{rgb}{1.000000,1.000000,1.000000}%
\pgfsetstrokecolor{currentstroke}%
\pgfsetdash{}{0pt}%
\pgfpathmoveto{\pgfqpoint{5.446576in}{1.675804in}}%
\pgfpathcurveto{\pgfqpoint{5.453709in}{1.675804in}}{\pgfqpoint{5.460550in}{1.678638in}}{\pgfqpoint{5.465594in}{1.683682in}}%
\pgfpathcurveto{\pgfqpoint{5.470638in}{1.688726in}}{\pgfqpoint{5.473472in}{1.695567in}}{\pgfqpoint{5.473472in}{1.702700in}}%
\pgfpathcurveto{\pgfqpoint{5.473472in}{1.709833in}}{\pgfqpoint{5.470638in}{1.716674in}}{\pgfqpoint{5.465594in}{1.721718in}}%
\pgfpathcurveto{\pgfqpoint{5.460550in}{1.726762in}}{\pgfqpoint{5.453709in}{1.729596in}}{\pgfqpoint{5.446576in}{1.729596in}}%
\pgfpathcurveto{\pgfqpoint{5.439443in}{1.729596in}}{\pgfqpoint{5.432601in}{1.726762in}}{\pgfqpoint{5.427558in}{1.721718in}}%
\pgfpathcurveto{\pgfqpoint{5.422514in}{1.716674in}}{\pgfqpoint{5.419680in}{1.709833in}}{\pgfqpoint{5.419680in}{1.702700in}}%
\pgfpathcurveto{\pgfqpoint{5.419680in}{1.695567in}}{\pgfqpoint{5.422514in}{1.688726in}}{\pgfqpoint{5.427558in}{1.683682in}}%
\pgfpathcurveto{\pgfqpoint{5.432601in}{1.678638in}}{\pgfqpoint{5.439443in}{1.675804in}}{\pgfqpoint{5.446576in}{1.675804in}}%
\pgfpathclose%
\pgfusepath{stroke,fill}%
\end{pgfscope}%
\begin{pgfscope}%
\pgfpathrectangle{\pgfqpoint{4.985294in}{0.500000in}}{\pgfqpoint{1.764706in}{1.700000in}}%
\pgfusepath{clip}%
\pgfsetbuttcap%
\pgfsetroundjoin%
\definecolor{currentfill}{rgb}{0.977657,0.891500,0.822809}%
\pgfsetfillcolor{currentfill}%
\pgfsetlinewidth{0.311001pt}%
\definecolor{currentstroke}{rgb}{1.000000,1.000000,1.000000}%
\pgfsetstrokecolor{currentstroke}%
\pgfsetdash{}{0pt}%
\pgfpathmoveto{\pgfqpoint{6.267005in}{1.126254in}}%
\pgfpathcurveto{\pgfqpoint{6.274138in}{1.126254in}}{\pgfqpoint{6.280980in}{1.129088in}}{\pgfqpoint{6.286023in}{1.134131in}}%
\pgfpathcurveto{\pgfqpoint{6.291067in}{1.139175in}}{\pgfqpoint{6.293901in}{1.146017in}}{\pgfqpoint{6.293901in}{1.153150in}}%
\pgfpathcurveto{\pgfqpoint{6.293901in}{1.160282in}}{\pgfqpoint{6.291067in}{1.167124in}}{\pgfqpoint{6.286023in}{1.172168in}}%
\pgfpathcurveto{\pgfqpoint{6.280980in}{1.177211in}}{\pgfqpoint{6.274138in}{1.180045in}}{\pgfqpoint{6.267005in}{1.180045in}}%
\pgfpathcurveto{\pgfqpoint{6.259873in}{1.180045in}}{\pgfqpoint{6.253031in}{1.177211in}}{\pgfqpoint{6.247987in}{1.172168in}}%
\pgfpathcurveto{\pgfqpoint{6.242944in}{1.167124in}}{\pgfqpoint{6.240110in}{1.160282in}}{\pgfqpoint{6.240110in}{1.153150in}}%
\pgfpathcurveto{\pgfqpoint{6.240110in}{1.146017in}}{\pgfqpoint{6.242944in}{1.139175in}}{\pgfqpoint{6.247987in}{1.134131in}}%
\pgfpathcurveto{\pgfqpoint{6.253031in}{1.129088in}}{\pgfqpoint{6.259873in}{1.126254in}}{\pgfqpoint{6.267005in}{1.126254in}}%
\pgfpathclose%
\pgfusepath{stroke,fill}%
\end{pgfscope}%
\begin{pgfscope}%
\pgfpathrectangle{\pgfqpoint{4.985294in}{0.500000in}}{\pgfqpoint{1.764706in}{1.700000in}}%
\pgfusepath{clip}%
\pgfsetbuttcap%
\pgfsetroundjoin%
\definecolor{currentfill}{rgb}{0.957344,0.505732,0.351309}%
\pgfsetfillcolor{currentfill}%
\pgfsetlinewidth{0.311001pt}%
\definecolor{currentstroke}{rgb}{1.000000,1.000000,1.000000}%
\pgfsetstrokecolor{currentstroke}%
\pgfsetdash{}{0pt}%
\pgfpathmoveto{\pgfqpoint{6.075140in}{0.889444in}}%
\pgfpathcurveto{\pgfqpoint{6.082273in}{0.889444in}}{\pgfqpoint{6.089115in}{0.892278in}}{\pgfqpoint{6.094158in}{0.897321in}}%
\pgfpathcurveto{\pgfqpoint{6.099202in}{0.902365in}}{\pgfqpoint{6.102036in}{0.909207in}}{\pgfqpoint{6.102036in}{0.916340in}}%
\pgfpathcurveto{\pgfqpoint{6.102036in}{0.923472in}}{\pgfqpoint{6.099202in}{0.930314in}}{\pgfqpoint{6.094158in}{0.935358in}}%
\pgfpathcurveto{\pgfqpoint{6.089115in}{0.940401in}}{\pgfqpoint{6.082273in}{0.943235in}}{\pgfqpoint{6.075140in}{0.943235in}}%
\pgfpathcurveto{\pgfqpoint{6.068007in}{0.943235in}}{\pgfqpoint{6.061166in}{0.940401in}}{\pgfqpoint{6.056122in}{0.935358in}}%
\pgfpathcurveto{\pgfqpoint{6.051078in}{0.930314in}}{\pgfqpoint{6.048244in}{0.923472in}}{\pgfqpoint{6.048244in}{0.916340in}}%
\pgfpathcurveto{\pgfqpoint{6.048244in}{0.909207in}}{\pgfqpoint{6.051078in}{0.902365in}}{\pgfqpoint{6.056122in}{0.897321in}}%
\pgfpathcurveto{\pgfqpoint{6.061166in}{0.892278in}}{\pgfqpoint{6.068007in}{0.889444in}}{\pgfqpoint{6.075140in}{0.889444in}}%
\pgfpathclose%
\pgfusepath{stroke,fill}%
\end{pgfscope}%
\begin{pgfscope}%
\pgfpathrectangle{\pgfqpoint{4.985294in}{0.500000in}}{\pgfqpoint{1.764706in}{1.700000in}}%
\pgfusepath{clip}%
\pgfsetbuttcap%
\pgfsetroundjoin%
\definecolor{currentfill}{rgb}{0.976287,0.879862,0.805788}%
\pgfsetfillcolor{currentfill}%
\pgfsetlinewidth{0.311001pt}%
\definecolor{currentstroke}{rgb}{1.000000,1.000000,1.000000}%
\pgfsetstrokecolor{currentstroke}%
\pgfsetdash{}{0pt}%
\pgfpathmoveto{\pgfqpoint{5.461421in}{1.374557in}}%
\pgfpathcurveto{\pgfqpoint{5.468554in}{1.374557in}}{\pgfqpoint{5.475396in}{1.377391in}}{\pgfqpoint{5.480439in}{1.382434in}}%
\pgfpathcurveto{\pgfqpoint{5.485483in}{1.387478in}}{\pgfqpoint{5.488317in}{1.394320in}}{\pgfqpoint{5.488317in}{1.401453in}}%
\pgfpathcurveto{\pgfqpoint{5.488317in}{1.408585in}}{\pgfqpoint{5.485483in}{1.415427in}}{\pgfqpoint{5.480439in}{1.420471in}}%
\pgfpathcurveto{\pgfqpoint{5.475396in}{1.425514in}}{\pgfqpoint{5.468554in}{1.428348in}}{\pgfqpoint{5.461421in}{1.428348in}}%
\pgfpathcurveto{\pgfqpoint{5.454288in}{1.428348in}}{\pgfqpoint{5.447447in}{1.425514in}}{\pgfqpoint{5.442403in}{1.420471in}}%
\pgfpathcurveto{\pgfqpoint{5.437359in}{1.415427in}}{\pgfqpoint{5.434525in}{1.408585in}}{\pgfqpoint{5.434525in}{1.401453in}}%
\pgfpathcurveto{\pgfqpoint{5.434525in}{1.394320in}}{\pgfqpoint{5.437359in}{1.387478in}}{\pgfqpoint{5.442403in}{1.382434in}}%
\pgfpathcurveto{\pgfqpoint{5.447447in}{1.377391in}}{\pgfqpoint{5.454288in}{1.374557in}}{\pgfqpoint{5.461421in}{1.374557in}}%
\pgfpathclose%
\pgfusepath{stroke,fill}%
\end{pgfscope}%
\begin{pgfscope}%
\pgfpathrectangle{\pgfqpoint{4.985294in}{0.500000in}}{\pgfqpoint{1.764706in}{1.700000in}}%
\pgfusepath{clip}%
\pgfsetbuttcap%
\pgfsetroundjoin%
\definecolor{currentfill}{rgb}{0.976961,0.885681,0.814303}%
\pgfsetfillcolor{currentfill}%
\pgfsetlinewidth{0.311001pt}%
\definecolor{currentstroke}{rgb}{1.000000,1.000000,1.000000}%
\pgfsetstrokecolor{currentstroke}%
\pgfsetdash{}{0pt}%
\pgfpathmoveto{\pgfqpoint{5.418177in}{1.479774in}}%
\pgfpathcurveto{\pgfqpoint{5.425310in}{1.479774in}}{\pgfqpoint{5.432151in}{1.482608in}}{\pgfqpoint{5.437195in}{1.487651in}}%
\pgfpathcurveto{\pgfqpoint{5.442239in}{1.492695in}}{\pgfqpoint{5.445072in}{1.499537in}}{\pgfqpoint{5.445072in}{1.506669in}}%
\pgfpathcurveto{\pgfqpoint{5.445072in}{1.513802in}}{\pgfqpoint{5.442239in}{1.520644in}}{\pgfqpoint{5.437195in}{1.525688in}}%
\pgfpathcurveto{\pgfqpoint{5.432151in}{1.530731in}}{\pgfqpoint{5.425310in}{1.533565in}}{\pgfqpoint{5.418177in}{1.533565in}}%
\pgfpathcurveto{\pgfqpoint{5.411044in}{1.533565in}}{\pgfqpoint{5.404202in}{1.530731in}}{\pgfqpoint{5.399159in}{1.525688in}}%
\pgfpathcurveto{\pgfqpoint{5.394115in}{1.520644in}}{\pgfqpoint{5.391281in}{1.513802in}}{\pgfqpoint{5.391281in}{1.506669in}}%
\pgfpathcurveto{\pgfqpoint{5.391281in}{1.499537in}}{\pgfqpoint{5.394115in}{1.492695in}}{\pgfqpoint{5.399159in}{1.487651in}}%
\pgfpathcurveto{\pgfqpoint{5.404202in}{1.482608in}}{\pgfqpoint{5.411044in}{1.479774in}}{\pgfqpoint{5.418177in}{1.479774in}}%
\pgfpathclose%
\pgfusepath{stroke,fill}%
\end{pgfscope}%
\begin{pgfscope}%
\pgfpathrectangle{\pgfqpoint{4.985294in}{0.500000in}}{\pgfqpoint{1.764706in}{1.700000in}}%
\pgfusepath{clip}%
\pgfsetbuttcap%
\pgfsetroundjoin%
\definecolor{currentfill}{rgb}{0.978376,0.897317,0.831308}%
\pgfsetfillcolor{currentfill}%
\pgfsetlinewidth{0.311001pt}%
\definecolor{currentstroke}{rgb}{1.000000,1.000000,1.000000}%
\pgfsetstrokecolor{currentstroke}%
\pgfsetdash{}{0pt}%
\pgfpathmoveto{\pgfqpoint{6.332915in}{1.216967in}}%
\pgfpathcurveto{\pgfqpoint{6.340048in}{1.216967in}}{\pgfqpoint{6.346889in}{1.219801in}}{\pgfqpoint{6.351933in}{1.224845in}}%
\pgfpathcurveto{\pgfqpoint{6.356977in}{1.229889in}}{\pgfqpoint{6.359811in}{1.236730in}}{\pgfqpoint{6.359811in}{1.243863in}}%
\pgfpathcurveto{\pgfqpoint{6.359811in}{1.250996in}}{\pgfqpoint{6.356977in}{1.257838in}}{\pgfqpoint{6.351933in}{1.262881in}}%
\pgfpathcurveto{\pgfqpoint{6.346889in}{1.267925in}}{\pgfqpoint{6.340048in}{1.270759in}}{\pgfqpoint{6.332915in}{1.270759in}}%
\pgfpathcurveto{\pgfqpoint{6.325782in}{1.270759in}}{\pgfqpoint{6.318940in}{1.267925in}}{\pgfqpoint{6.313897in}{1.262881in}}%
\pgfpathcurveto{\pgfqpoint{6.308853in}{1.257838in}}{\pgfqpoint{6.306019in}{1.250996in}}{\pgfqpoint{6.306019in}{1.243863in}}%
\pgfpathcurveto{\pgfqpoint{6.306019in}{1.236730in}}{\pgfqpoint{6.308853in}{1.229889in}}{\pgfqpoint{6.313897in}{1.224845in}}%
\pgfpathcurveto{\pgfqpoint{6.318940in}{1.219801in}}{\pgfqpoint{6.325782in}{1.216967in}}{\pgfqpoint{6.332915in}{1.216967in}}%
\pgfpathclose%
\pgfusepath{stroke,fill}%
\end{pgfscope}%
\begin{pgfscope}%
\pgfpathrectangle{\pgfqpoint{4.985294in}{0.500000in}}{\pgfqpoint{1.764706in}{1.700000in}}%
\pgfusepath{clip}%
\pgfsetbuttcap%
\pgfsetroundjoin%
\definecolor{currentfill}{rgb}{0.965753,0.732351,0.592427}%
\pgfsetfillcolor{currentfill}%
\pgfsetlinewidth{0.311001pt}%
\definecolor{currentstroke}{rgb}{1.000000,1.000000,1.000000}%
\pgfsetstrokecolor{currentstroke}%
\pgfsetdash{}{0pt}%
\pgfpathmoveto{\pgfqpoint{6.287577in}{0.989216in}}%
\pgfpathcurveto{\pgfqpoint{6.294710in}{0.989216in}}{\pgfqpoint{6.301552in}{0.992050in}}{\pgfqpoint{6.306595in}{0.997093in}}%
\pgfpathcurveto{\pgfqpoint{6.311639in}{1.002137in}}{\pgfqpoint{6.314473in}{1.008979in}}{\pgfqpoint{6.314473in}{1.016111in}}%
\pgfpathcurveto{\pgfqpoint{6.314473in}{1.023244in}}{\pgfqpoint{6.311639in}{1.030086in}}{\pgfqpoint{6.306595in}{1.035129in}}%
\pgfpathcurveto{\pgfqpoint{6.301552in}{1.040173in}}{\pgfqpoint{6.294710in}{1.043007in}}{\pgfqpoint{6.287577in}{1.043007in}}%
\pgfpathcurveto{\pgfqpoint{6.280444in}{1.043007in}}{\pgfqpoint{6.273603in}{1.040173in}}{\pgfqpoint{6.268559in}{1.035129in}}%
\pgfpathcurveto{\pgfqpoint{6.263515in}{1.030086in}}{\pgfqpoint{6.260682in}{1.023244in}}{\pgfqpoint{6.260682in}{1.016111in}}%
\pgfpathcurveto{\pgfqpoint{6.260682in}{1.008979in}}{\pgfqpoint{6.263515in}{1.002137in}}{\pgfqpoint{6.268559in}{0.997093in}}%
\pgfpathcurveto{\pgfqpoint{6.273603in}{0.992050in}}{\pgfqpoint{6.280444in}{0.989216in}}{\pgfqpoint{6.287577in}{0.989216in}}%
\pgfpathclose%
\pgfusepath{stroke,fill}%
\end{pgfscope}%
\begin{pgfscope}%
\pgfpathrectangle{\pgfqpoint{4.985294in}{0.500000in}}{\pgfqpoint{1.764706in}{1.700000in}}%
\pgfusepath{clip}%
\pgfsetbuttcap%
\pgfsetroundjoin%
\definecolor{currentfill}{rgb}{0.979124,0.903132,0.839793}%
\pgfsetfillcolor{currentfill}%
\pgfsetlinewidth{0.311001pt}%
\definecolor{currentstroke}{rgb}{1.000000,1.000000,1.000000}%
\pgfsetstrokecolor{currentstroke}%
\pgfsetdash{}{0pt}%
\pgfpathmoveto{\pgfqpoint{5.440985in}{1.194608in}}%
\pgfpathcurveto{\pgfqpoint{5.448118in}{1.194608in}}{\pgfqpoint{5.454960in}{1.197442in}}{\pgfqpoint{5.460003in}{1.202485in}}%
\pgfpathcurveto{\pgfqpoint{5.465047in}{1.207529in}}{\pgfqpoint{5.467881in}{1.214371in}}{\pgfqpoint{5.467881in}{1.221504in}}%
\pgfpathcurveto{\pgfqpoint{5.467881in}{1.228636in}}{\pgfqpoint{5.465047in}{1.235478in}}{\pgfqpoint{5.460003in}{1.240522in}}%
\pgfpathcurveto{\pgfqpoint{5.454960in}{1.245565in}}{\pgfqpoint{5.448118in}{1.248399in}}{\pgfqpoint{5.440985in}{1.248399in}}%
\pgfpathcurveto{\pgfqpoint{5.433852in}{1.248399in}}{\pgfqpoint{5.427011in}{1.245565in}}{\pgfqpoint{5.421967in}{1.240522in}}%
\pgfpathcurveto{\pgfqpoint{5.416923in}{1.235478in}}{\pgfqpoint{5.414089in}{1.228636in}}{\pgfqpoint{5.414089in}{1.221504in}}%
\pgfpathcurveto{\pgfqpoint{5.414089in}{1.214371in}}{\pgfqpoint{5.416923in}{1.207529in}}{\pgfqpoint{5.421967in}{1.202485in}}%
\pgfpathcurveto{\pgfqpoint{5.427011in}{1.197442in}}{\pgfqpoint{5.433852in}{1.194608in}}{\pgfqpoint{5.440985in}{1.194608in}}%
\pgfpathclose%
\pgfusepath{stroke,fill}%
\end{pgfscope}%
\begin{pgfscope}%
\pgfpathrectangle{\pgfqpoint{4.985294in}{0.500000in}}{\pgfqpoint{1.764706in}{1.700000in}}%
\pgfusepath{clip}%
\pgfsetbuttcap%
\pgfsetroundjoin%
\definecolor{currentfill}{rgb}{0.965928,0.738443,0.600540}%
\pgfsetfillcolor{currentfill}%
\pgfsetlinewidth{0.311001pt}%
\definecolor{currentstroke}{rgb}{1.000000,1.000000,1.000000}%
\pgfsetstrokecolor{currentstroke}%
\pgfsetdash{}{0pt}%
\pgfpathmoveto{\pgfqpoint{5.507875in}{1.707853in}}%
\pgfpathcurveto{\pgfqpoint{5.515008in}{1.707853in}}{\pgfqpoint{5.521850in}{1.710686in}}{\pgfqpoint{5.526893in}{1.715730in}}%
\pgfpathcurveto{\pgfqpoint{5.531937in}{1.720774in}}{\pgfqpoint{5.534771in}{1.727615in}}{\pgfqpoint{5.534771in}{1.734748in}}%
\pgfpathcurveto{\pgfqpoint{5.534771in}{1.741881in}}{\pgfqpoint{5.531937in}{1.748723in}}{\pgfqpoint{5.526893in}{1.753766in}}%
\pgfpathcurveto{\pgfqpoint{5.521850in}{1.758810in}}{\pgfqpoint{5.515008in}{1.761644in}}{\pgfqpoint{5.507875in}{1.761644in}}%
\pgfpathcurveto{\pgfqpoint{5.500743in}{1.761644in}}{\pgfqpoint{5.493901in}{1.758810in}}{\pgfqpoint{5.488857in}{1.753766in}}%
\pgfpathcurveto{\pgfqpoint{5.483814in}{1.748723in}}{\pgfqpoint{5.480980in}{1.741881in}}{\pgfqpoint{5.480980in}{1.734748in}}%
\pgfpathcurveto{\pgfqpoint{5.480980in}{1.727615in}}{\pgfqpoint{5.483814in}{1.720774in}}{\pgfqpoint{5.488857in}{1.715730in}}%
\pgfpathcurveto{\pgfqpoint{5.493901in}{1.710686in}}{\pgfqpoint{5.500743in}{1.707853in}}{\pgfqpoint{5.507875in}{1.707853in}}%
\pgfpathclose%
\pgfusepath{stroke,fill}%
\end{pgfscope}%
\begin{pgfscope}%
\pgfpathrectangle{\pgfqpoint{4.985294in}{0.500000in}}{\pgfqpoint{1.764706in}{1.700000in}}%
\pgfusepath{clip}%
\pgfsetbuttcap%
\pgfsetroundjoin%
\definecolor{currentfill}{rgb}{0.883342,0.198306,0.260142}%
\pgfsetfillcolor{currentfill}%
\pgfsetlinewidth{0.311001pt}%
\definecolor{currentstroke}{rgb}{1.000000,1.000000,1.000000}%
\pgfsetstrokecolor{currentstroke}%
\pgfsetdash{}{0pt}%
\pgfpathmoveto{\pgfqpoint{5.704487in}{1.021549in}}%
\pgfpathcurveto{\pgfqpoint{5.711620in}{1.021549in}}{\pgfqpoint{5.718462in}{1.024382in}}{\pgfqpoint{5.723505in}{1.029426in}}%
\pgfpathcurveto{\pgfqpoint{5.728549in}{1.034470in}}{\pgfqpoint{5.731383in}{1.041311in}}{\pgfqpoint{5.731383in}{1.048444in}}%
\pgfpathcurveto{\pgfqpoint{5.731383in}{1.055577in}}{\pgfqpoint{5.728549in}{1.062419in}}{\pgfqpoint{5.723505in}{1.067462in}}%
\pgfpathcurveto{\pgfqpoint{5.718462in}{1.072506in}}{\pgfqpoint{5.711620in}{1.075340in}}{\pgfqpoint{5.704487in}{1.075340in}}%
\pgfpathcurveto{\pgfqpoint{5.697354in}{1.075340in}}{\pgfqpoint{5.690513in}{1.072506in}}{\pgfqpoint{5.685469in}{1.067462in}}%
\pgfpathcurveto{\pgfqpoint{5.680425in}{1.062419in}}{\pgfqpoint{5.677591in}{1.055577in}}{\pgfqpoint{5.677591in}{1.048444in}}%
\pgfpathcurveto{\pgfqpoint{5.677591in}{1.041311in}}{\pgfqpoint{5.680425in}{1.034470in}}{\pgfqpoint{5.685469in}{1.029426in}}%
\pgfpathcurveto{\pgfqpoint{5.690513in}{1.024382in}}{\pgfqpoint{5.697354in}{1.021549in}}{\pgfqpoint{5.704487in}{1.021549in}}%
\pgfpathclose%
\pgfusepath{stroke,fill}%
\end{pgfscope}%
\begin{pgfscope}%
\pgfpathrectangle{\pgfqpoint{4.985294in}{0.500000in}}{\pgfqpoint{1.764706in}{1.700000in}}%
\pgfusepath{clip}%
\pgfsetbuttcap%
\pgfsetroundjoin%
\definecolor{currentfill}{rgb}{0.975644,0.874038,0.797253}%
\pgfsetfillcolor{currentfill}%
\pgfsetlinewidth{0.311001pt}%
\definecolor{currentstroke}{rgb}{1.000000,1.000000,1.000000}%
\pgfsetstrokecolor{currentstroke}%
\pgfsetdash{}{0pt}%
\pgfpathmoveto{\pgfqpoint{6.362449in}{1.324895in}}%
\pgfpathcurveto{\pgfqpoint{6.369581in}{1.324895in}}{\pgfqpoint{6.376423in}{1.327728in}}{\pgfqpoint{6.381467in}{1.332772in}}%
\pgfpathcurveto{\pgfqpoint{6.386510in}{1.337816in}}{\pgfqpoint{6.389344in}{1.344657in}}{\pgfqpoint{6.389344in}{1.351790in}}%
\pgfpathcurveto{\pgfqpoint{6.389344in}{1.358923in}}{\pgfqpoint{6.386510in}{1.365765in}}{\pgfqpoint{6.381467in}{1.370808in}}%
\pgfpathcurveto{\pgfqpoint{6.376423in}{1.375852in}}{\pgfqpoint{6.369581in}{1.378686in}}{\pgfqpoint{6.362449in}{1.378686in}}%
\pgfpathcurveto{\pgfqpoint{6.355316in}{1.378686in}}{\pgfqpoint{6.348474in}{1.375852in}}{\pgfqpoint{6.343430in}{1.370808in}}%
\pgfpathcurveto{\pgfqpoint{6.338387in}{1.365765in}}{\pgfqpoint{6.335553in}{1.358923in}}{\pgfqpoint{6.335553in}{1.351790in}}%
\pgfpathcurveto{\pgfqpoint{6.335553in}{1.344657in}}{\pgfqpoint{6.338387in}{1.337816in}}{\pgfqpoint{6.343430in}{1.332772in}}%
\pgfpathcurveto{\pgfqpoint{6.348474in}{1.327728in}}{\pgfqpoint{6.355316in}{1.324895in}}{\pgfqpoint{6.362449in}{1.324895in}}%
\pgfpathclose%
\pgfusepath{stroke,fill}%
\end{pgfscope}%
\begin{pgfscope}%
\pgfpathrectangle{\pgfqpoint{4.985294in}{0.500000in}}{\pgfqpoint{1.764706in}{1.700000in}}%
\pgfusepath{clip}%
\pgfsetbuttcap%
\pgfsetroundjoin%
\definecolor{currentfill}{rgb}{0.953126,0.456614,0.312398}%
\pgfsetfillcolor{currentfill}%
\pgfsetlinewidth{0.311001pt}%
\definecolor{currentstroke}{rgb}{1.000000,1.000000,1.000000}%
\pgfsetstrokecolor{currentstroke}%
\pgfsetdash{}{0pt}%
\pgfpathmoveto{\pgfqpoint{5.567241in}{1.331936in}}%
\pgfpathcurveto{\pgfqpoint{5.574373in}{1.331936in}}{\pgfqpoint{5.581215in}{1.334770in}}{\pgfqpoint{5.586259in}{1.339813in}}%
\pgfpathcurveto{\pgfqpoint{5.591302in}{1.344857in}}{\pgfqpoint{5.594136in}{1.351698in}}{\pgfqpoint{5.594136in}{1.358831in}}%
\pgfpathcurveto{\pgfqpoint{5.594136in}{1.365964in}}{\pgfqpoint{5.591302in}{1.372806in}}{\pgfqpoint{5.586259in}{1.377849in}}%
\pgfpathcurveto{\pgfqpoint{5.581215in}{1.382893in}}{\pgfqpoint{5.574373in}{1.385727in}}{\pgfqpoint{5.567241in}{1.385727in}}%
\pgfpathcurveto{\pgfqpoint{5.560108in}{1.385727in}}{\pgfqpoint{5.553266in}{1.382893in}}{\pgfqpoint{5.548222in}{1.377849in}}%
\pgfpathcurveto{\pgfqpoint{5.543179in}{1.372806in}}{\pgfqpoint{5.540345in}{1.365964in}}{\pgfqpoint{5.540345in}{1.358831in}}%
\pgfpathcurveto{\pgfqpoint{5.540345in}{1.351698in}}{\pgfqpoint{5.543179in}{1.344857in}}{\pgfqpoint{5.548222in}{1.339813in}}%
\pgfpathcurveto{\pgfqpoint{5.553266in}{1.334770in}}{\pgfqpoint{5.560108in}{1.331936in}}{\pgfqpoint{5.567241in}{1.331936in}}%
\pgfpathclose%
\pgfusepath{stroke,fill}%
\end{pgfscope}%
\begin{pgfscope}%
\pgfpathrectangle{\pgfqpoint{4.985294in}{0.500000in}}{\pgfqpoint{1.764706in}{1.700000in}}%
\pgfusepath{clip}%
\pgfsetbuttcap%
\pgfsetroundjoin%
\definecolor{currentfill}{rgb}{0.966560,0.756582,0.625273}%
\pgfsetfillcolor{currentfill}%
\pgfsetlinewidth{0.311001pt}%
\definecolor{currentstroke}{rgb}{1.000000,1.000000,1.000000}%
\pgfsetstrokecolor{currentstroke}%
\pgfsetdash{}{0pt}%
\pgfpathmoveto{\pgfqpoint{5.354021in}{1.169040in}}%
\pgfpathcurveto{\pgfqpoint{5.361154in}{1.169040in}}{\pgfqpoint{5.367996in}{1.171874in}}{\pgfqpoint{5.373039in}{1.176918in}}%
\pgfpathcurveto{\pgfqpoint{5.378083in}{1.181962in}}{\pgfqpoint{5.380917in}{1.188803in}}{\pgfqpoint{5.380917in}{1.195936in}}%
\pgfpathcurveto{\pgfqpoint{5.380917in}{1.203069in}}{\pgfqpoint{5.378083in}{1.209911in}}{\pgfqpoint{5.373039in}{1.214954in}}%
\pgfpathcurveto{\pgfqpoint{5.367996in}{1.219998in}}{\pgfqpoint{5.361154in}{1.222832in}}{\pgfqpoint{5.354021in}{1.222832in}}%
\pgfpathcurveto{\pgfqpoint{5.346888in}{1.222832in}}{\pgfqpoint{5.340047in}{1.219998in}}{\pgfqpoint{5.335003in}{1.214954in}}%
\pgfpathcurveto{\pgfqpoint{5.329959in}{1.209911in}}{\pgfqpoint{5.327125in}{1.203069in}}{\pgfqpoint{5.327125in}{1.195936in}}%
\pgfpathcurveto{\pgfqpoint{5.327125in}{1.188803in}}{\pgfqpoint{5.329959in}{1.181962in}}{\pgfqpoint{5.335003in}{1.176918in}}%
\pgfpathcurveto{\pgfqpoint{5.340047in}{1.171874in}}{\pgfqpoint{5.346888in}{1.169040in}}{\pgfqpoint{5.354021in}{1.169040in}}%
\pgfpathclose%
\pgfusepath{stroke,fill}%
\end{pgfscope}%
\begin{pgfscope}%
\pgfpathrectangle{\pgfqpoint{4.985294in}{0.500000in}}{\pgfqpoint{1.764706in}{1.700000in}}%
\pgfusepath{clip}%
\pgfsetbuttcap%
\pgfsetroundjoin%
\definecolor{currentfill}{rgb}{0.967092,0.768560,0.642079}%
\pgfsetfillcolor{currentfill}%
\pgfsetlinewidth{0.311001pt}%
\definecolor{currentstroke}{rgb}{1.000000,1.000000,1.000000}%
\pgfsetstrokecolor{currentstroke}%
\pgfsetdash{}{0pt}%
\pgfpathmoveto{\pgfqpoint{5.506975in}{1.339796in}}%
\pgfpathcurveto{\pgfqpoint{5.514108in}{1.339796in}}{\pgfqpoint{5.520949in}{1.342630in}}{\pgfqpoint{5.525993in}{1.347674in}}%
\pgfpathcurveto{\pgfqpoint{5.531037in}{1.352717in}}{\pgfqpoint{5.533871in}{1.359559in}}{\pgfqpoint{5.533871in}{1.366692in}}%
\pgfpathcurveto{\pgfqpoint{5.533871in}{1.373825in}}{\pgfqpoint{5.531037in}{1.380666in}}{\pgfqpoint{5.525993in}{1.385710in}}%
\pgfpathcurveto{\pgfqpoint{5.520949in}{1.390754in}}{\pgfqpoint{5.514108in}{1.393588in}}{\pgfqpoint{5.506975in}{1.393588in}}%
\pgfpathcurveto{\pgfqpoint{5.499842in}{1.393588in}}{\pgfqpoint{5.493000in}{1.390754in}}{\pgfqpoint{5.487957in}{1.385710in}}%
\pgfpathcurveto{\pgfqpoint{5.482913in}{1.380666in}}{\pgfqpoint{5.480079in}{1.373825in}}{\pgfqpoint{5.480079in}{1.366692in}}%
\pgfpathcurveto{\pgfqpoint{5.480079in}{1.359559in}}{\pgfqpoint{5.482913in}{1.352717in}}{\pgfqpoint{5.487957in}{1.347674in}}%
\pgfpathcurveto{\pgfqpoint{5.493000in}{1.342630in}}{\pgfqpoint{5.499842in}{1.339796in}}{\pgfqpoint{5.506975in}{1.339796in}}%
\pgfpathclose%
\pgfusepath{stroke,fill}%
\end{pgfscope}%
\begin{pgfscope}%
\pgfpathrectangle{\pgfqpoint{4.985294in}{0.500000in}}{\pgfqpoint{1.764706in}{1.700000in}}%
\pgfusepath{clip}%
\pgfsetbuttcap%
\pgfsetroundjoin%
\definecolor{currentfill}{rgb}{0.970718,0.821518,0.719872}%
\pgfsetfillcolor{currentfill}%
\pgfsetlinewidth{0.311001pt}%
\definecolor{currentstroke}{rgb}{1.000000,1.000000,1.000000}%
\pgfsetstrokecolor{currentstroke}%
\pgfsetdash{}{0pt}%
\pgfpathmoveto{\pgfqpoint{6.284664in}{1.040486in}}%
\pgfpathcurveto{\pgfqpoint{6.291797in}{1.040486in}}{\pgfqpoint{6.298638in}{1.043320in}}{\pgfqpoint{6.303682in}{1.048364in}}%
\pgfpathcurveto{\pgfqpoint{6.308726in}{1.053408in}}{\pgfqpoint{6.311560in}{1.060249in}}{\pgfqpoint{6.311560in}{1.067382in}}%
\pgfpathcurveto{\pgfqpoint{6.311560in}{1.074515in}}{\pgfqpoint{6.308726in}{1.081357in}}{\pgfqpoint{6.303682in}{1.086400in}}%
\pgfpathcurveto{\pgfqpoint{6.298638in}{1.091444in}}{\pgfqpoint{6.291797in}{1.094278in}}{\pgfqpoint{6.284664in}{1.094278in}}%
\pgfpathcurveto{\pgfqpoint{6.277531in}{1.094278in}}{\pgfqpoint{6.270690in}{1.091444in}}{\pgfqpoint{6.265646in}{1.086400in}}%
\pgfpathcurveto{\pgfqpoint{6.260602in}{1.081357in}}{\pgfqpoint{6.257768in}{1.074515in}}{\pgfqpoint{6.257768in}{1.067382in}}%
\pgfpathcurveto{\pgfqpoint{6.257768in}{1.060249in}}{\pgfqpoint{6.260602in}{1.053408in}}{\pgfqpoint{6.265646in}{1.048364in}}%
\pgfpathcurveto{\pgfqpoint{6.270690in}{1.043320in}}{\pgfqpoint{6.277531in}{1.040486in}}{\pgfqpoint{6.284664in}{1.040486in}}%
\pgfpathclose%
\pgfusepath{stroke,fill}%
\end{pgfscope}%
\begin{pgfscope}%
\pgfpathrectangle{\pgfqpoint{4.985294in}{0.500000in}}{\pgfqpoint{1.764706in}{1.700000in}}%
\pgfusepath{clip}%
\pgfsetbuttcap%
\pgfsetroundjoin%
\definecolor{currentfill}{rgb}{0.966812,0.762584,0.633643}%
\pgfsetfillcolor{currentfill}%
\pgfsetlinewidth{0.311001pt}%
\definecolor{currentstroke}{rgb}{1.000000,1.000000,1.000000}%
\pgfsetstrokecolor{currentstroke}%
\pgfsetdash{}{0pt}%
\pgfpathmoveto{\pgfqpoint{5.506855in}{1.261850in}}%
\pgfpathcurveto{\pgfqpoint{5.513988in}{1.261850in}}{\pgfqpoint{5.520829in}{1.264684in}}{\pgfqpoint{5.525873in}{1.269728in}}%
\pgfpathcurveto{\pgfqpoint{5.530917in}{1.274771in}}{\pgfqpoint{5.533750in}{1.281613in}}{\pgfqpoint{5.533750in}{1.288746in}}%
\pgfpathcurveto{\pgfqpoint{5.533750in}{1.295879in}}{\pgfqpoint{5.530917in}{1.302720in}}{\pgfqpoint{5.525873in}{1.307764in}}%
\pgfpathcurveto{\pgfqpoint{5.520829in}{1.312808in}}{\pgfqpoint{5.513988in}{1.315642in}}{\pgfqpoint{5.506855in}{1.315642in}}%
\pgfpathcurveto{\pgfqpoint{5.499722in}{1.315642in}}{\pgfqpoint{5.492880in}{1.312808in}}{\pgfqpoint{5.487837in}{1.307764in}}%
\pgfpathcurveto{\pgfqpoint{5.482793in}{1.302720in}}{\pgfqpoint{5.479959in}{1.295879in}}{\pgfqpoint{5.479959in}{1.288746in}}%
\pgfpathcurveto{\pgfqpoint{5.479959in}{1.281613in}}{\pgfqpoint{5.482793in}{1.274771in}}{\pgfqpoint{5.487837in}{1.269728in}}%
\pgfpathcurveto{\pgfqpoint{5.492880in}{1.264684in}}{\pgfqpoint{5.499722in}{1.261850in}}{\pgfqpoint{5.506855in}{1.261850in}}%
\pgfpathclose%
\pgfusepath{stroke,fill}%
\end{pgfscope}%
\begin{pgfscope}%
\pgfpathrectangle{\pgfqpoint{4.985294in}{0.500000in}}{\pgfqpoint{1.764706in}{1.700000in}}%
\pgfusepath{clip}%
\pgfsetbuttcap%
\pgfsetroundjoin%
\definecolor{currentfill}{rgb}{0.976961,0.885681,0.814303}%
\pgfsetfillcolor{currentfill}%
\pgfsetlinewidth{0.311001pt}%
\definecolor{currentstroke}{rgb}{1.000000,1.000000,1.000000}%
\pgfsetstrokecolor{currentstroke}%
\pgfsetdash{}{0pt}%
\pgfpathmoveto{\pgfqpoint{5.466032in}{1.122792in}}%
\pgfpathcurveto{\pgfqpoint{5.473165in}{1.122792in}}{\pgfqpoint{5.480007in}{1.125625in}}{\pgfqpoint{5.485050in}{1.130669in}}%
\pgfpathcurveto{\pgfqpoint{5.490094in}{1.135713in}}{\pgfqpoint{5.492928in}{1.142554in}}{\pgfqpoint{5.492928in}{1.149687in}}%
\pgfpathcurveto{\pgfqpoint{5.492928in}{1.156820in}}{\pgfqpoint{5.490094in}{1.163662in}}{\pgfqpoint{5.485050in}{1.168705in}}%
\pgfpathcurveto{\pgfqpoint{5.480007in}{1.173749in}}{\pgfqpoint{5.473165in}{1.176583in}}{\pgfqpoint{5.466032in}{1.176583in}}%
\pgfpathcurveto{\pgfqpoint{5.458899in}{1.176583in}}{\pgfqpoint{5.452058in}{1.173749in}}{\pgfqpoint{5.447014in}{1.168705in}}%
\pgfpathcurveto{\pgfqpoint{5.441970in}{1.163662in}}{\pgfqpoint{5.439136in}{1.156820in}}{\pgfqpoint{5.439136in}{1.149687in}}%
\pgfpathcurveto{\pgfqpoint{5.439136in}{1.142554in}}{\pgfqpoint{5.441970in}{1.135713in}}{\pgfqpoint{5.447014in}{1.130669in}}%
\pgfpathcurveto{\pgfqpoint{5.452058in}{1.125625in}}{\pgfqpoint{5.458899in}{1.122792in}}{\pgfqpoint{5.466032in}{1.122792in}}%
\pgfpathclose%
\pgfusepath{stroke,fill}%
\end{pgfscope}%
\begin{pgfscope}%
\pgfpathrectangle{\pgfqpoint{4.985294in}{0.500000in}}{\pgfqpoint{1.764706in}{1.700000in}}%
\pgfusepath{clip}%
\pgfsetbuttcap%
\pgfsetroundjoin%
\definecolor{currentfill}{rgb}{0.972726,0.844889,0.754401}%
\pgfsetfillcolor{currentfill}%
\pgfsetlinewidth{0.311001pt}%
\definecolor{currentstroke}{rgb}{1.000000,1.000000,1.000000}%
\pgfsetstrokecolor{currentstroke}%
\pgfsetdash{}{0pt}%
\pgfpathmoveto{\pgfqpoint{6.219376in}{1.140878in}}%
\pgfpathcurveto{\pgfqpoint{6.226509in}{1.140878in}}{\pgfqpoint{6.233351in}{1.143712in}}{\pgfqpoint{6.238395in}{1.148756in}}%
\pgfpathcurveto{\pgfqpoint{6.243438in}{1.153799in}}{\pgfqpoint{6.246272in}{1.160641in}}{\pgfqpoint{6.246272in}{1.167774in}}%
\pgfpathcurveto{\pgfqpoint{6.246272in}{1.174907in}}{\pgfqpoint{6.243438in}{1.181748in}}{\pgfqpoint{6.238395in}{1.186792in}}%
\pgfpathcurveto{\pgfqpoint{6.233351in}{1.191836in}}{\pgfqpoint{6.226509in}{1.194669in}}{\pgfqpoint{6.219376in}{1.194669in}}%
\pgfpathcurveto{\pgfqpoint{6.212244in}{1.194669in}}{\pgfqpoint{6.205402in}{1.191836in}}{\pgfqpoint{6.200358in}{1.186792in}}%
\pgfpathcurveto{\pgfqpoint{6.195315in}{1.181748in}}{\pgfqpoint{6.192481in}{1.174907in}}{\pgfqpoint{6.192481in}{1.167774in}}%
\pgfpathcurveto{\pgfqpoint{6.192481in}{1.160641in}}{\pgfqpoint{6.195315in}{1.153799in}}{\pgfqpoint{6.200358in}{1.148756in}}%
\pgfpathcurveto{\pgfqpoint{6.205402in}{1.143712in}}{\pgfqpoint{6.212244in}{1.140878in}}{\pgfqpoint{6.219376in}{1.140878in}}%
\pgfpathclose%
\pgfusepath{stroke,fill}%
\end{pgfscope}%
\begin{pgfscope}%
\pgfpathrectangle{\pgfqpoint{4.985294in}{0.500000in}}{\pgfqpoint{1.764706in}{1.700000in}}%
\pgfusepath{clip}%
\pgfsetbuttcap%
\pgfsetroundjoin%
\definecolor{currentfill}{rgb}{0.962985,0.612625,0.451451}%
\pgfsetfillcolor{currentfill}%
\pgfsetlinewidth{0.311001pt}%
\definecolor{currentstroke}{rgb}{1.000000,1.000000,1.000000}%
\pgfsetstrokecolor{currentstroke}%
\pgfsetdash{}{0pt}%
\pgfpathmoveto{\pgfqpoint{5.610750in}{1.602596in}}%
\pgfpathcurveto{\pgfqpoint{5.617883in}{1.602596in}}{\pgfqpoint{5.624724in}{1.605430in}}{\pgfqpoint{5.629768in}{1.610473in}}%
\pgfpathcurveto{\pgfqpoint{5.634812in}{1.615517in}}{\pgfqpoint{5.637646in}{1.622359in}}{\pgfqpoint{5.637646in}{1.629491in}}%
\pgfpathcurveto{\pgfqpoint{5.637646in}{1.636624in}}{\pgfqpoint{5.634812in}{1.643466in}}{\pgfqpoint{5.629768in}{1.648510in}}%
\pgfpathcurveto{\pgfqpoint{5.624724in}{1.653553in}}{\pgfqpoint{5.617883in}{1.656387in}}{\pgfqpoint{5.610750in}{1.656387in}}%
\pgfpathcurveto{\pgfqpoint{5.603617in}{1.656387in}}{\pgfqpoint{5.596776in}{1.653553in}}{\pgfqpoint{5.591732in}{1.648510in}}%
\pgfpathcurveto{\pgfqpoint{5.586688in}{1.643466in}}{\pgfqpoint{5.583854in}{1.636624in}}{\pgfqpoint{5.583854in}{1.629491in}}%
\pgfpathcurveto{\pgfqpoint{5.583854in}{1.622359in}}{\pgfqpoint{5.586688in}{1.615517in}}{\pgfqpoint{5.591732in}{1.610473in}}%
\pgfpathcurveto{\pgfqpoint{5.596776in}{1.605430in}}{\pgfqpoint{5.603617in}{1.602596in}}{\pgfqpoint{5.610750in}{1.602596in}}%
\pgfpathclose%
\pgfusepath{stroke,fill}%
\end{pgfscope}%
\begin{pgfscope}%
\pgfpathrectangle{\pgfqpoint{4.985294in}{0.500000in}}{\pgfqpoint{1.764706in}{1.700000in}}%
\pgfusepath{clip}%
\pgfsetbuttcap%
\pgfsetroundjoin%
\definecolor{currentfill}{rgb}{0.965928,0.738443,0.600540}%
\pgfsetfillcolor{currentfill}%
\pgfsetlinewidth{0.311001pt}%
\definecolor{currentstroke}{rgb}{1.000000,1.000000,1.000000}%
\pgfsetstrokecolor{currentstroke}%
\pgfsetdash{}{0pt}%
\pgfpathmoveto{\pgfqpoint{5.540784in}{1.134004in}}%
\pgfpathcurveto{\pgfqpoint{5.547917in}{1.134004in}}{\pgfqpoint{5.554758in}{1.136837in}}{\pgfqpoint{5.559802in}{1.141881in}}%
\pgfpathcurveto{\pgfqpoint{5.564846in}{1.146925in}}{\pgfqpoint{5.567680in}{1.153766in}}{\pgfqpoint{5.567680in}{1.160899in}}%
\pgfpathcurveto{\pgfqpoint{5.567680in}{1.168032in}}{\pgfqpoint{5.564846in}{1.174874in}}{\pgfqpoint{5.559802in}{1.179917in}}%
\pgfpathcurveto{\pgfqpoint{5.554758in}{1.184961in}}{\pgfqpoint{5.547917in}{1.187795in}}{\pgfqpoint{5.540784in}{1.187795in}}%
\pgfpathcurveto{\pgfqpoint{5.533651in}{1.187795in}}{\pgfqpoint{5.526809in}{1.184961in}}{\pgfqpoint{5.521766in}{1.179917in}}%
\pgfpathcurveto{\pgfqpoint{5.516722in}{1.174874in}}{\pgfqpoint{5.513888in}{1.168032in}}{\pgfqpoint{5.513888in}{1.160899in}}%
\pgfpathcurveto{\pgfqpoint{5.513888in}{1.153766in}}{\pgfqpoint{5.516722in}{1.146925in}}{\pgfqpoint{5.521766in}{1.141881in}}%
\pgfpathcurveto{\pgfqpoint{5.526809in}{1.136837in}}{\pgfqpoint{5.533651in}{1.134004in}}{\pgfqpoint{5.540784in}{1.134004in}}%
\pgfpathclose%
\pgfusepath{stroke,fill}%
\end{pgfscope}%
\begin{pgfscope}%
\pgfpathrectangle{\pgfqpoint{4.985294in}{0.500000in}}{\pgfqpoint{1.764706in}{1.700000in}}%
\pgfusepath{clip}%
\pgfsetbuttcap%
\pgfsetroundjoin%
\definecolor{currentfill}{rgb}{0.966120,0.744512,0.608720}%
\pgfsetfillcolor{currentfill}%
\pgfsetlinewidth{0.311001pt}%
\definecolor{currentstroke}{rgb}{1.000000,1.000000,1.000000}%
\pgfsetstrokecolor{currentstroke}%
\pgfsetdash{}{0pt}%
\pgfpathmoveto{\pgfqpoint{6.202624in}{1.263379in}}%
\pgfpathcurveto{\pgfqpoint{6.209757in}{1.263379in}}{\pgfqpoint{6.216598in}{1.266213in}}{\pgfqpoint{6.221642in}{1.271256in}}%
\pgfpathcurveto{\pgfqpoint{6.226686in}{1.276300in}}{\pgfqpoint{6.229520in}{1.283142in}}{\pgfqpoint{6.229520in}{1.290274in}}%
\pgfpathcurveto{\pgfqpoint{6.229520in}{1.297407in}}{\pgfqpoint{6.226686in}{1.304249in}}{\pgfqpoint{6.221642in}{1.309293in}}%
\pgfpathcurveto{\pgfqpoint{6.216598in}{1.314336in}}{\pgfqpoint{6.209757in}{1.317170in}}{\pgfqpoint{6.202624in}{1.317170in}}%
\pgfpathcurveto{\pgfqpoint{6.195491in}{1.317170in}}{\pgfqpoint{6.188650in}{1.314336in}}{\pgfqpoint{6.183606in}{1.309293in}}%
\pgfpathcurveto{\pgfqpoint{6.178562in}{1.304249in}}{\pgfqpoint{6.175728in}{1.297407in}}{\pgfqpoint{6.175728in}{1.290274in}}%
\pgfpathcurveto{\pgfqpoint{6.175728in}{1.283142in}}{\pgfqpoint{6.178562in}{1.276300in}}{\pgfqpoint{6.183606in}{1.271256in}}%
\pgfpathcurveto{\pgfqpoint{6.188650in}{1.266213in}}{\pgfqpoint{6.195491in}{1.263379in}}{\pgfqpoint{6.202624in}{1.263379in}}%
\pgfpathclose%
\pgfusepath{stroke,fill}%
\end{pgfscope}%
\begin{pgfscope}%
\pgfpathrectangle{\pgfqpoint{4.985294in}{0.500000in}}{\pgfqpoint{1.764706in}{1.700000in}}%
\pgfusepath{clip}%
\pgfsetbuttcap%
\pgfsetroundjoin%
\definecolor{currentfill}{rgb}{0.973271,0.850724,0.762998}%
\pgfsetfillcolor{currentfill}%
\pgfsetlinewidth{0.311001pt}%
\definecolor{currentstroke}{rgb}{1.000000,1.000000,1.000000}%
\pgfsetstrokecolor{currentstroke}%
\pgfsetdash{}{0pt}%
\pgfpathmoveto{\pgfqpoint{6.221907in}{1.035759in}}%
\pgfpathcurveto{\pgfqpoint{6.229040in}{1.035759in}}{\pgfqpoint{6.235882in}{1.038593in}}{\pgfqpoint{6.240925in}{1.043637in}}%
\pgfpathcurveto{\pgfqpoint{6.245969in}{1.048680in}}{\pgfqpoint{6.248803in}{1.055522in}}{\pgfqpoint{6.248803in}{1.062655in}}%
\pgfpathcurveto{\pgfqpoint{6.248803in}{1.069788in}}{\pgfqpoint{6.245969in}{1.076629in}}{\pgfqpoint{6.240925in}{1.081673in}}%
\pgfpathcurveto{\pgfqpoint{6.235882in}{1.086717in}}{\pgfqpoint{6.229040in}{1.089551in}}{\pgfqpoint{6.221907in}{1.089551in}}%
\pgfpathcurveto{\pgfqpoint{6.214774in}{1.089551in}}{\pgfqpoint{6.207933in}{1.086717in}}{\pgfqpoint{6.202889in}{1.081673in}}%
\pgfpathcurveto{\pgfqpoint{6.197845in}{1.076629in}}{\pgfqpoint{6.195012in}{1.069788in}}{\pgfqpoint{6.195012in}{1.062655in}}%
\pgfpathcurveto{\pgfqpoint{6.195012in}{1.055522in}}{\pgfqpoint{6.197845in}{1.048680in}}{\pgfqpoint{6.202889in}{1.043637in}}%
\pgfpathcurveto{\pgfqpoint{6.207933in}{1.038593in}}{\pgfqpoint{6.214774in}{1.035759in}}{\pgfqpoint{6.221907in}{1.035759in}}%
\pgfpathclose%
\pgfusepath{stroke,fill}%
\end{pgfscope}%
\begin{pgfscope}%
\pgfpathrectangle{\pgfqpoint{4.985294in}{0.500000in}}{\pgfqpoint{1.764706in}{1.700000in}}%
\pgfusepath{clip}%
\pgfsetbuttcap%
\pgfsetroundjoin%
\definecolor{currentfill}{rgb}{0.969359,0.803954,0.693832}%
\pgfsetfillcolor{currentfill}%
\pgfsetlinewidth{0.311001pt}%
\definecolor{currentstroke}{rgb}{1.000000,1.000000,1.000000}%
\pgfsetstrokecolor{currentstroke}%
\pgfsetdash{}{0pt}%
\pgfpathmoveto{\pgfqpoint{6.179774in}{1.672578in}}%
\pgfpathcurveto{\pgfqpoint{6.186906in}{1.672578in}}{\pgfqpoint{6.193748in}{1.675412in}}{\pgfqpoint{6.198792in}{1.680456in}}%
\pgfpathcurveto{\pgfqpoint{6.203835in}{1.685500in}}{\pgfqpoint{6.206669in}{1.692341in}}{\pgfqpoint{6.206669in}{1.699474in}}%
\pgfpathcurveto{\pgfqpoint{6.206669in}{1.706607in}}{\pgfqpoint{6.203835in}{1.713449in}}{\pgfqpoint{6.198792in}{1.718492in}}%
\pgfpathcurveto{\pgfqpoint{6.193748in}{1.723536in}}{\pgfqpoint{6.186906in}{1.726370in}}{\pgfqpoint{6.179774in}{1.726370in}}%
\pgfpathcurveto{\pgfqpoint{6.172641in}{1.726370in}}{\pgfqpoint{6.165799in}{1.723536in}}{\pgfqpoint{6.160755in}{1.718492in}}%
\pgfpathcurveto{\pgfqpoint{6.155712in}{1.713449in}}{\pgfqpoint{6.152878in}{1.706607in}}{\pgfqpoint{6.152878in}{1.699474in}}%
\pgfpathcurveto{\pgfqpoint{6.152878in}{1.692341in}}{\pgfqpoint{6.155712in}{1.685500in}}{\pgfqpoint{6.160755in}{1.680456in}}%
\pgfpathcurveto{\pgfqpoint{6.165799in}{1.675412in}}{\pgfqpoint{6.172641in}{1.672578in}}{\pgfqpoint{6.179774in}{1.672578in}}%
\pgfpathclose%
\pgfusepath{stroke,fill}%
\end{pgfscope}%
\begin{pgfscope}%
\pgfpathrectangle{\pgfqpoint{4.985294in}{0.500000in}}{\pgfqpoint{1.764706in}{1.700000in}}%
\pgfusepath{clip}%
\pgfsetbuttcap%
\pgfsetroundjoin%
\definecolor{currentfill}{rgb}{0.972726,0.844889,0.754401}%
\pgfsetfillcolor{currentfill}%
\pgfsetlinewidth{0.311001pt}%
\definecolor{currentstroke}{rgb}{1.000000,1.000000,1.000000}%
\pgfsetstrokecolor{currentstroke}%
\pgfsetdash{}{0pt}%
\pgfpathmoveto{\pgfqpoint{6.218376in}{1.549391in}}%
\pgfpathcurveto{\pgfqpoint{6.225509in}{1.549391in}}{\pgfqpoint{6.232351in}{1.552225in}}{\pgfqpoint{6.237394in}{1.557268in}}%
\pgfpathcurveto{\pgfqpoint{6.242438in}{1.562312in}}{\pgfqpoint{6.245272in}{1.569153in}}{\pgfqpoint{6.245272in}{1.576286in}}%
\pgfpathcurveto{\pgfqpoint{6.245272in}{1.583419in}}{\pgfqpoint{6.242438in}{1.590261in}}{\pgfqpoint{6.237394in}{1.595304in}}%
\pgfpathcurveto{\pgfqpoint{6.232351in}{1.600348in}}{\pgfqpoint{6.225509in}{1.603182in}}{\pgfqpoint{6.218376in}{1.603182in}}%
\pgfpathcurveto{\pgfqpoint{6.211243in}{1.603182in}}{\pgfqpoint{6.204402in}{1.600348in}}{\pgfqpoint{6.199358in}{1.595304in}}%
\pgfpathcurveto{\pgfqpoint{6.194314in}{1.590261in}}{\pgfqpoint{6.191481in}{1.583419in}}{\pgfqpoint{6.191481in}{1.576286in}}%
\pgfpathcurveto{\pgfqpoint{6.191481in}{1.569153in}}{\pgfqpoint{6.194314in}{1.562312in}}{\pgfqpoint{6.199358in}{1.557268in}}%
\pgfpathcurveto{\pgfqpoint{6.204402in}{1.552225in}}{\pgfqpoint{6.211243in}{1.549391in}}{\pgfqpoint{6.218376in}{1.549391in}}%
\pgfpathclose%
\pgfusepath{stroke,fill}%
\end{pgfscope}%
\begin{pgfscope}%
\pgfpathrectangle{\pgfqpoint{4.985294in}{0.500000in}}{\pgfqpoint{1.764706in}{1.700000in}}%
\pgfusepath{clip}%
\pgfsetbuttcap%
\pgfsetroundjoin%
\definecolor{currentfill}{rgb}{0.976961,0.885681,0.814303}%
\pgfsetfillcolor{currentfill}%
\pgfsetlinewidth{0.311001pt}%
\definecolor{currentstroke}{rgb}{1.000000,1.000000,1.000000}%
\pgfsetstrokecolor{currentstroke}%
\pgfsetdash{}{0pt}%
\pgfpathmoveto{\pgfqpoint{5.457670in}{1.359346in}}%
\pgfpathcurveto{\pgfqpoint{5.464802in}{1.359346in}}{\pgfqpoint{5.471644in}{1.362180in}}{\pgfqpoint{5.476688in}{1.367224in}}%
\pgfpathcurveto{\pgfqpoint{5.481731in}{1.372268in}}{\pgfqpoint{5.484565in}{1.379109in}}{\pgfqpoint{5.484565in}{1.386242in}}%
\pgfpathcurveto{\pgfqpoint{5.484565in}{1.393375in}}{\pgfqpoint{5.481731in}{1.400217in}}{\pgfqpoint{5.476688in}{1.405260in}}%
\pgfpathcurveto{\pgfqpoint{5.471644in}{1.410304in}}{\pgfqpoint{5.464802in}{1.413138in}}{\pgfqpoint{5.457670in}{1.413138in}}%
\pgfpathcurveto{\pgfqpoint{5.450537in}{1.413138in}}{\pgfqpoint{5.443695in}{1.410304in}}{\pgfqpoint{5.438651in}{1.405260in}}%
\pgfpathcurveto{\pgfqpoint{5.433608in}{1.400217in}}{\pgfqpoint{5.430774in}{1.393375in}}{\pgfqpoint{5.430774in}{1.386242in}}%
\pgfpathcurveto{\pgfqpoint{5.430774in}{1.379109in}}{\pgfqpoint{5.433608in}{1.372268in}}{\pgfqpoint{5.438651in}{1.367224in}}%
\pgfpathcurveto{\pgfqpoint{5.443695in}{1.362180in}}{\pgfqpoint{5.450537in}{1.359346in}}{\pgfqpoint{5.457670in}{1.359346in}}%
\pgfpathclose%
\pgfusepath{stroke,fill}%
\end{pgfscope}%
\begin{pgfscope}%
\pgfpathrectangle{\pgfqpoint{4.985294in}{0.500000in}}{\pgfqpoint{1.764706in}{1.700000in}}%
\pgfusepath{clip}%
\pgfsetbuttcap%
\pgfsetroundjoin%
\definecolor{currentfill}{rgb}{0.963559,0.632016,0.472047}%
\pgfsetfillcolor{currentfill}%
\pgfsetlinewidth{0.311001pt}%
\definecolor{currentstroke}{rgb}{1.000000,1.000000,1.000000}%
\pgfsetstrokecolor{currentstroke}%
\pgfsetdash{}{0pt}%
\pgfpathmoveto{\pgfqpoint{6.105474in}{1.720542in}}%
\pgfpathcurveto{\pgfqpoint{6.112606in}{1.720542in}}{\pgfqpoint{6.119448in}{1.723376in}}{\pgfqpoint{6.124492in}{1.728420in}}%
\pgfpathcurveto{\pgfqpoint{6.129535in}{1.733464in}}{\pgfqpoint{6.132369in}{1.740305in}}{\pgfqpoint{6.132369in}{1.747438in}}%
\pgfpathcurveto{\pgfqpoint{6.132369in}{1.754571in}}{\pgfqpoint{6.129535in}{1.761413in}}{\pgfqpoint{6.124492in}{1.766456in}}%
\pgfpathcurveto{\pgfqpoint{6.119448in}{1.771500in}}{\pgfqpoint{6.112606in}{1.774334in}}{\pgfqpoint{6.105474in}{1.774334in}}%
\pgfpathcurveto{\pgfqpoint{6.098341in}{1.774334in}}{\pgfqpoint{6.091499in}{1.771500in}}{\pgfqpoint{6.086455in}{1.766456in}}%
\pgfpathcurveto{\pgfqpoint{6.081412in}{1.761413in}}{\pgfqpoint{6.078578in}{1.754571in}}{\pgfqpoint{6.078578in}{1.747438in}}%
\pgfpathcurveto{\pgfqpoint{6.078578in}{1.740305in}}{\pgfqpoint{6.081412in}{1.733464in}}{\pgfqpoint{6.086455in}{1.728420in}}%
\pgfpathcurveto{\pgfqpoint{6.091499in}{1.723376in}}{\pgfqpoint{6.098341in}{1.720542in}}{\pgfqpoint{6.105474in}{1.720542in}}%
\pgfpathclose%
\pgfusepath{stroke,fill}%
\end{pgfscope}%
\begin{pgfscope}%
\pgfpathrectangle{\pgfqpoint{4.985294in}{0.500000in}}{\pgfqpoint{1.764706in}{1.700000in}}%
\pgfusepath{clip}%
\pgfsetbuttcap%
\pgfsetroundjoin%
\definecolor{currentfill}{rgb}{0.979124,0.903132,0.839793}%
\pgfsetfillcolor{currentfill}%
\pgfsetlinewidth{0.311001pt}%
\definecolor{currentstroke}{rgb}{1.000000,1.000000,1.000000}%
\pgfsetstrokecolor{currentstroke}%
\pgfsetdash{}{0pt}%
\pgfpathmoveto{\pgfqpoint{6.344278in}{1.282446in}}%
\pgfpathcurveto{\pgfqpoint{6.351411in}{1.282446in}}{\pgfqpoint{6.358253in}{1.285280in}}{\pgfqpoint{6.363297in}{1.290324in}}%
\pgfpathcurveto{\pgfqpoint{6.368340in}{1.295368in}}{\pgfqpoint{6.371174in}{1.302209in}}{\pgfqpoint{6.371174in}{1.309342in}}%
\pgfpathcurveto{\pgfqpoint{6.371174in}{1.316475in}}{\pgfqpoint{6.368340in}{1.323317in}}{\pgfqpoint{6.363297in}{1.328360in}}%
\pgfpathcurveto{\pgfqpoint{6.358253in}{1.333404in}}{\pgfqpoint{6.351411in}{1.336238in}}{\pgfqpoint{6.344278in}{1.336238in}}%
\pgfpathcurveto{\pgfqpoint{6.337146in}{1.336238in}}{\pgfqpoint{6.330304in}{1.333404in}}{\pgfqpoint{6.325260in}{1.328360in}}%
\pgfpathcurveto{\pgfqpoint{6.320217in}{1.323317in}}{\pgfqpoint{6.317383in}{1.316475in}}{\pgfqpoint{6.317383in}{1.309342in}}%
\pgfpathcurveto{\pgfqpoint{6.317383in}{1.302209in}}{\pgfqpoint{6.320217in}{1.295368in}}{\pgfqpoint{6.325260in}{1.290324in}}%
\pgfpathcurveto{\pgfqpoint{6.330304in}{1.285280in}}{\pgfqpoint{6.337146in}{1.282446in}}{\pgfqpoint{6.344278in}{1.282446in}}%
\pgfpathclose%
\pgfusepath{stroke,fill}%
\end{pgfscope}%
\begin{pgfscope}%
\pgfpathrectangle{\pgfqpoint{4.985294in}{0.500000in}}{\pgfqpoint{1.764706in}{1.700000in}}%
\pgfusepath{clip}%
\pgfsetbuttcap%
\pgfsetroundjoin%
\definecolor{currentfill}{rgb}{0.970718,0.821518,0.719872}%
\pgfsetfillcolor{currentfill}%
\pgfsetlinewidth{0.311001pt}%
\definecolor{currentstroke}{rgb}{1.000000,1.000000,1.000000}%
\pgfsetstrokecolor{currentstroke}%
\pgfsetdash{}{0pt}%
\pgfpathmoveto{\pgfqpoint{6.230760in}{1.240457in}}%
\pgfpathcurveto{\pgfqpoint{6.237893in}{1.240457in}}{\pgfqpoint{6.244735in}{1.243291in}}{\pgfqpoint{6.249778in}{1.248335in}}%
\pgfpathcurveto{\pgfqpoint{6.254822in}{1.253378in}}{\pgfqpoint{6.257656in}{1.260220in}}{\pgfqpoint{6.257656in}{1.267353in}}%
\pgfpathcurveto{\pgfqpoint{6.257656in}{1.274486in}}{\pgfqpoint{6.254822in}{1.281327in}}{\pgfqpoint{6.249778in}{1.286371in}}%
\pgfpathcurveto{\pgfqpoint{6.244735in}{1.291415in}}{\pgfqpoint{6.237893in}{1.294248in}}{\pgfqpoint{6.230760in}{1.294248in}}%
\pgfpathcurveto{\pgfqpoint{6.223627in}{1.294248in}}{\pgfqpoint{6.216786in}{1.291415in}}{\pgfqpoint{6.211742in}{1.286371in}}%
\pgfpathcurveto{\pgfqpoint{6.206698in}{1.281327in}}{\pgfqpoint{6.203865in}{1.274486in}}{\pgfqpoint{6.203865in}{1.267353in}}%
\pgfpathcurveto{\pgfqpoint{6.203865in}{1.260220in}}{\pgfqpoint{6.206698in}{1.253378in}}{\pgfqpoint{6.211742in}{1.248335in}}%
\pgfpathcurveto{\pgfqpoint{6.216786in}{1.243291in}}{\pgfqpoint{6.223627in}{1.240457in}}{\pgfqpoint{6.230760in}{1.240457in}}%
\pgfpathclose%
\pgfusepath{stroke,fill}%
\end{pgfscope}%
\begin{pgfscope}%
\pgfpathrectangle{\pgfqpoint{4.985294in}{0.500000in}}{\pgfqpoint{1.764706in}{1.700000in}}%
\pgfusepath{clip}%
\pgfsetbuttcap%
\pgfsetroundjoin%
\definecolor{currentfill}{rgb}{0.968105,0.786346,0.667739}%
\pgfsetfillcolor{currentfill}%
\pgfsetlinewidth{0.311001pt}%
\definecolor{currentstroke}{rgb}{1.000000,1.000000,1.000000}%
\pgfsetstrokecolor{currentstroke}%
\pgfsetdash{}{0pt}%
\pgfpathmoveto{\pgfqpoint{5.501524in}{1.250253in}}%
\pgfpathcurveto{\pgfqpoint{5.508657in}{1.250253in}}{\pgfqpoint{5.515499in}{1.253087in}}{\pgfqpoint{5.520542in}{1.258130in}}%
\pgfpathcurveto{\pgfqpoint{5.525586in}{1.263174in}}{\pgfqpoint{5.528420in}{1.270016in}}{\pgfqpoint{5.528420in}{1.277148in}}%
\pgfpathcurveto{\pgfqpoint{5.528420in}{1.284281in}}{\pgfqpoint{5.525586in}{1.291123in}}{\pgfqpoint{5.520542in}{1.296166in}}%
\pgfpathcurveto{\pgfqpoint{5.515499in}{1.301210in}}{\pgfqpoint{5.508657in}{1.304044in}}{\pgfqpoint{5.501524in}{1.304044in}}%
\pgfpathcurveto{\pgfqpoint{5.494391in}{1.304044in}}{\pgfqpoint{5.487550in}{1.301210in}}{\pgfqpoint{5.482506in}{1.296166in}}%
\pgfpathcurveto{\pgfqpoint{5.477462in}{1.291123in}}{\pgfqpoint{5.474628in}{1.284281in}}{\pgfqpoint{5.474628in}{1.277148in}}%
\pgfpathcurveto{\pgfqpoint{5.474628in}{1.270016in}}{\pgfqpoint{5.477462in}{1.263174in}}{\pgfqpoint{5.482506in}{1.258130in}}%
\pgfpathcurveto{\pgfqpoint{5.487550in}{1.253087in}}{\pgfqpoint{5.494391in}{1.250253in}}{\pgfqpoint{5.501524in}{1.250253in}}%
\pgfpathclose%
\pgfusepath{stroke,fill}%
\end{pgfscope}%
\begin{pgfscope}%
\pgfpathrectangle{\pgfqpoint{4.985294in}{0.500000in}}{\pgfqpoint{1.764706in}{1.700000in}}%
\pgfusepath{clip}%
\pgfsetbuttcap%
\pgfsetroundjoin%
\definecolor{currentfill}{rgb}{0.966328,0.750560,0.616961}%
\pgfsetfillcolor{currentfill}%
\pgfsetlinewidth{0.311001pt}%
\definecolor{currentstroke}{rgb}{1.000000,1.000000,1.000000}%
\pgfsetstrokecolor{currentstroke}%
\pgfsetdash{}{0pt}%
\pgfpathmoveto{\pgfqpoint{5.333223in}{1.322591in}}%
\pgfpathcurveto{\pgfqpoint{5.340356in}{1.322591in}}{\pgfqpoint{5.347198in}{1.325424in}}{\pgfqpoint{5.352241in}{1.330468in}}%
\pgfpathcurveto{\pgfqpoint{5.357285in}{1.335512in}}{\pgfqpoint{5.360119in}{1.342353in}}{\pgfqpoint{5.360119in}{1.349486in}}%
\pgfpathcurveto{\pgfqpoint{5.360119in}{1.356619in}}{\pgfqpoint{5.357285in}{1.363461in}}{\pgfqpoint{5.352241in}{1.368504in}}%
\pgfpathcurveto{\pgfqpoint{5.347198in}{1.373548in}}{\pgfqpoint{5.340356in}{1.376382in}}{\pgfqpoint{5.333223in}{1.376382in}}%
\pgfpathcurveto{\pgfqpoint{5.326090in}{1.376382in}}{\pgfqpoint{5.319249in}{1.373548in}}{\pgfqpoint{5.314205in}{1.368504in}}%
\pgfpathcurveto{\pgfqpoint{5.309161in}{1.363461in}}{\pgfqpoint{5.306327in}{1.356619in}}{\pgfqpoint{5.306327in}{1.349486in}}%
\pgfpathcurveto{\pgfqpoint{5.306327in}{1.342353in}}{\pgfqpoint{5.309161in}{1.335512in}}{\pgfqpoint{5.314205in}{1.330468in}}%
\pgfpathcurveto{\pgfqpoint{5.319249in}{1.325424in}}{\pgfqpoint{5.326090in}{1.322591in}}{\pgfqpoint{5.333223in}{1.322591in}}%
\pgfpathclose%
\pgfusepath{stroke,fill}%
\end{pgfscope}%
\begin{pgfscope}%
\pgfpathrectangle{\pgfqpoint{4.985294in}{0.500000in}}{\pgfqpoint{1.764706in}{1.700000in}}%
\pgfusepath{clip}%
\pgfsetbuttcap%
\pgfsetroundjoin%
\definecolor{currentfill}{rgb}{0.975644,0.874038,0.797253}%
\pgfsetfillcolor{currentfill}%
\pgfsetlinewidth{0.311001pt}%
\definecolor{currentstroke}{rgb}{1.000000,1.000000,1.000000}%
\pgfsetstrokecolor{currentstroke}%
\pgfsetdash{}{0pt}%
\pgfpathmoveto{\pgfqpoint{6.241260in}{1.122763in}}%
\pgfpathcurveto{\pgfqpoint{6.248393in}{1.122763in}}{\pgfqpoint{6.255235in}{1.125596in}}{\pgfqpoint{6.260278in}{1.130640in}}%
\pgfpathcurveto{\pgfqpoint{6.265322in}{1.135684in}}{\pgfqpoint{6.268156in}{1.142525in}}{\pgfqpoint{6.268156in}{1.149658in}}%
\pgfpathcurveto{\pgfqpoint{6.268156in}{1.156791in}}{\pgfqpoint{6.265322in}{1.163633in}}{\pgfqpoint{6.260278in}{1.168676in}}%
\pgfpathcurveto{\pgfqpoint{6.255235in}{1.173720in}}{\pgfqpoint{6.248393in}{1.176554in}}{\pgfqpoint{6.241260in}{1.176554in}}%
\pgfpathcurveto{\pgfqpoint{6.234127in}{1.176554in}}{\pgfqpoint{6.227286in}{1.173720in}}{\pgfqpoint{6.222242in}{1.168676in}}%
\pgfpathcurveto{\pgfqpoint{6.217198in}{1.163633in}}{\pgfqpoint{6.214365in}{1.156791in}}{\pgfqpoint{6.214365in}{1.149658in}}%
\pgfpathcurveto{\pgfqpoint{6.214365in}{1.142525in}}{\pgfqpoint{6.217198in}{1.135684in}}{\pgfqpoint{6.222242in}{1.130640in}}%
\pgfpathcurveto{\pgfqpoint{6.227286in}{1.125596in}}{\pgfqpoint{6.234127in}{1.122763in}}{\pgfqpoint{6.241260in}{1.122763in}}%
\pgfpathclose%
\pgfusepath{stroke,fill}%
\end{pgfscope}%
\begin{pgfscope}%
\pgfpathrectangle{\pgfqpoint{4.985294in}{0.500000in}}{\pgfqpoint{1.764706in}{1.700000in}}%
\pgfusepath{clip}%
\pgfsetbuttcap%
\pgfsetroundjoin%
\definecolor{currentfill}{rgb}{0.977657,0.891500,0.822809}%
\pgfsetfillcolor{currentfill}%
\pgfsetlinewidth{0.311001pt}%
\definecolor{currentstroke}{rgb}{1.000000,1.000000,1.000000}%
\pgfsetstrokecolor{currentstroke}%
\pgfsetdash{}{0pt}%
\pgfpathmoveto{\pgfqpoint{6.333764in}{1.451720in}}%
\pgfpathcurveto{\pgfqpoint{6.340897in}{1.451720in}}{\pgfqpoint{6.347738in}{1.454553in}}{\pgfqpoint{6.352782in}{1.459597in}}%
\pgfpathcurveto{\pgfqpoint{6.357826in}{1.464641in}}{\pgfqpoint{6.360660in}{1.471482in}}{\pgfqpoint{6.360660in}{1.478615in}}%
\pgfpathcurveto{\pgfqpoint{6.360660in}{1.485748in}}{\pgfqpoint{6.357826in}{1.492590in}}{\pgfqpoint{6.352782in}{1.497633in}}%
\pgfpathcurveto{\pgfqpoint{6.347738in}{1.502677in}}{\pgfqpoint{6.340897in}{1.505511in}}{\pgfqpoint{6.333764in}{1.505511in}}%
\pgfpathcurveto{\pgfqpoint{6.326631in}{1.505511in}}{\pgfqpoint{6.319789in}{1.502677in}}{\pgfqpoint{6.314746in}{1.497633in}}%
\pgfpathcurveto{\pgfqpoint{6.309702in}{1.492590in}}{\pgfqpoint{6.306868in}{1.485748in}}{\pgfqpoint{6.306868in}{1.478615in}}%
\pgfpathcurveto{\pgfqpoint{6.306868in}{1.471482in}}{\pgfqpoint{6.309702in}{1.464641in}}{\pgfqpoint{6.314746in}{1.459597in}}%
\pgfpathcurveto{\pgfqpoint{6.319789in}{1.454553in}}{\pgfqpoint{6.326631in}{1.451720in}}{\pgfqpoint{6.333764in}{1.451720in}}%
\pgfpathclose%
\pgfusepath{stroke,fill}%
\end{pgfscope}%
\begin{pgfscope}%
\pgfpathrectangle{\pgfqpoint{4.985294in}{0.500000in}}{\pgfqpoint{1.764706in}{1.700000in}}%
\pgfusepath{clip}%
\pgfsetbuttcap%
\pgfsetroundjoin%
\definecolor{currentfill}{rgb}{0.973271,0.850724,0.762998}%
\pgfsetfillcolor{currentfill}%
\pgfsetlinewidth{0.311001pt}%
\definecolor{currentstroke}{rgb}{1.000000,1.000000,1.000000}%
\pgfsetstrokecolor{currentstroke}%
\pgfsetdash{}{0pt}%
\pgfpathmoveto{\pgfqpoint{5.497846in}{1.016251in}}%
\pgfpathcurveto{\pgfqpoint{5.504979in}{1.016251in}}{\pgfqpoint{5.511820in}{1.019084in}}{\pgfqpoint{5.516864in}{1.024128in}}%
\pgfpathcurveto{\pgfqpoint{5.521908in}{1.029172in}}{\pgfqpoint{5.524742in}{1.036013in}}{\pgfqpoint{5.524742in}{1.043146in}}%
\pgfpathcurveto{\pgfqpoint{5.524742in}{1.050279in}}{\pgfqpoint{5.521908in}{1.057121in}}{\pgfqpoint{5.516864in}{1.062164in}}%
\pgfpathcurveto{\pgfqpoint{5.511820in}{1.067208in}}{\pgfqpoint{5.504979in}{1.070042in}}{\pgfqpoint{5.497846in}{1.070042in}}%
\pgfpathcurveto{\pgfqpoint{5.490713in}{1.070042in}}{\pgfqpoint{5.483871in}{1.067208in}}{\pgfqpoint{5.478828in}{1.062164in}}%
\pgfpathcurveto{\pgfqpoint{5.473784in}{1.057121in}}{\pgfqpoint{5.470950in}{1.050279in}}{\pgfqpoint{5.470950in}{1.043146in}}%
\pgfpathcurveto{\pgfqpoint{5.470950in}{1.036013in}}{\pgfqpoint{5.473784in}{1.029172in}}{\pgfqpoint{5.478828in}{1.024128in}}%
\pgfpathcurveto{\pgfqpoint{5.483871in}{1.019084in}}{\pgfqpoint{5.490713in}{1.016251in}}{\pgfqpoint{5.497846in}{1.016251in}}%
\pgfpathclose%
\pgfusepath{stroke,fill}%
\end{pgfscope}%
\begin{pgfscope}%
\pgfpathrectangle{\pgfqpoint{4.985294in}{0.500000in}}{\pgfqpoint{1.764706in}{1.700000in}}%
\pgfusepath{clip}%
\pgfsetbuttcap%
\pgfsetroundjoin%
\definecolor{currentfill}{rgb}{0.960778,0.559972,0.399412}%
\pgfsetfillcolor{currentfill}%
\pgfsetlinewidth{0.311001pt}%
\definecolor{currentstroke}{rgb}{1.000000,1.000000,1.000000}%
\pgfsetstrokecolor{currentstroke}%
\pgfsetdash{}{0pt}%
\pgfpathmoveto{\pgfqpoint{5.555378in}{1.248246in}}%
\pgfpathcurveto{\pgfqpoint{5.562511in}{1.248246in}}{\pgfqpoint{5.569353in}{1.251079in}}{\pgfqpoint{5.574396in}{1.256123in}}%
\pgfpathcurveto{\pgfqpoint{5.579440in}{1.261167in}}{\pgfqpoint{5.582274in}{1.268008in}}{\pgfqpoint{5.582274in}{1.275141in}}%
\pgfpathcurveto{\pgfqpoint{5.582274in}{1.282274in}}{\pgfqpoint{5.579440in}{1.289116in}}{\pgfqpoint{5.574396in}{1.294159in}}%
\pgfpathcurveto{\pgfqpoint{5.569353in}{1.299203in}}{\pgfqpoint{5.562511in}{1.302037in}}{\pgfqpoint{5.555378in}{1.302037in}}%
\pgfpathcurveto{\pgfqpoint{5.548245in}{1.302037in}}{\pgfqpoint{5.541404in}{1.299203in}}{\pgfqpoint{5.536360in}{1.294159in}}%
\pgfpathcurveto{\pgfqpoint{5.531316in}{1.289116in}}{\pgfqpoint{5.528482in}{1.282274in}}{\pgfqpoint{5.528482in}{1.275141in}}%
\pgfpathcurveto{\pgfqpoint{5.528482in}{1.268008in}}{\pgfqpoint{5.531316in}{1.261167in}}{\pgfqpoint{5.536360in}{1.256123in}}%
\pgfpathcurveto{\pgfqpoint{5.541404in}{1.251079in}}{\pgfqpoint{5.548245in}{1.248246in}}{\pgfqpoint{5.555378in}{1.248246in}}%
\pgfpathclose%
\pgfusepath{stroke,fill}%
\end{pgfscope}%
\begin{pgfscope}%
\pgfpathrectangle{\pgfqpoint{4.985294in}{0.500000in}}{\pgfqpoint{1.764706in}{1.700000in}}%
\pgfusepath{clip}%
\pgfsetbuttcap%
\pgfsetroundjoin%
\definecolor{currentfill}{rgb}{0.967398,0.774513,0.650573}%
\pgfsetfillcolor{currentfill}%
\pgfsetlinewidth{0.311001pt}%
\definecolor{currentstroke}{rgb}{1.000000,1.000000,1.000000}%
\pgfsetstrokecolor{currentstroke}%
\pgfsetdash{}{0pt}%
\pgfpathmoveto{\pgfqpoint{5.354694in}{1.189813in}}%
\pgfpathcurveto{\pgfqpoint{5.361827in}{1.189813in}}{\pgfqpoint{5.368668in}{1.192647in}}{\pgfqpoint{5.373712in}{1.197691in}}%
\pgfpathcurveto{\pgfqpoint{5.378756in}{1.202735in}}{\pgfqpoint{5.381590in}{1.209576in}}{\pgfqpoint{5.381590in}{1.216709in}}%
\pgfpathcurveto{\pgfqpoint{5.381590in}{1.223842in}}{\pgfqpoint{5.378756in}{1.230684in}}{\pgfqpoint{5.373712in}{1.235727in}}%
\pgfpathcurveto{\pgfqpoint{5.368668in}{1.240771in}}{\pgfqpoint{5.361827in}{1.243605in}}{\pgfqpoint{5.354694in}{1.243605in}}%
\pgfpathcurveto{\pgfqpoint{5.347561in}{1.243605in}}{\pgfqpoint{5.340719in}{1.240771in}}{\pgfqpoint{5.335676in}{1.235727in}}%
\pgfpathcurveto{\pgfqpoint{5.330632in}{1.230684in}}{\pgfqpoint{5.327798in}{1.223842in}}{\pgfqpoint{5.327798in}{1.216709in}}%
\pgfpathcurveto{\pgfqpoint{5.327798in}{1.209576in}}{\pgfqpoint{5.330632in}{1.202735in}}{\pgfqpoint{5.335676in}{1.197691in}}%
\pgfpathcurveto{\pgfqpoint{5.340719in}{1.192647in}}{\pgfqpoint{5.347561in}{1.189813in}}{\pgfqpoint{5.354694in}{1.189813in}}%
\pgfpathclose%
\pgfusepath{stroke,fill}%
\end{pgfscope}%
\begin{pgfscope}%
\pgfpathrectangle{\pgfqpoint{4.985294in}{0.500000in}}{\pgfqpoint{1.764706in}{1.700000in}}%
\pgfusepath{clip}%
\pgfsetbuttcap%
\pgfsetroundjoin%
\definecolor{currentfill}{rgb}{0.966120,0.744512,0.608720}%
\pgfsetfillcolor{currentfill}%
\pgfsetlinewidth{0.311001pt}%
\definecolor{currentstroke}{rgb}{1.000000,1.000000,1.000000}%
\pgfsetstrokecolor{currentstroke}%
\pgfsetdash{}{0pt}%
\pgfpathmoveto{\pgfqpoint{6.313483in}{1.024122in}}%
\pgfpathcurveto{\pgfqpoint{6.320616in}{1.024122in}}{\pgfqpoint{6.327458in}{1.026956in}}{\pgfqpoint{6.332501in}{1.031999in}}%
\pgfpathcurveto{\pgfqpoint{6.337545in}{1.037043in}}{\pgfqpoint{6.340379in}{1.043885in}}{\pgfqpoint{6.340379in}{1.051017in}}%
\pgfpathcurveto{\pgfqpoint{6.340379in}{1.058150in}}{\pgfqpoint{6.337545in}{1.064992in}}{\pgfqpoint{6.332501in}{1.070035in}}%
\pgfpathcurveto{\pgfqpoint{6.327458in}{1.075079in}}{\pgfqpoint{6.320616in}{1.077913in}}{\pgfqpoint{6.313483in}{1.077913in}}%
\pgfpathcurveto{\pgfqpoint{6.306350in}{1.077913in}}{\pgfqpoint{6.299509in}{1.075079in}}{\pgfqpoint{6.294465in}{1.070035in}}%
\pgfpathcurveto{\pgfqpoint{6.289422in}{1.064992in}}{\pgfqpoint{6.286588in}{1.058150in}}{\pgfqpoint{6.286588in}{1.051017in}}%
\pgfpathcurveto{\pgfqpoint{6.286588in}{1.043885in}}{\pgfqpoint{6.289422in}{1.037043in}}{\pgfqpoint{6.294465in}{1.031999in}}%
\pgfpathcurveto{\pgfqpoint{6.299509in}{1.026956in}}{\pgfqpoint{6.306350in}{1.024122in}}{\pgfqpoint{6.313483in}{1.024122in}}%
\pgfpathclose%
\pgfusepath{stroke,fill}%
\end{pgfscope}%
\begin{pgfscope}%
\pgfpathrectangle{\pgfqpoint{4.985294in}{0.500000in}}{\pgfqpoint{1.764706in}{1.700000in}}%
\pgfusepath{clip}%
\pgfsetbuttcap%
\pgfsetroundjoin%
\definecolor{currentfill}{rgb}{0.976287,0.879862,0.805788}%
\pgfsetfillcolor{currentfill}%
\pgfsetlinewidth{0.311001pt}%
\definecolor{currentstroke}{rgb}{1.000000,1.000000,1.000000}%
\pgfsetstrokecolor{currentstroke}%
\pgfsetdash{}{0pt}%
\pgfpathmoveto{\pgfqpoint{6.243704in}{1.094996in}}%
\pgfpathcurveto{\pgfqpoint{6.250837in}{1.094996in}}{\pgfqpoint{6.257679in}{1.097830in}}{\pgfqpoint{6.262722in}{1.102873in}}%
\pgfpathcurveto{\pgfqpoint{6.267766in}{1.107917in}}{\pgfqpoint{6.270600in}{1.114759in}}{\pgfqpoint{6.270600in}{1.121892in}}%
\pgfpathcurveto{\pgfqpoint{6.270600in}{1.129024in}}{\pgfqpoint{6.267766in}{1.135866in}}{\pgfqpoint{6.262722in}{1.140910in}}%
\pgfpathcurveto{\pgfqpoint{6.257679in}{1.145953in}}{\pgfqpoint{6.250837in}{1.148787in}}{\pgfqpoint{6.243704in}{1.148787in}}%
\pgfpathcurveto{\pgfqpoint{6.236571in}{1.148787in}}{\pgfqpoint{6.229730in}{1.145953in}}{\pgfqpoint{6.224686in}{1.140910in}}%
\pgfpathcurveto{\pgfqpoint{6.219642in}{1.135866in}}{\pgfqpoint{6.216808in}{1.129024in}}{\pgfqpoint{6.216808in}{1.121892in}}%
\pgfpathcurveto{\pgfqpoint{6.216808in}{1.114759in}}{\pgfqpoint{6.219642in}{1.107917in}}{\pgfqpoint{6.224686in}{1.102873in}}%
\pgfpathcurveto{\pgfqpoint{6.229730in}{1.097830in}}{\pgfqpoint{6.236571in}{1.094996in}}{\pgfqpoint{6.243704in}{1.094996in}}%
\pgfpathclose%
\pgfusepath{stroke,fill}%
\end{pgfscope}%
\begin{pgfscope}%
\pgfpathrectangle{\pgfqpoint{4.985294in}{0.500000in}}{\pgfqpoint{1.764706in}{1.700000in}}%
\pgfusepath{clip}%
\pgfsetbuttcap%
\pgfsetroundjoin%
\definecolor{currentfill}{rgb}{0.980678,0.914765,0.856766}%
\pgfsetfillcolor{currentfill}%
\pgfsetlinewidth{0.311001pt}%
\definecolor{currentstroke}{rgb}{1.000000,1.000000,1.000000}%
\pgfsetstrokecolor{currentstroke}%
\pgfsetdash{}{0pt}%
\pgfpathmoveto{\pgfqpoint{5.406057in}{1.357002in}}%
\pgfpathcurveto{\pgfqpoint{5.413189in}{1.357002in}}{\pgfqpoint{5.420031in}{1.359836in}}{\pgfqpoint{5.425075in}{1.364880in}}%
\pgfpathcurveto{\pgfqpoint{5.430118in}{1.369923in}}{\pgfqpoint{5.432952in}{1.376765in}}{\pgfqpoint{5.432952in}{1.383898in}}%
\pgfpathcurveto{\pgfqpoint{5.432952in}{1.391031in}}{\pgfqpoint{5.430118in}{1.397872in}}{\pgfqpoint{5.425075in}{1.402916in}}%
\pgfpathcurveto{\pgfqpoint{5.420031in}{1.407960in}}{\pgfqpoint{5.413189in}{1.410794in}}{\pgfqpoint{5.406057in}{1.410794in}}%
\pgfpathcurveto{\pgfqpoint{5.398924in}{1.410794in}}{\pgfqpoint{5.392082in}{1.407960in}}{\pgfqpoint{5.387039in}{1.402916in}}%
\pgfpathcurveto{\pgfqpoint{5.381995in}{1.397872in}}{\pgfqpoint{5.379161in}{1.391031in}}{\pgfqpoint{5.379161in}{1.383898in}}%
\pgfpathcurveto{\pgfqpoint{5.379161in}{1.376765in}}{\pgfqpoint{5.381995in}{1.369923in}}{\pgfqpoint{5.387039in}{1.364880in}}%
\pgfpathcurveto{\pgfqpoint{5.392082in}{1.359836in}}{\pgfqpoint{5.398924in}{1.357002in}}{\pgfqpoint{5.406057in}{1.357002in}}%
\pgfpathclose%
\pgfusepath{stroke,fill}%
\end{pgfscope}%
\begin{pgfscope}%
\pgfpathrectangle{\pgfqpoint{4.985294in}{0.500000in}}{\pgfqpoint{1.764706in}{1.700000in}}%
\pgfusepath{clip}%
\pgfsetbuttcap%
\pgfsetroundjoin%
\definecolor{currentfill}{rgb}{0.965042,0.701564,0.552889}%
\pgfsetfillcolor{currentfill}%
\pgfsetlinewidth{0.311001pt}%
\definecolor{currentstroke}{rgb}{1.000000,1.000000,1.000000}%
\pgfsetstrokecolor{currentstroke}%
\pgfsetdash{}{0pt}%
\pgfpathmoveto{\pgfqpoint{6.153490in}{1.540538in}}%
\pgfpathcurveto{\pgfqpoint{6.160623in}{1.540538in}}{\pgfqpoint{6.167464in}{1.543372in}}{\pgfqpoint{6.172508in}{1.548416in}}%
\pgfpathcurveto{\pgfqpoint{6.177552in}{1.553460in}}{\pgfqpoint{6.180385in}{1.560301in}}{\pgfqpoint{6.180385in}{1.567434in}}%
\pgfpathcurveto{\pgfqpoint{6.180385in}{1.574567in}}{\pgfqpoint{6.177552in}{1.581409in}}{\pgfqpoint{6.172508in}{1.586452in}}%
\pgfpathcurveto{\pgfqpoint{6.167464in}{1.591496in}}{\pgfqpoint{6.160623in}{1.594330in}}{\pgfqpoint{6.153490in}{1.594330in}}%
\pgfpathcurveto{\pgfqpoint{6.146357in}{1.594330in}}{\pgfqpoint{6.139515in}{1.591496in}}{\pgfqpoint{6.134472in}{1.586452in}}%
\pgfpathcurveto{\pgfqpoint{6.129428in}{1.581409in}}{\pgfqpoint{6.126594in}{1.574567in}}{\pgfqpoint{6.126594in}{1.567434in}}%
\pgfpathcurveto{\pgfqpoint{6.126594in}{1.560301in}}{\pgfqpoint{6.129428in}{1.553460in}}{\pgfqpoint{6.134472in}{1.548416in}}%
\pgfpathcurveto{\pgfqpoint{6.139515in}{1.543372in}}{\pgfqpoint{6.146357in}{1.540538in}}{\pgfqpoint{6.153490in}{1.540538in}}%
\pgfpathclose%
\pgfusepath{stroke,fill}%
\end{pgfscope}%
\begin{pgfscope}%
\pgfpathrectangle{\pgfqpoint{4.985294in}{0.500000in}}{\pgfqpoint{1.764706in}{1.700000in}}%
\pgfusepath{clip}%
\pgfsetbuttcap%
\pgfsetroundjoin%
\definecolor{currentfill}{rgb}{0.966812,0.762584,0.633643}%
\pgfsetfillcolor{currentfill}%
\pgfsetlinewidth{0.311001pt}%
\definecolor{currentstroke}{rgb}{1.000000,1.000000,1.000000}%
\pgfsetstrokecolor{currentstroke}%
\pgfsetdash{}{0pt}%
\pgfpathmoveto{\pgfqpoint{5.520283in}{1.180533in}}%
\pgfpathcurveto{\pgfqpoint{5.527416in}{1.180533in}}{\pgfqpoint{5.534258in}{1.183367in}}{\pgfqpoint{5.539301in}{1.188410in}}%
\pgfpathcurveto{\pgfqpoint{5.544345in}{1.193454in}}{\pgfqpoint{5.547179in}{1.200296in}}{\pgfqpoint{5.547179in}{1.207429in}}%
\pgfpathcurveto{\pgfqpoint{5.547179in}{1.214561in}}{\pgfqpoint{5.544345in}{1.221403in}}{\pgfqpoint{5.539301in}{1.226447in}}%
\pgfpathcurveto{\pgfqpoint{5.534258in}{1.231490in}}{\pgfqpoint{5.527416in}{1.234324in}}{\pgfqpoint{5.520283in}{1.234324in}}%
\pgfpathcurveto{\pgfqpoint{5.513150in}{1.234324in}}{\pgfqpoint{5.506309in}{1.231490in}}{\pgfqpoint{5.501265in}{1.226447in}}%
\pgfpathcurveto{\pgfqpoint{5.496221in}{1.221403in}}{\pgfqpoint{5.493387in}{1.214561in}}{\pgfqpoint{5.493387in}{1.207429in}}%
\pgfpathcurveto{\pgfqpoint{5.493387in}{1.200296in}}{\pgfqpoint{5.496221in}{1.193454in}}{\pgfqpoint{5.501265in}{1.188410in}}%
\pgfpathcurveto{\pgfqpoint{5.506309in}{1.183367in}}{\pgfqpoint{5.513150in}{1.180533in}}{\pgfqpoint{5.520283in}{1.180533in}}%
\pgfpathclose%
\pgfusepath{stroke,fill}%
\end{pgfscope}%
\begin{pgfscope}%
\pgfpathrectangle{\pgfqpoint{4.985294in}{0.500000in}}{\pgfqpoint{1.764706in}{1.700000in}}%
\pgfusepath{clip}%
\pgfsetbuttcap%
\pgfsetroundjoin%
\definecolor{currentfill}{rgb}{0.979124,0.903132,0.839793}%
\pgfsetfillcolor{currentfill}%
\pgfsetlinewidth{0.311001pt}%
\definecolor{currentstroke}{rgb}{1.000000,1.000000,1.000000}%
\pgfsetstrokecolor{currentstroke}%
\pgfsetdash{}{0pt}%
\pgfpathmoveto{\pgfqpoint{6.340262in}{1.264777in}}%
\pgfpathcurveto{\pgfqpoint{6.347395in}{1.264777in}}{\pgfqpoint{6.354237in}{1.267611in}}{\pgfqpoint{6.359280in}{1.272654in}}%
\pgfpathcurveto{\pgfqpoint{6.364324in}{1.277698in}}{\pgfqpoint{6.367158in}{1.284540in}}{\pgfqpoint{6.367158in}{1.291672in}}%
\pgfpathcurveto{\pgfqpoint{6.367158in}{1.298805in}}{\pgfqpoint{6.364324in}{1.305647in}}{\pgfqpoint{6.359280in}{1.310691in}}%
\pgfpathcurveto{\pgfqpoint{6.354237in}{1.315734in}}{\pgfqpoint{6.347395in}{1.318568in}}{\pgfqpoint{6.340262in}{1.318568in}}%
\pgfpathcurveto{\pgfqpoint{6.333130in}{1.318568in}}{\pgfqpoint{6.326288in}{1.315734in}}{\pgfqpoint{6.321244in}{1.310691in}}%
\pgfpathcurveto{\pgfqpoint{6.316201in}{1.305647in}}{\pgfqpoint{6.313367in}{1.298805in}}{\pgfqpoint{6.313367in}{1.291672in}}%
\pgfpathcurveto{\pgfqpoint{6.313367in}{1.284540in}}{\pgfqpoint{6.316201in}{1.277698in}}{\pgfqpoint{6.321244in}{1.272654in}}%
\pgfpathcurveto{\pgfqpoint{6.326288in}{1.267611in}}{\pgfqpoint{6.333130in}{1.264777in}}{\pgfqpoint{6.340262in}{1.264777in}}%
\pgfpathclose%
\pgfusepath{stroke,fill}%
\end{pgfscope}%
\begin{pgfscope}%
\pgfpathrectangle{\pgfqpoint{4.985294in}{0.500000in}}{\pgfqpoint{1.764706in}{1.700000in}}%
\pgfusepath{clip}%
\pgfsetbuttcap%
\pgfsetroundjoin%
\definecolor{currentfill}{rgb}{0.964032,0.651225,0.493258}%
\pgfsetfillcolor{currentfill}%
\pgfsetlinewidth{0.311001pt}%
\definecolor{currentstroke}{rgb}{1.000000,1.000000,1.000000}%
\pgfsetstrokecolor{currentstroke}%
\pgfsetdash{}{0pt}%
\pgfpathmoveto{\pgfqpoint{5.556970in}{1.743925in}}%
\pgfpathcurveto{\pgfqpoint{5.564103in}{1.743925in}}{\pgfqpoint{5.570944in}{1.746759in}}{\pgfqpoint{5.575988in}{1.751802in}}%
\pgfpathcurveto{\pgfqpoint{5.581032in}{1.756846in}}{\pgfqpoint{5.583865in}{1.763688in}}{\pgfqpoint{5.583865in}{1.770821in}}%
\pgfpathcurveto{\pgfqpoint{5.583865in}{1.777953in}}{\pgfqpoint{5.581032in}{1.784795in}}{\pgfqpoint{5.575988in}{1.789839in}}%
\pgfpathcurveto{\pgfqpoint{5.570944in}{1.794882in}}{\pgfqpoint{5.564103in}{1.797716in}}{\pgfqpoint{5.556970in}{1.797716in}}%
\pgfpathcurveto{\pgfqpoint{5.549837in}{1.797716in}}{\pgfqpoint{5.542995in}{1.794882in}}{\pgfqpoint{5.537952in}{1.789839in}}%
\pgfpathcurveto{\pgfqpoint{5.532908in}{1.784795in}}{\pgfqpoint{5.530074in}{1.777953in}}{\pgfqpoint{5.530074in}{1.770821in}}%
\pgfpathcurveto{\pgfqpoint{5.530074in}{1.763688in}}{\pgfqpoint{5.532908in}{1.756846in}}{\pgfqpoint{5.537952in}{1.751802in}}%
\pgfpathcurveto{\pgfqpoint{5.542995in}{1.746759in}}{\pgfqpoint{5.549837in}{1.743925in}}{\pgfqpoint{5.556970in}{1.743925in}}%
\pgfpathclose%
\pgfusepath{stroke,fill}%
\end{pgfscope}%
\begin{pgfscope}%
\pgfpathrectangle{\pgfqpoint{4.985294in}{0.500000in}}{\pgfqpoint{1.764706in}{1.700000in}}%
\pgfusepath{clip}%
\pgfsetbuttcap%
\pgfsetroundjoin%
\definecolor{currentfill}{rgb}{0.971694,0.833208,0.737161}%
\pgfsetfillcolor{currentfill}%
\pgfsetlinewidth{0.311001pt}%
\definecolor{currentstroke}{rgb}{1.000000,1.000000,1.000000}%
\pgfsetstrokecolor{currentstroke}%
\pgfsetdash{}{0pt}%
\pgfpathmoveto{\pgfqpoint{5.363994in}{1.238995in}}%
\pgfpathcurveto{\pgfqpoint{5.371127in}{1.238995in}}{\pgfqpoint{5.377969in}{1.241829in}}{\pgfqpoint{5.383013in}{1.246873in}}%
\pgfpathcurveto{\pgfqpoint{5.388056in}{1.251916in}}{\pgfqpoint{5.390890in}{1.258758in}}{\pgfqpoint{5.390890in}{1.265891in}}%
\pgfpathcurveto{\pgfqpoint{5.390890in}{1.273024in}}{\pgfqpoint{5.388056in}{1.279865in}}{\pgfqpoint{5.383013in}{1.284909in}}%
\pgfpathcurveto{\pgfqpoint{5.377969in}{1.289953in}}{\pgfqpoint{5.371127in}{1.292787in}}{\pgfqpoint{5.363994in}{1.292787in}}%
\pgfpathcurveto{\pgfqpoint{5.356862in}{1.292787in}}{\pgfqpoint{5.350020in}{1.289953in}}{\pgfqpoint{5.344976in}{1.284909in}}%
\pgfpathcurveto{\pgfqpoint{5.339933in}{1.279865in}}{\pgfqpoint{5.337099in}{1.273024in}}{\pgfqpoint{5.337099in}{1.265891in}}%
\pgfpathcurveto{\pgfqpoint{5.337099in}{1.258758in}}{\pgfqpoint{5.339933in}{1.251916in}}{\pgfqpoint{5.344976in}{1.246873in}}%
\pgfpathcurveto{\pgfqpoint{5.350020in}{1.241829in}}{\pgfqpoint{5.356862in}{1.238995in}}{\pgfqpoint{5.363994in}{1.238995in}}%
\pgfpathclose%
\pgfusepath{stroke,fill}%
\end{pgfscope}%
\begin{pgfscope}%
\pgfpathrectangle{\pgfqpoint{4.985294in}{0.500000in}}{\pgfqpoint{1.764706in}{1.700000in}}%
\pgfusepath{clip}%
\pgfsetbuttcap%
\pgfsetroundjoin%
\definecolor{currentfill}{rgb}{0.981377,0.920617,0.865369}%
\pgfsetfillcolor{currentfill}%
\pgfsetlinewidth{0.311001pt}%
\definecolor{currentstroke}{rgb}{1.000000,1.000000,1.000000}%
\pgfsetstrokecolor{currentstroke}%
\pgfsetdash{}{0pt}%
\pgfpathmoveto{\pgfqpoint{6.323115in}{1.299230in}}%
\pgfpathcurveto{\pgfqpoint{6.330247in}{1.299230in}}{\pgfqpoint{6.337089in}{1.302063in}}{\pgfqpoint{6.342133in}{1.307107in}}%
\pgfpathcurveto{\pgfqpoint{6.347176in}{1.312151in}}{\pgfqpoint{6.350010in}{1.318992in}}{\pgfqpoint{6.350010in}{1.326125in}}%
\pgfpathcurveto{\pgfqpoint{6.350010in}{1.333258in}}{\pgfqpoint{6.347176in}{1.340100in}}{\pgfqpoint{6.342133in}{1.345143in}}%
\pgfpathcurveto{\pgfqpoint{6.337089in}{1.350187in}}{\pgfqpoint{6.330247in}{1.353021in}}{\pgfqpoint{6.323115in}{1.353021in}}%
\pgfpathcurveto{\pgfqpoint{6.315982in}{1.353021in}}{\pgfqpoint{6.309140in}{1.350187in}}{\pgfqpoint{6.304096in}{1.345143in}}%
\pgfpathcurveto{\pgfqpoint{6.299053in}{1.340100in}}{\pgfqpoint{6.296219in}{1.333258in}}{\pgfqpoint{6.296219in}{1.326125in}}%
\pgfpathcurveto{\pgfqpoint{6.296219in}{1.318992in}}{\pgfqpoint{6.299053in}{1.312151in}}{\pgfqpoint{6.304096in}{1.307107in}}%
\pgfpathcurveto{\pgfqpoint{6.309140in}{1.302063in}}{\pgfqpoint{6.315982in}{1.299230in}}{\pgfqpoint{6.323115in}{1.299230in}}%
\pgfpathclose%
\pgfusepath{stroke,fill}%
\end{pgfscope}%
\begin{pgfscope}%
\pgfpathrectangle{\pgfqpoint{4.985294in}{0.500000in}}{\pgfqpoint{1.764706in}{1.700000in}}%
\pgfusepath{clip}%
\pgfsetbuttcap%
\pgfsetroundjoin%
\definecolor{currentfill}{rgb}{0.966560,0.756582,0.625273}%
\pgfsetfillcolor{currentfill}%
\pgfsetlinewidth{0.311001pt}%
\definecolor{currentstroke}{rgb}{1.000000,1.000000,1.000000}%
\pgfsetstrokecolor{currentstroke}%
\pgfsetdash{}{0pt}%
\pgfpathmoveto{\pgfqpoint{5.447272in}{0.971282in}}%
\pgfpathcurveto{\pgfqpoint{5.454404in}{0.971282in}}{\pgfqpoint{5.461246in}{0.974116in}}{\pgfqpoint{5.466290in}{0.979160in}}%
\pgfpathcurveto{\pgfqpoint{5.471333in}{0.984204in}}{\pgfqpoint{5.474167in}{0.991045in}}{\pgfqpoint{5.474167in}{0.998178in}}%
\pgfpathcurveto{\pgfqpoint{5.474167in}{1.005311in}}{\pgfqpoint{5.471333in}{1.012153in}}{\pgfqpoint{5.466290in}{1.017196in}}%
\pgfpathcurveto{\pgfqpoint{5.461246in}{1.022240in}}{\pgfqpoint{5.454404in}{1.025074in}}{\pgfqpoint{5.447272in}{1.025074in}}%
\pgfpathcurveto{\pgfqpoint{5.440139in}{1.025074in}}{\pgfqpoint{5.433297in}{1.022240in}}{\pgfqpoint{5.428254in}{1.017196in}}%
\pgfpathcurveto{\pgfqpoint{5.423210in}{1.012153in}}{\pgfqpoint{5.420376in}{1.005311in}}{\pgfqpoint{5.420376in}{0.998178in}}%
\pgfpathcurveto{\pgfqpoint{5.420376in}{0.991045in}}{\pgfqpoint{5.423210in}{0.984204in}}{\pgfqpoint{5.428254in}{0.979160in}}%
\pgfpathcurveto{\pgfqpoint{5.433297in}{0.974116in}}{\pgfqpoint{5.440139in}{0.971282in}}{\pgfqpoint{5.447272in}{0.971282in}}%
\pgfpathclose%
\pgfusepath{stroke,fill}%
\end{pgfscope}%
\begin{pgfscope}%
\pgfpathrectangle{\pgfqpoint{4.985294in}{0.500000in}}{\pgfqpoint{1.764706in}{1.700000in}}%
\pgfusepath{clip}%
\pgfsetbuttcap%
\pgfsetroundjoin%
\definecolor{currentfill}{rgb}{0.979124,0.903132,0.839793}%
\pgfsetfillcolor{currentfill}%
\pgfsetlinewidth{0.311001pt}%
\definecolor{currentstroke}{rgb}{1.000000,1.000000,1.000000}%
\pgfsetstrokecolor{currentstroke}%
\pgfsetdash{}{0pt}%
\pgfpathmoveto{\pgfqpoint{6.275640in}{1.196014in}}%
\pgfpathcurveto{\pgfqpoint{6.282773in}{1.196014in}}{\pgfqpoint{6.289615in}{1.198848in}}{\pgfqpoint{6.294658in}{1.203892in}}%
\pgfpathcurveto{\pgfqpoint{6.299702in}{1.208935in}}{\pgfqpoint{6.302536in}{1.215777in}}{\pgfqpoint{6.302536in}{1.222910in}}%
\pgfpathcurveto{\pgfqpoint{6.302536in}{1.230043in}}{\pgfqpoint{6.299702in}{1.236884in}}{\pgfqpoint{6.294658in}{1.241928in}}%
\pgfpathcurveto{\pgfqpoint{6.289615in}{1.246972in}}{\pgfqpoint{6.282773in}{1.249805in}}{\pgfqpoint{6.275640in}{1.249805in}}%
\pgfpathcurveto{\pgfqpoint{6.268508in}{1.249805in}}{\pgfqpoint{6.261666in}{1.246972in}}{\pgfqpoint{6.256622in}{1.241928in}}%
\pgfpathcurveto{\pgfqpoint{6.251579in}{1.236884in}}{\pgfqpoint{6.248745in}{1.230043in}}{\pgfqpoint{6.248745in}{1.222910in}}%
\pgfpathcurveto{\pgfqpoint{6.248745in}{1.215777in}}{\pgfqpoint{6.251579in}{1.208935in}}{\pgfqpoint{6.256622in}{1.203892in}}%
\pgfpathcurveto{\pgfqpoint{6.261666in}{1.198848in}}{\pgfqpoint{6.268508in}{1.196014in}}{\pgfqpoint{6.275640in}{1.196014in}}%
\pgfpathclose%
\pgfusepath{stroke,fill}%
\end{pgfscope}%
\begin{pgfscope}%
\pgfpathrectangle{\pgfqpoint{4.985294in}{0.500000in}}{\pgfqpoint{1.764706in}{1.700000in}}%
\pgfusepath{clip}%
\pgfsetbuttcap%
\pgfsetroundjoin%
\definecolor{currentfill}{rgb}{0.979891,0.908948,0.848279}%
\pgfsetfillcolor{currentfill}%
\pgfsetlinewidth{0.311001pt}%
\definecolor{currentstroke}{rgb}{1.000000,1.000000,1.000000}%
\pgfsetstrokecolor{currentstroke}%
\pgfsetdash{}{0pt}%
\pgfpathmoveto{\pgfqpoint{5.430154in}{1.224496in}}%
\pgfpathcurveto{\pgfqpoint{5.437286in}{1.224496in}}{\pgfqpoint{5.444128in}{1.227330in}}{\pgfqpoint{5.449172in}{1.232373in}}%
\pgfpathcurveto{\pgfqpoint{5.454215in}{1.237417in}}{\pgfqpoint{5.457049in}{1.244258in}}{\pgfqpoint{5.457049in}{1.251391in}}%
\pgfpathcurveto{\pgfqpoint{5.457049in}{1.258524in}}{\pgfqpoint{5.454215in}{1.265366in}}{\pgfqpoint{5.449172in}{1.270409in}}%
\pgfpathcurveto{\pgfqpoint{5.444128in}{1.275453in}}{\pgfqpoint{5.437286in}{1.278287in}}{\pgfqpoint{5.430154in}{1.278287in}}%
\pgfpathcurveto{\pgfqpoint{5.423021in}{1.278287in}}{\pgfqpoint{5.416179in}{1.275453in}}{\pgfqpoint{5.411136in}{1.270409in}}%
\pgfpathcurveto{\pgfqpoint{5.406092in}{1.265366in}}{\pgfqpoint{5.403258in}{1.258524in}}{\pgfqpoint{5.403258in}{1.251391in}}%
\pgfpathcurveto{\pgfqpoint{5.403258in}{1.244258in}}{\pgfqpoint{5.406092in}{1.237417in}}{\pgfqpoint{5.411136in}{1.232373in}}%
\pgfpathcurveto{\pgfqpoint{5.416179in}{1.227330in}}{\pgfqpoint{5.423021in}{1.224496in}}{\pgfqpoint{5.430154in}{1.224496in}}%
\pgfpathclose%
\pgfusepath{stroke,fill}%
\end{pgfscope}%
\begin{pgfscope}%
\pgfpathrectangle{\pgfqpoint{4.985294in}{0.500000in}}{\pgfqpoint{1.764706in}{1.700000in}}%
\pgfusepath{clip}%
\pgfsetbuttcap%
\pgfsetroundjoin%
\definecolor{currentfill}{rgb}{0.963884,0.644842,0.486120}%
\pgfsetfillcolor{currentfill}%
\pgfsetlinewidth{0.311001pt}%
\definecolor{currentstroke}{rgb}{1.000000,1.000000,1.000000}%
\pgfsetstrokecolor{currentstroke}%
\pgfsetdash{}{0pt}%
\pgfpathmoveto{\pgfqpoint{5.549112in}{1.748977in}}%
\pgfpathcurveto{\pgfqpoint{5.556245in}{1.748977in}}{\pgfqpoint{5.563087in}{1.751811in}}{\pgfqpoint{5.568130in}{1.756855in}}%
\pgfpathcurveto{\pgfqpoint{5.573174in}{1.761898in}}{\pgfqpoint{5.576008in}{1.768740in}}{\pgfqpoint{5.576008in}{1.775873in}}%
\pgfpathcurveto{\pgfqpoint{5.576008in}{1.783006in}}{\pgfqpoint{5.573174in}{1.789847in}}{\pgfqpoint{5.568130in}{1.794891in}}%
\pgfpathcurveto{\pgfqpoint{5.563087in}{1.799935in}}{\pgfqpoint{5.556245in}{1.802768in}}{\pgfqpoint{5.549112in}{1.802768in}}%
\pgfpathcurveto{\pgfqpoint{5.541979in}{1.802768in}}{\pgfqpoint{5.535138in}{1.799935in}}{\pgfqpoint{5.530094in}{1.794891in}}%
\pgfpathcurveto{\pgfqpoint{5.525050in}{1.789847in}}{\pgfqpoint{5.522216in}{1.783006in}}{\pgfqpoint{5.522216in}{1.775873in}}%
\pgfpathcurveto{\pgfqpoint{5.522216in}{1.768740in}}{\pgfqpoint{5.525050in}{1.761898in}}{\pgfqpoint{5.530094in}{1.756855in}}%
\pgfpathcurveto{\pgfqpoint{5.535138in}{1.751811in}}{\pgfqpoint{5.541979in}{1.748977in}}{\pgfqpoint{5.549112in}{1.748977in}}%
\pgfpathclose%
\pgfusepath{stroke,fill}%
\end{pgfscope}%
\begin{pgfscope}%
\pgfpathrectangle{\pgfqpoint{4.985294in}{0.500000in}}{\pgfqpoint{1.764706in}{1.700000in}}%
\pgfusepath{clip}%
\pgfsetbuttcap%
\pgfsetroundjoin%
\definecolor{currentfill}{rgb}{0.976961,0.885681,0.814303}%
\pgfsetfillcolor{currentfill}%
\pgfsetlinewidth{0.311001pt}%
\definecolor{currentstroke}{rgb}{1.000000,1.000000,1.000000}%
\pgfsetstrokecolor{currentstroke}%
\pgfsetdash{}{0pt}%
\pgfpathmoveto{\pgfqpoint{5.457511in}{1.376755in}}%
\pgfpathcurveto{\pgfqpoint{5.464643in}{1.376755in}}{\pgfqpoint{5.471485in}{1.379589in}}{\pgfqpoint{5.476529in}{1.384632in}}%
\pgfpathcurveto{\pgfqpoint{5.481572in}{1.389676in}}{\pgfqpoint{5.484406in}{1.396518in}}{\pgfqpoint{5.484406in}{1.403650in}}%
\pgfpathcurveto{\pgfqpoint{5.484406in}{1.410783in}}{\pgfqpoint{5.481572in}{1.417625in}}{\pgfqpoint{5.476529in}{1.422669in}}%
\pgfpathcurveto{\pgfqpoint{5.471485in}{1.427712in}}{\pgfqpoint{5.464643in}{1.430546in}}{\pgfqpoint{5.457511in}{1.430546in}}%
\pgfpathcurveto{\pgfqpoint{5.450378in}{1.430546in}}{\pgfqpoint{5.443536in}{1.427712in}}{\pgfqpoint{5.438492in}{1.422669in}}%
\pgfpathcurveto{\pgfqpoint{5.433449in}{1.417625in}}{\pgfqpoint{5.430615in}{1.410783in}}{\pgfqpoint{5.430615in}{1.403650in}}%
\pgfpathcurveto{\pgfqpoint{5.430615in}{1.396518in}}{\pgfqpoint{5.433449in}{1.389676in}}{\pgfqpoint{5.438492in}{1.384632in}}%
\pgfpathcurveto{\pgfqpoint{5.443536in}{1.379589in}}{\pgfqpoint{5.450378in}{1.376755in}}{\pgfqpoint{5.457511in}{1.376755in}}%
\pgfpathclose%
\pgfusepath{stroke,fill}%
\end{pgfscope}%
\begin{pgfscope}%
\pgfpathrectangle{\pgfqpoint{4.985294in}{0.500000in}}{\pgfqpoint{1.764706in}{1.700000in}}%
\pgfusepath{clip}%
\pgfsetbuttcap%
\pgfsetroundjoin%
\definecolor{currentfill}{rgb}{0.976961,0.885681,0.814303}%
\pgfsetfillcolor{currentfill}%
\pgfsetlinewidth{0.311001pt}%
\definecolor{currentstroke}{rgb}{1.000000,1.000000,1.000000}%
\pgfsetstrokecolor{currentstroke}%
\pgfsetdash{}{0pt}%
\pgfpathmoveto{\pgfqpoint{5.458853in}{1.181260in}}%
\pgfpathcurveto{\pgfqpoint{5.465986in}{1.181260in}}{\pgfqpoint{5.472828in}{1.184094in}}{\pgfqpoint{5.477871in}{1.189137in}}%
\pgfpathcurveto{\pgfqpoint{5.482915in}{1.194181in}}{\pgfqpoint{5.485749in}{1.201023in}}{\pgfqpoint{5.485749in}{1.208155in}}%
\pgfpathcurveto{\pgfqpoint{5.485749in}{1.215288in}}{\pgfqpoint{5.482915in}{1.222130in}}{\pgfqpoint{5.477871in}{1.227174in}}%
\pgfpathcurveto{\pgfqpoint{5.472828in}{1.232217in}}{\pgfqpoint{5.465986in}{1.235051in}}{\pgfqpoint{5.458853in}{1.235051in}}%
\pgfpathcurveto{\pgfqpoint{5.451720in}{1.235051in}}{\pgfqpoint{5.444879in}{1.232217in}}{\pgfqpoint{5.439835in}{1.227174in}}%
\pgfpathcurveto{\pgfqpoint{5.434791in}{1.222130in}}{\pgfqpoint{5.431957in}{1.215288in}}{\pgfqpoint{5.431957in}{1.208155in}}%
\pgfpathcurveto{\pgfqpoint{5.431957in}{1.201023in}}{\pgfqpoint{5.434791in}{1.194181in}}{\pgfqpoint{5.439835in}{1.189137in}}%
\pgfpathcurveto{\pgfqpoint{5.444879in}{1.184094in}}{\pgfqpoint{5.451720in}{1.181260in}}{\pgfqpoint{5.458853in}{1.181260in}}%
\pgfpathclose%
\pgfusepath{stroke,fill}%
\end{pgfscope}%
\begin{pgfscope}%
\pgfpathrectangle{\pgfqpoint{4.985294in}{0.500000in}}{\pgfqpoint{1.764706in}{1.700000in}}%
\pgfusepath{clip}%
\pgfsetbuttcap%
\pgfsetroundjoin%
\definecolor{currentfill}{rgb}{0.978376,0.897317,0.831308}%
\pgfsetfillcolor{currentfill}%
\pgfsetlinewidth{0.311001pt}%
\definecolor{currentstroke}{rgb}{1.000000,1.000000,1.000000}%
\pgfsetstrokecolor{currentstroke}%
\pgfsetdash{}{0pt}%
\pgfpathmoveto{\pgfqpoint{6.337894in}{1.233621in}}%
\pgfpathcurveto{\pgfqpoint{6.345026in}{1.233621in}}{\pgfqpoint{6.351868in}{1.236455in}}{\pgfqpoint{6.356912in}{1.241498in}}%
\pgfpathcurveto{\pgfqpoint{6.361955in}{1.246542in}}{\pgfqpoint{6.364789in}{1.253384in}}{\pgfqpoint{6.364789in}{1.260517in}}%
\pgfpathcurveto{\pgfqpoint{6.364789in}{1.267649in}}{\pgfqpoint{6.361955in}{1.274491in}}{\pgfqpoint{6.356912in}{1.279535in}}%
\pgfpathcurveto{\pgfqpoint{6.351868in}{1.284578in}}{\pgfqpoint{6.345026in}{1.287412in}}{\pgfqpoint{6.337894in}{1.287412in}}%
\pgfpathcurveto{\pgfqpoint{6.330761in}{1.287412in}}{\pgfqpoint{6.323919in}{1.284578in}}{\pgfqpoint{6.318875in}{1.279535in}}%
\pgfpathcurveto{\pgfqpoint{6.313832in}{1.274491in}}{\pgfqpoint{6.310998in}{1.267649in}}{\pgfqpoint{6.310998in}{1.260517in}}%
\pgfpathcurveto{\pgfqpoint{6.310998in}{1.253384in}}{\pgfqpoint{6.313832in}{1.246542in}}{\pgfqpoint{6.318875in}{1.241498in}}%
\pgfpathcurveto{\pgfqpoint{6.323919in}{1.236455in}}{\pgfqpoint{6.330761in}{1.233621in}}{\pgfqpoint{6.337894in}{1.233621in}}%
\pgfpathclose%
\pgfusepath{stroke,fill}%
\end{pgfscope}%
\begin{pgfscope}%
\pgfpathrectangle{\pgfqpoint{4.985294in}{0.500000in}}{\pgfqpoint{1.764706in}{1.700000in}}%
\pgfusepath{clip}%
\pgfsetbuttcap%
\pgfsetroundjoin%
\definecolor{currentfill}{rgb}{0.965440,0.720101,0.576404}%
\pgfsetfillcolor{currentfill}%
\pgfsetlinewidth{0.311001pt}%
\definecolor{currentstroke}{rgb}{1.000000,1.000000,1.000000}%
\pgfsetstrokecolor{currentstroke}%
\pgfsetdash{}{0pt}%
\pgfpathmoveto{\pgfqpoint{5.603950in}{0.936423in}}%
\pgfpathcurveto{\pgfqpoint{5.611083in}{0.936423in}}{\pgfqpoint{5.617924in}{0.939257in}}{\pgfqpoint{5.622968in}{0.944301in}}%
\pgfpathcurveto{\pgfqpoint{5.628012in}{0.949345in}}{\pgfqpoint{5.630845in}{0.956186in}}{\pgfqpoint{5.630845in}{0.963319in}}%
\pgfpathcurveto{\pgfqpoint{5.630845in}{0.970452in}}{\pgfqpoint{5.628012in}{0.977294in}}{\pgfqpoint{5.622968in}{0.982337in}}%
\pgfpathcurveto{\pgfqpoint{5.617924in}{0.987381in}}{\pgfqpoint{5.611083in}{0.990215in}}{\pgfqpoint{5.603950in}{0.990215in}}%
\pgfpathcurveto{\pgfqpoint{5.596817in}{0.990215in}}{\pgfqpoint{5.589975in}{0.987381in}}{\pgfqpoint{5.584932in}{0.982337in}}%
\pgfpathcurveto{\pgfqpoint{5.579888in}{0.977294in}}{\pgfqpoint{5.577054in}{0.970452in}}{\pgfqpoint{5.577054in}{0.963319in}}%
\pgfpathcurveto{\pgfqpoint{5.577054in}{0.956186in}}{\pgfqpoint{5.579888in}{0.949345in}}{\pgfqpoint{5.584932in}{0.944301in}}%
\pgfpathcurveto{\pgfqpoint{5.589975in}{0.939257in}}{\pgfqpoint{5.596817in}{0.936423in}}{\pgfqpoint{5.603950in}{0.936423in}}%
\pgfpathclose%
\pgfusepath{stroke,fill}%
\end{pgfscope}%
\begin{pgfscope}%
\pgfpathrectangle{\pgfqpoint{4.985294in}{0.500000in}}{\pgfqpoint{1.764706in}{1.700000in}}%
\pgfusepath{clip}%
\pgfsetbuttcap%
\pgfsetroundjoin%
\definecolor{currentfill}{rgb}{0.965042,0.701564,0.552889}%
\pgfsetfillcolor{currentfill}%
\pgfsetlinewidth{0.311001pt}%
\definecolor{currentstroke}{rgb}{1.000000,1.000000,1.000000}%
\pgfsetstrokecolor{currentstroke}%
\pgfsetdash{}{0pt}%
\pgfpathmoveto{\pgfqpoint{5.401125in}{1.007176in}}%
\pgfpathcurveto{\pgfqpoint{5.408258in}{1.007176in}}{\pgfqpoint{5.415100in}{1.010009in}}{\pgfqpoint{5.420143in}{1.015053in}}%
\pgfpathcurveto{\pgfqpoint{5.425187in}{1.020097in}}{\pgfqpoint{5.428021in}{1.026938in}}{\pgfqpoint{5.428021in}{1.034071in}}%
\pgfpathcurveto{\pgfqpoint{5.428021in}{1.041204in}}{\pgfqpoint{5.425187in}{1.048046in}}{\pgfqpoint{5.420143in}{1.053089in}}%
\pgfpathcurveto{\pgfqpoint{5.415100in}{1.058133in}}{\pgfqpoint{5.408258in}{1.060967in}}{\pgfqpoint{5.401125in}{1.060967in}}%
\pgfpathcurveto{\pgfqpoint{5.393992in}{1.060967in}}{\pgfqpoint{5.387151in}{1.058133in}}{\pgfqpoint{5.382107in}{1.053089in}}%
\pgfpathcurveto{\pgfqpoint{5.377063in}{1.048046in}}{\pgfqpoint{5.374230in}{1.041204in}}{\pgfqpoint{5.374230in}{1.034071in}}%
\pgfpathcurveto{\pgfqpoint{5.374230in}{1.026938in}}{\pgfqpoint{5.377063in}{1.020097in}}{\pgfqpoint{5.382107in}{1.015053in}}%
\pgfpathcurveto{\pgfqpoint{5.387151in}{1.010009in}}{\pgfqpoint{5.393992in}{1.007176in}}{\pgfqpoint{5.401125in}{1.007176in}}%
\pgfpathclose%
\pgfusepath{stroke,fill}%
\end{pgfscope}%
\begin{pgfscope}%
\pgfpathrectangle{\pgfqpoint{4.985294in}{0.500000in}}{\pgfqpoint{1.764706in}{1.700000in}}%
\pgfusepath{clip}%
\pgfsetbuttcap%
\pgfsetroundjoin%
\definecolor{currentfill}{rgb}{0.233340,0.102637,0.256977}%
\pgfsetfillcolor{currentfill}%
\pgfsetlinewidth{0.311001pt}%
\definecolor{currentstroke}{rgb}{1.000000,1.000000,1.000000}%
\pgfsetstrokecolor{currentstroke}%
\pgfsetdash{}{0pt}%
\pgfpathmoveto{\pgfqpoint{5.902230in}{1.568592in}}%
\pgfpathcurveto{\pgfqpoint{5.909362in}{1.568592in}}{\pgfqpoint{5.916204in}{1.571426in}}{\pgfqpoint{5.921248in}{1.576469in}}%
\pgfpathcurveto{\pgfqpoint{5.926291in}{1.581513in}}{\pgfqpoint{5.929125in}{1.588355in}}{\pgfqpoint{5.929125in}{1.595487in}}%
\pgfpathcurveto{\pgfqpoint{5.929125in}{1.602620in}}{\pgfqpoint{5.926291in}{1.609462in}}{\pgfqpoint{5.921248in}{1.614506in}}%
\pgfpathcurveto{\pgfqpoint{5.916204in}{1.619549in}}{\pgfqpoint{5.909362in}{1.622383in}}{\pgfqpoint{5.902230in}{1.622383in}}%
\pgfpathcurveto{\pgfqpoint{5.895097in}{1.622383in}}{\pgfqpoint{5.888255in}{1.619549in}}{\pgfqpoint{5.883211in}{1.614506in}}%
\pgfpathcurveto{\pgfqpoint{5.878168in}{1.609462in}}{\pgfqpoint{5.875334in}{1.602620in}}{\pgfqpoint{5.875334in}{1.595487in}}%
\pgfpathcurveto{\pgfqpoint{5.875334in}{1.588355in}}{\pgfqpoint{5.878168in}{1.581513in}}{\pgfqpoint{5.883211in}{1.576469in}}%
\pgfpathcurveto{\pgfqpoint{5.888255in}{1.571426in}}{\pgfqpoint{5.895097in}{1.568592in}}{\pgfqpoint{5.902230in}{1.568592in}}%
\pgfpathclose%
\pgfusepath{stroke,fill}%
\end{pgfscope}%
\begin{pgfscope}%
\pgfpathrectangle{\pgfqpoint{4.985294in}{0.500000in}}{\pgfqpoint{1.764706in}{1.700000in}}%
\pgfusepath{clip}%
\pgfsetbuttcap%
\pgfsetroundjoin%
\definecolor{currentfill}{rgb}{0.975018,0.868213,0.788710}%
\pgfsetfillcolor{currentfill}%
\pgfsetlinewidth{0.311001pt}%
\definecolor{currentstroke}{rgb}{1.000000,1.000000,1.000000}%
\pgfsetstrokecolor{currentstroke}%
\pgfsetdash{}{0pt}%
\pgfpathmoveto{\pgfqpoint{5.477555in}{1.145094in}}%
\pgfpathcurveto{\pgfqpoint{5.484688in}{1.145094in}}{\pgfqpoint{5.491529in}{1.147928in}}{\pgfqpoint{5.496573in}{1.152971in}}%
\pgfpathcurveto{\pgfqpoint{5.501617in}{1.158015in}}{\pgfqpoint{5.504451in}{1.164857in}}{\pgfqpoint{5.504451in}{1.171989in}}%
\pgfpathcurveto{\pgfqpoint{5.504451in}{1.179122in}}{\pgfqpoint{5.501617in}{1.185964in}}{\pgfqpoint{5.496573in}{1.191008in}}%
\pgfpathcurveto{\pgfqpoint{5.491529in}{1.196051in}}{\pgfqpoint{5.484688in}{1.198885in}}{\pgfqpoint{5.477555in}{1.198885in}}%
\pgfpathcurveto{\pgfqpoint{5.470422in}{1.198885in}}{\pgfqpoint{5.463580in}{1.196051in}}{\pgfqpoint{5.458537in}{1.191008in}}%
\pgfpathcurveto{\pgfqpoint{5.453493in}{1.185964in}}{\pgfqpoint{5.450659in}{1.179122in}}{\pgfqpoint{5.450659in}{1.171989in}}%
\pgfpathcurveto{\pgfqpoint{5.450659in}{1.164857in}}{\pgfqpoint{5.453493in}{1.158015in}}{\pgfqpoint{5.458537in}{1.152971in}}%
\pgfpathcurveto{\pgfqpoint{5.463580in}{1.147928in}}{\pgfqpoint{5.470422in}{1.145094in}}{\pgfqpoint{5.477555in}{1.145094in}}%
\pgfpathclose%
\pgfusepath{stroke,fill}%
\end{pgfscope}%
\begin{pgfscope}%
\pgfpathrectangle{\pgfqpoint{4.985294in}{0.500000in}}{\pgfqpoint{1.764706in}{1.700000in}}%
\pgfusepath{clip}%
\pgfsetbuttcap%
\pgfsetroundjoin%
\definecolor{currentfill}{rgb}{0.950851,0.435000,0.297228}%
\pgfsetfillcolor{currentfill}%
\pgfsetlinewidth{0.311001pt}%
\definecolor{currentstroke}{rgb}{1.000000,1.000000,1.000000}%
\pgfsetstrokecolor{currentstroke}%
\pgfsetdash{}{0pt}%
\pgfpathmoveto{\pgfqpoint{6.448308in}{1.388729in}}%
\pgfpathcurveto{\pgfqpoint{6.455441in}{1.388729in}}{\pgfqpoint{6.462282in}{1.391563in}}{\pgfqpoint{6.467326in}{1.396606in}}%
\pgfpathcurveto{\pgfqpoint{6.472370in}{1.401650in}}{\pgfqpoint{6.475204in}{1.408492in}}{\pgfqpoint{6.475204in}{1.415625in}}%
\pgfpathcurveto{\pgfqpoint{6.475204in}{1.422757in}}{\pgfqpoint{6.472370in}{1.429599in}}{\pgfqpoint{6.467326in}{1.434643in}}%
\pgfpathcurveto{\pgfqpoint{6.462282in}{1.439686in}}{\pgfqpoint{6.455441in}{1.442520in}}{\pgfqpoint{6.448308in}{1.442520in}}%
\pgfpathcurveto{\pgfqpoint{6.441175in}{1.442520in}}{\pgfqpoint{6.434333in}{1.439686in}}{\pgfqpoint{6.429290in}{1.434643in}}%
\pgfpathcurveto{\pgfqpoint{6.424246in}{1.429599in}}{\pgfqpoint{6.421412in}{1.422757in}}{\pgfqpoint{6.421412in}{1.415625in}}%
\pgfpathcurveto{\pgfqpoint{6.421412in}{1.408492in}}{\pgfqpoint{6.424246in}{1.401650in}}{\pgfqpoint{6.429290in}{1.396606in}}%
\pgfpathcurveto{\pgfqpoint{6.434333in}{1.391563in}}{\pgfqpoint{6.441175in}{1.388729in}}{\pgfqpoint{6.448308in}{1.388729in}}%
\pgfpathclose%
\pgfusepath{stroke,fill}%
\end{pgfscope}%
\begin{pgfscope}%
\pgfpathrectangle{\pgfqpoint{4.985294in}{0.500000in}}{\pgfqpoint{1.764706in}{1.700000in}}%
\pgfusepath{clip}%
\pgfsetbuttcap%
\pgfsetroundjoin%
\definecolor{currentfill}{rgb}{0.967092,0.768560,0.642079}%
\pgfsetfillcolor{currentfill}%
\pgfsetlinewidth{0.311001pt}%
\definecolor{currentstroke}{rgb}{1.000000,1.000000,1.000000}%
\pgfsetstrokecolor{currentstroke}%
\pgfsetdash{}{0pt}%
\pgfpathmoveto{\pgfqpoint{5.349110in}{1.208720in}}%
\pgfpathcurveto{\pgfqpoint{5.356243in}{1.208720in}}{\pgfqpoint{5.363084in}{1.211554in}}{\pgfqpoint{5.368128in}{1.216598in}}%
\pgfpathcurveto{\pgfqpoint{5.373171in}{1.221642in}}{\pgfqpoint{5.376005in}{1.228483in}}{\pgfqpoint{5.376005in}{1.235616in}}%
\pgfpathcurveto{\pgfqpoint{5.376005in}{1.242749in}}{\pgfqpoint{5.373171in}{1.249590in}}{\pgfqpoint{5.368128in}{1.254634in}}%
\pgfpathcurveto{\pgfqpoint{5.363084in}{1.259678in}}{\pgfqpoint{5.356243in}{1.262512in}}{\pgfqpoint{5.349110in}{1.262512in}}%
\pgfpathcurveto{\pgfqpoint{5.341977in}{1.262512in}}{\pgfqpoint{5.335135in}{1.259678in}}{\pgfqpoint{5.330092in}{1.254634in}}%
\pgfpathcurveto{\pgfqpoint{5.325048in}{1.249590in}}{\pgfqpoint{5.322214in}{1.242749in}}{\pgfqpoint{5.322214in}{1.235616in}}%
\pgfpathcurveto{\pgfqpoint{5.322214in}{1.228483in}}{\pgfqpoint{5.325048in}{1.221642in}}{\pgfqpoint{5.330092in}{1.216598in}}%
\pgfpathcurveto{\pgfqpoint{5.335135in}{1.211554in}}{\pgfqpoint{5.341977in}{1.208720in}}{\pgfqpoint{5.349110in}{1.208720in}}%
\pgfpathclose%
\pgfusepath{stroke,fill}%
\end{pgfscope}%
\begin{pgfscope}%
\pgfpathrectangle{\pgfqpoint{4.985294in}{0.500000in}}{\pgfqpoint{1.764706in}{1.700000in}}%
\pgfusepath{clip}%
\pgfsetbuttcap%
\pgfsetroundjoin%
\definecolor{currentfill}{rgb}{0.965928,0.738443,0.600540}%
\pgfsetfillcolor{currentfill}%
\pgfsetlinewidth{0.311001pt}%
\definecolor{currentstroke}{rgb}{1.000000,1.000000,1.000000}%
\pgfsetstrokecolor{currentstroke}%
\pgfsetdash{}{0pt}%
\pgfpathmoveto{\pgfqpoint{6.383993in}{1.459208in}}%
\pgfpathcurveto{\pgfqpoint{6.391126in}{1.459208in}}{\pgfqpoint{6.397968in}{1.462042in}}{\pgfqpoint{6.403012in}{1.467085in}}%
\pgfpathcurveto{\pgfqpoint{6.408055in}{1.472129in}}{\pgfqpoint{6.410889in}{1.478970in}}{\pgfqpoint{6.410889in}{1.486103in}}%
\pgfpathcurveto{\pgfqpoint{6.410889in}{1.493236in}}{\pgfqpoint{6.408055in}{1.500078in}}{\pgfqpoint{6.403012in}{1.505121in}}%
\pgfpathcurveto{\pgfqpoint{6.397968in}{1.510165in}}{\pgfqpoint{6.391126in}{1.512999in}}{\pgfqpoint{6.383993in}{1.512999in}}%
\pgfpathcurveto{\pgfqpoint{6.376861in}{1.512999in}}{\pgfqpoint{6.370019in}{1.510165in}}{\pgfqpoint{6.364975in}{1.505121in}}%
\pgfpathcurveto{\pgfqpoint{6.359932in}{1.500078in}}{\pgfqpoint{6.357098in}{1.493236in}}{\pgfqpoint{6.357098in}{1.486103in}}%
\pgfpathcurveto{\pgfqpoint{6.357098in}{1.478970in}}{\pgfqpoint{6.359932in}{1.472129in}}{\pgfqpoint{6.364975in}{1.467085in}}%
\pgfpathcurveto{\pgfqpoint{6.370019in}{1.462042in}}{\pgfqpoint{6.376861in}{1.459208in}}{\pgfqpoint{6.383993in}{1.459208in}}%
\pgfpathclose%
\pgfusepath{stroke,fill}%
\end{pgfscope}%
\begin{pgfscope}%
\pgfpathrectangle{\pgfqpoint{4.985294in}{0.500000in}}{\pgfqpoint{1.764706in}{1.700000in}}%
\pgfusepath{clip}%
\pgfsetbuttcap%
\pgfsetroundjoin%
\definecolor{currentfill}{rgb}{0.980678,0.914765,0.856766}%
\pgfsetfillcolor{currentfill}%
\pgfsetlinewidth{0.311001pt}%
\definecolor{currentstroke}{rgb}{1.000000,1.000000,1.000000}%
\pgfsetstrokecolor{currentstroke}%
\pgfsetdash{}{0pt}%
\pgfpathmoveto{\pgfqpoint{5.403773in}{1.315568in}}%
\pgfpathcurveto{\pgfqpoint{5.410906in}{1.315568in}}{\pgfqpoint{5.417747in}{1.318402in}}{\pgfqpoint{5.422791in}{1.323446in}}%
\pgfpathcurveto{\pgfqpoint{5.427835in}{1.328489in}}{\pgfqpoint{5.430668in}{1.335331in}}{\pgfqpoint{5.430668in}{1.342464in}}%
\pgfpathcurveto{\pgfqpoint{5.430668in}{1.349597in}}{\pgfqpoint{5.427835in}{1.356438in}}{\pgfqpoint{5.422791in}{1.361482in}}%
\pgfpathcurveto{\pgfqpoint{5.417747in}{1.366526in}}{\pgfqpoint{5.410906in}{1.369360in}}{\pgfqpoint{5.403773in}{1.369360in}}%
\pgfpathcurveto{\pgfqpoint{5.396640in}{1.369360in}}{\pgfqpoint{5.389798in}{1.366526in}}{\pgfqpoint{5.384755in}{1.361482in}}%
\pgfpathcurveto{\pgfqpoint{5.379711in}{1.356438in}}{\pgfqpoint{5.376877in}{1.349597in}}{\pgfqpoint{5.376877in}{1.342464in}}%
\pgfpathcurveto{\pgfqpoint{5.376877in}{1.335331in}}{\pgfqpoint{5.379711in}{1.328489in}}{\pgfqpoint{5.384755in}{1.323446in}}%
\pgfpathcurveto{\pgfqpoint{5.389798in}{1.318402in}}{\pgfqpoint{5.396640in}{1.315568in}}{\pgfqpoint{5.403773in}{1.315568in}}%
\pgfpathclose%
\pgfusepath{stroke,fill}%
\end{pgfscope}%
\begin{pgfscope}%
\pgfpathrectangle{\pgfqpoint{4.985294in}{0.500000in}}{\pgfqpoint{1.764706in}{1.700000in}}%
\pgfusepath{clip}%
\pgfsetbuttcap%
\pgfsetroundjoin%
\definecolor{currentfill}{rgb}{0.698038,0.088972,0.346299}%
\pgfsetfillcolor{currentfill}%
\pgfsetlinewidth{0.311001pt}%
\definecolor{currentstroke}{rgb}{1.000000,1.000000,1.000000}%
\pgfsetstrokecolor{currentstroke}%
\pgfsetdash{}{0pt}%
\pgfpathmoveto{\pgfqpoint{6.141820in}{1.902330in}}%
\pgfpathcurveto{\pgfqpoint{6.148952in}{1.902330in}}{\pgfqpoint{6.155794in}{1.905164in}}{\pgfqpoint{6.160838in}{1.910207in}}%
\pgfpathcurveto{\pgfqpoint{6.165881in}{1.915251in}}{\pgfqpoint{6.168715in}{1.922093in}}{\pgfqpoint{6.168715in}{1.929226in}}%
\pgfpathcurveto{\pgfqpoint{6.168715in}{1.936358in}}{\pgfqpoint{6.165881in}{1.943200in}}{\pgfqpoint{6.160838in}{1.948244in}}%
\pgfpathcurveto{\pgfqpoint{6.155794in}{1.953287in}}{\pgfqpoint{6.148952in}{1.956121in}}{\pgfqpoint{6.141820in}{1.956121in}}%
\pgfpathcurveto{\pgfqpoint{6.134687in}{1.956121in}}{\pgfqpoint{6.127845in}{1.953287in}}{\pgfqpoint{6.122801in}{1.948244in}}%
\pgfpathcurveto{\pgfqpoint{6.117758in}{1.943200in}}{\pgfqpoint{6.114924in}{1.936358in}}{\pgfqpoint{6.114924in}{1.929226in}}%
\pgfpathcurveto{\pgfqpoint{6.114924in}{1.922093in}}{\pgfqpoint{6.117758in}{1.915251in}}{\pgfqpoint{6.122801in}{1.910207in}}%
\pgfpathcurveto{\pgfqpoint{6.127845in}{1.905164in}}{\pgfqpoint{6.134687in}{1.902330in}}{\pgfqpoint{6.141820in}{1.902330in}}%
\pgfpathclose%
\pgfusepath{stroke,fill}%
\end{pgfscope}%
\begin{pgfscope}%
\pgfpathrectangle{\pgfqpoint{4.985294in}{0.500000in}}{\pgfqpoint{1.764706in}{1.700000in}}%
\pgfusepath{clip}%
\pgfsetbuttcap%
\pgfsetroundjoin%
\definecolor{currentfill}{rgb}{0.010608,0.018082,0.100187}%
\pgfsetfillcolor{currentfill}%
\pgfsetlinewidth{0.311001pt}%
\definecolor{currentstroke}{rgb}{1.000000,1.000000,1.000000}%
\pgfsetstrokecolor{currentstroke}%
\pgfsetdash{}{0pt}%
\pgfpathmoveto{\pgfqpoint{6.593933in}{1.299280in}}%
\pgfpathcurveto{\pgfqpoint{6.601066in}{1.299280in}}{\pgfqpoint{6.607908in}{1.302114in}}{\pgfqpoint{6.612952in}{1.307158in}}%
\pgfpathcurveto{\pgfqpoint{6.617995in}{1.312202in}}{\pgfqpoint{6.620829in}{1.319043in}}{\pgfqpoint{6.620829in}{1.326176in}}%
\pgfpathcurveto{\pgfqpoint{6.620829in}{1.333309in}}{\pgfqpoint{6.617995in}{1.340151in}}{\pgfqpoint{6.612952in}{1.345194in}}%
\pgfpathcurveto{\pgfqpoint{6.607908in}{1.350238in}}{\pgfqpoint{6.601066in}{1.353072in}}{\pgfqpoint{6.593933in}{1.353072in}}%
\pgfpathcurveto{\pgfqpoint{6.586801in}{1.353072in}}{\pgfqpoint{6.579959in}{1.350238in}}{\pgfqpoint{6.574915in}{1.345194in}}%
\pgfpathcurveto{\pgfqpoint{6.569872in}{1.340151in}}{\pgfqpoint{6.567038in}{1.333309in}}{\pgfqpoint{6.567038in}{1.326176in}}%
\pgfpathcurveto{\pgfqpoint{6.567038in}{1.319043in}}{\pgfqpoint{6.569872in}{1.312202in}}{\pgfqpoint{6.574915in}{1.307158in}}%
\pgfpathcurveto{\pgfqpoint{6.579959in}{1.302114in}}{\pgfqpoint{6.586801in}{1.299280in}}{\pgfqpoint{6.593933in}{1.299280in}}%
\pgfpathclose%
\pgfusepath{stroke,fill}%
\end{pgfscope}%
\begin{pgfscope}%
\pgfpathrectangle{\pgfqpoint{4.985294in}{0.500000in}}{\pgfqpoint{1.764706in}{1.700000in}}%
\pgfusepath{clip}%
\pgfsetbuttcap%
\pgfsetroundjoin%
\definecolor{currentfill}{rgb}{0.964306,0.663930,0.507747}%
\pgfsetfillcolor{currentfill}%
\pgfsetlinewidth{0.311001pt}%
\definecolor{currentstroke}{rgb}{1.000000,1.000000,1.000000}%
\pgfsetstrokecolor{currentstroke}%
\pgfsetdash{}{0pt}%
\pgfpathmoveto{\pgfqpoint{5.544988in}{1.184596in}}%
\pgfpathcurveto{\pgfqpoint{5.552121in}{1.184596in}}{\pgfqpoint{5.558963in}{1.187430in}}{\pgfqpoint{5.564006in}{1.192474in}}%
\pgfpathcurveto{\pgfqpoint{5.569050in}{1.197518in}}{\pgfqpoint{5.571884in}{1.204359in}}{\pgfqpoint{5.571884in}{1.211492in}}%
\pgfpathcurveto{\pgfqpoint{5.571884in}{1.218625in}}{\pgfqpoint{5.569050in}{1.225467in}}{\pgfqpoint{5.564006in}{1.230510in}}%
\pgfpathcurveto{\pgfqpoint{5.558963in}{1.235554in}}{\pgfqpoint{5.552121in}{1.238388in}}{\pgfqpoint{5.544988in}{1.238388in}}%
\pgfpathcurveto{\pgfqpoint{5.537855in}{1.238388in}}{\pgfqpoint{5.531014in}{1.235554in}}{\pgfqpoint{5.525970in}{1.230510in}}%
\pgfpathcurveto{\pgfqpoint{5.520926in}{1.225467in}}{\pgfqpoint{5.518093in}{1.218625in}}{\pgfqpoint{5.518093in}{1.211492in}}%
\pgfpathcurveto{\pgfqpoint{5.518093in}{1.204359in}}{\pgfqpoint{5.520926in}{1.197518in}}{\pgfqpoint{5.525970in}{1.192474in}}%
\pgfpathcurveto{\pgfqpoint{5.531014in}{1.187430in}}{\pgfqpoint{5.537855in}{1.184596in}}{\pgfqpoint{5.544988in}{1.184596in}}%
\pgfpathclose%
\pgfusepath{stroke,fill}%
\end{pgfscope}%
\begin{pgfscope}%
\pgfpathrectangle{\pgfqpoint{4.985294in}{0.500000in}}{\pgfqpoint{1.764706in}{1.700000in}}%
\pgfusepath{clip}%
\pgfsetbuttcap%
\pgfsetroundjoin%
\definecolor{currentfill}{rgb}{0.963559,0.632016,0.472047}%
\pgfsetfillcolor{currentfill}%
\pgfsetlinewidth{0.311001pt}%
\definecolor{currentstroke}{rgb}{1.000000,1.000000,1.000000}%
\pgfsetstrokecolor{currentstroke}%
\pgfsetdash{}{0pt}%
\pgfpathmoveto{\pgfqpoint{6.135429in}{1.082861in}}%
\pgfpathcurveto{\pgfqpoint{6.142562in}{1.082861in}}{\pgfqpoint{6.149404in}{1.085695in}}{\pgfqpoint{6.154448in}{1.090738in}}%
\pgfpathcurveto{\pgfqpoint{6.159491in}{1.095782in}}{\pgfqpoint{6.162325in}{1.102624in}}{\pgfqpoint{6.162325in}{1.109757in}}%
\pgfpathcurveto{\pgfqpoint{6.162325in}{1.116889in}}{\pgfqpoint{6.159491in}{1.123731in}}{\pgfqpoint{6.154448in}{1.128775in}}%
\pgfpathcurveto{\pgfqpoint{6.149404in}{1.133818in}}{\pgfqpoint{6.142562in}{1.136652in}}{\pgfqpoint{6.135429in}{1.136652in}}%
\pgfpathcurveto{\pgfqpoint{6.128297in}{1.136652in}}{\pgfqpoint{6.121455in}{1.133818in}}{\pgfqpoint{6.116411in}{1.128775in}}%
\pgfpathcurveto{\pgfqpoint{6.111368in}{1.123731in}}{\pgfqpoint{6.108534in}{1.116889in}}{\pgfqpoint{6.108534in}{1.109757in}}%
\pgfpathcurveto{\pgfqpoint{6.108534in}{1.102624in}}{\pgfqpoint{6.111368in}{1.095782in}}{\pgfqpoint{6.116411in}{1.090738in}}%
\pgfpathcurveto{\pgfqpoint{6.121455in}{1.085695in}}{\pgfqpoint{6.128297in}{1.082861in}}{\pgfqpoint{6.135429in}{1.082861in}}%
\pgfpathclose%
\pgfusepath{stroke,fill}%
\end{pgfscope}%
\begin{pgfscope}%
\pgfpathrectangle{\pgfqpoint{4.985294in}{0.500000in}}{\pgfqpoint{1.764706in}{1.700000in}}%
\pgfusepath{clip}%
\pgfsetbuttcap%
\pgfsetroundjoin%
\definecolor{currentfill}{rgb}{0.962018,0.586477,0.424918}%
\pgfsetfillcolor{currentfill}%
\pgfsetlinewidth{0.311001pt}%
\definecolor{currentstroke}{rgb}{1.000000,1.000000,1.000000}%
\pgfsetstrokecolor{currentstroke}%
\pgfsetdash{}{0pt}%
\pgfpathmoveto{\pgfqpoint{5.565631in}{1.768263in}}%
\pgfpathcurveto{\pgfqpoint{5.572764in}{1.768263in}}{\pgfqpoint{5.579606in}{1.771096in}}{\pgfqpoint{5.584649in}{1.776140in}}%
\pgfpathcurveto{\pgfqpoint{5.589693in}{1.781184in}}{\pgfqpoint{5.592527in}{1.788025in}}{\pgfqpoint{5.592527in}{1.795158in}}%
\pgfpathcurveto{\pgfqpoint{5.592527in}{1.802291in}}{\pgfqpoint{5.589693in}{1.809133in}}{\pgfqpoint{5.584649in}{1.814176in}}%
\pgfpathcurveto{\pgfqpoint{5.579606in}{1.819220in}}{\pgfqpoint{5.572764in}{1.822054in}}{\pgfqpoint{5.565631in}{1.822054in}}%
\pgfpathcurveto{\pgfqpoint{5.558498in}{1.822054in}}{\pgfqpoint{5.551657in}{1.819220in}}{\pgfqpoint{5.546613in}{1.814176in}}%
\pgfpathcurveto{\pgfqpoint{5.541569in}{1.809133in}}{\pgfqpoint{5.538736in}{1.802291in}}{\pgfqpoint{5.538736in}{1.795158in}}%
\pgfpathcurveto{\pgfqpoint{5.538736in}{1.788025in}}{\pgfqpoint{5.541569in}{1.781184in}}{\pgfqpoint{5.546613in}{1.776140in}}%
\pgfpathcurveto{\pgfqpoint{5.551657in}{1.771096in}}{\pgfqpoint{5.558498in}{1.768263in}}{\pgfqpoint{5.565631in}{1.768263in}}%
\pgfpathclose%
\pgfusepath{stroke,fill}%
\end{pgfscope}%
\begin{pgfscope}%
\pgfpathrectangle{\pgfqpoint{4.985294in}{0.500000in}}{\pgfqpoint{1.764706in}{1.700000in}}%
\pgfusepath{clip}%
\pgfsetbuttcap%
\pgfsetroundjoin%
\definecolor{currentfill}{rgb}{0.976287,0.879862,0.805788}%
\pgfsetfillcolor{currentfill}%
\pgfsetlinewidth{0.311001pt}%
\definecolor{currentstroke}{rgb}{1.000000,1.000000,1.000000}%
\pgfsetstrokecolor{currentstroke}%
\pgfsetdash{}{0pt}%
\pgfpathmoveto{\pgfqpoint{6.247461in}{1.095698in}}%
\pgfpathcurveto{\pgfqpoint{6.254593in}{1.095698in}}{\pgfqpoint{6.261435in}{1.098532in}}{\pgfqpoint{6.266479in}{1.103576in}}%
\pgfpathcurveto{\pgfqpoint{6.271522in}{1.108620in}}{\pgfqpoint{6.274356in}{1.115461in}}{\pgfqpoint{6.274356in}{1.122594in}}%
\pgfpathcurveto{\pgfqpoint{6.274356in}{1.129727in}}{\pgfqpoint{6.271522in}{1.136569in}}{\pgfqpoint{6.266479in}{1.141612in}}%
\pgfpathcurveto{\pgfqpoint{6.261435in}{1.146656in}}{\pgfqpoint{6.254593in}{1.149490in}}{\pgfqpoint{6.247461in}{1.149490in}}%
\pgfpathcurveto{\pgfqpoint{6.240328in}{1.149490in}}{\pgfqpoint{6.233486in}{1.146656in}}{\pgfqpoint{6.228442in}{1.141612in}}%
\pgfpathcurveto{\pgfqpoint{6.223399in}{1.136569in}}{\pgfqpoint{6.220565in}{1.129727in}}{\pgfqpoint{6.220565in}{1.122594in}}%
\pgfpathcurveto{\pgfqpoint{6.220565in}{1.115461in}}{\pgfqpoint{6.223399in}{1.108620in}}{\pgfqpoint{6.228442in}{1.103576in}}%
\pgfpathcurveto{\pgfqpoint{6.233486in}{1.098532in}}{\pgfqpoint{6.240328in}{1.095698in}}{\pgfqpoint{6.247461in}{1.095698in}}%
\pgfpathclose%
\pgfusepath{stroke,fill}%
\end{pgfscope}%
\begin{pgfscope}%
\pgfpathrectangle{\pgfqpoint{4.985294in}{0.500000in}}{\pgfqpoint{1.764706in}{1.700000in}}%
\pgfusepath{clip}%
\pgfsetbuttcap%
\pgfsetroundjoin%
\definecolor{currentfill}{rgb}{0.977657,0.891500,0.822809}%
\pgfsetfillcolor{currentfill}%
\pgfsetlinewidth{0.311001pt}%
\definecolor{currentstroke}{rgb}{1.000000,1.000000,1.000000}%
\pgfsetstrokecolor{currentstroke}%
\pgfsetdash{}{0pt}%
\pgfpathmoveto{\pgfqpoint{6.276254in}{1.125055in}}%
\pgfpathcurveto{\pgfqpoint{6.283387in}{1.125055in}}{\pgfqpoint{6.290228in}{1.127889in}}{\pgfqpoint{6.295272in}{1.132933in}}%
\pgfpathcurveto{\pgfqpoint{6.300316in}{1.137976in}}{\pgfqpoint{6.303149in}{1.144818in}}{\pgfqpoint{6.303149in}{1.151951in}}%
\pgfpathcurveto{\pgfqpoint{6.303149in}{1.159084in}}{\pgfqpoint{6.300316in}{1.165925in}}{\pgfqpoint{6.295272in}{1.170969in}}%
\pgfpathcurveto{\pgfqpoint{6.290228in}{1.176013in}}{\pgfqpoint{6.283387in}{1.178846in}}{\pgfqpoint{6.276254in}{1.178846in}}%
\pgfpathcurveto{\pgfqpoint{6.269121in}{1.178846in}}{\pgfqpoint{6.262279in}{1.176013in}}{\pgfqpoint{6.257236in}{1.170969in}}%
\pgfpathcurveto{\pgfqpoint{6.252192in}{1.165925in}}{\pgfqpoint{6.249358in}{1.159084in}}{\pgfqpoint{6.249358in}{1.151951in}}%
\pgfpathcurveto{\pgfqpoint{6.249358in}{1.144818in}}{\pgfqpoint{6.252192in}{1.137976in}}{\pgfqpoint{6.257236in}{1.132933in}}%
\pgfpathcurveto{\pgfqpoint{6.262279in}{1.127889in}}{\pgfqpoint{6.269121in}{1.125055in}}{\pgfqpoint{6.276254in}{1.125055in}}%
\pgfpathclose%
\pgfusepath{stroke,fill}%
\end{pgfscope}%
\begin{pgfscope}%
\pgfpathrectangle{\pgfqpoint{4.985294in}{0.500000in}}{\pgfqpoint{1.764706in}{1.700000in}}%
\pgfusepath{clip}%
\pgfsetbuttcap%
\pgfsetroundjoin%
\definecolor{currentfill}{rgb}{0.967092,0.768560,0.642079}%
\pgfsetfillcolor{currentfill}%
\pgfsetlinewidth{0.311001pt}%
\definecolor{currentstroke}{rgb}{1.000000,1.000000,1.000000}%
\pgfsetstrokecolor{currentstroke}%
\pgfsetdash{}{0pt}%
\pgfpathmoveto{\pgfqpoint{6.215922in}{1.330149in}}%
\pgfpathcurveto{\pgfqpoint{6.223055in}{1.330149in}}{\pgfqpoint{6.229897in}{1.332983in}}{\pgfqpoint{6.234940in}{1.338027in}}%
\pgfpathcurveto{\pgfqpoint{6.239984in}{1.343070in}}{\pgfqpoint{6.242818in}{1.349912in}}{\pgfqpoint{6.242818in}{1.357045in}}%
\pgfpathcurveto{\pgfqpoint{6.242818in}{1.364178in}}{\pgfqpoint{6.239984in}{1.371019in}}{\pgfqpoint{6.234940in}{1.376063in}}%
\pgfpathcurveto{\pgfqpoint{6.229897in}{1.381107in}}{\pgfqpoint{6.223055in}{1.383941in}}{\pgfqpoint{6.215922in}{1.383941in}}%
\pgfpathcurveto{\pgfqpoint{6.208790in}{1.383941in}}{\pgfqpoint{6.201948in}{1.381107in}}{\pgfqpoint{6.196904in}{1.376063in}}%
\pgfpathcurveto{\pgfqpoint{6.191861in}{1.371019in}}{\pgfqpoint{6.189027in}{1.364178in}}{\pgfqpoint{6.189027in}{1.357045in}}%
\pgfpathcurveto{\pgfqpoint{6.189027in}{1.349912in}}{\pgfqpoint{6.191861in}{1.343070in}}{\pgfqpoint{6.196904in}{1.338027in}}%
\pgfpathcurveto{\pgfqpoint{6.201948in}{1.332983in}}{\pgfqpoint{6.208790in}{1.330149in}}{\pgfqpoint{6.215922in}{1.330149in}}%
\pgfpathclose%
\pgfusepath{stroke,fill}%
\end{pgfscope}%
\begin{pgfscope}%
\pgfpathrectangle{\pgfqpoint{4.985294in}{0.500000in}}{\pgfqpoint{1.764706in}{1.700000in}}%
\pgfusepath{clip}%
\pgfsetbuttcap%
\pgfsetroundjoin%
\definecolor{currentfill}{rgb}{0.974412,0.862387,0.780156}%
\pgfsetfillcolor{currentfill}%
\pgfsetlinewidth{0.311001pt}%
\definecolor{currentstroke}{rgb}{1.000000,1.000000,1.000000}%
\pgfsetstrokecolor{currentstroke}%
\pgfsetdash{}{0pt}%
\pgfpathmoveto{\pgfqpoint{6.222910in}{1.087324in}}%
\pgfpathcurveto{\pgfqpoint{6.230043in}{1.087324in}}{\pgfqpoint{6.236885in}{1.090158in}}{\pgfqpoint{6.241928in}{1.095202in}}%
\pgfpathcurveto{\pgfqpoint{6.246972in}{1.100245in}}{\pgfqpoint{6.249806in}{1.107087in}}{\pgfqpoint{6.249806in}{1.114220in}}%
\pgfpathcurveto{\pgfqpoint{6.249806in}{1.121353in}}{\pgfqpoint{6.246972in}{1.128194in}}{\pgfqpoint{6.241928in}{1.133238in}}%
\pgfpathcurveto{\pgfqpoint{6.236885in}{1.138282in}}{\pgfqpoint{6.230043in}{1.141115in}}{\pgfqpoint{6.222910in}{1.141115in}}%
\pgfpathcurveto{\pgfqpoint{6.215777in}{1.141115in}}{\pgfqpoint{6.208936in}{1.138282in}}{\pgfqpoint{6.203892in}{1.133238in}}%
\pgfpathcurveto{\pgfqpoint{6.198848in}{1.128194in}}{\pgfqpoint{6.196014in}{1.121353in}}{\pgfqpoint{6.196014in}{1.114220in}}%
\pgfpathcurveto{\pgfqpoint{6.196014in}{1.107087in}}{\pgfqpoint{6.198848in}{1.100245in}}{\pgfqpoint{6.203892in}{1.095202in}}%
\pgfpathcurveto{\pgfqpoint{6.208936in}{1.090158in}}{\pgfqpoint{6.215777in}{1.087324in}}{\pgfqpoint{6.222910in}{1.087324in}}%
\pgfpathclose%
\pgfusepath{stroke,fill}%
\end{pgfscope}%
\begin{pgfscope}%
\pgfpathrectangle{\pgfqpoint{4.985294in}{0.500000in}}{\pgfqpoint{1.764706in}{1.700000in}}%
\pgfusepath{clip}%
\pgfsetbuttcap%
\pgfsetroundjoin%
\definecolor{currentfill}{rgb}{0.977657,0.891500,0.822809}%
\pgfsetfillcolor{currentfill}%
\pgfsetlinewidth{0.311001pt}%
\definecolor{currentstroke}{rgb}{1.000000,1.000000,1.000000}%
\pgfsetstrokecolor{currentstroke}%
\pgfsetdash{}{0pt}%
\pgfpathmoveto{\pgfqpoint{6.348352in}{1.375659in}}%
\pgfpathcurveto{\pgfqpoint{6.355485in}{1.375659in}}{\pgfqpoint{6.362326in}{1.378493in}}{\pgfqpoint{6.367370in}{1.383536in}}%
\pgfpathcurveto{\pgfqpoint{6.372414in}{1.388580in}}{\pgfqpoint{6.375248in}{1.395422in}}{\pgfqpoint{6.375248in}{1.402555in}}%
\pgfpathcurveto{\pgfqpoint{6.375248in}{1.409687in}}{\pgfqpoint{6.372414in}{1.416529in}}{\pgfqpoint{6.367370in}{1.421573in}}%
\pgfpathcurveto{\pgfqpoint{6.362326in}{1.426616in}}{\pgfqpoint{6.355485in}{1.429450in}}{\pgfqpoint{6.348352in}{1.429450in}}%
\pgfpathcurveto{\pgfqpoint{6.341219in}{1.429450in}}{\pgfqpoint{6.334378in}{1.426616in}}{\pgfqpoint{6.329334in}{1.421573in}}%
\pgfpathcurveto{\pgfqpoint{6.324290in}{1.416529in}}{\pgfqpoint{6.321456in}{1.409687in}}{\pgfqpoint{6.321456in}{1.402555in}}%
\pgfpathcurveto{\pgfqpoint{6.321456in}{1.395422in}}{\pgfqpoint{6.324290in}{1.388580in}}{\pgfqpoint{6.329334in}{1.383536in}}%
\pgfpathcurveto{\pgfqpoint{6.334378in}{1.378493in}}{\pgfqpoint{6.341219in}{1.375659in}}{\pgfqpoint{6.348352in}{1.375659in}}%
\pgfpathclose%
\pgfusepath{stroke,fill}%
\end{pgfscope}%
\begin{pgfscope}%
\pgfpathrectangle{\pgfqpoint{4.985294in}{0.500000in}}{\pgfqpoint{1.764706in}{1.700000in}}%
\pgfusepath{clip}%
\pgfsetbuttcap%
\pgfsetroundjoin%
\definecolor{currentfill}{rgb}{0.973832,0.856556,0.771584}%
\pgfsetfillcolor{currentfill}%
\pgfsetlinewidth{0.311001pt}%
\definecolor{currentstroke}{rgb}{1.000000,1.000000,1.000000}%
\pgfsetstrokecolor{currentstroke}%
\pgfsetdash{}{0pt}%
\pgfpathmoveto{\pgfqpoint{6.286636in}{1.077481in}}%
\pgfpathcurveto{\pgfqpoint{6.293769in}{1.077481in}}{\pgfqpoint{6.300611in}{1.080315in}}{\pgfqpoint{6.305655in}{1.085359in}}%
\pgfpathcurveto{\pgfqpoint{6.310698in}{1.090402in}}{\pgfqpoint{6.313532in}{1.097244in}}{\pgfqpoint{6.313532in}{1.104377in}}%
\pgfpathcurveto{\pgfqpoint{6.313532in}{1.111510in}}{\pgfqpoint{6.310698in}{1.118351in}}{\pgfqpoint{6.305655in}{1.123395in}}%
\pgfpathcurveto{\pgfqpoint{6.300611in}{1.128439in}}{\pgfqpoint{6.293769in}{1.131272in}}{\pgfqpoint{6.286636in}{1.131272in}}%
\pgfpathcurveto{\pgfqpoint{6.279504in}{1.131272in}}{\pgfqpoint{6.272662in}{1.128439in}}{\pgfqpoint{6.267618in}{1.123395in}}%
\pgfpathcurveto{\pgfqpoint{6.262575in}{1.118351in}}{\pgfqpoint{6.259741in}{1.111510in}}{\pgfqpoint{6.259741in}{1.104377in}}%
\pgfpathcurveto{\pgfqpoint{6.259741in}{1.097244in}}{\pgfqpoint{6.262575in}{1.090402in}}{\pgfqpoint{6.267618in}{1.085359in}}%
\pgfpathcurveto{\pgfqpoint{6.272662in}{1.080315in}}{\pgfqpoint{6.279504in}{1.077481in}}{\pgfqpoint{6.286636in}{1.077481in}}%
\pgfpathclose%
\pgfusepath{stroke,fill}%
\end{pgfscope}%
\begin{pgfscope}%
\pgfpathrectangle{\pgfqpoint{4.985294in}{0.500000in}}{\pgfqpoint{1.764706in}{1.700000in}}%
\pgfusepath{clip}%
\pgfsetbuttcap%
\pgfsetroundjoin%
\definecolor{currentfill}{rgb}{0.968509,0.792226,0.676405}%
\pgfsetfillcolor{currentfill}%
\pgfsetlinewidth{0.311001pt}%
\definecolor{currentstroke}{rgb}{1.000000,1.000000,1.000000}%
\pgfsetstrokecolor{currentstroke}%
\pgfsetdash{}{0pt}%
\pgfpathmoveto{\pgfqpoint{6.170064in}{1.031889in}}%
\pgfpathcurveto{\pgfqpoint{6.177197in}{1.031889in}}{\pgfqpoint{6.184038in}{1.034723in}}{\pgfqpoint{6.189082in}{1.039766in}}%
\pgfpathcurveto{\pgfqpoint{6.194126in}{1.044810in}}{\pgfqpoint{6.196960in}{1.051652in}}{\pgfqpoint{6.196960in}{1.058784in}}%
\pgfpathcurveto{\pgfqpoint{6.196960in}{1.065917in}}{\pgfqpoint{6.194126in}{1.072759in}}{\pgfqpoint{6.189082in}{1.077803in}}%
\pgfpathcurveto{\pgfqpoint{6.184038in}{1.082846in}}{\pgfqpoint{6.177197in}{1.085680in}}{\pgfqpoint{6.170064in}{1.085680in}}%
\pgfpathcurveto{\pgfqpoint{6.162931in}{1.085680in}}{\pgfqpoint{6.156089in}{1.082846in}}{\pgfqpoint{6.151046in}{1.077803in}}%
\pgfpathcurveto{\pgfqpoint{6.146002in}{1.072759in}}{\pgfqpoint{6.143168in}{1.065917in}}{\pgfqpoint{6.143168in}{1.058784in}}%
\pgfpathcurveto{\pgfqpoint{6.143168in}{1.051652in}}{\pgfqpoint{6.146002in}{1.044810in}}{\pgfqpoint{6.151046in}{1.039766in}}%
\pgfpathcurveto{\pgfqpoint{6.156089in}{1.034723in}}{\pgfqpoint{6.162931in}{1.031889in}}{\pgfqpoint{6.170064in}{1.031889in}}%
\pgfpathclose%
\pgfusepath{stroke,fill}%
\end{pgfscope}%
\begin{pgfscope}%
\pgfpathrectangle{\pgfqpoint{4.985294in}{0.500000in}}{\pgfqpoint{1.764706in}{1.700000in}}%
\pgfusepath{clip}%
\pgfsetbuttcap%
\pgfsetroundjoin%
\definecolor{currentfill}{rgb}{0.974412,0.862387,0.780156}%
\pgfsetfillcolor{currentfill}%
\pgfsetlinewidth{0.311001pt}%
\definecolor{currentstroke}{rgb}{1.000000,1.000000,1.000000}%
\pgfsetstrokecolor{currentstroke}%
\pgfsetdash{}{0pt}%
\pgfpathmoveto{\pgfqpoint{5.370446in}{1.351766in}}%
\pgfpathcurveto{\pgfqpoint{5.377579in}{1.351766in}}{\pgfqpoint{5.384420in}{1.354600in}}{\pgfqpoint{5.389464in}{1.359644in}}%
\pgfpathcurveto{\pgfqpoint{5.394508in}{1.364688in}}{\pgfqpoint{5.397342in}{1.371529in}}{\pgfqpoint{5.397342in}{1.378662in}}%
\pgfpathcurveto{\pgfqpoint{5.397342in}{1.385795in}}{\pgfqpoint{5.394508in}{1.392637in}}{\pgfqpoint{5.389464in}{1.397680in}}%
\pgfpathcurveto{\pgfqpoint{5.384420in}{1.402724in}}{\pgfqpoint{5.377579in}{1.405558in}}{\pgfqpoint{5.370446in}{1.405558in}}%
\pgfpathcurveto{\pgfqpoint{5.363313in}{1.405558in}}{\pgfqpoint{5.356471in}{1.402724in}}{\pgfqpoint{5.351428in}{1.397680in}}%
\pgfpathcurveto{\pgfqpoint{5.346384in}{1.392637in}}{\pgfqpoint{5.343550in}{1.385795in}}{\pgfqpoint{5.343550in}{1.378662in}}%
\pgfpathcurveto{\pgfqpoint{5.343550in}{1.371529in}}{\pgfqpoint{5.346384in}{1.364688in}}{\pgfqpoint{5.351428in}{1.359644in}}%
\pgfpathcurveto{\pgfqpoint{5.356471in}{1.354600in}}{\pgfqpoint{5.363313in}{1.351766in}}{\pgfqpoint{5.370446in}{1.351766in}}%
\pgfpathclose%
\pgfusepath{stroke,fill}%
\end{pgfscope}%
\begin{pgfscope}%
\pgfpathrectangle{\pgfqpoint{4.985294in}{0.500000in}}{\pgfqpoint{1.764706in}{1.700000in}}%
\pgfusepath{clip}%
\pgfsetbuttcap%
\pgfsetroundjoin%
\definecolor{currentfill}{rgb}{0.965169,0.707764,0.560659}%
\pgfsetfillcolor{currentfill}%
\pgfsetlinewidth{0.311001pt}%
\definecolor{currentstroke}{rgb}{1.000000,1.000000,1.000000}%
\pgfsetstrokecolor{currentstroke}%
\pgfsetdash{}{0pt}%
\pgfpathmoveto{\pgfqpoint{5.553873in}{1.712681in}}%
\pgfpathcurveto{\pgfqpoint{5.561006in}{1.712681in}}{\pgfqpoint{5.567847in}{1.715514in}}{\pgfqpoint{5.572891in}{1.720558in}}%
\pgfpathcurveto{\pgfqpoint{5.577935in}{1.725602in}}{\pgfqpoint{5.580769in}{1.732443in}}{\pgfqpoint{5.580769in}{1.739576in}}%
\pgfpathcurveto{\pgfqpoint{5.580769in}{1.746709in}}{\pgfqpoint{5.577935in}{1.753551in}}{\pgfqpoint{5.572891in}{1.758594in}}%
\pgfpathcurveto{\pgfqpoint{5.567847in}{1.763638in}}{\pgfqpoint{5.561006in}{1.766472in}}{\pgfqpoint{5.553873in}{1.766472in}}%
\pgfpathcurveto{\pgfqpoint{5.546740in}{1.766472in}}{\pgfqpoint{5.539898in}{1.763638in}}{\pgfqpoint{5.534855in}{1.758594in}}%
\pgfpathcurveto{\pgfqpoint{5.529811in}{1.753551in}}{\pgfqpoint{5.526977in}{1.746709in}}{\pgfqpoint{5.526977in}{1.739576in}}%
\pgfpathcurveto{\pgfqpoint{5.526977in}{1.732443in}}{\pgfqpoint{5.529811in}{1.725602in}}{\pgfqpoint{5.534855in}{1.720558in}}%
\pgfpathcurveto{\pgfqpoint{5.539898in}{1.715514in}}{\pgfqpoint{5.546740in}{1.712681in}}{\pgfqpoint{5.553873in}{1.712681in}}%
\pgfpathclose%
\pgfusepath{stroke,fill}%
\end{pgfscope}%
\begin{pgfscope}%
\pgfpathrectangle{\pgfqpoint{4.985294in}{0.500000in}}{\pgfqpoint{1.764706in}{1.700000in}}%
\pgfusepath{clip}%
\pgfsetbuttcap%
\pgfsetroundjoin%
\definecolor{currentfill}{rgb}{0.974412,0.862387,0.780156}%
\pgfsetfillcolor{currentfill}%
\pgfsetlinewidth{0.311001pt}%
\definecolor{currentstroke}{rgb}{1.000000,1.000000,1.000000}%
\pgfsetstrokecolor{currentstroke}%
\pgfsetdash{}{0pt}%
\pgfpathmoveto{\pgfqpoint{6.367349in}{1.301304in}}%
\pgfpathcurveto{\pgfqpoint{6.374482in}{1.301304in}}{\pgfqpoint{6.381323in}{1.304138in}}{\pgfqpoint{6.386367in}{1.309182in}}%
\pgfpathcurveto{\pgfqpoint{6.391411in}{1.314225in}}{\pgfqpoint{6.394245in}{1.321067in}}{\pgfqpoint{6.394245in}{1.328200in}}%
\pgfpathcurveto{\pgfqpoint{6.394245in}{1.335332in}}{\pgfqpoint{6.391411in}{1.342174in}}{\pgfqpoint{6.386367in}{1.347218in}}%
\pgfpathcurveto{\pgfqpoint{6.381323in}{1.352261in}}{\pgfqpoint{6.374482in}{1.355095in}}{\pgfqpoint{6.367349in}{1.355095in}}%
\pgfpathcurveto{\pgfqpoint{6.360216in}{1.355095in}}{\pgfqpoint{6.353374in}{1.352261in}}{\pgfqpoint{6.348331in}{1.347218in}}%
\pgfpathcurveto{\pgfqpoint{6.343287in}{1.342174in}}{\pgfqpoint{6.340453in}{1.335332in}}{\pgfqpoint{6.340453in}{1.328200in}}%
\pgfpathcurveto{\pgfqpoint{6.340453in}{1.321067in}}{\pgfqpoint{6.343287in}{1.314225in}}{\pgfqpoint{6.348331in}{1.309182in}}%
\pgfpathcurveto{\pgfqpoint{6.353374in}{1.304138in}}{\pgfqpoint{6.360216in}{1.301304in}}{\pgfqpoint{6.367349in}{1.301304in}}%
\pgfpathclose%
\pgfusepath{stroke,fill}%
\end{pgfscope}%
\begin{pgfscope}%
\pgfpathrectangle{\pgfqpoint{4.985294in}{0.500000in}}{\pgfqpoint{1.764706in}{1.700000in}}%
\pgfusepath{clip}%
\pgfsetbuttcap%
\pgfsetroundjoin%
\definecolor{currentfill}{rgb}{0.965302,0.713942,0.568499}%
\pgfsetfillcolor{currentfill}%
\pgfsetlinewidth{0.311001pt}%
\definecolor{currentstroke}{rgb}{1.000000,1.000000,1.000000}%
\pgfsetstrokecolor{currentstroke}%
\pgfsetdash{}{0pt}%
\pgfpathmoveto{\pgfqpoint{6.199761in}{1.360594in}}%
\pgfpathcurveto{\pgfqpoint{6.206894in}{1.360594in}}{\pgfqpoint{6.213735in}{1.363428in}}{\pgfqpoint{6.218779in}{1.368471in}}%
\pgfpathcurveto{\pgfqpoint{6.223823in}{1.373515in}}{\pgfqpoint{6.226657in}{1.380357in}}{\pgfqpoint{6.226657in}{1.387489in}}%
\pgfpathcurveto{\pgfqpoint{6.226657in}{1.394622in}}{\pgfqpoint{6.223823in}{1.401464in}}{\pgfqpoint{6.218779in}{1.406508in}}%
\pgfpathcurveto{\pgfqpoint{6.213735in}{1.411551in}}{\pgfqpoint{6.206894in}{1.414385in}}{\pgfqpoint{6.199761in}{1.414385in}}%
\pgfpathcurveto{\pgfqpoint{6.192628in}{1.414385in}}{\pgfqpoint{6.185787in}{1.411551in}}{\pgfqpoint{6.180743in}{1.406508in}}%
\pgfpathcurveto{\pgfqpoint{6.175699in}{1.401464in}}{\pgfqpoint{6.172865in}{1.394622in}}{\pgfqpoint{6.172865in}{1.387489in}}%
\pgfpathcurveto{\pgfqpoint{6.172865in}{1.380357in}}{\pgfqpoint{6.175699in}{1.373515in}}{\pgfqpoint{6.180743in}{1.368471in}}%
\pgfpathcurveto{\pgfqpoint{6.185787in}{1.363428in}}{\pgfqpoint{6.192628in}{1.360594in}}{\pgfqpoint{6.199761in}{1.360594in}}%
\pgfpathclose%
\pgfusepath{stroke,fill}%
\end{pgfscope}%
\begin{pgfscope}%
\pgfpathrectangle{\pgfqpoint{4.985294in}{0.500000in}}{\pgfqpoint{1.764706in}{1.700000in}}%
\pgfusepath{clip}%
\pgfsetbuttcap%
\pgfsetroundjoin%
\definecolor{currentfill}{rgb}{0.975018,0.868213,0.788710}%
\pgfsetfillcolor{currentfill}%
\pgfsetlinewidth{0.311001pt}%
\definecolor{currentstroke}{rgb}{1.000000,1.000000,1.000000}%
\pgfsetstrokecolor{currentstroke}%
\pgfsetdash{}{0pt}%
\pgfpathmoveto{\pgfqpoint{5.470025in}{1.553845in}}%
\pgfpathcurveto{\pgfqpoint{5.477158in}{1.553845in}}{\pgfqpoint{5.484000in}{1.556679in}}{\pgfqpoint{5.489043in}{1.561723in}}%
\pgfpathcurveto{\pgfqpoint{5.494087in}{1.566766in}}{\pgfqpoint{5.496921in}{1.573608in}}{\pgfqpoint{5.496921in}{1.580741in}}%
\pgfpathcurveto{\pgfqpoint{5.496921in}{1.587874in}}{\pgfqpoint{5.494087in}{1.594715in}}{\pgfqpoint{5.489043in}{1.599759in}}%
\pgfpathcurveto{\pgfqpoint{5.484000in}{1.604803in}}{\pgfqpoint{5.477158in}{1.607637in}}{\pgfqpoint{5.470025in}{1.607637in}}%
\pgfpathcurveto{\pgfqpoint{5.462892in}{1.607637in}}{\pgfqpoint{5.456051in}{1.604803in}}{\pgfqpoint{5.451007in}{1.599759in}}%
\pgfpathcurveto{\pgfqpoint{5.445964in}{1.594715in}}{\pgfqpoint{5.443130in}{1.587874in}}{\pgfqpoint{5.443130in}{1.580741in}}%
\pgfpathcurveto{\pgfqpoint{5.443130in}{1.573608in}}{\pgfqpoint{5.445964in}{1.566766in}}{\pgfqpoint{5.451007in}{1.561723in}}%
\pgfpathcurveto{\pgfqpoint{5.456051in}{1.556679in}}{\pgfqpoint{5.462892in}{1.553845in}}{\pgfqpoint{5.470025in}{1.553845in}}%
\pgfpathclose%
\pgfusepath{stroke,fill}%
\end{pgfscope}%
\begin{pgfscope}%
\pgfpathrectangle{\pgfqpoint{4.985294in}{0.500000in}}{\pgfqpoint{1.764706in}{1.700000in}}%
\pgfusepath{clip}%
\pgfsetbuttcap%
\pgfsetroundjoin%
\definecolor{currentfill}{rgb}{0.968105,0.786346,0.667739}%
\pgfsetfillcolor{currentfill}%
\pgfsetlinewidth{0.311001pt}%
\definecolor{currentstroke}{rgb}{1.000000,1.000000,1.000000}%
\pgfsetstrokecolor{currentstroke}%
\pgfsetdash{}{0pt}%
\pgfpathmoveto{\pgfqpoint{6.254677in}{0.987933in}}%
\pgfpathcurveto{\pgfqpoint{6.261809in}{0.987933in}}{\pgfqpoint{6.268651in}{0.990766in}}{\pgfqpoint{6.273695in}{0.995810in}}%
\pgfpathcurveto{\pgfqpoint{6.278738in}{1.000854in}}{\pgfqpoint{6.281572in}{1.007695in}}{\pgfqpoint{6.281572in}{1.014828in}}%
\pgfpathcurveto{\pgfqpoint{6.281572in}{1.021961in}}{\pgfqpoint{6.278738in}{1.028803in}}{\pgfqpoint{6.273695in}{1.033846in}}%
\pgfpathcurveto{\pgfqpoint{6.268651in}{1.038890in}}{\pgfqpoint{6.261809in}{1.041724in}}{\pgfqpoint{6.254677in}{1.041724in}}%
\pgfpathcurveto{\pgfqpoint{6.247544in}{1.041724in}}{\pgfqpoint{6.240702in}{1.038890in}}{\pgfqpoint{6.235659in}{1.033846in}}%
\pgfpathcurveto{\pgfqpoint{6.230615in}{1.028803in}}{\pgfqpoint{6.227781in}{1.021961in}}{\pgfqpoint{6.227781in}{1.014828in}}%
\pgfpathcurveto{\pgfqpoint{6.227781in}{1.007695in}}{\pgfqpoint{6.230615in}{1.000854in}}{\pgfqpoint{6.235659in}{0.995810in}}%
\pgfpathcurveto{\pgfqpoint{6.240702in}{0.990766in}}{\pgfqpoint{6.247544in}{0.987933in}}{\pgfqpoint{6.254677in}{0.987933in}}%
\pgfpathclose%
\pgfusepath{stroke,fill}%
\end{pgfscope}%
\begin{pgfscope}%
\pgfpathrectangle{\pgfqpoint{4.985294in}{0.500000in}}{\pgfqpoint{1.764706in}{1.700000in}}%
\pgfusepath{clip}%
\pgfsetbuttcap%
\pgfsetroundjoin%
\definecolor{currentfill}{rgb}{0.969803,0.809811,0.702523}%
\pgfsetfillcolor{currentfill}%
\pgfsetlinewidth{0.311001pt}%
\definecolor{currentstroke}{rgb}{1.000000,1.000000,1.000000}%
\pgfsetstrokecolor{currentstroke}%
\pgfsetdash{}{0pt}%
\pgfpathmoveto{\pgfqpoint{6.195820in}{1.127613in}}%
\pgfpathcurveto{\pgfqpoint{6.202952in}{1.127613in}}{\pgfqpoint{6.209794in}{1.130447in}}{\pgfqpoint{6.214838in}{1.135491in}}%
\pgfpathcurveto{\pgfqpoint{6.219881in}{1.140535in}}{\pgfqpoint{6.222715in}{1.147376in}}{\pgfqpoint{6.222715in}{1.154509in}}%
\pgfpathcurveto{\pgfqpoint{6.222715in}{1.161642in}}{\pgfqpoint{6.219881in}{1.168484in}}{\pgfqpoint{6.214838in}{1.173527in}}%
\pgfpathcurveto{\pgfqpoint{6.209794in}{1.178571in}}{\pgfqpoint{6.202952in}{1.181405in}}{\pgfqpoint{6.195820in}{1.181405in}}%
\pgfpathcurveto{\pgfqpoint{6.188687in}{1.181405in}}{\pgfqpoint{6.181845in}{1.178571in}}{\pgfqpoint{6.176801in}{1.173527in}}%
\pgfpathcurveto{\pgfqpoint{6.171758in}{1.168484in}}{\pgfqpoint{6.168924in}{1.161642in}}{\pgfqpoint{6.168924in}{1.154509in}}%
\pgfpathcurveto{\pgfqpoint{6.168924in}{1.147376in}}{\pgfqpoint{6.171758in}{1.140535in}}{\pgfqpoint{6.176801in}{1.135491in}}%
\pgfpathcurveto{\pgfqpoint{6.181845in}{1.130447in}}{\pgfqpoint{6.188687in}{1.127613in}}{\pgfqpoint{6.195820in}{1.127613in}}%
\pgfpathclose%
\pgfusepath{stroke,fill}%
\end{pgfscope}%
\begin{pgfscope}%
\pgfpathrectangle{\pgfqpoint{4.985294in}{0.500000in}}{\pgfqpoint{1.764706in}{1.700000in}}%
\pgfusepath{clip}%
\pgfsetbuttcap%
\pgfsetroundjoin%
\definecolor{currentfill}{rgb}{0.962532,0.599594,0.438051}%
\pgfsetfillcolor{currentfill}%
\pgfsetlinewidth{0.311001pt}%
\definecolor{currentstroke}{rgb}{1.000000,1.000000,1.000000}%
\pgfsetstrokecolor{currentstroke}%
\pgfsetdash{}{0pt}%
\pgfpathmoveto{\pgfqpoint{5.627778in}{1.645232in}}%
\pgfpathcurveto{\pgfqpoint{5.634910in}{1.645232in}}{\pgfqpoint{5.641752in}{1.648066in}}{\pgfqpoint{5.646796in}{1.653109in}}%
\pgfpathcurveto{\pgfqpoint{5.651839in}{1.658153in}}{\pgfqpoint{5.654673in}{1.664995in}}{\pgfqpoint{5.654673in}{1.672128in}}%
\pgfpathcurveto{\pgfqpoint{5.654673in}{1.679260in}}{\pgfqpoint{5.651839in}{1.686102in}}{\pgfqpoint{5.646796in}{1.691146in}}%
\pgfpathcurveto{\pgfqpoint{5.641752in}{1.696189in}}{\pgfqpoint{5.634910in}{1.699023in}}{\pgfqpoint{5.627778in}{1.699023in}}%
\pgfpathcurveto{\pgfqpoint{5.620645in}{1.699023in}}{\pgfqpoint{5.613803in}{1.696189in}}{\pgfqpoint{5.608760in}{1.691146in}}%
\pgfpathcurveto{\pgfqpoint{5.603716in}{1.686102in}}{\pgfqpoint{5.600882in}{1.679260in}}{\pgfqpoint{5.600882in}{1.672128in}}%
\pgfpathcurveto{\pgfqpoint{5.600882in}{1.664995in}}{\pgfqpoint{5.603716in}{1.658153in}}{\pgfqpoint{5.608760in}{1.653109in}}%
\pgfpathcurveto{\pgfqpoint{5.613803in}{1.648066in}}{\pgfqpoint{5.620645in}{1.645232in}}{\pgfqpoint{5.627778in}{1.645232in}}%
\pgfpathclose%
\pgfusepath{stroke,fill}%
\end{pgfscope}%
\begin{pgfscope}%
\pgfpathrectangle{\pgfqpoint{4.985294in}{0.500000in}}{\pgfqpoint{1.764706in}{1.700000in}}%
\pgfusepath{clip}%
\pgfsetbuttcap%
\pgfsetroundjoin%
\definecolor{currentfill}{rgb}{0.967092,0.768560,0.642079}%
\pgfsetfillcolor{currentfill}%
\pgfsetlinewidth{0.311001pt}%
\definecolor{currentstroke}{rgb}{1.000000,1.000000,1.000000}%
\pgfsetstrokecolor{currentstroke}%
\pgfsetdash{}{0pt}%
\pgfpathmoveto{\pgfqpoint{6.396744in}{1.267677in}}%
\pgfpathcurveto{\pgfqpoint{6.403877in}{1.267677in}}{\pgfqpoint{6.410719in}{1.270511in}}{\pgfqpoint{6.415762in}{1.275555in}}%
\pgfpathcurveto{\pgfqpoint{6.420806in}{1.280598in}}{\pgfqpoint{6.423640in}{1.287440in}}{\pgfqpoint{6.423640in}{1.294573in}}%
\pgfpathcurveto{\pgfqpoint{6.423640in}{1.301706in}}{\pgfqpoint{6.420806in}{1.308547in}}{\pgfqpoint{6.415762in}{1.313591in}}%
\pgfpathcurveto{\pgfqpoint{6.410719in}{1.318635in}}{\pgfqpoint{6.403877in}{1.321469in}}{\pgfqpoint{6.396744in}{1.321469in}}%
\pgfpathcurveto{\pgfqpoint{6.389611in}{1.321469in}}{\pgfqpoint{6.382770in}{1.318635in}}{\pgfqpoint{6.377726in}{1.313591in}}%
\pgfpathcurveto{\pgfqpoint{6.372682in}{1.308547in}}{\pgfqpoint{6.369848in}{1.301706in}}{\pgfqpoint{6.369848in}{1.294573in}}%
\pgfpathcurveto{\pgfqpoint{6.369848in}{1.287440in}}{\pgfqpoint{6.372682in}{1.280598in}}{\pgfqpoint{6.377726in}{1.275555in}}%
\pgfpathcurveto{\pgfqpoint{6.382770in}{1.270511in}}{\pgfqpoint{6.389611in}{1.267677in}}{\pgfqpoint{6.396744in}{1.267677in}}%
\pgfpathclose%
\pgfusepath{stroke,fill}%
\end{pgfscope}%
\begin{pgfscope}%
\pgfpathrectangle{\pgfqpoint{4.985294in}{0.500000in}}{\pgfqpoint{1.764706in}{1.700000in}}%
\pgfusepath{clip}%
\pgfsetbuttcap%
\pgfsetroundjoin%
\definecolor{currentfill}{rgb}{0.919781,0.275262,0.242460}%
\pgfsetfillcolor{currentfill}%
\pgfsetlinewidth{0.311001pt}%
\definecolor{currentstroke}{rgb}{1.000000,1.000000,1.000000}%
\pgfsetstrokecolor{currentstroke}%
\pgfsetdash{}{0pt}%
\pgfpathmoveto{\pgfqpoint{5.601269in}{1.835141in}}%
\pgfpathcurveto{\pgfqpoint{5.608402in}{1.835141in}}{\pgfqpoint{5.615244in}{1.837975in}}{\pgfqpoint{5.620287in}{1.843018in}}%
\pgfpathcurveto{\pgfqpoint{5.625331in}{1.848062in}}{\pgfqpoint{5.628165in}{1.854904in}}{\pgfqpoint{5.628165in}{1.862036in}}%
\pgfpathcurveto{\pgfqpoint{5.628165in}{1.869169in}}{\pgfqpoint{5.625331in}{1.876011in}}{\pgfqpoint{5.620287in}{1.881055in}}%
\pgfpathcurveto{\pgfqpoint{5.615244in}{1.886098in}}{\pgfqpoint{5.608402in}{1.888932in}}{\pgfqpoint{5.601269in}{1.888932in}}%
\pgfpathcurveto{\pgfqpoint{5.594136in}{1.888932in}}{\pgfqpoint{5.587295in}{1.886098in}}{\pgfqpoint{5.582251in}{1.881055in}}%
\pgfpathcurveto{\pgfqpoint{5.577207in}{1.876011in}}{\pgfqpoint{5.574373in}{1.869169in}}{\pgfqpoint{5.574373in}{1.862036in}}%
\pgfpathcurveto{\pgfqpoint{5.574373in}{1.854904in}}{\pgfqpoint{5.577207in}{1.848062in}}{\pgfqpoint{5.582251in}{1.843018in}}%
\pgfpathcurveto{\pgfqpoint{5.587295in}{1.837975in}}{\pgfqpoint{5.594136in}{1.835141in}}{\pgfqpoint{5.601269in}{1.835141in}}%
\pgfpathclose%
\pgfusepath{stroke,fill}%
\end{pgfscope}%
\begin{pgfscope}%
\pgfpathrectangle{\pgfqpoint{4.985294in}{0.500000in}}{\pgfqpoint{1.764706in}{1.700000in}}%
\pgfusepath{clip}%
\pgfsetbuttcap%
\pgfsetroundjoin%
\definecolor{currentfill}{rgb}{0.968105,0.786346,0.667739}%
\pgfsetfillcolor{currentfill}%
\pgfsetlinewidth{0.311001pt}%
\definecolor{currentstroke}{rgb}{1.000000,1.000000,1.000000}%
\pgfsetstrokecolor{currentstroke}%
\pgfsetdash{}{0pt}%
\pgfpathmoveto{\pgfqpoint{5.406904in}{1.051144in}}%
\pgfpathcurveto{\pgfqpoint{5.414036in}{1.051144in}}{\pgfqpoint{5.420878in}{1.053978in}}{\pgfqpoint{5.425922in}{1.059021in}}%
\pgfpathcurveto{\pgfqpoint{5.430965in}{1.064065in}}{\pgfqpoint{5.433799in}{1.070906in}}{\pgfqpoint{5.433799in}{1.078039in}}%
\pgfpathcurveto{\pgfqpoint{5.433799in}{1.085172in}}{\pgfqpoint{5.430965in}{1.092014in}}{\pgfqpoint{5.425922in}{1.097057in}}%
\pgfpathcurveto{\pgfqpoint{5.420878in}{1.102101in}}{\pgfqpoint{5.414036in}{1.104935in}}{\pgfqpoint{5.406904in}{1.104935in}}%
\pgfpathcurveto{\pgfqpoint{5.399771in}{1.104935in}}{\pgfqpoint{5.392929in}{1.102101in}}{\pgfqpoint{5.387885in}{1.097057in}}%
\pgfpathcurveto{\pgfqpoint{5.382842in}{1.092014in}}{\pgfqpoint{5.380008in}{1.085172in}}{\pgfqpoint{5.380008in}{1.078039in}}%
\pgfpathcurveto{\pgfqpoint{5.380008in}{1.070906in}}{\pgfqpoint{5.382842in}{1.064065in}}{\pgfqpoint{5.387885in}{1.059021in}}%
\pgfpathcurveto{\pgfqpoint{5.392929in}{1.053978in}}{\pgfqpoint{5.399771in}{1.051144in}}{\pgfqpoint{5.406904in}{1.051144in}}%
\pgfpathclose%
\pgfusepath{stroke,fill}%
\end{pgfscope}%
\begin{pgfscope}%
\pgfpathrectangle{\pgfqpoint{4.985294in}{0.500000in}}{\pgfqpoint{1.764706in}{1.700000in}}%
\pgfusepath{clip}%
\pgfsetbuttcap%
\pgfsetroundjoin%
\definecolor{currentfill}{rgb}{0.962018,0.586477,0.424918}%
\pgfsetfillcolor{currentfill}%
\pgfsetlinewidth{0.311001pt}%
\definecolor{currentstroke}{rgb}{1.000000,1.000000,1.000000}%
\pgfsetstrokecolor{currentstroke}%
\pgfsetdash{}{0pt}%
\pgfpathmoveto{\pgfqpoint{5.299513in}{1.322782in}}%
\pgfpathcurveto{\pgfqpoint{5.306646in}{1.322782in}}{\pgfqpoint{5.313488in}{1.325616in}}{\pgfqpoint{5.318531in}{1.330660in}}%
\pgfpathcurveto{\pgfqpoint{5.323575in}{1.335704in}}{\pgfqpoint{5.326409in}{1.342545in}}{\pgfqpoint{5.326409in}{1.349678in}}%
\pgfpathcurveto{\pgfqpoint{5.326409in}{1.356811in}}{\pgfqpoint{5.323575in}{1.363652in}}{\pgfqpoint{5.318531in}{1.368696in}}%
\pgfpathcurveto{\pgfqpoint{5.313488in}{1.373740in}}{\pgfqpoint{5.306646in}{1.376574in}}{\pgfqpoint{5.299513in}{1.376574in}}%
\pgfpathcurveto{\pgfqpoint{5.292380in}{1.376574in}}{\pgfqpoint{5.285539in}{1.373740in}}{\pgfqpoint{5.280495in}{1.368696in}}%
\pgfpathcurveto{\pgfqpoint{5.275452in}{1.363652in}}{\pgfqpoint{5.272618in}{1.356811in}}{\pgfqpoint{5.272618in}{1.349678in}}%
\pgfpathcurveto{\pgfqpoint{5.272618in}{1.342545in}}{\pgfqpoint{5.275452in}{1.335704in}}{\pgfqpoint{5.280495in}{1.330660in}}%
\pgfpathcurveto{\pgfqpoint{5.285539in}{1.325616in}}{\pgfqpoint{5.292380in}{1.322782in}}{\pgfqpoint{5.299513in}{1.322782in}}%
\pgfpathclose%
\pgfusepath{stroke,fill}%
\end{pgfscope}%
\begin{pgfscope}%
\pgfpathrectangle{\pgfqpoint{4.985294in}{0.500000in}}{\pgfqpoint{1.764706in}{1.700000in}}%
\pgfusepath{clip}%
\pgfsetbuttcap%
\pgfsetroundjoin%
\definecolor{currentfill}{rgb}{0.972201,0.839051,0.745789}%
\pgfsetfillcolor{currentfill}%
\pgfsetlinewidth{0.311001pt}%
\definecolor{currentstroke}{rgb}{1.000000,1.000000,1.000000}%
\pgfsetstrokecolor{currentstroke}%
\pgfsetdash{}{0pt}%
\pgfpathmoveto{\pgfqpoint{6.281699in}{1.627457in}}%
\pgfpathcurveto{\pgfqpoint{6.288832in}{1.627457in}}{\pgfqpoint{6.295674in}{1.630291in}}{\pgfqpoint{6.300718in}{1.635335in}}%
\pgfpathcurveto{\pgfqpoint{6.305761in}{1.640378in}}{\pgfqpoint{6.308595in}{1.647220in}}{\pgfqpoint{6.308595in}{1.654353in}}%
\pgfpathcurveto{\pgfqpoint{6.308595in}{1.661486in}}{\pgfqpoint{6.305761in}{1.668327in}}{\pgfqpoint{6.300718in}{1.673371in}}%
\pgfpathcurveto{\pgfqpoint{6.295674in}{1.678415in}}{\pgfqpoint{6.288832in}{1.681249in}}{\pgfqpoint{6.281699in}{1.681249in}}%
\pgfpathcurveto{\pgfqpoint{6.274567in}{1.681249in}}{\pgfqpoint{6.267725in}{1.678415in}}{\pgfqpoint{6.262681in}{1.673371in}}%
\pgfpathcurveto{\pgfqpoint{6.257638in}{1.668327in}}{\pgfqpoint{6.254804in}{1.661486in}}{\pgfqpoint{6.254804in}{1.654353in}}%
\pgfpathcurveto{\pgfqpoint{6.254804in}{1.647220in}}{\pgfqpoint{6.257638in}{1.640378in}}{\pgfqpoint{6.262681in}{1.635335in}}%
\pgfpathcurveto{\pgfqpoint{6.267725in}{1.630291in}}{\pgfqpoint{6.274567in}{1.627457in}}{\pgfqpoint{6.281699in}{1.627457in}}%
\pgfpathclose%
\pgfusepath{stroke,fill}%
\end{pgfscope}%
\begin{pgfscope}%
\pgfpathrectangle{\pgfqpoint{4.985294in}{0.500000in}}{\pgfqpoint{1.764706in}{1.700000in}}%
\pgfusepath{clip}%
\pgfsetbuttcap%
\pgfsetroundjoin%
\definecolor{currentfill}{rgb}{0.980678,0.914765,0.856766}%
\pgfsetfillcolor{currentfill}%
\pgfsetlinewidth{0.311001pt}%
\definecolor{currentstroke}{rgb}{1.000000,1.000000,1.000000}%
\pgfsetstrokecolor{currentstroke}%
\pgfsetdash{}{0pt}%
\pgfpathmoveto{\pgfqpoint{6.290019in}{1.457543in}}%
\pgfpathcurveto{\pgfqpoint{6.297152in}{1.457543in}}{\pgfqpoint{6.303993in}{1.460377in}}{\pgfqpoint{6.309037in}{1.465421in}}%
\pgfpathcurveto{\pgfqpoint{6.314081in}{1.470465in}}{\pgfqpoint{6.316914in}{1.477306in}}{\pgfqpoint{6.316914in}{1.484439in}}%
\pgfpathcurveto{\pgfqpoint{6.316914in}{1.491572in}}{\pgfqpoint{6.314081in}{1.498414in}}{\pgfqpoint{6.309037in}{1.503457in}}%
\pgfpathcurveto{\pgfqpoint{6.303993in}{1.508501in}}{\pgfqpoint{6.297152in}{1.511335in}}{\pgfqpoint{6.290019in}{1.511335in}}%
\pgfpathcurveto{\pgfqpoint{6.282886in}{1.511335in}}{\pgfqpoint{6.276044in}{1.508501in}}{\pgfqpoint{6.271001in}{1.503457in}}%
\pgfpathcurveto{\pgfqpoint{6.265957in}{1.498414in}}{\pgfqpoint{6.263123in}{1.491572in}}{\pgfqpoint{6.263123in}{1.484439in}}%
\pgfpathcurveto{\pgfqpoint{6.263123in}{1.477306in}}{\pgfqpoint{6.265957in}{1.470465in}}{\pgfqpoint{6.271001in}{1.465421in}}%
\pgfpathcurveto{\pgfqpoint{6.276044in}{1.460377in}}{\pgfqpoint{6.282886in}{1.457543in}}{\pgfqpoint{6.290019in}{1.457543in}}%
\pgfpathclose%
\pgfusepath{stroke,fill}%
\end{pgfscope}%
\begin{pgfscope}%
\pgfpathrectangle{\pgfqpoint{4.985294in}{0.500000in}}{\pgfqpoint{1.764706in}{1.700000in}}%
\pgfusepath{clip}%
\pgfsetbuttcap%
\pgfsetroundjoin%
\definecolor{currentfill}{rgb}{0.979891,0.908948,0.848279}%
\pgfsetfillcolor{currentfill}%
\pgfsetlinewidth{0.311001pt}%
\definecolor{currentstroke}{rgb}{1.000000,1.000000,1.000000}%
\pgfsetstrokecolor{currentstroke}%
\pgfsetdash{}{0pt}%
\pgfpathmoveto{\pgfqpoint{6.329402in}{1.248208in}}%
\pgfpathcurveto{\pgfqpoint{6.336534in}{1.248208in}}{\pgfqpoint{6.343376in}{1.251042in}}{\pgfqpoint{6.348420in}{1.256086in}}%
\pgfpathcurveto{\pgfqpoint{6.353463in}{1.261130in}}{\pgfqpoint{6.356297in}{1.267971in}}{\pgfqpoint{6.356297in}{1.275104in}}%
\pgfpathcurveto{\pgfqpoint{6.356297in}{1.282237in}}{\pgfqpoint{6.353463in}{1.289078in}}{\pgfqpoint{6.348420in}{1.294122in}}%
\pgfpathcurveto{\pgfqpoint{6.343376in}{1.299166in}}{\pgfqpoint{6.336534in}{1.302000in}}{\pgfqpoint{6.329402in}{1.302000in}}%
\pgfpathcurveto{\pgfqpoint{6.322269in}{1.302000in}}{\pgfqpoint{6.315427in}{1.299166in}}{\pgfqpoint{6.310384in}{1.294122in}}%
\pgfpathcurveto{\pgfqpoint{6.305340in}{1.289078in}}{\pgfqpoint{6.302506in}{1.282237in}}{\pgfqpoint{6.302506in}{1.275104in}}%
\pgfpathcurveto{\pgfqpoint{6.302506in}{1.267971in}}{\pgfqpoint{6.305340in}{1.261130in}}{\pgfqpoint{6.310384in}{1.256086in}}%
\pgfpathcurveto{\pgfqpoint{6.315427in}{1.251042in}}{\pgfqpoint{6.322269in}{1.248208in}}{\pgfqpoint{6.329402in}{1.248208in}}%
\pgfpathclose%
\pgfusepath{stroke,fill}%
\end{pgfscope}%
\begin{pgfscope}%
\pgfpathrectangle{\pgfqpoint{4.985294in}{0.500000in}}{\pgfqpoint{1.764706in}{1.700000in}}%
\pgfusepath{clip}%
\pgfsetbuttcap%
\pgfsetroundjoin%
\definecolor{currentfill}{rgb}{0.964920,0.695342,0.545192}%
\pgfsetfillcolor{currentfill}%
\pgfsetlinewidth{0.311001pt}%
\definecolor{currentstroke}{rgb}{1.000000,1.000000,1.000000}%
\pgfsetstrokecolor{currentstroke}%
\pgfsetdash{}{0pt}%
\pgfpathmoveto{\pgfqpoint{6.129132in}{0.981923in}}%
\pgfpathcurveto{\pgfqpoint{6.136265in}{0.981923in}}{\pgfqpoint{6.143106in}{0.984757in}}{\pgfqpoint{6.148150in}{0.989800in}}%
\pgfpathcurveto{\pgfqpoint{6.153194in}{0.994844in}}{\pgfqpoint{6.156028in}{1.001686in}}{\pgfqpoint{6.156028in}{1.008818in}}%
\pgfpathcurveto{\pgfqpoint{6.156028in}{1.015951in}}{\pgfqpoint{6.153194in}{1.022793in}}{\pgfqpoint{6.148150in}{1.027837in}}%
\pgfpathcurveto{\pgfqpoint{6.143106in}{1.032880in}}{\pgfqpoint{6.136265in}{1.035714in}}{\pgfqpoint{6.129132in}{1.035714in}}%
\pgfpathcurveto{\pgfqpoint{6.121999in}{1.035714in}}{\pgfqpoint{6.115157in}{1.032880in}}{\pgfqpoint{6.110114in}{1.027837in}}%
\pgfpathcurveto{\pgfqpoint{6.105070in}{1.022793in}}{\pgfqpoint{6.102236in}{1.015951in}}{\pgfqpoint{6.102236in}{1.008818in}}%
\pgfpathcurveto{\pgfqpoint{6.102236in}{1.001686in}}{\pgfqpoint{6.105070in}{0.994844in}}{\pgfqpoint{6.110114in}{0.989800in}}%
\pgfpathcurveto{\pgfqpoint{6.115157in}{0.984757in}}{\pgfqpoint{6.121999in}{0.981923in}}{\pgfqpoint{6.129132in}{0.981923in}}%
\pgfpathclose%
\pgfusepath{stroke,fill}%
\end{pgfscope}%
\begin{pgfscope}%
\pgfpathrectangle{\pgfqpoint{4.985294in}{0.500000in}}{\pgfqpoint{1.764706in}{1.700000in}}%
\pgfusepath{clip}%
\pgfsetbuttcap%
\pgfsetroundjoin%
\definecolor{currentfill}{rgb}{0.976287,0.879862,0.805788}%
\pgfsetfillcolor{currentfill}%
\pgfsetlinewidth{0.311001pt}%
\definecolor{currentstroke}{rgb}{1.000000,1.000000,1.000000}%
\pgfsetstrokecolor{currentstroke}%
\pgfsetdash{}{0pt}%
\pgfpathmoveto{\pgfqpoint{6.247032in}{1.107215in}}%
\pgfpathcurveto{\pgfqpoint{6.254165in}{1.107215in}}{\pgfqpoint{6.261006in}{1.110049in}}{\pgfqpoint{6.266050in}{1.115093in}}%
\pgfpathcurveto{\pgfqpoint{6.271094in}{1.120137in}}{\pgfqpoint{6.273928in}{1.126978in}}{\pgfqpoint{6.273928in}{1.134111in}}%
\pgfpathcurveto{\pgfqpoint{6.273928in}{1.141244in}}{\pgfqpoint{6.271094in}{1.148085in}}{\pgfqpoint{6.266050in}{1.153129in}}%
\pgfpathcurveto{\pgfqpoint{6.261006in}{1.158173in}}{\pgfqpoint{6.254165in}{1.161007in}}{\pgfqpoint{6.247032in}{1.161007in}}%
\pgfpathcurveto{\pgfqpoint{6.239899in}{1.161007in}}{\pgfqpoint{6.233057in}{1.158173in}}{\pgfqpoint{6.228014in}{1.153129in}}%
\pgfpathcurveto{\pgfqpoint{6.222970in}{1.148085in}}{\pgfqpoint{6.220136in}{1.141244in}}{\pgfqpoint{6.220136in}{1.134111in}}%
\pgfpathcurveto{\pgfqpoint{6.220136in}{1.126978in}}{\pgfqpoint{6.222970in}{1.120137in}}{\pgfqpoint{6.228014in}{1.115093in}}%
\pgfpathcurveto{\pgfqpoint{6.233057in}{1.110049in}}{\pgfqpoint{6.239899in}{1.107215in}}{\pgfqpoint{6.247032in}{1.107215in}}%
\pgfpathclose%
\pgfusepath{stroke,fill}%
\end{pgfscope}%
\begin{pgfscope}%
\pgfpathrectangle{\pgfqpoint{4.985294in}{0.500000in}}{\pgfqpoint{1.764706in}{1.700000in}}%
\pgfusepath{clip}%
\pgfsetbuttcap%
\pgfsetroundjoin%
\definecolor{currentfill}{rgb}{0.968931,0.798091,0.685123}%
\pgfsetfillcolor{currentfill}%
\pgfsetlinewidth{0.311001pt}%
\definecolor{currentstroke}{rgb}{1.000000,1.000000,1.000000}%
\pgfsetstrokecolor{currentstroke}%
\pgfsetdash{}{0pt}%
\pgfpathmoveto{\pgfqpoint{5.454781in}{1.635406in}}%
\pgfpathcurveto{\pgfqpoint{5.461914in}{1.635406in}}{\pgfqpoint{5.468756in}{1.638240in}}{\pgfqpoint{5.473799in}{1.643284in}}%
\pgfpathcurveto{\pgfqpoint{5.478843in}{1.648327in}}{\pgfqpoint{5.481677in}{1.655169in}}{\pgfqpoint{5.481677in}{1.662302in}}%
\pgfpathcurveto{\pgfqpoint{5.481677in}{1.669435in}}{\pgfqpoint{5.478843in}{1.676276in}}{\pgfqpoint{5.473799in}{1.681320in}}%
\pgfpathcurveto{\pgfqpoint{5.468756in}{1.686364in}}{\pgfqpoint{5.461914in}{1.689197in}}{\pgfqpoint{5.454781in}{1.689197in}}%
\pgfpathcurveto{\pgfqpoint{5.447648in}{1.689197in}}{\pgfqpoint{5.440807in}{1.686364in}}{\pgfqpoint{5.435763in}{1.681320in}}%
\pgfpathcurveto{\pgfqpoint{5.430719in}{1.676276in}}{\pgfqpoint{5.427886in}{1.669435in}}{\pgfqpoint{5.427886in}{1.662302in}}%
\pgfpathcurveto{\pgfqpoint{5.427886in}{1.655169in}}{\pgfqpoint{5.430719in}{1.648327in}}{\pgfqpoint{5.435763in}{1.643284in}}%
\pgfpathcurveto{\pgfqpoint{5.440807in}{1.638240in}}{\pgfqpoint{5.447648in}{1.635406in}}{\pgfqpoint{5.454781in}{1.635406in}}%
\pgfpathclose%
\pgfusepath{stroke,fill}%
\end{pgfscope}%
\begin{pgfscope}%
\pgfpathrectangle{\pgfqpoint{4.985294in}{0.500000in}}{\pgfqpoint{1.764706in}{1.700000in}}%
\pgfusepath{clip}%
\pgfsetbuttcap%
\pgfsetroundjoin%
\definecolor{currentfill}{rgb}{0.973271,0.850724,0.762998}%
\pgfsetfillcolor{currentfill}%
\pgfsetlinewidth{0.311001pt}%
\definecolor{currentstroke}{rgb}{1.000000,1.000000,1.000000}%
\pgfsetstrokecolor{currentstroke}%
\pgfsetdash{}{0pt}%
\pgfpathmoveto{\pgfqpoint{6.262179in}{1.375459in}}%
\pgfpathcurveto{\pgfqpoint{6.269312in}{1.375459in}}{\pgfqpoint{6.276154in}{1.378293in}}{\pgfqpoint{6.281197in}{1.383337in}}%
\pgfpathcurveto{\pgfqpoint{6.286241in}{1.388380in}}{\pgfqpoint{6.289075in}{1.395222in}}{\pgfqpoint{6.289075in}{1.402355in}}%
\pgfpathcurveto{\pgfqpoint{6.289075in}{1.409488in}}{\pgfqpoint{6.286241in}{1.416329in}}{\pgfqpoint{6.281197in}{1.421373in}}%
\pgfpathcurveto{\pgfqpoint{6.276154in}{1.426417in}}{\pgfqpoint{6.269312in}{1.429250in}}{\pgfqpoint{6.262179in}{1.429250in}}%
\pgfpathcurveto{\pgfqpoint{6.255046in}{1.429250in}}{\pgfqpoint{6.248205in}{1.426417in}}{\pgfqpoint{6.243161in}{1.421373in}}%
\pgfpathcurveto{\pgfqpoint{6.238117in}{1.416329in}}{\pgfqpoint{6.235283in}{1.409488in}}{\pgfqpoint{6.235283in}{1.402355in}}%
\pgfpathcurveto{\pgfqpoint{6.235283in}{1.395222in}}{\pgfqpoint{6.238117in}{1.388380in}}{\pgfqpoint{6.243161in}{1.383337in}}%
\pgfpathcurveto{\pgfqpoint{6.248205in}{1.378293in}}{\pgfqpoint{6.255046in}{1.375459in}}{\pgfqpoint{6.262179in}{1.375459in}}%
\pgfpathclose%
\pgfusepath{stroke,fill}%
\end{pgfscope}%
\begin{pgfscope}%
\pgfpathrectangle{\pgfqpoint{4.985294in}{0.500000in}}{\pgfqpoint{1.764706in}{1.700000in}}%
\pgfusepath{clip}%
\pgfsetbuttcap%
\pgfsetroundjoin%
\definecolor{currentfill}{rgb}{0.979891,0.908948,0.848279}%
\pgfsetfillcolor{currentfill}%
\pgfsetlinewidth{0.311001pt}%
\definecolor{currentstroke}{rgb}{1.000000,1.000000,1.000000}%
\pgfsetstrokecolor{currentstroke}%
\pgfsetdash{}{0pt}%
\pgfpathmoveto{\pgfqpoint{6.290352in}{1.228840in}}%
\pgfpathcurveto{\pgfqpoint{6.297485in}{1.228840in}}{\pgfqpoint{6.304326in}{1.231674in}}{\pgfqpoint{6.309370in}{1.236718in}}%
\pgfpathcurveto{\pgfqpoint{6.314414in}{1.241761in}}{\pgfqpoint{6.317248in}{1.248603in}}{\pgfqpoint{6.317248in}{1.255736in}}%
\pgfpathcurveto{\pgfqpoint{6.317248in}{1.262869in}}{\pgfqpoint{6.314414in}{1.269710in}}{\pgfqpoint{6.309370in}{1.274754in}}%
\pgfpathcurveto{\pgfqpoint{6.304326in}{1.279797in}}{\pgfqpoint{6.297485in}{1.282631in}}{\pgfqpoint{6.290352in}{1.282631in}}%
\pgfpathcurveto{\pgfqpoint{6.283219in}{1.282631in}}{\pgfqpoint{6.276377in}{1.279797in}}{\pgfqpoint{6.271334in}{1.274754in}}%
\pgfpathcurveto{\pgfqpoint{6.266290in}{1.269710in}}{\pgfqpoint{6.263456in}{1.262869in}}{\pgfqpoint{6.263456in}{1.255736in}}%
\pgfpathcurveto{\pgfqpoint{6.263456in}{1.248603in}}{\pgfqpoint{6.266290in}{1.241761in}}{\pgfqpoint{6.271334in}{1.236718in}}%
\pgfpathcurveto{\pgfqpoint{6.276377in}{1.231674in}}{\pgfqpoint{6.283219in}{1.228840in}}{\pgfqpoint{6.290352in}{1.228840in}}%
\pgfpathclose%
\pgfusepath{stroke,fill}%
\end{pgfscope}%
\begin{pgfscope}%
\pgfpathrectangle{\pgfqpoint{4.985294in}{0.500000in}}{\pgfqpoint{1.764706in}{1.700000in}}%
\pgfusepath{clip}%
\pgfsetbuttcap%
\pgfsetroundjoin%
\definecolor{currentfill}{rgb}{0.980678,0.914765,0.856766}%
\pgfsetfillcolor{currentfill}%
\pgfsetlinewidth{0.311001pt}%
\definecolor{currentstroke}{rgb}{1.000000,1.000000,1.000000}%
\pgfsetstrokecolor{currentstroke}%
\pgfsetdash{}{0pt}%
\pgfpathmoveto{\pgfqpoint{5.417021in}{1.387091in}}%
\pgfpathcurveto{\pgfqpoint{5.424154in}{1.387091in}}{\pgfqpoint{5.430995in}{1.389924in}}{\pgfqpoint{5.436039in}{1.394968in}}%
\pgfpathcurveto{\pgfqpoint{5.441083in}{1.400012in}}{\pgfqpoint{5.443917in}{1.406853in}}{\pgfqpoint{5.443917in}{1.413986in}}%
\pgfpathcurveto{\pgfqpoint{5.443917in}{1.421119in}}{\pgfqpoint{5.441083in}{1.427961in}}{\pgfqpoint{5.436039in}{1.433004in}}%
\pgfpathcurveto{\pgfqpoint{5.430995in}{1.438048in}}{\pgfqpoint{5.424154in}{1.440882in}}{\pgfqpoint{5.417021in}{1.440882in}}%
\pgfpathcurveto{\pgfqpoint{5.409888in}{1.440882in}}{\pgfqpoint{5.403046in}{1.438048in}}{\pgfqpoint{5.398003in}{1.433004in}}%
\pgfpathcurveto{\pgfqpoint{5.392959in}{1.427961in}}{\pgfqpoint{5.390125in}{1.421119in}}{\pgfqpoint{5.390125in}{1.413986in}}%
\pgfpathcurveto{\pgfqpoint{5.390125in}{1.406853in}}{\pgfqpoint{5.392959in}{1.400012in}}{\pgfqpoint{5.398003in}{1.394968in}}%
\pgfpathcurveto{\pgfqpoint{5.403046in}{1.389924in}}{\pgfqpoint{5.409888in}{1.387091in}}{\pgfqpoint{5.417021in}{1.387091in}}%
\pgfpathclose%
\pgfusepath{stroke,fill}%
\end{pgfscope}%
\begin{pgfscope}%
\pgfpathrectangle{\pgfqpoint{4.985294in}{0.500000in}}{\pgfqpoint{1.764706in}{1.700000in}}%
\pgfusepath{clip}%
\pgfsetbuttcap%
\pgfsetroundjoin%
\definecolor{currentfill}{rgb}{0.976287,0.879862,0.805788}%
\pgfsetfillcolor{currentfill}%
\pgfsetlinewidth{0.311001pt}%
\definecolor{currentstroke}{rgb}{1.000000,1.000000,1.000000}%
\pgfsetstrokecolor{currentstroke}%
\pgfsetdash{}{0pt}%
\pgfpathmoveto{\pgfqpoint{6.274996in}{1.338048in}}%
\pgfpathcurveto{\pgfqpoint{6.282129in}{1.338048in}}{\pgfqpoint{6.288970in}{1.340882in}}{\pgfqpoint{6.294014in}{1.345926in}}%
\pgfpathcurveto{\pgfqpoint{6.299058in}{1.350969in}}{\pgfqpoint{6.301892in}{1.357811in}}{\pgfqpoint{6.301892in}{1.364944in}}%
\pgfpathcurveto{\pgfqpoint{6.301892in}{1.372077in}}{\pgfqpoint{6.299058in}{1.378918in}}{\pgfqpoint{6.294014in}{1.383962in}}%
\pgfpathcurveto{\pgfqpoint{6.288970in}{1.389006in}}{\pgfqpoint{6.282129in}{1.391839in}}{\pgfqpoint{6.274996in}{1.391839in}}%
\pgfpathcurveto{\pgfqpoint{6.267863in}{1.391839in}}{\pgfqpoint{6.261021in}{1.389006in}}{\pgfqpoint{6.255978in}{1.383962in}}%
\pgfpathcurveto{\pgfqpoint{6.250934in}{1.378918in}}{\pgfqpoint{6.248100in}{1.372077in}}{\pgfqpoint{6.248100in}{1.364944in}}%
\pgfpathcurveto{\pgfqpoint{6.248100in}{1.357811in}}{\pgfqpoint{6.250934in}{1.350969in}}{\pgfqpoint{6.255978in}{1.345926in}}%
\pgfpathcurveto{\pgfqpoint{6.261021in}{1.340882in}}{\pgfqpoint{6.267863in}{1.338048in}}{\pgfqpoint{6.274996in}{1.338048in}}%
\pgfpathclose%
\pgfusepath{stroke,fill}%
\end{pgfscope}%
\begin{pgfscope}%
\pgfpathrectangle{\pgfqpoint{4.985294in}{0.500000in}}{\pgfqpoint{1.764706in}{1.700000in}}%
\pgfusepath{clip}%
\pgfsetbuttcap%
\pgfsetroundjoin%
\definecolor{currentfill}{rgb}{0.964799,0.689101,0.537560}%
\pgfsetfillcolor{currentfill}%
\pgfsetlinewidth{0.311001pt}%
\definecolor{currentstroke}{rgb}{1.000000,1.000000,1.000000}%
\pgfsetstrokecolor{currentstroke}%
\pgfsetdash{}{0pt}%
\pgfpathmoveto{\pgfqpoint{5.442169in}{1.680826in}}%
\pgfpathcurveto{\pgfqpoint{5.449302in}{1.680826in}}{\pgfqpoint{5.456144in}{1.683660in}}{\pgfqpoint{5.461188in}{1.688704in}}%
\pgfpathcurveto{\pgfqpoint{5.466231in}{1.693748in}}{\pgfqpoint{5.469065in}{1.700589in}}{\pgfqpoint{5.469065in}{1.707722in}}%
\pgfpathcurveto{\pgfqpoint{5.469065in}{1.714855in}}{\pgfqpoint{5.466231in}{1.721697in}}{\pgfqpoint{5.461188in}{1.726740in}}%
\pgfpathcurveto{\pgfqpoint{5.456144in}{1.731784in}}{\pgfqpoint{5.449302in}{1.734618in}}{\pgfqpoint{5.442169in}{1.734618in}}%
\pgfpathcurveto{\pgfqpoint{5.435037in}{1.734618in}}{\pgfqpoint{5.428195in}{1.731784in}}{\pgfqpoint{5.423151in}{1.726740in}}%
\pgfpathcurveto{\pgfqpoint{5.418108in}{1.721697in}}{\pgfqpoint{5.415274in}{1.714855in}}{\pgfqpoint{5.415274in}{1.707722in}}%
\pgfpathcurveto{\pgfqpoint{5.415274in}{1.700589in}}{\pgfqpoint{5.418108in}{1.693748in}}{\pgfqpoint{5.423151in}{1.688704in}}%
\pgfpathcurveto{\pgfqpoint{5.428195in}{1.683660in}}{\pgfqpoint{5.435037in}{1.680826in}}{\pgfqpoint{5.442169in}{1.680826in}}%
\pgfpathclose%
\pgfusepath{stroke,fill}%
\end{pgfscope}%
\begin{pgfscope}%
\pgfpathrectangle{\pgfqpoint{4.985294in}{0.500000in}}{\pgfqpoint{1.764706in}{1.700000in}}%
\pgfusepath{clip}%
\pgfsetbuttcap%
\pgfsetroundjoin%
\definecolor{currentfill}{rgb}{0.948235,0.413004,0.283323}%
\pgfsetfillcolor{currentfill}%
\pgfsetlinewidth{0.311001pt}%
\definecolor{currentstroke}{rgb}{1.000000,1.000000,1.000000}%
\pgfsetstrokecolor{currentstroke}%
\pgfsetdash{}{0pt}%
\pgfpathmoveto{\pgfqpoint{5.301993in}{1.108090in}}%
\pgfpathcurveto{\pgfqpoint{5.309126in}{1.108090in}}{\pgfqpoint{5.315967in}{1.110924in}}{\pgfqpoint{5.321011in}{1.115968in}}%
\pgfpathcurveto{\pgfqpoint{5.326055in}{1.121011in}}{\pgfqpoint{5.328888in}{1.127853in}}{\pgfqpoint{5.328888in}{1.134986in}}%
\pgfpathcurveto{\pgfqpoint{5.328888in}{1.142119in}}{\pgfqpoint{5.326055in}{1.148960in}}{\pgfqpoint{5.321011in}{1.154004in}}%
\pgfpathcurveto{\pgfqpoint{5.315967in}{1.159048in}}{\pgfqpoint{5.309126in}{1.161882in}}{\pgfqpoint{5.301993in}{1.161882in}}%
\pgfpathcurveto{\pgfqpoint{5.294860in}{1.161882in}}{\pgfqpoint{5.288018in}{1.159048in}}{\pgfqpoint{5.282975in}{1.154004in}}%
\pgfpathcurveto{\pgfqpoint{5.277931in}{1.148960in}}{\pgfqpoint{5.275097in}{1.142119in}}{\pgfqpoint{5.275097in}{1.134986in}}%
\pgfpathcurveto{\pgfqpoint{5.275097in}{1.127853in}}{\pgfqpoint{5.277931in}{1.121011in}}{\pgfqpoint{5.282975in}{1.115968in}}%
\pgfpathcurveto{\pgfqpoint{5.288018in}{1.110924in}}{\pgfqpoint{5.294860in}{1.108090in}}{\pgfqpoint{5.301993in}{1.108090in}}%
\pgfpathclose%
\pgfusepath{stroke,fill}%
\end{pgfscope}%
\begin{pgfscope}%
\pgfpathrectangle{\pgfqpoint{4.985294in}{0.500000in}}{\pgfqpoint{1.764706in}{1.700000in}}%
\pgfusepath{clip}%
\pgfsetbuttcap%
\pgfsetroundjoin%
\definecolor{currentfill}{rgb}{0.977657,0.891500,0.822809}%
\pgfsetfillcolor{currentfill}%
\pgfsetlinewidth{0.311001pt}%
\definecolor{currentstroke}{rgb}{1.000000,1.000000,1.000000}%
\pgfsetstrokecolor{currentstroke}%
\pgfsetdash{}{0pt}%
\pgfpathmoveto{\pgfqpoint{6.276046in}{1.473038in}}%
\pgfpathcurveto{\pgfqpoint{6.283179in}{1.473038in}}{\pgfqpoint{6.290021in}{1.475872in}}{\pgfqpoint{6.295064in}{1.480915in}}%
\pgfpathcurveto{\pgfqpoint{6.300108in}{1.485959in}}{\pgfqpoint{6.302942in}{1.492801in}}{\pgfqpoint{6.302942in}{1.499933in}}%
\pgfpathcurveto{\pgfqpoint{6.302942in}{1.507066in}}{\pgfqpoint{6.300108in}{1.513908in}}{\pgfqpoint{6.295064in}{1.518952in}}%
\pgfpathcurveto{\pgfqpoint{6.290021in}{1.523995in}}{\pgfqpoint{6.283179in}{1.526829in}}{\pgfqpoint{6.276046in}{1.526829in}}%
\pgfpathcurveto{\pgfqpoint{6.268914in}{1.526829in}}{\pgfqpoint{6.262072in}{1.523995in}}{\pgfqpoint{6.257028in}{1.518952in}}%
\pgfpathcurveto{\pgfqpoint{6.251985in}{1.513908in}}{\pgfqpoint{6.249151in}{1.507066in}}{\pgfqpoint{6.249151in}{1.499933in}}%
\pgfpathcurveto{\pgfqpoint{6.249151in}{1.492801in}}{\pgfqpoint{6.251985in}{1.485959in}}{\pgfqpoint{6.257028in}{1.480915in}}%
\pgfpathcurveto{\pgfqpoint{6.262072in}{1.475872in}}{\pgfqpoint{6.268914in}{1.473038in}}{\pgfqpoint{6.276046in}{1.473038in}}%
\pgfpathclose%
\pgfusepath{stroke,fill}%
\end{pgfscope}%
\begin{pgfscope}%
\pgfpathrectangle{\pgfqpoint{4.985294in}{0.500000in}}{\pgfqpoint{1.764706in}{1.700000in}}%
\pgfusepath{clip}%
\pgfsetbuttcap%
\pgfsetroundjoin%
\definecolor{currentfill}{rgb}{0.979124,0.903132,0.839793}%
\pgfsetfillcolor{currentfill}%
\pgfsetlinewidth{0.311001pt}%
\definecolor{currentstroke}{rgb}{1.000000,1.000000,1.000000}%
\pgfsetstrokecolor{currentstroke}%
\pgfsetdash{}{0pt}%
\pgfpathmoveto{\pgfqpoint{6.291097in}{1.363520in}}%
\pgfpathcurveto{\pgfqpoint{6.298230in}{1.363520in}}{\pgfqpoint{6.305071in}{1.366354in}}{\pgfqpoint{6.310115in}{1.371397in}}%
\pgfpathcurveto{\pgfqpoint{6.315158in}{1.376441in}}{\pgfqpoint{6.317992in}{1.383283in}}{\pgfqpoint{6.317992in}{1.390415in}}%
\pgfpathcurveto{\pgfqpoint{6.317992in}{1.397548in}}{\pgfqpoint{6.315158in}{1.404390in}}{\pgfqpoint{6.310115in}{1.409434in}}%
\pgfpathcurveto{\pgfqpoint{6.305071in}{1.414477in}}{\pgfqpoint{6.298230in}{1.417311in}}{\pgfqpoint{6.291097in}{1.417311in}}%
\pgfpathcurveto{\pgfqpoint{6.283964in}{1.417311in}}{\pgfqpoint{6.277122in}{1.414477in}}{\pgfqpoint{6.272079in}{1.409434in}}%
\pgfpathcurveto{\pgfqpoint{6.267035in}{1.404390in}}{\pgfqpoint{6.264201in}{1.397548in}}{\pgfqpoint{6.264201in}{1.390415in}}%
\pgfpathcurveto{\pgfqpoint{6.264201in}{1.383283in}}{\pgfqpoint{6.267035in}{1.376441in}}{\pgfqpoint{6.272079in}{1.371397in}}%
\pgfpathcurveto{\pgfqpoint{6.277122in}{1.366354in}}{\pgfqpoint{6.283964in}{1.363520in}}{\pgfqpoint{6.291097in}{1.363520in}}%
\pgfpathclose%
\pgfusepath{stroke,fill}%
\end{pgfscope}%
\begin{pgfscope}%
\pgfpathrectangle{\pgfqpoint{4.985294in}{0.500000in}}{\pgfqpoint{1.764706in}{1.700000in}}%
\pgfusepath{clip}%
\pgfsetbuttcap%
\pgfsetroundjoin%
\definecolor{currentfill}{rgb}{0.979891,0.908948,0.848279}%
\pgfsetfillcolor{currentfill}%
\pgfsetlinewidth{0.311001pt}%
\definecolor{currentstroke}{rgb}{1.000000,1.000000,1.000000}%
\pgfsetstrokecolor{currentstroke}%
\pgfsetdash{}{0pt}%
\pgfpathmoveto{\pgfqpoint{5.411031in}{1.398700in}}%
\pgfpathcurveto{\pgfqpoint{5.418163in}{1.398700in}}{\pgfqpoint{5.425005in}{1.401534in}}{\pgfqpoint{5.430049in}{1.406578in}}%
\pgfpathcurveto{\pgfqpoint{5.435092in}{1.411622in}}{\pgfqpoint{5.437926in}{1.418463in}}{\pgfqpoint{5.437926in}{1.425596in}}%
\pgfpathcurveto{\pgfqpoint{5.437926in}{1.432729in}}{\pgfqpoint{5.435092in}{1.439571in}}{\pgfqpoint{5.430049in}{1.444614in}}%
\pgfpathcurveto{\pgfqpoint{5.425005in}{1.449658in}}{\pgfqpoint{5.418163in}{1.452492in}}{\pgfqpoint{5.411031in}{1.452492in}}%
\pgfpathcurveto{\pgfqpoint{5.403898in}{1.452492in}}{\pgfqpoint{5.397056in}{1.449658in}}{\pgfqpoint{5.392012in}{1.444614in}}%
\pgfpathcurveto{\pgfqpoint{5.386969in}{1.439571in}}{\pgfqpoint{5.384135in}{1.432729in}}{\pgfqpoint{5.384135in}{1.425596in}}%
\pgfpathcurveto{\pgfqpoint{5.384135in}{1.418463in}}{\pgfqpoint{5.386969in}{1.411622in}}{\pgfqpoint{5.392012in}{1.406578in}}%
\pgfpathcurveto{\pgfqpoint{5.397056in}{1.401534in}}{\pgfqpoint{5.403898in}{1.398700in}}{\pgfqpoint{5.411031in}{1.398700in}}%
\pgfpathclose%
\pgfusepath{stroke,fill}%
\end{pgfscope}%
\begin{pgfscope}%
\pgfpathrectangle{\pgfqpoint{4.985294in}{0.500000in}}{\pgfqpoint{1.764706in}{1.700000in}}%
\pgfusepath{clip}%
\pgfsetbuttcap%
\pgfsetroundjoin%
\definecolor{currentfill}{rgb}{0.976287,0.879862,0.805788}%
\pgfsetfillcolor{currentfill}%
\pgfsetlinewidth{0.311001pt}%
\definecolor{currentstroke}{rgb}{1.000000,1.000000,1.000000}%
\pgfsetstrokecolor{currentstroke}%
\pgfsetdash{}{0pt}%
\pgfpathmoveto{\pgfqpoint{6.320206in}{1.515399in}}%
\pgfpathcurveto{\pgfqpoint{6.327339in}{1.515399in}}{\pgfqpoint{6.334180in}{1.518233in}}{\pgfqpoint{6.339224in}{1.523277in}}%
\pgfpathcurveto{\pgfqpoint{6.344268in}{1.528320in}}{\pgfqpoint{6.347102in}{1.535162in}}{\pgfqpoint{6.347102in}{1.542295in}}%
\pgfpathcurveto{\pgfqpoint{6.347102in}{1.549428in}}{\pgfqpoint{6.344268in}{1.556269in}}{\pgfqpoint{6.339224in}{1.561313in}}%
\pgfpathcurveto{\pgfqpoint{6.334180in}{1.566357in}}{\pgfqpoint{6.327339in}{1.569191in}}{\pgfqpoint{6.320206in}{1.569191in}}%
\pgfpathcurveto{\pgfqpoint{6.313073in}{1.569191in}}{\pgfqpoint{6.306231in}{1.566357in}}{\pgfqpoint{6.301188in}{1.561313in}}%
\pgfpathcurveto{\pgfqpoint{6.296144in}{1.556269in}}{\pgfqpoint{6.293310in}{1.549428in}}{\pgfqpoint{6.293310in}{1.542295in}}%
\pgfpathcurveto{\pgfqpoint{6.293310in}{1.535162in}}{\pgfqpoint{6.296144in}{1.528320in}}{\pgfqpoint{6.301188in}{1.523277in}}%
\pgfpathcurveto{\pgfqpoint{6.306231in}{1.518233in}}{\pgfqpoint{6.313073in}{1.515399in}}{\pgfqpoint{6.320206in}{1.515399in}}%
\pgfpathclose%
\pgfusepath{stroke,fill}%
\end{pgfscope}%
\begin{pgfscope}%
\pgfpathrectangle{\pgfqpoint{4.985294in}{0.500000in}}{\pgfqpoint{1.764706in}{1.700000in}}%
\pgfusepath{clip}%
\pgfsetbuttcap%
\pgfsetroundjoin%
\definecolor{currentfill}{rgb}{0.957344,0.505732,0.351309}%
\pgfsetfillcolor{currentfill}%
\pgfsetlinewidth{0.311001pt}%
\definecolor{currentstroke}{rgb}{1.000000,1.000000,1.000000}%
\pgfsetstrokecolor{currentstroke}%
\pgfsetdash{}{0pt}%
\pgfpathmoveto{\pgfqpoint{6.443656in}{1.331113in}}%
\pgfpathcurveto{\pgfqpoint{6.450789in}{1.331113in}}{\pgfqpoint{6.457631in}{1.333947in}}{\pgfqpoint{6.462674in}{1.338991in}}%
\pgfpathcurveto{\pgfqpoint{6.467718in}{1.344035in}}{\pgfqpoint{6.470552in}{1.350876in}}{\pgfqpoint{6.470552in}{1.358009in}}%
\pgfpathcurveto{\pgfqpoint{6.470552in}{1.365142in}}{\pgfqpoint{6.467718in}{1.371983in}}{\pgfqpoint{6.462674in}{1.377027in}}%
\pgfpathcurveto{\pgfqpoint{6.457631in}{1.382071in}}{\pgfqpoint{6.450789in}{1.384905in}}{\pgfqpoint{6.443656in}{1.384905in}}%
\pgfpathcurveto{\pgfqpoint{6.436524in}{1.384905in}}{\pgfqpoint{6.429682in}{1.382071in}}{\pgfqpoint{6.424638in}{1.377027in}}%
\pgfpathcurveto{\pgfqpoint{6.419595in}{1.371983in}}{\pgfqpoint{6.416761in}{1.365142in}}{\pgfqpoint{6.416761in}{1.358009in}}%
\pgfpathcurveto{\pgfqpoint{6.416761in}{1.350876in}}{\pgfqpoint{6.419595in}{1.344035in}}{\pgfqpoint{6.424638in}{1.338991in}}%
\pgfpathcurveto{\pgfqpoint{6.429682in}{1.333947in}}{\pgfqpoint{6.436524in}{1.331113in}}{\pgfqpoint{6.443656in}{1.331113in}}%
\pgfpathclose%
\pgfusepath{stroke,fill}%
\end{pgfscope}%
\begin{pgfscope}%
\pgfpathrectangle{\pgfqpoint{4.985294in}{0.500000in}}{\pgfqpoint{1.764706in}{1.700000in}}%
\pgfusepath{clip}%
\pgfsetbuttcap%
\pgfsetroundjoin%
\definecolor{currentfill}{rgb}{0.967398,0.774513,0.650573}%
\pgfsetfillcolor{currentfill}%
\pgfsetlinewidth{0.311001pt}%
\definecolor{currentstroke}{rgb}{1.000000,1.000000,1.000000}%
\pgfsetstrokecolor{currentstroke}%
\pgfsetdash{}{0pt}%
\pgfpathmoveto{\pgfqpoint{6.205911in}{1.232039in}}%
\pgfpathcurveto{\pgfqpoint{6.213044in}{1.232039in}}{\pgfqpoint{6.219886in}{1.234873in}}{\pgfqpoint{6.224930in}{1.239916in}}%
\pgfpathcurveto{\pgfqpoint{6.229973in}{1.244960in}}{\pgfqpoint{6.232807in}{1.251802in}}{\pgfqpoint{6.232807in}{1.258935in}}%
\pgfpathcurveto{\pgfqpoint{6.232807in}{1.266067in}}{\pgfqpoint{6.229973in}{1.272909in}}{\pgfqpoint{6.224930in}{1.277953in}}%
\pgfpathcurveto{\pgfqpoint{6.219886in}{1.282996in}}{\pgfqpoint{6.213044in}{1.285830in}}{\pgfqpoint{6.205911in}{1.285830in}}%
\pgfpathcurveto{\pgfqpoint{6.198779in}{1.285830in}}{\pgfqpoint{6.191937in}{1.282996in}}{\pgfqpoint{6.186893in}{1.277953in}}%
\pgfpathcurveto{\pgfqpoint{6.181850in}{1.272909in}}{\pgfqpoint{6.179016in}{1.266067in}}{\pgfqpoint{6.179016in}{1.258935in}}%
\pgfpathcurveto{\pgfqpoint{6.179016in}{1.251802in}}{\pgfqpoint{6.181850in}{1.244960in}}{\pgfqpoint{6.186893in}{1.239916in}}%
\pgfpathcurveto{\pgfqpoint{6.191937in}{1.234873in}}{\pgfqpoint{6.198779in}{1.232039in}}{\pgfqpoint{6.205911in}{1.232039in}}%
\pgfpathclose%
\pgfusepath{stroke,fill}%
\end{pgfscope}%
\begin{pgfscope}%
\pgfpathrectangle{\pgfqpoint{4.985294in}{0.500000in}}{\pgfqpoint{1.764706in}{1.700000in}}%
\pgfusepath{clip}%
\pgfsetbuttcap%
\pgfsetroundjoin%
\definecolor{currentfill}{rgb}{0.976287,0.879862,0.805788}%
\pgfsetfillcolor{currentfill}%
\pgfsetlinewidth{0.311001pt}%
\definecolor{currentstroke}{rgb}{1.000000,1.000000,1.000000}%
\pgfsetstrokecolor{currentstroke}%
\pgfsetdash{}{0pt}%
\pgfpathmoveto{\pgfqpoint{5.463529in}{1.207741in}}%
\pgfpathcurveto{\pgfqpoint{5.470662in}{1.207741in}}{\pgfqpoint{5.477504in}{1.210575in}}{\pgfqpoint{5.482547in}{1.215618in}}%
\pgfpathcurveto{\pgfqpoint{5.487591in}{1.220662in}}{\pgfqpoint{5.490425in}{1.227504in}}{\pgfqpoint{5.490425in}{1.234636in}}%
\pgfpathcurveto{\pgfqpoint{5.490425in}{1.241769in}}{\pgfqpoint{5.487591in}{1.248611in}}{\pgfqpoint{5.482547in}{1.253655in}}%
\pgfpathcurveto{\pgfqpoint{5.477504in}{1.258698in}}{\pgfqpoint{5.470662in}{1.261532in}}{\pgfqpoint{5.463529in}{1.261532in}}%
\pgfpathcurveto{\pgfqpoint{5.456396in}{1.261532in}}{\pgfqpoint{5.449555in}{1.258698in}}{\pgfqpoint{5.444511in}{1.253655in}}%
\pgfpathcurveto{\pgfqpoint{5.439467in}{1.248611in}}{\pgfqpoint{5.436633in}{1.241769in}}{\pgfqpoint{5.436633in}{1.234636in}}%
\pgfpathcurveto{\pgfqpoint{5.436633in}{1.227504in}}{\pgfqpoint{5.439467in}{1.220662in}}{\pgfqpoint{5.444511in}{1.215618in}}%
\pgfpathcurveto{\pgfqpoint{5.449555in}{1.210575in}}{\pgfqpoint{5.456396in}{1.207741in}}{\pgfqpoint{5.463529in}{1.207741in}}%
\pgfpathclose%
\pgfusepath{stroke,fill}%
\end{pgfscope}%
\begin{pgfscope}%
\pgfpathrectangle{\pgfqpoint{4.985294in}{0.500000in}}{\pgfqpoint{1.764706in}{1.700000in}}%
\pgfusepath{clip}%
\pgfsetbuttcap%
\pgfsetroundjoin%
\definecolor{currentfill}{rgb}{0.979891,0.908948,0.848279}%
\pgfsetfillcolor{currentfill}%
\pgfsetlinewidth{0.311001pt}%
\definecolor{currentstroke}{rgb}{1.000000,1.000000,1.000000}%
\pgfsetstrokecolor{currentstroke}%
\pgfsetdash{}{0pt}%
\pgfpathmoveto{\pgfqpoint{5.432516in}{1.242593in}}%
\pgfpathcurveto{\pgfqpoint{5.439648in}{1.242593in}}{\pgfqpoint{5.446490in}{1.245427in}}{\pgfqpoint{5.451534in}{1.250471in}}%
\pgfpathcurveto{\pgfqpoint{5.456577in}{1.255515in}}{\pgfqpoint{5.459411in}{1.262356in}}{\pgfqpoint{5.459411in}{1.269489in}}%
\pgfpathcurveto{\pgfqpoint{5.459411in}{1.276622in}}{\pgfqpoint{5.456577in}{1.283464in}}{\pgfqpoint{5.451534in}{1.288507in}}%
\pgfpathcurveto{\pgfqpoint{5.446490in}{1.293551in}}{\pgfqpoint{5.439648in}{1.296385in}}{\pgfqpoint{5.432516in}{1.296385in}}%
\pgfpathcurveto{\pgfqpoint{5.425383in}{1.296385in}}{\pgfqpoint{5.418541in}{1.293551in}}{\pgfqpoint{5.413497in}{1.288507in}}%
\pgfpathcurveto{\pgfqpoint{5.408454in}{1.283464in}}{\pgfqpoint{5.405620in}{1.276622in}}{\pgfqpoint{5.405620in}{1.269489in}}%
\pgfpathcurveto{\pgfqpoint{5.405620in}{1.262356in}}{\pgfqpoint{5.408454in}{1.255515in}}{\pgfqpoint{5.413497in}{1.250471in}}%
\pgfpathcurveto{\pgfqpoint{5.418541in}{1.245427in}}{\pgfqpoint{5.425383in}{1.242593in}}{\pgfqpoint{5.432516in}{1.242593in}}%
\pgfpathclose%
\pgfusepath{stroke,fill}%
\end{pgfscope}%
\begin{pgfscope}%
\pgfpathrectangle{\pgfqpoint{4.985294in}{0.500000in}}{\pgfqpoint{1.764706in}{1.700000in}}%
\pgfusepath{clip}%
\pgfsetbuttcap%
\pgfsetroundjoin%
\definecolor{currentfill}{rgb}{0.969359,0.803954,0.693832}%
\pgfsetfillcolor{currentfill}%
\pgfsetlinewidth{0.311001pt}%
\definecolor{currentstroke}{rgb}{1.000000,1.000000,1.000000}%
\pgfsetstrokecolor{currentstroke}%
\pgfsetdash{}{0pt}%
\pgfpathmoveto{\pgfqpoint{6.190552in}{0.982456in}}%
\pgfpathcurveto{\pgfqpoint{6.197685in}{0.982456in}}{\pgfqpoint{6.204526in}{0.985290in}}{\pgfqpoint{6.209570in}{0.990333in}}%
\pgfpathcurveto{\pgfqpoint{6.214614in}{0.995377in}}{\pgfqpoint{6.217447in}{1.002219in}}{\pgfqpoint{6.217447in}{1.009352in}}%
\pgfpathcurveto{\pgfqpoint{6.217447in}{1.016484in}}{\pgfqpoint{6.214614in}{1.023326in}}{\pgfqpoint{6.209570in}{1.028370in}}%
\pgfpathcurveto{\pgfqpoint{6.204526in}{1.033413in}}{\pgfqpoint{6.197685in}{1.036247in}}{\pgfqpoint{6.190552in}{1.036247in}}%
\pgfpathcurveto{\pgfqpoint{6.183419in}{1.036247in}}{\pgfqpoint{6.176577in}{1.033413in}}{\pgfqpoint{6.171534in}{1.028370in}}%
\pgfpathcurveto{\pgfqpoint{6.166490in}{1.023326in}}{\pgfqpoint{6.163656in}{1.016484in}}{\pgfqpoint{6.163656in}{1.009352in}}%
\pgfpathcurveto{\pgfqpoint{6.163656in}{1.002219in}}{\pgfqpoint{6.166490in}{0.995377in}}{\pgfqpoint{6.171534in}{0.990333in}}%
\pgfpathcurveto{\pgfqpoint{6.176577in}{0.985290in}}{\pgfqpoint{6.183419in}{0.982456in}}{\pgfqpoint{6.190552in}{0.982456in}}%
\pgfpathclose%
\pgfusepath{stroke,fill}%
\end{pgfscope}%
\begin{pgfscope}%
\pgfpathrectangle{\pgfqpoint{4.985294in}{0.500000in}}{\pgfqpoint{1.764706in}{1.700000in}}%
\pgfusepath{clip}%
\pgfsetbuttcap%
\pgfsetroundjoin%
\definecolor{currentfill}{rgb}{0.952404,0.449449,0.307210}%
\pgfsetfillcolor{currentfill}%
\pgfsetlinewidth{0.311001pt}%
\definecolor{currentstroke}{rgb}{1.000000,1.000000,1.000000}%
\pgfsetstrokecolor{currentstroke}%
\pgfsetdash{}{0pt}%
\pgfpathmoveto{\pgfqpoint{5.646822in}{1.056962in}}%
\pgfpathcurveto{\pgfqpoint{5.653955in}{1.056962in}}{\pgfqpoint{5.660796in}{1.059796in}}{\pgfqpoint{5.665840in}{1.064839in}}%
\pgfpathcurveto{\pgfqpoint{5.670884in}{1.069883in}}{\pgfqpoint{5.673718in}{1.076725in}}{\pgfqpoint{5.673718in}{1.083857in}}%
\pgfpathcurveto{\pgfqpoint{5.673718in}{1.090990in}}{\pgfqpoint{5.670884in}{1.097832in}}{\pgfqpoint{5.665840in}{1.102876in}}%
\pgfpathcurveto{\pgfqpoint{5.660796in}{1.107919in}}{\pgfqpoint{5.653955in}{1.110753in}}{\pgfqpoint{5.646822in}{1.110753in}}%
\pgfpathcurveto{\pgfqpoint{5.639689in}{1.110753in}}{\pgfqpoint{5.632848in}{1.107919in}}{\pgfqpoint{5.627804in}{1.102876in}}%
\pgfpathcurveto{\pgfqpoint{5.622760in}{1.097832in}}{\pgfqpoint{5.619926in}{1.090990in}}{\pgfqpoint{5.619926in}{1.083857in}}%
\pgfpathcurveto{\pgfqpoint{5.619926in}{1.076725in}}{\pgfqpoint{5.622760in}{1.069883in}}{\pgfqpoint{5.627804in}{1.064839in}}%
\pgfpathcurveto{\pgfqpoint{5.632848in}{1.059796in}}{\pgfqpoint{5.639689in}{1.056962in}}{\pgfqpoint{5.646822in}{1.056962in}}%
\pgfpathclose%
\pgfusepath{stroke,fill}%
\end{pgfscope}%
\begin{pgfscope}%
\pgfpathrectangle{\pgfqpoint{4.985294in}{0.500000in}}{\pgfqpoint{1.764706in}{1.700000in}}%
\pgfusepath{clip}%
\pgfsetbuttcap%
\pgfsetroundjoin%
\definecolor{currentfill}{rgb}{0.971202,0.827364,0.728520}%
\pgfsetfillcolor{currentfill}%
\pgfsetlinewidth{0.311001pt}%
\definecolor{currentstroke}{rgb}{1.000000,1.000000,1.000000}%
\pgfsetstrokecolor{currentstroke}%
\pgfsetdash{}{0pt}%
\pgfpathmoveto{\pgfqpoint{5.523490in}{1.073790in}}%
\pgfpathcurveto{\pgfqpoint{5.530623in}{1.073790in}}{\pgfqpoint{5.537465in}{1.076624in}}{\pgfqpoint{5.542509in}{1.081667in}}%
\pgfpathcurveto{\pgfqpoint{5.547552in}{1.086711in}}{\pgfqpoint{5.550386in}{1.093553in}}{\pgfqpoint{5.550386in}{1.100686in}}%
\pgfpathcurveto{\pgfqpoint{5.550386in}{1.107818in}}{\pgfqpoint{5.547552in}{1.114660in}}{\pgfqpoint{5.542509in}{1.119704in}}%
\pgfpathcurveto{\pgfqpoint{5.537465in}{1.124747in}}{\pgfqpoint{5.530623in}{1.127581in}}{\pgfqpoint{5.523490in}{1.127581in}}%
\pgfpathcurveto{\pgfqpoint{5.516358in}{1.127581in}}{\pgfqpoint{5.509516in}{1.124747in}}{\pgfqpoint{5.504472in}{1.119704in}}%
\pgfpathcurveto{\pgfqpoint{5.499429in}{1.114660in}}{\pgfqpoint{5.496595in}{1.107818in}}{\pgfqpoint{5.496595in}{1.100686in}}%
\pgfpathcurveto{\pgfqpoint{5.496595in}{1.093553in}}{\pgfqpoint{5.499429in}{1.086711in}}{\pgfqpoint{5.504472in}{1.081667in}}%
\pgfpathcurveto{\pgfqpoint{5.509516in}{1.076624in}}{\pgfqpoint{5.516358in}{1.073790in}}{\pgfqpoint{5.523490in}{1.073790in}}%
\pgfpathclose%
\pgfusepath{stroke,fill}%
\end{pgfscope}%
\begin{pgfscope}%
\pgfpathrectangle{\pgfqpoint{4.985294in}{0.500000in}}{\pgfqpoint{1.764706in}{1.700000in}}%
\pgfusepath{clip}%
\pgfsetbuttcap%
\pgfsetroundjoin%
\definecolor{currentfill}{rgb}{0.971202,0.827364,0.728520}%
\pgfsetfillcolor{currentfill}%
\pgfsetlinewidth{0.311001pt}%
\definecolor{currentstroke}{rgb}{1.000000,1.000000,1.000000}%
\pgfsetstrokecolor{currentstroke}%
\pgfsetdash{}{0pt}%
\pgfpathmoveto{\pgfqpoint{5.508381in}{1.626080in}}%
\pgfpathcurveto{\pgfqpoint{5.515514in}{1.626080in}}{\pgfqpoint{5.522356in}{1.628914in}}{\pgfqpoint{5.527400in}{1.633958in}}%
\pgfpathcurveto{\pgfqpoint{5.532443in}{1.639002in}}{\pgfqpoint{5.535277in}{1.645843in}}{\pgfqpoint{5.535277in}{1.652976in}}%
\pgfpathcurveto{\pgfqpoint{5.535277in}{1.660109in}}{\pgfqpoint{5.532443in}{1.666951in}}{\pgfqpoint{5.527400in}{1.671994in}}%
\pgfpathcurveto{\pgfqpoint{5.522356in}{1.677038in}}{\pgfqpoint{5.515514in}{1.679872in}}{\pgfqpoint{5.508381in}{1.679872in}}%
\pgfpathcurveto{\pgfqpoint{5.501249in}{1.679872in}}{\pgfqpoint{5.494407in}{1.677038in}}{\pgfqpoint{5.489363in}{1.671994in}}%
\pgfpathcurveto{\pgfqpoint{5.484320in}{1.666951in}}{\pgfqpoint{5.481486in}{1.660109in}}{\pgfqpoint{5.481486in}{1.652976in}}%
\pgfpathcurveto{\pgfqpoint{5.481486in}{1.645843in}}{\pgfqpoint{5.484320in}{1.639002in}}{\pgfqpoint{5.489363in}{1.633958in}}%
\pgfpathcurveto{\pgfqpoint{5.494407in}{1.628914in}}{\pgfqpoint{5.501249in}{1.626080in}}{\pgfqpoint{5.508381in}{1.626080in}}%
\pgfpathclose%
\pgfusepath{stroke,fill}%
\end{pgfscope}%
\begin{pgfscope}%
\pgfpathrectangle{\pgfqpoint{4.985294in}{0.500000in}}{\pgfqpoint{1.764706in}{1.700000in}}%
\pgfusepath{clip}%
\pgfsetbuttcap%
\pgfsetroundjoin%
\definecolor{currentfill}{rgb}{0.964679,0.682838,0.530002}%
\pgfsetfillcolor{currentfill}%
\pgfsetlinewidth{0.311001pt}%
\definecolor{currentstroke}{rgb}{1.000000,1.000000,1.000000}%
\pgfsetstrokecolor{currentstroke}%
\pgfsetdash{}{0pt}%
\pgfpathmoveto{\pgfqpoint{6.158284in}{0.900645in}}%
\pgfpathcurveto{\pgfqpoint{6.165417in}{0.900645in}}{\pgfqpoint{6.172258in}{0.903478in}}{\pgfqpoint{6.177302in}{0.908522in}}%
\pgfpathcurveto{\pgfqpoint{6.182346in}{0.913566in}}{\pgfqpoint{6.185180in}{0.920407in}}{\pgfqpoint{6.185180in}{0.927540in}}%
\pgfpathcurveto{\pgfqpoint{6.185180in}{0.934673in}}{\pgfqpoint{6.182346in}{0.941515in}}{\pgfqpoint{6.177302in}{0.946558in}}%
\pgfpathcurveto{\pgfqpoint{6.172258in}{0.951602in}}{\pgfqpoint{6.165417in}{0.954436in}}{\pgfqpoint{6.158284in}{0.954436in}}%
\pgfpathcurveto{\pgfqpoint{6.151151in}{0.954436in}}{\pgfqpoint{6.144309in}{0.951602in}}{\pgfqpoint{6.139266in}{0.946558in}}%
\pgfpathcurveto{\pgfqpoint{6.134222in}{0.941515in}}{\pgfqpoint{6.131388in}{0.934673in}}{\pgfqpoint{6.131388in}{0.927540in}}%
\pgfpathcurveto{\pgfqpoint{6.131388in}{0.920407in}}{\pgfqpoint{6.134222in}{0.913566in}}{\pgfqpoint{6.139266in}{0.908522in}}%
\pgfpathcurveto{\pgfqpoint{6.144309in}{0.903478in}}{\pgfqpoint{6.151151in}{0.900645in}}{\pgfqpoint{6.158284in}{0.900645in}}%
\pgfpathclose%
\pgfusepath{stroke,fill}%
\end{pgfscope}%
\begin{pgfscope}%
\pgfpathrectangle{\pgfqpoint{4.985294in}{0.500000in}}{\pgfqpoint{1.764706in}{1.700000in}}%
\pgfusepath{clip}%
\pgfsetbuttcap%
\pgfsetroundjoin%
\definecolor{currentfill}{rgb}{0.974412,0.862387,0.780156}%
\pgfsetfillcolor{currentfill}%
\pgfsetlinewidth{0.311001pt}%
\definecolor{currentstroke}{rgb}{1.000000,1.000000,1.000000}%
\pgfsetstrokecolor{currentstroke}%
\pgfsetdash{}{0pt}%
\pgfpathmoveto{\pgfqpoint{6.357737in}{1.211288in}}%
\pgfpathcurveto{\pgfqpoint{6.364870in}{1.211288in}}{\pgfqpoint{6.371711in}{1.214122in}}{\pgfqpoint{6.376755in}{1.219165in}}%
\pgfpathcurveto{\pgfqpoint{6.381799in}{1.224209in}}{\pgfqpoint{6.384633in}{1.231051in}}{\pgfqpoint{6.384633in}{1.238184in}}%
\pgfpathcurveto{\pgfqpoint{6.384633in}{1.245316in}}{\pgfqpoint{6.381799in}{1.252158in}}{\pgfqpoint{6.376755in}{1.257202in}}%
\pgfpathcurveto{\pgfqpoint{6.371711in}{1.262245in}}{\pgfqpoint{6.364870in}{1.265079in}}{\pgfqpoint{6.357737in}{1.265079in}}%
\pgfpathcurveto{\pgfqpoint{6.350604in}{1.265079in}}{\pgfqpoint{6.343762in}{1.262245in}}{\pgfqpoint{6.338719in}{1.257202in}}%
\pgfpathcurveto{\pgfqpoint{6.333675in}{1.252158in}}{\pgfqpoint{6.330841in}{1.245316in}}{\pgfqpoint{6.330841in}{1.238184in}}%
\pgfpathcurveto{\pgfqpoint{6.330841in}{1.231051in}}{\pgfqpoint{6.333675in}{1.224209in}}{\pgfqpoint{6.338719in}{1.219165in}}%
\pgfpathcurveto{\pgfqpoint{6.343762in}{1.214122in}}{\pgfqpoint{6.350604in}{1.211288in}}{\pgfqpoint{6.357737in}{1.211288in}}%
\pgfpathclose%
\pgfusepath{stroke,fill}%
\end{pgfscope}%
\begin{pgfscope}%
\pgfpathrectangle{\pgfqpoint{4.985294in}{0.500000in}}{\pgfqpoint{1.764706in}{1.700000in}}%
\pgfusepath{clip}%
\pgfsetbuttcap%
\pgfsetroundjoin%
\definecolor{currentfill}{rgb}{0.958791,0.526283,0.368909}%
\pgfsetfillcolor{currentfill}%
\pgfsetlinewidth{0.311001pt}%
\definecolor{currentstroke}{rgb}{1.000000,1.000000,1.000000}%
\pgfsetstrokecolor{currentstroke}%
\pgfsetdash{}{0pt}%
\pgfpathmoveto{\pgfqpoint{5.310059in}{1.466022in}}%
\pgfpathcurveto{\pgfqpoint{5.317192in}{1.466022in}}{\pgfqpoint{5.324034in}{1.468856in}}{\pgfqpoint{5.329077in}{1.473900in}}%
\pgfpathcurveto{\pgfqpoint{5.334121in}{1.478943in}}{\pgfqpoint{5.336955in}{1.485785in}}{\pgfqpoint{5.336955in}{1.492918in}}%
\pgfpathcurveto{\pgfqpoint{5.336955in}{1.500051in}}{\pgfqpoint{5.334121in}{1.506892in}}{\pgfqpoint{5.329077in}{1.511936in}}%
\pgfpathcurveto{\pgfqpoint{5.324034in}{1.516980in}}{\pgfqpoint{5.317192in}{1.519814in}}{\pgfqpoint{5.310059in}{1.519814in}}%
\pgfpathcurveto{\pgfqpoint{5.302926in}{1.519814in}}{\pgfqpoint{5.296085in}{1.516980in}}{\pgfqpoint{5.291041in}{1.511936in}}%
\pgfpathcurveto{\pgfqpoint{5.285997in}{1.506892in}}{\pgfqpoint{5.283163in}{1.500051in}}{\pgfqpoint{5.283163in}{1.492918in}}%
\pgfpathcurveto{\pgfqpoint{5.283163in}{1.485785in}}{\pgfqpoint{5.285997in}{1.478943in}}{\pgfqpoint{5.291041in}{1.473900in}}%
\pgfpathcurveto{\pgfqpoint{5.296085in}{1.468856in}}{\pgfqpoint{5.302926in}{1.466022in}}{\pgfqpoint{5.310059in}{1.466022in}}%
\pgfpathclose%
\pgfusepath{stroke,fill}%
\end{pgfscope}%
\begin{pgfscope}%
\pgfpathrectangle{\pgfqpoint{4.985294in}{0.500000in}}{\pgfqpoint{1.764706in}{1.700000in}}%
\pgfusepath{clip}%
\pgfsetbuttcap%
\pgfsetroundjoin%
\definecolor{currentfill}{rgb}{0.963379,0.625574,0.465113}%
\pgfsetfillcolor{currentfill}%
\pgfsetlinewidth{0.311001pt}%
\definecolor{currentstroke}{rgb}{1.000000,1.000000,1.000000}%
\pgfsetstrokecolor{currentstroke}%
\pgfsetdash{}{0pt}%
\pgfpathmoveto{\pgfqpoint{6.419013in}{1.213292in}}%
\pgfpathcurveto{\pgfqpoint{6.426145in}{1.213292in}}{\pgfqpoint{6.432987in}{1.216126in}}{\pgfqpoint{6.438031in}{1.221169in}}%
\pgfpathcurveto{\pgfqpoint{6.443074in}{1.226213in}}{\pgfqpoint{6.445908in}{1.233055in}}{\pgfqpoint{6.445908in}{1.240187in}}%
\pgfpathcurveto{\pgfqpoint{6.445908in}{1.247320in}}{\pgfqpoint{6.443074in}{1.254162in}}{\pgfqpoint{6.438031in}{1.259205in}}%
\pgfpathcurveto{\pgfqpoint{6.432987in}{1.264249in}}{\pgfqpoint{6.426145in}{1.267083in}}{\pgfqpoint{6.419013in}{1.267083in}}%
\pgfpathcurveto{\pgfqpoint{6.411880in}{1.267083in}}{\pgfqpoint{6.405038in}{1.264249in}}{\pgfqpoint{6.399994in}{1.259205in}}%
\pgfpathcurveto{\pgfqpoint{6.394951in}{1.254162in}}{\pgfqpoint{6.392117in}{1.247320in}}{\pgfqpoint{6.392117in}{1.240187in}}%
\pgfpathcurveto{\pgfqpoint{6.392117in}{1.233055in}}{\pgfqpoint{6.394951in}{1.226213in}}{\pgfqpoint{6.399994in}{1.221169in}}%
\pgfpathcurveto{\pgfqpoint{6.405038in}{1.216126in}}{\pgfqpoint{6.411880in}{1.213292in}}{\pgfqpoint{6.419013in}{1.213292in}}%
\pgfpathclose%
\pgfusepath{stroke,fill}%
\end{pgfscope}%
\begin{pgfscope}%
\pgfpathrectangle{\pgfqpoint{4.985294in}{0.500000in}}{\pgfqpoint{1.764706in}{1.700000in}}%
\pgfusepath{clip}%
\pgfsetbuttcap%
\pgfsetroundjoin%
\definecolor{currentfill}{rgb}{0.966560,0.756582,0.625273}%
\pgfsetfillcolor{currentfill}%
\pgfsetlinewidth{0.311001pt}%
\definecolor{currentstroke}{rgb}{1.000000,1.000000,1.000000}%
\pgfsetstrokecolor{currentstroke}%
\pgfsetdash{}{0pt}%
\pgfpathmoveto{\pgfqpoint{6.399468in}{1.265758in}}%
\pgfpathcurveto{\pgfqpoint{6.406601in}{1.265758in}}{\pgfqpoint{6.413443in}{1.268592in}}{\pgfqpoint{6.418486in}{1.273635in}}%
\pgfpathcurveto{\pgfqpoint{6.423530in}{1.278679in}}{\pgfqpoint{6.426364in}{1.285521in}}{\pgfqpoint{6.426364in}{1.292653in}}%
\pgfpathcurveto{\pgfqpoint{6.426364in}{1.299786in}}{\pgfqpoint{6.423530in}{1.306628in}}{\pgfqpoint{6.418486in}{1.311672in}}%
\pgfpathcurveto{\pgfqpoint{6.413443in}{1.316715in}}{\pgfqpoint{6.406601in}{1.319549in}}{\pgfqpoint{6.399468in}{1.319549in}}%
\pgfpathcurveto{\pgfqpoint{6.392335in}{1.319549in}}{\pgfqpoint{6.385494in}{1.316715in}}{\pgfqpoint{6.380450in}{1.311672in}}%
\pgfpathcurveto{\pgfqpoint{6.375406in}{1.306628in}}{\pgfqpoint{6.372573in}{1.299786in}}{\pgfqpoint{6.372573in}{1.292653in}}%
\pgfpathcurveto{\pgfqpoint{6.372573in}{1.285521in}}{\pgfqpoint{6.375406in}{1.278679in}}{\pgfqpoint{6.380450in}{1.273635in}}%
\pgfpathcurveto{\pgfqpoint{6.385494in}{1.268592in}}{\pgfqpoint{6.392335in}{1.265758in}}{\pgfqpoint{6.399468in}{1.265758in}}%
\pgfpathclose%
\pgfusepath{stroke,fill}%
\end{pgfscope}%
\begin{pgfscope}%
\pgfpathrectangle{\pgfqpoint{4.985294in}{0.500000in}}{\pgfqpoint{1.764706in}{1.700000in}}%
\pgfusepath{clip}%
\pgfsetbuttcap%
\pgfsetroundjoin%
\definecolor{currentfill}{rgb}{0.967398,0.774513,0.650573}%
\pgfsetfillcolor{currentfill}%
\pgfsetlinewidth{0.311001pt}%
\definecolor{currentstroke}{rgb}{1.000000,1.000000,1.000000}%
\pgfsetstrokecolor{currentstroke}%
\pgfsetdash{}{0pt}%
\pgfpathmoveto{\pgfqpoint{5.478161in}{1.671635in}}%
\pgfpathcurveto{\pgfqpoint{5.485294in}{1.671635in}}{\pgfqpoint{5.492136in}{1.674469in}}{\pgfqpoint{5.497180in}{1.679513in}}%
\pgfpathcurveto{\pgfqpoint{5.502223in}{1.684556in}}{\pgfqpoint{5.505057in}{1.691398in}}{\pgfqpoint{5.505057in}{1.698531in}}%
\pgfpathcurveto{\pgfqpoint{5.505057in}{1.705664in}}{\pgfqpoint{5.502223in}{1.712505in}}{\pgfqpoint{5.497180in}{1.717549in}}%
\pgfpathcurveto{\pgfqpoint{5.492136in}{1.722593in}}{\pgfqpoint{5.485294in}{1.725427in}}{\pgfqpoint{5.478161in}{1.725427in}}%
\pgfpathcurveto{\pgfqpoint{5.471029in}{1.725427in}}{\pgfqpoint{5.464187in}{1.722593in}}{\pgfqpoint{5.459143in}{1.717549in}}%
\pgfpathcurveto{\pgfqpoint{5.454100in}{1.712505in}}{\pgfqpoint{5.451266in}{1.705664in}}{\pgfqpoint{5.451266in}{1.698531in}}%
\pgfpathcurveto{\pgfqpoint{5.451266in}{1.691398in}}{\pgfqpoint{5.454100in}{1.684556in}}{\pgfqpoint{5.459143in}{1.679513in}}%
\pgfpathcurveto{\pgfqpoint{5.464187in}{1.674469in}}{\pgfqpoint{5.471029in}{1.671635in}}{\pgfqpoint{5.478161in}{1.671635in}}%
\pgfpathclose%
\pgfusepath{stroke,fill}%
\end{pgfscope}%
\begin{pgfscope}%
\pgfpathrectangle{\pgfqpoint{4.985294in}{0.500000in}}{\pgfqpoint{1.764706in}{1.700000in}}%
\pgfusepath{clip}%
\pgfsetbuttcap%
\pgfsetroundjoin%
\definecolor{currentfill}{rgb}{0.978376,0.897317,0.831308}%
\pgfsetfillcolor{currentfill}%
\pgfsetlinewidth{0.311001pt}%
\definecolor{currentstroke}{rgb}{1.000000,1.000000,1.000000}%
\pgfsetstrokecolor{currentstroke}%
\pgfsetdash{}{0pt}%
\pgfpathmoveto{\pgfqpoint{5.419222in}{1.180888in}}%
\pgfpathcurveto{\pgfqpoint{5.426355in}{1.180888in}}{\pgfqpoint{5.433197in}{1.183722in}}{\pgfqpoint{5.438240in}{1.188766in}}%
\pgfpathcurveto{\pgfqpoint{5.443284in}{1.193809in}}{\pgfqpoint{5.446118in}{1.200651in}}{\pgfqpoint{5.446118in}{1.207784in}}%
\pgfpathcurveto{\pgfqpoint{5.446118in}{1.214917in}}{\pgfqpoint{5.443284in}{1.221758in}}{\pgfqpoint{5.438240in}{1.226802in}}%
\pgfpathcurveto{\pgfqpoint{5.433197in}{1.231846in}}{\pgfqpoint{5.426355in}{1.234680in}}{\pgfqpoint{5.419222in}{1.234680in}}%
\pgfpathcurveto{\pgfqpoint{5.412089in}{1.234680in}}{\pgfqpoint{5.405248in}{1.231846in}}{\pgfqpoint{5.400204in}{1.226802in}}%
\pgfpathcurveto{\pgfqpoint{5.395160in}{1.221758in}}{\pgfqpoint{5.392326in}{1.214917in}}{\pgfqpoint{5.392326in}{1.207784in}}%
\pgfpathcurveto{\pgfqpoint{5.392326in}{1.200651in}}{\pgfqpoint{5.395160in}{1.193809in}}{\pgfqpoint{5.400204in}{1.188766in}}%
\pgfpathcurveto{\pgfqpoint{5.405248in}{1.183722in}}{\pgfqpoint{5.412089in}{1.180888in}}{\pgfqpoint{5.419222in}{1.180888in}}%
\pgfpathclose%
\pgfusepath{stroke,fill}%
\end{pgfscope}%
\begin{pgfscope}%
\pgfpathrectangle{\pgfqpoint{4.985294in}{0.500000in}}{\pgfqpoint{1.764706in}{1.700000in}}%
\pgfusepath{clip}%
\pgfsetbuttcap%
\pgfsetroundjoin%
\definecolor{currentfill}{rgb}{0.965592,0.726236,0.584384}%
\pgfsetfillcolor{currentfill}%
\pgfsetlinewidth{0.311001pt}%
\definecolor{currentstroke}{rgb}{1.000000,1.000000,1.000000}%
\pgfsetstrokecolor{currentstroke}%
\pgfsetdash{}{0pt}%
\pgfpathmoveto{\pgfqpoint{5.567139in}{1.556515in}}%
\pgfpathcurveto{\pgfqpoint{5.574272in}{1.556515in}}{\pgfqpoint{5.581113in}{1.559349in}}{\pgfqpoint{5.586157in}{1.564392in}}%
\pgfpathcurveto{\pgfqpoint{5.591201in}{1.569436in}}{\pgfqpoint{5.594034in}{1.576278in}}{\pgfqpoint{5.594034in}{1.583410in}}%
\pgfpathcurveto{\pgfqpoint{5.594034in}{1.590543in}}{\pgfqpoint{5.591201in}{1.597385in}}{\pgfqpoint{5.586157in}{1.602428in}}%
\pgfpathcurveto{\pgfqpoint{5.581113in}{1.607472in}}{\pgfqpoint{5.574272in}{1.610306in}}{\pgfqpoint{5.567139in}{1.610306in}}%
\pgfpathcurveto{\pgfqpoint{5.560006in}{1.610306in}}{\pgfqpoint{5.553164in}{1.607472in}}{\pgfqpoint{5.548121in}{1.602428in}}%
\pgfpathcurveto{\pgfqpoint{5.543077in}{1.597385in}}{\pgfqpoint{5.540243in}{1.590543in}}{\pgfqpoint{5.540243in}{1.583410in}}%
\pgfpathcurveto{\pgfqpoint{5.540243in}{1.576278in}}{\pgfqpoint{5.543077in}{1.569436in}}{\pgfqpoint{5.548121in}{1.564392in}}%
\pgfpathcurveto{\pgfqpoint{5.553164in}{1.559349in}}{\pgfqpoint{5.560006in}{1.556515in}}{\pgfqpoint{5.567139in}{1.556515in}}%
\pgfpathclose%
\pgfusepath{stroke,fill}%
\end{pgfscope}%
\begin{pgfscope}%
\pgfpathrectangle{\pgfqpoint{4.985294in}{0.500000in}}{\pgfqpoint{1.764706in}{1.700000in}}%
\pgfusepath{clip}%
\pgfsetbuttcap%
\pgfsetroundjoin%
\definecolor{currentfill}{rgb}{0.852817,0.156578,0.279098}%
\pgfsetfillcolor{currentfill}%
\pgfsetlinewidth{0.311001pt}%
\definecolor{currentstroke}{rgb}{1.000000,1.000000,1.000000}%
\pgfsetstrokecolor{currentstroke}%
\pgfsetdash{}{0pt}%
\pgfpathmoveto{\pgfqpoint{5.313907in}{1.624234in}}%
\pgfpathcurveto{\pgfqpoint{5.321040in}{1.624234in}}{\pgfqpoint{5.327882in}{1.627068in}}{\pgfqpoint{5.332925in}{1.632112in}}%
\pgfpathcurveto{\pgfqpoint{5.337969in}{1.637155in}}{\pgfqpoint{5.340803in}{1.643997in}}{\pgfqpoint{5.340803in}{1.651130in}}%
\pgfpathcurveto{\pgfqpoint{5.340803in}{1.658263in}}{\pgfqpoint{5.337969in}{1.665104in}}{\pgfqpoint{5.332925in}{1.670148in}}%
\pgfpathcurveto{\pgfqpoint{5.327882in}{1.675192in}}{\pgfqpoint{5.321040in}{1.678026in}}{\pgfqpoint{5.313907in}{1.678026in}}%
\pgfpathcurveto{\pgfqpoint{5.306774in}{1.678026in}}{\pgfqpoint{5.299933in}{1.675192in}}{\pgfqpoint{5.294889in}{1.670148in}}%
\pgfpathcurveto{\pgfqpoint{5.289845in}{1.665104in}}{\pgfqpoint{5.287012in}{1.658263in}}{\pgfqpoint{5.287012in}{1.651130in}}%
\pgfpathcurveto{\pgfqpoint{5.287012in}{1.643997in}}{\pgfqpoint{5.289845in}{1.637155in}}{\pgfqpoint{5.294889in}{1.632112in}}%
\pgfpathcurveto{\pgfqpoint{5.299933in}{1.627068in}}{\pgfqpoint{5.306774in}{1.624234in}}{\pgfqpoint{5.313907in}{1.624234in}}%
\pgfpathclose%
\pgfusepath{stroke,fill}%
\end{pgfscope}%
\begin{pgfscope}%
\pgfpathrectangle{\pgfqpoint{4.985294in}{0.500000in}}{\pgfqpoint{1.764706in}{1.700000in}}%
\pgfusepath{clip}%
\pgfsetbuttcap%
\pgfsetroundjoin%
\definecolor{currentfill}{rgb}{0.965302,0.713942,0.568499}%
\pgfsetfillcolor{currentfill}%
\pgfsetlinewidth{0.311001pt}%
\definecolor{currentstroke}{rgb}{1.000000,1.000000,1.000000}%
\pgfsetstrokecolor{currentstroke}%
\pgfsetdash{}{0pt}%
\pgfpathmoveto{\pgfqpoint{6.320532in}{1.627028in}}%
\pgfpathcurveto{\pgfqpoint{6.327665in}{1.627028in}}{\pgfqpoint{6.334507in}{1.629862in}}{\pgfqpoint{6.339550in}{1.634905in}}%
\pgfpathcurveto{\pgfqpoint{6.344594in}{1.639949in}}{\pgfqpoint{6.347428in}{1.646791in}}{\pgfqpoint{6.347428in}{1.653923in}}%
\pgfpathcurveto{\pgfqpoint{6.347428in}{1.661056in}}{\pgfqpoint{6.344594in}{1.667898in}}{\pgfqpoint{6.339550in}{1.672942in}}%
\pgfpathcurveto{\pgfqpoint{6.334507in}{1.677985in}}{\pgfqpoint{6.327665in}{1.680819in}}{\pgfqpoint{6.320532in}{1.680819in}}%
\pgfpathcurveto{\pgfqpoint{6.313399in}{1.680819in}}{\pgfqpoint{6.306558in}{1.677985in}}{\pgfqpoint{6.301514in}{1.672942in}}%
\pgfpathcurveto{\pgfqpoint{6.296471in}{1.667898in}}{\pgfqpoint{6.293637in}{1.661056in}}{\pgfqpoint{6.293637in}{1.653923in}}%
\pgfpathcurveto{\pgfqpoint{6.293637in}{1.646791in}}{\pgfqpoint{6.296471in}{1.639949in}}{\pgfqpoint{6.301514in}{1.634905in}}%
\pgfpathcurveto{\pgfqpoint{6.306558in}{1.629862in}}{\pgfqpoint{6.313399in}{1.627028in}}{\pgfqpoint{6.320532in}{1.627028in}}%
\pgfpathclose%
\pgfusepath{stroke,fill}%
\end{pgfscope}%
\begin{pgfscope}%
\pgfpathrectangle{\pgfqpoint{4.985294in}{0.500000in}}{\pgfqpoint{1.764706in}{1.700000in}}%
\pgfusepath{clip}%
\pgfsetbuttcap%
\pgfsetroundjoin%
\definecolor{currentfill}{rgb}{0.975644,0.874038,0.797253}%
\pgfsetfillcolor{currentfill}%
\pgfsetlinewidth{0.311001pt}%
\definecolor{currentstroke}{rgb}{1.000000,1.000000,1.000000}%
\pgfsetstrokecolor{currentstroke}%
\pgfsetdash{}{0pt}%
\pgfpathmoveto{\pgfqpoint{5.478956in}{1.124587in}}%
\pgfpathcurveto{\pgfqpoint{5.486089in}{1.124587in}}{\pgfqpoint{5.492931in}{1.127421in}}{\pgfqpoint{5.497974in}{1.132465in}}%
\pgfpathcurveto{\pgfqpoint{5.503018in}{1.137508in}}{\pgfqpoint{5.505852in}{1.144350in}}{\pgfqpoint{5.505852in}{1.151483in}}%
\pgfpathcurveto{\pgfqpoint{5.505852in}{1.158616in}}{\pgfqpoint{5.503018in}{1.165457in}}{\pgfqpoint{5.497974in}{1.170501in}}%
\pgfpathcurveto{\pgfqpoint{5.492931in}{1.175545in}}{\pgfqpoint{5.486089in}{1.178379in}}{\pgfqpoint{5.478956in}{1.178379in}}%
\pgfpathcurveto{\pgfqpoint{5.471823in}{1.178379in}}{\pgfqpoint{5.464982in}{1.175545in}}{\pgfqpoint{5.459938in}{1.170501in}}%
\pgfpathcurveto{\pgfqpoint{5.454894in}{1.165457in}}{\pgfqpoint{5.452060in}{1.158616in}}{\pgfqpoint{5.452060in}{1.151483in}}%
\pgfpathcurveto{\pgfqpoint{5.452060in}{1.144350in}}{\pgfqpoint{5.454894in}{1.137508in}}{\pgfqpoint{5.459938in}{1.132465in}}%
\pgfpathcurveto{\pgfqpoint{5.464982in}{1.127421in}}{\pgfqpoint{5.471823in}{1.124587in}}{\pgfqpoint{5.478956in}{1.124587in}}%
\pgfpathclose%
\pgfusepath{stroke,fill}%
\end{pgfscope}%
\begin{pgfscope}%
\pgfpathrectangle{\pgfqpoint{4.985294in}{0.500000in}}{\pgfqpoint{1.764706in}{1.700000in}}%
\pgfusepath{clip}%
\pgfsetbuttcap%
\pgfsetroundjoin%
\definecolor{currentfill}{rgb}{0.965928,0.738443,0.600540}%
\pgfsetfillcolor{currentfill}%
\pgfsetlinewidth{0.311001pt}%
\definecolor{currentstroke}{rgb}{1.000000,1.000000,1.000000}%
\pgfsetstrokecolor{currentstroke}%
\pgfsetdash{}{0pt}%
\pgfpathmoveto{\pgfqpoint{5.337340in}{1.400763in}}%
\pgfpathcurveto{\pgfqpoint{5.344472in}{1.400763in}}{\pgfqpoint{5.351314in}{1.403597in}}{\pgfqpoint{5.356358in}{1.408641in}}%
\pgfpathcurveto{\pgfqpoint{5.361401in}{1.413685in}}{\pgfqpoint{5.364235in}{1.420526in}}{\pgfqpoint{5.364235in}{1.427659in}}%
\pgfpathcurveto{\pgfqpoint{5.364235in}{1.434792in}}{\pgfqpoint{5.361401in}{1.441634in}}{\pgfqpoint{5.356358in}{1.446677in}}%
\pgfpathcurveto{\pgfqpoint{5.351314in}{1.451721in}}{\pgfqpoint{5.344472in}{1.454555in}}{\pgfqpoint{5.337340in}{1.454555in}}%
\pgfpathcurveto{\pgfqpoint{5.330207in}{1.454555in}}{\pgfqpoint{5.323365in}{1.451721in}}{\pgfqpoint{5.318321in}{1.446677in}}%
\pgfpathcurveto{\pgfqpoint{5.313278in}{1.441634in}}{\pgfqpoint{5.310444in}{1.434792in}}{\pgfqpoint{5.310444in}{1.427659in}}%
\pgfpathcurveto{\pgfqpoint{5.310444in}{1.420526in}}{\pgfqpoint{5.313278in}{1.413685in}}{\pgfqpoint{5.318321in}{1.408641in}}%
\pgfpathcurveto{\pgfqpoint{5.323365in}{1.403597in}}{\pgfqpoint{5.330207in}{1.400763in}}{\pgfqpoint{5.337340in}{1.400763in}}%
\pgfpathclose%
\pgfusepath{stroke,fill}%
\end{pgfscope}%
\begin{pgfscope}%
\pgfpathrectangle{\pgfqpoint{4.985294in}{0.500000in}}{\pgfqpoint{1.764706in}{1.700000in}}%
\pgfusepath{clip}%
\pgfsetbuttcap%
\pgfsetroundjoin%
\definecolor{currentfill}{rgb}{0.976961,0.885681,0.814303}%
\pgfsetfillcolor{currentfill}%
\pgfsetlinewidth{0.311001pt}%
\definecolor{currentstroke}{rgb}{1.000000,1.000000,1.000000}%
\pgfsetstrokecolor{currentstroke}%
\pgfsetdash{}{0pt}%
\pgfpathmoveto{\pgfqpoint{5.398930in}{1.209580in}}%
\pgfpathcurveto{\pgfqpoint{5.406063in}{1.209580in}}{\pgfqpoint{5.412905in}{1.212414in}}{\pgfqpoint{5.417948in}{1.217457in}}%
\pgfpathcurveto{\pgfqpoint{5.422992in}{1.222501in}}{\pgfqpoint{5.425826in}{1.229343in}}{\pgfqpoint{5.425826in}{1.236475in}}%
\pgfpathcurveto{\pgfqpoint{5.425826in}{1.243608in}}{\pgfqpoint{5.422992in}{1.250450in}}{\pgfqpoint{5.417948in}{1.255494in}}%
\pgfpathcurveto{\pgfqpoint{5.412905in}{1.260537in}}{\pgfqpoint{5.406063in}{1.263371in}}{\pgfqpoint{5.398930in}{1.263371in}}%
\pgfpathcurveto{\pgfqpoint{5.391797in}{1.263371in}}{\pgfqpoint{5.384956in}{1.260537in}}{\pgfqpoint{5.379912in}{1.255494in}}%
\pgfpathcurveto{\pgfqpoint{5.374868in}{1.250450in}}{\pgfqpoint{5.372035in}{1.243608in}}{\pgfqpoint{5.372035in}{1.236475in}}%
\pgfpathcurveto{\pgfqpoint{5.372035in}{1.229343in}}{\pgfqpoint{5.374868in}{1.222501in}}{\pgfqpoint{5.379912in}{1.217457in}}%
\pgfpathcurveto{\pgfqpoint{5.384956in}{1.212414in}}{\pgfqpoint{5.391797in}{1.209580in}}{\pgfqpoint{5.398930in}{1.209580in}}%
\pgfpathclose%
\pgfusepath{stroke,fill}%
\end{pgfscope}%
\begin{pgfscope}%
\pgfpathrectangle{\pgfqpoint{4.985294in}{0.500000in}}{\pgfqpoint{1.764706in}{1.700000in}}%
\pgfusepath{clip}%
\pgfsetbuttcap%
\pgfsetroundjoin%
\definecolor{currentfill}{rgb}{0.979891,0.908948,0.848279}%
\pgfsetfillcolor{currentfill}%
\pgfsetlinewidth{0.311001pt}%
\definecolor{currentstroke}{rgb}{1.000000,1.000000,1.000000}%
\pgfsetstrokecolor{currentstroke}%
\pgfsetdash{}{0pt}%
\pgfpathmoveto{\pgfqpoint{6.295347in}{1.208979in}}%
\pgfpathcurveto{\pgfqpoint{6.302480in}{1.208979in}}{\pgfqpoint{6.309322in}{1.211813in}}{\pgfqpoint{6.314365in}{1.216857in}}%
\pgfpathcurveto{\pgfqpoint{6.319409in}{1.221900in}}{\pgfqpoint{6.322243in}{1.228742in}}{\pgfqpoint{6.322243in}{1.235875in}}%
\pgfpathcurveto{\pgfqpoint{6.322243in}{1.243008in}}{\pgfqpoint{6.319409in}{1.249849in}}{\pgfqpoint{6.314365in}{1.254893in}}%
\pgfpathcurveto{\pgfqpoint{6.309322in}{1.259937in}}{\pgfqpoint{6.302480in}{1.262771in}}{\pgfqpoint{6.295347in}{1.262771in}}%
\pgfpathcurveto{\pgfqpoint{6.288214in}{1.262771in}}{\pgfqpoint{6.281373in}{1.259937in}}{\pgfqpoint{6.276329in}{1.254893in}}%
\pgfpathcurveto{\pgfqpoint{6.271286in}{1.249849in}}{\pgfqpoint{6.268452in}{1.243008in}}{\pgfqpoint{6.268452in}{1.235875in}}%
\pgfpathcurveto{\pgfqpoint{6.268452in}{1.228742in}}{\pgfqpoint{6.271286in}{1.221900in}}{\pgfqpoint{6.276329in}{1.216857in}}%
\pgfpathcurveto{\pgfqpoint{6.281373in}{1.211813in}}{\pgfqpoint{6.288214in}{1.208979in}}{\pgfqpoint{6.295347in}{1.208979in}}%
\pgfpathclose%
\pgfusepath{stroke,fill}%
\end{pgfscope}%
\begin{pgfscope}%
\pgfpathrectangle{\pgfqpoint{4.985294in}{0.500000in}}{\pgfqpoint{1.764706in}{1.700000in}}%
\pgfusepath{clip}%
\pgfsetbuttcap%
\pgfsetroundjoin%
\definecolor{currentfill}{rgb}{0.963884,0.644842,0.486120}%
\pgfsetfillcolor{currentfill}%
\pgfsetlinewidth{0.311001pt}%
\definecolor{currentstroke}{rgb}{1.000000,1.000000,1.000000}%
\pgfsetstrokecolor{currentstroke}%
\pgfsetdash{}{0pt}%
\pgfpathmoveto{\pgfqpoint{5.610678in}{1.673976in}}%
\pgfpathcurveto{\pgfqpoint{5.617811in}{1.673976in}}{\pgfqpoint{5.624652in}{1.676810in}}{\pgfqpoint{5.629696in}{1.681853in}}%
\pgfpathcurveto{\pgfqpoint{5.634740in}{1.686897in}}{\pgfqpoint{5.637573in}{1.693739in}}{\pgfqpoint{5.637573in}{1.700871in}}%
\pgfpathcurveto{\pgfqpoint{5.637573in}{1.708004in}}{\pgfqpoint{5.634740in}{1.714846in}}{\pgfqpoint{5.629696in}{1.719890in}}%
\pgfpathcurveto{\pgfqpoint{5.624652in}{1.724933in}}{\pgfqpoint{5.617811in}{1.727767in}}{\pgfqpoint{5.610678in}{1.727767in}}%
\pgfpathcurveto{\pgfqpoint{5.603545in}{1.727767in}}{\pgfqpoint{5.596703in}{1.724933in}}{\pgfqpoint{5.591660in}{1.719890in}}%
\pgfpathcurveto{\pgfqpoint{5.586616in}{1.714846in}}{\pgfqpoint{5.583782in}{1.708004in}}{\pgfqpoint{5.583782in}{1.700871in}}%
\pgfpathcurveto{\pgfqpoint{5.583782in}{1.693739in}}{\pgfqpoint{5.586616in}{1.686897in}}{\pgfqpoint{5.591660in}{1.681853in}}%
\pgfpathcurveto{\pgfqpoint{5.596703in}{1.676810in}}{\pgfqpoint{5.603545in}{1.673976in}}{\pgfqpoint{5.610678in}{1.673976in}}%
\pgfpathclose%
\pgfusepath{stroke,fill}%
\end{pgfscope}%
\begin{pgfscope}%
\pgfpathrectangle{\pgfqpoint{4.985294in}{0.500000in}}{\pgfqpoint{1.764706in}{1.700000in}}%
\pgfusepath{clip}%
\pgfsetbuttcap%
\pgfsetroundjoin%
\definecolor{currentfill}{rgb}{0.981377,0.920617,0.865369}%
\pgfsetfillcolor{currentfill}%
\pgfsetlinewidth{0.311001pt}%
\definecolor{currentstroke}{rgb}{1.000000,1.000000,1.000000}%
\pgfsetstrokecolor{currentstroke}%
\pgfsetdash{}{0pt}%
\pgfpathmoveto{\pgfqpoint{6.324646in}{1.355670in}}%
\pgfpathcurveto{\pgfqpoint{6.331779in}{1.355670in}}{\pgfqpoint{6.338620in}{1.358504in}}{\pgfqpoint{6.343664in}{1.363548in}}%
\pgfpathcurveto{\pgfqpoint{6.348708in}{1.368592in}}{\pgfqpoint{6.351542in}{1.375433in}}{\pgfqpoint{6.351542in}{1.382566in}}%
\pgfpathcurveto{\pgfqpoint{6.351542in}{1.389699in}}{\pgfqpoint{6.348708in}{1.396540in}}{\pgfqpoint{6.343664in}{1.401584in}}%
\pgfpathcurveto{\pgfqpoint{6.338620in}{1.406628in}}{\pgfqpoint{6.331779in}{1.409462in}}{\pgfqpoint{6.324646in}{1.409462in}}%
\pgfpathcurveto{\pgfqpoint{6.317513in}{1.409462in}}{\pgfqpoint{6.310671in}{1.406628in}}{\pgfqpoint{6.305628in}{1.401584in}}%
\pgfpathcurveto{\pgfqpoint{6.300584in}{1.396540in}}{\pgfqpoint{6.297750in}{1.389699in}}{\pgfqpoint{6.297750in}{1.382566in}}%
\pgfpathcurveto{\pgfqpoint{6.297750in}{1.375433in}}{\pgfqpoint{6.300584in}{1.368592in}}{\pgfqpoint{6.305628in}{1.363548in}}%
\pgfpathcurveto{\pgfqpoint{6.310671in}{1.358504in}}{\pgfqpoint{6.317513in}{1.355670in}}{\pgfqpoint{6.324646in}{1.355670in}}%
\pgfpathclose%
\pgfusepath{stroke,fill}%
\end{pgfscope}%
\begin{pgfscope}%
\pgfpathrectangle{\pgfqpoint{4.985294in}{0.500000in}}{\pgfqpoint{1.764706in}{1.700000in}}%
\pgfusepath{clip}%
\pgfsetbuttcap%
\pgfsetroundjoin%
\definecolor{currentfill}{rgb}{0.972201,0.839051,0.745789}%
\pgfsetfillcolor{currentfill}%
\pgfsetlinewidth{0.311001pt}%
\definecolor{currentstroke}{rgb}{1.000000,1.000000,1.000000}%
\pgfsetstrokecolor{currentstroke}%
\pgfsetdash{}{0pt}%
\pgfpathmoveto{\pgfqpoint{6.189570in}{1.655881in}}%
\pgfpathcurveto{\pgfqpoint{6.196703in}{1.655881in}}{\pgfqpoint{6.203545in}{1.658715in}}{\pgfqpoint{6.208589in}{1.663758in}}%
\pgfpathcurveto{\pgfqpoint{6.213632in}{1.668802in}}{\pgfqpoint{6.216466in}{1.675644in}}{\pgfqpoint{6.216466in}{1.682776in}}%
\pgfpathcurveto{\pgfqpoint{6.216466in}{1.689909in}}{\pgfqpoint{6.213632in}{1.696751in}}{\pgfqpoint{6.208589in}{1.701795in}}%
\pgfpathcurveto{\pgfqpoint{6.203545in}{1.706838in}}{\pgfqpoint{6.196703in}{1.709672in}}{\pgfqpoint{6.189570in}{1.709672in}}%
\pgfpathcurveto{\pgfqpoint{6.182438in}{1.709672in}}{\pgfqpoint{6.175596in}{1.706838in}}{\pgfqpoint{6.170552in}{1.701795in}}%
\pgfpathcurveto{\pgfqpoint{6.165509in}{1.696751in}}{\pgfqpoint{6.162675in}{1.689909in}}{\pgfqpoint{6.162675in}{1.682776in}}%
\pgfpathcurveto{\pgfqpoint{6.162675in}{1.675644in}}{\pgfqpoint{6.165509in}{1.668802in}}{\pgfqpoint{6.170552in}{1.663758in}}%
\pgfpathcurveto{\pgfqpoint{6.175596in}{1.658715in}}{\pgfqpoint{6.182438in}{1.655881in}}{\pgfqpoint{6.189570in}{1.655881in}}%
\pgfpathclose%
\pgfusepath{stroke,fill}%
\end{pgfscope}%
\begin{pgfscope}%
\pgfpathrectangle{\pgfqpoint{4.985294in}{0.500000in}}{\pgfqpoint{1.764706in}{1.700000in}}%
\pgfusepath{clip}%
\pgfsetbuttcap%
\pgfsetroundjoin%
\definecolor{currentfill}{rgb}{0.934351,0.329284,0.247753}%
\pgfsetfillcolor{currentfill}%
\pgfsetlinewidth{0.311001pt}%
\definecolor{currentstroke}{rgb}{1.000000,1.000000,1.000000}%
\pgfsetstrokecolor{currentstroke}%
\pgfsetdash{}{0pt}%
\pgfpathmoveto{\pgfqpoint{6.159666in}{1.825670in}}%
\pgfpathcurveto{\pgfqpoint{6.166799in}{1.825670in}}{\pgfqpoint{6.173641in}{1.828504in}}{\pgfqpoint{6.178684in}{1.833548in}}%
\pgfpathcurveto{\pgfqpoint{6.183728in}{1.838592in}}{\pgfqpoint{6.186562in}{1.845433in}}{\pgfqpoint{6.186562in}{1.852566in}}%
\pgfpathcurveto{\pgfqpoint{6.186562in}{1.859699in}}{\pgfqpoint{6.183728in}{1.866541in}}{\pgfqpoint{6.178684in}{1.871584in}}%
\pgfpathcurveto{\pgfqpoint{6.173641in}{1.876628in}}{\pgfqpoint{6.166799in}{1.879462in}}{\pgfqpoint{6.159666in}{1.879462in}}%
\pgfpathcurveto{\pgfqpoint{6.152533in}{1.879462in}}{\pgfqpoint{6.145692in}{1.876628in}}{\pgfqpoint{6.140648in}{1.871584in}}%
\pgfpathcurveto{\pgfqpoint{6.135604in}{1.866541in}}{\pgfqpoint{6.132770in}{1.859699in}}{\pgfqpoint{6.132770in}{1.852566in}}%
\pgfpathcurveto{\pgfqpoint{6.132770in}{1.845433in}}{\pgfqpoint{6.135604in}{1.838592in}}{\pgfqpoint{6.140648in}{1.833548in}}%
\pgfpathcurveto{\pgfqpoint{6.145692in}{1.828504in}}{\pgfqpoint{6.152533in}{1.825670in}}{\pgfqpoint{6.159666in}{1.825670in}}%
\pgfpathclose%
\pgfusepath{stroke,fill}%
\end{pgfscope}%
\begin{pgfscope}%
\pgfpathrectangle{\pgfqpoint{4.985294in}{0.500000in}}{\pgfqpoint{1.764706in}{1.700000in}}%
\pgfusepath{clip}%
\pgfsetbuttcap%
\pgfsetroundjoin%
\definecolor{currentfill}{rgb}{0.973271,0.850724,0.762998}%
\pgfsetfillcolor{currentfill}%
\pgfsetlinewidth{0.311001pt}%
\definecolor{currentstroke}{rgb}{1.000000,1.000000,1.000000}%
\pgfsetstrokecolor{currentstroke}%
\pgfsetdash{}{0pt}%
\pgfpathmoveto{\pgfqpoint{5.396985in}{1.488420in}}%
\pgfpathcurveto{\pgfqpoint{5.404118in}{1.488420in}}{\pgfqpoint{5.410959in}{1.491254in}}{\pgfqpoint{5.416003in}{1.496298in}}%
\pgfpathcurveto{\pgfqpoint{5.421047in}{1.501342in}}{\pgfqpoint{5.423881in}{1.508183in}}{\pgfqpoint{5.423881in}{1.515316in}}%
\pgfpathcurveto{\pgfqpoint{5.423881in}{1.522449in}}{\pgfqpoint{5.421047in}{1.529290in}}{\pgfqpoint{5.416003in}{1.534334in}}%
\pgfpathcurveto{\pgfqpoint{5.410959in}{1.539378in}}{\pgfqpoint{5.404118in}{1.542212in}}{\pgfqpoint{5.396985in}{1.542212in}}%
\pgfpathcurveto{\pgfqpoint{5.389852in}{1.542212in}}{\pgfqpoint{5.383010in}{1.539378in}}{\pgfqpoint{5.377967in}{1.534334in}}%
\pgfpathcurveto{\pgfqpoint{5.372923in}{1.529290in}}{\pgfqpoint{5.370089in}{1.522449in}}{\pgfqpoint{5.370089in}{1.515316in}}%
\pgfpathcurveto{\pgfqpoint{5.370089in}{1.508183in}}{\pgfqpoint{5.372923in}{1.501342in}}{\pgfqpoint{5.377967in}{1.496298in}}%
\pgfpathcurveto{\pgfqpoint{5.383010in}{1.491254in}}{\pgfqpoint{5.389852in}{1.488420in}}{\pgfqpoint{5.396985in}{1.488420in}}%
\pgfpathclose%
\pgfusepath{stroke,fill}%
\end{pgfscope}%
\begin{pgfscope}%
\pgfpathrectangle{\pgfqpoint{4.985294in}{0.500000in}}{\pgfqpoint{1.764706in}{1.700000in}}%
\pgfusepath{clip}%
\pgfsetbuttcap%
\pgfsetroundjoin%
\definecolor{currentfill}{rgb}{0.972726,0.844889,0.754401}%
\pgfsetfillcolor{currentfill}%
\pgfsetlinewidth{0.311001pt}%
\definecolor{currentstroke}{rgb}{1.000000,1.000000,1.000000}%
\pgfsetstrokecolor{currentstroke}%
\pgfsetdash{}{0pt}%
\pgfpathmoveto{\pgfqpoint{5.397425in}{1.499653in}}%
\pgfpathcurveto{\pgfqpoint{5.404558in}{1.499653in}}{\pgfqpoint{5.411399in}{1.502487in}}{\pgfqpoint{5.416443in}{1.507531in}}%
\pgfpathcurveto{\pgfqpoint{5.421487in}{1.512575in}}{\pgfqpoint{5.424321in}{1.519416in}}{\pgfqpoint{5.424321in}{1.526549in}}%
\pgfpathcurveto{\pgfqpoint{5.424321in}{1.533682in}}{\pgfqpoint{5.421487in}{1.540524in}}{\pgfqpoint{5.416443in}{1.545567in}}%
\pgfpathcurveto{\pgfqpoint{5.411399in}{1.550611in}}{\pgfqpoint{5.404558in}{1.553445in}}{\pgfqpoint{5.397425in}{1.553445in}}%
\pgfpathcurveto{\pgfqpoint{5.390292in}{1.553445in}}{\pgfqpoint{5.383450in}{1.550611in}}{\pgfqpoint{5.378407in}{1.545567in}}%
\pgfpathcurveto{\pgfqpoint{5.373363in}{1.540524in}}{\pgfqpoint{5.370529in}{1.533682in}}{\pgfqpoint{5.370529in}{1.526549in}}%
\pgfpathcurveto{\pgfqpoint{5.370529in}{1.519416in}}{\pgfqpoint{5.373363in}{1.512575in}}{\pgfqpoint{5.378407in}{1.507531in}}%
\pgfpathcurveto{\pgfqpoint{5.383450in}{1.502487in}}{\pgfqpoint{5.390292in}{1.499653in}}{\pgfqpoint{5.397425in}{1.499653in}}%
\pgfpathclose%
\pgfusepath{stroke,fill}%
\end{pgfscope}%
\begin{pgfscope}%
\pgfpathrectangle{\pgfqpoint{4.985294in}{0.500000in}}{\pgfqpoint{1.764706in}{1.700000in}}%
\pgfusepath{clip}%
\pgfsetbuttcap%
\pgfsetroundjoin%
\definecolor{currentfill}{rgb}{0.964920,0.695342,0.545192}%
\pgfsetfillcolor{currentfill}%
\pgfsetlinewidth{0.311001pt}%
\definecolor{currentstroke}{rgb}{1.000000,1.000000,1.000000}%
\pgfsetstrokecolor{currentstroke}%
\pgfsetdash{}{0pt}%
\pgfpathmoveto{\pgfqpoint{6.195360in}{1.335569in}}%
\pgfpathcurveto{\pgfqpoint{6.202493in}{1.335569in}}{\pgfqpoint{6.209335in}{1.338403in}}{\pgfqpoint{6.214378in}{1.343447in}}%
\pgfpathcurveto{\pgfqpoint{6.219422in}{1.348490in}}{\pgfqpoint{6.222256in}{1.355332in}}{\pgfqpoint{6.222256in}{1.362465in}}%
\pgfpathcurveto{\pgfqpoint{6.222256in}{1.369598in}}{\pgfqpoint{6.219422in}{1.376439in}}{\pgfqpoint{6.214378in}{1.381483in}}%
\pgfpathcurveto{\pgfqpoint{6.209335in}{1.386527in}}{\pgfqpoint{6.202493in}{1.389360in}}{\pgfqpoint{6.195360in}{1.389360in}}%
\pgfpathcurveto{\pgfqpoint{6.188227in}{1.389360in}}{\pgfqpoint{6.181386in}{1.386527in}}{\pgfqpoint{6.176342in}{1.381483in}}%
\pgfpathcurveto{\pgfqpoint{6.171298in}{1.376439in}}{\pgfqpoint{6.168464in}{1.369598in}}{\pgfqpoint{6.168464in}{1.362465in}}%
\pgfpathcurveto{\pgfqpoint{6.168464in}{1.355332in}}{\pgfqpoint{6.171298in}{1.348490in}}{\pgfqpoint{6.176342in}{1.343447in}}%
\pgfpathcurveto{\pgfqpoint{6.181386in}{1.338403in}}{\pgfqpoint{6.188227in}{1.335569in}}{\pgfqpoint{6.195360in}{1.335569in}}%
\pgfpathclose%
\pgfusepath{stroke,fill}%
\end{pgfscope}%
\begin{pgfscope}%
\pgfpathrectangle{\pgfqpoint{4.985294in}{0.500000in}}{\pgfqpoint{1.764706in}{1.700000in}}%
\pgfusepath{clip}%
\pgfsetbuttcap%
\pgfsetroundjoin%
\definecolor{currentfill}{rgb}{0.977657,0.891500,0.822809}%
\pgfsetfillcolor{currentfill}%
\pgfsetlinewidth{0.311001pt}%
\definecolor{currentstroke}{rgb}{1.000000,1.000000,1.000000}%
\pgfsetstrokecolor{currentstroke}%
\pgfsetdash{}{0pt}%
\pgfpathmoveto{\pgfqpoint{5.423726in}{1.163102in}}%
\pgfpathcurveto{\pgfqpoint{5.430859in}{1.163102in}}{\pgfqpoint{5.437701in}{1.165936in}}{\pgfqpoint{5.442745in}{1.170980in}}%
\pgfpathcurveto{\pgfqpoint{5.447788in}{1.176024in}}{\pgfqpoint{5.450622in}{1.182865in}}{\pgfqpoint{5.450622in}{1.189998in}}%
\pgfpathcurveto{\pgfqpoint{5.450622in}{1.197131in}}{\pgfqpoint{5.447788in}{1.203973in}}{\pgfqpoint{5.442745in}{1.209016in}}%
\pgfpathcurveto{\pgfqpoint{5.437701in}{1.214060in}}{\pgfqpoint{5.430859in}{1.216894in}}{\pgfqpoint{5.423726in}{1.216894in}}%
\pgfpathcurveto{\pgfqpoint{5.416594in}{1.216894in}}{\pgfqpoint{5.409752in}{1.214060in}}{\pgfqpoint{5.404708in}{1.209016in}}%
\pgfpathcurveto{\pgfqpoint{5.399665in}{1.203973in}}{\pgfqpoint{5.396831in}{1.197131in}}{\pgfqpoint{5.396831in}{1.189998in}}%
\pgfpathcurveto{\pgfqpoint{5.396831in}{1.182865in}}{\pgfqpoint{5.399665in}{1.176024in}}{\pgfqpoint{5.404708in}{1.170980in}}%
\pgfpathcurveto{\pgfqpoint{5.409752in}{1.165936in}}{\pgfqpoint{5.416594in}{1.163102in}}{\pgfqpoint{5.423726in}{1.163102in}}%
\pgfpathclose%
\pgfusepath{stroke,fill}%
\end{pgfscope}%
\begin{pgfscope}%
\pgfpathrectangle{\pgfqpoint{4.985294in}{0.500000in}}{\pgfqpoint{1.764706in}{1.700000in}}%
\pgfusepath{clip}%
\pgfsetbuttcap%
\pgfsetroundjoin%
\definecolor{currentfill}{rgb}{0.963728,0.638439,0.479050}%
\pgfsetfillcolor{currentfill}%
\pgfsetlinewidth{0.311001pt}%
\definecolor{currentstroke}{rgb}{1.000000,1.000000,1.000000}%
\pgfsetstrokecolor{currentstroke}%
\pgfsetdash{}{0pt}%
\pgfpathmoveto{\pgfqpoint{5.563349in}{0.877839in}}%
\pgfpathcurveto{\pgfqpoint{5.570482in}{0.877839in}}{\pgfqpoint{5.577324in}{0.880673in}}{\pgfqpoint{5.582368in}{0.885717in}}%
\pgfpathcurveto{\pgfqpoint{5.587411in}{0.890761in}}{\pgfqpoint{5.590245in}{0.897602in}}{\pgfqpoint{5.590245in}{0.904735in}}%
\pgfpathcurveto{\pgfqpoint{5.590245in}{0.911868in}}{\pgfqpoint{5.587411in}{0.918709in}}{\pgfqpoint{5.582368in}{0.923753in}}%
\pgfpathcurveto{\pgfqpoint{5.577324in}{0.928797in}}{\pgfqpoint{5.570482in}{0.931631in}}{\pgfqpoint{5.563349in}{0.931631in}}%
\pgfpathcurveto{\pgfqpoint{5.556217in}{0.931631in}}{\pgfqpoint{5.549375in}{0.928797in}}{\pgfqpoint{5.544331in}{0.923753in}}%
\pgfpathcurveto{\pgfqpoint{5.539288in}{0.918709in}}{\pgfqpoint{5.536454in}{0.911868in}}{\pgfqpoint{5.536454in}{0.904735in}}%
\pgfpathcurveto{\pgfqpoint{5.536454in}{0.897602in}}{\pgfqpoint{5.539288in}{0.890761in}}{\pgfqpoint{5.544331in}{0.885717in}}%
\pgfpathcurveto{\pgfqpoint{5.549375in}{0.880673in}}{\pgfqpoint{5.556217in}{0.877839in}}{\pgfqpoint{5.563349in}{0.877839in}}%
\pgfpathclose%
\pgfusepath{stroke,fill}%
\end{pgfscope}%
\begin{pgfscope}%
\pgfpathrectangle{\pgfqpoint{4.985294in}{0.500000in}}{\pgfqpoint{1.764706in}{1.700000in}}%
\pgfusepath{clip}%
\pgfsetbuttcap%
\pgfsetroundjoin%
\definecolor{currentfill}{rgb}{0.971694,0.833208,0.737161}%
\pgfsetfillcolor{currentfill}%
\pgfsetlinewidth{0.311001pt}%
\definecolor{currentstroke}{rgb}{1.000000,1.000000,1.000000}%
\pgfsetstrokecolor{currentstroke}%
\pgfsetdash{}{0pt}%
\pgfpathmoveto{\pgfqpoint{6.228496in}{1.507864in}}%
\pgfpathcurveto{\pgfqpoint{6.235629in}{1.507864in}}{\pgfqpoint{6.242471in}{1.510698in}}{\pgfqpoint{6.247514in}{1.515742in}}%
\pgfpathcurveto{\pgfqpoint{6.252558in}{1.520785in}}{\pgfqpoint{6.255392in}{1.527627in}}{\pgfqpoint{6.255392in}{1.534760in}}%
\pgfpathcurveto{\pgfqpoint{6.255392in}{1.541893in}}{\pgfqpoint{6.252558in}{1.548734in}}{\pgfqpoint{6.247514in}{1.553778in}}%
\pgfpathcurveto{\pgfqpoint{6.242471in}{1.558822in}}{\pgfqpoint{6.235629in}{1.561656in}}{\pgfqpoint{6.228496in}{1.561656in}}%
\pgfpathcurveto{\pgfqpoint{6.221363in}{1.561656in}}{\pgfqpoint{6.214522in}{1.558822in}}{\pgfqpoint{6.209478in}{1.553778in}}%
\pgfpathcurveto{\pgfqpoint{6.204434in}{1.548734in}}{\pgfqpoint{6.201601in}{1.541893in}}{\pgfqpoint{6.201601in}{1.534760in}}%
\pgfpathcurveto{\pgfqpoint{6.201601in}{1.527627in}}{\pgfqpoint{6.204434in}{1.520785in}}{\pgfqpoint{6.209478in}{1.515742in}}%
\pgfpathcurveto{\pgfqpoint{6.214522in}{1.510698in}}{\pgfqpoint{6.221363in}{1.507864in}}{\pgfqpoint{6.228496in}{1.507864in}}%
\pgfpathclose%
\pgfusepath{stroke,fill}%
\end{pgfscope}%
\begin{pgfscope}%
\pgfpathrectangle{\pgfqpoint{4.985294in}{0.500000in}}{\pgfqpoint{1.764706in}{1.700000in}}%
\pgfusepath{clip}%
\pgfsetbuttcap%
\pgfsetroundjoin%
\definecolor{currentfill}{rgb}{0.963379,0.625574,0.465113}%
\pgfsetfillcolor{currentfill}%
\pgfsetlinewidth{0.311001pt}%
\definecolor{currentstroke}{rgb}{1.000000,1.000000,1.000000}%
\pgfsetstrokecolor{currentstroke}%
\pgfsetdash{}{0pt}%
\pgfpathmoveto{\pgfqpoint{6.133981in}{1.082475in}}%
\pgfpathcurveto{\pgfqpoint{6.141114in}{1.082475in}}{\pgfqpoint{6.147955in}{1.085309in}}{\pgfqpoint{6.152999in}{1.090353in}}%
\pgfpathcurveto{\pgfqpoint{6.158043in}{1.095397in}}{\pgfqpoint{6.160877in}{1.102238in}}{\pgfqpoint{6.160877in}{1.109371in}}%
\pgfpathcurveto{\pgfqpoint{6.160877in}{1.116504in}}{\pgfqpoint{6.158043in}{1.123346in}}{\pgfqpoint{6.152999in}{1.128389in}}%
\pgfpathcurveto{\pgfqpoint{6.147955in}{1.133433in}}{\pgfqpoint{6.141114in}{1.136267in}}{\pgfqpoint{6.133981in}{1.136267in}}%
\pgfpathcurveto{\pgfqpoint{6.126848in}{1.136267in}}{\pgfqpoint{6.120007in}{1.133433in}}{\pgfqpoint{6.114963in}{1.128389in}}%
\pgfpathcurveto{\pgfqpoint{6.109919in}{1.123346in}}{\pgfqpoint{6.107085in}{1.116504in}}{\pgfqpoint{6.107085in}{1.109371in}}%
\pgfpathcurveto{\pgfqpoint{6.107085in}{1.102238in}}{\pgfqpoint{6.109919in}{1.095397in}}{\pgfqpoint{6.114963in}{1.090353in}}%
\pgfpathcurveto{\pgfqpoint{6.120007in}{1.085309in}}{\pgfqpoint{6.126848in}{1.082475in}}{\pgfqpoint{6.133981in}{1.082475in}}%
\pgfpathclose%
\pgfusepath{stroke,fill}%
\end{pgfscope}%
\begin{pgfscope}%
\pgfpathrectangle{\pgfqpoint{4.985294in}{0.500000in}}{\pgfqpoint{1.764706in}{1.700000in}}%
\pgfusepath{clip}%
\pgfsetbuttcap%
\pgfsetroundjoin%
\definecolor{currentfill}{rgb}{0.972726,0.844889,0.754401}%
\pgfsetfillcolor{currentfill}%
\pgfsetlinewidth{0.311001pt}%
\definecolor{currentstroke}{rgb}{1.000000,1.000000,1.000000}%
\pgfsetstrokecolor{currentstroke}%
\pgfsetdash{}{0pt}%
\pgfpathmoveto{\pgfqpoint{5.501854in}{1.114319in}}%
\pgfpathcurveto{\pgfqpoint{5.508987in}{1.114319in}}{\pgfqpoint{5.515829in}{1.117153in}}{\pgfqpoint{5.520873in}{1.122197in}}%
\pgfpathcurveto{\pgfqpoint{5.525916in}{1.127240in}}{\pgfqpoint{5.528750in}{1.134082in}}{\pgfqpoint{5.528750in}{1.141215in}}%
\pgfpathcurveto{\pgfqpoint{5.528750in}{1.148348in}}{\pgfqpoint{5.525916in}{1.155189in}}{\pgfqpoint{5.520873in}{1.160233in}}%
\pgfpathcurveto{\pgfqpoint{5.515829in}{1.165277in}}{\pgfqpoint{5.508987in}{1.168111in}}{\pgfqpoint{5.501854in}{1.168111in}}%
\pgfpathcurveto{\pgfqpoint{5.494722in}{1.168111in}}{\pgfqpoint{5.487880in}{1.165277in}}{\pgfqpoint{5.482836in}{1.160233in}}%
\pgfpathcurveto{\pgfqpoint{5.477793in}{1.155189in}}{\pgfqpoint{5.474959in}{1.148348in}}{\pgfqpoint{5.474959in}{1.141215in}}%
\pgfpathcurveto{\pgfqpoint{5.474959in}{1.134082in}}{\pgfqpoint{5.477793in}{1.127240in}}{\pgfqpoint{5.482836in}{1.122197in}}%
\pgfpathcurveto{\pgfqpoint{5.487880in}{1.117153in}}{\pgfqpoint{5.494722in}{1.114319in}}{\pgfqpoint{5.501854in}{1.114319in}}%
\pgfpathclose%
\pgfusepath{stroke,fill}%
\end{pgfscope}%
\begin{pgfscope}%
\pgfpathrectangle{\pgfqpoint{4.985294in}{0.500000in}}{\pgfqpoint{1.764706in}{1.700000in}}%
\pgfusepath{clip}%
\pgfsetbuttcap%
\pgfsetroundjoin%
\definecolor{currentfill}{rgb}{0.979891,0.908948,0.848279}%
\pgfsetfillcolor{currentfill}%
\pgfsetlinewidth{0.311001pt}%
\definecolor{currentstroke}{rgb}{1.000000,1.000000,1.000000}%
\pgfsetstrokecolor{currentstroke}%
\pgfsetdash{}{0pt}%
\pgfpathmoveto{\pgfqpoint{6.329363in}{1.240070in}}%
\pgfpathcurveto{\pgfqpoint{6.336496in}{1.240070in}}{\pgfqpoint{6.343337in}{1.242904in}}{\pgfqpoint{6.348381in}{1.247948in}}%
\pgfpathcurveto{\pgfqpoint{6.353424in}{1.252991in}}{\pgfqpoint{6.356258in}{1.259833in}}{\pgfqpoint{6.356258in}{1.266966in}}%
\pgfpathcurveto{\pgfqpoint{6.356258in}{1.274099in}}{\pgfqpoint{6.353424in}{1.280940in}}{\pgfqpoint{6.348381in}{1.285984in}}%
\pgfpathcurveto{\pgfqpoint{6.343337in}{1.291028in}}{\pgfqpoint{6.336496in}{1.293862in}}{\pgfqpoint{6.329363in}{1.293862in}}%
\pgfpathcurveto{\pgfqpoint{6.322230in}{1.293862in}}{\pgfqpoint{6.315388in}{1.291028in}}{\pgfqpoint{6.310345in}{1.285984in}}%
\pgfpathcurveto{\pgfqpoint{6.305301in}{1.280940in}}{\pgfqpoint{6.302467in}{1.274099in}}{\pgfqpoint{6.302467in}{1.266966in}}%
\pgfpathcurveto{\pgfqpoint{6.302467in}{1.259833in}}{\pgfqpoint{6.305301in}{1.252991in}}{\pgfqpoint{6.310345in}{1.247948in}}%
\pgfpathcurveto{\pgfqpoint{6.315388in}{1.242904in}}{\pgfqpoint{6.322230in}{1.240070in}}{\pgfqpoint{6.329363in}{1.240070in}}%
\pgfpathclose%
\pgfusepath{stroke,fill}%
\end{pgfscope}%
\begin{pgfscope}%
\pgfpathrectangle{\pgfqpoint{4.985294in}{0.500000in}}{\pgfqpoint{1.764706in}{1.700000in}}%
\pgfusepath{clip}%
\pgfsetbuttcap%
\pgfsetroundjoin%
\definecolor{currentfill}{rgb}{0.980678,0.914765,0.856766}%
\pgfsetfillcolor{currentfill}%
\pgfsetlinewidth{0.311001pt}%
\definecolor{currentstroke}{rgb}{1.000000,1.000000,1.000000}%
\pgfsetstrokecolor{currentstroke}%
\pgfsetdash{}{0pt}%
\pgfpathmoveto{\pgfqpoint{5.420786in}{1.307152in}}%
\pgfpathcurveto{\pgfqpoint{5.427919in}{1.307152in}}{\pgfqpoint{5.434761in}{1.309986in}}{\pgfqpoint{5.439805in}{1.315030in}}%
\pgfpathcurveto{\pgfqpoint{5.444848in}{1.320074in}}{\pgfqpoint{5.447682in}{1.326915in}}{\pgfqpoint{5.447682in}{1.334048in}}%
\pgfpathcurveto{\pgfqpoint{5.447682in}{1.341181in}}{\pgfqpoint{5.444848in}{1.348023in}}{\pgfqpoint{5.439805in}{1.353066in}}%
\pgfpathcurveto{\pgfqpoint{5.434761in}{1.358110in}}{\pgfqpoint{5.427919in}{1.360944in}}{\pgfqpoint{5.420786in}{1.360944in}}%
\pgfpathcurveto{\pgfqpoint{5.413654in}{1.360944in}}{\pgfqpoint{5.406812in}{1.358110in}}{\pgfqpoint{5.401768in}{1.353066in}}%
\pgfpathcurveto{\pgfqpoint{5.396725in}{1.348023in}}{\pgfqpoint{5.393891in}{1.341181in}}{\pgfqpoint{5.393891in}{1.334048in}}%
\pgfpathcurveto{\pgfqpoint{5.393891in}{1.326915in}}{\pgfqpoint{5.396725in}{1.320074in}}{\pgfqpoint{5.401768in}{1.315030in}}%
\pgfpathcurveto{\pgfqpoint{5.406812in}{1.309986in}}{\pgfqpoint{5.413654in}{1.307152in}}{\pgfqpoint{5.420786in}{1.307152in}}%
\pgfpathclose%
\pgfusepath{stroke,fill}%
\end{pgfscope}%
\begin{pgfscope}%
\pgfpathrectangle{\pgfqpoint{4.985294in}{0.500000in}}{\pgfqpoint{1.764706in}{1.700000in}}%
\pgfusepath{clip}%
\pgfsetbuttcap%
\pgfsetroundjoin%
\definecolor{currentfill}{rgb}{0.977657,0.891500,0.822809}%
\pgfsetfillcolor{currentfill}%
\pgfsetlinewidth{0.311001pt}%
\definecolor{currentstroke}{rgb}{1.000000,1.000000,1.000000}%
\pgfsetstrokecolor{currentstroke}%
\pgfsetdash{}{0pt}%
\pgfpathmoveto{\pgfqpoint{6.300245in}{1.140106in}}%
\pgfpathcurveto{\pgfqpoint{6.307377in}{1.140106in}}{\pgfqpoint{6.314219in}{1.142940in}}{\pgfqpoint{6.319263in}{1.147983in}}%
\pgfpathcurveto{\pgfqpoint{6.324306in}{1.153027in}}{\pgfqpoint{6.327140in}{1.159869in}}{\pgfqpoint{6.327140in}{1.167001in}}%
\pgfpathcurveto{\pgfqpoint{6.327140in}{1.174134in}}{\pgfqpoint{6.324306in}{1.180976in}}{\pgfqpoint{6.319263in}{1.186019in}}%
\pgfpathcurveto{\pgfqpoint{6.314219in}{1.191063in}}{\pgfqpoint{6.307377in}{1.193897in}}{\pgfqpoint{6.300245in}{1.193897in}}%
\pgfpathcurveto{\pgfqpoint{6.293112in}{1.193897in}}{\pgfqpoint{6.286270in}{1.191063in}}{\pgfqpoint{6.281226in}{1.186019in}}%
\pgfpathcurveto{\pgfqpoint{6.276183in}{1.180976in}}{\pgfqpoint{6.273349in}{1.174134in}}{\pgfqpoint{6.273349in}{1.167001in}}%
\pgfpathcurveto{\pgfqpoint{6.273349in}{1.159869in}}{\pgfqpoint{6.276183in}{1.153027in}}{\pgfqpoint{6.281226in}{1.147983in}}%
\pgfpathcurveto{\pgfqpoint{6.286270in}{1.142940in}}{\pgfqpoint{6.293112in}{1.140106in}}{\pgfqpoint{6.300245in}{1.140106in}}%
\pgfpathclose%
\pgfusepath{stroke,fill}%
\end{pgfscope}%
\begin{pgfscope}%
\pgfpathrectangle{\pgfqpoint{4.985294in}{0.500000in}}{\pgfqpoint{1.764706in}{1.700000in}}%
\pgfusepath{clip}%
\pgfsetbuttcap%
\pgfsetroundjoin%
\definecolor{currentfill}{rgb}{0.963884,0.644842,0.486120}%
\pgfsetfillcolor{currentfill}%
\pgfsetlinewidth{0.311001pt}%
\definecolor{currentstroke}{rgb}{1.000000,1.000000,1.000000}%
\pgfsetstrokecolor{currentstroke}%
\pgfsetdash{}{0pt}%
\pgfpathmoveto{\pgfqpoint{6.407219in}{1.447120in}}%
\pgfpathcurveto{\pgfqpoint{6.414352in}{1.447120in}}{\pgfqpoint{6.421194in}{1.449954in}}{\pgfqpoint{6.426238in}{1.454998in}}%
\pgfpathcurveto{\pgfqpoint{6.431281in}{1.460041in}}{\pgfqpoint{6.434115in}{1.466883in}}{\pgfqpoint{6.434115in}{1.474016in}}%
\pgfpathcurveto{\pgfqpoint{6.434115in}{1.481149in}}{\pgfqpoint{6.431281in}{1.487990in}}{\pgfqpoint{6.426238in}{1.493034in}}%
\pgfpathcurveto{\pgfqpoint{6.421194in}{1.498078in}}{\pgfqpoint{6.414352in}{1.500911in}}{\pgfqpoint{6.407219in}{1.500911in}}%
\pgfpathcurveto{\pgfqpoint{6.400087in}{1.500911in}}{\pgfqpoint{6.393245in}{1.498078in}}{\pgfqpoint{6.388201in}{1.493034in}}%
\pgfpathcurveto{\pgfqpoint{6.383158in}{1.487990in}}{\pgfqpoint{6.380324in}{1.481149in}}{\pgfqpoint{6.380324in}{1.474016in}}%
\pgfpathcurveto{\pgfqpoint{6.380324in}{1.466883in}}{\pgfqpoint{6.383158in}{1.460041in}}{\pgfqpoint{6.388201in}{1.454998in}}%
\pgfpathcurveto{\pgfqpoint{6.393245in}{1.449954in}}{\pgfqpoint{6.400087in}{1.447120in}}{\pgfqpoint{6.407219in}{1.447120in}}%
\pgfpathclose%
\pgfusepath{stroke,fill}%
\end{pgfscope}%
\begin{pgfscope}%
\pgfpathrectangle{\pgfqpoint{4.985294in}{0.500000in}}{\pgfqpoint{1.764706in}{1.700000in}}%
\pgfusepath{clip}%
\pgfsetbuttcap%
\pgfsetroundjoin%
\definecolor{currentfill}{rgb}{0.976961,0.885681,0.814303}%
\pgfsetfillcolor{currentfill}%
\pgfsetlinewidth{0.311001pt}%
\definecolor{currentstroke}{rgb}{1.000000,1.000000,1.000000}%
\pgfsetstrokecolor{currentstroke}%
\pgfsetdash{}{0pt}%
\pgfpathmoveto{\pgfqpoint{5.460506in}{1.447353in}}%
\pgfpathcurveto{\pgfqpoint{5.467639in}{1.447353in}}{\pgfqpoint{5.474481in}{1.450187in}}{\pgfqpoint{5.479524in}{1.455231in}}%
\pgfpathcurveto{\pgfqpoint{5.484568in}{1.460274in}}{\pgfqpoint{5.487402in}{1.467116in}}{\pgfqpoint{5.487402in}{1.474249in}}%
\pgfpathcurveto{\pgfqpoint{5.487402in}{1.481381in}}{\pgfqpoint{5.484568in}{1.488223in}}{\pgfqpoint{5.479524in}{1.493267in}}%
\pgfpathcurveto{\pgfqpoint{5.474481in}{1.498310in}}{\pgfqpoint{5.467639in}{1.501144in}}{\pgfqpoint{5.460506in}{1.501144in}}%
\pgfpathcurveto{\pgfqpoint{5.453373in}{1.501144in}}{\pgfqpoint{5.446532in}{1.498310in}}{\pgfqpoint{5.441488in}{1.493267in}}%
\pgfpathcurveto{\pgfqpoint{5.436444in}{1.488223in}}{\pgfqpoint{5.433610in}{1.481381in}}{\pgfqpoint{5.433610in}{1.474249in}}%
\pgfpathcurveto{\pgfqpoint{5.433610in}{1.467116in}}{\pgfqpoint{5.436444in}{1.460274in}}{\pgfqpoint{5.441488in}{1.455231in}}%
\pgfpathcurveto{\pgfqpoint{5.446532in}{1.450187in}}{\pgfqpoint{5.453373in}{1.447353in}}{\pgfqpoint{5.460506in}{1.447353in}}%
\pgfpathclose%
\pgfusepath{stroke,fill}%
\end{pgfscope}%
\begin{pgfscope}%
\pgfpathrectangle{\pgfqpoint{4.985294in}{0.500000in}}{\pgfqpoint{1.764706in}{1.700000in}}%
\pgfusepath{clip}%
\pgfsetbuttcap%
\pgfsetroundjoin%
\definecolor{currentfill}{rgb}{0.857426,0.162258,0.276275}%
\pgfsetfillcolor{currentfill}%
\pgfsetlinewidth{0.311001pt}%
\definecolor{currentstroke}{rgb}{1.000000,1.000000,1.000000}%
\pgfsetstrokecolor{currentstroke}%
\pgfsetdash{}{0pt}%
\pgfpathmoveto{\pgfqpoint{6.035908in}{1.826743in}}%
\pgfpathcurveto{\pgfqpoint{6.043041in}{1.826743in}}{\pgfqpoint{6.049882in}{1.829577in}}{\pgfqpoint{6.054926in}{1.834621in}}%
\pgfpathcurveto{\pgfqpoint{6.059969in}{1.839664in}}{\pgfqpoint{6.062803in}{1.846506in}}{\pgfqpoint{6.062803in}{1.853639in}}%
\pgfpathcurveto{\pgfqpoint{6.062803in}{1.860772in}}{\pgfqpoint{6.059969in}{1.867613in}}{\pgfqpoint{6.054926in}{1.872657in}}%
\pgfpathcurveto{\pgfqpoint{6.049882in}{1.877701in}}{\pgfqpoint{6.043041in}{1.880535in}}{\pgfqpoint{6.035908in}{1.880535in}}%
\pgfpathcurveto{\pgfqpoint{6.028775in}{1.880535in}}{\pgfqpoint{6.021933in}{1.877701in}}{\pgfqpoint{6.016890in}{1.872657in}}%
\pgfpathcurveto{\pgfqpoint{6.011846in}{1.867613in}}{\pgfqpoint{6.009012in}{1.860772in}}{\pgfqpoint{6.009012in}{1.853639in}}%
\pgfpathcurveto{\pgfqpoint{6.009012in}{1.846506in}}{\pgfqpoint{6.011846in}{1.839664in}}{\pgfqpoint{6.016890in}{1.834621in}}%
\pgfpathcurveto{\pgfqpoint{6.021933in}{1.829577in}}{\pgfqpoint{6.028775in}{1.826743in}}{\pgfqpoint{6.035908in}{1.826743in}}%
\pgfpathclose%
\pgfusepath{stroke,fill}%
\end{pgfscope}%
\begin{pgfscope}%
\pgfpathrectangle{\pgfqpoint{4.985294in}{0.500000in}}{\pgfqpoint{1.764706in}{1.700000in}}%
\pgfusepath{clip}%
\pgfsetbuttcap%
\pgfsetroundjoin%
\definecolor{currentfill}{rgb}{0.975018,0.868213,0.788710}%
\pgfsetfillcolor{currentfill}%
\pgfsetlinewidth{0.311001pt}%
\definecolor{currentstroke}{rgb}{1.000000,1.000000,1.000000}%
\pgfsetstrokecolor{currentstroke}%
\pgfsetdash{}{0pt}%
\pgfpathmoveto{\pgfqpoint{5.483654in}{1.133032in}}%
\pgfpathcurveto{\pgfqpoint{5.490787in}{1.133032in}}{\pgfqpoint{5.497629in}{1.135866in}}{\pgfqpoint{5.502673in}{1.140910in}}%
\pgfpathcurveto{\pgfqpoint{5.507716in}{1.145954in}}{\pgfqpoint{5.510550in}{1.152795in}}{\pgfqpoint{5.510550in}{1.159928in}}%
\pgfpathcurveto{\pgfqpoint{5.510550in}{1.167061in}}{\pgfqpoint{5.507716in}{1.173903in}}{\pgfqpoint{5.502673in}{1.178946in}}%
\pgfpathcurveto{\pgfqpoint{5.497629in}{1.183990in}}{\pgfqpoint{5.490787in}{1.186824in}}{\pgfqpoint{5.483654in}{1.186824in}}%
\pgfpathcurveto{\pgfqpoint{5.476522in}{1.186824in}}{\pgfqpoint{5.469680in}{1.183990in}}{\pgfqpoint{5.464636in}{1.178946in}}%
\pgfpathcurveto{\pgfqpoint{5.459593in}{1.173903in}}{\pgfqpoint{5.456759in}{1.167061in}}{\pgfqpoint{5.456759in}{1.159928in}}%
\pgfpathcurveto{\pgfqpoint{5.456759in}{1.152795in}}{\pgfqpoint{5.459593in}{1.145954in}}{\pgfqpoint{5.464636in}{1.140910in}}%
\pgfpathcurveto{\pgfqpoint{5.469680in}{1.135866in}}{\pgfqpoint{5.476522in}{1.133032in}}{\pgfqpoint{5.483654in}{1.133032in}}%
\pgfpathclose%
\pgfusepath{stroke,fill}%
\end{pgfscope}%
\begin{pgfscope}%
\pgfpathrectangle{\pgfqpoint{4.985294in}{0.500000in}}{\pgfqpoint{1.764706in}{1.700000in}}%
\pgfusepath{clip}%
\pgfsetbuttcap%
\pgfsetroundjoin%
\definecolor{currentfill}{rgb}{0.962765,0.606121,0.444717}%
\pgfsetfillcolor{currentfill}%
\pgfsetlinewidth{0.311001pt}%
\definecolor{currentstroke}{rgb}{1.000000,1.000000,1.000000}%
\pgfsetstrokecolor{currentstroke}%
\pgfsetdash{}{0pt}%
\pgfpathmoveto{\pgfqpoint{5.302943in}{1.297412in}}%
\pgfpathcurveto{\pgfqpoint{5.310076in}{1.297412in}}{\pgfqpoint{5.316918in}{1.300245in}}{\pgfqpoint{5.321961in}{1.305289in}}%
\pgfpathcurveto{\pgfqpoint{5.327005in}{1.310333in}}{\pgfqpoint{5.329839in}{1.317174in}}{\pgfqpoint{5.329839in}{1.324307in}}%
\pgfpathcurveto{\pgfqpoint{5.329839in}{1.331440in}}{\pgfqpoint{5.327005in}{1.338282in}}{\pgfqpoint{5.321961in}{1.343325in}}%
\pgfpathcurveto{\pgfqpoint{5.316918in}{1.348369in}}{\pgfqpoint{5.310076in}{1.351203in}}{\pgfqpoint{5.302943in}{1.351203in}}%
\pgfpathcurveto{\pgfqpoint{5.295810in}{1.351203in}}{\pgfqpoint{5.288969in}{1.348369in}}{\pgfqpoint{5.283925in}{1.343325in}}%
\pgfpathcurveto{\pgfqpoint{5.278881in}{1.338282in}}{\pgfqpoint{5.276048in}{1.331440in}}{\pgfqpoint{5.276048in}{1.324307in}}%
\pgfpathcurveto{\pgfqpoint{5.276048in}{1.317174in}}{\pgfqpoint{5.278881in}{1.310333in}}{\pgfqpoint{5.283925in}{1.305289in}}%
\pgfpathcurveto{\pgfqpoint{5.288969in}{1.300245in}}{\pgfqpoint{5.295810in}{1.297412in}}{\pgfqpoint{5.302943in}{1.297412in}}%
\pgfpathclose%
\pgfusepath{stroke,fill}%
\end{pgfscope}%
\begin{pgfscope}%
\pgfpathrectangle{\pgfqpoint{4.985294in}{0.500000in}}{\pgfqpoint{1.764706in}{1.700000in}}%
\pgfusepath{clip}%
\pgfsetbuttcap%
\pgfsetroundjoin%
\definecolor{currentfill}{rgb}{0.969359,0.803954,0.693832}%
\pgfsetfillcolor{currentfill}%
\pgfsetlinewidth{0.311001pt}%
\definecolor{currentstroke}{rgb}{1.000000,1.000000,1.000000}%
\pgfsetstrokecolor{currentstroke}%
\pgfsetdash{}{0pt}%
\pgfpathmoveto{\pgfqpoint{6.176222in}{1.667958in}}%
\pgfpathcurveto{\pgfqpoint{6.183355in}{1.667958in}}{\pgfqpoint{6.190196in}{1.670792in}}{\pgfqpoint{6.195240in}{1.675835in}}%
\pgfpathcurveto{\pgfqpoint{6.200284in}{1.680879in}}{\pgfqpoint{6.203118in}{1.687721in}}{\pgfqpoint{6.203118in}{1.694853in}}%
\pgfpathcurveto{\pgfqpoint{6.203118in}{1.701986in}}{\pgfqpoint{6.200284in}{1.708828in}}{\pgfqpoint{6.195240in}{1.713871in}}%
\pgfpathcurveto{\pgfqpoint{6.190196in}{1.718915in}}{\pgfqpoint{6.183355in}{1.721749in}}{\pgfqpoint{6.176222in}{1.721749in}}%
\pgfpathcurveto{\pgfqpoint{6.169089in}{1.721749in}}{\pgfqpoint{6.162247in}{1.718915in}}{\pgfqpoint{6.157204in}{1.713871in}}%
\pgfpathcurveto{\pgfqpoint{6.152160in}{1.708828in}}{\pgfqpoint{6.149326in}{1.701986in}}{\pgfqpoint{6.149326in}{1.694853in}}%
\pgfpathcurveto{\pgfqpoint{6.149326in}{1.687721in}}{\pgfqpoint{6.152160in}{1.680879in}}{\pgfqpoint{6.157204in}{1.675835in}}%
\pgfpathcurveto{\pgfqpoint{6.162247in}{1.670792in}}{\pgfqpoint{6.169089in}{1.667958in}}{\pgfqpoint{6.176222in}{1.667958in}}%
\pgfpathclose%
\pgfusepath{stroke,fill}%
\end{pgfscope}%
\begin{pgfscope}%
\pgfpathrectangle{\pgfqpoint{4.985294in}{0.500000in}}{\pgfqpoint{1.764706in}{1.700000in}}%
\pgfusepath{clip}%
\pgfsetbuttcap%
\pgfsetroundjoin%
\definecolor{currentfill}{rgb}{0.965753,0.732351,0.592427}%
\pgfsetfillcolor{currentfill}%
\pgfsetlinewidth{0.311001pt}%
\definecolor{currentstroke}{rgb}{1.000000,1.000000,1.000000}%
\pgfsetstrokecolor{currentstroke}%
\pgfsetdash{}{0pt}%
\pgfpathmoveto{\pgfqpoint{6.289459in}{0.989762in}}%
\pgfpathcurveto{\pgfqpoint{6.296591in}{0.989762in}}{\pgfqpoint{6.303433in}{0.992596in}}{\pgfqpoint{6.308477in}{0.997640in}}%
\pgfpathcurveto{\pgfqpoint{6.313520in}{1.002683in}}{\pgfqpoint{6.316354in}{1.009525in}}{\pgfqpoint{6.316354in}{1.016658in}}%
\pgfpathcurveto{\pgfqpoint{6.316354in}{1.023791in}}{\pgfqpoint{6.313520in}{1.030632in}}{\pgfqpoint{6.308477in}{1.035676in}}%
\pgfpathcurveto{\pgfqpoint{6.303433in}{1.040720in}}{\pgfqpoint{6.296591in}{1.043554in}}{\pgfqpoint{6.289459in}{1.043554in}}%
\pgfpathcurveto{\pgfqpoint{6.282326in}{1.043554in}}{\pgfqpoint{6.275484in}{1.040720in}}{\pgfqpoint{6.270440in}{1.035676in}}%
\pgfpathcurveto{\pgfqpoint{6.265397in}{1.030632in}}{\pgfqpoint{6.262563in}{1.023791in}}{\pgfqpoint{6.262563in}{1.016658in}}%
\pgfpathcurveto{\pgfqpoint{6.262563in}{1.009525in}}{\pgfqpoint{6.265397in}{1.002683in}}{\pgfqpoint{6.270440in}{0.997640in}}%
\pgfpathcurveto{\pgfqpoint{6.275484in}{0.992596in}}{\pgfqpoint{6.282326in}{0.989762in}}{\pgfqpoint{6.289459in}{0.989762in}}%
\pgfpathclose%
\pgfusepath{stroke,fill}%
\end{pgfscope}%
\begin{pgfscope}%
\pgfpathrectangle{\pgfqpoint{4.985294in}{0.500000in}}{\pgfqpoint{1.764706in}{1.700000in}}%
\pgfusepath{clip}%
\pgfsetbuttcap%
\pgfsetroundjoin%
\definecolor{currentfill}{rgb}{0.963559,0.632016,0.472047}%
\pgfsetfillcolor{currentfill}%
\pgfsetlinewidth{0.311001pt}%
\definecolor{currentstroke}{rgb}{1.000000,1.000000,1.000000}%
\pgfsetstrokecolor{currentstroke}%
\pgfsetdash{}{0pt}%
\pgfpathmoveto{\pgfqpoint{6.275198in}{0.937436in}}%
\pgfpathcurveto{\pgfqpoint{6.282331in}{0.937436in}}{\pgfqpoint{6.289173in}{0.940270in}}{\pgfqpoint{6.294216in}{0.945314in}}%
\pgfpathcurveto{\pgfqpoint{6.299260in}{0.950357in}}{\pgfqpoint{6.302094in}{0.957199in}}{\pgfqpoint{6.302094in}{0.964332in}}%
\pgfpathcurveto{\pgfqpoint{6.302094in}{0.971465in}}{\pgfqpoint{6.299260in}{0.978306in}}{\pgfqpoint{6.294216in}{0.983350in}}%
\pgfpathcurveto{\pgfqpoint{6.289173in}{0.988394in}}{\pgfqpoint{6.282331in}{0.991228in}}{\pgfqpoint{6.275198in}{0.991228in}}%
\pgfpathcurveto{\pgfqpoint{6.268065in}{0.991228in}}{\pgfqpoint{6.261224in}{0.988394in}}{\pgfqpoint{6.256180in}{0.983350in}}%
\pgfpathcurveto{\pgfqpoint{6.251136in}{0.978306in}}{\pgfqpoint{6.248302in}{0.971465in}}{\pgfqpoint{6.248302in}{0.964332in}}%
\pgfpathcurveto{\pgfqpoint{6.248302in}{0.957199in}}{\pgfqpoint{6.251136in}{0.950357in}}{\pgfqpoint{6.256180in}{0.945314in}}%
\pgfpathcurveto{\pgfqpoint{6.261224in}{0.940270in}}{\pgfqpoint{6.268065in}{0.937436in}}{\pgfqpoint{6.275198in}{0.937436in}}%
\pgfpathclose%
\pgfusepath{stroke,fill}%
\end{pgfscope}%
\begin{pgfscope}%
\pgfpathrectangle{\pgfqpoint{4.985294in}{0.500000in}}{\pgfqpoint{1.764706in}{1.700000in}}%
\pgfusepath{clip}%
\pgfsetbuttcap%
\pgfsetroundjoin%
\definecolor{currentfill}{rgb}{0.980678,0.914765,0.856766}%
\pgfsetfillcolor{currentfill}%
\pgfsetlinewidth{0.311001pt}%
\definecolor{currentstroke}{rgb}{1.000000,1.000000,1.000000}%
\pgfsetstrokecolor{currentstroke}%
\pgfsetdash{}{0pt}%
\pgfpathmoveto{\pgfqpoint{6.334984in}{1.303658in}}%
\pgfpathcurveto{\pgfqpoint{6.342117in}{1.303658in}}{\pgfqpoint{6.348958in}{1.306492in}}{\pgfqpoint{6.354002in}{1.311536in}}%
\pgfpathcurveto{\pgfqpoint{6.359046in}{1.316580in}}{\pgfqpoint{6.361880in}{1.323421in}}{\pgfqpoint{6.361880in}{1.330554in}}%
\pgfpathcurveto{\pgfqpoint{6.361880in}{1.337687in}}{\pgfqpoint{6.359046in}{1.344529in}}{\pgfqpoint{6.354002in}{1.349572in}}%
\pgfpathcurveto{\pgfqpoint{6.348958in}{1.354616in}}{\pgfqpoint{6.342117in}{1.357450in}}{\pgfqpoint{6.334984in}{1.357450in}}%
\pgfpathcurveto{\pgfqpoint{6.327851in}{1.357450in}}{\pgfqpoint{6.321009in}{1.354616in}}{\pgfqpoint{6.315966in}{1.349572in}}%
\pgfpathcurveto{\pgfqpoint{6.310922in}{1.344529in}}{\pgfqpoint{6.308088in}{1.337687in}}{\pgfqpoint{6.308088in}{1.330554in}}%
\pgfpathcurveto{\pgfqpoint{6.308088in}{1.323421in}}{\pgfqpoint{6.310922in}{1.316580in}}{\pgfqpoint{6.315966in}{1.311536in}}%
\pgfpathcurveto{\pgfqpoint{6.321009in}{1.306492in}}{\pgfqpoint{6.327851in}{1.303658in}}{\pgfqpoint{6.334984in}{1.303658in}}%
\pgfpathclose%
\pgfusepath{stroke,fill}%
\end{pgfscope}%
\begin{pgfscope}%
\pgfpathrectangle{\pgfqpoint{4.985294in}{0.500000in}}{\pgfqpoint{1.764706in}{1.700000in}}%
\pgfusepath{clip}%
\pgfsetbuttcap%
\pgfsetroundjoin%
\definecolor{currentfill}{rgb}{0.977657,0.891500,0.822809}%
\pgfsetfillcolor{currentfill}%
\pgfsetlinewidth{0.311001pt}%
\definecolor{currentstroke}{rgb}{1.000000,1.000000,1.000000}%
\pgfsetstrokecolor{currentstroke}%
\pgfsetdash{}{0pt}%
\pgfpathmoveto{\pgfqpoint{5.453518in}{1.172398in}}%
\pgfpathcurveto{\pgfqpoint{5.460651in}{1.172398in}}{\pgfqpoint{5.467492in}{1.175231in}}{\pgfqpoint{5.472536in}{1.180275in}}%
\pgfpathcurveto{\pgfqpoint{5.477580in}{1.185319in}}{\pgfqpoint{5.480414in}{1.192160in}}{\pgfqpoint{5.480414in}{1.199293in}}%
\pgfpathcurveto{\pgfqpoint{5.480414in}{1.206426in}}{\pgfqpoint{5.477580in}{1.213268in}}{\pgfqpoint{5.472536in}{1.218311in}}%
\pgfpathcurveto{\pgfqpoint{5.467492in}{1.223355in}}{\pgfqpoint{5.460651in}{1.226189in}}{\pgfqpoint{5.453518in}{1.226189in}}%
\pgfpathcurveto{\pgfqpoint{5.446385in}{1.226189in}}{\pgfqpoint{5.439543in}{1.223355in}}{\pgfqpoint{5.434500in}{1.218311in}}%
\pgfpathcurveto{\pgfqpoint{5.429456in}{1.213268in}}{\pgfqpoint{5.426622in}{1.206426in}}{\pgfqpoint{5.426622in}{1.199293in}}%
\pgfpathcurveto{\pgfqpoint{5.426622in}{1.192160in}}{\pgfqpoint{5.429456in}{1.185319in}}{\pgfqpoint{5.434500in}{1.180275in}}%
\pgfpathcurveto{\pgfqpoint{5.439543in}{1.175231in}}{\pgfqpoint{5.446385in}{1.172398in}}{\pgfqpoint{5.453518in}{1.172398in}}%
\pgfpathclose%
\pgfusepath{stroke,fill}%
\end{pgfscope}%
\begin{pgfscope}%
\pgfpathrectangle{\pgfqpoint{4.985294in}{0.500000in}}{\pgfqpoint{1.764706in}{1.700000in}}%
\pgfusepath{clip}%
\pgfsetbuttcap%
\pgfsetroundjoin%
\definecolor{currentfill}{rgb}{0.964799,0.689101,0.537560}%
\pgfsetfillcolor{currentfill}%
\pgfsetlinewidth{0.311001pt}%
\definecolor{currentstroke}{rgb}{1.000000,1.000000,1.000000}%
\pgfsetstrokecolor{currentstroke}%
\pgfsetdash{}{0pt}%
\pgfpathmoveto{\pgfqpoint{6.283417in}{0.965158in}}%
\pgfpathcurveto{\pgfqpoint{6.290550in}{0.965158in}}{\pgfqpoint{6.297392in}{0.967992in}}{\pgfqpoint{6.302435in}{0.973036in}}%
\pgfpathcurveto{\pgfqpoint{6.307479in}{0.978079in}}{\pgfqpoint{6.310313in}{0.984921in}}{\pgfqpoint{6.310313in}{0.992054in}}%
\pgfpathcurveto{\pgfqpoint{6.310313in}{0.999187in}}{\pgfqpoint{6.307479in}{1.006028in}}{\pgfqpoint{6.302435in}{1.011072in}}%
\pgfpathcurveto{\pgfqpoint{6.297392in}{1.016116in}}{\pgfqpoint{6.290550in}{1.018950in}}{\pgfqpoint{6.283417in}{1.018950in}}%
\pgfpathcurveto{\pgfqpoint{6.276284in}{1.018950in}}{\pgfqpoint{6.269443in}{1.016116in}}{\pgfqpoint{6.264399in}{1.011072in}}%
\pgfpathcurveto{\pgfqpoint{6.259355in}{1.006028in}}{\pgfqpoint{6.256521in}{0.999187in}}{\pgfqpoint{6.256521in}{0.992054in}}%
\pgfpathcurveto{\pgfqpoint{6.256521in}{0.984921in}}{\pgfqpoint{6.259355in}{0.978079in}}{\pgfqpoint{6.264399in}{0.973036in}}%
\pgfpathcurveto{\pgfqpoint{6.269443in}{0.967992in}}{\pgfqpoint{6.276284in}{0.965158in}}{\pgfqpoint{6.283417in}{0.965158in}}%
\pgfpathclose%
\pgfusepath{stroke,fill}%
\end{pgfscope}%
\begin{pgfscope}%
\pgfpathrectangle{\pgfqpoint{4.985294in}{0.500000in}}{\pgfqpoint{1.764706in}{1.700000in}}%
\pgfusepath{clip}%
\pgfsetbuttcap%
\pgfsetroundjoin%
\definecolor{currentfill}{rgb}{0.976287,0.879862,0.805788}%
\pgfsetfillcolor{currentfill}%
\pgfsetlinewidth{0.311001pt}%
\definecolor{currentstroke}{rgb}{1.000000,1.000000,1.000000}%
\pgfsetstrokecolor{currentstroke}%
\pgfsetdash{}{0pt}%
\pgfpathmoveto{\pgfqpoint{6.256296in}{1.537330in}}%
\pgfpathcurveto{\pgfqpoint{6.263429in}{1.537330in}}{\pgfqpoint{6.270271in}{1.540164in}}{\pgfqpoint{6.275314in}{1.545208in}}%
\pgfpathcurveto{\pgfqpoint{6.280358in}{1.550252in}}{\pgfqpoint{6.283192in}{1.557093in}}{\pgfqpoint{6.283192in}{1.564226in}}%
\pgfpathcurveto{\pgfqpoint{6.283192in}{1.571359in}}{\pgfqpoint{6.280358in}{1.578201in}}{\pgfqpoint{6.275314in}{1.583244in}}%
\pgfpathcurveto{\pgfqpoint{6.270271in}{1.588288in}}{\pgfqpoint{6.263429in}{1.591122in}}{\pgfqpoint{6.256296in}{1.591122in}}%
\pgfpathcurveto{\pgfqpoint{6.249163in}{1.591122in}}{\pgfqpoint{6.242322in}{1.588288in}}{\pgfqpoint{6.237278in}{1.583244in}}%
\pgfpathcurveto{\pgfqpoint{6.232234in}{1.578201in}}{\pgfqpoint{6.229400in}{1.571359in}}{\pgfqpoint{6.229400in}{1.564226in}}%
\pgfpathcurveto{\pgfqpoint{6.229400in}{1.557093in}}{\pgfqpoint{6.232234in}{1.550252in}}{\pgfqpoint{6.237278in}{1.545208in}}%
\pgfpathcurveto{\pgfqpoint{6.242322in}{1.540164in}}{\pgfqpoint{6.249163in}{1.537330in}}{\pgfqpoint{6.256296in}{1.537330in}}%
\pgfpathclose%
\pgfusepath{stroke,fill}%
\end{pgfscope}%
\begin{pgfscope}%
\pgfpathrectangle{\pgfqpoint{4.985294in}{0.500000in}}{\pgfqpoint{1.764706in}{1.700000in}}%
\pgfusepath{clip}%
\pgfsetbuttcap%
\pgfsetroundjoin%
\definecolor{currentfill}{rgb}{0.963379,0.625574,0.465113}%
\pgfsetfillcolor{currentfill}%
\pgfsetlinewidth{0.311001pt}%
\definecolor{currentstroke}{rgb}{1.000000,1.000000,1.000000}%
\pgfsetstrokecolor{currentstroke}%
\pgfsetdash{}{0pt}%
\pgfpathmoveto{\pgfqpoint{5.415994in}{0.948516in}}%
\pgfpathcurveto{\pgfqpoint{5.423127in}{0.948516in}}{\pgfqpoint{5.429969in}{0.951349in}}{\pgfqpoint{5.435012in}{0.956393in}}%
\pgfpathcurveto{\pgfqpoint{5.440056in}{0.961437in}}{\pgfqpoint{5.442890in}{0.968278in}}{\pgfqpoint{5.442890in}{0.975411in}}%
\pgfpathcurveto{\pgfqpoint{5.442890in}{0.982544in}}{\pgfqpoint{5.440056in}{0.989386in}}{\pgfqpoint{5.435012in}{0.994429in}}%
\pgfpathcurveto{\pgfqpoint{5.429969in}{0.999473in}}{\pgfqpoint{5.423127in}{1.002307in}}{\pgfqpoint{5.415994in}{1.002307in}}%
\pgfpathcurveto{\pgfqpoint{5.408861in}{1.002307in}}{\pgfqpoint{5.402020in}{0.999473in}}{\pgfqpoint{5.396976in}{0.994429in}}%
\pgfpathcurveto{\pgfqpoint{5.391932in}{0.989386in}}{\pgfqpoint{5.389098in}{0.982544in}}{\pgfqpoint{5.389098in}{0.975411in}}%
\pgfpathcurveto{\pgfqpoint{5.389098in}{0.968278in}}{\pgfqpoint{5.391932in}{0.961437in}}{\pgfqpoint{5.396976in}{0.956393in}}%
\pgfpathcurveto{\pgfqpoint{5.402020in}{0.951349in}}{\pgfqpoint{5.408861in}{0.948516in}}{\pgfqpoint{5.415994in}{0.948516in}}%
\pgfpathclose%
\pgfusepath{stroke,fill}%
\end{pgfscope}%
\begin{pgfscope}%
\pgfpathrectangle{\pgfqpoint{4.985294in}{0.500000in}}{\pgfqpoint{1.764706in}{1.700000in}}%
\pgfusepath{clip}%
\pgfsetbuttcap%
\pgfsetroundjoin%
\definecolor{currentfill}{rgb}{0.974412,0.862387,0.780156}%
\pgfsetfillcolor{currentfill}%
\pgfsetlinewidth{0.311001pt}%
\definecolor{currentstroke}{rgb}{1.000000,1.000000,1.000000}%
\pgfsetstrokecolor{currentstroke}%
\pgfsetdash{}{0pt}%
\pgfpathmoveto{\pgfqpoint{6.288242in}{1.601828in}}%
\pgfpathcurveto{\pgfqpoint{6.295375in}{1.601828in}}{\pgfqpoint{6.302217in}{1.604662in}}{\pgfqpoint{6.307260in}{1.609706in}}%
\pgfpathcurveto{\pgfqpoint{6.312304in}{1.614749in}}{\pgfqpoint{6.315138in}{1.621591in}}{\pgfqpoint{6.315138in}{1.628724in}}%
\pgfpathcurveto{\pgfqpoint{6.315138in}{1.635857in}}{\pgfqpoint{6.312304in}{1.642698in}}{\pgfqpoint{6.307260in}{1.647742in}}%
\pgfpathcurveto{\pgfqpoint{6.302217in}{1.652786in}}{\pgfqpoint{6.295375in}{1.655620in}}{\pgfqpoint{6.288242in}{1.655620in}}%
\pgfpathcurveto{\pgfqpoint{6.281109in}{1.655620in}}{\pgfqpoint{6.274268in}{1.652786in}}{\pgfqpoint{6.269224in}{1.647742in}}%
\pgfpathcurveto{\pgfqpoint{6.264180in}{1.642698in}}{\pgfqpoint{6.261347in}{1.635857in}}{\pgfqpoint{6.261347in}{1.628724in}}%
\pgfpathcurveto{\pgfqpoint{6.261347in}{1.621591in}}{\pgfqpoint{6.264180in}{1.614749in}}{\pgfqpoint{6.269224in}{1.609706in}}%
\pgfpathcurveto{\pgfqpoint{6.274268in}{1.604662in}}{\pgfqpoint{6.281109in}{1.601828in}}{\pgfqpoint{6.288242in}{1.601828in}}%
\pgfpathclose%
\pgfusepath{stroke,fill}%
\end{pgfscope}%
\begin{pgfscope}%
\pgfpathrectangle{\pgfqpoint{4.985294in}{0.500000in}}{\pgfqpoint{1.764706in}{1.700000in}}%
\pgfusepath{clip}%
\pgfsetbuttcap%
\pgfsetroundjoin%
\definecolor{currentfill}{rgb}{0.968105,0.786346,0.667739}%
\pgfsetfillcolor{currentfill}%
\pgfsetlinewidth{0.311001pt}%
\definecolor{currentstroke}{rgb}{1.000000,1.000000,1.000000}%
\pgfsetstrokecolor{currentstroke}%
\pgfsetdash{}{0pt}%
\pgfpathmoveto{\pgfqpoint{5.343579in}{1.349226in}}%
\pgfpathcurveto{\pgfqpoint{5.350712in}{1.349226in}}{\pgfqpoint{5.357554in}{1.352060in}}{\pgfqpoint{5.362598in}{1.357104in}}%
\pgfpathcurveto{\pgfqpoint{5.367641in}{1.362147in}}{\pgfqpoint{5.370475in}{1.368989in}}{\pgfqpoint{5.370475in}{1.376122in}}%
\pgfpathcurveto{\pgfqpoint{5.370475in}{1.383254in}}{\pgfqpoint{5.367641in}{1.390096in}}{\pgfqpoint{5.362598in}{1.395140in}}%
\pgfpathcurveto{\pgfqpoint{5.357554in}{1.400183in}}{\pgfqpoint{5.350712in}{1.403017in}}{\pgfqpoint{5.343579in}{1.403017in}}%
\pgfpathcurveto{\pgfqpoint{5.336447in}{1.403017in}}{\pgfqpoint{5.329605in}{1.400183in}}{\pgfqpoint{5.324561in}{1.395140in}}%
\pgfpathcurveto{\pgfqpoint{5.319518in}{1.390096in}}{\pgfqpoint{5.316684in}{1.383254in}}{\pgfqpoint{5.316684in}{1.376122in}}%
\pgfpathcurveto{\pgfqpoint{5.316684in}{1.368989in}}{\pgfqpoint{5.319518in}{1.362147in}}{\pgfqpoint{5.324561in}{1.357104in}}%
\pgfpathcurveto{\pgfqpoint{5.329605in}{1.352060in}}{\pgfqpoint{5.336447in}{1.349226in}}{\pgfqpoint{5.343579in}{1.349226in}}%
\pgfpathclose%
\pgfusepath{stroke,fill}%
\end{pgfscope}%
\begin{pgfscope}%
\pgfpathrectangle{\pgfqpoint{4.985294in}{0.500000in}}{\pgfqpoint{1.764706in}{1.700000in}}%
\pgfusepath{clip}%
\pgfsetbuttcap%
\pgfsetroundjoin%
\definecolor{currentfill}{rgb}{0.970718,0.821518,0.719872}%
\pgfsetfillcolor{currentfill}%
\pgfsetlinewidth{0.311001pt}%
\definecolor{currentstroke}{rgb}{1.000000,1.000000,1.000000}%
\pgfsetstrokecolor{currentstroke}%
\pgfsetdash{}{0pt}%
\pgfpathmoveto{\pgfqpoint{6.199190in}{1.117297in}}%
\pgfpathcurveto{\pgfqpoint{6.206323in}{1.117297in}}{\pgfqpoint{6.213164in}{1.120131in}}{\pgfqpoint{6.218208in}{1.125174in}}%
\pgfpathcurveto{\pgfqpoint{6.223252in}{1.130218in}}{\pgfqpoint{6.226085in}{1.137060in}}{\pgfqpoint{6.226085in}{1.144193in}}%
\pgfpathcurveto{\pgfqpoint{6.226085in}{1.151325in}}{\pgfqpoint{6.223252in}{1.158167in}}{\pgfqpoint{6.218208in}{1.163211in}}%
\pgfpathcurveto{\pgfqpoint{6.213164in}{1.168254in}}{\pgfqpoint{6.206323in}{1.171088in}}{\pgfqpoint{6.199190in}{1.171088in}}%
\pgfpathcurveto{\pgfqpoint{6.192057in}{1.171088in}}{\pgfqpoint{6.185215in}{1.168254in}}{\pgfqpoint{6.180172in}{1.163211in}}%
\pgfpathcurveto{\pgfqpoint{6.175128in}{1.158167in}}{\pgfqpoint{6.172294in}{1.151325in}}{\pgfqpoint{6.172294in}{1.144193in}}%
\pgfpathcurveto{\pgfqpoint{6.172294in}{1.137060in}}{\pgfqpoint{6.175128in}{1.130218in}}{\pgfqpoint{6.180172in}{1.125174in}}%
\pgfpathcurveto{\pgfqpoint{6.185215in}{1.120131in}}{\pgfqpoint{6.192057in}{1.117297in}}{\pgfqpoint{6.199190in}{1.117297in}}%
\pgfpathclose%
\pgfusepath{stroke,fill}%
\end{pgfscope}%
\begin{pgfscope}%
\pgfpathrectangle{\pgfqpoint{4.985294in}{0.500000in}}{\pgfqpoint{1.764706in}{1.700000in}}%
\pgfusepath{clip}%
\pgfsetbuttcap%
\pgfsetroundjoin%
\definecolor{currentfill}{rgb}{0.948235,0.413004,0.283323}%
\pgfsetfillcolor{currentfill}%
\pgfsetlinewidth{0.311001pt}%
\definecolor{currentstroke}{rgb}{1.000000,1.000000,1.000000}%
\pgfsetstrokecolor{currentstroke}%
\pgfsetdash{}{0pt}%
\pgfpathmoveto{\pgfqpoint{6.138072in}{1.424527in}}%
\pgfpathcurveto{\pgfqpoint{6.145205in}{1.424527in}}{\pgfqpoint{6.152047in}{1.427361in}}{\pgfqpoint{6.157091in}{1.432405in}}%
\pgfpathcurveto{\pgfqpoint{6.162134in}{1.437449in}}{\pgfqpoint{6.164968in}{1.444290in}}{\pgfqpoint{6.164968in}{1.451423in}}%
\pgfpathcurveto{\pgfqpoint{6.164968in}{1.458556in}}{\pgfqpoint{6.162134in}{1.465397in}}{\pgfqpoint{6.157091in}{1.470441in}}%
\pgfpathcurveto{\pgfqpoint{6.152047in}{1.475485in}}{\pgfqpoint{6.145205in}{1.478319in}}{\pgfqpoint{6.138072in}{1.478319in}}%
\pgfpathcurveto{\pgfqpoint{6.130940in}{1.478319in}}{\pgfqpoint{6.124098in}{1.475485in}}{\pgfqpoint{6.119054in}{1.470441in}}%
\pgfpathcurveto{\pgfqpoint{6.114011in}{1.465397in}}{\pgfqpoint{6.111177in}{1.458556in}}{\pgfqpoint{6.111177in}{1.451423in}}%
\pgfpathcurveto{\pgfqpoint{6.111177in}{1.444290in}}{\pgfqpoint{6.114011in}{1.437449in}}{\pgfqpoint{6.119054in}{1.432405in}}%
\pgfpathcurveto{\pgfqpoint{6.124098in}{1.427361in}}{\pgfqpoint{6.130940in}{1.424527in}}{\pgfqpoint{6.138072in}{1.424527in}}%
\pgfpathclose%
\pgfusepath{stroke,fill}%
\end{pgfscope}%
\begin{pgfscope}%
\pgfpathrectangle{\pgfqpoint{4.985294in}{0.500000in}}{\pgfqpoint{1.764706in}{1.700000in}}%
\pgfusepath{clip}%
\pgfsetbuttcap%
\pgfsetroundjoin%
\definecolor{currentfill}{rgb}{0.966328,0.750560,0.616961}%
\pgfsetfillcolor{currentfill}%
\pgfsetlinewidth{0.311001pt}%
\definecolor{currentstroke}{rgb}{1.000000,1.000000,1.000000}%
\pgfsetstrokecolor{currentstroke}%
\pgfsetdash{}{0pt}%
\pgfpathmoveto{\pgfqpoint{6.148712in}{0.998086in}}%
\pgfpathcurveto{\pgfqpoint{6.155845in}{0.998086in}}{\pgfqpoint{6.162687in}{1.000920in}}{\pgfqpoint{6.167731in}{1.005964in}}%
\pgfpathcurveto{\pgfqpoint{6.172774in}{1.011008in}}{\pgfqpoint{6.175608in}{1.017849in}}{\pgfqpoint{6.175608in}{1.024982in}}%
\pgfpathcurveto{\pgfqpoint{6.175608in}{1.032115in}}{\pgfqpoint{6.172774in}{1.038956in}}{\pgfqpoint{6.167731in}{1.044000in}}%
\pgfpathcurveto{\pgfqpoint{6.162687in}{1.049044in}}{\pgfqpoint{6.155845in}{1.051878in}}{\pgfqpoint{6.148712in}{1.051878in}}%
\pgfpathcurveto{\pgfqpoint{6.141580in}{1.051878in}}{\pgfqpoint{6.134738in}{1.049044in}}{\pgfqpoint{6.129694in}{1.044000in}}%
\pgfpathcurveto{\pgfqpoint{6.124651in}{1.038956in}}{\pgfqpoint{6.121817in}{1.032115in}}{\pgfqpoint{6.121817in}{1.024982in}}%
\pgfpathcurveto{\pgfqpoint{6.121817in}{1.017849in}}{\pgfqpoint{6.124651in}{1.011008in}}{\pgfqpoint{6.129694in}{1.005964in}}%
\pgfpathcurveto{\pgfqpoint{6.134738in}{1.000920in}}{\pgfqpoint{6.141580in}{0.998086in}}{\pgfqpoint{6.148712in}{0.998086in}}%
\pgfpathclose%
\pgfusepath{stroke,fill}%
\end{pgfscope}%
\begin{pgfscope}%
\pgfpathrectangle{\pgfqpoint{4.985294in}{0.500000in}}{\pgfqpoint{1.764706in}{1.700000in}}%
\pgfusepath{clip}%
\pgfsetbuttcap%
\pgfsetroundjoin%
\definecolor{currentfill}{rgb}{0.976961,0.885681,0.814303}%
\pgfsetfillcolor{currentfill}%
\pgfsetlinewidth{0.311001pt}%
\definecolor{currentstroke}{rgb}{1.000000,1.000000,1.000000}%
\pgfsetstrokecolor{currentstroke}%
\pgfsetdash{}{0pt}%
\pgfpathmoveto{\pgfqpoint{5.381849in}{1.354927in}}%
\pgfpathcurveto{\pgfqpoint{5.388982in}{1.354927in}}{\pgfqpoint{5.395824in}{1.357760in}}{\pgfqpoint{5.400867in}{1.362804in}}%
\pgfpathcurveto{\pgfqpoint{5.405911in}{1.367848in}}{\pgfqpoint{5.408745in}{1.374689in}}{\pgfqpoint{5.408745in}{1.381822in}}%
\pgfpathcurveto{\pgfqpoint{5.408745in}{1.388955in}}{\pgfqpoint{5.405911in}{1.395797in}}{\pgfqpoint{5.400867in}{1.400840in}}%
\pgfpathcurveto{\pgfqpoint{5.395824in}{1.405884in}}{\pgfqpoint{5.388982in}{1.408718in}}{\pgfqpoint{5.381849in}{1.408718in}}%
\pgfpathcurveto{\pgfqpoint{5.374716in}{1.408718in}}{\pgfqpoint{5.367875in}{1.405884in}}{\pgfqpoint{5.362831in}{1.400840in}}%
\pgfpathcurveto{\pgfqpoint{5.357787in}{1.395797in}}{\pgfqpoint{5.354953in}{1.388955in}}{\pgfqpoint{5.354953in}{1.381822in}}%
\pgfpathcurveto{\pgfqpoint{5.354953in}{1.374689in}}{\pgfqpoint{5.357787in}{1.367848in}}{\pgfqpoint{5.362831in}{1.362804in}}%
\pgfpathcurveto{\pgfqpoint{5.367875in}{1.357760in}}{\pgfqpoint{5.374716in}{1.354927in}}{\pgfqpoint{5.381849in}{1.354927in}}%
\pgfpathclose%
\pgfusepath{stroke,fill}%
\end{pgfscope}%
\begin{pgfscope}%
\pgfpathrectangle{\pgfqpoint{4.985294in}{0.500000in}}{\pgfqpoint{1.764706in}{1.700000in}}%
\pgfusepath{clip}%
\pgfsetbuttcap%
\pgfsetroundjoin%
\definecolor{currentfill}{rgb}{0.911533,0.252926,0.244703}%
\pgfsetfillcolor{currentfill}%
\pgfsetlinewidth{0.311001pt}%
\definecolor{currentstroke}{rgb}{1.000000,1.000000,1.000000}%
\pgfsetstrokecolor{currentstroke}%
\pgfsetdash{}{0pt}%
\pgfpathmoveto{\pgfqpoint{5.739960in}{1.681008in}}%
\pgfpathcurveto{\pgfqpoint{5.747093in}{1.681008in}}{\pgfqpoint{5.753934in}{1.683842in}}{\pgfqpoint{5.758978in}{1.688885in}}%
\pgfpathcurveto{\pgfqpoint{5.764022in}{1.693929in}}{\pgfqpoint{5.766855in}{1.700771in}}{\pgfqpoint{5.766855in}{1.707904in}}%
\pgfpathcurveto{\pgfqpoint{5.766855in}{1.715036in}}{\pgfqpoint{5.764022in}{1.721878in}}{\pgfqpoint{5.758978in}{1.726922in}}%
\pgfpathcurveto{\pgfqpoint{5.753934in}{1.731965in}}{\pgfqpoint{5.747093in}{1.734799in}}{\pgfqpoint{5.739960in}{1.734799in}}%
\pgfpathcurveto{\pgfqpoint{5.732827in}{1.734799in}}{\pgfqpoint{5.725985in}{1.731965in}}{\pgfqpoint{5.720942in}{1.726922in}}%
\pgfpathcurveto{\pgfqpoint{5.715898in}{1.721878in}}{\pgfqpoint{5.713064in}{1.715036in}}{\pgfqpoint{5.713064in}{1.707904in}}%
\pgfpathcurveto{\pgfqpoint{5.713064in}{1.700771in}}{\pgfqpoint{5.715898in}{1.693929in}}{\pgfqpoint{5.720942in}{1.688885in}}%
\pgfpathcurveto{\pgfqpoint{5.725985in}{1.683842in}}{\pgfqpoint{5.732827in}{1.681008in}}{\pgfqpoint{5.739960in}{1.681008in}}%
\pgfpathclose%
\pgfusepath{stroke,fill}%
\end{pgfscope}%
\begin{pgfscope}%
\pgfpathrectangle{\pgfqpoint{4.985294in}{0.500000in}}{\pgfqpoint{1.764706in}{1.700000in}}%
\pgfusepath{clip}%
\pgfsetbuttcap%
\pgfsetroundjoin%
\definecolor{currentfill}{rgb}{0.978376,0.897317,0.831308}%
\pgfsetfillcolor{currentfill}%
\pgfsetlinewidth{0.311001pt}%
\definecolor{currentstroke}{rgb}{1.000000,1.000000,1.000000}%
\pgfsetstrokecolor{currentstroke}%
\pgfsetdash{}{0pt}%
\pgfpathmoveto{\pgfqpoint{5.409822in}{1.438726in}}%
\pgfpathcurveto{\pgfqpoint{5.416955in}{1.438726in}}{\pgfqpoint{5.423797in}{1.441560in}}{\pgfqpoint{5.428841in}{1.446603in}}%
\pgfpathcurveto{\pgfqpoint{5.433884in}{1.451647in}}{\pgfqpoint{5.436718in}{1.458489in}}{\pgfqpoint{5.436718in}{1.465621in}}%
\pgfpathcurveto{\pgfqpoint{5.436718in}{1.472754in}}{\pgfqpoint{5.433884in}{1.479596in}}{\pgfqpoint{5.428841in}{1.484639in}}%
\pgfpathcurveto{\pgfqpoint{5.423797in}{1.489683in}}{\pgfqpoint{5.416955in}{1.492517in}}{\pgfqpoint{5.409822in}{1.492517in}}%
\pgfpathcurveto{\pgfqpoint{5.402690in}{1.492517in}}{\pgfqpoint{5.395848in}{1.489683in}}{\pgfqpoint{5.390804in}{1.484639in}}%
\pgfpathcurveto{\pgfqpoint{5.385761in}{1.479596in}}{\pgfqpoint{5.382927in}{1.472754in}}{\pgfqpoint{5.382927in}{1.465621in}}%
\pgfpathcurveto{\pgfqpoint{5.382927in}{1.458489in}}{\pgfqpoint{5.385761in}{1.451647in}}{\pgfqpoint{5.390804in}{1.446603in}}%
\pgfpathcurveto{\pgfqpoint{5.395848in}{1.441560in}}{\pgfqpoint{5.402690in}{1.438726in}}{\pgfqpoint{5.409822in}{1.438726in}}%
\pgfpathclose%
\pgfusepath{stroke,fill}%
\end{pgfscope}%
\begin{pgfscope}%
\pgfpathrectangle{\pgfqpoint{4.985294in}{0.500000in}}{\pgfqpoint{1.764706in}{1.700000in}}%
\pgfusepath{clip}%
\pgfsetbuttcap%
\pgfsetroundjoin%
\definecolor{currentfill}{rgb}{0.975018,0.868213,0.788710}%
\pgfsetfillcolor{currentfill}%
\pgfsetlinewidth{0.311001pt}%
\definecolor{currentstroke}{rgb}{1.000000,1.000000,1.000000}%
\pgfsetstrokecolor{currentstroke}%
\pgfsetdash{}{0pt}%
\pgfpathmoveto{\pgfqpoint{6.262349in}{1.260521in}}%
\pgfpathcurveto{\pgfqpoint{6.269482in}{1.260521in}}{\pgfqpoint{6.276324in}{1.263355in}}{\pgfqpoint{6.281367in}{1.268399in}}%
\pgfpathcurveto{\pgfqpoint{6.286411in}{1.273442in}}{\pgfqpoint{6.289245in}{1.280284in}}{\pgfqpoint{6.289245in}{1.287417in}}%
\pgfpathcurveto{\pgfqpoint{6.289245in}{1.294550in}}{\pgfqpoint{6.286411in}{1.301391in}}{\pgfqpoint{6.281367in}{1.306435in}}%
\pgfpathcurveto{\pgfqpoint{6.276324in}{1.311479in}}{\pgfqpoint{6.269482in}{1.314313in}}{\pgfqpoint{6.262349in}{1.314313in}}%
\pgfpathcurveto{\pgfqpoint{6.255216in}{1.314313in}}{\pgfqpoint{6.248375in}{1.311479in}}{\pgfqpoint{6.243331in}{1.306435in}}%
\pgfpathcurveto{\pgfqpoint{6.238287in}{1.301391in}}{\pgfqpoint{6.235454in}{1.294550in}}{\pgfqpoint{6.235454in}{1.287417in}}%
\pgfpathcurveto{\pgfqpoint{6.235454in}{1.280284in}}{\pgfqpoint{6.238287in}{1.273442in}}{\pgfqpoint{6.243331in}{1.268399in}}%
\pgfpathcurveto{\pgfqpoint{6.248375in}{1.263355in}}{\pgfqpoint{6.255216in}{1.260521in}}{\pgfqpoint{6.262349in}{1.260521in}}%
\pgfpathclose%
\pgfusepath{stroke,fill}%
\end{pgfscope}%
\begin{pgfscope}%
\pgfpathrectangle{\pgfqpoint{4.985294in}{0.500000in}}{\pgfqpoint{1.764706in}{1.700000in}}%
\pgfusepath{clip}%
\pgfsetbuttcap%
\pgfsetroundjoin%
\definecolor{currentfill}{rgb}{0.973832,0.856556,0.771584}%
\pgfsetfillcolor{currentfill}%
\pgfsetlinewidth{0.311001pt}%
\definecolor{currentstroke}{rgb}{1.000000,1.000000,1.000000}%
\pgfsetstrokecolor{currentstroke}%
\pgfsetdash{}{0pt}%
\pgfpathmoveto{\pgfqpoint{5.376709in}{1.220893in}}%
\pgfpathcurveto{\pgfqpoint{5.383842in}{1.220893in}}{\pgfqpoint{5.390684in}{1.223727in}}{\pgfqpoint{5.395728in}{1.228771in}}%
\pgfpathcurveto{\pgfqpoint{5.400771in}{1.233815in}}{\pgfqpoint{5.403605in}{1.240656in}}{\pgfqpoint{5.403605in}{1.247789in}}%
\pgfpathcurveto{\pgfqpoint{5.403605in}{1.254922in}}{\pgfqpoint{5.400771in}{1.261764in}}{\pgfqpoint{5.395728in}{1.266807in}}%
\pgfpathcurveto{\pgfqpoint{5.390684in}{1.271851in}}{\pgfqpoint{5.383842in}{1.274685in}}{\pgfqpoint{5.376709in}{1.274685in}}%
\pgfpathcurveto{\pgfqpoint{5.369577in}{1.274685in}}{\pgfqpoint{5.362735in}{1.271851in}}{\pgfqpoint{5.357691in}{1.266807in}}%
\pgfpathcurveto{\pgfqpoint{5.352648in}{1.261764in}}{\pgfqpoint{5.349814in}{1.254922in}}{\pgfqpoint{5.349814in}{1.247789in}}%
\pgfpathcurveto{\pgfqpoint{5.349814in}{1.240656in}}{\pgfqpoint{5.352648in}{1.233815in}}{\pgfqpoint{5.357691in}{1.228771in}}%
\pgfpathcurveto{\pgfqpoint{5.362735in}{1.223727in}}{\pgfqpoint{5.369577in}{1.220893in}}{\pgfqpoint{5.376709in}{1.220893in}}%
\pgfpathclose%
\pgfusepath{stroke,fill}%
\end{pgfscope}%
\begin{pgfscope}%
\pgfpathrectangle{\pgfqpoint{4.985294in}{0.500000in}}{\pgfqpoint{1.764706in}{1.700000in}}%
\pgfusepath{clip}%
\pgfsetbuttcap%
\pgfsetroundjoin%
\definecolor{currentfill}{rgb}{0.979891,0.908948,0.848279}%
\pgfsetfillcolor{currentfill}%
\pgfsetlinewidth{0.311001pt}%
\definecolor{currentstroke}{rgb}{1.000000,1.000000,1.000000}%
\pgfsetstrokecolor{currentstroke}%
\pgfsetdash{}{0pt}%
\pgfpathmoveto{\pgfqpoint{5.426554in}{1.400759in}}%
\pgfpathcurveto{\pgfqpoint{5.433687in}{1.400759in}}{\pgfqpoint{5.440529in}{1.403593in}}{\pgfqpoint{5.445573in}{1.408637in}}%
\pgfpathcurveto{\pgfqpoint{5.450616in}{1.413680in}}{\pgfqpoint{5.453450in}{1.420522in}}{\pgfqpoint{5.453450in}{1.427655in}}%
\pgfpathcurveto{\pgfqpoint{5.453450in}{1.434788in}}{\pgfqpoint{5.450616in}{1.441629in}}{\pgfqpoint{5.445573in}{1.446673in}}%
\pgfpathcurveto{\pgfqpoint{5.440529in}{1.451717in}}{\pgfqpoint{5.433687in}{1.454551in}}{\pgfqpoint{5.426554in}{1.454551in}}%
\pgfpathcurveto{\pgfqpoint{5.419422in}{1.454551in}}{\pgfqpoint{5.412580in}{1.451717in}}{\pgfqpoint{5.407536in}{1.446673in}}%
\pgfpathcurveto{\pgfqpoint{5.402493in}{1.441629in}}{\pgfqpoint{5.399659in}{1.434788in}}{\pgfqpoint{5.399659in}{1.427655in}}%
\pgfpathcurveto{\pgfqpoint{5.399659in}{1.420522in}}{\pgfqpoint{5.402493in}{1.413680in}}{\pgfqpoint{5.407536in}{1.408637in}}%
\pgfpathcurveto{\pgfqpoint{5.412580in}{1.403593in}}{\pgfqpoint{5.419422in}{1.400759in}}{\pgfqpoint{5.426554in}{1.400759in}}%
\pgfpathclose%
\pgfusepath{stroke,fill}%
\end{pgfscope}%
\begin{pgfscope}%
\pgfpathrectangle{\pgfqpoint{4.985294in}{0.500000in}}{\pgfqpoint{1.764706in}{1.700000in}}%
\pgfusepath{clip}%
\pgfsetbuttcap%
\pgfsetroundjoin%
\definecolor{currentfill}{rgb}{0.979124,0.903132,0.839793}%
\pgfsetfillcolor{currentfill}%
\pgfsetlinewidth{0.311001pt}%
\definecolor{currentstroke}{rgb}{1.000000,1.000000,1.000000}%
\pgfsetstrokecolor{currentstroke}%
\pgfsetdash{}{0pt}%
\pgfpathmoveto{\pgfqpoint{6.302103in}{1.531516in}}%
\pgfpathcurveto{\pgfqpoint{6.309236in}{1.531516in}}{\pgfqpoint{6.316078in}{1.534350in}}{\pgfqpoint{6.321122in}{1.539394in}}%
\pgfpathcurveto{\pgfqpoint{6.326165in}{1.544437in}}{\pgfqpoint{6.328999in}{1.551279in}}{\pgfqpoint{6.328999in}{1.558412in}}%
\pgfpathcurveto{\pgfqpoint{6.328999in}{1.565545in}}{\pgfqpoint{6.326165in}{1.572386in}}{\pgfqpoint{6.321122in}{1.577430in}}%
\pgfpathcurveto{\pgfqpoint{6.316078in}{1.582474in}}{\pgfqpoint{6.309236in}{1.585308in}}{\pgfqpoint{6.302103in}{1.585308in}}%
\pgfpathcurveto{\pgfqpoint{6.294971in}{1.585308in}}{\pgfqpoint{6.288129in}{1.582474in}}{\pgfqpoint{6.283085in}{1.577430in}}%
\pgfpathcurveto{\pgfqpoint{6.278042in}{1.572386in}}{\pgfqpoint{6.275208in}{1.565545in}}{\pgfqpoint{6.275208in}{1.558412in}}%
\pgfpathcurveto{\pgfqpoint{6.275208in}{1.551279in}}{\pgfqpoint{6.278042in}{1.544437in}}{\pgfqpoint{6.283085in}{1.539394in}}%
\pgfpathcurveto{\pgfqpoint{6.288129in}{1.534350in}}{\pgfqpoint{6.294971in}{1.531516in}}{\pgfqpoint{6.302103in}{1.531516in}}%
\pgfpathclose%
\pgfusepath{stroke,fill}%
\end{pgfscope}%
\begin{pgfscope}%
\pgfpathrectangle{\pgfqpoint{4.985294in}{0.500000in}}{\pgfqpoint{1.764706in}{1.700000in}}%
\pgfusepath{clip}%
\pgfsetbuttcap%
\pgfsetroundjoin%
\definecolor{currentfill}{rgb}{0.966812,0.762584,0.633643}%
\pgfsetfillcolor{currentfill}%
\pgfsetlinewidth{0.311001pt}%
\definecolor{currentstroke}{rgb}{1.000000,1.000000,1.000000}%
\pgfsetstrokecolor{currentstroke}%
\pgfsetdash{}{0pt}%
\pgfpathmoveto{\pgfqpoint{6.297974in}{1.015672in}}%
\pgfpathcurveto{\pgfqpoint{6.305107in}{1.015672in}}{\pgfqpoint{6.311948in}{1.018505in}}{\pgfqpoint{6.316992in}{1.023549in}}%
\pgfpathcurveto{\pgfqpoint{6.322036in}{1.028593in}}{\pgfqpoint{6.324870in}{1.035434in}}{\pgfqpoint{6.324870in}{1.042567in}}%
\pgfpathcurveto{\pgfqpoint{6.324870in}{1.049700in}}{\pgfqpoint{6.322036in}{1.056542in}}{\pgfqpoint{6.316992in}{1.061585in}}%
\pgfpathcurveto{\pgfqpoint{6.311948in}{1.066629in}}{\pgfqpoint{6.305107in}{1.069463in}}{\pgfqpoint{6.297974in}{1.069463in}}%
\pgfpathcurveto{\pgfqpoint{6.290841in}{1.069463in}}{\pgfqpoint{6.284000in}{1.066629in}}{\pgfqpoint{6.278956in}{1.061585in}}%
\pgfpathcurveto{\pgfqpoint{6.273912in}{1.056542in}}{\pgfqpoint{6.271078in}{1.049700in}}{\pgfqpoint{6.271078in}{1.042567in}}%
\pgfpathcurveto{\pgfqpoint{6.271078in}{1.035434in}}{\pgfqpoint{6.273912in}{1.028593in}}{\pgfqpoint{6.278956in}{1.023549in}}%
\pgfpathcurveto{\pgfqpoint{6.284000in}{1.018505in}}{\pgfqpoint{6.290841in}{1.015672in}}{\pgfqpoint{6.297974in}{1.015672in}}%
\pgfpathclose%
\pgfusepath{stroke,fill}%
\end{pgfscope}%
\begin{pgfscope}%
\pgfpathrectangle{\pgfqpoint{4.985294in}{0.500000in}}{\pgfqpoint{1.764706in}{1.700000in}}%
\pgfusepath{clip}%
\pgfsetbuttcap%
\pgfsetroundjoin%
\definecolor{currentfill}{rgb}{0.970718,0.821518,0.719872}%
\pgfsetfillcolor{currentfill}%
\pgfsetlinewidth{0.311001pt}%
\definecolor{currentstroke}{rgb}{1.000000,1.000000,1.000000}%
\pgfsetstrokecolor{currentstroke}%
\pgfsetdash{}{0pt}%
\pgfpathmoveto{\pgfqpoint{5.487883in}{1.269431in}}%
\pgfpathcurveto{\pgfqpoint{5.495016in}{1.269431in}}{\pgfqpoint{5.501857in}{1.272265in}}{\pgfqpoint{5.506901in}{1.277308in}}%
\pgfpathcurveto{\pgfqpoint{5.511945in}{1.282352in}}{\pgfqpoint{5.514778in}{1.289194in}}{\pgfqpoint{5.514778in}{1.296326in}}%
\pgfpathcurveto{\pgfqpoint{5.514778in}{1.303459in}}{\pgfqpoint{5.511945in}{1.310301in}}{\pgfqpoint{5.506901in}{1.315345in}}%
\pgfpathcurveto{\pgfqpoint{5.501857in}{1.320388in}}{\pgfqpoint{5.495016in}{1.323222in}}{\pgfqpoint{5.487883in}{1.323222in}}%
\pgfpathcurveto{\pgfqpoint{5.480750in}{1.323222in}}{\pgfqpoint{5.473908in}{1.320388in}}{\pgfqpoint{5.468865in}{1.315345in}}%
\pgfpathcurveto{\pgfqpoint{5.463821in}{1.310301in}}{\pgfqpoint{5.460987in}{1.303459in}}{\pgfqpoint{5.460987in}{1.296326in}}%
\pgfpathcurveto{\pgfqpoint{5.460987in}{1.289194in}}{\pgfqpoint{5.463821in}{1.282352in}}{\pgfqpoint{5.468865in}{1.277308in}}%
\pgfpathcurveto{\pgfqpoint{5.473908in}{1.272265in}}{\pgfqpoint{5.480750in}{1.269431in}}{\pgfqpoint{5.487883in}{1.269431in}}%
\pgfpathclose%
\pgfusepath{stroke,fill}%
\end{pgfscope}%
\begin{pgfscope}%
\pgfpathrectangle{\pgfqpoint{4.985294in}{0.500000in}}{\pgfqpoint{1.764706in}{1.700000in}}%
\pgfusepath{clip}%
\pgfsetbuttcap%
\pgfsetroundjoin%
\definecolor{currentfill}{rgb}{0.976961,0.885681,0.814303}%
\pgfsetfillcolor{currentfill}%
\pgfsetlinewidth{0.311001pt}%
\definecolor{currentstroke}{rgb}{1.000000,1.000000,1.000000}%
\pgfsetstrokecolor{currentstroke}%
\pgfsetdash{}{0pt}%
\pgfpathmoveto{\pgfqpoint{6.302004in}{1.129878in}}%
\pgfpathcurveto{\pgfqpoint{6.309137in}{1.129878in}}{\pgfqpoint{6.315979in}{1.132712in}}{\pgfqpoint{6.321022in}{1.137755in}}%
\pgfpathcurveto{\pgfqpoint{6.326066in}{1.142799in}}{\pgfqpoint{6.328900in}{1.149641in}}{\pgfqpoint{6.328900in}{1.156773in}}%
\pgfpathcurveto{\pgfqpoint{6.328900in}{1.163906in}}{\pgfqpoint{6.326066in}{1.170748in}}{\pgfqpoint{6.321022in}{1.175792in}}%
\pgfpathcurveto{\pgfqpoint{6.315979in}{1.180835in}}{\pgfqpoint{6.309137in}{1.183669in}}{\pgfqpoint{6.302004in}{1.183669in}}%
\pgfpathcurveto{\pgfqpoint{6.294871in}{1.183669in}}{\pgfqpoint{6.288030in}{1.180835in}}{\pgfqpoint{6.282986in}{1.175792in}}%
\pgfpathcurveto{\pgfqpoint{6.277942in}{1.170748in}}{\pgfqpoint{6.275108in}{1.163906in}}{\pgfqpoint{6.275108in}{1.156773in}}%
\pgfpathcurveto{\pgfqpoint{6.275108in}{1.149641in}}{\pgfqpoint{6.277942in}{1.142799in}}{\pgfqpoint{6.282986in}{1.137755in}}%
\pgfpathcurveto{\pgfqpoint{6.288030in}{1.132712in}}{\pgfqpoint{6.294871in}{1.129878in}}{\pgfqpoint{6.302004in}{1.129878in}}%
\pgfpathclose%
\pgfusepath{stroke,fill}%
\end{pgfscope}%
\begin{pgfscope}%
\pgfpathrectangle{\pgfqpoint{4.985294in}{0.500000in}}{\pgfqpoint{1.764706in}{1.700000in}}%
\pgfusepath{clip}%
\pgfsetbuttcap%
\pgfsetroundjoin%
\definecolor{currentfill}{rgb}{0.964173,0.657587,0.500469}%
\pgfsetfillcolor{currentfill}%
\pgfsetlinewidth{0.311001pt}%
\definecolor{currentstroke}{rgb}{1.000000,1.000000,1.000000}%
\pgfsetstrokecolor{currentstroke}%
\pgfsetdash{}{0pt}%
\pgfpathmoveto{\pgfqpoint{6.417970in}{1.245131in}}%
\pgfpathcurveto{\pgfqpoint{6.425103in}{1.245131in}}{\pgfqpoint{6.431944in}{1.247965in}}{\pgfqpoint{6.436988in}{1.253008in}}%
\pgfpathcurveto{\pgfqpoint{6.442032in}{1.258052in}}{\pgfqpoint{6.444866in}{1.264894in}}{\pgfqpoint{6.444866in}{1.272027in}}%
\pgfpathcurveto{\pgfqpoint{6.444866in}{1.279159in}}{\pgfqpoint{6.442032in}{1.286001in}}{\pgfqpoint{6.436988in}{1.291045in}}%
\pgfpathcurveto{\pgfqpoint{6.431944in}{1.296088in}}{\pgfqpoint{6.425103in}{1.298922in}}{\pgfqpoint{6.417970in}{1.298922in}}%
\pgfpathcurveto{\pgfqpoint{6.410837in}{1.298922in}}{\pgfqpoint{6.403995in}{1.296088in}}{\pgfqpoint{6.398952in}{1.291045in}}%
\pgfpathcurveto{\pgfqpoint{6.393908in}{1.286001in}}{\pgfqpoint{6.391074in}{1.279159in}}{\pgfqpoint{6.391074in}{1.272027in}}%
\pgfpathcurveto{\pgfqpoint{6.391074in}{1.264894in}}{\pgfqpoint{6.393908in}{1.258052in}}{\pgfqpoint{6.398952in}{1.253008in}}%
\pgfpathcurveto{\pgfqpoint{6.403995in}{1.247965in}}{\pgfqpoint{6.410837in}{1.245131in}}{\pgfqpoint{6.417970in}{1.245131in}}%
\pgfpathclose%
\pgfusepath{stroke,fill}%
\end{pgfscope}%
\begin{pgfscope}%
\pgfpathrectangle{\pgfqpoint{4.985294in}{0.500000in}}{\pgfqpoint{1.764706in}{1.700000in}}%
\pgfusepath{clip}%
\pgfsetbuttcap%
\pgfsetroundjoin%
\definecolor{currentfill}{rgb}{0.981377,0.920617,0.865369}%
\pgfsetfillcolor{currentfill}%
\pgfsetlinewidth{0.311001pt}%
\definecolor{currentstroke}{rgb}{1.000000,1.000000,1.000000}%
\pgfsetstrokecolor{currentstroke}%
\pgfsetdash{}{0pt}%
\pgfpathmoveto{\pgfqpoint{6.298237in}{1.463099in}}%
\pgfpathcurveto{\pgfqpoint{6.305370in}{1.463099in}}{\pgfqpoint{6.312211in}{1.465933in}}{\pgfqpoint{6.317255in}{1.470976in}}%
\pgfpathcurveto{\pgfqpoint{6.322299in}{1.476020in}}{\pgfqpoint{6.325132in}{1.482862in}}{\pgfqpoint{6.325132in}{1.489994in}}%
\pgfpathcurveto{\pgfqpoint{6.325132in}{1.497127in}}{\pgfqpoint{6.322299in}{1.503969in}}{\pgfqpoint{6.317255in}{1.509012in}}%
\pgfpathcurveto{\pgfqpoint{6.312211in}{1.514056in}}{\pgfqpoint{6.305370in}{1.516890in}}{\pgfqpoint{6.298237in}{1.516890in}}%
\pgfpathcurveto{\pgfqpoint{6.291104in}{1.516890in}}{\pgfqpoint{6.284262in}{1.514056in}}{\pgfqpoint{6.279219in}{1.509012in}}%
\pgfpathcurveto{\pgfqpoint{6.274175in}{1.503969in}}{\pgfqpoint{6.271341in}{1.497127in}}{\pgfqpoint{6.271341in}{1.489994in}}%
\pgfpathcurveto{\pgfqpoint{6.271341in}{1.482862in}}{\pgfqpoint{6.274175in}{1.476020in}}{\pgfqpoint{6.279219in}{1.470976in}}%
\pgfpathcurveto{\pgfqpoint{6.284262in}{1.465933in}}{\pgfqpoint{6.291104in}{1.463099in}}{\pgfqpoint{6.298237in}{1.463099in}}%
\pgfpathclose%
\pgfusepath{stroke,fill}%
\end{pgfscope}%
\begin{pgfscope}%
\pgfpathrectangle{\pgfqpoint{4.985294in}{0.500000in}}{\pgfqpoint{1.764706in}{1.700000in}}%
\pgfusepath{clip}%
\pgfsetbuttcap%
\pgfsetroundjoin%
\definecolor{currentfill}{rgb}{0.963884,0.644842,0.486120}%
\pgfsetfillcolor{currentfill}%
\pgfsetlinewidth{0.311001pt}%
\definecolor{currentstroke}{rgb}{1.000000,1.000000,1.000000}%
\pgfsetstrokecolor{currentstroke}%
\pgfsetdash{}{0pt}%
\pgfpathmoveto{\pgfqpoint{6.127014in}{1.739514in}}%
\pgfpathcurveto{\pgfqpoint{6.134146in}{1.739514in}}{\pgfqpoint{6.140988in}{1.742348in}}{\pgfqpoint{6.146032in}{1.747391in}}%
\pgfpathcurveto{\pgfqpoint{6.151075in}{1.752435in}}{\pgfqpoint{6.153909in}{1.759276in}}{\pgfqpoint{6.153909in}{1.766409in}}%
\pgfpathcurveto{\pgfqpoint{6.153909in}{1.773542in}}{\pgfqpoint{6.151075in}{1.780384in}}{\pgfqpoint{6.146032in}{1.785427in}}%
\pgfpathcurveto{\pgfqpoint{6.140988in}{1.790471in}}{\pgfqpoint{6.134146in}{1.793305in}}{\pgfqpoint{6.127014in}{1.793305in}}%
\pgfpathcurveto{\pgfqpoint{6.119881in}{1.793305in}}{\pgfqpoint{6.113039in}{1.790471in}}{\pgfqpoint{6.107995in}{1.785427in}}%
\pgfpathcurveto{\pgfqpoint{6.102952in}{1.780384in}}{\pgfqpoint{6.100118in}{1.773542in}}{\pgfqpoint{6.100118in}{1.766409in}}%
\pgfpathcurveto{\pgfqpoint{6.100118in}{1.759276in}}{\pgfqpoint{6.102952in}{1.752435in}}{\pgfqpoint{6.107995in}{1.747391in}}%
\pgfpathcurveto{\pgfqpoint{6.113039in}{1.742348in}}{\pgfqpoint{6.119881in}{1.739514in}}{\pgfqpoint{6.127014in}{1.739514in}}%
\pgfpathclose%
\pgfusepath{stroke,fill}%
\end{pgfscope}%
\begin{pgfscope}%
\pgfpathrectangle{\pgfqpoint{4.985294in}{0.500000in}}{\pgfqpoint{1.764706in}{1.700000in}}%
\pgfusepath{clip}%
\pgfsetbuttcap%
\pgfsetroundjoin%
\definecolor{currentfill}{rgb}{0.908486,0.245685,0.245983}%
\pgfsetfillcolor{currentfill}%
\pgfsetlinewidth{0.311001pt}%
\definecolor{currentstroke}{rgb}{1.000000,1.000000,1.000000}%
\pgfsetstrokecolor{currentstroke}%
\pgfsetdash{}{0pt}%
\pgfpathmoveto{\pgfqpoint{5.623062in}{1.837985in}}%
\pgfpathcurveto{\pgfqpoint{5.630195in}{1.837985in}}{\pgfqpoint{5.637037in}{1.840819in}}{\pgfqpoint{5.642081in}{1.845863in}}%
\pgfpathcurveto{\pgfqpoint{5.647124in}{1.850906in}}{\pgfqpoint{5.649958in}{1.857748in}}{\pgfqpoint{5.649958in}{1.864881in}}%
\pgfpathcurveto{\pgfqpoint{5.649958in}{1.872014in}}{\pgfqpoint{5.647124in}{1.878855in}}{\pgfqpoint{5.642081in}{1.883899in}}%
\pgfpathcurveto{\pgfqpoint{5.637037in}{1.888943in}}{\pgfqpoint{5.630195in}{1.891777in}}{\pgfqpoint{5.623062in}{1.891777in}}%
\pgfpathcurveto{\pgfqpoint{5.615930in}{1.891777in}}{\pgfqpoint{5.609088in}{1.888943in}}{\pgfqpoint{5.604044in}{1.883899in}}%
\pgfpathcurveto{\pgfqpoint{5.599001in}{1.878855in}}{\pgfqpoint{5.596167in}{1.872014in}}{\pgfqpoint{5.596167in}{1.864881in}}%
\pgfpathcurveto{\pgfqpoint{5.596167in}{1.857748in}}{\pgfqpoint{5.599001in}{1.850906in}}{\pgfqpoint{5.604044in}{1.845863in}}%
\pgfpathcurveto{\pgfqpoint{5.609088in}{1.840819in}}{\pgfqpoint{5.615930in}{1.837985in}}{\pgfqpoint{5.623062in}{1.837985in}}%
\pgfpathclose%
\pgfusepath{stroke,fill}%
\end{pgfscope}%
\begin{pgfscope}%
\pgfpathrectangle{\pgfqpoint{4.985294in}{0.500000in}}{\pgfqpoint{1.764706in}{1.700000in}}%
\pgfusepath{clip}%
\pgfsetbuttcap%
\pgfsetroundjoin%
\definecolor{currentfill}{rgb}{0.617923,0.103253,0.357601}%
\pgfsetfillcolor{currentfill}%
\pgfsetlinewidth{0.311001pt}%
\definecolor{currentstroke}{rgb}{1.000000,1.000000,1.000000}%
\pgfsetstrokecolor{currentstroke}%
\pgfsetdash{}{0pt}%
\pgfpathmoveto{\pgfqpoint{5.233986in}{1.094204in}}%
\pgfpathcurveto{\pgfqpoint{5.241119in}{1.094204in}}{\pgfqpoint{5.247961in}{1.097038in}}{\pgfqpoint{5.253005in}{1.102082in}}%
\pgfpathcurveto{\pgfqpoint{5.258048in}{1.107125in}}{\pgfqpoint{5.260882in}{1.113967in}}{\pgfqpoint{5.260882in}{1.121100in}}%
\pgfpathcurveto{\pgfqpoint{5.260882in}{1.128233in}}{\pgfqpoint{5.258048in}{1.135074in}}{\pgfqpoint{5.253005in}{1.140118in}}%
\pgfpathcurveto{\pgfqpoint{5.247961in}{1.145162in}}{\pgfqpoint{5.241119in}{1.147996in}}{\pgfqpoint{5.233986in}{1.147996in}}%
\pgfpathcurveto{\pgfqpoint{5.226854in}{1.147996in}}{\pgfqpoint{5.220012in}{1.145162in}}{\pgfqpoint{5.214968in}{1.140118in}}%
\pgfpathcurveto{\pgfqpoint{5.209925in}{1.135074in}}{\pgfqpoint{5.207091in}{1.128233in}}{\pgfqpoint{5.207091in}{1.121100in}}%
\pgfpathcurveto{\pgfqpoint{5.207091in}{1.113967in}}{\pgfqpoint{5.209925in}{1.107125in}}{\pgfqpoint{5.214968in}{1.102082in}}%
\pgfpathcurveto{\pgfqpoint{5.220012in}{1.097038in}}{\pgfqpoint{5.226854in}{1.094204in}}{\pgfqpoint{5.233986in}{1.094204in}}%
\pgfpathclose%
\pgfusepath{stroke,fill}%
\end{pgfscope}%
\begin{pgfscope}%
\pgfpathrectangle{\pgfqpoint{4.985294in}{0.500000in}}{\pgfqpoint{1.764706in}{1.700000in}}%
\pgfusepath{clip}%
\pgfsetbuttcap%
\pgfsetroundjoin%
\definecolor{currentfill}{rgb}{0.965302,0.713942,0.568499}%
\pgfsetfillcolor{currentfill}%
\pgfsetlinewidth{0.311001pt}%
\definecolor{currentstroke}{rgb}{1.000000,1.000000,1.000000}%
\pgfsetstrokecolor{currentstroke}%
\pgfsetdash{}{0pt}%
\pgfpathmoveto{\pgfqpoint{6.211824in}{0.922648in}}%
\pgfpathcurveto{\pgfqpoint{6.218956in}{0.922648in}}{\pgfqpoint{6.225798in}{0.925482in}}{\pgfqpoint{6.230842in}{0.930526in}}%
\pgfpathcurveto{\pgfqpoint{6.235885in}{0.935569in}}{\pgfqpoint{6.238719in}{0.942411in}}{\pgfqpoint{6.238719in}{0.949544in}}%
\pgfpathcurveto{\pgfqpoint{6.238719in}{0.956677in}}{\pgfqpoint{6.235885in}{0.963518in}}{\pgfqpoint{6.230842in}{0.968562in}}%
\pgfpathcurveto{\pgfqpoint{6.225798in}{0.973606in}}{\pgfqpoint{6.218956in}{0.976439in}}{\pgfqpoint{6.211824in}{0.976439in}}%
\pgfpathcurveto{\pgfqpoint{6.204691in}{0.976439in}}{\pgfqpoint{6.197849in}{0.973606in}}{\pgfqpoint{6.192805in}{0.968562in}}%
\pgfpathcurveto{\pgfqpoint{6.187762in}{0.963518in}}{\pgfqpoint{6.184928in}{0.956677in}}{\pgfqpoint{6.184928in}{0.949544in}}%
\pgfpathcurveto{\pgfqpoint{6.184928in}{0.942411in}}{\pgfqpoint{6.187762in}{0.935569in}}{\pgfqpoint{6.192805in}{0.930526in}}%
\pgfpathcurveto{\pgfqpoint{6.197849in}{0.925482in}}{\pgfqpoint{6.204691in}{0.922648in}}{\pgfqpoint{6.211824in}{0.922648in}}%
\pgfpathclose%
\pgfusepath{stroke,fill}%
\end{pgfscope}%
\begin{pgfscope}%
\pgfpathrectangle{\pgfqpoint{4.985294in}{0.500000in}}{\pgfqpoint{1.764706in}{1.700000in}}%
\pgfusepath{clip}%
\pgfsetbuttcap%
\pgfsetroundjoin%
\definecolor{currentfill}{rgb}{0.964173,0.657587,0.500469}%
\pgfsetfillcolor{currentfill}%
\pgfsetlinewidth{0.311001pt}%
\definecolor{currentstroke}{rgb}{1.000000,1.000000,1.000000}%
\pgfsetstrokecolor{currentstroke}%
\pgfsetdash{}{0pt}%
\pgfpathmoveto{\pgfqpoint{6.128437in}{1.041844in}}%
\pgfpathcurveto{\pgfqpoint{6.135570in}{1.041844in}}{\pgfqpoint{6.142411in}{1.044678in}}{\pgfqpoint{6.147455in}{1.049721in}}%
\pgfpathcurveto{\pgfqpoint{6.152499in}{1.054765in}}{\pgfqpoint{6.155332in}{1.061607in}}{\pgfqpoint{6.155332in}{1.068740in}}%
\pgfpathcurveto{\pgfqpoint{6.155332in}{1.075872in}}{\pgfqpoint{6.152499in}{1.082714in}}{\pgfqpoint{6.147455in}{1.087758in}}%
\pgfpathcurveto{\pgfqpoint{6.142411in}{1.092801in}}{\pgfqpoint{6.135570in}{1.095635in}}{\pgfqpoint{6.128437in}{1.095635in}}%
\pgfpathcurveto{\pgfqpoint{6.121304in}{1.095635in}}{\pgfqpoint{6.114462in}{1.092801in}}{\pgfqpoint{6.109419in}{1.087758in}}%
\pgfpathcurveto{\pgfqpoint{6.104375in}{1.082714in}}{\pgfqpoint{6.101541in}{1.075872in}}{\pgfqpoint{6.101541in}{1.068740in}}%
\pgfpathcurveto{\pgfqpoint{6.101541in}{1.061607in}}{\pgfqpoint{6.104375in}{1.054765in}}{\pgfqpoint{6.109419in}{1.049721in}}%
\pgfpathcurveto{\pgfqpoint{6.114462in}{1.044678in}}{\pgfqpoint{6.121304in}{1.041844in}}{\pgfqpoint{6.128437in}{1.041844in}}%
\pgfpathclose%
\pgfusepath{stroke,fill}%
\end{pgfscope}%
\begin{pgfscope}%
\pgfpathrectangle{\pgfqpoint{4.985294in}{0.500000in}}{\pgfqpoint{1.764706in}{1.700000in}}%
\pgfusepath{clip}%
\pgfsetbuttcap%
\pgfsetroundjoin%
\definecolor{currentfill}{rgb}{0.963190,0.619109,0.458249}%
\pgfsetfillcolor{currentfill}%
\pgfsetlinewidth{0.311001pt}%
\definecolor{currentstroke}{rgb}{1.000000,1.000000,1.000000}%
\pgfsetstrokecolor{currentstroke}%
\pgfsetdash{}{0pt}%
\pgfpathmoveto{\pgfqpoint{5.325801in}{1.154223in}}%
\pgfpathcurveto{\pgfqpoint{5.332934in}{1.154223in}}{\pgfqpoint{5.339775in}{1.157057in}}{\pgfqpoint{5.344819in}{1.162100in}}%
\pgfpathcurveto{\pgfqpoint{5.349863in}{1.167144in}}{\pgfqpoint{5.352697in}{1.173986in}}{\pgfqpoint{5.352697in}{1.181118in}}%
\pgfpathcurveto{\pgfqpoint{5.352697in}{1.188251in}}{\pgfqpoint{5.349863in}{1.195093in}}{\pgfqpoint{5.344819in}{1.200137in}}%
\pgfpathcurveto{\pgfqpoint{5.339775in}{1.205180in}}{\pgfqpoint{5.332934in}{1.208014in}}{\pgfqpoint{5.325801in}{1.208014in}}%
\pgfpathcurveto{\pgfqpoint{5.318668in}{1.208014in}}{\pgfqpoint{5.311826in}{1.205180in}}{\pgfqpoint{5.306783in}{1.200137in}}%
\pgfpathcurveto{\pgfqpoint{5.301739in}{1.195093in}}{\pgfqpoint{5.298905in}{1.188251in}}{\pgfqpoint{5.298905in}{1.181118in}}%
\pgfpathcurveto{\pgfqpoint{5.298905in}{1.173986in}}{\pgfqpoint{5.301739in}{1.167144in}}{\pgfqpoint{5.306783in}{1.162100in}}%
\pgfpathcurveto{\pgfqpoint{5.311826in}{1.157057in}}{\pgfqpoint{5.318668in}{1.154223in}}{\pgfqpoint{5.325801in}{1.154223in}}%
\pgfpathclose%
\pgfusepath{stroke,fill}%
\end{pgfscope}%
\begin{pgfscope}%
\pgfpathrectangle{\pgfqpoint{4.985294in}{0.500000in}}{\pgfqpoint{1.764706in}{1.700000in}}%
\pgfusepath{clip}%
\pgfsetbuttcap%
\pgfsetroundjoin%
\definecolor{currentfill}{rgb}{0.964433,0.670254,0.515093}%
\pgfsetfillcolor{currentfill}%
\pgfsetlinewidth{0.311001pt}%
\definecolor{currentstroke}{rgb}{1.000000,1.000000,1.000000}%
\pgfsetstrokecolor{currentstroke}%
\pgfsetdash{}{0pt}%
\pgfpathmoveto{\pgfqpoint{5.586331in}{0.894126in}}%
\pgfpathcurveto{\pgfqpoint{5.593464in}{0.894126in}}{\pgfqpoint{5.600306in}{0.896960in}}{\pgfqpoint{5.605350in}{0.902003in}}%
\pgfpathcurveto{\pgfqpoint{5.610393in}{0.907047in}}{\pgfqpoint{5.613227in}{0.913889in}}{\pgfqpoint{5.613227in}{0.921022in}}%
\pgfpathcurveto{\pgfqpoint{5.613227in}{0.928154in}}{\pgfqpoint{5.610393in}{0.934996in}}{\pgfqpoint{5.605350in}{0.940040in}}%
\pgfpathcurveto{\pgfqpoint{5.600306in}{0.945083in}}{\pgfqpoint{5.593464in}{0.947917in}}{\pgfqpoint{5.586331in}{0.947917in}}%
\pgfpathcurveto{\pgfqpoint{5.579199in}{0.947917in}}{\pgfqpoint{5.572357in}{0.945083in}}{\pgfqpoint{5.567313in}{0.940040in}}%
\pgfpathcurveto{\pgfqpoint{5.562270in}{0.934996in}}{\pgfqpoint{5.559436in}{0.928154in}}{\pgfqpoint{5.559436in}{0.921022in}}%
\pgfpathcurveto{\pgfqpoint{5.559436in}{0.913889in}}{\pgfqpoint{5.562270in}{0.907047in}}{\pgfqpoint{5.567313in}{0.902003in}}%
\pgfpathcurveto{\pgfqpoint{5.572357in}{0.896960in}}{\pgfqpoint{5.579199in}{0.894126in}}{\pgfqpoint{5.586331in}{0.894126in}}%
\pgfpathclose%
\pgfusepath{stroke,fill}%
\end{pgfscope}%
\begin{pgfscope}%
\pgfpathrectangle{\pgfqpoint{4.985294in}{0.500000in}}{\pgfqpoint{1.764706in}{1.700000in}}%
\pgfusepath{clip}%
\pgfsetbuttcap%
\pgfsetroundjoin%
\definecolor{currentfill}{rgb}{0.972726,0.844889,0.754401}%
\pgfsetfillcolor{currentfill}%
\pgfsetlinewidth{0.311001pt}%
\definecolor{currentstroke}{rgb}{1.000000,1.000000,1.000000}%
\pgfsetstrokecolor{currentstroke}%
\pgfsetdash{}{0pt}%
\pgfpathmoveto{\pgfqpoint{5.481761in}{1.373192in}}%
\pgfpathcurveto{\pgfqpoint{5.488894in}{1.373192in}}{\pgfqpoint{5.495736in}{1.376026in}}{\pgfqpoint{5.500780in}{1.381069in}}%
\pgfpathcurveto{\pgfqpoint{5.505823in}{1.386113in}}{\pgfqpoint{5.508657in}{1.392955in}}{\pgfqpoint{5.508657in}{1.400087in}}%
\pgfpathcurveto{\pgfqpoint{5.508657in}{1.407220in}}{\pgfqpoint{5.505823in}{1.414062in}}{\pgfqpoint{5.500780in}{1.419106in}}%
\pgfpathcurveto{\pgfqpoint{5.495736in}{1.424149in}}{\pgfqpoint{5.488894in}{1.426983in}}{\pgfqpoint{5.481761in}{1.426983in}}%
\pgfpathcurveto{\pgfqpoint{5.474629in}{1.426983in}}{\pgfqpoint{5.467787in}{1.424149in}}{\pgfqpoint{5.462743in}{1.419106in}}%
\pgfpathcurveto{\pgfqpoint{5.457700in}{1.414062in}}{\pgfqpoint{5.454866in}{1.407220in}}{\pgfqpoint{5.454866in}{1.400087in}}%
\pgfpathcurveto{\pgfqpoint{5.454866in}{1.392955in}}{\pgfqpoint{5.457700in}{1.386113in}}{\pgfqpoint{5.462743in}{1.381069in}}%
\pgfpathcurveto{\pgfqpoint{5.467787in}{1.376026in}}{\pgfqpoint{5.474629in}{1.373192in}}{\pgfqpoint{5.481761in}{1.373192in}}%
\pgfpathclose%
\pgfusepath{stroke,fill}%
\end{pgfscope}%
\begin{pgfscope}%
\pgfpathrectangle{\pgfqpoint{4.985294in}{0.500000in}}{\pgfqpoint{1.764706in}{1.700000in}}%
\pgfusepath{clip}%
\pgfsetbuttcap%
\pgfsetroundjoin%
\definecolor{currentfill}{rgb}{0.967735,0.780441,0.659127}%
\pgfsetfillcolor{currentfill}%
\pgfsetlinewidth{0.311001pt}%
\definecolor{currentstroke}{rgb}{1.000000,1.000000,1.000000}%
\pgfsetstrokecolor{currentstroke}%
\pgfsetdash{}{0pt}%
\pgfpathmoveto{\pgfqpoint{6.190676in}{0.954279in}}%
\pgfpathcurveto{\pgfqpoint{6.197809in}{0.954279in}}{\pgfqpoint{6.204651in}{0.957113in}}{\pgfqpoint{6.209694in}{0.962156in}}%
\pgfpathcurveto{\pgfqpoint{6.214738in}{0.967200in}}{\pgfqpoint{6.217572in}{0.974042in}}{\pgfqpoint{6.217572in}{0.981175in}}%
\pgfpathcurveto{\pgfqpoint{6.217572in}{0.988307in}}{\pgfqpoint{6.214738in}{0.995149in}}{\pgfqpoint{6.209694in}{1.000193in}}%
\pgfpathcurveto{\pgfqpoint{6.204651in}{1.005236in}}{\pgfqpoint{6.197809in}{1.008070in}}{\pgfqpoint{6.190676in}{1.008070in}}%
\pgfpathcurveto{\pgfqpoint{6.183543in}{1.008070in}}{\pgfqpoint{6.176702in}{1.005236in}}{\pgfqpoint{6.171658in}{1.000193in}}%
\pgfpathcurveto{\pgfqpoint{6.166615in}{0.995149in}}{\pgfqpoint{6.163781in}{0.988307in}}{\pgfqpoint{6.163781in}{0.981175in}}%
\pgfpathcurveto{\pgfqpoint{6.163781in}{0.974042in}}{\pgfqpoint{6.166615in}{0.967200in}}{\pgfqpoint{6.171658in}{0.962156in}}%
\pgfpathcurveto{\pgfqpoint{6.176702in}{0.957113in}}{\pgfqpoint{6.183543in}{0.954279in}}{\pgfqpoint{6.190676in}{0.954279in}}%
\pgfpathclose%
\pgfusepath{stroke,fill}%
\end{pgfscope}%
\begin{pgfscope}%
\pgfpathrectangle{\pgfqpoint{4.985294in}{0.500000in}}{\pgfqpoint{1.764706in}{1.700000in}}%
\pgfusepath{clip}%
\pgfsetbuttcap%
\pgfsetroundjoin%
\definecolor{currentfill}{rgb}{0.968509,0.792226,0.676405}%
\pgfsetfillcolor{currentfill}%
\pgfsetlinewidth{0.311001pt}%
\definecolor{currentstroke}{rgb}{1.000000,1.000000,1.000000}%
\pgfsetstrokecolor{currentstroke}%
\pgfsetdash{}{0pt}%
\pgfpathmoveto{\pgfqpoint{5.434931in}{1.011657in}}%
\pgfpathcurveto{\pgfqpoint{5.442064in}{1.011657in}}{\pgfqpoint{5.448905in}{1.014490in}}{\pgfqpoint{5.453949in}{1.019534in}}%
\pgfpathcurveto{\pgfqpoint{5.458993in}{1.024578in}}{\pgfqpoint{5.461827in}{1.031419in}}{\pgfqpoint{5.461827in}{1.038552in}}%
\pgfpathcurveto{\pgfqpoint{5.461827in}{1.045685in}}{\pgfqpoint{5.458993in}{1.052527in}}{\pgfqpoint{5.453949in}{1.057570in}}%
\pgfpathcurveto{\pgfqpoint{5.448905in}{1.062614in}}{\pgfqpoint{5.442064in}{1.065448in}}{\pgfqpoint{5.434931in}{1.065448in}}%
\pgfpathcurveto{\pgfqpoint{5.427798in}{1.065448in}}{\pgfqpoint{5.420956in}{1.062614in}}{\pgfqpoint{5.415913in}{1.057570in}}%
\pgfpathcurveto{\pgfqpoint{5.410869in}{1.052527in}}{\pgfqpoint{5.408035in}{1.045685in}}{\pgfqpoint{5.408035in}{1.038552in}}%
\pgfpathcurveto{\pgfqpoint{5.408035in}{1.031419in}}{\pgfqpoint{5.410869in}{1.024578in}}{\pgfqpoint{5.415913in}{1.019534in}}%
\pgfpathcurveto{\pgfqpoint{5.420956in}{1.014490in}}{\pgfqpoint{5.427798in}{1.011657in}}{\pgfqpoint{5.434931in}{1.011657in}}%
\pgfpathclose%
\pgfusepath{stroke,fill}%
\end{pgfscope}%
\begin{pgfscope}%
\pgfpathrectangle{\pgfqpoint{4.985294in}{0.500000in}}{\pgfqpoint{1.764706in}{1.700000in}}%
\pgfusepath{clip}%
\pgfsetbuttcap%
\pgfsetroundjoin%
\definecolor{currentfill}{rgb}{0.978376,0.897317,0.831308}%
\pgfsetfillcolor{currentfill}%
\pgfsetlinewidth{0.311001pt}%
\definecolor{currentstroke}{rgb}{1.000000,1.000000,1.000000}%
\pgfsetstrokecolor{currentstroke}%
\pgfsetdash{}{0pt}%
\pgfpathmoveto{\pgfqpoint{6.314122in}{1.175659in}}%
\pgfpathcurveto{\pgfqpoint{6.321255in}{1.175659in}}{\pgfqpoint{6.328096in}{1.178493in}}{\pgfqpoint{6.333140in}{1.183536in}}%
\pgfpathcurveto{\pgfqpoint{6.338184in}{1.188580in}}{\pgfqpoint{6.341017in}{1.195422in}}{\pgfqpoint{6.341017in}{1.202555in}}%
\pgfpathcurveto{\pgfqpoint{6.341017in}{1.209687in}}{\pgfqpoint{6.338184in}{1.216529in}}{\pgfqpoint{6.333140in}{1.221573in}}%
\pgfpathcurveto{\pgfqpoint{6.328096in}{1.226616in}}{\pgfqpoint{6.321255in}{1.229450in}}{\pgfqpoint{6.314122in}{1.229450in}}%
\pgfpathcurveto{\pgfqpoint{6.306989in}{1.229450in}}{\pgfqpoint{6.300147in}{1.226616in}}{\pgfqpoint{6.295104in}{1.221573in}}%
\pgfpathcurveto{\pgfqpoint{6.290060in}{1.216529in}}{\pgfqpoint{6.287226in}{1.209687in}}{\pgfqpoint{6.287226in}{1.202555in}}%
\pgfpathcurveto{\pgfqpoint{6.287226in}{1.195422in}}{\pgfqpoint{6.290060in}{1.188580in}}{\pgfqpoint{6.295104in}{1.183536in}}%
\pgfpathcurveto{\pgfqpoint{6.300147in}{1.178493in}}{\pgfqpoint{6.306989in}{1.175659in}}{\pgfqpoint{6.314122in}{1.175659in}}%
\pgfpathclose%
\pgfusepath{stroke,fill}%
\end{pgfscope}%
\begin{pgfscope}%
\pgfpathrectangle{\pgfqpoint{4.985294in}{0.500000in}}{\pgfqpoint{1.764706in}{1.700000in}}%
\pgfusepath{clip}%
\pgfsetbuttcap%
\pgfsetroundjoin%
\definecolor{currentfill}{rgb}{0.977657,0.891500,0.822809}%
\pgfsetfillcolor{currentfill}%
\pgfsetlinewidth{0.311001pt}%
\definecolor{currentstroke}{rgb}{1.000000,1.000000,1.000000}%
\pgfsetstrokecolor{currentstroke}%
\pgfsetdash{}{0pt}%
\pgfpathmoveto{\pgfqpoint{5.453770in}{1.215029in}}%
\pgfpathcurveto{\pgfqpoint{5.460903in}{1.215029in}}{\pgfqpoint{5.467744in}{1.217863in}}{\pgfqpoint{5.472788in}{1.222906in}}%
\pgfpathcurveto{\pgfqpoint{5.477832in}{1.227950in}}{\pgfqpoint{5.480666in}{1.234792in}}{\pgfqpoint{5.480666in}{1.241925in}}%
\pgfpathcurveto{\pgfqpoint{5.480666in}{1.249057in}}{\pgfqpoint{5.477832in}{1.255899in}}{\pgfqpoint{5.472788in}{1.260943in}}%
\pgfpathcurveto{\pgfqpoint{5.467744in}{1.265986in}}{\pgfqpoint{5.460903in}{1.268820in}}{\pgfqpoint{5.453770in}{1.268820in}}%
\pgfpathcurveto{\pgfqpoint{5.446637in}{1.268820in}}{\pgfqpoint{5.439795in}{1.265986in}}{\pgfqpoint{5.434752in}{1.260943in}}%
\pgfpathcurveto{\pgfqpoint{5.429708in}{1.255899in}}{\pgfqpoint{5.426874in}{1.249057in}}{\pgfqpoint{5.426874in}{1.241925in}}%
\pgfpathcurveto{\pgfqpoint{5.426874in}{1.234792in}}{\pgfqpoint{5.429708in}{1.227950in}}{\pgfqpoint{5.434752in}{1.222906in}}%
\pgfpathcurveto{\pgfqpoint{5.439795in}{1.217863in}}{\pgfqpoint{5.446637in}{1.215029in}}{\pgfqpoint{5.453770in}{1.215029in}}%
\pgfpathclose%
\pgfusepath{stroke,fill}%
\end{pgfscope}%
\begin{pgfscope}%
\pgfpathrectangle{\pgfqpoint{4.985294in}{0.500000in}}{\pgfqpoint{1.764706in}{1.700000in}}%
\pgfusepath{clip}%
\pgfsetbuttcap%
\pgfsetroundjoin%
\definecolor{currentfill}{rgb}{0.979124,0.903132,0.839793}%
\pgfsetfillcolor{currentfill}%
\pgfsetlinewidth{0.311001pt}%
\definecolor{currentstroke}{rgb}{1.000000,1.000000,1.000000}%
\pgfsetstrokecolor{currentstroke}%
\pgfsetdash{}{0pt}%
\pgfpathmoveto{\pgfqpoint{5.397909in}{1.264594in}}%
\pgfpathcurveto{\pgfqpoint{5.405042in}{1.264594in}}{\pgfqpoint{5.411884in}{1.267428in}}{\pgfqpoint{5.416927in}{1.272471in}}%
\pgfpathcurveto{\pgfqpoint{5.421971in}{1.277515in}}{\pgfqpoint{5.424805in}{1.284357in}}{\pgfqpoint{5.424805in}{1.291489in}}%
\pgfpathcurveto{\pgfqpoint{5.424805in}{1.298622in}}{\pgfqpoint{5.421971in}{1.305464in}}{\pgfqpoint{5.416927in}{1.310507in}}%
\pgfpathcurveto{\pgfqpoint{5.411884in}{1.315551in}}{\pgfqpoint{5.405042in}{1.318385in}}{\pgfqpoint{5.397909in}{1.318385in}}%
\pgfpathcurveto{\pgfqpoint{5.390776in}{1.318385in}}{\pgfqpoint{5.383935in}{1.315551in}}{\pgfqpoint{5.378891in}{1.310507in}}%
\pgfpathcurveto{\pgfqpoint{5.373847in}{1.305464in}}{\pgfqpoint{5.371013in}{1.298622in}}{\pgfqpoint{5.371013in}{1.291489in}}%
\pgfpathcurveto{\pgfqpoint{5.371013in}{1.284357in}}{\pgfqpoint{5.373847in}{1.277515in}}{\pgfqpoint{5.378891in}{1.272471in}}%
\pgfpathcurveto{\pgfqpoint{5.383935in}{1.267428in}}{\pgfqpoint{5.390776in}{1.264594in}}{\pgfqpoint{5.397909in}{1.264594in}}%
\pgfpathclose%
\pgfusepath{stroke,fill}%
\end{pgfscope}%
\begin{pgfscope}%
\pgfpathrectangle{\pgfqpoint{4.985294in}{0.500000in}}{\pgfqpoint{1.764706in}{1.700000in}}%
\pgfusepath{clip}%
\pgfsetbuttcap%
\pgfsetroundjoin%
\definecolor{currentfill}{rgb}{0.967735,0.780441,0.659127}%
\pgfsetfillcolor{currentfill}%
\pgfsetlinewidth{0.311001pt}%
\definecolor{currentstroke}{rgb}{1.000000,1.000000,1.000000}%
\pgfsetstrokecolor{currentstroke}%
\pgfsetdash{}{0pt}%
\pgfpathmoveto{\pgfqpoint{5.507575in}{1.215213in}}%
\pgfpathcurveto{\pgfqpoint{5.514708in}{1.215213in}}{\pgfqpoint{5.521549in}{1.218047in}}{\pgfqpoint{5.526593in}{1.223091in}}%
\pgfpathcurveto{\pgfqpoint{5.531637in}{1.228135in}}{\pgfqpoint{5.534470in}{1.234976in}}{\pgfqpoint{5.534470in}{1.242109in}}%
\pgfpathcurveto{\pgfqpoint{5.534470in}{1.249242in}}{\pgfqpoint{5.531637in}{1.256084in}}{\pgfqpoint{5.526593in}{1.261127in}}%
\pgfpathcurveto{\pgfqpoint{5.521549in}{1.266171in}}{\pgfqpoint{5.514708in}{1.269005in}}{\pgfqpoint{5.507575in}{1.269005in}}%
\pgfpathcurveto{\pgfqpoint{5.500442in}{1.269005in}}{\pgfqpoint{5.493600in}{1.266171in}}{\pgfqpoint{5.488557in}{1.261127in}}%
\pgfpathcurveto{\pgfqpoint{5.483513in}{1.256084in}}{\pgfqpoint{5.480679in}{1.249242in}}{\pgfqpoint{5.480679in}{1.242109in}}%
\pgfpathcurveto{\pgfqpoint{5.480679in}{1.234976in}}{\pgfqpoint{5.483513in}{1.228135in}}{\pgfqpoint{5.488557in}{1.223091in}}%
\pgfpathcurveto{\pgfqpoint{5.493600in}{1.218047in}}{\pgfqpoint{5.500442in}{1.215213in}}{\pgfqpoint{5.507575in}{1.215213in}}%
\pgfpathclose%
\pgfusepath{stroke,fill}%
\end{pgfscope}%
\begin{pgfscope}%
\pgfpathrectangle{\pgfqpoint{4.985294in}{0.500000in}}{\pgfqpoint{1.764706in}{1.700000in}}%
\pgfusepath{clip}%
\pgfsetbuttcap%
\pgfsetroundjoin%
\definecolor{currentfill}{rgb}{0.976961,0.885681,0.814303}%
\pgfsetfillcolor{currentfill}%
\pgfsetlinewidth{0.311001pt}%
\definecolor{currentstroke}{rgb}{1.000000,1.000000,1.000000}%
\pgfsetstrokecolor{currentstroke}%
\pgfsetdash{}{0pt}%
\pgfpathmoveto{\pgfqpoint{6.273839in}{1.286972in}}%
\pgfpathcurveto{\pgfqpoint{6.280972in}{1.286972in}}{\pgfqpoint{6.287814in}{1.289806in}}{\pgfqpoint{6.292858in}{1.294850in}}%
\pgfpathcurveto{\pgfqpoint{6.297901in}{1.299894in}}{\pgfqpoint{6.300735in}{1.306735in}}{\pgfqpoint{6.300735in}{1.313868in}}%
\pgfpathcurveto{\pgfqpoint{6.300735in}{1.321001in}}{\pgfqpoint{6.297901in}{1.327843in}}{\pgfqpoint{6.292858in}{1.332886in}}%
\pgfpathcurveto{\pgfqpoint{6.287814in}{1.337930in}}{\pgfqpoint{6.280972in}{1.340764in}}{\pgfqpoint{6.273839in}{1.340764in}}%
\pgfpathcurveto{\pgfqpoint{6.266707in}{1.340764in}}{\pgfqpoint{6.259865in}{1.337930in}}{\pgfqpoint{6.254821in}{1.332886in}}%
\pgfpathcurveto{\pgfqpoint{6.249778in}{1.327843in}}{\pgfqpoint{6.246944in}{1.321001in}}{\pgfqpoint{6.246944in}{1.313868in}}%
\pgfpathcurveto{\pgfqpoint{6.246944in}{1.306735in}}{\pgfqpoint{6.249778in}{1.299894in}}{\pgfqpoint{6.254821in}{1.294850in}}%
\pgfpathcurveto{\pgfqpoint{6.259865in}{1.289806in}}{\pgfqpoint{6.266707in}{1.286972in}}{\pgfqpoint{6.273839in}{1.286972in}}%
\pgfpathclose%
\pgfusepath{stroke,fill}%
\end{pgfscope}%
\begin{pgfscope}%
\pgfpathrectangle{\pgfqpoint{4.985294in}{0.500000in}}{\pgfqpoint{1.764706in}{1.700000in}}%
\pgfusepath{clip}%
\pgfsetbuttcap%
\pgfsetroundjoin%
\definecolor{currentfill}{rgb}{0.973832,0.856556,0.771584}%
\pgfsetfillcolor{currentfill}%
\pgfsetlinewidth{0.311001pt}%
\definecolor{currentstroke}{rgb}{1.000000,1.000000,1.000000}%
\pgfsetstrokecolor{currentstroke}%
\pgfsetdash{}{0pt}%
\pgfpathmoveto{\pgfqpoint{5.489945in}{1.461321in}}%
\pgfpathcurveto{\pgfqpoint{5.497078in}{1.461321in}}{\pgfqpoint{5.503919in}{1.464155in}}{\pgfqpoint{5.508963in}{1.469199in}}%
\pgfpathcurveto{\pgfqpoint{5.514007in}{1.474243in}}{\pgfqpoint{5.516841in}{1.481084in}}{\pgfqpoint{5.516841in}{1.488217in}}%
\pgfpathcurveto{\pgfqpoint{5.516841in}{1.495350in}}{\pgfqpoint{5.514007in}{1.502192in}}{\pgfqpoint{5.508963in}{1.507235in}}%
\pgfpathcurveto{\pgfqpoint{5.503919in}{1.512279in}}{\pgfqpoint{5.497078in}{1.515113in}}{\pgfqpoint{5.489945in}{1.515113in}}%
\pgfpathcurveto{\pgfqpoint{5.482812in}{1.515113in}}{\pgfqpoint{5.475970in}{1.512279in}}{\pgfqpoint{5.470927in}{1.507235in}}%
\pgfpathcurveto{\pgfqpoint{5.465883in}{1.502192in}}{\pgfqpoint{5.463049in}{1.495350in}}{\pgfqpoint{5.463049in}{1.488217in}}%
\pgfpathcurveto{\pgfqpoint{5.463049in}{1.481084in}}{\pgfqpoint{5.465883in}{1.474243in}}{\pgfqpoint{5.470927in}{1.469199in}}%
\pgfpathcurveto{\pgfqpoint{5.475970in}{1.464155in}}{\pgfqpoint{5.482812in}{1.461321in}}{\pgfqpoint{5.489945in}{1.461321in}}%
\pgfpathclose%
\pgfusepath{stroke,fill}%
\end{pgfscope}%
\begin{pgfscope}%
\pgfpathrectangle{\pgfqpoint{4.985294in}{0.500000in}}{\pgfqpoint{1.764706in}{1.700000in}}%
\pgfusepath{clip}%
\pgfsetbuttcap%
\pgfsetroundjoin%
\definecolor{currentfill}{rgb}{0.955103,0.477872,0.328626}%
\pgfsetfillcolor{currentfill}%
\pgfsetlinewidth{0.311001pt}%
\definecolor{currentstroke}{rgb}{1.000000,1.000000,1.000000}%
\pgfsetstrokecolor{currentstroke}%
\pgfsetdash{}{0pt}%
\pgfpathmoveto{\pgfqpoint{6.182098in}{1.777384in}}%
\pgfpathcurveto{\pgfqpoint{6.189231in}{1.777384in}}{\pgfqpoint{6.196073in}{1.780218in}}{\pgfqpoint{6.201116in}{1.785262in}}%
\pgfpathcurveto{\pgfqpoint{6.206160in}{1.790305in}}{\pgfqpoint{6.208994in}{1.797147in}}{\pgfqpoint{6.208994in}{1.804280in}}%
\pgfpathcurveto{\pgfqpoint{6.208994in}{1.811413in}}{\pgfqpoint{6.206160in}{1.818254in}}{\pgfqpoint{6.201116in}{1.823298in}}%
\pgfpathcurveto{\pgfqpoint{6.196073in}{1.828342in}}{\pgfqpoint{6.189231in}{1.831176in}}{\pgfqpoint{6.182098in}{1.831176in}}%
\pgfpathcurveto{\pgfqpoint{6.174965in}{1.831176in}}{\pgfqpoint{6.168124in}{1.828342in}}{\pgfqpoint{6.163080in}{1.823298in}}%
\pgfpathcurveto{\pgfqpoint{6.158036in}{1.818254in}}{\pgfqpoint{6.155203in}{1.811413in}}{\pgfqpoint{6.155203in}{1.804280in}}%
\pgfpathcurveto{\pgfqpoint{6.155203in}{1.797147in}}{\pgfqpoint{6.158036in}{1.790305in}}{\pgfqpoint{6.163080in}{1.785262in}}%
\pgfpathcurveto{\pgfqpoint{6.168124in}{1.780218in}}{\pgfqpoint{6.174965in}{1.777384in}}{\pgfqpoint{6.182098in}{1.777384in}}%
\pgfpathclose%
\pgfusepath{stroke,fill}%
\end{pgfscope}%
\begin{pgfscope}%
\pgfpathrectangle{\pgfqpoint{4.985294in}{0.500000in}}{\pgfqpoint{1.764706in}{1.700000in}}%
\pgfusepath{clip}%
\pgfsetbuttcap%
\pgfsetroundjoin%
\definecolor{currentfill}{rgb}{0.976961,0.885681,0.814303}%
\pgfsetfillcolor{currentfill}%
\pgfsetlinewidth{0.311001pt}%
\definecolor{currentstroke}{rgb}{1.000000,1.000000,1.000000}%
\pgfsetstrokecolor{currentstroke}%
\pgfsetdash{}{0pt}%
\pgfpathmoveto{\pgfqpoint{5.389043in}{1.247131in}}%
\pgfpathcurveto{\pgfqpoint{5.396176in}{1.247131in}}{\pgfqpoint{5.403018in}{1.249965in}}{\pgfqpoint{5.408061in}{1.255009in}}%
\pgfpathcurveto{\pgfqpoint{5.413105in}{1.260052in}}{\pgfqpoint{5.415939in}{1.266894in}}{\pgfqpoint{5.415939in}{1.274027in}}%
\pgfpathcurveto{\pgfqpoint{5.415939in}{1.281160in}}{\pgfqpoint{5.413105in}{1.288001in}}{\pgfqpoint{5.408061in}{1.293045in}}%
\pgfpathcurveto{\pgfqpoint{5.403018in}{1.298089in}}{\pgfqpoint{5.396176in}{1.300922in}}{\pgfqpoint{5.389043in}{1.300922in}}%
\pgfpathcurveto{\pgfqpoint{5.381910in}{1.300922in}}{\pgfqpoint{5.375069in}{1.298089in}}{\pgfqpoint{5.370025in}{1.293045in}}%
\pgfpathcurveto{\pgfqpoint{5.364981in}{1.288001in}}{\pgfqpoint{5.362147in}{1.281160in}}{\pgfqpoint{5.362147in}{1.274027in}}%
\pgfpathcurveto{\pgfqpoint{5.362147in}{1.266894in}}{\pgfqpoint{5.364981in}{1.260052in}}{\pgfqpoint{5.370025in}{1.255009in}}%
\pgfpathcurveto{\pgfqpoint{5.375069in}{1.249965in}}{\pgfqpoint{5.381910in}{1.247131in}}{\pgfqpoint{5.389043in}{1.247131in}}%
\pgfpathclose%
\pgfusepath{stroke,fill}%
\end{pgfscope}%
\begin{pgfscope}%
\pgfpathrectangle{\pgfqpoint{4.985294in}{0.500000in}}{\pgfqpoint{1.764706in}{1.700000in}}%
\pgfusepath{clip}%
\pgfsetbuttcap%
\pgfsetroundjoin%
\definecolor{currentfill}{rgb}{0.981377,0.920617,0.865369}%
\pgfsetfillcolor{currentfill}%
\pgfsetlinewidth{0.311001pt}%
\definecolor{currentstroke}{rgb}{1.000000,1.000000,1.000000}%
\pgfsetstrokecolor{currentstroke}%
\pgfsetdash{}{0pt}%
\pgfpathmoveto{\pgfqpoint{6.312653in}{1.431035in}}%
\pgfpathcurveto{\pgfqpoint{6.319786in}{1.431035in}}{\pgfqpoint{6.326627in}{1.433869in}}{\pgfqpoint{6.331671in}{1.438913in}}%
\pgfpathcurveto{\pgfqpoint{6.336715in}{1.443957in}}{\pgfqpoint{6.339549in}{1.450798in}}{\pgfqpoint{6.339549in}{1.457931in}}%
\pgfpathcurveto{\pgfqpoint{6.339549in}{1.465064in}}{\pgfqpoint{6.336715in}{1.471906in}}{\pgfqpoint{6.331671in}{1.476949in}}%
\pgfpathcurveto{\pgfqpoint{6.326627in}{1.481993in}}{\pgfqpoint{6.319786in}{1.484827in}}{\pgfqpoint{6.312653in}{1.484827in}}%
\pgfpathcurveto{\pgfqpoint{6.305520in}{1.484827in}}{\pgfqpoint{6.298679in}{1.481993in}}{\pgfqpoint{6.293635in}{1.476949in}}%
\pgfpathcurveto{\pgfqpoint{6.288591in}{1.471906in}}{\pgfqpoint{6.285757in}{1.465064in}}{\pgfqpoint{6.285757in}{1.457931in}}%
\pgfpathcurveto{\pgfqpoint{6.285757in}{1.450798in}}{\pgfqpoint{6.288591in}{1.443957in}}{\pgfqpoint{6.293635in}{1.438913in}}%
\pgfpathcurveto{\pgfqpoint{6.298679in}{1.433869in}}{\pgfqpoint{6.305520in}{1.431035in}}{\pgfqpoint{6.312653in}{1.431035in}}%
\pgfpathclose%
\pgfusepath{stroke,fill}%
\end{pgfscope}%
\begin{pgfscope}%
\pgfpathrectangle{\pgfqpoint{4.985294in}{0.500000in}}{\pgfqpoint{1.764706in}{1.700000in}}%
\pgfusepath{clip}%
\pgfsetbuttcap%
\pgfsetroundjoin%
\definecolor{currentfill}{rgb}{0.975018,0.868213,0.788710}%
\pgfsetfillcolor{currentfill}%
\pgfsetlinewidth{0.311001pt}%
\definecolor{currentstroke}{rgb}{1.000000,1.000000,1.000000}%
\pgfsetstrokecolor{currentstroke}%
\pgfsetdash{}{0pt}%
\pgfpathmoveto{\pgfqpoint{5.487919in}{1.103906in}}%
\pgfpathcurveto{\pgfqpoint{5.495052in}{1.103906in}}{\pgfqpoint{5.501893in}{1.106739in}}{\pgfqpoint{5.506937in}{1.111783in}}%
\pgfpathcurveto{\pgfqpoint{5.511981in}{1.116827in}}{\pgfqpoint{5.514815in}{1.123668in}}{\pgfqpoint{5.514815in}{1.130801in}}%
\pgfpathcurveto{\pgfqpoint{5.514815in}{1.137934in}}{\pgfqpoint{5.511981in}{1.144776in}}{\pgfqpoint{5.506937in}{1.149819in}}%
\pgfpathcurveto{\pgfqpoint{5.501893in}{1.154863in}}{\pgfqpoint{5.495052in}{1.157697in}}{\pgfqpoint{5.487919in}{1.157697in}}%
\pgfpathcurveto{\pgfqpoint{5.480786in}{1.157697in}}{\pgfqpoint{5.473944in}{1.154863in}}{\pgfqpoint{5.468901in}{1.149819in}}%
\pgfpathcurveto{\pgfqpoint{5.463857in}{1.144776in}}{\pgfqpoint{5.461023in}{1.137934in}}{\pgfqpoint{5.461023in}{1.130801in}}%
\pgfpathcurveto{\pgfqpoint{5.461023in}{1.123668in}}{\pgfqpoint{5.463857in}{1.116827in}}{\pgfqpoint{5.468901in}{1.111783in}}%
\pgfpathcurveto{\pgfqpoint{5.473944in}{1.106739in}}{\pgfqpoint{5.480786in}{1.103906in}}{\pgfqpoint{5.487919in}{1.103906in}}%
\pgfpathclose%
\pgfusepath{stroke,fill}%
\end{pgfscope}%
\begin{pgfscope}%
\pgfpathrectangle{\pgfqpoint{4.985294in}{0.500000in}}{\pgfqpoint{1.764706in}{1.700000in}}%
\pgfusepath{clip}%
\pgfsetbuttcap%
\pgfsetroundjoin%
\definecolor{currentfill}{rgb}{0.964679,0.682838,0.530002}%
\pgfsetfillcolor{currentfill}%
\pgfsetlinewidth{0.311001pt}%
\definecolor{currentstroke}{rgb}{1.000000,1.000000,1.000000}%
\pgfsetstrokecolor{currentstroke}%
\pgfsetdash{}{0pt}%
\pgfpathmoveto{\pgfqpoint{6.265431in}{1.703721in}}%
\pgfpathcurveto{\pgfqpoint{6.272564in}{1.703721in}}{\pgfqpoint{6.279406in}{1.706555in}}{\pgfqpoint{6.284450in}{1.711598in}}%
\pgfpathcurveto{\pgfqpoint{6.289493in}{1.716642in}}{\pgfqpoint{6.292327in}{1.723484in}}{\pgfqpoint{6.292327in}{1.730617in}}%
\pgfpathcurveto{\pgfqpoint{6.292327in}{1.737749in}}{\pgfqpoint{6.289493in}{1.744591in}}{\pgfqpoint{6.284450in}{1.749635in}}%
\pgfpathcurveto{\pgfqpoint{6.279406in}{1.754678in}}{\pgfqpoint{6.272564in}{1.757512in}}{\pgfqpoint{6.265431in}{1.757512in}}%
\pgfpathcurveto{\pgfqpoint{6.258299in}{1.757512in}}{\pgfqpoint{6.251457in}{1.754678in}}{\pgfqpoint{6.246413in}{1.749635in}}%
\pgfpathcurveto{\pgfqpoint{6.241370in}{1.744591in}}{\pgfqpoint{6.238536in}{1.737749in}}{\pgfqpoint{6.238536in}{1.730617in}}%
\pgfpathcurveto{\pgfqpoint{6.238536in}{1.723484in}}{\pgfqpoint{6.241370in}{1.716642in}}{\pgfqpoint{6.246413in}{1.711598in}}%
\pgfpathcurveto{\pgfqpoint{6.251457in}{1.706555in}}{\pgfqpoint{6.258299in}{1.703721in}}{\pgfqpoint{6.265431in}{1.703721in}}%
\pgfpathclose%
\pgfusepath{stroke,fill}%
\end{pgfscope}%
\begin{pgfscope}%
\pgfpathrectangle{\pgfqpoint{4.985294in}{0.500000in}}{\pgfqpoint{1.764706in}{1.700000in}}%
\pgfusepath{clip}%
\pgfsetbuttcap%
\pgfsetroundjoin%
\definecolor{currentfill}{rgb}{0.969359,0.803954,0.693832}%
\pgfsetfillcolor{currentfill}%
\pgfsetlinewidth{0.311001pt}%
\definecolor{currentstroke}{rgb}{1.000000,1.000000,1.000000}%
\pgfsetstrokecolor{currentstroke}%
\pgfsetdash{}{0pt}%
\pgfpathmoveto{\pgfqpoint{5.548452in}{0.970472in}}%
\pgfpathcurveto{\pgfqpoint{5.555584in}{0.970472in}}{\pgfqpoint{5.562426in}{0.973305in}}{\pgfqpoint{5.567470in}{0.978349in}}%
\pgfpathcurveto{\pgfqpoint{5.572513in}{0.983393in}}{\pgfqpoint{5.575347in}{0.990234in}}{\pgfqpoint{5.575347in}{0.997367in}}%
\pgfpathcurveto{\pgfqpoint{5.575347in}{1.004500in}}{\pgfqpoint{5.572513in}{1.011342in}}{\pgfqpoint{5.567470in}{1.016385in}}%
\pgfpathcurveto{\pgfqpoint{5.562426in}{1.021429in}}{\pgfqpoint{5.555584in}{1.024263in}}{\pgfqpoint{5.548452in}{1.024263in}}%
\pgfpathcurveto{\pgfqpoint{5.541319in}{1.024263in}}{\pgfqpoint{5.534477in}{1.021429in}}{\pgfqpoint{5.529433in}{1.016385in}}%
\pgfpathcurveto{\pgfqpoint{5.524390in}{1.011342in}}{\pgfqpoint{5.521556in}{1.004500in}}{\pgfqpoint{5.521556in}{0.997367in}}%
\pgfpathcurveto{\pgfqpoint{5.521556in}{0.990234in}}{\pgfqpoint{5.524390in}{0.983393in}}{\pgfqpoint{5.529433in}{0.978349in}}%
\pgfpathcurveto{\pgfqpoint{5.534477in}{0.973305in}}{\pgfqpoint{5.541319in}{0.970472in}}{\pgfqpoint{5.548452in}{0.970472in}}%
\pgfpathclose%
\pgfusepath{stroke,fill}%
\end{pgfscope}%
\begin{pgfscope}%
\pgfpathrectangle{\pgfqpoint{4.985294in}{0.500000in}}{\pgfqpoint{1.764706in}{1.700000in}}%
\pgfusepath{clip}%
\pgfsetbuttcap%
\pgfsetroundjoin%
\definecolor{currentfill}{rgb}{0.965302,0.713942,0.568499}%
\pgfsetfillcolor{currentfill}%
\pgfsetlinewidth{0.311001pt}%
\definecolor{currentstroke}{rgb}{1.000000,1.000000,1.000000}%
\pgfsetstrokecolor{currentstroke}%
\pgfsetdash{}{0pt}%
\pgfpathmoveto{\pgfqpoint{6.262549in}{0.958352in}}%
\pgfpathcurveto{\pgfqpoint{6.269682in}{0.958352in}}{\pgfqpoint{6.276523in}{0.961186in}}{\pgfqpoint{6.281567in}{0.966229in}}%
\pgfpathcurveto{\pgfqpoint{6.286611in}{0.971273in}}{\pgfqpoint{6.289445in}{0.978115in}}{\pgfqpoint{6.289445in}{0.985248in}}%
\pgfpathcurveto{\pgfqpoint{6.289445in}{0.992380in}}{\pgfqpoint{6.286611in}{0.999222in}}{\pgfqpoint{6.281567in}{1.004266in}}%
\pgfpathcurveto{\pgfqpoint{6.276523in}{1.009309in}}{\pgfqpoint{6.269682in}{1.012143in}}{\pgfqpoint{6.262549in}{1.012143in}}%
\pgfpathcurveto{\pgfqpoint{6.255416in}{1.012143in}}{\pgfqpoint{6.248575in}{1.009309in}}{\pgfqpoint{6.243531in}{1.004266in}}%
\pgfpathcurveto{\pgfqpoint{6.238487in}{0.999222in}}{\pgfqpoint{6.235653in}{0.992380in}}{\pgfqpoint{6.235653in}{0.985248in}}%
\pgfpathcurveto{\pgfqpoint{6.235653in}{0.978115in}}{\pgfqpoint{6.238487in}{0.971273in}}{\pgfqpoint{6.243531in}{0.966229in}}%
\pgfpathcurveto{\pgfqpoint{6.248575in}{0.961186in}}{\pgfqpoint{6.255416in}{0.958352in}}{\pgfqpoint{6.262549in}{0.958352in}}%
\pgfpathclose%
\pgfusepath{stroke,fill}%
\end{pgfscope}%
\begin{pgfscope}%
\pgfpathrectangle{\pgfqpoint{4.985294in}{0.500000in}}{\pgfqpoint{1.764706in}{1.700000in}}%
\pgfusepath{clip}%
\pgfsetbuttcap%
\pgfsetroundjoin%
\definecolor{currentfill}{rgb}{0.978376,0.897317,0.831308}%
\pgfsetfillcolor{currentfill}%
\pgfsetlinewidth{0.311001pt}%
\definecolor{currentstroke}{rgb}{1.000000,1.000000,1.000000}%
\pgfsetstrokecolor{currentstroke}%
\pgfsetdash{}{0pt}%
\pgfpathmoveto{\pgfqpoint{5.445482in}{1.341203in}}%
\pgfpathcurveto{\pgfqpoint{5.452614in}{1.341203in}}{\pgfqpoint{5.459456in}{1.344036in}}{\pgfqpoint{5.464500in}{1.349080in}}%
\pgfpathcurveto{\pgfqpoint{5.469543in}{1.354124in}}{\pgfqpoint{5.472377in}{1.360965in}}{\pgfqpoint{5.472377in}{1.368098in}}%
\pgfpathcurveto{\pgfqpoint{5.472377in}{1.375231in}}{\pgfqpoint{5.469543in}{1.382073in}}{\pgfqpoint{5.464500in}{1.387116in}}%
\pgfpathcurveto{\pgfqpoint{5.459456in}{1.392160in}}{\pgfqpoint{5.452614in}{1.394994in}}{\pgfqpoint{5.445482in}{1.394994in}}%
\pgfpathcurveto{\pgfqpoint{5.438349in}{1.394994in}}{\pgfqpoint{5.431507in}{1.392160in}}{\pgfqpoint{5.426463in}{1.387116in}}%
\pgfpathcurveto{\pgfqpoint{5.421420in}{1.382073in}}{\pgfqpoint{5.418586in}{1.375231in}}{\pgfqpoint{5.418586in}{1.368098in}}%
\pgfpathcurveto{\pgfqpoint{5.418586in}{1.360965in}}{\pgfqpoint{5.421420in}{1.354124in}}{\pgfqpoint{5.426463in}{1.349080in}}%
\pgfpathcurveto{\pgfqpoint{5.431507in}{1.344036in}}{\pgfqpoint{5.438349in}{1.341203in}}{\pgfqpoint{5.445482in}{1.341203in}}%
\pgfpathclose%
\pgfusepath{stroke,fill}%
\end{pgfscope}%
\begin{pgfscope}%
\pgfpathrectangle{\pgfqpoint{4.985294in}{0.500000in}}{\pgfqpoint{1.764706in}{1.700000in}}%
\pgfusepath{clip}%
\pgfsetbuttcap%
\pgfsetroundjoin%
\definecolor{currentfill}{rgb}{0.956268,0.491874,0.339856}%
\pgfsetfillcolor{currentfill}%
\pgfsetlinewidth{0.311001pt}%
\definecolor{currentstroke}{rgb}{1.000000,1.000000,1.000000}%
\pgfsetstrokecolor{currentstroke}%
\pgfsetdash{}{0pt}%
\pgfpathmoveto{\pgfqpoint{6.159394in}{1.322726in}}%
\pgfpathcurveto{\pgfqpoint{6.166526in}{1.322726in}}{\pgfqpoint{6.173368in}{1.325560in}}{\pgfqpoint{6.178412in}{1.330603in}}%
\pgfpathcurveto{\pgfqpoint{6.183455in}{1.335647in}}{\pgfqpoint{6.186289in}{1.342489in}}{\pgfqpoint{6.186289in}{1.349621in}}%
\pgfpathcurveto{\pgfqpoint{6.186289in}{1.356754in}}{\pgfqpoint{6.183455in}{1.363596in}}{\pgfqpoint{6.178412in}{1.368640in}}%
\pgfpathcurveto{\pgfqpoint{6.173368in}{1.373683in}}{\pgfqpoint{6.166526in}{1.376517in}}{\pgfqpoint{6.159394in}{1.376517in}}%
\pgfpathcurveto{\pgfqpoint{6.152261in}{1.376517in}}{\pgfqpoint{6.145419in}{1.373683in}}{\pgfqpoint{6.140375in}{1.368640in}}%
\pgfpathcurveto{\pgfqpoint{6.135332in}{1.363596in}}{\pgfqpoint{6.132498in}{1.356754in}}{\pgfqpoint{6.132498in}{1.349621in}}%
\pgfpathcurveto{\pgfqpoint{6.132498in}{1.342489in}}{\pgfqpoint{6.135332in}{1.335647in}}{\pgfqpoint{6.140375in}{1.330603in}}%
\pgfpathcurveto{\pgfqpoint{6.145419in}{1.325560in}}{\pgfqpoint{6.152261in}{1.322726in}}{\pgfqpoint{6.159394in}{1.322726in}}%
\pgfpathclose%
\pgfusepath{stroke,fill}%
\end{pgfscope}%
\begin{pgfscope}%
\pgfpathrectangle{\pgfqpoint{4.985294in}{0.500000in}}{\pgfqpoint{1.764706in}{1.700000in}}%
\pgfusepath{clip}%
\pgfsetbuttcap%
\pgfsetroundjoin%
\definecolor{currentfill}{rgb}{0.972201,0.839051,0.745789}%
\pgfsetfillcolor{currentfill}%
\pgfsetlinewidth{0.311001pt}%
\definecolor{currentstroke}{rgb}{1.000000,1.000000,1.000000}%
\pgfsetstrokecolor{currentstroke}%
\pgfsetdash{}{0pt}%
\pgfpathmoveto{\pgfqpoint{6.205731in}{1.024940in}}%
\pgfpathcurveto{\pgfqpoint{6.212864in}{1.024940in}}{\pgfqpoint{6.219705in}{1.027774in}}{\pgfqpoint{6.224749in}{1.032818in}}%
\pgfpathcurveto{\pgfqpoint{6.229793in}{1.037862in}}{\pgfqpoint{6.232627in}{1.044703in}}{\pgfqpoint{6.232627in}{1.051836in}}%
\pgfpathcurveto{\pgfqpoint{6.232627in}{1.058969in}}{\pgfqpoint{6.229793in}{1.065810in}}{\pgfqpoint{6.224749in}{1.070854in}}%
\pgfpathcurveto{\pgfqpoint{6.219705in}{1.075898in}}{\pgfqpoint{6.212864in}{1.078732in}}{\pgfqpoint{6.205731in}{1.078732in}}%
\pgfpathcurveto{\pgfqpoint{6.198598in}{1.078732in}}{\pgfqpoint{6.191756in}{1.075898in}}{\pgfqpoint{6.186713in}{1.070854in}}%
\pgfpathcurveto{\pgfqpoint{6.181669in}{1.065810in}}{\pgfqpoint{6.178835in}{1.058969in}}{\pgfqpoint{6.178835in}{1.051836in}}%
\pgfpathcurveto{\pgfqpoint{6.178835in}{1.044703in}}{\pgfqpoint{6.181669in}{1.037862in}}{\pgfqpoint{6.186713in}{1.032818in}}%
\pgfpathcurveto{\pgfqpoint{6.191756in}{1.027774in}}{\pgfqpoint{6.198598in}{1.024940in}}{\pgfqpoint{6.205731in}{1.024940in}}%
\pgfpathclose%
\pgfusepath{stroke,fill}%
\end{pgfscope}%
\begin{pgfscope}%
\pgfpathrectangle{\pgfqpoint{4.985294in}{0.500000in}}{\pgfqpoint{1.764706in}{1.700000in}}%
\pgfusepath{clip}%
\pgfsetbuttcap%
\pgfsetroundjoin%
\definecolor{currentfill}{rgb}{0.975644,0.874038,0.797253}%
\pgfsetfillcolor{currentfill}%
\pgfsetlinewidth{0.311001pt}%
\definecolor{currentstroke}{rgb}{1.000000,1.000000,1.000000}%
\pgfsetstrokecolor{currentstroke}%
\pgfsetdash{}{0pt}%
\pgfpathmoveto{\pgfqpoint{5.473584in}{1.433864in}}%
\pgfpathcurveto{\pgfqpoint{5.480717in}{1.433864in}}{\pgfqpoint{5.487559in}{1.436698in}}{\pgfqpoint{5.492602in}{1.441741in}}%
\pgfpathcurveto{\pgfqpoint{5.497646in}{1.446785in}}{\pgfqpoint{5.500480in}{1.453627in}}{\pgfqpoint{5.500480in}{1.460760in}}%
\pgfpathcurveto{\pgfqpoint{5.500480in}{1.467892in}}{\pgfqpoint{5.497646in}{1.474734in}}{\pgfqpoint{5.492602in}{1.479778in}}%
\pgfpathcurveto{\pgfqpoint{5.487559in}{1.484821in}}{\pgfqpoint{5.480717in}{1.487655in}}{\pgfqpoint{5.473584in}{1.487655in}}%
\pgfpathcurveto{\pgfqpoint{5.466451in}{1.487655in}}{\pgfqpoint{5.459610in}{1.484821in}}{\pgfqpoint{5.454566in}{1.479778in}}%
\pgfpathcurveto{\pgfqpoint{5.449523in}{1.474734in}}{\pgfqpoint{5.446689in}{1.467892in}}{\pgfqpoint{5.446689in}{1.460760in}}%
\pgfpathcurveto{\pgfqpoint{5.446689in}{1.453627in}}{\pgfqpoint{5.449523in}{1.446785in}}{\pgfqpoint{5.454566in}{1.441741in}}%
\pgfpathcurveto{\pgfqpoint{5.459610in}{1.436698in}}{\pgfqpoint{5.466451in}{1.433864in}}{\pgfqpoint{5.473584in}{1.433864in}}%
\pgfpathclose%
\pgfusepath{stroke,fill}%
\end{pgfscope}%
\begin{pgfscope}%
\pgfpathrectangle{\pgfqpoint{4.985294in}{0.500000in}}{\pgfqpoint{1.764706in}{1.700000in}}%
\pgfusepath{clip}%
\pgfsetbuttcap%
\pgfsetroundjoin%
\definecolor{currentfill}{rgb}{0.964799,0.689101,0.537560}%
\pgfsetfillcolor{currentfill}%
\pgfsetlinewidth{0.311001pt}%
\definecolor{currentstroke}{rgb}{1.000000,1.000000,1.000000}%
\pgfsetstrokecolor{currentstroke}%
\pgfsetdash{}{0pt}%
\pgfpathmoveto{\pgfqpoint{5.567826in}{0.896584in}}%
\pgfpathcurveto{\pgfqpoint{5.574959in}{0.896584in}}{\pgfqpoint{5.581800in}{0.899418in}}{\pgfqpoint{5.586844in}{0.904461in}}%
\pgfpathcurveto{\pgfqpoint{5.591888in}{0.909505in}}{\pgfqpoint{5.594722in}{0.916347in}}{\pgfqpoint{5.594722in}{0.923479in}}%
\pgfpathcurveto{\pgfqpoint{5.594722in}{0.930612in}}{\pgfqpoint{5.591888in}{0.937454in}}{\pgfqpoint{5.586844in}{0.942498in}}%
\pgfpathcurveto{\pgfqpoint{5.581800in}{0.947541in}}{\pgfqpoint{5.574959in}{0.950375in}}{\pgfqpoint{5.567826in}{0.950375in}}%
\pgfpathcurveto{\pgfqpoint{5.560693in}{0.950375in}}{\pgfqpoint{5.553851in}{0.947541in}}{\pgfqpoint{5.548808in}{0.942498in}}%
\pgfpathcurveto{\pgfqpoint{5.543764in}{0.937454in}}{\pgfqpoint{5.540930in}{0.930612in}}{\pgfqpoint{5.540930in}{0.923479in}}%
\pgfpathcurveto{\pgfqpoint{5.540930in}{0.916347in}}{\pgfqpoint{5.543764in}{0.909505in}}{\pgfqpoint{5.548808in}{0.904461in}}%
\pgfpathcurveto{\pgfqpoint{5.553851in}{0.899418in}}{\pgfqpoint{5.560693in}{0.896584in}}{\pgfqpoint{5.567826in}{0.896584in}}%
\pgfpathclose%
\pgfusepath{stroke,fill}%
\end{pgfscope}%
\begin{pgfscope}%
\pgfpathrectangle{\pgfqpoint{4.985294in}{0.500000in}}{\pgfqpoint{1.764706in}{1.700000in}}%
\pgfusepath{clip}%
\pgfsetbuttcap%
\pgfsetroundjoin%
\definecolor{currentfill}{rgb}{0.975018,0.868213,0.788710}%
\pgfsetfillcolor{currentfill}%
\pgfsetlinewidth{0.311001pt}%
\definecolor{currentstroke}{rgb}{1.000000,1.000000,1.000000}%
\pgfsetstrokecolor{currentstroke}%
\pgfsetdash{}{0pt}%
\pgfpathmoveto{\pgfqpoint{6.270323in}{1.390018in}}%
\pgfpathcurveto{\pgfqpoint{6.277456in}{1.390018in}}{\pgfqpoint{6.284298in}{1.392852in}}{\pgfqpoint{6.289342in}{1.397895in}}%
\pgfpathcurveto{\pgfqpoint{6.294385in}{1.402939in}}{\pgfqpoint{6.297219in}{1.409781in}}{\pgfqpoint{6.297219in}{1.416913in}}%
\pgfpathcurveto{\pgfqpoint{6.297219in}{1.424046in}}{\pgfqpoint{6.294385in}{1.430888in}}{\pgfqpoint{6.289342in}{1.435932in}}%
\pgfpathcurveto{\pgfqpoint{6.284298in}{1.440975in}}{\pgfqpoint{6.277456in}{1.443809in}}{\pgfqpoint{6.270323in}{1.443809in}}%
\pgfpathcurveto{\pgfqpoint{6.263191in}{1.443809in}}{\pgfqpoint{6.256349in}{1.440975in}}{\pgfqpoint{6.251305in}{1.435932in}}%
\pgfpathcurveto{\pgfqpoint{6.246262in}{1.430888in}}{\pgfqpoint{6.243428in}{1.424046in}}{\pgfqpoint{6.243428in}{1.416913in}}%
\pgfpathcurveto{\pgfqpoint{6.243428in}{1.409781in}}{\pgfqpoint{6.246262in}{1.402939in}}{\pgfqpoint{6.251305in}{1.397895in}}%
\pgfpathcurveto{\pgfqpoint{6.256349in}{1.392852in}}{\pgfqpoint{6.263191in}{1.390018in}}{\pgfqpoint{6.270323in}{1.390018in}}%
\pgfpathclose%
\pgfusepath{stroke,fill}%
\end{pgfscope}%
\begin{pgfscope}%
\pgfpathrectangle{\pgfqpoint{4.985294in}{0.500000in}}{\pgfqpoint{1.764706in}{1.700000in}}%
\pgfusepath{clip}%
\pgfsetbuttcap%
\pgfsetroundjoin%
\definecolor{currentfill}{rgb}{0.981377,0.920617,0.865369}%
\pgfsetfillcolor{currentfill}%
\pgfsetlinewidth{0.311001pt}%
\definecolor{currentstroke}{rgb}{1.000000,1.000000,1.000000}%
\pgfsetstrokecolor{currentstroke}%
\pgfsetdash{}{0pt}%
\pgfpathmoveto{\pgfqpoint{6.302868in}{1.278694in}}%
\pgfpathcurveto{\pgfqpoint{6.310001in}{1.278694in}}{\pgfqpoint{6.316843in}{1.281528in}}{\pgfqpoint{6.321886in}{1.286572in}}%
\pgfpathcurveto{\pgfqpoint{6.326930in}{1.291615in}}{\pgfqpoint{6.329764in}{1.298457in}}{\pgfqpoint{6.329764in}{1.305590in}}%
\pgfpathcurveto{\pgfqpoint{6.329764in}{1.312723in}}{\pgfqpoint{6.326930in}{1.319564in}}{\pgfqpoint{6.321886in}{1.324608in}}%
\pgfpathcurveto{\pgfqpoint{6.316843in}{1.329652in}}{\pgfqpoint{6.310001in}{1.332486in}}{\pgfqpoint{6.302868in}{1.332486in}}%
\pgfpathcurveto{\pgfqpoint{6.295736in}{1.332486in}}{\pgfqpoint{6.288894in}{1.329652in}}{\pgfqpoint{6.283850in}{1.324608in}}%
\pgfpathcurveto{\pgfqpoint{6.278807in}{1.319564in}}{\pgfqpoint{6.275973in}{1.312723in}}{\pgfqpoint{6.275973in}{1.305590in}}%
\pgfpathcurveto{\pgfqpoint{6.275973in}{1.298457in}}{\pgfqpoint{6.278807in}{1.291615in}}{\pgfqpoint{6.283850in}{1.286572in}}%
\pgfpathcurveto{\pgfqpoint{6.288894in}{1.281528in}}{\pgfqpoint{6.295736in}{1.278694in}}{\pgfqpoint{6.302868in}{1.278694in}}%
\pgfpathclose%
\pgfusepath{stroke,fill}%
\end{pgfscope}%
\begin{pgfscope}%
\pgfpathrectangle{\pgfqpoint{4.985294in}{0.500000in}}{\pgfqpoint{1.764706in}{1.700000in}}%
\pgfusepath{clip}%
\pgfsetbuttcap%
\pgfsetroundjoin%
\definecolor{currentfill}{rgb}{0.980678,0.914765,0.856766}%
\pgfsetfillcolor{currentfill}%
\pgfsetlinewidth{0.311001pt}%
\definecolor{currentstroke}{rgb}{1.000000,1.000000,1.000000}%
\pgfsetstrokecolor{currentstroke}%
\pgfsetdash{}{0pt}%
\pgfpathmoveto{\pgfqpoint{5.414749in}{1.326319in}}%
\pgfpathcurveto{\pgfqpoint{5.421882in}{1.326319in}}{\pgfqpoint{5.428724in}{1.329153in}}{\pgfqpoint{5.433767in}{1.334197in}}%
\pgfpathcurveto{\pgfqpoint{5.438811in}{1.339240in}}{\pgfqpoint{5.441645in}{1.346082in}}{\pgfqpoint{5.441645in}{1.353215in}}%
\pgfpathcurveto{\pgfqpoint{5.441645in}{1.360348in}}{\pgfqpoint{5.438811in}{1.367189in}}{\pgfqpoint{5.433767in}{1.372233in}}%
\pgfpathcurveto{\pgfqpoint{5.428724in}{1.377277in}}{\pgfqpoint{5.421882in}{1.380110in}}{\pgfqpoint{5.414749in}{1.380110in}}%
\pgfpathcurveto{\pgfqpoint{5.407616in}{1.380110in}}{\pgfqpoint{5.400775in}{1.377277in}}{\pgfqpoint{5.395731in}{1.372233in}}%
\pgfpathcurveto{\pgfqpoint{5.390687in}{1.367189in}}{\pgfqpoint{5.387853in}{1.360348in}}{\pgfqpoint{5.387853in}{1.353215in}}%
\pgfpathcurveto{\pgfqpoint{5.387853in}{1.346082in}}{\pgfqpoint{5.390687in}{1.339240in}}{\pgfqpoint{5.395731in}{1.334197in}}%
\pgfpathcurveto{\pgfqpoint{5.400775in}{1.329153in}}{\pgfqpoint{5.407616in}{1.326319in}}{\pgfqpoint{5.414749in}{1.326319in}}%
\pgfpathclose%
\pgfusepath{stroke,fill}%
\end{pgfscope}%
\begin{pgfscope}%
\pgfpathrectangle{\pgfqpoint{4.985294in}{0.500000in}}{\pgfqpoint{1.764706in}{1.700000in}}%
\pgfusepath{clip}%
\pgfsetbuttcap%
\pgfsetroundjoin%
\definecolor{currentfill}{rgb}{0.974412,0.862387,0.780156}%
\pgfsetfillcolor{currentfill}%
\pgfsetlinewidth{0.311001pt}%
\definecolor{currentstroke}{rgb}{1.000000,1.000000,1.000000}%
\pgfsetstrokecolor{currentstroke}%
\pgfsetdash{}{0pt}%
\pgfpathmoveto{\pgfqpoint{6.359830in}{1.387703in}}%
\pgfpathcurveto{\pgfqpoint{6.366963in}{1.387703in}}{\pgfqpoint{6.373805in}{1.390537in}}{\pgfqpoint{6.378848in}{1.395581in}}%
\pgfpathcurveto{\pgfqpoint{6.383892in}{1.400625in}}{\pgfqpoint{6.386726in}{1.407466in}}{\pgfqpoint{6.386726in}{1.414599in}}%
\pgfpathcurveto{\pgfqpoint{6.386726in}{1.421732in}}{\pgfqpoint{6.383892in}{1.428574in}}{\pgfqpoint{6.378848in}{1.433617in}}%
\pgfpathcurveto{\pgfqpoint{6.373805in}{1.438661in}}{\pgfqpoint{6.366963in}{1.441495in}}{\pgfqpoint{6.359830in}{1.441495in}}%
\pgfpathcurveto{\pgfqpoint{6.352697in}{1.441495in}}{\pgfqpoint{6.345856in}{1.438661in}}{\pgfqpoint{6.340812in}{1.433617in}}%
\pgfpathcurveto{\pgfqpoint{6.335768in}{1.428574in}}{\pgfqpoint{6.332934in}{1.421732in}}{\pgfqpoint{6.332934in}{1.414599in}}%
\pgfpathcurveto{\pgfqpoint{6.332934in}{1.407466in}}{\pgfqpoint{6.335768in}{1.400625in}}{\pgfqpoint{6.340812in}{1.395581in}}%
\pgfpathcurveto{\pgfqpoint{6.345856in}{1.390537in}}{\pgfqpoint{6.352697in}{1.387703in}}{\pgfqpoint{6.359830in}{1.387703in}}%
\pgfpathclose%
\pgfusepath{stroke,fill}%
\end{pgfscope}%
\begin{pgfscope}%
\pgfpathrectangle{\pgfqpoint{4.985294in}{0.500000in}}{\pgfqpoint{1.764706in}{1.700000in}}%
\pgfusepath{clip}%
\pgfsetbuttcap%
\pgfsetroundjoin%
\definecolor{currentfill}{rgb}{0.970718,0.821518,0.719872}%
\pgfsetfillcolor{currentfill}%
\pgfsetlinewidth{0.311001pt}%
\definecolor{currentstroke}{rgb}{1.000000,1.000000,1.000000}%
\pgfsetstrokecolor{currentstroke}%
\pgfsetdash{}{0pt}%
\pgfpathmoveto{\pgfqpoint{6.268203in}{1.649919in}}%
\pgfpathcurveto{\pgfqpoint{6.275336in}{1.649919in}}{\pgfqpoint{6.282178in}{1.652753in}}{\pgfqpoint{6.287222in}{1.657796in}}%
\pgfpathcurveto{\pgfqpoint{6.292265in}{1.662840in}}{\pgfqpoint{6.295099in}{1.669682in}}{\pgfqpoint{6.295099in}{1.676814in}}%
\pgfpathcurveto{\pgfqpoint{6.295099in}{1.683947in}}{\pgfqpoint{6.292265in}{1.690789in}}{\pgfqpoint{6.287222in}{1.695833in}}%
\pgfpathcurveto{\pgfqpoint{6.282178in}{1.700876in}}{\pgfqpoint{6.275336in}{1.703710in}}{\pgfqpoint{6.268203in}{1.703710in}}%
\pgfpathcurveto{\pgfqpoint{6.261071in}{1.703710in}}{\pgfqpoint{6.254229in}{1.700876in}}{\pgfqpoint{6.249185in}{1.695833in}}%
\pgfpathcurveto{\pgfqpoint{6.244142in}{1.690789in}}{\pgfqpoint{6.241308in}{1.683947in}}{\pgfqpoint{6.241308in}{1.676814in}}%
\pgfpathcurveto{\pgfqpoint{6.241308in}{1.669682in}}{\pgfqpoint{6.244142in}{1.662840in}}{\pgfqpoint{6.249185in}{1.657796in}}%
\pgfpathcurveto{\pgfqpoint{6.254229in}{1.652753in}}{\pgfqpoint{6.261071in}{1.649919in}}{\pgfqpoint{6.268203in}{1.649919in}}%
\pgfpathclose%
\pgfusepath{stroke,fill}%
\end{pgfscope}%
\begin{pgfscope}%
\pgfpathrectangle{\pgfqpoint{4.985294in}{0.500000in}}{\pgfqpoint{1.764706in}{1.700000in}}%
\pgfusepath{clip}%
\pgfsetbuttcap%
\pgfsetroundjoin%
\definecolor{currentfill}{rgb}{0.979124,0.903132,0.839793}%
\pgfsetfillcolor{currentfill}%
\pgfsetlinewidth{0.311001pt}%
\definecolor{currentstroke}{rgb}{1.000000,1.000000,1.000000}%
\pgfsetstrokecolor{currentstroke}%
\pgfsetdash{}{0pt}%
\pgfpathmoveto{\pgfqpoint{6.343662in}{1.289115in}}%
\pgfpathcurveto{\pgfqpoint{6.350794in}{1.289115in}}{\pgfqpoint{6.357636in}{1.291948in}}{\pgfqpoint{6.362680in}{1.296992in}}%
\pgfpathcurveto{\pgfqpoint{6.367723in}{1.302036in}}{\pgfqpoint{6.370557in}{1.308877in}}{\pgfqpoint{6.370557in}{1.316010in}}%
\pgfpathcurveto{\pgfqpoint{6.370557in}{1.323143in}}{\pgfqpoint{6.367723in}{1.329985in}}{\pgfqpoint{6.362680in}{1.335028in}}%
\pgfpathcurveto{\pgfqpoint{6.357636in}{1.340072in}}{\pgfqpoint{6.350794in}{1.342906in}}{\pgfqpoint{6.343662in}{1.342906in}}%
\pgfpathcurveto{\pgfqpoint{6.336529in}{1.342906in}}{\pgfqpoint{6.329687in}{1.340072in}}{\pgfqpoint{6.324643in}{1.335028in}}%
\pgfpathcurveto{\pgfqpoint{6.319600in}{1.329985in}}{\pgfqpoint{6.316766in}{1.323143in}}{\pgfqpoint{6.316766in}{1.316010in}}%
\pgfpathcurveto{\pgfqpoint{6.316766in}{1.308877in}}{\pgfqpoint{6.319600in}{1.302036in}}{\pgfqpoint{6.324643in}{1.296992in}}%
\pgfpathcurveto{\pgfqpoint{6.329687in}{1.291948in}}{\pgfqpoint{6.336529in}{1.289115in}}{\pgfqpoint{6.343662in}{1.289115in}}%
\pgfpathclose%
\pgfusepath{stroke,fill}%
\end{pgfscope}%
\begin{pgfscope}%
\pgfpathrectangle{\pgfqpoint{4.985294in}{0.500000in}}{\pgfqpoint{1.764706in}{1.700000in}}%
\pgfusepath{clip}%
\pgfsetbuttcap%
\pgfsetroundjoin%
\definecolor{currentfill}{rgb}{0.981377,0.920617,0.865369}%
\pgfsetfillcolor{currentfill}%
\pgfsetlinewidth{0.311001pt}%
\definecolor{currentstroke}{rgb}{1.000000,1.000000,1.000000}%
\pgfsetstrokecolor{currentstroke}%
\pgfsetdash{}{0pt}%
\pgfpathmoveto{\pgfqpoint{6.326505in}{1.358824in}}%
\pgfpathcurveto{\pgfqpoint{6.333638in}{1.358824in}}{\pgfqpoint{6.340479in}{1.361658in}}{\pgfqpoint{6.345523in}{1.366702in}}%
\pgfpathcurveto{\pgfqpoint{6.350567in}{1.371746in}}{\pgfqpoint{6.353400in}{1.378587in}}{\pgfqpoint{6.353400in}{1.385720in}}%
\pgfpathcurveto{\pgfqpoint{6.353400in}{1.392853in}}{\pgfqpoint{6.350567in}{1.399695in}}{\pgfqpoint{6.345523in}{1.404738in}}%
\pgfpathcurveto{\pgfqpoint{6.340479in}{1.409782in}}{\pgfqpoint{6.333638in}{1.412616in}}{\pgfqpoint{6.326505in}{1.412616in}}%
\pgfpathcurveto{\pgfqpoint{6.319372in}{1.412616in}}{\pgfqpoint{6.312530in}{1.409782in}}{\pgfqpoint{6.307487in}{1.404738in}}%
\pgfpathcurveto{\pgfqpoint{6.302443in}{1.399695in}}{\pgfqpoint{6.299609in}{1.392853in}}{\pgfqpoint{6.299609in}{1.385720in}}%
\pgfpathcurveto{\pgfqpoint{6.299609in}{1.378587in}}{\pgfqpoint{6.302443in}{1.371746in}}{\pgfqpoint{6.307487in}{1.366702in}}%
\pgfpathcurveto{\pgfqpoint{6.312530in}{1.361658in}}{\pgfqpoint{6.319372in}{1.358824in}}{\pgfqpoint{6.326505in}{1.358824in}}%
\pgfpathclose%
\pgfusepath{stroke,fill}%
\end{pgfscope}%
\begin{pgfscope}%
\pgfpathrectangle{\pgfqpoint{4.985294in}{0.500000in}}{\pgfqpoint{1.764706in}{1.700000in}}%
\pgfusepath{clip}%
\pgfsetbuttcap%
\pgfsetroundjoin%
\definecolor{currentfill}{rgb}{0.972201,0.839051,0.745789}%
\pgfsetfillcolor{currentfill}%
\pgfsetlinewidth{0.311001pt}%
\definecolor{currentstroke}{rgb}{1.000000,1.000000,1.000000}%
\pgfsetstrokecolor{currentstroke}%
\pgfsetdash{}{0pt}%
\pgfpathmoveto{\pgfqpoint{6.232277in}{1.205267in}}%
\pgfpathcurveto{\pgfqpoint{6.239410in}{1.205267in}}{\pgfqpoint{6.246251in}{1.208100in}}{\pgfqpoint{6.251295in}{1.213144in}}%
\pgfpathcurveto{\pgfqpoint{6.256339in}{1.218188in}}{\pgfqpoint{6.259173in}{1.225029in}}{\pgfqpoint{6.259173in}{1.232162in}}%
\pgfpathcurveto{\pgfqpoint{6.259173in}{1.239295in}}{\pgfqpoint{6.256339in}{1.246137in}}{\pgfqpoint{6.251295in}{1.251180in}}%
\pgfpathcurveto{\pgfqpoint{6.246251in}{1.256224in}}{\pgfqpoint{6.239410in}{1.259058in}}{\pgfqpoint{6.232277in}{1.259058in}}%
\pgfpathcurveto{\pgfqpoint{6.225144in}{1.259058in}}{\pgfqpoint{6.218302in}{1.256224in}}{\pgfqpoint{6.213259in}{1.251180in}}%
\pgfpathcurveto{\pgfqpoint{6.208215in}{1.246137in}}{\pgfqpoint{6.205381in}{1.239295in}}{\pgfqpoint{6.205381in}{1.232162in}}%
\pgfpathcurveto{\pgfqpoint{6.205381in}{1.225029in}}{\pgfqpoint{6.208215in}{1.218188in}}{\pgfqpoint{6.213259in}{1.213144in}}%
\pgfpathcurveto{\pgfqpoint{6.218302in}{1.208100in}}{\pgfqpoint{6.225144in}{1.205267in}}{\pgfqpoint{6.232277in}{1.205267in}}%
\pgfpathclose%
\pgfusepath{stroke,fill}%
\end{pgfscope}%
\begin{pgfscope}%
\pgfpathrectangle{\pgfqpoint{4.985294in}{0.500000in}}{\pgfqpoint{1.764706in}{1.700000in}}%
\pgfusepath{clip}%
\pgfsetbuttcap%
\pgfsetroundjoin%
\definecolor{currentfill}{rgb}{0.970718,0.821518,0.719872}%
\pgfsetfillcolor{currentfill}%
\pgfsetlinewidth{0.311001pt}%
\definecolor{currentstroke}{rgb}{1.000000,1.000000,1.000000}%
\pgfsetstrokecolor{currentstroke}%
\pgfsetdash{}{0pt}%
\pgfpathmoveto{\pgfqpoint{5.524604in}{1.080908in}}%
\pgfpathcurveto{\pgfqpoint{5.531737in}{1.080908in}}{\pgfqpoint{5.538579in}{1.083742in}}{\pgfqpoint{5.543622in}{1.088786in}}%
\pgfpathcurveto{\pgfqpoint{5.548666in}{1.093829in}}{\pgfqpoint{5.551500in}{1.100671in}}{\pgfqpoint{5.551500in}{1.107804in}}%
\pgfpathcurveto{\pgfqpoint{5.551500in}{1.114937in}}{\pgfqpoint{5.548666in}{1.121778in}}{\pgfqpoint{5.543622in}{1.126822in}}%
\pgfpathcurveto{\pgfqpoint{5.538579in}{1.131866in}}{\pgfqpoint{5.531737in}{1.134700in}}{\pgfqpoint{5.524604in}{1.134700in}}%
\pgfpathcurveto{\pgfqpoint{5.517471in}{1.134700in}}{\pgfqpoint{5.510630in}{1.131866in}}{\pgfqpoint{5.505586in}{1.126822in}}%
\pgfpathcurveto{\pgfqpoint{5.500542in}{1.121778in}}{\pgfqpoint{5.497709in}{1.114937in}}{\pgfqpoint{5.497709in}{1.107804in}}%
\pgfpathcurveto{\pgfqpoint{5.497709in}{1.100671in}}{\pgfqpoint{5.500542in}{1.093829in}}{\pgfqpoint{5.505586in}{1.088786in}}%
\pgfpathcurveto{\pgfqpoint{5.510630in}{1.083742in}}{\pgfqpoint{5.517471in}{1.080908in}}{\pgfqpoint{5.524604in}{1.080908in}}%
\pgfpathclose%
\pgfusepath{stroke,fill}%
\end{pgfscope}%
\begin{pgfscope}%
\pgfpathrectangle{\pgfqpoint{4.985294in}{0.500000in}}{\pgfqpoint{1.764706in}{1.700000in}}%
\pgfusepath{clip}%
\pgfsetbuttcap%
\pgfsetroundjoin%
\definecolor{currentfill}{rgb}{0.980678,0.914765,0.856766}%
\pgfsetfillcolor{currentfill}%
\pgfsetlinewidth{0.311001pt}%
\definecolor{currentstroke}{rgb}{1.000000,1.000000,1.000000}%
\pgfsetstrokecolor{currentstroke}%
\pgfsetdash{}{0pt}%
\pgfpathmoveto{\pgfqpoint{6.294584in}{1.414232in}}%
\pgfpathcurveto{\pgfqpoint{6.301717in}{1.414232in}}{\pgfqpoint{6.308559in}{1.417066in}}{\pgfqpoint{6.313603in}{1.422110in}}%
\pgfpathcurveto{\pgfqpoint{6.318646in}{1.427154in}}{\pgfqpoint{6.321480in}{1.433995in}}{\pgfqpoint{6.321480in}{1.441128in}}%
\pgfpathcurveto{\pgfqpoint{6.321480in}{1.448261in}}{\pgfqpoint{6.318646in}{1.455103in}}{\pgfqpoint{6.313603in}{1.460146in}}%
\pgfpathcurveto{\pgfqpoint{6.308559in}{1.465190in}}{\pgfqpoint{6.301717in}{1.468024in}}{\pgfqpoint{6.294584in}{1.468024in}}%
\pgfpathcurveto{\pgfqpoint{6.287452in}{1.468024in}}{\pgfqpoint{6.280610in}{1.465190in}}{\pgfqpoint{6.275566in}{1.460146in}}%
\pgfpathcurveto{\pgfqpoint{6.270523in}{1.455103in}}{\pgfqpoint{6.267689in}{1.448261in}}{\pgfqpoint{6.267689in}{1.441128in}}%
\pgfpathcurveto{\pgfqpoint{6.267689in}{1.433995in}}{\pgfqpoint{6.270523in}{1.427154in}}{\pgfqpoint{6.275566in}{1.422110in}}%
\pgfpathcurveto{\pgfqpoint{6.280610in}{1.417066in}}{\pgfqpoint{6.287452in}{1.414232in}}{\pgfqpoint{6.294584in}{1.414232in}}%
\pgfpathclose%
\pgfusepath{stroke,fill}%
\end{pgfscope}%
\begin{pgfscope}%
\pgfpathrectangle{\pgfqpoint{4.985294in}{0.500000in}}{\pgfqpoint{1.764706in}{1.700000in}}%
\pgfusepath{clip}%
\pgfsetbuttcap%
\pgfsetroundjoin%
\definecolor{currentfill}{rgb}{0.960421,0.553286,0.393191}%
\pgfsetfillcolor{currentfill}%
\pgfsetlinewidth{0.311001pt}%
\definecolor{currentstroke}{rgb}{1.000000,1.000000,1.000000}%
\pgfsetstrokecolor{currentstroke}%
\pgfsetdash{}{0pt}%
\pgfpathmoveto{\pgfqpoint{6.301195in}{0.935145in}}%
\pgfpathcurveto{\pgfqpoint{6.308328in}{0.935145in}}{\pgfqpoint{6.315169in}{0.937979in}}{\pgfqpoint{6.320213in}{0.943023in}}%
\pgfpathcurveto{\pgfqpoint{6.325257in}{0.948066in}}{\pgfqpoint{6.328091in}{0.954908in}}{\pgfqpoint{6.328091in}{0.962041in}}%
\pgfpathcurveto{\pgfqpoint{6.328091in}{0.969174in}}{\pgfqpoint{6.325257in}{0.976015in}}{\pgfqpoint{6.320213in}{0.981059in}}%
\pgfpathcurveto{\pgfqpoint{6.315169in}{0.986103in}}{\pgfqpoint{6.308328in}{0.988936in}}{\pgfqpoint{6.301195in}{0.988936in}}%
\pgfpathcurveto{\pgfqpoint{6.294062in}{0.988936in}}{\pgfqpoint{6.287220in}{0.986103in}}{\pgfqpoint{6.282177in}{0.981059in}}%
\pgfpathcurveto{\pgfqpoint{6.277133in}{0.976015in}}{\pgfqpoint{6.274299in}{0.969174in}}{\pgfqpoint{6.274299in}{0.962041in}}%
\pgfpathcurveto{\pgfqpoint{6.274299in}{0.954908in}}{\pgfqpoint{6.277133in}{0.948066in}}{\pgfqpoint{6.282177in}{0.943023in}}%
\pgfpathcurveto{\pgfqpoint{6.287220in}{0.937979in}}{\pgfqpoint{6.294062in}{0.935145in}}{\pgfqpoint{6.301195in}{0.935145in}}%
\pgfpathclose%
\pgfusepath{stroke,fill}%
\end{pgfscope}%
\begin{pgfscope}%
\pgfpathrectangle{\pgfqpoint{4.985294in}{0.500000in}}{\pgfqpoint{1.764706in}{1.700000in}}%
\pgfusepath{clip}%
\pgfsetbuttcap%
\pgfsetroundjoin%
\definecolor{currentfill}{rgb}{0.975644,0.874038,0.797253}%
\pgfsetfillcolor{currentfill}%
\pgfsetlinewidth{0.311001pt}%
\definecolor{currentstroke}{rgb}{1.000000,1.000000,1.000000}%
\pgfsetstrokecolor{currentstroke}%
\pgfsetdash{}{0pt}%
\pgfpathmoveto{\pgfqpoint{5.376944in}{1.364966in}}%
\pgfpathcurveto{\pgfqpoint{5.384077in}{1.364966in}}{\pgfqpoint{5.390918in}{1.367800in}}{\pgfqpoint{5.395962in}{1.372843in}}%
\pgfpathcurveto{\pgfqpoint{5.401006in}{1.377887in}}{\pgfqpoint{5.403840in}{1.384729in}}{\pgfqpoint{5.403840in}{1.391861in}}%
\pgfpathcurveto{\pgfqpoint{5.403840in}{1.398994in}}{\pgfqpoint{5.401006in}{1.405836in}}{\pgfqpoint{5.395962in}{1.410880in}}%
\pgfpathcurveto{\pgfqpoint{5.390918in}{1.415923in}}{\pgfqpoint{5.384077in}{1.418757in}}{\pgfqpoint{5.376944in}{1.418757in}}%
\pgfpathcurveto{\pgfqpoint{5.369811in}{1.418757in}}{\pgfqpoint{5.362969in}{1.415923in}}{\pgfqpoint{5.357926in}{1.410880in}}%
\pgfpathcurveto{\pgfqpoint{5.352882in}{1.405836in}}{\pgfqpoint{5.350048in}{1.398994in}}{\pgfqpoint{5.350048in}{1.391861in}}%
\pgfpathcurveto{\pgfqpoint{5.350048in}{1.384729in}}{\pgfqpoint{5.352882in}{1.377887in}}{\pgfqpoint{5.357926in}{1.372843in}}%
\pgfpathcurveto{\pgfqpoint{5.362969in}{1.367800in}}{\pgfqpoint{5.369811in}{1.364966in}}{\pgfqpoint{5.376944in}{1.364966in}}%
\pgfpathclose%
\pgfusepath{stroke,fill}%
\end{pgfscope}%
\begin{pgfscope}%
\pgfpathrectangle{\pgfqpoint{4.985294in}{0.500000in}}{\pgfqpoint{1.764706in}{1.700000in}}%
\pgfusepath{clip}%
\pgfsetbuttcap%
\pgfsetroundjoin%
\definecolor{currentfill}{rgb}{0.950017,0.427714,0.292447}%
\pgfsetfillcolor{currentfill}%
\pgfsetlinewidth{0.311001pt}%
\definecolor{currentstroke}{rgb}{1.000000,1.000000,1.000000}%
\pgfsetstrokecolor{currentstroke}%
\pgfsetdash{}{0pt}%
\pgfpathmoveto{\pgfqpoint{5.578639in}{1.237343in}}%
\pgfpathcurveto{\pgfqpoint{5.585772in}{1.237343in}}{\pgfqpoint{5.592614in}{1.240177in}}{\pgfqpoint{5.597657in}{1.245220in}}%
\pgfpathcurveto{\pgfqpoint{5.602701in}{1.250264in}}{\pgfqpoint{5.605535in}{1.257106in}}{\pgfqpoint{5.605535in}{1.264238in}}%
\pgfpathcurveto{\pgfqpoint{5.605535in}{1.271371in}}{\pgfqpoint{5.602701in}{1.278213in}}{\pgfqpoint{5.597657in}{1.283257in}}%
\pgfpathcurveto{\pgfqpoint{5.592614in}{1.288300in}}{\pgfqpoint{5.585772in}{1.291134in}}{\pgfqpoint{5.578639in}{1.291134in}}%
\pgfpathcurveto{\pgfqpoint{5.571506in}{1.291134in}}{\pgfqpoint{5.564665in}{1.288300in}}{\pgfqpoint{5.559621in}{1.283257in}}%
\pgfpathcurveto{\pgfqpoint{5.554577in}{1.278213in}}{\pgfqpoint{5.551743in}{1.271371in}}{\pgfqpoint{5.551743in}{1.264238in}}%
\pgfpathcurveto{\pgfqpoint{5.551743in}{1.257106in}}{\pgfqpoint{5.554577in}{1.250264in}}{\pgfqpoint{5.559621in}{1.245220in}}%
\pgfpathcurveto{\pgfqpoint{5.564665in}{1.240177in}}{\pgfqpoint{5.571506in}{1.237343in}}{\pgfqpoint{5.578639in}{1.237343in}}%
\pgfpathclose%
\pgfusepath{stroke,fill}%
\end{pgfscope}%
\begin{pgfscope}%
\pgfpathrectangle{\pgfqpoint{4.985294in}{0.500000in}}{\pgfqpoint{1.764706in}{1.700000in}}%
\pgfusepath{clip}%
\pgfsetbuttcap%
\pgfsetroundjoin%
\definecolor{currentfill}{rgb}{0.965440,0.720101,0.576404}%
\pgfsetfillcolor{currentfill}%
\pgfsetlinewidth{0.311001pt}%
\definecolor{currentstroke}{rgb}{1.000000,1.000000,1.000000}%
\pgfsetstrokecolor{currentstroke}%
\pgfsetdash{}{0pt}%
\pgfpathmoveto{\pgfqpoint{5.519602in}{1.260575in}}%
\pgfpathcurveto{\pgfqpoint{5.526735in}{1.260575in}}{\pgfqpoint{5.533576in}{1.263409in}}{\pgfqpoint{5.538620in}{1.268453in}}%
\pgfpathcurveto{\pgfqpoint{5.543664in}{1.273496in}}{\pgfqpoint{5.546498in}{1.280338in}}{\pgfqpoint{5.546498in}{1.287471in}}%
\pgfpathcurveto{\pgfqpoint{5.546498in}{1.294604in}}{\pgfqpoint{5.543664in}{1.301445in}}{\pgfqpoint{5.538620in}{1.306489in}}%
\pgfpathcurveto{\pgfqpoint{5.533576in}{1.311533in}}{\pgfqpoint{5.526735in}{1.314367in}}{\pgfqpoint{5.519602in}{1.314367in}}%
\pgfpathcurveto{\pgfqpoint{5.512469in}{1.314367in}}{\pgfqpoint{5.505627in}{1.311533in}}{\pgfqpoint{5.500584in}{1.306489in}}%
\pgfpathcurveto{\pgfqpoint{5.495540in}{1.301445in}}{\pgfqpoint{5.492706in}{1.294604in}}{\pgfqpoint{5.492706in}{1.287471in}}%
\pgfpathcurveto{\pgfqpoint{5.492706in}{1.280338in}}{\pgfqpoint{5.495540in}{1.273496in}}{\pgfqpoint{5.500584in}{1.268453in}}%
\pgfpathcurveto{\pgfqpoint{5.505627in}{1.263409in}}{\pgfqpoint{5.512469in}{1.260575in}}{\pgfqpoint{5.519602in}{1.260575in}}%
\pgfpathclose%
\pgfusepath{stroke,fill}%
\end{pgfscope}%
\begin{pgfscope}%
\pgfpathrectangle{\pgfqpoint{4.985294in}{0.500000in}}{\pgfqpoint{1.764706in}{1.700000in}}%
\pgfusepath{clip}%
\pgfsetbuttcap%
\pgfsetroundjoin%
\definecolor{currentfill}{rgb}{0.952404,0.449449,0.307210}%
\pgfsetfillcolor{currentfill}%
\pgfsetlinewidth{0.311001pt}%
\definecolor{currentstroke}{rgb}{1.000000,1.000000,1.000000}%
\pgfsetstrokecolor{currentstroke}%
\pgfsetdash{}{0pt}%
\pgfpathmoveto{\pgfqpoint{6.151338in}{1.368282in}}%
\pgfpathcurveto{\pgfqpoint{6.158471in}{1.368282in}}{\pgfqpoint{6.165313in}{1.371116in}}{\pgfqpoint{6.170356in}{1.376159in}}%
\pgfpathcurveto{\pgfqpoint{6.175400in}{1.381203in}}{\pgfqpoint{6.178234in}{1.388045in}}{\pgfqpoint{6.178234in}{1.395178in}}%
\pgfpathcurveto{\pgfqpoint{6.178234in}{1.402310in}}{\pgfqpoint{6.175400in}{1.409152in}}{\pgfqpoint{6.170356in}{1.414196in}}%
\pgfpathcurveto{\pgfqpoint{6.165313in}{1.419239in}}{\pgfqpoint{6.158471in}{1.422073in}}{\pgfqpoint{6.151338in}{1.422073in}}%
\pgfpathcurveto{\pgfqpoint{6.144205in}{1.422073in}}{\pgfqpoint{6.137364in}{1.419239in}}{\pgfqpoint{6.132320in}{1.414196in}}%
\pgfpathcurveto{\pgfqpoint{6.127276in}{1.409152in}}{\pgfqpoint{6.124443in}{1.402310in}}{\pgfqpoint{6.124443in}{1.395178in}}%
\pgfpathcurveto{\pgfqpoint{6.124443in}{1.388045in}}{\pgfqpoint{6.127276in}{1.381203in}}{\pgfqpoint{6.132320in}{1.376159in}}%
\pgfpathcurveto{\pgfqpoint{6.137364in}{1.371116in}}{\pgfqpoint{6.144205in}{1.368282in}}{\pgfqpoint{6.151338in}{1.368282in}}%
\pgfpathclose%
\pgfusepath{stroke,fill}%
\end{pgfscope}%
\begin{pgfscope}%
\pgfpathrectangle{\pgfqpoint{4.985294in}{0.500000in}}{\pgfqpoint{1.764706in}{1.700000in}}%
\pgfusepath{clip}%
\pgfsetbuttcap%
\pgfsetroundjoin%
\definecolor{currentfill}{rgb}{0.976287,0.879862,0.805788}%
\pgfsetfillcolor{currentfill}%
\pgfsetlinewidth{0.311001pt}%
\definecolor{currentstroke}{rgb}{1.000000,1.000000,1.000000}%
\pgfsetstrokecolor{currentstroke}%
\pgfsetdash{}{0pt}%
\pgfpathmoveto{\pgfqpoint{6.325118in}{1.154454in}}%
\pgfpathcurveto{\pgfqpoint{6.332251in}{1.154454in}}{\pgfqpoint{6.339092in}{1.157288in}}{\pgfqpoint{6.344136in}{1.162331in}}%
\pgfpathcurveto{\pgfqpoint{6.349180in}{1.167375in}}{\pgfqpoint{6.352014in}{1.174217in}}{\pgfqpoint{6.352014in}{1.181349in}}%
\pgfpathcurveto{\pgfqpoint{6.352014in}{1.188482in}}{\pgfqpoint{6.349180in}{1.195324in}}{\pgfqpoint{6.344136in}{1.200368in}}%
\pgfpathcurveto{\pgfqpoint{6.339092in}{1.205411in}}{\pgfqpoint{6.332251in}{1.208245in}}{\pgfqpoint{6.325118in}{1.208245in}}%
\pgfpathcurveto{\pgfqpoint{6.317985in}{1.208245in}}{\pgfqpoint{6.311143in}{1.205411in}}{\pgfqpoint{6.306100in}{1.200368in}}%
\pgfpathcurveto{\pgfqpoint{6.301056in}{1.195324in}}{\pgfqpoint{6.298222in}{1.188482in}}{\pgfqpoint{6.298222in}{1.181349in}}%
\pgfpathcurveto{\pgfqpoint{6.298222in}{1.174217in}}{\pgfqpoint{6.301056in}{1.167375in}}{\pgfqpoint{6.306100in}{1.162331in}}%
\pgfpathcurveto{\pgfqpoint{6.311143in}{1.157288in}}{\pgfqpoint{6.317985in}{1.154454in}}{\pgfqpoint{6.325118in}{1.154454in}}%
\pgfpathclose%
\pgfusepath{stroke,fill}%
\end{pgfscope}%
\begin{pgfscope}%
\pgfpathrectangle{\pgfqpoint{4.985294in}{0.500000in}}{\pgfqpoint{1.764706in}{1.700000in}}%
\pgfusepath{clip}%
\pgfsetbuttcap%
\pgfsetroundjoin%
\definecolor{currentfill}{rgb}{0.976961,0.885681,0.814303}%
\pgfsetfillcolor{currentfill}%
\pgfsetlinewidth{0.311001pt}%
\definecolor{currentstroke}{rgb}{1.000000,1.000000,1.000000}%
\pgfsetstrokecolor{currentstroke}%
\pgfsetdash{}{0pt}%
\pgfpathmoveto{\pgfqpoint{6.260084in}{1.596506in}}%
\pgfpathcurveto{\pgfqpoint{6.267216in}{1.596506in}}{\pgfqpoint{6.274058in}{1.599340in}}{\pgfqpoint{6.279102in}{1.604383in}}%
\pgfpathcurveto{\pgfqpoint{6.284145in}{1.609427in}}{\pgfqpoint{6.286979in}{1.616269in}}{\pgfqpoint{6.286979in}{1.623402in}}%
\pgfpathcurveto{\pgfqpoint{6.286979in}{1.630534in}}{\pgfqpoint{6.284145in}{1.637376in}}{\pgfqpoint{6.279102in}{1.642420in}}%
\pgfpathcurveto{\pgfqpoint{6.274058in}{1.647463in}}{\pgfqpoint{6.267216in}{1.650297in}}{\pgfqpoint{6.260084in}{1.650297in}}%
\pgfpathcurveto{\pgfqpoint{6.252951in}{1.650297in}}{\pgfqpoint{6.246109in}{1.647463in}}{\pgfqpoint{6.241065in}{1.642420in}}%
\pgfpathcurveto{\pgfqpoint{6.236022in}{1.637376in}}{\pgfqpoint{6.233188in}{1.630534in}}{\pgfqpoint{6.233188in}{1.623402in}}%
\pgfpathcurveto{\pgfqpoint{6.233188in}{1.616269in}}{\pgfqpoint{6.236022in}{1.609427in}}{\pgfqpoint{6.241065in}{1.604383in}}%
\pgfpathcurveto{\pgfqpoint{6.246109in}{1.599340in}}{\pgfqpoint{6.252951in}{1.596506in}}{\pgfqpoint{6.260084in}{1.596506in}}%
\pgfpathclose%
\pgfusepath{stroke,fill}%
\end{pgfscope}%
\begin{pgfscope}%
\pgfpathrectangle{\pgfqpoint{4.985294in}{0.500000in}}{\pgfqpoint{1.764706in}{1.700000in}}%
\pgfusepath{clip}%
\pgfsetbuttcap%
\pgfsetroundjoin%
\definecolor{currentfill}{rgb}{0.887314,0.204699,0.257695}%
\pgfsetfillcolor{currentfill}%
\pgfsetlinewidth{0.311001pt}%
\definecolor{currentstroke}{rgb}{1.000000,1.000000,1.000000}%
\pgfsetstrokecolor{currentstroke}%
\pgfsetdash{}{0pt}%
\pgfpathmoveto{\pgfqpoint{6.159865in}{0.788766in}}%
\pgfpathcurveto{\pgfqpoint{6.166998in}{0.788766in}}{\pgfqpoint{6.173840in}{0.791600in}}{\pgfqpoint{6.178883in}{0.796644in}}%
\pgfpathcurveto{\pgfqpoint{6.183927in}{0.801687in}}{\pgfqpoint{6.186761in}{0.808529in}}{\pgfqpoint{6.186761in}{0.815662in}}%
\pgfpathcurveto{\pgfqpoint{6.186761in}{0.822795in}}{\pgfqpoint{6.183927in}{0.829636in}}{\pgfqpoint{6.178883in}{0.834680in}}%
\pgfpathcurveto{\pgfqpoint{6.173840in}{0.839724in}}{\pgfqpoint{6.166998in}{0.842558in}}{\pgfqpoint{6.159865in}{0.842558in}}%
\pgfpathcurveto{\pgfqpoint{6.152732in}{0.842558in}}{\pgfqpoint{6.145891in}{0.839724in}}{\pgfqpoint{6.140847in}{0.834680in}}%
\pgfpathcurveto{\pgfqpoint{6.135803in}{0.829636in}}{\pgfqpoint{6.132970in}{0.822795in}}{\pgfqpoint{6.132970in}{0.815662in}}%
\pgfpathcurveto{\pgfqpoint{6.132970in}{0.808529in}}{\pgfqpoint{6.135803in}{0.801687in}}{\pgfqpoint{6.140847in}{0.796644in}}%
\pgfpathcurveto{\pgfqpoint{6.145891in}{0.791600in}}{\pgfqpoint{6.152732in}{0.788766in}}{\pgfqpoint{6.159865in}{0.788766in}}%
\pgfpathclose%
\pgfusepath{stroke,fill}%
\end{pgfscope}%
\begin{pgfscope}%
\pgfpathrectangle{\pgfqpoint{4.985294in}{0.500000in}}{\pgfqpoint{1.764706in}{1.700000in}}%
\pgfusepath{clip}%
\pgfsetbuttcap%
\pgfsetroundjoin%
\definecolor{currentfill}{rgb}{0.972726,0.844889,0.754401}%
\pgfsetfillcolor{currentfill}%
\pgfsetlinewidth{0.311001pt}%
\definecolor{currentstroke}{rgb}{1.000000,1.000000,1.000000}%
\pgfsetstrokecolor{currentstroke}%
\pgfsetdash{}{0pt}%
\pgfpathmoveto{\pgfqpoint{5.478160in}{1.331097in}}%
\pgfpathcurveto{\pgfqpoint{5.485293in}{1.331097in}}{\pgfqpoint{5.492135in}{1.333931in}}{\pgfqpoint{5.497178in}{1.338974in}}%
\pgfpathcurveto{\pgfqpoint{5.502222in}{1.344018in}}{\pgfqpoint{5.505056in}{1.350860in}}{\pgfqpoint{5.505056in}{1.357992in}}%
\pgfpathcurveto{\pgfqpoint{5.505056in}{1.365125in}}{\pgfqpoint{5.502222in}{1.371967in}}{\pgfqpoint{5.497178in}{1.377010in}}%
\pgfpathcurveto{\pgfqpoint{5.492135in}{1.382054in}}{\pgfqpoint{5.485293in}{1.384888in}}{\pgfqpoint{5.478160in}{1.384888in}}%
\pgfpathcurveto{\pgfqpoint{5.471027in}{1.384888in}}{\pgfqpoint{5.464186in}{1.382054in}}{\pgfqpoint{5.459142in}{1.377010in}}%
\pgfpathcurveto{\pgfqpoint{5.454098in}{1.371967in}}{\pgfqpoint{5.451264in}{1.365125in}}{\pgfqpoint{5.451264in}{1.357992in}}%
\pgfpathcurveto{\pgfqpoint{5.451264in}{1.350860in}}{\pgfqpoint{5.454098in}{1.344018in}}{\pgfqpoint{5.459142in}{1.338974in}}%
\pgfpathcurveto{\pgfqpoint{5.464186in}{1.333931in}}{\pgfqpoint{5.471027in}{1.331097in}}{\pgfqpoint{5.478160in}{1.331097in}}%
\pgfpathclose%
\pgfusepath{stroke,fill}%
\end{pgfscope}%
\begin{pgfscope}%
\pgfpathrectangle{\pgfqpoint{4.985294in}{0.500000in}}{\pgfqpoint{1.764706in}{1.700000in}}%
\pgfusepath{clip}%
\pgfsetbuttcap%
\pgfsetroundjoin%
\definecolor{currentfill}{rgb}{0.973832,0.856556,0.771584}%
\pgfsetfillcolor{currentfill}%
\pgfsetlinewidth{0.311001pt}%
\definecolor{currentstroke}{rgb}{1.000000,1.000000,1.000000}%
\pgfsetstrokecolor{currentstroke}%
\pgfsetdash{}{0pt}%
\pgfpathmoveto{\pgfqpoint{5.472754in}{1.275996in}}%
\pgfpathcurveto{\pgfqpoint{5.479887in}{1.275996in}}{\pgfqpoint{5.486729in}{1.278830in}}{\pgfqpoint{5.491773in}{1.283874in}}%
\pgfpathcurveto{\pgfqpoint{5.496816in}{1.288917in}}{\pgfqpoint{5.499650in}{1.295759in}}{\pgfqpoint{5.499650in}{1.302892in}}%
\pgfpathcurveto{\pgfqpoint{5.499650in}{1.310025in}}{\pgfqpoint{5.496816in}{1.316866in}}{\pgfqpoint{5.491773in}{1.321910in}}%
\pgfpathcurveto{\pgfqpoint{5.486729in}{1.326954in}}{\pgfqpoint{5.479887in}{1.329788in}}{\pgfqpoint{5.472754in}{1.329788in}}%
\pgfpathcurveto{\pgfqpoint{5.465622in}{1.329788in}}{\pgfqpoint{5.458780in}{1.326954in}}{\pgfqpoint{5.453736in}{1.321910in}}%
\pgfpathcurveto{\pgfqpoint{5.448693in}{1.316866in}}{\pgfqpoint{5.445859in}{1.310025in}}{\pgfqpoint{5.445859in}{1.302892in}}%
\pgfpathcurveto{\pgfqpoint{5.445859in}{1.295759in}}{\pgfqpoint{5.448693in}{1.288917in}}{\pgfqpoint{5.453736in}{1.283874in}}%
\pgfpathcurveto{\pgfqpoint{5.458780in}{1.278830in}}{\pgfqpoint{5.465622in}{1.275996in}}{\pgfqpoint{5.472754in}{1.275996in}}%
\pgfpathclose%
\pgfusepath{stroke,fill}%
\end{pgfscope}%
\begin{pgfscope}%
\pgfpathrectangle{\pgfqpoint{4.985294in}{0.500000in}}{\pgfqpoint{1.764706in}{1.700000in}}%
\pgfusepath{clip}%
\pgfsetbuttcap%
\pgfsetroundjoin%
\definecolor{currentfill}{rgb}{0.963559,0.632016,0.472047}%
\pgfsetfillcolor{currentfill}%
\pgfsetlinewidth{0.311001pt}%
\definecolor{currentstroke}{rgb}{1.000000,1.000000,1.000000}%
\pgfsetstrokecolor{currentstroke}%
\pgfsetdash{}{0pt}%
\pgfpathmoveto{\pgfqpoint{6.173539in}{1.429718in}}%
\pgfpathcurveto{\pgfqpoint{6.180672in}{1.429718in}}{\pgfqpoint{6.187514in}{1.432552in}}{\pgfqpoint{6.192557in}{1.437596in}}%
\pgfpathcurveto{\pgfqpoint{6.197601in}{1.442640in}}{\pgfqpoint{6.200435in}{1.449481in}}{\pgfqpoint{6.200435in}{1.456614in}}%
\pgfpathcurveto{\pgfqpoint{6.200435in}{1.463747in}}{\pgfqpoint{6.197601in}{1.470589in}}{\pgfqpoint{6.192557in}{1.475632in}}%
\pgfpathcurveto{\pgfqpoint{6.187514in}{1.480676in}}{\pgfqpoint{6.180672in}{1.483510in}}{\pgfqpoint{6.173539in}{1.483510in}}%
\pgfpathcurveto{\pgfqpoint{6.166406in}{1.483510in}}{\pgfqpoint{6.159565in}{1.480676in}}{\pgfqpoint{6.154521in}{1.475632in}}%
\pgfpathcurveto{\pgfqpoint{6.149478in}{1.470589in}}{\pgfqpoint{6.146644in}{1.463747in}}{\pgfqpoint{6.146644in}{1.456614in}}%
\pgfpathcurveto{\pgfqpoint{6.146644in}{1.449481in}}{\pgfqpoint{6.149478in}{1.442640in}}{\pgfqpoint{6.154521in}{1.437596in}}%
\pgfpathcurveto{\pgfqpoint{6.159565in}{1.432552in}}{\pgfqpoint{6.166406in}{1.429718in}}{\pgfqpoint{6.173539in}{1.429718in}}%
\pgfpathclose%
\pgfusepath{stroke,fill}%
\end{pgfscope}%
\begin{pgfscope}%
\pgfpathrectangle{\pgfqpoint{4.985294in}{0.500000in}}{\pgfqpoint{1.764706in}{1.700000in}}%
\pgfusepath{clip}%
\pgfsetbuttcap%
\pgfsetroundjoin%
\definecolor{currentfill}{rgb}{0.965440,0.720101,0.576404}%
\pgfsetfillcolor{currentfill}%
\pgfsetlinewidth{0.311001pt}%
\definecolor{currentstroke}{rgb}{1.000000,1.000000,1.000000}%
\pgfsetstrokecolor{currentstroke}%
\pgfsetdash{}{0pt}%
\pgfpathmoveto{\pgfqpoint{6.141814in}{1.693291in}}%
\pgfpathcurveto{\pgfqpoint{6.148947in}{1.693291in}}{\pgfqpoint{6.155788in}{1.696124in}}{\pgfqpoint{6.160832in}{1.701168in}}%
\pgfpathcurveto{\pgfqpoint{6.165875in}{1.706212in}}{\pgfqpoint{6.168709in}{1.713053in}}{\pgfqpoint{6.168709in}{1.720186in}}%
\pgfpathcurveto{\pgfqpoint{6.168709in}{1.727319in}}{\pgfqpoint{6.165875in}{1.734161in}}{\pgfqpoint{6.160832in}{1.739204in}}%
\pgfpathcurveto{\pgfqpoint{6.155788in}{1.744248in}}{\pgfqpoint{6.148947in}{1.747082in}}{\pgfqpoint{6.141814in}{1.747082in}}%
\pgfpathcurveto{\pgfqpoint{6.134681in}{1.747082in}}{\pgfqpoint{6.127839in}{1.744248in}}{\pgfqpoint{6.122796in}{1.739204in}}%
\pgfpathcurveto{\pgfqpoint{6.117752in}{1.734161in}}{\pgfqpoint{6.114918in}{1.727319in}}{\pgfqpoint{6.114918in}{1.720186in}}%
\pgfpathcurveto{\pgfqpoint{6.114918in}{1.713053in}}{\pgfqpoint{6.117752in}{1.706212in}}{\pgfqpoint{6.122796in}{1.701168in}}%
\pgfpathcurveto{\pgfqpoint{6.127839in}{1.696124in}}{\pgfqpoint{6.134681in}{1.693291in}}{\pgfqpoint{6.141814in}{1.693291in}}%
\pgfpathclose%
\pgfusepath{stroke,fill}%
\end{pgfscope}%
\begin{pgfscope}%
\pgfpathrectangle{\pgfqpoint{4.985294in}{0.500000in}}{\pgfqpoint{1.764706in}{1.700000in}}%
\pgfusepath{clip}%
\pgfsetbuttcap%
\pgfsetroundjoin%
\definecolor{currentfill}{rgb}{0.963728,0.638439,0.479050}%
\pgfsetfillcolor{currentfill}%
\pgfsetlinewidth{0.311001pt}%
\definecolor{currentstroke}{rgb}{1.000000,1.000000,1.000000}%
\pgfsetstrokecolor{currentstroke}%
\pgfsetdash{}{0pt}%
\pgfpathmoveto{\pgfqpoint{5.384130in}{1.611113in}}%
\pgfpathcurveto{\pgfqpoint{5.391263in}{1.611113in}}{\pgfqpoint{5.398105in}{1.613947in}}{\pgfqpoint{5.403148in}{1.618990in}}%
\pgfpathcurveto{\pgfqpoint{5.408192in}{1.624034in}}{\pgfqpoint{5.411026in}{1.630876in}}{\pgfqpoint{5.411026in}{1.638009in}}%
\pgfpathcurveto{\pgfqpoint{5.411026in}{1.645141in}}{\pgfqpoint{5.408192in}{1.651983in}}{\pgfqpoint{5.403148in}{1.657027in}}%
\pgfpathcurveto{\pgfqpoint{5.398105in}{1.662070in}}{\pgfqpoint{5.391263in}{1.664904in}}{\pgfqpoint{5.384130in}{1.664904in}}%
\pgfpathcurveto{\pgfqpoint{5.376997in}{1.664904in}}{\pgfqpoint{5.370156in}{1.662070in}}{\pgfqpoint{5.365112in}{1.657027in}}%
\pgfpathcurveto{\pgfqpoint{5.360068in}{1.651983in}}{\pgfqpoint{5.357235in}{1.645141in}}{\pgfqpoint{5.357235in}{1.638009in}}%
\pgfpathcurveto{\pgfqpoint{5.357235in}{1.630876in}}{\pgfqpoint{5.360068in}{1.624034in}}{\pgfqpoint{5.365112in}{1.618990in}}%
\pgfpathcurveto{\pgfqpoint{5.370156in}{1.613947in}}{\pgfqpoint{5.376997in}{1.611113in}}{\pgfqpoint{5.384130in}{1.611113in}}%
\pgfpathclose%
\pgfusepath{stroke,fill}%
\end{pgfscope}%
\begin{pgfscope}%
\pgfpathrectangle{\pgfqpoint{4.985294in}{0.500000in}}{\pgfqpoint{1.764706in}{1.700000in}}%
\pgfusepath{clip}%
\pgfsetbuttcap%
\pgfsetroundjoin%
\definecolor{currentfill}{rgb}{0.956817,0.498820,0.345554}%
\pgfsetfillcolor{currentfill}%
\pgfsetlinewidth{0.311001pt}%
\definecolor{currentstroke}{rgb}{1.000000,1.000000,1.000000}%
\pgfsetstrokecolor{currentstroke}%
\pgfsetdash{}{0pt}%
\pgfpathmoveto{\pgfqpoint{6.075459in}{1.027606in}}%
\pgfpathcurveto{\pgfqpoint{6.082592in}{1.027606in}}{\pgfqpoint{6.089434in}{1.030440in}}{\pgfqpoint{6.094477in}{1.035483in}}%
\pgfpathcurveto{\pgfqpoint{6.099521in}{1.040527in}}{\pgfqpoint{6.102355in}{1.047369in}}{\pgfqpoint{6.102355in}{1.054502in}}%
\pgfpathcurveto{\pgfqpoint{6.102355in}{1.061634in}}{\pgfqpoint{6.099521in}{1.068476in}}{\pgfqpoint{6.094477in}{1.073520in}}%
\pgfpathcurveto{\pgfqpoint{6.089434in}{1.078563in}}{\pgfqpoint{6.082592in}{1.081397in}}{\pgfqpoint{6.075459in}{1.081397in}}%
\pgfpathcurveto{\pgfqpoint{6.068326in}{1.081397in}}{\pgfqpoint{6.061485in}{1.078563in}}{\pgfqpoint{6.056441in}{1.073520in}}%
\pgfpathcurveto{\pgfqpoint{6.051397in}{1.068476in}}{\pgfqpoint{6.048564in}{1.061634in}}{\pgfqpoint{6.048564in}{1.054502in}}%
\pgfpathcurveto{\pgfqpoint{6.048564in}{1.047369in}}{\pgfqpoint{6.051397in}{1.040527in}}{\pgfqpoint{6.056441in}{1.035483in}}%
\pgfpathcurveto{\pgfqpoint{6.061485in}{1.030440in}}{\pgfqpoint{6.068326in}{1.027606in}}{\pgfqpoint{6.075459in}{1.027606in}}%
\pgfpathclose%
\pgfusepath{stroke,fill}%
\end{pgfscope}%
\begin{pgfscope}%
\pgfpathrectangle{\pgfqpoint{4.985294in}{0.500000in}}{\pgfqpoint{1.764706in}{1.700000in}}%
\pgfusepath{clip}%
\pgfsetbuttcap%
\pgfsetroundjoin%
\definecolor{currentfill}{rgb}{0.966328,0.750560,0.616961}%
\pgfsetfillcolor{currentfill}%
\pgfsetlinewidth{0.311001pt}%
\definecolor{currentstroke}{rgb}{1.000000,1.000000,1.000000}%
\pgfsetstrokecolor{currentstroke}%
\pgfsetdash{}{0pt}%
\pgfpathmoveto{\pgfqpoint{6.185609in}{1.167361in}}%
\pgfpathcurveto{\pgfqpoint{6.192741in}{1.167361in}}{\pgfqpoint{6.199583in}{1.170195in}}{\pgfqpoint{6.204627in}{1.175239in}}%
\pgfpathcurveto{\pgfqpoint{6.209670in}{1.180282in}}{\pgfqpoint{6.212504in}{1.187124in}}{\pgfqpoint{6.212504in}{1.194257in}}%
\pgfpathcurveto{\pgfqpoint{6.212504in}{1.201390in}}{\pgfqpoint{6.209670in}{1.208231in}}{\pgfqpoint{6.204627in}{1.213275in}}%
\pgfpathcurveto{\pgfqpoint{6.199583in}{1.218318in}}{\pgfqpoint{6.192741in}{1.221152in}}{\pgfqpoint{6.185609in}{1.221152in}}%
\pgfpathcurveto{\pgfqpoint{6.178476in}{1.221152in}}{\pgfqpoint{6.171634in}{1.218318in}}{\pgfqpoint{6.166590in}{1.213275in}}%
\pgfpathcurveto{\pgfqpoint{6.161547in}{1.208231in}}{\pgfqpoint{6.158713in}{1.201390in}}{\pgfqpoint{6.158713in}{1.194257in}}%
\pgfpathcurveto{\pgfqpoint{6.158713in}{1.187124in}}{\pgfqpoint{6.161547in}{1.180282in}}{\pgfqpoint{6.166590in}{1.175239in}}%
\pgfpathcurveto{\pgfqpoint{6.171634in}{1.170195in}}{\pgfqpoint{6.178476in}{1.167361in}}{\pgfqpoint{6.185609in}{1.167361in}}%
\pgfpathclose%
\pgfusepath{stroke,fill}%
\end{pgfscope}%
\begin{pgfscope}%
\pgfpathrectangle{\pgfqpoint{4.985294in}{0.500000in}}{\pgfqpoint{1.764706in}{1.700000in}}%
\pgfusepath{clip}%
\pgfsetbuttcap%
\pgfsetroundjoin%
\definecolor{currentfill}{rgb}{0.980678,0.914765,0.856766}%
\pgfsetfillcolor{currentfill}%
\pgfsetlinewidth{0.311001pt}%
\definecolor{currentstroke}{rgb}{1.000000,1.000000,1.000000}%
\pgfsetstrokecolor{currentstroke}%
\pgfsetdash{}{0pt}%
\pgfpathmoveto{\pgfqpoint{6.299431in}{1.368290in}}%
\pgfpathcurveto{\pgfqpoint{6.306564in}{1.368290in}}{\pgfqpoint{6.313406in}{1.371124in}}{\pgfqpoint{6.318450in}{1.376167in}}%
\pgfpathcurveto{\pgfqpoint{6.323493in}{1.381211in}}{\pgfqpoint{6.326327in}{1.388053in}}{\pgfqpoint{6.326327in}{1.395185in}}%
\pgfpathcurveto{\pgfqpoint{6.326327in}{1.402318in}}{\pgfqpoint{6.323493in}{1.409160in}}{\pgfqpoint{6.318450in}{1.414203in}}%
\pgfpathcurveto{\pgfqpoint{6.313406in}{1.419247in}}{\pgfqpoint{6.306564in}{1.422081in}}{\pgfqpoint{6.299431in}{1.422081in}}%
\pgfpathcurveto{\pgfqpoint{6.292299in}{1.422081in}}{\pgfqpoint{6.285457in}{1.419247in}}{\pgfqpoint{6.280413in}{1.414203in}}%
\pgfpathcurveto{\pgfqpoint{6.275370in}{1.409160in}}{\pgfqpoint{6.272536in}{1.402318in}}{\pgfqpoint{6.272536in}{1.395185in}}%
\pgfpathcurveto{\pgfqpoint{6.272536in}{1.388053in}}{\pgfqpoint{6.275370in}{1.381211in}}{\pgfqpoint{6.280413in}{1.376167in}}%
\pgfpathcurveto{\pgfqpoint{6.285457in}{1.371124in}}{\pgfqpoint{6.292299in}{1.368290in}}{\pgfqpoint{6.299431in}{1.368290in}}%
\pgfpathclose%
\pgfusepath{stroke,fill}%
\end{pgfscope}%
\begin{pgfscope}%
\pgfpathrectangle{\pgfqpoint{4.985294in}{0.500000in}}{\pgfqpoint{1.764706in}{1.700000in}}%
\pgfusepath{clip}%
\pgfsetbuttcap%
\pgfsetroundjoin%
\definecolor{currentfill}{rgb}{0.978376,0.897317,0.831308}%
\pgfsetfillcolor{currentfill}%
\pgfsetlinewidth{0.311001pt}%
\definecolor{currentstroke}{rgb}{1.000000,1.000000,1.000000}%
\pgfsetstrokecolor{currentstroke}%
\pgfsetdash{}{0pt}%
\pgfpathmoveto{\pgfqpoint{6.272077in}{1.199377in}}%
\pgfpathcurveto{\pgfqpoint{6.279210in}{1.199377in}}{\pgfqpoint{6.286051in}{1.202211in}}{\pgfqpoint{6.291095in}{1.207254in}}%
\pgfpathcurveto{\pgfqpoint{6.296139in}{1.212298in}}{\pgfqpoint{6.298972in}{1.219140in}}{\pgfqpoint{6.298972in}{1.226273in}}%
\pgfpathcurveto{\pgfqpoint{6.298972in}{1.233405in}}{\pgfqpoint{6.296139in}{1.240247in}}{\pgfqpoint{6.291095in}{1.245291in}}%
\pgfpathcurveto{\pgfqpoint{6.286051in}{1.250334in}}{\pgfqpoint{6.279210in}{1.253168in}}{\pgfqpoint{6.272077in}{1.253168in}}%
\pgfpathcurveto{\pgfqpoint{6.264944in}{1.253168in}}{\pgfqpoint{6.258102in}{1.250334in}}{\pgfqpoint{6.253059in}{1.245291in}}%
\pgfpathcurveto{\pgfqpoint{6.248015in}{1.240247in}}{\pgfqpoint{6.245181in}{1.233405in}}{\pgfqpoint{6.245181in}{1.226273in}}%
\pgfpathcurveto{\pgfqpoint{6.245181in}{1.219140in}}{\pgfqpoint{6.248015in}{1.212298in}}{\pgfqpoint{6.253059in}{1.207254in}}%
\pgfpathcurveto{\pgfqpoint{6.258102in}{1.202211in}}{\pgfqpoint{6.264944in}{1.199377in}}{\pgfqpoint{6.272077in}{1.199377in}}%
\pgfpathclose%
\pgfusepath{stroke,fill}%
\end{pgfscope}%
\begin{pgfscope}%
\pgfpathrectangle{\pgfqpoint{4.985294in}{0.500000in}}{\pgfqpoint{1.764706in}{1.700000in}}%
\pgfusepath{clip}%
\pgfsetbuttcap%
\pgfsetroundjoin%
\definecolor{currentfill}{rgb}{0.975018,0.868213,0.788710}%
\pgfsetfillcolor{currentfill}%
\pgfsetlinewidth{0.311001pt}%
\definecolor{currentstroke}{rgb}{1.000000,1.000000,1.000000}%
\pgfsetstrokecolor{currentstroke}%
\pgfsetdash{}{0pt}%
\pgfpathmoveto{\pgfqpoint{5.467955in}{1.361598in}}%
\pgfpathcurveto{\pgfqpoint{5.475088in}{1.361598in}}{\pgfqpoint{5.481930in}{1.364432in}}{\pgfqpoint{5.486974in}{1.369476in}}%
\pgfpathcurveto{\pgfqpoint{5.492017in}{1.374520in}}{\pgfqpoint{5.494851in}{1.381361in}}{\pgfqpoint{5.494851in}{1.388494in}}%
\pgfpathcurveto{\pgfqpoint{5.494851in}{1.395627in}}{\pgfqpoint{5.492017in}{1.402468in}}{\pgfqpoint{5.486974in}{1.407512in}}%
\pgfpathcurveto{\pgfqpoint{5.481930in}{1.412556in}}{\pgfqpoint{5.475088in}{1.415390in}}{\pgfqpoint{5.467955in}{1.415390in}}%
\pgfpathcurveto{\pgfqpoint{5.460823in}{1.415390in}}{\pgfqpoint{5.453981in}{1.412556in}}{\pgfqpoint{5.448937in}{1.407512in}}%
\pgfpathcurveto{\pgfqpoint{5.443894in}{1.402468in}}{\pgfqpoint{5.441060in}{1.395627in}}{\pgfqpoint{5.441060in}{1.388494in}}%
\pgfpathcurveto{\pgfqpoint{5.441060in}{1.381361in}}{\pgfqpoint{5.443894in}{1.374520in}}{\pgfqpoint{5.448937in}{1.369476in}}%
\pgfpathcurveto{\pgfqpoint{5.453981in}{1.364432in}}{\pgfqpoint{5.460823in}{1.361598in}}{\pgfqpoint{5.467955in}{1.361598in}}%
\pgfpathclose%
\pgfusepath{stroke,fill}%
\end{pgfscope}%
\begin{pgfscope}%
\pgfpathrectangle{\pgfqpoint{4.985294in}{0.500000in}}{\pgfqpoint{1.764706in}{1.700000in}}%
\pgfusepath{clip}%
\pgfsetbuttcap%
\pgfsetroundjoin%
\definecolor{currentfill}{rgb}{0.978376,0.897317,0.831308}%
\pgfsetfillcolor{currentfill}%
\pgfsetlinewidth{0.311001pt}%
\definecolor{currentstroke}{rgb}{1.000000,1.000000,1.000000}%
\pgfsetstrokecolor{currentstroke}%
\pgfsetdash{}{0pt}%
\pgfpathmoveto{\pgfqpoint{6.326010in}{1.460134in}}%
\pgfpathcurveto{\pgfqpoint{6.333143in}{1.460134in}}{\pgfqpoint{6.339985in}{1.462968in}}{\pgfqpoint{6.345029in}{1.468012in}}%
\pgfpathcurveto{\pgfqpoint{6.350072in}{1.473055in}}{\pgfqpoint{6.352906in}{1.479897in}}{\pgfqpoint{6.352906in}{1.487030in}}%
\pgfpathcurveto{\pgfqpoint{6.352906in}{1.494162in}}{\pgfqpoint{6.350072in}{1.501004in}}{\pgfqpoint{6.345029in}{1.506048in}}%
\pgfpathcurveto{\pgfqpoint{6.339985in}{1.511091in}}{\pgfqpoint{6.333143in}{1.513925in}}{\pgfqpoint{6.326010in}{1.513925in}}%
\pgfpathcurveto{\pgfqpoint{6.318878in}{1.513925in}}{\pgfqpoint{6.312036in}{1.511091in}}{\pgfqpoint{6.306992in}{1.506048in}}%
\pgfpathcurveto{\pgfqpoint{6.301949in}{1.501004in}}{\pgfqpoint{6.299115in}{1.494162in}}{\pgfqpoint{6.299115in}{1.487030in}}%
\pgfpathcurveto{\pgfqpoint{6.299115in}{1.479897in}}{\pgfqpoint{6.301949in}{1.473055in}}{\pgfqpoint{6.306992in}{1.468012in}}%
\pgfpathcurveto{\pgfqpoint{6.312036in}{1.462968in}}{\pgfqpoint{6.318878in}{1.460134in}}{\pgfqpoint{6.326010in}{1.460134in}}%
\pgfpathclose%
\pgfusepath{stroke,fill}%
\end{pgfscope}%
\begin{pgfscope}%
\pgfpathrectangle{\pgfqpoint{4.985294in}{0.500000in}}{\pgfqpoint{1.764706in}{1.700000in}}%
\pgfusepath{clip}%
\pgfsetbuttcap%
\pgfsetroundjoin%
\definecolor{currentfill}{rgb}{0.968509,0.792226,0.676405}%
\pgfsetfillcolor{currentfill}%
\pgfsetlinewidth{0.311001pt}%
\definecolor{currentstroke}{rgb}{1.000000,1.000000,1.000000}%
\pgfsetstrokecolor{currentstroke}%
\pgfsetdash{}{0pt}%
\pgfpathmoveto{\pgfqpoint{6.179585in}{1.081820in}}%
\pgfpathcurveto{\pgfqpoint{6.186718in}{1.081820in}}{\pgfqpoint{6.193559in}{1.084654in}}{\pgfqpoint{6.198603in}{1.089697in}}%
\pgfpathcurveto{\pgfqpoint{6.203647in}{1.094741in}}{\pgfqpoint{6.206481in}{1.101583in}}{\pgfqpoint{6.206481in}{1.108715in}}%
\pgfpathcurveto{\pgfqpoint{6.206481in}{1.115848in}}{\pgfqpoint{6.203647in}{1.122690in}}{\pgfqpoint{6.198603in}{1.127733in}}%
\pgfpathcurveto{\pgfqpoint{6.193559in}{1.132777in}}{\pgfqpoint{6.186718in}{1.135611in}}{\pgfqpoint{6.179585in}{1.135611in}}%
\pgfpathcurveto{\pgfqpoint{6.172452in}{1.135611in}}{\pgfqpoint{6.165611in}{1.132777in}}{\pgfqpoint{6.160567in}{1.127733in}}%
\pgfpathcurveto{\pgfqpoint{6.155523in}{1.122690in}}{\pgfqpoint{6.152689in}{1.115848in}}{\pgfqpoint{6.152689in}{1.108715in}}%
\pgfpathcurveto{\pgfqpoint{6.152689in}{1.101583in}}{\pgfqpoint{6.155523in}{1.094741in}}{\pgfqpoint{6.160567in}{1.089697in}}%
\pgfpathcurveto{\pgfqpoint{6.165611in}{1.084654in}}{\pgfqpoint{6.172452in}{1.081820in}}{\pgfqpoint{6.179585in}{1.081820in}}%
\pgfpathclose%
\pgfusepath{stroke,fill}%
\end{pgfscope}%
\begin{pgfscope}%
\pgfpathrectangle{\pgfqpoint{4.985294in}{0.500000in}}{\pgfqpoint{1.764706in}{1.700000in}}%
\pgfusepath{clip}%
\pgfsetbuttcap%
\pgfsetroundjoin%
\definecolor{currentfill}{rgb}{0.970255,0.815666,0.711203}%
\pgfsetfillcolor{currentfill}%
\pgfsetlinewidth{0.311001pt}%
\definecolor{currentstroke}{rgb}{1.000000,1.000000,1.000000}%
\pgfsetstrokecolor{currentstroke}%
\pgfsetdash{}{0pt}%
\pgfpathmoveto{\pgfqpoint{5.545385in}{0.987513in}}%
\pgfpathcurveto{\pgfqpoint{5.552518in}{0.987513in}}{\pgfqpoint{5.559359in}{0.990347in}}{\pgfqpoint{5.564403in}{0.995390in}}%
\pgfpathcurveto{\pgfqpoint{5.569447in}{1.000434in}}{\pgfqpoint{5.572281in}{1.007276in}}{\pgfqpoint{5.572281in}{1.014409in}}%
\pgfpathcurveto{\pgfqpoint{5.572281in}{1.021541in}}{\pgfqpoint{5.569447in}{1.028383in}}{\pgfqpoint{5.564403in}{1.033427in}}%
\pgfpathcurveto{\pgfqpoint{5.559359in}{1.038470in}}{\pgfqpoint{5.552518in}{1.041304in}}{\pgfqpoint{5.545385in}{1.041304in}}%
\pgfpathcurveto{\pgfqpoint{5.538252in}{1.041304in}}{\pgfqpoint{5.531410in}{1.038470in}}{\pgfqpoint{5.526367in}{1.033427in}}%
\pgfpathcurveto{\pgfqpoint{5.521323in}{1.028383in}}{\pgfqpoint{5.518489in}{1.021541in}}{\pgfqpoint{5.518489in}{1.014409in}}%
\pgfpathcurveto{\pgfqpoint{5.518489in}{1.007276in}}{\pgfqpoint{5.521323in}{1.000434in}}{\pgfqpoint{5.526367in}{0.995390in}}%
\pgfpathcurveto{\pgfqpoint{5.531410in}{0.990347in}}{\pgfqpoint{5.538252in}{0.987513in}}{\pgfqpoint{5.545385in}{0.987513in}}%
\pgfpathclose%
\pgfusepath{stroke,fill}%
\end{pgfscope}%
\begin{pgfscope}%
\pgfpathrectangle{\pgfqpoint{4.985294in}{0.500000in}}{\pgfqpoint{1.764706in}{1.700000in}}%
\pgfusepath{clip}%
\pgfsetbuttcap%
\pgfsetroundjoin%
\definecolor{currentfill}{rgb}{0.980678,0.914765,0.856766}%
\pgfsetfillcolor{currentfill}%
\pgfsetlinewidth{0.311001pt}%
\definecolor{currentstroke}{rgb}{1.000000,1.000000,1.000000}%
\pgfsetstrokecolor{currentstroke}%
\pgfsetdash{}{0pt}%
\pgfpathmoveto{\pgfqpoint{5.424667in}{1.330983in}}%
\pgfpathcurveto{\pgfqpoint{5.431800in}{1.330983in}}{\pgfqpoint{5.438641in}{1.333817in}}{\pgfqpoint{5.443685in}{1.338860in}}%
\pgfpathcurveto{\pgfqpoint{5.448729in}{1.343904in}}{\pgfqpoint{5.451562in}{1.350746in}}{\pgfqpoint{5.451562in}{1.357878in}}%
\pgfpathcurveto{\pgfqpoint{5.451562in}{1.365011in}}{\pgfqpoint{5.448729in}{1.371853in}}{\pgfqpoint{5.443685in}{1.376897in}}%
\pgfpathcurveto{\pgfqpoint{5.438641in}{1.381940in}}{\pgfqpoint{5.431800in}{1.384774in}}{\pgfqpoint{5.424667in}{1.384774in}}%
\pgfpathcurveto{\pgfqpoint{5.417534in}{1.384774in}}{\pgfqpoint{5.410692in}{1.381940in}}{\pgfqpoint{5.405649in}{1.376897in}}%
\pgfpathcurveto{\pgfqpoint{5.400605in}{1.371853in}}{\pgfqpoint{5.397771in}{1.365011in}}{\pgfqpoint{5.397771in}{1.357878in}}%
\pgfpathcurveto{\pgfqpoint{5.397771in}{1.350746in}}{\pgfqpoint{5.400605in}{1.343904in}}{\pgfqpoint{5.405649in}{1.338860in}}%
\pgfpathcurveto{\pgfqpoint{5.410692in}{1.333817in}}{\pgfqpoint{5.417534in}{1.330983in}}{\pgfqpoint{5.424667in}{1.330983in}}%
\pgfpathclose%
\pgfusepath{stroke,fill}%
\end{pgfscope}%
\begin{pgfscope}%
\pgfpathrectangle{\pgfqpoint{4.985294in}{0.500000in}}{\pgfqpoint{1.764706in}{1.700000in}}%
\pgfusepath{clip}%
\pgfsetbuttcap%
\pgfsetroundjoin%
\definecolor{currentfill}{rgb}{0.957344,0.505732,0.351309}%
\pgfsetfillcolor{currentfill}%
\pgfsetlinewidth{0.311001pt}%
\definecolor{currentstroke}{rgb}{1.000000,1.000000,1.000000}%
\pgfsetstrokecolor{currentstroke}%
\pgfsetdash{}{0pt}%
\pgfpathmoveto{\pgfqpoint{6.115620in}{1.102443in}}%
\pgfpathcurveto{\pgfqpoint{6.122753in}{1.102443in}}{\pgfqpoint{6.129595in}{1.105277in}}{\pgfqpoint{6.134638in}{1.110321in}}%
\pgfpathcurveto{\pgfqpoint{6.139682in}{1.115365in}}{\pgfqpoint{6.142516in}{1.122206in}}{\pgfqpoint{6.142516in}{1.129339in}}%
\pgfpathcurveto{\pgfqpoint{6.142516in}{1.136472in}}{\pgfqpoint{6.139682in}{1.143313in}}{\pgfqpoint{6.134638in}{1.148357in}}%
\pgfpathcurveto{\pgfqpoint{6.129595in}{1.153401in}}{\pgfqpoint{6.122753in}{1.156235in}}{\pgfqpoint{6.115620in}{1.156235in}}%
\pgfpathcurveto{\pgfqpoint{6.108487in}{1.156235in}}{\pgfqpoint{6.101646in}{1.153401in}}{\pgfqpoint{6.096602in}{1.148357in}}%
\pgfpathcurveto{\pgfqpoint{6.091558in}{1.143313in}}{\pgfqpoint{6.088725in}{1.136472in}}{\pgfqpoint{6.088725in}{1.129339in}}%
\pgfpathcurveto{\pgfqpoint{6.088725in}{1.122206in}}{\pgfqpoint{6.091558in}{1.115365in}}{\pgfqpoint{6.096602in}{1.110321in}}%
\pgfpathcurveto{\pgfqpoint{6.101646in}{1.105277in}}{\pgfqpoint{6.108487in}{1.102443in}}{\pgfqpoint{6.115620in}{1.102443in}}%
\pgfpathclose%
\pgfusepath{stroke,fill}%
\end{pgfscope}%
\begin{pgfscope}%
\pgfpathrectangle{\pgfqpoint{4.985294in}{0.500000in}}{\pgfqpoint{1.764706in}{1.700000in}}%
\pgfusepath{clip}%
\pgfsetbuttcap%
\pgfsetroundjoin%
\definecolor{currentfill}{rgb}{0.968931,0.798091,0.685123}%
\pgfsetfillcolor{currentfill}%
\pgfsetlinewidth{0.311001pt}%
\definecolor{currentstroke}{rgb}{1.000000,1.000000,1.000000}%
\pgfsetstrokecolor{currentstroke}%
\pgfsetdash{}{0pt}%
\pgfpathmoveto{\pgfqpoint{5.518486in}{1.659720in}}%
\pgfpathcurveto{\pgfqpoint{5.525619in}{1.659720in}}{\pgfqpoint{5.532460in}{1.662554in}}{\pgfqpoint{5.537504in}{1.667598in}}%
\pgfpathcurveto{\pgfqpoint{5.542548in}{1.672641in}}{\pgfqpoint{5.545381in}{1.679483in}}{\pgfqpoint{5.545381in}{1.686616in}}%
\pgfpathcurveto{\pgfqpoint{5.545381in}{1.693749in}}{\pgfqpoint{5.542548in}{1.700590in}}{\pgfqpoint{5.537504in}{1.705634in}}%
\pgfpathcurveto{\pgfqpoint{5.532460in}{1.710678in}}{\pgfqpoint{5.525619in}{1.713511in}}{\pgfqpoint{5.518486in}{1.713511in}}%
\pgfpathcurveto{\pgfqpoint{5.511353in}{1.713511in}}{\pgfqpoint{5.504511in}{1.710678in}}{\pgfqpoint{5.499468in}{1.705634in}}%
\pgfpathcurveto{\pgfqpoint{5.494424in}{1.700590in}}{\pgfqpoint{5.491590in}{1.693749in}}{\pgfqpoint{5.491590in}{1.686616in}}%
\pgfpathcurveto{\pgfqpoint{5.491590in}{1.679483in}}{\pgfqpoint{5.494424in}{1.672641in}}{\pgfqpoint{5.499468in}{1.667598in}}%
\pgfpathcurveto{\pgfqpoint{5.504511in}{1.662554in}}{\pgfqpoint{5.511353in}{1.659720in}}{\pgfqpoint{5.518486in}{1.659720in}}%
\pgfpathclose%
\pgfusepath{stroke,fill}%
\end{pgfscope}%
\begin{pgfscope}%
\pgfpathrectangle{\pgfqpoint{4.985294in}{0.500000in}}{\pgfqpoint{1.764706in}{1.700000in}}%
\pgfusepath{clip}%
\pgfsetbuttcap%
\pgfsetroundjoin%
\definecolor{currentfill}{rgb}{0.977657,0.891500,0.822809}%
\pgfsetfillcolor{currentfill}%
\pgfsetlinewidth{0.311001pt}%
\definecolor{currentstroke}{rgb}{1.000000,1.000000,1.000000}%
\pgfsetstrokecolor{currentstroke}%
\pgfsetdash{}{0pt}%
\pgfpathmoveto{\pgfqpoint{5.457852in}{1.434390in}}%
\pgfpathcurveto{\pgfqpoint{5.464985in}{1.434390in}}{\pgfqpoint{5.471827in}{1.437224in}}{\pgfqpoint{5.476870in}{1.442267in}}%
\pgfpathcurveto{\pgfqpoint{5.481914in}{1.447311in}}{\pgfqpoint{5.484748in}{1.454153in}}{\pgfqpoint{5.484748in}{1.461285in}}%
\pgfpathcurveto{\pgfqpoint{5.484748in}{1.468418in}}{\pgfqpoint{5.481914in}{1.475260in}}{\pgfqpoint{5.476870in}{1.480304in}}%
\pgfpathcurveto{\pgfqpoint{5.471827in}{1.485347in}}{\pgfqpoint{5.464985in}{1.488181in}}{\pgfqpoint{5.457852in}{1.488181in}}%
\pgfpathcurveto{\pgfqpoint{5.450719in}{1.488181in}}{\pgfqpoint{5.443878in}{1.485347in}}{\pgfqpoint{5.438834in}{1.480304in}}%
\pgfpathcurveto{\pgfqpoint{5.433790in}{1.475260in}}{\pgfqpoint{5.430957in}{1.468418in}}{\pgfqpoint{5.430957in}{1.461285in}}%
\pgfpathcurveto{\pgfqpoint{5.430957in}{1.454153in}}{\pgfqpoint{5.433790in}{1.447311in}}{\pgfqpoint{5.438834in}{1.442267in}}%
\pgfpathcurveto{\pgfqpoint{5.443878in}{1.437224in}}{\pgfqpoint{5.450719in}{1.434390in}}{\pgfqpoint{5.457852in}{1.434390in}}%
\pgfpathclose%
\pgfusepath{stroke,fill}%
\end{pgfscope}%
\begin{pgfscope}%
\pgfpathrectangle{\pgfqpoint{4.985294in}{0.500000in}}{\pgfqpoint{1.764706in}{1.700000in}}%
\pgfusepath{clip}%
\pgfsetbuttcap%
\pgfsetroundjoin%
\definecolor{currentfill}{rgb}{0.964433,0.670254,0.515093}%
\pgfsetfillcolor{currentfill}%
\pgfsetlinewidth{0.311001pt}%
\definecolor{currentstroke}{rgb}{1.000000,1.000000,1.000000}%
\pgfsetstrokecolor{currentstroke}%
\pgfsetdash{}{0pt}%
\pgfpathmoveto{\pgfqpoint{5.587491in}{1.697912in}}%
\pgfpathcurveto{\pgfqpoint{5.594624in}{1.697912in}}{\pgfqpoint{5.601466in}{1.700746in}}{\pgfqpoint{5.606509in}{1.705789in}}%
\pgfpathcurveto{\pgfqpoint{5.611553in}{1.710833in}}{\pgfqpoint{5.614387in}{1.717675in}}{\pgfqpoint{5.614387in}{1.724808in}}%
\pgfpathcurveto{\pgfqpoint{5.614387in}{1.731940in}}{\pgfqpoint{5.611553in}{1.738782in}}{\pgfqpoint{5.606509in}{1.743826in}}%
\pgfpathcurveto{\pgfqpoint{5.601466in}{1.748869in}}{\pgfqpoint{5.594624in}{1.751703in}}{\pgfqpoint{5.587491in}{1.751703in}}%
\pgfpathcurveto{\pgfqpoint{5.580358in}{1.751703in}}{\pgfqpoint{5.573517in}{1.748869in}}{\pgfqpoint{5.568473in}{1.743826in}}%
\pgfpathcurveto{\pgfqpoint{5.563429in}{1.738782in}}{\pgfqpoint{5.560595in}{1.731940in}}{\pgfqpoint{5.560595in}{1.724808in}}%
\pgfpathcurveto{\pgfqpoint{5.560595in}{1.717675in}}{\pgfqpoint{5.563429in}{1.710833in}}{\pgfqpoint{5.568473in}{1.705789in}}%
\pgfpathcurveto{\pgfqpoint{5.573517in}{1.700746in}}{\pgfqpoint{5.580358in}{1.697912in}}{\pgfqpoint{5.587491in}{1.697912in}}%
\pgfpathclose%
\pgfusepath{stroke,fill}%
\end{pgfscope}%
\begin{pgfscope}%
\pgfpathrectangle{\pgfqpoint{4.985294in}{0.500000in}}{\pgfqpoint{1.764706in}{1.700000in}}%
\pgfusepath{clip}%
\pgfsetbuttcap%
\pgfsetroundjoin%
\definecolor{currentfill}{rgb}{0.969803,0.809811,0.702523}%
\pgfsetfillcolor{currentfill}%
\pgfsetlinewidth{0.311001pt}%
\definecolor{currentstroke}{rgb}{1.000000,1.000000,1.000000}%
\pgfsetstrokecolor{currentstroke}%
\pgfsetdash{}{0pt}%
\pgfpathmoveto{\pgfqpoint{5.524221in}{1.098792in}}%
\pgfpathcurveto{\pgfqpoint{5.531354in}{1.098792in}}{\pgfqpoint{5.538196in}{1.101626in}}{\pgfqpoint{5.543239in}{1.106670in}}%
\pgfpathcurveto{\pgfqpoint{5.548283in}{1.111713in}}{\pgfqpoint{5.551117in}{1.118555in}}{\pgfqpoint{5.551117in}{1.125688in}}%
\pgfpathcurveto{\pgfqpoint{5.551117in}{1.132821in}}{\pgfqpoint{5.548283in}{1.139662in}}{\pgfqpoint{5.543239in}{1.144706in}}%
\pgfpathcurveto{\pgfqpoint{5.538196in}{1.149750in}}{\pgfqpoint{5.531354in}{1.152584in}}{\pgfqpoint{5.524221in}{1.152584in}}%
\pgfpathcurveto{\pgfqpoint{5.517088in}{1.152584in}}{\pgfqpoint{5.510247in}{1.149750in}}{\pgfqpoint{5.505203in}{1.144706in}}%
\pgfpathcurveto{\pgfqpoint{5.500159in}{1.139662in}}{\pgfqpoint{5.497326in}{1.132821in}}{\pgfqpoint{5.497326in}{1.125688in}}%
\pgfpathcurveto{\pgfqpoint{5.497326in}{1.118555in}}{\pgfqpoint{5.500159in}{1.111713in}}{\pgfqpoint{5.505203in}{1.106670in}}%
\pgfpathcurveto{\pgfqpoint{5.510247in}{1.101626in}}{\pgfqpoint{5.517088in}{1.098792in}}{\pgfqpoint{5.524221in}{1.098792in}}%
\pgfpathclose%
\pgfusepath{stroke,fill}%
\end{pgfscope}%
\begin{pgfscope}%
\pgfpathrectangle{\pgfqpoint{4.985294in}{0.500000in}}{\pgfqpoint{1.764706in}{1.700000in}}%
\pgfusepath{clip}%
\pgfsetbuttcap%
\pgfsetroundjoin%
\definecolor{currentfill}{rgb}{0.962765,0.606121,0.444717}%
\pgfsetfillcolor{currentfill}%
\pgfsetlinewidth{0.311001pt}%
\definecolor{currentstroke}{rgb}{1.000000,1.000000,1.000000}%
\pgfsetstrokecolor{currentstroke}%
\pgfsetdash{}{0pt}%
\pgfpathmoveto{\pgfqpoint{5.348250in}{1.544262in}}%
\pgfpathcurveto{\pgfqpoint{5.355383in}{1.544262in}}{\pgfqpoint{5.362225in}{1.547096in}}{\pgfqpoint{5.367269in}{1.552140in}}%
\pgfpathcurveto{\pgfqpoint{5.372312in}{1.557183in}}{\pgfqpoint{5.375146in}{1.564025in}}{\pgfqpoint{5.375146in}{1.571158in}}%
\pgfpathcurveto{\pgfqpoint{5.375146in}{1.578290in}}{\pgfqpoint{5.372312in}{1.585132in}}{\pgfqpoint{5.367269in}{1.590176in}}%
\pgfpathcurveto{\pgfqpoint{5.362225in}{1.595219in}}{\pgfqpoint{5.355383in}{1.598053in}}{\pgfqpoint{5.348250in}{1.598053in}}%
\pgfpathcurveto{\pgfqpoint{5.341118in}{1.598053in}}{\pgfqpoint{5.334276in}{1.595219in}}{\pgfqpoint{5.329232in}{1.590176in}}%
\pgfpathcurveto{\pgfqpoint{5.324189in}{1.585132in}}{\pgfqpoint{5.321355in}{1.578290in}}{\pgfqpoint{5.321355in}{1.571158in}}%
\pgfpathcurveto{\pgfqpoint{5.321355in}{1.564025in}}{\pgfqpoint{5.324189in}{1.557183in}}{\pgfqpoint{5.329232in}{1.552140in}}%
\pgfpathcurveto{\pgfqpoint{5.334276in}{1.547096in}}{\pgfqpoint{5.341118in}{1.544262in}}{\pgfqpoint{5.348250in}{1.544262in}}%
\pgfpathclose%
\pgfusepath{stroke,fill}%
\end{pgfscope}%
\begin{pgfscope}%
\pgfpathrectangle{\pgfqpoint{4.985294in}{0.500000in}}{\pgfqpoint{1.764706in}{1.700000in}}%
\pgfusepath{clip}%
\pgfsetbuttcap%
\pgfsetroundjoin%
\definecolor{currentfill}{rgb}{0.978376,0.897317,0.831308}%
\pgfsetfillcolor{currentfill}%
\pgfsetlinewidth{0.311001pt}%
\definecolor{currentstroke}{rgb}{1.000000,1.000000,1.000000}%
\pgfsetstrokecolor{currentstroke}%
\pgfsetdash{}{0pt}%
\pgfpathmoveto{\pgfqpoint{5.418518in}{1.192585in}}%
\pgfpathcurveto{\pgfqpoint{5.425651in}{1.192585in}}{\pgfqpoint{5.432493in}{1.195419in}}{\pgfqpoint{5.437536in}{1.200463in}}%
\pgfpathcurveto{\pgfqpoint{5.442580in}{1.205507in}}{\pgfqpoint{5.445414in}{1.212348in}}{\pgfqpoint{5.445414in}{1.219481in}}%
\pgfpathcurveto{\pgfqpoint{5.445414in}{1.226614in}}{\pgfqpoint{5.442580in}{1.233456in}}{\pgfqpoint{5.437536in}{1.238499in}}%
\pgfpathcurveto{\pgfqpoint{5.432493in}{1.243543in}}{\pgfqpoint{5.425651in}{1.246377in}}{\pgfqpoint{5.418518in}{1.246377in}}%
\pgfpathcurveto{\pgfqpoint{5.411385in}{1.246377in}}{\pgfqpoint{5.404544in}{1.243543in}}{\pgfqpoint{5.399500in}{1.238499in}}%
\pgfpathcurveto{\pgfqpoint{5.394456in}{1.233456in}}{\pgfqpoint{5.391622in}{1.226614in}}{\pgfqpoint{5.391622in}{1.219481in}}%
\pgfpathcurveto{\pgfqpoint{5.391622in}{1.212348in}}{\pgfqpoint{5.394456in}{1.205507in}}{\pgfqpoint{5.399500in}{1.200463in}}%
\pgfpathcurveto{\pgfqpoint{5.404544in}{1.195419in}}{\pgfqpoint{5.411385in}{1.192585in}}{\pgfqpoint{5.418518in}{1.192585in}}%
\pgfpathclose%
\pgfusepath{stroke,fill}%
\end{pgfscope}%
\begin{pgfscope}%
\pgfpathrectangle{\pgfqpoint{4.985294in}{0.500000in}}{\pgfqpoint{1.764706in}{1.700000in}}%
\pgfusepath{clip}%
\pgfsetbuttcap%
\pgfsetroundjoin%
\definecolor{currentfill}{rgb}{0.976961,0.885681,0.814303}%
\pgfsetfillcolor{currentfill}%
\pgfsetlinewidth{0.311001pt}%
\definecolor{currentstroke}{rgb}{1.000000,1.000000,1.000000}%
\pgfsetstrokecolor{currentstroke}%
\pgfsetdash{}{0pt}%
\pgfpathmoveto{\pgfqpoint{5.462067in}{1.178848in}}%
\pgfpathcurveto{\pgfqpoint{5.469200in}{1.178848in}}{\pgfqpoint{5.476041in}{1.181682in}}{\pgfqpoint{5.481085in}{1.186726in}}%
\pgfpathcurveto{\pgfqpoint{5.486129in}{1.191769in}}{\pgfqpoint{5.488962in}{1.198611in}}{\pgfqpoint{5.488962in}{1.205744in}}%
\pgfpathcurveto{\pgfqpoint{5.488962in}{1.212877in}}{\pgfqpoint{5.486129in}{1.219718in}}{\pgfqpoint{5.481085in}{1.224762in}}%
\pgfpathcurveto{\pgfqpoint{5.476041in}{1.229805in}}{\pgfqpoint{5.469200in}{1.232639in}}{\pgfqpoint{5.462067in}{1.232639in}}%
\pgfpathcurveto{\pgfqpoint{5.454934in}{1.232639in}}{\pgfqpoint{5.448092in}{1.229805in}}{\pgfqpoint{5.443049in}{1.224762in}}%
\pgfpathcurveto{\pgfqpoint{5.438005in}{1.219718in}}{\pgfqpoint{5.435171in}{1.212877in}}{\pgfqpoint{5.435171in}{1.205744in}}%
\pgfpathcurveto{\pgfqpoint{5.435171in}{1.198611in}}{\pgfqpoint{5.438005in}{1.191769in}}{\pgfqpoint{5.443049in}{1.186726in}}%
\pgfpathcurveto{\pgfqpoint{5.448092in}{1.181682in}}{\pgfqpoint{5.454934in}{1.178848in}}{\pgfqpoint{5.462067in}{1.178848in}}%
\pgfpathclose%
\pgfusepath{stroke,fill}%
\end{pgfscope}%
\begin{pgfscope}%
\pgfpathrectangle{\pgfqpoint{4.985294in}{0.500000in}}{\pgfqpoint{1.764706in}{1.700000in}}%
\pgfusepath{clip}%
\pgfsetbuttcap%
\pgfsetroundjoin%
\definecolor{currentfill}{rgb}{0.917171,0.267738,0.242941}%
\pgfsetfillcolor{currentfill}%
\pgfsetlinewidth{0.311001pt}%
\definecolor{currentstroke}{rgb}{1.000000,1.000000,1.000000}%
\pgfsetstrokecolor{currentstroke}%
\pgfsetdash{}{0pt}%
\pgfpathmoveto{\pgfqpoint{6.126303in}{1.313100in}}%
\pgfpathcurveto{\pgfqpoint{6.133436in}{1.313100in}}{\pgfqpoint{6.140277in}{1.315934in}}{\pgfqpoint{6.145321in}{1.320978in}}%
\pgfpathcurveto{\pgfqpoint{6.150365in}{1.326021in}}{\pgfqpoint{6.153198in}{1.332863in}}{\pgfqpoint{6.153198in}{1.339996in}}%
\pgfpathcurveto{\pgfqpoint{6.153198in}{1.347129in}}{\pgfqpoint{6.150365in}{1.353970in}}{\pgfqpoint{6.145321in}{1.359014in}}%
\pgfpathcurveto{\pgfqpoint{6.140277in}{1.364058in}}{\pgfqpoint{6.133436in}{1.366892in}}{\pgfqpoint{6.126303in}{1.366892in}}%
\pgfpathcurveto{\pgfqpoint{6.119170in}{1.366892in}}{\pgfqpoint{6.112328in}{1.364058in}}{\pgfqpoint{6.107285in}{1.359014in}}%
\pgfpathcurveto{\pgfqpoint{6.102241in}{1.353970in}}{\pgfqpoint{6.099407in}{1.347129in}}{\pgfqpoint{6.099407in}{1.339996in}}%
\pgfpathcurveto{\pgfqpoint{6.099407in}{1.332863in}}{\pgfqpoint{6.102241in}{1.326021in}}{\pgfqpoint{6.107285in}{1.320978in}}%
\pgfpathcurveto{\pgfqpoint{6.112328in}{1.315934in}}{\pgfqpoint{6.119170in}{1.313100in}}{\pgfqpoint{6.126303in}{1.313100in}}%
\pgfpathclose%
\pgfusepath{stroke,fill}%
\end{pgfscope}%
\begin{pgfscope}%
\pgfpathrectangle{\pgfqpoint{4.985294in}{0.500000in}}{\pgfqpoint{1.764706in}{1.700000in}}%
\pgfusepath{clip}%
\pgfsetbuttcap%
\pgfsetroundjoin%
\definecolor{currentfill}{rgb}{0.970255,0.815666,0.711203}%
\pgfsetfillcolor{currentfill}%
\pgfsetlinewidth{0.311001pt}%
\definecolor{currentstroke}{rgb}{1.000000,1.000000,1.000000}%
\pgfsetstrokecolor{currentstroke}%
\pgfsetdash{}{0pt}%
\pgfpathmoveto{\pgfqpoint{6.181445in}{1.597068in}}%
\pgfpathcurveto{\pgfqpoint{6.188577in}{1.597068in}}{\pgfqpoint{6.195419in}{1.599902in}}{\pgfqpoint{6.200463in}{1.604945in}}%
\pgfpathcurveto{\pgfqpoint{6.205506in}{1.609989in}}{\pgfqpoint{6.208340in}{1.616831in}}{\pgfqpoint{6.208340in}{1.623964in}}%
\pgfpathcurveto{\pgfqpoint{6.208340in}{1.631096in}}{\pgfqpoint{6.205506in}{1.637938in}}{\pgfqpoint{6.200463in}{1.642982in}}%
\pgfpathcurveto{\pgfqpoint{6.195419in}{1.648025in}}{\pgfqpoint{6.188577in}{1.650859in}}{\pgfqpoint{6.181445in}{1.650859in}}%
\pgfpathcurveto{\pgfqpoint{6.174312in}{1.650859in}}{\pgfqpoint{6.167470in}{1.648025in}}{\pgfqpoint{6.162426in}{1.642982in}}%
\pgfpathcurveto{\pgfqpoint{6.157383in}{1.637938in}}{\pgfqpoint{6.154549in}{1.631096in}}{\pgfqpoint{6.154549in}{1.623964in}}%
\pgfpathcurveto{\pgfqpoint{6.154549in}{1.616831in}}{\pgfqpoint{6.157383in}{1.609989in}}{\pgfqpoint{6.162426in}{1.604945in}}%
\pgfpathcurveto{\pgfqpoint{6.167470in}{1.599902in}}{\pgfqpoint{6.174312in}{1.597068in}}{\pgfqpoint{6.181445in}{1.597068in}}%
\pgfpathclose%
\pgfusepath{stroke,fill}%
\end{pgfscope}%
\begin{pgfscope}%
\pgfpathrectangle{\pgfqpoint{4.985294in}{0.500000in}}{\pgfqpoint{1.764706in}{1.700000in}}%
\pgfusepath{clip}%
\pgfsetbuttcap%
\pgfsetroundjoin%
\definecolor{currentfill}{rgb}{0.964799,0.689101,0.537560}%
\pgfsetfillcolor{currentfill}%
\pgfsetlinewidth{0.311001pt}%
\definecolor{currentstroke}{rgb}{1.000000,1.000000,1.000000}%
\pgfsetstrokecolor{currentstroke}%
\pgfsetdash{}{0pt}%
\pgfpathmoveto{\pgfqpoint{6.413418in}{1.265412in}}%
\pgfpathcurveto{\pgfqpoint{6.420551in}{1.265412in}}{\pgfqpoint{6.427393in}{1.268246in}}{\pgfqpoint{6.432437in}{1.273290in}}%
\pgfpathcurveto{\pgfqpoint{6.437480in}{1.278333in}}{\pgfqpoint{6.440314in}{1.285175in}}{\pgfqpoint{6.440314in}{1.292308in}}%
\pgfpathcurveto{\pgfqpoint{6.440314in}{1.299441in}}{\pgfqpoint{6.437480in}{1.306282in}}{\pgfqpoint{6.432437in}{1.311326in}}%
\pgfpathcurveto{\pgfqpoint{6.427393in}{1.316370in}}{\pgfqpoint{6.420551in}{1.319203in}}{\pgfqpoint{6.413418in}{1.319203in}}%
\pgfpathcurveto{\pgfqpoint{6.406286in}{1.319203in}}{\pgfqpoint{6.399444in}{1.316370in}}{\pgfqpoint{6.394400in}{1.311326in}}%
\pgfpathcurveto{\pgfqpoint{6.389357in}{1.306282in}}{\pgfqpoint{6.386523in}{1.299441in}}{\pgfqpoint{6.386523in}{1.292308in}}%
\pgfpathcurveto{\pgfqpoint{6.386523in}{1.285175in}}{\pgfqpoint{6.389357in}{1.278333in}}{\pgfqpoint{6.394400in}{1.273290in}}%
\pgfpathcurveto{\pgfqpoint{6.399444in}{1.268246in}}{\pgfqpoint{6.406286in}{1.265412in}}{\pgfqpoint{6.413418in}{1.265412in}}%
\pgfpathclose%
\pgfusepath{stroke,fill}%
\end{pgfscope}%
\begin{pgfscope}%
\pgfpathrectangle{\pgfqpoint{4.985294in}{0.500000in}}{\pgfqpoint{1.764706in}{1.700000in}}%
\pgfusepath{clip}%
\pgfsetbuttcap%
\pgfsetroundjoin%
\definecolor{currentfill}{rgb}{0.930781,0.313740,0.244688}%
\pgfsetfillcolor{currentfill}%
\pgfsetlinewidth{0.311001pt}%
\definecolor{currentstroke}{rgb}{1.000000,1.000000,1.000000}%
\pgfsetstrokecolor{currentstroke}%
\pgfsetdash{}{0pt}%
\pgfpathmoveto{\pgfqpoint{6.004165in}{1.717217in}}%
\pgfpathcurveto{\pgfqpoint{6.011298in}{1.717217in}}{\pgfqpoint{6.018139in}{1.720051in}}{\pgfqpoint{6.023183in}{1.725094in}}%
\pgfpathcurveto{\pgfqpoint{6.028227in}{1.730138in}}{\pgfqpoint{6.031060in}{1.736980in}}{\pgfqpoint{6.031060in}{1.744113in}}%
\pgfpathcurveto{\pgfqpoint{6.031060in}{1.751245in}}{\pgfqpoint{6.028227in}{1.758087in}}{\pgfqpoint{6.023183in}{1.763131in}}%
\pgfpathcurveto{\pgfqpoint{6.018139in}{1.768174in}}{\pgfqpoint{6.011298in}{1.771008in}}{\pgfqpoint{6.004165in}{1.771008in}}%
\pgfpathcurveto{\pgfqpoint{5.997032in}{1.771008in}}{\pgfqpoint{5.990190in}{1.768174in}}{\pgfqpoint{5.985147in}{1.763131in}}%
\pgfpathcurveto{\pgfqpoint{5.980103in}{1.758087in}}{\pgfqpoint{5.977269in}{1.751245in}}{\pgfqpoint{5.977269in}{1.744113in}}%
\pgfpathcurveto{\pgfqpoint{5.977269in}{1.736980in}}{\pgfqpoint{5.980103in}{1.730138in}}{\pgfqpoint{5.985147in}{1.725094in}}%
\pgfpathcurveto{\pgfqpoint{5.990190in}{1.720051in}}{\pgfqpoint{5.997032in}{1.717217in}}{\pgfqpoint{6.004165in}{1.717217in}}%
\pgfpathclose%
\pgfusepath{stroke,fill}%
\end{pgfscope}%
\begin{pgfscope}%
\pgfpathrectangle{\pgfqpoint{4.985294in}{0.500000in}}{\pgfqpoint{1.764706in}{1.700000in}}%
\pgfusepath{clip}%
\pgfsetbuttcap%
\pgfsetroundjoin%
\definecolor{currentfill}{rgb}{0.942910,0.375495,0.263698}%
\pgfsetfillcolor{currentfill}%
\pgfsetlinewidth{0.311001pt}%
\definecolor{currentstroke}{rgb}{1.000000,1.000000,1.000000}%
\pgfsetstrokecolor{currentstroke}%
\pgfsetdash{}{0pt}%
\pgfpathmoveto{\pgfqpoint{5.564737in}{1.819242in}}%
\pgfpathcurveto{\pgfqpoint{5.571870in}{1.819242in}}{\pgfqpoint{5.578711in}{1.822076in}}{\pgfqpoint{5.583755in}{1.827119in}}%
\pgfpathcurveto{\pgfqpoint{5.588799in}{1.832163in}}{\pgfqpoint{5.591633in}{1.839005in}}{\pgfqpoint{5.591633in}{1.846137in}}%
\pgfpathcurveto{\pgfqpoint{5.591633in}{1.853270in}}{\pgfqpoint{5.588799in}{1.860112in}}{\pgfqpoint{5.583755in}{1.865156in}}%
\pgfpathcurveto{\pgfqpoint{5.578711in}{1.870199in}}{\pgfqpoint{5.571870in}{1.873033in}}{\pgfqpoint{5.564737in}{1.873033in}}%
\pgfpathcurveto{\pgfqpoint{5.557604in}{1.873033in}}{\pgfqpoint{5.550762in}{1.870199in}}{\pgfqpoint{5.545719in}{1.865156in}}%
\pgfpathcurveto{\pgfqpoint{5.540675in}{1.860112in}}{\pgfqpoint{5.537841in}{1.853270in}}{\pgfqpoint{5.537841in}{1.846137in}}%
\pgfpathcurveto{\pgfqpoint{5.537841in}{1.839005in}}{\pgfqpoint{5.540675in}{1.832163in}}{\pgfqpoint{5.545719in}{1.827119in}}%
\pgfpathcurveto{\pgfqpoint{5.550762in}{1.822076in}}{\pgfqpoint{5.557604in}{1.819242in}}{\pgfqpoint{5.564737in}{1.819242in}}%
\pgfpathclose%
\pgfusepath{stroke,fill}%
\end{pgfscope}%
\begin{pgfscope}%
\pgfpathrectangle{\pgfqpoint{4.985294in}{0.500000in}}{\pgfqpoint{1.764706in}{1.700000in}}%
\pgfusepath{clip}%
\pgfsetbuttcap%
\pgfsetroundjoin%
\definecolor{currentfill}{rgb}{0.967092,0.768560,0.642079}%
\pgfsetfillcolor{currentfill}%
\pgfsetlinewidth{0.311001pt}%
\definecolor{currentstroke}{rgb}{1.000000,1.000000,1.000000}%
\pgfsetstrokecolor{currentstroke}%
\pgfsetdash{}{0pt}%
\pgfpathmoveto{\pgfqpoint{6.267249in}{0.988362in}}%
\pgfpathcurveto{\pgfqpoint{6.274381in}{0.988362in}}{\pgfqpoint{6.281223in}{0.991195in}}{\pgfqpoint{6.286267in}{0.996239in}}%
\pgfpathcurveto{\pgfqpoint{6.291310in}{1.001283in}}{\pgfqpoint{6.294144in}{1.008124in}}{\pgfqpoint{6.294144in}{1.015257in}}%
\pgfpathcurveto{\pgfqpoint{6.294144in}{1.022390in}}{\pgfqpoint{6.291310in}{1.029232in}}{\pgfqpoint{6.286267in}{1.034275in}}%
\pgfpathcurveto{\pgfqpoint{6.281223in}{1.039319in}}{\pgfqpoint{6.274381in}{1.042153in}}{\pgfqpoint{6.267249in}{1.042153in}}%
\pgfpathcurveto{\pgfqpoint{6.260116in}{1.042153in}}{\pgfqpoint{6.253274in}{1.039319in}}{\pgfqpoint{6.248230in}{1.034275in}}%
\pgfpathcurveto{\pgfqpoint{6.243187in}{1.029232in}}{\pgfqpoint{6.240353in}{1.022390in}}{\pgfqpoint{6.240353in}{1.015257in}}%
\pgfpathcurveto{\pgfqpoint{6.240353in}{1.008124in}}{\pgfqpoint{6.243187in}{1.001283in}}{\pgfqpoint{6.248230in}{0.996239in}}%
\pgfpathcurveto{\pgfqpoint{6.253274in}{0.991195in}}{\pgfqpoint{6.260116in}{0.988362in}}{\pgfqpoint{6.267249in}{0.988362in}}%
\pgfpathclose%
\pgfusepath{stroke,fill}%
\end{pgfscope}%
\begin{pgfscope}%
\pgfpathrectangle{\pgfqpoint{4.985294in}{0.500000in}}{\pgfqpoint{1.764706in}{1.700000in}}%
\pgfusepath{clip}%
\pgfsetbuttcap%
\pgfsetroundjoin%
\definecolor{currentfill}{rgb}{0.974412,0.862387,0.780156}%
\pgfsetfillcolor{currentfill}%
\pgfsetlinewidth{0.311001pt}%
\definecolor{currentstroke}{rgb}{1.000000,1.000000,1.000000}%
\pgfsetstrokecolor{currentstroke}%
\pgfsetdash{}{0pt}%
\pgfpathmoveto{\pgfqpoint{6.228093in}{1.062034in}}%
\pgfpathcurveto{\pgfqpoint{6.235226in}{1.062034in}}{\pgfqpoint{6.242067in}{1.064868in}}{\pgfqpoint{6.247111in}{1.069912in}}%
\pgfpathcurveto{\pgfqpoint{6.252155in}{1.074956in}}{\pgfqpoint{6.254989in}{1.081797in}}{\pgfqpoint{6.254989in}{1.088930in}}%
\pgfpathcurveto{\pgfqpoint{6.254989in}{1.096063in}}{\pgfqpoint{6.252155in}{1.102905in}}{\pgfqpoint{6.247111in}{1.107948in}}%
\pgfpathcurveto{\pgfqpoint{6.242067in}{1.112992in}}{\pgfqpoint{6.235226in}{1.115826in}}{\pgfqpoint{6.228093in}{1.115826in}}%
\pgfpathcurveto{\pgfqpoint{6.220960in}{1.115826in}}{\pgfqpoint{6.214118in}{1.112992in}}{\pgfqpoint{6.209075in}{1.107948in}}%
\pgfpathcurveto{\pgfqpoint{6.204031in}{1.102905in}}{\pgfqpoint{6.201197in}{1.096063in}}{\pgfqpoint{6.201197in}{1.088930in}}%
\pgfpathcurveto{\pgfqpoint{6.201197in}{1.081797in}}{\pgfqpoint{6.204031in}{1.074956in}}{\pgfqpoint{6.209075in}{1.069912in}}%
\pgfpathcurveto{\pgfqpoint{6.214118in}{1.064868in}}{\pgfqpoint{6.220960in}{1.062034in}}{\pgfqpoint{6.228093in}{1.062034in}}%
\pgfpathclose%
\pgfusepath{stroke,fill}%
\end{pgfscope}%
\begin{pgfscope}%
\pgfpathrectangle{\pgfqpoint{4.985294in}{0.500000in}}{\pgfqpoint{1.764706in}{1.700000in}}%
\pgfusepath{clip}%
\pgfsetbuttcap%
\pgfsetroundjoin%
\definecolor{currentfill}{rgb}{0.966328,0.750560,0.616961}%
\pgfsetfillcolor{currentfill}%
\pgfsetlinewidth{0.311001pt}%
\definecolor{currentstroke}{rgb}{1.000000,1.000000,1.000000}%
\pgfsetstrokecolor{currentstroke}%
\pgfsetdash{}{0pt}%
\pgfpathmoveto{\pgfqpoint{5.548380in}{1.680020in}}%
\pgfpathcurveto{\pgfqpoint{5.555513in}{1.680020in}}{\pgfqpoint{5.562354in}{1.682854in}}{\pgfqpoint{5.567398in}{1.687897in}}%
\pgfpathcurveto{\pgfqpoint{5.572442in}{1.692941in}}{\pgfqpoint{5.575276in}{1.699783in}}{\pgfqpoint{5.575276in}{1.706915in}}%
\pgfpathcurveto{\pgfqpoint{5.575276in}{1.714048in}}{\pgfqpoint{5.572442in}{1.720890in}}{\pgfqpoint{5.567398in}{1.725934in}}%
\pgfpathcurveto{\pgfqpoint{5.562354in}{1.730977in}}{\pgfqpoint{5.555513in}{1.733811in}}{\pgfqpoint{5.548380in}{1.733811in}}%
\pgfpathcurveto{\pgfqpoint{5.541247in}{1.733811in}}{\pgfqpoint{5.534405in}{1.730977in}}{\pgfqpoint{5.529362in}{1.725934in}}%
\pgfpathcurveto{\pgfqpoint{5.524318in}{1.720890in}}{\pgfqpoint{5.521484in}{1.714048in}}{\pgfqpoint{5.521484in}{1.706915in}}%
\pgfpathcurveto{\pgfqpoint{5.521484in}{1.699783in}}{\pgfqpoint{5.524318in}{1.692941in}}{\pgfqpoint{5.529362in}{1.687897in}}%
\pgfpathcurveto{\pgfqpoint{5.534405in}{1.682854in}}{\pgfqpoint{5.541247in}{1.680020in}}{\pgfqpoint{5.548380in}{1.680020in}}%
\pgfpathclose%
\pgfusepath{stroke,fill}%
\end{pgfscope}%
\begin{pgfscope}%
\pgfpathrectangle{\pgfqpoint{4.985294in}{0.500000in}}{\pgfqpoint{1.764706in}{1.700000in}}%
\pgfusepath{clip}%
\pgfsetbuttcap%
\pgfsetroundjoin%
\definecolor{currentfill}{rgb}{0.968105,0.786346,0.667739}%
\pgfsetfillcolor{currentfill}%
\pgfsetlinewidth{0.311001pt}%
\definecolor{currentstroke}{rgb}{1.000000,1.000000,1.000000}%
\pgfsetstrokecolor{currentstroke}%
\pgfsetdash{}{0pt}%
\pgfpathmoveto{\pgfqpoint{6.201612in}{1.192552in}}%
\pgfpathcurveto{\pgfqpoint{6.208745in}{1.192552in}}{\pgfqpoint{6.215586in}{1.195386in}}{\pgfqpoint{6.220630in}{1.200430in}}%
\pgfpathcurveto{\pgfqpoint{6.225674in}{1.205474in}}{\pgfqpoint{6.228508in}{1.212315in}}{\pgfqpoint{6.228508in}{1.219448in}}%
\pgfpathcurveto{\pgfqpoint{6.228508in}{1.226581in}}{\pgfqpoint{6.225674in}{1.233423in}}{\pgfqpoint{6.220630in}{1.238466in}}%
\pgfpathcurveto{\pgfqpoint{6.215586in}{1.243510in}}{\pgfqpoint{6.208745in}{1.246344in}}{\pgfqpoint{6.201612in}{1.246344in}}%
\pgfpathcurveto{\pgfqpoint{6.194479in}{1.246344in}}{\pgfqpoint{6.187638in}{1.243510in}}{\pgfqpoint{6.182594in}{1.238466in}}%
\pgfpathcurveto{\pgfqpoint{6.177550in}{1.233423in}}{\pgfqpoint{6.174716in}{1.226581in}}{\pgfqpoint{6.174716in}{1.219448in}}%
\pgfpathcurveto{\pgfqpoint{6.174716in}{1.212315in}}{\pgfqpoint{6.177550in}{1.205474in}}{\pgfqpoint{6.182594in}{1.200430in}}%
\pgfpathcurveto{\pgfqpoint{6.187638in}{1.195386in}}{\pgfqpoint{6.194479in}{1.192552in}}{\pgfqpoint{6.201612in}{1.192552in}}%
\pgfpathclose%
\pgfusepath{stroke,fill}%
\end{pgfscope}%
\begin{pgfscope}%
\pgfpathrectangle{\pgfqpoint{4.985294in}{0.500000in}}{\pgfqpoint{1.764706in}{1.700000in}}%
\pgfusepath{clip}%
\pgfsetbuttcap%
\pgfsetroundjoin%
\definecolor{currentfill}{rgb}{0.972726,0.844889,0.754401}%
\pgfsetfillcolor{currentfill}%
\pgfsetlinewidth{0.311001pt}%
\definecolor{currentstroke}{rgb}{1.000000,1.000000,1.000000}%
\pgfsetstrokecolor{currentstroke}%
\pgfsetdash{}{0pt}%
\pgfpathmoveto{\pgfqpoint{5.368706in}{1.402582in}}%
\pgfpathcurveto{\pgfqpoint{5.375838in}{1.402582in}}{\pgfqpoint{5.382680in}{1.405416in}}{\pgfqpoint{5.387724in}{1.410460in}}%
\pgfpathcurveto{\pgfqpoint{5.392767in}{1.415503in}}{\pgfqpoint{5.395601in}{1.422345in}}{\pgfqpoint{5.395601in}{1.429478in}}%
\pgfpathcurveto{\pgfqpoint{5.395601in}{1.436611in}}{\pgfqpoint{5.392767in}{1.443452in}}{\pgfqpoint{5.387724in}{1.448496in}}%
\pgfpathcurveto{\pgfqpoint{5.382680in}{1.453540in}}{\pgfqpoint{5.375838in}{1.456374in}}{\pgfqpoint{5.368706in}{1.456374in}}%
\pgfpathcurveto{\pgfqpoint{5.361573in}{1.456374in}}{\pgfqpoint{5.354731in}{1.453540in}}{\pgfqpoint{5.349687in}{1.448496in}}%
\pgfpathcurveto{\pgfqpoint{5.344644in}{1.443452in}}{\pgfqpoint{5.341810in}{1.436611in}}{\pgfqpoint{5.341810in}{1.429478in}}%
\pgfpathcurveto{\pgfqpoint{5.341810in}{1.422345in}}{\pgfqpoint{5.344644in}{1.415503in}}{\pgfqpoint{5.349687in}{1.410460in}}%
\pgfpathcurveto{\pgfqpoint{5.354731in}{1.405416in}}{\pgfqpoint{5.361573in}{1.402582in}}{\pgfqpoint{5.368706in}{1.402582in}}%
\pgfpathclose%
\pgfusepath{stroke,fill}%
\end{pgfscope}%
\begin{pgfscope}%
\pgfpathrectangle{\pgfqpoint{4.985294in}{0.500000in}}{\pgfqpoint{1.764706in}{1.700000in}}%
\pgfusepath{clip}%
\pgfsetbuttcap%
\pgfsetroundjoin%
\definecolor{currentfill}{rgb}{0.818205,0.120806,0.299261}%
\pgfsetfillcolor{currentfill}%
\pgfsetlinewidth{0.311001pt}%
\definecolor{currentstroke}{rgb}{1.000000,1.000000,1.000000}%
\pgfsetstrokecolor{currentstroke}%
\pgfsetdash{}{0pt}%
\pgfpathmoveto{\pgfqpoint{6.066855in}{1.128483in}}%
\pgfpathcurveto{\pgfqpoint{6.073988in}{1.128483in}}{\pgfqpoint{6.080829in}{1.131317in}}{\pgfqpoint{6.085873in}{1.136360in}}%
\pgfpathcurveto{\pgfqpoint{6.090917in}{1.141404in}}{\pgfqpoint{6.093750in}{1.148246in}}{\pgfqpoint{6.093750in}{1.155378in}}%
\pgfpathcurveto{\pgfqpoint{6.093750in}{1.162511in}}{\pgfqpoint{6.090917in}{1.169353in}}{\pgfqpoint{6.085873in}{1.174397in}}%
\pgfpathcurveto{\pgfqpoint{6.080829in}{1.179440in}}{\pgfqpoint{6.073988in}{1.182274in}}{\pgfqpoint{6.066855in}{1.182274in}}%
\pgfpathcurveto{\pgfqpoint{6.059722in}{1.182274in}}{\pgfqpoint{6.052880in}{1.179440in}}{\pgfqpoint{6.047837in}{1.174397in}}%
\pgfpathcurveto{\pgfqpoint{6.042793in}{1.169353in}}{\pgfqpoint{6.039959in}{1.162511in}}{\pgfqpoint{6.039959in}{1.155378in}}%
\pgfpathcurveto{\pgfqpoint{6.039959in}{1.148246in}}{\pgfqpoint{6.042793in}{1.141404in}}{\pgfqpoint{6.047837in}{1.136360in}}%
\pgfpathcurveto{\pgfqpoint{6.052880in}{1.131317in}}{\pgfqpoint{6.059722in}{1.128483in}}{\pgfqpoint{6.066855in}{1.128483in}}%
\pgfpathclose%
\pgfusepath{stroke,fill}%
\end{pgfscope}%
\begin{pgfscope}%
\pgfpathrectangle{\pgfqpoint{4.985294in}{0.500000in}}{\pgfqpoint{1.764706in}{1.700000in}}%
\pgfusepath{clip}%
\pgfsetbuttcap%
\pgfsetroundjoin%
\definecolor{currentfill}{rgb}{0.970255,0.815666,0.711203}%
\pgfsetfillcolor{currentfill}%
\pgfsetlinewidth{0.311001pt}%
\definecolor{currentstroke}{rgb}{1.000000,1.000000,1.000000}%
\pgfsetstrokecolor{currentstroke}%
\pgfsetdash{}{0pt}%
\pgfpathmoveto{\pgfqpoint{6.186128in}{1.055414in}}%
\pgfpathcurveto{\pgfqpoint{6.193261in}{1.055414in}}{\pgfqpoint{6.200103in}{1.058248in}}{\pgfqpoint{6.205146in}{1.063292in}}%
\pgfpathcurveto{\pgfqpoint{6.210190in}{1.068335in}}{\pgfqpoint{6.213024in}{1.075177in}}{\pgfqpoint{6.213024in}{1.082310in}}%
\pgfpathcurveto{\pgfqpoint{6.213024in}{1.089443in}}{\pgfqpoint{6.210190in}{1.096284in}}{\pgfqpoint{6.205146in}{1.101328in}}%
\pgfpathcurveto{\pgfqpoint{6.200103in}{1.106372in}}{\pgfqpoint{6.193261in}{1.109206in}}{\pgfqpoint{6.186128in}{1.109206in}}%
\pgfpathcurveto{\pgfqpoint{6.178996in}{1.109206in}}{\pgfqpoint{6.172154in}{1.106372in}}{\pgfqpoint{6.167110in}{1.101328in}}%
\pgfpathcurveto{\pgfqpoint{6.162067in}{1.096284in}}{\pgfqpoint{6.159233in}{1.089443in}}{\pgfqpoint{6.159233in}{1.082310in}}%
\pgfpathcurveto{\pgfqpoint{6.159233in}{1.075177in}}{\pgfqpoint{6.162067in}{1.068335in}}{\pgfqpoint{6.167110in}{1.063292in}}%
\pgfpathcurveto{\pgfqpoint{6.172154in}{1.058248in}}{\pgfqpoint{6.178996in}{1.055414in}}{\pgfqpoint{6.186128in}{1.055414in}}%
\pgfpathclose%
\pgfusepath{stroke,fill}%
\end{pgfscope}%
\begin{pgfscope}%
\pgfpathrectangle{\pgfqpoint{4.985294in}{0.500000in}}{\pgfqpoint{1.764706in}{1.700000in}}%
\pgfusepath{clip}%
\pgfsetbuttcap%
\pgfsetroundjoin%
\definecolor{currentfill}{rgb}{0.966812,0.762584,0.633643}%
\pgfsetfillcolor{currentfill}%
\pgfsetlinewidth{0.311001pt}%
\definecolor{currentstroke}{rgb}{1.000000,1.000000,1.000000}%
\pgfsetstrokecolor{currentstroke}%
\pgfsetdash{}{0pt}%
\pgfpathmoveto{\pgfqpoint{6.161061in}{1.045533in}}%
\pgfpathcurveto{\pgfqpoint{6.168194in}{1.045533in}}{\pgfqpoint{6.175036in}{1.048367in}}{\pgfqpoint{6.180079in}{1.053411in}}%
\pgfpathcurveto{\pgfqpoint{6.185123in}{1.058455in}}{\pgfqpoint{6.187957in}{1.065296in}}{\pgfqpoint{6.187957in}{1.072429in}}%
\pgfpathcurveto{\pgfqpoint{6.187957in}{1.079562in}}{\pgfqpoint{6.185123in}{1.086403in}}{\pgfqpoint{6.180079in}{1.091447in}}%
\pgfpathcurveto{\pgfqpoint{6.175036in}{1.096491in}}{\pgfqpoint{6.168194in}{1.099325in}}{\pgfqpoint{6.161061in}{1.099325in}}%
\pgfpathcurveto{\pgfqpoint{6.153928in}{1.099325in}}{\pgfqpoint{6.147087in}{1.096491in}}{\pgfqpoint{6.142043in}{1.091447in}}%
\pgfpathcurveto{\pgfqpoint{6.136999in}{1.086403in}}{\pgfqpoint{6.134165in}{1.079562in}}{\pgfqpoint{6.134165in}{1.072429in}}%
\pgfpathcurveto{\pgfqpoint{6.134165in}{1.065296in}}{\pgfqpoint{6.136999in}{1.058455in}}{\pgfqpoint{6.142043in}{1.053411in}}%
\pgfpathcurveto{\pgfqpoint{6.147087in}{1.048367in}}{\pgfqpoint{6.153928in}{1.045533in}}{\pgfqpoint{6.161061in}{1.045533in}}%
\pgfpathclose%
\pgfusepath{stroke,fill}%
\end{pgfscope}%
\begin{pgfscope}%
\pgfpathrectangle{\pgfqpoint{4.985294in}{0.500000in}}{\pgfqpoint{1.764706in}{1.700000in}}%
\pgfusepath{clip}%
\pgfsetbuttcap%
\pgfsetroundjoin%
\definecolor{currentfill}{rgb}{0.967092,0.768560,0.642079}%
\pgfsetfillcolor{currentfill}%
\pgfsetlinewidth{0.311001pt}%
\definecolor{currentstroke}{rgb}{1.000000,1.000000,1.000000}%
\pgfsetstrokecolor{currentstroke}%
\pgfsetdash{}{0pt}%
\pgfpathmoveto{\pgfqpoint{5.543438in}{1.094411in}}%
\pgfpathcurveto{\pgfqpoint{5.550570in}{1.094411in}}{\pgfqpoint{5.557412in}{1.097244in}}{\pgfqpoint{5.562456in}{1.102288in}}%
\pgfpathcurveto{\pgfqpoint{5.567499in}{1.107332in}}{\pgfqpoint{5.570333in}{1.114173in}}{\pgfqpoint{5.570333in}{1.121306in}}%
\pgfpathcurveto{\pgfqpoint{5.570333in}{1.128439in}}{\pgfqpoint{5.567499in}{1.135281in}}{\pgfqpoint{5.562456in}{1.140324in}}%
\pgfpathcurveto{\pgfqpoint{5.557412in}{1.145368in}}{\pgfqpoint{5.550570in}{1.148202in}}{\pgfqpoint{5.543438in}{1.148202in}}%
\pgfpathcurveto{\pgfqpoint{5.536305in}{1.148202in}}{\pgfqpoint{5.529463in}{1.145368in}}{\pgfqpoint{5.524419in}{1.140324in}}%
\pgfpathcurveto{\pgfqpoint{5.519376in}{1.135281in}}{\pgfqpoint{5.516542in}{1.128439in}}{\pgfqpoint{5.516542in}{1.121306in}}%
\pgfpathcurveto{\pgfqpoint{5.516542in}{1.114173in}}{\pgfqpoint{5.519376in}{1.107332in}}{\pgfqpoint{5.524419in}{1.102288in}}%
\pgfpathcurveto{\pgfqpoint{5.529463in}{1.097244in}}{\pgfqpoint{5.536305in}{1.094411in}}{\pgfqpoint{5.543438in}{1.094411in}}%
\pgfpathclose%
\pgfusepath{stroke,fill}%
\end{pgfscope}%
\begin{pgfscope}%
\pgfpathrectangle{\pgfqpoint{4.985294in}{0.500000in}}{\pgfqpoint{1.764706in}{1.700000in}}%
\pgfusepath{clip}%
\pgfsetbuttcap%
\pgfsetroundjoin%
\definecolor{currentfill}{rgb}{0.979124,0.903132,0.839793}%
\pgfsetfillcolor{currentfill}%
\pgfsetlinewidth{0.311001pt}%
\definecolor{currentstroke}{rgb}{1.000000,1.000000,1.000000}%
\pgfsetstrokecolor{currentstroke}%
\pgfsetdash{}{0pt}%
\pgfpathmoveto{\pgfqpoint{6.323361in}{1.456659in}}%
\pgfpathcurveto{\pgfqpoint{6.330494in}{1.456659in}}{\pgfqpoint{6.337336in}{1.459493in}}{\pgfqpoint{6.342379in}{1.464536in}}%
\pgfpathcurveto{\pgfqpoint{6.347423in}{1.469580in}}{\pgfqpoint{6.350257in}{1.476422in}}{\pgfqpoint{6.350257in}{1.483555in}}%
\pgfpathcurveto{\pgfqpoint{6.350257in}{1.490687in}}{\pgfqpoint{6.347423in}{1.497529in}}{\pgfqpoint{6.342379in}{1.502573in}}%
\pgfpathcurveto{\pgfqpoint{6.337336in}{1.507616in}}{\pgfqpoint{6.330494in}{1.510450in}}{\pgfqpoint{6.323361in}{1.510450in}}%
\pgfpathcurveto{\pgfqpoint{6.316228in}{1.510450in}}{\pgfqpoint{6.309387in}{1.507616in}}{\pgfqpoint{6.304343in}{1.502573in}}%
\pgfpathcurveto{\pgfqpoint{6.299299in}{1.497529in}}{\pgfqpoint{6.296465in}{1.490687in}}{\pgfqpoint{6.296465in}{1.483555in}}%
\pgfpathcurveto{\pgfqpoint{6.296465in}{1.476422in}}{\pgfqpoint{6.299299in}{1.469580in}}{\pgfqpoint{6.304343in}{1.464536in}}%
\pgfpathcurveto{\pgfqpoint{6.309387in}{1.459493in}}{\pgfqpoint{6.316228in}{1.456659in}}{\pgfqpoint{6.323361in}{1.456659in}}%
\pgfpathclose%
\pgfusepath{stroke,fill}%
\end{pgfscope}%
\begin{pgfscope}%
\pgfpathrectangle{\pgfqpoint{4.985294in}{0.500000in}}{\pgfqpoint{1.764706in}{1.700000in}}%
\pgfusepath{clip}%
\pgfsetbuttcap%
\pgfsetroundjoin%
\definecolor{currentfill}{rgb}{0.981377,0.920617,0.865369}%
\pgfsetfillcolor{currentfill}%
\pgfsetlinewidth{0.311001pt}%
\definecolor{currentstroke}{rgb}{1.000000,1.000000,1.000000}%
\pgfsetstrokecolor{currentstroke}%
\pgfsetdash{}{0pt}%
\pgfpathmoveto{\pgfqpoint{6.313848in}{1.310232in}}%
\pgfpathcurveto{\pgfqpoint{6.320981in}{1.310232in}}{\pgfqpoint{6.327823in}{1.313066in}}{\pgfqpoint{6.332866in}{1.318110in}}%
\pgfpathcurveto{\pgfqpoint{6.337910in}{1.323154in}}{\pgfqpoint{6.340744in}{1.329995in}}{\pgfqpoint{6.340744in}{1.337128in}}%
\pgfpathcurveto{\pgfqpoint{6.340744in}{1.344261in}}{\pgfqpoint{6.337910in}{1.351102in}}{\pgfqpoint{6.332866in}{1.356146in}}%
\pgfpathcurveto{\pgfqpoint{6.327823in}{1.361190in}}{\pgfqpoint{6.320981in}{1.364024in}}{\pgfqpoint{6.313848in}{1.364024in}}%
\pgfpathcurveto{\pgfqpoint{6.306715in}{1.364024in}}{\pgfqpoint{6.299874in}{1.361190in}}{\pgfqpoint{6.294830in}{1.356146in}}%
\pgfpathcurveto{\pgfqpoint{6.289786in}{1.351102in}}{\pgfqpoint{6.286952in}{1.344261in}}{\pgfqpoint{6.286952in}{1.337128in}}%
\pgfpathcurveto{\pgfqpoint{6.286952in}{1.329995in}}{\pgfqpoint{6.289786in}{1.323154in}}{\pgfqpoint{6.294830in}{1.318110in}}%
\pgfpathcurveto{\pgfqpoint{6.299874in}{1.313066in}}{\pgfqpoint{6.306715in}{1.310232in}}{\pgfqpoint{6.313848in}{1.310232in}}%
\pgfpathclose%
\pgfusepath{stroke,fill}%
\end{pgfscope}%
\begin{pgfscope}%
\pgfpathrectangle{\pgfqpoint{4.985294in}{0.500000in}}{\pgfqpoint{1.764706in}{1.700000in}}%
\pgfusepath{clip}%
\pgfsetbuttcap%
\pgfsetroundjoin%
\definecolor{currentfill}{rgb}{0.980678,0.914765,0.856766}%
\pgfsetfillcolor{currentfill}%
\pgfsetlinewidth{0.311001pt}%
\definecolor{currentstroke}{rgb}{1.000000,1.000000,1.000000}%
\pgfsetstrokecolor{currentstroke}%
\pgfsetdash{}{0pt}%
\pgfpathmoveto{\pgfqpoint{6.336181in}{1.347368in}}%
\pgfpathcurveto{\pgfqpoint{6.343314in}{1.347368in}}{\pgfqpoint{6.350156in}{1.350202in}}{\pgfqpoint{6.355199in}{1.355246in}}%
\pgfpathcurveto{\pgfqpoint{6.360243in}{1.360289in}}{\pgfqpoint{6.363077in}{1.367131in}}{\pgfqpoint{6.363077in}{1.374264in}}%
\pgfpathcurveto{\pgfqpoint{6.363077in}{1.381396in}}{\pgfqpoint{6.360243in}{1.388238in}}{\pgfqpoint{6.355199in}{1.393282in}}%
\pgfpathcurveto{\pgfqpoint{6.350156in}{1.398325in}}{\pgfqpoint{6.343314in}{1.401159in}}{\pgfqpoint{6.336181in}{1.401159in}}%
\pgfpathcurveto{\pgfqpoint{6.329048in}{1.401159in}}{\pgfqpoint{6.322207in}{1.398325in}}{\pgfqpoint{6.317163in}{1.393282in}}%
\pgfpathcurveto{\pgfqpoint{6.312119in}{1.388238in}}{\pgfqpoint{6.309285in}{1.381396in}}{\pgfqpoint{6.309285in}{1.374264in}}%
\pgfpathcurveto{\pgfqpoint{6.309285in}{1.367131in}}{\pgfqpoint{6.312119in}{1.360289in}}{\pgfqpoint{6.317163in}{1.355246in}}%
\pgfpathcurveto{\pgfqpoint{6.322207in}{1.350202in}}{\pgfqpoint{6.329048in}{1.347368in}}{\pgfqpoint{6.336181in}{1.347368in}}%
\pgfpathclose%
\pgfusepath{stroke,fill}%
\end{pgfscope}%
\begin{pgfscope}%
\pgfpathrectangle{\pgfqpoint{4.985294in}{0.500000in}}{\pgfqpoint{1.764706in}{1.700000in}}%
\pgfusepath{clip}%
\pgfsetbuttcap%
\pgfsetroundjoin%
\definecolor{currentfill}{rgb}{0.960421,0.553286,0.393191}%
\pgfsetfillcolor{currentfill}%
\pgfsetlinewidth{0.311001pt}%
\definecolor{currentstroke}{rgb}{1.000000,1.000000,1.000000}%
\pgfsetstrokecolor{currentstroke}%
\pgfsetdash{}{0pt}%
\pgfpathmoveto{\pgfqpoint{5.609565in}{0.862683in}}%
\pgfpathcurveto{\pgfqpoint{5.616698in}{0.862683in}}{\pgfqpoint{5.623539in}{0.865517in}}{\pgfqpoint{5.628583in}{0.870561in}}%
\pgfpathcurveto{\pgfqpoint{5.633627in}{0.875604in}}{\pgfqpoint{5.636460in}{0.882446in}}{\pgfqpoint{5.636460in}{0.889579in}}%
\pgfpathcurveto{\pgfqpoint{5.636460in}{0.896712in}}{\pgfqpoint{5.633627in}{0.903553in}}{\pgfqpoint{5.628583in}{0.908597in}}%
\pgfpathcurveto{\pgfqpoint{5.623539in}{0.913641in}}{\pgfqpoint{5.616698in}{0.916475in}}{\pgfqpoint{5.609565in}{0.916475in}}%
\pgfpathcurveto{\pgfqpoint{5.602432in}{0.916475in}}{\pgfqpoint{5.595590in}{0.913641in}}{\pgfqpoint{5.590547in}{0.908597in}}%
\pgfpathcurveto{\pgfqpoint{5.585503in}{0.903553in}}{\pgfqpoint{5.582669in}{0.896712in}}{\pgfqpoint{5.582669in}{0.889579in}}%
\pgfpathcurveto{\pgfqpoint{5.582669in}{0.882446in}}{\pgfqpoint{5.585503in}{0.875604in}}{\pgfqpoint{5.590547in}{0.870561in}}%
\pgfpathcurveto{\pgfqpoint{5.595590in}{0.865517in}}{\pgfqpoint{5.602432in}{0.862683in}}{\pgfqpoint{5.609565in}{0.862683in}}%
\pgfpathclose%
\pgfusepath{stroke,fill}%
\end{pgfscope}%
\begin{pgfscope}%
\pgfpathrectangle{\pgfqpoint{4.985294in}{0.500000in}}{\pgfqpoint{1.764706in}{1.700000in}}%
\pgfusepath{clip}%
\pgfsetbuttcap%
\pgfsetroundjoin%
\definecolor{currentfill}{rgb}{0.941676,0.367866,0.260395}%
\pgfsetfillcolor{currentfill}%
\pgfsetlinewidth{0.311001pt}%
\definecolor{currentstroke}{rgb}{1.000000,1.000000,1.000000}%
\pgfsetstrokecolor{currentstroke}%
\pgfsetdash{}{0pt}%
\pgfpathmoveto{\pgfqpoint{5.501853in}{0.825848in}}%
\pgfpathcurveto{\pgfqpoint{5.508986in}{0.825848in}}{\pgfqpoint{5.515828in}{0.828682in}}{\pgfqpoint{5.520871in}{0.833726in}}%
\pgfpathcurveto{\pgfqpoint{5.525915in}{0.838770in}}{\pgfqpoint{5.528749in}{0.845611in}}{\pgfqpoint{5.528749in}{0.852744in}}%
\pgfpathcurveto{\pgfqpoint{5.528749in}{0.859877in}}{\pgfqpoint{5.525915in}{0.866719in}}{\pgfqpoint{5.520871in}{0.871762in}}%
\pgfpathcurveto{\pgfqpoint{5.515828in}{0.876806in}}{\pgfqpoint{5.508986in}{0.879640in}}{\pgfqpoint{5.501853in}{0.879640in}}%
\pgfpathcurveto{\pgfqpoint{5.494721in}{0.879640in}}{\pgfqpoint{5.487879in}{0.876806in}}{\pgfqpoint{5.482835in}{0.871762in}}%
\pgfpathcurveto{\pgfqpoint{5.477792in}{0.866719in}}{\pgfqpoint{5.474958in}{0.859877in}}{\pgfqpoint{5.474958in}{0.852744in}}%
\pgfpathcurveto{\pgfqpoint{5.474958in}{0.845611in}}{\pgfqpoint{5.477792in}{0.838770in}}{\pgfqpoint{5.482835in}{0.833726in}}%
\pgfpathcurveto{\pgfqpoint{5.487879in}{0.828682in}}{\pgfqpoint{5.494721in}{0.825848in}}{\pgfqpoint{5.501853in}{0.825848in}}%
\pgfpathclose%
\pgfusepath{stroke,fill}%
\end{pgfscope}%
\begin{pgfscope}%
\pgfpathrectangle{\pgfqpoint{4.985294in}{0.500000in}}{\pgfqpoint{1.764706in}{1.700000in}}%
\pgfusepath{clip}%
\pgfsetbuttcap%
\pgfsetroundjoin%
\definecolor{currentfill}{rgb}{0.981377,0.920617,0.865369}%
\pgfsetfillcolor{currentfill}%
\pgfsetlinewidth{0.311001pt}%
\definecolor{currentstroke}{rgb}{1.000000,1.000000,1.000000}%
\pgfsetstrokecolor{currentstroke}%
\pgfsetdash{}{0pt}%
\pgfpathmoveto{\pgfqpoint{6.304780in}{1.422974in}}%
\pgfpathcurveto{\pgfqpoint{6.311913in}{1.422974in}}{\pgfqpoint{6.318755in}{1.425808in}}{\pgfqpoint{6.323798in}{1.430852in}}%
\pgfpathcurveto{\pgfqpoint{6.328842in}{1.435896in}}{\pgfqpoint{6.331676in}{1.442737in}}{\pgfqpoint{6.331676in}{1.449870in}}%
\pgfpathcurveto{\pgfqpoint{6.331676in}{1.457003in}}{\pgfqpoint{6.328842in}{1.463845in}}{\pgfqpoint{6.323798in}{1.468888in}}%
\pgfpathcurveto{\pgfqpoint{6.318755in}{1.473932in}}{\pgfqpoint{6.311913in}{1.476766in}}{\pgfqpoint{6.304780in}{1.476766in}}%
\pgfpathcurveto{\pgfqpoint{6.297647in}{1.476766in}}{\pgfqpoint{6.290806in}{1.473932in}}{\pgfqpoint{6.285762in}{1.468888in}}%
\pgfpathcurveto{\pgfqpoint{6.280718in}{1.463845in}}{\pgfqpoint{6.277884in}{1.457003in}}{\pgfqpoint{6.277884in}{1.449870in}}%
\pgfpathcurveto{\pgfqpoint{6.277884in}{1.442737in}}{\pgfqpoint{6.280718in}{1.435896in}}{\pgfqpoint{6.285762in}{1.430852in}}%
\pgfpathcurveto{\pgfqpoint{6.290806in}{1.425808in}}{\pgfqpoint{6.297647in}{1.422974in}}{\pgfqpoint{6.304780in}{1.422974in}}%
\pgfpathclose%
\pgfusepath{stroke,fill}%
\end{pgfscope}%
\begin{pgfscope}%
\pgfpathrectangle{\pgfqpoint{4.985294in}{0.500000in}}{\pgfqpoint{1.764706in}{1.700000in}}%
\pgfusepath{clip}%
\pgfsetbuttcap%
\pgfsetroundjoin%
\definecolor{currentfill}{rgb}{0.969803,0.809811,0.702523}%
\pgfsetfillcolor{currentfill}%
\pgfsetlinewidth{0.311001pt}%
\definecolor{currentstroke}{rgb}{1.000000,1.000000,1.000000}%
\pgfsetstrokecolor{currentstroke}%
\pgfsetdash{}{0pt}%
\pgfpathmoveto{\pgfqpoint{6.381607in}{1.373929in}}%
\pgfpathcurveto{\pgfqpoint{6.388740in}{1.373929in}}{\pgfqpoint{6.395582in}{1.376762in}}{\pgfqpoint{6.400625in}{1.381806in}}%
\pgfpathcurveto{\pgfqpoint{6.405669in}{1.386850in}}{\pgfqpoint{6.408503in}{1.393691in}}{\pgfqpoint{6.408503in}{1.400824in}}%
\pgfpathcurveto{\pgfqpoint{6.408503in}{1.407957in}}{\pgfqpoint{6.405669in}{1.414799in}}{\pgfqpoint{6.400625in}{1.419842in}}%
\pgfpathcurveto{\pgfqpoint{6.395582in}{1.424886in}}{\pgfqpoint{6.388740in}{1.427720in}}{\pgfqpoint{6.381607in}{1.427720in}}%
\pgfpathcurveto{\pgfqpoint{6.374474in}{1.427720in}}{\pgfqpoint{6.367633in}{1.424886in}}{\pgfqpoint{6.362589in}{1.419842in}}%
\pgfpathcurveto{\pgfqpoint{6.357545in}{1.414799in}}{\pgfqpoint{6.354711in}{1.407957in}}{\pgfqpoint{6.354711in}{1.400824in}}%
\pgfpathcurveto{\pgfqpoint{6.354711in}{1.393691in}}{\pgfqpoint{6.357545in}{1.386850in}}{\pgfqpoint{6.362589in}{1.381806in}}%
\pgfpathcurveto{\pgfqpoint{6.367633in}{1.376762in}}{\pgfqpoint{6.374474in}{1.373929in}}{\pgfqpoint{6.381607in}{1.373929in}}%
\pgfpathclose%
\pgfusepath{stroke,fill}%
\end{pgfscope}%
\begin{pgfscope}%
\pgfpathrectangle{\pgfqpoint{4.985294in}{0.500000in}}{\pgfqpoint{1.764706in}{1.700000in}}%
\pgfusepath{clip}%
\pgfsetbuttcap%
\pgfsetroundjoin%
\definecolor{currentfill}{rgb}{0.964920,0.695342,0.545192}%
\pgfsetfillcolor{currentfill}%
\pgfsetlinewidth{0.311001pt}%
\definecolor{currentstroke}{rgb}{1.000000,1.000000,1.000000}%
\pgfsetstrokecolor{currentstroke}%
\pgfsetdash{}{0pt}%
\pgfpathmoveto{\pgfqpoint{6.146219in}{0.916482in}}%
\pgfpathcurveto{\pgfqpoint{6.153352in}{0.916482in}}{\pgfqpoint{6.160193in}{0.919316in}}{\pgfqpoint{6.165237in}{0.924360in}}%
\pgfpathcurveto{\pgfqpoint{6.170281in}{0.929404in}}{\pgfqpoint{6.173114in}{0.936245in}}{\pgfqpoint{6.173114in}{0.943378in}}%
\pgfpathcurveto{\pgfqpoint{6.173114in}{0.950511in}}{\pgfqpoint{6.170281in}{0.957353in}}{\pgfqpoint{6.165237in}{0.962396in}}%
\pgfpathcurveto{\pgfqpoint{6.160193in}{0.967440in}}{\pgfqpoint{6.153352in}{0.970274in}}{\pgfqpoint{6.146219in}{0.970274in}}%
\pgfpathcurveto{\pgfqpoint{6.139086in}{0.970274in}}{\pgfqpoint{6.132244in}{0.967440in}}{\pgfqpoint{6.127201in}{0.962396in}}%
\pgfpathcurveto{\pgfqpoint{6.122157in}{0.957353in}}{\pgfqpoint{6.119323in}{0.950511in}}{\pgfqpoint{6.119323in}{0.943378in}}%
\pgfpathcurveto{\pgfqpoint{6.119323in}{0.936245in}}{\pgfqpoint{6.122157in}{0.929404in}}{\pgfqpoint{6.127201in}{0.924360in}}%
\pgfpathcurveto{\pgfqpoint{6.132244in}{0.919316in}}{\pgfqpoint{6.139086in}{0.916482in}}{\pgfqpoint{6.146219in}{0.916482in}}%
\pgfpathclose%
\pgfusepath{stroke,fill}%
\end{pgfscope}%
\begin{pgfscope}%
\pgfpathrectangle{\pgfqpoint{4.985294in}{0.500000in}}{\pgfqpoint{1.764706in}{1.700000in}}%
\pgfusepath{clip}%
\pgfsetbuttcap%
\pgfsetroundjoin%
\definecolor{currentfill}{rgb}{0.978376,0.897317,0.831308}%
\pgfsetfillcolor{currentfill}%
\pgfsetlinewidth{0.311001pt}%
\definecolor{currentstroke}{rgb}{1.000000,1.000000,1.000000}%
\pgfsetstrokecolor{currentstroke}%
\pgfsetdash{}{0pt}%
\pgfpathmoveto{\pgfqpoint{6.280636in}{1.578205in}}%
\pgfpathcurveto{\pgfqpoint{6.287769in}{1.578205in}}{\pgfqpoint{6.294611in}{1.581039in}}{\pgfqpoint{6.299654in}{1.586083in}}%
\pgfpathcurveto{\pgfqpoint{6.304698in}{1.591127in}}{\pgfqpoint{6.307532in}{1.597968in}}{\pgfqpoint{6.307532in}{1.605101in}}%
\pgfpathcurveto{\pgfqpoint{6.307532in}{1.612234in}}{\pgfqpoint{6.304698in}{1.619076in}}{\pgfqpoint{6.299654in}{1.624119in}}%
\pgfpathcurveto{\pgfqpoint{6.294611in}{1.629163in}}{\pgfqpoint{6.287769in}{1.631997in}}{\pgfqpoint{6.280636in}{1.631997in}}%
\pgfpathcurveto{\pgfqpoint{6.273503in}{1.631997in}}{\pgfqpoint{6.266662in}{1.629163in}}{\pgfqpoint{6.261618in}{1.624119in}}%
\pgfpathcurveto{\pgfqpoint{6.256574in}{1.619076in}}{\pgfqpoint{6.253741in}{1.612234in}}{\pgfqpoint{6.253741in}{1.605101in}}%
\pgfpathcurveto{\pgfqpoint{6.253741in}{1.597968in}}{\pgfqpoint{6.256574in}{1.591127in}}{\pgfqpoint{6.261618in}{1.586083in}}%
\pgfpathcurveto{\pgfqpoint{6.266662in}{1.581039in}}{\pgfqpoint{6.273503in}{1.578205in}}{\pgfqpoint{6.280636in}{1.578205in}}%
\pgfpathclose%
\pgfusepath{stroke,fill}%
\end{pgfscope}%
\begin{pgfscope}%
\pgfpathrectangle{\pgfqpoint{4.985294in}{0.500000in}}{\pgfqpoint{1.764706in}{1.700000in}}%
\pgfusepath{clip}%
\pgfsetbuttcap%
\pgfsetroundjoin%
\definecolor{currentfill}{rgb}{0.954476,0.470822,0.323110}%
\pgfsetfillcolor{currentfill}%
\pgfsetlinewidth{0.311001pt}%
\definecolor{currentstroke}{rgb}{1.000000,1.000000,1.000000}%
\pgfsetstrokecolor{currentstroke}%
\pgfsetdash{}{0pt}%
\pgfpathmoveto{\pgfqpoint{5.676818in}{0.887075in}}%
\pgfpathcurveto{\pgfqpoint{5.683951in}{0.887075in}}{\pgfqpoint{5.690793in}{0.889909in}}{\pgfqpoint{5.695836in}{0.894953in}}%
\pgfpathcurveto{\pgfqpoint{5.700880in}{0.899996in}}{\pgfqpoint{5.703714in}{0.906838in}}{\pgfqpoint{5.703714in}{0.913971in}}%
\pgfpathcurveto{\pgfqpoint{5.703714in}{0.921104in}}{\pgfqpoint{5.700880in}{0.927945in}}{\pgfqpoint{5.695836in}{0.932989in}}%
\pgfpathcurveto{\pgfqpoint{5.690793in}{0.938033in}}{\pgfqpoint{5.683951in}{0.940867in}}{\pgfqpoint{5.676818in}{0.940867in}}%
\pgfpathcurveto{\pgfqpoint{5.669685in}{0.940867in}}{\pgfqpoint{5.662844in}{0.938033in}}{\pgfqpoint{5.657800in}{0.932989in}}%
\pgfpathcurveto{\pgfqpoint{5.652756in}{0.927945in}}{\pgfqpoint{5.649923in}{0.921104in}}{\pgfqpoint{5.649923in}{0.913971in}}%
\pgfpathcurveto{\pgfqpoint{5.649923in}{0.906838in}}{\pgfqpoint{5.652756in}{0.899996in}}{\pgfqpoint{5.657800in}{0.894953in}}%
\pgfpathcurveto{\pgfqpoint{5.662844in}{0.889909in}}{\pgfqpoint{5.669685in}{0.887075in}}{\pgfqpoint{5.676818in}{0.887075in}}%
\pgfpathclose%
\pgfusepath{stroke,fill}%
\end{pgfscope}%
\begin{pgfscope}%
\pgfpathrectangle{\pgfqpoint{4.985294in}{0.500000in}}{\pgfqpoint{1.764706in}{1.700000in}}%
\pgfusepath{clip}%
\pgfsetbuttcap%
\pgfsetroundjoin%
\definecolor{currentfill}{rgb}{0.979124,0.903132,0.839793}%
\pgfsetfillcolor{currentfill}%
\pgfsetlinewidth{0.311001pt}%
\definecolor{currentstroke}{rgb}{1.000000,1.000000,1.000000}%
\pgfsetstrokecolor{currentstroke}%
\pgfsetdash{}{0pt}%
\pgfpathmoveto{\pgfqpoint{6.289712in}{1.308940in}}%
\pgfpathcurveto{\pgfqpoint{6.296845in}{1.308940in}}{\pgfqpoint{6.303687in}{1.311774in}}{\pgfqpoint{6.308730in}{1.316818in}}%
\pgfpathcurveto{\pgfqpoint{6.313774in}{1.321862in}}{\pgfqpoint{6.316608in}{1.328703in}}{\pgfqpoint{6.316608in}{1.335836in}}%
\pgfpathcurveto{\pgfqpoint{6.316608in}{1.342969in}}{\pgfqpoint{6.313774in}{1.349811in}}{\pgfqpoint{6.308730in}{1.354854in}}%
\pgfpathcurveto{\pgfqpoint{6.303687in}{1.359898in}}{\pgfqpoint{6.296845in}{1.362732in}}{\pgfqpoint{6.289712in}{1.362732in}}%
\pgfpathcurveto{\pgfqpoint{6.282579in}{1.362732in}}{\pgfqpoint{6.275738in}{1.359898in}}{\pgfqpoint{6.270694in}{1.354854in}}%
\pgfpathcurveto{\pgfqpoint{6.265650in}{1.349811in}}{\pgfqpoint{6.262816in}{1.342969in}}{\pgfqpoint{6.262816in}{1.335836in}}%
\pgfpathcurveto{\pgfqpoint{6.262816in}{1.328703in}}{\pgfqpoint{6.265650in}{1.321862in}}{\pgfqpoint{6.270694in}{1.316818in}}%
\pgfpathcurveto{\pgfqpoint{6.275738in}{1.311774in}}{\pgfqpoint{6.282579in}{1.308940in}}{\pgfqpoint{6.289712in}{1.308940in}}%
\pgfpathclose%
\pgfusepath{stroke,fill}%
\end{pgfscope}%
\begin{pgfscope}%
\pgfpathrectangle{\pgfqpoint{4.985294in}{0.500000in}}{\pgfqpoint{1.764706in}{1.700000in}}%
\pgfusepath{clip}%
\pgfsetbuttcap%
\pgfsetroundjoin%
\definecolor{currentfill}{rgb}{0.980678,0.914765,0.856766}%
\pgfsetfillcolor{currentfill}%
\pgfsetlinewidth{0.311001pt}%
\definecolor{currentstroke}{rgb}{1.000000,1.000000,1.000000}%
\pgfsetstrokecolor{currentstroke}%
\pgfsetdash{}{0pt}%
\pgfpathmoveto{\pgfqpoint{6.324491in}{1.244324in}}%
\pgfpathcurveto{\pgfqpoint{6.331624in}{1.244324in}}{\pgfqpoint{6.338465in}{1.247158in}}{\pgfqpoint{6.343509in}{1.252201in}}%
\pgfpathcurveto{\pgfqpoint{6.348553in}{1.257245in}}{\pgfqpoint{6.351387in}{1.264087in}}{\pgfqpoint{6.351387in}{1.271220in}}%
\pgfpathcurveto{\pgfqpoint{6.351387in}{1.278352in}}{\pgfqpoint{6.348553in}{1.285194in}}{\pgfqpoint{6.343509in}{1.290238in}}%
\pgfpathcurveto{\pgfqpoint{6.338465in}{1.295281in}}{\pgfqpoint{6.331624in}{1.298115in}}{\pgfqpoint{6.324491in}{1.298115in}}%
\pgfpathcurveto{\pgfqpoint{6.317358in}{1.298115in}}{\pgfqpoint{6.310516in}{1.295281in}}{\pgfqpoint{6.305473in}{1.290238in}}%
\pgfpathcurveto{\pgfqpoint{6.300429in}{1.285194in}}{\pgfqpoint{6.297595in}{1.278352in}}{\pgfqpoint{6.297595in}{1.271220in}}%
\pgfpathcurveto{\pgfqpoint{6.297595in}{1.264087in}}{\pgfqpoint{6.300429in}{1.257245in}}{\pgfqpoint{6.305473in}{1.252201in}}%
\pgfpathcurveto{\pgfqpoint{6.310516in}{1.247158in}}{\pgfqpoint{6.317358in}{1.244324in}}{\pgfqpoint{6.324491in}{1.244324in}}%
\pgfpathclose%
\pgfusepath{stroke,fill}%
\end{pgfscope}%
\begin{pgfscope}%
\pgfpathrectangle{\pgfqpoint{4.985294in}{0.500000in}}{\pgfqpoint{1.764706in}{1.700000in}}%
\pgfusepath{clip}%
\pgfsetbuttcap%
\pgfsetroundjoin%
\definecolor{currentfill}{rgb}{0.980678,0.914765,0.856766}%
\pgfsetfillcolor{currentfill}%
\pgfsetlinewidth{0.311001pt}%
\definecolor{currentstroke}{rgb}{1.000000,1.000000,1.000000}%
\pgfsetstrokecolor{currentstroke}%
\pgfsetdash{}{0pt}%
\pgfpathmoveto{\pgfqpoint{5.421899in}{1.322243in}}%
\pgfpathcurveto{\pgfqpoint{5.429032in}{1.322243in}}{\pgfqpoint{5.435873in}{1.325076in}}{\pgfqpoint{5.440917in}{1.330120in}}%
\pgfpathcurveto{\pgfqpoint{5.445961in}{1.335164in}}{\pgfqpoint{5.448794in}{1.342005in}}{\pgfqpoint{5.448794in}{1.349138in}}%
\pgfpathcurveto{\pgfqpoint{5.448794in}{1.356271in}}{\pgfqpoint{5.445961in}{1.363113in}}{\pgfqpoint{5.440917in}{1.368156in}}%
\pgfpathcurveto{\pgfqpoint{5.435873in}{1.373200in}}{\pgfqpoint{5.429032in}{1.376034in}}{\pgfqpoint{5.421899in}{1.376034in}}%
\pgfpathcurveto{\pgfqpoint{5.414766in}{1.376034in}}{\pgfqpoint{5.407924in}{1.373200in}}{\pgfqpoint{5.402881in}{1.368156in}}%
\pgfpathcurveto{\pgfqpoint{5.397837in}{1.363113in}}{\pgfqpoint{5.395003in}{1.356271in}}{\pgfqpoint{5.395003in}{1.349138in}}%
\pgfpathcurveto{\pgfqpoint{5.395003in}{1.342005in}}{\pgfqpoint{5.397837in}{1.335164in}}{\pgfqpoint{5.402881in}{1.330120in}}%
\pgfpathcurveto{\pgfqpoint{5.407924in}{1.325076in}}{\pgfqpoint{5.414766in}{1.322243in}}{\pgfqpoint{5.421899in}{1.322243in}}%
\pgfpathclose%
\pgfusepath{stroke,fill}%
\end{pgfscope}%
\begin{pgfscope}%
\pgfpathrectangle{\pgfqpoint{4.985294in}{0.500000in}}{\pgfqpoint{1.764706in}{1.700000in}}%
\pgfusepath{clip}%
\pgfsetbuttcap%
\pgfsetroundjoin%
\definecolor{currentfill}{rgb}{0.967398,0.774513,0.650573}%
\pgfsetfillcolor{currentfill}%
\pgfsetlinewidth{0.311001pt}%
\definecolor{currentstroke}{rgb}{1.000000,1.000000,1.000000}%
\pgfsetstrokecolor{currentstroke}%
\pgfsetdash{}{0pt}%
\pgfpathmoveto{\pgfqpoint{5.365910in}{1.147755in}}%
\pgfpathcurveto{\pgfqpoint{5.373043in}{1.147755in}}{\pgfqpoint{5.379884in}{1.150588in}}{\pgfqpoint{5.384928in}{1.155632in}}%
\pgfpathcurveto{\pgfqpoint{5.389972in}{1.160676in}}{\pgfqpoint{5.392805in}{1.167517in}}{\pgfqpoint{5.392805in}{1.174650in}}%
\pgfpathcurveto{\pgfqpoint{5.392805in}{1.181783in}}{\pgfqpoint{5.389972in}{1.188625in}}{\pgfqpoint{5.384928in}{1.193668in}}%
\pgfpathcurveto{\pgfqpoint{5.379884in}{1.198712in}}{\pgfqpoint{5.373043in}{1.201546in}}{\pgfqpoint{5.365910in}{1.201546in}}%
\pgfpathcurveto{\pgfqpoint{5.358777in}{1.201546in}}{\pgfqpoint{5.351935in}{1.198712in}}{\pgfqpoint{5.346892in}{1.193668in}}%
\pgfpathcurveto{\pgfqpoint{5.341848in}{1.188625in}}{\pgfqpoint{5.339014in}{1.181783in}}{\pgfqpoint{5.339014in}{1.174650in}}%
\pgfpathcurveto{\pgfqpoint{5.339014in}{1.167517in}}{\pgfqpoint{5.341848in}{1.160676in}}{\pgfqpoint{5.346892in}{1.155632in}}%
\pgfpathcurveto{\pgfqpoint{5.351935in}{1.150588in}}{\pgfqpoint{5.358777in}{1.147755in}}{\pgfqpoint{5.365910in}{1.147755in}}%
\pgfpathclose%
\pgfusepath{stroke,fill}%
\end{pgfscope}%
\begin{pgfscope}%
\pgfpathrectangle{\pgfqpoint{4.985294in}{0.500000in}}{\pgfqpoint{1.764706in}{1.700000in}}%
\pgfusepath{clip}%
\pgfsetbuttcap%
\pgfsetroundjoin%
\definecolor{currentfill}{rgb}{0.691463,0.089868,0.347769}%
\pgfsetfillcolor{currentfill}%
\pgfsetlinewidth{0.311001pt}%
\definecolor{currentstroke}{rgb}{1.000000,1.000000,1.000000}%
\pgfsetstrokecolor{currentstroke}%
\pgfsetdash{}{0pt}%
\pgfpathmoveto{\pgfqpoint{5.654872in}{1.193159in}}%
\pgfpathcurveto{\pgfqpoint{5.662005in}{1.193159in}}{\pgfqpoint{5.668847in}{1.195993in}}{\pgfqpoint{5.673891in}{1.201037in}}%
\pgfpathcurveto{\pgfqpoint{5.678934in}{1.206080in}}{\pgfqpoint{5.681768in}{1.212922in}}{\pgfqpoint{5.681768in}{1.220055in}}%
\pgfpathcurveto{\pgfqpoint{5.681768in}{1.227188in}}{\pgfqpoint{5.678934in}{1.234029in}}{\pgfqpoint{5.673891in}{1.239073in}}%
\pgfpathcurveto{\pgfqpoint{5.668847in}{1.244117in}}{\pgfqpoint{5.662005in}{1.246951in}}{\pgfqpoint{5.654872in}{1.246951in}}%
\pgfpathcurveto{\pgfqpoint{5.647740in}{1.246951in}}{\pgfqpoint{5.640898in}{1.244117in}}{\pgfqpoint{5.635854in}{1.239073in}}%
\pgfpathcurveto{\pgfqpoint{5.630811in}{1.234029in}}{\pgfqpoint{5.627977in}{1.227188in}}{\pgfqpoint{5.627977in}{1.220055in}}%
\pgfpathcurveto{\pgfqpoint{5.627977in}{1.212922in}}{\pgfqpoint{5.630811in}{1.206080in}}{\pgfqpoint{5.635854in}{1.201037in}}%
\pgfpathcurveto{\pgfqpoint{5.640898in}{1.195993in}}{\pgfqpoint{5.647740in}{1.193159in}}{\pgfqpoint{5.654872in}{1.193159in}}%
\pgfpathclose%
\pgfusepath{stroke,fill}%
\end{pgfscope}%
\begin{pgfscope}%
\pgfpathrectangle{\pgfqpoint{4.985294in}{0.500000in}}{\pgfqpoint{1.764706in}{1.700000in}}%
\pgfusepath{clip}%
\pgfsetbuttcap%
\pgfsetroundjoin%
\definecolor{currentfill}{rgb}{0.966120,0.744512,0.608720}%
\pgfsetfillcolor{currentfill}%
\pgfsetlinewidth{0.311001pt}%
\definecolor{currentstroke}{rgb}{1.000000,1.000000,1.000000}%
\pgfsetstrokecolor{currentstroke}%
\pgfsetdash{}{0pt}%
\pgfpathmoveto{\pgfqpoint{6.309125in}{1.632670in}}%
\pgfpathcurveto{\pgfqpoint{6.316258in}{1.632670in}}{\pgfqpoint{6.323099in}{1.635504in}}{\pgfqpoint{6.328143in}{1.640548in}}%
\pgfpathcurveto{\pgfqpoint{6.333187in}{1.645591in}}{\pgfqpoint{6.336021in}{1.652433in}}{\pgfqpoint{6.336021in}{1.659566in}}%
\pgfpathcurveto{\pgfqpoint{6.336021in}{1.666699in}}{\pgfqpoint{6.333187in}{1.673540in}}{\pgfqpoint{6.328143in}{1.678584in}}%
\pgfpathcurveto{\pgfqpoint{6.323099in}{1.683628in}}{\pgfqpoint{6.316258in}{1.686461in}}{\pgfqpoint{6.309125in}{1.686461in}}%
\pgfpathcurveto{\pgfqpoint{6.301992in}{1.686461in}}{\pgfqpoint{6.295150in}{1.683628in}}{\pgfqpoint{6.290107in}{1.678584in}}%
\pgfpathcurveto{\pgfqpoint{6.285063in}{1.673540in}}{\pgfqpoint{6.282229in}{1.666699in}}{\pgfqpoint{6.282229in}{1.659566in}}%
\pgfpathcurveto{\pgfqpoint{6.282229in}{1.652433in}}{\pgfqpoint{6.285063in}{1.645591in}}{\pgfqpoint{6.290107in}{1.640548in}}%
\pgfpathcurveto{\pgfqpoint{6.295150in}{1.635504in}}{\pgfqpoint{6.301992in}{1.632670in}}{\pgfqpoint{6.309125in}{1.632670in}}%
\pgfpathclose%
\pgfusepath{stroke,fill}%
\end{pgfscope}%
\begin{pgfscope}%
\pgfpathrectangle{\pgfqpoint{4.985294in}{0.500000in}}{\pgfqpoint{1.764706in}{1.700000in}}%
\pgfusepath{clip}%
\pgfsetbuttcap%
\pgfsetroundjoin%
\definecolor{currentfill}{rgb}{0.973832,0.856556,0.771584}%
\pgfsetfillcolor{currentfill}%
\pgfsetlinewidth{0.311001pt}%
\definecolor{currentstroke}{rgb}{1.000000,1.000000,1.000000}%
\pgfsetstrokecolor{currentstroke}%
\pgfsetdash{}{0pt}%
\pgfpathmoveto{\pgfqpoint{5.489317in}{1.444403in}}%
\pgfpathcurveto{\pgfqpoint{5.496450in}{1.444403in}}{\pgfqpoint{5.503291in}{1.447237in}}{\pgfqpoint{5.508335in}{1.452281in}}%
\pgfpathcurveto{\pgfqpoint{5.513379in}{1.457325in}}{\pgfqpoint{5.516213in}{1.464166in}}{\pgfqpoint{5.516213in}{1.471299in}}%
\pgfpathcurveto{\pgfqpoint{5.516213in}{1.478432in}}{\pgfqpoint{5.513379in}{1.485273in}}{\pgfqpoint{5.508335in}{1.490317in}}%
\pgfpathcurveto{\pgfqpoint{5.503291in}{1.495361in}}{\pgfqpoint{5.496450in}{1.498195in}}{\pgfqpoint{5.489317in}{1.498195in}}%
\pgfpathcurveto{\pgfqpoint{5.482184in}{1.498195in}}{\pgfqpoint{5.475342in}{1.495361in}}{\pgfqpoint{5.470299in}{1.490317in}}%
\pgfpathcurveto{\pgfqpoint{5.465255in}{1.485273in}}{\pgfqpoint{5.462421in}{1.478432in}}{\pgfqpoint{5.462421in}{1.471299in}}%
\pgfpathcurveto{\pgfqpoint{5.462421in}{1.464166in}}{\pgfqpoint{5.465255in}{1.457325in}}{\pgfqpoint{5.470299in}{1.452281in}}%
\pgfpathcurveto{\pgfqpoint{5.475342in}{1.447237in}}{\pgfqpoint{5.482184in}{1.444403in}}{\pgfqpoint{5.489317in}{1.444403in}}%
\pgfpathclose%
\pgfusepath{stroke,fill}%
\end{pgfscope}%
\begin{pgfscope}%
\pgfpathrectangle{\pgfqpoint{4.985294in}{0.500000in}}{\pgfqpoint{1.764706in}{1.700000in}}%
\pgfusepath{clip}%
\pgfsetbuttcap%
\pgfsetroundjoin%
\definecolor{currentfill}{rgb}{0.972726,0.844889,0.754401}%
\pgfsetfillcolor{currentfill}%
\pgfsetlinewidth{0.311001pt}%
\definecolor{currentstroke}{rgb}{1.000000,1.000000,1.000000}%
\pgfsetstrokecolor{currentstroke}%
\pgfsetdash{}{0pt}%
\pgfpathmoveto{\pgfqpoint{6.364628in}{1.213098in}}%
\pgfpathcurveto{\pgfqpoint{6.371761in}{1.213098in}}{\pgfqpoint{6.378602in}{1.215932in}}{\pgfqpoint{6.383646in}{1.220975in}}%
\pgfpathcurveto{\pgfqpoint{6.388690in}{1.226019in}}{\pgfqpoint{6.391523in}{1.232861in}}{\pgfqpoint{6.391523in}{1.239994in}}%
\pgfpathcurveto{\pgfqpoint{6.391523in}{1.247126in}}{\pgfqpoint{6.388690in}{1.253968in}}{\pgfqpoint{6.383646in}{1.259012in}}%
\pgfpathcurveto{\pgfqpoint{6.378602in}{1.264055in}}{\pgfqpoint{6.371761in}{1.266889in}}{\pgfqpoint{6.364628in}{1.266889in}}%
\pgfpathcurveto{\pgfqpoint{6.357495in}{1.266889in}}{\pgfqpoint{6.350653in}{1.264055in}}{\pgfqpoint{6.345610in}{1.259012in}}%
\pgfpathcurveto{\pgfqpoint{6.340566in}{1.253968in}}{\pgfqpoint{6.337732in}{1.247126in}}{\pgfqpoint{6.337732in}{1.239994in}}%
\pgfpathcurveto{\pgfqpoint{6.337732in}{1.232861in}}{\pgfqpoint{6.340566in}{1.226019in}}{\pgfqpoint{6.345610in}{1.220975in}}%
\pgfpathcurveto{\pgfqpoint{6.350653in}{1.215932in}}{\pgfqpoint{6.357495in}{1.213098in}}{\pgfqpoint{6.364628in}{1.213098in}}%
\pgfpathclose%
\pgfusepath{stroke,fill}%
\end{pgfscope}%
\begin{pgfscope}%
\pgfpathrectangle{\pgfqpoint{4.985294in}{0.500000in}}{\pgfqpoint{1.764706in}{1.700000in}}%
\pgfusepath{clip}%
\pgfsetbuttcap%
\pgfsetroundjoin%
\definecolor{currentfill}{rgb}{0.965302,0.713942,0.568499}%
\pgfsetfillcolor{currentfill}%
\pgfsetlinewidth{0.311001pt}%
\definecolor{currentstroke}{rgb}{1.000000,1.000000,1.000000}%
\pgfsetstrokecolor{currentstroke}%
\pgfsetdash{}{0pt}%
\pgfpathmoveto{\pgfqpoint{6.348084in}{1.577683in}}%
\pgfpathcurveto{\pgfqpoint{6.355217in}{1.577683in}}{\pgfqpoint{6.362058in}{1.580517in}}{\pgfqpoint{6.367102in}{1.585561in}}%
\pgfpathcurveto{\pgfqpoint{6.372146in}{1.590605in}}{\pgfqpoint{6.374980in}{1.597446in}}{\pgfqpoint{6.374980in}{1.604579in}}%
\pgfpathcurveto{\pgfqpoint{6.374980in}{1.611712in}}{\pgfqpoint{6.372146in}{1.618554in}}{\pgfqpoint{6.367102in}{1.623597in}}%
\pgfpathcurveto{\pgfqpoint{6.362058in}{1.628641in}}{\pgfqpoint{6.355217in}{1.631475in}}{\pgfqpoint{6.348084in}{1.631475in}}%
\pgfpathcurveto{\pgfqpoint{6.340951in}{1.631475in}}{\pgfqpoint{6.334109in}{1.628641in}}{\pgfqpoint{6.329066in}{1.623597in}}%
\pgfpathcurveto{\pgfqpoint{6.324022in}{1.618554in}}{\pgfqpoint{6.321188in}{1.611712in}}{\pgfqpoint{6.321188in}{1.604579in}}%
\pgfpathcurveto{\pgfqpoint{6.321188in}{1.597446in}}{\pgfqpoint{6.324022in}{1.590605in}}{\pgfqpoint{6.329066in}{1.585561in}}%
\pgfpathcurveto{\pgfqpoint{6.334109in}{1.580517in}}{\pgfqpoint{6.340951in}{1.577683in}}{\pgfqpoint{6.348084in}{1.577683in}}%
\pgfpathclose%
\pgfusepath{stroke,fill}%
\end{pgfscope}%
\begin{pgfscope}%
\pgfpathrectangle{\pgfqpoint{4.985294in}{0.500000in}}{\pgfqpoint{1.764706in}{1.700000in}}%
\pgfusepath{clip}%
\pgfsetbuttcap%
\pgfsetroundjoin%
\definecolor{currentfill}{rgb}{0.973832,0.856556,0.771584}%
\pgfsetfillcolor{currentfill}%
\pgfsetlinewidth{0.311001pt}%
\definecolor{currentstroke}{rgb}{1.000000,1.000000,1.000000}%
\pgfsetstrokecolor{currentstroke}%
\pgfsetdash{}{0pt}%
\pgfpathmoveto{\pgfqpoint{5.458561in}{1.575915in}}%
\pgfpathcurveto{\pgfqpoint{5.465694in}{1.575915in}}{\pgfqpoint{5.472536in}{1.578749in}}{\pgfqpoint{5.477579in}{1.583792in}}%
\pgfpathcurveto{\pgfqpoint{5.482623in}{1.588836in}}{\pgfqpoint{5.485457in}{1.595678in}}{\pgfqpoint{5.485457in}{1.602810in}}%
\pgfpathcurveto{\pgfqpoint{5.485457in}{1.609943in}}{\pgfqpoint{5.482623in}{1.616785in}}{\pgfqpoint{5.477579in}{1.621829in}}%
\pgfpathcurveto{\pgfqpoint{5.472536in}{1.626872in}}{\pgfqpoint{5.465694in}{1.629706in}}{\pgfqpoint{5.458561in}{1.629706in}}%
\pgfpathcurveto{\pgfqpoint{5.451428in}{1.629706in}}{\pgfqpoint{5.444587in}{1.626872in}}{\pgfqpoint{5.439543in}{1.621829in}}%
\pgfpathcurveto{\pgfqpoint{5.434499in}{1.616785in}}{\pgfqpoint{5.431665in}{1.609943in}}{\pgfqpoint{5.431665in}{1.602810in}}%
\pgfpathcurveto{\pgfqpoint{5.431665in}{1.595678in}}{\pgfqpoint{5.434499in}{1.588836in}}{\pgfqpoint{5.439543in}{1.583792in}}%
\pgfpathcurveto{\pgfqpoint{5.444587in}{1.578749in}}{\pgfqpoint{5.451428in}{1.575915in}}{\pgfqpoint{5.458561in}{1.575915in}}%
\pgfpathclose%
\pgfusepath{stroke,fill}%
\end{pgfscope}%
\begin{pgfscope}%
\pgfpathrectangle{\pgfqpoint{4.985294in}{0.500000in}}{\pgfqpoint{1.764706in}{1.700000in}}%
\pgfusepath{clip}%
\pgfsetbuttcap%
\pgfsetroundjoin%
\definecolor{currentfill}{rgb}{0.937528,0.344792,0.251999}%
\pgfsetfillcolor{currentfill}%
\pgfsetlinewidth{0.311001pt}%
\definecolor{currentstroke}{rgb}{1.000000,1.000000,1.000000}%
\pgfsetstrokecolor{currentstroke}%
\pgfsetdash{}{0pt}%
\pgfpathmoveto{\pgfqpoint{6.131813in}{0.811560in}}%
\pgfpathcurveto{\pgfqpoint{6.138946in}{0.811560in}}{\pgfqpoint{6.145787in}{0.814394in}}{\pgfqpoint{6.150831in}{0.819438in}}%
\pgfpathcurveto{\pgfqpoint{6.155875in}{0.824481in}}{\pgfqpoint{6.158709in}{0.831323in}}{\pgfqpoint{6.158709in}{0.838456in}}%
\pgfpathcurveto{\pgfqpoint{6.158709in}{0.845589in}}{\pgfqpoint{6.155875in}{0.852430in}}{\pgfqpoint{6.150831in}{0.857474in}}%
\pgfpathcurveto{\pgfqpoint{6.145787in}{0.862518in}}{\pgfqpoint{6.138946in}{0.865352in}}{\pgfqpoint{6.131813in}{0.865352in}}%
\pgfpathcurveto{\pgfqpoint{6.124680in}{0.865352in}}{\pgfqpoint{6.117838in}{0.862518in}}{\pgfqpoint{6.112795in}{0.857474in}}%
\pgfpathcurveto{\pgfqpoint{6.107751in}{0.852430in}}{\pgfqpoint{6.104917in}{0.845589in}}{\pgfqpoint{6.104917in}{0.838456in}}%
\pgfpathcurveto{\pgfqpoint{6.104917in}{0.831323in}}{\pgfqpoint{6.107751in}{0.824481in}}{\pgfqpoint{6.112795in}{0.819438in}}%
\pgfpathcurveto{\pgfqpoint{6.117838in}{0.814394in}}{\pgfqpoint{6.124680in}{0.811560in}}{\pgfqpoint{6.131813in}{0.811560in}}%
\pgfpathclose%
\pgfusepath{stroke,fill}%
\end{pgfscope}%
\begin{pgfscope}%
\pgfpathrectangle{\pgfqpoint{4.985294in}{0.500000in}}{\pgfqpoint{1.764706in}{1.700000in}}%
\pgfusepath{clip}%
\pgfsetbuttcap%
\pgfsetroundjoin%
\definecolor{currentfill}{rgb}{0.976961,0.885681,0.814303}%
\pgfsetfillcolor{currentfill}%
\pgfsetlinewidth{0.311001pt}%
\definecolor{currentstroke}{rgb}{1.000000,1.000000,1.000000}%
\pgfsetstrokecolor{currentstroke}%
\pgfsetdash{}{0pt}%
\pgfpathmoveto{\pgfqpoint{5.415265in}{1.169173in}}%
\pgfpathcurveto{\pgfqpoint{5.422398in}{1.169173in}}{\pgfqpoint{5.429240in}{1.172007in}}{\pgfqpoint{5.434284in}{1.177051in}}%
\pgfpathcurveto{\pgfqpoint{5.439327in}{1.182094in}}{\pgfqpoint{5.442161in}{1.188936in}}{\pgfqpoint{5.442161in}{1.196069in}}%
\pgfpathcurveto{\pgfqpoint{5.442161in}{1.203202in}}{\pgfqpoint{5.439327in}{1.210043in}}{\pgfqpoint{5.434284in}{1.215087in}}%
\pgfpathcurveto{\pgfqpoint{5.429240in}{1.220131in}}{\pgfqpoint{5.422398in}{1.222964in}}{\pgfqpoint{5.415265in}{1.222964in}}%
\pgfpathcurveto{\pgfqpoint{5.408133in}{1.222964in}}{\pgfqpoint{5.401291in}{1.220131in}}{\pgfqpoint{5.396247in}{1.215087in}}%
\pgfpathcurveto{\pgfqpoint{5.391204in}{1.210043in}}{\pgfqpoint{5.388370in}{1.203202in}}{\pgfqpoint{5.388370in}{1.196069in}}%
\pgfpathcurveto{\pgfqpoint{5.388370in}{1.188936in}}{\pgfqpoint{5.391204in}{1.182094in}}{\pgfqpoint{5.396247in}{1.177051in}}%
\pgfpathcurveto{\pgfqpoint{5.401291in}{1.172007in}}{\pgfqpoint{5.408133in}{1.169173in}}{\pgfqpoint{5.415265in}{1.169173in}}%
\pgfpathclose%
\pgfusepath{stroke,fill}%
\end{pgfscope}%
\begin{pgfscope}%
\pgfpathrectangle{\pgfqpoint{4.985294in}{0.500000in}}{\pgfqpoint{1.764706in}{1.700000in}}%
\pgfusepath{clip}%
\pgfsetbuttcap%
\pgfsetroundjoin%
\definecolor{currentfill}{rgb}{0.965440,0.720101,0.576404}%
\pgfsetfillcolor{currentfill}%
\pgfsetlinewidth{0.311001pt}%
\definecolor{currentstroke}{rgb}{1.000000,1.000000,1.000000}%
\pgfsetstrokecolor{currentstroke}%
\pgfsetdash{}{0pt}%
\pgfpathmoveto{\pgfqpoint{5.565888in}{1.546436in}}%
\pgfpathcurveto{\pgfqpoint{5.573021in}{1.546436in}}{\pgfqpoint{5.579863in}{1.549270in}}{\pgfqpoint{5.584906in}{1.554314in}}%
\pgfpathcurveto{\pgfqpoint{5.589950in}{1.559357in}}{\pgfqpoint{5.592784in}{1.566199in}}{\pgfqpoint{5.592784in}{1.573332in}}%
\pgfpathcurveto{\pgfqpoint{5.592784in}{1.580465in}}{\pgfqpoint{5.589950in}{1.587306in}}{\pgfqpoint{5.584906in}{1.592350in}}%
\pgfpathcurveto{\pgfqpoint{5.579863in}{1.597394in}}{\pgfqpoint{5.573021in}{1.600228in}}{\pgfqpoint{5.565888in}{1.600228in}}%
\pgfpathcurveto{\pgfqpoint{5.558755in}{1.600228in}}{\pgfqpoint{5.551914in}{1.597394in}}{\pgfqpoint{5.546870in}{1.592350in}}%
\pgfpathcurveto{\pgfqpoint{5.541826in}{1.587306in}}{\pgfqpoint{5.538992in}{1.580465in}}{\pgfqpoint{5.538992in}{1.573332in}}%
\pgfpathcurveto{\pgfqpoint{5.538992in}{1.566199in}}{\pgfqpoint{5.541826in}{1.559357in}}{\pgfqpoint{5.546870in}{1.554314in}}%
\pgfpathcurveto{\pgfqpoint{5.551914in}{1.549270in}}{\pgfqpoint{5.558755in}{1.546436in}}{\pgfqpoint{5.565888in}{1.546436in}}%
\pgfpathclose%
\pgfusepath{stroke,fill}%
\end{pgfscope}%
\begin{pgfscope}%
\pgfpathrectangle{\pgfqpoint{4.985294in}{0.500000in}}{\pgfqpoint{1.764706in}{1.700000in}}%
\pgfusepath{clip}%
\pgfsetbuttcap%
\pgfsetroundjoin%
\definecolor{currentfill}{rgb}{0.977657,0.891500,0.822809}%
\pgfsetfillcolor{currentfill}%
\pgfsetlinewidth{0.311001pt}%
\definecolor{currentstroke}{rgb}{1.000000,1.000000,1.000000}%
\pgfsetstrokecolor{currentstroke}%
\pgfsetdash{}{0pt}%
\pgfpathmoveto{\pgfqpoint{5.424415in}{1.153438in}}%
\pgfpathcurveto{\pgfqpoint{5.431548in}{1.153438in}}{\pgfqpoint{5.438389in}{1.156272in}}{\pgfqpoint{5.443433in}{1.161316in}}%
\pgfpathcurveto{\pgfqpoint{5.448477in}{1.166360in}}{\pgfqpoint{5.451311in}{1.173201in}}{\pgfqpoint{5.451311in}{1.180334in}}%
\pgfpathcurveto{\pgfqpoint{5.451311in}{1.187467in}}{\pgfqpoint{5.448477in}{1.194309in}}{\pgfqpoint{5.443433in}{1.199352in}}%
\pgfpathcurveto{\pgfqpoint{5.438389in}{1.204396in}}{\pgfqpoint{5.431548in}{1.207230in}}{\pgfqpoint{5.424415in}{1.207230in}}%
\pgfpathcurveto{\pgfqpoint{5.417282in}{1.207230in}}{\pgfqpoint{5.410440in}{1.204396in}}{\pgfqpoint{5.405397in}{1.199352in}}%
\pgfpathcurveto{\pgfqpoint{5.400353in}{1.194309in}}{\pgfqpoint{5.397519in}{1.187467in}}{\pgfqpoint{5.397519in}{1.180334in}}%
\pgfpathcurveto{\pgfqpoint{5.397519in}{1.173201in}}{\pgfqpoint{5.400353in}{1.166360in}}{\pgfqpoint{5.405397in}{1.161316in}}%
\pgfpathcurveto{\pgfqpoint{5.410440in}{1.156272in}}{\pgfqpoint{5.417282in}{1.153438in}}{\pgfqpoint{5.424415in}{1.153438in}}%
\pgfpathclose%
\pgfusepath{stroke,fill}%
\end{pgfscope}%
\begin{pgfscope}%
\pgfpathrectangle{\pgfqpoint{4.985294in}{0.500000in}}{\pgfqpoint{1.764706in}{1.700000in}}%
\pgfusepath{clip}%
\pgfsetbuttcap%
\pgfsetroundjoin%
\definecolor{currentfill}{rgb}{0.972201,0.839051,0.745789}%
\pgfsetfillcolor{currentfill}%
\pgfsetlinewidth{0.311001pt}%
\definecolor{currentstroke}{rgb}{1.000000,1.000000,1.000000}%
\pgfsetstrokecolor{currentstroke}%
\pgfsetdash{}{0pt}%
\pgfpathmoveto{\pgfqpoint{5.473424in}{1.611785in}}%
\pgfpathcurveto{\pgfqpoint{5.480557in}{1.611785in}}{\pgfqpoint{5.487398in}{1.614619in}}{\pgfqpoint{5.492442in}{1.619663in}}%
\pgfpathcurveto{\pgfqpoint{5.497486in}{1.624707in}}{\pgfqpoint{5.500319in}{1.631548in}}{\pgfqpoint{5.500319in}{1.638681in}}%
\pgfpathcurveto{\pgfqpoint{5.500319in}{1.645814in}}{\pgfqpoint{5.497486in}{1.652656in}}{\pgfqpoint{5.492442in}{1.657699in}}%
\pgfpathcurveto{\pgfqpoint{5.487398in}{1.662743in}}{\pgfqpoint{5.480557in}{1.665577in}}{\pgfqpoint{5.473424in}{1.665577in}}%
\pgfpathcurveto{\pgfqpoint{5.466291in}{1.665577in}}{\pgfqpoint{5.459449in}{1.662743in}}{\pgfqpoint{5.454406in}{1.657699in}}%
\pgfpathcurveto{\pgfqpoint{5.449362in}{1.652656in}}{\pgfqpoint{5.446528in}{1.645814in}}{\pgfqpoint{5.446528in}{1.638681in}}%
\pgfpathcurveto{\pgfqpoint{5.446528in}{1.631548in}}{\pgfqpoint{5.449362in}{1.624707in}}{\pgfqpoint{5.454406in}{1.619663in}}%
\pgfpathcurveto{\pgfqpoint{5.459449in}{1.614619in}}{\pgfqpoint{5.466291in}{1.611785in}}{\pgfqpoint{5.473424in}{1.611785in}}%
\pgfpathclose%
\pgfusepath{stroke,fill}%
\end{pgfscope}%
\begin{pgfscope}%
\pgfpathrectangle{\pgfqpoint{4.985294in}{0.500000in}}{\pgfqpoint{1.764706in}{1.700000in}}%
\pgfusepath{clip}%
\pgfsetbuttcap%
\pgfsetroundjoin%
\definecolor{currentfill}{rgb}{0.970718,0.821518,0.719872}%
\pgfsetfillcolor{currentfill}%
\pgfsetlinewidth{0.311001pt}%
\definecolor{currentstroke}{rgb}{1.000000,1.000000,1.000000}%
\pgfsetstrokecolor{currentstroke}%
\pgfsetdash{}{0pt}%
\pgfpathmoveto{\pgfqpoint{6.213653in}{1.173665in}}%
\pgfpathcurveto{\pgfqpoint{6.220786in}{1.173665in}}{\pgfqpoint{6.227627in}{1.176499in}}{\pgfqpoint{6.232671in}{1.181542in}}%
\pgfpathcurveto{\pgfqpoint{6.237715in}{1.186586in}}{\pgfqpoint{6.240549in}{1.193428in}}{\pgfqpoint{6.240549in}{1.200560in}}%
\pgfpathcurveto{\pgfqpoint{6.240549in}{1.207693in}}{\pgfqpoint{6.237715in}{1.214535in}}{\pgfqpoint{6.232671in}{1.219579in}}%
\pgfpathcurveto{\pgfqpoint{6.227627in}{1.224622in}}{\pgfqpoint{6.220786in}{1.227456in}}{\pgfqpoint{6.213653in}{1.227456in}}%
\pgfpathcurveto{\pgfqpoint{6.206520in}{1.227456in}}{\pgfqpoint{6.199679in}{1.224622in}}{\pgfqpoint{6.194635in}{1.219579in}}%
\pgfpathcurveto{\pgfqpoint{6.189591in}{1.214535in}}{\pgfqpoint{6.186757in}{1.207693in}}{\pgfqpoint{6.186757in}{1.200560in}}%
\pgfpathcurveto{\pgfqpoint{6.186757in}{1.193428in}}{\pgfqpoint{6.189591in}{1.186586in}}{\pgfqpoint{6.194635in}{1.181542in}}%
\pgfpathcurveto{\pgfqpoint{6.199679in}{1.176499in}}{\pgfqpoint{6.206520in}{1.173665in}}{\pgfqpoint{6.213653in}{1.173665in}}%
\pgfpathclose%
\pgfusepath{stroke,fill}%
\end{pgfscope}%
\begin{pgfscope}%
\pgfpathrectangle{\pgfqpoint{4.985294in}{0.500000in}}{\pgfqpoint{1.764706in}{1.700000in}}%
\pgfusepath{clip}%
\pgfsetbuttcap%
\pgfsetroundjoin%
\definecolor{currentfill}{rgb}{0.964920,0.695342,0.545192}%
\pgfsetfillcolor{currentfill}%
\pgfsetlinewidth{0.311001pt}%
\definecolor{currentstroke}{rgb}{1.000000,1.000000,1.000000}%
\pgfsetstrokecolor{currentstroke}%
\pgfsetdash{}{0pt}%
\pgfpathmoveto{\pgfqpoint{6.361755in}{1.560577in}}%
\pgfpathcurveto{\pgfqpoint{6.368887in}{1.560577in}}{\pgfqpoint{6.375729in}{1.563411in}}{\pgfqpoint{6.380773in}{1.568454in}}%
\pgfpathcurveto{\pgfqpoint{6.385816in}{1.573498in}}{\pgfqpoint{6.388650in}{1.580340in}}{\pgfqpoint{6.388650in}{1.587472in}}%
\pgfpathcurveto{\pgfqpoint{6.388650in}{1.594605in}}{\pgfqpoint{6.385816in}{1.601447in}}{\pgfqpoint{6.380773in}{1.606491in}}%
\pgfpathcurveto{\pgfqpoint{6.375729in}{1.611534in}}{\pgfqpoint{6.368887in}{1.614368in}}{\pgfqpoint{6.361755in}{1.614368in}}%
\pgfpathcurveto{\pgfqpoint{6.354622in}{1.614368in}}{\pgfqpoint{6.347780in}{1.611534in}}{\pgfqpoint{6.342737in}{1.606491in}}%
\pgfpathcurveto{\pgfqpoint{6.337693in}{1.601447in}}{\pgfqpoint{6.334859in}{1.594605in}}{\pgfqpoint{6.334859in}{1.587472in}}%
\pgfpathcurveto{\pgfqpoint{6.334859in}{1.580340in}}{\pgfqpoint{6.337693in}{1.573498in}}{\pgfqpoint{6.342737in}{1.568454in}}%
\pgfpathcurveto{\pgfqpoint{6.347780in}{1.563411in}}{\pgfqpoint{6.354622in}{1.560577in}}{\pgfqpoint{6.361755in}{1.560577in}}%
\pgfpathclose%
\pgfusepath{stroke,fill}%
\end{pgfscope}%
\begin{pgfscope}%
\pgfpathrectangle{\pgfqpoint{4.985294in}{0.500000in}}{\pgfqpoint{1.764706in}{1.700000in}}%
\pgfusepath{clip}%
\pgfsetbuttcap%
\pgfsetroundjoin%
\definecolor{currentfill}{rgb}{0.966328,0.750560,0.616961}%
\pgfsetfillcolor{currentfill}%
\pgfsetlinewidth{0.311001pt}%
\definecolor{currentstroke}{rgb}{1.000000,1.000000,1.000000}%
\pgfsetstrokecolor{currentstroke}%
\pgfsetdash{}{0pt}%
\pgfpathmoveto{\pgfqpoint{6.194980in}{1.481250in}}%
\pgfpathcurveto{\pgfqpoint{6.202112in}{1.481250in}}{\pgfqpoint{6.208954in}{1.484083in}}{\pgfqpoint{6.213998in}{1.489127in}}%
\pgfpathcurveto{\pgfqpoint{6.219041in}{1.494171in}}{\pgfqpoint{6.221875in}{1.501012in}}{\pgfqpoint{6.221875in}{1.508145in}}%
\pgfpathcurveto{\pgfqpoint{6.221875in}{1.515278in}}{\pgfqpoint{6.219041in}{1.522120in}}{\pgfqpoint{6.213998in}{1.527163in}}%
\pgfpathcurveto{\pgfqpoint{6.208954in}{1.532207in}}{\pgfqpoint{6.202112in}{1.535041in}}{\pgfqpoint{6.194980in}{1.535041in}}%
\pgfpathcurveto{\pgfqpoint{6.187847in}{1.535041in}}{\pgfqpoint{6.181005in}{1.532207in}}{\pgfqpoint{6.175961in}{1.527163in}}%
\pgfpathcurveto{\pgfqpoint{6.170918in}{1.522120in}}{\pgfqpoint{6.168084in}{1.515278in}}{\pgfqpoint{6.168084in}{1.508145in}}%
\pgfpathcurveto{\pgfqpoint{6.168084in}{1.501012in}}{\pgfqpoint{6.170918in}{1.494171in}}{\pgfqpoint{6.175961in}{1.489127in}}%
\pgfpathcurveto{\pgfqpoint{6.181005in}{1.484083in}}{\pgfqpoint{6.187847in}{1.481250in}}{\pgfqpoint{6.194980in}{1.481250in}}%
\pgfpathclose%
\pgfusepath{stroke,fill}%
\end{pgfscope}%
\begin{pgfscope}%
\pgfpathrectangle{\pgfqpoint{4.985294in}{0.500000in}}{\pgfqpoint{1.764706in}{1.700000in}}%
\pgfusepath{clip}%
\pgfsetbuttcap%
\pgfsetroundjoin%
\definecolor{currentfill}{rgb}{0.976287,0.879862,0.805788}%
\pgfsetfillcolor{currentfill}%
\pgfsetlinewidth{0.311001pt}%
\definecolor{currentstroke}{rgb}{1.000000,1.000000,1.000000}%
\pgfsetstrokecolor{currentstroke}%
\pgfsetdash{}{0pt}%
\pgfpathmoveto{\pgfqpoint{5.459139in}{1.347704in}}%
\pgfpathcurveto{\pgfqpoint{5.466272in}{1.347704in}}{\pgfqpoint{5.473113in}{1.350538in}}{\pgfqpoint{5.478157in}{1.355582in}}%
\pgfpathcurveto{\pgfqpoint{5.483201in}{1.360626in}}{\pgfqpoint{5.486034in}{1.367467in}}{\pgfqpoint{5.486034in}{1.374600in}}%
\pgfpathcurveto{\pgfqpoint{5.486034in}{1.381733in}}{\pgfqpoint{5.483201in}{1.388575in}}{\pgfqpoint{5.478157in}{1.393618in}}%
\pgfpathcurveto{\pgfqpoint{5.473113in}{1.398662in}}{\pgfqpoint{5.466272in}{1.401496in}}{\pgfqpoint{5.459139in}{1.401496in}}%
\pgfpathcurveto{\pgfqpoint{5.452006in}{1.401496in}}{\pgfqpoint{5.445164in}{1.398662in}}{\pgfqpoint{5.440121in}{1.393618in}}%
\pgfpathcurveto{\pgfqpoint{5.435077in}{1.388575in}}{\pgfqpoint{5.432243in}{1.381733in}}{\pgfqpoint{5.432243in}{1.374600in}}%
\pgfpathcurveto{\pgfqpoint{5.432243in}{1.367467in}}{\pgfqpoint{5.435077in}{1.360626in}}{\pgfqpoint{5.440121in}{1.355582in}}%
\pgfpathcurveto{\pgfqpoint{5.445164in}{1.350538in}}{\pgfqpoint{5.452006in}{1.347704in}}{\pgfqpoint{5.459139in}{1.347704in}}%
\pgfpathclose%
\pgfusepath{stroke,fill}%
\end{pgfscope}%
\begin{pgfscope}%
\pgfpathrectangle{\pgfqpoint{4.985294in}{0.500000in}}{\pgfqpoint{1.764706in}{1.700000in}}%
\pgfusepath{clip}%
\pgfsetbuttcap%
\pgfsetroundjoin%
\definecolor{currentfill}{rgb}{0.965169,0.707764,0.560659}%
\pgfsetfillcolor{currentfill}%
\pgfsetlinewidth{0.311001pt}%
\definecolor{currentstroke}{rgb}{1.000000,1.000000,1.000000}%
\pgfsetstrokecolor{currentstroke}%
\pgfsetdash{}{0pt}%
\pgfpathmoveto{\pgfqpoint{5.570923in}{0.904585in}}%
\pgfpathcurveto{\pgfqpoint{5.578055in}{0.904585in}}{\pgfqpoint{5.584897in}{0.907419in}}{\pgfqpoint{5.589941in}{0.912463in}}%
\pgfpathcurveto{\pgfqpoint{5.594984in}{0.917507in}}{\pgfqpoint{5.597818in}{0.924348in}}{\pgfqpoint{5.597818in}{0.931481in}}%
\pgfpathcurveto{\pgfqpoint{5.597818in}{0.938614in}}{\pgfqpoint{5.594984in}{0.945456in}}{\pgfqpoint{5.589941in}{0.950499in}}%
\pgfpathcurveto{\pgfqpoint{5.584897in}{0.955543in}}{\pgfqpoint{5.578055in}{0.958377in}}{\pgfqpoint{5.570923in}{0.958377in}}%
\pgfpathcurveto{\pgfqpoint{5.563790in}{0.958377in}}{\pgfqpoint{5.556948in}{0.955543in}}{\pgfqpoint{5.551904in}{0.950499in}}%
\pgfpathcurveto{\pgfqpoint{5.546861in}{0.945456in}}{\pgfqpoint{5.544027in}{0.938614in}}{\pgfqpoint{5.544027in}{0.931481in}}%
\pgfpathcurveto{\pgfqpoint{5.544027in}{0.924348in}}{\pgfqpoint{5.546861in}{0.917507in}}{\pgfqpoint{5.551904in}{0.912463in}}%
\pgfpathcurveto{\pgfqpoint{5.556948in}{0.907419in}}{\pgfqpoint{5.563790in}{0.904585in}}{\pgfqpoint{5.570923in}{0.904585in}}%
\pgfpathclose%
\pgfusepath{stroke,fill}%
\end{pgfscope}%
\begin{pgfscope}%
\pgfpathrectangle{\pgfqpoint{4.985294in}{0.500000in}}{\pgfqpoint{1.764706in}{1.700000in}}%
\pgfusepath{clip}%
\pgfsetbuttcap%
\pgfsetroundjoin%
\definecolor{currentfill}{rgb}{0.957344,0.505732,0.351309}%
\pgfsetfillcolor{currentfill}%
\pgfsetlinewidth{0.311001pt}%
\definecolor{currentstroke}{rgb}{1.000000,1.000000,1.000000}%
\pgfsetstrokecolor{currentstroke}%
\pgfsetdash{}{0pt}%
\pgfpathmoveto{\pgfqpoint{6.412984in}{1.529243in}}%
\pgfpathcurveto{\pgfqpoint{6.420117in}{1.529243in}}{\pgfqpoint{6.426959in}{1.532077in}}{\pgfqpoint{6.432003in}{1.537120in}}%
\pgfpathcurveto{\pgfqpoint{6.437046in}{1.542164in}}{\pgfqpoint{6.439880in}{1.549006in}}{\pgfqpoint{6.439880in}{1.556138in}}%
\pgfpathcurveto{\pgfqpoint{6.439880in}{1.563271in}}{\pgfqpoint{6.437046in}{1.570113in}}{\pgfqpoint{6.432003in}{1.575157in}}%
\pgfpathcurveto{\pgfqpoint{6.426959in}{1.580200in}}{\pgfqpoint{6.420117in}{1.583034in}}{\pgfqpoint{6.412984in}{1.583034in}}%
\pgfpathcurveto{\pgfqpoint{6.405852in}{1.583034in}}{\pgfqpoint{6.399010in}{1.580200in}}{\pgfqpoint{6.393966in}{1.575157in}}%
\pgfpathcurveto{\pgfqpoint{6.388923in}{1.570113in}}{\pgfqpoint{6.386089in}{1.563271in}}{\pgfqpoint{6.386089in}{1.556138in}}%
\pgfpathcurveto{\pgfqpoint{6.386089in}{1.549006in}}{\pgfqpoint{6.388923in}{1.542164in}}{\pgfqpoint{6.393966in}{1.537120in}}%
\pgfpathcurveto{\pgfqpoint{6.399010in}{1.532077in}}{\pgfqpoint{6.405852in}{1.529243in}}{\pgfqpoint{6.412984in}{1.529243in}}%
\pgfpathclose%
\pgfusepath{stroke,fill}%
\end{pgfscope}%
\begin{pgfscope}%
\pgfpathrectangle{\pgfqpoint{4.985294in}{0.500000in}}{\pgfqpoint{1.764706in}{1.700000in}}%
\pgfusepath{clip}%
\pgfsetbuttcap%
\pgfsetroundjoin%
\definecolor{currentfill}{rgb}{0.963190,0.619109,0.458249}%
\pgfsetfillcolor{currentfill}%
\pgfsetlinewidth{0.311001pt}%
\definecolor{currentstroke}{rgb}{1.000000,1.000000,1.000000}%
\pgfsetstrokecolor{currentstroke}%
\pgfsetdash{}{0pt}%
\pgfpathmoveto{\pgfqpoint{5.576042in}{1.749359in}}%
\pgfpathcurveto{\pgfqpoint{5.583175in}{1.749359in}}{\pgfqpoint{5.590016in}{1.752193in}}{\pgfqpoint{5.595060in}{1.757236in}}%
\pgfpathcurveto{\pgfqpoint{5.600104in}{1.762280in}}{\pgfqpoint{5.602938in}{1.769122in}}{\pgfqpoint{5.602938in}{1.776254in}}%
\pgfpathcurveto{\pgfqpoint{5.602938in}{1.783387in}}{\pgfqpoint{5.600104in}{1.790229in}}{\pgfqpoint{5.595060in}{1.795273in}}%
\pgfpathcurveto{\pgfqpoint{5.590016in}{1.800316in}}{\pgfqpoint{5.583175in}{1.803150in}}{\pgfqpoint{5.576042in}{1.803150in}}%
\pgfpathcurveto{\pgfqpoint{5.568909in}{1.803150in}}{\pgfqpoint{5.562067in}{1.800316in}}{\pgfqpoint{5.557024in}{1.795273in}}%
\pgfpathcurveto{\pgfqpoint{5.551980in}{1.790229in}}{\pgfqpoint{5.549146in}{1.783387in}}{\pgfqpoint{5.549146in}{1.776254in}}%
\pgfpathcurveto{\pgfqpoint{5.549146in}{1.769122in}}{\pgfqpoint{5.551980in}{1.762280in}}{\pgfqpoint{5.557024in}{1.757236in}}%
\pgfpathcurveto{\pgfqpoint{5.562067in}{1.752193in}}{\pgfqpoint{5.568909in}{1.749359in}}{\pgfqpoint{5.576042in}{1.749359in}}%
\pgfpathclose%
\pgfusepath{stroke,fill}%
\end{pgfscope}%
\begin{pgfscope}%
\pgfpathrectangle{\pgfqpoint{4.985294in}{0.500000in}}{\pgfqpoint{1.764706in}{1.700000in}}%
\pgfusepath{clip}%
\pgfsetbuttcap%
\pgfsetroundjoin%
\definecolor{currentfill}{rgb}{0.981377,0.920617,0.865369}%
\pgfsetfillcolor{currentfill}%
\pgfsetlinewidth{0.311001pt}%
\definecolor{currentstroke}{rgb}{1.000000,1.000000,1.000000}%
\pgfsetstrokecolor{currentstroke}%
\pgfsetdash{}{0pt}%
\pgfpathmoveto{\pgfqpoint{6.314209in}{1.442572in}}%
\pgfpathcurveto{\pgfqpoint{6.321342in}{1.442572in}}{\pgfqpoint{6.328183in}{1.445406in}}{\pgfqpoint{6.333227in}{1.450450in}}%
\pgfpathcurveto{\pgfqpoint{6.338271in}{1.455493in}}{\pgfqpoint{6.341105in}{1.462335in}}{\pgfqpoint{6.341105in}{1.469468in}}%
\pgfpathcurveto{\pgfqpoint{6.341105in}{1.476601in}}{\pgfqpoint{6.338271in}{1.483442in}}{\pgfqpoint{6.333227in}{1.488486in}}%
\pgfpathcurveto{\pgfqpoint{6.328183in}{1.493530in}}{\pgfqpoint{6.321342in}{1.496363in}}{\pgfqpoint{6.314209in}{1.496363in}}%
\pgfpathcurveto{\pgfqpoint{6.307076in}{1.496363in}}{\pgfqpoint{6.300234in}{1.493530in}}{\pgfqpoint{6.295191in}{1.488486in}}%
\pgfpathcurveto{\pgfqpoint{6.290147in}{1.483442in}}{\pgfqpoint{6.287313in}{1.476601in}}{\pgfqpoint{6.287313in}{1.469468in}}%
\pgfpathcurveto{\pgfqpoint{6.287313in}{1.462335in}}{\pgfqpoint{6.290147in}{1.455493in}}{\pgfqpoint{6.295191in}{1.450450in}}%
\pgfpathcurveto{\pgfqpoint{6.300234in}{1.445406in}}{\pgfqpoint{6.307076in}{1.442572in}}{\pgfqpoint{6.314209in}{1.442572in}}%
\pgfpathclose%
\pgfusepath{stroke,fill}%
\end{pgfscope}%
\begin{pgfscope}%
\pgfpathrectangle{\pgfqpoint{4.985294in}{0.500000in}}{\pgfqpoint{1.764706in}{1.700000in}}%
\pgfusepath{clip}%
\pgfsetbuttcap%
\pgfsetroundjoin%
\definecolor{currentfill}{rgb}{0.976961,0.885681,0.814303}%
\pgfsetfillcolor{currentfill}%
\pgfsetlinewidth{0.311001pt}%
\definecolor{currentstroke}{rgb}{1.000000,1.000000,1.000000}%
\pgfsetstrokecolor{currentstroke}%
\pgfsetdash{}{0pt}%
\pgfpathmoveto{\pgfqpoint{6.267081in}{1.496371in}}%
\pgfpathcurveto{\pgfqpoint{6.274214in}{1.496371in}}{\pgfqpoint{6.281056in}{1.499204in}}{\pgfqpoint{6.286100in}{1.504248in}}%
\pgfpathcurveto{\pgfqpoint{6.291143in}{1.509292in}}{\pgfqpoint{6.293977in}{1.516133in}}{\pgfqpoint{6.293977in}{1.523266in}}%
\pgfpathcurveto{\pgfqpoint{6.293977in}{1.530399in}}{\pgfqpoint{6.291143in}{1.537241in}}{\pgfqpoint{6.286100in}{1.542284in}}%
\pgfpathcurveto{\pgfqpoint{6.281056in}{1.547328in}}{\pgfqpoint{6.274214in}{1.550162in}}{\pgfqpoint{6.267081in}{1.550162in}}%
\pgfpathcurveto{\pgfqpoint{6.259949in}{1.550162in}}{\pgfqpoint{6.253107in}{1.547328in}}{\pgfqpoint{6.248063in}{1.542284in}}%
\pgfpathcurveto{\pgfqpoint{6.243020in}{1.537241in}}{\pgfqpoint{6.240186in}{1.530399in}}{\pgfqpoint{6.240186in}{1.523266in}}%
\pgfpathcurveto{\pgfqpoint{6.240186in}{1.516133in}}{\pgfqpoint{6.243020in}{1.509292in}}{\pgfqpoint{6.248063in}{1.504248in}}%
\pgfpathcurveto{\pgfqpoint{6.253107in}{1.499204in}}{\pgfqpoint{6.259949in}{1.496371in}}{\pgfqpoint{6.267081in}{1.496371in}}%
\pgfpathclose%
\pgfusepath{stroke,fill}%
\end{pgfscope}%
\begin{pgfscope}%
\pgfpathrectangle{\pgfqpoint{4.985294in}{0.500000in}}{\pgfqpoint{1.764706in}{1.700000in}}%
\pgfusepath{clip}%
\pgfsetbuttcap%
\pgfsetroundjoin%
\definecolor{currentfill}{rgb}{0.962765,0.606121,0.444717}%
\pgfsetfillcolor{currentfill}%
\pgfsetlinewidth{0.311001pt}%
\definecolor{currentstroke}{rgb}{1.000000,1.000000,1.000000}%
\pgfsetstrokecolor{currentstroke}%
\pgfsetdash{}{0pt}%
\pgfpathmoveto{\pgfqpoint{5.314852in}{1.192590in}}%
\pgfpathcurveto{\pgfqpoint{5.321985in}{1.192590in}}{\pgfqpoint{5.328826in}{1.195424in}}{\pgfqpoint{5.333870in}{1.200468in}}%
\pgfpathcurveto{\pgfqpoint{5.338914in}{1.205511in}}{\pgfqpoint{5.341747in}{1.212353in}}{\pgfqpoint{5.341747in}{1.219486in}}%
\pgfpathcurveto{\pgfqpoint{5.341747in}{1.226619in}}{\pgfqpoint{5.338914in}{1.233460in}}{\pgfqpoint{5.333870in}{1.238504in}}%
\pgfpathcurveto{\pgfqpoint{5.328826in}{1.243548in}}{\pgfqpoint{5.321985in}{1.246382in}}{\pgfqpoint{5.314852in}{1.246382in}}%
\pgfpathcurveto{\pgfqpoint{5.307719in}{1.246382in}}{\pgfqpoint{5.300877in}{1.243548in}}{\pgfqpoint{5.295834in}{1.238504in}}%
\pgfpathcurveto{\pgfqpoint{5.290790in}{1.233460in}}{\pgfqpoint{5.287956in}{1.226619in}}{\pgfqpoint{5.287956in}{1.219486in}}%
\pgfpathcurveto{\pgfqpoint{5.287956in}{1.212353in}}{\pgfqpoint{5.290790in}{1.205511in}}{\pgfqpoint{5.295834in}{1.200468in}}%
\pgfpathcurveto{\pgfqpoint{5.300877in}{1.195424in}}{\pgfqpoint{5.307719in}{1.192590in}}{\pgfqpoint{5.314852in}{1.192590in}}%
\pgfpathclose%
\pgfusepath{stroke,fill}%
\end{pgfscope}%
\begin{pgfscope}%
\pgfpathrectangle{\pgfqpoint{4.985294in}{0.500000in}}{\pgfqpoint{1.764706in}{1.700000in}}%
\pgfusepath{clip}%
\pgfsetbuttcap%
\pgfsetroundjoin%
\definecolor{currentfill}{rgb}{0.978376,0.897317,0.831308}%
\pgfsetfillcolor{currentfill}%
\pgfsetlinewidth{0.311001pt}%
\definecolor{currentstroke}{rgb}{1.000000,1.000000,1.000000}%
\pgfsetstrokecolor{currentstroke}%
\pgfsetdash{}{0pt}%
\pgfpathmoveto{\pgfqpoint{5.408937in}{1.440388in}}%
\pgfpathcurveto{\pgfqpoint{5.416070in}{1.440388in}}{\pgfqpoint{5.422912in}{1.443222in}}{\pgfqpoint{5.427955in}{1.448266in}}%
\pgfpathcurveto{\pgfqpoint{5.432999in}{1.453310in}}{\pgfqpoint{5.435833in}{1.460151in}}{\pgfqpoint{5.435833in}{1.467284in}}%
\pgfpathcurveto{\pgfqpoint{5.435833in}{1.474417in}}{\pgfqpoint{5.432999in}{1.481259in}}{\pgfqpoint{5.427955in}{1.486302in}}%
\pgfpathcurveto{\pgfqpoint{5.422912in}{1.491346in}}{\pgfqpoint{5.416070in}{1.494180in}}{\pgfqpoint{5.408937in}{1.494180in}}%
\pgfpathcurveto{\pgfqpoint{5.401804in}{1.494180in}}{\pgfqpoint{5.394963in}{1.491346in}}{\pgfqpoint{5.389919in}{1.486302in}}%
\pgfpathcurveto{\pgfqpoint{5.384875in}{1.481259in}}{\pgfqpoint{5.382041in}{1.474417in}}{\pgfqpoint{5.382041in}{1.467284in}}%
\pgfpathcurveto{\pgfqpoint{5.382041in}{1.460151in}}{\pgfqpoint{5.384875in}{1.453310in}}{\pgfqpoint{5.389919in}{1.448266in}}%
\pgfpathcurveto{\pgfqpoint{5.394963in}{1.443222in}}{\pgfqpoint{5.401804in}{1.440388in}}{\pgfqpoint{5.408937in}{1.440388in}}%
\pgfpathclose%
\pgfusepath{stroke,fill}%
\end{pgfscope}%
\begin{pgfscope}%
\pgfpathrectangle{\pgfqpoint{4.985294in}{0.500000in}}{\pgfqpoint{1.764706in}{1.700000in}}%
\pgfusepath{clip}%
\pgfsetbuttcap%
\pgfsetroundjoin%
\definecolor{currentfill}{rgb}{0.971694,0.833208,0.737161}%
\pgfsetfillcolor{currentfill}%
\pgfsetlinewidth{0.311001pt}%
\definecolor{currentstroke}{rgb}{1.000000,1.000000,1.000000}%
\pgfsetstrokecolor{currentstroke}%
\pgfsetdash{}{0pt}%
\pgfpathmoveto{\pgfqpoint{6.244352in}{1.465003in}}%
\pgfpathcurveto{\pgfqpoint{6.251484in}{1.465003in}}{\pgfqpoint{6.258326in}{1.467837in}}{\pgfqpoint{6.263370in}{1.472880in}}%
\pgfpathcurveto{\pgfqpoint{6.268413in}{1.477924in}}{\pgfqpoint{6.271247in}{1.484766in}}{\pgfqpoint{6.271247in}{1.491898in}}%
\pgfpathcurveto{\pgfqpoint{6.271247in}{1.499031in}}{\pgfqpoint{6.268413in}{1.505873in}}{\pgfqpoint{6.263370in}{1.510917in}}%
\pgfpathcurveto{\pgfqpoint{6.258326in}{1.515960in}}{\pgfqpoint{6.251484in}{1.518794in}}{\pgfqpoint{6.244352in}{1.518794in}}%
\pgfpathcurveto{\pgfqpoint{6.237219in}{1.518794in}}{\pgfqpoint{6.230377in}{1.515960in}}{\pgfqpoint{6.225334in}{1.510917in}}%
\pgfpathcurveto{\pgfqpoint{6.220290in}{1.505873in}}{\pgfqpoint{6.217456in}{1.499031in}}{\pgfqpoint{6.217456in}{1.491898in}}%
\pgfpathcurveto{\pgfqpoint{6.217456in}{1.484766in}}{\pgfqpoint{6.220290in}{1.477924in}}{\pgfqpoint{6.225334in}{1.472880in}}%
\pgfpathcurveto{\pgfqpoint{6.230377in}{1.467837in}}{\pgfqpoint{6.237219in}{1.465003in}}{\pgfqpoint{6.244352in}{1.465003in}}%
\pgfpathclose%
\pgfusepath{stroke,fill}%
\end{pgfscope}%
\begin{pgfscope}%
\pgfpathrectangle{\pgfqpoint{4.985294in}{0.500000in}}{\pgfqpoint{1.764706in}{1.700000in}}%
\pgfusepath{clip}%
\pgfsetbuttcap%
\pgfsetroundjoin%
\definecolor{currentfill}{rgb}{0.969359,0.803954,0.693832}%
\pgfsetfillcolor{currentfill}%
\pgfsetlinewidth{0.311001pt}%
\definecolor{currentstroke}{rgb}{1.000000,1.000000,1.000000}%
\pgfsetstrokecolor{currentstroke}%
\pgfsetdash{}{0pt}%
\pgfpathmoveto{\pgfqpoint{5.525704in}{1.102162in}}%
\pgfpathcurveto{\pgfqpoint{5.532837in}{1.102162in}}{\pgfqpoint{5.539679in}{1.104996in}}{\pgfqpoint{5.544722in}{1.110040in}}%
\pgfpathcurveto{\pgfqpoint{5.549766in}{1.115083in}}{\pgfqpoint{5.552600in}{1.121925in}}{\pgfqpoint{5.552600in}{1.129058in}}%
\pgfpathcurveto{\pgfqpoint{5.552600in}{1.136191in}}{\pgfqpoint{5.549766in}{1.143032in}}{\pgfqpoint{5.544722in}{1.148076in}}%
\pgfpathcurveto{\pgfqpoint{5.539679in}{1.153120in}}{\pgfqpoint{5.532837in}{1.155954in}}{\pgfqpoint{5.525704in}{1.155954in}}%
\pgfpathcurveto{\pgfqpoint{5.518572in}{1.155954in}}{\pgfqpoint{5.511730in}{1.153120in}}{\pgfqpoint{5.506686in}{1.148076in}}%
\pgfpathcurveto{\pgfqpoint{5.501643in}{1.143032in}}{\pgfqpoint{5.498809in}{1.136191in}}{\pgfqpoint{5.498809in}{1.129058in}}%
\pgfpathcurveto{\pgfqpoint{5.498809in}{1.121925in}}{\pgfqpoint{5.501643in}{1.115083in}}{\pgfqpoint{5.506686in}{1.110040in}}%
\pgfpathcurveto{\pgfqpoint{5.511730in}{1.104996in}}{\pgfqpoint{5.518572in}{1.102162in}}{\pgfqpoint{5.525704in}{1.102162in}}%
\pgfpathclose%
\pgfusepath{stroke,fill}%
\end{pgfscope}%
\begin{pgfscope}%
\pgfpathrectangle{\pgfqpoint{4.985294in}{0.500000in}}{\pgfqpoint{1.764706in}{1.700000in}}%
\pgfusepath{clip}%
\pgfsetbuttcap%
\pgfsetroundjoin%
\definecolor{currentfill}{rgb}{0.969803,0.809811,0.702523}%
\pgfsetfillcolor{currentfill}%
\pgfsetlinewidth{0.311001pt}%
\definecolor{currentstroke}{rgb}{1.000000,1.000000,1.000000}%
\pgfsetstrokecolor{currentstroke}%
\pgfsetdash{}{0pt}%
\pgfpathmoveto{\pgfqpoint{6.386325in}{1.302022in}}%
\pgfpathcurveto{\pgfqpoint{6.393458in}{1.302022in}}{\pgfqpoint{6.400300in}{1.304855in}}{\pgfqpoint{6.405344in}{1.309899in}}%
\pgfpathcurveto{\pgfqpoint{6.410387in}{1.314943in}}{\pgfqpoint{6.413221in}{1.321784in}}{\pgfqpoint{6.413221in}{1.328917in}}%
\pgfpathcurveto{\pgfqpoint{6.413221in}{1.336050in}}{\pgfqpoint{6.410387in}{1.342892in}}{\pgfqpoint{6.405344in}{1.347935in}}%
\pgfpathcurveto{\pgfqpoint{6.400300in}{1.352979in}}{\pgfqpoint{6.393458in}{1.355813in}}{\pgfqpoint{6.386325in}{1.355813in}}%
\pgfpathcurveto{\pgfqpoint{6.379193in}{1.355813in}}{\pgfqpoint{6.372351in}{1.352979in}}{\pgfqpoint{6.367307in}{1.347935in}}%
\pgfpathcurveto{\pgfqpoint{6.362264in}{1.342892in}}{\pgfqpoint{6.359430in}{1.336050in}}{\pgfqpoint{6.359430in}{1.328917in}}%
\pgfpathcurveto{\pgfqpoint{6.359430in}{1.321784in}}{\pgfqpoint{6.362264in}{1.314943in}}{\pgfqpoint{6.367307in}{1.309899in}}%
\pgfpathcurveto{\pgfqpoint{6.372351in}{1.304855in}}{\pgfqpoint{6.379193in}{1.302022in}}{\pgfqpoint{6.386325in}{1.302022in}}%
\pgfpathclose%
\pgfusepath{stroke,fill}%
\end{pgfscope}%
\begin{pgfscope}%
\pgfpathrectangle{\pgfqpoint{4.985294in}{0.500000in}}{\pgfqpoint{1.764706in}{1.700000in}}%
\pgfusepath{clip}%
\pgfsetbuttcap%
\pgfsetroundjoin%
\definecolor{currentfill}{rgb}{0.972726,0.844889,0.754401}%
\pgfsetfillcolor{currentfill}%
\pgfsetlinewidth{0.311001pt}%
\definecolor{currentstroke}{rgb}{1.000000,1.000000,1.000000}%
\pgfsetstrokecolor{currentstroke}%
\pgfsetdash{}{0pt}%
\pgfpathmoveto{\pgfqpoint{5.476381in}{1.275714in}}%
\pgfpathcurveto{\pgfqpoint{5.483514in}{1.275714in}}{\pgfqpoint{5.490355in}{1.278548in}}{\pgfqpoint{5.495399in}{1.283592in}}%
\pgfpathcurveto{\pgfqpoint{5.500443in}{1.288635in}}{\pgfqpoint{5.503276in}{1.295477in}}{\pgfqpoint{5.503276in}{1.302610in}}%
\pgfpathcurveto{\pgfqpoint{5.503276in}{1.309743in}}{\pgfqpoint{5.500443in}{1.316584in}}{\pgfqpoint{5.495399in}{1.321628in}}%
\pgfpathcurveto{\pgfqpoint{5.490355in}{1.326672in}}{\pgfqpoint{5.483514in}{1.329505in}}{\pgfqpoint{5.476381in}{1.329505in}}%
\pgfpathcurveto{\pgfqpoint{5.469248in}{1.329505in}}{\pgfqpoint{5.462406in}{1.326672in}}{\pgfqpoint{5.457363in}{1.321628in}}%
\pgfpathcurveto{\pgfqpoint{5.452319in}{1.316584in}}{\pgfqpoint{5.449485in}{1.309743in}}{\pgfqpoint{5.449485in}{1.302610in}}%
\pgfpathcurveto{\pgfqpoint{5.449485in}{1.295477in}}{\pgfqpoint{5.452319in}{1.288635in}}{\pgfqpoint{5.457363in}{1.283592in}}%
\pgfpathcurveto{\pgfqpoint{5.462406in}{1.278548in}}{\pgfqpoint{5.469248in}{1.275714in}}{\pgfqpoint{5.476381in}{1.275714in}}%
\pgfpathclose%
\pgfusepath{stroke,fill}%
\end{pgfscope}%
\begin{pgfscope}%
\pgfpathrectangle{\pgfqpoint{4.985294in}{0.500000in}}{\pgfqpoint{1.764706in}{1.700000in}}%
\pgfusepath{clip}%
\pgfsetbuttcap%
\pgfsetroundjoin%
\definecolor{currentfill}{rgb}{0.981377,0.920617,0.865369}%
\pgfsetfillcolor{currentfill}%
\pgfsetlinewidth{0.311001pt}%
\definecolor{currentstroke}{rgb}{1.000000,1.000000,1.000000}%
\pgfsetstrokecolor{currentstroke}%
\pgfsetdash{}{0pt}%
\pgfpathmoveto{\pgfqpoint{6.318860in}{1.410854in}}%
\pgfpathcurveto{\pgfqpoint{6.325993in}{1.410854in}}{\pgfqpoint{6.332835in}{1.413688in}}{\pgfqpoint{6.337878in}{1.418732in}}%
\pgfpathcurveto{\pgfqpoint{6.342922in}{1.423776in}}{\pgfqpoint{6.345756in}{1.430617in}}{\pgfqpoint{6.345756in}{1.437750in}}%
\pgfpathcurveto{\pgfqpoint{6.345756in}{1.444883in}}{\pgfqpoint{6.342922in}{1.451725in}}{\pgfqpoint{6.337878in}{1.456768in}}%
\pgfpathcurveto{\pgfqpoint{6.332835in}{1.461812in}}{\pgfqpoint{6.325993in}{1.464646in}}{\pgfqpoint{6.318860in}{1.464646in}}%
\pgfpathcurveto{\pgfqpoint{6.311727in}{1.464646in}}{\pgfqpoint{6.304886in}{1.461812in}}{\pgfqpoint{6.299842in}{1.456768in}}%
\pgfpathcurveto{\pgfqpoint{6.294798in}{1.451725in}}{\pgfqpoint{6.291964in}{1.444883in}}{\pgfqpoint{6.291964in}{1.437750in}}%
\pgfpathcurveto{\pgfqpoint{6.291964in}{1.430617in}}{\pgfqpoint{6.294798in}{1.423776in}}{\pgfqpoint{6.299842in}{1.418732in}}%
\pgfpathcurveto{\pgfqpoint{6.304886in}{1.413688in}}{\pgfqpoint{6.311727in}{1.410854in}}{\pgfqpoint{6.318860in}{1.410854in}}%
\pgfpathclose%
\pgfusepath{stroke,fill}%
\end{pgfscope}%
\begin{pgfscope}%
\pgfpathrectangle{\pgfqpoint{4.985294in}{0.500000in}}{\pgfqpoint{1.764706in}{1.700000in}}%
\pgfusepath{clip}%
\pgfsetbuttcap%
\pgfsetroundjoin%
\definecolor{currentfill}{rgb}{0.977657,0.891500,0.822809}%
\pgfsetfillcolor{currentfill}%
\pgfsetlinewidth{0.311001pt}%
\definecolor{currentstroke}{rgb}{1.000000,1.000000,1.000000}%
\pgfsetstrokecolor{currentstroke}%
\pgfsetdash{}{0pt}%
\pgfpathmoveto{\pgfqpoint{6.335456in}{1.437731in}}%
\pgfpathcurveto{\pgfqpoint{6.342589in}{1.437731in}}{\pgfqpoint{6.349431in}{1.440565in}}{\pgfqpoint{6.354474in}{1.445609in}}%
\pgfpathcurveto{\pgfqpoint{6.359518in}{1.450652in}}{\pgfqpoint{6.362352in}{1.457494in}}{\pgfqpoint{6.362352in}{1.464627in}}%
\pgfpathcurveto{\pgfqpoint{6.362352in}{1.471760in}}{\pgfqpoint{6.359518in}{1.478601in}}{\pgfqpoint{6.354474in}{1.483645in}}%
\pgfpathcurveto{\pgfqpoint{6.349431in}{1.488689in}}{\pgfqpoint{6.342589in}{1.491523in}}{\pgfqpoint{6.335456in}{1.491523in}}%
\pgfpathcurveto{\pgfqpoint{6.328323in}{1.491523in}}{\pgfqpoint{6.321482in}{1.488689in}}{\pgfqpoint{6.316438in}{1.483645in}}%
\pgfpathcurveto{\pgfqpoint{6.311394in}{1.478601in}}{\pgfqpoint{6.308561in}{1.471760in}}{\pgfqpoint{6.308561in}{1.464627in}}%
\pgfpathcurveto{\pgfqpoint{6.308561in}{1.457494in}}{\pgfqpoint{6.311394in}{1.450652in}}{\pgfqpoint{6.316438in}{1.445609in}}%
\pgfpathcurveto{\pgfqpoint{6.321482in}{1.440565in}}{\pgfqpoint{6.328323in}{1.437731in}}{\pgfqpoint{6.335456in}{1.437731in}}%
\pgfpathclose%
\pgfusepath{stroke,fill}%
\end{pgfscope}%
\begin{pgfscope}%
\pgfpathrectangle{\pgfqpoint{4.985294in}{0.500000in}}{\pgfqpoint{1.764706in}{1.700000in}}%
\pgfusepath{clip}%
\pgfsetbuttcap%
\pgfsetroundjoin%
\definecolor{currentfill}{rgb}{0.968105,0.786346,0.667739}%
\pgfsetfillcolor{currentfill}%
\pgfsetlinewidth{0.311001pt}%
\definecolor{currentstroke}{rgb}{1.000000,1.000000,1.000000}%
\pgfsetstrokecolor{currentstroke}%
\pgfsetdash{}{0pt}%
\pgfpathmoveto{\pgfqpoint{6.211499in}{1.471570in}}%
\pgfpathcurveto{\pgfqpoint{6.218632in}{1.471570in}}{\pgfqpoint{6.225474in}{1.474404in}}{\pgfqpoint{6.230518in}{1.479447in}}%
\pgfpathcurveto{\pgfqpoint{6.235561in}{1.484491in}}{\pgfqpoint{6.238395in}{1.491333in}}{\pgfqpoint{6.238395in}{1.498466in}}%
\pgfpathcurveto{\pgfqpoint{6.238395in}{1.505598in}}{\pgfqpoint{6.235561in}{1.512440in}}{\pgfqpoint{6.230518in}{1.517484in}}%
\pgfpathcurveto{\pgfqpoint{6.225474in}{1.522527in}}{\pgfqpoint{6.218632in}{1.525361in}}{\pgfqpoint{6.211499in}{1.525361in}}%
\pgfpathcurveto{\pgfqpoint{6.204367in}{1.525361in}}{\pgfqpoint{6.197525in}{1.522527in}}{\pgfqpoint{6.192481in}{1.517484in}}%
\pgfpathcurveto{\pgfqpoint{6.187438in}{1.512440in}}{\pgfqpoint{6.184604in}{1.505598in}}{\pgfqpoint{6.184604in}{1.498466in}}%
\pgfpathcurveto{\pgfqpoint{6.184604in}{1.491333in}}{\pgfqpoint{6.187438in}{1.484491in}}{\pgfqpoint{6.192481in}{1.479447in}}%
\pgfpathcurveto{\pgfqpoint{6.197525in}{1.474404in}}{\pgfqpoint{6.204367in}{1.471570in}}{\pgfqpoint{6.211499in}{1.471570in}}%
\pgfpathclose%
\pgfusepath{stroke,fill}%
\end{pgfscope}%
\begin{pgfscope}%
\pgfpathrectangle{\pgfqpoint{4.985294in}{0.500000in}}{\pgfqpoint{1.764706in}{1.700000in}}%
\pgfusepath{clip}%
\pgfsetbuttcap%
\pgfsetroundjoin%
\definecolor{currentfill}{rgb}{0.964920,0.695342,0.545192}%
\pgfsetfillcolor{currentfill}%
\pgfsetlinewidth{0.311001pt}%
\definecolor{currentstroke}{rgb}{1.000000,1.000000,1.000000}%
\pgfsetstrokecolor{currentstroke}%
\pgfsetdash{}{0pt}%
\pgfpathmoveto{\pgfqpoint{6.398922in}{1.438013in}}%
\pgfpathcurveto{\pgfqpoint{6.406055in}{1.438013in}}{\pgfqpoint{6.412896in}{1.440847in}}{\pgfqpoint{6.417940in}{1.445891in}}%
\pgfpathcurveto{\pgfqpoint{6.422984in}{1.450934in}}{\pgfqpoint{6.425818in}{1.457776in}}{\pgfqpoint{6.425818in}{1.464909in}}%
\pgfpathcurveto{\pgfqpoint{6.425818in}{1.472042in}}{\pgfqpoint{6.422984in}{1.478883in}}{\pgfqpoint{6.417940in}{1.483927in}}%
\pgfpathcurveto{\pgfqpoint{6.412896in}{1.488971in}}{\pgfqpoint{6.406055in}{1.491804in}}{\pgfqpoint{6.398922in}{1.491804in}}%
\pgfpathcurveto{\pgfqpoint{6.391789in}{1.491804in}}{\pgfqpoint{6.384948in}{1.488971in}}{\pgfqpoint{6.379904in}{1.483927in}}%
\pgfpathcurveto{\pgfqpoint{6.374860in}{1.478883in}}{\pgfqpoint{6.372026in}{1.472042in}}{\pgfqpoint{6.372026in}{1.464909in}}%
\pgfpathcurveto{\pgfqpoint{6.372026in}{1.457776in}}{\pgfqpoint{6.374860in}{1.450934in}}{\pgfqpoint{6.379904in}{1.445891in}}%
\pgfpathcurveto{\pgfqpoint{6.384948in}{1.440847in}}{\pgfqpoint{6.391789in}{1.438013in}}{\pgfqpoint{6.398922in}{1.438013in}}%
\pgfpathclose%
\pgfusepath{stroke,fill}%
\end{pgfscope}%
\begin{pgfscope}%
\pgfpathrectangle{\pgfqpoint{4.985294in}{0.500000in}}{\pgfqpoint{1.764706in}{1.700000in}}%
\pgfusepath{clip}%
\pgfsetbuttcap%
\pgfsetroundjoin%
\definecolor{currentfill}{rgb}{0.973832,0.856556,0.771584}%
\pgfsetfillcolor{currentfill}%
\pgfsetlinewidth{0.311001pt}%
\definecolor{currentstroke}{rgb}{1.000000,1.000000,1.000000}%
\pgfsetstrokecolor{currentstroke}%
\pgfsetdash{}{0pt}%
\pgfpathmoveto{\pgfqpoint{6.333044in}{1.500922in}}%
\pgfpathcurveto{\pgfqpoint{6.340177in}{1.500922in}}{\pgfqpoint{6.347019in}{1.503756in}}{\pgfqpoint{6.352062in}{1.508800in}}%
\pgfpathcurveto{\pgfqpoint{6.357106in}{1.513843in}}{\pgfqpoint{6.359940in}{1.520685in}}{\pgfqpoint{6.359940in}{1.527818in}}%
\pgfpathcurveto{\pgfqpoint{6.359940in}{1.534951in}}{\pgfqpoint{6.357106in}{1.541792in}}{\pgfqpoint{6.352062in}{1.546836in}}%
\pgfpathcurveto{\pgfqpoint{6.347019in}{1.551880in}}{\pgfqpoint{6.340177in}{1.554714in}}{\pgfqpoint{6.333044in}{1.554714in}}%
\pgfpathcurveto{\pgfqpoint{6.325911in}{1.554714in}}{\pgfqpoint{6.319070in}{1.551880in}}{\pgfqpoint{6.314026in}{1.546836in}}%
\pgfpathcurveto{\pgfqpoint{6.308982in}{1.541792in}}{\pgfqpoint{6.306149in}{1.534951in}}{\pgfqpoint{6.306149in}{1.527818in}}%
\pgfpathcurveto{\pgfqpoint{6.306149in}{1.520685in}}{\pgfqpoint{6.308982in}{1.513843in}}{\pgfqpoint{6.314026in}{1.508800in}}%
\pgfpathcurveto{\pgfqpoint{6.319070in}{1.503756in}}{\pgfqpoint{6.325911in}{1.500922in}}{\pgfqpoint{6.333044in}{1.500922in}}%
\pgfpathclose%
\pgfusepath{stroke,fill}%
\end{pgfscope}%
\begin{pgfscope}%
\pgfpathrectangle{\pgfqpoint{4.985294in}{0.500000in}}{\pgfqpoint{1.764706in}{1.700000in}}%
\pgfusepath{clip}%
\pgfsetbuttcap%
\pgfsetroundjoin%
\definecolor{currentfill}{rgb}{0.972726,0.844889,0.754401}%
\pgfsetfillcolor{currentfill}%
\pgfsetlinewidth{0.311001pt}%
\definecolor{currentstroke}{rgb}{1.000000,1.000000,1.000000}%
\pgfsetstrokecolor{currentstroke}%
\pgfsetdash{}{0pt}%
\pgfpathmoveto{\pgfqpoint{5.478959in}{1.265706in}}%
\pgfpathcurveto{\pgfqpoint{5.486092in}{1.265706in}}{\pgfqpoint{5.492933in}{1.268540in}}{\pgfqpoint{5.497977in}{1.273583in}}%
\pgfpathcurveto{\pgfqpoint{5.503021in}{1.278627in}}{\pgfqpoint{5.505855in}{1.285469in}}{\pgfqpoint{5.505855in}{1.292601in}}%
\pgfpathcurveto{\pgfqpoint{5.505855in}{1.299734in}}{\pgfqpoint{5.503021in}{1.306576in}}{\pgfqpoint{5.497977in}{1.311620in}}%
\pgfpathcurveto{\pgfqpoint{5.492933in}{1.316663in}}{\pgfqpoint{5.486092in}{1.319497in}}{\pgfqpoint{5.478959in}{1.319497in}}%
\pgfpathcurveto{\pgfqpoint{5.471826in}{1.319497in}}{\pgfqpoint{5.464984in}{1.316663in}}{\pgfqpoint{5.459941in}{1.311620in}}%
\pgfpathcurveto{\pgfqpoint{5.454897in}{1.306576in}}{\pgfqpoint{5.452063in}{1.299734in}}{\pgfqpoint{5.452063in}{1.292601in}}%
\pgfpathcurveto{\pgfqpoint{5.452063in}{1.285469in}}{\pgfqpoint{5.454897in}{1.278627in}}{\pgfqpoint{5.459941in}{1.273583in}}%
\pgfpathcurveto{\pgfqpoint{5.464984in}{1.268540in}}{\pgfqpoint{5.471826in}{1.265706in}}{\pgfqpoint{5.478959in}{1.265706in}}%
\pgfpathclose%
\pgfusepath{stroke,fill}%
\end{pgfscope}%
\begin{pgfscope}%
\pgfpathrectangle{\pgfqpoint{4.985294in}{0.500000in}}{\pgfqpoint{1.764706in}{1.700000in}}%
\pgfusepath{clip}%
\pgfsetbuttcap%
\pgfsetroundjoin%
\definecolor{currentfill}{rgb}{0.965928,0.738443,0.600540}%
\pgfsetfillcolor{currentfill}%
\pgfsetlinewidth{0.311001pt}%
\definecolor{currentstroke}{rgb}{1.000000,1.000000,1.000000}%
\pgfsetstrokecolor{currentstroke}%
\pgfsetdash{}{0pt}%
\pgfpathmoveto{\pgfqpoint{6.385436in}{1.161667in}}%
\pgfpathcurveto{\pgfqpoint{6.392569in}{1.161667in}}{\pgfqpoint{6.399410in}{1.164501in}}{\pgfqpoint{6.404454in}{1.169544in}}%
\pgfpathcurveto{\pgfqpoint{6.409498in}{1.174588in}}{\pgfqpoint{6.412332in}{1.181430in}}{\pgfqpoint{6.412332in}{1.188563in}}%
\pgfpathcurveto{\pgfqpoint{6.412332in}{1.195695in}}{\pgfqpoint{6.409498in}{1.202537in}}{\pgfqpoint{6.404454in}{1.207581in}}%
\pgfpathcurveto{\pgfqpoint{6.399410in}{1.212624in}}{\pgfqpoint{6.392569in}{1.215458in}}{\pgfqpoint{6.385436in}{1.215458in}}%
\pgfpathcurveto{\pgfqpoint{6.378303in}{1.215458in}}{\pgfqpoint{6.371461in}{1.212624in}}{\pgfqpoint{6.366418in}{1.207581in}}%
\pgfpathcurveto{\pgfqpoint{6.361374in}{1.202537in}}{\pgfqpoint{6.358540in}{1.195695in}}{\pgfqpoint{6.358540in}{1.188563in}}%
\pgfpathcurveto{\pgfqpoint{6.358540in}{1.181430in}}{\pgfqpoint{6.361374in}{1.174588in}}{\pgfqpoint{6.366418in}{1.169544in}}%
\pgfpathcurveto{\pgfqpoint{6.371461in}{1.164501in}}{\pgfqpoint{6.378303in}{1.161667in}}{\pgfqpoint{6.385436in}{1.161667in}}%
\pgfpathclose%
\pgfusepath{stroke,fill}%
\end{pgfscope}%
\begin{pgfscope}%
\pgfpathrectangle{\pgfqpoint{4.985294in}{0.500000in}}{\pgfqpoint{1.764706in}{1.700000in}}%
\pgfusepath{clip}%
\pgfsetbuttcap%
\pgfsetroundjoin%
\definecolor{currentfill}{rgb}{0.966812,0.762584,0.633643}%
\pgfsetfillcolor{currentfill}%
\pgfsetlinewidth{0.311001pt}%
\definecolor{currentstroke}{rgb}{1.000000,1.000000,1.000000}%
\pgfsetstrokecolor{currentstroke}%
\pgfsetdash{}{0pt}%
\pgfpathmoveto{\pgfqpoint{5.347179in}{1.422544in}}%
\pgfpathcurveto{\pgfqpoint{5.354312in}{1.422544in}}{\pgfqpoint{5.361154in}{1.425378in}}{\pgfqpoint{5.366198in}{1.430421in}}%
\pgfpathcurveto{\pgfqpoint{5.371241in}{1.435465in}}{\pgfqpoint{5.374075in}{1.442307in}}{\pgfqpoint{5.374075in}{1.449440in}}%
\pgfpathcurveto{\pgfqpoint{5.374075in}{1.456572in}}{\pgfqpoint{5.371241in}{1.463414in}}{\pgfqpoint{5.366198in}{1.468458in}}%
\pgfpathcurveto{\pgfqpoint{5.361154in}{1.473501in}}{\pgfqpoint{5.354312in}{1.476335in}}{\pgfqpoint{5.347179in}{1.476335in}}%
\pgfpathcurveto{\pgfqpoint{5.340047in}{1.476335in}}{\pgfqpoint{5.333205in}{1.473501in}}{\pgfqpoint{5.328161in}{1.468458in}}%
\pgfpathcurveto{\pgfqpoint{5.323118in}{1.463414in}}{\pgfqpoint{5.320284in}{1.456572in}}{\pgfqpoint{5.320284in}{1.449440in}}%
\pgfpathcurveto{\pgfqpoint{5.320284in}{1.442307in}}{\pgfqpoint{5.323118in}{1.435465in}}{\pgfqpoint{5.328161in}{1.430421in}}%
\pgfpathcurveto{\pgfqpoint{5.333205in}{1.425378in}}{\pgfqpoint{5.340047in}{1.422544in}}{\pgfqpoint{5.347179in}{1.422544in}}%
\pgfpathclose%
\pgfusepath{stroke,fill}%
\end{pgfscope}%
\begin{pgfscope}%
\pgfpathrectangle{\pgfqpoint{4.985294in}{0.500000in}}{\pgfqpoint{1.764706in}{1.700000in}}%
\pgfusepath{clip}%
\pgfsetbuttcap%
\pgfsetroundjoin%
\definecolor{currentfill}{rgb}{0.964558,0.676556,0.522514}%
\pgfsetfillcolor{currentfill}%
\pgfsetlinewidth{0.311001pt}%
\definecolor{currentstroke}{rgb}{1.000000,1.000000,1.000000}%
\pgfsetstrokecolor{currentstroke}%
\pgfsetdash{}{0pt}%
\pgfpathmoveto{\pgfqpoint{5.621362in}{0.992910in}}%
\pgfpathcurveto{\pgfqpoint{5.628495in}{0.992910in}}{\pgfqpoint{5.635337in}{0.995744in}}{\pgfqpoint{5.640380in}{1.000788in}}%
\pgfpathcurveto{\pgfqpoint{5.645424in}{1.005832in}}{\pgfqpoint{5.648258in}{1.012673in}}{\pgfqpoint{5.648258in}{1.019806in}}%
\pgfpathcurveto{\pgfqpoint{5.648258in}{1.026939in}}{\pgfqpoint{5.645424in}{1.033781in}}{\pgfqpoint{5.640380in}{1.038824in}}%
\pgfpathcurveto{\pgfqpoint{5.635337in}{1.043868in}}{\pgfqpoint{5.628495in}{1.046702in}}{\pgfqpoint{5.621362in}{1.046702in}}%
\pgfpathcurveto{\pgfqpoint{5.614229in}{1.046702in}}{\pgfqpoint{5.607388in}{1.043868in}}{\pgfqpoint{5.602344in}{1.038824in}}%
\pgfpathcurveto{\pgfqpoint{5.597300in}{1.033781in}}{\pgfqpoint{5.594467in}{1.026939in}}{\pgfqpoint{5.594467in}{1.019806in}}%
\pgfpathcurveto{\pgfqpoint{5.594467in}{1.012673in}}{\pgfqpoint{5.597300in}{1.005832in}}{\pgfqpoint{5.602344in}{1.000788in}}%
\pgfpathcurveto{\pgfqpoint{5.607388in}{0.995744in}}{\pgfqpoint{5.614229in}{0.992910in}}{\pgfqpoint{5.621362in}{0.992910in}}%
\pgfpathclose%
\pgfusepath{stroke,fill}%
\end{pgfscope}%
\begin{pgfscope}%
\pgfpathrectangle{\pgfqpoint{4.985294in}{0.500000in}}{\pgfqpoint{1.764706in}{1.700000in}}%
\pgfusepath{clip}%
\pgfsetbuttcap%
\pgfsetroundjoin%
\definecolor{currentfill}{rgb}{0.964799,0.689101,0.537560}%
\pgfsetfillcolor{currentfill}%
\pgfsetlinewidth{0.311001pt}%
\definecolor{currentstroke}{rgb}{1.000000,1.000000,1.000000}%
\pgfsetstrokecolor{currentstroke}%
\pgfsetdash{}{0pt}%
\pgfpathmoveto{\pgfqpoint{5.325266in}{1.395087in}}%
\pgfpathcurveto{\pgfqpoint{5.332399in}{1.395087in}}{\pgfqpoint{5.339240in}{1.397921in}}{\pgfqpoint{5.344284in}{1.402965in}}%
\pgfpathcurveto{\pgfqpoint{5.349328in}{1.408008in}}{\pgfqpoint{5.352161in}{1.414850in}}{\pgfqpoint{5.352161in}{1.421983in}}%
\pgfpathcurveto{\pgfqpoint{5.352161in}{1.429115in}}{\pgfqpoint{5.349328in}{1.435957in}}{\pgfqpoint{5.344284in}{1.441001in}}%
\pgfpathcurveto{\pgfqpoint{5.339240in}{1.446044in}}{\pgfqpoint{5.332399in}{1.448878in}}{\pgfqpoint{5.325266in}{1.448878in}}%
\pgfpathcurveto{\pgfqpoint{5.318133in}{1.448878in}}{\pgfqpoint{5.311291in}{1.446044in}}{\pgfqpoint{5.306248in}{1.441001in}}%
\pgfpathcurveto{\pgfqpoint{5.301204in}{1.435957in}}{\pgfqpoint{5.298370in}{1.429115in}}{\pgfqpoint{5.298370in}{1.421983in}}%
\pgfpathcurveto{\pgfqpoint{5.298370in}{1.414850in}}{\pgfqpoint{5.301204in}{1.408008in}}{\pgfqpoint{5.306248in}{1.402965in}}%
\pgfpathcurveto{\pgfqpoint{5.311291in}{1.397921in}}{\pgfqpoint{5.318133in}{1.395087in}}{\pgfqpoint{5.325266in}{1.395087in}}%
\pgfpathclose%
\pgfusepath{stroke,fill}%
\end{pgfscope}%
\begin{pgfscope}%
\pgfpathrectangle{\pgfqpoint{4.985294in}{0.500000in}}{\pgfqpoint{1.764706in}{1.700000in}}%
\pgfusepath{clip}%
\pgfsetbuttcap%
\pgfsetroundjoin%
\definecolor{currentfill}{rgb}{0.979124,0.903132,0.839793}%
\pgfsetfillcolor{currentfill}%
\pgfsetlinewidth{0.311001pt}%
\definecolor{currentstroke}{rgb}{1.000000,1.000000,1.000000}%
\pgfsetstrokecolor{currentstroke}%
\pgfsetdash{}{0pt}%
\pgfpathmoveto{\pgfqpoint{6.318381in}{1.483505in}}%
\pgfpathcurveto{\pgfqpoint{6.325514in}{1.483505in}}{\pgfqpoint{6.332356in}{1.486339in}}{\pgfqpoint{6.337399in}{1.491383in}}%
\pgfpathcurveto{\pgfqpoint{6.342443in}{1.496427in}}{\pgfqpoint{6.345277in}{1.503268in}}{\pgfqpoint{6.345277in}{1.510401in}}%
\pgfpathcurveto{\pgfqpoint{6.345277in}{1.517534in}}{\pgfqpoint{6.342443in}{1.524375in}}{\pgfqpoint{6.337399in}{1.529419in}}%
\pgfpathcurveto{\pgfqpoint{6.332356in}{1.534463in}}{\pgfqpoint{6.325514in}{1.537297in}}{\pgfqpoint{6.318381in}{1.537297in}}%
\pgfpathcurveto{\pgfqpoint{6.311248in}{1.537297in}}{\pgfqpoint{6.304407in}{1.534463in}}{\pgfqpoint{6.299363in}{1.529419in}}%
\pgfpathcurveto{\pgfqpoint{6.294319in}{1.524375in}}{\pgfqpoint{6.291485in}{1.517534in}}{\pgfqpoint{6.291485in}{1.510401in}}%
\pgfpathcurveto{\pgfqpoint{6.291485in}{1.503268in}}{\pgfqpoint{6.294319in}{1.496427in}}{\pgfqpoint{6.299363in}{1.491383in}}%
\pgfpathcurveto{\pgfqpoint{6.304407in}{1.486339in}}{\pgfqpoint{6.311248in}{1.483505in}}{\pgfqpoint{6.318381in}{1.483505in}}%
\pgfpathclose%
\pgfusepath{stroke,fill}%
\end{pgfscope}%
\begin{pgfscope}%
\pgfpathrectangle{\pgfqpoint{4.985294in}{0.500000in}}{\pgfqpoint{1.764706in}{1.700000in}}%
\pgfusepath{clip}%
\pgfsetbuttcap%
\pgfsetroundjoin%
\definecolor{currentfill}{rgb}{0.950017,0.427714,0.292447}%
\pgfsetfillcolor{currentfill}%
\pgfsetlinewidth{0.311001pt}%
\definecolor{currentstroke}{rgb}{1.000000,1.000000,1.000000}%
\pgfsetstrokecolor{currentstroke}%
\pgfsetdash{}{0pt}%
\pgfpathmoveto{\pgfqpoint{6.316808in}{1.717736in}}%
\pgfpathcurveto{\pgfqpoint{6.323941in}{1.717736in}}{\pgfqpoint{6.330782in}{1.720570in}}{\pgfqpoint{6.335826in}{1.725613in}}%
\pgfpathcurveto{\pgfqpoint{6.340870in}{1.730657in}}{\pgfqpoint{6.343704in}{1.737498in}}{\pgfqpoint{6.343704in}{1.744631in}}%
\pgfpathcurveto{\pgfqpoint{6.343704in}{1.751764in}}{\pgfqpoint{6.340870in}{1.758606in}}{\pgfqpoint{6.335826in}{1.763649in}}%
\pgfpathcurveto{\pgfqpoint{6.330782in}{1.768693in}}{\pgfqpoint{6.323941in}{1.771527in}}{\pgfqpoint{6.316808in}{1.771527in}}%
\pgfpathcurveto{\pgfqpoint{6.309675in}{1.771527in}}{\pgfqpoint{6.302833in}{1.768693in}}{\pgfqpoint{6.297790in}{1.763649in}}%
\pgfpathcurveto{\pgfqpoint{6.292746in}{1.758606in}}{\pgfqpoint{6.289912in}{1.751764in}}{\pgfqpoint{6.289912in}{1.744631in}}%
\pgfpathcurveto{\pgfqpoint{6.289912in}{1.737498in}}{\pgfqpoint{6.292746in}{1.730657in}}{\pgfqpoint{6.297790in}{1.725613in}}%
\pgfpathcurveto{\pgfqpoint{6.302833in}{1.720570in}}{\pgfqpoint{6.309675in}{1.717736in}}{\pgfqpoint{6.316808in}{1.717736in}}%
\pgfpathclose%
\pgfusepath{stroke,fill}%
\end{pgfscope}%
\begin{pgfscope}%
\pgfpathrectangle{\pgfqpoint{4.985294in}{0.500000in}}{\pgfqpoint{1.764706in}{1.700000in}}%
\pgfusepath{clip}%
\pgfsetbuttcap%
\pgfsetroundjoin%
\definecolor{currentfill}{rgb}{0.966120,0.744512,0.608720}%
\pgfsetfillcolor{currentfill}%
\pgfsetlinewidth{0.311001pt}%
\definecolor{currentstroke}{rgb}{1.000000,1.000000,1.000000}%
\pgfsetstrokecolor{currentstroke}%
\pgfsetdash{}{0pt}%
\pgfpathmoveto{\pgfqpoint{5.511973in}{1.318845in}}%
\pgfpathcurveto{\pgfqpoint{5.519106in}{1.318845in}}{\pgfqpoint{5.525947in}{1.321678in}}{\pgfqpoint{5.530991in}{1.326722in}}%
\pgfpathcurveto{\pgfqpoint{5.536035in}{1.331766in}}{\pgfqpoint{5.538869in}{1.338607in}}{\pgfqpoint{5.538869in}{1.345740in}}%
\pgfpathcurveto{\pgfqpoint{5.538869in}{1.352873in}}{\pgfqpoint{5.536035in}{1.359715in}}{\pgfqpoint{5.530991in}{1.364758in}}%
\pgfpathcurveto{\pgfqpoint{5.525947in}{1.369802in}}{\pgfqpoint{5.519106in}{1.372636in}}{\pgfqpoint{5.511973in}{1.372636in}}%
\pgfpathcurveto{\pgfqpoint{5.504840in}{1.372636in}}{\pgfqpoint{5.497998in}{1.369802in}}{\pgfqpoint{5.492955in}{1.364758in}}%
\pgfpathcurveto{\pgfqpoint{5.487911in}{1.359715in}}{\pgfqpoint{5.485077in}{1.352873in}}{\pgfqpoint{5.485077in}{1.345740in}}%
\pgfpathcurveto{\pgfqpoint{5.485077in}{1.338607in}}{\pgfqpoint{5.487911in}{1.331766in}}{\pgfqpoint{5.492955in}{1.326722in}}%
\pgfpathcurveto{\pgfqpoint{5.497998in}{1.321678in}}{\pgfqpoint{5.504840in}{1.318845in}}{\pgfqpoint{5.511973in}{1.318845in}}%
\pgfpathclose%
\pgfusepath{stroke,fill}%
\end{pgfscope}%
\begin{pgfscope}%
\pgfpathrectangle{\pgfqpoint{4.985294in}{0.500000in}}{\pgfqpoint{1.764706in}{1.700000in}}%
\pgfusepath{clip}%
\pgfsetbuttcap%
\pgfsetroundjoin%
\definecolor{currentfill}{rgb}{0.973832,0.856556,0.771584}%
\pgfsetfillcolor{currentfill}%
\pgfsetlinewidth{0.311001pt}%
\definecolor{currentstroke}{rgb}{1.000000,1.000000,1.000000}%
\pgfsetstrokecolor{currentstroke}%
\pgfsetdash{}{0pt}%
\pgfpathmoveto{\pgfqpoint{5.498245in}{1.528744in}}%
\pgfpathcurveto{\pgfqpoint{5.505378in}{1.528744in}}{\pgfqpoint{5.512220in}{1.531578in}}{\pgfqpoint{5.517263in}{1.536622in}}%
\pgfpathcurveto{\pgfqpoint{5.522307in}{1.541665in}}{\pgfqpoint{5.525141in}{1.548507in}}{\pgfqpoint{5.525141in}{1.555640in}}%
\pgfpathcurveto{\pgfqpoint{5.525141in}{1.562773in}}{\pgfqpoint{5.522307in}{1.569614in}}{\pgfqpoint{5.517263in}{1.574658in}}%
\pgfpathcurveto{\pgfqpoint{5.512220in}{1.579702in}}{\pgfqpoint{5.505378in}{1.582536in}}{\pgfqpoint{5.498245in}{1.582536in}}%
\pgfpathcurveto{\pgfqpoint{5.491112in}{1.582536in}}{\pgfqpoint{5.484271in}{1.579702in}}{\pgfqpoint{5.479227in}{1.574658in}}%
\pgfpathcurveto{\pgfqpoint{5.474183in}{1.569614in}}{\pgfqpoint{5.471349in}{1.562773in}}{\pgfqpoint{5.471349in}{1.555640in}}%
\pgfpathcurveto{\pgfqpoint{5.471349in}{1.548507in}}{\pgfqpoint{5.474183in}{1.541665in}}{\pgfqpoint{5.479227in}{1.536622in}}%
\pgfpathcurveto{\pgfqpoint{5.484271in}{1.531578in}}{\pgfqpoint{5.491112in}{1.528744in}}{\pgfqpoint{5.498245in}{1.528744in}}%
\pgfpathclose%
\pgfusepath{stroke,fill}%
\end{pgfscope}%
\begin{pgfscope}%
\pgfpathrectangle{\pgfqpoint{4.985294in}{0.500000in}}{\pgfqpoint{1.764706in}{1.700000in}}%
\pgfusepath{clip}%
\pgfsetbuttcap%
\pgfsetroundjoin%
\definecolor{currentfill}{rgb}{0.973832,0.856556,0.771584}%
\pgfsetfillcolor{currentfill}%
\pgfsetlinewidth{0.311001pt}%
\definecolor{currentstroke}{rgb}{1.000000,1.000000,1.000000}%
\pgfsetstrokecolor{currentstroke}%
\pgfsetdash{}{0pt}%
\pgfpathmoveto{\pgfqpoint{6.360819in}{1.219298in}}%
\pgfpathcurveto{\pgfqpoint{6.367952in}{1.219298in}}{\pgfqpoint{6.374794in}{1.222132in}}{\pgfqpoint{6.379838in}{1.227175in}}%
\pgfpathcurveto{\pgfqpoint{6.384881in}{1.232219in}}{\pgfqpoint{6.387715in}{1.239061in}}{\pgfqpoint{6.387715in}{1.246194in}}%
\pgfpathcurveto{\pgfqpoint{6.387715in}{1.253326in}}{\pgfqpoint{6.384881in}{1.260168in}}{\pgfqpoint{6.379838in}{1.265212in}}%
\pgfpathcurveto{\pgfqpoint{6.374794in}{1.270255in}}{\pgfqpoint{6.367952in}{1.273089in}}{\pgfqpoint{6.360819in}{1.273089in}}%
\pgfpathcurveto{\pgfqpoint{6.353687in}{1.273089in}}{\pgfqpoint{6.346845in}{1.270255in}}{\pgfqpoint{6.341801in}{1.265212in}}%
\pgfpathcurveto{\pgfqpoint{6.336758in}{1.260168in}}{\pgfqpoint{6.333924in}{1.253326in}}{\pgfqpoint{6.333924in}{1.246194in}}%
\pgfpathcurveto{\pgfqpoint{6.333924in}{1.239061in}}{\pgfqpoint{6.336758in}{1.232219in}}{\pgfqpoint{6.341801in}{1.227175in}}%
\pgfpathcurveto{\pgfqpoint{6.346845in}{1.222132in}}{\pgfqpoint{6.353687in}{1.219298in}}{\pgfqpoint{6.360819in}{1.219298in}}%
\pgfpathclose%
\pgfusepath{stroke,fill}%
\end{pgfscope}%
\begin{pgfscope}%
\pgfpathrectangle{\pgfqpoint{4.985294in}{0.500000in}}{\pgfqpoint{1.764706in}{1.700000in}}%
\pgfusepath{clip}%
\pgfsetbuttcap%
\pgfsetroundjoin%
\definecolor{currentfill}{rgb}{0.930781,0.313740,0.244688}%
\pgfsetfillcolor{currentfill}%
\pgfsetlinewidth{0.311001pt}%
\definecolor{currentstroke}{rgb}{1.000000,1.000000,1.000000}%
\pgfsetstrokecolor{currentstroke}%
\pgfsetdash{}{0pt}%
\pgfpathmoveto{\pgfqpoint{6.467640in}{1.279445in}}%
\pgfpathcurveto{\pgfqpoint{6.474773in}{1.279445in}}{\pgfqpoint{6.481614in}{1.282279in}}{\pgfqpoint{6.486658in}{1.287323in}}%
\pgfpathcurveto{\pgfqpoint{6.491702in}{1.292367in}}{\pgfqpoint{6.494536in}{1.299208in}}{\pgfqpoint{6.494536in}{1.306341in}}%
\pgfpathcurveto{\pgfqpoint{6.494536in}{1.313474in}}{\pgfqpoint{6.491702in}{1.320316in}}{\pgfqpoint{6.486658in}{1.325359in}}%
\pgfpathcurveto{\pgfqpoint{6.481614in}{1.330403in}}{\pgfqpoint{6.474773in}{1.333237in}}{\pgfqpoint{6.467640in}{1.333237in}}%
\pgfpathcurveto{\pgfqpoint{6.460507in}{1.333237in}}{\pgfqpoint{6.453665in}{1.330403in}}{\pgfqpoint{6.448622in}{1.325359in}}%
\pgfpathcurveto{\pgfqpoint{6.443578in}{1.320316in}}{\pgfqpoint{6.440744in}{1.313474in}}{\pgfqpoint{6.440744in}{1.306341in}}%
\pgfpathcurveto{\pgfqpoint{6.440744in}{1.299208in}}{\pgfqpoint{6.443578in}{1.292367in}}{\pgfqpoint{6.448622in}{1.287323in}}%
\pgfpathcurveto{\pgfqpoint{6.453665in}{1.282279in}}{\pgfqpoint{6.460507in}{1.279445in}}{\pgfqpoint{6.467640in}{1.279445in}}%
\pgfpathclose%
\pgfusepath{stroke,fill}%
\end{pgfscope}%
\begin{pgfscope}%
\pgfpathrectangle{\pgfqpoint{4.985294in}{0.500000in}}{\pgfqpoint{1.764706in}{1.700000in}}%
\pgfusepath{clip}%
\pgfsetbuttcap%
\pgfsetroundjoin%
\definecolor{currentfill}{rgb}{0.963559,0.632016,0.472047}%
\pgfsetfillcolor{currentfill}%
\pgfsetlinewidth{0.311001pt}%
\definecolor{currentstroke}{rgb}{1.000000,1.000000,1.000000}%
\pgfsetstrokecolor{currentstroke}%
\pgfsetdash{}{0pt}%
\pgfpathmoveto{\pgfqpoint{5.308669in}{1.325469in}}%
\pgfpathcurveto{\pgfqpoint{5.315802in}{1.325469in}}{\pgfqpoint{5.322643in}{1.328303in}}{\pgfqpoint{5.327687in}{1.333347in}}%
\pgfpathcurveto{\pgfqpoint{5.332731in}{1.338390in}}{\pgfqpoint{5.335565in}{1.345232in}}{\pgfqpoint{5.335565in}{1.352365in}}%
\pgfpathcurveto{\pgfqpoint{5.335565in}{1.359498in}}{\pgfqpoint{5.332731in}{1.366339in}}{\pgfqpoint{5.327687in}{1.371383in}}%
\pgfpathcurveto{\pgfqpoint{5.322643in}{1.376427in}}{\pgfqpoint{5.315802in}{1.379260in}}{\pgfqpoint{5.308669in}{1.379260in}}%
\pgfpathcurveto{\pgfqpoint{5.301536in}{1.379260in}}{\pgfqpoint{5.294695in}{1.376427in}}{\pgfqpoint{5.289651in}{1.371383in}}%
\pgfpathcurveto{\pgfqpoint{5.284607in}{1.366339in}}{\pgfqpoint{5.281773in}{1.359498in}}{\pgfqpoint{5.281773in}{1.352365in}}%
\pgfpathcurveto{\pgfqpoint{5.281773in}{1.345232in}}{\pgfqpoint{5.284607in}{1.338390in}}{\pgfqpoint{5.289651in}{1.333347in}}%
\pgfpathcurveto{\pgfqpoint{5.294695in}{1.328303in}}{\pgfqpoint{5.301536in}{1.325469in}}{\pgfqpoint{5.308669in}{1.325469in}}%
\pgfpathclose%
\pgfusepath{stroke,fill}%
\end{pgfscope}%
\begin{pgfscope}%
\pgfpathrectangle{\pgfqpoint{4.985294in}{0.500000in}}{\pgfqpoint{1.764706in}{1.700000in}}%
\pgfusepath{clip}%
\pgfsetbuttcap%
\pgfsetroundjoin%
\definecolor{currentfill}{rgb}{0.968931,0.798091,0.685123}%
\pgfsetfillcolor{currentfill}%
\pgfsetlinewidth{0.311001pt}%
\definecolor{currentstroke}{rgb}{1.000000,1.000000,1.000000}%
\pgfsetstrokecolor{currentstroke}%
\pgfsetdash{}{0pt}%
\pgfpathmoveto{\pgfqpoint{6.296695in}{1.629220in}}%
\pgfpathcurveto{\pgfqpoint{6.303827in}{1.629220in}}{\pgfqpoint{6.310669in}{1.632054in}}{\pgfqpoint{6.315713in}{1.637098in}}%
\pgfpathcurveto{\pgfqpoint{6.320756in}{1.642141in}}{\pgfqpoint{6.323590in}{1.648983in}}{\pgfqpoint{6.323590in}{1.656116in}}%
\pgfpathcurveto{\pgfqpoint{6.323590in}{1.663249in}}{\pgfqpoint{6.320756in}{1.670090in}}{\pgfqpoint{6.315713in}{1.675134in}}%
\pgfpathcurveto{\pgfqpoint{6.310669in}{1.680177in}}{\pgfqpoint{6.303827in}{1.683011in}}{\pgfqpoint{6.296695in}{1.683011in}}%
\pgfpathcurveto{\pgfqpoint{6.289562in}{1.683011in}}{\pgfqpoint{6.282720in}{1.680177in}}{\pgfqpoint{6.277676in}{1.675134in}}%
\pgfpathcurveto{\pgfqpoint{6.272633in}{1.670090in}}{\pgfqpoint{6.269799in}{1.663249in}}{\pgfqpoint{6.269799in}{1.656116in}}%
\pgfpathcurveto{\pgfqpoint{6.269799in}{1.648983in}}{\pgfqpoint{6.272633in}{1.642141in}}{\pgfqpoint{6.277676in}{1.637098in}}%
\pgfpathcurveto{\pgfqpoint{6.282720in}{1.632054in}}{\pgfqpoint{6.289562in}{1.629220in}}{\pgfqpoint{6.296695in}{1.629220in}}%
\pgfpathclose%
\pgfusepath{stroke,fill}%
\end{pgfscope}%
\begin{pgfscope}%
\pgfpathrectangle{\pgfqpoint{4.985294in}{0.500000in}}{\pgfqpoint{1.764706in}{1.700000in}}%
\pgfusepath{clip}%
\pgfsetbuttcap%
\pgfsetroundjoin%
\definecolor{currentfill}{rgb}{0.968105,0.786346,0.667739}%
\pgfsetfillcolor{currentfill}%
\pgfsetlinewidth{0.311001pt}%
\definecolor{currentstroke}{rgb}{1.000000,1.000000,1.000000}%
\pgfsetstrokecolor{currentstroke}%
\pgfsetdash{}{0pt}%
\pgfpathmoveto{\pgfqpoint{5.563734in}{1.023784in}}%
\pgfpathcurveto{\pgfqpoint{5.570867in}{1.023784in}}{\pgfqpoint{5.577708in}{1.026618in}}{\pgfqpoint{5.582752in}{1.031661in}}%
\pgfpathcurveto{\pgfqpoint{5.587796in}{1.036705in}}{\pgfqpoint{5.590630in}{1.043547in}}{\pgfqpoint{5.590630in}{1.050679in}}%
\pgfpathcurveto{\pgfqpoint{5.590630in}{1.057812in}}{\pgfqpoint{5.587796in}{1.064654in}}{\pgfqpoint{5.582752in}{1.069698in}}%
\pgfpathcurveto{\pgfqpoint{5.577708in}{1.074741in}}{\pgfqpoint{5.570867in}{1.077575in}}{\pgfqpoint{5.563734in}{1.077575in}}%
\pgfpathcurveto{\pgfqpoint{5.556601in}{1.077575in}}{\pgfqpoint{5.549759in}{1.074741in}}{\pgfqpoint{5.544716in}{1.069698in}}%
\pgfpathcurveto{\pgfqpoint{5.539672in}{1.064654in}}{\pgfqpoint{5.536838in}{1.057812in}}{\pgfqpoint{5.536838in}{1.050679in}}%
\pgfpathcurveto{\pgfqpoint{5.536838in}{1.043547in}}{\pgfqpoint{5.539672in}{1.036705in}}{\pgfqpoint{5.544716in}{1.031661in}}%
\pgfpathcurveto{\pgfqpoint{5.549759in}{1.026618in}}{\pgfqpoint{5.556601in}{1.023784in}}{\pgfqpoint{5.563734in}{1.023784in}}%
\pgfpathclose%
\pgfusepath{stroke,fill}%
\end{pgfscope}%
\begin{pgfscope}%
\pgfpathrectangle{\pgfqpoint{4.985294in}{0.500000in}}{\pgfqpoint{1.764706in}{1.700000in}}%
\pgfusepath{clip}%
\pgfsetbuttcap%
\pgfsetroundjoin%
\definecolor{currentfill}{rgb}{0.938993,0.352507,0.254528}%
\pgfsetfillcolor{currentfill}%
\pgfsetlinewidth{0.311001pt}%
\definecolor{currentstroke}{rgb}{1.000000,1.000000,1.000000}%
\pgfsetstrokecolor{currentstroke}%
\pgfsetdash{}{0pt}%
\pgfpathmoveto{\pgfqpoint{5.624923in}{1.129501in}}%
\pgfpathcurveto{\pgfqpoint{5.632056in}{1.129501in}}{\pgfqpoint{5.638898in}{1.132335in}}{\pgfqpoint{5.643941in}{1.137378in}}%
\pgfpathcurveto{\pgfqpoint{5.648985in}{1.142422in}}{\pgfqpoint{5.651819in}{1.149264in}}{\pgfqpoint{5.651819in}{1.156397in}}%
\pgfpathcurveto{\pgfqpoint{5.651819in}{1.163529in}}{\pgfqpoint{5.648985in}{1.170371in}}{\pgfqpoint{5.643941in}{1.175415in}}%
\pgfpathcurveto{\pgfqpoint{5.638898in}{1.180458in}}{\pgfqpoint{5.632056in}{1.183292in}}{\pgfqpoint{5.624923in}{1.183292in}}%
\pgfpathcurveto{\pgfqpoint{5.617790in}{1.183292in}}{\pgfqpoint{5.610949in}{1.180458in}}{\pgfqpoint{5.605905in}{1.175415in}}%
\pgfpathcurveto{\pgfqpoint{5.600861in}{1.170371in}}{\pgfqpoint{5.598028in}{1.163529in}}{\pgfqpoint{5.598028in}{1.156397in}}%
\pgfpathcurveto{\pgfqpoint{5.598028in}{1.149264in}}{\pgfqpoint{5.600861in}{1.142422in}}{\pgfqpoint{5.605905in}{1.137378in}}%
\pgfpathcurveto{\pgfqpoint{5.610949in}{1.132335in}}{\pgfqpoint{5.617790in}{1.129501in}}{\pgfqpoint{5.624923in}{1.129501in}}%
\pgfpathclose%
\pgfusepath{stroke,fill}%
\end{pgfscope}%
\begin{pgfscope}%
\pgfpathrectangle{\pgfqpoint{4.985294in}{0.500000in}}{\pgfqpoint{1.764706in}{1.700000in}}%
\pgfusepath{clip}%
\pgfsetbuttcap%
\pgfsetroundjoin%
\definecolor{currentfill}{rgb}{0.979891,0.908948,0.848279}%
\pgfsetfillcolor{currentfill}%
\pgfsetlinewidth{0.311001pt}%
\definecolor{currentstroke}{rgb}{1.000000,1.000000,1.000000}%
\pgfsetstrokecolor{currentstroke}%
\pgfsetdash{}{0pt}%
\pgfpathmoveto{\pgfqpoint{5.421395in}{1.414688in}}%
\pgfpathcurveto{\pgfqpoint{5.428528in}{1.414688in}}{\pgfqpoint{5.435370in}{1.417522in}}{\pgfqpoint{5.440413in}{1.422566in}}%
\pgfpathcurveto{\pgfqpoint{5.445457in}{1.427610in}}{\pgfqpoint{5.448291in}{1.434451in}}{\pgfqpoint{5.448291in}{1.441584in}}%
\pgfpathcurveto{\pgfqpoint{5.448291in}{1.448717in}}{\pgfqpoint{5.445457in}{1.455559in}}{\pgfqpoint{5.440413in}{1.460602in}}%
\pgfpathcurveto{\pgfqpoint{5.435370in}{1.465646in}}{\pgfqpoint{5.428528in}{1.468480in}}{\pgfqpoint{5.421395in}{1.468480in}}%
\pgfpathcurveto{\pgfqpoint{5.414262in}{1.468480in}}{\pgfqpoint{5.407421in}{1.465646in}}{\pgfqpoint{5.402377in}{1.460602in}}%
\pgfpathcurveto{\pgfqpoint{5.397333in}{1.455559in}}{\pgfqpoint{5.394499in}{1.448717in}}{\pgfqpoint{5.394499in}{1.441584in}}%
\pgfpathcurveto{\pgfqpoint{5.394499in}{1.434451in}}{\pgfqpoint{5.397333in}{1.427610in}}{\pgfqpoint{5.402377in}{1.422566in}}%
\pgfpathcurveto{\pgfqpoint{5.407421in}{1.417522in}}{\pgfqpoint{5.414262in}{1.414688in}}{\pgfqpoint{5.421395in}{1.414688in}}%
\pgfpathclose%
\pgfusepath{stroke,fill}%
\end{pgfscope}%
\begin{pgfscope}%
\pgfpathrectangle{\pgfqpoint{4.985294in}{0.500000in}}{\pgfqpoint{1.764706in}{1.700000in}}%
\pgfusepath{clip}%
\pgfsetbuttcap%
\pgfsetroundjoin%
\definecolor{currentfill}{rgb}{0.967735,0.780441,0.659127}%
\pgfsetfillcolor{currentfill}%
\pgfsetlinewidth{0.311001pt}%
\definecolor{currentstroke}{rgb}{1.000000,1.000000,1.000000}%
\pgfsetstrokecolor{currentstroke}%
\pgfsetdash{}{0pt}%
\pgfpathmoveto{\pgfqpoint{6.391160in}{1.357018in}}%
\pgfpathcurveto{\pgfqpoint{6.398293in}{1.357018in}}{\pgfqpoint{6.405134in}{1.359851in}}{\pgfqpoint{6.410178in}{1.364895in}}%
\pgfpathcurveto{\pgfqpoint{6.415222in}{1.369939in}}{\pgfqpoint{6.418055in}{1.376780in}}{\pgfqpoint{6.418055in}{1.383913in}}%
\pgfpathcurveto{\pgfqpoint{6.418055in}{1.391046in}}{\pgfqpoint{6.415222in}{1.397888in}}{\pgfqpoint{6.410178in}{1.402931in}}%
\pgfpathcurveto{\pgfqpoint{6.405134in}{1.407975in}}{\pgfqpoint{6.398293in}{1.410809in}}{\pgfqpoint{6.391160in}{1.410809in}}%
\pgfpathcurveto{\pgfqpoint{6.384027in}{1.410809in}}{\pgfqpoint{6.377185in}{1.407975in}}{\pgfqpoint{6.372142in}{1.402931in}}%
\pgfpathcurveto{\pgfqpoint{6.367098in}{1.397888in}}{\pgfqpoint{6.364264in}{1.391046in}}{\pgfqpoint{6.364264in}{1.383913in}}%
\pgfpathcurveto{\pgfqpoint{6.364264in}{1.376780in}}{\pgfqpoint{6.367098in}{1.369939in}}{\pgfqpoint{6.372142in}{1.364895in}}%
\pgfpathcurveto{\pgfqpoint{6.377185in}{1.359851in}}{\pgfqpoint{6.384027in}{1.357018in}}{\pgfqpoint{6.391160in}{1.357018in}}%
\pgfpathclose%
\pgfusepath{stroke,fill}%
\end{pgfscope}%
\begin{pgfscope}%
\pgfpathrectangle{\pgfqpoint{4.985294in}{0.500000in}}{\pgfqpoint{1.764706in}{1.700000in}}%
\pgfusepath{clip}%
\pgfsetbuttcap%
\pgfsetroundjoin%
\definecolor{currentfill}{rgb}{0.977657,0.891500,0.822809}%
\pgfsetfillcolor{currentfill}%
\pgfsetlinewidth{0.311001pt}%
\definecolor{currentstroke}{rgb}{1.000000,1.000000,1.000000}%
\pgfsetstrokecolor{currentstroke}%
\pgfsetdash{}{0pt}%
\pgfpathmoveto{\pgfqpoint{5.451648in}{1.248358in}}%
\pgfpathcurveto{\pgfqpoint{5.458781in}{1.248358in}}{\pgfqpoint{5.465622in}{1.251192in}}{\pgfqpoint{5.470666in}{1.256236in}}%
\pgfpathcurveto{\pgfqpoint{5.475710in}{1.261280in}}{\pgfqpoint{5.478543in}{1.268121in}}{\pgfqpoint{5.478543in}{1.275254in}}%
\pgfpathcurveto{\pgfqpoint{5.478543in}{1.282387in}}{\pgfqpoint{5.475710in}{1.289229in}}{\pgfqpoint{5.470666in}{1.294272in}}%
\pgfpathcurveto{\pgfqpoint{5.465622in}{1.299316in}}{\pgfqpoint{5.458781in}{1.302150in}}{\pgfqpoint{5.451648in}{1.302150in}}%
\pgfpathcurveto{\pgfqpoint{5.444515in}{1.302150in}}{\pgfqpoint{5.437673in}{1.299316in}}{\pgfqpoint{5.432630in}{1.294272in}}%
\pgfpathcurveto{\pgfqpoint{5.427586in}{1.289229in}}{\pgfqpoint{5.424752in}{1.282387in}}{\pgfqpoint{5.424752in}{1.275254in}}%
\pgfpathcurveto{\pgfqpoint{5.424752in}{1.268121in}}{\pgfqpoint{5.427586in}{1.261280in}}{\pgfqpoint{5.432630in}{1.256236in}}%
\pgfpathcurveto{\pgfqpoint{5.437673in}{1.251192in}}{\pgfqpoint{5.444515in}{1.248358in}}{\pgfqpoint{5.451648in}{1.248358in}}%
\pgfpathclose%
\pgfusepath{stroke,fill}%
\end{pgfscope}%
\begin{pgfscope}%
\pgfpathrectangle{\pgfqpoint{4.985294in}{0.500000in}}{\pgfqpoint{1.764706in}{1.700000in}}%
\pgfusepath{clip}%
\pgfsetbuttcap%
\pgfsetroundjoin%
\definecolor{currentfill}{rgb}{0.973271,0.850724,0.762998}%
\pgfsetfillcolor{currentfill}%
\pgfsetlinewidth{0.311001pt}%
\definecolor{currentstroke}{rgb}{1.000000,1.000000,1.000000}%
\pgfsetstrokecolor{currentstroke}%
\pgfsetdash{}{0pt}%
\pgfpathmoveto{\pgfqpoint{5.498946in}{1.015215in}}%
\pgfpathcurveto{\pgfqpoint{5.506079in}{1.015215in}}{\pgfqpoint{5.512921in}{1.018048in}}{\pgfqpoint{5.517965in}{1.023092in}}%
\pgfpathcurveto{\pgfqpoint{5.523008in}{1.028136in}}{\pgfqpoint{5.525842in}{1.034977in}}{\pgfqpoint{5.525842in}{1.042110in}}%
\pgfpathcurveto{\pgfqpoint{5.525842in}{1.049243in}}{\pgfqpoint{5.523008in}{1.056085in}}{\pgfqpoint{5.517965in}{1.061128in}}%
\pgfpathcurveto{\pgfqpoint{5.512921in}{1.066172in}}{\pgfqpoint{5.506079in}{1.069006in}}{\pgfqpoint{5.498946in}{1.069006in}}%
\pgfpathcurveto{\pgfqpoint{5.491814in}{1.069006in}}{\pgfqpoint{5.484972in}{1.066172in}}{\pgfqpoint{5.479928in}{1.061128in}}%
\pgfpathcurveto{\pgfqpoint{5.474885in}{1.056085in}}{\pgfqpoint{5.472051in}{1.049243in}}{\pgfqpoint{5.472051in}{1.042110in}}%
\pgfpathcurveto{\pgfqpoint{5.472051in}{1.034977in}}{\pgfqpoint{5.474885in}{1.028136in}}{\pgfqpoint{5.479928in}{1.023092in}}%
\pgfpathcurveto{\pgfqpoint{5.484972in}{1.018048in}}{\pgfqpoint{5.491814in}{1.015215in}}{\pgfqpoint{5.498946in}{1.015215in}}%
\pgfpathclose%
\pgfusepath{stroke,fill}%
\end{pgfscope}%
\begin{pgfscope}%
\pgfpathrectangle{\pgfqpoint{4.985294in}{0.500000in}}{\pgfqpoint{1.764706in}{1.700000in}}%
\pgfusepath{clip}%
\pgfsetbuttcap%
\pgfsetroundjoin%
\definecolor{currentfill}{rgb}{0.972201,0.839051,0.745789}%
\pgfsetfillcolor{currentfill}%
\pgfsetlinewidth{0.311001pt}%
\definecolor{currentstroke}{rgb}{1.000000,1.000000,1.000000}%
\pgfsetstrokecolor{currentstroke}%
\pgfsetdash{}{0pt}%
\pgfpathmoveto{\pgfqpoint{5.519295in}{1.568614in}}%
\pgfpathcurveto{\pgfqpoint{5.526428in}{1.568614in}}{\pgfqpoint{5.533269in}{1.571448in}}{\pgfqpoint{5.538313in}{1.576492in}}%
\pgfpathcurveto{\pgfqpoint{5.543357in}{1.581536in}}{\pgfqpoint{5.546190in}{1.588377in}}{\pgfqpoint{5.546190in}{1.595510in}}%
\pgfpathcurveto{\pgfqpoint{5.546190in}{1.602643in}}{\pgfqpoint{5.543357in}{1.609484in}}{\pgfqpoint{5.538313in}{1.614528in}}%
\pgfpathcurveto{\pgfqpoint{5.533269in}{1.619572in}}{\pgfqpoint{5.526428in}{1.622406in}}{\pgfqpoint{5.519295in}{1.622406in}}%
\pgfpathcurveto{\pgfqpoint{5.512162in}{1.622406in}}{\pgfqpoint{5.505320in}{1.619572in}}{\pgfqpoint{5.500277in}{1.614528in}}%
\pgfpathcurveto{\pgfqpoint{5.495233in}{1.609484in}}{\pgfqpoint{5.492399in}{1.602643in}}{\pgfqpoint{5.492399in}{1.595510in}}%
\pgfpathcurveto{\pgfqpoint{5.492399in}{1.588377in}}{\pgfqpoint{5.495233in}{1.581536in}}{\pgfqpoint{5.500277in}{1.576492in}}%
\pgfpathcurveto{\pgfqpoint{5.505320in}{1.571448in}}{\pgfqpoint{5.512162in}{1.568614in}}{\pgfqpoint{5.519295in}{1.568614in}}%
\pgfpathclose%
\pgfusepath{stroke,fill}%
\end{pgfscope}%
\begin{pgfscope}%
\pgfpathrectangle{\pgfqpoint{4.985294in}{0.500000in}}{\pgfqpoint{1.764706in}{1.700000in}}%
\pgfusepath{clip}%
\pgfsetbuttcap%
\pgfsetroundjoin%
\definecolor{currentfill}{rgb}{0.954476,0.470822,0.323110}%
\pgfsetfillcolor{currentfill}%
\pgfsetlinewidth{0.311001pt}%
\definecolor{currentstroke}{rgb}{1.000000,1.000000,1.000000}%
\pgfsetstrokecolor{currentstroke}%
\pgfsetdash{}{0pt}%
\pgfpathmoveto{\pgfqpoint{5.658225in}{0.864146in}}%
\pgfpathcurveto{\pgfqpoint{5.665357in}{0.864146in}}{\pgfqpoint{5.672199in}{0.866980in}}{\pgfqpoint{5.677243in}{0.872023in}}%
\pgfpathcurveto{\pgfqpoint{5.682286in}{0.877067in}}{\pgfqpoint{5.685120in}{0.883908in}}{\pgfqpoint{5.685120in}{0.891041in}}%
\pgfpathcurveto{\pgfqpoint{5.685120in}{0.898174in}}{\pgfqpoint{5.682286in}{0.905016in}}{\pgfqpoint{5.677243in}{0.910059in}}%
\pgfpathcurveto{\pgfqpoint{5.672199in}{0.915103in}}{\pgfqpoint{5.665357in}{0.917937in}}{\pgfqpoint{5.658225in}{0.917937in}}%
\pgfpathcurveto{\pgfqpoint{5.651092in}{0.917937in}}{\pgfqpoint{5.644250in}{0.915103in}}{\pgfqpoint{5.639207in}{0.910059in}}%
\pgfpathcurveto{\pgfqpoint{5.634163in}{0.905016in}}{\pgfqpoint{5.631329in}{0.898174in}}{\pgfqpoint{5.631329in}{0.891041in}}%
\pgfpathcurveto{\pgfqpoint{5.631329in}{0.883908in}}{\pgfqpoint{5.634163in}{0.877067in}}{\pgfqpoint{5.639207in}{0.872023in}}%
\pgfpathcurveto{\pgfqpoint{5.644250in}{0.866980in}}{\pgfqpoint{5.651092in}{0.864146in}}{\pgfqpoint{5.658225in}{0.864146in}}%
\pgfpathclose%
\pgfusepath{stroke,fill}%
\end{pgfscope}%
\begin{pgfscope}%
\pgfpathrectangle{\pgfqpoint{4.985294in}{0.500000in}}{\pgfqpoint{1.764706in}{1.700000in}}%
\pgfusepath{clip}%
\pgfsetbuttcap%
\pgfsetroundjoin%
\definecolor{currentfill}{rgb}{0.962532,0.599594,0.438051}%
\pgfsetfillcolor{currentfill}%
\pgfsetlinewidth{0.311001pt}%
\definecolor{currentstroke}{rgb}{1.000000,1.000000,1.000000}%
\pgfsetstrokecolor{currentstroke}%
\pgfsetdash{}{0pt}%
\pgfpathmoveto{\pgfqpoint{6.103039in}{1.735014in}}%
\pgfpathcurveto{\pgfqpoint{6.110171in}{1.735014in}}{\pgfqpoint{6.117013in}{1.737848in}}{\pgfqpoint{6.122057in}{1.742892in}}%
\pgfpathcurveto{\pgfqpoint{6.127100in}{1.747935in}}{\pgfqpoint{6.129934in}{1.754777in}}{\pgfqpoint{6.129934in}{1.761910in}}%
\pgfpathcurveto{\pgfqpoint{6.129934in}{1.769043in}}{\pgfqpoint{6.127100in}{1.775884in}}{\pgfqpoint{6.122057in}{1.780928in}}%
\pgfpathcurveto{\pgfqpoint{6.117013in}{1.785971in}}{\pgfqpoint{6.110171in}{1.788805in}}{\pgfqpoint{6.103039in}{1.788805in}}%
\pgfpathcurveto{\pgfqpoint{6.095906in}{1.788805in}}{\pgfqpoint{6.089064in}{1.785971in}}{\pgfqpoint{6.084020in}{1.780928in}}%
\pgfpathcurveto{\pgfqpoint{6.078977in}{1.775884in}}{\pgfqpoint{6.076143in}{1.769043in}}{\pgfqpoint{6.076143in}{1.761910in}}%
\pgfpathcurveto{\pgfqpoint{6.076143in}{1.754777in}}{\pgfqpoint{6.078977in}{1.747935in}}{\pgfqpoint{6.084020in}{1.742892in}}%
\pgfpathcurveto{\pgfqpoint{6.089064in}{1.737848in}}{\pgfqpoint{6.095906in}{1.735014in}}{\pgfqpoint{6.103039in}{1.735014in}}%
\pgfpathclose%
\pgfusepath{stroke,fill}%
\end{pgfscope}%
\begin{pgfscope}%
\pgfpathrectangle{\pgfqpoint{4.985294in}{0.500000in}}{\pgfqpoint{1.764706in}{1.700000in}}%
\pgfusepath{clip}%
\pgfsetbuttcap%
\pgfsetroundjoin%
\definecolor{currentfill}{rgb}{0.977657,0.891500,0.822809}%
\pgfsetfillcolor{currentfill}%
\pgfsetlinewidth{0.311001pt}%
\definecolor{currentstroke}{rgb}{1.000000,1.000000,1.000000}%
\pgfsetstrokecolor{currentstroke}%
\pgfsetdash{}{0pt}%
\pgfpathmoveto{\pgfqpoint{6.336360in}{1.441071in}}%
\pgfpathcurveto{\pgfqpoint{6.343493in}{1.441071in}}{\pgfqpoint{6.350334in}{1.443905in}}{\pgfqpoint{6.355378in}{1.448948in}}%
\pgfpathcurveto{\pgfqpoint{6.360422in}{1.453992in}}{\pgfqpoint{6.363256in}{1.460834in}}{\pgfqpoint{6.363256in}{1.467966in}}%
\pgfpathcurveto{\pgfqpoint{6.363256in}{1.475099in}}{\pgfqpoint{6.360422in}{1.481941in}}{\pgfqpoint{6.355378in}{1.486984in}}%
\pgfpathcurveto{\pgfqpoint{6.350334in}{1.492028in}}{\pgfqpoint{6.343493in}{1.494862in}}{\pgfqpoint{6.336360in}{1.494862in}}%
\pgfpathcurveto{\pgfqpoint{6.329227in}{1.494862in}}{\pgfqpoint{6.322385in}{1.492028in}}{\pgfqpoint{6.317342in}{1.486984in}}%
\pgfpathcurveto{\pgfqpoint{6.312298in}{1.481941in}}{\pgfqpoint{6.309464in}{1.475099in}}{\pgfqpoint{6.309464in}{1.467966in}}%
\pgfpathcurveto{\pgfqpoint{6.309464in}{1.460834in}}{\pgfqpoint{6.312298in}{1.453992in}}{\pgfqpoint{6.317342in}{1.448948in}}%
\pgfpathcurveto{\pgfqpoint{6.322385in}{1.443905in}}{\pgfqpoint{6.329227in}{1.441071in}}{\pgfqpoint{6.336360in}{1.441071in}}%
\pgfpathclose%
\pgfusepath{stroke,fill}%
\end{pgfscope}%
\begin{pgfscope}%
\pgfpathrectangle{\pgfqpoint{4.985294in}{0.500000in}}{\pgfqpoint{1.764706in}{1.700000in}}%
\pgfusepath{clip}%
\pgfsetbuttcap%
\pgfsetroundjoin%
\definecolor{currentfill}{rgb}{0.971202,0.827364,0.728520}%
\pgfsetfillcolor{currentfill}%
\pgfsetlinewidth{0.311001pt}%
\definecolor{currentstroke}{rgb}{1.000000,1.000000,1.000000}%
\pgfsetstrokecolor{currentstroke}%
\pgfsetdash{}{0pt}%
\pgfpathmoveto{\pgfqpoint{5.368542in}{1.206663in}}%
\pgfpathcurveto{\pgfqpoint{5.375675in}{1.206663in}}{\pgfqpoint{5.382517in}{1.209497in}}{\pgfqpoint{5.387560in}{1.214541in}}%
\pgfpathcurveto{\pgfqpoint{5.392604in}{1.219584in}}{\pgfqpoint{5.395438in}{1.226426in}}{\pgfqpoint{5.395438in}{1.233559in}}%
\pgfpathcurveto{\pgfqpoint{5.395438in}{1.240692in}}{\pgfqpoint{5.392604in}{1.247533in}}{\pgfqpoint{5.387560in}{1.252577in}}%
\pgfpathcurveto{\pgfqpoint{5.382517in}{1.257620in}}{\pgfqpoint{5.375675in}{1.260454in}}{\pgfqpoint{5.368542in}{1.260454in}}%
\pgfpathcurveto{\pgfqpoint{5.361409in}{1.260454in}}{\pgfqpoint{5.354568in}{1.257620in}}{\pgfqpoint{5.349524in}{1.252577in}}%
\pgfpathcurveto{\pgfqpoint{5.344480in}{1.247533in}}{\pgfqpoint{5.341647in}{1.240692in}}{\pgfqpoint{5.341647in}{1.233559in}}%
\pgfpathcurveto{\pgfqpoint{5.341647in}{1.226426in}}{\pgfqpoint{5.344480in}{1.219584in}}{\pgfqpoint{5.349524in}{1.214541in}}%
\pgfpathcurveto{\pgfqpoint{5.354568in}{1.209497in}}{\pgfqpoint{5.361409in}{1.206663in}}{\pgfqpoint{5.368542in}{1.206663in}}%
\pgfpathclose%
\pgfusepath{stroke,fill}%
\end{pgfscope}%
\begin{pgfscope}%
\pgfpathrectangle{\pgfqpoint{4.985294in}{0.500000in}}{\pgfqpoint{1.764706in}{1.700000in}}%
\pgfusepath{clip}%
\pgfsetbuttcap%
\pgfsetroundjoin%
\definecolor{currentfill}{rgb}{0.965302,0.713942,0.568499}%
\pgfsetfillcolor{currentfill}%
\pgfsetlinewidth{0.311001pt}%
\definecolor{currentstroke}{rgb}{1.000000,1.000000,1.000000}%
\pgfsetstrokecolor{currentstroke}%
\pgfsetdash{}{0pt}%
\pgfpathmoveto{\pgfqpoint{6.141145in}{1.029041in}}%
\pgfpathcurveto{\pgfqpoint{6.148278in}{1.029041in}}{\pgfqpoint{6.155120in}{1.031875in}}{\pgfqpoint{6.160163in}{1.036919in}}%
\pgfpathcurveto{\pgfqpoint{6.165207in}{1.041962in}}{\pgfqpoint{6.168041in}{1.048804in}}{\pgfqpoint{6.168041in}{1.055937in}}%
\pgfpathcurveto{\pgfqpoint{6.168041in}{1.063070in}}{\pgfqpoint{6.165207in}{1.069911in}}{\pgfqpoint{6.160163in}{1.074955in}}%
\pgfpathcurveto{\pgfqpoint{6.155120in}{1.079999in}}{\pgfqpoint{6.148278in}{1.082832in}}{\pgfqpoint{6.141145in}{1.082832in}}%
\pgfpathcurveto{\pgfqpoint{6.134012in}{1.082832in}}{\pgfqpoint{6.127171in}{1.079999in}}{\pgfqpoint{6.122127in}{1.074955in}}%
\pgfpathcurveto{\pgfqpoint{6.117083in}{1.069911in}}{\pgfqpoint{6.114249in}{1.063070in}}{\pgfqpoint{6.114249in}{1.055937in}}%
\pgfpathcurveto{\pgfqpoint{6.114249in}{1.048804in}}{\pgfqpoint{6.117083in}{1.041962in}}{\pgfqpoint{6.122127in}{1.036919in}}%
\pgfpathcurveto{\pgfqpoint{6.127171in}{1.031875in}}{\pgfqpoint{6.134012in}{1.029041in}}{\pgfqpoint{6.141145in}{1.029041in}}%
\pgfpathclose%
\pgfusepath{stroke,fill}%
\end{pgfscope}%
\begin{pgfscope}%
\pgfpathrectangle{\pgfqpoint{4.985294in}{0.500000in}}{\pgfqpoint{1.764706in}{1.700000in}}%
\pgfusepath{clip}%
\pgfsetbuttcap%
\pgfsetroundjoin%
\definecolor{currentfill}{rgb}{0.970255,0.815666,0.711203}%
\pgfsetfillcolor{currentfill}%
\pgfsetlinewidth{0.311001pt}%
\definecolor{currentstroke}{rgb}{1.000000,1.000000,1.000000}%
\pgfsetstrokecolor{currentstroke}%
\pgfsetdash{}{0pt}%
\pgfpathmoveto{\pgfqpoint{5.503961in}{1.646149in}}%
\pgfpathcurveto{\pgfqpoint{5.511094in}{1.646149in}}{\pgfqpoint{5.517936in}{1.648983in}}{\pgfqpoint{5.522979in}{1.654027in}}%
\pgfpathcurveto{\pgfqpoint{5.528023in}{1.659070in}}{\pgfqpoint{5.530857in}{1.665912in}}{\pgfqpoint{5.530857in}{1.673045in}}%
\pgfpathcurveto{\pgfqpoint{5.530857in}{1.680178in}}{\pgfqpoint{5.528023in}{1.687019in}}{\pgfqpoint{5.522979in}{1.692063in}}%
\pgfpathcurveto{\pgfqpoint{5.517936in}{1.697107in}}{\pgfqpoint{5.511094in}{1.699941in}}{\pgfqpoint{5.503961in}{1.699941in}}%
\pgfpathcurveto{\pgfqpoint{5.496828in}{1.699941in}}{\pgfqpoint{5.489987in}{1.697107in}}{\pgfqpoint{5.484943in}{1.692063in}}%
\pgfpathcurveto{\pgfqpoint{5.479899in}{1.687019in}}{\pgfqpoint{5.477065in}{1.680178in}}{\pgfqpoint{5.477065in}{1.673045in}}%
\pgfpathcurveto{\pgfqpoint{5.477065in}{1.665912in}}{\pgfqpoint{5.479899in}{1.659070in}}{\pgfqpoint{5.484943in}{1.654027in}}%
\pgfpathcurveto{\pgfqpoint{5.489987in}{1.648983in}}{\pgfqpoint{5.496828in}{1.646149in}}{\pgfqpoint{5.503961in}{1.646149in}}%
\pgfpathclose%
\pgfusepath{stroke,fill}%
\end{pgfscope}%
\begin{pgfscope}%
\pgfpathrectangle{\pgfqpoint{4.985294in}{0.500000in}}{\pgfqpoint{1.764706in}{1.700000in}}%
\pgfusepath{clip}%
\pgfsetbuttcap%
\pgfsetroundjoin%
\definecolor{currentfill}{rgb}{0.962283,0.593046,0.431453}%
\pgfsetfillcolor{currentfill}%
\pgfsetlinewidth{0.311001pt}%
\definecolor{currentstroke}{rgb}{1.000000,1.000000,1.000000}%
\pgfsetstrokecolor{currentstroke}%
\pgfsetdash{}{0pt}%
\pgfpathmoveto{\pgfqpoint{6.429779in}{1.260796in}}%
\pgfpathcurveto{\pgfqpoint{6.436912in}{1.260796in}}{\pgfqpoint{6.443754in}{1.263629in}}{\pgfqpoint{6.448797in}{1.268673in}}%
\pgfpathcurveto{\pgfqpoint{6.453841in}{1.273717in}}{\pgfqpoint{6.456675in}{1.280558in}}{\pgfqpoint{6.456675in}{1.287691in}}%
\pgfpathcurveto{\pgfqpoint{6.456675in}{1.294824in}}{\pgfqpoint{6.453841in}{1.301666in}}{\pgfqpoint{6.448797in}{1.306709in}}%
\pgfpathcurveto{\pgfqpoint{6.443754in}{1.311753in}}{\pgfqpoint{6.436912in}{1.314587in}}{\pgfqpoint{6.429779in}{1.314587in}}%
\pgfpathcurveto{\pgfqpoint{6.422646in}{1.314587in}}{\pgfqpoint{6.415805in}{1.311753in}}{\pgfqpoint{6.410761in}{1.306709in}}%
\pgfpathcurveto{\pgfqpoint{6.405717in}{1.301666in}}{\pgfqpoint{6.402883in}{1.294824in}}{\pgfqpoint{6.402883in}{1.287691in}}%
\pgfpathcurveto{\pgfqpoint{6.402883in}{1.280558in}}{\pgfqpoint{6.405717in}{1.273717in}}{\pgfqpoint{6.410761in}{1.268673in}}%
\pgfpathcurveto{\pgfqpoint{6.415805in}{1.263629in}}{\pgfqpoint{6.422646in}{1.260796in}}{\pgfqpoint{6.429779in}{1.260796in}}%
\pgfpathclose%
\pgfusepath{stroke,fill}%
\end{pgfscope}%
\begin{pgfscope}%
\pgfpathrectangle{\pgfqpoint{4.985294in}{0.500000in}}{\pgfqpoint{1.764706in}{1.700000in}}%
\pgfusepath{clip}%
\pgfsetbuttcap%
\pgfsetroundjoin%
\definecolor{currentfill}{rgb}{0.963379,0.625574,0.465113}%
\pgfsetfillcolor{currentfill}%
\pgfsetlinewidth{0.311001pt}%
\definecolor{currentstroke}{rgb}{1.000000,1.000000,1.000000}%
\pgfsetstrokecolor{currentstroke}%
\pgfsetdash{}{0pt}%
\pgfpathmoveto{\pgfqpoint{6.393695in}{1.518710in}}%
\pgfpathcurveto{\pgfqpoint{6.400828in}{1.518710in}}{\pgfqpoint{6.407670in}{1.521544in}}{\pgfqpoint{6.412714in}{1.526588in}}%
\pgfpathcurveto{\pgfqpoint{6.417757in}{1.531631in}}{\pgfqpoint{6.420591in}{1.538473in}}{\pgfqpoint{6.420591in}{1.545606in}}%
\pgfpathcurveto{\pgfqpoint{6.420591in}{1.552739in}}{\pgfqpoint{6.417757in}{1.559580in}}{\pgfqpoint{6.412714in}{1.564624in}}%
\pgfpathcurveto{\pgfqpoint{6.407670in}{1.569667in}}{\pgfqpoint{6.400828in}{1.572501in}}{\pgfqpoint{6.393695in}{1.572501in}}%
\pgfpathcurveto{\pgfqpoint{6.386563in}{1.572501in}}{\pgfqpoint{6.379721in}{1.569667in}}{\pgfqpoint{6.374677in}{1.564624in}}%
\pgfpathcurveto{\pgfqpoint{6.369634in}{1.559580in}}{\pgfqpoint{6.366800in}{1.552739in}}{\pgfqpoint{6.366800in}{1.545606in}}%
\pgfpathcurveto{\pgfqpoint{6.366800in}{1.538473in}}{\pgfqpoint{6.369634in}{1.531631in}}{\pgfqpoint{6.374677in}{1.526588in}}%
\pgfpathcurveto{\pgfqpoint{6.379721in}{1.521544in}}{\pgfqpoint{6.386563in}{1.518710in}}{\pgfqpoint{6.393695in}{1.518710in}}%
\pgfpathclose%
\pgfusepath{stroke,fill}%
\end{pgfscope}%
\begin{pgfscope}%
\pgfpathrectangle{\pgfqpoint{4.985294in}{0.500000in}}{\pgfqpoint{1.764706in}{1.700000in}}%
\pgfusepath{clip}%
\pgfsetbuttcap%
\pgfsetroundjoin%
\definecolor{currentfill}{rgb}{0.978376,0.897317,0.831308}%
\pgfsetfillcolor{currentfill}%
\pgfsetlinewidth{0.311001pt}%
\definecolor{currentstroke}{rgb}{1.000000,1.000000,1.000000}%
\pgfsetstrokecolor{currentstroke}%
\pgfsetdash{}{0pt}%
\pgfpathmoveto{\pgfqpoint{6.267927in}{1.141812in}}%
\pgfpathcurveto{\pgfqpoint{6.275060in}{1.141812in}}{\pgfqpoint{6.281902in}{1.144646in}}{\pgfqpoint{6.286945in}{1.149690in}}%
\pgfpathcurveto{\pgfqpoint{6.291989in}{1.154733in}}{\pgfqpoint{6.294823in}{1.161575in}}{\pgfqpoint{6.294823in}{1.168708in}}%
\pgfpathcurveto{\pgfqpoint{6.294823in}{1.175841in}}{\pgfqpoint{6.291989in}{1.182682in}}{\pgfqpoint{6.286945in}{1.187726in}}%
\pgfpathcurveto{\pgfqpoint{6.281902in}{1.192770in}}{\pgfqpoint{6.275060in}{1.195604in}}{\pgfqpoint{6.267927in}{1.195604in}}%
\pgfpathcurveto{\pgfqpoint{6.260794in}{1.195604in}}{\pgfqpoint{6.253953in}{1.192770in}}{\pgfqpoint{6.248909in}{1.187726in}}%
\pgfpathcurveto{\pgfqpoint{6.243865in}{1.182682in}}{\pgfqpoint{6.241032in}{1.175841in}}{\pgfqpoint{6.241032in}{1.168708in}}%
\pgfpathcurveto{\pgfqpoint{6.241032in}{1.161575in}}{\pgfqpoint{6.243865in}{1.154733in}}{\pgfqpoint{6.248909in}{1.149690in}}%
\pgfpathcurveto{\pgfqpoint{6.253953in}{1.144646in}}{\pgfqpoint{6.260794in}{1.141812in}}{\pgfqpoint{6.267927in}{1.141812in}}%
\pgfpathclose%
\pgfusepath{stroke,fill}%
\end{pgfscope}%
\begin{pgfscope}%
\pgfpathrectangle{\pgfqpoint{4.985294in}{0.500000in}}{\pgfqpoint{1.764706in}{1.700000in}}%
\pgfusepath{clip}%
\pgfsetbuttcap%
\pgfsetroundjoin%
\definecolor{currentfill}{rgb}{0.950017,0.427714,0.292447}%
\pgfsetfillcolor{currentfill}%
\pgfsetlinewidth{0.311001pt}%
\definecolor{currentstroke}{rgb}{1.000000,1.000000,1.000000}%
\pgfsetstrokecolor{currentstroke}%
\pgfsetdash{}{0pt}%
\pgfpathmoveto{\pgfqpoint{5.634799in}{1.084000in}}%
\pgfpathcurveto{\pgfqpoint{5.641931in}{1.084000in}}{\pgfqpoint{5.648773in}{1.086834in}}{\pgfqpoint{5.653817in}{1.091878in}}%
\pgfpathcurveto{\pgfqpoint{5.658860in}{1.096921in}}{\pgfqpoint{5.661694in}{1.103763in}}{\pgfqpoint{5.661694in}{1.110896in}}%
\pgfpathcurveto{\pgfqpoint{5.661694in}{1.118029in}}{\pgfqpoint{5.658860in}{1.124870in}}{\pgfqpoint{5.653817in}{1.129914in}}%
\pgfpathcurveto{\pgfqpoint{5.648773in}{1.134958in}}{\pgfqpoint{5.641931in}{1.137792in}}{\pgfqpoint{5.634799in}{1.137792in}}%
\pgfpathcurveto{\pgfqpoint{5.627666in}{1.137792in}}{\pgfqpoint{5.620824in}{1.134958in}}{\pgfqpoint{5.615781in}{1.129914in}}%
\pgfpathcurveto{\pgfqpoint{5.610737in}{1.124870in}}{\pgfqpoint{5.607903in}{1.118029in}}{\pgfqpoint{5.607903in}{1.110896in}}%
\pgfpathcurveto{\pgfqpoint{5.607903in}{1.103763in}}{\pgfqpoint{5.610737in}{1.096921in}}{\pgfqpoint{5.615781in}{1.091878in}}%
\pgfpathcurveto{\pgfqpoint{5.620824in}{1.086834in}}{\pgfqpoint{5.627666in}{1.084000in}}{\pgfqpoint{5.634799in}{1.084000in}}%
\pgfpathclose%
\pgfusepath{stroke,fill}%
\end{pgfscope}%
\begin{pgfscope}%
\pgfpathrectangle{\pgfqpoint{4.985294in}{0.500000in}}{\pgfqpoint{1.764706in}{1.700000in}}%
\pgfusepath{clip}%
\pgfsetbuttcap%
\pgfsetroundjoin%
\definecolor{currentfill}{rgb}{0.968105,0.786346,0.667739}%
\pgfsetfillcolor{currentfill}%
\pgfsetlinewidth{0.311001pt}%
\definecolor{currentstroke}{rgb}{1.000000,1.000000,1.000000}%
\pgfsetstrokecolor{currentstroke}%
\pgfsetdash{}{0pt}%
\pgfpathmoveto{\pgfqpoint{5.351558in}{1.218765in}}%
\pgfpathcurveto{\pgfqpoint{5.358691in}{1.218765in}}{\pgfqpoint{5.365533in}{1.221599in}}{\pgfqpoint{5.370577in}{1.226643in}}%
\pgfpathcurveto{\pgfqpoint{5.375620in}{1.231687in}}{\pgfqpoint{5.378454in}{1.238528in}}{\pgfqpoint{5.378454in}{1.245661in}}%
\pgfpathcurveto{\pgfqpoint{5.378454in}{1.252794in}}{\pgfqpoint{5.375620in}{1.259636in}}{\pgfqpoint{5.370577in}{1.264679in}}%
\pgfpathcurveto{\pgfqpoint{5.365533in}{1.269723in}}{\pgfqpoint{5.358691in}{1.272557in}}{\pgfqpoint{5.351558in}{1.272557in}}%
\pgfpathcurveto{\pgfqpoint{5.344426in}{1.272557in}}{\pgfqpoint{5.337584in}{1.269723in}}{\pgfqpoint{5.332540in}{1.264679in}}%
\pgfpathcurveto{\pgfqpoint{5.327497in}{1.259636in}}{\pgfqpoint{5.324663in}{1.252794in}}{\pgfqpoint{5.324663in}{1.245661in}}%
\pgfpathcurveto{\pgfqpoint{5.324663in}{1.238528in}}{\pgfqpoint{5.327497in}{1.231687in}}{\pgfqpoint{5.332540in}{1.226643in}}%
\pgfpathcurveto{\pgfqpoint{5.337584in}{1.221599in}}{\pgfqpoint{5.344426in}{1.218765in}}{\pgfqpoint{5.351558in}{1.218765in}}%
\pgfpathclose%
\pgfusepath{stroke,fill}%
\end{pgfscope}%
\begin{pgfscope}%
\pgfpathrectangle{\pgfqpoint{4.985294in}{0.500000in}}{\pgfqpoint{1.764706in}{1.700000in}}%
\pgfusepath{clip}%
\pgfsetbuttcap%
\pgfsetroundjoin%
\definecolor{currentfill}{rgb}{0.973832,0.856556,0.771584}%
\pgfsetfillcolor{currentfill}%
\pgfsetlinewidth{0.311001pt}%
\definecolor{currentstroke}{rgb}{1.000000,1.000000,1.000000}%
\pgfsetstrokecolor{currentstroke}%
\pgfsetdash{}{0pt}%
\pgfpathmoveto{\pgfqpoint{5.428173in}{1.544799in}}%
\pgfpathcurveto{\pgfqpoint{5.435306in}{1.544799in}}{\pgfqpoint{5.442148in}{1.547633in}}{\pgfqpoint{5.447191in}{1.552676in}}%
\pgfpathcurveto{\pgfqpoint{5.452235in}{1.557720in}}{\pgfqpoint{5.455069in}{1.564561in}}{\pgfqpoint{5.455069in}{1.571694in}}%
\pgfpathcurveto{\pgfqpoint{5.455069in}{1.578827in}}{\pgfqpoint{5.452235in}{1.585669in}}{\pgfqpoint{5.447191in}{1.590712in}}%
\pgfpathcurveto{\pgfqpoint{5.442148in}{1.595756in}}{\pgfqpoint{5.435306in}{1.598590in}}{\pgfqpoint{5.428173in}{1.598590in}}%
\pgfpathcurveto{\pgfqpoint{5.421040in}{1.598590in}}{\pgfqpoint{5.414199in}{1.595756in}}{\pgfqpoint{5.409155in}{1.590712in}}%
\pgfpathcurveto{\pgfqpoint{5.404111in}{1.585669in}}{\pgfqpoint{5.401278in}{1.578827in}}{\pgfqpoint{5.401278in}{1.571694in}}%
\pgfpathcurveto{\pgfqpoint{5.401278in}{1.564561in}}{\pgfqpoint{5.404111in}{1.557720in}}{\pgfqpoint{5.409155in}{1.552676in}}%
\pgfpathcurveto{\pgfqpoint{5.414199in}{1.547633in}}{\pgfqpoint{5.421040in}{1.544799in}}{\pgfqpoint{5.428173in}{1.544799in}}%
\pgfpathclose%
\pgfusepath{stroke,fill}%
\end{pgfscope}%
\begin{pgfscope}%
\pgfpathrectangle{\pgfqpoint{4.985294in}{0.500000in}}{\pgfqpoint{1.764706in}{1.700000in}}%
\pgfusepath{clip}%
\pgfsetbuttcap%
\pgfsetroundjoin%
\definecolor{currentfill}{rgb}{0.965169,0.707764,0.560659}%
\pgfsetfillcolor{currentfill}%
\pgfsetlinewidth{0.311001pt}%
\definecolor{currentstroke}{rgb}{1.000000,1.000000,1.000000}%
\pgfsetstrokecolor{currentstroke}%
\pgfsetdash{}{0pt}%
\pgfpathmoveto{\pgfqpoint{5.341351in}{1.456872in}}%
\pgfpathcurveto{\pgfqpoint{5.348484in}{1.456872in}}{\pgfqpoint{5.355325in}{1.459706in}}{\pgfqpoint{5.360369in}{1.464749in}}%
\pgfpathcurveto{\pgfqpoint{5.365413in}{1.469793in}}{\pgfqpoint{5.368247in}{1.476635in}}{\pgfqpoint{5.368247in}{1.483768in}}%
\pgfpathcurveto{\pgfqpoint{5.368247in}{1.490900in}}{\pgfqpoint{5.365413in}{1.497742in}}{\pgfqpoint{5.360369in}{1.502786in}}%
\pgfpathcurveto{\pgfqpoint{5.355325in}{1.507829in}}{\pgfqpoint{5.348484in}{1.510663in}}{\pgfqpoint{5.341351in}{1.510663in}}%
\pgfpathcurveto{\pgfqpoint{5.334218in}{1.510663in}}{\pgfqpoint{5.327376in}{1.507829in}}{\pgfqpoint{5.322333in}{1.502786in}}%
\pgfpathcurveto{\pgfqpoint{5.317289in}{1.497742in}}{\pgfqpoint{5.314455in}{1.490900in}}{\pgfqpoint{5.314455in}{1.483768in}}%
\pgfpathcurveto{\pgfqpoint{5.314455in}{1.476635in}}{\pgfqpoint{5.317289in}{1.469793in}}{\pgfqpoint{5.322333in}{1.464749in}}%
\pgfpathcurveto{\pgfqpoint{5.327376in}{1.459706in}}{\pgfqpoint{5.334218in}{1.456872in}}{\pgfqpoint{5.341351in}{1.456872in}}%
\pgfpathclose%
\pgfusepath{stroke,fill}%
\end{pgfscope}%
\begin{pgfscope}%
\pgfpathrectangle{\pgfqpoint{4.985294in}{0.500000in}}{\pgfqpoint{1.764706in}{1.700000in}}%
\pgfusepath{clip}%
\pgfsetbuttcap%
\pgfsetroundjoin%
\definecolor{currentfill}{rgb}{0.964920,0.695342,0.545192}%
\pgfsetfillcolor{currentfill}%
\pgfsetlinewidth{0.311001pt}%
\definecolor{currentstroke}{rgb}{1.000000,1.000000,1.000000}%
\pgfsetstrokecolor{currentstroke}%
\pgfsetdash{}{0pt}%
\pgfpathmoveto{\pgfqpoint{6.393511in}{1.463666in}}%
\pgfpathcurveto{\pgfqpoint{6.400643in}{1.463666in}}{\pgfqpoint{6.407485in}{1.466500in}}{\pgfqpoint{6.412529in}{1.471544in}}%
\pgfpathcurveto{\pgfqpoint{6.417572in}{1.476587in}}{\pgfqpoint{6.420406in}{1.483429in}}{\pgfqpoint{6.420406in}{1.490562in}}%
\pgfpathcurveto{\pgfqpoint{6.420406in}{1.497695in}}{\pgfqpoint{6.417572in}{1.504536in}}{\pgfqpoint{6.412529in}{1.509580in}}%
\pgfpathcurveto{\pgfqpoint{6.407485in}{1.514624in}}{\pgfqpoint{6.400643in}{1.517457in}}{\pgfqpoint{6.393511in}{1.517457in}}%
\pgfpathcurveto{\pgfqpoint{6.386378in}{1.517457in}}{\pgfqpoint{6.379536in}{1.514624in}}{\pgfqpoint{6.374493in}{1.509580in}}%
\pgfpathcurveto{\pgfqpoint{6.369449in}{1.504536in}}{\pgfqpoint{6.366615in}{1.497695in}}{\pgfqpoint{6.366615in}{1.490562in}}%
\pgfpathcurveto{\pgfqpoint{6.366615in}{1.483429in}}{\pgfqpoint{6.369449in}{1.476587in}}{\pgfqpoint{6.374493in}{1.471544in}}%
\pgfpathcurveto{\pgfqpoint{6.379536in}{1.466500in}}{\pgfqpoint{6.386378in}{1.463666in}}{\pgfqpoint{6.393511in}{1.463666in}}%
\pgfpathclose%
\pgfusepath{stroke,fill}%
\end{pgfscope}%
\begin{pgfscope}%
\pgfpathrectangle{\pgfqpoint{4.985294in}{0.500000in}}{\pgfqpoint{1.764706in}{1.700000in}}%
\pgfusepath{clip}%
\pgfsetbuttcap%
\pgfsetroundjoin%
\definecolor{currentfill}{rgb}{0.962765,0.606121,0.444717}%
\pgfsetfillcolor{currentfill}%
\pgfsetlinewidth{0.311001pt}%
\definecolor{currentstroke}{rgb}{1.000000,1.000000,1.000000}%
\pgfsetstrokecolor{currentstroke}%
\pgfsetdash{}{0pt}%
\pgfpathmoveto{\pgfqpoint{5.524344in}{0.871338in}}%
\pgfpathcurveto{\pgfqpoint{5.531477in}{0.871338in}}{\pgfqpoint{5.538319in}{0.874172in}}{\pgfqpoint{5.543362in}{0.879215in}}%
\pgfpathcurveto{\pgfqpoint{5.548406in}{0.884259in}}{\pgfqpoint{5.551240in}{0.891101in}}{\pgfqpoint{5.551240in}{0.898233in}}%
\pgfpathcurveto{\pgfqpoint{5.551240in}{0.905366in}}{\pgfqpoint{5.548406in}{0.912208in}}{\pgfqpoint{5.543362in}{0.917252in}}%
\pgfpathcurveto{\pgfqpoint{5.538319in}{0.922295in}}{\pgfqpoint{5.531477in}{0.925129in}}{\pgfqpoint{5.524344in}{0.925129in}}%
\pgfpathcurveto{\pgfqpoint{5.517211in}{0.925129in}}{\pgfqpoint{5.510370in}{0.922295in}}{\pgfqpoint{5.505326in}{0.917252in}}%
\pgfpathcurveto{\pgfqpoint{5.500282in}{0.912208in}}{\pgfqpoint{5.497448in}{0.905366in}}{\pgfqpoint{5.497448in}{0.898233in}}%
\pgfpathcurveto{\pgfqpoint{5.497448in}{0.891101in}}{\pgfqpoint{5.500282in}{0.884259in}}{\pgfqpoint{5.505326in}{0.879215in}}%
\pgfpathcurveto{\pgfqpoint{5.510370in}{0.874172in}}{\pgfqpoint{5.517211in}{0.871338in}}{\pgfqpoint{5.524344in}{0.871338in}}%
\pgfpathclose%
\pgfusepath{stroke,fill}%
\end{pgfscope}%
\begin{pgfscope}%
\pgfpathrectangle{\pgfqpoint{4.985294in}{0.500000in}}{\pgfqpoint{1.764706in}{1.700000in}}%
\pgfusepath{clip}%
\pgfsetbuttcap%
\pgfsetroundjoin%
\definecolor{currentfill}{rgb}{0.973271,0.850724,0.762998}%
\pgfsetfillcolor{currentfill}%
\pgfsetlinewidth{0.311001pt}%
\definecolor{currentstroke}{rgb}{1.000000,1.000000,1.000000}%
\pgfsetstrokecolor{currentstroke}%
\pgfsetdash{}{0pt}%
\pgfpathmoveto{\pgfqpoint{6.235864in}{1.637927in}}%
\pgfpathcurveto{\pgfqpoint{6.242997in}{1.637927in}}{\pgfqpoint{6.249838in}{1.640761in}}{\pgfqpoint{6.254882in}{1.645805in}}%
\pgfpathcurveto{\pgfqpoint{6.259926in}{1.650849in}}{\pgfqpoint{6.262759in}{1.657690in}}{\pgfqpoint{6.262759in}{1.664823in}}%
\pgfpathcurveto{\pgfqpoint{6.262759in}{1.671956in}}{\pgfqpoint{6.259926in}{1.678797in}}{\pgfqpoint{6.254882in}{1.683841in}}%
\pgfpathcurveto{\pgfqpoint{6.249838in}{1.688885in}}{\pgfqpoint{6.242997in}{1.691719in}}{\pgfqpoint{6.235864in}{1.691719in}}%
\pgfpathcurveto{\pgfqpoint{6.228731in}{1.691719in}}{\pgfqpoint{6.221889in}{1.688885in}}{\pgfqpoint{6.216846in}{1.683841in}}%
\pgfpathcurveto{\pgfqpoint{6.211802in}{1.678797in}}{\pgfqpoint{6.208968in}{1.671956in}}{\pgfqpoint{6.208968in}{1.664823in}}%
\pgfpathcurveto{\pgfqpoint{6.208968in}{1.657690in}}{\pgfqpoint{6.211802in}{1.650849in}}{\pgfqpoint{6.216846in}{1.645805in}}%
\pgfpathcurveto{\pgfqpoint{6.221889in}{1.640761in}}{\pgfqpoint{6.228731in}{1.637927in}}{\pgfqpoint{6.235864in}{1.637927in}}%
\pgfpathclose%
\pgfusepath{stroke,fill}%
\end{pgfscope}%
\begin{pgfscope}%
\pgfpathrectangle{\pgfqpoint{4.985294in}{0.500000in}}{\pgfqpoint{1.764706in}{1.700000in}}%
\pgfusepath{clip}%
\pgfsetbuttcap%
\pgfsetroundjoin%
\definecolor{currentfill}{rgb}{0.972201,0.839051,0.745789}%
\pgfsetfillcolor{currentfill}%
\pgfsetlinewidth{0.311001pt}%
\definecolor{currentstroke}{rgb}{1.000000,1.000000,1.000000}%
\pgfsetstrokecolor{currentstroke}%
\pgfsetdash{}{0pt}%
\pgfpathmoveto{\pgfqpoint{5.514591in}{0.996768in}}%
\pgfpathcurveto{\pgfqpoint{5.521723in}{0.996768in}}{\pgfqpoint{5.528565in}{0.999602in}}{\pgfqpoint{5.533609in}{1.004645in}}%
\pgfpathcurveto{\pgfqpoint{5.538652in}{1.009689in}}{\pgfqpoint{5.541486in}{1.016531in}}{\pgfqpoint{5.541486in}{1.023664in}}%
\pgfpathcurveto{\pgfqpoint{5.541486in}{1.030796in}}{\pgfqpoint{5.538652in}{1.037638in}}{\pgfqpoint{5.533609in}{1.042682in}}%
\pgfpathcurveto{\pgfqpoint{5.528565in}{1.047725in}}{\pgfqpoint{5.521723in}{1.050559in}}{\pgfqpoint{5.514591in}{1.050559in}}%
\pgfpathcurveto{\pgfqpoint{5.507458in}{1.050559in}}{\pgfqpoint{5.500616in}{1.047725in}}{\pgfqpoint{5.495573in}{1.042682in}}%
\pgfpathcurveto{\pgfqpoint{5.490529in}{1.037638in}}{\pgfqpoint{5.487695in}{1.030796in}}{\pgfqpoint{5.487695in}{1.023664in}}%
\pgfpathcurveto{\pgfqpoint{5.487695in}{1.016531in}}{\pgfqpoint{5.490529in}{1.009689in}}{\pgfqpoint{5.495573in}{1.004645in}}%
\pgfpathcurveto{\pgfqpoint{5.500616in}{0.999602in}}{\pgfqpoint{5.507458in}{0.996768in}}{\pgfqpoint{5.514591in}{0.996768in}}%
\pgfpathclose%
\pgfusepath{stroke,fill}%
\end{pgfscope}%
\begin{pgfscope}%
\pgfpathrectangle{\pgfqpoint{4.985294in}{0.500000in}}{\pgfqpoint{1.764706in}{1.700000in}}%
\pgfusepath{clip}%
\pgfsetbuttcap%
\pgfsetroundjoin%
\definecolor{currentfill}{rgb}{0.953126,0.456614,0.312398}%
\pgfsetfillcolor{currentfill}%
\pgfsetlinewidth{0.311001pt}%
\definecolor{currentstroke}{rgb}{1.000000,1.000000,1.000000}%
\pgfsetstrokecolor{currentstroke}%
\pgfsetdash{}{0pt}%
\pgfpathmoveto{\pgfqpoint{5.646436in}{1.607105in}}%
\pgfpathcurveto{\pgfqpoint{5.653568in}{1.607105in}}{\pgfqpoint{5.660410in}{1.609939in}}{\pgfqpoint{5.665454in}{1.614983in}}%
\pgfpathcurveto{\pgfqpoint{5.670497in}{1.620027in}}{\pgfqpoint{5.673331in}{1.626868in}}{\pgfqpoint{5.673331in}{1.634001in}}%
\pgfpathcurveto{\pgfqpoint{5.673331in}{1.641134in}}{\pgfqpoint{5.670497in}{1.647976in}}{\pgfqpoint{5.665454in}{1.653019in}}%
\pgfpathcurveto{\pgfqpoint{5.660410in}{1.658063in}}{\pgfqpoint{5.653568in}{1.660897in}}{\pgfqpoint{5.646436in}{1.660897in}}%
\pgfpathcurveto{\pgfqpoint{5.639303in}{1.660897in}}{\pgfqpoint{5.632461in}{1.658063in}}{\pgfqpoint{5.627417in}{1.653019in}}%
\pgfpathcurveto{\pgfqpoint{5.622374in}{1.647976in}}{\pgfqpoint{5.619540in}{1.641134in}}{\pgfqpoint{5.619540in}{1.634001in}}%
\pgfpathcurveto{\pgfqpoint{5.619540in}{1.626868in}}{\pgfqpoint{5.622374in}{1.620027in}}{\pgfqpoint{5.627417in}{1.614983in}}%
\pgfpathcurveto{\pgfqpoint{5.632461in}{1.609939in}}{\pgfqpoint{5.639303in}{1.607105in}}{\pgfqpoint{5.646436in}{1.607105in}}%
\pgfpathclose%
\pgfusepath{stroke,fill}%
\end{pgfscope}%
\begin{pgfscope}%
\pgfpathrectangle{\pgfqpoint{4.985294in}{0.500000in}}{\pgfqpoint{1.764706in}{1.700000in}}%
\pgfusepath{clip}%
\pgfsetbuttcap%
\pgfsetroundjoin%
\definecolor{currentfill}{rgb}{0.969359,0.803954,0.693832}%
\pgfsetfillcolor{currentfill}%
\pgfsetlinewidth{0.311001pt}%
\definecolor{currentstroke}{rgb}{1.000000,1.000000,1.000000}%
\pgfsetstrokecolor{currentstroke}%
\pgfsetdash{}{0pt}%
\pgfpathmoveto{\pgfqpoint{5.405450in}{1.558870in}}%
\pgfpathcurveto{\pgfqpoint{5.412583in}{1.558870in}}{\pgfqpoint{5.419425in}{1.561704in}}{\pgfqpoint{5.424468in}{1.566748in}}%
\pgfpathcurveto{\pgfqpoint{5.429512in}{1.571792in}}{\pgfqpoint{5.432346in}{1.578633in}}{\pgfqpoint{5.432346in}{1.585766in}}%
\pgfpathcurveto{\pgfqpoint{5.432346in}{1.592899in}}{\pgfqpoint{5.429512in}{1.599740in}}{\pgfqpoint{5.424468in}{1.604784in}}%
\pgfpathcurveto{\pgfqpoint{5.419425in}{1.609828in}}{\pgfqpoint{5.412583in}{1.612662in}}{\pgfqpoint{5.405450in}{1.612662in}}%
\pgfpathcurveto{\pgfqpoint{5.398317in}{1.612662in}}{\pgfqpoint{5.391476in}{1.609828in}}{\pgfqpoint{5.386432in}{1.604784in}}%
\pgfpathcurveto{\pgfqpoint{5.381388in}{1.599740in}}{\pgfqpoint{5.378554in}{1.592899in}}{\pgfqpoint{5.378554in}{1.585766in}}%
\pgfpathcurveto{\pgfqpoint{5.378554in}{1.578633in}}{\pgfqpoint{5.381388in}{1.571792in}}{\pgfqpoint{5.386432in}{1.566748in}}%
\pgfpathcurveto{\pgfqpoint{5.391476in}{1.561704in}}{\pgfqpoint{5.398317in}{1.558870in}}{\pgfqpoint{5.405450in}{1.558870in}}%
\pgfpathclose%
\pgfusepath{stroke,fill}%
\end{pgfscope}%
\begin{pgfscope}%
\pgfpathrectangle{\pgfqpoint{4.985294in}{0.500000in}}{\pgfqpoint{1.764706in}{1.700000in}}%
\pgfusepath{clip}%
\pgfsetbuttcap%
\pgfsetroundjoin%
\definecolor{currentfill}{rgb}{0.965042,0.701564,0.552889}%
\pgfsetfillcolor{currentfill}%
\pgfsetlinewidth{0.311001pt}%
\definecolor{currentstroke}{rgb}{1.000000,1.000000,1.000000}%
\pgfsetstrokecolor{currentstroke}%
\pgfsetdash{}{0pt}%
\pgfpathmoveto{\pgfqpoint{5.425801in}{1.655171in}}%
\pgfpathcurveto{\pgfqpoint{5.432934in}{1.655171in}}{\pgfqpoint{5.439776in}{1.658005in}}{\pgfqpoint{5.444820in}{1.663049in}}%
\pgfpathcurveto{\pgfqpoint{5.449863in}{1.668093in}}{\pgfqpoint{5.452697in}{1.674934in}}{\pgfqpoint{5.452697in}{1.682067in}}%
\pgfpathcurveto{\pgfqpoint{5.452697in}{1.689200in}}{\pgfqpoint{5.449863in}{1.696041in}}{\pgfqpoint{5.444820in}{1.701085in}}%
\pgfpathcurveto{\pgfqpoint{5.439776in}{1.706129in}}{\pgfqpoint{5.432934in}{1.708963in}}{\pgfqpoint{5.425801in}{1.708963in}}%
\pgfpathcurveto{\pgfqpoint{5.418669in}{1.708963in}}{\pgfqpoint{5.411827in}{1.706129in}}{\pgfqpoint{5.406783in}{1.701085in}}%
\pgfpathcurveto{\pgfqpoint{5.401740in}{1.696041in}}{\pgfqpoint{5.398906in}{1.689200in}}{\pgfqpoint{5.398906in}{1.682067in}}%
\pgfpathcurveto{\pgfqpoint{5.398906in}{1.674934in}}{\pgfqpoint{5.401740in}{1.668093in}}{\pgfqpoint{5.406783in}{1.663049in}}%
\pgfpathcurveto{\pgfqpoint{5.411827in}{1.658005in}}{\pgfqpoint{5.418669in}{1.655171in}}{\pgfqpoint{5.425801in}{1.655171in}}%
\pgfpathclose%
\pgfusepath{stroke,fill}%
\end{pgfscope}%
\begin{pgfscope}%
\pgfpathrectangle{\pgfqpoint{4.985294in}{0.500000in}}{\pgfqpoint{1.764706in}{1.700000in}}%
\pgfusepath{clip}%
\pgfsetbuttcap%
\pgfsetroundjoin%
\definecolor{currentfill}{rgb}{0.972201,0.839051,0.745789}%
\pgfsetfillcolor{currentfill}%
\pgfsetlinewidth{0.311001pt}%
\definecolor{currentstroke}{rgb}{1.000000,1.000000,1.000000}%
\pgfsetstrokecolor{currentstroke}%
\pgfsetdash{}{0pt}%
\pgfpathmoveto{\pgfqpoint{6.374039in}{1.356716in}}%
\pgfpathcurveto{\pgfqpoint{6.381172in}{1.356716in}}{\pgfqpoint{6.388014in}{1.359549in}}{\pgfqpoint{6.393057in}{1.364593in}}%
\pgfpathcurveto{\pgfqpoint{6.398101in}{1.369637in}}{\pgfqpoint{6.400935in}{1.376478in}}{\pgfqpoint{6.400935in}{1.383611in}}%
\pgfpathcurveto{\pgfqpoint{6.400935in}{1.390744in}}{\pgfqpoint{6.398101in}{1.397586in}}{\pgfqpoint{6.393057in}{1.402629in}}%
\pgfpathcurveto{\pgfqpoint{6.388014in}{1.407673in}}{\pgfqpoint{6.381172in}{1.410507in}}{\pgfqpoint{6.374039in}{1.410507in}}%
\pgfpathcurveto{\pgfqpoint{6.366906in}{1.410507in}}{\pgfqpoint{6.360065in}{1.407673in}}{\pgfqpoint{6.355021in}{1.402629in}}%
\pgfpathcurveto{\pgfqpoint{6.349977in}{1.397586in}}{\pgfqpoint{6.347143in}{1.390744in}}{\pgfqpoint{6.347143in}{1.383611in}}%
\pgfpathcurveto{\pgfqpoint{6.347143in}{1.376478in}}{\pgfqpoint{6.349977in}{1.369637in}}{\pgfqpoint{6.355021in}{1.364593in}}%
\pgfpathcurveto{\pgfqpoint{6.360065in}{1.359549in}}{\pgfqpoint{6.366906in}{1.356716in}}{\pgfqpoint{6.374039in}{1.356716in}}%
\pgfpathclose%
\pgfusepath{stroke,fill}%
\end{pgfscope}%
\begin{pgfscope}%
\pgfpathrectangle{\pgfqpoint{4.985294in}{0.500000in}}{\pgfqpoint{1.764706in}{1.700000in}}%
\pgfusepath{clip}%
\pgfsetbuttcap%
\pgfsetroundjoin%
\definecolor{currentfill}{rgb}{0.975018,0.868213,0.788710}%
\pgfsetfillcolor{currentfill}%
\pgfsetlinewidth{0.311001pt}%
\definecolor{currentstroke}{rgb}{1.000000,1.000000,1.000000}%
\pgfsetstrokecolor{currentstroke}%
\pgfsetdash{}{0pt}%
\pgfpathmoveto{\pgfqpoint{6.233116in}{1.095468in}}%
\pgfpathcurveto{\pgfqpoint{6.240248in}{1.095468in}}{\pgfqpoint{6.247090in}{1.098302in}}{\pgfqpoint{6.252134in}{1.103345in}}%
\pgfpathcurveto{\pgfqpoint{6.257177in}{1.108389in}}{\pgfqpoint{6.260011in}{1.115231in}}{\pgfqpoint{6.260011in}{1.122363in}}%
\pgfpathcurveto{\pgfqpoint{6.260011in}{1.129496in}}{\pgfqpoint{6.257177in}{1.136338in}}{\pgfqpoint{6.252134in}{1.141382in}}%
\pgfpathcurveto{\pgfqpoint{6.247090in}{1.146425in}}{\pgfqpoint{6.240248in}{1.149259in}}{\pgfqpoint{6.233116in}{1.149259in}}%
\pgfpathcurveto{\pgfqpoint{6.225983in}{1.149259in}}{\pgfqpoint{6.219141in}{1.146425in}}{\pgfqpoint{6.214097in}{1.141382in}}%
\pgfpathcurveto{\pgfqpoint{6.209054in}{1.136338in}}{\pgfqpoint{6.206220in}{1.129496in}}{\pgfqpoint{6.206220in}{1.122363in}}%
\pgfpathcurveto{\pgfqpoint{6.206220in}{1.115231in}}{\pgfqpoint{6.209054in}{1.108389in}}{\pgfqpoint{6.214097in}{1.103345in}}%
\pgfpathcurveto{\pgfqpoint{6.219141in}{1.098302in}}{\pgfqpoint{6.225983in}{1.095468in}}{\pgfqpoint{6.233116in}{1.095468in}}%
\pgfpathclose%
\pgfusepath{stroke,fill}%
\end{pgfscope}%
\begin{pgfscope}%
\pgfpathrectangle{\pgfqpoint{4.985294in}{0.500000in}}{\pgfqpoint{1.764706in}{1.700000in}}%
\pgfusepath{clip}%
\pgfsetbuttcap%
\pgfsetroundjoin%
\definecolor{currentfill}{rgb}{0.730358,0.086862,0.337485}%
\pgfsetfillcolor{currentfill}%
\pgfsetlinewidth{0.311001pt}%
\definecolor{currentstroke}{rgb}{1.000000,1.000000,1.000000}%
\pgfsetstrokecolor{currentstroke}%
\pgfsetdash{}{0pt}%
\pgfpathmoveto{\pgfqpoint{6.099105in}{1.245361in}}%
\pgfpathcurveto{\pgfqpoint{6.106238in}{1.245361in}}{\pgfqpoint{6.113080in}{1.248195in}}{\pgfqpoint{6.118124in}{1.253239in}}%
\pgfpathcurveto{\pgfqpoint{6.123167in}{1.258283in}}{\pgfqpoint{6.126001in}{1.265124in}}{\pgfqpoint{6.126001in}{1.272257in}}%
\pgfpathcurveto{\pgfqpoint{6.126001in}{1.279390in}}{\pgfqpoint{6.123167in}{1.286232in}}{\pgfqpoint{6.118124in}{1.291275in}}%
\pgfpathcurveto{\pgfqpoint{6.113080in}{1.296319in}}{\pgfqpoint{6.106238in}{1.299153in}}{\pgfqpoint{6.099105in}{1.299153in}}%
\pgfpathcurveto{\pgfqpoint{6.091973in}{1.299153in}}{\pgfqpoint{6.085131in}{1.296319in}}{\pgfqpoint{6.080087in}{1.291275in}}%
\pgfpathcurveto{\pgfqpoint{6.075044in}{1.286232in}}{\pgfqpoint{6.072210in}{1.279390in}}{\pgfqpoint{6.072210in}{1.272257in}}%
\pgfpathcurveto{\pgfqpoint{6.072210in}{1.265124in}}{\pgfqpoint{6.075044in}{1.258283in}}{\pgfqpoint{6.080087in}{1.253239in}}%
\pgfpathcurveto{\pgfqpoint{6.085131in}{1.248195in}}{\pgfqpoint{6.091973in}{1.245361in}}{\pgfqpoint{6.099105in}{1.245361in}}%
\pgfpathclose%
\pgfusepath{stroke,fill}%
\end{pgfscope}%
\begin{pgfscope}%
\pgfpathrectangle{\pgfqpoint{4.985294in}{0.500000in}}{\pgfqpoint{1.764706in}{1.700000in}}%
\pgfusepath{clip}%
\pgfsetbuttcap%
\pgfsetroundjoin%
\definecolor{currentfill}{rgb}{0.944085,0.383081,0.267220}%
\pgfsetfillcolor{currentfill}%
\pgfsetlinewidth{0.311001pt}%
\definecolor{currentstroke}{rgb}{1.000000,1.000000,1.000000}%
\pgfsetstrokecolor{currentstroke}%
\pgfsetdash{}{0pt}%
\pgfpathmoveto{\pgfqpoint{6.137842in}{1.271815in}}%
\pgfpathcurveto{\pgfqpoint{6.144975in}{1.271815in}}{\pgfqpoint{6.151816in}{1.274649in}}{\pgfqpoint{6.156860in}{1.279692in}}%
\pgfpathcurveto{\pgfqpoint{6.161904in}{1.284736in}}{\pgfqpoint{6.164738in}{1.291578in}}{\pgfqpoint{6.164738in}{1.298711in}}%
\pgfpathcurveto{\pgfqpoint{6.164738in}{1.305843in}}{\pgfqpoint{6.161904in}{1.312685in}}{\pgfqpoint{6.156860in}{1.317729in}}%
\pgfpathcurveto{\pgfqpoint{6.151816in}{1.322772in}}{\pgfqpoint{6.144975in}{1.325606in}}{\pgfqpoint{6.137842in}{1.325606in}}%
\pgfpathcurveto{\pgfqpoint{6.130709in}{1.325606in}}{\pgfqpoint{6.123867in}{1.322772in}}{\pgfqpoint{6.118824in}{1.317729in}}%
\pgfpathcurveto{\pgfqpoint{6.113780in}{1.312685in}}{\pgfqpoint{6.110946in}{1.305843in}}{\pgfqpoint{6.110946in}{1.298711in}}%
\pgfpathcurveto{\pgfqpoint{6.110946in}{1.291578in}}{\pgfqpoint{6.113780in}{1.284736in}}{\pgfqpoint{6.118824in}{1.279692in}}%
\pgfpathcurveto{\pgfqpoint{6.123867in}{1.274649in}}{\pgfqpoint{6.130709in}{1.271815in}}{\pgfqpoint{6.137842in}{1.271815in}}%
\pgfpathclose%
\pgfusepath{stroke,fill}%
\end{pgfscope}%
\begin{pgfscope}%
\pgfpathrectangle{\pgfqpoint{4.985294in}{0.500000in}}{\pgfqpoint{1.764706in}{1.700000in}}%
\pgfusepath{clip}%
\pgfsetbuttcap%
\pgfsetroundjoin%
\definecolor{currentfill}{rgb}{0.975018,0.868213,0.788710}%
\pgfsetfillcolor{currentfill}%
\pgfsetlinewidth{0.311001pt}%
\definecolor{currentstroke}{rgb}{1.000000,1.000000,1.000000}%
\pgfsetstrokecolor{currentstroke}%
\pgfsetdash{}{0pt}%
\pgfpathmoveto{\pgfqpoint{5.490067in}{1.069572in}}%
\pgfpathcurveto{\pgfqpoint{5.497199in}{1.069572in}}{\pgfqpoint{5.504041in}{1.072406in}}{\pgfqpoint{5.509085in}{1.077449in}}%
\pgfpathcurveto{\pgfqpoint{5.514128in}{1.082493in}}{\pgfqpoint{5.516962in}{1.089335in}}{\pgfqpoint{5.516962in}{1.096467in}}%
\pgfpathcurveto{\pgfqpoint{5.516962in}{1.103600in}}{\pgfqpoint{5.514128in}{1.110442in}}{\pgfqpoint{5.509085in}{1.115486in}}%
\pgfpathcurveto{\pgfqpoint{5.504041in}{1.120529in}}{\pgfqpoint{5.497199in}{1.123363in}}{\pgfqpoint{5.490067in}{1.123363in}}%
\pgfpathcurveto{\pgfqpoint{5.482934in}{1.123363in}}{\pgfqpoint{5.476092in}{1.120529in}}{\pgfqpoint{5.471048in}{1.115486in}}%
\pgfpathcurveto{\pgfqpoint{5.466005in}{1.110442in}}{\pgfqpoint{5.463171in}{1.103600in}}{\pgfqpoint{5.463171in}{1.096467in}}%
\pgfpathcurveto{\pgfqpoint{5.463171in}{1.089335in}}{\pgfqpoint{5.466005in}{1.082493in}}{\pgfqpoint{5.471048in}{1.077449in}}%
\pgfpathcurveto{\pgfqpoint{5.476092in}{1.072406in}}{\pgfqpoint{5.482934in}{1.069572in}}{\pgfqpoint{5.490067in}{1.069572in}}%
\pgfpathclose%
\pgfusepath{stroke,fill}%
\end{pgfscope}%
\begin{pgfscope}%
\pgfpathrectangle{\pgfqpoint{4.985294in}{0.500000in}}{\pgfqpoint{1.764706in}{1.700000in}}%
\pgfusepath{clip}%
\pgfsetbuttcap%
\pgfsetroundjoin%
\definecolor{currentfill}{rgb}{0.969359,0.803954,0.693832}%
\pgfsetfillcolor{currentfill}%
\pgfsetlinewidth{0.311001pt}%
\definecolor{currentstroke}{rgb}{1.000000,1.000000,1.000000}%
\pgfsetstrokecolor{currentstroke}%
\pgfsetdash{}{0pt}%
\pgfpathmoveto{\pgfqpoint{5.543081in}{1.054099in}}%
\pgfpathcurveto{\pgfqpoint{5.550214in}{1.054099in}}{\pgfqpoint{5.557056in}{1.056933in}}{\pgfqpoint{5.562099in}{1.061977in}}%
\pgfpathcurveto{\pgfqpoint{5.567143in}{1.067020in}}{\pgfqpoint{5.569977in}{1.073862in}}{\pgfqpoint{5.569977in}{1.080995in}}%
\pgfpathcurveto{\pgfqpoint{5.569977in}{1.088128in}}{\pgfqpoint{5.567143in}{1.094969in}}{\pgfqpoint{5.562099in}{1.100013in}}%
\pgfpathcurveto{\pgfqpoint{5.557056in}{1.105057in}}{\pgfqpoint{5.550214in}{1.107890in}}{\pgfqpoint{5.543081in}{1.107890in}}%
\pgfpathcurveto{\pgfqpoint{5.535948in}{1.107890in}}{\pgfqpoint{5.529107in}{1.105057in}}{\pgfqpoint{5.524063in}{1.100013in}}%
\pgfpathcurveto{\pgfqpoint{5.519019in}{1.094969in}}{\pgfqpoint{5.516186in}{1.088128in}}{\pgfqpoint{5.516186in}{1.080995in}}%
\pgfpathcurveto{\pgfqpoint{5.516186in}{1.073862in}}{\pgfqpoint{5.519019in}{1.067020in}}{\pgfqpoint{5.524063in}{1.061977in}}%
\pgfpathcurveto{\pgfqpoint{5.529107in}{1.056933in}}{\pgfqpoint{5.535948in}{1.054099in}}{\pgfqpoint{5.543081in}{1.054099in}}%
\pgfpathclose%
\pgfusepath{stroke,fill}%
\end{pgfscope}%
\begin{pgfscope}%
\pgfpathrectangle{\pgfqpoint{4.985294in}{0.500000in}}{\pgfqpoint{1.764706in}{1.700000in}}%
\pgfusepath{clip}%
\pgfsetbuttcap%
\pgfsetroundjoin%
\definecolor{currentfill}{rgb}{0.975018,0.868213,0.788710}%
\pgfsetfillcolor{currentfill}%
\pgfsetlinewidth{0.311001pt}%
\definecolor{currentstroke}{rgb}{1.000000,1.000000,1.000000}%
\pgfsetstrokecolor{currentstroke}%
\pgfsetdash{}{0pt}%
\pgfpathmoveto{\pgfqpoint{5.456273in}{1.550791in}}%
\pgfpathcurveto{\pgfqpoint{5.463406in}{1.550791in}}{\pgfqpoint{5.470248in}{1.553625in}}{\pgfqpoint{5.475291in}{1.558668in}}%
\pgfpathcurveto{\pgfqpoint{5.480335in}{1.563712in}}{\pgfqpoint{5.483169in}{1.570554in}}{\pgfqpoint{5.483169in}{1.577686in}}%
\pgfpathcurveto{\pgfqpoint{5.483169in}{1.584819in}}{\pgfqpoint{5.480335in}{1.591661in}}{\pgfqpoint{5.475291in}{1.596705in}}%
\pgfpathcurveto{\pgfqpoint{5.470248in}{1.601748in}}{\pgfqpoint{5.463406in}{1.604582in}}{\pgfqpoint{5.456273in}{1.604582in}}%
\pgfpathcurveto{\pgfqpoint{5.449140in}{1.604582in}}{\pgfqpoint{5.442299in}{1.601748in}}{\pgfqpoint{5.437255in}{1.596705in}}%
\pgfpathcurveto{\pgfqpoint{5.432211in}{1.591661in}}{\pgfqpoint{5.429377in}{1.584819in}}{\pgfqpoint{5.429377in}{1.577686in}}%
\pgfpathcurveto{\pgfqpoint{5.429377in}{1.570554in}}{\pgfqpoint{5.432211in}{1.563712in}}{\pgfqpoint{5.437255in}{1.558668in}}%
\pgfpathcurveto{\pgfqpoint{5.442299in}{1.553625in}}{\pgfqpoint{5.449140in}{1.550791in}}{\pgfqpoint{5.456273in}{1.550791in}}%
\pgfpathclose%
\pgfusepath{stroke,fill}%
\end{pgfscope}%
\begin{pgfscope}%
\pgfpathrectangle{\pgfqpoint{4.985294in}{0.500000in}}{\pgfqpoint{1.764706in}{1.700000in}}%
\pgfusepath{clip}%
\pgfsetbuttcap%
\pgfsetroundjoin%
\definecolor{currentfill}{rgb}{0.965169,0.707764,0.560659}%
\pgfsetfillcolor{currentfill}%
\pgfsetlinewidth{0.311001pt}%
\definecolor{currentstroke}{rgb}{1.000000,1.000000,1.000000}%
\pgfsetstrokecolor{currentstroke}%
\pgfsetdash{}{0pt}%
\pgfpathmoveto{\pgfqpoint{6.195394in}{1.413298in}}%
\pgfpathcurveto{\pgfqpoint{6.202527in}{1.413298in}}{\pgfqpoint{6.209368in}{1.416132in}}{\pgfqpoint{6.214412in}{1.421176in}}%
\pgfpathcurveto{\pgfqpoint{6.219456in}{1.426220in}}{\pgfqpoint{6.222289in}{1.433061in}}{\pgfqpoint{6.222289in}{1.440194in}}%
\pgfpathcurveto{\pgfqpoint{6.222289in}{1.447327in}}{\pgfqpoint{6.219456in}{1.454169in}}{\pgfqpoint{6.214412in}{1.459212in}}%
\pgfpathcurveto{\pgfqpoint{6.209368in}{1.464256in}}{\pgfqpoint{6.202527in}{1.467090in}}{\pgfqpoint{6.195394in}{1.467090in}}%
\pgfpathcurveto{\pgfqpoint{6.188261in}{1.467090in}}{\pgfqpoint{6.181419in}{1.464256in}}{\pgfqpoint{6.176376in}{1.459212in}}%
\pgfpathcurveto{\pgfqpoint{6.171332in}{1.454169in}}{\pgfqpoint{6.168498in}{1.447327in}}{\pgfqpoint{6.168498in}{1.440194in}}%
\pgfpathcurveto{\pgfqpoint{6.168498in}{1.433061in}}{\pgfqpoint{6.171332in}{1.426220in}}{\pgfqpoint{6.176376in}{1.421176in}}%
\pgfpathcurveto{\pgfqpoint{6.181419in}{1.416132in}}{\pgfqpoint{6.188261in}{1.413298in}}{\pgfqpoint{6.195394in}{1.413298in}}%
\pgfpathclose%
\pgfusepath{stroke,fill}%
\end{pgfscope}%
\begin{pgfscope}%
\pgfpathrectangle{\pgfqpoint{4.985294in}{0.500000in}}{\pgfqpoint{1.764706in}{1.700000in}}%
\pgfusepath{clip}%
\pgfsetbuttcap%
\pgfsetroundjoin%
\definecolor{currentfill}{rgb}{0.973832,0.856556,0.771584}%
\pgfsetfillcolor{currentfill}%
\pgfsetlinewidth{0.311001pt}%
\definecolor{currentstroke}{rgb}{1.000000,1.000000,1.000000}%
\pgfsetstrokecolor{currentstroke}%
\pgfsetdash{}{0pt}%
\pgfpathmoveto{\pgfqpoint{6.362789in}{1.380396in}}%
\pgfpathcurveto{\pgfqpoint{6.369922in}{1.380396in}}{\pgfqpoint{6.376763in}{1.383230in}}{\pgfqpoint{6.381807in}{1.388274in}}%
\pgfpathcurveto{\pgfqpoint{6.386851in}{1.393318in}}{\pgfqpoint{6.389684in}{1.400159in}}{\pgfqpoint{6.389684in}{1.407292in}}%
\pgfpathcurveto{\pgfqpoint{6.389684in}{1.414425in}}{\pgfqpoint{6.386851in}{1.421267in}}{\pgfqpoint{6.381807in}{1.426310in}}%
\pgfpathcurveto{\pgfqpoint{6.376763in}{1.431354in}}{\pgfqpoint{6.369922in}{1.434188in}}{\pgfqpoint{6.362789in}{1.434188in}}%
\pgfpathcurveto{\pgfqpoint{6.355656in}{1.434188in}}{\pgfqpoint{6.348814in}{1.431354in}}{\pgfqpoint{6.343771in}{1.426310in}}%
\pgfpathcurveto{\pgfqpoint{6.338727in}{1.421267in}}{\pgfqpoint{6.335893in}{1.414425in}}{\pgfqpoint{6.335893in}{1.407292in}}%
\pgfpathcurveto{\pgfqpoint{6.335893in}{1.400159in}}{\pgfqpoint{6.338727in}{1.393318in}}{\pgfqpoint{6.343771in}{1.388274in}}%
\pgfpathcurveto{\pgfqpoint{6.348814in}{1.383230in}}{\pgfqpoint{6.355656in}{1.380396in}}{\pgfqpoint{6.362789in}{1.380396in}}%
\pgfpathclose%
\pgfusepath{stroke,fill}%
\end{pgfscope}%
\begin{pgfscope}%
\pgfpathrectangle{\pgfqpoint{4.985294in}{0.500000in}}{\pgfqpoint{1.764706in}{1.700000in}}%
\pgfusepath{clip}%
\pgfsetbuttcap%
\pgfsetroundjoin%
\definecolor{currentfill}{rgb}{0.975018,0.868213,0.788710}%
\pgfsetfillcolor{currentfill}%
\pgfsetlinewidth{0.311001pt}%
\definecolor{currentstroke}{rgb}{1.000000,1.000000,1.000000}%
\pgfsetstrokecolor{currentstroke}%
\pgfsetdash{}{0pt}%
\pgfpathmoveto{\pgfqpoint{5.371937in}{1.317901in}}%
\pgfpathcurveto{\pgfqpoint{5.379070in}{1.317901in}}{\pgfqpoint{5.385912in}{1.320735in}}{\pgfqpoint{5.390956in}{1.325779in}}%
\pgfpathcurveto{\pgfqpoint{5.395999in}{1.330823in}}{\pgfqpoint{5.398833in}{1.337664in}}{\pgfqpoint{5.398833in}{1.344797in}}%
\pgfpathcurveto{\pgfqpoint{5.398833in}{1.351930in}}{\pgfqpoint{5.395999in}{1.358772in}}{\pgfqpoint{5.390956in}{1.363815in}}%
\pgfpathcurveto{\pgfqpoint{5.385912in}{1.368859in}}{\pgfqpoint{5.379070in}{1.371693in}}{\pgfqpoint{5.371937in}{1.371693in}}%
\pgfpathcurveto{\pgfqpoint{5.364805in}{1.371693in}}{\pgfqpoint{5.357963in}{1.368859in}}{\pgfqpoint{5.352919in}{1.363815in}}%
\pgfpathcurveto{\pgfqpoint{5.347876in}{1.358772in}}{\pgfqpoint{5.345042in}{1.351930in}}{\pgfqpoint{5.345042in}{1.344797in}}%
\pgfpathcurveto{\pgfqpoint{5.345042in}{1.337664in}}{\pgfqpoint{5.347876in}{1.330823in}}{\pgfqpoint{5.352919in}{1.325779in}}%
\pgfpathcurveto{\pgfqpoint{5.357963in}{1.320735in}}{\pgfqpoint{5.364805in}{1.317901in}}{\pgfqpoint{5.371937in}{1.317901in}}%
\pgfpathclose%
\pgfusepath{stroke,fill}%
\end{pgfscope}%
\begin{pgfscope}%
\pgfpathrectangle{\pgfqpoint{4.985294in}{0.500000in}}{\pgfqpoint{1.764706in}{1.700000in}}%
\pgfusepath{clip}%
\pgfsetbuttcap%
\pgfsetroundjoin%
\definecolor{currentfill}{rgb}{0.973832,0.856556,0.771584}%
\pgfsetfillcolor{currentfill}%
\pgfsetlinewidth{0.311001pt}%
\definecolor{currentstroke}{rgb}{1.000000,1.000000,1.000000}%
\pgfsetstrokecolor{currentstroke}%
\pgfsetdash{}{0pt}%
\pgfpathmoveto{\pgfqpoint{6.196934in}{1.615482in}}%
\pgfpathcurveto{\pgfqpoint{6.204067in}{1.615482in}}{\pgfqpoint{6.210909in}{1.618315in}}{\pgfqpoint{6.215953in}{1.623359in}}%
\pgfpathcurveto{\pgfqpoint{6.220996in}{1.628403in}}{\pgfqpoint{6.223830in}{1.635244in}}{\pgfqpoint{6.223830in}{1.642377in}}%
\pgfpathcurveto{\pgfqpoint{6.223830in}{1.649510in}}{\pgfqpoint{6.220996in}{1.656352in}}{\pgfqpoint{6.215953in}{1.661395in}}%
\pgfpathcurveto{\pgfqpoint{6.210909in}{1.666439in}}{\pgfqpoint{6.204067in}{1.669273in}}{\pgfqpoint{6.196934in}{1.669273in}}%
\pgfpathcurveto{\pgfqpoint{6.189802in}{1.669273in}}{\pgfqpoint{6.182960in}{1.666439in}}{\pgfqpoint{6.177916in}{1.661395in}}%
\pgfpathcurveto{\pgfqpoint{6.172873in}{1.656352in}}{\pgfqpoint{6.170039in}{1.649510in}}{\pgfqpoint{6.170039in}{1.642377in}}%
\pgfpathcurveto{\pgfqpoint{6.170039in}{1.635244in}}{\pgfqpoint{6.172873in}{1.628403in}}{\pgfqpoint{6.177916in}{1.623359in}}%
\pgfpathcurveto{\pgfqpoint{6.182960in}{1.618315in}}{\pgfqpoint{6.189802in}{1.615482in}}{\pgfqpoint{6.196934in}{1.615482in}}%
\pgfpathclose%
\pgfusepath{stroke,fill}%
\end{pgfscope}%
\begin{pgfscope}%
\pgfpathrectangle{\pgfqpoint{4.985294in}{0.500000in}}{\pgfqpoint{1.764706in}{1.700000in}}%
\pgfusepath{clip}%
\pgfsetbuttcap%
\pgfsetroundjoin%
\definecolor{currentfill}{rgb}{0.961734,0.579886,0.418445}%
\pgfsetfillcolor{currentfill}%
\pgfsetlinewidth{0.311001pt}%
\definecolor{currentstroke}{rgb}{1.000000,1.000000,1.000000}%
\pgfsetstrokecolor{currentstroke}%
\pgfsetdash{}{0pt}%
\pgfpathmoveto{\pgfqpoint{6.231373in}{0.887208in}}%
\pgfpathcurveto{\pgfqpoint{6.238506in}{0.887208in}}{\pgfqpoint{6.245347in}{0.890042in}}{\pgfqpoint{6.250391in}{0.895085in}}%
\pgfpathcurveto{\pgfqpoint{6.255435in}{0.900129in}}{\pgfqpoint{6.258269in}{0.906971in}}{\pgfqpoint{6.258269in}{0.914103in}}%
\pgfpathcurveto{\pgfqpoint{6.258269in}{0.921236in}}{\pgfqpoint{6.255435in}{0.928078in}}{\pgfqpoint{6.250391in}{0.933122in}}%
\pgfpathcurveto{\pgfqpoint{6.245347in}{0.938165in}}{\pgfqpoint{6.238506in}{0.940999in}}{\pgfqpoint{6.231373in}{0.940999in}}%
\pgfpathcurveto{\pgfqpoint{6.224240in}{0.940999in}}{\pgfqpoint{6.217399in}{0.938165in}}{\pgfqpoint{6.212355in}{0.933122in}}%
\pgfpathcurveto{\pgfqpoint{6.207311in}{0.928078in}}{\pgfqpoint{6.204477in}{0.921236in}}{\pgfqpoint{6.204477in}{0.914103in}}%
\pgfpathcurveto{\pgfqpoint{6.204477in}{0.906971in}}{\pgfqpoint{6.207311in}{0.900129in}}{\pgfqpoint{6.212355in}{0.895085in}}%
\pgfpathcurveto{\pgfqpoint{6.217399in}{0.890042in}}{\pgfqpoint{6.224240in}{0.887208in}}{\pgfqpoint{6.231373in}{0.887208in}}%
\pgfpathclose%
\pgfusepath{stroke,fill}%
\end{pgfscope}%
\begin{pgfscope}%
\pgfpathrectangle{\pgfqpoint{4.985294in}{0.500000in}}{\pgfqpoint{1.764706in}{1.700000in}}%
\pgfusepath{clip}%
\pgfsetbuttcap%
\pgfsetroundjoin%
\definecolor{currentfill}{rgb}{0.968105,0.786346,0.667739}%
\pgfsetfillcolor{currentfill}%
\pgfsetlinewidth{0.311001pt}%
\definecolor{currentstroke}{rgb}{1.000000,1.000000,1.000000}%
\pgfsetstrokecolor{currentstroke}%
\pgfsetdash{}{0pt}%
\pgfpathmoveto{\pgfqpoint{5.515047in}{1.426792in}}%
\pgfpathcurveto{\pgfqpoint{5.522180in}{1.426792in}}{\pgfqpoint{5.529021in}{1.429626in}}{\pgfqpoint{5.534065in}{1.434669in}}%
\pgfpathcurveto{\pgfqpoint{5.539109in}{1.439713in}}{\pgfqpoint{5.541943in}{1.446555in}}{\pgfqpoint{5.541943in}{1.453688in}}%
\pgfpathcurveto{\pgfqpoint{5.541943in}{1.460820in}}{\pgfqpoint{5.539109in}{1.467662in}}{\pgfqpoint{5.534065in}{1.472706in}}%
\pgfpathcurveto{\pgfqpoint{5.529021in}{1.477749in}}{\pgfqpoint{5.522180in}{1.480583in}}{\pgfqpoint{5.515047in}{1.480583in}}%
\pgfpathcurveto{\pgfqpoint{5.507914in}{1.480583in}}{\pgfqpoint{5.501072in}{1.477749in}}{\pgfqpoint{5.496029in}{1.472706in}}%
\pgfpathcurveto{\pgfqpoint{5.490985in}{1.467662in}}{\pgfqpoint{5.488151in}{1.460820in}}{\pgfqpoint{5.488151in}{1.453688in}}%
\pgfpathcurveto{\pgfqpoint{5.488151in}{1.446555in}}{\pgfqpoint{5.490985in}{1.439713in}}{\pgfqpoint{5.496029in}{1.434669in}}%
\pgfpathcurveto{\pgfqpoint{5.501072in}{1.429626in}}{\pgfqpoint{5.507914in}{1.426792in}}{\pgfqpoint{5.515047in}{1.426792in}}%
\pgfpathclose%
\pgfusepath{stroke,fill}%
\end{pgfscope}%
\begin{pgfscope}%
\pgfpathrectangle{\pgfqpoint{4.985294in}{0.500000in}}{\pgfqpoint{1.764706in}{1.700000in}}%
\pgfusepath{clip}%
\pgfsetbuttcap%
\pgfsetroundjoin%
\definecolor{currentfill}{rgb}{0.966328,0.750560,0.616961}%
\pgfsetfillcolor{currentfill}%
\pgfsetlinewidth{0.311001pt}%
\definecolor{currentstroke}{rgb}{1.000000,1.000000,1.000000}%
\pgfsetstrokecolor{currentstroke}%
\pgfsetdash{}{0pt}%
\pgfpathmoveto{\pgfqpoint{6.154479in}{0.960493in}}%
\pgfpathcurveto{\pgfqpoint{6.161612in}{0.960493in}}{\pgfqpoint{6.168454in}{0.963327in}}{\pgfqpoint{6.173497in}{0.968371in}}%
\pgfpathcurveto{\pgfqpoint{6.178541in}{0.973414in}}{\pgfqpoint{6.181375in}{0.980256in}}{\pgfqpoint{6.181375in}{0.987389in}}%
\pgfpathcurveto{\pgfqpoint{6.181375in}{0.994522in}}{\pgfqpoint{6.178541in}{1.001363in}}{\pgfqpoint{6.173497in}{1.006407in}}%
\pgfpathcurveto{\pgfqpoint{6.168454in}{1.011451in}}{\pgfqpoint{6.161612in}{1.014284in}}{\pgfqpoint{6.154479in}{1.014284in}}%
\pgfpathcurveto{\pgfqpoint{6.147346in}{1.014284in}}{\pgfqpoint{6.140505in}{1.011451in}}{\pgfqpoint{6.135461in}{1.006407in}}%
\pgfpathcurveto{\pgfqpoint{6.130417in}{1.001363in}}{\pgfqpoint{6.127584in}{0.994522in}}{\pgfqpoint{6.127584in}{0.987389in}}%
\pgfpathcurveto{\pgfqpoint{6.127584in}{0.980256in}}{\pgfqpoint{6.130417in}{0.973414in}}{\pgfqpoint{6.135461in}{0.968371in}}%
\pgfpathcurveto{\pgfqpoint{6.140505in}{0.963327in}}{\pgfqpoint{6.147346in}{0.960493in}}{\pgfqpoint{6.154479in}{0.960493in}}%
\pgfpathclose%
\pgfusepath{stroke,fill}%
\end{pgfscope}%
\begin{pgfscope}%
\pgfpathrectangle{\pgfqpoint{4.985294in}{0.500000in}}{\pgfqpoint{1.764706in}{1.700000in}}%
\pgfusepath{clip}%
\pgfsetbuttcap%
\pgfsetroundjoin%
\definecolor{currentfill}{rgb}{0.971694,0.833208,0.737161}%
\pgfsetfillcolor{currentfill}%
\pgfsetlinewidth{0.311001pt}%
\definecolor{currentstroke}{rgb}{1.000000,1.000000,1.000000}%
\pgfsetstrokecolor{currentstroke}%
\pgfsetdash{}{0pt}%
\pgfpathmoveto{\pgfqpoint{5.518429in}{1.523246in}}%
\pgfpathcurveto{\pgfqpoint{5.525562in}{1.523246in}}{\pgfqpoint{5.532404in}{1.526080in}}{\pgfqpoint{5.537448in}{1.531123in}}%
\pgfpathcurveto{\pgfqpoint{5.542491in}{1.536167in}}{\pgfqpoint{5.545325in}{1.543009in}}{\pgfqpoint{5.545325in}{1.550141in}}%
\pgfpathcurveto{\pgfqpoint{5.545325in}{1.557274in}}{\pgfqpoint{5.542491in}{1.564116in}}{\pgfqpoint{5.537448in}{1.569160in}}%
\pgfpathcurveto{\pgfqpoint{5.532404in}{1.574203in}}{\pgfqpoint{5.525562in}{1.577037in}}{\pgfqpoint{5.518429in}{1.577037in}}%
\pgfpathcurveto{\pgfqpoint{5.511297in}{1.577037in}}{\pgfqpoint{5.504455in}{1.574203in}}{\pgfqpoint{5.499411in}{1.569160in}}%
\pgfpathcurveto{\pgfqpoint{5.494368in}{1.564116in}}{\pgfqpoint{5.491534in}{1.557274in}}{\pgfqpoint{5.491534in}{1.550141in}}%
\pgfpathcurveto{\pgfqpoint{5.491534in}{1.543009in}}{\pgfqpoint{5.494368in}{1.536167in}}{\pgfqpoint{5.499411in}{1.531123in}}%
\pgfpathcurveto{\pgfqpoint{5.504455in}{1.526080in}}{\pgfqpoint{5.511297in}{1.523246in}}{\pgfqpoint{5.518429in}{1.523246in}}%
\pgfpathclose%
\pgfusepath{stroke,fill}%
\end{pgfscope}%
\begin{pgfscope}%
\pgfpathrectangle{\pgfqpoint{4.985294in}{0.500000in}}{\pgfqpoint{1.764706in}{1.700000in}}%
\pgfusepath{clip}%
\pgfsetbuttcap%
\pgfsetroundjoin%
\definecolor{currentfill}{rgb}{0.975018,0.868213,0.788710}%
\pgfsetfillcolor{currentfill}%
\pgfsetlinewidth{0.311001pt}%
\definecolor{currentstroke}{rgb}{1.000000,1.000000,1.000000}%
\pgfsetstrokecolor{currentstroke}%
\pgfsetdash{}{0pt}%
\pgfpathmoveto{\pgfqpoint{6.308427in}{1.560803in}}%
\pgfpathcurveto{\pgfqpoint{6.315560in}{1.560803in}}{\pgfqpoint{6.322402in}{1.563637in}}{\pgfqpoint{6.327445in}{1.568681in}}%
\pgfpathcurveto{\pgfqpoint{6.332489in}{1.573724in}}{\pgfqpoint{6.335323in}{1.580566in}}{\pgfqpoint{6.335323in}{1.587699in}}%
\pgfpathcurveto{\pgfqpoint{6.335323in}{1.594831in}}{\pgfqpoint{6.332489in}{1.601673in}}{\pgfqpoint{6.327445in}{1.606717in}}%
\pgfpathcurveto{\pgfqpoint{6.322402in}{1.611760in}}{\pgfqpoint{6.315560in}{1.614594in}}{\pgfqpoint{6.308427in}{1.614594in}}%
\pgfpathcurveto{\pgfqpoint{6.301294in}{1.614594in}}{\pgfqpoint{6.294453in}{1.611760in}}{\pgfqpoint{6.289409in}{1.606717in}}%
\pgfpathcurveto{\pgfqpoint{6.284365in}{1.601673in}}{\pgfqpoint{6.281531in}{1.594831in}}{\pgfqpoint{6.281531in}{1.587699in}}%
\pgfpathcurveto{\pgfqpoint{6.281531in}{1.580566in}}{\pgfqpoint{6.284365in}{1.573724in}}{\pgfqpoint{6.289409in}{1.568681in}}%
\pgfpathcurveto{\pgfqpoint{6.294453in}{1.563637in}}{\pgfqpoint{6.301294in}{1.560803in}}{\pgfqpoint{6.308427in}{1.560803in}}%
\pgfpathclose%
\pgfusepath{stroke,fill}%
\end{pgfscope}%
\begin{pgfscope}%
\pgfpathrectangle{\pgfqpoint{4.985294in}{0.500000in}}{\pgfqpoint{1.764706in}{1.700000in}}%
\pgfusepath{clip}%
\pgfsetbuttcap%
\pgfsetroundjoin%
\definecolor{currentfill}{rgb}{0.975644,0.874038,0.797253}%
\pgfsetfillcolor{currentfill}%
\pgfsetlinewidth{0.311001pt}%
\definecolor{currentstroke}{rgb}{1.000000,1.000000,1.000000}%
\pgfsetstrokecolor{currentstroke}%
\pgfsetdash{}{0pt}%
\pgfpathmoveto{\pgfqpoint{5.470739in}{1.423456in}}%
\pgfpathcurveto{\pgfqpoint{5.477872in}{1.423456in}}{\pgfqpoint{5.484714in}{1.426289in}}{\pgfqpoint{5.489757in}{1.431333in}}%
\pgfpathcurveto{\pgfqpoint{5.494801in}{1.436377in}}{\pgfqpoint{5.497635in}{1.443218in}}{\pgfqpoint{5.497635in}{1.450351in}}%
\pgfpathcurveto{\pgfqpoint{5.497635in}{1.457484in}}{\pgfqpoint{5.494801in}{1.464326in}}{\pgfqpoint{5.489757in}{1.469369in}}%
\pgfpathcurveto{\pgfqpoint{5.484714in}{1.474413in}}{\pgfqpoint{5.477872in}{1.477247in}}{\pgfqpoint{5.470739in}{1.477247in}}%
\pgfpathcurveto{\pgfqpoint{5.463606in}{1.477247in}}{\pgfqpoint{5.456765in}{1.474413in}}{\pgfqpoint{5.451721in}{1.469369in}}%
\pgfpathcurveto{\pgfqpoint{5.446677in}{1.464326in}}{\pgfqpoint{5.443844in}{1.457484in}}{\pgfqpoint{5.443844in}{1.450351in}}%
\pgfpathcurveto{\pgfqpoint{5.443844in}{1.443218in}}{\pgfqpoint{5.446677in}{1.436377in}}{\pgfqpoint{5.451721in}{1.431333in}}%
\pgfpathcurveto{\pgfqpoint{5.456765in}{1.426289in}}{\pgfqpoint{5.463606in}{1.423456in}}{\pgfqpoint{5.470739in}{1.423456in}}%
\pgfpathclose%
\pgfusepath{stroke,fill}%
\end{pgfscope}%
\begin{pgfscope}%
\pgfpathrectangle{\pgfqpoint{4.985294in}{0.500000in}}{\pgfqpoint{1.764706in}{1.700000in}}%
\pgfusepath{clip}%
\pgfsetbuttcap%
\pgfsetroundjoin%
\definecolor{currentfill}{rgb}{0.978376,0.897317,0.831308}%
\pgfsetfillcolor{currentfill}%
\pgfsetlinewidth{0.311001pt}%
\definecolor{currentstroke}{rgb}{1.000000,1.000000,1.000000}%
\pgfsetstrokecolor{currentstroke}%
\pgfsetdash{}{0pt}%
\pgfpathmoveto{\pgfqpoint{5.428014in}{1.470992in}}%
\pgfpathcurveto{\pgfqpoint{5.435146in}{1.470992in}}{\pgfqpoint{5.441988in}{1.473826in}}{\pgfqpoint{5.447032in}{1.478869in}}%
\pgfpathcurveto{\pgfqpoint{5.452075in}{1.483913in}}{\pgfqpoint{5.454909in}{1.490754in}}{\pgfqpoint{5.454909in}{1.497887in}}%
\pgfpathcurveto{\pgfqpoint{5.454909in}{1.505020in}}{\pgfqpoint{5.452075in}{1.511862in}}{\pgfqpoint{5.447032in}{1.516905in}}%
\pgfpathcurveto{\pgfqpoint{5.441988in}{1.521949in}}{\pgfqpoint{5.435146in}{1.524783in}}{\pgfqpoint{5.428014in}{1.524783in}}%
\pgfpathcurveto{\pgfqpoint{5.420881in}{1.524783in}}{\pgfqpoint{5.414039in}{1.521949in}}{\pgfqpoint{5.408995in}{1.516905in}}%
\pgfpathcurveto{\pgfqpoint{5.403952in}{1.511862in}}{\pgfqpoint{5.401118in}{1.505020in}}{\pgfqpoint{5.401118in}{1.497887in}}%
\pgfpathcurveto{\pgfqpoint{5.401118in}{1.490754in}}{\pgfqpoint{5.403952in}{1.483913in}}{\pgfqpoint{5.408995in}{1.478869in}}%
\pgfpathcurveto{\pgfqpoint{5.414039in}{1.473826in}}{\pgfqpoint{5.420881in}{1.470992in}}{\pgfqpoint{5.428014in}{1.470992in}}%
\pgfpathclose%
\pgfusepath{stroke,fill}%
\end{pgfscope}%
\begin{pgfscope}%
\pgfpathrectangle{\pgfqpoint{4.985294in}{0.500000in}}{\pgfqpoint{1.764706in}{1.700000in}}%
\pgfusepath{clip}%
\pgfsetbuttcap%
\pgfsetroundjoin%
\definecolor{currentfill}{rgb}{0.979891,0.908948,0.848279}%
\pgfsetfillcolor{currentfill}%
\pgfsetlinewidth{0.311001pt}%
\definecolor{currentstroke}{rgb}{1.000000,1.000000,1.000000}%
\pgfsetstrokecolor{currentstroke}%
\pgfsetdash{}{0pt}%
\pgfpathmoveto{\pgfqpoint{6.334734in}{1.249527in}}%
\pgfpathcurveto{\pgfqpoint{6.341866in}{1.249527in}}{\pgfqpoint{6.348708in}{1.252361in}}{\pgfqpoint{6.353752in}{1.257405in}}%
\pgfpathcurveto{\pgfqpoint{6.358795in}{1.262449in}}{\pgfqpoint{6.361629in}{1.269290in}}{\pgfqpoint{6.361629in}{1.276423in}}%
\pgfpathcurveto{\pgfqpoint{6.361629in}{1.283556in}}{\pgfqpoint{6.358795in}{1.290397in}}{\pgfqpoint{6.353752in}{1.295441in}}%
\pgfpathcurveto{\pgfqpoint{6.348708in}{1.300485in}}{\pgfqpoint{6.341866in}{1.303319in}}{\pgfqpoint{6.334734in}{1.303319in}}%
\pgfpathcurveto{\pgfqpoint{6.327601in}{1.303319in}}{\pgfqpoint{6.320759in}{1.300485in}}{\pgfqpoint{6.315715in}{1.295441in}}%
\pgfpathcurveto{\pgfqpoint{6.310672in}{1.290397in}}{\pgfqpoint{6.307838in}{1.283556in}}{\pgfqpoint{6.307838in}{1.276423in}}%
\pgfpathcurveto{\pgfqpoint{6.307838in}{1.269290in}}{\pgfqpoint{6.310672in}{1.262449in}}{\pgfqpoint{6.315715in}{1.257405in}}%
\pgfpathcurveto{\pgfqpoint{6.320759in}{1.252361in}}{\pgfqpoint{6.327601in}{1.249527in}}{\pgfqpoint{6.334734in}{1.249527in}}%
\pgfpathclose%
\pgfusepath{stroke,fill}%
\end{pgfscope}%
\begin{pgfscope}%
\pgfpathrectangle{\pgfqpoint{4.985294in}{0.500000in}}{\pgfqpoint{1.764706in}{1.700000in}}%
\pgfusepath{clip}%
\pgfsetbuttcap%
\pgfsetroundjoin%
\definecolor{currentfill}{rgb}{0.963559,0.632016,0.472047}%
\pgfsetfillcolor{currentfill}%
\pgfsetlinewidth{0.311001pt}%
\definecolor{currentstroke}{rgb}{1.000000,1.000000,1.000000}%
\pgfsetstrokecolor{currentstroke}%
\pgfsetdash{}{0pt}%
\pgfpathmoveto{\pgfqpoint{5.618007in}{1.655059in}}%
\pgfpathcurveto{\pgfqpoint{5.625140in}{1.655059in}}{\pgfqpoint{5.631981in}{1.657893in}}{\pgfqpoint{5.637025in}{1.662937in}}%
\pgfpathcurveto{\pgfqpoint{5.642069in}{1.667980in}}{\pgfqpoint{5.644903in}{1.674822in}}{\pgfqpoint{5.644903in}{1.681955in}}%
\pgfpathcurveto{\pgfqpoint{5.644903in}{1.689088in}}{\pgfqpoint{5.642069in}{1.695929in}}{\pgfqpoint{5.637025in}{1.700973in}}%
\pgfpathcurveto{\pgfqpoint{5.631981in}{1.706017in}}{\pgfqpoint{5.625140in}{1.708851in}}{\pgfqpoint{5.618007in}{1.708851in}}%
\pgfpathcurveto{\pgfqpoint{5.610874in}{1.708851in}}{\pgfqpoint{5.604032in}{1.706017in}}{\pgfqpoint{5.598989in}{1.700973in}}%
\pgfpathcurveto{\pgfqpoint{5.593945in}{1.695929in}}{\pgfqpoint{5.591111in}{1.689088in}}{\pgfqpoint{5.591111in}{1.681955in}}%
\pgfpathcurveto{\pgfqpoint{5.591111in}{1.674822in}}{\pgfqpoint{5.593945in}{1.667980in}}{\pgfqpoint{5.598989in}{1.662937in}}%
\pgfpathcurveto{\pgfqpoint{5.604032in}{1.657893in}}{\pgfqpoint{5.610874in}{1.655059in}}{\pgfqpoint{5.618007in}{1.655059in}}%
\pgfpathclose%
\pgfusepath{stroke,fill}%
\end{pgfscope}%
\begin{pgfscope}%
\pgfpathrectangle{\pgfqpoint{4.985294in}{0.500000in}}{\pgfqpoint{1.764706in}{1.700000in}}%
\pgfusepath{clip}%
\pgfsetbuttcap%
\pgfsetroundjoin%
\definecolor{currentfill}{rgb}{0.971202,0.827364,0.728520}%
\pgfsetfillcolor{currentfill}%
\pgfsetlinewidth{0.311001pt}%
\definecolor{currentstroke}{rgb}{1.000000,1.000000,1.000000}%
\pgfsetstrokecolor{currentstroke}%
\pgfsetdash{}{0pt}%
\pgfpathmoveto{\pgfqpoint{6.305002in}{1.595790in}}%
\pgfpathcurveto{\pgfqpoint{6.312135in}{1.595790in}}{\pgfqpoint{6.318976in}{1.598624in}}{\pgfqpoint{6.324020in}{1.603668in}}%
\pgfpathcurveto{\pgfqpoint{6.329064in}{1.608711in}}{\pgfqpoint{6.331898in}{1.615553in}}{\pgfqpoint{6.331898in}{1.622686in}}%
\pgfpathcurveto{\pgfqpoint{6.331898in}{1.629819in}}{\pgfqpoint{6.329064in}{1.636660in}}{\pgfqpoint{6.324020in}{1.641704in}}%
\pgfpathcurveto{\pgfqpoint{6.318976in}{1.646748in}}{\pgfqpoint{6.312135in}{1.649582in}}{\pgfqpoint{6.305002in}{1.649582in}}%
\pgfpathcurveto{\pgfqpoint{6.297869in}{1.649582in}}{\pgfqpoint{6.291028in}{1.646748in}}{\pgfqpoint{6.285984in}{1.641704in}}%
\pgfpathcurveto{\pgfqpoint{6.280940in}{1.636660in}}{\pgfqpoint{6.278106in}{1.629819in}}{\pgfqpoint{6.278106in}{1.622686in}}%
\pgfpathcurveto{\pgfqpoint{6.278106in}{1.615553in}}{\pgfqpoint{6.280940in}{1.608711in}}{\pgfqpoint{6.285984in}{1.603668in}}%
\pgfpathcurveto{\pgfqpoint{6.291028in}{1.598624in}}{\pgfqpoint{6.297869in}{1.595790in}}{\pgfqpoint{6.305002in}{1.595790in}}%
\pgfpathclose%
\pgfusepath{stroke,fill}%
\end{pgfscope}%
\begin{pgfscope}%
\pgfpathrectangle{\pgfqpoint{4.985294in}{0.500000in}}{\pgfqpoint{1.764706in}{1.700000in}}%
\pgfusepath{clip}%
\pgfsetbuttcap%
\pgfsetroundjoin%
\definecolor{currentfill}{rgb}{0.977657,0.891500,0.822809}%
\pgfsetfillcolor{currentfill}%
\pgfsetlinewidth{0.311001pt}%
\definecolor{currentstroke}{rgb}{1.000000,1.000000,1.000000}%
\pgfsetstrokecolor{currentstroke}%
\pgfsetdash{}{0pt}%
\pgfpathmoveto{\pgfqpoint{6.347250in}{1.248659in}}%
\pgfpathcurveto{\pgfqpoint{6.354382in}{1.248659in}}{\pgfqpoint{6.361224in}{1.251493in}}{\pgfqpoint{6.366268in}{1.256536in}}%
\pgfpathcurveto{\pgfqpoint{6.371311in}{1.261580in}}{\pgfqpoint{6.374145in}{1.268422in}}{\pgfqpoint{6.374145in}{1.275555in}}%
\pgfpathcurveto{\pgfqpoint{6.374145in}{1.282687in}}{\pgfqpoint{6.371311in}{1.289529in}}{\pgfqpoint{6.366268in}{1.294573in}}%
\pgfpathcurveto{\pgfqpoint{6.361224in}{1.299616in}}{\pgfqpoint{6.354382in}{1.302450in}}{\pgfqpoint{6.347250in}{1.302450in}}%
\pgfpathcurveto{\pgfqpoint{6.340117in}{1.302450in}}{\pgfqpoint{6.333275in}{1.299616in}}{\pgfqpoint{6.328232in}{1.294573in}}%
\pgfpathcurveto{\pgfqpoint{6.323188in}{1.289529in}}{\pgfqpoint{6.320354in}{1.282687in}}{\pgfqpoint{6.320354in}{1.275555in}}%
\pgfpathcurveto{\pgfqpoint{6.320354in}{1.268422in}}{\pgfqpoint{6.323188in}{1.261580in}}{\pgfqpoint{6.328232in}{1.256536in}}%
\pgfpathcurveto{\pgfqpoint{6.333275in}{1.251493in}}{\pgfqpoint{6.340117in}{1.248659in}}{\pgfqpoint{6.347250in}{1.248659in}}%
\pgfpathclose%
\pgfusepath{stroke,fill}%
\end{pgfscope}%
\begin{pgfscope}%
\pgfpathrectangle{\pgfqpoint{4.985294in}{0.500000in}}{\pgfqpoint{1.764706in}{1.700000in}}%
\pgfusepath{clip}%
\pgfsetbuttcap%
\pgfsetroundjoin%
\definecolor{currentfill}{rgb}{0.974412,0.862387,0.780156}%
\pgfsetfillcolor{currentfill}%
\pgfsetlinewidth{0.311001pt}%
\definecolor{currentstroke}{rgb}{1.000000,1.000000,1.000000}%
\pgfsetstrokecolor{currentstroke}%
\pgfsetdash{}{0pt}%
\pgfpathmoveto{\pgfqpoint{6.226406in}{1.606740in}}%
\pgfpathcurveto{\pgfqpoint{6.233539in}{1.606740in}}{\pgfqpoint{6.240381in}{1.609574in}}{\pgfqpoint{6.245424in}{1.614618in}}%
\pgfpathcurveto{\pgfqpoint{6.250468in}{1.619661in}}{\pgfqpoint{6.253302in}{1.626503in}}{\pgfqpoint{6.253302in}{1.633636in}}%
\pgfpathcurveto{\pgfqpoint{6.253302in}{1.640769in}}{\pgfqpoint{6.250468in}{1.647610in}}{\pgfqpoint{6.245424in}{1.652654in}}%
\pgfpathcurveto{\pgfqpoint{6.240381in}{1.657698in}}{\pgfqpoint{6.233539in}{1.660532in}}{\pgfqpoint{6.226406in}{1.660532in}}%
\pgfpathcurveto{\pgfqpoint{6.219273in}{1.660532in}}{\pgfqpoint{6.212432in}{1.657698in}}{\pgfqpoint{6.207388in}{1.652654in}}%
\pgfpathcurveto{\pgfqpoint{6.202344in}{1.647610in}}{\pgfqpoint{6.199510in}{1.640769in}}{\pgfqpoint{6.199510in}{1.633636in}}%
\pgfpathcurveto{\pgfqpoint{6.199510in}{1.626503in}}{\pgfqpoint{6.202344in}{1.619661in}}{\pgfqpoint{6.207388in}{1.614618in}}%
\pgfpathcurveto{\pgfqpoint{6.212432in}{1.609574in}}{\pgfqpoint{6.219273in}{1.606740in}}{\pgfqpoint{6.226406in}{1.606740in}}%
\pgfpathclose%
\pgfusepath{stroke,fill}%
\end{pgfscope}%
\begin{pgfscope}%
\pgfpathrectangle{\pgfqpoint{4.985294in}{0.500000in}}{\pgfqpoint{1.764706in}{1.700000in}}%
\pgfusepath{clip}%
\pgfsetbuttcap%
\pgfsetroundjoin%
\definecolor{currentfill}{rgb}{0.980678,0.914765,0.856766}%
\pgfsetfillcolor{currentfill}%
\pgfsetlinewidth{0.311001pt}%
\definecolor{currentstroke}{rgb}{1.000000,1.000000,1.000000}%
\pgfsetstrokecolor{currentstroke}%
\pgfsetdash{}{0pt}%
\pgfpathmoveto{\pgfqpoint{5.416149in}{1.357176in}}%
\pgfpathcurveto{\pgfqpoint{5.423282in}{1.357176in}}{\pgfqpoint{5.430123in}{1.360010in}}{\pgfqpoint{5.435167in}{1.365053in}}%
\pgfpathcurveto{\pgfqpoint{5.440211in}{1.370097in}}{\pgfqpoint{5.443044in}{1.376939in}}{\pgfqpoint{5.443044in}{1.384071in}}%
\pgfpathcurveto{\pgfqpoint{5.443044in}{1.391204in}}{\pgfqpoint{5.440211in}{1.398046in}}{\pgfqpoint{5.435167in}{1.403090in}}%
\pgfpathcurveto{\pgfqpoint{5.430123in}{1.408133in}}{\pgfqpoint{5.423282in}{1.410967in}}{\pgfqpoint{5.416149in}{1.410967in}}%
\pgfpathcurveto{\pgfqpoint{5.409016in}{1.410967in}}{\pgfqpoint{5.402174in}{1.408133in}}{\pgfqpoint{5.397131in}{1.403090in}}%
\pgfpathcurveto{\pgfqpoint{5.392087in}{1.398046in}}{\pgfqpoint{5.389253in}{1.391204in}}{\pgfqpoint{5.389253in}{1.384071in}}%
\pgfpathcurveto{\pgfqpoint{5.389253in}{1.376939in}}{\pgfqpoint{5.392087in}{1.370097in}}{\pgfqpoint{5.397131in}{1.365053in}}%
\pgfpathcurveto{\pgfqpoint{5.402174in}{1.360010in}}{\pgfqpoint{5.409016in}{1.357176in}}{\pgfqpoint{5.416149in}{1.357176in}}%
\pgfpathclose%
\pgfusepath{stroke,fill}%
\end{pgfscope}%
\begin{pgfscope}%
\pgfpathrectangle{\pgfqpoint{4.985294in}{0.500000in}}{\pgfqpoint{1.764706in}{1.700000in}}%
\pgfusepath{clip}%
\pgfsetbuttcap%
\pgfsetroundjoin%
\definecolor{currentfill}{rgb}{0.966560,0.756582,0.625273}%
\pgfsetfillcolor{currentfill}%
\pgfsetlinewidth{0.311001pt}%
\definecolor{currentstroke}{rgb}{1.000000,1.000000,1.000000}%
\pgfsetstrokecolor{currentstroke}%
\pgfsetdash{}{0pt}%
\pgfpathmoveto{\pgfqpoint{5.524458in}{1.165920in}}%
\pgfpathcurveto{\pgfqpoint{5.531591in}{1.165920in}}{\pgfqpoint{5.538433in}{1.168754in}}{\pgfqpoint{5.543477in}{1.173797in}}%
\pgfpathcurveto{\pgfqpoint{5.548520in}{1.178841in}}{\pgfqpoint{5.551354in}{1.185683in}}{\pgfqpoint{5.551354in}{1.192815in}}%
\pgfpathcurveto{\pgfqpoint{5.551354in}{1.199948in}}{\pgfqpoint{5.548520in}{1.206790in}}{\pgfqpoint{5.543477in}{1.211834in}}%
\pgfpathcurveto{\pgfqpoint{5.538433in}{1.216877in}}{\pgfqpoint{5.531591in}{1.219711in}}{\pgfqpoint{5.524458in}{1.219711in}}%
\pgfpathcurveto{\pgfqpoint{5.517326in}{1.219711in}}{\pgfqpoint{5.510484in}{1.216877in}}{\pgfqpoint{5.505440in}{1.211834in}}%
\pgfpathcurveto{\pgfqpoint{5.500397in}{1.206790in}}{\pgfqpoint{5.497563in}{1.199948in}}{\pgfqpoint{5.497563in}{1.192815in}}%
\pgfpathcurveto{\pgfqpoint{5.497563in}{1.185683in}}{\pgfqpoint{5.500397in}{1.178841in}}{\pgfqpoint{5.505440in}{1.173797in}}%
\pgfpathcurveto{\pgfqpoint{5.510484in}{1.168754in}}{\pgfqpoint{5.517326in}{1.165920in}}{\pgfqpoint{5.524458in}{1.165920in}}%
\pgfpathclose%
\pgfusepath{stroke,fill}%
\end{pgfscope}%
\begin{pgfscope}%
\pgfpathrectangle{\pgfqpoint{4.985294in}{0.500000in}}{\pgfqpoint{1.764706in}{1.700000in}}%
\pgfusepath{clip}%
\pgfsetbuttcap%
\pgfsetroundjoin%
\definecolor{currentfill}{rgb}{0.796501,0.105066,0.310630}%
\pgfsetfillcolor{currentfill}%
\pgfsetlinewidth{0.311001pt}%
\definecolor{currentstroke}{rgb}{1.000000,1.000000,1.000000}%
\pgfsetstrokecolor{currentstroke}%
\pgfsetdash{}{0pt}%
\pgfpathmoveto{\pgfqpoint{5.990270in}{0.992481in}}%
\pgfpathcurveto{\pgfqpoint{5.997403in}{0.992481in}}{\pgfqpoint{6.004245in}{0.995315in}}{\pgfqpoint{6.009288in}{1.000358in}}%
\pgfpathcurveto{\pgfqpoint{6.014332in}{1.005402in}}{\pgfqpoint{6.017166in}{1.012244in}}{\pgfqpoint{6.017166in}{1.019376in}}%
\pgfpathcurveto{\pgfqpoint{6.017166in}{1.026509in}}{\pgfqpoint{6.014332in}{1.033351in}}{\pgfqpoint{6.009288in}{1.038394in}}%
\pgfpathcurveto{\pgfqpoint{6.004245in}{1.043438in}}{\pgfqpoint{5.997403in}{1.046272in}}{\pgfqpoint{5.990270in}{1.046272in}}%
\pgfpathcurveto{\pgfqpoint{5.983137in}{1.046272in}}{\pgfqpoint{5.976296in}{1.043438in}}{\pgfqpoint{5.971252in}{1.038394in}}%
\pgfpathcurveto{\pgfqpoint{5.966208in}{1.033351in}}{\pgfqpoint{5.963374in}{1.026509in}}{\pgfqpoint{5.963374in}{1.019376in}}%
\pgfpathcurveto{\pgfqpoint{5.963374in}{1.012244in}}{\pgfqpoint{5.966208in}{1.005402in}}{\pgfqpoint{5.971252in}{1.000358in}}%
\pgfpathcurveto{\pgfqpoint{5.976296in}{0.995315in}}{\pgfqpoint{5.983137in}{0.992481in}}{\pgfqpoint{5.990270in}{0.992481in}}%
\pgfpathclose%
\pgfusepath{stroke,fill}%
\end{pgfscope}%
\begin{pgfscope}%
\pgfpathrectangle{\pgfqpoint{4.985294in}{0.500000in}}{\pgfqpoint{1.764706in}{1.700000in}}%
\pgfusepath{clip}%
\pgfsetbuttcap%
\pgfsetroundjoin%
\definecolor{currentfill}{rgb}{0.942910,0.375495,0.263698}%
\pgfsetfillcolor{currentfill}%
\pgfsetlinewidth{0.311001pt}%
\definecolor{currentstroke}{rgb}{1.000000,1.000000,1.000000}%
\pgfsetstrokecolor{currentstroke}%
\pgfsetdash{}{0pt}%
\pgfpathmoveto{\pgfqpoint{6.072928in}{1.793765in}}%
\pgfpathcurveto{\pgfqpoint{6.080061in}{1.793765in}}{\pgfqpoint{6.086903in}{1.796599in}}{\pgfqpoint{6.091947in}{1.801643in}}%
\pgfpathcurveto{\pgfqpoint{6.096990in}{1.806686in}}{\pgfqpoint{6.099824in}{1.813528in}}{\pgfqpoint{6.099824in}{1.820661in}}%
\pgfpathcurveto{\pgfqpoint{6.099824in}{1.827794in}}{\pgfqpoint{6.096990in}{1.834635in}}{\pgfqpoint{6.091947in}{1.839679in}}%
\pgfpathcurveto{\pgfqpoint{6.086903in}{1.844723in}}{\pgfqpoint{6.080061in}{1.847557in}}{\pgfqpoint{6.072928in}{1.847557in}}%
\pgfpathcurveto{\pgfqpoint{6.065796in}{1.847557in}}{\pgfqpoint{6.058954in}{1.844723in}}{\pgfqpoint{6.053910in}{1.839679in}}%
\pgfpathcurveto{\pgfqpoint{6.048867in}{1.834635in}}{\pgfqpoint{6.046033in}{1.827794in}}{\pgfqpoint{6.046033in}{1.820661in}}%
\pgfpathcurveto{\pgfqpoint{6.046033in}{1.813528in}}{\pgfqpoint{6.048867in}{1.806686in}}{\pgfqpoint{6.053910in}{1.801643in}}%
\pgfpathcurveto{\pgfqpoint{6.058954in}{1.796599in}}{\pgfqpoint{6.065796in}{1.793765in}}{\pgfqpoint{6.072928in}{1.793765in}}%
\pgfpathclose%
\pgfusepath{stroke,fill}%
\end{pgfscope}%
\begin{pgfscope}%
\pgfpathrectangle{\pgfqpoint{4.985294in}{0.500000in}}{\pgfqpoint{1.764706in}{1.700000in}}%
\pgfusepath{clip}%
\pgfsetbuttcap%
\pgfsetroundjoin%
\definecolor{currentfill}{rgb}{0.978376,0.897317,0.831308}%
\pgfsetfillcolor{currentfill}%
\pgfsetlinewidth{0.311001pt}%
\definecolor{currentstroke}{rgb}{1.000000,1.000000,1.000000}%
\pgfsetstrokecolor{currentstroke}%
\pgfsetdash{}{0pt}%
\pgfpathmoveto{\pgfqpoint{6.280787in}{1.269627in}}%
\pgfpathcurveto{\pgfqpoint{6.287920in}{1.269627in}}{\pgfqpoint{6.294762in}{1.272460in}}{\pgfqpoint{6.299806in}{1.277504in}}%
\pgfpathcurveto{\pgfqpoint{6.304849in}{1.282548in}}{\pgfqpoint{6.307683in}{1.289389in}}{\pgfqpoint{6.307683in}{1.296522in}}%
\pgfpathcurveto{\pgfqpoint{6.307683in}{1.303655in}}{\pgfqpoint{6.304849in}{1.310497in}}{\pgfqpoint{6.299806in}{1.315540in}}%
\pgfpathcurveto{\pgfqpoint{6.294762in}{1.320584in}}{\pgfqpoint{6.287920in}{1.323418in}}{\pgfqpoint{6.280787in}{1.323418in}}%
\pgfpathcurveto{\pgfqpoint{6.273655in}{1.323418in}}{\pgfqpoint{6.266813in}{1.320584in}}{\pgfqpoint{6.261769in}{1.315540in}}%
\pgfpathcurveto{\pgfqpoint{6.256726in}{1.310497in}}{\pgfqpoint{6.253892in}{1.303655in}}{\pgfqpoint{6.253892in}{1.296522in}}%
\pgfpathcurveto{\pgfqpoint{6.253892in}{1.289389in}}{\pgfqpoint{6.256726in}{1.282548in}}{\pgfqpoint{6.261769in}{1.277504in}}%
\pgfpathcurveto{\pgfqpoint{6.266813in}{1.272460in}}{\pgfqpoint{6.273655in}{1.269627in}}{\pgfqpoint{6.280787in}{1.269627in}}%
\pgfpathclose%
\pgfusepath{stroke,fill}%
\end{pgfscope}%
\begin{pgfscope}%
\pgfpathrectangle{\pgfqpoint{4.985294in}{0.500000in}}{\pgfqpoint{1.764706in}{1.700000in}}%
\pgfusepath{clip}%
\pgfsetbuttcap%
\pgfsetroundjoin%
\definecolor{currentfill}{rgb}{0.973271,0.850724,0.762998}%
\pgfsetfillcolor{currentfill}%
\pgfsetlinewidth{0.311001pt}%
\definecolor{currentstroke}{rgb}{1.000000,1.000000,1.000000}%
\pgfsetstrokecolor{currentstroke}%
\pgfsetdash{}{0pt}%
\pgfpathmoveto{\pgfqpoint{5.510874in}{1.540816in}}%
\pgfpathcurveto{\pgfqpoint{5.518007in}{1.540816in}}{\pgfqpoint{5.524848in}{1.543650in}}{\pgfqpoint{5.529892in}{1.548694in}}%
\pgfpathcurveto{\pgfqpoint{5.534935in}{1.553738in}}{\pgfqpoint{5.537769in}{1.560579in}}{\pgfqpoint{5.537769in}{1.567712in}}%
\pgfpathcurveto{\pgfqpoint{5.537769in}{1.574845in}}{\pgfqpoint{5.534935in}{1.581687in}}{\pgfqpoint{5.529892in}{1.586730in}}%
\pgfpathcurveto{\pgfqpoint{5.524848in}{1.591774in}}{\pgfqpoint{5.518007in}{1.594608in}}{\pgfqpoint{5.510874in}{1.594608in}}%
\pgfpathcurveto{\pgfqpoint{5.503741in}{1.594608in}}{\pgfqpoint{5.496899in}{1.591774in}}{\pgfqpoint{5.491856in}{1.586730in}}%
\pgfpathcurveto{\pgfqpoint{5.486812in}{1.581687in}}{\pgfqpoint{5.483978in}{1.574845in}}{\pgfqpoint{5.483978in}{1.567712in}}%
\pgfpathcurveto{\pgfqpoint{5.483978in}{1.560579in}}{\pgfqpoint{5.486812in}{1.553738in}}{\pgfqpoint{5.491856in}{1.548694in}}%
\pgfpathcurveto{\pgfqpoint{5.496899in}{1.543650in}}{\pgfqpoint{5.503741in}{1.540816in}}{\pgfqpoint{5.510874in}{1.540816in}}%
\pgfpathclose%
\pgfusepath{stroke,fill}%
\end{pgfscope}%
\begin{pgfscope}%
\pgfpathrectangle{\pgfqpoint{4.985294in}{0.500000in}}{\pgfqpoint{1.764706in}{1.700000in}}%
\pgfusepath{clip}%
\pgfsetbuttcap%
\pgfsetroundjoin%
\definecolor{currentfill}{rgb}{0.970255,0.815666,0.711203}%
\pgfsetfillcolor{currentfill}%
\pgfsetlinewidth{0.311001pt}%
\definecolor{currentstroke}{rgb}{1.000000,1.000000,1.000000}%
\pgfsetstrokecolor{currentstroke}%
\pgfsetdash{}{0pt}%
\pgfpathmoveto{\pgfqpoint{5.521370in}{0.972507in}}%
\pgfpathcurveto{\pgfqpoint{5.528503in}{0.972507in}}{\pgfqpoint{5.535344in}{0.975341in}}{\pgfqpoint{5.540388in}{0.980385in}}%
\pgfpathcurveto{\pgfqpoint{5.545432in}{0.985429in}}{\pgfqpoint{5.548266in}{0.992270in}}{\pgfqpoint{5.548266in}{0.999403in}}%
\pgfpathcurveto{\pgfqpoint{5.548266in}{1.006536in}}{\pgfqpoint{5.545432in}{1.013378in}}{\pgfqpoint{5.540388in}{1.018421in}}%
\pgfpathcurveto{\pgfqpoint{5.535344in}{1.023465in}}{\pgfqpoint{5.528503in}{1.026299in}}{\pgfqpoint{5.521370in}{1.026299in}}%
\pgfpathcurveto{\pgfqpoint{5.514237in}{1.026299in}}{\pgfqpoint{5.507395in}{1.023465in}}{\pgfqpoint{5.502352in}{1.018421in}}%
\pgfpathcurveto{\pgfqpoint{5.497308in}{1.013378in}}{\pgfqpoint{5.494474in}{1.006536in}}{\pgfqpoint{5.494474in}{0.999403in}}%
\pgfpathcurveto{\pgfqpoint{5.494474in}{0.992270in}}{\pgfqpoint{5.497308in}{0.985429in}}{\pgfqpoint{5.502352in}{0.980385in}}%
\pgfpathcurveto{\pgfqpoint{5.507395in}{0.975341in}}{\pgfqpoint{5.514237in}{0.972507in}}{\pgfqpoint{5.521370in}{0.972507in}}%
\pgfpathclose%
\pgfusepath{stroke,fill}%
\end{pgfscope}%
\begin{pgfscope}%
\pgfpathrectangle{\pgfqpoint{4.985294in}{0.500000in}}{\pgfqpoint{1.764706in}{1.700000in}}%
\pgfusepath{clip}%
\pgfsetbuttcap%
\pgfsetroundjoin%
\definecolor{currentfill}{rgb}{0.978376,0.897317,0.831308}%
\pgfsetfillcolor{currentfill}%
\pgfsetlinewidth{0.311001pt}%
\definecolor{currentstroke}{rgb}{1.000000,1.000000,1.000000}%
\pgfsetstrokecolor{currentstroke}%
\pgfsetdash{}{0pt}%
\pgfpathmoveto{\pgfqpoint{6.280828in}{1.266007in}}%
\pgfpathcurveto{\pgfqpoint{6.287961in}{1.266007in}}{\pgfqpoint{6.294802in}{1.268841in}}{\pgfqpoint{6.299846in}{1.273885in}}%
\pgfpathcurveto{\pgfqpoint{6.304890in}{1.278928in}}{\pgfqpoint{6.307724in}{1.285770in}}{\pgfqpoint{6.307724in}{1.292903in}}%
\pgfpathcurveto{\pgfqpoint{6.307724in}{1.300036in}}{\pgfqpoint{6.304890in}{1.306877in}}{\pgfqpoint{6.299846in}{1.311921in}}%
\pgfpathcurveto{\pgfqpoint{6.294802in}{1.316965in}}{\pgfqpoint{6.287961in}{1.319799in}}{\pgfqpoint{6.280828in}{1.319799in}}%
\pgfpathcurveto{\pgfqpoint{6.273695in}{1.319799in}}{\pgfqpoint{6.266853in}{1.316965in}}{\pgfqpoint{6.261810in}{1.311921in}}%
\pgfpathcurveto{\pgfqpoint{6.256766in}{1.306877in}}{\pgfqpoint{6.253932in}{1.300036in}}{\pgfqpoint{6.253932in}{1.292903in}}%
\pgfpathcurveto{\pgfqpoint{6.253932in}{1.285770in}}{\pgfqpoint{6.256766in}{1.278928in}}{\pgfqpoint{6.261810in}{1.273885in}}%
\pgfpathcurveto{\pgfqpoint{6.266853in}{1.268841in}}{\pgfqpoint{6.273695in}{1.266007in}}{\pgfqpoint{6.280828in}{1.266007in}}%
\pgfpathclose%
\pgfusepath{stroke,fill}%
\end{pgfscope}%
\begin{pgfscope}%
\pgfpathrectangle{\pgfqpoint{4.985294in}{0.500000in}}{\pgfqpoint{1.764706in}{1.700000in}}%
\pgfusepath{clip}%
\pgfsetbuttcap%
\pgfsetroundjoin%
\definecolor{currentfill}{rgb}{0.966328,0.750560,0.616961}%
\pgfsetfillcolor{currentfill}%
\pgfsetlinewidth{0.311001pt}%
\definecolor{currentstroke}{rgb}{1.000000,1.000000,1.000000}%
\pgfsetstrokecolor{currentstroke}%
\pgfsetdash{}{0pt}%
\pgfpathmoveto{\pgfqpoint{6.192593in}{1.486773in}}%
\pgfpathcurveto{\pgfqpoint{6.199726in}{1.486773in}}{\pgfqpoint{6.206567in}{1.489607in}}{\pgfqpoint{6.211611in}{1.494651in}}%
\pgfpathcurveto{\pgfqpoint{6.216655in}{1.499695in}}{\pgfqpoint{6.219488in}{1.506536in}}{\pgfqpoint{6.219488in}{1.513669in}}%
\pgfpathcurveto{\pgfqpoint{6.219488in}{1.520802in}}{\pgfqpoint{6.216655in}{1.527644in}}{\pgfqpoint{6.211611in}{1.532687in}}%
\pgfpathcurveto{\pgfqpoint{6.206567in}{1.537731in}}{\pgfqpoint{6.199726in}{1.540565in}}{\pgfqpoint{6.192593in}{1.540565in}}%
\pgfpathcurveto{\pgfqpoint{6.185460in}{1.540565in}}{\pgfqpoint{6.178618in}{1.537731in}}{\pgfqpoint{6.173575in}{1.532687in}}%
\pgfpathcurveto{\pgfqpoint{6.168531in}{1.527644in}}{\pgfqpoint{6.165697in}{1.520802in}}{\pgfqpoint{6.165697in}{1.513669in}}%
\pgfpathcurveto{\pgfqpoint{6.165697in}{1.506536in}}{\pgfqpoint{6.168531in}{1.499695in}}{\pgfqpoint{6.173575in}{1.494651in}}%
\pgfpathcurveto{\pgfqpoint{6.178618in}{1.489607in}}{\pgfqpoint{6.185460in}{1.486773in}}{\pgfqpoint{6.192593in}{1.486773in}}%
\pgfpathclose%
\pgfusepath{stroke,fill}%
\end{pgfscope}%
\begin{pgfscope}%
\pgfpathrectangle{\pgfqpoint{4.985294in}{0.500000in}}{\pgfqpoint{1.764706in}{1.700000in}}%
\pgfusepath{clip}%
\pgfsetbuttcap%
\pgfsetroundjoin%
\definecolor{currentfill}{rgb}{0.965753,0.732351,0.592427}%
\pgfsetfillcolor{currentfill}%
\pgfsetlinewidth{0.311001pt}%
\definecolor{currentstroke}{rgb}{1.000000,1.000000,1.000000}%
\pgfsetstrokecolor{currentstroke}%
\pgfsetdash{}{0pt}%
\pgfpathmoveto{\pgfqpoint{6.404960in}{1.328449in}}%
\pgfpathcurveto{\pgfqpoint{6.412093in}{1.328449in}}{\pgfqpoint{6.418934in}{1.331282in}}{\pgfqpoint{6.423978in}{1.336326in}}%
\pgfpathcurveto{\pgfqpoint{6.429022in}{1.341370in}}{\pgfqpoint{6.431855in}{1.348211in}}{\pgfqpoint{6.431855in}{1.355344in}}%
\pgfpathcurveto{\pgfqpoint{6.431855in}{1.362477in}}{\pgfqpoint{6.429022in}{1.369319in}}{\pgfqpoint{6.423978in}{1.374362in}}%
\pgfpathcurveto{\pgfqpoint{6.418934in}{1.379406in}}{\pgfqpoint{6.412093in}{1.382240in}}{\pgfqpoint{6.404960in}{1.382240in}}%
\pgfpathcurveto{\pgfqpoint{6.397827in}{1.382240in}}{\pgfqpoint{6.390985in}{1.379406in}}{\pgfqpoint{6.385942in}{1.374362in}}%
\pgfpathcurveto{\pgfqpoint{6.380898in}{1.369319in}}{\pgfqpoint{6.378064in}{1.362477in}}{\pgfqpoint{6.378064in}{1.355344in}}%
\pgfpathcurveto{\pgfqpoint{6.378064in}{1.348211in}}{\pgfqpoint{6.380898in}{1.341370in}}{\pgfqpoint{6.385942in}{1.336326in}}%
\pgfpathcurveto{\pgfqpoint{6.390985in}{1.331282in}}{\pgfqpoint{6.397827in}{1.328449in}}{\pgfqpoint{6.404960in}{1.328449in}}%
\pgfpathclose%
\pgfusepath{stroke,fill}%
\end{pgfscope}%
\begin{pgfscope}%
\pgfpathrectangle{\pgfqpoint{4.985294in}{0.500000in}}{\pgfqpoint{1.764706in}{1.700000in}}%
\pgfusepath{clip}%
\pgfsetbuttcap%
\pgfsetroundjoin%
\definecolor{currentfill}{rgb}{0.962283,0.593046,0.431453}%
\pgfsetfillcolor{currentfill}%
\pgfsetlinewidth{0.311001pt}%
\definecolor{currentstroke}{rgb}{1.000000,1.000000,1.000000}%
\pgfsetstrokecolor{currentstroke}%
\pgfsetdash{}{0pt}%
\pgfpathmoveto{\pgfqpoint{6.278180in}{1.714686in}}%
\pgfpathcurveto{\pgfqpoint{6.285313in}{1.714686in}}{\pgfqpoint{6.292154in}{1.717520in}}{\pgfqpoint{6.297198in}{1.722564in}}%
\pgfpathcurveto{\pgfqpoint{6.302242in}{1.727607in}}{\pgfqpoint{6.305076in}{1.734449in}}{\pgfqpoint{6.305076in}{1.741582in}}%
\pgfpathcurveto{\pgfqpoint{6.305076in}{1.748715in}}{\pgfqpoint{6.302242in}{1.755556in}}{\pgfqpoint{6.297198in}{1.760600in}}%
\pgfpathcurveto{\pgfqpoint{6.292154in}{1.765644in}}{\pgfqpoint{6.285313in}{1.768478in}}{\pgfqpoint{6.278180in}{1.768478in}}%
\pgfpathcurveto{\pgfqpoint{6.271047in}{1.768478in}}{\pgfqpoint{6.264205in}{1.765644in}}{\pgfqpoint{6.259162in}{1.760600in}}%
\pgfpathcurveto{\pgfqpoint{6.254118in}{1.755556in}}{\pgfqpoint{6.251284in}{1.748715in}}{\pgfqpoint{6.251284in}{1.741582in}}%
\pgfpathcurveto{\pgfqpoint{6.251284in}{1.734449in}}{\pgfqpoint{6.254118in}{1.727607in}}{\pgfqpoint{6.259162in}{1.722564in}}%
\pgfpathcurveto{\pgfqpoint{6.264205in}{1.717520in}}{\pgfqpoint{6.271047in}{1.714686in}}{\pgfqpoint{6.278180in}{1.714686in}}%
\pgfpathclose%
\pgfusepath{stroke,fill}%
\end{pgfscope}%
\begin{pgfscope}%
\pgfpathrectangle{\pgfqpoint{4.985294in}{0.500000in}}{\pgfqpoint{1.764706in}{1.700000in}}%
\pgfusepath{clip}%
\pgfsetbuttcap%
\pgfsetroundjoin%
\definecolor{currentfill}{rgb}{0.965753,0.732351,0.592427}%
\pgfsetfillcolor{currentfill}%
\pgfsetlinewidth{0.311001pt}%
\definecolor{currentstroke}{rgb}{1.000000,1.000000,1.000000}%
\pgfsetstrokecolor{currentstroke}%
\pgfsetdash{}{0pt}%
\pgfpathmoveto{\pgfqpoint{5.450634in}{1.672618in}}%
\pgfpathcurveto{\pgfqpoint{5.457767in}{1.672618in}}{\pgfqpoint{5.464609in}{1.675452in}}{\pgfqpoint{5.469652in}{1.680495in}}%
\pgfpathcurveto{\pgfqpoint{5.474696in}{1.685539in}}{\pgfqpoint{5.477530in}{1.692381in}}{\pgfqpoint{5.477530in}{1.699513in}}%
\pgfpathcurveto{\pgfqpoint{5.477530in}{1.706646in}}{\pgfqpoint{5.474696in}{1.713488in}}{\pgfqpoint{5.469652in}{1.718532in}}%
\pgfpathcurveto{\pgfqpoint{5.464609in}{1.723575in}}{\pgfqpoint{5.457767in}{1.726409in}}{\pgfqpoint{5.450634in}{1.726409in}}%
\pgfpathcurveto{\pgfqpoint{5.443501in}{1.726409in}}{\pgfqpoint{5.436660in}{1.723575in}}{\pgfqpoint{5.431616in}{1.718532in}}%
\pgfpathcurveto{\pgfqpoint{5.426572in}{1.713488in}}{\pgfqpoint{5.423739in}{1.706646in}}{\pgfqpoint{5.423739in}{1.699513in}}%
\pgfpathcurveto{\pgfqpoint{5.423739in}{1.692381in}}{\pgfqpoint{5.426572in}{1.685539in}}{\pgfqpoint{5.431616in}{1.680495in}}%
\pgfpathcurveto{\pgfqpoint{5.436660in}{1.675452in}}{\pgfqpoint{5.443501in}{1.672618in}}{\pgfqpoint{5.450634in}{1.672618in}}%
\pgfpathclose%
\pgfusepath{stroke,fill}%
\end{pgfscope}%
\begin{pgfscope}%
\pgfpathrectangle{\pgfqpoint{4.985294in}{0.500000in}}{\pgfqpoint{1.764706in}{1.700000in}}%
\pgfusepath{clip}%
\pgfsetbuttcap%
\pgfsetroundjoin%
\definecolor{currentfill}{rgb}{0.818205,0.120806,0.299261}%
\pgfsetfillcolor{currentfill}%
\pgfsetlinewidth{0.311001pt}%
\definecolor{currentstroke}{rgb}{1.000000,1.000000,1.000000}%
\pgfsetstrokecolor{currentstroke}%
\pgfsetdash{}{0pt}%
\pgfpathmoveto{\pgfqpoint{6.309889in}{0.848409in}}%
\pgfpathcurveto{\pgfqpoint{6.317022in}{0.848409in}}{\pgfqpoint{6.323863in}{0.851243in}}{\pgfqpoint{6.328907in}{0.856287in}}%
\pgfpathcurveto{\pgfqpoint{6.333951in}{0.861330in}}{\pgfqpoint{6.336785in}{0.868172in}}{\pgfqpoint{6.336785in}{0.875305in}}%
\pgfpathcurveto{\pgfqpoint{6.336785in}{0.882438in}}{\pgfqpoint{6.333951in}{0.889279in}}{\pgfqpoint{6.328907in}{0.894323in}}%
\pgfpathcurveto{\pgfqpoint{6.323863in}{0.899367in}}{\pgfqpoint{6.317022in}{0.902200in}}{\pgfqpoint{6.309889in}{0.902200in}}%
\pgfpathcurveto{\pgfqpoint{6.302756in}{0.902200in}}{\pgfqpoint{6.295914in}{0.899367in}}{\pgfqpoint{6.290871in}{0.894323in}}%
\pgfpathcurveto{\pgfqpoint{6.285827in}{0.889279in}}{\pgfqpoint{6.282993in}{0.882438in}}{\pgfqpoint{6.282993in}{0.875305in}}%
\pgfpathcurveto{\pgfqpoint{6.282993in}{0.868172in}}{\pgfqpoint{6.285827in}{0.861330in}}{\pgfqpoint{6.290871in}{0.856287in}}%
\pgfpathcurveto{\pgfqpoint{6.295914in}{0.851243in}}{\pgfqpoint{6.302756in}{0.848409in}}{\pgfqpoint{6.309889in}{0.848409in}}%
\pgfpathclose%
\pgfusepath{stroke,fill}%
\end{pgfscope}%
\begin{pgfscope}%
\pgfpathrectangle{\pgfqpoint{4.985294in}{0.500000in}}{\pgfqpoint{1.764706in}{1.700000in}}%
\pgfusepath{clip}%
\pgfsetbuttcap%
\pgfsetroundjoin%
\definecolor{currentfill}{rgb}{0.953816,0.463738,0.317699}%
\pgfsetfillcolor{currentfill}%
\pgfsetlinewidth{0.311001pt}%
\definecolor{currentstroke}{rgb}{1.000000,1.000000,1.000000}%
\pgfsetstrokecolor{currentstroke}%
\pgfsetdash{}{0pt}%
\pgfpathmoveto{\pgfqpoint{5.565771in}{1.801365in}}%
\pgfpathcurveto{\pgfqpoint{5.572904in}{1.801365in}}{\pgfqpoint{5.579745in}{1.804199in}}{\pgfqpoint{5.584789in}{1.809242in}}%
\pgfpathcurveto{\pgfqpoint{5.589833in}{1.814286in}}{\pgfqpoint{5.592666in}{1.821128in}}{\pgfqpoint{5.592666in}{1.828260in}}%
\pgfpathcurveto{\pgfqpoint{5.592666in}{1.835393in}}{\pgfqpoint{5.589833in}{1.842235in}}{\pgfqpoint{5.584789in}{1.847279in}}%
\pgfpathcurveto{\pgfqpoint{5.579745in}{1.852322in}}{\pgfqpoint{5.572904in}{1.855156in}}{\pgfqpoint{5.565771in}{1.855156in}}%
\pgfpathcurveto{\pgfqpoint{5.558638in}{1.855156in}}{\pgfqpoint{5.551796in}{1.852322in}}{\pgfqpoint{5.546753in}{1.847279in}}%
\pgfpathcurveto{\pgfqpoint{5.541709in}{1.842235in}}{\pgfqpoint{5.538875in}{1.835393in}}{\pgfqpoint{5.538875in}{1.828260in}}%
\pgfpathcurveto{\pgfqpoint{5.538875in}{1.821128in}}{\pgfqpoint{5.541709in}{1.814286in}}{\pgfqpoint{5.546753in}{1.809242in}}%
\pgfpathcurveto{\pgfqpoint{5.551796in}{1.804199in}}{\pgfqpoint{5.558638in}{1.801365in}}{\pgfqpoint{5.565771in}{1.801365in}}%
\pgfpathclose%
\pgfusepath{stroke,fill}%
\end{pgfscope}%
\begin{pgfscope}%
\pgfpathrectangle{\pgfqpoint{4.985294in}{0.500000in}}{\pgfqpoint{1.764706in}{1.700000in}}%
\pgfusepath{clip}%
\pgfsetbuttcap%
\pgfsetroundjoin%
\definecolor{currentfill}{rgb}{0.972726,0.844889,0.754401}%
\pgfsetfillcolor{currentfill}%
\pgfsetlinewidth{0.311001pt}%
\definecolor{currentstroke}{rgb}{1.000000,1.000000,1.000000}%
\pgfsetstrokecolor{currentstroke}%
\pgfsetdash{}{0pt}%
\pgfpathmoveto{\pgfqpoint{6.349864in}{1.164701in}}%
\pgfpathcurveto{\pgfqpoint{6.356997in}{1.164701in}}{\pgfqpoint{6.363839in}{1.167535in}}{\pgfqpoint{6.368883in}{1.172579in}}%
\pgfpathcurveto{\pgfqpoint{6.373926in}{1.177623in}}{\pgfqpoint{6.376760in}{1.184464in}}{\pgfqpoint{6.376760in}{1.191597in}}%
\pgfpathcurveto{\pgfqpoint{6.376760in}{1.198730in}}{\pgfqpoint{6.373926in}{1.205571in}}{\pgfqpoint{6.368883in}{1.210615in}}%
\pgfpathcurveto{\pgfqpoint{6.363839in}{1.215659in}}{\pgfqpoint{6.356997in}{1.218493in}}{\pgfqpoint{6.349864in}{1.218493in}}%
\pgfpathcurveto{\pgfqpoint{6.342732in}{1.218493in}}{\pgfqpoint{6.335890in}{1.215659in}}{\pgfqpoint{6.330846in}{1.210615in}}%
\pgfpathcurveto{\pgfqpoint{6.325803in}{1.205571in}}{\pgfqpoint{6.322969in}{1.198730in}}{\pgfqpoint{6.322969in}{1.191597in}}%
\pgfpathcurveto{\pgfqpoint{6.322969in}{1.184464in}}{\pgfqpoint{6.325803in}{1.177623in}}{\pgfqpoint{6.330846in}{1.172579in}}%
\pgfpathcurveto{\pgfqpoint{6.335890in}{1.167535in}}{\pgfqpoint{6.342732in}{1.164701in}}{\pgfqpoint{6.349864in}{1.164701in}}%
\pgfpathclose%
\pgfusepath{stroke,fill}%
\end{pgfscope}%
\begin{pgfscope}%
\pgfpathrectangle{\pgfqpoint{4.985294in}{0.500000in}}{\pgfqpoint{1.764706in}{1.700000in}}%
\pgfusepath{clip}%
\pgfsetbuttcap%
\pgfsetroundjoin%
\definecolor{currentfill}{rgb}{0.965302,0.713942,0.568499}%
\pgfsetfillcolor{currentfill}%
\pgfsetlinewidth{0.311001pt}%
\definecolor{currentstroke}{rgb}{1.000000,1.000000,1.000000}%
\pgfsetstrokecolor{currentstroke}%
\pgfsetdash{}{0pt}%
\pgfpathmoveto{\pgfqpoint{6.402694in}{1.216596in}}%
\pgfpathcurveto{\pgfqpoint{6.409826in}{1.216596in}}{\pgfqpoint{6.416668in}{1.219430in}}{\pgfqpoint{6.421712in}{1.224474in}}%
\pgfpathcurveto{\pgfqpoint{6.426755in}{1.229517in}}{\pgfqpoint{6.429589in}{1.236359in}}{\pgfqpoint{6.429589in}{1.243492in}}%
\pgfpathcurveto{\pgfqpoint{6.429589in}{1.250625in}}{\pgfqpoint{6.426755in}{1.257466in}}{\pgfqpoint{6.421712in}{1.262510in}}%
\pgfpathcurveto{\pgfqpoint{6.416668in}{1.267554in}}{\pgfqpoint{6.409826in}{1.270388in}}{\pgfqpoint{6.402694in}{1.270388in}}%
\pgfpathcurveto{\pgfqpoint{6.395561in}{1.270388in}}{\pgfqpoint{6.388719in}{1.267554in}}{\pgfqpoint{6.383675in}{1.262510in}}%
\pgfpathcurveto{\pgfqpoint{6.378632in}{1.257466in}}{\pgfqpoint{6.375798in}{1.250625in}}{\pgfqpoint{6.375798in}{1.243492in}}%
\pgfpathcurveto{\pgfqpoint{6.375798in}{1.236359in}}{\pgfqpoint{6.378632in}{1.229517in}}{\pgfqpoint{6.383675in}{1.224474in}}%
\pgfpathcurveto{\pgfqpoint{6.388719in}{1.219430in}}{\pgfqpoint{6.395561in}{1.216596in}}{\pgfqpoint{6.402694in}{1.216596in}}%
\pgfpathclose%
\pgfusepath{stroke,fill}%
\end{pgfscope}%
\begin{pgfscope}%
\pgfpathrectangle{\pgfqpoint{4.985294in}{0.500000in}}{\pgfqpoint{1.764706in}{1.700000in}}%
\pgfusepath{clip}%
\pgfsetbuttcap%
\pgfsetroundjoin%
\definecolor{currentfill}{rgb}{0.965928,0.738443,0.600540}%
\pgfsetfillcolor{currentfill}%
\pgfsetlinewidth{0.311001pt}%
\definecolor{currentstroke}{rgb}{1.000000,1.000000,1.000000}%
\pgfsetstrokecolor{currentstroke}%
\pgfsetdash{}{0pt}%
\pgfpathmoveto{\pgfqpoint{5.572881in}{1.052576in}}%
\pgfpathcurveto{\pgfqpoint{5.580014in}{1.052576in}}{\pgfqpoint{5.586856in}{1.055410in}}{\pgfqpoint{5.591900in}{1.060453in}}%
\pgfpathcurveto{\pgfqpoint{5.596943in}{1.065497in}}{\pgfqpoint{5.599777in}{1.072339in}}{\pgfqpoint{5.599777in}{1.079471in}}%
\pgfpathcurveto{\pgfqpoint{5.599777in}{1.086604in}}{\pgfqpoint{5.596943in}{1.093446in}}{\pgfqpoint{5.591900in}{1.098490in}}%
\pgfpathcurveto{\pgfqpoint{5.586856in}{1.103533in}}{\pgfqpoint{5.580014in}{1.106367in}}{\pgfqpoint{5.572881in}{1.106367in}}%
\pgfpathcurveto{\pgfqpoint{5.565749in}{1.106367in}}{\pgfqpoint{5.558907in}{1.103533in}}{\pgfqpoint{5.553863in}{1.098490in}}%
\pgfpathcurveto{\pgfqpoint{5.548820in}{1.093446in}}{\pgfqpoint{5.545986in}{1.086604in}}{\pgfqpoint{5.545986in}{1.079471in}}%
\pgfpathcurveto{\pgfqpoint{5.545986in}{1.072339in}}{\pgfqpoint{5.548820in}{1.065497in}}{\pgfqpoint{5.553863in}{1.060453in}}%
\pgfpathcurveto{\pgfqpoint{5.558907in}{1.055410in}}{\pgfqpoint{5.565749in}{1.052576in}}{\pgfqpoint{5.572881in}{1.052576in}}%
\pgfpathclose%
\pgfusepath{stroke,fill}%
\end{pgfscope}%
\begin{pgfscope}%
\pgfpathrectangle{\pgfqpoint{4.985294in}{0.500000in}}{\pgfqpoint{1.764706in}{1.700000in}}%
\pgfusepath{clip}%
\pgfsetbuttcap%
\pgfsetroundjoin%
\definecolor{currentfill}{rgb}{0.698038,0.088972,0.346299}%
\pgfsetfillcolor{currentfill}%
\pgfsetlinewidth{0.311001pt}%
\definecolor{currentstroke}{rgb}{1.000000,1.000000,1.000000}%
\pgfsetstrokecolor{currentstroke}%
\pgfsetdash{}{0pt}%
\pgfpathmoveto{\pgfqpoint{6.403966in}{1.766837in}}%
\pgfpathcurveto{\pgfqpoint{6.411099in}{1.766837in}}{\pgfqpoint{6.417941in}{1.769671in}}{\pgfqpoint{6.422984in}{1.774714in}}%
\pgfpathcurveto{\pgfqpoint{6.428028in}{1.779758in}}{\pgfqpoint{6.430862in}{1.786600in}}{\pgfqpoint{6.430862in}{1.793733in}}%
\pgfpathcurveto{\pgfqpoint{6.430862in}{1.800865in}}{\pgfqpoint{6.428028in}{1.807707in}}{\pgfqpoint{6.422984in}{1.812751in}}%
\pgfpathcurveto{\pgfqpoint{6.417941in}{1.817794in}}{\pgfqpoint{6.411099in}{1.820628in}}{\pgfqpoint{6.403966in}{1.820628in}}%
\pgfpathcurveto{\pgfqpoint{6.396833in}{1.820628in}}{\pgfqpoint{6.389992in}{1.817794in}}{\pgfqpoint{6.384948in}{1.812751in}}%
\pgfpathcurveto{\pgfqpoint{6.379904in}{1.807707in}}{\pgfqpoint{6.377070in}{1.800865in}}{\pgfqpoint{6.377070in}{1.793733in}}%
\pgfpathcurveto{\pgfqpoint{6.377070in}{1.786600in}}{\pgfqpoint{6.379904in}{1.779758in}}{\pgfqpoint{6.384948in}{1.774714in}}%
\pgfpathcurveto{\pgfqpoint{6.389992in}{1.769671in}}{\pgfqpoint{6.396833in}{1.766837in}}{\pgfqpoint{6.403966in}{1.766837in}}%
\pgfpathclose%
\pgfusepath{stroke,fill}%
\end{pgfscope}%
\begin{pgfscope}%
\pgfpathrectangle{\pgfqpoint{4.985294in}{0.500000in}}{\pgfqpoint{1.764706in}{1.700000in}}%
\pgfusepath{clip}%
\pgfsetbuttcap%
\pgfsetroundjoin%
\definecolor{currentfill}{rgb}{0.975018,0.868213,0.788710}%
\pgfsetfillcolor{currentfill}%
\pgfsetlinewidth{0.311001pt}%
\definecolor{currentstroke}{rgb}{1.000000,1.000000,1.000000}%
\pgfsetstrokecolor{currentstroke}%
\pgfsetdash{}{0pt}%
\pgfpathmoveto{\pgfqpoint{6.278780in}{1.612755in}}%
\pgfpathcurveto{\pgfqpoint{6.285913in}{1.612755in}}{\pgfqpoint{6.292755in}{1.615589in}}{\pgfqpoint{6.297798in}{1.620633in}}%
\pgfpathcurveto{\pgfqpoint{6.302842in}{1.625676in}}{\pgfqpoint{6.305676in}{1.632518in}}{\pgfqpoint{6.305676in}{1.639651in}}%
\pgfpathcurveto{\pgfqpoint{6.305676in}{1.646784in}}{\pgfqpoint{6.302842in}{1.653625in}}{\pgfqpoint{6.297798in}{1.658669in}}%
\pgfpathcurveto{\pgfqpoint{6.292755in}{1.663713in}}{\pgfqpoint{6.285913in}{1.666547in}}{\pgfqpoint{6.278780in}{1.666547in}}%
\pgfpathcurveto{\pgfqpoint{6.271647in}{1.666547in}}{\pgfqpoint{6.264806in}{1.663713in}}{\pgfqpoint{6.259762in}{1.658669in}}%
\pgfpathcurveto{\pgfqpoint{6.254718in}{1.653625in}}{\pgfqpoint{6.251884in}{1.646784in}}{\pgfqpoint{6.251884in}{1.639651in}}%
\pgfpathcurveto{\pgfqpoint{6.251884in}{1.632518in}}{\pgfqpoint{6.254718in}{1.625676in}}{\pgfqpoint{6.259762in}{1.620633in}}%
\pgfpathcurveto{\pgfqpoint{6.264806in}{1.615589in}}{\pgfqpoint{6.271647in}{1.612755in}}{\pgfqpoint{6.278780in}{1.612755in}}%
\pgfpathclose%
\pgfusepath{stroke,fill}%
\end{pgfscope}%
\begin{pgfscope}%
\pgfpathrectangle{\pgfqpoint{4.985294in}{0.500000in}}{\pgfqpoint{1.764706in}{1.700000in}}%
\pgfusepath{clip}%
\pgfsetbuttcap%
\pgfsetroundjoin%
\definecolor{currentfill}{rgb}{0.973271,0.850724,0.762998}%
\pgfsetfillcolor{currentfill}%
\pgfsetlinewidth{0.311001pt}%
\definecolor{currentstroke}{rgb}{1.000000,1.000000,1.000000}%
\pgfsetstrokecolor{currentstroke}%
\pgfsetdash{}{0pt}%
\pgfpathmoveto{\pgfqpoint{6.241762in}{1.508968in}}%
\pgfpathcurveto{\pgfqpoint{6.248895in}{1.508968in}}{\pgfqpoint{6.255737in}{1.511802in}}{\pgfqpoint{6.260780in}{1.516846in}}%
\pgfpathcurveto{\pgfqpoint{6.265824in}{1.521889in}}{\pgfqpoint{6.268658in}{1.528731in}}{\pgfqpoint{6.268658in}{1.535864in}}%
\pgfpathcurveto{\pgfqpoint{6.268658in}{1.542996in}}{\pgfqpoint{6.265824in}{1.549838in}}{\pgfqpoint{6.260780in}{1.554882in}}%
\pgfpathcurveto{\pgfqpoint{6.255737in}{1.559925in}}{\pgfqpoint{6.248895in}{1.562759in}}{\pgfqpoint{6.241762in}{1.562759in}}%
\pgfpathcurveto{\pgfqpoint{6.234629in}{1.562759in}}{\pgfqpoint{6.227788in}{1.559925in}}{\pgfqpoint{6.222744in}{1.554882in}}%
\pgfpathcurveto{\pgfqpoint{6.217700in}{1.549838in}}{\pgfqpoint{6.214867in}{1.542996in}}{\pgfqpoint{6.214867in}{1.535864in}}%
\pgfpathcurveto{\pgfqpoint{6.214867in}{1.528731in}}{\pgfqpoint{6.217700in}{1.521889in}}{\pgfqpoint{6.222744in}{1.516846in}}%
\pgfpathcurveto{\pgfqpoint{6.227788in}{1.511802in}}{\pgfqpoint{6.234629in}{1.508968in}}{\pgfqpoint{6.241762in}{1.508968in}}%
\pgfpathclose%
\pgfusepath{stroke,fill}%
\end{pgfscope}%
\begin{pgfscope}%
\pgfpathrectangle{\pgfqpoint{4.985294in}{0.500000in}}{\pgfqpoint{1.764706in}{1.700000in}}%
\pgfusepath{clip}%
\pgfsetbuttcap%
\pgfsetroundjoin%
\definecolor{currentfill}{rgb}{0.975644,0.874038,0.797253}%
\pgfsetfillcolor{currentfill}%
\pgfsetlinewidth{0.311001pt}%
\definecolor{currentstroke}{rgb}{1.000000,1.000000,1.000000}%
\pgfsetstrokecolor{currentstroke}%
\pgfsetdash{}{0pt}%
\pgfpathmoveto{\pgfqpoint{5.467069in}{1.540785in}}%
\pgfpathcurveto{\pgfqpoint{5.474202in}{1.540785in}}{\pgfqpoint{5.481043in}{1.543619in}}{\pgfqpoint{5.486087in}{1.548662in}}%
\pgfpathcurveto{\pgfqpoint{5.491131in}{1.553706in}}{\pgfqpoint{5.493965in}{1.560548in}}{\pgfqpoint{5.493965in}{1.567680in}}%
\pgfpathcurveto{\pgfqpoint{5.493965in}{1.574813in}}{\pgfqpoint{5.491131in}{1.581655in}}{\pgfqpoint{5.486087in}{1.586699in}}%
\pgfpathcurveto{\pgfqpoint{5.481043in}{1.591742in}}{\pgfqpoint{5.474202in}{1.594576in}}{\pgfqpoint{5.467069in}{1.594576in}}%
\pgfpathcurveto{\pgfqpoint{5.459936in}{1.594576in}}{\pgfqpoint{5.453094in}{1.591742in}}{\pgfqpoint{5.448051in}{1.586699in}}%
\pgfpathcurveto{\pgfqpoint{5.443007in}{1.581655in}}{\pgfqpoint{5.440173in}{1.574813in}}{\pgfqpoint{5.440173in}{1.567680in}}%
\pgfpathcurveto{\pgfqpoint{5.440173in}{1.560548in}}{\pgfqpoint{5.443007in}{1.553706in}}{\pgfqpoint{5.448051in}{1.548662in}}%
\pgfpathcurveto{\pgfqpoint{5.453094in}{1.543619in}}{\pgfqpoint{5.459936in}{1.540785in}}{\pgfqpoint{5.467069in}{1.540785in}}%
\pgfpathclose%
\pgfusepath{stroke,fill}%
\end{pgfscope}%
\begin{pgfscope}%
\pgfpathrectangle{\pgfqpoint{4.985294in}{0.500000in}}{\pgfqpoint{1.764706in}{1.700000in}}%
\pgfusepath{clip}%
\pgfsetbuttcap%
\pgfsetroundjoin%
\definecolor{currentfill}{rgb}{0.958791,0.526283,0.368909}%
\pgfsetfillcolor{currentfill}%
\pgfsetlinewidth{0.311001pt}%
\definecolor{currentstroke}{rgb}{1.000000,1.000000,1.000000}%
\pgfsetstrokecolor{currentstroke}%
\pgfsetdash{}{0pt}%
\pgfpathmoveto{\pgfqpoint{6.292082in}{1.716619in}}%
\pgfpathcurveto{\pgfqpoint{6.299215in}{1.716619in}}{\pgfqpoint{6.306056in}{1.719453in}}{\pgfqpoint{6.311100in}{1.724497in}}%
\pgfpathcurveto{\pgfqpoint{6.316144in}{1.729540in}}{\pgfqpoint{6.318977in}{1.736382in}}{\pgfqpoint{6.318977in}{1.743515in}}%
\pgfpathcurveto{\pgfqpoint{6.318977in}{1.750648in}}{\pgfqpoint{6.316144in}{1.757489in}}{\pgfqpoint{6.311100in}{1.762533in}}%
\pgfpathcurveto{\pgfqpoint{6.306056in}{1.767577in}}{\pgfqpoint{6.299215in}{1.770411in}}{\pgfqpoint{6.292082in}{1.770411in}}%
\pgfpathcurveto{\pgfqpoint{6.284949in}{1.770411in}}{\pgfqpoint{6.278107in}{1.767577in}}{\pgfqpoint{6.273064in}{1.762533in}}%
\pgfpathcurveto{\pgfqpoint{6.268020in}{1.757489in}}{\pgfqpoint{6.265186in}{1.750648in}}{\pgfqpoint{6.265186in}{1.743515in}}%
\pgfpathcurveto{\pgfqpoint{6.265186in}{1.736382in}}{\pgfqpoint{6.268020in}{1.729540in}}{\pgfqpoint{6.273064in}{1.724497in}}%
\pgfpathcurveto{\pgfqpoint{6.278107in}{1.719453in}}{\pgfqpoint{6.284949in}{1.716619in}}{\pgfqpoint{6.292082in}{1.716619in}}%
\pgfpathclose%
\pgfusepath{stroke,fill}%
\end{pgfscope}%
\begin{pgfscope}%
\pgfpathrectangle{\pgfqpoint{4.985294in}{0.500000in}}{\pgfqpoint{1.764706in}{1.700000in}}%
\pgfusepath{clip}%
\pgfsetbuttcap%
\pgfsetroundjoin%
\definecolor{currentfill}{rgb}{0.980678,0.914765,0.856766}%
\pgfsetfillcolor{currentfill}%
\pgfsetlinewidth{0.311001pt}%
\definecolor{currentstroke}{rgb}{1.000000,1.000000,1.000000}%
\pgfsetstrokecolor{currentstroke}%
\pgfsetdash{}{0pt}%
\pgfpathmoveto{\pgfqpoint{6.287016in}{1.498160in}}%
\pgfpathcurveto{\pgfqpoint{6.294149in}{1.498160in}}{\pgfqpoint{6.300991in}{1.500994in}}{\pgfqpoint{6.306034in}{1.506038in}}%
\pgfpathcurveto{\pgfqpoint{6.311078in}{1.511081in}}{\pgfqpoint{6.313912in}{1.517923in}}{\pgfqpoint{6.313912in}{1.525056in}}%
\pgfpathcurveto{\pgfqpoint{6.313912in}{1.532188in}}{\pgfqpoint{6.311078in}{1.539030in}}{\pgfqpoint{6.306034in}{1.544074in}}%
\pgfpathcurveto{\pgfqpoint{6.300991in}{1.549117in}}{\pgfqpoint{6.294149in}{1.551951in}}{\pgfqpoint{6.287016in}{1.551951in}}%
\pgfpathcurveto{\pgfqpoint{6.279883in}{1.551951in}}{\pgfqpoint{6.273042in}{1.549117in}}{\pgfqpoint{6.267998in}{1.544074in}}%
\pgfpathcurveto{\pgfqpoint{6.262954in}{1.539030in}}{\pgfqpoint{6.260121in}{1.532188in}}{\pgfqpoint{6.260121in}{1.525056in}}%
\pgfpathcurveto{\pgfqpoint{6.260121in}{1.517923in}}{\pgfqpoint{6.262954in}{1.511081in}}{\pgfqpoint{6.267998in}{1.506038in}}%
\pgfpathcurveto{\pgfqpoint{6.273042in}{1.500994in}}{\pgfqpoint{6.279883in}{1.498160in}}{\pgfqpoint{6.287016in}{1.498160in}}%
\pgfpathclose%
\pgfusepath{stroke,fill}%
\end{pgfscope}%
\begin{pgfscope}%
\pgfpathrectangle{\pgfqpoint{4.985294in}{0.500000in}}{\pgfqpoint{1.764706in}{1.700000in}}%
\pgfusepath{clip}%
\pgfsetbuttcap%
\pgfsetroundjoin%
\definecolor{currentfill}{rgb}{0.981377,0.920617,0.865369}%
\pgfsetfillcolor{currentfill}%
\pgfsetlinewidth{0.311001pt}%
\definecolor{currentstroke}{rgb}{1.000000,1.000000,1.000000}%
\pgfsetstrokecolor{currentstroke}%
\pgfsetdash{}{0pt}%
\pgfpathmoveto{\pgfqpoint{6.299635in}{1.451417in}}%
\pgfpathcurveto{\pgfqpoint{6.306768in}{1.451417in}}{\pgfqpoint{6.313610in}{1.454251in}}{\pgfqpoint{6.318653in}{1.459295in}}%
\pgfpathcurveto{\pgfqpoint{6.323697in}{1.464339in}}{\pgfqpoint{6.326531in}{1.471180in}}{\pgfqpoint{6.326531in}{1.478313in}}%
\pgfpathcurveto{\pgfqpoint{6.326531in}{1.485446in}}{\pgfqpoint{6.323697in}{1.492288in}}{\pgfqpoint{6.318653in}{1.497331in}}%
\pgfpathcurveto{\pgfqpoint{6.313610in}{1.502375in}}{\pgfqpoint{6.306768in}{1.505209in}}{\pgfqpoint{6.299635in}{1.505209in}}%
\pgfpathcurveto{\pgfqpoint{6.292502in}{1.505209in}}{\pgfqpoint{6.285661in}{1.502375in}}{\pgfqpoint{6.280617in}{1.497331in}}%
\pgfpathcurveto{\pgfqpoint{6.275573in}{1.492288in}}{\pgfqpoint{6.272739in}{1.485446in}}{\pgfqpoint{6.272739in}{1.478313in}}%
\pgfpathcurveto{\pgfqpoint{6.272739in}{1.471180in}}{\pgfqpoint{6.275573in}{1.464339in}}{\pgfqpoint{6.280617in}{1.459295in}}%
\pgfpathcurveto{\pgfqpoint{6.285661in}{1.454251in}}{\pgfqpoint{6.292502in}{1.451417in}}{\pgfqpoint{6.299635in}{1.451417in}}%
\pgfpathclose%
\pgfusepath{stroke,fill}%
\end{pgfscope}%
\begin{pgfscope}%
\pgfpathrectangle{\pgfqpoint{4.985294in}{0.500000in}}{\pgfqpoint{1.764706in}{1.700000in}}%
\pgfusepath{clip}%
\pgfsetbuttcap%
\pgfsetroundjoin%
\definecolor{currentfill}{rgb}{0.969803,0.809811,0.702523}%
\pgfsetfillcolor{currentfill}%
\pgfsetlinewidth{0.311001pt}%
\definecolor{currentstroke}{rgb}{1.000000,1.000000,1.000000}%
\pgfsetstrokecolor{currentstroke}%
\pgfsetdash{}{0pt}%
\pgfpathmoveto{\pgfqpoint{6.203014in}{1.153181in}}%
\pgfpathcurveto{\pgfqpoint{6.210147in}{1.153181in}}{\pgfqpoint{6.216988in}{1.156015in}}{\pgfqpoint{6.222032in}{1.161059in}}%
\pgfpathcurveto{\pgfqpoint{6.227076in}{1.166103in}}{\pgfqpoint{6.229909in}{1.172944in}}{\pgfqpoint{6.229909in}{1.180077in}}%
\pgfpathcurveto{\pgfqpoint{6.229909in}{1.187210in}}{\pgfqpoint{6.227076in}{1.194052in}}{\pgfqpoint{6.222032in}{1.199095in}}%
\pgfpathcurveto{\pgfqpoint{6.216988in}{1.204139in}}{\pgfqpoint{6.210147in}{1.206973in}}{\pgfqpoint{6.203014in}{1.206973in}}%
\pgfpathcurveto{\pgfqpoint{6.195881in}{1.206973in}}{\pgfqpoint{6.189039in}{1.204139in}}{\pgfqpoint{6.183996in}{1.199095in}}%
\pgfpathcurveto{\pgfqpoint{6.178952in}{1.194052in}}{\pgfqpoint{6.176118in}{1.187210in}}{\pgfqpoint{6.176118in}{1.180077in}}%
\pgfpathcurveto{\pgfqpoint{6.176118in}{1.172944in}}{\pgfqpoint{6.178952in}{1.166103in}}{\pgfqpoint{6.183996in}{1.161059in}}%
\pgfpathcurveto{\pgfqpoint{6.189039in}{1.156015in}}{\pgfqpoint{6.195881in}{1.153181in}}{\pgfqpoint{6.203014in}{1.153181in}}%
\pgfpathclose%
\pgfusepath{stroke,fill}%
\end{pgfscope}%
\begin{pgfscope}%
\pgfpathrectangle{\pgfqpoint{4.985294in}{0.500000in}}{\pgfqpoint{1.764706in}{1.700000in}}%
\pgfusepath{clip}%
\pgfsetbuttcap%
\pgfsetroundjoin%
\definecolor{currentfill}{rgb}{0.953126,0.456614,0.312398}%
\pgfsetfillcolor{currentfill}%
\pgfsetlinewidth{0.311001pt}%
\definecolor{currentstroke}{rgb}{1.000000,1.000000,1.000000}%
\pgfsetstrokecolor{currentstroke}%
\pgfsetdash{}{0pt}%
\pgfpathmoveto{\pgfqpoint{6.214090in}{0.848006in}}%
\pgfpathcurveto{\pgfqpoint{6.221223in}{0.848006in}}{\pgfqpoint{6.228065in}{0.850840in}}{\pgfqpoint{6.233109in}{0.855883in}}%
\pgfpathcurveto{\pgfqpoint{6.238152in}{0.860927in}}{\pgfqpoint{6.240986in}{0.867769in}}{\pgfqpoint{6.240986in}{0.874901in}}%
\pgfpathcurveto{\pgfqpoint{6.240986in}{0.882034in}}{\pgfqpoint{6.238152in}{0.888876in}}{\pgfqpoint{6.233109in}{0.893920in}}%
\pgfpathcurveto{\pgfqpoint{6.228065in}{0.898963in}}{\pgfqpoint{6.221223in}{0.901797in}}{\pgfqpoint{6.214090in}{0.901797in}}%
\pgfpathcurveto{\pgfqpoint{6.206958in}{0.901797in}}{\pgfqpoint{6.200116in}{0.898963in}}{\pgfqpoint{6.195072in}{0.893920in}}%
\pgfpathcurveto{\pgfqpoint{6.190029in}{0.888876in}}{\pgfqpoint{6.187195in}{0.882034in}}{\pgfqpoint{6.187195in}{0.874901in}}%
\pgfpathcurveto{\pgfqpoint{6.187195in}{0.867769in}}{\pgfqpoint{6.190029in}{0.860927in}}{\pgfqpoint{6.195072in}{0.855883in}}%
\pgfpathcurveto{\pgfqpoint{6.200116in}{0.850840in}}{\pgfqpoint{6.206958in}{0.848006in}}{\pgfqpoint{6.214090in}{0.848006in}}%
\pgfpathclose%
\pgfusepath{stroke,fill}%
\end{pgfscope}%
\begin{pgfscope}%
\pgfpathrectangle{\pgfqpoint{4.985294in}{0.500000in}}{\pgfqpoint{1.764706in}{1.700000in}}%
\pgfusepath{clip}%
\pgfsetbuttcap%
\pgfsetroundjoin%
\definecolor{currentfill}{rgb}{0.976961,0.885681,0.814303}%
\pgfsetfillcolor{currentfill}%
\pgfsetlinewidth{0.311001pt}%
\definecolor{currentstroke}{rgb}{1.000000,1.000000,1.000000}%
\pgfsetstrokecolor{currentstroke}%
\pgfsetdash{}{0pt}%
\pgfpathmoveto{\pgfqpoint{6.319600in}{1.508657in}}%
\pgfpathcurveto{\pgfqpoint{6.326733in}{1.508657in}}{\pgfqpoint{6.333574in}{1.511490in}}{\pgfqpoint{6.338618in}{1.516534in}}%
\pgfpathcurveto{\pgfqpoint{6.343662in}{1.521578in}}{\pgfqpoint{6.346496in}{1.528419in}}{\pgfqpoint{6.346496in}{1.535552in}}%
\pgfpathcurveto{\pgfqpoint{6.346496in}{1.542685in}}{\pgfqpoint{6.343662in}{1.549527in}}{\pgfqpoint{6.338618in}{1.554570in}}%
\pgfpathcurveto{\pgfqpoint{6.333574in}{1.559614in}}{\pgfqpoint{6.326733in}{1.562448in}}{\pgfqpoint{6.319600in}{1.562448in}}%
\pgfpathcurveto{\pgfqpoint{6.312467in}{1.562448in}}{\pgfqpoint{6.305625in}{1.559614in}}{\pgfqpoint{6.300582in}{1.554570in}}%
\pgfpathcurveto{\pgfqpoint{6.295538in}{1.549527in}}{\pgfqpoint{6.292704in}{1.542685in}}{\pgfqpoint{6.292704in}{1.535552in}}%
\pgfpathcurveto{\pgfqpoint{6.292704in}{1.528419in}}{\pgfqpoint{6.295538in}{1.521578in}}{\pgfqpoint{6.300582in}{1.516534in}}%
\pgfpathcurveto{\pgfqpoint{6.305625in}{1.511490in}}{\pgfqpoint{6.312467in}{1.508657in}}{\pgfqpoint{6.319600in}{1.508657in}}%
\pgfpathclose%
\pgfusepath{stroke,fill}%
\end{pgfscope}%
\begin{pgfscope}%
\pgfpathrectangle{\pgfqpoint{4.985294in}{0.500000in}}{\pgfqpoint{1.764706in}{1.700000in}}%
\pgfusepath{clip}%
\pgfsetbuttcap%
\pgfsetroundjoin%
\definecolor{currentfill}{rgb}{0.914423,0.260289,0.243694}%
\pgfsetfillcolor{currentfill}%
\pgfsetlinewidth{0.311001pt}%
\definecolor{currentstroke}{rgb}{1.000000,1.000000,1.000000}%
\pgfsetstrokecolor{currentstroke}%
\pgfsetdash{}{0pt}%
\pgfpathmoveto{\pgfqpoint{5.749996in}{1.697010in}}%
\pgfpathcurveto{\pgfqpoint{5.757129in}{1.697010in}}{\pgfqpoint{5.763971in}{1.699844in}}{\pgfqpoint{5.769014in}{1.704888in}}%
\pgfpathcurveto{\pgfqpoint{5.774058in}{1.709932in}}{\pgfqpoint{5.776892in}{1.716773in}}{\pgfqpoint{5.776892in}{1.723906in}}%
\pgfpathcurveto{\pgfqpoint{5.776892in}{1.731039in}}{\pgfqpoint{5.774058in}{1.737880in}}{\pgfqpoint{5.769014in}{1.742924in}}%
\pgfpathcurveto{\pgfqpoint{5.763971in}{1.747968in}}{\pgfqpoint{5.757129in}{1.750802in}}{\pgfqpoint{5.749996in}{1.750802in}}%
\pgfpathcurveto{\pgfqpoint{5.742863in}{1.750802in}}{\pgfqpoint{5.736022in}{1.747968in}}{\pgfqpoint{5.730978in}{1.742924in}}%
\pgfpathcurveto{\pgfqpoint{5.725934in}{1.737880in}}{\pgfqpoint{5.723101in}{1.731039in}}{\pgfqpoint{5.723101in}{1.723906in}}%
\pgfpathcurveto{\pgfqpoint{5.723101in}{1.716773in}}{\pgfqpoint{5.725934in}{1.709932in}}{\pgfqpoint{5.730978in}{1.704888in}}%
\pgfpathcurveto{\pgfqpoint{5.736022in}{1.699844in}}{\pgfqpoint{5.742863in}{1.697010in}}{\pgfqpoint{5.749996in}{1.697010in}}%
\pgfpathclose%
\pgfusepath{stroke,fill}%
\end{pgfscope}%
\begin{pgfscope}%
\pgfpathrectangle{\pgfqpoint{4.985294in}{0.500000in}}{\pgfqpoint{1.764706in}{1.700000in}}%
\pgfusepath{clip}%
\pgfsetbuttcap%
\pgfsetroundjoin%
\definecolor{currentfill}{rgb}{0.972201,0.839051,0.745789}%
\pgfsetfillcolor{currentfill}%
\pgfsetlinewidth{0.311001pt}%
\definecolor{currentstroke}{rgb}{1.000000,1.000000,1.000000}%
\pgfsetstrokecolor{currentstroke}%
\pgfsetdash{}{0pt}%
\pgfpathmoveto{\pgfqpoint{6.206933in}{1.558982in}}%
\pgfpathcurveto{\pgfqpoint{6.214065in}{1.558982in}}{\pgfqpoint{6.220907in}{1.561816in}}{\pgfqpoint{6.225951in}{1.566860in}}%
\pgfpathcurveto{\pgfqpoint{6.230994in}{1.571903in}}{\pgfqpoint{6.233828in}{1.578745in}}{\pgfqpoint{6.233828in}{1.585878in}}%
\pgfpathcurveto{\pgfqpoint{6.233828in}{1.593011in}}{\pgfqpoint{6.230994in}{1.599852in}}{\pgfqpoint{6.225951in}{1.604896in}}%
\pgfpathcurveto{\pgfqpoint{6.220907in}{1.609940in}}{\pgfqpoint{6.214065in}{1.612774in}}{\pgfqpoint{6.206933in}{1.612774in}}%
\pgfpathcurveto{\pgfqpoint{6.199800in}{1.612774in}}{\pgfqpoint{6.192958in}{1.609940in}}{\pgfqpoint{6.187914in}{1.604896in}}%
\pgfpathcurveto{\pgfqpoint{6.182871in}{1.599852in}}{\pgfqpoint{6.180037in}{1.593011in}}{\pgfqpoint{6.180037in}{1.585878in}}%
\pgfpathcurveto{\pgfqpoint{6.180037in}{1.578745in}}{\pgfqpoint{6.182871in}{1.571903in}}{\pgfqpoint{6.187914in}{1.566860in}}%
\pgfpathcurveto{\pgfqpoint{6.192958in}{1.561816in}}{\pgfqpoint{6.199800in}{1.558982in}}{\pgfqpoint{6.206933in}{1.558982in}}%
\pgfpathclose%
\pgfusepath{stroke,fill}%
\end{pgfscope}%
\begin{pgfscope}%
\pgfpathrectangle{\pgfqpoint{4.985294in}{0.500000in}}{\pgfqpoint{1.764706in}{1.700000in}}%
\pgfusepath{clip}%
\pgfsetbuttcap%
\pgfsetroundjoin%
\definecolor{currentfill}{rgb}{0.979891,0.908948,0.848279}%
\pgfsetfillcolor{currentfill}%
\pgfsetlinewidth{0.311001pt}%
\definecolor{currentstroke}{rgb}{1.000000,1.000000,1.000000}%
\pgfsetstrokecolor{currentstroke}%
\pgfsetdash{}{0pt}%
\pgfpathmoveto{\pgfqpoint{6.292427in}{1.232070in}}%
\pgfpathcurveto{\pgfqpoint{6.299560in}{1.232070in}}{\pgfqpoint{6.306402in}{1.234904in}}{\pgfqpoint{6.311445in}{1.239947in}}%
\pgfpathcurveto{\pgfqpoint{6.316489in}{1.244991in}}{\pgfqpoint{6.319323in}{1.251833in}}{\pgfqpoint{6.319323in}{1.258966in}}%
\pgfpathcurveto{\pgfqpoint{6.319323in}{1.266098in}}{\pgfqpoint{6.316489in}{1.272940in}}{\pgfqpoint{6.311445in}{1.277984in}}%
\pgfpathcurveto{\pgfqpoint{6.306402in}{1.283027in}}{\pgfqpoint{6.299560in}{1.285861in}}{\pgfqpoint{6.292427in}{1.285861in}}%
\pgfpathcurveto{\pgfqpoint{6.285294in}{1.285861in}}{\pgfqpoint{6.278453in}{1.283027in}}{\pgfqpoint{6.273409in}{1.277984in}}%
\pgfpathcurveto{\pgfqpoint{6.268365in}{1.272940in}}{\pgfqpoint{6.265531in}{1.266098in}}{\pgfqpoint{6.265531in}{1.258966in}}%
\pgfpathcurveto{\pgfqpoint{6.265531in}{1.251833in}}{\pgfqpoint{6.268365in}{1.244991in}}{\pgfqpoint{6.273409in}{1.239947in}}%
\pgfpathcurveto{\pgfqpoint{6.278453in}{1.234904in}}{\pgfqpoint{6.285294in}{1.232070in}}{\pgfqpoint{6.292427in}{1.232070in}}%
\pgfpathclose%
\pgfusepath{stroke,fill}%
\end{pgfscope}%
\begin{pgfscope}%
\pgfpathrectangle{\pgfqpoint{4.985294in}{0.500000in}}{\pgfqpoint{1.764706in}{1.700000in}}%
\pgfusepath{clip}%
\pgfsetbuttcap%
\pgfsetroundjoin%
\definecolor{currentfill}{rgb}{0.972201,0.839051,0.745789}%
\pgfsetfillcolor{currentfill}%
\pgfsetlinewidth{0.311001pt}%
\definecolor{currentstroke}{rgb}{1.000000,1.000000,1.000000}%
\pgfsetstrokecolor{currentstroke}%
\pgfsetdash{}{0pt}%
\pgfpathmoveto{\pgfqpoint{5.516716in}{1.521466in}}%
\pgfpathcurveto{\pgfqpoint{5.523849in}{1.521466in}}{\pgfqpoint{5.530691in}{1.524300in}}{\pgfqpoint{5.535734in}{1.529344in}}%
\pgfpathcurveto{\pgfqpoint{5.540778in}{1.534387in}}{\pgfqpoint{5.543612in}{1.541229in}}{\pgfqpoint{5.543612in}{1.548362in}}%
\pgfpathcurveto{\pgfqpoint{5.543612in}{1.555495in}}{\pgfqpoint{5.540778in}{1.562336in}}{\pgfqpoint{5.535734in}{1.567380in}}%
\pgfpathcurveto{\pgfqpoint{5.530691in}{1.572424in}}{\pgfqpoint{5.523849in}{1.575257in}}{\pgfqpoint{5.516716in}{1.575257in}}%
\pgfpathcurveto{\pgfqpoint{5.509583in}{1.575257in}}{\pgfqpoint{5.502742in}{1.572424in}}{\pgfqpoint{5.497698in}{1.567380in}}%
\pgfpathcurveto{\pgfqpoint{5.492654in}{1.562336in}}{\pgfqpoint{5.489820in}{1.555495in}}{\pgfqpoint{5.489820in}{1.548362in}}%
\pgfpathcurveto{\pgfqpoint{5.489820in}{1.541229in}}{\pgfqpoint{5.492654in}{1.534387in}}{\pgfqpoint{5.497698in}{1.529344in}}%
\pgfpathcurveto{\pgfqpoint{5.502742in}{1.524300in}}{\pgfqpoint{5.509583in}{1.521466in}}{\pgfqpoint{5.516716in}{1.521466in}}%
\pgfpathclose%
\pgfusepath{stroke,fill}%
\end{pgfscope}%
\begin{pgfscope}%
\pgfpathrectangle{\pgfqpoint{4.985294in}{0.500000in}}{\pgfqpoint{1.764706in}{1.700000in}}%
\pgfusepath{clip}%
\pgfsetbuttcap%
\pgfsetroundjoin%
\definecolor{currentfill}{rgb}{0.978376,0.897317,0.831308}%
\pgfsetfillcolor{currentfill}%
\pgfsetlinewidth{0.311001pt}%
\definecolor{currentstroke}{rgb}{1.000000,1.000000,1.000000}%
\pgfsetstrokecolor{currentstroke}%
\pgfsetdash{}{0pt}%
\pgfpathmoveto{\pgfqpoint{6.280681in}{1.140479in}}%
\pgfpathcurveto{\pgfqpoint{6.287814in}{1.140479in}}{\pgfqpoint{6.294656in}{1.143312in}}{\pgfqpoint{6.299699in}{1.148356in}}%
\pgfpathcurveto{\pgfqpoint{6.304743in}{1.153400in}}{\pgfqpoint{6.307577in}{1.160241in}}{\pgfqpoint{6.307577in}{1.167374in}}%
\pgfpathcurveto{\pgfqpoint{6.307577in}{1.174507in}}{\pgfqpoint{6.304743in}{1.181349in}}{\pgfqpoint{6.299699in}{1.186392in}}%
\pgfpathcurveto{\pgfqpoint{6.294656in}{1.191436in}}{\pgfqpoint{6.287814in}{1.194270in}}{\pgfqpoint{6.280681in}{1.194270in}}%
\pgfpathcurveto{\pgfqpoint{6.273548in}{1.194270in}}{\pgfqpoint{6.266707in}{1.191436in}}{\pgfqpoint{6.261663in}{1.186392in}}%
\pgfpathcurveto{\pgfqpoint{6.256619in}{1.181349in}}{\pgfqpoint{6.253785in}{1.174507in}}{\pgfqpoint{6.253785in}{1.167374in}}%
\pgfpathcurveto{\pgfqpoint{6.253785in}{1.160241in}}{\pgfqpoint{6.256619in}{1.153400in}}{\pgfqpoint{6.261663in}{1.148356in}}%
\pgfpathcurveto{\pgfqpoint{6.266707in}{1.143312in}}{\pgfqpoint{6.273548in}{1.140479in}}{\pgfqpoint{6.280681in}{1.140479in}}%
\pgfpathclose%
\pgfusepath{stroke,fill}%
\end{pgfscope}%
\begin{pgfscope}%
\pgfpathrectangle{\pgfqpoint{4.985294in}{0.500000in}}{\pgfqpoint{1.764706in}{1.700000in}}%
\pgfusepath{clip}%
\pgfsetbuttcap%
\pgfsetroundjoin%
\definecolor{currentfill}{rgb}{0.966328,0.750560,0.616961}%
\pgfsetfillcolor{currentfill}%
\pgfsetlinewidth{0.311001pt}%
\definecolor{currentstroke}{rgb}{1.000000,1.000000,1.000000}%
\pgfsetstrokecolor{currentstroke}%
\pgfsetdash{}{0pt}%
\pgfpathmoveto{\pgfqpoint{5.544127in}{0.923459in}}%
\pgfpathcurveto{\pgfqpoint{5.551260in}{0.923459in}}{\pgfqpoint{5.558102in}{0.926293in}}{\pgfqpoint{5.563145in}{0.931336in}}%
\pgfpathcurveto{\pgfqpoint{5.568189in}{0.936380in}}{\pgfqpoint{5.571023in}{0.943222in}}{\pgfqpoint{5.571023in}{0.950354in}}%
\pgfpathcurveto{\pgfqpoint{5.571023in}{0.957487in}}{\pgfqpoint{5.568189in}{0.964329in}}{\pgfqpoint{5.563145in}{0.969373in}}%
\pgfpathcurveto{\pgfqpoint{5.558102in}{0.974416in}}{\pgfqpoint{5.551260in}{0.977250in}}{\pgfqpoint{5.544127in}{0.977250in}}%
\pgfpathcurveto{\pgfqpoint{5.536994in}{0.977250in}}{\pgfqpoint{5.530153in}{0.974416in}}{\pgfqpoint{5.525109in}{0.969373in}}%
\pgfpathcurveto{\pgfqpoint{5.520065in}{0.964329in}}{\pgfqpoint{5.517231in}{0.957487in}}{\pgfqpoint{5.517231in}{0.950354in}}%
\pgfpathcurveto{\pgfqpoint{5.517231in}{0.943222in}}{\pgfqpoint{5.520065in}{0.936380in}}{\pgfqpoint{5.525109in}{0.931336in}}%
\pgfpathcurveto{\pgfqpoint{5.530153in}{0.926293in}}{\pgfqpoint{5.536994in}{0.923459in}}{\pgfqpoint{5.544127in}{0.923459in}}%
\pgfpathclose%
\pgfusepath{stroke,fill}%
\end{pgfscope}%
\begin{pgfscope}%
\pgfpathrectangle{\pgfqpoint{4.985294in}{0.500000in}}{\pgfqpoint{1.764706in}{1.700000in}}%
\pgfusepath{clip}%
\pgfsetbuttcap%
\pgfsetroundjoin%
\definecolor{currentfill}{rgb}{0.942910,0.375495,0.263698}%
\pgfsetfillcolor{currentfill}%
\pgfsetlinewidth{0.311001pt}%
\definecolor{currentstroke}{rgb}{1.000000,1.000000,1.000000}%
\pgfsetstrokecolor{currentstroke}%
\pgfsetdash{}{0pt}%
\pgfpathmoveto{\pgfqpoint{6.022663in}{0.908866in}}%
\pgfpathcurveto{\pgfqpoint{6.029796in}{0.908866in}}{\pgfqpoint{6.036638in}{0.911700in}}{\pgfqpoint{6.041681in}{0.916743in}}%
\pgfpathcurveto{\pgfqpoint{6.046725in}{0.921787in}}{\pgfqpoint{6.049559in}{0.928629in}}{\pgfqpoint{6.049559in}{0.935762in}}%
\pgfpathcurveto{\pgfqpoint{6.049559in}{0.942894in}}{\pgfqpoint{6.046725in}{0.949736in}}{\pgfqpoint{6.041681in}{0.954780in}}%
\pgfpathcurveto{\pgfqpoint{6.036638in}{0.959823in}}{\pgfqpoint{6.029796in}{0.962657in}}{\pgfqpoint{6.022663in}{0.962657in}}%
\pgfpathcurveto{\pgfqpoint{6.015530in}{0.962657in}}{\pgfqpoint{6.008689in}{0.959823in}}{\pgfqpoint{6.003645in}{0.954780in}}%
\pgfpathcurveto{\pgfqpoint{5.998601in}{0.949736in}}{\pgfqpoint{5.995767in}{0.942894in}}{\pgfqpoint{5.995767in}{0.935762in}}%
\pgfpathcurveto{\pgfqpoint{5.995767in}{0.928629in}}{\pgfqpoint{5.998601in}{0.921787in}}{\pgfqpoint{6.003645in}{0.916743in}}%
\pgfpathcurveto{\pgfqpoint{6.008689in}{0.911700in}}{\pgfqpoint{6.015530in}{0.908866in}}{\pgfqpoint{6.022663in}{0.908866in}}%
\pgfpathclose%
\pgfusepath{stroke,fill}%
\end{pgfscope}%
\begin{pgfscope}%
\pgfpathrectangle{\pgfqpoint{4.985294in}{0.500000in}}{\pgfqpoint{1.764706in}{1.700000in}}%
\pgfusepath{clip}%
\pgfsetbuttcap%
\pgfsetroundjoin%
\definecolor{currentfill}{rgb}{0.948235,0.413004,0.283323}%
\pgfsetfillcolor{currentfill}%
\pgfsetlinewidth{0.311001pt}%
\definecolor{currentstroke}{rgb}{1.000000,1.000000,1.000000}%
\pgfsetstrokecolor{currentstroke}%
\pgfsetdash{}{0pt}%
\pgfpathmoveto{\pgfqpoint{6.350717in}{1.697460in}}%
\pgfpathcurveto{\pgfqpoint{6.357849in}{1.697460in}}{\pgfqpoint{6.364691in}{1.700294in}}{\pgfqpoint{6.369735in}{1.705337in}}%
\pgfpathcurveto{\pgfqpoint{6.374778in}{1.710381in}}{\pgfqpoint{6.377612in}{1.717223in}}{\pgfqpoint{6.377612in}{1.724355in}}%
\pgfpathcurveto{\pgfqpoint{6.377612in}{1.731488in}}{\pgfqpoint{6.374778in}{1.738330in}}{\pgfqpoint{6.369735in}{1.743374in}}%
\pgfpathcurveto{\pgfqpoint{6.364691in}{1.748417in}}{\pgfqpoint{6.357849in}{1.751251in}}{\pgfqpoint{6.350717in}{1.751251in}}%
\pgfpathcurveto{\pgfqpoint{6.343584in}{1.751251in}}{\pgfqpoint{6.336742in}{1.748417in}}{\pgfqpoint{6.331698in}{1.743374in}}%
\pgfpathcurveto{\pgfqpoint{6.326655in}{1.738330in}}{\pgfqpoint{6.323821in}{1.731488in}}{\pgfqpoint{6.323821in}{1.724355in}}%
\pgfpathcurveto{\pgfqpoint{6.323821in}{1.717223in}}{\pgfqpoint{6.326655in}{1.710381in}}{\pgfqpoint{6.331698in}{1.705337in}}%
\pgfpathcurveto{\pgfqpoint{6.336742in}{1.700294in}}{\pgfqpoint{6.343584in}{1.697460in}}{\pgfqpoint{6.350717in}{1.697460in}}%
\pgfpathclose%
\pgfusepath{stroke,fill}%
\end{pgfscope}%
\begin{pgfscope}%
\pgfpathrectangle{\pgfqpoint{4.985294in}{0.500000in}}{\pgfqpoint{1.764706in}{1.700000in}}%
\pgfusepath{clip}%
\pgfsetbuttcap%
\pgfsetroundjoin%
\definecolor{currentfill}{rgb}{0.922239,0.282873,0.242296}%
\pgfsetfillcolor{currentfill}%
\pgfsetlinewidth{0.311001pt}%
\definecolor{currentstroke}{rgb}{1.000000,1.000000,1.000000}%
\pgfsetstrokecolor{currentstroke}%
\pgfsetdash{}{0pt}%
\pgfpathmoveto{\pgfqpoint{6.178689in}{0.803970in}}%
\pgfpathcurveto{\pgfqpoint{6.185821in}{0.803970in}}{\pgfqpoint{6.192663in}{0.806804in}}{\pgfqpoint{6.197707in}{0.811848in}}%
\pgfpathcurveto{\pgfqpoint{6.202750in}{0.816891in}}{\pgfqpoint{6.205584in}{0.823733in}}{\pgfqpoint{6.205584in}{0.830866in}}%
\pgfpathcurveto{\pgfqpoint{6.205584in}{0.837999in}}{\pgfqpoint{6.202750in}{0.844840in}}{\pgfqpoint{6.197707in}{0.849884in}}%
\pgfpathcurveto{\pgfqpoint{6.192663in}{0.854928in}}{\pgfqpoint{6.185821in}{0.857762in}}{\pgfqpoint{6.178689in}{0.857762in}}%
\pgfpathcurveto{\pgfqpoint{6.171556in}{0.857762in}}{\pgfqpoint{6.164714in}{0.854928in}}{\pgfqpoint{6.159670in}{0.849884in}}%
\pgfpathcurveto{\pgfqpoint{6.154627in}{0.844840in}}{\pgfqpoint{6.151793in}{0.837999in}}{\pgfqpoint{6.151793in}{0.830866in}}%
\pgfpathcurveto{\pgfqpoint{6.151793in}{0.823733in}}{\pgfqpoint{6.154627in}{0.816891in}}{\pgfqpoint{6.159670in}{0.811848in}}%
\pgfpathcurveto{\pgfqpoint{6.164714in}{0.806804in}}{\pgfqpoint{6.171556in}{0.803970in}}{\pgfqpoint{6.178689in}{0.803970in}}%
\pgfpathclose%
\pgfusepath{stroke,fill}%
\end{pgfscope}%
\begin{pgfscope}%
\pgfpathrectangle{\pgfqpoint{4.985294in}{0.500000in}}{\pgfqpoint{1.764706in}{1.700000in}}%
\pgfusepath{clip}%
\pgfsetbuttcap%
\pgfsetroundjoin%
\definecolor{currentfill}{rgb}{0.970255,0.815666,0.711203}%
\pgfsetfillcolor{currentfill}%
\pgfsetlinewidth{0.311001pt}%
\definecolor{currentstroke}{rgb}{1.000000,1.000000,1.000000}%
\pgfsetstrokecolor{currentstroke}%
\pgfsetdash{}{0pt}%
\pgfpathmoveto{\pgfqpoint{6.382075in}{1.255915in}}%
\pgfpathcurveto{\pgfqpoint{6.389208in}{1.255915in}}{\pgfqpoint{6.396050in}{1.258749in}}{\pgfqpoint{6.401094in}{1.263793in}}%
\pgfpathcurveto{\pgfqpoint{6.406137in}{1.268837in}}{\pgfqpoint{6.408971in}{1.275678in}}{\pgfqpoint{6.408971in}{1.282811in}}%
\pgfpathcurveto{\pgfqpoint{6.408971in}{1.289944in}}{\pgfqpoint{6.406137in}{1.296785in}}{\pgfqpoint{6.401094in}{1.301829in}}%
\pgfpathcurveto{\pgfqpoint{6.396050in}{1.306873in}}{\pgfqpoint{6.389208in}{1.309707in}}{\pgfqpoint{6.382075in}{1.309707in}}%
\pgfpathcurveto{\pgfqpoint{6.374943in}{1.309707in}}{\pgfqpoint{6.368101in}{1.306873in}}{\pgfqpoint{6.363057in}{1.301829in}}%
\pgfpathcurveto{\pgfqpoint{6.358014in}{1.296785in}}{\pgfqpoint{6.355180in}{1.289944in}}{\pgfqpoint{6.355180in}{1.282811in}}%
\pgfpathcurveto{\pgfqpoint{6.355180in}{1.275678in}}{\pgfqpoint{6.358014in}{1.268837in}}{\pgfqpoint{6.363057in}{1.263793in}}%
\pgfpathcurveto{\pgfqpoint{6.368101in}{1.258749in}}{\pgfqpoint{6.374943in}{1.255915in}}{\pgfqpoint{6.382075in}{1.255915in}}%
\pgfpathclose%
\pgfusepath{stroke,fill}%
\end{pgfscope}%
\begin{pgfscope}%
\pgfpathrectangle{\pgfqpoint{4.985294in}{0.500000in}}{\pgfqpoint{1.764706in}{1.700000in}}%
\pgfusepath{clip}%
\pgfsetbuttcap%
\pgfsetroundjoin%
\definecolor{currentfill}{rgb}{0.966328,0.750560,0.616961}%
\pgfsetfillcolor{currentfill}%
\pgfsetlinewidth{0.311001pt}%
\definecolor{currentstroke}{rgb}{1.000000,1.000000,1.000000}%
\pgfsetstrokecolor{currentstroke}%
\pgfsetdash{}{0pt}%
\pgfpathmoveto{\pgfqpoint{5.522353in}{1.699880in}}%
\pgfpathcurveto{\pgfqpoint{5.529485in}{1.699880in}}{\pgfqpoint{5.536327in}{1.702714in}}{\pgfqpoint{5.541371in}{1.707757in}}%
\pgfpathcurveto{\pgfqpoint{5.546414in}{1.712801in}}{\pgfqpoint{5.549248in}{1.719643in}}{\pgfqpoint{5.549248in}{1.726776in}}%
\pgfpathcurveto{\pgfqpoint{5.549248in}{1.733908in}}{\pgfqpoint{5.546414in}{1.740750in}}{\pgfqpoint{5.541371in}{1.745794in}}%
\pgfpathcurveto{\pgfqpoint{5.536327in}{1.750837in}}{\pgfqpoint{5.529485in}{1.753671in}}{\pgfqpoint{5.522353in}{1.753671in}}%
\pgfpathcurveto{\pgfqpoint{5.515220in}{1.753671in}}{\pgfqpoint{5.508378in}{1.750837in}}{\pgfqpoint{5.503334in}{1.745794in}}%
\pgfpathcurveto{\pgfqpoint{5.498291in}{1.740750in}}{\pgfqpoint{5.495457in}{1.733908in}}{\pgfqpoint{5.495457in}{1.726776in}}%
\pgfpathcurveto{\pgfqpoint{5.495457in}{1.719643in}}{\pgfqpoint{5.498291in}{1.712801in}}{\pgfqpoint{5.503334in}{1.707757in}}%
\pgfpathcurveto{\pgfqpoint{5.508378in}{1.702714in}}{\pgfqpoint{5.515220in}{1.699880in}}{\pgfqpoint{5.522353in}{1.699880in}}%
\pgfpathclose%
\pgfusepath{stroke,fill}%
\end{pgfscope}%
\begin{pgfscope}%
\pgfpathrectangle{\pgfqpoint{4.985294in}{0.500000in}}{\pgfqpoint{1.764706in}{1.700000in}}%
\pgfusepath{clip}%
\pgfsetbuttcap%
\pgfsetroundjoin%
\definecolor{currentfill}{rgb}{0.976287,0.879862,0.805788}%
\pgfsetfillcolor{currentfill}%
\pgfsetlinewidth{0.311001pt}%
\definecolor{currentstroke}{rgb}{1.000000,1.000000,1.000000}%
\pgfsetstrokecolor{currentstroke}%
\pgfsetdash{}{0pt}%
\pgfpathmoveto{\pgfqpoint{6.354500in}{1.377516in}}%
\pgfpathcurveto{\pgfqpoint{6.361633in}{1.377516in}}{\pgfqpoint{6.368475in}{1.380349in}}{\pgfqpoint{6.373518in}{1.385393in}}%
\pgfpathcurveto{\pgfqpoint{6.378562in}{1.390437in}}{\pgfqpoint{6.381396in}{1.397278in}}{\pgfqpoint{6.381396in}{1.404411in}}%
\pgfpathcurveto{\pgfqpoint{6.381396in}{1.411544in}}{\pgfqpoint{6.378562in}{1.418386in}}{\pgfqpoint{6.373518in}{1.423429in}}%
\pgfpathcurveto{\pgfqpoint{6.368475in}{1.428473in}}{\pgfqpoint{6.361633in}{1.431307in}}{\pgfqpoint{6.354500in}{1.431307in}}%
\pgfpathcurveto{\pgfqpoint{6.347367in}{1.431307in}}{\pgfqpoint{6.340526in}{1.428473in}}{\pgfqpoint{6.335482in}{1.423429in}}%
\pgfpathcurveto{\pgfqpoint{6.330438in}{1.418386in}}{\pgfqpoint{6.327604in}{1.411544in}}{\pgfqpoint{6.327604in}{1.404411in}}%
\pgfpathcurveto{\pgfqpoint{6.327604in}{1.397278in}}{\pgfqpoint{6.330438in}{1.390437in}}{\pgfqpoint{6.335482in}{1.385393in}}%
\pgfpathcurveto{\pgfqpoint{6.340526in}{1.380349in}}{\pgfqpoint{6.347367in}{1.377516in}}{\pgfqpoint{6.354500in}{1.377516in}}%
\pgfpathclose%
\pgfusepath{stroke,fill}%
\end{pgfscope}%
\begin{pgfscope}%
\pgfpathrectangle{\pgfqpoint{4.985294in}{0.500000in}}{\pgfqpoint{1.764706in}{1.700000in}}%
\pgfusepath{clip}%
\pgfsetbuttcap%
\pgfsetroundjoin%
\definecolor{currentfill}{rgb}{0.971202,0.827364,0.728520}%
\pgfsetfillcolor{currentfill}%
\pgfsetlinewidth{0.311001pt}%
\definecolor{currentstroke}{rgb}{1.000000,1.000000,1.000000}%
\pgfsetstrokecolor{currentstroke}%
\pgfsetdash{}{0pt}%
\pgfpathmoveto{\pgfqpoint{5.505369in}{1.624877in}}%
\pgfpathcurveto{\pgfqpoint{5.512502in}{1.624877in}}{\pgfqpoint{5.519343in}{1.627711in}}{\pgfqpoint{5.524387in}{1.632755in}}%
\pgfpathcurveto{\pgfqpoint{5.529431in}{1.637798in}}{\pgfqpoint{5.532264in}{1.644640in}}{\pgfqpoint{5.532264in}{1.651773in}}%
\pgfpathcurveto{\pgfqpoint{5.532264in}{1.658906in}}{\pgfqpoint{5.529431in}{1.665747in}}{\pgfqpoint{5.524387in}{1.670791in}}%
\pgfpathcurveto{\pgfqpoint{5.519343in}{1.675835in}}{\pgfqpoint{5.512502in}{1.678669in}}{\pgfqpoint{5.505369in}{1.678669in}}%
\pgfpathcurveto{\pgfqpoint{5.498236in}{1.678669in}}{\pgfqpoint{5.491394in}{1.675835in}}{\pgfqpoint{5.486351in}{1.670791in}}%
\pgfpathcurveto{\pgfqpoint{5.481307in}{1.665747in}}{\pgfqpoint{5.478473in}{1.658906in}}{\pgfqpoint{5.478473in}{1.651773in}}%
\pgfpathcurveto{\pgfqpoint{5.478473in}{1.644640in}}{\pgfqpoint{5.481307in}{1.637798in}}{\pgfqpoint{5.486351in}{1.632755in}}%
\pgfpathcurveto{\pgfqpoint{5.491394in}{1.627711in}}{\pgfqpoint{5.498236in}{1.624877in}}{\pgfqpoint{5.505369in}{1.624877in}}%
\pgfpathclose%
\pgfusepath{stroke,fill}%
\end{pgfscope}%
\begin{pgfscope}%
\pgfpathrectangle{\pgfqpoint{4.985294in}{0.500000in}}{\pgfqpoint{1.764706in}{1.700000in}}%
\pgfusepath{clip}%
\pgfsetbuttcap%
\pgfsetroundjoin%
\definecolor{currentfill}{rgb}{0.971694,0.833208,0.737161}%
\pgfsetfillcolor{currentfill}%
\pgfsetlinewidth{0.311001pt}%
\definecolor{currentstroke}{rgb}{1.000000,1.000000,1.000000}%
\pgfsetstrokecolor{currentstroke}%
\pgfsetdash{}{0pt}%
\pgfpathmoveto{\pgfqpoint{6.358182in}{1.450964in}}%
\pgfpathcurveto{\pgfqpoint{6.365315in}{1.450964in}}{\pgfqpoint{6.372157in}{1.453797in}}{\pgfqpoint{6.377200in}{1.458841in}}%
\pgfpathcurveto{\pgfqpoint{6.382244in}{1.463885in}}{\pgfqpoint{6.385078in}{1.470726in}}{\pgfqpoint{6.385078in}{1.477859in}}%
\pgfpathcurveto{\pgfqpoint{6.385078in}{1.484992in}}{\pgfqpoint{6.382244in}{1.491834in}}{\pgfqpoint{6.377200in}{1.496877in}}%
\pgfpathcurveto{\pgfqpoint{6.372157in}{1.501921in}}{\pgfqpoint{6.365315in}{1.504755in}}{\pgfqpoint{6.358182in}{1.504755in}}%
\pgfpathcurveto{\pgfqpoint{6.351049in}{1.504755in}}{\pgfqpoint{6.344208in}{1.501921in}}{\pgfqpoint{6.339164in}{1.496877in}}%
\pgfpathcurveto{\pgfqpoint{6.334120in}{1.491834in}}{\pgfqpoint{6.331287in}{1.484992in}}{\pgfqpoint{6.331287in}{1.477859in}}%
\pgfpathcurveto{\pgfqpoint{6.331287in}{1.470726in}}{\pgfqpoint{6.334120in}{1.463885in}}{\pgfqpoint{6.339164in}{1.458841in}}%
\pgfpathcurveto{\pgfqpoint{6.344208in}{1.453797in}}{\pgfqpoint{6.351049in}{1.450964in}}{\pgfqpoint{6.358182in}{1.450964in}}%
\pgfpathclose%
\pgfusepath{stroke,fill}%
\end{pgfscope}%
\begin{pgfscope}%
\pgfpathrectangle{\pgfqpoint{4.985294in}{0.500000in}}{\pgfqpoint{1.764706in}{1.700000in}}%
\pgfusepath{clip}%
\pgfsetbuttcap%
\pgfsetroundjoin%
\definecolor{currentfill}{rgb}{0.976961,0.885681,0.814303}%
\pgfsetfillcolor{currentfill}%
\pgfsetlinewidth{0.311001pt}%
\definecolor{currentstroke}{rgb}{1.000000,1.000000,1.000000}%
\pgfsetstrokecolor{currentstroke}%
\pgfsetdash{}{0pt}%
\pgfpathmoveto{\pgfqpoint{5.459986in}{1.188632in}}%
\pgfpathcurveto{\pgfqpoint{5.467119in}{1.188632in}}{\pgfqpoint{5.473961in}{1.191466in}}{\pgfqpoint{5.479005in}{1.196510in}}%
\pgfpathcurveto{\pgfqpoint{5.484048in}{1.201553in}}{\pgfqpoint{5.486882in}{1.208395in}}{\pgfqpoint{5.486882in}{1.215528in}}%
\pgfpathcurveto{\pgfqpoint{5.486882in}{1.222661in}}{\pgfqpoint{5.484048in}{1.229502in}}{\pgfqpoint{5.479005in}{1.234546in}}%
\pgfpathcurveto{\pgfqpoint{5.473961in}{1.239590in}}{\pgfqpoint{5.467119in}{1.242423in}}{\pgfqpoint{5.459986in}{1.242423in}}%
\pgfpathcurveto{\pgfqpoint{5.452854in}{1.242423in}}{\pgfqpoint{5.446012in}{1.239590in}}{\pgfqpoint{5.440968in}{1.234546in}}%
\pgfpathcurveto{\pgfqpoint{5.435925in}{1.229502in}}{\pgfqpoint{5.433091in}{1.222661in}}{\pgfqpoint{5.433091in}{1.215528in}}%
\pgfpathcurveto{\pgfqpoint{5.433091in}{1.208395in}}{\pgfqpoint{5.435925in}{1.201553in}}{\pgfqpoint{5.440968in}{1.196510in}}%
\pgfpathcurveto{\pgfqpoint{5.446012in}{1.191466in}}{\pgfqpoint{5.452854in}{1.188632in}}{\pgfqpoint{5.459986in}{1.188632in}}%
\pgfpathclose%
\pgfusepath{stroke,fill}%
\end{pgfscope}%
\begin{pgfscope}%
\pgfpathrectangle{\pgfqpoint{4.985294in}{0.500000in}}{\pgfqpoint{1.764706in}{1.700000in}}%
\pgfusepath{clip}%
\pgfsetbuttcap%
\pgfsetroundjoin%
\definecolor{currentfill}{rgb}{0.432143,0.121800,0.339663}%
\pgfsetfillcolor{currentfill}%
\pgfsetlinewidth{0.311001pt}%
\definecolor{currentstroke}{rgb}{1.000000,1.000000,1.000000}%
\pgfsetstrokecolor{currentstroke}%
\pgfsetdash{}{0pt}%
\pgfpathmoveto{\pgfqpoint{5.667445in}{1.287658in}}%
\pgfpathcurveto{\pgfqpoint{5.674578in}{1.287658in}}{\pgfqpoint{5.681419in}{1.290492in}}{\pgfqpoint{5.686463in}{1.295535in}}%
\pgfpathcurveto{\pgfqpoint{5.691507in}{1.300579in}}{\pgfqpoint{5.694341in}{1.307421in}}{\pgfqpoint{5.694341in}{1.314554in}}%
\pgfpathcurveto{\pgfqpoint{5.694341in}{1.321686in}}{\pgfqpoint{5.691507in}{1.328528in}}{\pgfqpoint{5.686463in}{1.333572in}}%
\pgfpathcurveto{\pgfqpoint{5.681419in}{1.338615in}}{\pgfqpoint{5.674578in}{1.341449in}}{\pgfqpoint{5.667445in}{1.341449in}}%
\pgfpathcurveto{\pgfqpoint{5.660312in}{1.341449in}}{\pgfqpoint{5.653470in}{1.338615in}}{\pgfqpoint{5.648427in}{1.333572in}}%
\pgfpathcurveto{\pgfqpoint{5.643383in}{1.328528in}}{\pgfqpoint{5.640549in}{1.321686in}}{\pgfqpoint{5.640549in}{1.314554in}}%
\pgfpathcurveto{\pgfqpoint{5.640549in}{1.307421in}}{\pgfqpoint{5.643383in}{1.300579in}}{\pgfqpoint{5.648427in}{1.295535in}}%
\pgfpathcurveto{\pgfqpoint{5.653470in}{1.290492in}}{\pgfqpoint{5.660312in}{1.287658in}}{\pgfqpoint{5.667445in}{1.287658in}}%
\pgfpathclose%
\pgfusepath{stroke,fill}%
\end{pgfscope}%
\begin{pgfscope}%
\pgfpathrectangle{\pgfqpoint{4.985294in}{0.500000in}}{\pgfqpoint{1.764706in}{1.700000in}}%
\pgfusepath{clip}%
\pgfsetbuttcap%
\pgfsetroundjoin%
\definecolor{currentfill}{rgb}{0.979891,0.908948,0.848279}%
\pgfsetfillcolor{currentfill}%
\pgfsetlinewidth{0.311001pt}%
\definecolor{currentstroke}{rgb}{1.000000,1.000000,1.000000}%
\pgfsetstrokecolor{currentstroke}%
\pgfsetdash{}{0pt}%
\pgfpathmoveto{\pgfqpoint{5.410433in}{1.256706in}}%
\pgfpathcurveto{\pgfqpoint{5.417566in}{1.256706in}}{\pgfqpoint{5.424407in}{1.259540in}}{\pgfqpoint{5.429451in}{1.264583in}}%
\pgfpathcurveto{\pgfqpoint{5.434495in}{1.269627in}}{\pgfqpoint{5.437328in}{1.276469in}}{\pgfqpoint{5.437328in}{1.283602in}}%
\pgfpathcurveto{\pgfqpoint{5.437328in}{1.290734in}}{\pgfqpoint{5.434495in}{1.297576in}}{\pgfqpoint{5.429451in}{1.302620in}}%
\pgfpathcurveto{\pgfqpoint{5.424407in}{1.307663in}}{\pgfqpoint{5.417566in}{1.310497in}}{\pgfqpoint{5.410433in}{1.310497in}}%
\pgfpathcurveto{\pgfqpoint{5.403300in}{1.310497in}}{\pgfqpoint{5.396458in}{1.307663in}}{\pgfqpoint{5.391415in}{1.302620in}}%
\pgfpathcurveto{\pgfqpoint{5.386371in}{1.297576in}}{\pgfqpoint{5.383537in}{1.290734in}}{\pgfqpoint{5.383537in}{1.283602in}}%
\pgfpathcurveto{\pgfqpoint{5.383537in}{1.276469in}}{\pgfqpoint{5.386371in}{1.269627in}}{\pgfqpoint{5.391415in}{1.264583in}}%
\pgfpathcurveto{\pgfqpoint{5.396458in}{1.259540in}}{\pgfqpoint{5.403300in}{1.256706in}}{\pgfqpoint{5.410433in}{1.256706in}}%
\pgfpathclose%
\pgfusepath{stroke,fill}%
\end{pgfscope}%
\begin{pgfscope}%
\pgfpathrectangle{\pgfqpoint{4.985294in}{0.500000in}}{\pgfqpoint{1.764706in}{1.700000in}}%
\pgfusepath{clip}%
\pgfsetbuttcap%
\pgfsetroundjoin%
\definecolor{currentfill}{rgb}{0.978376,0.897317,0.831308}%
\pgfsetfillcolor{currentfill}%
\pgfsetlinewidth{0.311001pt}%
\definecolor{currentstroke}{rgb}{1.000000,1.000000,1.000000}%
\pgfsetstrokecolor{currentstroke}%
\pgfsetdash{}{0pt}%
\pgfpathmoveto{\pgfqpoint{5.448028in}{1.431188in}}%
\pgfpathcurveto{\pgfqpoint{5.455161in}{1.431188in}}{\pgfqpoint{5.462003in}{1.434022in}}{\pgfqpoint{5.467047in}{1.439066in}}%
\pgfpathcurveto{\pgfqpoint{5.472090in}{1.444109in}}{\pgfqpoint{5.474924in}{1.450951in}}{\pgfqpoint{5.474924in}{1.458084in}}%
\pgfpathcurveto{\pgfqpoint{5.474924in}{1.465217in}}{\pgfqpoint{5.472090in}{1.472058in}}{\pgfqpoint{5.467047in}{1.477102in}}%
\pgfpathcurveto{\pgfqpoint{5.462003in}{1.482146in}}{\pgfqpoint{5.455161in}{1.484980in}}{\pgfqpoint{5.448028in}{1.484980in}}%
\pgfpathcurveto{\pgfqpoint{5.440896in}{1.484980in}}{\pgfqpoint{5.434054in}{1.482146in}}{\pgfqpoint{5.429010in}{1.477102in}}%
\pgfpathcurveto{\pgfqpoint{5.423967in}{1.472058in}}{\pgfqpoint{5.421133in}{1.465217in}}{\pgfqpoint{5.421133in}{1.458084in}}%
\pgfpathcurveto{\pgfqpoint{5.421133in}{1.450951in}}{\pgfqpoint{5.423967in}{1.444109in}}{\pgfqpoint{5.429010in}{1.439066in}}%
\pgfpathcurveto{\pgfqpoint{5.434054in}{1.434022in}}{\pgfqpoint{5.440896in}{1.431188in}}{\pgfqpoint{5.448028in}{1.431188in}}%
\pgfpathclose%
\pgfusepath{stroke,fill}%
\end{pgfscope}%
\begin{pgfscope}%
\pgfpathrectangle{\pgfqpoint{4.985294in}{0.500000in}}{\pgfqpoint{1.764706in}{1.700000in}}%
\pgfusepath{clip}%
\pgfsetbuttcap%
\pgfsetroundjoin%
\definecolor{currentfill}{rgb}{0.966328,0.750560,0.616961}%
\pgfsetfillcolor{currentfill}%
\pgfsetlinewidth{0.311001pt}%
\definecolor{currentstroke}{rgb}{1.000000,1.000000,1.000000}%
\pgfsetstrokecolor{currentstroke}%
\pgfsetdash{}{0pt}%
\pgfpathmoveto{\pgfqpoint{5.561674in}{1.068249in}}%
\pgfpathcurveto{\pgfqpoint{5.568807in}{1.068249in}}{\pgfqpoint{5.575648in}{1.071083in}}{\pgfqpoint{5.580692in}{1.076127in}}%
\pgfpathcurveto{\pgfqpoint{5.585736in}{1.081170in}}{\pgfqpoint{5.588569in}{1.088012in}}{\pgfqpoint{5.588569in}{1.095145in}}%
\pgfpathcurveto{\pgfqpoint{5.588569in}{1.102277in}}{\pgfqpoint{5.585736in}{1.109119in}}{\pgfqpoint{5.580692in}{1.114163in}}%
\pgfpathcurveto{\pgfqpoint{5.575648in}{1.119206in}}{\pgfqpoint{5.568807in}{1.122040in}}{\pgfqpoint{5.561674in}{1.122040in}}%
\pgfpathcurveto{\pgfqpoint{5.554541in}{1.122040in}}{\pgfqpoint{5.547699in}{1.119206in}}{\pgfqpoint{5.542656in}{1.114163in}}%
\pgfpathcurveto{\pgfqpoint{5.537612in}{1.109119in}}{\pgfqpoint{5.534778in}{1.102277in}}{\pgfqpoint{5.534778in}{1.095145in}}%
\pgfpathcurveto{\pgfqpoint{5.534778in}{1.088012in}}{\pgfqpoint{5.537612in}{1.081170in}}{\pgfqpoint{5.542656in}{1.076127in}}%
\pgfpathcurveto{\pgfqpoint{5.547699in}{1.071083in}}{\pgfqpoint{5.554541in}{1.068249in}}{\pgfqpoint{5.561674in}{1.068249in}}%
\pgfpathclose%
\pgfusepath{stroke,fill}%
\end{pgfscope}%
\begin{pgfscope}%
\pgfpathrectangle{\pgfqpoint{4.985294in}{0.500000in}}{\pgfqpoint{1.764706in}{1.700000in}}%
\pgfusepath{clip}%
\pgfsetbuttcap%
\pgfsetroundjoin%
\definecolor{currentfill}{rgb}{0.979891,0.908948,0.848279}%
\pgfsetfillcolor{currentfill}%
\pgfsetlinewidth{0.311001pt}%
\definecolor{currentstroke}{rgb}{1.000000,1.000000,1.000000}%
\pgfsetstrokecolor{currentstroke}%
\pgfsetdash{}{0pt}%
\pgfpathmoveto{\pgfqpoint{6.284072in}{1.497946in}}%
\pgfpathcurveto{\pgfqpoint{6.291205in}{1.497946in}}{\pgfqpoint{6.298046in}{1.500780in}}{\pgfqpoint{6.303090in}{1.505824in}}%
\pgfpathcurveto{\pgfqpoint{6.308134in}{1.510867in}}{\pgfqpoint{6.310968in}{1.517709in}}{\pgfqpoint{6.310968in}{1.524842in}}%
\pgfpathcurveto{\pgfqpoint{6.310968in}{1.531975in}}{\pgfqpoint{6.308134in}{1.538816in}}{\pgfqpoint{6.303090in}{1.543860in}}%
\pgfpathcurveto{\pgfqpoint{6.298046in}{1.548904in}}{\pgfqpoint{6.291205in}{1.551738in}}{\pgfqpoint{6.284072in}{1.551738in}}%
\pgfpathcurveto{\pgfqpoint{6.276939in}{1.551738in}}{\pgfqpoint{6.270097in}{1.548904in}}{\pgfqpoint{6.265054in}{1.543860in}}%
\pgfpathcurveto{\pgfqpoint{6.260010in}{1.538816in}}{\pgfqpoint{6.257176in}{1.531975in}}{\pgfqpoint{6.257176in}{1.524842in}}%
\pgfpathcurveto{\pgfqpoint{6.257176in}{1.517709in}}{\pgfqpoint{6.260010in}{1.510867in}}{\pgfqpoint{6.265054in}{1.505824in}}%
\pgfpathcurveto{\pgfqpoint{6.270097in}{1.500780in}}{\pgfqpoint{6.276939in}{1.497946in}}{\pgfqpoint{6.284072in}{1.497946in}}%
\pgfpathclose%
\pgfusepath{stroke,fill}%
\end{pgfscope}%
\begin{pgfscope}%
\pgfpathrectangle{\pgfqpoint{4.985294in}{0.500000in}}{\pgfqpoint{1.764706in}{1.700000in}}%
\pgfusepath{clip}%
\pgfsetbuttcap%
\pgfsetroundjoin%
\definecolor{currentfill}{rgb}{0.973271,0.850724,0.762998}%
\pgfsetfillcolor{currentfill}%
\pgfsetlinewidth{0.311001pt}%
\definecolor{currentstroke}{rgb}{1.000000,1.000000,1.000000}%
\pgfsetstrokecolor{currentstroke}%
\pgfsetdash{}{0pt}%
\pgfpathmoveto{\pgfqpoint{6.320220in}{1.110666in}}%
\pgfpathcurveto{\pgfqpoint{6.327353in}{1.110666in}}{\pgfqpoint{6.334195in}{1.113500in}}{\pgfqpoint{6.339238in}{1.118544in}}%
\pgfpathcurveto{\pgfqpoint{6.344282in}{1.123588in}}{\pgfqpoint{6.347116in}{1.130429in}}{\pgfqpoint{6.347116in}{1.137562in}}%
\pgfpathcurveto{\pgfqpoint{6.347116in}{1.144695in}}{\pgfqpoint{6.344282in}{1.151536in}}{\pgfqpoint{6.339238in}{1.156580in}}%
\pgfpathcurveto{\pgfqpoint{6.334195in}{1.161624in}}{\pgfqpoint{6.327353in}{1.164458in}}{\pgfqpoint{6.320220in}{1.164458in}}%
\pgfpathcurveto{\pgfqpoint{6.313087in}{1.164458in}}{\pgfqpoint{6.306246in}{1.161624in}}{\pgfqpoint{6.301202in}{1.156580in}}%
\pgfpathcurveto{\pgfqpoint{6.296158in}{1.151536in}}{\pgfqpoint{6.293324in}{1.144695in}}{\pgfqpoint{6.293324in}{1.137562in}}%
\pgfpathcurveto{\pgfqpoint{6.293324in}{1.130429in}}{\pgfqpoint{6.296158in}{1.123588in}}{\pgfqpoint{6.301202in}{1.118544in}}%
\pgfpathcurveto{\pgfqpoint{6.306246in}{1.113500in}}{\pgfqpoint{6.313087in}{1.110666in}}{\pgfqpoint{6.320220in}{1.110666in}}%
\pgfpathclose%
\pgfusepath{stroke,fill}%
\end{pgfscope}%
\begin{pgfscope}%
\pgfpathrectangle{\pgfqpoint{4.985294in}{0.500000in}}{\pgfqpoint{1.764706in}{1.700000in}}%
\pgfusepath{clip}%
\pgfsetbuttcap%
\pgfsetroundjoin%
\definecolor{currentfill}{rgb}{0.597702,0.106938,0.358380}%
\pgfsetfillcolor{currentfill}%
\pgfsetlinewidth{0.311001pt}%
\definecolor{currentstroke}{rgb}{1.000000,1.000000,1.000000}%
\pgfsetstrokecolor{currentstroke}%
\pgfsetdash{}{0pt}%
\pgfpathmoveto{\pgfqpoint{5.990102in}{1.050100in}}%
\pgfpathcurveto{\pgfqpoint{5.997235in}{1.050100in}}{\pgfqpoint{6.004077in}{1.052934in}}{\pgfqpoint{6.009120in}{1.057978in}}%
\pgfpathcurveto{\pgfqpoint{6.014164in}{1.063022in}}{\pgfqpoint{6.016998in}{1.069863in}}{\pgfqpoint{6.016998in}{1.076996in}}%
\pgfpathcurveto{\pgfqpoint{6.016998in}{1.084129in}}{\pgfqpoint{6.014164in}{1.090971in}}{\pgfqpoint{6.009120in}{1.096014in}}%
\pgfpathcurveto{\pgfqpoint{6.004077in}{1.101058in}}{\pgfqpoint{5.997235in}{1.103892in}}{\pgfqpoint{5.990102in}{1.103892in}}%
\pgfpathcurveto{\pgfqpoint{5.982969in}{1.103892in}}{\pgfqpoint{5.976128in}{1.101058in}}{\pgfqpoint{5.971084in}{1.096014in}}%
\pgfpathcurveto{\pgfqpoint{5.966040in}{1.090971in}}{\pgfqpoint{5.963206in}{1.084129in}}{\pgfqpoint{5.963206in}{1.076996in}}%
\pgfpathcurveto{\pgfqpoint{5.963206in}{1.069863in}}{\pgfqpoint{5.966040in}{1.063022in}}{\pgfqpoint{5.971084in}{1.057978in}}%
\pgfpathcurveto{\pgfqpoint{5.976128in}{1.052934in}}{\pgfqpoint{5.982969in}{1.050100in}}{\pgfqpoint{5.990102in}{1.050100in}}%
\pgfpathclose%
\pgfusepath{stroke,fill}%
\end{pgfscope}%
\begin{pgfscope}%
\pgfpathrectangle{\pgfqpoint{4.985294in}{0.500000in}}{\pgfqpoint{1.764706in}{1.700000in}}%
\pgfusepath{clip}%
\pgfsetbuttcap%
\pgfsetroundjoin%
\definecolor{currentfill}{rgb}{0.976287,0.879862,0.805788}%
\pgfsetfillcolor{currentfill}%
\pgfsetlinewidth{0.311001pt}%
\definecolor{currentstroke}{rgb}{1.000000,1.000000,1.000000}%
\pgfsetstrokecolor{currentstroke}%
\pgfsetdash{}{0pt}%
\pgfpathmoveto{\pgfqpoint{6.308891in}{1.131023in}}%
\pgfpathcurveto{\pgfqpoint{6.316023in}{1.131023in}}{\pgfqpoint{6.322865in}{1.133857in}}{\pgfqpoint{6.327909in}{1.138901in}}%
\pgfpathcurveto{\pgfqpoint{6.332952in}{1.143944in}}{\pgfqpoint{6.335786in}{1.150786in}}{\pgfqpoint{6.335786in}{1.157919in}}%
\pgfpathcurveto{\pgfqpoint{6.335786in}{1.165052in}}{\pgfqpoint{6.332952in}{1.171893in}}{\pgfqpoint{6.327909in}{1.176937in}}%
\pgfpathcurveto{\pgfqpoint{6.322865in}{1.181981in}}{\pgfqpoint{6.316023in}{1.184815in}}{\pgfqpoint{6.308891in}{1.184815in}}%
\pgfpathcurveto{\pgfqpoint{6.301758in}{1.184815in}}{\pgfqpoint{6.294916in}{1.181981in}}{\pgfqpoint{6.289872in}{1.176937in}}%
\pgfpathcurveto{\pgfqpoint{6.284829in}{1.171893in}}{\pgfqpoint{6.281995in}{1.165052in}}{\pgfqpoint{6.281995in}{1.157919in}}%
\pgfpathcurveto{\pgfqpoint{6.281995in}{1.150786in}}{\pgfqpoint{6.284829in}{1.143944in}}{\pgfqpoint{6.289872in}{1.138901in}}%
\pgfpathcurveto{\pgfqpoint{6.294916in}{1.133857in}}{\pgfqpoint{6.301758in}{1.131023in}}{\pgfqpoint{6.308891in}{1.131023in}}%
\pgfpathclose%
\pgfusepath{stroke,fill}%
\end{pgfscope}%
\begin{pgfscope}%
\pgfpathrectangle{\pgfqpoint{4.985294in}{0.500000in}}{\pgfqpoint{1.764706in}{1.700000in}}%
\pgfusepath{clip}%
\pgfsetbuttcap%
\pgfsetroundjoin%
\definecolor{currentfill}{rgb}{0.950017,0.427714,0.292447}%
\pgfsetfillcolor{currentfill}%
\pgfsetlinewidth{0.311001pt}%
\definecolor{currentstroke}{rgb}{1.000000,1.000000,1.000000}%
\pgfsetstrokecolor{currentstroke}%
\pgfsetdash{}{0pt}%
\pgfpathmoveto{\pgfqpoint{5.645360in}{0.843607in}}%
\pgfpathcurveto{\pgfqpoint{5.652493in}{0.843607in}}{\pgfqpoint{5.659334in}{0.846441in}}{\pgfqpoint{5.664378in}{0.851485in}}%
\pgfpathcurveto{\pgfqpoint{5.669422in}{0.856528in}}{\pgfqpoint{5.672255in}{0.863370in}}{\pgfqpoint{5.672255in}{0.870503in}}%
\pgfpathcurveto{\pgfqpoint{5.672255in}{0.877636in}}{\pgfqpoint{5.669422in}{0.884477in}}{\pgfqpoint{5.664378in}{0.889521in}}%
\pgfpathcurveto{\pgfqpoint{5.659334in}{0.894565in}}{\pgfqpoint{5.652493in}{0.897399in}}{\pgfqpoint{5.645360in}{0.897399in}}%
\pgfpathcurveto{\pgfqpoint{5.638227in}{0.897399in}}{\pgfqpoint{5.631385in}{0.894565in}}{\pgfqpoint{5.626342in}{0.889521in}}%
\pgfpathcurveto{\pgfqpoint{5.621298in}{0.884477in}}{\pgfqpoint{5.618464in}{0.877636in}}{\pgfqpoint{5.618464in}{0.870503in}}%
\pgfpathcurveto{\pgfqpoint{5.618464in}{0.863370in}}{\pgfqpoint{5.621298in}{0.856528in}}{\pgfqpoint{5.626342in}{0.851485in}}%
\pgfpathcurveto{\pgfqpoint{5.631385in}{0.846441in}}{\pgfqpoint{5.638227in}{0.843607in}}{\pgfqpoint{5.645360in}{0.843607in}}%
\pgfpathclose%
\pgfusepath{stroke,fill}%
\end{pgfscope}%
\begin{pgfscope}%
\pgfpathrectangle{\pgfqpoint{4.985294in}{0.500000in}}{\pgfqpoint{1.764706in}{1.700000in}}%
\pgfusepath{clip}%
\pgfsetbuttcap%
\pgfsetroundjoin%
\definecolor{currentfill}{rgb}{0.704578,0.088213,0.344730}%
\pgfsetfillcolor{currentfill}%
\pgfsetlinewidth{0.311001pt}%
\definecolor{currentstroke}{rgb}{1.000000,1.000000,1.000000}%
\pgfsetstrokecolor{currentstroke}%
\pgfsetdash{}{0pt}%
\pgfpathmoveto{\pgfqpoint{6.339184in}{1.793387in}}%
\pgfpathcurveto{\pgfqpoint{6.346317in}{1.793387in}}{\pgfqpoint{6.353158in}{1.796221in}}{\pgfqpoint{6.358202in}{1.801265in}}%
\pgfpathcurveto{\pgfqpoint{6.363246in}{1.806308in}}{\pgfqpoint{6.366080in}{1.813150in}}{\pgfqpoint{6.366080in}{1.820283in}}%
\pgfpathcurveto{\pgfqpoint{6.366080in}{1.827416in}}{\pgfqpoint{6.363246in}{1.834257in}}{\pgfqpoint{6.358202in}{1.839301in}}%
\pgfpathcurveto{\pgfqpoint{6.353158in}{1.844345in}}{\pgfqpoint{6.346317in}{1.847179in}}{\pgfqpoint{6.339184in}{1.847179in}}%
\pgfpathcurveto{\pgfqpoint{6.332051in}{1.847179in}}{\pgfqpoint{6.325209in}{1.844345in}}{\pgfqpoint{6.320166in}{1.839301in}}%
\pgfpathcurveto{\pgfqpoint{6.315122in}{1.834257in}}{\pgfqpoint{6.312288in}{1.827416in}}{\pgfqpoint{6.312288in}{1.820283in}}%
\pgfpathcurveto{\pgfqpoint{6.312288in}{1.813150in}}{\pgfqpoint{6.315122in}{1.806308in}}{\pgfqpoint{6.320166in}{1.801265in}}%
\pgfpathcurveto{\pgfqpoint{6.325209in}{1.796221in}}{\pgfqpoint{6.332051in}{1.793387in}}{\pgfqpoint{6.339184in}{1.793387in}}%
\pgfpathclose%
\pgfusepath{stroke,fill}%
\end{pgfscope}%
\begin{pgfscope}%
\pgfpathrectangle{\pgfqpoint{4.985294in}{0.500000in}}{\pgfqpoint{1.764706in}{1.700000in}}%
\pgfusepath{clip}%
\pgfsetbuttcap%
\pgfsetroundjoin%
\definecolor{currentfill}{rgb}{0.965753,0.732351,0.592427}%
\pgfsetfillcolor{currentfill}%
\pgfsetlinewidth{0.311001pt}%
\definecolor{currentstroke}{rgb}{1.000000,1.000000,1.000000}%
\pgfsetstrokecolor{currentstroke}%
\pgfsetdash{}{0pt}%
\pgfpathmoveto{\pgfqpoint{6.128859in}{1.649283in}}%
\pgfpathcurveto{\pgfqpoint{6.135992in}{1.649283in}}{\pgfqpoint{6.142833in}{1.652117in}}{\pgfqpoint{6.147877in}{1.657161in}}%
\pgfpathcurveto{\pgfqpoint{6.152921in}{1.662204in}}{\pgfqpoint{6.155754in}{1.669046in}}{\pgfqpoint{6.155754in}{1.676179in}}%
\pgfpathcurveto{\pgfqpoint{6.155754in}{1.683312in}}{\pgfqpoint{6.152921in}{1.690153in}}{\pgfqpoint{6.147877in}{1.695197in}}%
\pgfpathcurveto{\pgfqpoint{6.142833in}{1.700240in}}{\pgfqpoint{6.135992in}{1.703074in}}{\pgfqpoint{6.128859in}{1.703074in}}%
\pgfpathcurveto{\pgfqpoint{6.121726in}{1.703074in}}{\pgfqpoint{6.114884in}{1.700240in}}{\pgfqpoint{6.109841in}{1.695197in}}%
\pgfpathcurveto{\pgfqpoint{6.104797in}{1.690153in}}{\pgfqpoint{6.101963in}{1.683312in}}{\pgfqpoint{6.101963in}{1.676179in}}%
\pgfpathcurveto{\pgfqpoint{6.101963in}{1.669046in}}{\pgfqpoint{6.104797in}{1.662204in}}{\pgfqpoint{6.109841in}{1.657161in}}%
\pgfpathcurveto{\pgfqpoint{6.114884in}{1.652117in}}{\pgfqpoint{6.121726in}{1.649283in}}{\pgfqpoint{6.128859in}{1.649283in}}%
\pgfpathclose%
\pgfusepath{stroke,fill}%
\end{pgfscope}%
\begin{pgfscope}%
\pgfpathrectangle{\pgfqpoint{4.985294in}{0.500000in}}{\pgfqpoint{1.764706in}{1.700000in}}%
\pgfusepath{clip}%
\pgfsetbuttcap%
\pgfsetroundjoin%
\definecolor{currentfill}{rgb}{0.977657,0.891500,0.822809}%
\pgfsetfillcolor{currentfill}%
\pgfsetlinewidth{0.311001pt}%
\definecolor{currentstroke}{rgb}{1.000000,1.000000,1.000000}%
\pgfsetstrokecolor{currentstroke}%
\pgfsetdash{}{0pt}%
\pgfpathmoveto{\pgfqpoint{6.326814in}{1.181789in}}%
\pgfpathcurveto{\pgfqpoint{6.333947in}{1.181789in}}{\pgfqpoint{6.340789in}{1.184623in}}{\pgfqpoint{6.345832in}{1.189667in}}%
\pgfpathcurveto{\pgfqpoint{6.350876in}{1.194710in}}{\pgfqpoint{6.353710in}{1.201552in}}{\pgfqpoint{6.353710in}{1.208685in}}%
\pgfpathcurveto{\pgfqpoint{6.353710in}{1.215818in}}{\pgfqpoint{6.350876in}{1.222659in}}{\pgfqpoint{6.345832in}{1.227703in}}%
\pgfpathcurveto{\pgfqpoint{6.340789in}{1.232746in}}{\pgfqpoint{6.333947in}{1.235580in}}{\pgfqpoint{6.326814in}{1.235580in}}%
\pgfpathcurveto{\pgfqpoint{6.319681in}{1.235580in}}{\pgfqpoint{6.312840in}{1.232746in}}{\pgfqpoint{6.307796in}{1.227703in}}%
\pgfpathcurveto{\pgfqpoint{6.302752in}{1.222659in}}{\pgfqpoint{6.299918in}{1.215818in}}{\pgfqpoint{6.299918in}{1.208685in}}%
\pgfpathcurveto{\pgfqpoint{6.299918in}{1.201552in}}{\pgfqpoint{6.302752in}{1.194710in}}{\pgfqpoint{6.307796in}{1.189667in}}%
\pgfpathcurveto{\pgfqpoint{6.312840in}{1.184623in}}{\pgfqpoint{6.319681in}{1.181789in}}{\pgfqpoint{6.326814in}{1.181789in}}%
\pgfpathclose%
\pgfusepath{stroke,fill}%
\end{pgfscope}%
\begin{pgfscope}%
\pgfpathrectangle{\pgfqpoint{4.985294in}{0.500000in}}{\pgfqpoint{1.764706in}{1.700000in}}%
\pgfusepath{clip}%
\pgfsetbuttcap%
\pgfsetroundjoin%
\definecolor{currentfill}{rgb}{0.956817,0.498820,0.345554}%
\pgfsetfillcolor{currentfill}%
\pgfsetlinewidth{0.311001pt}%
\definecolor{currentstroke}{rgb}{1.000000,1.000000,1.000000}%
\pgfsetstrokecolor{currentstroke}%
\pgfsetdash{}{0pt}%
\pgfpathmoveto{\pgfqpoint{5.671496in}{1.692483in}}%
\pgfpathcurveto{\pgfqpoint{5.678629in}{1.692483in}}{\pgfqpoint{5.685470in}{1.695316in}}{\pgfqpoint{5.690514in}{1.700360in}}%
\pgfpathcurveto{\pgfqpoint{5.695558in}{1.705404in}}{\pgfqpoint{5.698392in}{1.712245in}}{\pgfqpoint{5.698392in}{1.719378in}}%
\pgfpathcurveto{\pgfqpoint{5.698392in}{1.726511in}}{\pgfqpoint{5.695558in}{1.733353in}}{\pgfqpoint{5.690514in}{1.738396in}}%
\pgfpathcurveto{\pgfqpoint{5.685470in}{1.743440in}}{\pgfqpoint{5.678629in}{1.746274in}}{\pgfqpoint{5.671496in}{1.746274in}}%
\pgfpathcurveto{\pgfqpoint{5.664363in}{1.746274in}}{\pgfqpoint{5.657521in}{1.743440in}}{\pgfqpoint{5.652478in}{1.738396in}}%
\pgfpathcurveto{\pgfqpoint{5.647434in}{1.733353in}}{\pgfqpoint{5.644600in}{1.726511in}}{\pgfqpoint{5.644600in}{1.719378in}}%
\pgfpathcurveto{\pgfqpoint{5.644600in}{1.712245in}}{\pgfqpoint{5.647434in}{1.705404in}}{\pgfqpoint{5.652478in}{1.700360in}}%
\pgfpathcurveto{\pgfqpoint{5.657521in}{1.695316in}}{\pgfqpoint{5.664363in}{1.692483in}}{\pgfqpoint{5.671496in}{1.692483in}}%
\pgfpathclose%
\pgfusepath{stroke,fill}%
\end{pgfscope}%
\begin{pgfscope}%
\pgfpathrectangle{\pgfqpoint{4.985294in}{0.500000in}}{\pgfqpoint{1.764706in}{1.700000in}}%
\pgfusepath{clip}%
\pgfsetbuttcap%
\pgfsetroundjoin%
\definecolor{currentfill}{rgb}{0.967398,0.774513,0.650573}%
\pgfsetfillcolor{currentfill}%
\pgfsetlinewidth{0.311001pt}%
\definecolor{currentstroke}{rgb}{1.000000,1.000000,1.000000}%
\pgfsetstrokecolor{currentstroke}%
\pgfsetdash{}{0pt}%
\pgfpathmoveto{\pgfqpoint{6.207278in}{1.470304in}}%
\pgfpathcurveto{\pgfqpoint{6.214411in}{1.470304in}}{\pgfqpoint{6.221253in}{1.473138in}}{\pgfqpoint{6.226296in}{1.478182in}}%
\pgfpathcurveto{\pgfqpoint{6.231340in}{1.483225in}}{\pgfqpoint{6.234174in}{1.490067in}}{\pgfqpoint{6.234174in}{1.497200in}}%
\pgfpathcurveto{\pgfqpoint{6.234174in}{1.504333in}}{\pgfqpoint{6.231340in}{1.511174in}}{\pgfqpoint{6.226296in}{1.516218in}}%
\pgfpathcurveto{\pgfqpoint{6.221253in}{1.521262in}}{\pgfqpoint{6.214411in}{1.524095in}}{\pgfqpoint{6.207278in}{1.524095in}}%
\pgfpathcurveto{\pgfqpoint{6.200145in}{1.524095in}}{\pgfqpoint{6.193304in}{1.521262in}}{\pgfqpoint{6.188260in}{1.516218in}}%
\pgfpathcurveto{\pgfqpoint{6.183216in}{1.511174in}}{\pgfqpoint{6.180382in}{1.504333in}}{\pgfqpoint{6.180382in}{1.497200in}}%
\pgfpathcurveto{\pgfqpoint{6.180382in}{1.490067in}}{\pgfqpoint{6.183216in}{1.483225in}}{\pgfqpoint{6.188260in}{1.478182in}}%
\pgfpathcurveto{\pgfqpoint{6.193304in}{1.473138in}}{\pgfqpoint{6.200145in}{1.470304in}}{\pgfqpoint{6.207278in}{1.470304in}}%
\pgfpathclose%
\pgfusepath{stroke,fill}%
\end{pgfscope}%
\begin{pgfscope}%
\pgfpathrectangle{\pgfqpoint{4.985294in}{0.500000in}}{\pgfqpoint{1.764706in}{1.700000in}}%
\pgfusepath{clip}%
\pgfsetbuttcap%
\pgfsetroundjoin%
\definecolor{currentfill}{rgb}{0.963884,0.644842,0.486120}%
\pgfsetfillcolor{currentfill}%
\pgfsetlinewidth{0.311001pt}%
\definecolor{currentstroke}{rgb}{1.000000,1.000000,1.000000}%
\pgfsetstrokecolor{currentstroke}%
\pgfsetdash{}{0pt}%
\pgfpathmoveto{\pgfqpoint{5.549498in}{1.427522in}}%
\pgfpathcurveto{\pgfqpoint{5.556631in}{1.427522in}}{\pgfqpoint{5.563472in}{1.430356in}}{\pgfqpoint{5.568516in}{1.435399in}}%
\pgfpathcurveto{\pgfqpoint{5.573559in}{1.440443in}}{\pgfqpoint{5.576393in}{1.447285in}}{\pgfqpoint{5.576393in}{1.454417in}}%
\pgfpathcurveto{\pgfqpoint{5.576393in}{1.461550in}}{\pgfqpoint{5.573559in}{1.468392in}}{\pgfqpoint{5.568516in}{1.473436in}}%
\pgfpathcurveto{\pgfqpoint{5.563472in}{1.478479in}}{\pgfqpoint{5.556631in}{1.481313in}}{\pgfqpoint{5.549498in}{1.481313in}}%
\pgfpathcurveto{\pgfqpoint{5.542365in}{1.481313in}}{\pgfqpoint{5.535523in}{1.478479in}}{\pgfqpoint{5.530480in}{1.473436in}}%
\pgfpathcurveto{\pgfqpoint{5.525436in}{1.468392in}}{\pgfqpoint{5.522602in}{1.461550in}}{\pgfqpoint{5.522602in}{1.454417in}}%
\pgfpathcurveto{\pgfqpoint{5.522602in}{1.447285in}}{\pgfqpoint{5.525436in}{1.440443in}}{\pgfqpoint{5.530480in}{1.435399in}}%
\pgfpathcurveto{\pgfqpoint{5.535523in}{1.430356in}}{\pgfqpoint{5.542365in}{1.427522in}}{\pgfqpoint{5.549498in}{1.427522in}}%
\pgfpathclose%
\pgfusepath{stroke,fill}%
\end{pgfscope}%
\begin{pgfscope}%
\pgfpathrectangle{\pgfqpoint{4.985294in}{0.500000in}}{\pgfqpoint{1.764706in}{1.700000in}}%
\pgfusepath{clip}%
\pgfsetbuttcap%
\pgfsetroundjoin%
\definecolor{currentfill}{rgb}{0.971202,0.827364,0.728520}%
\pgfsetfillcolor{currentfill}%
\pgfsetlinewidth{0.311001pt}%
\definecolor{currentstroke}{rgb}{1.000000,1.000000,1.000000}%
\pgfsetstrokecolor{currentstroke}%
\pgfsetdash{}{0pt}%
\pgfpathmoveto{\pgfqpoint{6.184724in}{1.660260in}}%
\pgfpathcurveto{\pgfqpoint{6.191857in}{1.660260in}}{\pgfqpoint{6.198699in}{1.663094in}}{\pgfqpoint{6.203742in}{1.668137in}}%
\pgfpathcurveto{\pgfqpoint{6.208786in}{1.673181in}}{\pgfqpoint{6.211620in}{1.680023in}}{\pgfqpoint{6.211620in}{1.687155in}}%
\pgfpathcurveto{\pgfqpoint{6.211620in}{1.694288in}}{\pgfqpoint{6.208786in}{1.701130in}}{\pgfqpoint{6.203742in}{1.706174in}}%
\pgfpathcurveto{\pgfqpoint{6.198699in}{1.711217in}}{\pgfqpoint{6.191857in}{1.714051in}}{\pgfqpoint{6.184724in}{1.714051in}}%
\pgfpathcurveto{\pgfqpoint{6.177592in}{1.714051in}}{\pgfqpoint{6.170750in}{1.711217in}}{\pgfqpoint{6.165706in}{1.706174in}}%
\pgfpathcurveto{\pgfqpoint{6.160663in}{1.701130in}}{\pgfqpoint{6.157829in}{1.694288in}}{\pgfqpoint{6.157829in}{1.687155in}}%
\pgfpathcurveto{\pgfqpoint{6.157829in}{1.680023in}}{\pgfqpoint{6.160663in}{1.673181in}}{\pgfqpoint{6.165706in}{1.668137in}}%
\pgfpathcurveto{\pgfqpoint{6.170750in}{1.663094in}}{\pgfqpoint{6.177592in}{1.660260in}}{\pgfqpoint{6.184724in}{1.660260in}}%
\pgfpathclose%
\pgfusepath{stroke,fill}%
\end{pgfscope}%
\begin{pgfscope}%
\pgfpathrectangle{\pgfqpoint{4.985294in}{0.500000in}}{\pgfqpoint{1.764706in}{1.700000in}}%
\pgfusepath{clip}%
\pgfsetbuttcap%
\pgfsetroundjoin%
\definecolor{currentfill}{rgb}{0.952404,0.449449,0.307210}%
\pgfsetfillcolor{currentfill}%
\pgfsetlinewidth{0.311001pt}%
\definecolor{currentstroke}{rgb}{1.000000,1.000000,1.000000}%
\pgfsetstrokecolor{currentstroke}%
\pgfsetdash{}{0pt}%
\pgfpathmoveto{\pgfqpoint{6.420531in}{1.533302in}}%
\pgfpathcurveto{\pgfqpoint{6.427664in}{1.533302in}}{\pgfqpoint{6.434505in}{1.536136in}}{\pgfqpoint{6.439549in}{1.541180in}}%
\pgfpathcurveto{\pgfqpoint{6.444593in}{1.546223in}}{\pgfqpoint{6.447427in}{1.553065in}}{\pgfqpoint{6.447427in}{1.560198in}}%
\pgfpathcurveto{\pgfqpoint{6.447427in}{1.567331in}}{\pgfqpoint{6.444593in}{1.574172in}}{\pgfqpoint{6.439549in}{1.579216in}}%
\pgfpathcurveto{\pgfqpoint{6.434505in}{1.584260in}}{\pgfqpoint{6.427664in}{1.587094in}}{\pgfqpoint{6.420531in}{1.587094in}}%
\pgfpathcurveto{\pgfqpoint{6.413398in}{1.587094in}}{\pgfqpoint{6.406556in}{1.584260in}}{\pgfqpoint{6.401513in}{1.579216in}}%
\pgfpathcurveto{\pgfqpoint{6.396469in}{1.574172in}}{\pgfqpoint{6.393635in}{1.567331in}}{\pgfqpoint{6.393635in}{1.560198in}}%
\pgfpathcurveto{\pgfqpoint{6.393635in}{1.553065in}}{\pgfqpoint{6.396469in}{1.546223in}}{\pgfqpoint{6.401513in}{1.541180in}}%
\pgfpathcurveto{\pgfqpoint{6.406556in}{1.536136in}}{\pgfqpoint{6.413398in}{1.533302in}}{\pgfqpoint{6.420531in}{1.533302in}}%
\pgfpathclose%
\pgfusepath{stroke,fill}%
\end{pgfscope}%
\begin{pgfscope}%
\pgfpathrectangle{\pgfqpoint{4.985294in}{0.500000in}}{\pgfqpoint{1.764706in}{1.700000in}}%
\pgfusepath{clip}%
\pgfsetbuttcap%
\pgfsetroundjoin%
\definecolor{currentfill}{rgb}{0.969803,0.809811,0.702523}%
\pgfsetfillcolor{currentfill}%
\pgfsetlinewidth{0.311001pt}%
\definecolor{currentstroke}{rgb}{1.000000,1.000000,1.000000}%
\pgfsetstrokecolor{currentstroke}%
\pgfsetdash{}{0pt}%
\pgfpathmoveto{\pgfqpoint{6.227923in}{1.255374in}}%
\pgfpathcurveto{\pgfqpoint{6.235056in}{1.255374in}}{\pgfqpoint{6.241897in}{1.258208in}}{\pgfqpoint{6.246941in}{1.263252in}}%
\pgfpathcurveto{\pgfqpoint{6.251985in}{1.268295in}}{\pgfqpoint{6.254818in}{1.275137in}}{\pgfqpoint{6.254818in}{1.282270in}}%
\pgfpathcurveto{\pgfqpoint{6.254818in}{1.289403in}}{\pgfqpoint{6.251985in}{1.296244in}}{\pgfqpoint{6.246941in}{1.301288in}}%
\pgfpathcurveto{\pgfqpoint{6.241897in}{1.306332in}}{\pgfqpoint{6.235056in}{1.309166in}}{\pgfqpoint{6.227923in}{1.309166in}}%
\pgfpathcurveto{\pgfqpoint{6.220790in}{1.309166in}}{\pgfqpoint{6.213948in}{1.306332in}}{\pgfqpoint{6.208905in}{1.301288in}}%
\pgfpathcurveto{\pgfqpoint{6.203861in}{1.296244in}}{\pgfqpoint{6.201027in}{1.289403in}}{\pgfqpoint{6.201027in}{1.282270in}}%
\pgfpathcurveto{\pgfqpoint{6.201027in}{1.275137in}}{\pgfqpoint{6.203861in}{1.268295in}}{\pgfqpoint{6.208905in}{1.263252in}}%
\pgfpathcurveto{\pgfqpoint{6.213948in}{1.258208in}}{\pgfqpoint{6.220790in}{1.255374in}}{\pgfqpoint{6.227923in}{1.255374in}}%
\pgfpathclose%
\pgfusepath{stroke,fill}%
\end{pgfscope}%
\begin{pgfscope}%
\pgfpathrectangle{\pgfqpoint{4.985294in}{0.500000in}}{\pgfqpoint{1.764706in}{1.700000in}}%
\pgfusepath{clip}%
\pgfsetbuttcap%
\pgfsetroundjoin%
\definecolor{currentfill}{rgb}{0.966328,0.750560,0.616961}%
\pgfsetfillcolor{currentfill}%
\pgfsetlinewidth{0.311001pt}%
\definecolor{currentstroke}{rgb}{1.000000,1.000000,1.000000}%
\pgfsetstrokecolor{currentstroke}%
\pgfsetdash{}{0pt}%
\pgfpathmoveto{\pgfqpoint{5.382696in}{1.076182in}}%
\pgfpathcurveto{\pgfqpoint{5.389829in}{1.076182in}}{\pgfqpoint{5.396670in}{1.079016in}}{\pgfqpoint{5.401714in}{1.084060in}}%
\pgfpathcurveto{\pgfqpoint{5.406758in}{1.089104in}}{\pgfqpoint{5.409592in}{1.095945in}}{\pgfqpoint{5.409592in}{1.103078in}}%
\pgfpathcurveto{\pgfqpoint{5.409592in}{1.110211in}}{\pgfqpoint{5.406758in}{1.117053in}}{\pgfqpoint{5.401714in}{1.122096in}}%
\pgfpathcurveto{\pgfqpoint{5.396670in}{1.127140in}}{\pgfqpoint{5.389829in}{1.129974in}}{\pgfqpoint{5.382696in}{1.129974in}}%
\pgfpathcurveto{\pgfqpoint{5.375563in}{1.129974in}}{\pgfqpoint{5.368721in}{1.127140in}}{\pgfqpoint{5.363678in}{1.122096in}}%
\pgfpathcurveto{\pgfqpoint{5.358634in}{1.117053in}}{\pgfqpoint{5.355800in}{1.110211in}}{\pgfqpoint{5.355800in}{1.103078in}}%
\pgfpathcurveto{\pgfqpoint{5.355800in}{1.095945in}}{\pgfqpoint{5.358634in}{1.089104in}}{\pgfqpoint{5.363678in}{1.084060in}}%
\pgfpathcurveto{\pgfqpoint{5.368721in}{1.079016in}}{\pgfqpoint{5.375563in}{1.076182in}}{\pgfqpoint{5.382696in}{1.076182in}}%
\pgfpathclose%
\pgfusepath{stroke,fill}%
\end{pgfscope}%
\begin{pgfscope}%
\pgfpathrectangle{\pgfqpoint{4.985294in}{0.500000in}}{\pgfqpoint{1.764706in}{1.700000in}}%
\pgfusepath{clip}%
\pgfsetbuttcap%
\pgfsetroundjoin%
\definecolor{currentfill}{rgb}{0.955103,0.477872,0.328626}%
\pgfsetfillcolor{currentfill}%
\pgfsetlinewidth{0.311001pt}%
\definecolor{currentstroke}{rgb}{1.000000,1.000000,1.000000}%
\pgfsetstrokecolor{currentstroke}%
\pgfsetdash{}{0pt}%
\pgfpathmoveto{\pgfqpoint{5.575641in}{1.424882in}}%
\pgfpathcurveto{\pgfqpoint{5.582774in}{1.424882in}}{\pgfqpoint{5.589615in}{1.427716in}}{\pgfqpoint{5.594659in}{1.432760in}}%
\pgfpathcurveto{\pgfqpoint{5.599703in}{1.437803in}}{\pgfqpoint{5.602537in}{1.444645in}}{\pgfqpoint{5.602537in}{1.451778in}}%
\pgfpathcurveto{\pgfqpoint{5.602537in}{1.458911in}}{\pgfqpoint{5.599703in}{1.465752in}}{\pgfqpoint{5.594659in}{1.470796in}}%
\pgfpathcurveto{\pgfqpoint{5.589615in}{1.475840in}}{\pgfqpoint{5.582774in}{1.478673in}}{\pgfqpoint{5.575641in}{1.478673in}}%
\pgfpathcurveto{\pgfqpoint{5.568508in}{1.478673in}}{\pgfqpoint{5.561666in}{1.475840in}}{\pgfqpoint{5.556623in}{1.470796in}}%
\pgfpathcurveto{\pgfqpoint{5.551579in}{1.465752in}}{\pgfqpoint{5.548745in}{1.458911in}}{\pgfqpoint{5.548745in}{1.451778in}}%
\pgfpathcurveto{\pgfqpoint{5.548745in}{1.444645in}}{\pgfqpoint{5.551579in}{1.437803in}}{\pgfqpoint{5.556623in}{1.432760in}}%
\pgfpathcurveto{\pgfqpoint{5.561666in}{1.427716in}}{\pgfqpoint{5.568508in}{1.424882in}}{\pgfqpoint{5.575641in}{1.424882in}}%
\pgfpathclose%
\pgfusepath{stroke,fill}%
\end{pgfscope}%
\begin{pgfscope}%
\pgfpathrectangle{\pgfqpoint{4.985294in}{0.500000in}}{\pgfqpoint{1.764706in}{1.700000in}}%
\pgfusepath{clip}%
\pgfsetbuttcap%
\pgfsetroundjoin%
\definecolor{currentfill}{rgb}{0.967735,0.780441,0.659127}%
\pgfsetfillcolor{currentfill}%
\pgfsetlinewidth{0.311001pt}%
\definecolor{currentstroke}{rgb}{1.000000,1.000000,1.000000}%
\pgfsetstrokecolor{currentstroke}%
\pgfsetdash{}{0pt}%
\pgfpathmoveto{\pgfqpoint{5.529088in}{1.124463in}}%
\pgfpathcurveto{\pgfqpoint{5.536220in}{1.124463in}}{\pgfqpoint{5.543062in}{1.127297in}}{\pgfqpoint{5.548106in}{1.132341in}}%
\pgfpathcurveto{\pgfqpoint{5.553149in}{1.137384in}}{\pgfqpoint{5.555983in}{1.144226in}}{\pgfqpoint{5.555983in}{1.151359in}}%
\pgfpathcurveto{\pgfqpoint{5.555983in}{1.158492in}}{\pgfqpoint{5.553149in}{1.165333in}}{\pgfqpoint{5.548106in}{1.170377in}}%
\pgfpathcurveto{\pgfqpoint{5.543062in}{1.175421in}}{\pgfqpoint{5.536220in}{1.178254in}}{\pgfqpoint{5.529088in}{1.178254in}}%
\pgfpathcurveto{\pgfqpoint{5.521955in}{1.178254in}}{\pgfqpoint{5.515113in}{1.175421in}}{\pgfqpoint{5.510069in}{1.170377in}}%
\pgfpathcurveto{\pgfqpoint{5.505026in}{1.165333in}}{\pgfqpoint{5.502192in}{1.158492in}}{\pgfqpoint{5.502192in}{1.151359in}}%
\pgfpathcurveto{\pgfqpoint{5.502192in}{1.144226in}}{\pgfqpoint{5.505026in}{1.137384in}}{\pgfqpoint{5.510069in}{1.132341in}}%
\pgfpathcurveto{\pgfqpoint{5.515113in}{1.127297in}}{\pgfqpoint{5.521955in}{1.124463in}}{\pgfqpoint{5.529088in}{1.124463in}}%
\pgfpathclose%
\pgfusepath{stroke,fill}%
\end{pgfscope}%
\begin{pgfscope}%
\pgfpathrectangle{\pgfqpoint{4.985294in}{0.500000in}}{\pgfqpoint{1.764706in}{1.700000in}}%
\pgfusepath{clip}%
\pgfsetbuttcap%
\pgfsetroundjoin%
\definecolor{currentfill}{rgb}{0.879259,0.192033,0.262681}%
\pgfsetfillcolor{currentfill}%
\pgfsetlinewidth{0.311001pt}%
\definecolor{currentstroke}{rgb}{1.000000,1.000000,1.000000}%
\pgfsetstrokecolor{currentstroke}%
\pgfsetdash{}{0pt}%
\pgfpathmoveto{\pgfqpoint{5.244888in}{1.310977in}}%
\pgfpathcurveto{\pgfqpoint{5.252021in}{1.310977in}}{\pgfqpoint{5.258863in}{1.313811in}}{\pgfqpoint{5.263906in}{1.318855in}}%
\pgfpathcurveto{\pgfqpoint{5.268950in}{1.323899in}}{\pgfqpoint{5.271784in}{1.330740in}}{\pgfqpoint{5.271784in}{1.337873in}}%
\pgfpathcurveto{\pgfqpoint{5.271784in}{1.345006in}}{\pgfqpoint{5.268950in}{1.351848in}}{\pgfqpoint{5.263906in}{1.356891in}}%
\pgfpathcurveto{\pgfqpoint{5.258863in}{1.361935in}}{\pgfqpoint{5.252021in}{1.364769in}}{\pgfqpoint{5.244888in}{1.364769in}}%
\pgfpathcurveto{\pgfqpoint{5.237755in}{1.364769in}}{\pgfqpoint{5.230914in}{1.361935in}}{\pgfqpoint{5.225870in}{1.356891in}}%
\pgfpathcurveto{\pgfqpoint{5.220826in}{1.351848in}}{\pgfqpoint{5.217992in}{1.345006in}}{\pgfqpoint{5.217992in}{1.337873in}}%
\pgfpathcurveto{\pgfqpoint{5.217992in}{1.330740in}}{\pgfqpoint{5.220826in}{1.323899in}}{\pgfqpoint{5.225870in}{1.318855in}}%
\pgfpathcurveto{\pgfqpoint{5.230914in}{1.313811in}}{\pgfqpoint{5.237755in}{1.310977in}}{\pgfqpoint{5.244888in}{1.310977in}}%
\pgfpathclose%
\pgfusepath{stroke,fill}%
\end{pgfscope}%
\begin{pgfscope}%
\pgfpathrectangle{\pgfqpoint{4.985294in}{0.500000in}}{\pgfqpoint{1.764706in}{1.700000in}}%
\pgfusepath{clip}%
\pgfsetbuttcap%
\pgfsetroundjoin%
\definecolor{currentfill}{rgb}{0.970718,0.821518,0.719872}%
\pgfsetfillcolor{currentfill}%
\pgfsetlinewidth{0.311001pt}%
\definecolor{currentstroke}{rgb}{1.000000,1.000000,1.000000}%
\pgfsetstrokecolor{currentstroke}%
\pgfsetdash{}{0pt}%
\pgfpathmoveto{\pgfqpoint{6.244725in}{1.338158in}}%
\pgfpathcurveto{\pgfqpoint{6.251858in}{1.338158in}}{\pgfqpoint{6.258700in}{1.340991in}}{\pgfqpoint{6.263743in}{1.346035in}}%
\pgfpathcurveto{\pgfqpoint{6.268787in}{1.351079in}}{\pgfqpoint{6.271621in}{1.357920in}}{\pgfqpoint{6.271621in}{1.365053in}}%
\pgfpathcurveto{\pgfqpoint{6.271621in}{1.372186in}}{\pgfqpoint{6.268787in}{1.379028in}}{\pgfqpoint{6.263743in}{1.384071in}}%
\pgfpathcurveto{\pgfqpoint{6.258700in}{1.389115in}}{\pgfqpoint{6.251858in}{1.391949in}}{\pgfqpoint{6.244725in}{1.391949in}}%
\pgfpathcurveto{\pgfqpoint{6.237592in}{1.391949in}}{\pgfqpoint{6.230751in}{1.389115in}}{\pgfqpoint{6.225707in}{1.384071in}}%
\pgfpathcurveto{\pgfqpoint{6.220663in}{1.379028in}}{\pgfqpoint{6.217829in}{1.372186in}}{\pgfqpoint{6.217829in}{1.365053in}}%
\pgfpathcurveto{\pgfqpoint{6.217829in}{1.357920in}}{\pgfqpoint{6.220663in}{1.351079in}}{\pgfqpoint{6.225707in}{1.346035in}}%
\pgfpathcurveto{\pgfqpoint{6.230751in}{1.340991in}}{\pgfqpoint{6.237592in}{1.338158in}}{\pgfqpoint{6.244725in}{1.338158in}}%
\pgfpathclose%
\pgfusepath{stroke,fill}%
\end{pgfscope}%
\begin{pgfscope}%
\pgfpathrectangle{\pgfqpoint{4.985294in}{0.500000in}}{\pgfqpoint{1.764706in}{1.700000in}}%
\pgfusepath{clip}%
\pgfsetbuttcap%
\pgfsetroundjoin%
\definecolor{currentfill}{rgb}{0.965592,0.726236,0.584384}%
\pgfsetfillcolor{currentfill}%
\pgfsetlinewidth{0.311001pt}%
\definecolor{currentstroke}{rgb}{1.000000,1.000000,1.000000}%
\pgfsetstrokecolor{currentstroke}%
\pgfsetdash{}{0pt}%
\pgfpathmoveto{\pgfqpoint{5.328724in}{1.285844in}}%
\pgfpathcurveto{\pgfqpoint{5.335856in}{1.285844in}}{\pgfqpoint{5.342698in}{1.288678in}}{\pgfqpoint{5.347742in}{1.293722in}}%
\pgfpathcurveto{\pgfqpoint{5.352785in}{1.298765in}}{\pgfqpoint{5.355619in}{1.305607in}}{\pgfqpoint{5.355619in}{1.312740in}}%
\pgfpathcurveto{\pgfqpoint{5.355619in}{1.319873in}}{\pgfqpoint{5.352785in}{1.326714in}}{\pgfqpoint{5.347742in}{1.331758in}}%
\pgfpathcurveto{\pgfqpoint{5.342698in}{1.336801in}}{\pgfqpoint{5.335856in}{1.339635in}}{\pgfqpoint{5.328724in}{1.339635in}}%
\pgfpathcurveto{\pgfqpoint{5.321591in}{1.339635in}}{\pgfqpoint{5.314749in}{1.336801in}}{\pgfqpoint{5.309705in}{1.331758in}}%
\pgfpathcurveto{\pgfqpoint{5.304662in}{1.326714in}}{\pgfqpoint{5.301828in}{1.319873in}}{\pgfqpoint{5.301828in}{1.312740in}}%
\pgfpathcurveto{\pgfqpoint{5.301828in}{1.305607in}}{\pgfqpoint{5.304662in}{1.298765in}}{\pgfqpoint{5.309705in}{1.293722in}}%
\pgfpathcurveto{\pgfqpoint{5.314749in}{1.288678in}}{\pgfqpoint{5.321591in}{1.285844in}}{\pgfqpoint{5.328724in}{1.285844in}}%
\pgfpathclose%
\pgfusepath{stroke,fill}%
\end{pgfscope}%
\begin{pgfscope}%
\pgfpathrectangle{\pgfqpoint{4.985294in}{0.500000in}}{\pgfqpoint{1.764706in}{1.700000in}}%
\pgfusepath{clip}%
\pgfsetbuttcap%
\pgfsetroundjoin%
\definecolor{currentfill}{rgb}{0.981377,0.920617,0.865369}%
\pgfsetfillcolor{currentfill}%
\pgfsetlinewidth{0.311001pt}%
\definecolor{currentstroke}{rgb}{1.000000,1.000000,1.000000}%
\pgfsetstrokecolor{currentstroke}%
\pgfsetdash{}{0pt}%
\pgfpathmoveto{\pgfqpoint{6.310316in}{1.269648in}}%
\pgfpathcurveto{\pgfqpoint{6.317449in}{1.269648in}}{\pgfqpoint{6.324291in}{1.272481in}}{\pgfqpoint{6.329335in}{1.277525in}}%
\pgfpathcurveto{\pgfqpoint{6.334378in}{1.282569in}}{\pgfqpoint{6.337212in}{1.289410in}}{\pgfqpoint{6.337212in}{1.296543in}}%
\pgfpathcurveto{\pgfqpoint{6.337212in}{1.303676in}}{\pgfqpoint{6.334378in}{1.310518in}}{\pgfqpoint{6.329335in}{1.315561in}}%
\pgfpathcurveto{\pgfqpoint{6.324291in}{1.320605in}}{\pgfqpoint{6.317449in}{1.323439in}}{\pgfqpoint{6.310316in}{1.323439in}}%
\pgfpathcurveto{\pgfqpoint{6.303184in}{1.323439in}}{\pgfqpoint{6.296342in}{1.320605in}}{\pgfqpoint{6.291298in}{1.315561in}}%
\pgfpathcurveto{\pgfqpoint{6.286255in}{1.310518in}}{\pgfqpoint{6.283421in}{1.303676in}}{\pgfqpoint{6.283421in}{1.296543in}}%
\pgfpathcurveto{\pgfqpoint{6.283421in}{1.289410in}}{\pgfqpoint{6.286255in}{1.282569in}}{\pgfqpoint{6.291298in}{1.277525in}}%
\pgfpathcurveto{\pgfqpoint{6.296342in}{1.272481in}}{\pgfqpoint{6.303184in}{1.269648in}}{\pgfqpoint{6.310316in}{1.269648in}}%
\pgfpathclose%
\pgfusepath{stroke,fill}%
\end{pgfscope}%
\begin{pgfscope}%
\pgfpathrectangle{\pgfqpoint{4.985294in}{0.500000in}}{\pgfqpoint{1.764706in}{1.700000in}}%
\pgfusepath{clip}%
\pgfsetbuttcap%
\pgfsetroundjoin%
\definecolor{currentfill}{rgb}{0.966328,0.750560,0.616961}%
\pgfsetfillcolor{currentfill}%
\pgfsetlinewidth{0.311001pt}%
\definecolor{currentstroke}{rgb}{1.000000,1.000000,1.000000}%
\pgfsetstrokecolor{currentstroke}%
\pgfsetdash{}{0pt}%
\pgfpathmoveto{\pgfqpoint{5.523757in}{0.923959in}}%
\pgfpathcurveto{\pgfqpoint{5.530890in}{0.923959in}}{\pgfqpoint{5.537731in}{0.926792in}}{\pgfqpoint{5.542775in}{0.931836in}}%
\pgfpathcurveto{\pgfqpoint{5.547819in}{0.936880in}}{\pgfqpoint{5.550652in}{0.943721in}}{\pgfqpoint{5.550652in}{0.950854in}}%
\pgfpathcurveto{\pgfqpoint{5.550652in}{0.957987in}}{\pgfqpoint{5.547819in}{0.964829in}}{\pgfqpoint{5.542775in}{0.969872in}}%
\pgfpathcurveto{\pgfqpoint{5.537731in}{0.974916in}}{\pgfqpoint{5.530890in}{0.977750in}}{\pgfqpoint{5.523757in}{0.977750in}}%
\pgfpathcurveto{\pgfqpoint{5.516624in}{0.977750in}}{\pgfqpoint{5.509782in}{0.974916in}}{\pgfqpoint{5.504739in}{0.969872in}}%
\pgfpathcurveto{\pgfqpoint{5.499695in}{0.964829in}}{\pgfqpoint{5.496861in}{0.957987in}}{\pgfqpoint{5.496861in}{0.950854in}}%
\pgfpathcurveto{\pgfqpoint{5.496861in}{0.943721in}}{\pgfqpoint{5.499695in}{0.936880in}}{\pgfqpoint{5.504739in}{0.931836in}}%
\pgfpathcurveto{\pgfqpoint{5.509782in}{0.926792in}}{\pgfqpoint{5.516624in}{0.923959in}}{\pgfqpoint{5.523757in}{0.923959in}}%
\pgfpathclose%
\pgfusepath{stroke,fill}%
\end{pgfscope}%
\begin{pgfscope}%
\pgfpathrectangle{\pgfqpoint{4.985294in}{0.500000in}}{\pgfqpoint{1.764706in}{1.700000in}}%
\pgfusepath{clip}%
\pgfsetbuttcap%
\pgfsetroundjoin%
\definecolor{currentfill}{rgb}{0.966812,0.762584,0.633643}%
\pgfsetfillcolor{currentfill}%
\pgfsetlinewidth{0.311001pt}%
\definecolor{currentstroke}{rgb}{1.000000,1.000000,1.000000}%
\pgfsetstrokecolor{currentstroke}%
\pgfsetdash{}{0pt}%
\pgfpathmoveto{\pgfqpoint{5.380491in}{1.091270in}}%
\pgfpathcurveto{\pgfqpoint{5.387624in}{1.091270in}}{\pgfqpoint{5.394465in}{1.094103in}}{\pgfqpoint{5.399509in}{1.099147in}}%
\pgfpathcurveto{\pgfqpoint{5.404553in}{1.104191in}}{\pgfqpoint{5.407386in}{1.111032in}}{\pgfqpoint{5.407386in}{1.118165in}}%
\pgfpathcurveto{\pgfqpoint{5.407386in}{1.125298in}}{\pgfqpoint{5.404553in}{1.132140in}}{\pgfqpoint{5.399509in}{1.137183in}}%
\pgfpathcurveto{\pgfqpoint{5.394465in}{1.142227in}}{\pgfqpoint{5.387624in}{1.145061in}}{\pgfqpoint{5.380491in}{1.145061in}}%
\pgfpathcurveto{\pgfqpoint{5.373358in}{1.145061in}}{\pgfqpoint{5.366516in}{1.142227in}}{\pgfqpoint{5.361473in}{1.137183in}}%
\pgfpathcurveto{\pgfqpoint{5.356429in}{1.132140in}}{\pgfqpoint{5.353595in}{1.125298in}}{\pgfqpoint{5.353595in}{1.118165in}}%
\pgfpathcurveto{\pgfqpoint{5.353595in}{1.111032in}}{\pgfqpoint{5.356429in}{1.104191in}}{\pgfqpoint{5.361473in}{1.099147in}}%
\pgfpathcurveto{\pgfqpoint{5.366516in}{1.094103in}}{\pgfqpoint{5.373358in}{1.091270in}}{\pgfqpoint{5.380491in}{1.091270in}}%
\pgfpathclose%
\pgfusepath{stroke,fill}%
\end{pgfscope}%
\begin{pgfscope}%
\pgfpathrectangle{\pgfqpoint{4.985294in}{0.500000in}}{\pgfqpoint{1.764706in}{1.700000in}}%
\pgfusepath{clip}%
\pgfsetbuttcap%
\pgfsetroundjoin%
\definecolor{currentfill}{rgb}{0.976287,0.879862,0.805788}%
\pgfsetfillcolor{currentfill}%
\pgfsetlinewidth{0.311001pt}%
\definecolor{currentstroke}{rgb}{1.000000,1.000000,1.000000}%
\pgfsetstrokecolor{currentstroke}%
\pgfsetdash{}{0pt}%
\pgfpathmoveto{\pgfqpoint{6.260151in}{1.215349in}}%
\pgfpathcurveto{\pgfqpoint{6.267284in}{1.215349in}}{\pgfqpoint{6.274125in}{1.218183in}}{\pgfqpoint{6.279169in}{1.223226in}}%
\pgfpathcurveto{\pgfqpoint{6.284213in}{1.228270in}}{\pgfqpoint{6.287046in}{1.235112in}}{\pgfqpoint{6.287046in}{1.242244in}}%
\pgfpathcurveto{\pgfqpoint{6.287046in}{1.249377in}}{\pgfqpoint{6.284213in}{1.256219in}}{\pgfqpoint{6.279169in}{1.261263in}}%
\pgfpathcurveto{\pgfqpoint{6.274125in}{1.266306in}}{\pgfqpoint{6.267284in}{1.269140in}}{\pgfqpoint{6.260151in}{1.269140in}}%
\pgfpathcurveto{\pgfqpoint{6.253018in}{1.269140in}}{\pgfqpoint{6.246176in}{1.266306in}}{\pgfqpoint{6.241133in}{1.261263in}}%
\pgfpathcurveto{\pgfqpoint{6.236089in}{1.256219in}}{\pgfqpoint{6.233255in}{1.249377in}}{\pgfqpoint{6.233255in}{1.242244in}}%
\pgfpathcurveto{\pgfqpoint{6.233255in}{1.235112in}}{\pgfqpoint{6.236089in}{1.228270in}}{\pgfqpoint{6.241133in}{1.223226in}}%
\pgfpathcurveto{\pgfqpoint{6.246176in}{1.218183in}}{\pgfqpoint{6.253018in}{1.215349in}}{\pgfqpoint{6.260151in}{1.215349in}}%
\pgfpathclose%
\pgfusepath{stroke,fill}%
\end{pgfscope}%
\begin{pgfscope}%
\pgfpathrectangle{\pgfqpoint{4.985294in}{0.500000in}}{\pgfqpoint{1.764706in}{1.700000in}}%
\pgfusepath{clip}%
\pgfsetbuttcap%
\pgfsetroundjoin%
\definecolor{currentfill}{rgb}{0.958331,0.519463,0.362986}%
\pgfsetfillcolor{currentfill}%
\pgfsetlinewidth{0.311001pt}%
\definecolor{currentstroke}{rgb}{1.000000,1.000000,1.000000}%
\pgfsetstrokecolor{currentstroke}%
\pgfsetdash{}{0pt}%
\pgfpathmoveto{\pgfqpoint{5.621655in}{1.761458in}}%
\pgfpathcurveto{\pgfqpoint{5.628788in}{1.761458in}}{\pgfqpoint{5.635629in}{1.764292in}}{\pgfqpoint{5.640673in}{1.769335in}}%
\pgfpathcurveto{\pgfqpoint{5.645717in}{1.774379in}}{\pgfqpoint{5.648551in}{1.781221in}}{\pgfqpoint{5.648551in}{1.788354in}}%
\pgfpathcurveto{\pgfqpoint{5.648551in}{1.795486in}}{\pgfqpoint{5.645717in}{1.802328in}}{\pgfqpoint{5.640673in}{1.807372in}}%
\pgfpathcurveto{\pgfqpoint{5.635629in}{1.812415in}}{\pgfqpoint{5.628788in}{1.815249in}}{\pgfqpoint{5.621655in}{1.815249in}}%
\pgfpathcurveto{\pgfqpoint{5.614522in}{1.815249in}}{\pgfqpoint{5.607681in}{1.812415in}}{\pgfqpoint{5.602637in}{1.807372in}}%
\pgfpathcurveto{\pgfqpoint{5.597593in}{1.802328in}}{\pgfqpoint{5.594759in}{1.795486in}}{\pgfqpoint{5.594759in}{1.788354in}}%
\pgfpathcurveto{\pgfqpoint{5.594759in}{1.781221in}}{\pgfqpoint{5.597593in}{1.774379in}}{\pgfqpoint{5.602637in}{1.769335in}}%
\pgfpathcurveto{\pgfqpoint{5.607681in}{1.764292in}}{\pgfqpoint{5.614522in}{1.761458in}}{\pgfqpoint{5.621655in}{1.761458in}}%
\pgfpathclose%
\pgfusepath{stroke,fill}%
\end{pgfscope}%
\begin{pgfscope}%
\pgfpathrectangle{\pgfqpoint{4.985294in}{0.500000in}}{\pgfqpoint{1.764706in}{1.700000in}}%
\pgfusepath{clip}%
\pgfsetbuttcap%
\pgfsetroundjoin%
\definecolor{currentfill}{rgb}{0.973271,0.850724,0.762998}%
\pgfsetfillcolor{currentfill}%
\pgfsetlinewidth{0.311001pt}%
\definecolor{currentstroke}{rgb}{1.000000,1.000000,1.000000}%
\pgfsetstrokecolor{currentstroke}%
\pgfsetdash{}{0pt}%
\pgfpathmoveto{\pgfqpoint{6.363136in}{1.214405in}}%
\pgfpathcurveto{\pgfqpoint{6.370269in}{1.214405in}}{\pgfqpoint{6.377111in}{1.217239in}}{\pgfqpoint{6.382155in}{1.222283in}}%
\pgfpathcurveto{\pgfqpoint{6.387198in}{1.227326in}}{\pgfqpoint{6.390032in}{1.234168in}}{\pgfqpoint{6.390032in}{1.241301in}}%
\pgfpathcurveto{\pgfqpoint{6.390032in}{1.248433in}}{\pgfqpoint{6.387198in}{1.255275in}}{\pgfqpoint{6.382155in}{1.260319in}}%
\pgfpathcurveto{\pgfqpoint{6.377111in}{1.265362in}}{\pgfqpoint{6.370269in}{1.268196in}}{\pgfqpoint{6.363136in}{1.268196in}}%
\pgfpathcurveto{\pgfqpoint{6.356004in}{1.268196in}}{\pgfqpoint{6.349162in}{1.265362in}}{\pgfqpoint{6.344118in}{1.260319in}}%
\pgfpathcurveto{\pgfqpoint{6.339075in}{1.255275in}}{\pgfqpoint{6.336241in}{1.248433in}}{\pgfqpoint{6.336241in}{1.241301in}}%
\pgfpathcurveto{\pgfqpoint{6.336241in}{1.234168in}}{\pgfqpoint{6.339075in}{1.227326in}}{\pgfqpoint{6.344118in}{1.222283in}}%
\pgfpathcurveto{\pgfqpoint{6.349162in}{1.217239in}}{\pgfqpoint{6.356004in}{1.214405in}}{\pgfqpoint{6.363136in}{1.214405in}}%
\pgfpathclose%
\pgfusepath{stroke,fill}%
\end{pgfscope}%
\begin{pgfscope}%
\pgfpathrectangle{\pgfqpoint{4.985294in}{0.500000in}}{\pgfqpoint{1.764706in}{1.700000in}}%
\pgfusepath{clip}%
\pgfsetbuttcap%
\pgfsetroundjoin%
\definecolor{currentfill}{rgb}{0.970255,0.815666,0.711203}%
\pgfsetfillcolor{currentfill}%
\pgfsetlinewidth{0.311001pt}%
\definecolor{currentstroke}{rgb}{1.000000,1.000000,1.000000}%
\pgfsetstrokecolor{currentstroke}%
\pgfsetdash{}{0pt}%
\pgfpathmoveto{\pgfqpoint{6.383199in}{1.257808in}}%
\pgfpathcurveto{\pgfqpoint{6.390332in}{1.257808in}}{\pgfqpoint{6.397174in}{1.260641in}}{\pgfqpoint{6.402217in}{1.265685in}}%
\pgfpathcurveto{\pgfqpoint{6.407261in}{1.270729in}}{\pgfqpoint{6.410095in}{1.277570in}}{\pgfqpoint{6.410095in}{1.284703in}}%
\pgfpathcurveto{\pgfqpoint{6.410095in}{1.291836in}}{\pgfqpoint{6.407261in}{1.298678in}}{\pgfqpoint{6.402217in}{1.303721in}}%
\pgfpathcurveto{\pgfqpoint{6.397174in}{1.308765in}}{\pgfqpoint{6.390332in}{1.311599in}}{\pgfqpoint{6.383199in}{1.311599in}}%
\pgfpathcurveto{\pgfqpoint{6.376066in}{1.311599in}}{\pgfqpoint{6.369225in}{1.308765in}}{\pgfqpoint{6.364181in}{1.303721in}}%
\pgfpathcurveto{\pgfqpoint{6.359137in}{1.298678in}}{\pgfqpoint{6.356304in}{1.291836in}}{\pgfqpoint{6.356304in}{1.284703in}}%
\pgfpathcurveto{\pgfqpoint{6.356304in}{1.277570in}}{\pgfqpoint{6.359137in}{1.270729in}}{\pgfqpoint{6.364181in}{1.265685in}}%
\pgfpathcurveto{\pgfqpoint{6.369225in}{1.260641in}}{\pgfqpoint{6.376066in}{1.257808in}}{\pgfqpoint{6.383199in}{1.257808in}}%
\pgfpathclose%
\pgfusepath{stroke,fill}%
\end{pgfscope}%
\begin{pgfscope}%
\pgfpathrectangle{\pgfqpoint{4.985294in}{0.500000in}}{\pgfqpoint{1.764706in}{1.700000in}}%
\pgfusepath{clip}%
\pgfsetbuttcap%
\pgfsetroundjoin%
\definecolor{currentfill}{rgb}{0.973832,0.856556,0.771584}%
\pgfsetfillcolor{currentfill}%
\pgfsetlinewidth{0.311001pt}%
\definecolor{currentstroke}{rgb}{1.000000,1.000000,1.000000}%
\pgfsetstrokecolor{currentstroke}%
\pgfsetdash{}{0pt}%
\pgfpathmoveto{\pgfqpoint{6.248983in}{1.047749in}}%
\pgfpathcurveto{\pgfqpoint{6.256115in}{1.047749in}}{\pgfqpoint{6.262957in}{1.050583in}}{\pgfqpoint{6.268001in}{1.055627in}}%
\pgfpathcurveto{\pgfqpoint{6.273044in}{1.060670in}}{\pgfqpoint{6.275878in}{1.067512in}}{\pgfqpoint{6.275878in}{1.074645in}}%
\pgfpathcurveto{\pgfqpoint{6.275878in}{1.081777in}}{\pgfqpoint{6.273044in}{1.088619in}}{\pgfqpoint{6.268001in}{1.093663in}}%
\pgfpathcurveto{\pgfqpoint{6.262957in}{1.098706in}}{\pgfqpoint{6.256115in}{1.101540in}}{\pgfqpoint{6.248983in}{1.101540in}}%
\pgfpathcurveto{\pgfqpoint{6.241850in}{1.101540in}}{\pgfqpoint{6.235008in}{1.098706in}}{\pgfqpoint{6.229964in}{1.093663in}}%
\pgfpathcurveto{\pgfqpoint{6.224921in}{1.088619in}}{\pgfqpoint{6.222087in}{1.081777in}}{\pgfqpoint{6.222087in}{1.074645in}}%
\pgfpathcurveto{\pgfqpoint{6.222087in}{1.067512in}}{\pgfqpoint{6.224921in}{1.060670in}}{\pgfqpoint{6.229964in}{1.055627in}}%
\pgfpathcurveto{\pgfqpoint{6.235008in}{1.050583in}}{\pgfqpoint{6.241850in}{1.047749in}}{\pgfqpoint{6.248983in}{1.047749in}}%
\pgfpathclose%
\pgfusepath{stroke,fill}%
\end{pgfscope}%
\begin{pgfscope}%
\pgfpathrectangle{\pgfqpoint{4.985294in}{0.500000in}}{\pgfqpoint{1.764706in}{1.700000in}}%
\pgfusepath{clip}%
\pgfsetbuttcap%
\pgfsetroundjoin%
\definecolor{currentfill}{rgb}{0.968105,0.786346,0.667739}%
\pgfsetfillcolor{currentfill}%
\pgfsetlinewidth{0.311001pt}%
\definecolor{currentstroke}{rgb}{1.000000,1.000000,1.000000}%
\pgfsetstrokecolor{currentstroke}%
\pgfsetdash{}{0pt}%
\pgfpathmoveto{\pgfqpoint{5.499201in}{1.674995in}}%
\pgfpathcurveto{\pgfqpoint{5.506334in}{1.674995in}}{\pgfqpoint{5.513175in}{1.677829in}}{\pgfqpoint{5.518219in}{1.682872in}}%
\pgfpathcurveto{\pgfqpoint{5.523263in}{1.687916in}}{\pgfqpoint{5.526096in}{1.694758in}}{\pgfqpoint{5.526096in}{1.701890in}}%
\pgfpathcurveto{\pgfqpoint{5.526096in}{1.709023in}}{\pgfqpoint{5.523263in}{1.715865in}}{\pgfqpoint{5.518219in}{1.720908in}}%
\pgfpathcurveto{\pgfqpoint{5.513175in}{1.725952in}}{\pgfqpoint{5.506334in}{1.728786in}}{\pgfqpoint{5.499201in}{1.728786in}}%
\pgfpathcurveto{\pgfqpoint{5.492068in}{1.728786in}}{\pgfqpoint{5.485226in}{1.725952in}}{\pgfqpoint{5.480183in}{1.720908in}}%
\pgfpathcurveto{\pgfqpoint{5.475139in}{1.715865in}}{\pgfqpoint{5.472305in}{1.709023in}}{\pgfqpoint{5.472305in}{1.701890in}}%
\pgfpathcurveto{\pgfqpoint{5.472305in}{1.694758in}}{\pgfqpoint{5.475139in}{1.687916in}}{\pgfqpoint{5.480183in}{1.682872in}}%
\pgfpathcurveto{\pgfqpoint{5.485226in}{1.677829in}}{\pgfqpoint{5.492068in}{1.674995in}}{\pgfqpoint{5.499201in}{1.674995in}}%
\pgfpathclose%
\pgfusepath{stroke,fill}%
\end{pgfscope}%
\begin{pgfscope}%
\pgfpathrectangle{\pgfqpoint{4.985294in}{0.500000in}}{\pgfqpoint{1.764706in}{1.700000in}}%
\pgfusepath{clip}%
\pgfsetbuttcap%
\pgfsetroundjoin%
\definecolor{currentfill}{rgb}{0.956817,0.498820,0.345554}%
\pgfsetfillcolor{currentfill}%
\pgfsetlinewidth{0.311001pt}%
\definecolor{currentstroke}{rgb}{1.000000,1.000000,1.000000}%
\pgfsetstrokecolor{currentstroke}%
\pgfsetdash{}{0pt}%
\pgfpathmoveto{\pgfqpoint{6.266445in}{1.746948in}}%
\pgfpathcurveto{\pgfqpoint{6.273578in}{1.746948in}}{\pgfqpoint{6.280420in}{1.749782in}}{\pgfqpoint{6.285463in}{1.754826in}}%
\pgfpathcurveto{\pgfqpoint{6.290507in}{1.759869in}}{\pgfqpoint{6.293341in}{1.766711in}}{\pgfqpoint{6.293341in}{1.773844in}}%
\pgfpathcurveto{\pgfqpoint{6.293341in}{1.780976in}}{\pgfqpoint{6.290507in}{1.787818in}}{\pgfqpoint{6.285463in}{1.792862in}}%
\pgfpathcurveto{\pgfqpoint{6.280420in}{1.797905in}}{\pgfqpoint{6.273578in}{1.800739in}}{\pgfqpoint{6.266445in}{1.800739in}}%
\pgfpathcurveto{\pgfqpoint{6.259312in}{1.800739in}}{\pgfqpoint{6.252471in}{1.797905in}}{\pgfqpoint{6.247427in}{1.792862in}}%
\pgfpathcurveto{\pgfqpoint{6.242383in}{1.787818in}}{\pgfqpoint{6.239550in}{1.780976in}}{\pgfqpoint{6.239550in}{1.773844in}}%
\pgfpathcurveto{\pgfqpoint{6.239550in}{1.766711in}}{\pgfqpoint{6.242383in}{1.759869in}}{\pgfqpoint{6.247427in}{1.754826in}}%
\pgfpathcurveto{\pgfqpoint{6.252471in}{1.749782in}}{\pgfqpoint{6.259312in}{1.746948in}}{\pgfqpoint{6.266445in}{1.746948in}}%
\pgfpathclose%
\pgfusepath{stroke,fill}%
\end{pgfscope}%
\begin{pgfscope}%
\pgfpathrectangle{\pgfqpoint{4.985294in}{0.500000in}}{\pgfqpoint{1.764706in}{1.700000in}}%
\pgfusepath{clip}%
\pgfsetbuttcap%
\pgfsetroundjoin%
\definecolor{currentfill}{rgb}{0.978376,0.897317,0.831308}%
\pgfsetfillcolor{currentfill}%
\pgfsetlinewidth{0.311001pt}%
\definecolor{currentstroke}{rgb}{1.000000,1.000000,1.000000}%
\pgfsetstrokecolor{currentstroke}%
\pgfsetdash{}{0pt}%
\pgfpathmoveto{\pgfqpoint{6.285561in}{1.349068in}}%
\pgfpathcurveto{\pgfqpoint{6.292694in}{1.349068in}}{\pgfqpoint{6.299536in}{1.351902in}}{\pgfqpoint{6.304579in}{1.356946in}}%
\pgfpathcurveto{\pgfqpoint{6.309623in}{1.361989in}}{\pgfqpoint{6.312457in}{1.368831in}}{\pgfqpoint{6.312457in}{1.375964in}}%
\pgfpathcurveto{\pgfqpoint{6.312457in}{1.383097in}}{\pgfqpoint{6.309623in}{1.389938in}}{\pgfqpoint{6.304579in}{1.394982in}}%
\pgfpathcurveto{\pgfqpoint{6.299536in}{1.400026in}}{\pgfqpoint{6.292694in}{1.402860in}}{\pgfqpoint{6.285561in}{1.402860in}}%
\pgfpathcurveto{\pgfqpoint{6.278428in}{1.402860in}}{\pgfqpoint{6.271587in}{1.400026in}}{\pgfqpoint{6.266543in}{1.394982in}}%
\pgfpathcurveto{\pgfqpoint{6.261499in}{1.389938in}}{\pgfqpoint{6.258665in}{1.383097in}}{\pgfqpoint{6.258665in}{1.375964in}}%
\pgfpathcurveto{\pgfqpoint{6.258665in}{1.368831in}}{\pgfqpoint{6.261499in}{1.361989in}}{\pgfqpoint{6.266543in}{1.356946in}}%
\pgfpathcurveto{\pgfqpoint{6.271587in}{1.351902in}}{\pgfqpoint{6.278428in}{1.349068in}}{\pgfqpoint{6.285561in}{1.349068in}}%
\pgfpathclose%
\pgfusepath{stroke,fill}%
\end{pgfscope}%
\begin{pgfscope}%
\pgfpathrectangle{\pgfqpoint{4.985294in}{0.500000in}}{\pgfqpoint{1.764706in}{1.700000in}}%
\pgfusepath{clip}%
\pgfsetbuttcap%
\pgfsetroundjoin%
\definecolor{currentfill}{rgb}{0.981377,0.920617,0.865369}%
\pgfsetfillcolor{currentfill}%
\pgfsetlinewidth{0.311001pt}%
\definecolor{currentstroke}{rgb}{1.000000,1.000000,1.000000}%
\pgfsetstrokecolor{currentstroke}%
\pgfsetdash{}{0pt}%
\pgfpathmoveto{\pgfqpoint{6.310832in}{1.382327in}}%
\pgfpathcurveto{\pgfqpoint{6.317965in}{1.382327in}}{\pgfqpoint{6.324807in}{1.385161in}}{\pgfqpoint{6.329850in}{1.390205in}}%
\pgfpathcurveto{\pgfqpoint{6.334894in}{1.395249in}}{\pgfqpoint{6.337728in}{1.402090in}}{\pgfqpoint{6.337728in}{1.409223in}}%
\pgfpathcurveto{\pgfqpoint{6.337728in}{1.416356in}}{\pgfqpoint{6.334894in}{1.423197in}}{\pgfqpoint{6.329850in}{1.428241in}}%
\pgfpathcurveto{\pgfqpoint{6.324807in}{1.433285in}}{\pgfqpoint{6.317965in}{1.436119in}}{\pgfqpoint{6.310832in}{1.436119in}}%
\pgfpathcurveto{\pgfqpoint{6.303700in}{1.436119in}}{\pgfqpoint{6.296858in}{1.433285in}}{\pgfqpoint{6.291814in}{1.428241in}}%
\pgfpathcurveto{\pgfqpoint{6.286771in}{1.423197in}}{\pgfqpoint{6.283937in}{1.416356in}}{\pgfqpoint{6.283937in}{1.409223in}}%
\pgfpathcurveto{\pgfqpoint{6.283937in}{1.402090in}}{\pgfqpoint{6.286771in}{1.395249in}}{\pgfqpoint{6.291814in}{1.390205in}}%
\pgfpathcurveto{\pgfqpoint{6.296858in}{1.385161in}}{\pgfqpoint{6.303700in}{1.382327in}}{\pgfqpoint{6.310832in}{1.382327in}}%
\pgfpathclose%
\pgfusepath{stroke,fill}%
\end{pgfscope}%
\begin{pgfscope}%
\pgfpathrectangle{\pgfqpoint{4.985294in}{0.500000in}}{\pgfqpoint{1.764706in}{1.700000in}}%
\pgfusepath{clip}%
\pgfsetbuttcap%
\pgfsetroundjoin%
\definecolor{currentfill}{rgb}{0.980678,0.914765,0.856766}%
\pgfsetfillcolor{currentfill}%
\pgfsetlinewidth{0.311001pt}%
\definecolor{currentstroke}{rgb}{1.000000,1.000000,1.000000}%
\pgfsetstrokecolor{currentstroke}%
\pgfsetdash{}{0pt}%
\pgfpathmoveto{\pgfqpoint{6.301985in}{1.241922in}}%
\pgfpathcurveto{\pgfqpoint{6.309118in}{1.241922in}}{\pgfqpoint{6.315959in}{1.244756in}}{\pgfqpoint{6.321003in}{1.249800in}}%
\pgfpathcurveto{\pgfqpoint{6.326047in}{1.254843in}}{\pgfqpoint{6.328880in}{1.261685in}}{\pgfqpoint{6.328880in}{1.268818in}}%
\pgfpathcurveto{\pgfqpoint{6.328880in}{1.275951in}}{\pgfqpoint{6.326047in}{1.282792in}}{\pgfqpoint{6.321003in}{1.287836in}}%
\pgfpathcurveto{\pgfqpoint{6.315959in}{1.292880in}}{\pgfqpoint{6.309118in}{1.295714in}}{\pgfqpoint{6.301985in}{1.295714in}}%
\pgfpathcurveto{\pgfqpoint{6.294852in}{1.295714in}}{\pgfqpoint{6.288010in}{1.292880in}}{\pgfqpoint{6.282967in}{1.287836in}}%
\pgfpathcurveto{\pgfqpoint{6.277923in}{1.282792in}}{\pgfqpoint{6.275089in}{1.275951in}}{\pgfqpoint{6.275089in}{1.268818in}}%
\pgfpathcurveto{\pgfqpoint{6.275089in}{1.261685in}}{\pgfqpoint{6.277923in}{1.254843in}}{\pgfqpoint{6.282967in}{1.249800in}}%
\pgfpathcurveto{\pgfqpoint{6.288010in}{1.244756in}}{\pgfqpoint{6.294852in}{1.241922in}}{\pgfqpoint{6.301985in}{1.241922in}}%
\pgfpathclose%
\pgfusepath{stroke,fill}%
\end{pgfscope}%
\begin{pgfscope}%
\pgfpathrectangle{\pgfqpoint{4.985294in}{0.500000in}}{\pgfqpoint{1.764706in}{1.700000in}}%
\pgfusepath{clip}%
\pgfsetbuttcap%
\pgfsetroundjoin%
\definecolor{currentfill}{rgb}{0.976287,0.879862,0.805788}%
\pgfsetfillcolor{currentfill}%
\pgfsetlinewidth{0.311001pt}%
\definecolor{currentstroke}{rgb}{1.000000,1.000000,1.000000}%
\pgfsetstrokecolor{currentstroke}%
\pgfsetdash{}{0pt}%
\pgfpathmoveto{\pgfqpoint{5.461870in}{1.233642in}}%
\pgfpathcurveto{\pgfqpoint{5.469003in}{1.233642in}}{\pgfqpoint{5.475844in}{1.236476in}}{\pgfqpoint{5.480888in}{1.241520in}}%
\pgfpathcurveto{\pgfqpoint{5.485932in}{1.246564in}}{\pgfqpoint{5.488766in}{1.253405in}}{\pgfqpoint{5.488766in}{1.260538in}}%
\pgfpathcurveto{\pgfqpoint{5.488766in}{1.267671in}}{\pgfqpoint{5.485932in}{1.274512in}}{\pgfqpoint{5.480888in}{1.279556in}}%
\pgfpathcurveto{\pgfqpoint{5.475844in}{1.284600in}}{\pgfqpoint{5.469003in}{1.287434in}}{\pgfqpoint{5.461870in}{1.287434in}}%
\pgfpathcurveto{\pgfqpoint{5.454737in}{1.287434in}}{\pgfqpoint{5.447895in}{1.284600in}}{\pgfqpoint{5.442852in}{1.279556in}}%
\pgfpathcurveto{\pgfqpoint{5.437808in}{1.274512in}}{\pgfqpoint{5.434974in}{1.267671in}}{\pgfqpoint{5.434974in}{1.260538in}}%
\pgfpathcurveto{\pgfqpoint{5.434974in}{1.253405in}}{\pgfqpoint{5.437808in}{1.246564in}}{\pgfqpoint{5.442852in}{1.241520in}}%
\pgfpathcurveto{\pgfqpoint{5.447895in}{1.236476in}}{\pgfqpoint{5.454737in}{1.233642in}}{\pgfqpoint{5.461870in}{1.233642in}}%
\pgfpathclose%
\pgfusepath{stroke,fill}%
\end{pgfscope}%
\begin{pgfscope}%
\pgfpathrectangle{\pgfqpoint{4.985294in}{0.500000in}}{\pgfqpoint{1.764706in}{1.700000in}}%
\pgfusepath{clip}%
\pgfsetbuttcap%
\pgfsetroundjoin%
\definecolor{currentfill}{rgb}{0.973271,0.850724,0.762998}%
\pgfsetfillcolor{currentfill}%
\pgfsetlinewidth{0.311001pt}%
\definecolor{currentstroke}{rgb}{1.000000,1.000000,1.000000}%
\pgfsetstrokecolor{currentstroke}%
\pgfsetdash{}{0pt}%
\pgfpathmoveto{\pgfqpoint{6.238257in}{1.514575in}}%
\pgfpathcurveto{\pgfqpoint{6.245390in}{1.514575in}}{\pgfqpoint{6.252232in}{1.517409in}}{\pgfqpoint{6.257275in}{1.522453in}}%
\pgfpathcurveto{\pgfqpoint{6.262319in}{1.527496in}}{\pgfqpoint{6.265153in}{1.534338in}}{\pgfqpoint{6.265153in}{1.541471in}}%
\pgfpathcurveto{\pgfqpoint{6.265153in}{1.548604in}}{\pgfqpoint{6.262319in}{1.555445in}}{\pgfqpoint{6.257275in}{1.560489in}}%
\pgfpathcurveto{\pgfqpoint{6.252232in}{1.565533in}}{\pgfqpoint{6.245390in}{1.568366in}}{\pgfqpoint{6.238257in}{1.568366in}}%
\pgfpathcurveto{\pgfqpoint{6.231124in}{1.568366in}}{\pgfqpoint{6.224283in}{1.565533in}}{\pgfqpoint{6.219239in}{1.560489in}}%
\pgfpathcurveto{\pgfqpoint{6.214195in}{1.555445in}}{\pgfqpoint{6.211362in}{1.548604in}}{\pgfqpoint{6.211362in}{1.541471in}}%
\pgfpathcurveto{\pgfqpoint{6.211362in}{1.534338in}}{\pgfqpoint{6.214195in}{1.527496in}}{\pgfqpoint{6.219239in}{1.522453in}}%
\pgfpathcurveto{\pgfqpoint{6.224283in}{1.517409in}}{\pgfqpoint{6.231124in}{1.514575in}}{\pgfqpoint{6.238257in}{1.514575in}}%
\pgfpathclose%
\pgfusepath{stroke,fill}%
\end{pgfscope}%
\begin{pgfscope}%
\pgfpathrectangle{\pgfqpoint{4.985294in}{0.500000in}}{\pgfqpoint{1.764706in}{1.700000in}}%
\pgfusepath{clip}%
\pgfsetbuttcap%
\pgfsetroundjoin%
\definecolor{currentfill}{rgb}{0.960043,0.546576,0.387029}%
\pgfsetfillcolor{currentfill}%
\pgfsetlinewidth{0.311001pt}%
\definecolor{currentstroke}{rgb}{1.000000,1.000000,1.000000}%
\pgfsetstrokecolor{currentstroke}%
\pgfsetdash{}{0pt}%
\pgfpathmoveto{\pgfqpoint{5.325618in}{1.508102in}}%
\pgfpathcurveto{\pgfqpoint{5.332751in}{1.508102in}}{\pgfqpoint{5.339592in}{1.510936in}}{\pgfqpoint{5.344636in}{1.515979in}}%
\pgfpathcurveto{\pgfqpoint{5.349680in}{1.521023in}}{\pgfqpoint{5.352514in}{1.527865in}}{\pgfqpoint{5.352514in}{1.534998in}}%
\pgfpathcurveto{\pgfqpoint{5.352514in}{1.542130in}}{\pgfqpoint{5.349680in}{1.548972in}}{\pgfqpoint{5.344636in}{1.554016in}}%
\pgfpathcurveto{\pgfqpoint{5.339592in}{1.559059in}}{\pgfqpoint{5.332751in}{1.561893in}}{\pgfqpoint{5.325618in}{1.561893in}}%
\pgfpathcurveto{\pgfqpoint{5.318485in}{1.561893in}}{\pgfqpoint{5.311643in}{1.559059in}}{\pgfqpoint{5.306600in}{1.554016in}}%
\pgfpathcurveto{\pgfqpoint{5.301556in}{1.548972in}}{\pgfqpoint{5.298722in}{1.542130in}}{\pgfqpoint{5.298722in}{1.534998in}}%
\pgfpathcurveto{\pgfqpoint{5.298722in}{1.527865in}}{\pgfqpoint{5.301556in}{1.521023in}}{\pgfqpoint{5.306600in}{1.515979in}}%
\pgfpathcurveto{\pgfqpoint{5.311643in}{1.510936in}}{\pgfqpoint{5.318485in}{1.508102in}}{\pgfqpoint{5.325618in}{1.508102in}}%
\pgfpathclose%
\pgfusepath{stroke,fill}%
\end{pgfscope}%
\begin{pgfscope}%
\pgfpathrectangle{\pgfqpoint{4.985294in}{0.500000in}}{\pgfqpoint{1.764706in}{1.700000in}}%
\pgfusepath{clip}%
\pgfsetbuttcap%
\pgfsetroundjoin%
\definecolor{currentfill}{rgb}{0.966328,0.750560,0.616961}%
\pgfsetfillcolor{currentfill}%
\pgfsetlinewidth{0.311001pt}%
\definecolor{currentstroke}{rgb}{1.000000,1.000000,1.000000}%
\pgfsetstrokecolor{currentstroke}%
\pgfsetdash{}{0pt}%
\pgfpathmoveto{\pgfqpoint{5.371857in}{1.520412in}}%
\pgfpathcurveto{\pgfqpoint{5.378990in}{1.520412in}}{\pgfqpoint{5.385831in}{1.523246in}}{\pgfqpoint{5.390875in}{1.528290in}}%
\pgfpathcurveto{\pgfqpoint{5.395919in}{1.533333in}}{\pgfqpoint{5.398753in}{1.540175in}}{\pgfqpoint{5.398753in}{1.547308in}}%
\pgfpathcurveto{\pgfqpoint{5.398753in}{1.554441in}}{\pgfqpoint{5.395919in}{1.561282in}}{\pgfqpoint{5.390875in}{1.566326in}}%
\pgfpathcurveto{\pgfqpoint{5.385831in}{1.571370in}}{\pgfqpoint{5.378990in}{1.574203in}}{\pgfqpoint{5.371857in}{1.574203in}}%
\pgfpathcurveto{\pgfqpoint{5.364724in}{1.574203in}}{\pgfqpoint{5.357882in}{1.571370in}}{\pgfqpoint{5.352839in}{1.566326in}}%
\pgfpathcurveto{\pgfqpoint{5.347795in}{1.561282in}}{\pgfqpoint{5.344961in}{1.554441in}}{\pgfqpoint{5.344961in}{1.547308in}}%
\pgfpathcurveto{\pgfqpoint{5.344961in}{1.540175in}}{\pgfqpoint{5.347795in}{1.533333in}}{\pgfqpoint{5.352839in}{1.528290in}}%
\pgfpathcurveto{\pgfqpoint{5.357882in}{1.523246in}}{\pgfqpoint{5.364724in}{1.520412in}}{\pgfqpoint{5.371857in}{1.520412in}}%
\pgfpathclose%
\pgfusepath{stroke,fill}%
\end{pgfscope}%
\begin{pgfscope}%
\pgfpathrectangle{\pgfqpoint{4.985294in}{0.500000in}}{\pgfqpoint{1.764706in}{1.700000in}}%
\pgfusepath{clip}%
\pgfsetbuttcap%
\pgfsetroundjoin%
\definecolor{currentfill}{rgb}{0.979124,0.903132,0.839793}%
\pgfsetfillcolor{currentfill}%
\pgfsetlinewidth{0.311001pt}%
\definecolor{currentstroke}{rgb}{1.000000,1.000000,1.000000}%
\pgfsetstrokecolor{currentstroke}%
\pgfsetdash{}{0pt}%
\pgfpathmoveto{\pgfqpoint{6.271500in}{1.561965in}}%
\pgfpathcurveto{\pgfqpoint{6.278633in}{1.561965in}}{\pgfqpoint{6.285475in}{1.564799in}}{\pgfqpoint{6.290518in}{1.569842in}}%
\pgfpathcurveto{\pgfqpoint{6.295562in}{1.574886in}}{\pgfqpoint{6.298396in}{1.581728in}}{\pgfqpoint{6.298396in}{1.588860in}}%
\pgfpathcurveto{\pgfqpoint{6.298396in}{1.595993in}}{\pgfqpoint{6.295562in}{1.602835in}}{\pgfqpoint{6.290518in}{1.607879in}}%
\pgfpathcurveto{\pgfqpoint{6.285475in}{1.612922in}}{\pgfqpoint{6.278633in}{1.615756in}}{\pgfqpoint{6.271500in}{1.615756in}}%
\pgfpathcurveto{\pgfqpoint{6.264368in}{1.615756in}}{\pgfqpoint{6.257526in}{1.612922in}}{\pgfqpoint{6.252482in}{1.607879in}}%
\pgfpathcurveto{\pgfqpoint{6.247439in}{1.602835in}}{\pgfqpoint{6.244605in}{1.595993in}}{\pgfqpoint{6.244605in}{1.588860in}}%
\pgfpathcurveto{\pgfqpoint{6.244605in}{1.581728in}}{\pgfqpoint{6.247439in}{1.574886in}}{\pgfqpoint{6.252482in}{1.569842in}}%
\pgfpathcurveto{\pgfqpoint{6.257526in}{1.564799in}}{\pgfqpoint{6.264368in}{1.561965in}}{\pgfqpoint{6.271500in}{1.561965in}}%
\pgfpathclose%
\pgfusepath{stroke,fill}%
\end{pgfscope}%
\begin{pgfscope}%
\pgfpathrectangle{\pgfqpoint{4.985294in}{0.500000in}}{\pgfqpoint{1.764706in}{1.700000in}}%
\pgfusepath{clip}%
\pgfsetbuttcap%
\pgfsetroundjoin%
\definecolor{currentfill}{rgb}{0.967735,0.780441,0.659127}%
\pgfsetfillcolor{currentfill}%
\pgfsetlinewidth{0.311001pt}%
\definecolor{currentstroke}{rgb}{1.000000,1.000000,1.000000}%
\pgfsetstrokecolor{currentstroke}%
\pgfsetdash{}{0pt}%
\pgfpathmoveto{\pgfqpoint{5.452209in}{1.645124in}}%
\pgfpathcurveto{\pgfqpoint{5.459342in}{1.645124in}}{\pgfqpoint{5.466183in}{1.647958in}}{\pgfqpoint{5.471227in}{1.653002in}}%
\pgfpathcurveto{\pgfqpoint{5.476271in}{1.658045in}}{\pgfqpoint{5.479105in}{1.664887in}}{\pgfqpoint{5.479105in}{1.672020in}}%
\pgfpathcurveto{\pgfqpoint{5.479105in}{1.679153in}}{\pgfqpoint{5.476271in}{1.685994in}}{\pgfqpoint{5.471227in}{1.691038in}}%
\pgfpathcurveto{\pgfqpoint{5.466183in}{1.696082in}}{\pgfqpoint{5.459342in}{1.698915in}}{\pgfqpoint{5.452209in}{1.698915in}}%
\pgfpathcurveto{\pgfqpoint{5.445076in}{1.698915in}}{\pgfqpoint{5.438234in}{1.696082in}}{\pgfqpoint{5.433191in}{1.691038in}}%
\pgfpathcurveto{\pgfqpoint{5.428147in}{1.685994in}}{\pgfqpoint{5.425313in}{1.679153in}}{\pgfqpoint{5.425313in}{1.672020in}}%
\pgfpathcurveto{\pgfqpoint{5.425313in}{1.664887in}}{\pgfqpoint{5.428147in}{1.658045in}}{\pgfqpoint{5.433191in}{1.653002in}}%
\pgfpathcurveto{\pgfqpoint{5.438234in}{1.647958in}}{\pgfqpoint{5.445076in}{1.645124in}}{\pgfqpoint{5.452209in}{1.645124in}}%
\pgfpathclose%
\pgfusepath{stroke,fill}%
\end{pgfscope}%
\begin{pgfscope}%
\pgfpathrectangle{\pgfqpoint{4.985294in}{0.500000in}}{\pgfqpoint{1.764706in}{1.700000in}}%
\pgfusepath{clip}%
\pgfsetbuttcap%
\pgfsetroundjoin%
\definecolor{currentfill}{rgb}{0.967398,0.774513,0.650573}%
\pgfsetfillcolor{currentfill}%
\pgfsetlinewidth{0.311001pt}%
\definecolor{currentstroke}{rgb}{1.000000,1.000000,1.000000}%
\pgfsetstrokecolor{currentstroke}%
\pgfsetdash{}{0pt}%
\pgfpathmoveto{\pgfqpoint{6.171870in}{1.576067in}}%
\pgfpathcurveto{\pgfqpoint{6.179003in}{1.576067in}}{\pgfqpoint{6.185844in}{1.578901in}}{\pgfqpoint{6.190888in}{1.583945in}}%
\pgfpathcurveto{\pgfqpoint{6.195932in}{1.588989in}}{\pgfqpoint{6.198765in}{1.595830in}}{\pgfqpoint{6.198765in}{1.602963in}}%
\pgfpathcurveto{\pgfqpoint{6.198765in}{1.610096in}}{\pgfqpoint{6.195932in}{1.616938in}}{\pgfqpoint{6.190888in}{1.621981in}}%
\pgfpathcurveto{\pgfqpoint{6.185844in}{1.627025in}}{\pgfqpoint{6.179003in}{1.629859in}}{\pgfqpoint{6.171870in}{1.629859in}}%
\pgfpathcurveto{\pgfqpoint{6.164737in}{1.629859in}}{\pgfqpoint{6.157895in}{1.627025in}}{\pgfqpoint{6.152852in}{1.621981in}}%
\pgfpathcurveto{\pgfqpoint{6.147808in}{1.616938in}}{\pgfqpoint{6.144974in}{1.610096in}}{\pgfqpoint{6.144974in}{1.602963in}}%
\pgfpathcurveto{\pgfqpoint{6.144974in}{1.595830in}}{\pgfqpoint{6.147808in}{1.588989in}}{\pgfqpoint{6.152852in}{1.583945in}}%
\pgfpathcurveto{\pgfqpoint{6.157895in}{1.578901in}}{\pgfqpoint{6.164737in}{1.576067in}}{\pgfqpoint{6.171870in}{1.576067in}}%
\pgfpathclose%
\pgfusepath{stroke,fill}%
\end{pgfscope}%
\begin{pgfscope}%
\pgfpathrectangle{\pgfqpoint{4.985294in}{0.500000in}}{\pgfqpoint{1.764706in}{1.700000in}}%
\pgfusepath{clip}%
\pgfsetbuttcap%
\pgfsetroundjoin%
\definecolor{currentfill}{rgb}{0.973832,0.856556,0.771584}%
\pgfsetfillcolor{currentfill}%
\pgfsetlinewidth{0.311001pt}%
\definecolor{currentstroke}{rgb}{1.000000,1.000000,1.000000}%
\pgfsetstrokecolor{currentstroke}%
\pgfsetdash{}{0pt}%
\pgfpathmoveto{\pgfqpoint{5.482497in}{1.585944in}}%
\pgfpathcurveto{\pgfqpoint{5.489630in}{1.585944in}}{\pgfqpoint{5.496471in}{1.588777in}}{\pgfqpoint{5.501515in}{1.593821in}}%
\pgfpathcurveto{\pgfqpoint{5.506558in}{1.598865in}}{\pgfqpoint{5.509392in}{1.605706in}}{\pgfqpoint{5.509392in}{1.612839in}}%
\pgfpathcurveto{\pgfqpoint{5.509392in}{1.619972in}}{\pgfqpoint{5.506558in}{1.626814in}}{\pgfqpoint{5.501515in}{1.631857in}}%
\pgfpathcurveto{\pgfqpoint{5.496471in}{1.636901in}}{\pgfqpoint{5.489630in}{1.639735in}}{\pgfqpoint{5.482497in}{1.639735in}}%
\pgfpathcurveto{\pgfqpoint{5.475364in}{1.639735in}}{\pgfqpoint{5.468522in}{1.636901in}}{\pgfqpoint{5.463479in}{1.631857in}}%
\pgfpathcurveto{\pgfqpoint{5.458435in}{1.626814in}}{\pgfqpoint{5.455601in}{1.619972in}}{\pgfqpoint{5.455601in}{1.612839in}}%
\pgfpathcurveto{\pgfqpoint{5.455601in}{1.605706in}}{\pgfqpoint{5.458435in}{1.598865in}}{\pgfqpoint{5.463479in}{1.593821in}}%
\pgfpathcurveto{\pgfqpoint{5.468522in}{1.588777in}}{\pgfqpoint{5.475364in}{1.585944in}}{\pgfqpoint{5.482497in}{1.585944in}}%
\pgfpathclose%
\pgfusepath{stroke,fill}%
\end{pgfscope}%
\begin{pgfscope}%
\pgfpathrectangle{\pgfqpoint{4.985294in}{0.500000in}}{\pgfqpoint{1.764706in}{1.700000in}}%
\pgfusepath{clip}%
\pgfsetbuttcap%
\pgfsetroundjoin%
\definecolor{currentfill}{rgb}{0.962765,0.606121,0.444717}%
\pgfsetfillcolor{currentfill}%
\pgfsetlinewidth{0.311001pt}%
\definecolor{currentstroke}{rgb}{1.000000,1.000000,1.000000}%
\pgfsetstrokecolor{currentstroke}%
\pgfsetdash{}{0pt}%
\pgfpathmoveto{\pgfqpoint{5.344400in}{1.081034in}}%
\pgfpathcurveto{\pgfqpoint{5.351533in}{1.081034in}}{\pgfqpoint{5.358375in}{1.083868in}}{\pgfqpoint{5.363418in}{1.088911in}}%
\pgfpathcurveto{\pgfqpoint{5.368462in}{1.093955in}}{\pgfqpoint{5.371296in}{1.100797in}}{\pgfqpoint{5.371296in}{1.107930in}}%
\pgfpathcurveto{\pgfqpoint{5.371296in}{1.115062in}}{\pgfqpoint{5.368462in}{1.121904in}}{\pgfqpoint{5.363418in}{1.126948in}}%
\pgfpathcurveto{\pgfqpoint{5.358375in}{1.131991in}}{\pgfqpoint{5.351533in}{1.134825in}}{\pgfqpoint{5.344400in}{1.134825in}}%
\pgfpathcurveto{\pgfqpoint{5.337267in}{1.134825in}}{\pgfqpoint{5.330426in}{1.131991in}}{\pgfqpoint{5.325382in}{1.126948in}}%
\pgfpathcurveto{\pgfqpoint{5.320338in}{1.121904in}}{\pgfqpoint{5.317504in}{1.115062in}}{\pgfqpoint{5.317504in}{1.107930in}}%
\pgfpathcurveto{\pgfqpoint{5.317504in}{1.100797in}}{\pgfqpoint{5.320338in}{1.093955in}}{\pgfqpoint{5.325382in}{1.088911in}}%
\pgfpathcurveto{\pgfqpoint{5.330426in}{1.083868in}}{\pgfqpoint{5.337267in}{1.081034in}}{\pgfqpoint{5.344400in}{1.081034in}}%
\pgfpathclose%
\pgfusepath{stroke,fill}%
\end{pgfscope}%
\begin{pgfscope}%
\pgfpathrectangle{\pgfqpoint{4.985294in}{0.500000in}}{\pgfqpoint{1.764706in}{1.700000in}}%
\pgfusepath{clip}%
\pgfsetbuttcap%
\pgfsetroundjoin%
\definecolor{currentfill}{rgb}{0.973832,0.856556,0.771584}%
\pgfsetfillcolor{currentfill}%
\pgfsetlinewidth{0.311001pt}%
\definecolor{currentstroke}{rgb}{1.000000,1.000000,1.000000}%
\pgfsetstrokecolor{currentstroke}%
\pgfsetdash{}{0pt}%
\pgfpathmoveto{\pgfqpoint{6.263595in}{1.630039in}}%
\pgfpathcurveto{\pgfqpoint{6.270728in}{1.630039in}}{\pgfqpoint{6.277570in}{1.632873in}}{\pgfqpoint{6.282614in}{1.637917in}}%
\pgfpathcurveto{\pgfqpoint{6.287657in}{1.642960in}}{\pgfqpoint{6.290491in}{1.649802in}}{\pgfqpoint{6.290491in}{1.656935in}}%
\pgfpathcurveto{\pgfqpoint{6.290491in}{1.664068in}}{\pgfqpoint{6.287657in}{1.670909in}}{\pgfqpoint{6.282614in}{1.675953in}}%
\pgfpathcurveto{\pgfqpoint{6.277570in}{1.680997in}}{\pgfqpoint{6.270728in}{1.683831in}}{\pgfqpoint{6.263595in}{1.683831in}}%
\pgfpathcurveto{\pgfqpoint{6.256463in}{1.683831in}}{\pgfqpoint{6.249621in}{1.680997in}}{\pgfqpoint{6.244577in}{1.675953in}}%
\pgfpathcurveto{\pgfqpoint{6.239534in}{1.670909in}}{\pgfqpoint{6.236700in}{1.664068in}}{\pgfqpoint{6.236700in}{1.656935in}}%
\pgfpathcurveto{\pgfqpoint{6.236700in}{1.649802in}}{\pgfqpoint{6.239534in}{1.642960in}}{\pgfqpoint{6.244577in}{1.637917in}}%
\pgfpathcurveto{\pgfqpoint{6.249621in}{1.632873in}}{\pgfqpoint{6.256463in}{1.630039in}}{\pgfqpoint{6.263595in}{1.630039in}}%
\pgfpathclose%
\pgfusepath{stroke,fill}%
\end{pgfscope}%
\begin{pgfscope}%
\pgfpathrectangle{\pgfqpoint{4.985294in}{0.500000in}}{\pgfqpoint{1.764706in}{1.700000in}}%
\pgfusepath{clip}%
\pgfsetbuttcap%
\pgfsetroundjoin%
\definecolor{currentfill}{rgb}{0.972726,0.844889,0.754401}%
\pgfsetfillcolor{currentfill}%
\pgfsetlinewidth{0.311001pt}%
\definecolor{currentstroke}{rgb}{1.000000,1.000000,1.000000}%
\pgfsetstrokecolor{currentstroke}%
\pgfsetdash{}{0pt}%
\pgfpathmoveto{\pgfqpoint{5.513209in}{1.065865in}}%
\pgfpathcurveto{\pgfqpoint{5.520341in}{1.065865in}}{\pgfqpoint{5.527183in}{1.068699in}}{\pgfqpoint{5.532227in}{1.073743in}}%
\pgfpathcurveto{\pgfqpoint{5.537270in}{1.078786in}}{\pgfqpoint{5.540104in}{1.085628in}}{\pgfqpoint{5.540104in}{1.092761in}}%
\pgfpathcurveto{\pgfqpoint{5.540104in}{1.099894in}}{\pgfqpoint{5.537270in}{1.106735in}}{\pgfqpoint{5.532227in}{1.111779in}}%
\pgfpathcurveto{\pgfqpoint{5.527183in}{1.116823in}}{\pgfqpoint{5.520341in}{1.119657in}}{\pgfqpoint{5.513209in}{1.119657in}}%
\pgfpathcurveto{\pgfqpoint{5.506076in}{1.119657in}}{\pgfqpoint{5.499234in}{1.116823in}}{\pgfqpoint{5.494190in}{1.111779in}}%
\pgfpathcurveto{\pgfqpoint{5.489147in}{1.106735in}}{\pgfqpoint{5.486313in}{1.099894in}}{\pgfqpoint{5.486313in}{1.092761in}}%
\pgfpathcurveto{\pgfqpoint{5.486313in}{1.085628in}}{\pgfqpoint{5.489147in}{1.078786in}}{\pgfqpoint{5.494190in}{1.073743in}}%
\pgfpathcurveto{\pgfqpoint{5.499234in}{1.068699in}}{\pgfqpoint{5.506076in}{1.065865in}}{\pgfqpoint{5.513209in}{1.065865in}}%
\pgfpathclose%
\pgfusepath{stroke,fill}%
\end{pgfscope}%
\begin{pgfscope}%
\pgfpathrectangle{\pgfqpoint{4.985294in}{0.500000in}}{\pgfqpoint{1.764706in}{1.700000in}}%
\pgfusepath{clip}%
\pgfsetbuttcap%
\pgfsetroundjoin%
\definecolor{currentfill}{rgb}{0.975018,0.868213,0.788710}%
\pgfsetfillcolor{currentfill}%
\pgfsetlinewidth{0.311001pt}%
\definecolor{currentstroke}{rgb}{1.000000,1.000000,1.000000}%
\pgfsetstrokecolor{currentstroke}%
\pgfsetdash{}{0pt}%
\pgfpathmoveto{\pgfqpoint{5.482632in}{1.132118in}}%
\pgfpathcurveto{\pgfqpoint{5.489765in}{1.132118in}}{\pgfqpoint{5.496606in}{1.134952in}}{\pgfqpoint{5.501650in}{1.139996in}}%
\pgfpathcurveto{\pgfqpoint{5.506694in}{1.145040in}}{\pgfqpoint{5.509528in}{1.151881in}}{\pgfqpoint{5.509528in}{1.159014in}}%
\pgfpathcurveto{\pgfqpoint{5.509528in}{1.166147in}}{\pgfqpoint{5.506694in}{1.172988in}}{\pgfqpoint{5.501650in}{1.178032in}}%
\pgfpathcurveto{\pgfqpoint{5.496606in}{1.183076in}}{\pgfqpoint{5.489765in}{1.185910in}}{\pgfqpoint{5.482632in}{1.185910in}}%
\pgfpathcurveto{\pgfqpoint{5.475499in}{1.185910in}}{\pgfqpoint{5.468657in}{1.183076in}}{\pgfqpoint{5.463614in}{1.178032in}}%
\pgfpathcurveto{\pgfqpoint{5.458570in}{1.172988in}}{\pgfqpoint{5.455736in}{1.166147in}}{\pgfqpoint{5.455736in}{1.159014in}}%
\pgfpathcurveto{\pgfqpoint{5.455736in}{1.151881in}}{\pgfqpoint{5.458570in}{1.145040in}}{\pgfqpoint{5.463614in}{1.139996in}}%
\pgfpathcurveto{\pgfqpoint{5.468657in}{1.134952in}}{\pgfqpoint{5.475499in}{1.132118in}}{\pgfqpoint{5.482632in}{1.132118in}}%
\pgfpathclose%
\pgfusepath{stroke,fill}%
\end{pgfscope}%
\begin{pgfscope}%
\pgfpathrectangle{\pgfqpoint{4.985294in}{0.500000in}}{\pgfqpoint{1.764706in}{1.700000in}}%
\pgfusepath{clip}%
\pgfsetbuttcap%
\pgfsetroundjoin%
\definecolor{currentfill}{rgb}{0.950851,0.435000,0.297228}%
\pgfsetfillcolor{currentfill}%
\pgfsetlinewidth{0.311001pt}%
\definecolor{currentstroke}{rgb}{1.000000,1.000000,1.000000}%
\pgfsetstrokecolor{currentstroke}%
\pgfsetdash{}{0pt}%
\pgfpathmoveto{\pgfqpoint{6.351440in}{1.689398in}}%
\pgfpathcurveto{\pgfqpoint{6.358573in}{1.689398in}}{\pgfqpoint{6.365414in}{1.692232in}}{\pgfqpoint{6.370458in}{1.697276in}}%
\pgfpathcurveto{\pgfqpoint{6.375502in}{1.702320in}}{\pgfqpoint{6.378336in}{1.709161in}}{\pgfqpoint{6.378336in}{1.716294in}}%
\pgfpathcurveto{\pgfqpoint{6.378336in}{1.723427in}}{\pgfqpoint{6.375502in}{1.730268in}}{\pgfqpoint{6.370458in}{1.735312in}}%
\pgfpathcurveto{\pgfqpoint{6.365414in}{1.740356in}}{\pgfqpoint{6.358573in}{1.743190in}}{\pgfqpoint{6.351440in}{1.743190in}}%
\pgfpathcurveto{\pgfqpoint{6.344307in}{1.743190in}}{\pgfqpoint{6.337465in}{1.740356in}}{\pgfqpoint{6.332422in}{1.735312in}}%
\pgfpathcurveto{\pgfqpoint{6.327378in}{1.730268in}}{\pgfqpoint{6.324544in}{1.723427in}}{\pgfqpoint{6.324544in}{1.716294in}}%
\pgfpathcurveto{\pgfqpoint{6.324544in}{1.709161in}}{\pgfqpoint{6.327378in}{1.702320in}}{\pgfqpoint{6.332422in}{1.697276in}}%
\pgfpathcurveto{\pgfqpoint{6.337465in}{1.692232in}}{\pgfqpoint{6.344307in}{1.689398in}}{\pgfqpoint{6.351440in}{1.689398in}}%
\pgfpathclose%
\pgfusepath{stroke,fill}%
\end{pgfscope}%
\begin{pgfscope}%
\pgfpathrectangle{\pgfqpoint{4.985294in}{0.500000in}}{\pgfqpoint{1.764706in}{1.700000in}}%
\pgfusepath{clip}%
\pgfsetbuttcap%
\pgfsetroundjoin%
\definecolor{currentfill}{rgb}{0.968931,0.798091,0.685123}%
\pgfsetfillcolor{currentfill}%
\pgfsetlinewidth{0.311001pt}%
\definecolor{currentstroke}{rgb}{1.000000,1.000000,1.000000}%
\pgfsetstrokecolor{currentstroke}%
\pgfsetdash{}{0pt}%
\pgfpathmoveto{\pgfqpoint{6.363144in}{1.471940in}}%
\pgfpathcurveto{\pgfqpoint{6.370277in}{1.471940in}}{\pgfqpoint{6.377119in}{1.474774in}}{\pgfqpoint{6.382162in}{1.479818in}}%
\pgfpathcurveto{\pgfqpoint{6.387206in}{1.484861in}}{\pgfqpoint{6.390040in}{1.491703in}}{\pgfqpoint{6.390040in}{1.498836in}}%
\pgfpathcurveto{\pgfqpoint{6.390040in}{1.505969in}}{\pgfqpoint{6.387206in}{1.512810in}}{\pgfqpoint{6.382162in}{1.517854in}}%
\pgfpathcurveto{\pgfqpoint{6.377119in}{1.522898in}}{\pgfqpoint{6.370277in}{1.525732in}}{\pgfqpoint{6.363144in}{1.525732in}}%
\pgfpathcurveto{\pgfqpoint{6.356011in}{1.525732in}}{\pgfqpoint{6.349170in}{1.522898in}}{\pgfqpoint{6.344126in}{1.517854in}}%
\pgfpathcurveto{\pgfqpoint{6.339082in}{1.512810in}}{\pgfqpoint{6.336248in}{1.505969in}}{\pgfqpoint{6.336248in}{1.498836in}}%
\pgfpathcurveto{\pgfqpoint{6.336248in}{1.491703in}}{\pgfqpoint{6.339082in}{1.484861in}}{\pgfqpoint{6.344126in}{1.479818in}}%
\pgfpathcurveto{\pgfqpoint{6.349170in}{1.474774in}}{\pgfqpoint{6.356011in}{1.471940in}}{\pgfqpoint{6.363144in}{1.471940in}}%
\pgfpathclose%
\pgfusepath{stroke,fill}%
\end{pgfscope}%
\begin{pgfscope}%
\pgfpathrectangle{\pgfqpoint{4.985294in}{0.500000in}}{\pgfqpoint{1.764706in}{1.700000in}}%
\pgfusepath{clip}%
\pgfsetbuttcap%
\pgfsetroundjoin%
\definecolor{currentfill}{rgb}{0.962283,0.593046,0.431453}%
\pgfsetfillcolor{currentfill}%
\pgfsetlinewidth{0.311001pt}%
\definecolor{currentstroke}{rgb}{1.000000,1.000000,1.000000}%
\pgfsetstrokecolor{currentstroke}%
\pgfsetdash{}{0pt}%
\pgfpathmoveto{\pgfqpoint{6.136754in}{1.518821in}}%
\pgfpathcurveto{\pgfqpoint{6.143887in}{1.518821in}}{\pgfqpoint{6.150729in}{1.521655in}}{\pgfqpoint{6.155773in}{1.526698in}}%
\pgfpathcurveto{\pgfqpoint{6.160816in}{1.531742in}}{\pgfqpoint{6.163650in}{1.538584in}}{\pgfqpoint{6.163650in}{1.545716in}}%
\pgfpathcurveto{\pgfqpoint{6.163650in}{1.552849in}}{\pgfqpoint{6.160816in}{1.559691in}}{\pgfqpoint{6.155773in}{1.564735in}}%
\pgfpathcurveto{\pgfqpoint{6.150729in}{1.569778in}}{\pgfqpoint{6.143887in}{1.572612in}}{\pgfqpoint{6.136754in}{1.572612in}}%
\pgfpathcurveto{\pgfqpoint{6.129622in}{1.572612in}}{\pgfqpoint{6.122780in}{1.569778in}}{\pgfqpoint{6.117736in}{1.564735in}}%
\pgfpathcurveto{\pgfqpoint{6.112693in}{1.559691in}}{\pgfqpoint{6.109859in}{1.552849in}}{\pgfqpoint{6.109859in}{1.545716in}}%
\pgfpathcurveto{\pgfqpoint{6.109859in}{1.538584in}}{\pgfqpoint{6.112693in}{1.531742in}}{\pgfqpoint{6.117736in}{1.526698in}}%
\pgfpathcurveto{\pgfqpoint{6.122780in}{1.521655in}}{\pgfqpoint{6.129622in}{1.518821in}}{\pgfqpoint{6.136754in}{1.518821in}}%
\pgfpathclose%
\pgfusepath{stroke,fill}%
\end{pgfscope}%
\begin{pgfscope}%
\pgfpathrectangle{\pgfqpoint{4.985294in}{0.500000in}}{\pgfqpoint{1.764706in}{1.700000in}}%
\pgfusepath{clip}%
\pgfsetbuttcap%
\pgfsetroundjoin%
\definecolor{currentfill}{rgb}{0.966560,0.756582,0.625273}%
\pgfsetfillcolor{currentfill}%
\pgfsetlinewidth{0.311001pt}%
\definecolor{currentstroke}{rgb}{1.000000,1.000000,1.000000}%
\pgfsetstrokecolor{currentstroke}%
\pgfsetdash{}{0pt}%
\pgfpathmoveto{\pgfqpoint{5.449451in}{0.970324in}}%
\pgfpathcurveto{\pgfqpoint{5.456584in}{0.970324in}}{\pgfqpoint{5.463426in}{0.973158in}}{\pgfqpoint{5.468469in}{0.978201in}}%
\pgfpathcurveto{\pgfqpoint{5.473513in}{0.983245in}}{\pgfqpoint{5.476347in}{0.990087in}}{\pgfqpoint{5.476347in}{0.997220in}}%
\pgfpathcurveto{\pgfqpoint{5.476347in}{1.004352in}}{\pgfqpoint{5.473513in}{1.011194in}}{\pgfqpoint{5.468469in}{1.016238in}}%
\pgfpathcurveto{\pgfqpoint{5.463426in}{1.021281in}}{\pgfqpoint{5.456584in}{1.024115in}}{\pgfqpoint{5.449451in}{1.024115in}}%
\pgfpathcurveto{\pgfqpoint{5.442318in}{1.024115in}}{\pgfqpoint{5.435477in}{1.021281in}}{\pgfqpoint{5.430433in}{1.016238in}}%
\pgfpathcurveto{\pgfqpoint{5.425389in}{1.011194in}}{\pgfqpoint{5.422555in}{1.004352in}}{\pgfqpoint{5.422555in}{0.997220in}}%
\pgfpathcurveto{\pgfqpoint{5.422555in}{0.990087in}}{\pgfqpoint{5.425389in}{0.983245in}}{\pgfqpoint{5.430433in}{0.978201in}}%
\pgfpathcurveto{\pgfqpoint{5.435477in}{0.973158in}}{\pgfqpoint{5.442318in}{0.970324in}}{\pgfqpoint{5.449451in}{0.970324in}}%
\pgfpathclose%
\pgfusepath{stroke,fill}%
\end{pgfscope}%
\begin{pgfscope}%
\pgfpathrectangle{\pgfqpoint{4.985294in}{0.500000in}}{\pgfqpoint{1.764706in}{1.700000in}}%
\pgfusepath{clip}%
\pgfsetbuttcap%
\pgfsetroundjoin%
\definecolor{currentfill}{rgb}{0.953816,0.463738,0.317699}%
\pgfsetfillcolor{currentfill}%
\pgfsetlinewidth{0.311001pt}%
\definecolor{currentstroke}{rgb}{1.000000,1.000000,1.000000}%
\pgfsetstrokecolor{currentstroke}%
\pgfsetdash{}{0pt}%
\pgfpathmoveto{\pgfqpoint{5.369349in}{1.648261in}}%
\pgfpathcurveto{\pgfqpoint{5.376482in}{1.648261in}}{\pgfqpoint{5.383323in}{1.651095in}}{\pgfqpoint{5.388367in}{1.656139in}}%
\pgfpathcurveto{\pgfqpoint{5.393411in}{1.661182in}}{\pgfqpoint{5.396245in}{1.668024in}}{\pgfqpoint{5.396245in}{1.675157in}}%
\pgfpathcurveto{\pgfqpoint{5.396245in}{1.682290in}}{\pgfqpoint{5.393411in}{1.689131in}}{\pgfqpoint{5.388367in}{1.694175in}}%
\pgfpathcurveto{\pgfqpoint{5.383323in}{1.699219in}}{\pgfqpoint{5.376482in}{1.702053in}}{\pgfqpoint{5.369349in}{1.702053in}}%
\pgfpathcurveto{\pgfqpoint{5.362216in}{1.702053in}}{\pgfqpoint{5.355374in}{1.699219in}}{\pgfqpoint{5.350331in}{1.694175in}}%
\pgfpathcurveto{\pgfqpoint{5.345287in}{1.689131in}}{\pgfqpoint{5.342453in}{1.682290in}}{\pgfqpoint{5.342453in}{1.675157in}}%
\pgfpathcurveto{\pgfqpoint{5.342453in}{1.668024in}}{\pgfqpoint{5.345287in}{1.661182in}}{\pgfqpoint{5.350331in}{1.656139in}}%
\pgfpathcurveto{\pgfqpoint{5.355374in}{1.651095in}}{\pgfqpoint{5.362216in}{1.648261in}}{\pgfqpoint{5.369349in}{1.648261in}}%
\pgfpathclose%
\pgfusepath{stroke,fill}%
\end{pgfscope}%
\begin{pgfscope}%
\pgfpathrectangle{\pgfqpoint{4.985294in}{0.500000in}}{\pgfqpoint{1.764706in}{1.700000in}}%
\pgfusepath{clip}%
\pgfsetbuttcap%
\pgfsetroundjoin%
\definecolor{currentfill}{rgb}{0.979891,0.908948,0.848279}%
\pgfsetfillcolor{currentfill}%
\pgfsetlinewidth{0.311001pt}%
\definecolor{currentstroke}{rgb}{1.000000,1.000000,1.000000}%
\pgfsetstrokecolor{currentstroke}%
\pgfsetdash{}{0pt}%
\pgfpathmoveto{\pgfqpoint{6.341865in}{1.327498in}}%
\pgfpathcurveto{\pgfqpoint{6.348998in}{1.327498in}}{\pgfqpoint{6.355840in}{1.330332in}}{\pgfqpoint{6.360883in}{1.335376in}}%
\pgfpathcurveto{\pgfqpoint{6.365927in}{1.340419in}}{\pgfqpoint{6.368761in}{1.347261in}}{\pgfqpoint{6.368761in}{1.354394in}}%
\pgfpathcurveto{\pgfqpoint{6.368761in}{1.361527in}}{\pgfqpoint{6.365927in}{1.368368in}}{\pgfqpoint{6.360883in}{1.373412in}}%
\pgfpathcurveto{\pgfqpoint{6.355840in}{1.378456in}}{\pgfqpoint{6.348998in}{1.381289in}}{\pgfqpoint{6.341865in}{1.381289in}}%
\pgfpathcurveto{\pgfqpoint{6.334732in}{1.381289in}}{\pgfqpoint{6.327891in}{1.378456in}}{\pgfqpoint{6.322847in}{1.373412in}}%
\pgfpathcurveto{\pgfqpoint{6.317803in}{1.368368in}}{\pgfqpoint{6.314969in}{1.361527in}}{\pgfqpoint{6.314969in}{1.354394in}}%
\pgfpathcurveto{\pgfqpoint{6.314969in}{1.347261in}}{\pgfqpoint{6.317803in}{1.340419in}}{\pgfqpoint{6.322847in}{1.335376in}}%
\pgfpathcurveto{\pgfqpoint{6.327891in}{1.330332in}}{\pgfqpoint{6.334732in}{1.327498in}}{\pgfqpoint{6.341865in}{1.327498in}}%
\pgfpathclose%
\pgfusepath{stroke,fill}%
\end{pgfscope}%
\begin{pgfscope}%
\pgfpathrectangle{\pgfqpoint{4.985294in}{0.500000in}}{\pgfqpoint{1.764706in}{1.700000in}}%
\pgfusepath{clip}%
\pgfsetbuttcap%
\pgfsetroundjoin%
\definecolor{currentfill}{rgb}{0.977657,0.891500,0.822809}%
\pgfsetfillcolor{currentfill}%
\pgfsetlinewidth{0.311001pt}%
\definecolor{currentstroke}{rgb}{1.000000,1.000000,1.000000}%
\pgfsetstrokecolor{currentstroke}%
\pgfsetdash{}{0pt}%
\pgfpathmoveto{\pgfqpoint{5.454687in}{1.463079in}}%
\pgfpathcurveto{\pgfqpoint{5.461820in}{1.463079in}}{\pgfqpoint{5.468661in}{1.465913in}}{\pgfqpoint{5.473705in}{1.470956in}}%
\pgfpathcurveto{\pgfqpoint{5.478749in}{1.476000in}}{\pgfqpoint{5.481583in}{1.482842in}}{\pgfqpoint{5.481583in}{1.489974in}}%
\pgfpathcurveto{\pgfqpoint{5.481583in}{1.497107in}}{\pgfqpoint{5.478749in}{1.503949in}}{\pgfqpoint{5.473705in}{1.508993in}}%
\pgfpathcurveto{\pgfqpoint{5.468661in}{1.514036in}}{\pgfqpoint{5.461820in}{1.516870in}}{\pgfqpoint{5.454687in}{1.516870in}}%
\pgfpathcurveto{\pgfqpoint{5.447554in}{1.516870in}}{\pgfqpoint{5.440712in}{1.514036in}}{\pgfqpoint{5.435669in}{1.508993in}}%
\pgfpathcurveto{\pgfqpoint{5.430625in}{1.503949in}}{\pgfqpoint{5.427791in}{1.497107in}}{\pgfqpoint{5.427791in}{1.489974in}}%
\pgfpathcurveto{\pgfqpoint{5.427791in}{1.482842in}}{\pgfqpoint{5.430625in}{1.476000in}}{\pgfqpoint{5.435669in}{1.470956in}}%
\pgfpathcurveto{\pgfqpoint{5.440712in}{1.465913in}}{\pgfqpoint{5.447554in}{1.463079in}}{\pgfqpoint{5.454687in}{1.463079in}}%
\pgfpathclose%
\pgfusepath{stroke,fill}%
\end{pgfscope}%
\begin{pgfscope}%
\pgfpathrectangle{\pgfqpoint{4.985294in}{0.500000in}}{\pgfqpoint{1.764706in}{1.700000in}}%
\pgfusepath{clip}%
\pgfsetbuttcap%
\pgfsetroundjoin%
\definecolor{currentfill}{rgb}{0.965042,0.701564,0.552889}%
\pgfsetfillcolor{currentfill}%
\pgfsetlinewidth{0.311001pt}%
\definecolor{currentstroke}{rgb}{1.000000,1.000000,1.000000}%
\pgfsetstrokecolor{currentstroke}%
\pgfsetdash{}{0pt}%
\pgfpathmoveto{\pgfqpoint{5.365939in}{1.538191in}}%
\pgfpathcurveto{\pgfqpoint{5.373072in}{1.538191in}}{\pgfqpoint{5.379914in}{1.541025in}}{\pgfqpoint{5.384957in}{1.546069in}}%
\pgfpathcurveto{\pgfqpoint{5.390001in}{1.551112in}}{\pgfqpoint{5.392835in}{1.557954in}}{\pgfqpoint{5.392835in}{1.565087in}}%
\pgfpathcurveto{\pgfqpoint{5.392835in}{1.572219in}}{\pgfqpoint{5.390001in}{1.579061in}}{\pgfqpoint{5.384957in}{1.584105in}}%
\pgfpathcurveto{\pgfqpoint{5.379914in}{1.589148in}}{\pgfqpoint{5.373072in}{1.591982in}}{\pgfqpoint{5.365939in}{1.591982in}}%
\pgfpathcurveto{\pgfqpoint{5.358806in}{1.591982in}}{\pgfqpoint{5.351965in}{1.589148in}}{\pgfqpoint{5.346921in}{1.584105in}}%
\pgfpathcurveto{\pgfqpoint{5.341877in}{1.579061in}}{\pgfqpoint{5.339044in}{1.572219in}}{\pgfqpoint{5.339044in}{1.565087in}}%
\pgfpathcurveto{\pgfqpoint{5.339044in}{1.557954in}}{\pgfqpoint{5.341877in}{1.551112in}}{\pgfqpoint{5.346921in}{1.546069in}}%
\pgfpathcurveto{\pgfqpoint{5.351965in}{1.541025in}}{\pgfqpoint{5.358806in}{1.538191in}}{\pgfqpoint{5.365939in}{1.538191in}}%
\pgfpathclose%
\pgfusepath{stroke,fill}%
\end{pgfscope}%
\begin{pgfscope}%
\pgfpathrectangle{\pgfqpoint{4.985294in}{0.500000in}}{\pgfqpoint{1.764706in}{1.700000in}}%
\pgfusepath{clip}%
\pgfsetbuttcap%
\pgfsetroundjoin%
\definecolor{currentfill}{rgb}{0.958791,0.526283,0.368909}%
\pgfsetfillcolor{currentfill}%
\pgfsetlinewidth{0.311001pt}%
\definecolor{currentstroke}{rgb}{1.000000,1.000000,1.000000}%
\pgfsetstrokecolor{currentstroke}%
\pgfsetdash{}{0pt}%
\pgfpathmoveto{\pgfqpoint{6.422250in}{1.476573in}}%
\pgfpathcurveto{\pgfqpoint{6.429383in}{1.476573in}}{\pgfqpoint{6.436225in}{1.479407in}}{\pgfqpoint{6.441268in}{1.484450in}}%
\pgfpathcurveto{\pgfqpoint{6.446312in}{1.489494in}}{\pgfqpoint{6.449146in}{1.496336in}}{\pgfqpoint{6.449146in}{1.503469in}}%
\pgfpathcurveto{\pgfqpoint{6.449146in}{1.510601in}}{\pgfqpoint{6.446312in}{1.517443in}}{\pgfqpoint{6.441268in}{1.522487in}}%
\pgfpathcurveto{\pgfqpoint{6.436225in}{1.527530in}}{\pgfqpoint{6.429383in}{1.530364in}}{\pgfqpoint{6.422250in}{1.530364in}}%
\pgfpathcurveto{\pgfqpoint{6.415117in}{1.530364in}}{\pgfqpoint{6.408276in}{1.527530in}}{\pgfqpoint{6.403232in}{1.522487in}}%
\pgfpathcurveto{\pgfqpoint{6.398188in}{1.517443in}}{\pgfqpoint{6.395354in}{1.510601in}}{\pgfqpoint{6.395354in}{1.503469in}}%
\pgfpathcurveto{\pgfqpoint{6.395354in}{1.496336in}}{\pgfqpoint{6.398188in}{1.489494in}}{\pgfqpoint{6.403232in}{1.484450in}}%
\pgfpathcurveto{\pgfqpoint{6.408276in}{1.479407in}}{\pgfqpoint{6.415117in}{1.476573in}}{\pgfqpoint{6.422250in}{1.476573in}}%
\pgfpathclose%
\pgfusepath{stroke,fill}%
\end{pgfscope}%
\begin{pgfscope}%
\pgfpathrectangle{\pgfqpoint{4.985294in}{0.500000in}}{\pgfqpoint{1.764706in}{1.700000in}}%
\pgfusepath{clip}%
\pgfsetbuttcap%
\pgfsetroundjoin%
\definecolor{currentfill}{rgb}{0.287537,0.111919,0.286609}%
\pgfsetfillcolor{currentfill}%
\pgfsetlinewidth{0.803000pt}%
\definecolor{currentstroke}{rgb}{0.287537,0.111919,0.286609}%
\pgfsetstrokecolor{currentstroke}%
\pgfsetdash{}{0pt}%
\pgfsys@defobject{currentmarker}{\pgfqpoint{-0.033333in}{-0.033333in}}{\pgfqpoint{0.033333in}{0.033333in}}{%
\pgfpathmoveto{\pgfqpoint{0.000000in}{-0.033333in}}%
\pgfpathcurveto{\pgfqpoint{0.008840in}{-0.033333in}}{\pgfqpoint{0.017319in}{-0.029821in}}{\pgfqpoint{0.023570in}{-0.023570in}}%
\pgfpathcurveto{\pgfqpoint{0.029821in}{-0.017319in}}{\pgfqpoint{0.033333in}{-0.008840in}}{\pgfqpoint{0.033333in}{0.000000in}}%
\pgfpathcurveto{\pgfqpoint{0.033333in}{0.008840in}}{\pgfqpoint{0.029821in}{0.017319in}}{\pgfqpoint{0.023570in}{0.023570in}}%
\pgfpathcurveto{\pgfqpoint{0.017319in}{0.029821in}}{\pgfqpoint{0.008840in}{0.033333in}}{\pgfqpoint{0.000000in}{0.033333in}}%
\pgfpathcurveto{\pgfqpoint{-0.008840in}{0.033333in}}{\pgfqpoint{-0.017319in}{0.029821in}}{\pgfqpoint{-0.023570in}{0.023570in}}%
\pgfpathcurveto{\pgfqpoint{-0.029821in}{0.017319in}}{\pgfqpoint{-0.033333in}{0.008840in}}{\pgfqpoint{-0.033333in}{0.000000in}}%
\pgfpathcurveto{\pgfqpoint{-0.033333in}{-0.008840in}}{\pgfqpoint{-0.029821in}{-0.017319in}}{\pgfqpoint{-0.023570in}{-0.023570in}}%
\pgfpathcurveto{\pgfqpoint{-0.017319in}{-0.029821in}}{\pgfqpoint{-0.008840in}{-0.033333in}}{\pgfqpoint{0.000000in}{-0.033333in}}%
\pgfpathclose%
\pgfusepath{stroke,fill}%
}%
\end{pgfscope}%
\begin{pgfscope}%
\pgfpathrectangle{\pgfqpoint{4.985294in}{0.500000in}}{\pgfqpoint{1.764706in}{1.700000in}}%
\pgfusepath{clip}%
\pgfsetbuttcap%
\pgfsetroundjoin%
\definecolor{currentfill}{rgb}{0.638121,0.099382,0.356038}%
\pgfsetfillcolor{currentfill}%
\pgfsetlinewidth{0.803000pt}%
\definecolor{currentstroke}{rgb}{0.638121,0.099382,0.356038}%
\pgfsetstrokecolor{currentstroke}%
\pgfsetdash{}{0pt}%
\pgfsys@defobject{currentmarker}{\pgfqpoint{-0.033333in}{-0.033333in}}{\pgfqpoint{0.033333in}{0.033333in}}{%
\pgfpathmoveto{\pgfqpoint{0.000000in}{-0.033333in}}%
\pgfpathcurveto{\pgfqpoint{0.008840in}{-0.033333in}}{\pgfqpoint{0.017319in}{-0.029821in}}{\pgfqpoint{0.023570in}{-0.023570in}}%
\pgfpathcurveto{\pgfqpoint{0.029821in}{-0.017319in}}{\pgfqpoint{0.033333in}{-0.008840in}}{\pgfqpoint{0.033333in}{0.000000in}}%
\pgfpathcurveto{\pgfqpoint{0.033333in}{0.008840in}}{\pgfqpoint{0.029821in}{0.017319in}}{\pgfqpoint{0.023570in}{0.023570in}}%
\pgfpathcurveto{\pgfqpoint{0.017319in}{0.029821in}}{\pgfqpoint{0.008840in}{0.033333in}}{\pgfqpoint{0.000000in}{0.033333in}}%
\pgfpathcurveto{\pgfqpoint{-0.008840in}{0.033333in}}{\pgfqpoint{-0.017319in}{0.029821in}}{\pgfqpoint{-0.023570in}{0.023570in}}%
\pgfpathcurveto{\pgfqpoint{-0.029821in}{0.017319in}}{\pgfqpoint{-0.033333in}{0.008840in}}{\pgfqpoint{-0.033333in}{0.000000in}}%
\pgfpathcurveto{\pgfqpoint{-0.033333in}{-0.008840in}}{\pgfqpoint{-0.029821in}{-0.017319in}}{\pgfqpoint{-0.023570in}{-0.023570in}}%
\pgfpathcurveto{\pgfqpoint{-0.017319in}{-0.029821in}}{\pgfqpoint{-0.008840in}{-0.033333in}}{\pgfqpoint{0.000000in}{-0.033333in}}%
\pgfpathclose%
\pgfusepath{stroke,fill}%
}%
\end{pgfscope}%
\begin{pgfscope}%
\pgfpathrectangle{\pgfqpoint{4.985294in}{0.500000in}}{\pgfqpoint{1.764706in}{1.700000in}}%
\pgfusepath{clip}%
\pgfsetbuttcap%
\pgfsetroundjoin%
\definecolor{currentfill}{rgb}{0.919781,0.275262,0.242460}%
\pgfsetfillcolor{currentfill}%
\pgfsetlinewidth{0.803000pt}%
\definecolor{currentstroke}{rgb}{0.919781,0.275262,0.242460}%
\pgfsetstrokecolor{currentstroke}%
\pgfsetdash{}{0pt}%
\pgfsys@defobject{currentmarker}{\pgfqpoint{-0.033333in}{-0.033333in}}{\pgfqpoint{0.033333in}{0.033333in}}{%
\pgfpathmoveto{\pgfqpoint{0.000000in}{-0.033333in}}%
\pgfpathcurveto{\pgfqpoint{0.008840in}{-0.033333in}}{\pgfqpoint{0.017319in}{-0.029821in}}{\pgfqpoint{0.023570in}{-0.023570in}}%
\pgfpathcurveto{\pgfqpoint{0.029821in}{-0.017319in}}{\pgfqpoint{0.033333in}{-0.008840in}}{\pgfqpoint{0.033333in}{0.000000in}}%
\pgfpathcurveto{\pgfqpoint{0.033333in}{0.008840in}}{\pgfqpoint{0.029821in}{0.017319in}}{\pgfqpoint{0.023570in}{0.023570in}}%
\pgfpathcurveto{\pgfqpoint{0.017319in}{0.029821in}}{\pgfqpoint{0.008840in}{0.033333in}}{\pgfqpoint{0.000000in}{0.033333in}}%
\pgfpathcurveto{\pgfqpoint{-0.008840in}{0.033333in}}{\pgfqpoint{-0.017319in}{0.029821in}}{\pgfqpoint{-0.023570in}{0.023570in}}%
\pgfpathcurveto{\pgfqpoint{-0.029821in}{0.017319in}}{\pgfqpoint{-0.033333in}{0.008840in}}{\pgfqpoint{-0.033333in}{0.000000in}}%
\pgfpathcurveto{\pgfqpoint{-0.033333in}{-0.008840in}}{\pgfqpoint{-0.029821in}{-0.017319in}}{\pgfqpoint{-0.023570in}{-0.023570in}}%
\pgfpathcurveto{\pgfqpoint{-0.017319in}{-0.029821in}}{\pgfqpoint{-0.008840in}{-0.033333in}}{\pgfqpoint{0.000000in}{-0.033333in}}%
\pgfpathclose%
\pgfusepath{stroke,fill}%
}%
\end{pgfscope}%
\begin{pgfscope}%
\pgfpathrectangle{\pgfqpoint{4.985294in}{0.500000in}}{\pgfqpoint{1.764706in}{1.700000in}}%
\pgfusepath{clip}%
\pgfsetbuttcap%
\pgfsetroundjoin%
\definecolor{currentfill}{rgb}{0.964306,0.663930,0.507747}%
\pgfsetfillcolor{currentfill}%
\pgfsetlinewidth{0.803000pt}%
\definecolor{currentstroke}{rgb}{0.964306,0.663930,0.507747}%
\pgfsetstrokecolor{currentstroke}%
\pgfsetdash{}{0pt}%
\pgfsys@defobject{currentmarker}{\pgfqpoint{-0.033333in}{-0.033333in}}{\pgfqpoint{0.033333in}{0.033333in}}{%
\pgfpathmoveto{\pgfqpoint{0.000000in}{-0.033333in}}%
\pgfpathcurveto{\pgfqpoint{0.008840in}{-0.033333in}}{\pgfqpoint{0.017319in}{-0.029821in}}{\pgfqpoint{0.023570in}{-0.023570in}}%
\pgfpathcurveto{\pgfqpoint{0.029821in}{-0.017319in}}{\pgfqpoint{0.033333in}{-0.008840in}}{\pgfqpoint{0.033333in}{0.000000in}}%
\pgfpathcurveto{\pgfqpoint{0.033333in}{0.008840in}}{\pgfqpoint{0.029821in}{0.017319in}}{\pgfqpoint{0.023570in}{0.023570in}}%
\pgfpathcurveto{\pgfqpoint{0.017319in}{0.029821in}}{\pgfqpoint{0.008840in}{0.033333in}}{\pgfqpoint{0.000000in}{0.033333in}}%
\pgfpathcurveto{\pgfqpoint{-0.008840in}{0.033333in}}{\pgfqpoint{-0.017319in}{0.029821in}}{\pgfqpoint{-0.023570in}{0.023570in}}%
\pgfpathcurveto{\pgfqpoint{-0.029821in}{0.017319in}}{\pgfqpoint{-0.033333in}{0.008840in}}{\pgfqpoint{-0.033333in}{0.000000in}}%
\pgfpathcurveto{\pgfqpoint{-0.033333in}{-0.008840in}}{\pgfqpoint{-0.029821in}{-0.017319in}}{\pgfqpoint{-0.023570in}{-0.023570in}}%
\pgfpathcurveto{\pgfqpoint{-0.017319in}{-0.029821in}}{\pgfqpoint{-0.008840in}{-0.033333in}}{\pgfqpoint{0.000000in}{-0.033333in}}%
\pgfpathclose%
\pgfusepath{stroke,fill}%
}%
\end{pgfscope}%
\begin{pgfscope}%
\pgfsetrectcap%
\pgfsetmiterjoin%
\pgfsetlinewidth{1.003750pt}%
\definecolor{currentstroke}{rgb}{0.150000,0.150000,0.150000}%
\pgfsetstrokecolor{currentstroke}%
\pgfsetdash{}{0pt}%
\pgfpathmoveto{\pgfqpoint{4.985294in}{0.500000in}}%
\pgfpathlineto{\pgfqpoint{4.985294in}{2.200000in}}%
\pgfusepath{stroke}%
\end{pgfscope}%
\begin{pgfscope}%
\pgfsetrectcap%
\pgfsetmiterjoin%
\pgfsetlinewidth{1.003750pt}%
\definecolor{currentstroke}{rgb}{0.150000,0.150000,0.150000}%
\pgfsetstrokecolor{currentstroke}%
\pgfsetdash{}{0pt}%
\pgfpathmoveto{\pgfqpoint{6.750000in}{0.500000in}}%
\pgfpathlineto{\pgfqpoint{6.750000in}{2.200000in}}%
\pgfusepath{stroke}%
\end{pgfscope}%
\begin{pgfscope}%
\pgfsetrectcap%
\pgfsetmiterjoin%
\pgfsetlinewidth{1.003750pt}%
\definecolor{currentstroke}{rgb}{0.150000,0.150000,0.150000}%
\pgfsetstrokecolor{currentstroke}%
\pgfsetdash{}{0pt}%
\pgfpathmoveto{\pgfqpoint{4.985294in}{0.500000in}}%
\pgfpathlineto{\pgfqpoint{6.750000in}{0.500000in}}%
\pgfusepath{stroke}%
\end{pgfscope}%
\begin{pgfscope}%
\pgfsetrectcap%
\pgfsetmiterjoin%
\pgfsetlinewidth{1.003750pt}%
\definecolor{currentstroke}{rgb}{0.150000,0.150000,0.150000}%
\pgfsetstrokecolor{currentstroke}%
\pgfsetdash{}{0pt}%
\pgfpathmoveto{\pgfqpoint{4.985294in}{2.200000in}}%
\pgfpathlineto{\pgfqpoint{6.750000in}{2.200000in}}%
\pgfusepath{stroke}%
\end{pgfscope}%
\begin{pgfscope}%
\definecolor{textcolor}{rgb}{0.150000,0.150000,0.150000}%
\pgfsetstrokecolor{textcolor}%
\pgfsetfillcolor{textcolor}%
\pgftext[x=5.867647in,y=2.283333in,,base]{\color{textcolor}\rmfamily\fontsize{9.600000}{11.520000}\selectfont Iteration 6000}%
\end{pgfscope}%
\end{pgfpicture}%
\makeatother%
\endgroup%

%     %  \includegraphics{figures/elbow_example/elbow.pdf}
%     \caption{Three simple graphs}
%     \label{fig:three graphs}
%     \end{adjustwidth}
% \end{figure}

% \begin{figure}
%     \begin{adjustwidth}{-2cm}{-2cm}
%     \centering
%     \begin{subfigure}[t!]{0.3\linewidth}
%         \centering
%          % This file was created with tikzplotlib v0.9.15.
\begin{tikzpicture}

\begin{axis}[
axis line style={white!80!black},
height=\figureheight,
tick align=outside,
tick pos=left,
width=\figurewidth,
x grid style={white!80!black},
xlabel=\textcolor{white!15!black}{Iterations},
xmajorgrids,
xmin=4500, xmax=6000,
xtick style={draw=none},
y grid style={white!80!black},
ylabel=\textcolor{white!15!black}{ELBO},
ymajorgrids,
ymin=-50, ymax=350,
ytick style={draw=none}
]
\addplot [line width=0.48pt, white!10!black]
table {%
4500 101.750427246094
4501 67.9201354980469
4502 92.0496826171875
4503 73.9893951416016
4504 116.447143554688
4505 76.0513305664062
4506 68.4491577148438
4507 114.679962158203
4508 118.37451171875
4509 130.132781982422
4510 114.398162841797
4511 87.0838012695312
4512 110.592422485352
4513 121.500778198242
4514 98.3008575439453
4515 101.775039672852
4516 95.0255737304688
4517 83.6160583496094
4518 86.1577301025391
4519 89.2458190917969
4520 101.25617980957
4521 81.4906311035156
4522 57.1505432128906
4523 124.393737792969
4524 96.6603851318359
4525 106.896377563477
4526 92.2795867919922
4527 117.034759521484
4529 86.7299652099609
4530 111.980850219727
4531 108.337600708008
4532 90.7833404541016
4533 102.016326904297
4534 93.5084228515625
4535 75.1885375976562
4536 107.309844970703
4537 93.5156097412109
4538 83.0310516357422
4539 98.8274230957031
4540 56.7922515869141
4541 108.004165649414
4543 90.5948028564453
4544 97.3419952392578
4545 78.5833435058594
4546 98.4247894287109
4547 97.5113677978516
4548 76.9806213378906
4549 84.8318481445312
4550 111.040328979492
4551 108.052261352539
4552 111.871017456055
4553 116.267517089844
4554 91.7277069091797
4555 92.7515258789062
4556 87.1015625
4557 87.1900329589844
4558 89.1749114990234
4559 85.1879119873047
4560 77.9956970214844
4561 119.977066040039
4562 125.21418762207
4563 119.220565795898
4564 110.346115112305
4565 99.7146453857422
4566 114.5107421875
4567 112.827331542969
4568 83.1006927490234
4570 100.863327026367
4571 69.0715942382812
4572 65.3992309570312
4573 112.35302734375
4574 111.144332885742
4575 102.386291503906
4576 98.52490234375
4577 109.488906860352
4579 101.812911987305
4580 118.083145141602
4581 84.9849243164062
4582 76.0251312255859
4583 91.0150451660156
4584 92.1390686035156
4585 87.7391357421875
4586 94.0987396240234
4587 64.0050354003906
4588 94.0579223632812
4589 101.305252075195
4590 144.26628112793
4591 129.563262939453
4592 66.1572875976562
4593 91.9671325683594
4594 83.9291839599609
4595 106.135604858398
4596 113.134124755859
4597 109.822341918945
4598 86.5562286376953
4599 93.4699249267578
4600 107.512420654297
4601 114.534393310547
4602 67.6553955078125
4603 104.896743774414
4604 123.273880004883
4605 96.2346038818359
4606 104.612365722656
4607 117.306716918945
4608 107.239364624023
4609 71.9975738525391
4610 50.7969512939453
4611 106.807632446289
4612 84.8825378417969
4613 108.007522583008
4614 98.205810546875
4615 114.336456298828
4616 107.101531982422
4617 94.9555816650391
4618 102.269973754883
4619 84.843505859375
4621 106.369003295898
4622 75.5888977050781
4623 102.040969848633
4624 117.757339477539
4625 105.916076660156
4626 86.280517578125
4627 92.9723510742188
4628 107.442749023438
4629 103.846481323242
4630 101.32649230957
4631 97.0056610107422
4632 89.4305572509766
4633 104.400283813477
4634 106.287841796875
4635 90.0572052001953
4637 103.782165527344
4638 99.1288299560547
4639 115.327011108398
4640 98.078125
4641 126.612609863281
4642 84.6270294189453
4643 87.8264465332031
4644 94.3647918701172
4645 108.238464355469
4646 112.302047729492
4647 82.7541046142578
4648 100.374786376953
4649 92.8629455566406
4650 117.508117675781
4651 99.9753265380859
4652 109.875198364258
4653 76.6199951171875
4654 100.954696655273
4655 102.119476318359
4656 72.9318695068359
4657 69.6340637207031
4658 91.1833343505859
4659 92.23046875
4660 101.757034301758
4661 66.4280548095703
4662 102.34635925293
4663 122.287628173828
4664 105.751159667969
4665 120.792495727539
4666 95.7997894287109
4667 105.673782348633
4668 84.3228607177734
4669 110.482879638672
4670 88.4451293945312
4671 106.682830810547
4672 100.848770141602
4673 115.868270874023
4674 112.878799438477
4675 86.8812408447266
4676 86.4564666748047
4677 75.8909301757812
4678 112.289855957031
4679 97.6737213134766
4680 100.689468383789
4681 84.4916229248047
4682 106.79296875
4683 105.277008056641
4684 100.721237182617
4685 100.637649536133
4686 106.239700317383
4687 73.9027557373047
4688 122.632247924805
4689 96.9595947265625
4690 83.8398132324219
4692 109.627212524414
4693 98.2780914306641
4694 102.190139770508
4695 113.963821411133
4696 97.8961639404297
4697 89.1865386962891
4698 90.2901458740234
4699 88.8102111816406
4700 138.869415283203
4701 79.0336151123047
4702 118.553451538086
4703 110.015548706055
4705 81.9468536376953
4706 81.6823120117188
4707 85.2867736816406
4708 72.5743255615234
4709 87.4443511962891
4710 106.283340454102
4711 92.7206573486328
4712 74.4678192138672
4713 105.758316040039
4714 108.226928710938
4715 113.330978393555
4716 99.7837524414062
4717 99.9927978515625
4718 120.82780456543
4719 77.3975982666016
4720 104.772811889648
4721 78.5120849609375
4722 74.2274475097656
4723 89.6259155273438
4724 71.1983337402344
4725 83.9042205810547
4726 87.0911865234375
4727 75.5747222900391
4729 127.633193969727
4730 86.0520324707031
4731 91.6051025390625
4732 102.332580566406
4733 81.0534973144531
4734 85.3021392822266
4735 119.628463745117
4736 83.0200500488281
4737 111.746063232422
4738 95.7325744628906
4739 89.1133880615234
4740 94.7066192626953
4741 52.0473327636719
4742 96.1876068115234
4743 103.691833496094
4744 113.990463256836
4745 75.4209594726562
4746 101.163848876953
4747 109.288177490234
4748 102.451370239258
4750 98.0648193359375
4751 90.8467712402344
4752 89.9432525634766
4753 90.8345642089844
4754 73.5106353759766
4755 84.9059753417969
4756 56.4894714355469
4757 81.2941589355469
4758 121.697326660156
4759 88.218994140625
4760 82.4981536865234
4761 114.878814697266
4762 113.018600463867
4763 82.1737213134766
4764 100.815795898438
4765 90.6063690185547
4766 112.69465637207
4767 101.95637512207
4769 71.9876861572266
4770 79.0710144042969
4771 122.059188842773
4772 117.158081054688
4773 106.927505493164
4774 95.1229400634766
4775 86.8625640869141
4776 97.8158874511719
4777 101.458282470703
4778 86.3455810546875
4779 140.089706420898
4780 119.039672851562
4781 113.99267578125
4782 104.053756713867
4783 81.1412200927734
4784 92.3651123046875
4785 91.7598419189453
4786 87.2319793701172
4787 92.3754119873047
4788 92.7523651123047
4789 109.714752197266
4790 91.5606231689453
4791 104.589248657227
4792 73.8482666015625
4793 79.2301635742188
4794 76.5770568847656
4795 112.326858520508
4796 83.965087890625
4797 86.7905731201172
4798 95.9963073730469
4799 84.7737274169922
4800 121.107055664062
4801 88.3129577636719
4802 122.26057434082
4803 79.7699432373047
4804 88.1414794921875
4805 82.178466796875
4806 109.459716796875
4807 110.140823364258
4808 78.6355133056641
4809 88.8456115722656
4810 104.0546875
4811 104.280197143555
4812 81.2879943847656
4813 78.4560394287109
4814 87.6364135742188
4815 66.9937591552734
4816 119.68896484375
4817 95.1939239501953
4818 113.60334777832
4819 122.047302246094
4820 77.5413360595703
4822 96.5421295166016
4823 77.5269470214844
4824 102.950805664062
4825 83.6946411132812
4826 85.6174774169922
4827 104.772766113281
4828 128.221099853516
4829 83.0643920898438
4830 106.24739074707
4831 117.377685546875
4832 99.5176239013672
4833 71.1045379638672
4834 112.339904785156
4835 106.156921386719
4836 119.359848022461
4837 91.7895202636719
4838 92.0645904541016
4839 83.8745574951172
4840 63.3819122314453
4841 109.736587524414
4842 78.9352111816406
4843 59.0948486328125
4844 84.8487396240234
4845 120.946502685547
4846 88.7387542724609
4847 91.8920135498047
4848 88.8221588134766
4849 105.839462280273
4850 103.627990722656
4851 100.851501464844
4852 70.7960815429688
4853 77.5666046142578
4854 101.946212768555
4855 101.63737487793
4856 92.9249267578125
4857 93.3481750488281
4858 91.8699035644531
4859 98.9880828857422
4860 108.567993164062
4861 88.1689910888672
4862 91.8557739257812
4863 118.683059692383
4864 92.4336547851562
4865 108.61994934082
4866 87.9242401123047
4867 90.3033599853516
4868 104.677764892578
4869 96.2734985351562
4870 93.2132720947266
4871 91.7325897216797
4872 80.966796875
4873 77.798095703125
4874 103.692306518555
4875 121.982696533203
4876 113.969924926758
4877 89.6076965332031
4878 92.5866394042969
4879 100.051330566406
4880 85.9131622314453
4881 97.8296356201172
4882 78.9758758544922
4883 93.8455505371094
4884 99.2568054199219
4885 95.048095703125
4887 98.2253112792969
4888 94.66845703125
4889 86.658203125
4890 99.5444641113281
4891 100.807769775391
4892 101.07438659668
4893 104.10075378418
4894 116.036422729492
4895 106.741760253906
4896 128.06819152832
4897 120.291778564453
4898 105.496932983398
4899 108.171112060547
4900 106.103225708008
4901 111.904510498047
4902 93.3792266845703
4903 90.4171905517578
4904 78.6505737304688
4905 103.841262817383
4906 99.4296569824219
4907 84.4976501464844
4908 85.5638275146484
4909 84.0115966796875
4910 103.911483764648
4911 91.5060272216797
4912 112.18962097168
4913 99.6941070556641
4914 95.1019897460938
4916 119.688491821289
4917 100.040328979492
4918 83.7755584716797
4919 103.666488647461
4921 77.4875335693359
4922 113.22004699707
4923 122.587783813477
4924 88.4512481689453
4926 102.810287475586
4927 98.5417327880859
4928 107.878067016602
4929 71.5486450195312
4930 125.195602416992
4931 117.701354980469
4932 108.165908813477
4933 108.231109619141
4934 76.2493438720703
4935 68.0259552001953
4936 83.8709564208984
4937 106.849273681641
4938 82.9072570800781
4939 104.156356811523
4940 91.7919006347656
4941 109.019088745117
4942 84.5819549560547
4943 115.129486083984
4944 102.39729309082
4945 115.096908569336
4946 117.291717529297
4947 111.020401000977
4948 106.815567016602
4949 93.2091674804688
4950 74.8162384033203
4951 103.896041870117
4952 89.6591796875
4953 106.382186889648
4954 88.3465576171875
4955 90.5676879882812
4956 121.698181152344
4957 75.3652191162109
4958 94.3040618896484
4959 94.9741973876953
4960 88.6288909912109
4961 91.9351959228516
4962 102.946243286133
4963 68.1016693115234
4964 92.9033813476562
4965 78.4787445068359
4966 86.0002288818359
4967 96.7220611572266
4968 115.072402954102
4969 114.997680664062
4970 96.5900573730469
4971 62.0355224609375
4972 104.357192993164
4973 98.0667419433594
4974 79.3812713623047
4975 102.753692626953
4976 105.670791625977
4977 78.1369781494141
4978 93.4021301269531
4979 88.8896636962891
4980 94.1348266601562
4981 92.2778167724609
4982 78.2042694091797
4983 123.920059204102
4984 108.438583374023
4985 86.3204956054688
4986 84.9196929931641
4987 92.2153778076172
4988 120.078399658203
4989 106.627914428711
4990 114.20622253418
4991 118.832641601562
4992 118.418258666992
4993 107.588104248047
4994 108.265594482422
4995 86.7568969726562
4996 97.0220642089844
4997 97.2984466552734
4998 69.5943603515625
4999 98.4752807617188
5000 34.7579498291016
5001 83.6075592041016
5002 99.1755981445312
5003 111.529327392578
5004 107.214477539062
5005 99.335205078125
5006 100.939804077148
5007 98.7259368896484
5008 119.362609863281
5009 120.537048339844
5010 101.710784912109
5011 71.4829406738281
5012 107.497497558594
5013 71.935546875
5014 89.1594848632812
5015 102.472640991211
5016 92.0039672851562
5017 97.9520721435547
5018 67.7194213867188
5019 109.422653198242
5020 110.577011108398
5021 90.0861053466797
5022 80.1429290771484
5023 122.49235534668
5024 95.054931640625
5025 80.6958312988281
5026 90.8635406494141
5027 90.356201171875
5028 108.404205322266
5029 107.69889831543
5030 98.8728790283203
5031 83.8882598876953
5032 106.81282043457
5033 112.252593994141
5034 78.3504180908203
5035 87.2766876220703
5036 75.7885437011719
5037 113.008529663086
5038 82.9196014404297
5039 84.8080596923828
5040 88.1121673583984
5041 106.640594482422
5042 91.56396484375
5043 117.129211425781
5044 83.8787231445312
5045 86.8888092041016
5046 77.2144927978516
5047 122.368698120117
5048 94.3670043945312
5049 107.675521850586
5050 94.9369506835938
5051 94.9002990722656
5052 92.1088409423828
5053 115.933929443359
5054 78.5716705322266
5055 85.8223876953125
5056 116.311599731445
5057 93.0823669433594
5058 98.2785034179688
5059 112.497055053711
5060 96.9072570800781
5061 84.6894989013672
5062 118.647903442383
5063 77.0376281738281
5064 77.8331909179688
5065 101.689483642578
5066 94.3112335205078
5067 100.889846801758
5068 103.391830444336
5069 62.3896179199219
5070 81.4766235351562
5071 83.5956726074219
5072 105.388931274414
5073 86.3762359619141
5074 101.054260253906
5075 94.7462463378906
5076 82.7994995117188
5077 102.052444458008
5078 68.3404998779297
5079 102.464447021484
5080 98.1790161132812
5081 80.3755493164062
5082 109.31169128418
5083 113.586990356445
5084 91.5858764648438
5085 111.909042358398
5086 94.2143096923828
5087 86.3076019287109
5088 37.7453765869141
5089 92.3644714355469
5090 105.201385498047
5091 85.9314727783203
5092 82.4431762695312
5093 67.8749084472656
5094 80.1214141845703
5096 93.6600036621094
5097 91.2170867919922
5098 77.7159271240234
5099 112.584121704102
5100 111.028060913086
5101 105.849258422852
5102 85.3865661621094
5103 106.325454711914
5104 82.0862731933594
5105 105.436264038086
5106 111.022583007812
5107 121.870193481445
5108 78.6088409423828
5109 66.0112457275391
5110 86.6979675292969
5111 111.410095214844
5112 84.0507965087891
5113 117.902465820312
5114 95.5112762451172
5115 77.9755859375
5116 123.124877929688
5117 107.615859985352
5118 85.9048156738281
5119 83.6401214599609
5120 94.0849761962891
5121 73.8949127197266
5122 96.9149627685547
5123 88.8192901611328
5124 91.7730407714844
5125 63.2076110839844
5126 91.16845703125
5127 71.5387573242188
5128 90.4858703613281
5129 72.6418304443359
5130 103.241439819336
5132 85.3180694580078
5133 73.6002655029297
5134 55.5879516601562
5135 52.77978515625
5136 65.8206787109375
5137 96.8194732666016
5138 80.2813415527344
5139 82.4093933105469
5140 83.1907501220703
5141 70.0891876220703
5142 94.5509490966797
5143 107.462265014648
5144 65.0999908447266
5145 88.8759002685547
5146 69.1695709228516
5147 97.7158203125
5148 121.886917114258
5149 81.9359436035156
5150 87.9840240478516
5151 80.0098114013672
5152 89.1904296875
5153 67.550048828125
5154 69.3420715332031
5155 79.1152954101562
5156 81.1872100830078
5157 90.3768005371094
5159 76.0278625488281
5160 84.1871490478516
5162 112.748641967773
5163 121.263320922852
5164 79.7181243896484
5165 92.0785980224609
5166 98.6683807373047
5167 97.8621978759766
5168 95.6253814697266
5169 73.3975982666016
5170 90.3177185058594
5171 111.248229980469
5172 71.9591064453125
5173 44.0755004882812
5174 95.6096343994141
5175 63.5918731689453
5176 100.443695068359
5177 86.9976196289062
5178 91.0826110839844
5179 100.730178833008
5180 64.2348937988281
5181 68.7620086669922
5182 37.9540252685547
5183 84.9390869140625
5184 74.6619567871094
5185 83.1951599121094
5186 80.8492431640625
5187 70.0348052978516
5188 80.7153167724609
5189 68.0530853271484
5191 83.0214385986328
5192 36.2343597412109
5193 76.4176788330078
5194 80.869384765625
5195 74.7137756347656
5196 71.0614013671875
5197 68.4794464111328
5198 61.3241271972656
5199 93.3144836425781
5200 78.4416656494141
5201 61.0790710449219
5202 73.1800842285156
5203 60.5014190673828
5204 87.2151794433594
5205 58.9328765869141
5206 60.4416961669922
5207 102.928909301758
5208 58.1656494140625
5209 83.1820220947266
5210 84.6681976318359
5211 55.5574035644531
5212 82.1687469482422
5213 68.5773773193359
5214 83.2882690429688
5215 113.474304199219
5216 63.1764068603516
5217 73.4053192138672
5218 99.2986450195312
5219 58.888427734375
5220 74.3465576171875
5221 54.1761474609375
5222 95.6558685302734
5223 79.3511962890625
5224 103.990005493164
5225 38.8924560546875
5226 82.7912750244141
5227 86.3600616455078
5228 83.5649566650391
5229 82.4957733154297
5230 90.5884246826172
5231 40.0370178222656
5232 71.3241729736328
5233 69.1302795410156
5234 25.5801086425781
5235 67.7892761230469
5236 83.0135955810547
5237 64.4560546875
5238 70.1953430175781
5239 48.7945556640625
5240 70.3068542480469
5241 53.9056091308594
5242 109.886764526367
5243 83.8960876464844
5244 80.9392700195312
5245 64.0837707519531
5246 86.4188842773438
5247 119.235824584961
5248 67.9833374023438
5249 79.5022888183594
5251 48.6497802734375
5252 72.7451782226562
5253 93.1867980957031
5254 69.2225036621094
5255 79.0496520996094
5256 53.8119201660156
5257 113.185455322266
5258 115.518829345703
5259 93.0438232421875
5260 119.712310791016
5261 89.7743835449219
5262 121.451568603516
5263 128.591003417969
5264 151.559646606445
5265 113.120910644531
5267 176.252990722656
5268 138.928802490234
5269 125.676605224609
5270 181.376770019531
5271 169.534332275391
5272 223.052886962891
5273 209.863082885742
5274 222.359161376953
5275 229.069854736328
5276 224.838150024414
5277 199.84928894043
5278 204.717163085938
5279 226.719757080078
5280 208.194946289062
5281 164.550109863281
5282 223.613067626953
5283 185.345565795898
5284 186.282363891602
5285 197.146102905273
5286 228.216674804688
5287 215.175109863281
5288 257.57568359375
5289 209.035217285156
5290 223.737289428711
5291 266.801391601562
5292 265.497650146484
5293 243.983840942383
5294 257.677062988281
5295 232.826217651367
5296 224.41015625
5297 223.783721923828
5298 259.143798828125
5299 244.505554199219
5300 273.373229980469
5301 225.376968383789
5302 228.01237487793
5303 269.174682617188
5304 255.142166137695
5305 257.317016601562
5306 223.218444824219
5307 236.910629272461
5308 258.226135253906
5309 227.408660888672
5310 268.198486328125
5311 250.052459716797
5312 247.482162475586
5313 252.84033203125
5314 241.686218261719
5315 235.271881103516
5316 249.580947875977
5318 261.226745605469
5319 274.104614257812
5320 219.691772460938
5321 199.106796264648
5322 237.921203613281
5323 244.662628173828
5324 265.780944824219
5325 241.097900390625
5326 230.472595214844
5327 261.648620605469
5328 226.066696166992
5329 268.5751953125
5331 241.719314575195
5332 262.892272949219
5334 276.623352050781
5335 258.052185058594
5336 230.953384399414
5337 243.256881713867
5338 273.223724365234
5339 261.858642578125
5340 231.360809326172
5341 261.177093505859
5342 275.6806640625
5343 228.802612304688
5344 263.123779296875
5345 254.596267700195
5346 272.695251464844
5347 221.274505615234
5348 270.505493164062
5349 216.117004394531
5350 254.470764160156
5351 257.932861328125
5352 290.495910644531
5353 251.968902587891
5354 246.325561523438
5355 274.13525390625
5356 258.313842773438
5357 263.499237060547
5358 278.799499511719
5359 276.55419921875
5360 237.094146728516
5361 258.21630859375
5362 250.141448974609
5363 240.935424804688
5364 266.469329833984
5365 257.054626464844
5366 278.931488037109
5368 239.776184082031
5369 270.627197265625
5370 268.753387451172
5371 243.839767456055
5372 260.456176757812
5373 265.837829589844
5374 277.376342773438
5375 273.481079101562
5376 262.860229492188
5377 258.80029296875
5378 245.480270385742
5379 274.075622558594
5380 250.863662719727
5381 249.020217895508
5382 257.280700683594
5383 279.307556152344
5384 264.087097167969
5385 273.896667480469
5386 278.700988769531
5387 266.650573730469
5388 236.840484619141
5389 231.092224121094
5390 250.020263671875
5391 255.813034057617
5392 248.293243408203
5393 269.053985595703
5394 260.853881835938
5395 243.433883666992
5396 249.348617553711
5398 295.845825195312
5399 249.617370605469
5400 238.320297241211
5401 230.081634521484
5402 286.297729492188
5403 279.580932617188
5404 277.395751953125
5405 266.651977539062
5406 272.538879394531
5407 251.269943237305
5408 287.501831054688
5409 277.704711914062
5410 282.403503417969
5411 251.607009887695
5412 270.252197265625
5413 267.894836425781
5414 247.215423583984
5415 295.644592285156
5416 251.261734008789
5417 241.435012817383
5418 275.774444580078
5419 245.679626464844
5420 269.083374023438
5421 253.246948242188
5422 231.436462402344
5423 266.096130371094
5424 280.879577636719
5425 274.301574707031
5426 280.482818603516
5427 267.147827148438
5428 250.719314575195
5429 218.151519775391
5430 301.825622558594
5431 290.337890625
5432 295.026397705078
5433 256.4111328125
5434 257.958160400391
5435 251.315139770508
5436 270.991821289062
5437 267.682434082031
5438 253.41162109375
5439 265.453796386719
5440 246.682815551758
5441 236.006713867188
5442 288.167785644531
5443 241.629852294922
5444 261.794219970703
5445 258.329833984375
5446 274.545349121094
5448 236.563629150391
5449 282.558227539062
5450 241.655960083008
5451 302.364990234375
5452 268.828979492188
5453 263.603149414062
5454 272.02197265625
5455 267.931365966797
5456 255.638397216797
5457 275.766784667969
5458 282.531494140625
5459 302.710205078125
5460 260.339477539062
5461 266.076385498047
5462 256.802795410156
5463 257.246185302734
5464 253.673034667969
5465 264.761047363281
5466 267.022216796875
5467 290.999450683594
5468 278.643737792969
5469 259.06396484375
5470 284.448425292969
5471 282.775695800781
5472 274.399688720703
5473 262.162994384766
5474 258.380828857422
5475 276.858581542969
5476 267.375183105469
5477 268.158020019531
5478 260.690551757812
5479 263.328796386719
5480 264.334167480469
5481 278.626678466797
5482 290.517791748047
5483 272.425659179688
5484 285.848419189453
5485 279.585327148438
5486 270.607360839844
5487 241.959716796875
5488 241.827224731445
5490 255.337707519531
5491 295.688232421875
5492 266.72216796875
5493 283.456359863281
5494 257.884368896484
5495 292.497009277344
5496 269.342376708984
5497 256.456390380859
5498 280.297607421875
5499 274.875396728516
5500 274.809020996094
5501 258.428405761719
5502 282.429260253906
5503 291.238891601562
5504 284.2587890625
5505 283.751007080078
5506 300.970275878906
5507 263.137023925781
5508 276.877471923828
5509 282.483551025391
5510 286.045318603516
5511 255.688003540039
5512 268.955902099609
5513 262.979156494141
5514 282.452087402344
5515 283.191619873047
5516 257.73828125
5517 261.441345214844
5518 278.681335449219
5519 256.025817871094
5520 264.998291015625
5521 262.741333007812
5522 274.325347900391
5523 263.011840820312
5524 249.448440551758
5525 290.750274658203
5526 296.398376464844
5527 265.932983398438
5528 284.714630126953
5529 293.491607666016
5530 278.034545898438
5531 281.672973632812
5532 279.675842285156
5533 265.196838378906
5534 246.426681518555
5535 262.864868164062
5536 261.24609375
5537 294.032470703125
5538 279.962280273438
5539 291.95947265625
5540 253.922424316406
5541 264.724548339844
5542 279.567260742188
5543 260.782348632812
5544 276.986022949219
5545 266.481689453125
5546 281.806579589844
5548 237.908081054688
5549 277.366149902344
5550 257.026977539062
5551 265.511108398438
5552 286.094421386719
5554 293.593231201172
5555 271.546264648438
5556 266.898132324219
5557 282.54541015625
5558 290.8955078125
5559 254.75341796875
5561 275.475219726562
5562 267.572937011719
5563 282.929351806641
5564 273.319152832031
5565 273.490356445312
5566 236.931396484375
5567 264.650604248047
5568 272.573547363281
5569 254.705383300781
5570 265.144409179688
5571 271.788299560547
5572 253.469192504883
5573 280.969665527344
5574 265.33203125
5575 265.22607421875
5576 273.896301269531
5577 266.297790527344
5578 286.153503417969
5579 229.7109375
5580 248.337936401367
5581 258.443420410156
5582 272.801208496094
5583 244.838409423828
5584 284.240539550781
5585 285.510803222656
5586 288.101654052734
5587 279.002532958984
5588 260.302520751953
5589 260.447845458984
5590 263.661804199219
5591 245.481216430664
5592 271.3955078125
5593 271.023498535156
5594 269.8623046875
5595 288.454040527344
5596 283.68505859375
5597 274.904418945312
5598 300.716003417969
5599 300.291961669922
5600 294.292816162109
5601 287.244567871094
5602 282.034912109375
5603 268.299896240234
5604 269.393249511719
5605 265.731262207031
5607 290.979522705078
5608 271.30126953125
5609 295.432434082031
5610 261.085388183594
5611 259.047698974609
5612 281.969940185547
5613 288.260528564453
5614 260.080322265625
5615 279.185852050781
5616 290.669128417969
5617 293.398681640625
5618 274.794128417969
5619 290.813201904297
5620 276.269409179688
5621 278.507598876953
5622 257.793884277344
5623 266.542510986328
5624 262.194702148438
5625 284.490661621094
5626 271.402038574219
5627 266.167572021484
5628 294.823913574219
5629 268.899475097656
5630 282.896362304688
5631 276.047088623047
5632 273.157531738281
5633 287.979309082031
5634 274.255340576172
5635 273.298400878906
5636 278.596160888672
5637 274.826019287109
5638 288.176391601562
5639 297.602355957031
5640 289.786560058594
5641 253.812896728516
5642 278.77685546875
5643 240.65657043457
5644 255.269088745117
5645 293.082733154297
5646 266.967224121094
5647 265.937133789062
5648 237.926055908203
5649 279.643371582031
5650 295.082336425781
5651 265.055908203125
5652 252.495025634766
5653 276.168884277344
5654 294.8251953125
5655 288.507537841797
5656 281.143432617188
5657 282.995880126953
5658 286.437927246094
5659 262.409851074219
5660 263.796752929688
5661 291.145446777344
5662 283.486267089844
5663 291.584167480469
5664 266.913146972656
5665 284.482879638672
5666 271.832977294922
5667 273.088439941406
5668 276.3896484375
5669 278.938171386719
5670 274.941741943359
5671 269.47265625
5672 234.274719238281
5673 263.736694335938
5674 276.253051757812
5675 282.072570800781
5676 265.170593261719
5677 243.332977294922
5678 251.334121704102
5679 289.686767578125
5680 260.002258300781
5681 262.424621582031
5682 279.515167236328
5683 288.352081298828
5684 262.141204833984
5685 269.718963623047
5686 263.928070068359
5687 249.551559448242
5688 268.112701416016
5689 255.533843994141
5690 263.755523681641
5691 281.584716796875
5692 282.89697265625
5693 287.299072265625
5694 252.085250854492
5695 288.7890625
5696 259.946166992188
5697 269.377990722656
5698 285.849914550781
5699 257.540649414062
5700 272.135131835938
5701 227.124786376953
5702 252.357833862305
5705 282.170593261719
5706 298.245422363281
5707 271.813781738281
5708 293.757873535156
5709 278.800201416016
5710 261.526794433594
5711 290.9560546875
5712 285.389129638672
5713 270.062744140625
5714 266.113037109375
5715 254.921478271484
5716 248.282485961914
5717 254.843994140625
5718 259.186340332031
5719 260.984802246094
5720 263.966461181641
5721 231.297576904297
5722 293.034332275391
5723 281.236968994141
5724 273.161773681641
5725 281.897491455078
5726 259.028076171875
5727 244.37971496582
5728 273.929504394531
5729 277.315002441406
5730 268.533203125
5731 290.768676757812
5732 258.413879394531
5733 275.568511962891
5734 269.467407226562
5735 265.726684570312
5736 258.026214599609
5737 272.0224609375
5739 248.399368286133
5740 239.265899658203
5741 285.986022949219
5742 278.415100097656
5743 276.762145996094
5744 274.362426757812
5745 300.451110839844
5746 259.518341064453
5747 250.819885253906
5748 275.234802246094
5749 259.040252685547
5750 285.775817871094
5751 244.126068115234
5752 283.762512207031
5753 296.164245605469
5754 261.663208007812
5755 293.846801757812
5756 255.558746337891
5757 300.426055908203
5759 265.802520751953
5760 237.690322875977
5761 257.775634765625
5762 274.558013916016
5763 283.31298828125
5764 275.707092285156
5765 283.285339355469
5766 267.555541992188
5767 266.775756835938
5768 267.096740722656
5769 296.119720458984
5770 284.074035644531
5771 249.003341674805
5772 270.709594726562
5773 258.032592773438
5774 287.699890136719
5775 268.029968261719
5776 244.095779418945
5777 262.657531738281
5778 272.87646484375
5779 264.557312011719
5780 275.679565429688
5781 268.701110839844
5782 276.221130371094
5783 279.891723632812
5784 296.791931152344
5785 257.330017089844
5786 265.724182128906
5787 304.639495849609
5788 297.909118652344
5789 275.851226806641
5790 276.10595703125
5791 287.551239013672
5792 288.817687988281
5793 256.040496826172
5794 270.005859375
5795 299.923370361328
5796 262.927124023438
5797 301.35009765625
5798 277.460083007812
5799 278.497741699219
5800 270.260681152344
5801 256.544067382812
5802 253.124435424805
5803 282.349761962891
5804 272.949401855469
5805 270.641967773438
5806 293.741271972656
5807 279.112243652344
5808 254.99267578125
5809 291.655364990234
5810 284.053283691406
5811 247.013031005859
5812 307.886810302734
5813 302.477661132812
5814 287.847290039062
5815 282.528137207031
5816 274.874877929688
5817 305.941162109375
5818 252.397521972656
5819 273.433349609375
5820 306.502502441406
5821 289.247436523438
5822 283.335754394531
5823 256.784851074219
5824 280.673187255859
5825 267.976470947266
5826 296.253204345703
5827 283.437438964844
5828 233.000778198242
5829 255.541870117188
5830 257.192779541016
5831 273.188110351562
5832 286.080535888672
5833 301.33837890625
5834 268.058288574219
5835 271.281433105469
5836 284.216461181641
5837 287.992431640625
5838 263.873718261719
5839 259.803527832031
5840 265.645874023438
5841 274.143493652344
5842 232.841918945312
5843 272.543884277344
5844 265.354278564453
5845 246.375244140625
5846 263.781616210938
5847 276.155029296875
5849 286.103759765625
5850 282.624481201172
5851 251.376907348633
5852 271.814697265625
5853 267.920104980469
5854 281.221069335938
5855 272.55126953125
5856 255.238815307617
5857 282.223236083984
5858 267.557373046875
5859 305.7822265625
5860 256.298309326172
5861 243.962097167969
5862 277.356567382812
5863 303.464172363281
5864 269.5654296875
5865 280.951904296875
5866 270.123870849609
5867 283.763061523438
5868 275.867645263672
5869 249.128173828125
5870 268.544189453125
5871 244.413970947266
5872 262.178314208984
5873 260.494812011719
5874 295.512512207031
5875 256.597290039062
5876 238.200424194336
5877 265.863098144531
5878 285.695831298828
5879 271.123352050781
5880 276.175811767578
5881 275.676513671875
5882 266.9326171875
5883 262.407043457031
5884 276.465393066406
5885 279.757629394531
5886 231.169021606445
5887 283.416015625
5888 265.427368164062
5889 292.279327392578
5890 284.072631835938
5891 288.9912109375
5892 291.147674560547
5893 294.50341796875
5894 276.474884033203
5895 265.525787353516
5896 288.868286132812
5897 247.093719482422
5898 260.667114257812
5899 279.017333984375
5900 286.86767578125
5901 252.213912963867
5902 274.305847167969
5903 275.752319335938
5904 256.784362792969
5905 273.101440429688
5906 274.216583251953
5907 274.595886230469
5908 287.725280761719
5909 276.18408203125
5910 285.07666015625
5911 278.734619140625
5912 290.494171142578
5913 283.530578613281
5914 300.271240234375
5915 298.163879394531
5916 296.597778320312
5917 280.706451416016
5918 284.189880371094
5919 234.928146362305
5920 262.810119628906
5921 247.471923828125
5922 281.844879150391
5923 281.652404785156
5924 264.200988769531
5925 252.566864013672
5926 284.024658203125
5927 307.904296875
5928 274.016296386719
5929 296.932189941406
5930 248.64225769043
5931 278.275817871094
5932 291.604858398438
5933 271.390808105469
5934 280.206604003906
5935 265.77099609375
5936 281.01025390625
5937 253.867721557617
5938 278.060699462891
5939 272.27978515625
5940 295.685241699219
5941 284.541198730469
5942 293.590148925781
5943 283.478454589844
5944 275.298248291016
5945 258.582214355469
5946 245.996078491211
5947 273.767578125
5948 286.764129638672
5949 279.062530517578
5950 265.322631835938
5951 274.733032226562
5952 267.024780273438
5953 294.032104492188
5954 258.971130371094
5956 287.778472900391
5957 244.344940185547
5958 275.399353027344
5959 281.081481933594
5960 272.601379394531
5961 277.39208984375
5962 289.823211669922
5963 287.425872802734
5964 283.692810058594
5965 247.778686523438
5966 259.398742675781
5967 296.011657714844
5968 283.033386230469
5969 284.047790527344
5970 296.256652832031
5971 297.32373046875
5972 257.699096679688
5973 274.76953125
5975 290.299987792969
5976 289.142578125
5978 260.439575195312
5979 286.7294921875
5980 266.263916015625
5981 248.938140869141
5982 273.588806152344
5983 278.821166992188
5984 293.993072509766
5986 252.990020751953
5987 271.014892578125
5988 262.067993164062
5989 287.602844238281
5990 252.90348815918
5991 281.147399902344
5992 281.526275634766
5993 253.593811035156
5994 277.752075195312
5995 265.839477539062
5996 290.411071777344
5997 281.984985351562
5998 266.696838378906
5999 277.935791015625
6000 260.684906005859
};
\end{axis}

\end{tikzpicture}

%         \caption{Training curve}
%         \label{fig:elbow_example_9000}
%     \end{subfigure}
%     \hfill
%     \begin{subfigure}[t!]{0.3\linewidth}
%         \centering
%          % This file was created with tikzplotlib v0.9.15.
\begin{tikzpicture}

\definecolor{color0}{rgb}{0.15201338,0.08159108,0.20270523}
\definecolor{color1}{rgb}{0.41925278,0.12173833,0.33640407}
\definecolor{color2}{rgb}{0.71108138,0.08761223,0.3430635}
\definecolor{color3}{rgb}{0.92223947,0.28287251,0.24229568}
\definecolor{color4}{rgb}{0.96253168,0.5995944,0.43805131}
\definecolor{color5}{rgb}{0.97564391,0.87403763,0.79725261}

\begin{axis}[
axis line style={white!80!black},
height=\figureheight,
tick align=outside,
tick pos=left,
width=\figurewidth,
x grid style={white!80!black},
xmajorgrids,
xmin=-4, xmax=4,
xtick style={draw=none},
y grid style={white!80!black},
ymajorgrids,
ymin=-4, ymax=4,
ytick style={draw=none}
]
\addplot [
  draw=white,
  mark=*,
  only marks,
  scatter,
  scatter/@post marker code/.code={%
  \endscope
},
  scatter/@pre marker code/.code={%
  \expanded{%
  \noexpand\definecolor{thispointfillcolor}{RGB}{\fillcolor}%
  }%
  \scope[fill=thispointfillcolor]%
},
  visualization depends on={value \thisrow{fill} \as \fillcolor}
]
table[row sep=\\]{%
x  y  fill \\
1.83493781089783 0.725460708141327 250.0727982,233.26498575,218.47535805 \\
2.47890329360962 0.0141632556915283 246.4137216,191.3927133,157.3251468 \\
0.838586211204529 2.13736271858215 244.99843845,142.7928192,101.84999115 \\
2.77428579330444 0.11805933713913 242.2543452,109.0670649,74.5740564 \\
1.71238541603088 -0.263743102550507 248.6295441,221.39436855,201.1211367 \\
2.51340484619141 1.0128002166748 245.084376,144.4915935,103.45182465 \\
1.25608277320862 -1.46656894683838 246.7723791,199.01257995,168.07746405 \\
2.39540243148804 0.260011166334152 247.0774866,203.5132356,174.7062783 \\
1.79510951042175 0.830318927764893 250.0727982,233.26498575,218.47535805 \\
1.45968067646027 1.87097644805908 246.86670105,200.51813565,170.27341695 \\
2.11751079559326 1.64140915870667 245.8641966,167.68475895,127.6195797 \\
1.71888327598572 1.65995573997498 247.65640545,210.97774605,185.77259745 \\
1.50661873817444 -0.97752970457077 248.04512235,215.44664655,192.37235955 \\
1.17419612407684 -1.29485821723938 246.1520712,182.0552406,144.9672297 \\
2.13461017608643 -1.52672004699707 246.11813835,180.47969505,142.9681215 \\
2.23056244850159 0.225129842758179 248.78919705,222.87959565,203.29941555 \\
1.28114151954651 1.62061429023743 246.96984345,202.01754075,172.483173 \\
2.13569784164429 0.837346076965332 248.6295441,221.39436855,201.1211367 \\
1.47786581516266 -0.219341814517975 246.1872816,183.6258162,146.98295625 \\
1.79438972473145 -0.807079553604126 250.0727982,233.26498575,218.47535805 \\
1.67692494392395 -1.69286513328552 247.53321495,209.48702625,183.56726565 \\
-0.679823517799377 -2.81820058822632 37.3151853,20.33215215,50.6468505 \\
1.73324298858643 -1.74858927726746 246.96984345,202.01754075,172.483173 \\
1.27620625495911 0.638411164283752 245.70760365,161.1641412,120.3721023 \\
1.80800747871399 -0.423600435256958 249.6765537,230.29867785,214.14730935 \\
2.14898324012756 -0.332263290882111 249.6765537,230.29867785,214.14730935 \\
2.14709806442261 -0.0159289240837097 249.6765537,230.29867785,214.14730935 \\
1.49979615211487 0.789771199226379 247.65640545,210.97774605,185.77259745 \\
1.68610692024231 -1.10275912284851 249.48584175,228.81590385,211.9835043 \\
1.925705909729 0.736966013908386 250.0727982,233.26498575,218.47535805 \\
1.96460139751434 0.613875687122345 250.0727982,233.26498575,218.47535805 \\
1.80677604675293 -0.359685182571411 249.6765537,230.29867785,214.14730935 \\
1.33573222160339 1.73018097877502 247.0774866,203.5132356,174.7062783 \\
0.698629379272461 1.45964133739471 242.03209995,107.19754005,73.39147095 \\
1.71073722839355 -1.10594940185547 249.48584175,228.81590385,211.9835043 \\
1.86870276927948 -1.17872953414917 249.48584175,228.81590385,211.9835043 \\
2.056391954422 -1.55680966377258 246.36063315,189.8505141,155.2234827 \\
2.19041299819946 1.03260838985443 247.41505305,207.99484275,181.3567548 \\
1.20762097835541 1.44448435306549 246.53702175,194.45887155,161.57890635 \\
1.23877382278442 -1.56786227226257 246.68642115,197.50080735,165.8961201 \\
1.73772847652435 -1.56551218032837 248.04512235,215.44664655,192.37235955 \\
1.37099874019623 1.85769534111023 246.7723791,199.01257995,168.07746405 \\
1.72981798648834 0.442098319530487 248.9531187,224.36478195,205.4760165 \\
2.08965349197388 0.502794623374939 249.6765537,230.29867785,214.14730935 \\
1.22796308994293 1.58731544017792 246.68642115,197.50080735,165.8961201 \\
1.88257241249084 -1.55952382087708 247.41505305,207.99484275,181.3567548 \\
2.21686577796936 1.06918740272522 247.0774866,203.5132356,174.7062783 \\
1.26081335544586 1.89736557006836 246.36063315,189.8505141,155.2234827 \\
2.08235573768616 0.0277327299118042 250.0727982,233.26498575,218.47535805 \\
1.62583947181702 -1.31006169319153 249.1250907,225.84872895,207.6473262 \\
1.67523539066315 1.01056098937988 249.6765537,230.29867785,214.14730935 \\
1.80608987808228 -0.148907005786896 249.3025911,227.33242605,209.8164174 \\
1.72118997573853 0.372249245643616 248.78919705,222.87959565,203.29941555 \\
1.61979198455811 -1.75288963317871 247.299714,206.50190445,179.14325025 \\
1.53889393806458 -1.40211236476898 248.6295441,221.39436855,201.1211367 \\
1.78111243247986 1.02198648452759 250.0727982,233.26498575,218.47535805 \\
2.47247982025146 1.59737372398376 238.6776438,85.9450521,63.67900035 \\
2.05931234359741 -1.1777241230011 248.18406675,216.9345027,194.56450275 \\
1.39426481723785 1.28964757919312 247.91115555,213.9578826,190.17620265 \\
1.47930884361267 -1.63390517234802 247.78194195,212.46815985,187.97598105 \\
1.62595808506012 1.83210575580597 246.86670105,200.51813565,170.27341695 \\
1.1755359172821 2.10400104522705 245.8981065,169.30226475,129.4754085 \\
2.12253546714783 0.966886281967163 248.3271753,218.42187945,196.75380015 \\
1.84716236591339 0.365802079439163 249.6765537,230.29867785,214.14730935 \\
1.7033634185791 0.392698705196381 248.6295441,221.39436855,201.1211367 \\
1.53107213973999 -1.21314370632172 248.6295441,221.39436855,201.1211367 \\
1.55486726760864 1.95867073535919 246.4137216,191.3927133,157.3251468 \\
1.15858912467957 -0.866644859313965 245.2421496,147.8709147,106.70345205 \\
1.42712795734406 -1.21623516082764 247.78194195,212.46815985,187.97598105 \\
1.8409560918808 1.44005310535431 248.3271753,218.42187945,196.75380015 \\
1.5023136138916 -0.927915275096893 247.78194195,212.46815985,187.97598105 \\
1.42946743965149 -1.52591717243195 247.91115555,213.9578826,190.17620265 \\
2.08487868309021 0.598751604557037 249.48584175,228.81590385,211.9835043 \\
1.83709299564362 -1.79785764217377 246.31171395,188.3028579,153.13767705 \\
1.51140737533569 -2.20810556411743 245.3822772,151.2267249,110.02044105 \\
1.68031048774719 -0.175533771514893 248.04512235,215.44664655,192.37235955 \\
1.43134021759033 -1.00433421134949 247.299714,206.50190445,179.14325025 \\
2.26639723777771 1.08906996250153 246.53702175,194.45887155,161.57890635 \\
1.45973372459412 -0.13953560590744 246.05468925,177.3121896,139.0238835 \\
1.40210592746735 0.0907034575939178 245.7904863,164.4347457,123.960498 \\
1.41126883029938 2.09856414794922 246.05468925,177.3121896,139.0238835 \\
2.49102783203125 1.50292694568634 240.4421571,95.75117145,67.24304355 \\
1.46361494064331 -1.06273818016052 247.78194195,212.46815985,187.97598105 \\
1.60295808315277 0.773286461830139 248.6295441,221.39436855,201.1211367 \\
1.82715249061584 -0.565376937389374 250.0727982,233.26498575,218.47535805 \\
1.87156653404236 -1.14102268218994 249.48584175,228.81590385,211.9835043 \\
2.10370492935181 -0.625641167163849 249.6765537,230.29867785,214.14730935 \\
1.35943233966827 1.06232452392578 247.0774866,203.5132356,174.7062783 \\
2.03768181800842 0.335181146860123 250.0727982,233.26498575,218.47535805 \\
1.36284780502319 -2.93753242492676 147.2622705,28.12964415,91.39628145 \\
2.07779717445374 0.745592832565308 249.3025911,227.33242605,209.8164174 \\
2.01880598068237 -1.97045660018921 244.6033236,135.93415815,95.5968225 \\
1.43544769287109 -1.05850374698639 247.53321495,209.48702625,183.56726565 \\
1.32607460021973 0.779948830604553 246.1520712,182.0552406,144.9672297 \\
1.45659303665161 -0.744751513004303 246.86670105,200.51813565,170.27341695 \\
1.44389224052429 -1.35990643501282 248.04512235,215.44664655,192.37235955 \\
1.83623862266541 -0.243631362915039 249.6765537,230.29867785,214.14730935 \\
1.62537693977356 0.768461942672729 248.78919705,222.87959565,203.29941555 \\
2.07043147087097 0.186775118112564 250.0727982,233.26498575,218.47535805 \\
2.34945726394653 0.39002200961113 247.41505305,207.99484275,181.3567548 \\
1.55896818637848 -1.29892468452454 248.78919705,222.87959565,203.29941555 \\
1.65582585334778 -1.60538291931152 248.04512235,215.44664655,192.37235955 \\
1.56413841247559 -1.88891649246216 246.60841665,195.98272095,163.73019855 \\
2.14405369758606 -0.338227391242981 249.6765537,230.29867785,214.14730935 \\
1.37431502342224 1.91631007194519 246.53702175,194.45887155,161.57890635 \\
1.5863481760025 0.800411462783813 248.4751161,219.90865695,198.93982845 \\
1.49583685398102 -0.580070376396179 246.7723791,199.01257995,168.07746405 \\
1.954474568367 -1.40748190879822 247.91115555,213.9578826,190.17620265 \\
0.964897990226746 1.0565071105957 242.863122,114.60937515,78.3386775 \\
1.85752630233765 0.722483038902283 250.0727982,233.26498575,218.47535805 \\
1.48238968849182 -1.51298749446869 248.18406675,216.9345027,194.56450275 \\
1.52748799324036 -0.359226942062378 246.60841665,195.98272095,163.73019855 \\
1.60782849788666 1.13795304298401 249.3025911,227.33242605,209.8164174 \\
1.0074315071106 0.331252068281174 222.05177895,45.8543907,68.3333139 \\
1.66444230079651 0.25106942653656 248.04512235,215.44664655,192.37235955 \\
1.55539512634277 -1.20052647590637 248.78919705,222.87959565,203.29941555 \\
1.19637513160706 -1.65209054946899 246.4728561,192.92848905,159.44460225 \\
2.31751298904419 -0.212527811527252 248.04512235,215.44664655,192.37235955 \\
1.66106081008911 -1.88235330581665 246.4137216,191.3927133,157.3251468 \\
1.54681050777435 -1.94546842575073 246.36063315,189.8505141,155.2234827 \\
1.36510491371155 1.54390215873718 247.53321495,209.48702625,183.56726565 \\
2.35158061981201 0.32586732506752 247.41505305,207.99484275,181.3567548 \\
1.99051713943481 -0.273563086986542 250.25125995,234.75740895,220.66913325 \\
2.68587946891785 -1.57152676582336 209.9707536,31.96512465,75.5744775 \\
1.79333937168121 -0.32314133644104 249.48584175,228.81590385,211.9835043 \\
2.0001916885376 1.00642967224121 249.3025911,227.33242605,209.8164174 \\
2.10871171951294 -0.678967118263245 249.48584175,228.81590385,211.9835043 \\
1.86505496501923 0.364754319190979 249.872205,231.7816839,216.3110379 \\
2.05858302116394 0.495341837406158 249.872205,231.7816839,216.3110379 \\
1.00172078609467 1.79157364368439 245.96240475,172.5216882,133.24118985 \\
1.49726331233978 -1.75695443153381 247.299714,206.50190445,179.14325025 \\
1.21599042415619 -0.562989592552185 244.4915877,134.20217775,94.0718052 \\
1.46587514877319 -1.56500577926636 247.91115555,213.9578826,190.17620265 \\
2.20308971405029 -0.608695924282074 248.78919705,222.87959565,203.29941555 \\
1.71921467781067 -1.23620414733887 249.48584175,228.81590385,211.9835043 \\
2.23721981048584 -0.313624918460846 248.78919705,222.87959565,203.29941555 \\
1.69608962535858 -2.41664361953735 238.6776438,85.9450521,63.67900035 \\
1.71173620223999 1.46943855285645 248.78919705,222.87959565,203.29941555 \\
2.09960460662842 0.350934088230133 249.6765537,230.29867785,214.14730935 \\
1.18565285205841 1.66516375541687 246.4728561,192.92848905,159.44460225 \\
1.77932107448578 0.337900012731552 249.1250907,225.84872895,207.6473262 \\
1.5939359664917 2.07347464561462 245.9932011,174.1238073,135.15058905 \\
1.05056536197662 1.51787960529327 245.9932011,174.1238073,135.15058905 \\
-0.0353641211986542 2.40330982208252 198.72823455,24.5687247,81.2854371 \\
1.53029179573059 -1.66686987876892 247.78194195,212.46815985,187.97598105 \\
2.22380828857422 -1.64421367645264 245.2421496,147.8709147,106.70345205 \\
1.43343424797058 -1.36256194114685 248.04512235,215.44664655,192.37235955 \\
2.1378710269928 -0.526881992816925 249.48584175,228.81590385,211.9835043 \\
2.55186033248901 0.334899604320526 245.9304711,170.9147751,131.3488017 \\
2.37956929206848 0.543156147003174 246.7723791,199.01257995,168.07746405 \\
1.8098372220993 -0.725945651531219 250.0727982,233.26498575,218.47535805 \\
1.94710373878479 1.31976091861725 248.3271753,218.42187945,196.75380015 \\
1.70636343955994 -1.99498009681702 245.9932011,174.1238073,135.15058905 \\
2.08484077453613 -0.645355880260468 249.6765537,230.29867785,214.14730935 \\
2.15649223327637 0.867783963680267 248.3271753,218.42187945,196.75380015 \\
2.14027690887451 0.496562898159027 249.3025911,227.33242605,209.8164174 \\
2.06839156150818 -0.125012278556824 250.25125995,234.75740895,220.66913325 \\
1.01863217353821 2.24997520446777 245.084376,144.4915935,103.45182465 \\
2.24399399757385 0.527491569519043 248.3271753,218.42187945,196.75380015 \\
1.3352484703064 0.679933369159698 246.05468925,177.3121896,139.0238835 \\
1.84639430046082 -0.565219521522522 250.0727982,233.26498575,218.47535805 \\
1.75405406951904 -1.51119160652161 248.3271753,218.42187945,196.75380015 \\
2.01125264167786 -0.00551056861877441 250.25125995,234.75740895,220.66913325 \\
1.15062212944031 -1.91241490840912 246.11813835,180.47969505,142.9681215 \\
1.91370761394501 2.02300381660461 244.90734735,141.0879912,100.26359535 \\
2.21994161605835 -0.331811904907227 248.9531187,224.36478195,205.4760165 \\
1.81357836723328 -0.137316048145294 249.48584175,228.81590385,211.9835043 \\
0.608729243278503 -1.76346325874329 241.5538908,103.4256795,71.1509211 \\
1.46845197677612 1.69410479068756 247.65640545,210.97774605,185.77259745 \\
1.92406558990479 0.182008355855942 250.0727982,233.26498575,218.47535805 \\
1.84011244773865 -0.177184820175171 249.6765537,230.29867785,214.14730935 \\
1.40354347229004 1.19858133792877 247.78194195,212.46815985,187.97598105 \\
1.10719525814056 2.04875993728638 245.9304711,170.9147751,131.3488017 \\
1.23942375183105 -1.60235071182251 246.68642115,197.50080735,165.8961201 \\
1.55562222003937 -0.572086215019226 247.41505305,207.99484275,181.3567548 \\
2.17247629165649 1.13624501228333 247.1866113,205.00830825,176.92716255 \\
1.6864287853241 0.429827868938446 248.4751161,219.90865695,198.93982845 \\
1.7818431854248 -1.67686486244202 247.1866113,205.00830825,176.92716255 \\
2.52675461769104 0.44034019112587 245.96240475,172.5216882,133.24118985 \\
2.7246208190918 -0.204757750034332 243.98829675,127.1991204,88.11634905 \\
1.52495718002319 1.39670884609222 248.78919705,222.87959565,203.29941555 \\
1.72533738613129 -2.0235755443573 245.8641966,167.68475895,127.6195797 \\
0.701671838760376 -1.91741943359375 243.55115025,121.8574518,83.79951015 \\
2.13243556022644 -1.00490474700928 248.3271753,218.42187945,196.75380015 \\
0.758987903594971 -1.09994626045227 233.8786203,68.27324355,61.95004935 \\
1.74637305736542 -1.33266830444336 249.1250907,225.84872895,207.6473262 \\
1.85884487628937 -0.0801013112068176 249.872205,231.7816839,216.3110379 \\
1.45068097114563 -0.920997500419617 247.299714,206.50190445,179.14325025 \\
1.97130799293518 -0.0242555737495422 250.25125995,234.75740895,220.66913325 \\
2.03382706642151 0.796483516693115 249.48584175,228.81590385,211.9835043 \\
1.83758401870728 0.615691781044006 249.872205,231.7816839,216.3110379 \\
1.96228337287903 -1.3290958404541 248.18406675,216.9345027,194.56450275 \\
1.89879477024078 -0.582611262798309 250.25125995,234.75740895,220.66913325 \\
1.51545810699463 -1.70320963859558 247.65640545,210.97774605,185.77259745 \\
0.157061785459518 -2.3224732875824 204.52526685,27.68874915,78.49665 \\
2.56255316734314 -0.188063621520996 245.96240475,172.5216882,133.24118985 \\
1.99952340126038 -0.287446737289429 250.25125995,234.75740895,220.66913325 \\
1.81099271774292 1.57145702838898 247.78194195,212.46815985,187.97598105 \\
2.83934950828552 -0.546430706977844 233.8786203,68.27324355,61.95004935 \\
2.31597566604614 1.374152302742 245.70760365,161.1641412,120.3721023 \\
1.94817721843719 0.397285282611847 250.0727982,233.26498575,218.47535805 \\
2.33375883102417 0.191773235797882 247.78194195,212.46815985,187.97598105 \\
2.29947733879089 0.384705185890198 247.91115555,213.9578826,190.17620265 \\
0.918456614017487 -1.30251932144165 244.4915877,134.20217775,94.0718052 \\
0.995507717132568 1.74787044525146 245.96240475,172.5216882,133.24118985 \\
2.01607036590576 1.68356871604919 246.05468925,177.3121896,139.0238835 \\
1.76719486713409 1.6108170747757 247.65640545,210.97774605,185.77259745 \\
2.66285181045532 0.642043769359589 243.70268145,123.64724325,85.2245802 \\
1.91417837142944 0.454166680574417 250.0727982,233.26498575,218.47535805 \\
1.93458008766174 -1.31376838684082 248.4751161,219.90865695,198.93982845 \\
2.19161915779114 1.53151941299438 245.8641966,167.68475895,127.6195797 \\
1.45804071426392 -2.77226591110229 192.61635375,22.7454696,83.8342692 \\
0.914652109146118 2.0624852180481 245.50510815,154.5607581,113.4028299 \\
1.21860063076019 1.72652471065521 246.53702175,194.45887155,161.57890635 \\
1.12438011169434 -1.7727484703064 246.1872816,183.6258162,146.98295625 \\
1.86686956882477 -1.43407881259918 248.3271753,218.42187945,196.75380015 \\
1.70700597763062 0.0263800621032715 248.3271753,218.42187945,196.75380015 \\
2.3641951084137 1.84778702259064 235.17106485,72.13249005,61.7853984 \\
2.19984459877014 0.862285017967224 247.91115555,213.9578826,190.17620265 \\
1.79452681541443 -0.1917564868927 249.3025911,227.33242605,209.8164174 \\
1.73930299282074 0.966355085372925 249.872205,231.7816839,216.3110379 \\
1.62062895298004 -2.204256772995 245.084376,144.4915935,103.45182465 \\
1.69436001777649 -1.63079524040222 247.91115555,213.9578826,190.17620265 \\
1.29998409748077 0.670020461082458 245.8981065,169.30226475,129.4754085 \\
1.4350723028183 2.00577211380005 246.31171395,188.3028579,153.13767705 \\
1.97641849517822 -0.0620557069778442 250.25125995,234.75740895,220.66913325 \\
1.33672201633453 0.624618232250214 245.9932011,174.1238073,135.15058905 \\
1.85568368434906 0.317342042922974 249.6765537,230.29867785,214.14730935 \\
2.16412162780762 -0.33842945098877 249.48584175,228.81590385,211.9835043 \\
1.84070038795471 -0.0672029256820679 249.6765537,230.29867785,214.14730935 \\
1.58349275588989 -1.91970562934875 246.4137216,191.3927133,157.3251468 \\
2.10372400283813 1.23432791233063 247.41505305,207.99484275,181.3567548 \\
2.08930516242981 -1.22809863090515 247.78194195,212.46815985,187.97598105 \\
1.70289468765259 0.606834173202515 249.1250907,225.84872895,207.6473262 \\
2.0275182723999 -0.0747324824333191 250.25125995,234.75740895,220.66913325 \\
1.72415935993195 0.334454864263535 248.6295441,221.39436855,201.1211367 \\
2.49340009689331 0.548011481761932 246.05468925,177.3121896,139.0238835 \\
2.18119215965271 -0.194259703159332 249.48584175,228.81590385,211.9835043 \\
2.09508800506592 -1.42411494255066 246.68642115,197.50080735,165.8961201 \\
1.82551956176758 1.58686494827271 247.53321495,209.48702625,183.56726565 \\
1.54901885986328 0.278418660163879 246.7723791,199.01257995,168.07746405 \\
1.26198267936707 -2.22660160064697 245.4455784,152.896572,111.70308405 \\
1.84298694133759 0.270878463983536 249.6765537,230.29867785,214.14730935 \\
2.18582224845886 0.0155065059661865 249.48584175,228.81590385,211.9835043 \\
2.0998866558075 0.667247414588928 249.3025911,227.33242605,209.8164174 \\
1.54754531383514 1.99581432342529 246.26699715,186.7493979,151.06898445 \\
1.6197350025177 -0.599872946739197 248.18406675,216.9345027,194.56450275 \\
0.808260321617126 -2.08476495742798 244.2513471,130.71627165,91.0654521 \\
2.31953120231628 0.213486790657043 247.91115555,213.9578826,190.17620265 \\
1.91814863681793 1.87895154953003 245.7904863,164.4347457,123.960498 \\
2.25438570976257 -0.634360730648041 248.18406675,216.9345027,194.56450275 \\
1.81134963035583 0.846497476100922 250.0727982,233.26498575,218.47535805 \\
1.34294581413269 -0.773568034172058 246.11813835,180.47969505,142.9681215 \\
2.34138655662537 0.268020421266556 247.65640545,210.97774605,185.77259745 \\
0.994910955429077 -1.80972576141357 245.8641966,167.68475895,127.6195797 \\
1.76740860939026 -1.45133209228516 248.6295441,221.39436855,201.1211367 \\
2.02403593063354 -0.47223287820816 250.25125995,234.75740895,220.66913325 \\
2.05516242980957 -1.52416610717773 246.4728561,192.92848905,159.44460225 \\
1.68837881088257 -0.640256226062775 248.9531187,224.36478195,205.4760165 \\
2.25552487373352 -0.70874011516571 247.91115555,213.9578826,190.17620265 \\
2.38696599006653 1.16118121147156 245.7904863,164.4347457,123.960498 \\
1.89023947715759 -1.4914186000824 247.78194195,212.46815985,187.97598105 \\
1.63588428497314 -1.0189391374588 249.1250907,225.84872895,207.6473262 \\
2.50361824035645 0.315472424030304 246.1520712,182.0552406,144.9672297 \\
2.3231258392334 0.397569566965103 247.65640545,210.97774605,185.77259745 \\
1.18191397190094 1.12288796901703 246.02364555,175.72078815,137.0778663 \\
1.82564401626587 -0.846313893795013 250.0727982,233.26498575,218.47535805 \\
1.91328048706055 -1.17536664009094 249.3025911,227.33242605,209.8164174 \\
2.3274040222168 0.943475127220154 246.4137216,191.3927133,157.3251468 \\
1.59333753585815 1.6300653219223 248.18406675,216.9345027,194.56450275 \\
2.00441551208496 -0.827923595905304 249.872205,231.7816839,216.3110379 \\
1.31395292282104 -2.02205419540405 246.1520712,182.0552406,144.9672297 \\
1.36164331436157 -0.622058570384979 246.02364555,175.72078815,137.0778663 \\
1.55311822891235 -0.690030217170715 247.78194195,212.46815985,187.97598105 \\
2.24897718429565 -1.41243934631348 245.96240475,172.5216882,133.24118985 \\
2.02187585830688 -0.123045027256012 250.25125995,234.75740895,220.66913325 \\
2.29320025444031 -0.962411224842072 246.7723791,199.01257995,168.07746405 \\
1.72022581100464 -1.27313578128815 249.3025911,227.33242605,209.8164174 \\
2.58730101585388 1.11039066314697 242.46705855,110.92501275,75.79318335 \\
1.52233397960663 1.74114370346069 247.53321495,209.48702625,183.56726565 \\
1.80844759941101 0.135281383991241 249.3025911,227.33242605,209.8164174 \\
0.890662014484406 -1.60370659828186 245.2421496,147.8709147,106.70345205 \\
1.06460332870483 -1.35588550567627 245.82816765,166.06246425,125.780739 \\
1.37655341625214 -0.464507281780243 245.8981065,169.30226475,129.4754085 \\
1.3635847568512 -1.10050225257874 246.96984345,202.01754075,172.483173 \\
2.03923726081848 0.824947595596313 249.48584175,228.81590385,211.9835043 \\
2.24200344085693 -0.0481866002082825 248.9531187,224.36478195,205.4760165 \\
2.42483115196228 0.0431095957756042 246.86670105,200.51813565,170.27341695 \\
1.64509463310242 1.34418392181396 249.3025911,227.33242605,209.8164174 \\
2.30031156539917 -0.643957555294037 247.65640545,210.97774605,185.77259745 \\
1.64847803115845 -1.34953546524048 248.9531187,224.36478195,205.4760165 \\
1.90633392333984 -1.15446209907532 249.3025911,227.33242605,209.8164174 \\
1.98755466938019 0.385589241981506 250.0727982,233.26498575,218.47535805 \\
2.58310914039612 0.0463674068450928 245.8641966,167.68475895,127.6195797 \\
1.87163710594177 -1.58738875389099 247.299714,206.50190445,179.14325025 \\
2.06303644180298 0.172862887382507 250.0727982,233.26498575,218.47535805 \\
2.32566213607788 -0.711709499359131 247.1866113,205.00830825,176.92716255 \\
1.94159388542175 -0.556514382362366 250.25125995,234.75740895,220.66913325 \\
2.21633195877075 -0.54091739654541 248.78919705,222.87959565,203.29941555 \\
1.5814493894577 -1.38290858268738 248.78919705,222.87959565,203.29941555 \\
1.57044386863708 0.140593260526657 246.86670105,200.51813565,170.27341695 \\
1.88266062736511 1.36793601512909 248.4751161,219.90865695,198.93982845 \\
1.53995442390442 1.42616426944733 248.78919705,222.87959565,203.29941555 \\
1.99530601501465 1.56626653671265 246.53702175,194.45887155,161.57890635 \\
2.3618950843811 0.656022429466248 246.7723791,199.01257995,168.07746405 \\
1.10452675819397 0.217911332845688 235.17106485,72.13249005,61.7853984 \\
1.46789026260376 -1.17921185493469 248.04512235,215.44664655,192.37235955 \\
1.80093860626221 -1.38972640037537 248.78919705,222.87959565,203.29941555 \\
2.02781367301941 0.237354040145874 250.0727982,233.26498575,218.47535805 \\
1.00156199932098 -2.09969139099121 245.4455784,152.896572,111.70308405 \\
2.09039187431335 -0.0530625581741333 250.0727982,233.26498575,218.47535805 \\
1.64376854896545 -0.980647563934326 249.1250907,225.84872895,207.6473262 \\
-0.252036243677139 2.40218257904053 174.640065,23.16836415,89.03085555 \\
1.88459968566895 -0.493893921375275 250.25125995,234.75740895,220.66913325 \\
1.81385374069214 0.309007197618484 249.48584175,228.81590385,211.9835043 \\
1.43304657936096 1.23146283626556 248.18406675,216.9345027,194.56450275 \\
1.65350270271301 1.3230596780777 249.3025911,227.33242605,209.8164174 \\
1.81448554992676 -0.584631562232971 249.872205,231.7816839,216.3110379 \\
0.971031785011292 -2.18909311294556 244.90734735,141.0879912,100.26359535 \\
1.1578027009964 -0.081982284784317 239.06971905,87.92190135,64.2598164 \\
1.14245796203613 -1.435471534729 246.1872816,183.6258162,146.98295625 \\
1.78987407684326 -0.405869483947754 249.48584175,228.81590385,211.9835043 \\
1.373783826828 -0.946357727050781 246.60841665,195.98272095,163.73019855 \\
0.761721670627594 2.0377516746521 244.99843845,142.7928192,101.84999115 \\
1.20412874221802 -1.6348295211792 246.53702175,194.45887155,161.57890635 \\
1.63379955291748 -0.692668199539185 248.6295441,221.39436855,201.1211367 \\
1.70454800128937 0.0862687528133392 248.3271753,218.42187945,196.75380015 \\
1.76244759559631 0.640798091888428 249.48584175,228.81590385,211.9835043 \\
2.31458234786987 1.19355356693268 246.08576865,178.898718,140.986746 \\
2.25490236282349 0.277086168527603 248.4751161,219.90865695,198.93982845 \\
1.43327188491821 -1.55583882331848 247.91115555,213.9578826,190.17620265 \\
1.487020611763 1.08166909217834 248.3271753,218.42187945,196.75380015 \\
1.60832118988037 -1.6047431230545 248.04512235,215.44664655,192.37235955 \\
1.91830205917358 0.214151233434677 250.0727982,233.26498575,218.47535805 \\
1.78709006309509 -0.262397110462189 249.3025911,227.33242605,209.8164174 \\
1.83836221694946 0.160919398069382 249.48584175,228.81590385,211.9835043 \\
1.97337937355042 1.41240990161896 247.65640545,210.97774605,185.77259745 \\
2.21836447715759 0.506040990352631 248.6295441,221.39436855,201.1211367 \\
2.05877494812012 0.35681140422821 249.872205,231.7816839,216.3110379 \\
1.77807521820068 -0.674190163612366 249.872205,231.7816839,216.3110379 \\
1.33653926849365 1.53478467464447 247.41505305,207.99484275,181.3567548 \\
1.23336219787598 -0.887089848518372 245.82816765,166.06246425,125.780739 \\
2.01319289207458 1.05192267894745 248.9531187,224.36478195,205.4760165 \\
1.46956646442413 0.452501267194748 246.4137216,191.3927133,157.3251468 \\
1.34894871711731 -1.30280542373657 247.299714,206.50190445,179.14325025 \\
1.76279532909393 1.36192154884338 249.1250907,225.84872895,207.6473262 \\
1.56694972515106 1.4085887670517 248.9531187,224.36478195,205.4760165 \\
2.25120997428894 -1.24959409236908 246.26699715,186.7493979,151.06898445 \\
1.52177429199219 0.757739543914795 247.78194195,212.46815985,187.97598105 \\
2.04423570632935 -1.23452734947205 248.04512235,215.44664655,192.37235955 \\
2.09785795211792 -0.528278052806854 249.872205,231.7816839,216.3110379 \\
1.8308070898056 0.614418387413025 249.872205,231.7816839,216.3110379 \\
1.60588991641998 0.124842405319214 247.1866113,205.00830825,176.92716255 \\
2.23262596130371 -1.29477512836456 246.26699715,186.7493979,151.06898445 \\
1.78201603889465 -0.0205931663513184 249.1250907,225.84872895,207.6473262 \\
1.38892602920532 -1.9478919506073 246.36063315,189.8505141,155.2234827 \\
0.0583504438400269 2.14325356483459 228.20035935,55.5533157,64.5516231 \\
1.22981190681458 1.86926698684692 246.36063315,189.8505141,155.2234827 \\
2.17985200881958 -1.47234976291656 246.08576865,178.898718,140.986746 \\
1.78987574577332 0.0501962602138519 249.1250907,225.84872895,207.6473262 \\
1.88862824440002 1.99795341491699 245.3822772,151.2267249,110.02044105 \\
0.600891590118408 -1.33622145652771 230.85174735,60.82907955,63.1076499 \\
1.91797411441803 0.537098109722137 250.0727982,233.26498575,218.47535805 \\
1.77801334857941 -0.554003179073334 249.6765537,230.29867785,214.14730935 \\
1.31118297576904 -1.1409102678299 246.60841665,195.98272095,163.73019855 \\
1.44603395462036 0.912949562072754 247.53321495,209.48702625,183.56726565 \\
1.86844658851624 -0.987893998622894 249.872205,231.7816839,216.3110379 \\
0.807831764221191 1.61786925792694 244.99843845,142.7928192,101.84999115 \\
2.18674898147583 -1.39505100250244 246.2259753,185.1901596,149.01788685 \\
2.07627749443054 -1.16905641555786 248.18406675,216.9345027,194.56450275 \\
1.68206262588501 -1.3220990896225 249.1250907,225.84872895,207.6473262 \\
0.795299172401428 -0.296332269906998 152.41404825,27.2691237,91.38697395 \\
1.33681583404541 1.13743996620178 247.0774866,203.5132356,174.7062783 \\
1.92163181304932 0.0448712706565857 250.0727982,233.26498575,218.47535805 \\
2.18954133987427 0.592822551727295 248.6295441,221.39436855,201.1211367 \\
1.92526745796204 -0.699018478393555 250.25125995,234.75740895,220.66913325 \\
2.08719968795776 1.55158710479736 246.1872816,183.6258162,146.98295625 \\
2.14328575134277 0.692368030548096 248.9531187,224.36478195,205.4760165 \\
1.48674607276917 -1.54651939868927 248.04512235,215.44664655,192.37235955 \\
2.30263161659241 -0.00096207857131958 248.18406675,216.9345027,194.56450275 \\
1.57861948013306 1.25032353401184 249.1250907,225.84872895,207.6473262 \\
1.78705275058746 0.102170705795288 249.1250907,225.84872895,207.6473262 \\
2.37496304512024 0.255022883415222 247.299714,206.50190445,179.14325025 \\
2.09895944595337 -0.201847910881042 250.0727982,233.26498575,218.47535805 \\
2.02277565002441 -0.773902833461761 249.872205,231.7816839,216.3110379 \\
1.89168429374695 -0.321432530879974 250.25125995,234.75740895,220.66913325 \\
2.10063123703003 0.279865950345993 249.872205,231.7816839,216.3110379 \\
1.5738605260849 -1.78886866569519 247.1866113,205.00830825,176.92716255 \\
1.82351338863373 1.9241281747818 245.96240475,172.5216882,133.24118985 \\
2.14172196388245 0.824599266052246 248.6295441,221.39436855,201.1211367 \\
1.26369547843933 1.80396270751953 246.60841665,195.98272095,163.73019855 \\
2.08015608787537 -1.36306977272034 247.0774866,203.5132356,174.7062783 \\
1.59090256690979 -1.51814007759094 248.4751161,219.90865695,198.93982845 \\
2.28943777084351 0.0426746606826782 248.3271753,218.42187945,196.75380015 \\
2.0427393913269 -0.07863849401474 250.25125995,234.75740895,220.66913325 \\
1.74930989742279 2.28700923919678 243.22306725,118.25314155,81.01330365 \\
1.59736466407776 0.20311164855957 247.1866113,205.00830825,176.92716255 \\
1.05821347236633 -2.33348560333252 243.98829675,127.1991204,88.11634905 \\
0.526191473007202 2.43580055236816 235.17106485,72.13249005,61.7853984 \\
2.38062620162964 -0.189689517021179 247.299714,206.50190445,179.14325025 \\
1.42219626903534 1.04693794250488 247.65640545,210.97774605,185.77259745 \\
2.3100860118866 0.196010828018188 248.04512235,215.44664655,192.37235955 \\
2.28683686256409 -0.203972220420837 248.3271753,218.42187945,196.75380015 \\
1.68649125099182 1.24783074855804 249.6765537,230.29867785,214.14730935 \\
1.48921394348145 -2.07864904403687 245.9932011,174.1238073,135.15058905 \\
2.2571907043457 0.422377288341522 248.3271753,218.42187945,196.75380015 \\
1.68041944503784 -0.077566921710968 248.04512235,215.44664655,192.37235955 \\
1.59576082229614 2.04613828659058 246.08576865,178.898718,140.986746 \\
1.7138032913208 -1.04554843902588 249.6765537,230.29867785,214.14730935 \\
1.7295480966568 -1.64794778823853 247.65640545,210.97774605,185.77259745 \\
1.79754829406738 -1.20777583122253 249.48584175,228.81590385,211.9835043 \\
2.35742211341858 1.39677894115448 245.31448035,149.55157125,108.35396505 \\
0.807329773902893 -1.7692631483078 244.81107975,139.37686215,98.692446 \\
1.8055317401886 1.12336659431458 249.872205,231.7816839,216.3110379 \\
1.91014587879181 -1.63344931602478 246.7723791,199.01257995,168.07746405 \\
2.14614057540894 0.0488486289978027 249.6765537,230.29867785,214.14730935 \\
1.02030539512634 1.69297850131989 246.02364555,175.72078815,137.0778663 \\
2.58738970756531 0.140874087810516 245.82816765,166.06246425,125.780739 \\
2.14011573791504 -1.07512545585632 247.91115555,213.9578826,190.17620265 \\
1.99029612541199 0.59303092956543 250.0727982,233.26498575,218.47535805 \\
1.65071785449982 0.991764307022095 249.48584175,228.81590385,211.9835043 \\
1.39129257202148 -1.5091540813446 247.65640545,210.97774605,185.77259745 \\
2.32241797447205 0.80077862739563 246.7723791,199.01257995,168.07746405 \\
1.79867947101593 -1.43984246253967 248.6295441,221.39436855,201.1211367 \\
1.35600519180298 2.00732755661011 246.26699715,186.7493979,151.06898445 \\
1.45616149902344 2.30055642127991 245.16534615,146.18443905,105.06918765 \\
1.66377329826355 -1.98313760757446 246.08576865,178.898718,140.986746 \\
2.05264616012573 -1.00021743774414 248.9531187,224.36478195,205.4760165 \\
2.18248629570007 -0.333937525749207 249.3025911,227.33242605,209.8164174 \\
2.25476551055908 0.629198729991913 247.91115555,213.9578826,190.17620265 \\
1.30713582038879 -1.70098483562469 246.96984345,202.01754075,172.483173 \\
2.18259334564209 -1.79061532020569 244.4915877,134.20217775,94.0718052 \\
1.71020936965942 -1.00544548034668 249.6765537,230.29867785,214.14730935 \\
2.2814154624939 -0.301448881626129 248.3271753,218.42187945,196.75380015 \\
1.88976359367371 0.170714229345322 249.872205,231.7816839,216.3110379 \\
1.64917159080505 1.37983596324921 249.3025911,227.33242605,209.8164174 \\
2.34126281738281 1.29157364368439 245.7904863,164.4347457,123.960498 \\
1.94153690338135 -1.50705337524414 247.41505305,207.99484275,181.3567548 \\
2.79759907722473 -0.521064877510071 239.06971905,87.92190135,64.2598164 \\
0.588855922222137 1.88169836997986 243.98829675,127.1991204,88.11634905 \\
0.369808793067932 -2.28746485710144 224.2109889,48.96831045,66.9837468 \\
2.16914820671082 1.00441384315491 247.78194195,212.46815985,187.97598105 \\
1.10340666770935 -1.43963956832886 246.05468925,177.3121896,139.0238835 \\
1.95432567596436 -0.243198394775391 250.25125995,234.75740895,220.66913325 \\
2.23253178596497 -0.772730886936188 248.04512235,215.44664655,192.37235955 \\
1.5428501367569 1.51716470718384 248.4751161,219.90865695,198.93982845 \\
1.22368991374969 1.75471591949463 246.53702175,194.45887155,161.57890635 \\
1.49304378032684 -0.838517427444458 247.53321495,209.48702625,183.56726565 \\
1.97437930107117 -0.67895382642746 250.25125995,234.75740895,220.66913325 \\
2.03141975402832 -0.00457799434661865 250.25125995,234.75740895,220.66913325 \\
1.4473739862442 -1.45288467407227 248.04512235,215.44664655,192.37235955 \\
1.48217725753784 0.770947575569153 247.41505305,207.99484275,181.3567548 \\
1.90318500995636 1.0795636177063 249.6765537,230.29867785,214.14730935 \\
1.05403327941895 2.47834396362305 242.2543452,109.0670649,74.5740564 \\
1.6518087387085 -0.546836972236633 248.4751161,219.90865695,198.93982845 \\
2.09671115875244 0.16248631477356 249.872205,231.7816839,216.3110379 \\
0.989424109458923 -1.69972956180573 245.82816765,166.06246425,125.780739 \\
1.50939309597015 -1.96744418144226 246.31171395,188.3028579,153.13767705 \\
2.37930798530579 0.792580664157867 246.36063315,189.8505141,155.2234827 \\
1.44340419769287 1.63306391239166 247.78194195,212.46815985,187.97598105 \\
1.98251986503601 -0.645146489143372 250.25125995,234.75740895,220.66913325 \\
1.51352691650391 -0.818407297134399 247.65640545,210.97774605,185.77259745 \\
1.70901012420654 -0.265491724014282 248.4751161,219.90865695,198.93982845 \\
1.73310303688049 -1.14274311065674 249.6765537,230.29867785,214.14730935 \\
1.58235120773315 0.410580396652222 247.41505305,207.99484275,181.3567548 \\
0.422294080257416 1.75206446647644 241.5538908,103.4256795,71.1509211 \\
1.93207359313965 -0.934231281280518 249.872205,231.7816839,216.3110379 \\
1.05270421504974 -2.1174213886261 245.56115205,156.2192985,115.1199387 \\
1.52934241294861 0.650030732154846 247.53321495,209.48702625,183.56726565 \\
1.88331353664398 -1.42453503608704 248.18406675,216.9345027,194.56450275 \\
1.95750820636749 -0.310150623321533 250.25125995,234.75740895,220.66913325 \\
2.15873408317566 0.392710566520691 249.3025911,227.33242605,209.8164174 \\
0.454862236976624 1.9575070142746 242.67077295,112.7715672,77.0469597 \\
1.22630429267883 1.30736124515533 246.4728561,192.92848905,159.44460225 \\
1.62814712524414 -0.219236612319946 247.53321495,209.48702625,183.56726565 \\
2.45525002479553 0.484381198883057 246.26699715,186.7493979,151.06898445 \\
1.94527840614319 -1.72261691093445 246.2259753,185.1901596,149.01788685 \\
2.20821189880371 0.448860347270966 248.78919705,222.87959565,203.29941555 \\
2.593013048172 -0.0734830498695374 245.82816765,166.06246425,125.780739 \\
1.79680037498474 0.631193041801453 249.872205,231.7816839,216.3110379 \\
1.66594898700714 1.52638518810272 248.6295441,221.39436855,201.1211367 \\
2.29878425598145 -0.710089802742004 247.53321495,209.48702625,183.56726565 \\
2.31508350372314 0.787302136421204 246.96984345,202.01754075,172.483173 \\
1.6673309803009 1.77962183952332 247.0774866,203.5132356,174.7062783 \\
1.77172589302063 0.0766711831092834 248.9531187,224.36478195,205.4760165 \\
1.65802049636841 0.539418816566467 248.4751161,219.90865695,198.93982845 \\
1.95177412033081 0.163542985916138 250.25125995,234.75740895,220.66913325 \\
1.05925047397614 -2.12930035591125 245.56115205,156.2192985,115.1199387 \\
2.16801357269287 0.69639003276825 248.6295441,221.39436855,201.1211367 \\
1.68688607215881 -1.68534398078918 247.53321495,209.48702625,183.56726565 \\
1.70180118083954 0.767398953437805 249.3025911,227.33242605,209.8164174 \\
1.61274695396423 1.23848366737366 249.3025911,227.33242605,209.8164174 \\
1.7540340423584 0.313770741224289 248.9531187,224.36478195,205.4760165 \\
2.04558539390564 0.983898758888245 248.9531187,224.36478195,205.4760165 \\
1.84027910232544 2.06763339042664 245.16534615,146.18443905,105.06918765 \\
2.66036176681519 -0.688616394996643 244.1226945,128.96176965,89.5838256 \\
1.66997647285461 1.88060390949249 246.4728561,192.92848905,159.44460225 \\
2.11274099349976 0.690360248088837 249.1250907,225.84872895,207.6473262 \\
1.34749341011047 -1.98462283611298 246.26699715,186.7493979,151.06898445 \\
2.1970112323761 -0.380943477153778 249.1250907,225.84872895,207.6473262 \\
1.41803991794586 -1.86147463321686 246.7723791,199.01257995,168.07746405 \\
1.67755198478699 0.263181507587433 248.18406675,216.9345027,194.56450275 \\
1.53471636772156 1.80084025859833 247.1866113,205.00830825,176.92716255 \\
1.34503972530365 0.116805642843246 245.31448035,149.55157125,108.35396505 \\
1.32800447940826 -2.19215250015259 245.66176995,159.521268,118.60374105 \\
1.85193860530853 -0.658770680427551 250.25125995,234.75740895,220.66913325 \\
2.12211322784424 -1.40648341178894 246.53702175,194.45887155,161.57890635 \\
0.757905662059784 -2.82236528396606 177.99958035,22.6879263,88.306143 \\
1.86036109924316 1.36064648628235 248.6295441,221.39436855,201.1211367 \\
0.851461112499237 -1.23355638980865 242.67077295,112.7715672,77.0469597 \\
1.22093892097473 1.01252567768097 246.02364555,175.72078815,137.0778663 \\
1.6105922460556 -1.50122880935669 248.4751161,219.90865695,198.93982845 \\
2.08524966239929 1.21726942062378 247.65640545,210.97774605,185.77259745 \\
2.19963073730469 0.0279435515403748 249.3025911,227.33242605,209.8164174 \\
1.9812126159668 -0.63957804441452 250.25125995,234.75740895,220.66913325 \\
1.37664771080017 1.55410468578339 247.65640545,210.97774605,185.77259745 \\
1.69778883457184 0.315902292728424 248.4751161,219.90865695,198.93982845 \\
2.52833795547485 -0.643405199050903 245.8641966,167.68475895,127.6195797 \\
2.08034181594849 0.301942497491837 249.872205,231.7816839,216.3110379 \\
1.54220771789551 1.11318135261536 248.78919705,222.87959565,203.29941555 \\
1.61525130271912 -0.518652677536011 247.91115555,213.9578826,190.17620265 \\
1.456467628479 0.0975814461708069 246.05468925,177.3121896,139.0238835 \\
1.80410957336426 -0.607716798782349 249.872205,231.7816839,216.3110379 \\
2.53210520744324 -1.05173742771149 244.81107975,139.37686215,98.692446 \\
2.16340851783752 -0.0953818559646606 249.6765537,230.29867785,214.14730935 \\
1.95749413967133 -0.178070724010468 250.25125995,234.75740895,220.66913325 \\
1.27044713497162 1.37009394168854 246.86670105,200.51813565,170.27341695 \\
0.587867856025696 -1.82993805408478 241.29624135,101.5235784,70.0988421 \\
1.70093631744385 -0.17360657453537 248.3271753,218.42187945,196.75380015 \\
1.24453067779541 -0.500458240509033 244.7095719,137.65917705,97.1365992 \\
1.54806220531464 -2.00240087509155 246.1520712,182.0552406,144.9672297 \\
1.97503089904785 0.40978616476059 250.0727982,233.26498575,218.47535805 \\
2.03532910346985 -1.79017281532288 245.75060175,162.8018226,122.1578214 \\
1.90406429767609 -0.304155111312866 250.25125995,234.75740895,220.66913325 \\
1.65149629116058 -0.945951581001282 249.1250907,225.84872895,207.6473262 \\
1.81890296936035 -1.80125188827515 246.36063315,189.8505141,155.2234827 \\
2.17942476272583 -1.34658765792847 246.4137216,191.3927133,157.3251468 \\
2.04722309112549 -0.31633472442627 250.25125995,234.75740895,220.66913325 \\
1.62350916862488 -1.81337022781372 246.86670105,200.51813565,170.27341695 \\
1.16989755630493 -1.82381248474121 246.26699715,186.7493979,151.06898445 \\
0.643190085887909 0.636384844779968 135.30007515,29.73923475,90.7442541 \\
2.03511548042297 -0.677646815776825 249.872205,231.7816839,216.3110379 \\
2.67869710922241 0.672576308250427 243.0470178,116.43654195,79.6614288 \\
1.92741060256958 0.0188726186752319 250.0727982,233.26498575,218.47535805 \\
2.92868494987488 0.357245147228241 217.46842935,39.9273186,71.1700563 \\
1.78588438034058 0.614577889442444 249.6765537,230.29867785,214.14730935 \\
1.3967102766037 1.39399886131287 247.91115555,213.9578826,190.17620265 \\
0.436998128890991 2.06200337409973 242.03209995,107.19754005,73.39147095 \\
2.14390563964844 -0.080964982509613 249.6765537,230.29867785,214.14730935 \\
1.41617107391357 0.675931215286255 246.4728561,192.92848905,159.44460225 \\
2.12109899520874 1.59492456912994 245.96240475,172.5216882,133.24118985 \\
1.98368453979492 -0.784880042076111 250.0727982,233.26498575,218.47535805 \\
1.57146155834198 1.12195038795471 249.1250907,225.84872895,207.6473262 \\
1.91267263889313 0.56970226764679 250.0727982,233.26498575,218.47535805 \\
1.23222351074219 1.34723377227783 246.60841665,195.98272095,163.73019855 \\
2.11522364616394 -0.513000965118408 249.6765537,230.29867785,214.14730935 \\
-0.157353729009628 -1.87283492088318 137.00174115,29.54779095,90.88940265 \\
0.003132164478302 2.21192169189453 220.9361514,44.3388798,69.0305859 \\
0.91094708442688 1.99018812179565 245.61336585,157.87274145,116.8535001 \\
1.53285729885101 0.119822949171066 246.4728561,192.92848905,159.44460225 \\
2.48611879348755 0.751238167285919 245.8981065,169.30226475,129.4754085 \\
1.37387502193451 -0.341355323791504 245.75060175,162.8018226,122.1578214 \\
1.70831489562988 -1.49686217308044 248.4751161,219.90865695,198.93982845 \\
2.83193254470825 -1.02847290039062 215.05537455,37.11954165,72.62608845 \\
0.98858642578125 1.28966212272644 245.16534615,146.18443905,105.06918765 \\
1.47205591201782 -1.25311899185181 248.18406675,216.9345027,194.56450275 \\
1.19683039188385 2.4234561920166 243.8483094,125.42774505,86.66328255 \\
1.35629916191101 -1.49331116676331 247.53321495,209.48702625,183.56726565 \\
1.38218021392822 -1.27995121479034 247.53321495,209.48702625,183.56726565 \\
1.36631560325623 -2.35724949836731 243.98829675,127.1991204,88.11634905 \\
1.75027525424957 0.738681316375732 249.6765537,230.29867785,214.14730935 \\
1.47753202915192 -1.54270792007446 248.04512235,215.44664655,192.37235955 \\
1.25874531269073 0.777941405773163 245.8641966,167.68475895,127.6195797 \\
1.86402940750122 -0.568671822547913 250.25125995,234.75740895,220.66913325 \\
3.49558448791504 -1.25509703159332 2.70507825,4.61094825,25.5475677 \\
1.53487801551819 -1.75188398361206 247.41505305,207.99484275,181.3567548 \\
2.29143667221069 -0.545045077800751 247.91115555,213.9578826,190.17620265 \\
1.77167844772339 -1.54692208766937 248.04512235,215.44664655,192.37235955 \\
1.90822601318359 -0.380950570106506 250.25125995,234.75740895,220.66913325 \\
2.35989546775818 -0.187809884548187 247.53321495,209.48702625,183.56726565 \\
1.84837782382965 1.58856666088104 247.41505305,207.99484275,181.3567548 \\
1.55564022064209 -1.20714259147644 248.78919705,222.87959565,203.29941555 \\
0.503868341445923 1.44611525535583 234.54423405,70.19178705,61.82723115 \\
2.1190824508667 -0.0837098360061646 249.872205,231.7816839,216.3110379 \\
1.8731609582901 0.420235097408295 249.872205,231.7816839,216.3110379 \\
0.940121352672577 2.12679743766785 245.3822772,151.2267249,110.02044105 \\
1.68042230606079 0.303039014339447 248.18406675,216.9345027,194.56450275 \\
1.64685797691345 1.10133302211761 249.48584175,228.81590385,211.9835043 \\
1.82209467887878 0.795164406299591 250.0727982,233.26498575,218.47535805 \\
1.7811313867569 1.38891804218292 248.9531187,224.36478195,205.4760165 \\
2.16749811172485 -0.553350031375885 249.1250907,225.84872895,207.6473262 \\
1.95529067516327 -1.37860333919525 248.04512235,215.44664655,192.37235955 \\
1.58193624019623 1.77945053577423 247.299714,206.50190445,179.14325025 \\
2.0330548286438 1.46739649772644 246.7723791,199.01257995,168.07746405 \\
2.21039414405823 -1.25875639915466 246.4728561,192.92848905,159.44460225 \\
1.40855169296265 1.23367166519165 247.91115555,213.9578826,190.17620265 \\
1.06897592544556 0.879396140575409 243.8483094,125.42774505,86.66328255 \\
0.781986832618713 1.69508576393127 245.084376,144.4915935,103.45182465 \\
1.82999885082245 -0.7776780128479 250.0727982,233.26498575,218.47535805 \\
1.62273979187012 -1.82365751266479 246.86670105,200.51813565,170.27341695 \\
2.74465823173523 -0.165860950946808 243.55115025,121.8574518,83.79951015 \\
2.74990963935852 0.790006279945374 236.8515582,78.02511675,62.1237987 \\
2.26136159896851 0.416734665632248 248.3271753,218.42187945,196.75380015 \\
1.08468627929688 1.48211288452148 246.05468925,177.3121896,139.0238835 \\
1.40988039970398 1.81497824192047 246.96984345,202.01754075,172.483173 \\
1.54678964614868 -1.22739589214325 248.6295441,221.39436855,201.1211367 \\
1.00378823280334 2.05768013000488 245.75060175,162.8018226,122.1578214 \\
1.71404790878296 -0.882233262062073 249.6765537,230.29867785,214.14730935 \\
1.69437766075134 -1.30234217643738 249.3025911,227.33242605,209.8164174 \\
1.71669101715088 -0.326413094997406 248.6295441,221.39436855,201.1211367 \\
2.07012677192688 0.816876888275146 249.3025911,227.33242605,209.8164174 \\
1.69661498069763 -0.441569864749908 248.6295441,221.39436855,201.1211367 \\
2.14441704750061 0.862068176269531 248.4751161,219.90865695,198.93982845 \\
1.61459600925446 -1.47878503799438 248.6295441,221.39436855,201.1211367 \\
0.591274261474609 -2.02550601959229 241.02706845,99.60893895,69.0920562 \\
1.86931800842285 -0.653702914714813 250.25125995,234.75740895,220.66913325 \\
1.85334742069244 0.000805675983428955 249.6765537,230.29867785,214.14730935 \\
1.372802734375 -1.65318179130554 247.41505305,207.99484275,181.3567548 \\
1.58266294002533 0.534758925437927 247.78194195,212.46815985,187.97598105 \\
1.8274906873703 -1.17537593841553 249.48584175,228.81590385,211.9835043 \\
1.01056957244873 -1.61910176277161 245.8981065,169.30226475,129.4754085 \\
1.48024642467499 -0.265983462333679 246.2259753,185.1901596,149.01788685 \\
1.93698000907898 -0.80314826965332 250.0727982,233.26498575,218.47535805 \\
1.67601478099823 2.03603935241699 245.9932011,174.1238073,135.15058905 \\
1.85433006286621 0.883996486663818 250.0727982,233.26498575,218.47535805 \\
1.39373087882996 -1.40877366065979 247.78194195,212.46815985,187.97598105 \\
1.96437764167786 -1.25858092308044 248.6295441,221.39436855,201.1211367 \\
1.79089891910553 1.0835018157959 249.872205,231.7816839,216.3110379 \\
1.84631955623627 -1.1339693069458 249.6765537,230.29867785,214.14730935 \\
1.67022347450256 -0.00515079498291016 247.91115555,213.9578826,190.17620265 \\
1.95324277877808 0.260309159755707 250.0727982,233.26498575,218.47535805 \\
1.85449409484863 -1.45863246917725 248.18406675,216.9345027,194.56450275 \\
0.582411766052246 -2.46532034873962 225.2521335,50.5679178,66.33616155 \\
1.67461061477661 -1.74795603752136 247.1866113,205.00830825,176.92716255 \\
1.93261075019836 0.0174285769462585 250.0727982,233.26498575,218.47535805 \\
2.22217607498169 0.0565142631530762 249.1250907,225.84872895,207.6473262 \\
1.63736140727997 1.1032303571701 249.48584175,228.81590385,211.9835043 \\
1.25712609291077 1.03982424736023 246.2259753,185.1901596,149.01788685 \\
1.72495031356812 0.603094279766083 249.1250907,225.84872895,207.6473262 \\
2.29365611076355 -1.32954597473145 245.9304711,170.9147751,131.3488017 \\
1.29036068916321 2.46170425415039 243.22306725,118.25314155,81.01330365 \\
2.13413691520691 0.65978217124939 248.9531187,224.36478195,205.4760165 \\
1.56533288955688 -0.579684019088745 247.53321495,209.48702625,183.56726565 \\
1.77372670173645 -1.88980507850647 246.1520712,182.0552406,144.9672297 \\
2.70359992980957 -0.548575162887573 243.55115025,121.8574518,83.79951015 \\
2.24676418304443 0.454093813896179 248.3271753,218.42187945,196.75380015 \\
2.20378947257996 -0.294496953487396 249.1250907,225.84872895,207.6473262 \\
2.26142287254333 -2.45758581161499 166.1441331,24.6844896,90.40797285 \\
1.50092697143555 1.71523940563202 247.65640545,210.97774605,185.77259745 \\
0.981879353523254 1.72432279586792 245.9304711,170.9147751,131.3488017 \\
2.35928773880005 0.238139271736145 247.41505305,207.99484275,181.3567548 \\
1.14540553092957 1.91337382793427 246.1520712,182.0552406,144.9672297 \\
2.00971865653992 -0.176542282104492 250.25125995,234.75740895,220.66913325 \\
2.12813830375671 -0.598439455032349 249.48584175,228.81590385,211.9835043 \\
1.68996739387512 -0.474424302577972 248.6295441,221.39436855,201.1211367 \\
1.85347163677216 0.758149743080139 250.0727982,233.26498575,218.47535805 \\
2.0132794380188 -0.542115032672882 250.25125995,234.75740895,220.66913325 \\
1.46442174911499 0.740583419799805 247.1866113,205.00830825,176.92716255 \\
2.19935488700867 0.42854768037796 248.78919705,222.87959565,203.29941555 \\
1.90738582611084 -1.42888998985291 248.04512235,215.44664655,192.37235955 \\
2.32781434059143 -1.03495812416077 246.36063315,189.8505141,155.2234827 \\
1.27449989318848 1.29004943370819 246.7723791,199.01257995,168.07746405 \\
1.76563572883606 1.13965129852295 249.872205,231.7816839,216.3110379 \\
1.86978387832642 -0.324812054634094 250.0727982,233.26498575,218.47535805 \\
2.0435996055603 0.116270512342453 250.0727982,233.26498575,218.47535805 \\
1.73907625675201 -1.19910430908203 249.48584175,228.81590385,211.9835043 \\
1.85993885993958 0.118026882410049 249.6765537,230.29867785,214.14730935 \\
2.16221308708191 0.813788115978241 248.4751161,219.90865695,198.93982845 \\
1.82548296451569 1.52694272994995 247.91115555,213.9578826,190.17620265 \\
1.81641256809235 -0.258702218532562 249.6765537,230.29867785,214.14730935 \\
1.99984288215637 -0.561495244503021 250.25125995,234.75740895,220.66913325 \\
2.23803544044495 -0.802429139614105 247.91115555,213.9578826,190.17620265 \\
1.43139135837555 -1.54458892345428 247.91115555,213.9578826,190.17620265 \\
2.77417993545532 -0.706115245819092 238.25942085,83.9673231,63.1770864 \\
1.80679309368134 0.308377861976624 249.48584175,228.81590385,211.9835043 \\
2.02637624740601 0.551774740219116 249.872205,231.7816839,216.3110379 \\
1.6544930934906 0.852072954177856 249.3025911,227.33242605,209.8164174 \\
2.57752966880798 0.613117456436157 245.4455784,152.896572,111.70308405 \\
1.63887453079224 -0.546759068965912 248.3271753,218.42187945,196.75380015 \\
1.45011472702026 0.535057425498962 246.4137216,191.3927133,157.3251468 \\
1.50383317470551 0.978511035442352 248.3271753,218.42187945,196.75380015 \\
1.78864979743958 -1.00870788097382 249.872205,231.7816839,216.3110379 \\
1.88303136825562 0.425604701042175 250.0727982,233.26498575,218.47535805 \\
1.81960272789001 0.365746259689331 249.48584175,228.81590385,211.9835043 \\
1.65632903575897 2.267733335495 244.6033236,135.93415815,95.5968225 \\
2.11490964889526 1.25862789154053 247.1866113,205.00830825,176.92716255 \\
1.03139019012451 2.42170166969299 243.22306725,118.25314155,81.01330365 \\
1.58146786689758 1.94554603099823 246.4137216,191.3927133,157.3251468 \\
2.31067442893982 -0.635268330574036 247.53321495,209.48702625,183.56726565 \\
2.13234400749207 -0.16079705953598 249.872205,231.7816839,216.3110379 \\
2.32756567001343 0.0684897303581238 247.91115555,213.9578826,190.17620265 \\
1.55769097805023 1.0215812921524 248.78919705,222.87959565,203.29941555 \\
2.23392558097839 -0.0146951675415039 248.9531187,224.36478195,205.4760165 \\
1.89675223827362 0.624090731143951 250.0727982,233.26498575,218.47535805 \\
2.05602312088013 -0.987080097198486 248.9531187,224.36478195,205.4760165 \\
2.08292555809021 -0.523671329021454 249.872205,231.7816839,216.3110379 \\
1.17320287227631 -1.33890104293823 246.1872816,183.6258162,146.98295625 \\
2.09665155410767 0.2890485227108 249.872205,231.7816839,216.3110379 \\
1.81954622268677 1.10746741294861 249.872205,231.7816839,216.3110379 \\
2.01245641708374 2.33035707473755 224.2109889,48.96831045,66.9837468 \\
1.60454273223877 -1.59087765216827 248.18406675,216.9345027,194.56450275 \\
2.04115724563599 0.00287550687789917 250.25125995,234.75740895,220.66913325 \\
1.73718929290771 -1.00273597240448 249.6765537,230.29867785,214.14730935 \\
1.55782413482666 1.0259964466095 248.78919705,222.87959565,203.29941555 \\
1.81783747673035 1.08003735542297 249.872205,231.7816839,216.3110379 \\
2.25015783309937 1.08889675140381 246.68642115,197.50080735,165.8961201 \\
1.56831741333008 -1.35000538825989 248.78919705,222.87959565,203.29941555 \\
1.97179341316223 0.67327344417572 250.0727982,233.26498575,218.47535805 \\
1.64825010299683 0.0227621793746948 247.65640545,210.97774605,185.77259745 \\
2.38595485687256 -0.665373623371124 246.68642115,197.50080735,165.8961201 \\
2.09746909141541 -0.361933290958405 250.0727982,233.26498575,218.47535805 \\
2.35730528831482 0.320093870162964 247.41505305,207.99484275,181.3567548 \\
1.42422020435333 0.435743629932404 246.1520712,182.0552406,144.9672297 \\
1.95328855514526 0.369243264198303 250.0727982,233.26498575,218.47535805 \\
1.18881225585938 -2.29233026504517 244.90734735,141.0879912,100.26359535 \\
1.63868927955627 1.01033174991608 249.48584175,228.81590385,211.9835043 \\
2.33147573471069 -0.627565026283264 247.299714,206.50190445,179.14325025 \\
1.72492933273315 0.661676287651062 249.3025911,227.33242605,209.8164174 \\
2.35142755508423 0.393612146377563 247.41505305,207.99484275,181.3567548 \\
2.31316471099854 0.465995341539383 247.65640545,210.97774605,185.77259745 \\
1.49392640590668 -0.165199816226959 246.2259753,185.1901596,149.01788685 \\
1.64360344409943 0.76830518245697 248.9531187,224.36478195,205.4760165 \\
1.76017725467682 -0.224611401557922 248.9531187,224.36478195,205.4760165 \\
1.71339869499207 0.187850952148438 248.4751161,219.90865695,198.93982845 \\
1.08646547794342 1.82894837856293 246.11813835,180.47969505,142.9681215 \\
2.54578399658203 0.876665711402893 245.084376,144.4915935,103.45182465 \\
1.04753625392914 -2.05895090103149 245.70760365,161.1641412,120.3721023 \\
2.09484767913818 1.67303669452667 245.8641966,167.68475895,127.6195797 \\
1.33879363536835 1.98558878898621 246.31171395,188.3028579,153.13767705 \\
1.81532156467438 -0.154222905635834 249.48584175,228.81590385,211.9835043 \\
2.04831600189209 1.07188832759857 248.6295441,221.39436855,201.1211367 \\
1.59451806545258 1.60103511810303 248.3271753,218.42187945,196.75380015 \\
1.84390640258789 1.04011416435242 249.872205,231.7816839,216.3110379 \\
1.65796995162964 -1.66264283657074 247.78194195,212.46815985,187.97598105 \\
1.92026662826538 -0.353531241416931 250.25125995,234.75740895,220.66913325 \\
2.1128716468811 0.372382283210754 249.6765537,230.29867785,214.14730935 \\
1.96616065502167 -0.744734942913055 250.0727982,233.26498575,218.47535805 \\
2.37080121040344 0.813012480735779 246.4137216,191.3927133,157.3251468 \\
1.62349653244019 -1.16136062145233 249.1250907,225.84872895,207.6473262 \\
1.62880825996399 -0.374426424503326 247.78194195,212.46815985,187.97598105 \\
1.76506519317627 0.690693616867065 249.6765537,230.29867785,214.14730935 \\
1.67583858966827 -2.10188627243042 245.66176995,159.521268,118.60374105 \\
0.800133526325226 1.83377707004547 245.3822772,151.2267249,110.02044105 \\
1.18641412258148 1.12743377685547 246.05468925,177.3121896,139.0238835 \\
0.907258093357086 -1.61346089839935 245.3822772,151.2267249,110.02044105 \\
1.8093249797821 -0.403139650821686 249.6765537,230.29867785,214.14730935 \\
1.92798066139221 -0.677013695240021 250.25125995,234.75740895,220.66913325 \\
2.507648229599 0.439263880252838 246.05468925,177.3121896,139.0238835 \\
2.21362400054932 -1.57758450508118 245.66176995,159.521268,118.60374105 \\
2.91715049743652 0.279617547988892 223.14356655,47.3977884,67.65072735 \\
1.51189935207367 1.65109503269196 247.91115555,213.9578826,190.17620265 \\
2.17218518257141 1.58958196640015 245.7904863,164.4347457,123.960498 \\
2.01529884338379 1.23548579216003 248.18406675,216.9345027,194.56450275 \\
1.82153296470642 -1.82497358322144 246.26699715,186.7493979,151.06898445 \\
1.64539539813995 0.172270953655243 247.65640545,210.97774605,185.77259745 \\
1.42732501029968 -2.12908315658569 245.8641966,167.68475895,127.6195797 \\
1.51822543144226 -1.90670049190521 246.53702175,194.45887155,161.57890635 \\
2.22730350494385 -0.209742903709412 248.9531187,224.36478195,205.4760165 \\
0.877105474472046 -1.64385080337524 245.2421496,147.8709147,106.70345205 \\
1.73434674739838 1.36760234832764 249.3025911,227.33242605,209.8164174 \\
1.29806125164032 -1.78649091720581 246.7723791,199.01257995,168.07746405 \\
0.350962609052658 2.65100836753845 195.7084812,23.4946851,82.59583875 \\
1.8629424571991 1.21538984775543 249.3025911,227.33242605,209.8164174 \\
2.40479707717896 -1.18868565559387 245.70760365,161.1641412,120.3721023 \\
1.28172826766968 2.2502875328064 245.56115205,156.2192985,115.1199387 \\
1.53466856479645 -0.818996965885162 247.91115555,213.9578826,190.17620265 \\
1.66201341152191 1.73645555973053 247.41505305,207.99484275,181.3567548 \\
2.00361251831055 -1.37149226665497 247.65640545,210.97774605,185.77259745 \\
1.68073105812073 -0.989970743656158 249.48584175,228.81590385,211.9835043 \\
2.09827327728271 -0.911172688007355 248.9531187,224.36478195,205.4760165 \\
1.10669612884521 -1.73457217216492 246.1520712,182.0552406,144.9672297 \\
1.91454243659973 -0.175669729709625 250.25125995,234.75740895,220.66913325 \\
2.07701921463013 0.130897790193558 250.0727982,233.26498575,218.47535805 \\
0.635067760944366 1.79919242858887 244.2513471,130.71627165,91.0654521 \\
1.94022524356842 -0.702544450759888 250.25125995,234.75740895,220.66913325 \\
2.082688331604 1.0389267206192 248.4751161,219.90865695,198.93982845 \\
1.50231575965881 1.11309885978699 248.4751161,219.90865695,198.93982845 \\
2.35982370376587 0.00255668163299561 247.65640545,210.97774605,185.77259745 \\
1.71336078643799 -0.724073827266693 249.48584175,228.81590385,211.9835043 \\
2.07136631011963 0.977380037307739 248.78919705,222.87959565,203.29941555 \\
1.05657148361206 -1.7538013458252 246.02364555,175.72078815,137.0778663 \\
2.07358074188232 0.793340086936951 249.3025911,227.33242605,209.8164174 \\
1.39431309700012 1.84370505809784 246.86670105,200.51813565,170.27341695 \\
2.05237102508545 0.936960220336914 249.1250907,225.84872895,207.6473262 \\
1.33381021022797 0.815559864044189 246.26699715,186.7493979,151.06898445 \\
2.18694090843201 0.552648723125458 248.78919705,222.87959565,203.29941555 \\
1.69726693630219 0.182531505823135 248.3271753,218.42187945,196.75380015 \\
2.17196273803711 -0.0142271518707275 249.48584175,228.81590385,211.9835043 \\
0.473345071077347 1.65944600105286 241.29624135,101.5235784,70.0988421 \\
1.8424733877182 -0.811858057975769 250.0727982,233.26498575,218.47535805 \\
1.96640253067017 -0.112751364707947 250.25125995,234.75740895,220.66913325 \\
1.86236763000488 -1.43832647800446 248.3271753,218.42187945,196.75380015 \\
1.97093272209167 -0.480836868286133 250.25125995,234.75740895,220.66913325 \\
1.19227051734924 -1.53054332733154 246.4728561,192.92848905,159.44460225 \\
2.095219373703 1.61751246452332 245.9932011,174.1238073,135.15058905 \\
2.01762199401855 0.171933889389038 250.25125995,234.75740895,220.66913325 \\
2.54510092735291 -0.279628098011017 246.02364555,175.72078815,137.0778663 \\
1.28003644943237 -1.87715935707092 246.4728561,192.92848905,159.44460225 \\
1.79938936233521 0.199466049671173 249.3025911,227.33242605,209.8164174 \\
2.05865573883057 -0.0408080220222473 250.25125995,234.75740895,220.66913325 \\
2.59065294265747 0.500093758106232 245.50510815,154.5607581,113.4028299 \\
1.7314156293869 -1.50410544872284 248.4751161,219.90865695,198.93982845 \\
1.76145970821381 -1.15470135211945 249.6765537,230.29867785,214.14730935 \\
1.24030351638794 -1.47704124450684 246.68642115,197.50080735,165.8961201 \\
2.37361264228821 -1.044753074646 246.11813835,180.47969505,142.9681215 \\
2.0519642829895 1.0625946521759 248.6295441,221.39436855,201.1211367 \\
2.10345649719238 -2.21950888633728 232.440915,64.49618865,62.3991579 \\
1.46280765533447 -1.7993198633194 247.0774866,203.5132356,174.7062783 \\
1.10510611534119 1.67676067352295 246.2259753,185.1901596,149.01788685 \\
2.60693001747131 -0.993089199066162 243.55115025,121.8574518,83.79951015 \\
2.05556488037109 -2.24278807640076 234.54423405,70.19178705,61.82723115 \\
1.61031937599182 -0.825603246688843 248.6295441,221.39436855,201.1211367 \\
2.27114200592041 -0.123771369457245 248.6295441,221.39436855,201.1211367 \\
1.64676833152771 0.3189637362957 247.91115555,213.9578826,190.17620265 \\
2.05820465087891 -0.651134848594666 249.872205,231.7816839,216.3110379 \\
1.95988762378693 -0.358391880989075 250.25125995,234.75740895,220.66913325 \\
2.41817498207092 0.728176414966583 246.2259753,185.1901596,149.01788685 \\
1.67091298103333 1.85560584068298 246.60841665,195.98272095,163.73019855 \\
2.07225298881531 0.348519414663315 249.872205,231.7816839,216.3110379 \\
1.73801279067993 1.40387213230133 248.9531187,224.36478195,205.4760165 \\
2.57871246337891 0.38598820567131 245.75060175,162.8018226,122.1578214 \\
1.49043834209442 -0.345569908618927 246.36063315,189.8505141,155.2234827 \\
1.66476273536682 1.56328094005585 248.3271753,218.42187945,196.75380015 \\
1.95572972297668 -0.164654076099396 250.25125995,234.75740895,220.66913325 \\
1.17085373401642 1.44148147106171 246.36063315,189.8505141,155.2234827 \\
1.990931391716 -0.567860126495361 250.25125995,234.75740895,220.66913325 \\
2.02938842773438 1.00821375846863 248.9531187,224.36478195,205.4760165 \\
1.90441417694092 -0.394656240940094 250.25125995,234.75740895,220.66913325 \\
2.06033635139465 -0.516509771347046 250.0727982,233.26498575,218.47535805 \\
2.07187485694885 -0.528949499130249 249.872205,231.7816839,216.3110379 \\
1.06550824642181 -1.3989953994751 245.8981065,169.30226475,129.4754085 \\
2.47606182098389 0.0469394326210022 246.4137216,191.3927133,157.3251468 \\
2.0151641368866 -0.555357158184052 250.25125995,234.75740895,220.66913325 \\
1.2929310798645 0.971307337284088 246.31171395,188.3028579,153.13767705 \\
2.14764547348022 0.0944250524044037 249.6765537,230.29867785,214.14730935 \\
2.10630440711975 -0.823184549808502 249.1250907,225.84872895,207.6473262 \\
2.35786008834839 -0.501173436641693 247.299714,206.50190445,179.14325025 \\
1.67241096496582 -0.16228061914444 248.04512235,215.44664655,192.37235955 \\
2.06195378303528 0.211884438991547 250.0727982,233.26498575,218.47535805 \\
1.36563551425934 1.84242343902588 246.7723791,199.01257995,168.07746405 \\
2.26379442214966 -1.17197108268738 246.4137216,191.3927133,157.3251468 \\
1.10679602622986 0.988766729831696 245.16534615,146.18443905,105.06918765 \\
2.18445539474487 -0.658128678798676 248.78919705,222.87959565,203.29941555 \\
2.03242087364197 -0.962635219097137 249.3025911,227.33242605,209.8164174 \\
1.89863252639771 -0.842590510845184 250.0727982,233.26498575,218.47535805 \\
1.58944630622864 0.288805723190308 247.299714,206.50190445,179.14325025 \\
2.46119332313538 0.539354622364044 246.1872816,183.6258162,146.98295625 \\
2.4059693813324 -1.00301921367645 246.02364555,175.72078815,137.0778663 \\
1.99454963207245 -1.48866772651672 247.0774866,203.5132356,174.7062783 \\
1.71742141246796 -1.01091814041138 249.6765537,230.29867785,214.14730935 \\
0.520343899726868 1.60292840003967 241.02706845,99.60893895,69.0920562 \\
1.36457812786102 -1.98302459716797 246.26699715,186.7493979,151.06898445 \\
1.63038122653961 1.29846429824829 249.3025911,227.33242605,209.8164174 \\
1.70118272304535 -1.70245265960693 247.41505305,207.99484275,181.3567548 \\
1.55586194992065 -0.660678744316101 247.65640545,210.97774605,185.77259745 \\
1.5978729724884 0.909986734390259 248.9531187,224.36478195,205.4760165 \\
1.50213479995728 1.6749746799469 247.78194195,212.46815985,187.97598105 \\
1.80886912345886 0.153353780508041 249.3025911,227.33242605,209.8164174 \\
1.43646371364594 2.19305157661438 245.7904863,164.4347457,123.960498 \\
2.1648211479187 0.347455829381943 249.3025911,227.33242605,209.8164174 \\
1.38124144077301 0.620426774024963 246.1872816,183.6258162,146.98295625 \\
1.77341961860657 -0.518494427204132 249.6765537,230.29867785,214.14730935 \\
2.28603768348694 -0.581130027770996 247.91115555,213.9578826,190.17620265 \\
1.16856062412262 1.16843044757843 246.02364555,175.72078815,137.0778663 \\
1.69166445732117 -0.18923145532608 248.18406675,216.9345027,194.56450275 \\
2.15082979202271 -0.149572730064392 249.6765537,230.29867785,214.14730935 \\
1.52391612529755 -0.961849272251129 248.18406675,216.9345027,194.56450275 \\
1.33164966106415 2.19174957275391 245.82816765,166.06246425,125.780739 \\
1.95267260074615 1.04951167106628 249.48584175,228.81590385,211.9835043 \\
1.01255619525909 1.19820690155029 244.99843845,142.7928192,101.84999115 \\
1.52160274982452 1.14360058307648 248.78919705,222.87959565,203.29941555 \\
0.964728534221649 -1.85271286964417 245.75060175,162.8018226,122.1578214 \\
2.10250377655029 0.58160811662674 249.48584175,228.81590385,211.9835043 \\
1.38561141490936 2.33698701858521 244.90734735,141.0879912,100.26359535 \\
2.11798095703125 -1.83643460273743 244.81107975,139.37686215,98.692446 \\
1.83729219436646 -0.590149581432343 250.0727982,233.26498575,218.47535805 \\
1.84667885303497 1.30851757526398 248.9531187,224.36478195,205.4760165 \\
1.7500479221344 -0.66059410572052 249.6765537,230.29867785,214.14730935 \\
1.91008961200714 0.409836769104004 250.0727982,233.26498575,218.47535805 \\
0.482711493968964 1.74250328540802 242.2543452,109.0670649,74.5740564 \\
1.36428380012512 0.743379533290863 246.31171395,188.3028579,153.13767705 \\
2.65026473999023 0.705544352531433 243.70268145,123.64724325,85.2245802 \\
2.00784230232239 1.02763366699219 249.1250907,225.84872895,207.6473262 \\
1.35924363136292 -1.69907581806183 247.1866113,205.00830825,176.92716255 \\
1.10396718978882 1.71663475036621 246.1872816,183.6258162,146.98295625 \\
2.046715259552 0.357448130846024 250.0727982,233.26498575,218.47535805 \\
0.458182036876678 1.47288906574249 233.8786203,68.27324355,61.95004935 \\
1.5385639667511 0.928030371665955 248.4751161,219.90865695,198.93982845 \\
1.53237295150757 -1.21713864803314 248.6295441,221.39436855,201.1211367 \\
1.58509945869446 -1.5712023973465 248.18406675,216.9345027,194.56450275 \\
1.23281049728394 1.15547275543213 246.26699715,186.7493979,151.06898445 \\
1.32095563411713 -2.04420614242554 246.08576865,178.898718,140.986746 \\
2.27909278869629 0.0892218351364136 248.4751161,219.90865695,198.93982845 \\
1.05606329441071 1.54067742824554 246.02364555,175.72078815,137.0778663 \\
1.59741997718811 -1.39101421833038 248.78919705,222.87959565,203.29941555 \\
1.58059763908386 -0.465960502624512 247.41505305,207.99484275,181.3567548 \\
1.37091612815857 1.23775172233582 247.65640545,210.97774605,185.77259745 \\
1.07812929153442 -1.72134041786194 246.08576865,178.898718,140.986746 \\
1.19876992702484 -2.07278490066528 245.9304711,170.9147751,131.3488017 \\
2.02430891990662 -0.578043103218079 250.0727982,233.26498575,218.47535805 \\
2.15418314933777 -1.24156033992767 247.0774866,203.5132356,174.7062783 \\
1.33475911617279 0.447673678398132 245.70760365,161.1641412,120.3721023 \\
1.32513976097107 0.823028385639191 246.2259753,185.1901596,149.01788685 \\
2.22796964645386 0.226439774036407 248.78919705,222.87959565,203.29941555 \\
1.42442667484283 -2.14057326316833 245.82816765,166.06246425,125.780739 \\
1.97396767139435 0.958536148071289 249.48584175,228.81590385,211.9835043 \\
1.66235017776489 -0.248025298118591 248.04512235,215.44664655,192.37235955 \\
2.57618832588196 1.09243619441986 243.0470178,116.43654195,79.6614288 \\
1.65876471996307 1.45769345760345 248.9531187,224.36478195,205.4760165 \\
1.7995810508728 -0.13195139169693 249.3025911,227.33242605,209.8164174 \\
1.93232977390289 -1.48914563655853 247.53321495,209.48702625,183.56726565 \\
2.11705088615417 0.0445360541343689 249.872205,231.7816839,216.3110379 \\
1.96396696567535 0.17787903547287 250.25125995,234.75740895,220.66913325 \\
1.96742141246796 -0.869126856327057 249.872205,231.7816839,216.3110379 \\
1.30566048622131 -2.25314927101135 245.31448035,149.55157125,108.35396505 \\
1.19325315952301 -0.765125334262848 245.16534615,146.18443905,105.06918765 \\
1.6294013261795 -0.981126189231873 249.1250907,225.84872895,207.6473262 \\
1.28829526901245 -1.16263484954834 246.53702175,194.45887155,161.57890635 \\
1.7083740234375 1.7988760471344 246.7723791,199.01257995,168.07746405 \\
0.822330355644226 2.6490170955658 227.24799045,53.8605594,65.116596 \\
1.41050469875336 -0.683832049369812 246.31171395,188.3028579,153.13767705 \\
1.87218689918518 0.365444779396057 249.872205,231.7816839,216.3110379 \\
2.08756160736084 0.0111192464828491 250.0727982,233.26498575,218.47535805 \\
2.21625447273254 0.705580592155457 248.18406675,216.9345027,194.56450275 \\
1.84977245330811 0.73176646232605 250.0727982,233.26498575,218.47535805 \\
1.88757157325745 0.869310200214386 250.0727982,233.26498575,218.47535805 \\
2.13082933425903 -0.752511858940125 249.1250907,225.84872895,207.6473262 \\
2.40640687942505 0.240642011165619 246.96984345,202.01754075,172.483173 \\
2.07652640342712 -0.44209885597229 250.0727982,233.26498575,218.47535805 \\
1.52084684371948 -2.10284662246704 245.8981065,169.30226475,129.4754085 \\
0.707728147506714 -1.68042635917664 243.22306725,118.25314155,81.01330365 \\
1.99405264854431 -0.477522909641266 250.25125995,234.75740895,220.66913325 \\
2.18979692459106 0.830355167388916 248.04512235,215.44664655,192.37235955 \\
2.00459599494934 -0.702378392219543 250.0727982,233.26498575,218.47535805 \\
1.80860650539398 -0.395701467990875 249.6765537,230.29867785,214.14730935 \\
1.84260976314545 1.64611315727234 247.0774866,203.5132356,174.7062783 \\
1.4422721862793 1.36636352539062 248.18406675,216.9345027,194.56450275 \\
1.93717300891876 -1.66607546806335 246.4728561,192.92848905,159.44460225 \\
0.874971807003021 -2.03508114814758 244.99843845,142.7928192,101.84999115 \\
2.36333060264587 -0.521020770072937 247.1866113,205.00830825,176.92716255 \\
1.8059253692627 0.591712892055511 249.6765537,230.29867785,214.14730935 \\
1.32900166511536 0.869504034519196 246.31171395,188.3028579,153.13767705 \\
0.957028746604919 -2.23334836959839 244.4915877,134.20217775,94.0718052 \\
2.70831251144409 -0.180525302886963 244.37428005,132.46298085,92.56140195 \\
1.65608751773834 -0.0593448877334595 247.78194195,212.46815985,187.97598105 \\
1.51462268829346 -0.700278341770172 247.299714,206.50190445,179.14325025 \\
1.6099978685379 -0.453232705593109 247.78194195,212.46815985,187.97598105 \\
1.90870296955109 -1.13453304767609 249.48584175,228.81590385,211.9835043 \\
1.79121351242065 -0.384451687335968 249.48584175,228.81590385,211.9835043 \\
1.46638751029968 0.429998487234116 246.36063315,189.8505141,155.2234827 \\
2.05384135246277 0.374592334032059 249.872205,231.7816839,216.3110379 \\
1.40793895721436 1.48236358165741 247.91115555,213.9578826,190.17620265 \\
1.96558499336243 0.826970994472504 249.872205,231.7816839,216.3110379 \\
1.90613889694214 -1.66987943649292 246.60841665,195.98272095,163.73019855 \\
2.60000014305115 -0.671993494033813 245.2421496,147.8709147,106.70345205 \\
2.05879592895508 1.34083926677704 247.299714,206.50190445,179.14325025 \\
1.81584632396698 -1.17248260974884 249.48584175,228.81590385,211.9835043 \\
1.76736378669739 -1.09038782119751 249.6765537,230.29867785,214.14730935 \\
1.58577859401703 -0.511397957801819 247.65640545,210.97774605,185.77259745 \\
2.28465247154236 0.387859344482422 248.04512235,215.44664655,192.37235955 \\
2.48073053359985 0.53659725189209 246.11813835,180.47969505,142.9681215 \\
1.65250527858734 1.16358733177185 249.6765537,230.29867785,214.14730935 \\
1.67231750488281 -0.503216326236725 248.6295441,221.39436855,201.1211367 \\
1.83654356002808 1.27296710014343 249.3025911,227.33242605,209.8164174 \\
2.04025316238403 -1.03869378566742 248.9531187,224.36478195,205.4760165 \\
2.05907225608826 -0.907095670700073 249.3025911,227.33242605,209.8164174 \\
1.92497038841248 -0.542972028255463 250.25125995,234.75740895,220.66913325 \\
1.18036293983459 -1.06461143493652 245.8981065,169.30226475,129.4754085 \\
1.55160486698151 0.671368002891541 247.78194195,212.46815985,187.97598105 \\
1.19834887981415 -0.955895721912384 245.7904863,164.4347457,123.960498 \\
1.41173148155212 0.00547894835472107 245.7904863,164.4347457,123.960498 \\
2.52048659324646 -1.42969965934753 240.74170815,97.6856805,68.1411102 \\
1.74508333206177 -0.476938009262085 249.3025911,227.33242605,209.8164174 \\
1.75968503952026 -1.19485473632812 249.48584175,228.81590385,211.9835043 \\
1.96038997173309 2.25525498390198 237.34924425,80.00364135,62.39544765 \\
1.66954874992371 -0.927556693553925 249.3025911,227.33242605,209.8164174 \\
2.14552164077759 -0.788046002388 248.9531187,224.36478195,205.4760165 \\
1.35810732841492 -2.23196196556091 245.4455784,152.896572,111.70308405 \\
1.6352926492691 0.91919481754303 249.3025911,227.33242605,209.8164174 \\
1.51354146003723 0.639862656593323 247.299714,206.50190445,179.14325025 \\
0.145589917898178 1.92098212242126 236.32547535,76.0493691,61.9281168 \\
0.946347951889038 1.86287176609039 245.82816765,166.06246425,125.780739 \\
2.25961017608643 0.604042053222656 247.91115555,213.9578826,190.17620265 \\
1.95147371292114 -0.105408072471619 250.25125995,234.75740895,220.66913325 \\
1.91496539115906 -0.0543718934059143 250.0727982,233.26498575,218.47535805 \\
2.51127362251282 -0.120999872684479 246.2259753,185.1901596,149.01788685 \\
1.68249821662903 0.672617077827454 248.9531187,224.36478195,205.4760165 \\
1.34334373474121 -1.42209720611572 247.41505305,207.99484275,181.3567548 \\
1.16449987888336 1.95259177684784 246.11813835,180.47969505,142.9681215 \\
1.93268823623657 0.286761701107025 250.0727982,233.26498575,218.47535805 \\
1.87658619880676 0.505125224590302 250.0727982,233.26498575,218.47535805 \\
2.33785438537598 -1.27595221996307 245.8641966,167.68475895,127.6195797 \\
2.36624908447266 -0.334377348423004 247.41505305,207.99484275,181.3567548 \\
1.72907269001007 -1.91705572605133 246.1872816,183.6258162,146.98295625 \\
1.62880611419678 0.869491696357727 249.1250907,225.84872895,207.6473262 \\
1.90936756134033 0.76085376739502 250.0727982,233.26498575,218.47535805 \\
1.80072474479675 0.0264031291007996 249.3025911,227.33242605,209.8164174 \\
1.04935359954834 -0.998920500278473 244.4915877,134.20217775,94.0718052 \\
1.03037762641907 -1.50866317749023 245.8641966,167.68475895,127.6195797 \\
1.4913215637207 -1.34483432769775 248.3271753,218.42187945,196.75380015 \\
1.83000755310059 0.818961024284363 250.0727982,233.26498575,218.47535805 \\
2.13006520271301 0.708470165729523 248.9531187,224.36478195,205.4760165 \\
2.09686374664307 1.23238468170166 247.53321495,209.48702625,183.56726565 \\
1.12150502204895 2.2103545665741 245.56115205,156.2192985,115.1199387 \\
2.21080827713013 1.06942987442017 247.0774866,203.5132356,174.7062783 \\
1.40959203243256 -1.16303086280823 247.53321495,209.48702625,183.56726565 \\
0.977188348770142 -1.25722169876099 244.99843845,142.7928192,101.84999115 \\
2.10576367378235 -0.561927974224091 249.6765537,230.29867785,214.14730935 \\
2.27709746360779 1.06541633605957 246.53702175,194.45887155,161.57890635 \\
1.1418229341507 -1.86723279953003 246.1520712,182.0552406,144.9672297 \\
1.42712903022766 -1.58120548725128 247.78194195,212.46815985,187.97598105 \\
1.66683912277222 0.50584888458252 248.4751161,219.90865695,198.93982845 \\
0.901186466217041 -1.89763140678406 245.4455784,152.896572,111.70308405 \\
1.87525475025177 1.56180620193481 247.41505305,207.99484275,181.3567548 \\
1.93949151039124 -0.758541882038116 250.0727982,233.26498575,218.47535805 \\
1.79915273189545 -1.26015830039978 249.3025911,227.33242605,209.8164174 \\
1.09422385692596 -0.745702385902405 243.22306725,118.25314155,81.01330365 \\
2.16363072395325 -1.90370404720306 243.22306725,118.25314155,81.01330365 \\
2.20766949653625 1.16139554977417 246.7723791,199.01257995,168.07746405 \\
1.76298975944519 -0.865320980548859 249.872205,231.7816839,216.3110379 \\
1.61906313896179 -0.609756588935852 248.18406675,216.9345027,194.56450275 \\
2.13075876235962 -0.205740034580231 249.872205,231.7816839,216.3110379 \\
1.26308858394623 1.9714607000351 246.2259753,185.1901596,149.01788685 \\
1.71743214130402 2.01666402816772 245.96240475,172.5216882,133.24118985 \\
};
\addplot [draw=color0, fill=color0, mark=*, only marks]
table{%
x  y
};
\addplot [draw=color1, fill=color1, mark=*, only marks]
table{%
x  y
};
\addplot [draw=color2, fill=color2, mark=*, only marks]
table{%
x  y
};
\addplot [draw=color3, fill=color3, mark=*, only marks]
table{%
x  y
};
\addplot [draw=color4, fill=color4, mark=*, only marks]
table{%
x  y
};
\addplot [draw=color5, fill=color5, mark=*, only marks]
table{%
x  y
};
\end{axis}

\end{tikzpicture}

%         \caption{$y=3sinx$}
%         \label{fig:elbow_example_10000}
%     \end{subfigure}
%     \hfill
%     \begin{subfigure}[t!]{0.3\linewidth}
%         \centering
%         % This file was created with tikzplotlib v0.9.15.
\begin{tikzpicture}

\definecolor{color0}{rgb}{0.2875374,0.11191896,0.28660853}
\definecolor{color1}{rgb}{0.63812122,0.09938212,0.35603757}
\definecolor{color2}{rgb}{0.91978131,0.27526191,0.24245973}
\definecolor{color3}{rgb}{0.9643063,0.66393045,0.5077467}

\begin{axis}[
axis line style={white!80!black},
height=\figureheight,
tick align=outside,
tick pos=left,
width=\figurewidth,
x grid style={white!80!black},
xmajorgrids,
xmin=-4, xmax=4,
xtick style={draw=none},
y grid style={white!80!black},
ymajorgrids,
ymin=-4, ymax=4,
ytick style={draw=none}
]
\addplot [
  draw=white,
  mark=*,
  only marks,
  scatter,
  scatter/@post marker code/.code={%
  \endscope
},
  scatter/@pre marker code/.code={%
  \expanded{%
  \noexpand\definecolor{thispointfillcolor}{RGB}{\fillcolor}%
  }%
  \scope[fill=thispointfillcolor]%
},
  visualization depends on={value \thisrow{fill} \as \fillcolor}
]
table[row sep=\\]{%
x  y  fill \\
-2.10335874557495 -0.810960650444031 248.6295441,221.39436855,201.1211367 \\
-2.30438566207886 0.383602380752563 247.41505305,207.99484275,181.3567548 \\
1.98662149906158 -0.192034304141998 250.25125995,234.75740895,220.66913325 \\
2.04351449012756 0.00661367177963257 250.25125995,234.75740895,220.66913325 \\
-1.79283595085144 0.259220540523529 248.4751161,219.90865695,198.93982845 \\
-1.46105909347534 -1.29428684711456 247.1866113,205.00830825,176.92716255 \\
-1.98523616790771 -1.13970923423767 248.3271753,218.42187945,196.75380015 \\
-1.25453352928162 -1.27313232421875 246.02364555,175.72078815,137.0778663 \\
-2.12055253982544 0.309750497341156 249.6765537,230.29867785,214.14730935 \\
1.78509998321533 1.80029571056366 245.9932011,174.1238073,135.15058905 \\
1.76572823524475 -0.282869935035706 248.4751161,219.90865695,198.93982845 \\
1.28499114513397 1.6605966091156 246.36063315,189.8505141,155.2234827 \\
2.049485206604 0.892917811870575 248.9531187,224.36478195,205.4760165 \\
2.05853199958801 1.06211829185486 248.04512235,215.44664655,192.37235955 \\
-2.08757925033569 0.961524188518524 247.91115555,213.9578826,190.17620265 \\
1.81199383735657 0.1563521027565 248.4751161,219.90865695,198.93982845 \\
-1.75200819969177 -1.19499146938324 248.6295441,221.39436855,201.1211367 \\
1.56916928291321 0.0853671431541443 246.53702175,194.45887155,161.57890635 \\
-2.0421781539917 -0.294032871723175 250.0727982,233.26498575,218.47535805 \\
2.39569926261902 -0.472548604011536 246.4137216,191.3927133,157.3251468 \\
-2.49742317199707 0.178378045558929 245.8981065,169.30226475,129.4754085 \\
1.75175631046295 -0.39627879858017 248.4751161,219.90865695,198.93982845 \\
-2.37915229797363 0.664897620677948 246.08576865,178.898718,140.986746 \\
-2.21370816230774 -1.08396434783936 246.53702175,194.45887155,161.57890635 \\
-1.75134122371674 1.72866439819336 246.4137216,191.3927133,157.3251468 \\
1.96956264972687 1.0541467666626 249.3025911,227.33242605,209.8164174 \\
1.97028863430023 -0.778905272483826 249.48584175,228.81590385,211.9835043 \\
1.47504425048828 1.78150379657745 246.60841665,195.98272095,163.73019855 \\
-2.26058292388916 -0.107413709163666 248.4751161,219.90865695,198.93982845 \\
2.39730548858643 0.193877130746841 246.4728561,192.92848905,159.44460225 \\
-1.33856701850891 -1.62909138202667 246.86670105,200.51813565,170.27341695 \\
1.4930899143219 0.75999790430069 246.4728561,192.92848905,159.44460225 \\
-2.04301929473877 0.057812511920929 250.0727982,233.26498575,218.47535805 \\
-2.32974576950073 0.834281921386719 246.11813835,180.47969505,142.9681215 \\
1.70322489738464 -0.418643891811371 248.04512235,215.44664655,192.37235955 \\
1.88593602180481 0.338063627481461 249.48584175,228.81590385,211.9835043 \\
1.38426399230957 -0.62423175573349 245.9304711,170.9147751,131.3488017 \\
1.86669445037842 1.58265936374664 246.7723791,199.01257995,168.07746405 \\
1.57475566864014 2.16911673545837 243.39135705,120.0597069,82.39293015 \\
1.50262916088104 1.34491157531738 248.3271753,218.42187945,196.75380015 \\
-2.12473154067993 -0.00519353151321411 249.872205,231.7816839,216.3110379 \\
-1.82671189308167 -0.728840410709381 249.1250907,225.84872895,207.6473262 \\
1.62919008731842 -0.203139901161194 247.1866113,205.00830825,176.92716255 \\
-1.5912938117981 -0.39940333366394 246.31171395,188.3028579,153.13767705 \\
-1.80724275112152 1.44958329200745 247.53321495,209.48702625,183.56726565 \\
-1.67355215549469 -1.29001235961914 248.4751161,219.90865695,198.93982845 \\
2.08542394638062 -0.200338423252106 250.0727982,233.26498575,218.47535805 \\
-1.01074171066284 2.13517332077026 242.67077295,112.7715672,77.0469597 \\
-1.71918344497681 -1.16820645332336 248.6295441,221.39436855,201.1211367 \\
2.05843544006348 0.573968291282654 249.872205,231.7816839,216.3110379 \\
1.18521904945374 -0.981928586959839 244.90734735,141.0879912,100.26359535 \\
-1.35515594482422 1.43070709705353 246.4137216,191.3927133,157.3251468 \\
2.20066237449646 0.935905933380127 246.7723791,199.01257995,168.07746405 \\
-1.57236576080322 -1.97961246967316 246.08576865,178.898718,140.986746 \\
1.89179241657257 0.854650378227234 249.872205,231.7816839,216.3110379 \\
-2.05964016914368 -1.10382866859436 247.91115555,213.9578826,190.17620265 \\
1.32435607910156 0.603648900985718 245.31448035,149.55157125,108.35396505 \\
2.37398457527161 0.0326827168464661 246.86670105,200.51813565,170.27341695 \\
-2.74828767776489 0.44030898809433 231.6640269,62.64960615,62.7257262 \\
2.18888974189758 0.315166890621185 248.9531187,224.36478195,205.4760165 \\
-2.11613321304321 -0.770033717155457 248.6295441,221.39436855,201.1211367 \\
-1.33124685287476 -1.7574063539505 246.60841665,195.98272095,163.73019855 \\
-2.60010147094727 -0.464907288551331 244.37428005,132.46298085,92.56140195 \\
-1.79823577404022 1.15765058994293 248.4751161,219.90865695,198.93982845 \\
1.89545786380768 0.847991704940796 250.0727982,233.26498575,218.47535805 \\
2.51071286201477 0.491723954677582 245.31448035,149.55157125,108.35396505 \\
1.89394426345825 0.197663992643356 249.48584175,228.81590385,211.9835043 \\
2.25758004188538 0.146543800830841 248.4751161,219.90865695,198.93982845 \\
1.58224427700043 -0.997059464454651 248.18406675,216.9345027,194.56450275 \\
1.94536459445953 -0.48664790391922 250.0727982,233.26498575,218.47535805 \\
2.20985698699951 -0.857466697692871 247.299714,206.50190445,179.14325025 \\
2.17500782012939 -0.975188314914703 247.1866113,205.00830825,176.92716255 \\
1.94556307792664 -1.68910002708435 245.82816765,166.06246425,125.780739 \\
1.33066546916962 0.791401386260986 245.8641966,167.68475895,127.6195797 \\
-2.94271063804626 -0.354947566986084 179.6674767,22.4942895,87.9062673 \\
1.77936911582947 1.54946005344391 247.65640545,210.97774605,185.77259745 \\
1.06494152545929 -1.66790401935577 245.66176995,159.521268,118.60374105 \\
-1.22345125675201 -1.61251485347748 246.36063315,189.8505141,155.2234827 \\
1.19188022613525 -1.05537343025208 245.31448035,149.55157125,108.35396505 \\
1.68278503417969 -1.15086770057678 248.78919705,222.87959565,203.29941555 \\
1.79960870742798 -0.672399997711182 249.3025911,227.33242605,209.8164174 \\
-1.93810451030731 -1.54655241966248 246.7723791,199.01257995,168.07746405 \\
2.26519012451172 -0.125415444374084 248.4751161,219.90865695,198.93982845 \\
-1.79468560218811 -1.10093188285828 248.9531187,224.36478195,205.4760165 \\
2.34691715240479 0.0897878408432007 247.1866113,205.00830825,176.92716255 \\
2.44025206565857 -0.0820176601409912 246.26699715,186.7493979,151.06898445 \\
-1.56102108955383 -1.66024208068848 247.41505305,207.99484275,181.3567548 \\
1.40058326721191 -0.283113300800323 245.61336585,157.87274145,116.8535001 \\
1.19291710853577 -1.21621823310852 245.75060175,162.8018226,122.1578214 \\
2.07848024368286 0.506753921508789 249.872205,231.7816839,216.3110379 \\
0.961115717887878 -1.20205426216125 241.79993775,105.3160812,72.24732165 \\
1.6803765296936 -1.29445195198059 248.3271753,218.42187945,196.75380015 \\
1.91341435909271 -1.03677201271057 248.78919705,222.87959565,203.29941555 \\
1.8284786939621 1.89295423030853 245.2421496,147.8709147,106.70345205 \\
2.07200527191162 -0.567244827747345 249.6765537,230.29867785,214.14730935 \\
2.02375078201294 -1.28864502906799 246.86670105,200.51813565,170.27341695 \\
1.91824865341187 0.367420196533203 249.872205,231.7816839,216.3110379 \\
-1.63131511211395 -1.49767518043518 248.04512235,215.44664655,192.37235955 \\
1.13841068744659 1.01967561244965 244.4915877,134.20217775,94.0718052 \\
2.48088908195496 -1.28387117385864 238.25942085,83.9673231,63.1770864 \\
2.03295493125916 -1.31953907012939 246.60841665,195.98272095,163.73019855 \\
1.62881147861481 -0.253020167350769 247.1866113,205.00830825,176.92716255 \\
-1.86395990848541 0.674085259437561 249.3025911,227.33242605,209.8164174 \\
1.79661190509796 -0.0249200463294983 248.4751161,219.90865695,198.93982845 \\
1.13001227378845 1.39726328849792 245.9932011,174.1238073,135.15058905 \\
-1.59265899658203 0.508150696754456 246.86670105,200.51813565,170.27341695 \\
-1.77785158157349 -0.471260249614716 248.3271753,218.42187945,196.75380015 \\
2.73897838592529 0.820741593837738 208.64222655,30.8055453,76.31156775 \\
2.3969554901123 0.0832496285438538 246.60841665,195.98272095,163.73019855 \\
-1.2491067647934 -1.60860812664032 246.4137216,191.3927133,157.3251468 \\
-1.67501366138458 -0.380874216556549 247.0774866,203.5132356,174.7062783 \\
1.526123046875 1.95075857639313 245.8641966,167.68475895,127.6195797 \\
1.91073155403137 0.379989773035049 249.6765537,230.29867785,214.14730935 \\
-2.11650776863098 -0.561450242996216 249.1250907,225.84872895,207.6473262 \\
-2.36161327362061 0.00556439161300659 247.0774866,203.5132356,174.7062783 \\
1.67444837093353 -1.53744840621948 247.41505305,207.99484275,181.3567548 \\
-1.476597905159 1.80242967605591 246.26699715,186.7493979,151.06898445 \\
1.03602778911591 0.502941787242889 213.81796665,35.7639999,73.360389 \\
2.05985808372498 -0.0651458501815796 250.25125995,234.75740895,220.66913325 \\
-0.476229518651962 2.35635828971863 215.05537455,37.11954165,72.62608845 \\
-1.92742156982422 -1.1233925819397 248.6295441,221.39436855,201.1211367 \\
-2.19514012336731 0.673286914825439 247.91115555,213.9578826,190.17620265 \\
1.7559027671814 -1.70204162597656 246.26699715,186.7493979,151.06898445 \\
-1.31167125701904 1.51136422157288 246.2259753,185.1901596,149.01788685 \\
0.855374157428741 -0.92938768863678 203.107857,26.79192435,79.21072905 \\
2.61951017379761 0.636473834514618 240.4421571,95.75117145,67.24304355 \\
-2.54500436782837 -0.597103536128998 245.084376,144.4915935,103.45182465 \\
-1.75735402107239 0.677042007446289 248.6295441,221.39436855,201.1211367 \\
0.202361553907394 1.93425917625427 213.81796665,35.7639999,73.360389 \\
1.00273442268372 -0.950340509414673 233.1777375,66.3737001,62.14186545 \\
-0.932496607303619 2.50803995132446 211.2746961,33.1860009,74.8360587 \\
1.75056636333466 1.26673078536987 249.1250907,225.84872895,207.6473262 \\
-1.58730983734131 -1.54647636413574 247.78194195,212.46815985,187.97598105 \\
-1.72219586372375 0.431963443756104 248.04512235,215.44664655,192.37235955 \\
0.481439769268036 2.23310375213623 213.81796665,35.7639999,73.360389 \\
-1.74599146842957 1.03046727180481 248.6295441,221.39436855,201.1211367 \\
-0.981783509254456 -1.86016345024109 245.56115205,156.2192985,115.1199387 \\
1.95784389972687 0.190891951322556 250.0727982,233.26498575,218.47535805 \\
1.45020639896393 0.0429087579250336 245.8641966,167.68475895,127.6195797 \\
-2.0157470703125 0.598184764385223 249.6765537,230.29867785,214.14730935 \\
-2.30145406723022 -0.78804075717926 246.53702175,194.45887155,161.57890635 \\
1.8959196805954 -0.139800667762756 249.6765537,230.29867785,214.14730935 \\
-2.084552526474 1.15052962303162 247.0774866,203.5132356,174.7062783 \\
2.44883012771606 0.830394625663757 245.2421496,147.8709147,106.70345205 \\
1.22526657581329 0.733166992664337 244.1226945,128.96176965,89.5838256 \\
2.20491623878479 0.101353704929352 249.1250907,225.84872895,207.6473262 \\
-1.43743968009949 1.95014762878418 245.9304711,170.9147751,131.3488017 \\
1.55943322181702 1.41360402107239 248.4751161,219.90865695,198.93982845 \\
0.246426731348038 2.35228586196899 198.72823455,24.5687247,81.2854371 \\
-1.65508651733398 -0.713078796863556 247.41505305,207.99484275,181.3567548 \\
1.33953022956848 1.07392489910126 246.26699715,186.7493979,151.06898445 \\
-2.12105917930603 -0.102906465530396 249.872205,231.7816839,216.3110379 \\
-1.00810289382935 -1.86158657073975 245.66176995,159.521268,118.60374105 \\
1.31629240512848 0.109361320734024 243.70268145,123.64724325,85.2245802 \\
-2.02698183059692 1.31339049339294 246.7723791,199.01257995,168.07746405 \\
-1.76012408733368 1.52966547012329 247.299714,206.50190445,179.14325025 \\
1.70069670677185 1.82486319541931 246.08576865,178.898718,140.986746 \\
-1.37920928001404 1.86185383796692 246.02364555,175.72078815,137.0778663 \\
2.02700614929199 0.755126059055328 249.872205,231.7816839,216.3110379 \\
-1.90014958381653 1.01992356777191 248.78919705,222.87959565,203.29941555 \\
-1.50987601280212 -0.787874937057495 246.26699715,186.7493979,151.06898445 \\
1.08836364746094 1.76842784881592 245.8981065,169.30226475,129.4754085 \\
1.75732171535492 -0.344503819942474 248.4751161,219.90865695,198.93982845 \\
-1.24887728691101 -1.933966755867 246.08576865,178.898718,140.986746 \\
1.94942808151245 -1.25188934803009 247.53321495,209.48702625,183.56726565 \\
-2.17067050933838 -1.09051489830017 246.86670105,200.51813565,170.27341695 \\
-1.91269385814667 -0.887241184711456 249.3025911,227.33242605,209.8164174 \\
-1.78642058372498 -1.35751557350159 248.3271753,218.42187945,196.75380015 \\
-1.55599820613861 2.21700954437256 243.39135705,120.0597069,82.39293015 \\
-1.90606486797333 1.18984866142273 248.18406675,216.9345027,194.56450275 \\
2.21663570404053 1.12629413604736 246.1520712,182.0552406,144.9672297 \\
-1.6919002532959 -1.382328748703 248.3271753,218.42187945,196.75380015 \\
-1.26827573776245 1.18101859092712 245.8981065,169.30226475,129.4754085 \\
-1.5550149679184 -1.45076012611389 247.91115555,213.9578826,190.17620265 \\
-2.26376986503601 0.362649500370026 248.04512235,215.44664655,192.37235955 \\
1.99411797523499 1.06414294242859 248.9531187,224.36478195,205.4760165 \\
-1.76518893241882 0.430691003799438 248.4751161,219.90865695,198.93982845 \\
-1.89014899730682 1.74021530151367 245.9932011,174.1238073,135.15058905 \\
1.77929162979126 0.781959772109985 248.6295441,221.39436855,201.1211367 \\
-1.97610652446747 -1.37844181060791 247.299714,206.50190445,179.14325025 \\
2.08942294120789 -0.923970997333527 248.18406675,216.9345027,194.56450275 \\
2.26203584671021 0.514206051826477 247.53321495,209.48702625,183.56726565 \\
-1.9313987493515 1.63192200660706 246.1520712,182.0552406,144.9672297 \\
-1.0655837059021 1.32162249088287 244.37428005,132.46298085,92.56140195 \\
-1.54823470115662 -0.73057758808136 246.4137216,191.3927133,157.3251468 \\
-1.69086945056915 1.26745223999023 248.04512235,215.44664655,192.37235955 \\
1.99201798439026 1.86504423618317 242.863122,114.60937515,78.3386775 \\
-1.00627827644348 -1.34567475318909 244.1226945,128.96176965,89.5838256 \\
2.07434558868408 0.494244486093521 249.872205,231.7816839,216.3110379 \\
-1.40577459335327 1.19384813308716 246.68642115,197.50080735,165.8961201 \\
1.95998191833496 -1.1377968788147 248.04512235,215.44664655,192.37235955 \\
-1.87823522090912 -0.284660756587982 249.3025911,227.33242605,209.8164174 \\
2.19545793533325 0.322985261678696 248.9531187,224.36478195,205.4760165 \\
-2.8298544883728 0.0709713101387024 222.05177895,45.8543907,68.3333139 \\
1.97591829299927 0.75807797908783 250.25125995,234.75740895,220.66913325 \\
-1.46515905857086 -0.25400573015213 245.50510815,154.5607581,113.4028299 \\
1.83460199832916 0.391400247812271 248.78919705,222.87959565,203.29941555 \\
-1.19083666801453 -1.27047681808472 245.75060175,162.8018226,122.1578214 \\
-1.77588033676147 0.929884552955627 248.78919705,222.87959565,203.29941555 \\
1.70356798171997 0.0528141856193542 247.53321495,209.48702625,183.56726565 \\
-1.87168562412262 -1.34086132049561 248.04512235,215.44664655,192.37235955 \\
1.89239382743835 -0.278072834014893 249.6765537,230.29867785,214.14730935 \\
2.12518525123596 0.700793743133545 248.78919705,222.87959565,203.29941555 \\
-1.96649217605591 0.389243960380554 249.872205,231.7816839,216.3110379 \\
-1.61393523216248 -1.35701525211334 248.18406675,216.9345027,194.56450275 \\
-2.60420870780945 -0.627176284790039 243.55115025,121.8574518,83.79951015 \\
-2.35701870918274 -1.0078661441803 245.8981065,169.30226475,129.4754085 \\
1.89699065685272 0.646061420440674 249.872205,231.7816839,216.3110379 \\
2.38571834564209 0.673184394836426 246.02364555,175.72078815,137.0778663 \\
1.43075180053711 0.268058061599731 245.7904863,164.4347457,123.960498 \\
-2.30043244361877 -0.582414507865906 247.0774866,203.5132356,174.7062783 \\
1.78348779678345 -1.14114987850189 248.78919705,222.87959565,203.29941555 \\
2.32323670387268 -0.442114174365997 247.299714,206.50190445,179.14325025 \\
-1.97629475593567 -0.774162232875824 249.48584175,228.81590385,211.9835043 \\
-1.35915172100067 -2.00606179237366 246.02364555,175.72078815,137.0778663 \\
-1.80746412277222 1.47240042686462 247.41505305,207.99484275,181.3567548 \\
1.65932154655457 1.63361763954163 247.299714,206.50190445,179.14325025 \\
-1.36196637153625 -1.52680635452271 246.96984345,202.01754075,172.483173 \\
-1.65811586380005 1.32650530338287 247.91115555,213.9578826,190.17620265 \\
1.81142675876617 -0.35437136888504 248.9531187,224.36478195,205.4760165 \\
-0.96763664484024 1.45602774620056 243.70268145,123.64724325,85.2245802 \\
-1.93074882030487 1.82020497322083 245.50510815,154.5607581,113.4028299 \\
1.93560123443604 0.685820937156677 250.0727982,233.26498575,218.47535805 \\
-1.48130893707275 0.0848366022109985 245.61336585,157.87274145,116.8535001 \\
1.6986962556839 0.951390981674194 248.3271753,218.42187945,196.75380015 \\
2.20784759521484 0.917253851890564 246.7723791,199.01257995,168.07746405 \\
-1.75293910503387 1.39714002609253 247.78194195,212.46815985,187.97598105 \\
-1.80523920059204 0.102234423160553 248.4751161,219.90865695,198.93982845 \\
-1.63552165031433 -2.16787004470825 244.90734735,141.0879912,100.26359535 \\
-1.58271789550781 -1.05696582794189 247.53321495,209.48702625,183.56726565 \\
-2.35519742965698 -0.85700798034668 246.11813835,180.47969505,142.9681215 \\
-1.67281150817871 1.8752738237381 246.08576865,178.898718,140.986746 \\
-1.82947134971619 -0.391138732433319 248.78919705,222.87959565,203.29941555 \\
-1.92041385173798 -1.25500345230103 248.18406675,216.9345027,194.56450275 \\
1.49776768684387 -0.0934752821922302 246.1520712,182.0552406,144.9672297 \\
1.55180847644806 -0.69616311788559 247.299714,206.50190445,179.14325025 \\
-1.22478783130646 -1.64221608638763 246.36063315,189.8505141,155.2234827 \\
-1.60306024551392 -0.568741142749786 246.53702175,194.45887155,161.57890635 \\
1.8530091047287 0.0109168291091919 248.9531187,224.36478195,205.4760165 \\
1.77805852890015 -1.47562658786774 247.299714,206.50190445,179.14325025 \\
1.14045643806458 0.477467089891434 235.76429685,74.0861394,61.8186861 \\
-2.13815402984619 -0.187715530395508 249.6765537,230.29867785,214.14730935 \\
1.53082633018494 1.42714715003967 248.3271753,218.42187945,196.75380015 \\
1.24315679073334 1.36904060840607 246.36063315,189.8505141,155.2234827 \\
-1.4380989074707 -0.958871483802795 246.1872816,183.6258162,146.98295625 \\
-1.83702611923218 0.462522685527802 249.1250907,225.84872895,207.6473262 \\
-1.27827644348145 1.29907965660095 246.08576865,178.898718,140.986746 \\
-2.09124946594238 0.0342530608177185 250.0727982,233.26498575,218.47535805 \\
1.87859356403351 -1.56502342224121 246.36063315,189.8505141,155.2234827 \\
2.00712275505066 -0.441881954669952 250.0727982,233.26498575,218.47535805 \\
-0.765364587306976 -1.50216901302338 235.76429685,74.0861394,61.8186861 \\
1.32328307628632 1.07025277614594 246.2259753,185.1901596,149.01788685 \\
-1.76985788345337 0.649919450283051 248.78919705,222.87959565,203.29941555 \\
-1.7842766046524 -0.782478630542755 248.78919705,222.87959565,203.29941555 \\
-2.4428551197052 0.807525932788849 245.3822772,151.2267249,110.02044105 \\
2.33402967453003 0.989961087703705 245.8641966,167.68475895,127.6195797 \\
-1.60665166378021 -2.11318063735962 245.50510815,154.5607581,113.4028299 \\
1.29451930522919 2.01339530944824 245.56115205,156.2192985,115.1199387 \\
1.4870777130127 1.69772672653198 247.1866113,205.00830825,176.92716255 \\
-1.53615617752075 0.0604889392852783 245.9932011,174.1238073,135.15058905 \\
2.1833815574646 0.458977162837982 248.78919705,222.87959565,203.29941555 \\
2.48827171325684 0.805725157260895 244.6033236,135.93415815,95.5968225 \\
-1.34199154376984 1.30388712882996 246.36063315,189.8505141,155.2234827 \\
-1.79903244972229 -1.73957502841949 246.53702175,194.45887155,161.57890635 \\
1.80348920822144 0.705898404121399 248.78919705,222.87959565,203.29941555 \\
2.124831199646 0.438176244497299 249.48584175,228.81590385,211.9835043 \\
-1.6977686882019 0.898118674755096 248.4751161,219.90865695,198.93982845 \\
2.22501277923584 1.22457766532898 245.9304711,170.9147751,131.3488017 \\
-1.55013978481293 1.60475420951843 246.86670105,200.51813565,170.27341695 \\
1.44660651683807 -1.10731089115143 247.299714,206.50190445,179.14325025 \\
-1.4859561920166 1.66094923019409 246.53702175,194.45887155,161.57890635 \\
1.40280270576477 -2.3023054599762 243.0470178,116.43654195,79.6614288 \\
-1.86030173301697 1.49234759807587 247.0774866,203.5132356,174.7062783 \\
-2.46579051017761 -0.112281918525696 246.11813835,180.47969505,142.9681215 \\
-0.943805277347565 -1.67271637916565 245.16534615,146.18443905,105.06918765 \\
1.46225535869598 1.45837259292603 248.18406675,216.9345027,194.56450275 \\
-2.0108859539032 -0.156960785388947 250.0727982,233.26498575,218.47535805 \\
1.56387794017792 -0.927083909511566 247.91115555,213.9578826,190.17620265 \\
-2.19054460525513 0.13281774520874 249.3025911,227.33242605,209.8164174 \\
-1.54976427555084 1.29982435703278 247.65640545,210.97774605,185.77259745 \\
2.11914157867432 1.48680651187897 245.75060175,162.8018226,122.1578214 \\
-2.10666346549988 -1.79128456115723 245.084376,144.4915935,103.45182465 \\
-1.54146766662598 -0.894885301589966 246.68642115,197.50080735,165.8961201 \\
1.82166576385498 0.880838096141815 249.3025911,227.33242605,209.8164174 \\
1.71111619472504 1.67544198036194 246.86670105,200.51813565,170.27341695 \\
-2.13383269309998 -1.03289830684662 247.41505305,207.99484275,181.3567548 \\
-1.74676656723022 -1.18859565258026 248.6295441,221.39436855,201.1211367 \\
2.11488032341003 -0.243139326572418 249.872205,231.7816839,216.3110379 \\
1.11000239849091 -2.13900566101074 244.90734735,141.0879912,100.26359535 \\
-1.92154979705811 0.429364085197449 249.6765537,230.29867785,214.14730935 \\
2.18062400817871 0.950744330883026 246.86670105,200.51813565,170.27341695 \\
1.17882335186005 2.11228561401367 245.16534615,146.18443905,105.06918765 \\
-2.13412380218506 -0.435534358024597 249.3025911,227.33242605,209.8164174 \\
-2.11529946327209 0.939536035060883 247.78194195,212.46815985,187.97598105 \\
1.11363255977631 -0.887613654136658 241.79993775,105.3160812,72.24732165 \\
1.22626996040344 -1.26698064804077 245.9932011,174.1238073,135.15058905 \\
-1.26079380512238 -1.4180223941803 246.26699715,186.7493979,151.06898445 \\
1.5661393404007 -0.380891740322113 246.86670105,200.51813565,170.27341695 \\
-1.78209006786346 0.473222136497498 248.6295441,221.39436855,201.1211367 \\
1.95937502384186 -0.023838222026825 250.0727982,233.26498575,218.47535805 \\
1.56804573535919 0.809558510780334 247.1866113,205.00830825,176.92716255 \\
2.42568516731262 1.22736525535583 243.55115025,121.8574518,83.79951015 \\
-1.94564473628998 -1.24865937232971 248.04512235,215.44664655,192.37235955 \\
1.9234561920166 -0.205914080142975 249.872205,231.7816839,216.3110379 \\
-1.99466061592102 -0.247613430023193 249.872205,231.7816839,216.3110379 \\
-2.17572259902954 -0.820223808288574 247.78194195,212.46815985,187.97598105 \\
-1.7246710062027 1.81621265411377 246.1520712,182.0552406,144.9672297 \\
1.7407191991806 0.914510786533356 248.6295441,221.39436855,201.1211367 \\
2.31977701187134 -1.73583245277405 230.85174735,60.82907955,63.1076499 \\
1.41715908050537 -1.85998201370239 246.31171395,188.3028579,153.13767705 \\
-2.6171612739563 0.166323781013489 244.37428005,132.46298085,92.56140195 \\
1.97932720184326 0.658036708831787 250.25125995,234.75740895,220.66913325 \\
-1.96993911266327 0.800636112689972 249.3025911,227.33242605,209.8164174 \\
1.89163362979889 -1.4007967710495 247.0774866,203.5132356,174.7062783 \\
2.06826233863831 -0.949901700019836 248.18406675,216.9345027,194.56450275 \\
1.98004078865051 1.19893264770508 248.3271753,218.42187945,196.75380015 \\
-1.45216035842896 -0.51516205072403 245.66176995,159.521268,118.60374105 \\
-2.08667778968811 -1.5106714963913 246.1520712,182.0552406,144.9672297 \\
1.89032769203186 -0.422529637813568 249.6765537,230.29867785,214.14730935 \\
1.43094873428345 -1.06868028640747 247.0774866,203.5132356,174.7062783 \\
-1.99115109443665 -0.218103587627411 249.872205,231.7816839,216.3110379 \\
1.80305182933807 1.47711062431335 248.18406675,216.9345027,194.56450275 \\
2.01783490180969 -1.69617080688477 245.4455784,152.896572,111.70308405 \\
-2.12324547767639 0.431492745876312 249.48584175,228.81590385,211.9835043 \\
-2.02843403816223 1.34641408920288 246.60841665,195.98272095,163.73019855 \\
-1.97720086574554 -0.775431871414185 249.3025911,227.33242605,209.8164174 \\
1.99489057064056 0.70895516872406 250.25125995,234.75740895,220.66913325 \\
-1.51453566551208 -1.13038635253906 247.1866113,205.00830825,176.92716255 \\
2.46931767463684 -1.25550246238708 240.1273494,93.8058147,66.4006995 \\
-2.46655559539795 -1.28633904457092 242.46705855,110.92501275,75.79318335 \\
-1.58211731910706 -0.590360105037689 246.4137216,191.3927133,157.3251468 \\
-1.40718674659729 1.02014780044556 246.36063315,189.8505141,155.2234827 \\
-2.34952020645142 0.151059806346893 247.1866113,205.00830825,176.92716255 \\
-2.00891399383545 0.85748702287674 248.9531187,224.36478195,205.4760165 \\
2.50922083854675 -0.0875787138938904 245.82816765,166.06246425,125.780739 \\
-2.07572674751282 -0.334763407707214 249.872205,231.7816839,216.3110379 \\
2.20177388191223 -0.0618106126785278 249.3025911,227.33242605,209.8164174 \\
-1.13814318180084 -1.29928386211395 245.50510815,154.5607581,113.4028299 \\
-2.2136058807373 -0.913030743598938 247.0774866,203.5132356,174.7062783 \\
2.58983421325684 -0.366150856018066 244.4915877,134.20217775,94.0718052 \\
1.79143810272217 -0.221313059329987 248.6295441,221.39436855,201.1211367 \\
0.848085761070251 1.93579936027527 243.39135705,120.0597069,82.39293015 \\
-1.49858784675598 -1.77597975730896 246.86670105,200.51813565,170.27341695 \\
-1.99916565418243 -1.25428247451782 247.65640545,210.97774605,185.77259745 \\
-1.02777588367462 -0.977201223373413 235.76429685,74.0861394,61.8186861 \\
2.11874079704285 -1.38486921787262 246.05468925,177.3121896,139.0238835 \\
2.58642721176147 0.051109790802002 244.7095719,137.65917705,97.1365992 \\
1.62386584281921 0.909747123718262 247.78194195,212.46815985,187.97598105 \\
1.14310050010681 -1.47065460681915 245.8641966,167.68475895,127.6195797 \\
2.17278480529785 -1.4974809885025 245.3822772,151.2267249,110.02044105 \\
2.16661763191223 1.30170655250549 246.02364555,175.72078815,137.0778663 \\
1.2003378868103 1.69176471233368 246.1872816,183.6258162,146.98295625 \\
-2.33437657356262 -0.305637180805206 247.299714,206.50190445,179.14325025 \\
2.66173815727234 0.122195243835449 241.5538908,103.4256795,71.1509211 \\
1.67287158966064 0.583843469619751 247.53321495,209.48702625,183.56726565 \\
1.25469481945038 -0.175160765647888 242.03209995,107.19754005,73.39147095 \\
1.22766697406769 -0.954074442386627 245.3822772,151.2267249,110.02044105 \\
-1.79217076301575 0.106409847736359 248.3271753,218.42187945,196.75380015 \\
1.46190643310547 -0.573471903800964 246.31171395,188.3028579,153.13767705 \\
1.53304612636566 0.964673399925232 247.41505305,207.99484275,181.3567548 \\
1.79366743564606 -0.0887033939361572 248.4751161,219.90865695,198.93982845 \\
-1.96950650215149 1.07564747333527 248.3271753,218.42187945,196.75380015 \\
-2.60627412796021 0.41926109790802 243.70268145,123.64724325,85.2245802 \\
-2.011878490448 1.2351188659668 247.299714,206.50190445,179.14325025 \\
-2.10643267631531 0.982198715209961 247.65640545,210.97774605,185.77259745 \\
-1.80747306346893 -0.668740510940552 248.78919705,222.87959565,203.29941555 \\
-2.07424807548523 -1.18663477897644 247.41505305,207.99484275,181.3567548 \\
-1.7894971370697 0.126208305358887 248.3271753,218.42187945,196.75380015 \\
1.54129147529602 0.827900052070618 247.0774866,203.5132356,174.7062783 \\
2.25114989280701 -0.425083756446838 248.3271753,218.42187945,196.75380015 \\
-0.0354030132293701 2.14693307876587 217.46842935,39.9273186,71.1700563 \\
-1.72825217247009 -0.723900735378265 248.18406675,216.9345027,194.56450275 \\
1.96068751811981 -0.992353677749634 248.78919705,222.87959565,203.29941555 \\
-1.71773207187653 0.467935025691986 248.04512235,215.44664655,192.37235955 \\
-2.29478645324707 1.09646511077881 245.82816765,166.06246425,125.780739 \\
-1.43281507492065 0.798199892044067 246.1520712,182.0552406,144.9672297 \\
1.28712749481201 1.31055760383606 246.4728561,192.92848905,159.44460225 \\
-1.66793036460876 0.0794868469238281 246.96984345,202.01754075,172.483173 \\
-1.76066088676453 -0.746844291687012 248.4751161,219.90865695,198.93982845 \\
-1.61421620845795 -1.82398879528046 246.60841665,195.98272095,163.73019855 \\
1.70732688903809 -0.318724393844604 247.91115555,213.9578826,190.17620265 \\
2.37792110443115 1.06209111213684 245.3822772,151.2267249,110.02044105 \\
1.24281573295593 2.07868242263794 245.4455784,152.896572,111.70308405 \\
1.96948981285095 0.8617262840271 250.0727982,233.26498575,218.47535805 \\
2.09552145004272 0.597920656204224 249.48584175,228.81590385,211.9835043 \\
2.69290041923523 -0.245634019374847 240.1273494,93.8058147,66.4006995 \\
-1.93954050540924 -0.0574820637702942 249.6765537,230.29867785,214.14730935 \\
-1.32787334918976 0.0224805474281311 241.29624135,101.5235784,70.0988421 \\
1.79666256904602 1.49265217781067 248.04512235,215.44664655,192.37235955 \\
-2.03462696075439 1.08280897140503 247.91115555,213.9578826,190.17620265 \\
1.30884742736816 -1.97038650512695 246.02364555,175.72078815,137.0778663 \\
1.1958304643631 2.18063402175903 244.6033236,135.93415815,95.5968225 \\
-1.26842856407166 -0.464670300483704 240.4421571,95.75117145,67.24304355 \\
-1.87834739685059 -0.782218277454376 249.3025911,227.33242605,209.8164174 \\
-2.0899965763092 -1.32354879379272 246.68642115,197.50080735,165.8961201 \\
1.91493666172028 0.308309137821198 249.6765537,230.29867785,214.14730935 \\
-1.65407145023346 1.66717708110809 246.7723791,199.01257995,168.07746405 \\
2.20952343940735 -0.940494775772095 246.96984345,202.01754075,172.483173 \\
-1.96256160736084 0.964243590831757 248.78919705,222.87959565,203.29941555 \\
-1.30166971683502 1.55242371559143 246.1872816,183.6258162,146.98295625 \\
-1.73334455490112 0.553060114383698 248.3271753,218.42187945,196.75380015 \\
-1.92249393463135 -1.09420490264893 248.78919705,222.87959565,203.29941555 \\
0.984591424465179 1.76511192321777 245.3822772,151.2267249,110.02044105 \\
-1.78488659858704 2.21930432319641 238.25942085,83.9673231,63.1770864 \\
2.00737261772156 0.970465302467346 249.3025911,227.33242605,209.8164174 \\
1.83330607414246 1.1669704914093 249.48584175,228.81590385,211.9835043 \\
-2.43117761611938 -1.149334192276 244.7095719,137.65917705,97.1365992 \\
-2.09508895874023 0.329589545726776 249.872205,231.7816839,216.3110379 \\
-1.26400721073151 -0.376921832561493 239.06971905,87.92190135,64.2598164 \\
-1.89345860481262 0.882060348987579 249.1250907,225.84872895,207.6473262 \\
2.41621971130371 1.69983172416687 231.6640269,62.64960615,62.7257262 \\
1.98498594760895 1.22024428844452 248.18406675,216.9345027,194.56450275 \\
-2.30485510826111 -0.105398774147034 247.91115555,213.9578826,190.17620265 \\
-2.45244550704956 -0.704998254776001 245.7904863,164.4347457,123.960498 \\
1.51757383346558 0.431248664855957 246.26699715,186.7493979,151.06898445 \\
-2.0239417552948 -0.645441114902496 249.48584175,228.81590385,211.9835043 \\
1.52227342128754 1.08153474330902 247.65640545,210.97774605,185.77259745 \\
0.37205845117569 -2.09700012207031 140.413863,29.12293035,91.13155065 \\
-1.8560426235199 1.05366110801697 248.78919705,222.87959565,203.29941555 \\
1.11565160751343 -1.18187856674194 244.90734735,141.0879912,100.26359535 \\
2.40564107894897 0.550180017948151 246.05468925,177.3121896,139.0238835 \\
-2.24172019958496 -0.0128298997879028 248.6295441,221.39436855,201.1211367 \\
-1.90452873706818 0.71844744682312 249.3025911,227.33242605,209.8164174 \\
-1.35326516628265 2.34318423271179 239.79323055,91.85329245,65.62338555 \\
-2.2371301651001 0.539353907108307 247.91115555,213.9578826,190.17620265 \\
2.05521464347839 -0.573036313056946 249.6765537,230.29867785,214.14730935 \\
-0.812761843204498 -2.17159008979797 239.44318695,89.8893717,64.9046604 \\
-1.65592980384827 2.46370887756348 203.107857,26.79192435,79.21072905 \\
-1.59117817878723 -0.502668976783752 246.4137216,191.3927133,157.3251468 \\
1.31551587581635 -0.987691819667816 246.02364555,175.72078815,137.0778663 \\
-1.55512058734894 -0.65889984369278 246.36063315,189.8505141,155.2234827 \\
2.05941390991211 -2.07674169540405 229.118163,57.28129005,64.02729975 \\
-1.78990578651428 1.34662425518036 247.91115555,213.9578826,190.17620265 \\
-1.80778896808624 -1.21284377574921 248.6295441,221.39436855,201.1211367 \\
-1.61011338233948 1.16422116756439 248.04512235,215.44664655,192.37235955 \\
-2.48677229881287 -0.969598650932312 244.6033236,135.93415815,95.5968225 \\
1.90353226661682 0.500965714454651 249.6765537,230.29867785,214.14730935 \\
1.6567747592926 1.69306755065918 246.7723791,199.01257995,168.07746405 \\
1.94751620292664 -1.8139271736145 245.16534615,146.18443905,105.06918765 \\
1.77272856235504 0.478019267320633 248.18406675,216.9345027,194.56450275 \\
-2.45832347869873 1.11575269699097 242.2543452,109.0670649,74.5740564 \\
-2.03304052352905 -0.308831632137299 249.872205,231.7816839,216.3110379 \\
1.6693274974823 1.06742775440216 248.4751161,219.90865695,198.93982845 \\
-1.16781485080719 2.5474648475647 208.64222655,30.8055453,76.31156775 \\
-1.63688099384308 1.38764882087708 247.78194195,212.46815985,187.97598105 \\
2.06731629371643 0.581284940242767 249.872205,231.7816839,216.3110379 \\
1.47473406791687 -1.0009537935257 247.299714,206.50190445,179.14325025 \\
-1.88627064228058 0.571755647659302 249.3025911,227.33242605,209.8164174 \\
2.12259602546692 1.26636815071106 246.26699715,186.7493979,151.06898445 \\
-2.01838898658752 -0.201952755451202 250.0727982,233.26498575,218.47535805 \\
2.30866003036499 -0.63215708732605 247.0774866,203.5132356,174.7062783 \\
1.64843928813934 1.31394445896149 248.6295441,221.39436855,201.1211367 \\
1.53655791282654 0.201420843601227 246.31171395,188.3028579,153.13767705 \\
1.85695040225983 1.32121610641479 248.9531187,224.36478195,205.4760165 \\
1.52051854133606 -1.24802768230438 247.91115555,213.9578826,190.17620265 \\
2.01107931137085 -1.00178980827332 248.4751161,219.90865695,198.93982845 \\
-1.19855809211731 2.07043814659119 244.81107975,139.37686215,98.692446 \\
2.19588208198547 -0.806151390075684 247.78194195,212.46815985,187.97598105 \\
2.32030987739563 1.67437815666199 240.1273494,93.8058147,66.4006995 \\
2.42011737823486 1.67616784572601 232.440915,64.49618865,62.3991579 \\
-1.15770065784454 -3.35884952545166 4.67144955,5.859033,27.38320305 \\
-1.19488608837128 -1.42772126197815 246.05468925,177.3121896,139.0238835 \\
2.0643367767334 0.0660830140113831 250.25125995,234.75740895,220.66913325 \\
-1.49830341339111 2.37797689437866 235.17106485,72.13249005,61.7853984 \\
-1.28731620311737 -1.35689294338226 246.26699715,186.7493979,151.06898445 \\
2.45492458343506 -0.76689350605011 245.61336585,157.87274145,116.8535001 \\
1.73825192451477 -0.413273930549622 248.3271753,218.42187945,196.75380015 \\
1.52118051052094 0.342036604881287 246.26699715,186.7493979,151.06898445 \\
-2.08654761314392 0.227408945560455 249.872205,231.7816839,216.3110379 \\
1.36297810077667 1.0286191701889 246.31171395,188.3028579,153.13767705 \\
1.95558190345764 0.982532978057861 249.872205,231.7816839,216.3110379 \\
1.23147141933441 1.21459352970123 246.1520712,182.0552406,144.9672297 \\
-1.57104074954987 -0.601061403751373 246.36063315,189.8505141,155.2234827 \\
-1.95749258995056 -0.906054973602295 249.1250907,225.84872895,207.6473262 \\
2.22327065467834 -0.662497222423553 248.04512235,215.44664655,192.37235955 \\
-1.34461069107056 1.01039052009583 246.02364555,175.72078815,137.0778663 \\
1.6566321849823 2.0135281085968 245.16534615,146.18443905,105.06918765 \\
1.87210762500763 0.239041268825531 249.1250907,225.84872895,207.6473262 \\
-1.35418009757996 2.02289080619812 245.61336585,157.87274145,116.8535001 \\
-2.25007939338684 0.83168762922287 246.68642115,197.50080735,165.8961201 \\
-1.13797104358673 -1.89548420906067 245.9932011,174.1238073,135.15058905 \\
1.77921319007874 2.44408583641052 205.91963745,28.6619082,77.7743319 \\
-1.80480432510376 -0.407056272029877 248.6295441,221.39436855,201.1211367 \\
-2.61123538017273 -0.800546169281006 242.03209995,107.19754005,73.39147095 \\
2.04766011238098 -0.956361472606659 248.3271753,218.42187945,196.75380015 \\
-1.19584119319916 -1.09263849258423 245.084376,144.4915935,103.45182465 \\
-1.44032752513885 -1.04827272891998 246.36063315,189.8505141,155.2234827 \\
1.10348558425903 -1.39422821998596 245.56115205,156.2192985,115.1199387 \\
-1.85625040531158 0.987522304058075 248.9531187,224.36478195,205.4760165 \\
-2.34202766418457 -0.28349232673645 247.1866113,205.00830825,176.92716255 \\
-2.81280326843262 -0.201258301734924 226.26503685,52.19834955,65.71232445 \\
-2.43893027305603 -1.79778742790222 222.05177895,45.8543907,68.3333139 \\
-2.03867077827454 0.716543018817902 249.3025911,227.33242605,209.8164174 \\
-1.35928952693939 0.818815469741821 245.75060175,162.8018226,122.1578214 \\
-0.595871686935425 1.52779352664948 218.6436351,41.3758665,70.4500281 \\
1.73244726657867 -0.484057366847992 248.4751161,219.90865695,198.93982845 \\
-2.0888831615448 0.387447535991669 249.6765537,230.29867785,214.14730935 \\
-0.403108268976212 -1.22007191181183 24.4096098,15.42782385,41.31129285 \\
-1.71769297122955 2.24122977256775 239.44318695,89.8893717,64.9046604 \\
1.52449774742126 2.21648693084717 242.2543452,109.0670649,74.5740564 \\
-1.96116602420807 0.0632709860801697 249.872205,231.7816839,216.3110379 \\
1.16128075122833 1.34530377388 246.08576865,178.898718,140.986746 \\
-2.27646088600159 0.120419442653656 248.18406675,216.9345027,194.56450275 \\
1.6236344575882 -1.75882577896118 246.4137216,191.3927133,157.3251468 \\
1.36983263492584 1.22860419750214 246.7723791,199.01257995,168.07746405 \\
2.27950406074524 -0.661543548107147 247.299714,206.50190445,179.14325025 \\
2.58649778366089 0.111716151237488 244.6033236,135.93415815,95.5968225 \\
1.27545714378357 -1.1199939250946 246.02364555,175.72078815,137.0778663 \\
1.74307262897491 -0.138685047626495 248.04512235,215.44664655,192.37235955 \\
-1.80155277252197 -1.63212084770203 247.0774866,203.5132356,174.7062783 \\
1.93605446815491 -0.0971820950508118 249.872205,231.7816839,216.3110379 \\
1.66690123081207 -1.00800323486328 248.78919705,222.87959565,203.29941555 \\
-1.70057213306427 -0.529693007469177 247.53321495,209.48702625,183.56726565 \\
1.7222740650177 0.932026624679565 248.4751161,219.90865695,198.93982845 \\
1.45019292831421 1.61236071586609 247.65640545,210.97774605,185.77259745 \\
2.18957567214966 -0.660133898258209 248.3271753,218.42187945,196.75380015 \\
1.42502892017365 0.020344614982605 245.66176995,159.521268,118.60374105 \\
1.35479319095612 -1.60553419589996 246.7723791,199.01257995,168.07746405 \\
-2.90877985954285 0.339224576950073 186.24136395,22.1498865,86.0585526 \\
1.57439374923706 1.03380334377289 247.91115555,213.9578826,190.17620265 \\
2.47556233406067 -0.323232710361481 245.9932011,174.1238073,135.15058905 \\
1.80427002906799 0.0894767045974731 248.4751161,219.90865695,198.93982845 \\
-1.76344585418701 -1.50044322013855 247.78194195,212.46815985,187.97598105 \\
1.31099581718445 -1.71172940731049 246.4728561,192.92848905,159.44460225 \\
1.88825595378876 1.35972130298615 248.3271753,218.42187945,196.75380015 \\
-1.97972881793976 0.782937347888947 249.3025911,227.33242605,209.8164174 \\
-1.59706771373749 0.862987279891968 247.78194195,212.46815985,187.97598105 \\
-1.58115017414093 1.19142603874207 247.91115555,213.9578826,190.17620265 \\
-2.31768608093262 0.0956005454063416 247.65640545,210.97774605,185.77259745 \\
1.27825653553009 1.33483242988586 246.4137216,191.3927133,157.3251468 \\
-2.03736543655396 2.19491910934448 211.2746961,33.1860009,74.8360587 \\
2.04354906082153 -0.632344901561737 249.6765537,230.29867785,214.14730935 \\
-0.281890690326691 1.83382606506348 217.46842935,39.9273186,71.1700563 \\
-0.723656833171844 1.92728495597839 241.02706845,99.60893895,69.0920562 \\
1.3100733757019 -1.84136176109314 246.26699715,186.7493979,151.06898445 \\
-2.22888612747192 -0.00370830297470093 248.78919705,222.87959565,203.29941555 \\
-0.660729110240936 -2.01507115364075 227.24799045,53.8605594,65.116596 \\
1.90193426609039 0.146382659673691 249.48584175,228.81590385,211.9835043 \\
-2.09596753120422 -0.958219468593597 248.04512235,215.44664655,192.37235955 \\
1.22801387310028 -1.87318515777588 246.05468925,177.3121896,139.0238835 \\
-1.19832503795624 -1.70437479019165 246.26699715,186.7493979,151.06898445 \\
-2.78599715232849 -1.14868593215942 186.24136395,22.1498865,86.0585526 \\
-1.79475843906403 -1.43635714054108 248.04512235,215.44664655,192.37235955 \\
-2.03723311424255 -0.241518139839172 250.0727982,233.26498575,218.47535805 \\
2.07432079315186 -0.28836989402771 250.0727982,233.26498575,218.47535805 \\
-1.90951156616211 -0.08542400598526 249.48584175,228.81590385,211.9835043 \\
2.3860034942627 -0.555282950401306 246.36063315,189.8505141,155.2234827 \\
1.70477747917175 -0.737141489982605 248.6295441,221.39436855,201.1211367 \\
-1.13734531402588 1.91611528396606 245.2421496,147.8709147,106.70345205 \\
-1.8172219991684 1.27142667770386 248.18406675,216.9345027,194.56450275 \\
-2.03851938247681 -0.947414875030518 248.6295441,221.39436855,201.1211367 \\
-2.1339271068573 -0.514521360397339 249.1250907,225.84872895,207.6473262 \\
-1.33013033866882 1.98524475097656 245.70760365,161.1641412,120.3721023 \\
-2.77251482009888 -0.168382346630096 234.54423405,70.19178705,61.82723115 \\
-1.9475781917572 1.11301934719086 248.3271753,218.42187945,196.75380015 \\
2.20897030830383 -0.671970188617706 248.18406675,216.9345027,194.56450275 \\
1.8212183713913 1.73043620586395 246.1520712,182.0552406,144.9672297 \\
-1.89205133914948 -1.7878475189209 246.11813835,180.47969505,142.9681215 \\
0.863698065280914 0.528403043746948 148.97850975,27.85290285,91.41359595 \\
-1.15608692169189 -1.28775155544281 245.61336585,157.87274145,116.8535001 \\
1.30480909347534 -2.00714111328125 245.9304711,170.9147751,131.3488017 \\
-2.16541004180908 -1.64475631713867 245.4455784,152.896572,111.70308405 \\
2.01177549362183 -1.97430491447449 240.74170815,97.6856805,68.1411102 \\
1.39689779281616 -1.54892015457153 247.0774866,203.5132356,174.7062783 \\
-1.55766904354095 0.878081440925598 247.53321495,209.48702625,183.56726565 \\
-1.5693119764328 0.631574332714081 246.96984345,202.01754075,172.483173 \\
-2.09924554824829 -1.48813700675964 246.1520712,182.0552406,144.9672297 \\
-1.51385462284088 1.31234860420227 247.41505305,207.99484275,181.3567548 \\
-1.908855676651 1.65976476669312 246.1520712,182.0552406,144.9672297 \\
1.81042420864105 -0.926354944705963 249.3025911,227.33242605,209.8164174 \\
0.940635204315186 -2.04075527191162 244.1226945,128.96176965,89.5838256 \\
-1.84155762195587 0.242129623889923 248.9531187,224.36478195,205.4760165 \\
-2.03759860992432 0.737267911434174 249.1250907,225.84872895,207.6473262 \\
2.10921454429626 -0.499467968940735 249.48584175,228.81590385,211.9835043 \\
1.90368366241455 -1.57124078273773 246.26699715,186.7493979,151.06898445 \\
-1.93420076370239 -0.604689240455627 249.6765537,230.29867785,214.14730935 \\
-1.63096511363983 1.81057989597321 246.31171395,188.3028579,153.13767705 \\
-0.739658534526825 -1.41908574104309 225.2521335,50.5679178,66.33616155 \\
2.24310040473938 0.00842493772506714 248.78919705,222.87959565,203.29941555 \\
-1.36184275150299 0.0415591597557068 243.0470178,116.43654195,79.6614288 \\
-2.32843732833862 -0.725006461143494 246.4728561,192.92848905,159.44460225 \\
-1.6350474357605 0.0785503387451172 246.60841665,195.98272095,163.73019855 \\
1.8904767036438 -1.32996690273285 247.53321495,209.48702625,183.56726565 \\
-1.63559174537659 -0.288254916667938 246.53702175,194.45887155,161.57890635 \\
-1.82065451145172 -0.942648351192474 249.1250907,225.84872895,207.6473262 \\
1.59450626373291 -0.857535481452942 248.04512235,215.44664655,192.37235955 \\
-1.16459989547729 1.31525373458862 245.56115205,156.2192985,115.1199387 \\
-1.48177993297577 -0.889885604381561 246.31171395,188.3028579,153.13767705 \\
1.51856207847595 -0.281061232089996 246.36063315,189.8505141,155.2234827 \\
1.60597956180573 -1.35221219062805 248.18406675,216.9345027,194.56450275 \\
1.41497373580933 1.64458453655243 247.1866113,205.00830825,176.92716255 \\
1.58997237682343 1.06487679481506 248.04512235,215.44664655,192.37235955 \\
-1.85856437683105 0.170551061630249 249.1250907,225.84872895,207.6473262 \\
1.07814693450928 1.87029671669006 245.70760365,161.1641412,120.3721023 \\
2.16072916984558 -0.191330790519714 249.6765537,230.29867785,214.14730935 \\
1.64611315727234 -0.388928174972534 247.53321495,209.48702625,183.56726565 \\
-1.65975689888 -0.342831373214722 246.86670105,200.51813565,170.27341695 \\
-2.42272210121155 -0.00241750478744507 246.4137216,191.3927133,157.3251468 \\
1.69371330738068 -0.942784547805786 248.78919705,222.87959565,203.29941555 \\
2.11306285858154 0.605248212814331 249.3025911,227.33242605,209.8164174 \\
-1.67643177509308 -1.44401776790619 248.18406675,216.9345027,194.56450275 \\
-1.41561949253082 -0.352276563644409 244.99843845,142.7928192,101.84999115 \\
-2.32538795471191 -0.627251505851746 246.68642115,197.50080735,165.8961201 \\
2.02112436294556 -1.40697717666626 246.36063315,189.8505141,155.2234827 \\
1.70479226112366 -1.07345128059387 248.9531187,224.36478195,205.4760165 \\
-2.09254312515259 0.159519255161285 250.0727982,233.26498575,218.47535805 \\
1.29582023620605 1.02321910858154 246.08576865,178.898718,140.986746 \\
-1.57471680641174 -0.670924603939056 246.53702175,194.45887155,161.57890635 \\
2.14252257347107 -0.274482727050781 249.6765537,230.29867785,214.14730935 \\
-1.40840375423431 1.98033213615417 245.82816765,166.06246425,125.780739 \\
-2.28322505950928 -0.395807266235352 247.78194195,212.46815985,187.97598105 \\
2.06478595733643 -0.112351477146149 250.25125995,234.75740895,220.66913325 \\
-1.90570175647736 -1.65563201904297 246.4728561,192.92848905,159.44460225 \\
1.84956955909729 -0.598071813583374 249.6765537,230.29867785,214.14730935 \\
-1.98330342769623 -0.464040815830231 249.872205,231.7816839,216.3110379 \\
-1.44402468204498 2.0041069984436 245.7904863,164.4347457,123.960498 \\
-1.85928511619568 0.252472758293152 249.1250907,225.84872895,207.6473262 \\
-1.85319900512695 -0.667503595352173 249.1250907,225.84872895,207.6473262 \\
2.13178396224976 -0.421098709106445 249.48584175,228.81590385,211.9835043 \\
-1.19542765617371 -1.81967508792877 246.1872816,183.6258162,146.98295625 \\
-2.1148989200592 -1.48672366142273 246.08576865,178.898718,140.986746 \\
0.156774058938026 1.15523493289948 59.50181985,26.17252935,65.5292268 \\
-1.7684178352356 -0.837696850299835 248.6295441,221.39436855,201.1211367 \\
2.632328748703 0.308821797370911 242.46705855,110.92501275,75.79318335 \\
-2.35070276260376 -0.538277685642242 246.60841665,195.98272095,163.73019855 \\
2.34077000617981 0.640486240386963 246.31171395,188.3028579,153.13767705 \\
-2.10289692878723 -0.0354636907577515 250.0727982,233.26498575,218.47535805 \\
1.24291527271271 2.72576761245728 177.99958035,22.6879263,88.306143 \\
3.29249811172485 -0.112112045288086 2.70507825,4.61094825,25.5475677 \\
-1.46271979808807 -0.651801824569702 245.8981065,169.30226475,129.4754085 \\
1.21394646167755 -1.13055753707886 245.70760365,161.1641412,120.3721023 \\
-1.3691383600235 2.09486246109009 245.31448035,149.55157125,108.35396505 \\
1.72182106971741 -1.07014489173889 248.9531187,224.36478195,205.4760165 \\
1.85235011577606 -0.931996762752533 249.3025911,227.33242605,209.8164174 \\
1.57884800434113 0.0331527590751648 246.60841665,195.98272095,163.73019855 \\
1.61052584648132 -1.10955417156219 248.4751161,219.90865695,198.93982845 \\
2.17919564247131 0.247315913438797 249.3025911,227.33242605,209.8164174 \\
1.89941847324371 -1.15587425231934 248.3271753,218.42187945,196.75380015 \\
1.37095606327057 -1.37042617797852 246.96984345,202.01754075,172.483173 \\
-2.25397896766663 0.134880185127258 248.4751161,219.90865695,198.93982845 \\
-1.42244291305542 1.83330011367798 246.11813835,180.47969505,142.9681215 \\
2.26531457901001 -0.102589786052704 248.4751161,219.90865695,198.93982845 \\
1.50558316707611 0.176420718431473 246.1520712,182.0552406,144.9672297 \\
-1.8025518655777 1.08583903312683 248.6295441,221.39436855,201.1211367 \\
1.75453424453735 -1.57727873325348 246.86670105,200.51813565,170.27341695 \\
1.48771524429321 -0.919957518577576 247.299714,206.50190445,179.14325025 \\
-1.08740794658661 1.51589477062225 245.4455784,152.896572,111.70308405 \\
2.39857363700867 -0.260833203792572 246.60841665,195.98272095,163.73019855 \\
-1.20758020877838 2.40958333015442 234.54423405,70.19178705,61.82723115 \\
-2.08870387077332 -1.27981495857239 246.86670105,200.51813565,170.27341695 \\
-2.57553958892822 -0.00151526927947998 245.31448035,149.55157125,108.35396505 \\
1.87703716754913 1.43224883079529 247.91115555,213.9578826,190.17620265 \\
1.91475176811218 0.632655024528503 250.0727982,233.26498575,218.47535805 \\
2.09328746795654 -0.352451682090759 249.872205,231.7816839,216.3110379 \\
1.18539786338806 -1.60556054115295 246.05468925,177.3121896,139.0238835 \\
1.71987771987915 -1.01594829559326 248.9531187,224.36478195,205.4760165 \\
-1.87165820598602 1.46965539455414 247.0774866,203.5132356,174.7062783 \\
1.7885457277298 0.246375381946564 248.18406675,216.9345027,194.56450275 \\
1.91626226902008 -0.44359678030014 249.872205,231.7816839,216.3110379 \\
-2.04283881187439 0.301111817359924 250.0727982,233.26498575,218.47535805 \\
1.84664762020111 0.0703232884407043 248.9531187,224.36478195,205.4760165 \\
-1.9288318157196 1.68339800834656 246.02364555,175.72078815,137.0778663 \\
-2.56429958343506 -1.01183128356934 241.79993775,105.3160812,72.24732165 \\
1.85141003131866 0.705569088459015 249.3025911,227.33242605,209.8164174 \\
1.91963827610016 0.190190136432648 249.6765537,230.29867785,214.14730935 \\
-2.06999468803406 0.355746209621429 249.872205,231.7816839,216.3110379 \\
2.05160021781921 0.904916763305664 248.9531187,224.36478195,205.4760165 \\
2.61124205589294 0.0376894474029541 244.1226945,128.96176965,89.5838256 \\
1.53346514701843 -0.428543388843536 246.68642115,197.50080735,165.8961201 \\
-1.83200109004974 -0.542887389659882 248.9531187,224.36478195,205.4760165 \\
-1.97259593009949 -0.37887442111969 249.872205,231.7816839,216.3110379 \\
1.46383452415466 -1.60305118560791 247.1866113,205.00830825,176.92716255 \\
-1.00107359886169 -1.25243592262268 242.863122,114.60937515,78.3386775 \\
-1.56017661094666 -1.17324423789978 247.65640545,210.97774605,185.77259745 \\
-1.62867069244385 1.42576956748962 247.65640545,210.97774605,185.77259745 \\
1.31755399703979 -1.98804593086243 245.9932011,174.1238073,135.15058905 \\
2.22174048423767 -0.526194930076599 248.4751161,219.90865695,198.93982845 \\
-2.52773237228394 0.672555148601532 244.4915877,134.20217775,94.0718052 \\
2.49952340126038 -0.516765415668488 245.66176995,159.521268,118.60374105 \\
2.41092300415039 -0.269866168498993 246.4728561,192.92848905,159.44460225 \\
-1.76566851139069 1.64014565944672 246.68642115,197.50080735,165.8961201 \\
-2.03285956382751 -0.669252455234528 249.48584175,228.81590385,211.9835043 \\
-1.36230444908142 1.09840154647827 246.2259753,185.1901596,149.01788685 \\
-2.5102870464325 1.41708171367645 217.46842935,39.9273186,71.1700563 \\
2.05307984352112 1.43022787570953 246.1520712,182.0552406,144.9672297 \\
-1.76206600666046 -0.934198617935181 248.78919705,222.87959565,203.29941555 \\
-2.40406084060669 0.365454614162445 246.31171395,188.3028579,153.13767705 \\
-2.12484955787659 -0.534233391284943 249.1250907,225.84872895,207.6473262 \\
1.93890786170959 -0.537059724330902 249.872205,231.7816839,216.3110379 \\
-1.16492748260498 1.65115976333618 245.7904863,164.4347457,123.960498 \\
2.07172822952271 0.153251886367798 250.25125995,234.75740895,220.66913325 \\
1.45938575267792 1.56600689888 247.91115555,213.9578826,190.17620265 \\
1.32381963729858 2.36501741409302 238.25942085,83.9673231,63.1770864 \\
-2.13366889953613 0.777957677841187 248.18406675,216.9345027,194.56450275 \\
-2.13167357444763 0.830819606781006 248.04512235,215.44664655,192.37235955 \\
1.48563253879547 0.0586573779582977 246.05468925,177.3121896,139.0238835 \\
-2.01244020462036 -0.752949655056 249.3025911,227.33242605,209.8164174 \\
-1.37948226928711 -2.09536480903625 245.75060175,162.8018226,122.1578214 \\
1.63584971427917 0.869458079338074 247.78194195,212.46815985,187.97598105 \\
1.20738053321838 -1.13237142562866 245.66176995,159.521268,118.60374105 \\
-1.65826022624969 -0.982518374919891 248.04512235,215.44664655,192.37235955 \\
2.09311079978943 -0.390748381614685 249.872205,231.7816839,216.3110379 \\
-2.02576851844788 -0.0750674605369568 250.0727982,233.26498575,218.47535805 \\
1.96110844612122 -0.861170172691345 249.3025911,227.33242605,209.8164174 \\
2.44606113433838 0.58360368013382 245.7904863,164.4347457,123.960498 \\
-1.8457053899765 0.584699511528015 249.1250907,225.84872895,207.6473262 \\
0.762781500816345 2.37006568908691 218.6436351,41.3758665,70.4500281 \\
-1.74076676368713 -0.89445561170578 248.6295441,221.39436855,201.1211367 \\
-2.55999040603638 -0.120907187461853 245.50510815,154.5607581,113.4028299 \\
1.39887273311615 1.6228392124176 247.1866113,205.00830825,176.92716255 \\
1.91221189498901 -1.56866836547852 246.26699715,186.7493979,151.06898445 \\
1.8475649356842 -1.81490886211395 245.70760365,161.1641412,120.3721023 \\
2.11859393119812 -0.0915102362632751 250.0727982,233.26498575,218.47535805 \\
-1.87738597393036 -0.709208190441132 249.3025911,227.33242605,209.8164174 \\
1.88482427597046 -1.68445229530334 246.02364555,175.72078815,137.0778663 \\
1.76187598705292 1.00812327861786 248.9531187,224.36478195,205.4760165 \\
-2.04749369621277 -1.76277053356171 245.66176995,159.521268,118.60374105 \\
1.90669822692871 1.31164181232452 248.4751161,219.90865695,198.93982845 \\
-2.37577295303345 0.122925400733948 246.86670105,200.51813565,170.27341695 \\
1.50299346446991 -0.968505620956421 247.53321495,209.48702625,183.56726565 \\
1.2259281873703 0.477284789085388 241.79993775,105.3160812,72.24732165 \\
1.27416300773621 -1.52949643135071 246.4137216,191.3927133,157.3251468 \\
-2.20228362083435 0.149752020835876 249.1250907,225.84872895,207.6473262 \\
-0.578849196434021 1.68425178527832 232.440915,64.49618865,62.3991579 \\
-2.07547187805176 0.544100403785706 249.48584175,228.81590385,211.9835043 \\
1.78931665420532 -0.294508457183838 248.6295441,221.39436855,201.1211367 \\
-2.22558403015137 -0.48099273443222 248.3271753,218.42187945,196.75380015 \\
-1.99961960315704 0.365435063838959 249.872205,231.7816839,216.3110379 \\
1.96953558921814 0.98076194524765 249.6765537,230.29867785,214.14730935 \\
1.95081543922424 -1.44674205780029 246.53702175,194.45887155,161.57890635 \\
-1.72159826755524 -0.252581238746643 247.53321495,209.48702625,183.56726565 \\
1.96908497810364 -0.9093017578125 249.1250907,225.84872895,207.6473262 \\
2.49479651451111 -0.366933882236481 245.8641966,167.68475895,127.6195797 \\
1.95200657844543 0.658796846866608 250.25125995,234.75740895,220.66913325 \\
1.1757949590683 1.95957326889038 245.7904863,164.4347457,123.960498 \\
-1.10878384113312 2.42296838760376 231.6640269,62.64960615,62.7257262 \\
-2.8725950717926 -1.07717728614807 157.5703803,26.3294844,91.1882754 \\
1.56026685237885 -1.88450002670288 246.1520712,182.0552406,144.9672297 \\
1.18224632740021 -1.32357835769653 245.8641966,167.68475895,127.6195797 \\
-2.45636963844299 -0.794736862182617 245.61336585,157.87274145,116.8535001 \\
-1.27529764175415 -2.01872181892395 245.9304711,170.9147751,131.3488017 \\
-1.74934840202332 0.235705614089966 248.04512235,215.44664655,192.37235955 \\
1.46439933776855 -1.7356493473053 246.7723791,199.01257995,168.07746405 \\
-1.96164691448212 -1.4656366109848 246.96984345,202.01754075,172.483173 \\
2.02401852607727 -0.69386088848114 249.48584175,228.81590385,211.9835043 \\
-1.87624287605286 -0.508590340614319 249.3025911,227.33242605,209.8164174 \\
-2.12947869300842 -0.27534431219101 249.6765537,230.29867785,214.14730935 \\
-1.63232779502869 -0.507721543312073 246.7723791,199.01257995,168.07746405 \\
1.8414055109024 -0.170032680034637 249.1250907,225.84872895,207.6473262 \\
-1.71224975585938 0.650433957576752 248.3271753,218.42187945,196.75380015 \\
1.4255119562149 2.13778781890869 243.55115025,121.8574518,83.79951015 \\
-2.16967105865479 -0.35752135515213 249.1250907,225.84872895,207.6473262 \\
2.0173602104187 0.507911026477814 250.25125995,234.75740895,220.66913325 \\
-1.72143459320068 -1.03152346611023 248.6295441,221.39436855,201.1211367 \\
1.80328905582428 1.79113698005676 245.9932011,174.1238073,135.15058905 \\
-1.4470192193985 -1.65944826602936 247.1866113,205.00830825,176.92716255 \\
1.79022204875946 -1.71648180484772 246.1520712,182.0552406,144.9672297 \\
-1.91381692886353 0.0851683616638184 249.48584175,228.81590385,211.9835043 \\
1.32258439064026 -0.00178128480911255 243.8483094,125.42774505,86.66328255 \\
1.5326464176178 -1.40312480926514 247.91115555,213.9578826,190.17620265 \\
-1.78641772270203 0.521221399307251 248.78919705,222.87959565,203.29941555 \\
-1.35918974876404 -2.00715565681458 246.02364555,175.72078815,137.0778663 \\
1.82546651363373 0.314886838197708 248.6295441,221.39436855,201.1211367 \\
1.97300314903259 -0.208988904953003 250.25125995,234.75740895,220.66913325 \\
-2.05313754081726 0.0151283740997314 250.0727982,233.26498575,218.47535805 \\
2.23122954368591 0.303995728492737 248.4751161,219.90865695,198.93982845 \\
1.81585538387299 1.53795027732849 247.53321495,209.48702625,183.56726565 \\
2.15793228149414 -0.159951627254486 249.6765537,230.29867785,214.14730935 \\
2.08015465736389 0.168095022439957 250.25125995,234.75740895,220.66913325 \\
1.65298891067505 -0.554530441761017 247.91115555,213.9578826,190.17620265 \\
-1.55512726306915 -1.1397465467453 247.53321495,209.48702625,183.56726565 \\
1.93544960021973 0.428838342428207 250.0727982,233.26498575,218.47535805 \\
1.96541678905487 -1.82569074630737 244.90734735,141.0879912,100.26359535 \\
-2.22452092170715 0.196995198726654 248.78919705,222.87959565,203.29941555 \\
-1.31016910076141 -0.403583824634552 242.2543452,109.0670649,74.5740564 \\
-1.57780468463898 -0.294254660606384 246.1872816,183.6258162,146.98295625 \\
1.28606700897217 0.21260042488575 242.863122,114.60937515,78.3386775 \\
2.07386779785156 -0.793650090694427 248.9531187,224.36478195,205.4760165 \\
1.77904582023621 1.28659582138062 249.1250907,225.84872895,207.6473262 \\
1.32472264766693 -2.5145320892334 226.26503685,52.19834955,65.71232445 \\
-1.76567411422729 0.0376110672950745 248.04512235,215.44664655,192.37235955 \\
-1.79017961025238 -0.221685290336609 248.3271753,218.42187945,196.75380015 \\
1.38671159744263 0.501713633537292 245.70760365,161.1641412,120.3721023 \\
1.24288868904114 1.74205303192139 246.1872816,183.6258162,146.98295625 \\
-2.19194293022156 1.35533428192139 245.75060175,162.8018226,122.1578214 \\
0.942081928253174 -1.39058065414429 243.98829675,127.1991204,88.11634905 \\
1.44142556190491 -0.732909739017487 246.4137216,191.3927133,157.3251468 \\
1.95742249488831 0.212636888027191 250.0727982,233.26498575,218.47535805 \\
1.83341455459595 -0.582246601581573 249.48584175,228.81590385,211.9835043 \\
-1.81193542480469 0.1811483502388 248.6295441,221.39436855,201.1211367 \\
2.07791423797607 0.644845485687256 249.48584175,228.81590385,211.9835043 \\
1.41411876678467 -1.13545727729797 246.96984345,202.01754075,172.483173 \\
-1.46092164516449 -1.57925403118134 247.41505305,207.99484275,181.3567548 \\
-2.0081775188446 0.0370747447013855 250.0727982,233.26498575,218.47535805 \\
1.12414538860321 -1.03840470314026 244.1226945,128.96176965,89.5838256 \\
-1.58286452293396 1.58407425880432 247.0774866,203.5132356,174.7062783 \\
-1.85773658752441 0.52369612455368 249.3025911,227.33242605,209.8164174 \\
-1.27004039287567 1.76380002498627 245.9304711,170.9147751,131.3488017 \\
-1.55686366558075 -1.0555864572525 247.299714,206.50190445,179.14325025 \\
-2.35459780693054 1.04074192047119 245.50510815,154.5607581,113.4028299 \\
-2.03605079650879 -0.614206194877625 249.48584175,228.81590385,211.9835043 \\
-1.83863091468811 -0.678853273391724 249.1250907,225.84872895,207.6473262 \\
1.17257261276245 -0.0470783114433289 233.8786203,68.27324355,61.95004935 \\
1.42254853248596 1.28924059867859 247.41505305,207.99484275,181.3567548 \\
2.47416377067566 -0.271492898464203 246.02364555,175.72078815,137.0778663 \\
0.618879973888397 1.85464751720428 237.34924425,80.00364135,62.39544765 \\
-1.37319242954254 2.33476448059082 240.4421571,95.75117145,67.24304355 \\
1.81152701377869 -1.57525992393494 246.60841665,195.98272095,163.73019855 \\
1.63402128219604 -1.22856390476227 248.4751161,219.90865695,198.93982845 \\
-1.44734454154968 1.67960178852081 246.4137216,191.3927133,157.3251468 \\
1.51397442817688 -0.614361882209778 246.86670105,200.51813565,170.27341695 \\
-2.26186800003052 0.374013841152191 248.04512235,215.44664655,192.37235955 \\
0.903074622154236 -0.915866374969482 208.64222655,30.8055453,76.31156775 \\
1.44378185272217 -1.25971817970276 247.41505305,207.99484275,181.3567548 \\
1.33014416694641 -1.30621647834778 246.53702175,194.45887155,161.57890635 \\
-1.46974968910217 -1.07620573043823 246.60841665,195.98272095,163.73019855 \\
2.06590390205383 0.628491997718811 249.6765537,230.29867785,214.14730935 \\
2.02277827262878 -0.0605742335319519 250.25125995,234.75740895,220.66913325 \\
2.12402129173279 0.114181876182556 250.0727982,233.26498575,218.47535805 \\
-1.16997337341309 -2.16668796539307 244.90734735,141.0879912,100.26359535 \\
-1.65826487541199 -2.34002804756165 240.1273494,93.8058147,66.4006995 \\
1.98166990280151 0.469977378845215 250.25125995,234.75740895,220.66913325 \\
2.32995247840881 0.239172965288162 247.299714,206.50190445,179.14325025 \\
1.26285827159882 -1.91351509094238 246.05468925,177.3121896,139.0238835 \\
1.87221765518188 1.20047557353973 249.48584175,228.81590385,211.9835043 \\
-0.865090608596802 -2.05190181732178 243.39135705,120.0597069,82.39293015 \\
1.91336166858673 -0.066653311252594 249.6765537,230.29867785,214.14730935 \\
2.07102513313293 -0.370731353759766 250.0727982,233.26498575,218.47535805 \\
-2.0207257270813 -0.00405526161193848 250.0727982,233.26498575,218.47535805 \\
-2.27454257011414 -0.825175046920776 246.68642115,197.50080735,165.8961201 \\
-0.96457827091217 -0.611506342887878 176.3229834,22.9162125,88.6810032 \\
2.00136590003967 1.45677995681763 246.36063315,189.8505141,155.2234827 \\
-1.71509718894958 0.570818722248077 248.3271753,218.42187945,196.75380015 \\
2.25297904014587 -0.517677128314972 248.04512235,215.44664655,192.37235955 \\
2.17798066139221 1.19801914691925 246.1520712,182.0552406,144.9672297 \\
-1.85452330112457 1.18969643115997 248.3271753,218.42187945,196.75380015 \\
1.1975519657135 -2.40726613998413 239.06971905,87.92190135,64.2598164 \\
-2.05079674720764 -0.724382519721985 249.1250907,225.84872895,207.6473262 \\
-1.36797380447388 1.05097341537476 246.1872816,183.6258162,146.98295625 \\
-2.00931930541992 -0.79842746257782 249.3025911,227.33242605,209.8164174 \\
-1.78714573383331 1.35849916934967 247.91115555,213.9578826,190.17620265 \\
1.56856036186218 -0.703245341777802 247.53321495,209.48702625,183.56726565 \\
2.23995447158813 1.11751711368561 246.05468925,177.3121896,139.0238835 \\
1.48390734195709 0.744212925434113 246.4137216,191.3927133,157.3251468 \\
-1.85190439224243 0.115765511989594 248.9531187,224.36478195,205.4760165 \\
-1.3451509475708 -1.96950089931488 246.11813835,180.47969505,142.9681215 \\
2.47219634056091 0.970063090324402 244.1226945,128.96176965,89.5838256 \\
-1.32194375991821 2.00590348243713 245.61336585,157.87274145,116.8535001 \\
2.02441382408142 0.562201142311096 250.25125995,234.75740895,220.66913325 \\
1.81076908111572 0.815370798110962 249.1250907,225.84872895,207.6473262 \\
-2.50600552558899 -0.614184319972992 245.50510815,154.5607581,113.4028299 \\
-2.07948517799377 0.551925301551819 249.48584175,228.81590385,211.9835043 \\
1.70772755146027 0.667757213115692 247.78194195,212.46815985,187.97598105 \\
-1.55014026165009 -1.03972780704498 247.1866113,205.00830825,176.92716255 \\
2.35134172439575 -0.0992128252983093 247.299714,206.50190445,179.14325025 \\
-1.77374076843262 -0.223012804985046 248.04512235,215.44664655,192.37235955 \\
2.04549884796143 0.412942111492157 250.25125995,234.75740895,220.66913325 \\
2.12073516845703 0.539420485496521 249.3025911,227.33242605,209.8164174 \\
1.55879771709442 0.698661327362061 246.86670105,200.51813565,170.27341695 \\
2.40844631195068 0.540746867656708 246.05468925,177.3121896,139.0238835 \\
2.1098005771637 0.836789786815643 248.3271753,218.42187945,196.75380015 \\
-1.76205277442932 -0.27011102437973 248.04512235,215.44664655,192.37235955 \\
2.34730935096741 -0.759705543518066 246.31171395,188.3028579,153.13767705 \\
-2.35945296287537 0.467950880527496 246.53702175,194.45887155,161.57890635 \\
-1.11649084091187 -1.55385363101959 245.96240475,172.5216882,133.24118985 \\
-2.45879530906677 0.338741898536682 246.02364555,175.72078815,137.0778663 \\
2.04332804679871 0.754828214645386 249.6765537,230.29867785,214.14730935 \\
2.03619575500488 1.85708856582642 242.2543452,109.0670649,74.5740564 \\
-1.61239004135132 -0.0200458168983459 246.36063315,189.8505141,155.2234827 \\
-1.67462241649628 0.967716991901398 248.3271753,218.42187945,196.75380015 \\
2.23571443557739 -0.488500893115997 248.3271753,218.42187945,196.75380015 \\
2.71996736526489 -0.205453097820282 237.34924425,80.00364135,62.39544765 \\
-2.53403377532959 0.0111280679702759 245.70760365,161.1641412,120.3721023 \\
1.94501507282257 1.44054436683655 247.0774866,203.5132356,174.7062783 \\
-1.37773978710175 -1.40856754779816 246.86670105,200.51813565,170.27341695 \\
-1.10034775733948 -0.911074757575989 239.44318695,89.8893717,64.9046604 \\
-2.0230085849762 0.430984079837799 249.872205,231.7816839,216.3110379 \\
2.37325739860535 0.159591734409332 246.7723791,199.01257995,168.07746405 \\
-1.88586366176605 -0.351745545864105 249.3025911,227.33242605,209.8164174 \\
-1.6714426279068 -1.44889295101166 248.18406675,216.9345027,194.56450275 \\
-1.57919704914093 1.15534126758575 247.91115555,213.9578826,190.17620265 \\
-0.949381589889526 -2.15980553627014 243.39135705,120.0597069,82.39293015 \\
1.06710827350616 1.93839848041534 245.4455784,152.896572,111.70308405 \\
2.124831199646 0.555135726928711 249.3025911,227.33242605,209.8164174 \\
-2.26260828971863 -0.54795914888382 247.65640545,210.97774605,185.77259745 \\
1.2398579120636 -1.38382697105408 246.1520712,182.0552406,144.9672297 \\
-1.64870977401733 1.52021133899689 247.41505305,207.99484275,181.3567548 \\
2.54833197593689 -0.293217658996582 245.3822772,151.2267249,110.02044105 \\
2.38475251197815 0.920497298240662 245.66176995,159.521268,118.60374105 \\
1.81460356712341 -0.853139817714691 249.48584175,228.81590385,211.9835043 \\
-1.05557942390442 -1.12519598007202 242.2543452,109.0670649,74.5740564 \\
-2.33960175514221 -0.491006851196289 246.86670105,200.51813565,170.27341695 \\
-1.99228131771088 1.04326736927032 248.3271753,218.42187945,196.75380015 \\
-2.3858757019043 0.629494607448578 246.11813835,180.47969505,142.9681215 \\
2.38391494750977 0.661466956138611 246.05468925,177.3121896,139.0238835 \\
-1.55630660057068 -2.12596035003662 245.50510815,154.5607581,113.4028299 \\
1.66924893856049 1.48152005672455 248.18406675,216.9345027,194.56450275 \\
-1.60052239894867 -1.53570067882538 247.91115555,213.9578826,190.17620265 \\
-1.00282537937164 1.33647608757019 243.0470178,116.43654195,79.6614288 \\
-2.09529280662537 1.10948693752289 247.1866113,205.00830825,176.92716255 \\
-2.00303339958191 1.56266832351685 246.08576865,178.898718,140.986746 \\
2.29564380645752 0.158170461654663 247.91115555,213.9578826,190.17620265 \\
1.6567907333374 -1.07123064994812 248.6295441,221.39436855,201.1211367 \\
1.04927825927734 -0.365848958492279 186.24136395,22.1498865,86.0585526 \\
1.22488307952881 -0.241361796855927 240.74170815,97.6856805,68.1411102 \\
-1.71169829368591 -1.19309461116791 248.6295441,221.39436855,201.1211367 \\
-1.47136497497559 -1.26590728759766 247.1866113,205.00830825,176.92716255 \\
-1.86489510536194 1.07146537303925 248.6295441,221.39436855,201.1211367 \\
1.48578500747681 0.424443185329437 246.11813835,180.47969505,142.9681215 \\
2.24464225769043 0.269609570503235 248.3271753,218.42187945,196.75380015 \\
-2.24721717834473 -0.0244840979576111 248.6295441,221.39436855,201.1211367 \\
1.49276971817017 1.37589299678802 248.3271753,218.42187945,196.75380015 \\
1.64889097213745 -2.05127811431885 245.2421496,147.8709147,106.70345205 \\
-1.59845399856567 0.487941324710846 246.86670105,200.51813565,170.27341695 \\
1.30030620098114 -1.70640599727631 246.4137216,191.3927133,157.3251468 \\
-1.58312022686005 0.941842257976532 247.78194195,212.46815985,187.97598105 \\
1.99820291996002 1.11858189105988 248.6295441,221.39436855,201.1211367 \\
-1.7993152141571 0.472241282463074 248.78919705,222.87959565,203.29941555 \\
-1.99300515651703 0.695940315723419 249.48584175,228.81590385,211.9835043 \\
2.1174590587616 -0.346244633197784 249.872205,231.7816839,216.3110379 \\
-1.13170230388641 1.56214022636414 245.70760365,161.1641412,120.3721023 \\
1.98267567157745 1.28322768211365 247.65640545,210.97774605,185.77259745 \\
2.17419838905334 -0.350331366062164 249.3025911,227.33242605,209.8164174 \\
1.62637460231781 1.33475720882416 248.4751161,219.90865695,198.93982845 \\
-2.04679250717163 0.160336375236511 250.0727982,233.26498575,218.47535805 \\
-1.55578863620758 -0.739692270755768 246.4728561,192.92848905,159.44460225 \\
0.555891394615173 -1.55587601661682 203.107857,26.79192435,79.21072905 \\
0.930608987808228 2.21487474441528 240.4421571,95.75117145,67.24304355 \\
1.87290322780609 -0.25165992975235 249.48584175,228.81590385,211.9835043 \\
-1.61737263202667 1.02452802658081 248.18406675,216.9345027,194.56450275 \\
-1.56978988647461 -1.64986753463745 247.41505305,207.99484275,181.3567548 \\
1.87308633327484 -0.268692433834076 249.48584175,228.81590385,211.9835043 \\
1.47308707237244 0.770207285881042 246.4137216,191.3927133,157.3251468 \\
2.43581748008728 0.0251495838165283 246.26699715,186.7493979,151.06898445 \\
1.86108183860779 1.84273862838745 245.3822772,151.2267249,110.02044105 \\
-1.89045786857605 1.64476883411407 246.26699715,186.7493979,151.06898445 \\
2.00482988357544 -2.23386025428772 208.64222655,30.8055453,76.31156775 \\
-1.36850607395172 2.25063753128052 243.22306725,118.25314155,81.01330365 \\
2.18605184555054 -0.745425999164581 248.04512235,215.44664655,192.37235955 \\
2.42554426193237 -0.501214623451233 246.1520712,182.0552406,144.9672297 \\
-1.336270570755 -1.2730758190155 246.31171395,188.3028579,153.13767705 \\
2.43131327629089 2.08815360069275 177.99958035,22.6879263,88.306143 \\
1.86380290985107 1.3630633354187 248.6295441,221.39436855,201.1211367 \\
1.69598889350891 0.874652564525604 248.18406675,216.9345027,194.56450275 \\
-1.81595456600189 1.02437818050385 248.78919705,222.87959565,203.29941555 \\
1.92410385608673 1.85183489322662 244.4915877,134.20217775,94.0718052 \\
1.90114033222198 0.82379138469696 250.0727982,233.26498575,218.47535805 \\
1.95834600925446 0.603826880455017 250.25125995,234.75740895,220.66913325 \\
1.52032876014709 -0.799637317657471 247.299714,206.50190445,179.14325025 \\
1.57054352760315 -2.23575830459595 243.0470178,116.43654195,79.6614288 \\
2.04885244369507 0.873187303543091 249.1250907,225.84872895,207.6473262 \\
-0.533350229263306 1.75955760478973 233.1777375,66.3737001,62.14186545 \\
1.53809416294098 1.11001396179199 247.91115555,213.9578826,190.17620265 \\
1.92566931247711 -0.428397476673126 249.872205,231.7816839,216.3110379 \\
-1.59088671207428 0.933467149734497 247.91115555,213.9578826,190.17620265 \\
1.87242114543915 -0.859415352344513 249.48584175,228.81590385,211.9835043 \\
-1.46662354469299 -1.88068521022797 246.4137216,191.3927133,157.3251468 \\
0.702739179134369 -1.94935750961304 240.4421571,95.75117145,67.24304355 \\
2.18991494178772 1.76167249679565 241.79993775,105.3160812,72.24732165 \\
1.41005456447601 -2.44298434257507 235.17106485,72.13249005,61.7853984 \\
2.33207511901855 -0.316183626651764 247.41505305,207.99484275,181.3567548 \\
-1.56533527374268 1.77306151390076 246.4137216,191.3927133,157.3251468 \\
2.20706748962402 0.25605320930481 248.9531187,224.36478195,205.4760165 \\
-1.6423282623291 1.4201078414917 247.65640545,210.97774605,185.77259745 \\
2.2237594127655 0.601690888404846 247.78194195,212.46815985,187.97598105 \\
-1.84806168079376 -0.632810652256012 249.1250907,225.84872895,207.6473262 \\
-0.907583057880402 -0.166806995868683 110.19637065,31.05893115,86.6140242 \\
-2.07270503044128 -0.312463521957397 249.872205,231.7816839,216.3110379 \\
-1.90227115154266 0.508630454540253 249.48584175,228.81590385,211.9835043 \\
-1.38707911968231 -1.19931924343109 246.4137216,191.3927133,157.3251468 \\
1.88779294490814 0.822785317897797 249.872205,231.7816839,216.3110379 \\
2.05166435241699 -0.999708116054535 248.18406675,216.9345027,194.56450275 \\
0.555129408836365 -1.28472375869751 152.41404825,27.2691237,91.38697395 \\
2.00030398368835 -0.903911590576172 248.9531187,224.36478195,205.4760165 \\
-1.00770270824432 -2.25645685195923 242.2543452,109.0670649,74.5740564 \\
2.13763380050659 2.21309590339661 179.6674767,22.4942895,87.9062673 \\
1.1841596364975 1.53495848178864 246.26699715,186.7493979,151.06898445 \\
2.0815577507019 -0.665013253688812 249.3025911,227.33242605,209.8164174 \\
-0.88921844959259 1.73825061321259 243.98829675,127.1991204,88.11634905 \\
1.53966116905212 0.692704439163208 246.68642115,197.50080735,165.8961201 \\
-1.44227719306946 0.491375803947449 245.7904863,164.4347457,123.960498 \\
1.43741703033447 1.58661365509033 247.65640545,210.97774605,185.77259745 \\
2.50640630722046 0.989166617393494 242.863122,114.60937515,78.3386775 \\
1.63324987888336 -0.318730235099792 247.299714,206.50190445,179.14325025 \\
-2.19844532012939 -1.16198527812958 246.4137216,191.3927133,157.3251468 \\
-1.3237612247467 0.478953838348389 243.55115025,121.8574518,83.79951015 \\
-1.53480291366577 -0.934782266616821 246.7723791,199.01257995,168.07746405 \\
-2.82317399978638 -0.057067334651947 224.2109889,48.96831045,66.9837468 \\
1.70942068099976 0.0708386898040771 247.53321495,209.48702625,183.56726565 \\
-2.44312000274658 -0.175342679023743 246.2259753,185.1901596,149.01788685 \\
2.00676774978638 -0.251561224460602 250.25125995,234.75740895,220.66913325 \\
-1.55896949768066 -1.87833273410797 246.4137216,191.3927133,157.3251468 \\
-2.20844197273254 -1.09098720550537 246.53702175,194.45887155,161.57890635 \\
1.77934992313385 -0.507084727287292 248.9531187,224.36478195,205.4760165 \\
-1.11516404151917 2.06284022331238 244.37428005,132.46298085,92.56140195 \\
2.24621868133545 -0.511526346206665 248.18406675,216.9345027,194.56450275 \\
2.33716988563538 -0.307278871536255 247.41505305,207.99484275,181.3567548 \\
1.72872126102448 -1.29578983783722 248.3271753,218.42187945,196.75380015 \\
-1.67028987407684 1.65595459938049 246.86670105,200.51813565,170.27341695 \\
1.80788540840149 1.99455833435059 243.98829675,127.1991204,88.11634905 \\
1.89454412460327 0.122182995080948 249.48584175,228.81590385,211.9835043 \\
2.00910663604736 0.2786965072155 250.25125995,234.75740895,220.66913325 \\
1.9689975976944 -0.38203352689743 250.0727982,233.26498575,218.47535805 \\
-1.83952307701111 -0.420997679233551 248.9531187,224.36478195,205.4760165 \\
1.68009948730469 0.901038765907288 248.18406675,216.9345027,194.56450275 \\
-2.45719909667969 0.870576739311218 244.81107975,139.37686215,98.692446 \\
-2.24758172035217 0.928506970405579 246.4137216,191.3927133,157.3251468 \\
1.83080160617828 1.12404882907867 249.6765537,230.29867785,214.14730935 \\
-1.88331985473633 1.51538717746735 246.7723791,199.01257995,168.07746405 \\
1.37914264202118 1.19041430950165 246.68642115,197.50080735,165.8961201 \\
-1.7460150718689 1.2368905544281 248.3271753,218.42187945,196.75380015 \\
-2.37205266952515 -1.13915491104126 245.50510815,154.5607581,113.4028299 \\
1.79496610164642 1.44439899921417 248.3271753,218.42187945,196.75380015 \\
-1.60678780078888 -1.2105370759964 248.04512235,215.44664655,192.37235955 \\
-1.74540221691132 -0.898757576942444 248.6295441,221.39436855,201.1211367 \\
2.1931939125061 1.72373640537262 242.46705855,110.92501275,75.79318335 \\
2.24625301361084 0.700403690338135 247.0774866,203.5132356,174.7062783 \\
1.21995317935944 0.921018362045288 245.3822772,151.2267249,110.02044105 \\
-1.8958215713501 -1.66014289855957 246.4728561,192.92848905,159.44460225 \\
-2.25895166397095 1.53015041351318 243.22306725,118.25314155,81.01330365 \\
2.14978814125061 0.0206763744354248 249.872205,231.7816839,216.3110379 \\
-1.87208616733551 0.658703327178955 249.3025911,227.33242605,209.8164174 \\
-2.27440857887268 1.01217246055603 246.08576865,178.898718,140.986746 \\
2.51420068740845 0.722205281257629 244.4915877,134.20217775,94.0718052 \\
};
\addplot [draw=color0, fill=color0, mark=*, only marks]
table{%
x  y
};
\addplot [draw=color1, fill=color1, mark=*, only marks]
table{%
x  y
};
\addplot [draw=color2, fill=color2, mark=*, only marks]
table{%
x  y
};
\addplot [draw=color3, fill=color3, mark=*, only marks]
table{%
x  y
};
\end{axis}

\end{tikzpicture}

%         \caption{$y=5/x$}
%         \label{fig:elbow_example_11000}
%     \end{subfigure}
%     \caption{\the\linewidth}
%     \label{fig:three graphs}
%     \end{adjustwidth}
% \end{figure}

\begin{figure}
    \begin{adjustwidth}{-2cm}{-2cm}
    \centering
     %% Creator: Matplotlib, PGF backend
%%
%% To include the figure in your LaTeX document, write
%%   \input{<filename>.pgf}
%%
%% Make sure the required packages are loaded in your preamble
%%   \usepackage{pgf}
%%
%% Figures using additional raster images can only be included by \input if
%% they are in the same directory as the main LaTeX file. For loading figures
%% from other directories you can use the `import` package
%%   \usepackage{import}
%%
%% and then include the figures with
%%   \import{<path to file>}{<filename>.pgf}
%%
%% Matplotlib used the following preamble
%%
\begingroup%
\makeatletter%
\begin{pgfpicture}%
\pgfpathrectangle{\pgfpointorigin}{\pgfqpoint{7.500000in}{2.500000in}}%
\pgfusepath{use as bounding box, clip}%
\begin{pgfscope}%
\pgfsetbuttcap%
\pgfsetmiterjoin%
\definecolor{currentfill}{rgb}{1.000000,1.000000,1.000000}%
\pgfsetfillcolor{currentfill}%
\pgfsetlinewidth{0.000000pt}%
\definecolor{currentstroke}{rgb}{1.000000,1.000000,1.000000}%
\pgfsetstrokecolor{currentstroke}%
\pgfsetdash{}{0pt}%
\pgfpathmoveto{\pgfqpoint{0.000000in}{0.000000in}}%
\pgfpathlineto{\pgfqpoint{7.500000in}{0.000000in}}%
\pgfpathlineto{\pgfqpoint{7.500000in}{2.500000in}}%
\pgfpathlineto{\pgfqpoint{0.000000in}{2.500000in}}%
\pgfpathclose%
\pgfusepath{fill}%
\end{pgfscope}%
\begin{pgfscope}%
\pgfsetbuttcap%
\pgfsetmiterjoin%
\definecolor{currentfill}{rgb}{1.000000,1.000000,1.000000}%
\pgfsetfillcolor{currentfill}%
\pgfsetlinewidth{0.000000pt}%
\definecolor{currentstroke}{rgb}{0.000000,0.000000,0.000000}%
\pgfsetstrokecolor{currentstroke}%
\pgfsetstrokeopacity{0.000000}%
\pgfsetdash{}{0pt}%
\pgfpathmoveto{\pgfqpoint{0.750000in}{0.500000in}}%
\pgfpathlineto{\pgfqpoint{2.514706in}{0.500000in}}%
\pgfpathlineto{\pgfqpoint{2.514706in}{2.200000in}}%
\pgfpathlineto{\pgfqpoint{0.750000in}{2.200000in}}%
\pgfpathclose%
\pgfusepath{fill}%
\end{pgfscope}%
\begin{pgfscope}%
\pgfsetbuttcap%
\pgfsetroundjoin%
\definecolor{currentfill}{rgb}{0.150000,0.150000,0.150000}%
\pgfsetfillcolor{currentfill}%
\pgfsetlinewidth{1.003750pt}%
\definecolor{currentstroke}{rgb}{0.150000,0.150000,0.150000}%
\pgfsetstrokecolor{currentstroke}%
\pgfsetdash{}{0pt}%
\pgfsys@defobject{currentmarker}{\pgfqpoint{0.000000in}{-0.066667in}}{\pgfqpoint{0.000000in}{0.000000in}}{%
\pgfpathmoveto{\pgfqpoint{0.000000in}{0.000000in}}%
\pgfpathlineto{\pgfqpoint{0.000000in}{-0.066667in}}%
\pgfusepath{stroke,fill}%
}%
\begin{pgfscope}%
\pgfsys@transformshift{0.750000in}{0.500000in}%
\pgfsys@useobject{currentmarker}{}%
\end{pgfscope}%
\end{pgfscope}%
\begin{pgfscope}%
\definecolor{textcolor}{rgb}{0.150000,0.150000,0.150000}%
\pgfsetstrokecolor{textcolor}%
\pgfsetfillcolor{textcolor}%
\pgftext[x=0.750000in,y=0.384722in,,top]{\color{textcolor}\rmfamily\fontsize{8.800000}{10.560000}\selectfont \(\displaystyle {4500}\)}%
\end{pgfscope}%
\begin{pgfscope}%
\pgfsetbuttcap%
\pgfsetroundjoin%
\definecolor{currentfill}{rgb}{0.150000,0.150000,0.150000}%
\pgfsetfillcolor{currentfill}%
\pgfsetlinewidth{1.003750pt}%
\definecolor{currentstroke}{rgb}{0.150000,0.150000,0.150000}%
\pgfsetstrokecolor{currentstroke}%
\pgfsetdash{}{0pt}%
\pgfsys@defobject{currentmarker}{\pgfqpoint{0.000000in}{-0.066667in}}{\pgfqpoint{0.000000in}{0.000000in}}{%
\pgfpathmoveto{\pgfqpoint{0.000000in}{0.000000in}}%
\pgfpathlineto{\pgfqpoint{0.000000in}{-0.066667in}}%
\pgfusepath{stroke,fill}%
}%
\begin{pgfscope}%
\pgfsys@transformshift{1.338235in}{0.500000in}%
\pgfsys@useobject{currentmarker}{}%
\end{pgfscope}%
\end{pgfscope}%
\begin{pgfscope}%
\definecolor{textcolor}{rgb}{0.150000,0.150000,0.150000}%
\pgfsetstrokecolor{textcolor}%
\pgfsetfillcolor{textcolor}%
\pgftext[x=1.338235in,y=0.384722in,,top]{\color{textcolor}\rmfamily\fontsize{8.800000}{10.560000}\selectfont \(\displaystyle {5000}\)}%
\end{pgfscope}%
\begin{pgfscope}%
\pgfsetbuttcap%
\pgfsetroundjoin%
\definecolor{currentfill}{rgb}{0.150000,0.150000,0.150000}%
\pgfsetfillcolor{currentfill}%
\pgfsetlinewidth{1.003750pt}%
\definecolor{currentstroke}{rgb}{0.150000,0.150000,0.150000}%
\pgfsetstrokecolor{currentstroke}%
\pgfsetdash{}{0pt}%
\pgfsys@defobject{currentmarker}{\pgfqpoint{0.000000in}{-0.066667in}}{\pgfqpoint{0.000000in}{0.000000in}}{%
\pgfpathmoveto{\pgfqpoint{0.000000in}{0.000000in}}%
\pgfpathlineto{\pgfqpoint{0.000000in}{-0.066667in}}%
\pgfusepath{stroke,fill}%
}%
\begin{pgfscope}%
\pgfsys@transformshift{1.926471in}{0.500000in}%
\pgfsys@useobject{currentmarker}{}%
\end{pgfscope}%
\end{pgfscope}%
\begin{pgfscope}%
\definecolor{textcolor}{rgb}{0.150000,0.150000,0.150000}%
\pgfsetstrokecolor{textcolor}%
\pgfsetfillcolor{textcolor}%
\pgftext[x=1.926471in,y=0.384722in,,top]{\color{textcolor}\rmfamily\fontsize{8.800000}{10.560000}\selectfont \(\displaystyle {5500}\)}%
\end{pgfscope}%
\begin{pgfscope}%
\pgfsetbuttcap%
\pgfsetroundjoin%
\definecolor{currentfill}{rgb}{0.150000,0.150000,0.150000}%
\pgfsetfillcolor{currentfill}%
\pgfsetlinewidth{1.003750pt}%
\definecolor{currentstroke}{rgb}{0.150000,0.150000,0.150000}%
\pgfsetstrokecolor{currentstroke}%
\pgfsetdash{}{0pt}%
\pgfsys@defobject{currentmarker}{\pgfqpoint{0.000000in}{-0.066667in}}{\pgfqpoint{0.000000in}{0.000000in}}{%
\pgfpathmoveto{\pgfqpoint{0.000000in}{0.000000in}}%
\pgfpathlineto{\pgfqpoint{0.000000in}{-0.066667in}}%
\pgfusepath{stroke,fill}%
}%
\begin{pgfscope}%
\pgfsys@transformshift{2.514706in}{0.500000in}%
\pgfsys@useobject{currentmarker}{}%
\end{pgfscope}%
\end{pgfscope}%
\begin{pgfscope}%
\definecolor{textcolor}{rgb}{0.150000,0.150000,0.150000}%
\pgfsetstrokecolor{textcolor}%
\pgfsetfillcolor{textcolor}%
\pgftext[x=2.514706in,y=0.384722in,,top]{\color{textcolor}\rmfamily\fontsize{8.800000}{10.560000}\selectfont \(\displaystyle {6000}\)}%
\end{pgfscope}%
\begin{pgfscope}%
\definecolor{textcolor}{rgb}{0.150000,0.150000,0.150000}%
\pgfsetstrokecolor{textcolor}%
\pgfsetfillcolor{textcolor}%
\pgftext[x=1.632353in,y=0.218056in,,top]{\color{textcolor}\rmfamily\fontsize{9.600000}{11.520000}\selectfont Iterations}%
\end{pgfscope}%
\begin{pgfscope}%
\pgfsetbuttcap%
\pgfsetroundjoin%
\definecolor{currentfill}{rgb}{0.150000,0.150000,0.150000}%
\pgfsetfillcolor{currentfill}%
\pgfsetlinewidth{1.003750pt}%
\definecolor{currentstroke}{rgb}{0.150000,0.150000,0.150000}%
\pgfsetstrokecolor{currentstroke}%
\pgfsetdash{}{0pt}%
\pgfsys@defobject{currentmarker}{\pgfqpoint{-0.066667in}{0.000000in}}{\pgfqpoint{-0.000000in}{0.000000in}}{%
\pgfpathmoveto{\pgfqpoint{-0.000000in}{0.000000in}}%
\pgfpathlineto{\pgfqpoint{-0.066667in}{0.000000in}}%
\pgfusepath{stroke,fill}%
}%
\begin{pgfscope}%
\pgfsys@transformshift{0.750000in}{0.712500in}%
\pgfsys@useobject{currentmarker}{}%
\end{pgfscope}%
\end{pgfscope}%
\begin{pgfscope}%
\definecolor{textcolor}{rgb}{0.150000,0.150000,0.150000}%
\pgfsetstrokecolor{textcolor}%
\pgfsetfillcolor{textcolor}%
\pgftext[x=0.570487in, y=0.669097in, left, base]{\color{textcolor}\rmfamily\fontsize{8.800000}{10.560000}\selectfont \(\displaystyle {0}\)}%
\end{pgfscope}%
\begin{pgfscope}%
\pgfsetbuttcap%
\pgfsetroundjoin%
\definecolor{currentfill}{rgb}{0.150000,0.150000,0.150000}%
\pgfsetfillcolor{currentfill}%
\pgfsetlinewidth{1.003750pt}%
\definecolor{currentstroke}{rgb}{0.150000,0.150000,0.150000}%
\pgfsetstrokecolor{currentstroke}%
\pgfsetdash{}{0pt}%
\pgfsys@defobject{currentmarker}{\pgfqpoint{-0.066667in}{0.000000in}}{\pgfqpoint{-0.000000in}{0.000000in}}{%
\pgfpathmoveto{\pgfqpoint{-0.000000in}{0.000000in}}%
\pgfpathlineto{\pgfqpoint{-0.066667in}{0.000000in}}%
\pgfusepath{stroke,fill}%
}%
\begin{pgfscope}%
\pgfsys@transformshift{0.750000in}{1.137500in}%
\pgfsys@useobject{currentmarker}{}%
\end{pgfscope}%
\end{pgfscope}%
\begin{pgfscope}%
\definecolor{textcolor}{rgb}{0.150000,0.150000,0.150000}%
\pgfsetstrokecolor{textcolor}%
\pgfsetfillcolor{textcolor}%
\pgftext[x=0.442015in, y=1.094097in, left, base]{\color{textcolor}\rmfamily\fontsize{8.800000}{10.560000}\selectfont \(\displaystyle {100}\)}%
\end{pgfscope}%
\begin{pgfscope}%
\pgfsetbuttcap%
\pgfsetroundjoin%
\definecolor{currentfill}{rgb}{0.150000,0.150000,0.150000}%
\pgfsetfillcolor{currentfill}%
\pgfsetlinewidth{1.003750pt}%
\definecolor{currentstroke}{rgb}{0.150000,0.150000,0.150000}%
\pgfsetstrokecolor{currentstroke}%
\pgfsetdash{}{0pt}%
\pgfsys@defobject{currentmarker}{\pgfqpoint{-0.066667in}{0.000000in}}{\pgfqpoint{-0.000000in}{0.000000in}}{%
\pgfpathmoveto{\pgfqpoint{-0.000000in}{0.000000in}}%
\pgfpathlineto{\pgfqpoint{-0.066667in}{0.000000in}}%
\pgfusepath{stroke,fill}%
}%
\begin{pgfscope}%
\pgfsys@transformshift{0.750000in}{1.562500in}%
\pgfsys@useobject{currentmarker}{}%
\end{pgfscope}%
\end{pgfscope}%
\begin{pgfscope}%
\definecolor{textcolor}{rgb}{0.150000,0.150000,0.150000}%
\pgfsetstrokecolor{textcolor}%
\pgfsetfillcolor{textcolor}%
\pgftext[x=0.442015in, y=1.519097in, left, base]{\color{textcolor}\rmfamily\fontsize{8.800000}{10.560000}\selectfont \(\displaystyle {200}\)}%
\end{pgfscope}%
\begin{pgfscope}%
\pgfsetbuttcap%
\pgfsetroundjoin%
\definecolor{currentfill}{rgb}{0.150000,0.150000,0.150000}%
\pgfsetfillcolor{currentfill}%
\pgfsetlinewidth{1.003750pt}%
\definecolor{currentstroke}{rgb}{0.150000,0.150000,0.150000}%
\pgfsetstrokecolor{currentstroke}%
\pgfsetdash{}{0pt}%
\pgfsys@defobject{currentmarker}{\pgfqpoint{-0.066667in}{0.000000in}}{\pgfqpoint{-0.000000in}{0.000000in}}{%
\pgfpathmoveto{\pgfqpoint{-0.000000in}{0.000000in}}%
\pgfpathlineto{\pgfqpoint{-0.066667in}{0.000000in}}%
\pgfusepath{stroke,fill}%
}%
\begin{pgfscope}%
\pgfsys@transformshift{0.750000in}{1.987500in}%
\pgfsys@useobject{currentmarker}{}%
\end{pgfscope}%
\end{pgfscope}%
\begin{pgfscope}%
\definecolor{textcolor}{rgb}{0.150000,0.150000,0.150000}%
\pgfsetstrokecolor{textcolor}%
\pgfsetfillcolor{textcolor}%
\pgftext[x=0.442015in, y=1.944097in, left, base]{\color{textcolor}\rmfamily\fontsize{8.800000}{10.560000}\selectfont \(\displaystyle {300}\)}%
\end{pgfscope}%
\begin{pgfscope}%
\definecolor{textcolor}{rgb}{0.150000,0.150000,0.150000}%
\pgfsetstrokecolor{textcolor}%
\pgfsetfillcolor{textcolor}%
\pgftext[x=0.386460in,y=1.350000in,,bottom,rotate=90.000000]{\color{textcolor}\rmfamily\fontsize{9.600000}{11.520000}\selectfont ELBO}%
\end{pgfscope}%
\begin{pgfscope}%
\pgfpathrectangle{\pgfqpoint{0.750000in}{0.500000in}}{\pgfqpoint{1.764706in}{1.700000in}}%
\pgfusepath{clip}%
\pgfsetroundcap%
\pgfsetroundjoin%
\pgfsetlinewidth{1.204500pt}%
\definecolor{currentstroke}{rgb}{0.100000,0.100000,0.100000}%
\pgfsetstrokecolor{currentstroke}%
\pgfsetdash{}{0pt}%
\pgfpathmoveto{\pgfqpoint{0.748824in}{1.097109in}}%
\pgfpathlineto{\pgfqpoint{0.750000in}{1.144939in}}%
\pgfpathlineto{\pgfqpoint{0.751176in}{1.001161in}}%
\pgfpathlineto{\pgfqpoint{0.752353in}{1.103711in}}%
\pgfpathlineto{\pgfqpoint{0.753529in}{1.026955in}}%
\pgfpathlineto{\pgfqpoint{0.754706in}{1.207400in}}%
\pgfpathlineto{\pgfqpoint{0.757059in}{1.003409in}}%
\pgfpathlineto{\pgfqpoint{0.759412in}{1.215592in}}%
\pgfpathlineto{\pgfqpoint{0.760588in}{1.265564in}}%
\pgfpathlineto{\pgfqpoint{0.762941in}{1.082606in}}%
\pgfpathlineto{\pgfqpoint{0.765294in}{1.228878in}}%
\pgfpathlineto{\pgfqpoint{0.766471in}{1.130279in}}%
\pgfpathlineto{\pgfqpoint{0.767647in}{1.145044in}}%
\pgfpathlineto{\pgfqpoint{0.770000in}{1.067868in}}%
\pgfpathlineto{\pgfqpoint{0.771176in}{1.078670in}}%
\pgfpathlineto{\pgfqpoint{0.772353in}{1.091795in}}%
\pgfpathlineto{\pgfqpoint{0.773529in}{1.142839in}}%
\pgfpathlineto{\pgfqpoint{0.775882in}{0.955390in}}%
\pgfpathlineto{\pgfqpoint{0.777059in}{1.241173in}}%
\pgfpathlineto{\pgfqpoint{0.778235in}{1.123307in}}%
\pgfpathlineto{\pgfqpoint{0.779412in}{1.166810in}}%
\pgfpathlineto{\pgfqpoint{0.780588in}{1.104688in}}%
\pgfpathlineto{\pgfqpoint{0.781765in}{1.209898in}}%
\pgfpathlineto{\pgfqpoint{0.784118in}{1.081102in}}%
\pgfpathlineto{\pgfqpoint{0.785294in}{1.188419in}}%
\pgfpathlineto{\pgfqpoint{0.786471in}{1.172935in}}%
\pgfpathlineto{\pgfqpoint{0.787647in}{1.098329in}}%
\pgfpathlineto{\pgfqpoint{0.788824in}{1.146069in}}%
\pgfpathlineto{\pgfqpoint{0.790000in}{1.109911in}}%
\pgfpathlineto{\pgfqpoint{0.791176in}{1.032051in}}%
\pgfpathlineto{\pgfqpoint{0.792353in}{1.168567in}}%
\pgfpathlineto{\pgfqpoint{0.794706in}{1.065382in}}%
\pgfpathlineto{\pgfqpoint{0.795882in}{1.132517in}}%
\pgfpathlineto{\pgfqpoint{0.797059in}{0.953867in}}%
\pgfpathlineto{\pgfqpoint{0.798235in}{1.171518in}}%
\pgfpathlineto{\pgfqpoint{0.800588in}{1.097528in}}%
\pgfpathlineto{\pgfqpoint{0.801765in}{1.126203in}}%
\pgfpathlineto{\pgfqpoint{0.802941in}{1.046479in}}%
\pgfpathlineto{\pgfqpoint{0.804118in}{1.130805in}}%
\pgfpathlineto{\pgfqpoint{0.805294in}{1.126923in}}%
\pgfpathlineto{\pgfqpoint{0.806471in}{1.039668in}}%
\pgfpathlineto{\pgfqpoint{0.807647in}{1.073035in}}%
\pgfpathlineto{\pgfqpoint{0.808824in}{1.184421in}}%
\pgfpathlineto{\pgfqpoint{0.810000in}{1.171722in}}%
\pgfpathlineto{\pgfqpoint{0.812353in}{1.206637in}}%
\pgfpathlineto{\pgfqpoint{0.813529in}{1.102343in}}%
\pgfpathlineto{\pgfqpoint{0.814706in}{1.106694in}}%
\pgfpathlineto{\pgfqpoint{0.815882in}{1.082682in}}%
\pgfpathlineto{\pgfqpoint{0.817059in}{1.083058in}}%
\pgfpathlineto{\pgfqpoint{0.818235in}{1.091493in}}%
\pgfpathlineto{\pgfqpoint{0.820588in}{1.043982in}}%
\pgfpathlineto{\pgfqpoint{0.822941in}{1.244660in}}%
\pgfpathlineto{\pgfqpoint{0.825294in}{1.181471in}}%
\pgfpathlineto{\pgfqpoint{0.826471in}{1.136287in}}%
\pgfpathlineto{\pgfqpoint{0.827647in}{1.199171in}}%
\pgfpathlineto{\pgfqpoint{0.828824in}{1.192016in}}%
\pgfpathlineto{\pgfqpoint{0.830000in}{1.065678in}}%
\pgfpathlineto{\pgfqpoint{0.832353in}{1.141169in}}%
\pgfpathlineto{\pgfqpoint{0.834706in}{0.990447in}}%
\pgfpathlineto{\pgfqpoint{0.835882in}{1.190000in}}%
\pgfpathlineto{\pgfqpoint{0.837059in}{1.184863in}}%
\pgfpathlineto{\pgfqpoint{0.839412in}{1.131231in}}%
\pgfpathlineto{\pgfqpoint{0.840588in}{1.177828in}}%
\pgfpathlineto{\pgfqpoint{0.842941in}{1.145205in}}%
\pgfpathlineto{\pgfqpoint{0.844118in}{1.214353in}}%
\pgfpathlineto{\pgfqpoint{0.846471in}{1.035607in}}%
\pgfpathlineto{\pgfqpoint{0.848824in}{1.104091in}}%
\pgfpathlineto{\pgfqpoint{0.850000in}{1.085391in}}%
\pgfpathlineto{\pgfqpoint{0.851176in}{1.112420in}}%
\pgfpathlineto{\pgfqpoint{0.852353in}{0.984521in}}%
\pgfpathlineto{\pgfqpoint{0.855882in}{1.325632in}}%
\pgfpathlineto{\pgfqpoint{0.857059in}{1.263144in}}%
\pgfpathlineto{\pgfqpoint{0.858235in}{0.993668in}}%
\pgfpathlineto{\pgfqpoint{0.859412in}{1.103360in}}%
\pgfpathlineto{\pgfqpoint{0.860588in}{1.069199in}}%
\pgfpathlineto{\pgfqpoint{0.862941in}{1.193320in}}%
\pgfpathlineto{\pgfqpoint{0.864118in}{1.179245in}}%
\pgfpathlineto{\pgfqpoint{0.865294in}{1.080364in}}%
\pgfpathlineto{\pgfqpoint{0.866471in}{1.109747in}}%
\pgfpathlineto{\pgfqpoint{0.868824in}{1.199271in}}%
\pgfpathlineto{\pgfqpoint{0.870000in}{1.000035in}}%
\pgfpathlineto{\pgfqpoint{0.872353in}{1.236414in}}%
\pgfpathlineto{\pgfqpoint{0.873529in}{1.121497in}}%
\pgfpathlineto{\pgfqpoint{0.875882in}{1.211054in}}%
\pgfpathlineto{\pgfqpoint{0.877059in}{1.168267in}}%
\pgfpathlineto{\pgfqpoint{0.879412in}{0.928387in}}%
\pgfpathlineto{\pgfqpoint{0.880588in}{1.166432in}}%
\pgfpathlineto{\pgfqpoint{0.881765in}{1.073251in}}%
\pgfpathlineto{\pgfqpoint{0.882941in}{1.171532in}}%
\pgfpathlineto{\pgfqpoint{0.884118in}{1.129875in}}%
\pgfpathlineto{\pgfqpoint{0.885294in}{1.198430in}}%
\pgfpathlineto{\pgfqpoint{0.887647in}{1.116061in}}%
\pgfpathlineto{\pgfqpoint{0.888824in}{1.147147in}}%
\pgfpathlineto{\pgfqpoint{0.890000in}{1.073085in}}%
\pgfpathlineto{\pgfqpoint{0.892353in}{1.164568in}}%
\pgfpathlineto{\pgfqpoint{0.893529in}{1.033753in}}%
\pgfpathlineto{\pgfqpoint{0.895882in}{1.212969in}}%
\pgfpathlineto{\pgfqpoint{0.898235in}{1.079192in}}%
\pgfpathlineto{\pgfqpoint{0.899412in}{1.107632in}}%
\pgfpathlineto{\pgfqpoint{0.900588in}{1.169132in}}%
\pgfpathlineto{\pgfqpoint{0.904118in}{1.124774in}}%
\pgfpathlineto{\pgfqpoint{0.905294in}{1.092580in}}%
\pgfpathlineto{\pgfqpoint{0.907647in}{1.164223in}}%
\pgfpathlineto{\pgfqpoint{0.908824in}{1.095243in}}%
\pgfpathlineto{\pgfqpoint{0.911176in}{1.153574in}}%
\pgfpathlineto{\pgfqpoint{0.912353in}{1.133798in}}%
\pgfpathlineto{\pgfqpoint{0.913529in}{1.202640in}}%
\pgfpathlineto{\pgfqpoint{0.914706in}{1.129332in}}%
\pgfpathlineto{\pgfqpoint{0.915882in}{1.250604in}}%
\pgfpathlineto{\pgfqpoint{0.917059in}{1.072165in}}%
\pgfpathlineto{\pgfqpoint{0.918235in}{1.085762in}}%
\pgfpathlineto{\pgfqpoint{0.919412in}{1.113550in}}%
\pgfpathlineto{\pgfqpoint{0.921765in}{1.189784in}}%
\pgfpathlineto{\pgfqpoint{0.922941in}{1.064205in}}%
\pgfpathlineto{\pgfqpoint{0.924118in}{1.139093in}}%
\pgfpathlineto{\pgfqpoint{0.925294in}{1.107168in}}%
\pgfpathlineto{\pgfqpoint{0.926471in}{1.211910in}}%
\pgfpathlineto{\pgfqpoint{0.927647in}{1.137395in}}%
\pgfpathlineto{\pgfqpoint{0.928824in}{1.179470in}}%
\pgfpathlineto{\pgfqpoint{0.930000in}{1.038135in}}%
\pgfpathlineto{\pgfqpoint{0.931176in}{1.141557in}}%
\pgfpathlineto{\pgfqpoint{0.932353in}{1.146508in}}%
\pgfpathlineto{\pgfqpoint{0.934706in}{1.008445in}}%
\pgfpathlineto{\pgfqpoint{0.935882in}{1.100029in}}%
\pgfpathlineto{\pgfqpoint{0.937059in}{1.104479in}}%
\pgfpathlineto{\pgfqpoint{0.938235in}{1.144967in}}%
\pgfpathlineto{\pgfqpoint{0.939412in}{0.994819in}}%
\pgfpathlineto{\pgfqpoint{0.941765in}{1.232222in}}%
\pgfpathlineto{\pgfqpoint{0.942941in}{1.161942in}}%
\pgfpathlineto{\pgfqpoint{0.944118in}{1.225868in}}%
\pgfpathlineto{\pgfqpoint{0.945294in}{1.119649in}}%
\pgfpathlineto{\pgfqpoint{0.946471in}{1.161614in}}%
\pgfpathlineto{\pgfqpoint{0.947647in}{1.070872in}}%
\pgfpathlineto{\pgfqpoint{0.948824in}{1.182052in}}%
\pgfpathlineto{\pgfqpoint{0.950000in}{1.088392in}}%
\pgfpathlineto{\pgfqpoint{0.951176in}{1.165902in}}%
\pgfpathlineto{\pgfqpoint{0.952353in}{1.141107in}}%
\pgfpathlineto{\pgfqpoint{0.953529in}{1.204940in}}%
\pgfpathlineto{\pgfqpoint{0.954706in}{1.192235in}}%
\pgfpathlineto{\pgfqpoint{0.955882in}{1.081745in}}%
\pgfpathlineto{\pgfqpoint{0.957059in}{1.079940in}}%
\pgfpathlineto{\pgfqpoint{0.958235in}{1.035036in}}%
\pgfpathlineto{\pgfqpoint{0.959412in}{1.189732in}}%
\pgfpathlineto{\pgfqpoint{0.960588in}{1.127613in}}%
\pgfpathlineto{\pgfqpoint{0.961765in}{1.140430in}}%
\pgfpathlineto{\pgfqpoint{0.962941in}{1.071589in}}%
\pgfpathlineto{\pgfqpoint{0.964118in}{1.166370in}}%
\pgfpathlineto{\pgfqpoint{0.965294in}{1.159927in}}%
\pgfpathlineto{\pgfqpoint{0.966471in}{1.140565in}}%
\pgfpathlineto{\pgfqpoint{0.967647in}{1.140210in}}%
\pgfpathlineto{\pgfqpoint{0.968824in}{1.164019in}}%
\pgfpathlineto{\pgfqpoint{0.970000in}{1.026587in}}%
\pgfpathlineto{\pgfqpoint{0.971176in}{1.233687in}}%
\pgfpathlineto{\pgfqpoint{0.973529in}{1.068819in}}%
\pgfpathlineto{\pgfqpoint{0.975882in}{1.178416in}}%
\pgfpathlineto{\pgfqpoint{0.977059in}{1.130182in}}%
\pgfpathlineto{\pgfqpoint{0.978235in}{1.146808in}}%
\pgfpathlineto{\pgfqpoint{0.979412in}{1.196846in}}%
\pgfpathlineto{\pgfqpoint{0.981765in}{1.091543in}}%
\pgfpathlineto{\pgfqpoint{0.982941in}{1.096233in}}%
\pgfpathlineto{\pgfqpoint{0.984118in}{1.089943in}}%
\pgfpathlineto{\pgfqpoint{0.985294in}{1.302695in}}%
\pgfpathlineto{\pgfqpoint{0.986471in}{1.048393in}}%
\pgfpathlineto{\pgfqpoint{0.987647in}{1.216352in}}%
\pgfpathlineto{\pgfqpoint{0.990000in}{1.116257in}}%
\pgfpathlineto{\pgfqpoint{0.991176in}{1.060774in}}%
\pgfpathlineto{\pgfqpoint{0.992353in}{1.059650in}}%
\pgfpathlineto{\pgfqpoint{0.993529in}{1.074969in}}%
\pgfpathlineto{\pgfqpoint{0.994706in}{1.020941in}}%
\pgfpathlineto{\pgfqpoint{0.997059in}{1.164204in}}%
\pgfpathlineto{\pgfqpoint{0.999412in}{1.028988in}}%
\pgfpathlineto{\pgfqpoint{1.001765in}{1.172464in}}%
\pgfpathlineto{\pgfqpoint{1.002941in}{1.194157in}}%
\pgfpathlineto{\pgfqpoint{1.004118in}{1.136581in}}%
\pgfpathlineto{\pgfqpoint{1.005294in}{1.137469in}}%
\pgfpathlineto{\pgfqpoint{1.006471in}{1.226018in}}%
\pgfpathlineto{\pgfqpoint{1.007647in}{1.041440in}}%
\pgfpathlineto{\pgfqpoint{1.008824in}{1.157784in}}%
\pgfpathlineto{\pgfqpoint{1.011176in}{1.027967in}}%
\pgfpathlineto{\pgfqpoint{1.012353in}{1.093410in}}%
\pgfpathlineto{\pgfqpoint{1.013529in}{1.015093in}}%
\pgfpathlineto{\pgfqpoint{1.015882in}{1.082638in}}%
\pgfpathlineto{\pgfqpoint{1.017059in}{1.033693in}}%
\pgfpathlineto{\pgfqpoint{1.019412in}{1.254941in}}%
\pgfpathlineto{\pgfqpoint{1.020588in}{1.078221in}}%
\pgfpathlineto{\pgfqpoint{1.022941in}{1.147413in}}%
\pgfpathlineto{\pgfqpoint{1.024118in}{1.056977in}}%
\pgfpathlineto{\pgfqpoint{1.025294in}{1.075034in}}%
\pgfpathlineto{\pgfqpoint{1.026471in}{1.220921in}}%
\pgfpathlineto{\pgfqpoint{1.027647in}{1.065335in}}%
\pgfpathlineto{\pgfqpoint{1.028824in}{1.187421in}}%
\pgfpathlineto{\pgfqpoint{1.031176in}{1.091232in}}%
\pgfpathlineto{\pgfqpoint{1.032353in}{1.115003in}}%
\pgfpathlineto{\pgfqpoint{1.033529in}{0.933701in}}%
\pgfpathlineto{\pgfqpoint{1.035882in}{1.153190in}}%
\pgfpathlineto{\pgfqpoint{1.037059in}{1.196959in}}%
\pgfpathlineto{\pgfqpoint{1.038235in}{1.033039in}}%
\pgfpathlineto{\pgfqpoint{1.040588in}{1.176975in}}%
\pgfpathlineto{\pgfqpoint{1.042941in}{1.138578in}}%
\pgfpathlineto{\pgfqpoint{1.044118in}{1.129275in}}%
\pgfpathlineto{\pgfqpoint{1.046471in}{1.094759in}}%
\pgfpathlineto{\pgfqpoint{1.047647in}{1.098547in}}%
\pgfpathlineto{\pgfqpoint{1.048824in}{1.024920in}}%
\pgfpathlineto{\pgfqpoint{1.050000in}{1.073350in}}%
\pgfpathlineto{\pgfqpoint{1.051176in}{0.952580in}}%
\pgfpathlineto{\pgfqpoint{1.053529in}{1.229714in}}%
\pgfpathlineto{\pgfqpoint{1.055882in}{1.063117in}}%
\pgfpathlineto{\pgfqpoint{1.057059in}{1.200735in}}%
\pgfpathlineto{\pgfqpoint{1.058235in}{1.192829in}}%
\pgfpathlineto{\pgfqpoint{1.059412in}{1.061738in}}%
\pgfpathlineto{\pgfqpoint{1.060588in}{1.140967in}}%
\pgfpathlineto{\pgfqpoint{1.061765in}{1.097577in}}%
\pgfpathlineto{\pgfqpoint{1.062941in}{1.191452in}}%
\pgfpathlineto{\pgfqpoint{1.066471in}{1.018448in}}%
\pgfpathlineto{\pgfqpoint{1.067647in}{1.048552in}}%
\pgfpathlineto{\pgfqpoint{1.068824in}{1.231252in}}%
\pgfpathlineto{\pgfqpoint{1.070000in}{1.210422in}}%
\pgfpathlineto{\pgfqpoint{1.073529in}{1.081666in}}%
\pgfpathlineto{\pgfqpoint{1.075882in}{1.143698in}}%
\pgfpathlineto{\pgfqpoint{1.077059in}{1.079469in}}%
\pgfpathlineto{\pgfqpoint{1.078235in}{1.307881in}}%
\pgfpathlineto{\pgfqpoint{1.080588in}{1.196969in}}%
\pgfpathlineto{\pgfqpoint{1.081765in}{1.154728in}}%
\pgfpathlineto{\pgfqpoint{1.082941in}{1.057350in}}%
\pgfpathlineto{\pgfqpoint{1.084118in}{1.105052in}}%
\pgfpathlineto{\pgfqpoint{1.085294in}{1.102479in}}%
\pgfpathlineto{\pgfqpoint{1.086471in}{1.083236in}}%
\pgfpathlineto{\pgfqpoint{1.088824in}{1.106698in}}%
\pgfpathlineto{\pgfqpoint{1.090000in}{1.178788in}}%
\pgfpathlineto{\pgfqpoint{1.091176in}{1.101633in}}%
\pgfpathlineto{\pgfqpoint{1.092353in}{1.157004in}}%
\pgfpathlineto{\pgfqpoint{1.093529in}{1.026355in}}%
\pgfpathlineto{\pgfqpoint{1.094706in}{1.049228in}}%
\pgfpathlineto{\pgfqpoint{1.095882in}{1.037952in}}%
\pgfpathlineto{\pgfqpoint{1.097059in}{1.189889in}}%
\pgfpathlineto{\pgfqpoint{1.098235in}{1.069352in}}%
\pgfpathlineto{\pgfqpoint{1.099412in}{1.081360in}}%
\pgfpathlineto{\pgfqpoint{1.100588in}{1.120484in}}%
\pgfpathlineto{\pgfqpoint{1.101765in}{1.072788in}}%
\pgfpathlineto{\pgfqpoint{1.102941in}{1.227205in}}%
\pgfpathlineto{\pgfqpoint{1.104118in}{1.087830in}}%
\pgfpathlineto{\pgfqpoint{1.105294in}{1.232107in}}%
\pgfpathlineto{\pgfqpoint{1.106471in}{1.051522in}}%
\pgfpathlineto{\pgfqpoint{1.107647in}{1.087101in}}%
\pgfpathlineto{\pgfqpoint{1.108824in}{1.061758in}}%
\pgfpathlineto{\pgfqpoint{1.110000in}{1.177704in}}%
\pgfpathlineto{\pgfqpoint{1.111176in}{1.180598in}}%
\pgfpathlineto{\pgfqpoint{1.112353in}{1.046701in}}%
\pgfpathlineto{\pgfqpoint{1.115882in}{1.155691in}}%
\pgfpathlineto{\pgfqpoint{1.118235in}{1.045938in}}%
\pgfpathlineto{\pgfqpoint{1.119412in}{1.084955in}}%
\pgfpathlineto{\pgfqpoint{1.120588in}{0.997223in}}%
\pgfpathlineto{\pgfqpoint{1.121765in}{1.221178in}}%
\pgfpathlineto{\pgfqpoint{1.122941in}{1.117074in}}%
\pgfpathlineto{\pgfqpoint{1.125294in}{1.231201in}}%
\pgfpathlineto{\pgfqpoint{1.126471in}{1.042051in}}%
\pgfpathlineto{\pgfqpoint{1.128824in}{1.122804in}}%
\pgfpathlineto{\pgfqpoint{1.130000in}{1.041990in}}%
\pgfpathlineto{\pgfqpoint{1.131176in}{1.150041in}}%
\pgfpathlineto{\pgfqpoint{1.132353in}{1.068202in}}%
\pgfpathlineto{\pgfqpoint{1.133529in}{1.076374in}}%
\pgfpathlineto{\pgfqpoint{1.135882in}{1.257440in}}%
\pgfpathlineto{\pgfqpoint{1.137059in}{1.065524in}}%
\pgfpathlineto{\pgfqpoint{1.139412in}{1.211355in}}%
\pgfpathlineto{\pgfqpoint{1.141765in}{1.014694in}}%
\pgfpathlineto{\pgfqpoint{1.142941in}{1.189945in}}%
\pgfpathlineto{\pgfqpoint{1.144118in}{1.163667in}}%
\pgfpathlineto{\pgfqpoint{1.145294in}{1.219779in}}%
\pgfpathlineto{\pgfqpoint{1.146471in}{1.102605in}}%
\pgfpathlineto{\pgfqpoint{1.147647in}{1.103775in}}%
\pgfpathlineto{\pgfqpoint{1.148824in}{1.068967in}}%
\pgfpathlineto{\pgfqpoint{1.150000in}{0.981873in}}%
\pgfpathlineto{\pgfqpoint{1.151176in}{1.178880in}}%
\pgfpathlineto{\pgfqpoint{1.153529in}{0.963653in}}%
\pgfpathlineto{\pgfqpoint{1.155882in}{1.226523in}}%
\pgfpathlineto{\pgfqpoint{1.157059in}{1.089640in}}%
\pgfpathlineto{\pgfqpoint{1.158235in}{1.103041in}}%
\pgfpathlineto{\pgfqpoint{1.159412in}{1.089994in}}%
\pgfpathlineto{\pgfqpoint{1.160588in}{1.162318in}}%
\pgfpathlineto{\pgfqpoint{1.162941in}{1.141119in}}%
\pgfpathlineto{\pgfqpoint{1.164118in}{1.013383in}}%
\pgfpathlineto{\pgfqpoint{1.165294in}{1.042158in}}%
\pgfpathlineto{\pgfqpoint{1.166471in}{1.145771in}}%
\pgfpathlineto{\pgfqpoint{1.167647in}{1.144459in}}%
\pgfpathlineto{\pgfqpoint{1.168824in}{1.107431in}}%
\pgfpathlineto{\pgfqpoint{1.170000in}{1.109230in}}%
\pgfpathlineto{\pgfqpoint{1.171176in}{1.102947in}}%
\pgfpathlineto{\pgfqpoint{1.173529in}{1.173914in}}%
\pgfpathlineto{\pgfqpoint{1.174706in}{1.087218in}}%
\pgfpathlineto{\pgfqpoint{1.175882in}{1.102887in}}%
\pgfpathlineto{\pgfqpoint{1.177059in}{1.216903in}}%
\pgfpathlineto{\pgfqpoint{1.178235in}{1.105343in}}%
\pgfpathlineto{\pgfqpoint{1.179412in}{1.174135in}}%
\pgfpathlineto{\pgfqpoint{1.180588in}{1.086178in}}%
\pgfpathlineto{\pgfqpoint{1.181765in}{1.096289in}}%
\pgfpathlineto{\pgfqpoint{1.182941in}{1.157381in}}%
\pgfpathlineto{\pgfqpoint{1.185294in}{1.108656in}}%
\pgfpathlineto{\pgfqpoint{1.186471in}{1.102364in}}%
\pgfpathlineto{\pgfqpoint{1.188824in}{1.043142in}}%
\pgfpathlineto{\pgfqpoint{1.191176in}{1.230926in}}%
\pgfpathlineto{\pgfqpoint{1.192353in}{1.196872in}}%
\pgfpathlineto{\pgfqpoint{1.193529in}{1.093333in}}%
\pgfpathlineto{\pgfqpoint{1.194706in}{1.105993in}}%
\pgfpathlineto{\pgfqpoint{1.195882in}{1.137718in}}%
\pgfpathlineto{\pgfqpoint{1.197059in}{1.077631in}}%
\pgfpathlineto{\pgfqpoint{1.198235in}{1.128276in}}%
\pgfpathlineto{\pgfqpoint{1.199412in}{1.048147in}}%
\pgfpathlineto{\pgfqpoint{1.201765in}{1.134341in}}%
\pgfpathlineto{\pgfqpoint{1.202941in}{1.116454in}}%
\pgfpathlineto{\pgfqpoint{1.205294in}{1.129958in}}%
\pgfpathlineto{\pgfqpoint{1.206471in}{1.114841in}}%
\pgfpathlineto{\pgfqpoint{1.207647in}{1.080797in}}%
\pgfpathlineto{\pgfqpoint{1.210000in}{1.140933in}}%
\pgfpathlineto{\pgfqpoint{1.211176in}{1.142066in}}%
\pgfpathlineto{\pgfqpoint{1.212353in}{1.154928in}}%
\pgfpathlineto{\pgfqpoint{1.213529in}{1.205655in}}%
\pgfpathlineto{\pgfqpoint{1.214706in}{1.166152in}}%
\pgfpathlineto{\pgfqpoint{1.215882in}{1.256790in}}%
\pgfpathlineto{\pgfqpoint{1.218235in}{1.160862in}}%
\pgfpathlineto{\pgfqpoint{1.219412in}{1.172227in}}%
\pgfpathlineto{\pgfqpoint{1.220588in}{1.163439in}}%
\pgfpathlineto{\pgfqpoint{1.221765in}{1.188094in}}%
\pgfpathlineto{\pgfqpoint{1.224118in}{1.096773in}}%
\pgfpathlineto{\pgfqpoint{1.225294in}{1.046765in}}%
\pgfpathlineto{\pgfqpoint{1.226471in}{1.153825in}}%
\pgfpathlineto{\pgfqpoint{1.227647in}{1.135076in}}%
\pgfpathlineto{\pgfqpoint{1.228824in}{1.071615in}}%
\pgfpathlineto{\pgfqpoint{1.230000in}{1.076146in}}%
\pgfpathlineto{\pgfqpoint{1.231176in}{1.069549in}}%
\pgfpathlineto{\pgfqpoint{1.232353in}{1.154124in}}%
\pgfpathlineto{\pgfqpoint{1.233529in}{1.101401in}}%
\pgfpathlineto{\pgfqpoint{1.234706in}{1.189306in}}%
\pgfpathlineto{\pgfqpoint{1.237059in}{1.116683in}}%
\pgfpathlineto{\pgfqpoint{1.239412in}{1.221176in}}%
\pgfpathlineto{\pgfqpoint{1.241765in}{1.068546in}}%
\pgfpathlineto{\pgfqpoint{1.242941in}{1.153083in}}%
\pgfpathlineto{\pgfqpoint{1.245294in}{1.041822in}}%
\pgfpathlineto{\pgfqpoint{1.247647in}{1.233498in}}%
\pgfpathlineto{\pgfqpoint{1.248824in}{1.088418in}}%
\pgfpathlineto{\pgfqpoint{1.251176in}{1.149444in}}%
\pgfpathlineto{\pgfqpoint{1.252353in}{1.131302in}}%
\pgfpathlineto{\pgfqpoint{1.253529in}{1.170982in}}%
\pgfpathlineto{\pgfqpoint{1.254706in}{1.016582in}}%
\pgfpathlineto{\pgfqpoint{1.255882in}{1.244581in}}%
\pgfpathlineto{\pgfqpoint{1.258235in}{1.172205in}}%
\pgfpathlineto{\pgfqpoint{1.259412in}{1.172482in}}%
\pgfpathlineto{\pgfqpoint{1.261765in}{1.001610in}}%
\pgfpathlineto{\pgfqpoint{1.264118in}{1.166609in}}%
\pgfpathlineto{\pgfqpoint{1.265294in}{1.064856in}}%
\pgfpathlineto{\pgfqpoint{1.266471in}{1.155165in}}%
\pgfpathlineto{\pgfqpoint{1.267647in}{1.102616in}}%
\pgfpathlineto{\pgfqpoint{1.268824in}{1.175831in}}%
\pgfpathlineto{\pgfqpoint{1.270000in}{1.071973in}}%
\pgfpathlineto{\pgfqpoint{1.271176in}{1.201800in}}%
\pgfpathlineto{\pgfqpoint{1.272353in}{1.147688in}}%
\pgfpathlineto{\pgfqpoint{1.274706in}{1.210990in}}%
\pgfpathlineto{\pgfqpoint{1.278235in}{1.108639in}}%
\pgfpathlineto{\pgfqpoint{1.279412in}{1.030469in}}%
\pgfpathlineto{\pgfqpoint{1.280588in}{1.154058in}}%
\pgfpathlineto{\pgfqpoint{1.281765in}{1.093552in}}%
\pgfpathlineto{\pgfqpoint{1.282941in}{1.164624in}}%
\pgfpathlineto{\pgfqpoint{1.284118in}{1.087973in}}%
\pgfpathlineto{\pgfqpoint{1.285294in}{1.097413in}}%
\pgfpathlineto{\pgfqpoint{1.286471in}{1.229717in}}%
\pgfpathlineto{\pgfqpoint{1.287647in}{1.032802in}}%
\pgfpathlineto{\pgfqpoint{1.288824in}{1.113292in}}%
\pgfpathlineto{\pgfqpoint{1.290000in}{1.116140in}}%
\pgfpathlineto{\pgfqpoint{1.291176in}{1.089173in}}%
\pgfpathlineto{\pgfqpoint{1.292353in}{1.103225in}}%
\pgfpathlineto{\pgfqpoint{1.293529in}{1.150022in}}%
\pgfpathlineto{\pgfqpoint{1.294706in}{1.001932in}}%
\pgfpathlineto{\pgfqpoint{1.295882in}{1.107339in}}%
\pgfpathlineto{\pgfqpoint{1.297059in}{1.046035in}}%
\pgfpathlineto{\pgfqpoint{1.299412in}{1.123569in}}%
\pgfpathlineto{\pgfqpoint{1.300588in}{1.201558in}}%
\pgfpathlineto{\pgfqpoint{1.301765in}{1.201240in}}%
\pgfpathlineto{\pgfqpoint{1.304118in}{0.976151in}}%
\pgfpathlineto{\pgfqpoint{1.305294in}{1.156018in}}%
\pgfpathlineto{\pgfqpoint{1.306471in}{1.129284in}}%
\pgfpathlineto{\pgfqpoint{1.307647in}{1.049870in}}%
\pgfpathlineto{\pgfqpoint{1.310000in}{1.161601in}}%
\pgfpathlineto{\pgfqpoint{1.311176in}{1.044582in}}%
\pgfpathlineto{\pgfqpoint{1.312353in}{1.109459in}}%
\pgfpathlineto{\pgfqpoint{1.313529in}{1.090281in}}%
\pgfpathlineto{\pgfqpoint{1.314706in}{1.112573in}}%
\pgfpathlineto{\pgfqpoint{1.315882in}{1.104681in}}%
\pgfpathlineto{\pgfqpoint{1.317059in}{1.044868in}}%
\pgfpathlineto{\pgfqpoint{1.318235in}{1.239160in}}%
\pgfpathlineto{\pgfqpoint{1.321765in}{1.073409in}}%
\pgfpathlineto{\pgfqpoint{1.322941in}{1.104415in}}%
\pgfpathlineto{\pgfqpoint{1.324118in}{1.222833in}}%
\pgfpathlineto{\pgfqpoint{1.325294in}{1.165669in}}%
\pgfpathlineto{\pgfqpoint{1.327647in}{1.217539in}}%
\pgfpathlineto{\pgfqpoint{1.328824in}{1.215778in}}%
\pgfpathlineto{\pgfqpoint{1.330000in}{1.169749in}}%
\pgfpathlineto{\pgfqpoint{1.331176in}{1.172629in}}%
\pgfpathlineto{\pgfqpoint{1.332353in}{1.081217in}}%
\pgfpathlineto{\pgfqpoint{1.333529in}{1.124844in}}%
\pgfpathlineto{\pgfqpoint{1.334706in}{1.126018in}}%
\pgfpathlineto{\pgfqpoint{1.335882in}{1.008276in}}%
\pgfpathlineto{\pgfqpoint{1.337059in}{1.131020in}}%
\pgfpathlineto{\pgfqpoint{1.338235in}{0.860221in}}%
\pgfpathlineto{\pgfqpoint{1.340588in}{1.133996in}}%
\pgfpathlineto{\pgfqpoint{1.341765in}{1.186500in}}%
\pgfpathlineto{\pgfqpoint{1.344118in}{1.134675in}}%
\pgfpathlineto{\pgfqpoint{1.345294in}{1.141494in}}%
\pgfpathlineto{\pgfqpoint{1.346471in}{1.132085in}}%
\pgfpathlineto{\pgfqpoint{1.348824in}{1.224782in}}%
\pgfpathlineto{\pgfqpoint{1.351176in}{1.016302in}}%
\pgfpathlineto{\pgfqpoint{1.352353in}{1.169364in}}%
\pgfpathlineto{\pgfqpoint{1.353529in}{1.018226in}}%
\pgfpathlineto{\pgfqpoint{1.355882in}{1.148009in}}%
\pgfpathlineto{\pgfqpoint{1.357059in}{1.103517in}}%
\pgfpathlineto{\pgfqpoint{1.358235in}{1.128796in}}%
\pgfpathlineto{\pgfqpoint{1.359412in}{1.000308in}}%
\pgfpathlineto{\pgfqpoint{1.360588in}{1.177546in}}%
\pgfpathlineto{\pgfqpoint{1.361765in}{1.182452in}}%
\pgfpathlineto{\pgfqpoint{1.364118in}{1.053107in}}%
\pgfpathlineto{\pgfqpoint{1.365294in}{1.233093in}}%
\pgfpathlineto{\pgfqpoint{1.367647in}{1.055457in}}%
\pgfpathlineto{\pgfqpoint{1.368824in}{1.098670in}}%
\pgfpathlineto{\pgfqpoint{1.370000in}{1.096514in}}%
\pgfpathlineto{\pgfqpoint{1.371176in}{1.173218in}}%
\pgfpathlineto{\pgfqpoint{1.372353in}{1.170220in}}%
\pgfpathlineto{\pgfqpoint{1.374706in}{1.069025in}}%
\pgfpathlineto{\pgfqpoint{1.377059in}{1.189574in}}%
\pgfpathlineto{\pgfqpoint{1.378235in}{1.045489in}}%
\pgfpathlineto{\pgfqpoint{1.379412in}{1.083426in}}%
\pgfpathlineto{\pgfqpoint{1.380588in}{1.034601in}}%
\pgfpathlineto{\pgfqpoint{1.381765in}{1.192786in}}%
\pgfpathlineto{\pgfqpoint{1.382941in}{1.064908in}}%
\pgfpathlineto{\pgfqpoint{1.385294in}{1.086977in}}%
\pgfpathlineto{\pgfqpoint{1.386471in}{1.165723in}}%
\pgfpathlineto{\pgfqpoint{1.387647in}{1.101647in}}%
\pgfpathlineto{\pgfqpoint{1.388824in}{1.210299in}}%
\pgfpathlineto{\pgfqpoint{1.390000in}{1.068985in}}%
\pgfpathlineto{\pgfqpoint{1.391176in}{1.081777in}}%
\pgfpathlineto{\pgfqpoint{1.392353in}{1.040662in}}%
\pgfpathlineto{\pgfqpoint{1.393529in}{1.232567in}}%
\pgfpathlineto{\pgfqpoint{1.394706in}{1.113560in}}%
\pgfpathlineto{\pgfqpoint{1.395882in}{1.170121in}}%
\pgfpathlineto{\pgfqpoint{1.397059in}{1.115982in}}%
\pgfpathlineto{\pgfqpoint{1.398235in}{1.115826in}}%
\pgfpathlineto{\pgfqpoint{1.399412in}{1.103963in}}%
\pgfpathlineto{\pgfqpoint{1.400588in}{1.205219in}}%
\pgfpathlineto{\pgfqpoint{1.401765in}{1.046430in}}%
\pgfpathlineto{\pgfqpoint{1.402941in}{1.077245in}}%
\pgfpathlineto{\pgfqpoint{1.404118in}{1.206824in}}%
\pgfpathlineto{\pgfqpoint{1.405294in}{1.108100in}}%
\pgfpathlineto{\pgfqpoint{1.406471in}{1.130184in}}%
\pgfpathlineto{\pgfqpoint{1.407647in}{1.190612in}}%
\pgfpathlineto{\pgfqpoint{1.410000in}{1.072430in}}%
\pgfpathlineto{\pgfqpoint{1.411176in}{1.216754in}}%
\pgfpathlineto{\pgfqpoint{1.412353in}{1.039910in}}%
\pgfpathlineto{\pgfqpoint{1.413529in}{1.043291in}}%
\pgfpathlineto{\pgfqpoint{1.414706in}{1.144680in}}%
\pgfpathlineto{\pgfqpoint{1.415882in}{1.113323in}}%
\pgfpathlineto{\pgfqpoint{1.418235in}{1.151915in}}%
\pgfpathlineto{\pgfqpoint{1.419412in}{0.977656in}}%
\pgfpathlineto{\pgfqpoint{1.422941in}{1.160403in}}%
\pgfpathlineto{\pgfqpoint{1.424118in}{1.079599in}}%
\pgfpathlineto{\pgfqpoint{1.425294in}{1.141981in}}%
\pgfpathlineto{\pgfqpoint{1.427647in}{1.064398in}}%
\pgfpathlineto{\pgfqpoint{1.428824in}{1.146223in}}%
\pgfpathlineto{\pgfqpoint{1.430000in}{1.002947in}}%
\pgfpathlineto{\pgfqpoint{1.431176in}{1.147974in}}%
\pgfpathlineto{\pgfqpoint{1.432353in}{1.129761in}}%
\pgfpathlineto{\pgfqpoint{1.433529in}{1.054096in}}%
\pgfpathlineto{\pgfqpoint{1.435882in}{1.195245in}}%
\pgfpathlineto{\pgfqpoint{1.437059in}{1.101740in}}%
\pgfpathlineto{\pgfqpoint{1.438235in}{1.188113in}}%
\pgfpathlineto{\pgfqpoint{1.440588in}{1.079307in}}%
\pgfpathlineto{\pgfqpoint{1.441765in}{0.872918in}}%
\pgfpathlineto{\pgfqpoint{1.444118in}{1.159606in}}%
\pgfpathlineto{\pgfqpoint{1.446471in}{1.062883in}}%
\pgfpathlineto{\pgfqpoint{1.447647in}{1.000968in}}%
\pgfpathlineto{\pgfqpoint{1.451176in}{1.110555in}}%
\pgfpathlineto{\pgfqpoint{1.452353in}{1.100173in}}%
\pgfpathlineto{\pgfqpoint{1.453529in}{1.042793in}}%
\pgfpathlineto{\pgfqpoint{1.454706in}{1.190983in}}%
\pgfpathlineto{\pgfqpoint{1.455882in}{1.184369in}}%
\pgfpathlineto{\pgfqpoint{1.457059in}{1.162359in}}%
\pgfpathlineto{\pgfqpoint{1.458235in}{1.075393in}}%
\pgfpathlineto{\pgfqpoint{1.459412in}{1.164383in}}%
\pgfpathlineto{\pgfqpoint{1.460588in}{1.061367in}}%
\pgfpathlineto{\pgfqpoint{1.462941in}{1.184346in}}%
\pgfpathlineto{\pgfqpoint{1.464118in}{1.230448in}}%
\pgfpathlineto{\pgfqpoint{1.466471in}{0.993048in}}%
\pgfpathlineto{\pgfqpoint{1.468824in}{1.185993in}}%
\pgfpathlineto{\pgfqpoint{1.470000in}{1.069716in}}%
\pgfpathlineto{\pgfqpoint{1.471176in}{1.213585in}}%
\pgfpathlineto{\pgfqpoint{1.473529in}{1.043896in}}%
\pgfpathlineto{\pgfqpoint{1.474706in}{1.235781in}}%
\pgfpathlineto{\pgfqpoint{1.478235in}{1.067971in}}%
\pgfpathlineto{\pgfqpoint{1.479412in}{1.112361in}}%
\pgfpathlineto{\pgfqpoint{1.480588in}{1.026553in}}%
\pgfpathlineto{\pgfqpoint{1.481765in}{1.124389in}}%
\pgfpathlineto{\pgfqpoint{1.482941in}{1.089982in}}%
\pgfpathlineto{\pgfqpoint{1.484118in}{1.102535in}}%
\pgfpathlineto{\pgfqpoint{1.485294in}{0.981132in}}%
\pgfpathlineto{\pgfqpoint{1.486471in}{1.099966in}}%
\pgfpathlineto{\pgfqpoint{1.487647in}{1.016540in}}%
\pgfpathlineto{\pgfqpoint{1.488824in}{1.097065in}}%
\pgfpathlineto{\pgfqpoint{1.490000in}{1.021228in}}%
\pgfpathlineto{\pgfqpoint{1.491176in}{1.151276in}}%
\pgfpathlineto{\pgfqpoint{1.494706in}{1.025301in}}%
\pgfpathlineto{\pgfqpoint{1.497059in}{0.936814in}}%
\pgfpathlineto{\pgfqpoint{1.498235in}{0.992238in}}%
\pgfpathlineto{\pgfqpoint{1.499412in}{1.123983in}}%
\pgfpathlineto{\pgfqpoint{1.500588in}{1.053696in}}%
\pgfpathlineto{\pgfqpoint{1.502941in}{1.066061in}}%
\pgfpathlineto{\pgfqpoint{1.504118in}{1.010379in}}%
\pgfpathlineto{\pgfqpoint{1.506471in}{1.169215in}}%
\pgfpathlineto{\pgfqpoint{1.507647in}{0.989175in}}%
\pgfpathlineto{\pgfqpoint{1.508824in}{1.090223in}}%
\pgfpathlineto{\pgfqpoint{1.510000in}{1.006471in}}%
\pgfpathlineto{\pgfqpoint{1.512353in}{1.230519in}}%
\pgfpathlineto{\pgfqpoint{1.513529in}{1.060728in}}%
\pgfpathlineto{\pgfqpoint{1.514706in}{1.086432in}}%
\pgfpathlineto{\pgfqpoint{1.515882in}{1.052542in}}%
\pgfpathlineto{\pgfqpoint{1.517059in}{1.091559in}}%
\pgfpathlineto{\pgfqpoint{1.518235in}{0.999588in}}%
\pgfpathlineto{\pgfqpoint{1.519412in}{1.007204in}}%
\pgfpathlineto{\pgfqpoint{1.522941in}{1.096601in}}%
\pgfpathlineto{\pgfqpoint{1.525294in}{1.035618in}}%
\pgfpathlineto{\pgfqpoint{1.527647in}{1.131900in}}%
\pgfpathlineto{\pgfqpoint{1.530000in}{1.227869in}}%
\pgfpathlineto{\pgfqpoint{1.531176in}{1.051302in}}%
\pgfpathlineto{\pgfqpoint{1.533529in}{1.131841in}}%
\pgfpathlineto{\pgfqpoint{1.534706in}{1.128414in}}%
\pgfpathlineto{\pgfqpoint{1.535882in}{1.118908in}}%
\pgfpathlineto{\pgfqpoint{1.537059in}{1.024440in}}%
\pgfpathlineto{\pgfqpoint{1.539412in}{1.185305in}}%
\pgfpathlineto{\pgfqpoint{1.541765in}{0.899821in}}%
\pgfpathlineto{\pgfqpoint{1.542941in}{1.118841in}}%
\pgfpathlineto{\pgfqpoint{1.544118in}{0.982765in}}%
\pgfpathlineto{\pgfqpoint{1.545294in}{1.139386in}}%
\pgfpathlineto{\pgfqpoint{1.546471in}{1.082240in}}%
\pgfpathlineto{\pgfqpoint{1.547647in}{1.099601in}}%
\pgfpathlineto{\pgfqpoint{1.548824in}{1.140603in}}%
\pgfpathlineto{\pgfqpoint{1.550000in}{0.985498in}}%
\pgfpathlineto{\pgfqpoint{1.551176in}{1.004739in}}%
\pgfpathlineto{\pgfqpoint{1.552353in}{0.873805in}}%
\pgfpathlineto{\pgfqpoint{1.553529in}{1.073491in}}%
\pgfpathlineto{\pgfqpoint{1.554706in}{1.029813in}}%
\pgfpathlineto{\pgfqpoint{1.555882in}{1.066079in}}%
\pgfpathlineto{\pgfqpoint{1.557059in}{1.056109in}}%
\pgfpathlineto{\pgfqpoint{1.558235in}{1.010148in}}%
\pgfpathlineto{\pgfqpoint{1.559412in}{1.055540in}}%
\pgfpathlineto{\pgfqpoint{1.560588in}{1.001726in}}%
\pgfpathlineto{\pgfqpoint{1.562941in}{1.065341in}}%
\pgfpathlineto{\pgfqpoint{1.564118in}{0.866496in}}%
\pgfpathlineto{\pgfqpoint{1.566471in}{1.056195in}}%
\pgfpathlineto{\pgfqpoint{1.568824in}{1.014511in}}%
\pgfpathlineto{\pgfqpoint{1.570000in}{1.003538in}}%
\pgfpathlineto{\pgfqpoint{1.571176in}{0.973128in}}%
\pgfpathlineto{\pgfqpoint{1.572353in}{1.109087in}}%
\pgfpathlineto{\pgfqpoint{1.574706in}{0.972086in}}%
\pgfpathlineto{\pgfqpoint{1.575882in}{1.023515in}}%
\pgfpathlineto{\pgfqpoint{1.577059in}{0.969631in}}%
\pgfpathlineto{\pgfqpoint{1.578235in}{1.083165in}}%
\pgfpathlineto{\pgfqpoint{1.579412in}{0.962965in}}%
\pgfpathlineto{\pgfqpoint{1.580588in}{0.969377in}}%
\pgfpathlineto{\pgfqpoint{1.581765in}{1.149948in}}%
\pgfpathlineto{\pgfqpoint{1.582941in}{0.959704in}}%
\pgfpathlineto{\pgfqpoint{1.585294in}{1.072340in}}%
\pgfpathlineto{\pgfqpoint{1.586471in}{0.948619in}}%
\pgfpathlineto{\pgfqpoint{1.587647in}{1.061717in}}%
\pgfpathlineto{\pgfqpoint{1.588824in}{1.003954in}}%
\pgfpathlineto{\pgfqpoint{1.590000in}{1.066475in}}%
\pgfpathlineto{\pgfqpoint{1.591176in}{1.194766in}}%
\pgfpathlineto{\pgfqpoint{1.592353in}{0.981000in}}%
\pgfpathlineto{\pgfqpoint{1.593529in}{1.024473in}}%
\pgfpathlineto{\pgfqpoint{1.594706in}{1.134519in}}%
\pgfpathlineto{\pgfqpoint{1.595882in}{0.962776in}}%
\pgfpathlineto{\pgfqpoint{1.597059in}{1.028473in}}%
\pgfpathlineto{\pgfqpoint{1.598235in}{0.942749in}}%
\pgfpathlineto{\pgfqpoint{1.599412in}{1.119037in}}%
\pgfpathlineto{\pgfqpoint{1.600588in}{1.049743in}}%
\pgfpathlineto{\pgfqpoint{1.601765in}{1.154458in}}%
\pgfpathlineto{\pgfqpoint{1.602941in}{0.877793in}}%
\pgfpathlineto{\pgfqpoint{1.605294in}{1.079530in}}%
\pgfpathlineto{\pgfqpoint{1.607647in}{1.063107in}}%
\pgfpathlineto{\pgfqpoint{1.608824in}{1.097501in}}%
\pgfpathlineto{\pgfqpoint{1.610000in}{0.882657in}}%
\pgfpathlineto{\pgfqpoint{1.611176in}{1.015628in}}%
\pgfpathlineto{\pgfqpoint{1.612353in}{1.006304in}}%
\pgfpathlineto{\pgfqpoint{1.613529in}{0.821215in}}%
\pgfpathlineto{\pgfqpoint{1.615882in}{1.065308in}}%
\pgfpathlineto{\pgfqpoint{1.617059in}{0.986438in}}%
\pgfpathlineto{\pgfqpoint{1.618235in}{1.010830in}}%
\pgfpathlineto{\pgfqpoint{1.619412in}{0.919877in}}%
\pgfpathlineto{\pgfqpoint{1.620588in}{1.011304in}}%
\pgfpathlineto{\pgfqpoint{1.621765in}{0.941599in}}%
\pgfpathlineto{\pgfqpoint{1.622941in}{1.179519in}}%
\pgfpathlineto{\pgfqpoint{1.625294in}{1.056492in}}%
\pgfpathlineto{\pgfqpoint{1.626471in}{0.984856in}}%
\pgfpathlineto{\pgfqpoint{1.628824in}{1.219252in}}%
\pgfpathlineto{\pgfqpoint{1.630000in}{1.001429in}}%
\pgfpathlineto{\pgfqpoint{1.631176in}{1.050385in}}%
\pgfpathlineto{\pgfqpoint{1.633529in}{0.919262in}}%
\pgfpathlineto{\pgfqpoint{1.635882in}{1.108544in}}%
\pgfpathlineto{\pgfqpoint{1.637059in}{1.006696in}}%
\pgfpathlineto{\pgfqpoint{1.638235in}{1.048461in}}%
\pgfpathlineto{\pgfqpoint{1.639412in}{0.941201in}}%
\pgfpathlineto{\pgfqpoint{1.640588in}{1.193538in}}%
\pgfpathlineto{\pgfqpoint{1.641765in}{1.203455in}}%
\pgfpathlineto{\pgfqpoint{1.642941in}{1.107936in}}%
\pgfpathlineto{\pgfqpoint{1.644118in}{1.221277in}}%
\pgfpathlineto{\pgfqpoint{1.645294in}{1.094041in}}%
\pgfpathlineto{\pgfqpoint{1.647647in}{1.259012in}}%
\pgfpathlineto{\pgfqpoint{1.648824in}{1.356628in}}%
\pgfpathlineto{\pgfqpoint{1.650000in}{1.193264in}}%
\pgfpathlineto{\pgfqpoint{1.652353in}{1.461575in}}%
\pgfpathlineto{\pgfqpoint{1.654706in}{1.246626in}}%
\pgfpathlineto{\pgfqpoint{1.655882in}{1.483351in}}%
\pgfpathlineto{\pgfqpoint{1.657059in}{1.433021in}}%
\pgfpathlineto{\pgfqpoint{1.658235in}{1.660475in}}%
\pgfpathlineto{\pgfqpoint{1.659412in}{1.604418in}}%
\pgfpathlineto{\pgfqpoint{1.661765in}{1.686047in}}%
\pgfpathlineto{\pgfqpoint{1.662941in}{1.668062in}}%
\pgfpathlineto{\pgfqpoint{1.664118in}{1.561859in}}%
\pgfpathlineto{\pgfqpoint{1.665294in}{1.582548in}}%
\pgfpathlineto{\pgfqpoint{1.666471in}{1.676059in}}%
\pgfpathlineto{\pgfqpoint{1.667647in}{1.597329in}}%
\pgfpathlineto{\pgfqpoint{1.668824in}{1.411838in}}%
\pgfpathlineto{\pgfqpoint{1.670000in}{1.662856in}}%
\pgfpathlineto{\pgfqpoint{1.671176in}{1.500219in}}%
\pgfpathlineto{\pgfqpoint{1.672353in}{1.504200in}}%
\pgfpathlineto{\pgfqpoint{1.673529in}{1.550371in}}%
\pgfpathlineto{\pgfqpoint{1.674706in}{1.682421in}}%
\pgfpathlineto{\pgfqpoint{1.675882in}{1.626994in}}%
\pgfpathlineto{\pgfqpoint{1.677059in}{1.807197in}}%
\pgfpathlineto{\pgfqpoint{1.678235in}{1.600900in}}%
\pgfpathlineto{\pgfqpoint{1.679412in}{1.663383in}}%
\pgfpathlineto{\pgfqpoint{1.680588in}{1.846406in}}%
\pgfpathlineto{\pgfqpoint{1.681765in}{1.840865in}}%
\pgfpathlineto{\pgfqpoint{1.682941in}{1.749431in}}%
\pgfpathlineto{\pgfqpoint{1.684118in}{1.807628in}}%
\pgfpathlineto{\pgfqpoint{1.686471in}{1.666243in}}%
\pgfpathlineto{\pgfqpoint{1.687647in}{1.663581in}}%
\pgfpathlineto{\pgfqpoint{1.688824in}{1.813861in}}%
\pgfpathlineto{\pgfqpoint{1.690000in}{1.751649in}}%
\pgfpathlineto{\pgfqpoint{1.691176in}{1.874336in}}%
\pgfpathlineto{\pgfqpoint{1.692353in}{1.670352in}}%
\pgfpathlineto{\pgfqpoint{1.693529in}{1.681553in}}%
\pgfpathlineto{\pgfqpoint{1.694706in}{1.856492in}}%
\pgfpathlineto{\pgfqpoint{1.695882in}{1.796854in}}%
\pgfpathlineto{\pgfqpoint{1.697059in}{1.806097in}}%
\pgfpathlineto{\pgfqpoint{1.698235in}{1.661178in}}%
\pgfpathlineto{\pgfqpoint{1.700588in}{1.809961in}}%
\pgfpathlineto{\pgfqpoint{1.701765in}{1.678987in}}%
\pgfpathlineto{\pgfqpoint{1.702941in}{1.852344in}}%
\pgfpathlineto{\pgfqpoint{1.705294in}{1.764299in}}%
\pgfpathlineto{\pgfqpoint{1.706471in}{1.787071in}}%
\pgfpathlineto{\pgfqpoint{1.708824in}{1.712405in}}%
\pgfpathlineto{\pgfqpoint{1.711176in}{1.797193in}}%
\pgfpathlineto{\pgfqpoint{1.712353in}{1.822714in}}%
\pgfpathlineto{\pgfqpoint{1.713529in}{1.877445in}}%
\pgfpathlineto{\pgfqpoint{1.715882in}{1.558704in}}%
\pgfpathlineto{\pgfqpoint{1.718235in}{1.752316in}}%
\pgfpathlineto{\pgfqpoint{1.719412in}{1.842069in}}%
\pgfpathlineto{\pgfqpoint{1.721765in}{1.692009in}}%
\pgfpathlineto{\pgfqpoint{1.722941in}{1.824507in}}%
\pgfpathlineto{\pgfqpoint{1.724118in}{1.673283in}}%
\pgfpathlineto{\pgfqpoint{1.725294in}{1.853945in}}%
\pgfpathlineto{\pgfqpoint{1.727647in}{1.739807in}}%
\pgfpathlineto{\pgfqpoint{1.730000in}{1.860579in}}%
\pgfpathlineto{\pgfqpoint{1.731176in}{1.888149in}}%
\pgfpathlineto{\pgfqpoint{1.733529in}{1.694052in}}%
\pgfpathlineto{\pgfqpoint{1.734706in}{1.746342in}}%
\pgfpathlineto{\pgfqpoint{1.735882in}{1.873701in}}%
\pgfpathlineto{\pgfqpoint{1.737059in}{1.825399in}}%
\pgfpathlineto{\pgfqpoint{1.738235in}{1.695783in}}%
\pgfpathlineto{\pgfqpoint{1.740588in}{1.884143in}}%
\pgfpathlineto{\pgfqpoint{1.741765in}{1.684911in}}%
\pgfpathlineto{\pgfqpoint{1.742941in}{1.830776in}}%
\pgfpathlineto{\pgfqpoint{1.744118in}{1.794534in}}%
\pgfpathlineto{\pgfqpoint{1.745294in}{1.871455in}}%
\pgfpathlineto{\pgfqpoint{1.746471in}{1.652917in}}%
\pgfpathlineto{\pgfqpoint{1.747647in}{1.862148in}}%
\pgfpathlineto{\pgfqpoint{1.748824in}{1.630997in}}%
\pgfpathlineto{\pgfqpoint{1.751176in}{1.808715in}}%
\pgfpathlineto{\pgfqpoint{1.752353in}{1.947108in}}%
\pgfpathlineto{\pgfqpoint{1.754706in}{1.759384in}}%
\pgfpathlineto{\pgfqpoint{1.755882in}{1.877575in}}%
\pgfpathlineto{\pgfqpoint{1.757059in}{1.810334in}}%
\pgfpathlineto{\pgfqpoint{1.758235in}{1.832372in}}%
\pgfpathlineto{\pgfqpoint{1.759412in}{1.897398in}}%
\pgfpathlineto{\pgfqpoint{1.760588in}{1.887855in}}%
\pgfpathlineto{\pgfqpoint{1.761765in}{1.720150in}}%
\pgfpathlineto{\pgfqpoint{1.762941in}{1.809919in}}%
\pgfpathlineto{\pgfqpoint{1.765294in}{1.736476in}}%
\pgfpathlineto{\pgfqpoint{1.766471in}{1.844995in}}%
\pgfpathlineto{\pgfqpoint{1.767647in}{1.804982in}}%
\pgfpathlineto{\pgfqpoint{1.768824in}{1.897959in}}%
\pgfpathlineto{\pgfqpoint{1.771176in}{1.731549in}}%
\pgfpathlineto{\pgfqpoint{1.772353in}{1.862666in}}%
\pgfpathlineto{\pgfqpoint{1.773529in}{1.854702in}}%
\pgfpathlineto{\pgfqpoint{1.774706in}{1.748819in}}%
\pgfpathlineto{\pgfqpoint{1.777059in}{1.842311in}}%
\pgfpathlineto{\pgfqpoint{1.778235in}{1.891349in}}%
\pgfpathlineto{\pgfqpoint{1.779412in}{1.874795in}}%
\pgfpathlineto{\pgfqpoint{1.782941in}{1.755791in}}%
\pgfpathlineto{\pgfqpoint{1.784118in}{1.877321in}}%
\pgfpathlineto{\pgfqpoint{1.786471in}{1.770836in}}%
\pgfpathlineto{\pgfqpoint{1.787647in}{1.805943in}}%
\pgfpathlineto{\pgfqpoint{1.788824in}{1.899557in}}%
\pgfpathlineto{\pgfqpoint{1.790000in}{1.834870in}}%
\pgfpathlineto{\pgfqpoint{1.792353in}{1.896979in}}%
\pgfpathlineto{\pgfqpoint{1.793529in}{1.845765in}}%
\pgfpathlineto{\pgfqpoint{1.795882in}{1.694642in}}%
\pgfpathlineto{\pgfqpoint{1.798235in}{1.799705in}}%
\pgfpathlineto{\pgfqpoint{1.799412in}{1.767746in}}%
\pgfpathlineto{\pgfqpoint{1.800588in}{1.855979in}}%
\pgfpathlineto{\pgfqpoint{1.801765in}{1.821129in}}%
\pgfpathlineto{\pgfqpoint{1.802941in}{1.747094in}}%
\pgfpathlineto{\pgfqpoint{1.804118in}{1.772232in}}%
\pgfpathlineto{\pgfqpoint{1.806471in}{1.969845in}}%
\pgfpathlineto{\pgfqpoint{1.808824in}{1.725361in}}%
\pgfpathlineto{\pgfqpoint{1.810000in}{1.690347in}}%
\pgfpathlineto{\pgfqpoint{1.811176in}{1.929265in}}%
\pgfpathlineto{\pgfqpoint{1.814706in}{1.845771in}}%
\pgfpathlineto{\pgfqpoint{1.815882in}{1.870790in}}%
\pgfpathlineto{\pgfqpoint{1.817059in}{1.780397in}}%
\pgfpathlineto{\pgfqpoint{1.818235in}{1.934383in}}%
\pgfpathlineto{\pgfqpoint{1.819412in}{1.892745in}}%
\pgfpathlineto{\pgfqpoint{1.820588in}{1.912715in}}%
\pgfpathlineto{\pgfqpoint{1.821765in}{1.781830in}}%
\pgfpathlineto{\pgfqpoint{1.822941in}{1.861072in}}%
\pgfpathlineto{\pgfqpoint{1.824118in}{1.851053in}}%
\pgfpathlineto{\pgfqpoint{1.825294in}{1.763166in}}%
\pgfpathlineto{\pgfqpoint{1.826471in}{1.968990in}}%
\pgfpathlineto{\pgfqpoint{1.828824in}{1.738599in}}%
\pgfpathlineto{\pgfqpoint{1.830000in}{1.884541in}}%
\pgfpathlineto{\pgfqpoint{1.831176in}{1.756638in}}%
\pgfpathlineto{\pgfqpoint{1.832353in}{1.856104in}}%
\pgfpathlineto{\pgfqpoint{1.834706in}{1.696105in}}%
\pgfpathlineto{\pgfqpoint{1.837059in}{1.906238in}}%
\pgfpathlineto{\pgfqpoint{1.838235in}{1.878282in}}%
\pgfpathlineto{\pgfqpoint{1.839412in}{1.904552in}}%
\pgfpathlineto{\pgfqpoint{1.841765in}{1.778057in}}%
\pgfpathlineto{\pgfqpoint{1.842941in}{1.639644in}}%
\pgfpathlineto{\pgfqpoint{1.844118in}{1.995259in}}%
\pgfpathlineto{\pgfqpoint{1.845294in}{1.946436in}}%
\pgfpathlineto{\pgfqpoint{1.846471in}{1.966362in}}%
\pgfpathlineto{\pgfqpoint{1.847647in}{1.802247in}}%
\pgfpathlineto{\pgfqpoint{1.848824in}{1.808822in}}%
\pgfpathlineto{\pgfqpoint{1.850000in}{1.780589in}}%
\pgfpathlineto{\pgfqpoint{1.851176in}{1.864215in}}%
\pgfpathlineto{\pgfqpoint{1.852353in}{1.850150in}}%
\pgfpathlineto{\pgfqpoint{1.853529in}{1.789499in}}%
\pgfpathlineto{\pgfqpoint{1.854706in}{1.840679in}}%
\pgfpathlineto{\pgfqpoint{1.857059in}{1.715529in}}%
\pgfpathlineto{\pgfqpoint{1.858235in}{1.937213in}}%
\pgfpathlineto{\pgfqpoint{1.859412in}{1.739427in}}%
\pgfpathlineto{\pgfqpoint{1.860588in}{1.825125in}}%
\pgfpathlineto{\pgfqpoint{1.861765in}{1.810402in}}%
\pgfpathlineto{\pgfqpoint{1.862941in}{1.879318in}}%
\pgfpathlineto{\pgfqpoint{1.865294in}{1.717895in}}%
\pgfpathlineto{\pgfqpoint{1.866471in}{1.913373in}}%
\pgfpathlineto{\pgfqpoint{1.867647in}{1.739538in}}%
\pgfpathlineto{\pgfqpoint{1.868824in}{1.997551in}}%
\pgfpathlineto{\pgfqpoint{1.871176in}{1.832813in}}%
\pgfpathlineto{\pgfqpoint{1.872353in}{1.868593in}}%
\pgfpathlineto{\pgfqpoint{1.873529in}{1.851208in}}%
\pgfpathlineto{\pgfqpoint{1.874706in}{1.798963in}}%
\pgfpathlineto{\pgfqpoint{1.878235in}{1.999018in}}%
\pgfpathlineto{\pgfqpoint{1.879412in}{1.818943in}}%
\pgfpathlineto{\pgfqpoint{1.880588in}{1.843325in}}%
\pgfpathlineto{\pgfqpoint{1.881765in}{1.803912in}}%
\pgfpathlineto{\pgfqpoint{1.882941in}{1.805796in}}%
\pgfpathlineto{\pgfqpoint{1.884118in}{1.790610in}}%
\pgfpathlineto{\pgfqpoint{1.887647in}{1.949248in}}%
\pgfpathlineto{\pgfqpoint{1.890000in}{1.813522in}}%
\pgfpathlineto{\pgfqpoint{1.891176in}{1.921406in}}%
\pgfpathlineto{\pgfqpoint{1.892353in}{1.914297in}}%
\pgfpathlineto{\pgfqpoint{1.895882in}{1.810619in}}%
\pgfpathlineto{\pgfqpoint{1.897059in}{1.889149in}}%
\pgfpathlineto{\pgfqpoint{1.898235in}{1.848845in}}%
\pgfpathlineto{\pgfqpoint{1.899412in}{1.852172in}}%
\pgfpathlineto{\pgfqpoint{1.900588in}{1.820435in}}%
\pgfpathlineto{\pgfqpoint{1.902941in}{1.835920in}}%
\pgfpathlineto{\pgfqpoint{1.905294in}{1.947201in}}%
\pgfpathlineto{\pgfqpoint{1.906471in}{1.870309in}}%
\pgfpathlineto{\pgfqpoint{1.907647in}{1.927356in}}%
\pgfpathlineto{\pgfqpoint{1.910000in}{1.862581in}}%
\pgfpathlineto{\pgfqpoint{1.911176in}{1.740829in}}%
\pgfpathlineto{\pgfqpoint{1.912353in}{1.740266in}}%
\pgfpathlineto{\pgfqpoint{1.914706in}{1.797685in}}%
\pgfpathlineto{\pgfqpoint{1.915882in}{1.969175in}}%
\pgfpathlineto{\pgfqpoint{1.917059in}{1.846069in}}%
\pgfpathlineto{\pgfqpoint{1.918235in}{1.917190in}}%
\pgfpathlineto{\pgfqpoint{1.919412in}{1.808509in}}%
\pgfpathlineto{\pgfqpoint{1.920588in}{1.955612in}}%
\pgfpathlineto{\pgfqpoint{1.922941in}{1.802440in}}%
\pgfpathlineto{\pgfqpoint{1.924118in}{1.903765in}}%
\pgfpathlineto{\pgfqpoint{1.925294in}{1.880720in}}%
\pgfpathlineto{\pgfqpoint{1.926471in}{1.880438in}}%
\pgfpathlineto{\pgfqpoint{1.927647in}{1.810821in}}%
\pgfpathlineto{\pgfqpoint{1.930000in}{1.950265in}}%
\pgfpathlineto{\pgfqpoint{1.932353in}{1.918442in}}%
\pgfpathlineto{\pgfqpoint{1.933529in}{1.991624in}}%
\pgfpathlineto{\pgfqpoint{1.934706in}{1.830832in}}%
\pgfpathlineto{\pgfqpoint{1.937059in}{1.913055in}}%
\pgfpathlineto{\pgfqpoint{1.938235in}{1.928193in}}%
\pgfpathlineto{\pgfqpoint{1.939412in}{1.799174in}}%
\pgfpathlineto{\pgfqpoint{1.940588in}{1.855563in}}%
\pgfpathlineto{\pgfqpoint{1.941765in}{1.830161in}}%
\pgfpathlineto{\pgfqpoint{1.942941in}{1.912921in}}%
\pgfpathlineto{\pgfqpoint{1.944118in}{1.916064in}}%
\pgfpathlineto{\pgfqpoint{1.945294in}{1.807888in}}%
\pgfpathlineto{\pgfqpoint{1.946471in}{1.823626in}}%
\pgfpathlineto{\pgfqpoint{1.947647in}{1.896896in}}%
\pgfpathlineto{\pgfqpoint{1.948824in}{1.800610in}}%
\pgfpathlineto{\pgfqpoint{1.950000in}{1.838743in}}%
\pgfpathlineto{\pgfqpoint{1.951176in}{1.829151in}}%
\pgfpathlineto{\pgfqpoint{1.952353in}{1.878383in}}%
\pgfpathlineto{\pgfqpoint{1.954706in}{1.772656in}}%
\pgfpathlineto{\pgfqpoint{1.957059in}{1.972193in}}%
\pgfpathlineto{\pgfqpoint{1.958235in}{1.842715in}}%
\pgfpathlineto{\pgfqpoint{1.960588in}{1.959839in}}%
\pgfpathlineto{\pgfqpoint{1.961765in}{1.894147in}}%
\pgfpathlineto{\pgfqpoint{1.962941in}{1.909610in}}%
\pgfpathlineto{\pgfqpoint{1.964118in}{1.901122in}}%
\pgfpathlineto{\pgfqpoint{1.966471in}{1.759813in}}%
\pgfpathlineto{\pgfqpoint{1.967647in}{1.829676in}}%
\pgfpathlineto{\pgfqpoint{1.968824in}{1.822796in}}%
\pgfpathlineto{\pgfqpoint{1.970000in}{1.962138in}}%
\pgfpathlineto{\pgfqpoint{1.971176in}{1.902340in}}%
\pgfpathlineto{\pgfqpoint{1.972353in}{1.953328in}}%
\pgfpathlineto{\pgfqpoint{1.973529in}{1.791670in}}%
\pgfpathlineto{\pgfqpoint{1.975882in}{1.900661in}}%
\pgfpathlineto{\pgfqpoint{1.977059in}{1.820825in}}%
\pgfpathlineto{\pgfqpoint{1.978235in}{1.889691in}}%
\pgfpathlineto{\pgfqpoint{1.979412in}{1.845047in}}%
\pgfpathlineto{\pgfqpoint{1.980588in}{1.910178in}}%
\pgfpathlineto{\pgfqpoint{1.982941in}{1.723609in}}%
\pgfpathlineto{\pgfqpoint{1.984118in}{1.891306in}}%
\pgfpathlineto{\pgfqpoint{1.985294in}{1.804865in}}%
\pgfpathlineto{\pgfqpoint{1.986471in}{1.840922in}}%
\pgfpathlineto{\pgfqpoint{1.988824in}{1.944804in}}%
\pgfpathlineto{\pgfqpoint{1.990000in}{1.960271in}}%
\pgfpathlineto{\pgfqpoint{1.992353in}{1.846817in}}%
\pgfpathlineto{\pgfqpoint{1.994706in}{1.948806in}}%
\pgfpathlineto{\pgfqpoint{1.995882in}{1.795202in}}%
\pgfpathlineto{\pgfqpoint{1.998235in}{1.883270in}}%
\pgfpathlineto{\pgfqpoint{1.999412in}{1.849685in}}%
\pgfpathlineto{\pgfqpoint{2.000588in}{1.914950in}}%
\pgfpathlineto{\pgfqpoint{2.001765in}{1.874106in}}%
\pgfpathlineto{\pgfqpoint{2.002941in}{1.874834in}}%
\pgfpathlineto{\pgfqpoint{2.004118in}{1.719458in}}%
\pgfpathlineto{\pgfqpoint{2.006471in}{1.870938in}}%
\pgfpathlineto{\pgfqpoint{2.007647in}{1.794998in}}%
\pgfpathlineto{\pgfqpoint{2.010000in}{1.867600in}}%
\pgfpathlineto{\pgfqpoint{2.011176in}{1.789744in}}%
\pgfpathlineto{\pgfqpoint{2.012353in}{1.906621in}}%
\pgfpathlineto{\pgfqpoint{2.013529in}{1.840161in}}%
\pgfpathlineto{\pgfqpoint{2.014706in}{1.839711in}}%
\pgfpathlineto{\pgfqpoint{2.015882in}{1.876559in}}%
\pgfpathlineto{\pgfqpoint{2.017059in}{1.844266in}}%
\pgfpathlineto{\pgfqpoint{2.018235in}{1.928652in}}%
\pgfpathlineto{\pgfqpoint{2.019412in}{1.688771in}}%
\pgfpathlineto{\pgfqpoint{2.022941in}{1.871905in}}%
\pgfpathlineto{\pgfqpoint{2.024118in}{1.753063in}}%
\pgfpathlineto{\pgfqpoint{2.025294in}{1.920522in}}%
\pgfpathlineto{\pgfqpoint{2.026471in}{1.925921in}}%
\pgfpathlineto{\pgfqpoint{2.027647in}{1.936932in}}%
\pgfpathlineto{\pgfqpoint{2.028824in}{1.898261in}}%
\pgfpathlineto{\pgfqpoint{2.030000in}{1.818786in}}%
\pgfpathlineto{\pgfqpoint{2.031176in}{1.819403in}}%
\pgfpathlineto{\pgfqpoint{2.032353in}{1.833063in}}%
\pgfpathlineto{\pgfqpoint{2.033529in}{1.755795in}}%
\pgfpathlineto{\pgfqpoint{2.034706in}{1.865931in}}%
\pgfpathlineto{\pgfqpoint{2.035882in}{1.864350in}}%
\pgfpathlineto{\pgfqpoint{2.037059in}{1.859415in}}%
\pgfpathlineto{\pgfqpoint{2.038235in}{1.938430in}}%
\pgfpathlineto{\pgfqpoint{2.040588in}{1.880844in}}%
\pgfpathlineto{\pgfqpoint{2.041765in}{1.990543in}}%
\pgfpathlineto{\pgfqpoint{2.042941in}{1.988741in}}%
\pgfpathlineto{\pgfqpoint{2.046471in}{1.911148in}}%
\pgfpathlineto{\pgfqpoint{2.047647in}{1.852775in}}%
\pgfpathlineto{\pgfqpoint{2.048824in}{1.857421in}}%
\pgfpathlineto{\pgfqpoint{2.050000in}{1.841858in}}%
\pgfpathlineto{\pgfqpoint{2.052353in}{1.949163in}}%
\pgfpathlineto{\pgfqpoint{2.053529in}{1.865530in}}%
\pgfpathlineto{\pgfqpoint{2.054706in}{1.968088in}}%
\pgfpathlineto{\pgfqpoint{2.057059in}{1.813453in}}%
\pgfpathlineto{\pgfqpoint{2.059412in}{1.937607in}}%
\pgfpathlineto{\pgfqpoint{2.060588in}{1.817841in}}%
\pgfpathlineto{\pgfqpoint{2.062941in}{1.947844in}}%
\pgfpathlineto{\pgfqpoint{2.064118in}{1.959444in}}%
\pgfpathlineto{\pgfqpoint{2.065294in}{1.880375in}}%
\pgfpathlineto{\pgfqpoint{2.066471in}{1.948456in}}%
\pgfpathlineto{\pgfqpoint{2.067647in}{1.886645in}}%
\pgfpathlineto{\pgfqpoint{2.068824in}{1.896157in}}%
\pgfpathlineto{\pgfqpoint{2.070000in}{1.808124in}}%
\pgfpathlineto{\pgfqpoint{2.071176in}{1.845306in}}%
\pgfpathlineto{\pgfqpoint{2.072353in}{1.826828in}}%
\pgfpathlineto{\pgfqpoint{2.073529in}{1.921585in}}%
\pgfpathlineto{\pgfqpoint{2.075882in}{1.843712in}}%
\pgfpathlineto{\pgfqpoint{2.077059in}{1.965502in}}%
\pgfpathlineto{\pgfqpoint{2.078235in}{1.855323in}}%
\pgfpathlineto{\pgfqpoint{2.079412in}{1.914810in}}%
\pgfpathlineto{\pgfqpoint{2.081765in}{1.873419in}}%
\pgfpathlineto{\pgfqpoint{2.082941in}{1.936412in}}%
\pgfpathlineto{\pgfqpoint{2.085294in}{1.874018in}}%
\pgfpathlineto{\pgfqpoint{2.086471in}{1.896534in}}%
\pgfpathlineto{\pgfqpoint{2.087647in}{1.880511in}}%
\pgfpathlineto{\pgfqpoint{2.090000in}{1.977310in}}%
\pgfpathlineto{\pgfqpoint{2.091176in}{1.944093in}}%
\pgfpathlineto{\pgfqpoint{2.092353in}{1.791205in}}%
\pgfpathlineto{\pgfqpoint{2.093529in}{1.897302in}}%
\pgfpathlineto{\pgfqpoint{2.094706in}{1.735290in}}%
\pgfpathlineto{\pgfqpoint{2.095882in}{1.797394in}}%
\pgfpathlineto{\pgfqpoint{2.097059in}{1.958102in}}%
\pgfpathlineto{\pgfqpoint{2.098235in}{1.847111in}}%
\pgfpathlineto{\pgfqpoint{2.099412in}{1.842733in}}%
\pgfpathlineto{\pgfqpoint{2.100588in}{1.723686in}}%
\pgfpathlineto{\pgfqpoint{2.102941in}{1.966600in}}%
\pgfpathlineto{\pgfqpoint{2.105294in}{1.785604in}}%
\pgfpathlineto{\pgfqpoint{2.107647in}{1.965507in}}%
\pgfpathlineto{\pgfqpoint{2.110000in}{1.907360in}}%
\pgfpathlineto{\pgfqpoint{2.112353in}{1.929861in}}%
\pgfpathlineto{\pgfqpoint{2.113529in}{1.827742in}}%
\pgfpathlineto{\pgfqpoint{2.114706in}{1.833636in}}%
\pgfpathlineto{\pgfqpoint{2.115882in}{1.949868in}}%
\pgfpathlineto{\pgfqpoint{2.117059in}{1.917317in}}%
\pgfpathlineto{\pgfqpoint{2.118235in}{1.951733in}}%
\pgfpathlineto{\pgfqpoint{2.119412in}{1.846881in}}%
\pgfpathlineto{\pgfqpoint{2.120588in}{1.921552in}}%
\pgfpathlineto{\pgfqpoint{2.121765in}{1.867790in}}%
\pgfpathlineto{\pgfqpoint{2.122941in}{1.873126in}}%
\pgfpathlineto{\pgfqpoint{2.125294in}{1.897987in}}%
\pgfpathlineto{\pgfqpoint{2.127647in}{1.857759in}}%
\pgfpathlineto{\pgfqpoint{2.128824in}{1.708168in}}%
\pgfpathlineto{\pgfqpoint{2.131176in}{1.886575in}}%
\pgfpathlineto{\pgfqpoint{2.132353in}{1.911308in}}%
\pgfpathlineto{\pgfqpoint{2.134706in}{1.746665in}}%
\pgfpathlineto{\pgfqpoint{2.135882in}{1.780670in}}%
\pgfpathlineto{\pgfqpoint{2.137059in}{1.943669in}}%
\pgfpathlineto{\pgfqpoint{2.138235in}{1.817510in}}%
\pgfpathlineto{\pgfqpoint{2.139412in}{1.827805in}}%
\pgfpathlineto{\pgfqpoint{2.141765in}{1.937996in}}%
\pgfpathlineto{\pgfqpoint{2.142941in}{1.826600in}}%
\pgfpathlineto{\pgfqpoint{2.144118in}{1.858806in}}%
\pgfpathlineto{\pgfqpoint{2.145294in}{1.834194in}}%
\pgfpathlineto{\pgfqpoint{2.146471in}{1.773094in}}%
\pgfpathlineto{\pgfqpoint{2.147647in}{1.851979in}}%
\pgfpathlineto{\pgfqpoint{2.148824in}{1.798519in}}%
\pgfpathlineto{\pgfqpoint{2.150000in}{1.833461in}}%
\pgfpathlineto{\pgfqpoint{2.152353in}{1.914812in}}%
\pgfpathlineto{\pgfqpoint{2.153529in}{1.933521in}}%
\pgfpathlineto{\pgfqpoint{2.154706in}{1.783862in}}%
\pgfpathlineto{\pgfqpoint{2.155882in}{1.939853in}}%
\pgfpathlineto{\pgfqpoint{2.157059in}{1.817271in}}%
\pgfpathlineto{\pgfqpoint{2.159412in}{1.927362in}}%
\pgfpathlineto{\pgfqpoint{2.160588in}{1.807048in}}%
\pgfpathlineto{\pgfqpoint{2.161765in}{1.869074in}}%
\pgfpathlineto{\pgfqpoint{2.162941in}{1.677780in}}%
\pgfpathlineto{\pgfqpoint{2.165294in}{1.825867in}}%
\pgfpathlineto{\pgfqpoint{2.167647in}{1.911725in}}%
\pgfpathlineto{\pgfqpoint{2.168824in}{1.980043in}}%
\pgfpathlineto{\pgfqpoint{2.170000in}{1.867709in}}%
\pgfpathlineto{\pgfqpoint{2.171176in}{1.960971in}}%
\pgfpathlineto{\pgfqpoint{2.173529in}{1.823989in}}%
\pgfpathlineto{\pgfqpoint{2.174706in}{1.949063in}}%
\pgfpathlineto{\pgfqpoint{2.175882in}{1.925404in}}%
\pgfpathlineto{\pgfqpoint{2.178235in}{1.843480in}}%
\pgfpathlineto{\pgfqpoint{2.180588in}{1.767701in}}%
\pgfpathlineto{\pgfqpoint{2.182941in}{1.814042in}}%
\pgfpathlineto{\pgfqpoint{2.185294in}{1.834357in}}%
\pgfpathlineto{\pgfqpoint{2.186471in}{1.695515in}}%
\pgfpathlineto{\pgfqpoint{2.187647in}{1.957896in}}%
\pgfpathlineto{\pgfqpoint{2.190000in}{1.873438in}}%
\pgfpathlineto{\pgfqpoint{2.191176in}{1.910564in}}%
\pgfpathlineto{\pgfqpoint{2.193529in}{1.751114in}}%
\pgfpathlineto{\pgfqpoint{2.195882in}{1.891089in}}%
\pgfpathlineto{\pgfqpoint{2.197059in}{1.853766in}}%
\pgfpathlineto{\pgfqpoint{2.198235in}{1.948267in}}%
\pgfpathlineto{\pgfqpoint{2.199412in}{1.810759in}}%
\pgfpathlineto{\pgfqpoint{2.200588in}{1.883666in}}%
\pgfpathlineto{\pgfqpoint{2.204118in}{1.809111in}}%
\pgfpathlineto{\pgfqpoint{2.205294in}{1.868596in}}%
\pgfpathlineto{\pgfqpoint{2.208824in}{1.729380in}}%
\pgfpathlineto{\pgfqpoint{2.210000in}{1.927941in}}%
\pgfpathlineto{\pgfqpoint{2.212353in}{1.888739in}}%
\pgfpathlineto{\pgfqpoint{2.213529in}{1.878540in}}%
\pgfpathlineto{\pgfqpoint{2.214706in}{1.989417in}}%
\pgfpathlineto{\pgfqpoint{2.217059in}{1.778485in}}%
\pgfpathlineto{\pgfqpoint{2.218235in}{1.882248in}}%
\pgfpathlineto{\pgfqpoint{2.219412in}{1.813421in}}%
\pgfpathlineto{\pgfqpoint{2.220588in}{1.927047in}}%
\pgfpathlineto{\pgfqpoint{2.221765in}{1.750036in}}%
\pgfpathlineto{\pgfqpoint{2.224118in}{1.971198in}}%
\pgfpathlineto{\pgfqpoint{2.225294in}{1.824569in}}%
\pgfpathlineto{\pgfqpoint{2.226471in}{1.961349in}}%
\pgfpathlineto{\pgfqpoint{2.227647in}{1.798625in}}%
\pgfpathlineto{\pgfqpoint{2.228824in}{1.989311in}}%
\pgfpathlineto{\pgfqpoint{2.232353in}{1.722684in}}%
\pgfpathlineto{\pgfqpoint{2.235882in}{1.916580in}}%
\pgfpathlineto{\pgfqpoint{2.237059in}{1.884255in}}%
\pgfpathlineto{\pgfqpoint{2.238235in}{1.916463in}}%
\pgfpathlineto{\pgfqpoint{2.239412in}{1.849611in}}%
\pgfpathlineto{\pgfqpoint{2.240588in}{1.846297in}}%
\pgfpathlineto{\pgfqpoint{2.241765in}{1.847661in}}%
\pgfpathlineto{\pgfqpoint{2.242941in}{1.971009in}}%
\pgfpathlineto{\pgfqpoint{2.244118in}{1.919815in}}%
\pgfpathlineto{\pgfqpoint{2.245294in}{1.770764in}}%
\pgfpathlineto{\pgfqpoint{2.246471in}{1.863016in}}%
\pgfpathlineto{\pgfqpoint{2.247647in}{1.809139in}}%
\pgfpathlineto{\pgfqpoint{2.248824in}{1.935225in}}%
\pgfpathlineto{\pgfqpoint{2.251176in}{1.749907in}}%
\pgfpathlineto{\pgfqpoint{2.253529in}{1.872225in}}%
\pgfpathlineto{\pgfqpoint{2.254706in}{1.836869in}}%
\pgfpathlineto{\pgfqpoint{2.255882in}{1.884138in}}%
\pgfpathlineto{\pgfqpoint{2.257059in}{1.854480in}}%
\pgfpathlineto{\pgfqpoint{2.260588in}{1.973866in}}%
\pgfpathlineto{\pgfqpoint{2.261765in}{1.806153in}}%
\pgfpathlineto{\pgfqpoint{2.262941in}{1.841828in}}%
\pgfpathlineto{\pgfqpoint{2.264118in}{2.007218in}}%
\pgfpathlineto{\pgfqpoint{2.265294in}{1.978614in}}%
\pgfpathlineto{\pgfqpoint{2.266471in}{1.884868in}}%
\pgfpathlineto{\pgfqpoint{2.267647in}{1.885950in}}%
\pgfpathlineto{\pgfqpoint{2.270000in}{1.939975in}}%
\pgfpathlineto{\pgfqpoint{2.271176in}{1.800672in}}%
\pgfpathlineto{\pgfqpoint{2.272353in}{1.860025in}}%
\pgfpathlineto{\pgfqpoint{2.273529in}{1.987174in}}%
\pgfpathlineto{\pgfqpoint{2.274706in}{1.829940in}}%
\pgfpathlineto{\pgfqpoint{2.275882in}{1.993238in}}%
\pgfpathlineto{\pgfqpoint{2.277059in}{1.891705in}}%
\pgfpathlineto{\pgfqpoint{2.278235in}{1.896115in}}%
\pgfpathlineto{\pgfqpoint{2.281765in}{1.788279in}}%
\pgfpathlineto{\pgfqpoint{2.282941in}{1.912486in}}%
\pgfpathlineto{\pgfqpoint{2.285294in}{1.862728in}}%
\pgfpathlineto{\pgfqpoint{2.286471in}{1.960900in}}%
\pgfpathlineto{\pgfqpoint{2.288824in}{1.796219in}}%
\pgfpathlineto{\pgfqpoint{2.290000in}{1.952035in}}%
\pgfpathlineto{\pgfqpoint{2.291176in}{1.919726in}}%
\pgfpathlineto{\pgfqpoint{2.292353in}{1.762305in}}%
\pgfpathlineto{\pgfqpoint{2.293529in}{2.021019in}}%
\pgfpathlineto{\pgfqpoint{2.294706in}{1.998030in}}%
\pgfpathlineto{\pgfqpoint{2.297059in}{1.913245in}}%
\pgfpathlineto{\pgfqpoint{2.298235in}{1.880718in}}%
\pgfpathlineto{\pgfqpoint{2.299412in}{2.012750in}}%
\pgfpathlineto{\pgfqpoint{2.300588in}{1.785189in}}%
\pgfpathlineto{\pgfqpoint{2.302941in}{2.015136in}}%
\pgfpathlineto{\pgfqpoint{2.306471in}{1.803836in}}%
\pgfpathlineto{\pgfqpoint{2.307647in}{1.905361in}}%
\pgfpathlineto{\pgfqpoint{2.308824in}{1.851400in}}%
\pgfpathlineto{\pgfqpoint{2.310000in}{1.971576in}}%
\pgfpathlineto{\pgfqpoint{2.311176in}{1.917109in}}%
\pgfpathlineto{\pgfqpoint{2.312353in}{1.702753in}}%
\pgfpathlineto{\pgfqpoint{2.314706in}{1.805569in}}%
\pgfpathlineto{\pgfqpoint{2.318235in}{1.993188in}}%
\pgfpathlineto{\pgfqpoint{2.319412in}{1.851748in}}%
\pgfpathlineto{\pgfqpoint{2.320588in}{1.865446in}}%
\pgfpathlineto{\pgfqpoint{2.322941in}{1.936468in}}%
\pgfpathlineto{\pgfqpoint{2.325294in}{1.816665in}}%
\pgfpathlineto{\pgfqpoint{2.327647in}{1.877610in}}%
\pgfpathlineto{\pgfqpoint{2.328824in}{1.702078in}}%
\pgfpathlineto{\pgfqpoint{2.330000in}{1.870811in}}%
\pgfpathlineto{\pgfqpoint{2.331176in}{1.840256in}}%
\pgfpathlineto{\pgfqpoint{2.332353in}{1.759595in}}%
\pgfpathlineto{\pgfqpoint{2.334706in}{1.886159in}}%
\pgfpathlineto{\pgfqpoint{2.337059in}{1.928441in}}%
\pgfpathlineto{\pgfqpoint{2.338235in}{1.913654in}}%
\pgfpathlineto{\pgfqpoint{2.339412in}{1.780852in}}%
\pgfpathlineto{\pgfqpoint{2.340588in}{1.867712in}}%
\pgfpathlineto{\pgfqpoint{2.341765in}{1.851161in}}%
\pgfpathlineto{\pgfqpoint{2.342941in}{1.907690in}}%
\pgfpathlineto{\pgfqpoint{2.344118in}{1.870843in}}%
\pgfpathlineto{\pgfqpoint{2.345294in}{1.797265in}}%
\pgfpathlineto{\pgfqpoint{2.346471in}{1.911949in}}%
\pgfpathlineto{\pgfqpoint{2.347647in}{1.849619in}}%
\pgfpathlineto{\pgfqpoint{2.348824in}{2.012075in}}%
\pgfpathlineto{\pgfqpoint{2.351176in}{1.749339in}}%
\pgfpathlineto{\pgfqpoint{2.353529in}{2.002223in}}%
\pgfpathlineto{\pgfqpoint{2.354706in}{1.858153in}}%
\pgfpathlineto{\pgfqpoint{2.355882in}{1.906546in}}%
\pgfpathlineto{\pgfqpoint{2.357059in}{1.860526in}}%
\pgfpathlineto{\pgfqpoint{2.358235in}{1.918493in}}%
\pgfpathlineto{\pgfqpoint{2.359412in}{1.884937in}}%
\pgfpathlineto{\pgfqpoint{2.360588in}{1.771295in}}%
\pgfpathlineto{\pgfqpoint{2.361765in}{1.853813in}}%
\pgfpathlineto{\pgfqpoint{2.362941in}{1.751259in}}%
\pgfpathlineto{\pgfqpoint{2.364118in}{1.826758in}}%
\pgfpathlineto{\pgfqpoint{2.365294in}{1.819603in}}%
\pgfpathlineto{\pgfqpoint{2.366471in}{1.968428in}}%
\pgfpathlineto{\pgfqpoint{2.368824in}{1.724852in}}%
\pgfpathlineto{\pgfqpoint{2.371176in}{1.926707in}}%
\pgfpathlineto{\pgfqpoint{2.372353in}{1.864774in}}%
\pgfpathlineto{\pgfqpoint{2.373529in}{1.886247in}}%
\pgfpathlineto{\pgfqpoint{2.374706in}{1.884125in}}%
\pgfpathlineto{\pgfqpoint{2.377059in}{1.827730in}}%
\pgfpathlineto{\pgfqpoint{2.379412in}{1.901470in}}%
\pgfpathlineto{\pgfqpoint{2.380588in}{1.694968in}}%
\pgfpathlineto{\pgfqpoint{2.381765in}{1.917018in}}%
\pgfpathlineto{\pgfqpoint{2.382941in}{1.840566in}}%
\pgfpathlineto{\pgfqpoint{2.384118in}{1.954687in}}%
\pgfpathlineto{\pgfqpoint{2.385294in}{1.919809in}}%
\pgfpathlineto{\pgfqpoint{2.388824in}{1.964139in}}%
\pgfpathlineto{\pgfqpoint{2.391176in}{1.840985in}}%
\pgfpathlineto{\pgfqpoint{2.392353in}{1.940190in}}%
\pgfpathlineto{\pgfqpoint{2.393529in}{1.762648in}}%
\pgfpathlineto{\pgfqpoint{2.397059in}{1.931688in}}%
\pgfpathlineto{\pgfqpoint{2.398235in}{1.784409in}}%
\pgfpathlineto{\pgfqpoint{2.400588in}{1.884447in}}%
\pgfpathlineto{\pgfqpoint{2.401765in}{1.803833in}}%
\pgfpathlineto{\pgfqpoint{2.404118in}{1.877920in}}%
\pgfpathlineto{\pgfqpoint{2.405294in}{1.879532in}}%
\pgfpathlineto{\pgfqpoint{2.406471in}{1.935332in}}%
\pgfpathlineto{\pgfqpoint{2.407647in}{1.886282in}}%
\pgfpathlineto{\pgfqpoint{2.408824in}{1.924076in}}%
\pgfpathlineto{\pgfqpoint{2.410000in}{1.897122in}}%
\pgfpathlineto{\pgfqpoint{2.411176in}{1.947100in}}%
\pgfpathlineto{\pgfqpoint{2.412353in}{1.917505in}}%
\pgfpathlineto{\pgfqpoint{2.413529in}{1.988653in}}%
\pgfpathlineto{\pgfqpoint{2.415882in}{1.973041in}}%
\pgfpathlineto{\pgfqpoint{2.417059in}{1.905502in}}%
\pgfpathlineto{\pgfqpoint{2.418235in}{1.920307in}}%
\pgfpathlineto{\pgfqpoint{2.419412in}{1.710945in}}%
\pgfpathlineto{\pgfqpoint{2.420588in}{1.829443in}}%
\pgfpathlineto{\pgfqpoint{2.421765in}{1.764256in}}%
\pgfpathlineto{\pgfqpoint{2.422941in}{1.910341in}}%
\pgfpathlineto{\pgfqpoint{2.424118in}{1.909523in}}%
\pgfpathlineto{\pgfqpoint{2.426471in}{1.785909in}}%
\pgfpathlineto{\pgfqpoint{2.428824in}{2.021093in}}%
\pgfpathlineto{\pgfqpoint{2.430000in}{1.877069in}}%
\pgfpathlineto{\pgfqpoint{2.431176in}{1.974462in}}%
\pgfpathlineto{\pgfqpoint{2.432353in}{1.769230in}}%
\pgfpathlineto{\pgfqpoint{2.434706in}{1.951821in}}%
\pgfpathlineto{\pgfqpoint{2.435882in}{1.865911in}}%
\pgfpathlineto{\pgfqpoint{2.437059in}{1.903378in}}%
\pgfpathlineto{\pgfqpoint{2.438235in}{1.842027in}}%
\pgfpathlineto{\pgfqpoint{2.439412in}{1.906794in}}%
\pgfpathlineto{\pgfqpoint{2.440588in}{1.791438in}}%
\pgfpathlineto{\pgfqpoint{2.441765in}{1.894258in}}%
\pgfpathlineto{\pgfqpoint{2.442941in}{1.869689in}}%
\pgfpathlineto{\pgfqpoint{2.444118in}{1.969162in}}%
\pgfpathlineto{\pgfqpoint{2.445294in}{1.921800in}}%
\pgfpathlineto{\pgfqpoint{2.446471in}{1.960258in}}%
\pgfpathlineto{\pgfqpoint{2.451176in}{1.757983in}}%
\pgfpathlineto{\pgfqpoint{2.453529in}{1.931248in}}%
\pgfpathlineto{\pgfqpoint{2.455882in}{1.840121in}}%
\pgfpathlineto{\pgfqpoint{2.457059in}{1.880115in}}%
\pgfpathlineto{\pgfqpoint{2.458235in}{1.847355in}}%
\pgfpathlineto{\pgfqpoint{2.459412in}{1.962137in}}%
\pgfpathlineto{\pgfqpoint{2.460588in}{1.813127in}}%
\pgfpathlineto{\pgfqpoint{2.462941in}{1.935559in}}%
\pgfpathlineto{\pgfqpoint{2.464118in}{1.750966in}}%
\pgfpathlineto{\pgfqpoint{2.466471in}{1.907096in}}%
\pgfpathlineto{\pgfqpoint{2.467647in}{1.871056in}}%
\pgfpathlineto{\pgfqpoint{2.468824in}{1.891416in}}%
\pgfpathlineto{\pgfqpoint{2.470000in}{1.944249in}}%
\pgfpathlineto{\pgfqpoint{2.472353in}{1.918194in}}%
\pgfpathlineto{\pgfqpoint{2.473529in}{1.765559in}}%
\pgfpathlineto{\pgfqpoint{2.474706in}{1.814945in}}%
\pgfpathlineto{\pgfqpoint{2.475882in}{1.970550in}}%
\pgfpathlineto{\pgfqpoint{2.477059in}{1.915392in}}%
\pgfpathlineto{\pgfqpoint{2.478235in}{1.919703in}}%
\pgfpathlineto{\pgfqpoint{2.480588in}{1.976126in}}%
\pgfpathlineto{\pgfqpoint{2.481765in}{1.807721in}}%
\pgfpathlineto{\pgfqpoint{2.484118in}{1.911902in}}%
\pgfpathlineto{\pgfqpoint{2.485294in}{1.946275in}}%
\pgfpathlineto{\pgfqpoint{2.486471in}{1.941356in}}%
\pgfpathlineto{\pgfqpoint{2.488824in}{1.819368in}}%
\pgfpathlineto{\pgfqpoint{2.490000in}{1.931100in}}%
\pgfpathlineto{\pgfqpoint{2.492353in}{1.770487in}}%
\pgfpathlineto{\pgfqpoint{2.494706in}{1.897490in}}%
\pgfpathlineto{\pgfqpoint{2.495882in}{1.961971in}}%
\pgfpathlineto{\pgfqpoint{2.498235in}{1.787708in}}%
\pgfpathlineto{\pgfqpoint{2.499412in}{1.864313in}}%
\pgfpathlineto{\pgfqpoint{2.500588in}{1.826289in}}%
\pgfpathlineto{\pgfqpoint{2.501765in}{1.934812in}}%
\pgfpathlineto{\pgfqpoint{2.502941in}{1.787340in}}%
\pgfpathlineto{\pgfqpoint{2.504118in}{1.907377in}}%
\pgfpathlineto{\pgfqpoint{2.505294in}{1.908987in}}%
\pgfpathlineto{\pgfqpoint{2.506471in}{1.790274in}}%
\pgfpathlineto{\pgfqpoint{2.507647in}{1.892946in}}%
\pgfpathlineto{\pgfqpoint{2.508824in}{1.842318in}}%
\pgfpathlineto{\pgfqpoint{2.510000in}{1.946747in}}%
\pgfpathlineto{\pgfqpoint{2.512353in}{1.845962in}}%
\pgfpathlineto{\pgfqpoint{2.513529in}{1.893727in}}%
\pgfpathlineto{\pgfqpoint{2.514706in}{1.820411in}}%
\pgfpathlineto{\pgfqpoint{2.515882in}{1.889943in}}%
\pgfpathlineto{\pgfqpoint{2.515882in}{1.889943in}}%
\pgfusepath{stroke}%
\end{pgfscope}%
\begin{pgfscope}%
\pgfsetrectcap%
\pgfsetmiterjoin%
\pgfsetlinewidth{1.003750pt}%
\definecolor{currentstroke}{rgb}{0.150000,0.150000,0.150000}%
\pgfsetstrokecolor{currentstroke}%
\pgfsetdash{}{0pt}%
\pgfpathmoveto{\pgfqpoint{0.750000in}{0.500000in}}%
\pgfpathlineto{\pgfqpoint{0.750000in}{2.200000in}}%
\pgfusepath{stroke}%
\end{pgfscope}%
\begin{pgfscope}%
\pgfsetrectcap%
\pgfsetmiterjoin%
\pgfsetlinewidth{1.003750pt}%
\definecolor{currentstroke}{rgb}{0.150000,0.150000,0.150000}%
\pgfsetstrokecolor{currentstroke}%
\pgfsetdash{}{0pt}%
\pgfpathmoveto{\pgfqpoint{2.514706in}{0.500000in}}%
\pgfpathlineto{\pgfqpoint{2.514706in}{2.200000in}}%
\pgfusepath{stroke}%
\end{pgfscope}%
\begin{pgfscope}%
\pgfsetrectcap%
\pgfsetmiterjoin%
\pgfsetlinewidth{1.003750pt}%
\definecolor{currentstroke}{rgb}{0.150000,0.150000,0.150000}%
\pgfsetstrokecolor{currentstroke}%
\pgfsetdash{}{0pt}%
\pgfpathmoveto{\pgfqpoint{0.750000in}{0.500000in}}%
\pgfpathlineto{\pgfqpoint{2.514706in}{0.500000in}}%
\pgfusepath{stroke}%
\end{pgfscope}%
\begin{pgfscope}%
\pgfsetrectcap%
\pgfsetmiterjoin%
\pgfsetlinewidth{1.003750pt}%
\definecolor{currentstroke}{rgb}{0.150000,0.150000,0.150000}%
\pgfsetstrokecolor{currentstroke}%
\pgfsetdash{}{0pt}%
\pgfpathmoveto{\pgfqpoint{0.750000in}{2.200000in}}%
\pgfpathlineto{\pgfqpoint{2.514706in}{2.200000in}}%
\pgfusepath{stroke}%
\end{pgfscope}%
\begin{pgfscope}%
\definecolor{textcolor}{rgb}{0.150000,0.150000,0.150000}%
\pgfsetstrokecolor{textcolor}%
\pgfsetfillcolor{textcolor}%
\pgftext[x=1.632353in,y=2.283333in,,base]{\color{textcolor}\rmfamily\fontsize{9.600000}{11.520000}\selectfont Training curve}%
\end{pgfscope}%
\begin{pgfscope}%
\pgfsetbuttcap%
\pgfsetmiterjoin%
\definecolor{currentfill}{rgb}{1.000000,1.000000,1.000000}%
\pgfsetfillcolor{currentfill}%
\pgfsetlinewidth{0.000000pt}%
\definecolor{currentstroke}{rgb}{0.000000,0.000000,0.000000}%
\pgfsetstrokecolor{currentstroke}%
\pgfsetstrokeopacity{0.000000}%
\pgfsetdash{}{0pt}%
\pgfpathmoveto{\pgfqpoint{2.867647in}{0.500000in}}%
\pgfpathlineto{\pgfqpoint{4.632353in}{0.500000in}}%
\pgfpathlineto{\pgfqpoint{4.632353in}{2.200000in}}%
\pgfpathlineto{\pgfqpoint{2.867647in}{2.200000in}}%
\pgfpathclose%
\pgfusepath{fill}%
\end{pgfscope}%
\begin{pgfscope}%
\pgfpathrectangle{\pgfqpoint{2.867647in}{0.500000in}}{\pgfqpoint{1.764706in}{1.700000in}}%
\pgfusepath{clip}%
\pgfsetbuttcap%
\pgfsetroundjoin%
\definecolor{currentfill}{rgb}{0.980678,0.914765,0.856766}%
\pgfsetfillcolor{currentfill}%
\pgfsetlinewidth{0.311001pt}%
\definecolor{currentstroke}{rgb}{1.000000,1.000000,1.000000}%
\pgfsetstrokecolor{currentstroke}%
\pgfsetdash{}{0pt}%
\pgfpathmoveto{\pgfqpoint{4.154766in}{1.477265in}}%
\pgfpathcurveto{\pgfqpoint{4.161899in}{1.477265in}}{\pgfqpoint{4.168740in}{1.480099in}}{\pgfqpoint{4.173784in}{1.485142in}}%
\pgfpathcurveto{\pgfqpoint{4.178828in}{1.490186in}}{\pgfqpoint{4.181661in}{1.497028in}}{\pgfqpoint{4.181661in}{1.504160in}}%
\pgfpathcurveto{\pgfqpoint{4.181661in}{1.511293in}}{\pgfqpoint{4.178828in}{1.518135in}}{\pgfqpoint{4.173784in}{1.523179in}}%
\pgfpathcurveto{\pgfqpoint{4.168740in}{1.528222in}}{\pgfqpoint{4.161899in}{1.531056in}}{\pgfqpoint{4.154766in}{1.531056in}}%
\pgfpathcurveto{\pgfqpoint{4.147633in}{1.531056in}}{\pgfqpoint{4.140791in}{1.528222in}}{\pgfqpoint{4.135748in}{1.523179in}}%
\pgfpathcurveto{\pgfqpoint{4.130704in}{1.518135in}}{\pgfqpoint{4.127870in}{1.511293in}}{\pgfqpoint{4.127870in}{1.504160in}}%
\pgfpathcurveto{\pgfqpoint{4.127870in}{1.497028in}}{\pgfqpoint{4.130704in}{1.490186in}}{\pgfqpoint{4.135748in}{1.485142in}}%
\pgfpathcurveto{\pgfqpoint{4.140791in}{1.480099in}}{\pgfqpoint{4.147633in}{1.477265in}}{\pgfqpoint{4.154766in}{1.477265in}}%
\pgfpathclose%
\pgfusepath{stroke,fill}%
\end{pgfscope}%
\begin{pgfscope}%
\pgfpathrectangle{\pgfqpoint{2.867647in}{0.500000in}}{\pgfqpoint{1.764706in}{1.700000in}}%
\pgfusepath{clip}%
\pgfsetbuttcap%
\pgfsetroundjoin%
\definecolor{currentfill}{rgb}{0.966328,0.750560,0.616961}%
\pgfsetfillcolor{currentfill}%
\pgfsetlinewidth{0.311001pt}%
\definecolor{currentstroke}{rgb}{1.000000,1.000000,1.000000}%
\pgfsetstrokecolor{currentstroke}%
\pgfsetdash{}{0pt}%
\pgfpathmoveto{\pgfqpoint{4.296817in}{1.326114in}}%
\pgfpathcurveto{\pgfqpoint{4.303950in}{1.326114in}}{\pgfqpoint{4.310791in}{1.328948in}}{\pgfqpoint{4.315835in}{1.333992in}}%
\pgfpathcurveto{\pgfqpoint{4.320879in}{1.339035in}}{\pgfqpoint{4.323713in}{1.345877in}}{\pgfqpoint{4.323713in}{1.353010in}}%
\pgfpathcurveto{\pgfqpoint{4.323713in}{1.360143in}}{\pgfqpoint{4.320879in}{1.366984in}}{\pgfqpoint{4.315835in}{1.372028in}}%
\pgfpathcurveto{\pgfqpoint{4.310791in}{1.377072in}}{\pgfqpoint{4.303950in}{1.379905in}}{\pgfqpoint{4.296817in}{1.379905in}}%
\pgfpathcurveto{\pgfqpoint{4.289684in}{1.379905in}}{\pgfqpoint{4.282842in}{1.377072in}}{\pgfqpoint{4.277799in}{1.372028in}}%
\pgfpathcurveto{\pgfqpoint{4.272755in}{1.366984in}}{\pgfqpoint{4.269921in}{1.360143in}}{\pgfqpoint{4.269921in}{1.353010in}}%
\pgfpathcurveto{\pgfqpoint{4.269921in}{1.345877in}}{\pgfqpoint{4.272755in}{1.339035in}}{\pgfqpoint{4.277799in}{1.333992in}}%
\pgfpathcurveto{\pgfqpoint{4.282842in}{1.328948in}}{\pgfqpoint{4.289684in}{1.326114in}}{\pgfqpoint{4.296817in}{1.326114in}}%
\pgfpathclose%
\pgfusepath{stroke,fill}%
\end{pgfscope}%
\begin{pgfscope}%
\pgfpathrectangle{\pgfqpoint{2.867647in}{0.500000in}}{\pgfqpoint{1.764706in}{1.700000in}}%
\pgfusepath{clip}%
\pgfsetbuttcap%
\pgfsetroundjoin%
\definecolor{currentfill}{rgb}{0.960778,0.559972,0.399412}%
\pgfsetfillcolor{currentfill}%
\pgfsetlinewidth{0.311001pt}%
\definecolor{currentstroke}{rgb}{1.000000,1.000000,1.000000}%
\pgfsetstrokecolor{currentstroke}%
\pgfsetdash{}{0pt}%
\pgfpathmoveto{\pgfqpoint{3.934982in}{1.777294in}}%
\pgfpathcurveto{\pgfqpoint{3.942115in}{1.777294in}}{\pgfqpoint{3.948957in}{1.780128in}}{\pgfqpoint{3.954000in}{1.785171in}}%
\pgfpathcurveto{\pgfqpoint{3.959044in}{1.790215in}}{\pgfqpoint{3.961878in}{1.797057in}}{\pgfqpoint{3.961878in}{1.804190in}}%
\pgfpathcurveto{\pgfqpoint{3.961878in}{1.811322in}}{\pgfqpoint{3.959044in}{1.818164in}}{\pgfqpoint{3.954000in}{1.823208in}}%
\pgfpathcurveto{\pgfqpoint{3.948957in}{1.828251in}}{\pgfqpoint{3.942115in}{1.831085in}}{\pgfqpoint{3.934982in}{1.831085in}}%
\pgfpathcurveto{\pgfqpoint{3.927849in}{1.831085in}}{\pgfqpoint{3.921008in}{1.828251in}}{\pgfqpoint{3.915964in}{1.823208in}}%
\pgfpathcurveto{\pgfqpoint{3.910920in}{1.818164in}}{\pgfqpoint{3.908087in}{1.811322in}}{\pgfqpoint{3.908087in}{1.804190in}}%
\pgfpathcurveto{\pgfqpoint{3.908087in}{1.797057in}}{\pgfqpoint{3.910920in}{1.790215in}}{\pgfqpoint{3.915964in}{1.785171in}}%
\pgfpathcurveto{\pgfqpoint{3.921008in}{1.780128in}}{\pgfqpoint{3.927849in}{1.777294in}}{\pgfqpoint{3.934982in}{1.777294in}}%
\pgfpathclose%
\pgfusepath{stroke,fill}%
\end{pgfscope}%
\begin{pgfscope}%
\pgfpathrectangle{\pgfqpoint{2.867647in}{0.500000in}}{\pgfqpoint{1.764706in}{1.700000in}}%
\pgfusepath{clip}%
\pgfsetbuttcap%
\pgfsetroundjoin%
\definecolor{currentfill}{rgb}{0.950017,0.427714,0.292447}%
\pgfsetfillcolor{currentfill}%
\pgfsetlinewidth{0.311001pt}%
\definecolor{currentstroke}{rgb}{1.000000,1.000000,1.000000}%
\pgfsetstrokecolor{currentstroke}%
\pgfsetdash{}{0pt}%
\pgfpathmoveto{\pgfqpoint{4.361975in}{1.348192in}}%
\pgfpathcurveto{\pgfqpoint{4.369108in}{1.348192in}}{\pgfqpoint{4.375949in}{1.351026in}}{\pgfqpoint{4.380993in}{1.356069in}}%
\pgfpathcurveto{\pgfqpoint{4.386037in}{1.361113in}}{\pgfqpoint{4.388871in}{1.367955in}}{\pgfqpoint{4.388871in}{1.375088in}}%
\pgfpathcurveto{\pgfqpoint{4.388871in}{1.382220in}}{\pgfqpoint{4.386037in}{1.389062in}}{\pgfqpoint{4.380993in}{1.394106in}}%
\pgfpathcurveto{\pgfqpoint{4.375949in}{1.399149in}}{\pgfqpoint{4.369108in}{1.401983in}}{\pgfqpoint{4.361975in}{1.401983in}}%
\pgfpathcurveto{\pgfqpoint{4.354842in}{1.401983in}}{\pgfqpoint{4.348000in}{1.399149in}}{\pgfqpoint{4.342957in}{1.394106in}}%
\pgfpathcurveto{\pgfqpoint{4.337913in}{1.389062in}}{\pgfqpoint{4.335079in}{1.382220in}}{\pgfqpoint{4.335079in}{1.375088in}}%
\pgfpathcurveto{\pgfqpoint{4.335079in}{1.367955in}}{\pgfqpoint{4.337913in}{1.361113in}}{\pgfqpoint{4.342957in}{1.356069in}}%
\pgfpathcurveto{\pgfqpoint{4.348000in}{1.351026in}}{\pgfqpoint{4.354842in}{1.348192in}}{\pgfqpoint{4.361975in}{1.348192in}}%
\pgfpathclose%
\pgfusepath{stroke,fill}%
\end{pgfscope}%
\begin{pgfscope}%
\pgfpathrectangle{\pgfqpoint{2.867647in}{0.500000in}}{\pgfqpoint{1.764706in}{1.700000in}}%
\pgfusepath{clip}%
\pgfsetbuttcap%
\pgfsetroundjoin%
\definecolor{currentfill}{rgb}{0.975018,0.868213,0.788710}%
\pgfsetfillcolor{currentfill}%
\pgfsetlinewidth{0.311001pt}%
\definecolor{currentstroke}{rgb}{1.000000,1.000000,1.000000}%
\pgfsetstrokecolor{currentstroke}%
\pgfsetdash{}{0pt}%
\pgfpathmoveto{\pgfqpoint{4.127732in}{1.267059in}}%
\pgfpathcurveto{\pgfqpoint{4.134865in}{1.267059in}}{\pgfqpoint{4.141707in}{1.269893in}}{\pgfqpoint{4.146750in}{1.274936in}}%
\pgfpathcurveto{\pgfqpoint{4.151794in}{1.279980in}}{\pgfqpoint{4.154628in}{1.286822in}}{\pgfqpoint{4.154628in}{1.293955in}}%
\pgfpathcurveto{\pgfqpoint{4.154628in}{1.301087in}}{\pgfqpoint{4.151794in}{1.307929in}}{\pgfqpoint{4.146750in}{1.312973in}}%
\pgfpathcurveto{\pgfqpoint{4.141707in}{1.318016in}}{\pgfqpoint{4.134865in}{1.320850in}}{\pgfqpoint{4.127732in}{1.320850in}}%
\pgfpathcurveto{\pgfqpoint{4.120599in}{1.320850in}}{\pgfqpoint{4.113758in}{1.318016in}}{\pgfqpoint{4.108714in}{1.312973in}}%
\pgfpathcurveto{\pgfqpoint{4.103670in}{1.307929in}}{\pgfqpoint{4.100836in}{1.301087in}}{\pgfqpoint{4.100836in}{1.293955in}}%
\pgfpathcurveto{\pgfqpoint{4.100836in}{1.286822in}}{\pgfqpoint{4.103670in}{1.279980in}}{\pgfqpoint{4.108714in}{1.274936in}}%
\pgfpathcurveto{\pgfqpoint{4.113758in}{1.269893in}}{\pgfqpoint{4.120599in}{1.267059in}}{\pgfqpoint{4.127732in}{1.267059in}}%
\pgfpathclose%
\pgfusepath{stroke,fill}%
\end{pgfscope}%
\begin{pgfscope}%
\pgfpathrectangle{\pgfqpoint{2.867647in}{0.500000in}}{\pgfqpoint{1.764706in}{1.700000in}}%
\pgfusepath{clip}%
\pgfsetbuttcap%
\pgfsetroundjoin%
\definecolor{currentfill}{rgb}{0.961115,0.566634,0.405693}%
\pgfsetfillcolor{currentfill}%
\pgfsetlinewidth{0.311001pt}%
\definecolor{currentstroke}{rgb}{1.000000,1.000000,1.000000}%
\pgfsetstrokecolor{currentstroke}%
\pgfsetdash{}{0pt}%
\pgfpathmoveto{\pgfqpoint{4.304428in}{1.538324in}}%
\pgfpathcurveto{\pgfqpoint{4.311560in}{1.538324in}}{\pgfqpoint{4.318402in}{1.541158in}}{\pgfqpoint{4.323446in}{1.546202in}}%
\pgfpathcurveto{\pgfqpoint{4.328489in}{1.551246in}}{\pgfqpoint{4.331323in}{1.558087in}}{\pgfqpoint{4.331323in}{1.565220in}}%
\pgfpathcurveto{\pgfqpoint{4.331323in}{1.572353in}}{\pgfqpoint{4.328489in}{1.579195in}}{\pgfqpoint{4.323446in}{1.584238in}}%
\pgfpathcurveto{\pgfqpoint{4.318402in}{1.589282in}}{\pgfqpoint{4.311560in}{1.592116in}}{\pgfqpoint{4.304428in}{1.592116in}}%
\pgfpathcurveto{\pgfqpoint{4.297295in}{1.592116in}}{\pgfqpoint{4.290453in}{1.589282in}}{\pgfqpoint{4.285409in}{1.584238in}}%
\pgfpathcurveto{\pgfqpoint{4.280366in}{1.579195in}}{\pgfqpoint{4.277532in}{1.572353in}}{\pgfqpoint{4.277532in}{1.565220in}}%
\pgfpathcurveto{\pgfqpoint{4.277532in}{1.558087in}}{\pgfqpoint{4.280366in}{1.551246in}}{\pgfqpoint{4.285409in}{1.546202in}}%
\pgfpathcurveto{\pgfqpoint{4.290453in}{1.541158in}}{\pgfqpoint{4.297295in}{1.538324in}}{\pgfqpoint{4.304428in}{1.538324in}}%
\pgfpathclose%
\pgfusepath{stroke,fill}%
\end{pgfscope}%
\begin{pgfscope}%
\pgfpathrectangle{\pgfqpoint{2.867647in}{0.500000in}}{\pgfqpoint{1.764706in}{1.700000in}}%
\pgfusepath{clip}%
\pgfsetbuttcap%
\pgfsetroundjoin%
\definecolor{currentfill}{rgb}{0.967735,0.780441,0.659127}%
\pgfsetfillcolor{currentfill}%
\pgfsetlinewidth{0.311001pt}%
\definecolor{currentstroke}{rgb}{1.000000,1.000000,1.000000}%
\pgfsetstrokecolor{currentstroke}%
\pgfsetdash{}{0pt}%
\pgfpathmoveto{\pgfqpoint{4.027077in}{1.011458in}}%
\pgfpathcurveto{\pgfqpoint{4.034210in}{1.011458in}}{\pgfqpoint{4.041052in}{1.014292in}}{\pgfqpoint{4.046095in}{1.019336in}}%
\pgfpathcurveto{\pgfqpoint{4.051139in}{1.024380in}}{\pgfqpoint{4.053973in}{1.031221in}}{\pgfqpoint{4.053973in}{1.038354in}}%
\pgfpathcurveto{\pgfqpoint{4.053973in}{1.045487in}}{\pgfqpoint{4.051139in}{1.052329in}}{\pgfqpoint{4.046095in}{1.057372in}}%
\pgfpathcurveto{\pgfqpoint{4.041052in}{1.062416in}}{\pgfqpoint{4.034210in}{1.065250in}}{\pgfqpoint{4.027077in}{1.065250in}}%
\pgfpathcurveto{\pgfqpoint{4.019944in}{1.065250in}}{\pgfqpoint{4.013103in}{1.062416in}}{\pgfqpoint{4.008059in}{1.057372in}}%
\pgfpathcurveto{\pgfqpoint{4.003015in}{1.052329in}}{\pgfqpoint{4.000181in}{1.045487in}}{\pgfqpoint{4.000181in}{1.038354in}}%
\pgfpathcurveto{\pgfqpoint{4.000181in}{1.031221in}}{\pgfqpoint{4.003015in}{1.024380in}}{\pgfqpoint{4.008059in}{1.019336in}}%
\pgfpathcurveto{\pgfqpoint{4.013103in}{1.014292in}}{\pgfqpoint{4.019944in}{1.011458in}}{\pgfqpoint{4.027077in}{1.011458in}}%
\pgfpathclose%
\pgfusepath{stroke,fill}%
\end{pgfscope}%
\begin{pgfscope}%
\pgfpathrectangle{\pgfqpoint{2.867647in}{0.500000in}}{\pgfqpoint{1.764706in}{1.700000in}}%
\pgfusepath{clip}%
\pgfsetbuttcap%
\pgfsetroundjoin%
\definecolor{currentfill}{rgb}{0.968931,0.798091,0.685123}%
\pgfsetfillcolor{currentfill}%
\pgfsetlinewidth{0.311001pt}%
\definecolor{currentstroke}{rgb}{1.000000,1.000000,1.000000}%
\pgfsetstrokecolor{currentstroke}%
\pgfsetdash{}{0pt}%
\pgfpathmoveto{\pgfqpoint{4.278398in}{1.378357in}}%
\pgfpathcurveto{\pgfqpoint{4.285530in}{1.378357in}}{\pgfqpoint{4.292372in}{1.381191in}}{\pgfqpoint{4.297416in}{1.386234in}}%
\pgfpathcurveto{\pgfqpoint{4.302459in}{1.391278in}}{\pgfqpoint{4.305293in}{1.398120in}}{\pgfqpoint{4.305293in}{1.405252in}}%
\pgfpathcurveto{\pgfqpoint{4.305293in}{1.412385in}}{\pgfqpoint{4.302459in}{1.419227in}}{\pgfqpoint{4.297416in}{1.424271in}}%
\pgfpathcurveto{\pgfqpoint{4.292372in}{1.429314in}}{\pgfqpoint{4.285530in}{1.432148in}}{\pgfqpoint{4.278398in}{1.432148in}}%
\pgfpathcurveto{\pgfqpoint{4.271265in}{1.432148in}}{\pgfqpoint{4.264423in}{1.429314in}}{\pgfqpoint{4.259379in}{1.424271in}}%
\pgfpathcurveto{\pgfqpoint{4.254336in}{1.419227in}}{\pgfqpoint{4.251502in}{1.412385in}}{\pgfqpoint{4.251502in}{1.405252in}}%
\pgfpathcurveto{\pgfqpoint{4.251502in}{1.398120in}}{\pgfqpoint{4.254336in}{1.391278in}}{\pgfqpoint{4.259379in}{1.386234in}}%
\pgfpathcurveto{\pgfqpoint{4.264423in}{1.381191in}}{\pgfqpoint{4.271265in}{1.378357in}}{\pgfqpoint{4.278398in}{1.378357in}}%
\pgfpathclose%
\pgfusepath{stroke,fill}%
\end{pgfscope}%
\begin{pgfscope}%
\pgfpathrectangle{\pgfqpoint{2.867647in}{0.500000in}}{\pgfqpoint{1.764706in}{1.700000in}}%
\pgfusepath{clip}%
\pgfsetbuttcap%
\pgfsetroundjoin%
\definecolor{currentfill}{rgb}{0.980678,0.914765,0.856766}%
\pgfsetfillcolor{currentfill}%
\pgfsetlinewidth{0.311001pt}%
\definecolor{currentstroke}{rgb}{1.000000,1.000000,1.000000}%
\pgfsetstrokecolor{currentstroke}%
\pgfsetdash{}{0pt}%
\pgfpathmoveto{\pgfqpoint{4.145980in}{1.499547in}}%
\pgfpathcurveto{\pgfqpoint{4.153113in}{1.499547in}}{\pgfqpoint{4.159955in}{1.502381in}}{\pgfqpoint{4.164998in}{1.507425in}}%
\pgfpathcurveto{\pgfqpoint{4.170042in}{1.512468in}}{\pgfqpoint{4.172876in}{1.519310in}}{\pgfqpoint{4.172876in}{1.526443in}}%
\pgfpathcurveto{\pgfqpoint{4.172876in}{1.533576in}}{\pgfqpoint{4.170042in}{1.540417in}}{\pgfqpoint{4.164998in}{1.545461in}}%
\pgfpathcurveto{\pgfqpoint{4.159955in}{1.550505in}}{\pgfqpoint{4.153113in}{1.553338in}}{\pgfqpoint{4.145980in}{1.553338in}}%
\pgfpathcurveto{\pgfqpoint{4.138847in}{1.553338in}}{\pgfqpoint{4.132006in}{1.550505in}}{\pgfqpoint{4.126962in}{1.545461in}}%
\pgfpathcurveto{\pgfqpoint{4.121918in}{1.540417in}}{\pgfqpoint{4.119084in}{1.533576in}}{\pgfqpoint{4.119084in}{1.526443in}}%
\pgfpathcurveto{\pgfqpoint{4.119084in}{1.519310in}}{\pgfqpoint{4.121918in}{1.512468in}}{\pgfqpoint{4.126962in}{1.507425in}}%
\pgfpathcurveto{\pgfqpoint{4.132006in}{1.502381in}}{\pgfqpoint{4.138847in}{1.499547in}}{\pgfqpoint{4.145980in}{1.499547in}}%
\pgfpathclose%
\pgfusepath{stroke,fill}%
\end{pgfscope}%
\begin{pgfscope}%
\pgfpathrectangle{\pgfqpoint{2.867647in}{0.500000in}}{\pgfqpoint{1.764706in}{1.700000in}}%
\pgfusepath{clip}%
\pgfsetbuttcap%
\pgfsetroundjoin%
\definecolor{currentfill}{rgb}{0.968105,0.786346,0.667739}%
\pgfsetfillcolor{currentfill}%
\pgfsetlinewidth{0.311001pt}%
\definecolor{currentstroke}{rgb}{1.000000,1.000000,1.000000}%
\pgfsetstrokecolor{currentstroke}%
\pgfsetdash{}{0pt}%
\pgfpathmoveto{\pgfqpoint{4.071988in}{1.720687in}}%
\pgfpathcurveto{\pgfqpoint{4.079121in}{1.720687in}}{\pgfqpoint{4.085963in}{1.723521in}}{\pgfqpoint{4.091007in}{1.728564in}}%
\pgfpathcurveto{\pgfqpoint{4.096050in}{1.733608in}}{\pgfqpoint{4.098884in}{1.740450in}}{\pgfqpoint{4.098884in}{1.747582in}}%
\pgfpathcurveto{\pgfqpoint{4.098884in}{1.754715in}}{\pgfqpoint{4.096050in}{1.761557in}}{\pgfqpoint{4.091007in}{1.766601in}}%
\pgfpathcurveto{\pgfqpoint{4.085963in}{1.771644in}}{\pgfqpoint{4.079121in}{1.774478in}}{\pgfqpoint{4.071988in}{1.774478in}}%
\pgfpathcurveto{\pgfqpoint{4.064856in}{1.774478in}}{\pgfqpoint{4.058014in}{1.771644in}}{\pgfqpoint{4.052970in}{1.766601in}}%
\pgfpathcurveto{\pgfqpoint{4.047927in}{1.761557in}}{\pgfqpoint{4.045093in}{1.754715in}}{\pgfqpoint{4.045093in}{1.747582in}}%
\pgfpathcurveto{\pgfqpoint{4.045093in}{1.740450in}}{\pgfqpoint{4.047927in}{1.733608in}}{\pgfqpoint{4.052970in}{1.728564in}}%
\pgfpathcurveto{\pgfqpoint{4.058014in}{1.723521in}}{\pgfqpoint{4.064856in}{1.720687in}}{\pgfqpoint{4.071988in}{1.720687in}}%
\pgfpathclose%
\pgfusepath{stroke,fill}%
\end{pgfscope}%
\begin{pgfscope}%
\pgfpathrectangle{\pgfqpoint{2.867647in}{0.500000in}}{\pgfqpoint{1.764706in}{1.700000in}}%
\pgfusepath{clip}%
\pgfsetbuttcap%
\pgfsetroundjoin%
\definecolor{currentfill}{rgb}{0.964173,0.657587,0.500469}%
\pgfsetfillcolor{currentfill}%
\pgfsetlinewidth{0.311001pt}%
\definecolor{currentstroke}{rgb}{1.000000,1.000000,1.000000}%
\pgfsetstrokecolor{currentstroke}%
\pgfsetdash{}{0pt}%
\pgfpathmoveto{\pgfqpoint{4.217098in}{1.671904in}}%
\pgfpathcurveto{\pgfqpoint{4.224231in}{1.671904in}}{\pgfqpoint{4.231072in}{1.674738in}}{\pgfqpoint{4.236116in}{1.679781in}}%
\pgfpathcurveto{\pgfqpoint{4.241160in}{1.684825in}}{\pgfqpoint{4.243994in}{1.691667in}}{\pgfqpoint{4.243994in}{1.698799in}}%
\pgfpathcurveto{\pgfqpoint{4.243994in}{1.705932in}}{\pgfqpoint{4.241160in}{1.712774in}}{\pgfqpoint{4.236116in}{1.717818in}}%
\pgfpathcurveto{\pgfqpoint{4.231072in}{1.722861in}}{\pgfqpoint{4.224231in}{1.725695in}}{\pgfqpoint{4.217098in}{1.725695in}}%
\pgfpathcurveto{\pgfqpoint{4.209965in}{1.725695in}}{\pgfqpoint{4.203123in}{1.722861in}}{\pgfqpoint{4.198080in}{1.717818in}}%
\pgfpathcurveto{\pgfqpoint{4.193036in}{1.712774in}}{\pgfqpoint{4.190202in}{1.705932in}}{\pgfqpoint{4.190202in}{1.698799in}}%
\pgfpathcurveto{\pgfqpoint{4.190202in}{1.691667in}}{\pgfqpoint{4.193036in}{1.684825in}}{\pgfqpoint{4.198080in}{1.679781in}}%
\pgfpathcurveto{\pgfqpoint{4.203123in}{1.674738in}}{\pgfqpoint{4.209965in}{1.671904in}}{\pgfqpoint{4.217098in}{1.671904in}}%
\pgfpathclose%
\pgfusepath{stroke,fill}%
\end{pgfscope}%
\begin{pgfscope}%
\pgfpathrectangle{\pgfqpoint{2.867647in}{0.500000in}}{\pgfqpoint{1.764706in}{1.700000in}}%
\pgfusepath{clip}%
\pgfsetbuttcap%
\pgfsetroundjoin%
\definecolor{currentfill}{rgb}{0.971202,0.827364,0.728520}%
\pgfsetfillcolor{currentfill}%
\pgfsetlinewidth{0.311001pt}%
\definecolor{currentstroke}{rgb}{1.000000,1.000000,1.000000}%
\pgfsetstrokecolor{currentstroke}%
\pgfsetdash{}{0pt}%
\pgfpathmoveto{\pgfqpoint{4.129165in}{1.675845in}}%
\pgfpathcurveto{\pgfqpoint{4.136298in}{1.675845in}}{\pgfqpoint{4.143140in}{1.678679in}}{\pgfqpoint{4.148184in}{1.683722in}}%
\pgfpathcurveto{\pgfqpoint{4.153227in}{1.688766in}}{\pgfqpoint{4.156061in}{1.695608in}}{\pgfqpoint{4.156061in}{1.702741in}}%
\pgfpathcurveto{\pgfqpoint{4.156061in}{1.709873in}}{\pgfqpoint{4.153227in}{1.716715in}}{\pgfqpoint{4.148184in}{1.721759in}}%
\pgfpathcurveto{\pgfqpoint{4.143140in}{1.726802in}}{\pgfqpoint{4.136298in}{1.729636in}}{\pgfqpoint{4.129165in}{1.729636in}}%
\pgfpathcurveto{\pgfqpoint{4.122033in}{1.729636in}}{\pgfqpoint{4.115191in}{1.726802in}}{\pgfqpoint{4.110147in}{1.721759in}}%
\pgfpathcurveto{\pgfqpoint{4.105104in}{1.716715in}}{\pgfqpoint{4.102270in}{1.709873in}}{\pgfqpoint{4.102270in}{1.702741in}}%
\pgfpathcurveto{\pgfqpoint{4.102270in}{1.695608in}}{\pgfqpoint{4.105104in}{1.688766in}}{\pgfqpoint{4.110147in}{1.683722in}}%
\pgfpathcurveto{\pgfqpoint{4.115191in}{1.678679in}}{\pgfqpoint{4.122033in}{1.675845in}}{\pgfqpoint{4.129165in}{1.675845in}}%
\pgfpathclose%
\pgfusepath{stroke,fill}%
\end{pgfscope}%
\begin{pgfscope}%
\pgfpathrectangle{\pgfqpoint{2.867647in}{0.500000in}}{\pgfqpoint{1.764706in}{1.700000in}}%
\pgfusepath{clip}%
\pgfsetbuttcap%
\pgfsetroundjoin%
\definecolor{currentfill}{rgb}{0.972726,0.844889,0.754401}%
\pgfsetfillcolor{currentfill}%
\pgfsetlinewidth{0.311001pt}%
\definecolor{currentstroke}{rgb}{1.000000,1.000000,1.000000}%
\pgfsetstrokecolor{currentstroke}%
\pgfsetdash{}{0pt}%
\pgfpathmoveto{\pgfqpoint{4.082342in}{1.115379in}}%
\pgfpathcurveto{\pgfqpoint{4.089475in}{1.115379in}}{\pgfqpoint{4.096317in}{1.118213in}}{\pgfqpoint{4.101361in}{1.123257in}}%
\pgfpathcurveto{\pgfqpoint{4.106404in}{1.128300in}}{\pgfqpoint{4.109238in}{1.135142in}}{\pgfqpoint{4.109238in}{1.142275in}}%
\pgfpathcurveto{\pgfqpoint{4.109238in}{1.149408in}}{\pgfqpoint{4.106404in}{1.156249in}}{\pgfqpoint{4.101361in}{1.161293in}}%
\pgfpathcurveto{\pgfqpoint{4.096317in}{1.166337in}}{\pgfqpoint{4.089475in}{1.169171in}}{\pgfqpoint{4.082342in}{1.169171in}}%
\pgfpathcurveto{\pgfqpoint{4.075210in}{1.169171in}}{\pgfqpoint{4.068368in}{1.166337in}}{\pgfqpoint{4.063324in}{1.161293in}}%
\pgfpathcurveto{\pgfqpoint{4.058281in}{1.156249in}}{\pgfqpoint{4.055447in}{1.149408in}}{\pgfqpoint{4.055447in}{1.142275in}}%
\pgfpathcurveto{\pgfqpoint{4.055447in}{1.135142in}}{\pgfqpoint{4.058281in}{1.128300in}}{\pgfqpoint{4.063324in}{1.123257in}}%
\pgfpathcurveto{\pgfqpoint{4.068368in}{1.118213in}}{\pgfqpoint{4.075210in}{1.115379in}}{\pgfqpoint{4.082342in}{1.115379in}}%
\pgfpathclose%
\pgfusepath{stroke,fill}%
\end{pgfscope}%
\begin{pgfscope}%
\pgfpathrectangle{\pgfqpoint{2.867647in}{0.500000in}}{\pgfqpoint{1.764706in}{1.700000in}}%
\pgfusepath{clip}%
\pgfsetbuttcap%
\pgfsetroundjoin%
\definecolor{currentfill}{rgb}{0.965302,0.713942,0.568499}%
\pgfsetfillcolor{currentfill}%
\pgfsetlinewidth{0.311001pt}%
\definecolor{currentstroke}{rgb}{1.000000,1.000000,1.000000}%
\pgfsetstrokecolor{currentstroke}%
\pgfsetdash{}{0pt}%
\pgfpathmoveto{\pgfqpoint{4.009014in}{1.047947in}}%
\pgfpathcurveto{\pgfqpoint{4.016147in}{1.047947in}}{\pgfqpoint{4.022988in}{1.050781in}}{\pgfqpoint{4.028032in}{1.055824in}}%
\pgfpathcurveto{\pgfqpoint{4.033076in}{1.060868in}}{\pgfqpoint{4.035910in}{1.067710in}}{\pgfqpoint{4.035910in}{1.074843in}}%
\pgfpathcurveto{\pgfqpoint{4.035910in}{1.081975in}}{\pgfqpoint{4.033076in}{1.088817in}}{\pgfqpoint{4.028032in}{1.093861in}}%
\pgfpathcurveto{\pgfqpoint{4.022988in}{1.098904in}}{\pgfqpoint{4.016147in}{1.101738in}}{\pgfqpoint{4.009014in}{1.101738in}}%
\pgfpathcurveto{\pgfqpoint{4.001881in}{1.101738in}}{\pgfqpoint{3.995039in}{1.098904in}}{\pgfqpoint{3.989996in}{1.093861in}}%
\pgfpathcurveto{\pgfqpoint{3.984952in}{1.088817in}}{\pgfqpoint{3.982118in}{1.081975in}}{\pgfqpoint{3.982118in}{1.074843in}}%
\pgfpathcurveto{\pgfqpoint{3.982118in}{1.067710in}}{\pgfqpoint{3.984952in}{1.060868in}}{\pgfqpoint{3.989996in}{1.055824in}}%
\pgfpathcurveto{\pgfqpoint{3.995039in}{1.050781in}}{\pgfqpoint{4.001881in}{1.047947in}}{\pgfqpoint{4.009014in}{1.047947in}}%
\pgfpathclose%
\pgfusepath{stroke,fill}%
\end{pgfscope}%
\begin{pgfscope}%
\pgfpathrectangle{\pgfqpoint{2.867647in}{0.500000in}}{\pgfqpoint{1.764706in}{1.700000in}}%
\pgfusepath{clip}%
\pgfsetbuttcap%
\pgfsetroundjoin%
\definecolor{currentfill}{rgb}{0.965169,0.707764,0.560659}%
\pgfsetfillcolor{currentfill}%
\pgfsetlinewidth{0.311001pt}%
\definecolor{currentstroke}{rgb}{1.000000,1.000000,1.000000}%
\pgfsetstrokecolor{currentstroke}%
\pgfsetdash{}{0pt}%
\pgfpathmoveto{\pgfqpoint{4.220870in}{0.998676in}}%
\pgfpathcurveto{\pgfqpoint{4.228003in}{0.998676in}}{\pgfqpoint{4.234844in}{1.001510in}}{\pgfqpoint{4.239888in}{1.006554in}}%
\pgfpathcurveto{\pgfqpoint{4.244932in}{1.011598in}}{\pgfqpoint{4.247766in}{1.018439in}}{\pgfqpoint{4.247766in}{1.025572in}}%
\pgfpathcurveto{\pgfqpoint{4.247766in}{1.032705in}}{\pgfqpoint{4.244932in}{1.039546in}}{\pgfqpoint{4.239888in}{1.044590in}}%
\pgfpathcurveto{\pgfqpoint{4.234844in}{1.049634in}}{\pgfqpoint{4.228003in}{1.052468in}}{\pgfqpoint{4.220870in}{1.052468in}}%
\pgfpathcurveto{\pgfqpoint{4.213737in}{1.052468in}}{\pgfqpoint{4.206895in}{1.049634in}}{\pgfqpoint{4.201852in}{1.044590in}}%
\pgfpathcurveto{\pgfqpoint{4.196808in}{1.039546in}}{\pgfqpoint{4.193974in}{1.032705in}}{\pgfqpoint{4.193974in}{1.025572in}}%
\pgfpathcurveto{\pgfqpoint{4.193974in}{1.018439in}}{\pgfqpoint{4.196808in}{1.011598in}}{\pgfqpoint{4.201852in}{1.006554in}}%
\pgfpathcurveto{\pgfqpoint{4.206895in}{1.001510in}}{\pgfqpoint{4.213737in}{0.998676in}}{\pgfqpoint{4.220870in}{0.998676in}}%
\pgfpathclose%
\pgfusepath{stroke,fill}%
\end{pgfscope}%
\begin{pgfscope}%
\pgfpathrectangle{\pgfqpoint{2.867647in}{0.500000in}}{\pgfqpoint{1.764706in}{1.700000in}}%
\pgfusepath{clip}%
\pgfsetbuttcap%
\pgfsetroundjoin%
\definecolor{currentfill}{rgb}{0.975644,0.874038,0.797253}%
\pgfsetfillcolor{currentfill}%
\pgfsetlinewidth{0.311001pt}%
\definecolor{currentstroke}{rgb}{1.000000,1.000000,1.000000}%
\pgfsetstrokecolor{currentstroke}%
\pgfsetdash{}{0pt}%
\pgfpathmoveto{\pgfqpoint{4.242036in}{1.370944in}}%
\pgfpathcurveto{\pgfqpoint{4.249169in}{1.370944in}}{\pgfqpoint{4.256010in}{1.373778in}}{\pgfqpoint{4.261054in}{1.378822in}}%
\pgfpathcurveto{\pgfqpoint{4.266098in}{1.383866in}}{\pgfqpoint{4.268932in}{1.390707in}}{\pgfqpoint{4.268932in}{1.397840in}}%
\pgfpathcurveto{\pgfqpoint{4.268932in}{1.404973in}}{\pgfqpoint{4.266098in}{1.411815in}}{\pgfqpoint{4.261054in}{1.416858in}}%
\pgfpathcurveto{\pgfqpoint{4.256010in}{1.421902in}}{\pgfqpoint{4.249169in}{1.424736in}}{\pgfqpoint{4.242036in}{1.424736in}}%
\pgfpathcurveto{\pgfqpoint{4.234903in}{1.424736in}}{\pgfqpoint{4.228061in}{1.421902in}}{\pgfqpoint{4.223018in}{1.416858in}}%
\pgfpathcurveto{\pgfqpoint{4.217974in}{1.411815in}}{\pgfqpoint{4.215140in}{1.404973in}}{\pgfqpoint{4.215140in}{1.397840in}}%
\pgfpathcurveto{\pgfqpoint{4.215140in}{1.390707in}}{\pgfqpoint{4.217974in}{1.383866in}}{\pgfqpoint{4.223018in}{1.378822in}}%
\pgfpathcurveto{\pgfqpoint{4.228061in}{1.373778in}}{\pgfqpoint{4.234903in}{1.370944in}}{\pgfqpoint{4.242036in}{1.370944in}}%
\pgfpathclose%
\pgfusepath{stroke,fill}%
\end{pgfscope}%
\begin{pgfscope}%
\pgfpathrectangle{\pgfqpoint{2.867647in}{0.500000in}}{\pgfqpoint{1.764706in}{1.700000in}}%
\pgfusepath{clip}%
\pgfsetbuttcap%
\pgfsetroundjoin%
\definecolor{currentfill}{rgb}{0.968509,0.792226,0.676405}%
\pgfsetfillcolor{currentfill}%
\pgfsetlinewidth{0.311001pt}%
\definecolor{currentstroke}{rgb}{1.000000,1.000000,1.000000}%
\pgfsetstrokecolor{currentstroke}%
\pgfsetdash{}{0pt}%
\pgfpathmoveto{\pgfqpoint{4.032605in}{1.667485in}}%
\pgfpathcurveto{\pgfqpoint{4.039738in}{1.667485in}}{\pgfqpoint{4.046579in}{1.670319in}}{\pgfqpoint{4.051623in}{1.675362in}}%
\pgfpathcurveto{\pgfqpoint{4.056667in}{1.680406in}}{\pgfqpoint{4.059500in}{1.687248in}}{\pgfqpoint{4.059500in}{1.694381in}}%
\pgfpathcurveto{\pgfqpoint{4.059500in}{1.701513in}}{\pgfqpoint{4.056667in}{1.708355in}}{\pgfqpoint{4.051623in}{1.713399in}}%
\pgfpathcurveto{\pgfqpoint{4.046579in}{1.718442in}}{\pgfqpoint{4.039738in}{1.721276in}}{\pgfqpoint{4.032605in}{1.721276in}}%
\pgfpathcurveto{\pgfqpoint{4.025472in}{1.721276in}}{\pgfqpoint{4.018630in}{1.718442in}}{\pgfqpoint{4.013587in}{1.713399in}}%
\pgfpathcurveto{\pgfqpoint{4.008543in}{1.708355in}}{\pgfqpoint{4.005709in}{1.701513in}}{\pgfqpoint{4.005709in}{1.694381in}}%
\pgfpathcurveto{\pgfqpoint{4.005709in}{1.687248in}}{\pgfqpoint{4.008543in}{1.680406in}}{\pgfqpoint{4.013587in}{1.675362in}}%
\pgfpathcurveto{\pgfqpoint{4.018630in}{1.670319in}}{\pgfqpoint{4.025472in}{1.667485in}}{\pgfqpoint{4.032605in}{1.667485in}}%
\pgfpathclose%
\pgfusepath{stroke,fill}%
\end{pgfscope}%
\begin{pgfscope}%
\pgfpathrectangle{\pgfqpoint{2.867647in}{0.500000in}}{\pgfqpoint{1.764706in}{1.700000in}}%
\pgfusepath{clip}%
\pgfsetbuttcap%
\pgfsetroundjoin%
\definecolor{currentfill}{rgb}{0.975018,0.868213,0.788710}%
\pgfsetfillcolor{currentfill}%
\pgfsetlinewidth{0.311001pt}%
\definecolor{currentstroke}{rgb}{1.000000,1.000000,1.000000}%
\pgfsetstrokecolor{currentstroke}%
\pgfsetdash{}{0pt}%
\pgfpathmoveto{\pgfqpoint{4.221110in}{1.501040in}}%
\pgfpathcurveto{\pgfqpoint{4.228243in}{1.501040in}}{\pgfqpoint{4.235084in}{1.503874in}}{\pgfqpoint{4.240128in}{1.508918in}}%
\pgfpathcurveto{\pgfqpoint{4.245172in}{1.513962in}}{\pgfqpoint{4.248006in}{1.520803in}}{\pgfqpoint{4.248006in}{1.527936in}}%
\pgfpathcurveto{\pgfqpoint{4.248006in}{1.535069in}}{\pgfqpoint{4.245172in}{1.541911in}}{\pgfqpoint{4.240128in}{1.546954in}}%
\pgfpathcurveto{\pgfqpoint{4.235084in}{1.551998in}}{\pgfqpoint{4.228243in}{1.554832in}}{\pgfqpoint{4.221110in}{1.554832in}}%
\pgfpathcurveto{\pgfqpoint{4.213977in}{1.554832in}}{\pgfqpoint{4.207135in}{1.551998in}}{\pgfqpoint{4.202092in}{1.546954in}}%
\pgfpathcurveto{\pgfqpoint{4.197048in}{1.541911in}}{\pgfqpoint{4.194214in}{1.535069in}}{\pgfqpoint{4.194214in}{1.527936in}}%
\pgfpathcurveto{\pgfqpoint{4.194214in}{1.520803in}}{\pgfqpoint{4.197048in}{1.513962in}}{\pgfqpoint{4.202092in}{1.508918in}}%
\pgfpathcurveto{\pgfqpoint{4.207135in}{1.503874in}}{\pgfqpoint{4.213977in}{1.501040in}}{\pgfqpoint{4.221110in}{1.501040in}}%
\pgfpathclose%
\pgfusepath{stroke,fill}%
\end{pgfscope}%
\begin{pgfscope}%
\pgfpathrectangle{\pgfqpoint{2.867647in}{0.500000in}}{\pgfqpoint{1.764706in}{1.700000in}}%
\pgfusepath{clip}%
\pgfsetbuttcap%
\pgfsetroundjoin%
\definecolor{currentfill}{rgb}{0.965440,0.720101,0.576404}%
\pgfsetfillcolor{currentfill}%
\pgfsetlinewidth{0.311001pt}%
\definecolor{currentstroke}{rgb}{1.000000,1.000000,1.000000}%
\pgfsetstrokecolor{currentstroke}%
\pgfsetdash{}{0pt}%
\pgfpathmoveto{\pgfqpoint{4.076000in}{1.276494in}}%
\pgfpathcurveto{\pgfqpoint{4.083133in}{1.276494in}}{\pgfqpoint{4.089974in}{1.279328in}}{\pgfqpoint{4.095018in}{1.284372in}}%
\pgfpathcurveto{\pgfqpoint{4.100062in}{1.289415in}}{\pgfqpoint{4.102896in}{1.296257in}}{\pgfqpoint{4.102896in}{1.303390in}}%
\pgfpathcurveto{\pgfqpoint{4.102896in}{1.310523in}}{\pgfqpoint{4.100062in}{1.317364in}}{\pgfqpoint{4.095018in}{1.322408in}}%
\pgfpathcurveto{\pgfqpoint{4.089974in}{1.327452in}}{\pgfqpoint{4.083133in}{1.330286in}}{\pgfqpoint{4.076000in}{1.330286in}}%
\pgfpathcurveto{\pgfqpoint{4.068867in}{1.330286in}}{\pgfqpoint{4.062025in}{1.327452in}}{\pgfqpoint{4.056982in}{1.322408in}}%
\pgfpathcurveto{\pgfqpoint{4.051938in}{1.317364in}}{\pgfqpoint{4.049104in}{1.310523in}}{\pgfqpoint{4.049104in}{1.303390in}}%
\pgfpathcurveto{\pgfqpoint{4.049104in}{1.296257in}}{\pgfqpoint{4.051938in}{1.289415in}}{\pgfqpoint{4.056982in}{1.284372in}}%
\pgfpathcurveto{\pgfqpoint{4.062025in}{1.279328in}}{\pgfqpoint{4.068867in}{1.276494in}}{\pgfqpoint{4.076000in}{1.276494in}}%
\pgfpathclose%
\pgfusepath{stroke,fill}%
\end{pgfscope}%
\begin{pgfscope}%
\pgfpathrectangle{\pgfqpoint{2.867647in}{0.500000in}}{\pgfqpoint{1.764706in}{1.700000in}}%
\pgfusepath{clip}%
\pgfsetbuttcap%
\pgfsetroundjoin%
\definecolor{currentfill}{rgb}{0.980678,0.914765,0.856766}%
\pgfsetfillcolor{currentfill}%
\pgfsetlinewidth{0.311001pt}%
\definecolor{currentstroke}{rgb}{1.000000,1.000000,1.000000}%
\pgfsetstrokecolor{currentstroke}%
\pgfsetdash{}{0pt}%
\pgfpathmoveto{\pgfqpoint{4.145821in}{1.151600in}}%
\pgfpathcurveto{\pgfqpoint{4.152954in}{1.151600in}}{\pgfqpoint{4.159796in}{1.154434in}}{\pgfqpoint{4.164839in}{1.159477in}}%
\pgfpathcurveto{\pgfqpoint{4.169883in}{1.164521in}}{\pgfqpoint{4.172717in}{1.171363in}}{\pgfqpoint{4.172717in}{1.178496in}}%
\pgfpathcurveto{\pgfqpoint{4.172717in}{1.185628in}}{\pgfqpoint{4.169883in}{1.192470in}}{\pgfqpoint{4.164839in}{1.197514in}}%
\pgfpathcurveto{\pgfqpoint{4.159796in}{1.202557in}}{\pgfqpoint{4.152954in}{1.205391in}}{\pgfqpoint{4.145821in}{1.205391in}}%
\pgfpathcurveto{\pgfqpoint{4.138688in}{1.205391in}}{\pgfqpoint{4.131847in}{1.202557in}}{\pgfqpoint{4.126803in}{1.197514in}}%
\pgfpathcurveto{\pgfqpoint{4.121759in}{1.192470in}}{\pgfqpoint{4.118926in}{1.185628in}}{\pgfqpoint{4.118926in}{1.178496in}}%
\pgfpathcurveto{\pgfqpoint{4.118926in}{1.171363in}}{\pgfqpoint{4.121759in}{1.164521in}}{\pgfqpoint{4.126803in}{1.159477in}}%
\pgfpathcurveto{\pgfqpoint{4.131847in}{1.154434in}}{\pgfqpoint{4.138688in}{1.151600in}}{\pgfqpoint{4.145821in}{1.151600in}}%
\pgfpathclose%
\pgfusepath{stroke,fill}%
\end{pgfscope}%
\begin{pgfscope}%
\pgfpathrectangle{\pgfqpoint{2.867647in}{0.500000in}}{\pgfqpoint{1.764706in}{1.700000in}}%
\pgfusepath{clip}%
\pgfsetbuttcap%
\pgfsetroundjoin%
\definecolor{currentfill}{rgb}{0.970718,0.821518,0.719872}%
\pgfsetfillcolor{currentfill}%
\pgfsetlinewidth{0.311001pt}%
\definecolor{currentstroke}{rgb}{1.000000,1.000000,1.000000}%
\pgfsetstrokecolor{currentstroke}%
\pgfsetdash{}{0pt}%
\pgfpathmoveto{\pgfqpoint{4.119910in}{0.963370in}}%
\pgfpathcurveto{\pgfqpoint{4.127043in}{0.963370in}}{\pgfqpoint{4.133884in}{0.966204in}}{\pgfqpoint{4.138928in}{0.971248in}}%
\pgfpathcurveto{\pgfqpoint{4.143972in}{0.976292in}}{\pgfqpoint{4.146806in}{0.983133in}}{\pgfqpoint{4.146806in}{0.990266in}}%
\pgfpathcurveto{\pgfqpoint{4.146806in}{0.997399in}}{\pgfqpoint{4.143972in}{1.004241in}}{\pgfqpoint{4.138928in}{1.009284in}}%
\pgfpathcurveto{\pgfqpoint{4.133884in}{1.014328in}}{\pgfqpoint{4.127043in}{1.017162in}}{\pgfqpoint{4.119910in}{1.017162in}}%
\pgfpathcurveto{\pgfqpoint{4.112777in}{1.017162in}}{\pgfqpoint{4.105935in}{1.014328in}}{\pgfqpoint{4.100892in}{1.009284in}}%
\pgfpathcurveto{\pgfqpoint{4.095848in}{1.004241in}}{\pgfqpoint{4.093014in}{0.997399in}}{\pgfqpoint{4.093014in}{0.990266in}}%
\pgfpathcurveto{\pgfqpoint{4.093014in}{0.983133in}}{\pgfqpoint{4.095848in}{0.976292in}}{\pgfqpoint{4.100892in}{0.971248in}}%
\pgfpathcurveto{\pgfqpoint{4.105935in}{0.966204in}}{\pgfqpoint{4.112777in}{0.963370in}}{\pgfqpoint{4.119910in}{0.963370in}}%
\pgfpathclose%
\pgfusepath{stroke,fill}%
\end{pgfscope}%
\begin{pgfscope}%
\pgfpathrectangle{\pgfqpoint{2.867647in}{0.500000in}}{\pgfqpoint{1.764706in}{1.700000in}}%
\pgfusepath{clip}%
\pgfsetbuttcap%
\pgfsetroundjoin%
\definecolor{currentfill}{rgb}{0.146334,0.079734,0.198615}%
\pgfsetfillcolor{currentfill}%
\pgfsetlinewidth{0.311001pt}%
\definecolor{currentstroke}{rgb}{1.000000,1.000000,1.000000}%
\pgfsetstrokecolor{currentstroke}%
\pgfsetdash{}{0pt}%
\pgfpathmoveto{\pgfqpoint{3.600039in}{0.724237in}}%
\pgfpathcurveto{\pgfqpoint{3.607172in}{0.724237in}}{\pgfqpoint{3.614013in}{0.727071in}}{\pgfqpoint{3.619057in}{0.732114in}}%
\pgfpathcurveto{\pgfqpoint{3.624101in}{0.737158in}}{\pgfqpoint{3.626935in}{0.744000in}}{\pgfqpoint{3.626935in}{0.751132in}}%
\pgfpathcurveto{\pgfqpoint{3.626935in}{0.758265in}}{\pgfqpoint{3.624101in}{0.765107in}}{\pgfqpoint{3.619057in}{0.770151in}}%
\pgfpathcurveto{\pgfqpoint{3.614013in}{0.775194in}}{\pgfqpoint{3.607172in}{0.778028in}}{\pgfqpoint{3.600039in}{0.778028in}}%
\pgfpathcurveto{\pgfqpoint{3.592906in}{0.778028in}}{\pgfqpoint{3.586064in}{0.775194in}}{\pgfqpoint{3.581021in}{0.770151in}}%
\pgfpathcurveto{\pgfqpoint{3.575977in}{0.765107in}}{\pgfqpoint{3.573143in}{0.758265in}}{\pgfqpoint{3.573143in}{0.751132in}}%
\pgfpathcurveto{\pgfqpoint{3.573143in}{0.744000in}}{\pgfqpoint{3.575977in}{0.737158in}}{\pgfqpoint{3.581021in}{0.732114in}}%
\pgfpathcurveto{\pgfqpoint{3.586064in}{0.727071in}}{\pgfqpoint{3.592906in}{0.724237in}}{\pgfqpoint{3.600039in}{0.724237in}}%
\pgfpathclose%
\pgfusepath{stroke,fill}%
\end{pgfscope}%
\begin{pgfscope}%
\pgfpathrectangle{\pgfqpoint{2.867647in}{0.500000in}}{\pgfqpoint{1.764706in}{1.700000in}}%
\pgfusepath{clip}%
\pgfsetbuttcap%
\pgfsetroundjoin%
\definecolor{currentfill}{rgb}{0.968509,0.792226,0.676405}%
\pgfsetfillcolor{currentfill}%
\pgfsetlinewidth{0.311001pt}%
\definecolor{currentstroke}{rgb}{1.000000,1.000000,1.000000}%
\pgfsetstrokecolor{currentstroke}%
\pgfsetdash{}{0pt}%
\pgfpathmoveto{\pgfqpoint{4.132333in}{0.951529in}}%
\pgfpathcurveto{\pgfqpoint{4.139466in}{0.951529in}}{\pgfqpoint{4.146307in}{0.954363in}}{\pgfqpoint{4.151351in}{0.959407in}}%
\pgfpathcurveto{\pgfqpoint{4.156395in}{0.964450in}}{\pgfqpoint{4.159229in}{0.971292in}}{\pgfqpoint{4.159229in}{0.978425in}}%
\pgfpathcurveto{\pgfqpoint{4.159229in}{0.985558in}}{\pgfqpoint{4.156395in}{0.992399in}}{\pgfqpoint{4.151351in}{0.997443in}}%
\pgfpathcurveto{\pgfqpoint{4.146307in}{1.002487in}}{\pgfqpoint{4.139466in}{1.005320in}}{\pgfqpoint{4.132333in}{1.005320in}}%
\pgfpathcurveto{\pgfqpoint{4.125200in}{1.005320in}}{\pgfqpoint{4.118359in}{1.002487in}}{\pgfqpoint{4.113315in}{0.997443in}}%
\pgfpathcurveto{\pgfqpoint{4.108271in}{0.992399in}}{\pgfqpoint{4.105437in}{0.985558in}}{\pgfqpoint{4.105437in}{0.978425in}}%
\pgfpathcurveto{\pgfqpoint{4.105437in}{0.971292in}}{\pgfqpoint{4.108271in}{0.964450in}}{\pgfqpoint{4.113315in}{0.959407in}}%
\pgfpathcurveto{\pgfqpoint{4.118359in}{0.954363in}}{\pgfqpoint{4.125200in}{0.951529in}}{\pgfqpoint{4.132333in}{0.951529in}}%
\pgfpathclose%
\pgfusepath{stroke,fill}%
\end{pgfscope}%
\begin{pgfscope}%
\pgfpathrectangle{\pgfqpoint{2.867647in}{0.500000in}}{\pgfqpoint{1.764706in}{1.700000in}}%
\pgfusepath{clip}%
\pgfsetbuttcap%
\pgfsetroundjoin%
\definecolor{currentfill}{rgb}{0.963559,0.632016,0.472047}%
\pgfsetfillcolor{currentfill}%
\pgfsetlinewidth{0.311001pt}%
\definecolor{currentstroke}{rgb}{1.000000,1.000000,1.000000}%
\pgfsetstrokecolor{currentstroke}%
\pgfsetdash{}{0pt}%
\pgfpathmoveto{\pgfqpoint{4.031516in}{1.458767in}}%
\pgfpathcurveto{\pgfqpoint{4.038649in}{1.458767in}}{\pgfqpoint{4.045491in}{1.461601in}}{\pgfqpoint{4.050534in}{1.466644in}}%
\pgfpathcurveto{\pgfqpoint{4.055578in}{1.471688in}}{\pgfqpoint{4.058412in}{1.478530in}}{\pgfqpoint{4.058412in}{1.485662in}}%
\pgfpathcurveto{\pgfqpoint{4.058412in}{1.492795in}}{\pgfqpoint{4.055578in}{1.499637in}}{\pgfqpoint{4.050534in}{1.504681in}}%
\pgfpathcurveto{\pgfqpoint{4.045491in}{1.509724in}}{\pgfqpoint{4.038649in}{1.512558in}}{\pgfqpoint{4.031516in}{1.512558in}}%
\pgfpathcurveto{\pgfqpoint{4.024383in}{1.512558in}}{\pgfqpoint{4.017542in}{1.509724in}}{\pgfqpoint{4.012498in}{1.504681in}}%
\pgfpathcurveto{\pgfqpoint{4.007454in}{1.499637in}}{\pgfqpoint{4.004620in}{1.492795in}}{\pgfqpoint{4.004620in}{1.485662in}}%
\pgfpathcurveto{\pgfqpoint{4.004620in}{1.478530in}}{\pgfqpoint{4.007454in}{1.471688in}}{\pgfqpoint{4.012498in}{1.466644in}}%
\pgfpathcurveto{\pgfqpoint{4.017542in}{1.461601in}}{\pgfqpoint{4.024383in}{1.458767in}}{\pgfqpoint{4.031516in}{1.458767in}}%
\pgfpathclose%
\pgfusepath{stroke,fill}%
\end{pgfscope}%
\begin{pgfscope}%
\pgfpathrectangle{\pgfqpoint{2.867647in}{0.500000in}}{\pgfqpoint{1.764706in}{1.700000in}}%
\pgfusepath{clip}%
\pgfsetbuttcap%
\pgfsetroundjoin%
\definecolor{currentfill}{rgb}{0.979124,0.903132,0.839793}%
\pgfsetfillcolor{currentfill}%
\pgfsetlinewidth{0.311001pt}%
\definecolor{currentstroke}{rgb}{1.000000,1.000000,1.000000}%
\pgfsetstrokecolor{currentstroke}%
\pgfsetdash{}{0pt}%
\pgfpathmoveto{\pgfqpoint{4.148825in}{1.233089in}}%
\pgfpathcurveto{\pgfqpoint{4.155958in}{1.233089in}}{\pgfqpoint{4.162800in}{1.235923in}}{\pgfqpoint{4.167843in}{1.240967in}}%
\pgfpathcurveto{\pgfqpoint{4.172887in}{1.246010in}}{\pgfqpoint{4.175721in}{1.252852in}}{\pgfqpoint{4.175721in}{1.259985in}}%
\pgfpathcurveto{\pgfqpoint{4.175721in}{1.267118in}}{\pgfqpoint{4.172887in}{1.273959in}}{\pgfqpoint{4.167843in}{1.279003in}}%
\pgfpathcurveto{\pgfqpoint{4.162800in}{1.284047in}}{\pgfqpoint{4.155958in}{1.286881in}}{\pgfqpoint{4.148825in}{1.286881in}}%
\pgfpathcurveto{\pgfqpoint{4.141692in}{1.286881in}}{\pgfqpoint{4.134851in}{1.284047in}}{\pgfqpoint{4.129807in}{1.279003in}}%
\pgfpathcurveto{\pgfqpoint{4.124763in}{1.273959in}}{\pgfqpoint{4.121929in}{1.267118in}}{\pgfqpoint{4.121929in}{1.259985in}}%
\pgfpathcurveto{\pgfqpoint{4.121929in}{1.252852in}}{\pgfqpoint{4.124763in}{1.246010in}}{\pgfqpoint{4.129807in}{1.240967in}}%
\pgfpathcurveto{\pgfqpoint{4.134851in}{1.235923in}}{\pgfqpoint{4.141692in}{1.233089in}}{\pgfqpoint{4.148825in}{1.233089in}}%
\pgfpathclose%
\pgfusepath{stroke,fill}%
\end{pgfscope}%
\begin{pgfscope}%
\pgfpathrectangle{\pgfqpoint{2.867647in}{0.500000in}}{\pgfqpoint{1.764706in}{1.700000in}}%
\pgfusepath{clip}%
\pgfsetbuttcap%
\pgfsetroundjoin%
\definecolor{currentfill}{rgb}{0.979124,0.903132,0.839793}%
\pgfsetfillcolor{currentfill}%
\pgfsetlinewidth{0.311001pt}%
\definecolor{currentstroke}{rgb}{1.000000,1.000000,1.000000}%
\pgfsetstrokecolor{currentstroke}%
\pgfsetdash{}{0pt}%
\pgfpathmoveto{\pgfqpoint{4.224040in}{1.252498in}}%
\pgfpathcurveto{\pgfqpoint{4.231173in}{1.252498in}}{\pgfqpoint{4.238015in}{1.255332in}}{\pgfqpoint{4.243059in}{1.260376in}}%
\pgfpathcurveto{\pgfqpoint{4.248102in}{1.265420in}}{\pgfqpoint{4.250936in}{1.272261in}}{\pgfqpoint{4.250936in}{1.279394in}}%
\pgfpathcurveto{\pgfqpoint{4.250936in}{1.286527in}}{\pgfqpoint{4.248102in}{1.293369in}}{\pgfqpoint{4.243059in}{1.298412in}}%
\pgfpathcurveto{\pgfqpoint{4.238015in}{1.303456in}}{\pgfqpoint{4.231173in}{1.306290in}}{\pgfqpoint{4.224040in}{1.306290in}}%
\pgfpathcurveto{\pgfqpoint{4.216908in}{1.306290in}}{\pgfqpoint{4.210066in}{1.303456in}}{\pgfqpoint{4.205022in}{1.298412in}}%
\pgfpathcurveto{\pgfqpoint{4.199979in}{1.293369in}}{\pgfqpoint{4.197145in}{1.286527in}}{\pgfqpoint{4.197145in}{1.279394in}}%
\pgfpathcurveto{\pgfqpoint{4.197145in}{1.272261in}}{\pgfqpoint{4.199979in}{1.265420in}}{\pgfqpoint{4.205022in}{1.260376in}}%
\pgfpathcurveto{\pgfqpoint{4.210066in}{1.255332in}}{\pgfqpoint{4.216908in}{1.252498in}}{\pgfqpoint{4.224040in}{1.252498in}}%
\pgfpathclose%
\pgfusepath{stroke,fill}%
\end{pgfscope}%
\begin{pgfscope}%
\pgfpathrectangle{\pgfqpoint{2.867647in}{0.500000in}}{\pgfqpoint{1.764706in}{1.700000in}}%
\pgfusepath{clip}%
\pgfsetbuttcap%
\pgfsetroundjoin%
\definecolor{currentfill}{rgb}{0.979124,0.903132,0.839793}%
\pgfsetfillcolor{currentfill}%
\pgfsetlinewidth{0.311001pt}%
\definecolor{currentstroke}{rgb}{1.000000,1.000000,1.000000}%
\pgfsetstrokecolor{currentstroke}%
\pgfsetdash{}{0pt}%
\pgfpathmoveto{\pgfqpoint{4.223625in}{1.319719in}}%
\pgfpathcurveto{\pgfqpoint{4.230757in}{1.319719in}}{\pgfqpoint{4.237599in}{1.322553in}}{\pgfqpoint{4.242643in}{1.327597in}}%
\pgfpathcurveto{\pgfqpoint{4.247686in}{1.332641in}}{\pgfqpoint{4.250520in}{1.339482in}}{\pgfqpoint{4.250520in}{1.346615in}}%
\pgfpathcurveto{\pgfqpoint{4.250520in}{1.353748in}}{\pgfqpoint{4.247686in}{1.360590in}}{\pgfqpoint{4.242643in}{1.365633in}}%
\pgfpathcurveto{\pgfqpoint{4.237599in}{1.370677in}}{\pgfqpoint{4.230757in}{1.373511in}}{\pgfqpoint{4.223625in}{1.373511in}}%
\pgfpathcurveto{\pgfqpoint{4.216492in}{1.373511in}}{\pgfqpoint{4.209650in}{1.370677in}}{\pgfqpoint{4.204606in}{1.365633in}}%
\pgfpathcurveto{\pgfqpoint{4.199563in}{1.360590in}}{\pgfqpoint{4.196729in}{1.353748in}}{\pgfqpoint{4.196729in}{1.346615in}}%
\pgfpathcurveto{\pgfqpoint{4.196729in}{1.339482in}}{\pgfqpoint{4.199563in}{1.332641in}}{\pgfqpoint{4.204606in}{1.327597in}}%
\pgfpathcurveto{\pgfqpoint{4.209650in}{1.322553in}}{\pgfqpoint{4.216492in}{1.319719in}}{\pgfqpoint{4.223625in}{1.319719in}}%
\pgfpathclose%
\pgfusepath{stroke,fill}%
\end{pgfscope}%
\begin{pgfscope}%
\pgfpathrectangle{\pgfqpoint{2.867647in}{0.500000in}}{\pgfqpoint{1.764706in}{1.700000in}}%
\pgfusepath{clip}%
\pgfsetbuttcap%
\pgfsetroundjoin%
\definecolor{currentfill}{rgb}{0.971202,0.827364,0.728520}%
\pgfsetfillcolor{currentfill}%
\pgfsetlinewidth{0.311001pt}%
\definecolor{currentstroke}{rgb}{1.000000,1.000000,1.000000}%
\pgfsetstrokecolor{currentstroke}%
\pgfsetdash{}{0pt}%
\pgfpathmoveto{\pgfqpoint{4.080837in}{1.490931in}}%
\pgfpathcurveto{\pgfqpoint{4.087970in}{1.490931in}}{\pgfqpoint{4.094812in}{1.493765in}}{\pgfqpoint{4.099856in}{1.498808in}}%
\pgfpathcurveto{\pgfqpoint{4.104899in}{1.503852in}}{\pgfqpoint{4.107733in}{1.510694in}}{\pgfqpoint{4.107733in}{1.517826in}}%
\pgfpathcurveto{\pgfqpoint{4.107733in}{1.524959in}}{\pgfqpoint{4.104899in}{1.531801in}}{\pgfqpoint{4.099856in}{1.536845in}}%
\pgfpathcurveto{\pgfqpoint{4.094812in}{1.541888in}}{\pgfqpoint{4.087970in}{1.544722in}}{\pgfqpoint{4.080837in}{1.544722in}}%
\pgfpathcurveto{\pgfqpoint{4.073705in}{1.544722in}}{\pgfqpoint{4.066863in}{1.541888in}}{\pgfqpoint{4.061819in}{1.536845in}}%
\pgfpathcurveto{\pgfqpoint{4.056776in}{1.531801in}}{\pgfqpoint{4.053942in}{1.524959in}}{\pgfqpoint{4.053942in}{1.517826in}}%
\pgfpathcurveto{\pgfqpoint{4.053942in}{1.510694in}}{\pgfqpoint{4.056776in}{1.503852in}}{\pgfqpoint{4.061819in}{1.498808in}}%
\pgfpathcurveto{\pgfqpoint{4.066863in}{1.493765in}}{\pgfqpoint{4.073705in}{1.490931in}}{\pgfqpoint{4.080837in}{1.490931in}}%
\pgfpathclose%
\pgfusepath{stroke,fill}%
\end{pgfscope}%
\begin{pgfscope}%
\pgfpathrectangle{\pgfqpoint{2.867647in}{0.500000in}}{\pgfqpoint{1.764706in}{1.700000in}}%
\pgfusepath{clip}%
\pgfsetbuttcap%
\pgfsetroundjoin%
\definecolor{currentfill}{rgb}{0.978376,0.897317,0.831308}%
\pgfsetfillcolor{currentfill}%
\pgfsetlinewidth{0.311001pt}%
\definecolor{currentstroke}{rgb}{1.000000,1.000000,1.000000}%
\pgfsetstrokecolor{currentstroke}%
\pgfsetdash{}{0pt}%
\pgfpathmoveto{\pgfqpoint{4.121935in}{1.088768in}}%
\pgfpathcurveto{\pgfqpoint{4.129068in}{1.088768in}}{\pgfqpoint{4.135910in}{1.091602in}}{\pgfqpoint{4.140953in}{1.096646in}}%
\pgfpathcurveto{\pgfqpoint{4.145997in}{1.101689in}}{\pgfqpoint{4.148831in}{1.108531in}}{\pgfqpoint{4.148831in}{1.115664in}}%
\pgfpathcurveto{\pgfqpoint{4.148831in}{1.122797in}}{\pgfqpoint{4.145997in}{1.129638in}}{\pgfqpoint{4.140953in}{1.134682in}}%
\pgfpathcurveto{\pgfqpoint{4.135910in}{1.139726in}}{\pgfqpoint{4.129068in}{1.142559in}}{\pgfqpoint{4.121935in}{1.142559in}}%
\pgfpathcurveto{\pgfqpoint{4.114803in}{1.142559in}}{\pgfqpoint{4.107961in}{1.139726in}}{\pgfqpoint{4.102917in}{1.134682in}}%
\pgfpathcurveto{\pgfqpoint{4.097874in}{1.129638in}}{\pgfqpoint{4.095040in}{1.122797in}}{\pgfqpoint{4.095040in}{1.115664in}}%
\pgfpathcurveto{\pgfqpoint{4.095040in}{1.108531in}}{\pgfqpoint{4.097874in}{1.101689in}}{\pgfqpoint{4.102917in}{1.096646in}}%
\pgfpathcurveto{\pgfqpoint{4.107961in}{1.091602in}}{\pgfqpoint{4.114803in}{1.088768in}}{\pgfqpoint{4.121935in}{1.088768in}}%
\pgfpathclose%
\pgfusepath{stroke,fill}%
\end{pgfscope}%
\begin{pgfscope}%
\pgfpathrectangle{\pgfqpoint{2.867647in}{0.500000in}}{\pgfqpoint{1.764706in}{1.700000in}}%
\pgfusepath{clip}%
\pgfsetbuttcap%
\pgfsetroundjoin%
\definecolor{currentfill}{rgb}{0.980678,0.914765,0.856766}%
\pgfsetfillcolor{currentfill}%
\pgfsetlinewidth{0.311001pt}%
\definecolor{currentstroke}{rgb}{1.000000,1.000000,1.000000}%
\pgfsetstrokecolor{currentstroke}%
\pgfsetdash{}{0pt}%
\pgfpathmoveto{\pgfqpoint{4.174788in}{1.479710in}}%
\pgfpathcurveto{\pgfqpoint{4.181921in}{1.479710in}}{\pgfqpoint{4.188763in}{1.482543in}}{\pgfqpoint{4.193806in}{1.487587in}}%
\pgfpathcurveto{\pgfqpoint{4.198850in}{1.492631in}}{\pgfqpoint{4.201684in}{1.499472in}}{\pgfqpoint{4.201684in}{1.506605in}}%
\pgfpathcurveto{\pgfqpoint{4.201684in}{1.513738in}}{\pgfqpoint{4.198850in}{1.520580in}}{\pgfqpoint{4.193806in}{1.525623in}}%
\pgfpathcurveto{\pgfqpoint{4.188763in}{1.530667in}}{\pgfqpoint{4.181921in}{1.533501in}}{\pgfqpoint{4.174788in}{1.533501in}}%
\pgfpathcurveto{\pgfqpoint{4.167655in}{1.533501in}}{\pgfqpoint{4.160814in}{1.530667in}}{\pgfqpoint{4.155770in}{1.525623in}}%
\pgfpathcurveto{\pgfqpoint{4.150726in}{1.520580in}}{\pgfqpoint{4.147892in}{1.513738in}}{\pgfqpoint{4.147892in}{1.506605in}}%
\pgfpathcurveto{\pgfqpoint{4.147892in}{1.499472in}}{\pgfqpoint{4.150726in}{1.492631in}}{\pgfqpoint{4.155770in}{1.487587in}}%
\pgfpathcurveto{\pgfqpoint{4.160814in}{1.482543in}}{\pgfqpoint{4.167655in}{1.479710in}}{\pgfqpoint{4.174788in}{1.479710in}}%
\pgfpathclose%
\pgfusepath{stroke,fill}%
\end{pgfscope}%
\begin{pgfscope}%
\pgfpathrectangle{\pgfqpoint{2.867647in}{0.500000in}}{\pgfqpoint{1.764706in}{1.700000in}}%
\pgfusepath{clip}%
\pgfsetbuttcap%
\pgfsetroundjoin%
\definecolor{currentfill}{rgb}{0.980678,0.914765,0.856766}%
\pgfsetfillcolor{currentfill}%
\pgfsetlinewidth{0.311001pt}%
\definecolor{currentstroke}{rgb}{1.000000,1.000000,1.000000}%
\pgfsetstrokecolor{currentstroke}%
\pgfsetdash{}{0pt}%
\pgfpathmoveto{\pgfqpoint{4.183368in}{1.453553in}}%
\pgfpathcurveto{\pgfqpoint{4.190501in}{1.453553in}}{\pgfqpoint{4.197342in}{1.456387in}}{\pgfqpoint{4.202386in}{1.461430in}}%
\pgfpathcurveto{\pgfqpoint{4.207430in}{1.466474in}}{\pgfqpoint{4.210264in}{1.473316in}}{\pgfqpoint{4.210264in}{1.480449in}}%
\pgfpathcurveto{\pgfqpoint{4.210264in}{1.487581in}}{\pgfqpoint{4.207430in}{1.494423in}}{\pgfqpoint{4.202386in}{1.499467in}}%
\pgfpathcurveto{\pgfqpoint{4.197342in}{1.504510in}}{\pgfqpoint{4.190501in}{1.507344in}}{\pgfqpoint{4.183368in}{1.507344in}}%
\pgfpathcurveto{\pgfqpoint{4.176235in}{1.507344in}}{\pgfqpoint{4.169393in}{1.504510in}}{\pgfqpoint{4.164350in}{1.499467in}}%
\pgfpathcurveto{\pgfqpoint{4.159306in}{1.494423in}}{\pgfqpoint{4.156472in}{1.487581in}}{\pgfqpoint{4.156472in}{1.480449in}}%
\pgfpathcurveto{\pgfqpoint{4.156472in}{1.473316in}}{\pgfqpoint{4.159306in}{1.466474in}}{\pgfqpoint{4.164350in}{1.461430in}}%
\pgfpathcurveto{\pgfqpoint{4.169393in}{1.456387in}}{\pgfqpoint{4.176235in}{1.453553in}}{\pgfqpoint{4.183368in}{1.453553in}}%
\pgfpathclose%
\pgfusepath{stroke,fill}%
\end{pgfscope}%
\begin{pgfscope}%
\pgfpathrectangle{\pgfqpoint{2.867647in}{0.500000in}}{\pgfqpoint{1.764706in}{1.700000in}}%
\pgfusepath{clip}%
\pgfsetbuttcap%
\pgfsetroundjoin%
\definecolor{currentfill}{rgb}{0.979124,0.903132,0.839793}%
\pgfsetfillcolor{currentfill}%
\pgfsetlinewidth{0.311001pt}%
\definecolor{currentstroke}{rgb}{1.000000,1.000000,1.000000}%
\pgfsetstrokecolor{currentstroke}%
\pgfsetdash{}{0pt}%
\pgfpathmoveto{\pgfqpoint{4.148554in}{1.246671in}}%
\pgfpathcurveto{\pgfqpoint{4.155686in}{1.246671in}}{\pgfqpoint{4.162528in}{1.249505in}}{\pgfqpoint{4.167572in}{1.254549in}}%
\pgfpathcurveto{\pgfqpoint{4.172615in}{1.259592in}}{\pgfqpoint{4.175449in}{1.266434in}}{\pgfqpoint{4.175449in}{1.273567in}}%
\pgfpathcurveto{\pgfqpoint{4.175449in}{1.280700in}}{\pgfqpoint{4.172615in}{1.287541in}}{\pgfqpoint{4.167572in}{1.292585in}}%
\pgfpathcurveto{\pgfqpoint{4.162528in}{1.297629in}}{\pgfqpoint{4.155686in}{1.300463in}}{\pgfqpoint{4.148554in}{1.300463in}}%
\pgfpathcurveto{\pgfqpoint{4.141421in}{1.300463in}}{\pgfqpoint{4.134579in}{1.297629in}}{\pgfqpoint{4.129535in}{1.292585in}}%
\pgfpathcurveto{\pgfqpoint{4.124492in}{1.287541in}}{\pgfqpoint{4.121658in}{1.280700in}}{\pgfqpoint{4.121658in}{1.273567in}}%
\pgfpathcurveto{\pgfqpoint{4.121658in}{1.266434in}}{\pgfqpoint{4.124492in}{1.259592in}}{\pgfqpoint{4.129535in}{1.254549in}}%
\pgfpathcurveto{\pgfqpoint{4.134579in}{1.249505in}}{\pgfqpoint{4.141421in}{1.246671in}}{\pgfqpoint{4.148554in}{1.246671in}}%
\pgfpathclose%
\pgfusepath{stroke,fill}%
\end{pgfscope}%
\begin{pgfscope}%
\pgfpathrectangle{\pgfqpoint{2.867647in}{0.500000in}}{\pgfqpoint{1.764706in}{1.700000in}}%
\pgfusepath{clip}%
\pgfsetbuttcap%
\pgfsetroundjoin%
\definecolor{currentfill}{rgb}{0.968931,0.798091,0.685123}%
\pgfsetfillcolor{currentfill}%
\pgfsetlinewidth{0.311001pt}%
\definecolor{currentstroke}{rgb}{1.000000,1.000000,1.000000}%
\pgfsetstrokecolor{currentstroke}%
\pgfsetdash{}{0pt}%
\pgfpathmoveto{\pgfqpoint{4.044647in}{1.690768in}}%
\pgfpathcurveto{\pgfqpoint{4.051780in}{1.690768in}}{\pgfqpoint{4.058621in}{1.693602in}}{\pgfqpoint{4.063665in}{1.698645in}}%
\pgfpathcurveto{\pgfqpoint{4.068709in}{1.703689in}}{\pgfqpoint{4.071543in}{1.710531in}}{\pgfqpoint{4.071543in}{1.717663in}}%
\pgfpathcurveto{\pgfqpoint{4.071543in}{1.724796in}}{\pgfqpoint{4.068709in}{1.731638in}}{\pgfqpoint{4.063665in}{1.736682in}}%
\pgfpathcurveto{\pgfqpoint{4.058621in}{1.741725in}}{\pgfqpoint{4.051780in}{1.744559in}}{\pgfqpoint{4.044647in}{1.744559in}}%
\pgfpathcurveto{\pgfqpoint{4.037514in}{1.744559in}}{\pgfqpoint{4.030672in}{1.741725in}}{\pgfqpoint{4.025629in}{1.736682in}}%
\pgfpathcurveto{\pgfqpoint{4.020585in}{1.731638in}}{\pgfqpoint{4.017751in}{1.724796in}}{\pgfqpoint{4.017751in}{1.717663in}}%
\pgfpathcurveto{\pgfqpoint{4.017751in}{1.710531in}}{\pgfqpoint{4.020585in}{1.703689in}}{\pgfqpoint{4.025629in}{1.698645in}}%
\pgfpathcurveto{\pgfqpoint{4.030672in}{1.693602in}}{\pgfqpoint{4.037514in}{1.690768in}}{\pgfqpoint{4.044647in}{1.690768in}}%
\pgfpathclose%
\pgfusepath{stroke,fill}%
\end{pgfscope}%
\begin{pgfscope}%
\pgfpathrectangle{\pgfqpoint{2.867647in}{0.500000in}}{\pgfqpoint{1.764706in}{1.700000in}}%
\pgfusepath{clip}%
\pgfsetbuttcap%
\pgfsetroundjoin%
\definecolor{currentfill}{rgb}{0.949145,0.420383,0.287810}%
\pgfsetfillcolor{currentfill}%
\pgfsetlinewidth{0.311001pt}%
\definecolor{currentstroke}{rgb}{1.000000,1.000000,1.000000}%
\pgfsetstrokecolor{currentstroke}%
\pgfsetdash{}{0pt}%
\pgfpathmoveto{\pgfqpoint{3.904109in}{1.633278in}}%
\pgfpathcurveto{\pgfqpoint{3.911242in}{1.633278in}}{\pgfqpoint{3.918084in}{1.636112in}}{\pgfqpoint{3.923128in}{1.641156in}}%
\pgfpathcurveto{\pgfqpoint{3.928171in}{1.646199in}}{\pgfqpoint{3.931005in}{1.653041in}}{\pgfqpoint{3.931005in}{1.660174in}}%
\pgfpathcurveto{\pgfqpoint{3.931005in}{1.667307in}}{\pgfqpoint{3.928171in}{1.674148in}}{\pgfqpoint{3.923128in}{1.679192in}}%
\pgfpathcurveto{\pgfqpoint{3.918084in}{1.684236in}}{\pgfqpoint{3.911242in}{1.687070in}}{\pgfqpoint{3.904109in}{1.687070in}}%
\pgfpathcurveto{\pgfqpoint{3.896977in}{1.687070in}}{\pgfqpoint{3.890135in}{1.684236in}}{\pgfqpoint{3.885091in}{1.679192in}}%
\pgfpathcurveto{\pgfqpoint{3.880048in}{1.674148in}}{\pgfqpoint{3.877214in}{1.667307in}}{\pgfqpoint{3.877214in}{1.660174in}}%
\pgfpathcurveto{\pgfqpoint{3.877214in}{1.653041in}}{\pgfqpoint{3.880048in}{1.646199in}}{\pgfqpoint{3.885091in}{1.641156in}}%
\pgfpathcurveto{\pgfqpoint{3.890135in}{1.636112in}}{\pgfqpoint{3.896977in}{1.633278in}}{\pgfqpoint{3.904109in}{1.633278in}}%
\pgfpathclose%
\pgfusepath{stroke,fill}%
\end{pgfscope}%
\begin{pgfscope}%
\pgfpathrectangle{\pgfqpoint{2.867647in}{0.500000in}}{\pgfqpoint{1.764706in}{1.700000in}}%
\pgfusepath{clip}%
\pgfsetbuttcap%
\pgfsetroundjoin%
\definecolor{currentfill}{rgb}{0.978376,0.897317,0.831308}%
\pgfsetfillcolor{currentfill}%
\pgfsetlinewidth{0.311001pt}%
\definecolor{currentstroke}{rgb}{1.000000,1.000000,1.000000}%
\pgfsetstrokecolor{currentstroke}%
\pgfsetdash{}{0pt}%
\pgfpathmoveto{\pgfqpoint{4.127369in}{1.088090in}}%
\pgfpathcurveto{\pgfqpoint{4.134501in}{1.088090in}}{\pgfqpoint{4.141343in}{1.090924in}}{\pgfqpoint{4.146387in}{1.095968in}}%
\pgfpathcurveto{\pgfqpoint{4.151430in}{1.101011in}}{\pgfqpoint{4.154264in}{1.107853in}}{\pgfqpoint{4.154264in}{1.114986in}}%
\pgfpathcurveto{\pgfqpoint{4.154264in}{1.122119in}}{\pgfqpoint{4.151430in}{1.128960in}}{\pgfqpoint{4.146387in}{1.134004in}}%
\pgfpathcurveto{\pgfqpoint{4.141343in}{1.139048in}}{\pgfqpoint{4.134501in}{1.141881in}}{\pgfqpoint{4.127369in}{1.141881in}}%
\pgfpathcurveto{\pgfqpoint{4.120236in}{1.141881in}}{\pgfqpoint{4.113394in}{1.139048in}}{\pgfqpoint{4.108350in}{1.134004in}}%
\pgfpathcurveto{\pgfqpoint{4.103307in}{1.128960in}}{\pgfqpoint{4.100473in}{1.122119in}}{\pgfqpoint{4.100473in}{1.114986in}}%
\pgfpathcurveto{\pgfqpoint{4.100473in}{1.107853in}}{\pgfqpoint{4.103307in}{1.101011in}}{\pgfqpoint{4.108350in}{1.095968in}}%
\pgfpathcurveto{\pgfqpoint{4.113394in}{1.090924in}}{\pgfqpoint{4.120236in}{1.088090in}}{\pgfqpoint{4.127369in}{1.088090in}}%
\pgfpathclose%
\pgfusepath{stroke,fill}%
\end{pgfscope}%
\begin{pgfscope}%
\pgfpathrectangle{\pgfqpoint{2.867647in}{0.500000in}}{\pgfqpoint{1.764706in}{1.700000in}}%
\pgfusepath{clip}%
\pgfsetbuttcap%
\pgfsetroundjoin%
\definecolor{currentfill}{rgb}{0.978376,0.897317,0.831308}%
\pgfsetfillcolor{currentfill}%
\pgfsetlinewidth{0.311001pt}%
\definecolor{currentstroke}{rgb}{1.000000,1.000000,1.000000}%
\pgfsetstrokecolor{currentstroke}%
\pgfsetdash{}{0pt}%
\pgfpathmoveto{\pgfqpoint{4.162214in}{1.072624in}}%
\pgfpathcurveto{\pgfqpoint{4.169347in}{1.072624in}}{\pgfqpoint{4.176188in}{1.075458in}}{\pgfqpoint{4.181232in}{1.080502in}}%
\pgfpathcurveto{\pgfqpoint{4.186276in}{1.085546in}}{\pgfqpoint{4.189110in}{1.092387in}}{\pgfqpoint{4.189110in}{1.099520in}}%
\pgfpathcurveto{\pgfqpoint{4.189110in}{1.106653in}}{\pgfqpoint{4.186276in}{1.113494in}}{\pgfqpoint{4.181232in}{1.118538in}}%
\pgfpathcurveto{\pgfqpoint{4.176188in}{1.123582in}}{\pgfqpoint{4.169347in}{1.126416in}}{\pgfqpoint{4.162214in}{1.126416in}}%
\pgfpathcurveto{\pgfqpoint{4.155081in}{1.126416in}}{\pgfqpoint{4.148239in}{1.123582in}}{\pgfqpoint{4.143196in}{1.118538in}}%
\pgfpathcurveto{\pgfqpoint{4.138152in}{1.113494in}}{\pgfqpoint{4.135318in}{1.106653in}}{\pgfqpoint{4.135318in}{1.099520in}}%
\pgfpathcurveto{\pgfqpoint{4.135318in}{1.092387in}}{\pgfqpoint{4.138152in}{1.085546in}}{\pgfqpoint{4.143196in}{1.080502in}}%
\pgfpathcurveto{\pgfqpoint{4.148239in}{1.075458in}}{\pgfqpoint{4.155081in}{1.072624in}}{\pgfqpoint{4.162214in}{1.072624in}}%
\pgfpathclose%
\pgfusepath{stroke,fill}%
\end{pgfscope}%
\begin{pgfscope}%
\pgfpathrectangle{\pgfqpoint{2.867647in}{0.500000in}}{\pgfqpoint{1.764706in}{1.700000in}}%
\pgfusepath{clip}%
\pgfsetbuttcap%
\pgfsetroundjoin%
\definecolor{currentfill}{rgb}{0.966120,0.744512,0.608720}%
\pgfsetfillcolor{currentfill}%
\pgfsetlinewidth{0.311001pt}%
\definecolor{currentstroke}{rgb}{1.000000,1.000000,1.000000}%
\pgfsetstrokecolor{currentstroke}%
\pgfsetdash{}{0pt}%
\pgfpathmoveto{\pgfqpoint{4.203616in}{0.992282in}}%
\pgfpathcurveto{\pgfqpoint{4.210749in}{0.992282in}}{\pgfqpoint{4.217590in}{0.995116in}}{\pgfqpoint{4.222634in}{1.000160in}}%
\pgfpathcurveto{\pgfqpoint{4.227678in}{1.005203in}}{\pgfqpoint{4.230512in}{1.012045in}}{\pgfqpoint{4.230512in}{1.019178in}}%
\pgfpathcurveto{\pgfqpoint{4.230512in}{1.026311in}}{\pgfqpoint{4.227678in}{1.033152in}}{\pgfqpoint{4.222634in}{1.038196in}}%
\pgfpathcurveto{\pgfqpoint{4.217590in}{1.043240in}}{\pgfqpoint{4.210749in}{1.046074in}}{\pgfqpoint{4.203616in}{1.046074in}}%
\pgfpathcurveto{\pgfqpoint{4.196483in}{1.046074in}}{\pgfqpoint{4.189641in}{1.043240in}}{\pgfqpoint{4.184598in}{1.038196in}}%
\pgfpathcurveto{\pgfqpoint{4.179554in}{1.033152in}}{\pgfqpoint{4.176720in}{1.026311in}}{\pgfqpoint{4.176720in}{1.019178in}}%
\pgfpathcurveto{\pgfqpoint{4.176720in}{1.012045in}}{\pgfqpoint{4.179554in}{1.005203in}}{\pgfqpoint{4.184598in}{1.000160in}}%
\pgfpathcurveto{\pgfqpoint{4.189641in}{0.995116in}}{\pgfqpoint{4.196483in}{0.992282in}}{\pgfqpoint{4.203616in}{0.992282in}}%
\pgfpathclose%
\pgfusepath{stroke,fill}%
\end{pgfscope}%
\begin{pgfscope}%
\pgfpathrectangle{\pgfqpoint{2.867647in}{0.500000in}}{\pgfqpoint{1.764706in}{1.700000in}}%
\pgfusepath{clip}%
\pgfsetbuttcap%
\pgfsetroundjoin%
\definecolor{currentfill}{rgb}{0.970255,0.815666,0.711203}%
\pgfsetfillcolor{currentfill}%
\pgfsetlinewidth{0.311001pt}%
\definecolor{currentstroke}{rgb}{1.000000,1.000000,1.000000}%
\pgfsetstrokecolor{currentstroke}%
\pgfsetdash{}{0pt}%
\pgfpathmoveto{\pgfqpoint{4.233179in}{1.542534in}}%
\pgfpathcurveto{\pgfqpoint{4.240312in}{1.542534in}}{\pgfqpoint{4.247154in}{1.545367in}}{\pgfqpoint{4.252197in}{1.550411in}}%
\pgfpathcurveto{\pgfqpoint{4.257241in}{1.555455in}}{\pgfqpoint{4.260075in}{1.562296in}}{\pgfqpoint{4.260075in}{1.569429in}}%
\pgfpathcurveto{\pgfqpoint{4.260075in}{1.576562in}}{\pgfqpoint{4.257241in}{1.583404in}}{\pgfqpoint{4.252197in}{1.588447in}}%
\pgfpathcurveto{\pgfqpoint{4.247154in}{1.593491in}}{\pgfqpoint{4.240312in}{1.596325in}}{\pgfqpoint{4.233179in}{1.596325in}}%
\pgfpathcurveto{\pgfqpoint{4.226047in}{1.596325in}}{\pgfqpoint{4.219205in}{1.593491in}}{\pgfqpoint{4.214161in}{1.588447in}}%
\pgfpathcurveto{\pgfqpoint{4.209118in}{1.583404in}}{\pgfqpoint{4.206284in}{1.576562in}}{\pgfqpoint{4.206284in}{1.569429in}}%
\pgfpathcurveto{\pgfqpoint{4.206284in}{1.562296in}}{\pgfqpoint{4.209118in}{1.555455in}}{\pgfqpoint{4.214161in}{1.550411in}}%
\pgfpathcurveto{\pgfqpoint{4.219205in}{1.545367in}}{\pgfqpoint{4.226047in}{1.542534in}}{\pgfqpoint{4.233179in}{1.542534in}}%
\pgfpathclose%
\pgfusepath{stroke,fill}%
\end{pgfscope}%
\begin{pgfscope}%
\pgfpathrectangle{\pgfqpoint{2.867647in}{0.500000in}}{\pgfqpoint{1.764706in}{1.700000in}}%
\pgfusepath{clip}%
\pgfsetbuttcap%
\pgfsetroundjoin%
\definecolor{currentfill}{rgb}{0.966812,0.762584,0.633643}%
\pgfsetfillcolor{currentfill}%
\pgfsetlinewidth{0.311001pt}%
\definecolor{currentstroke}{rgb}{1.000000,1.000000,1.000000}%
\pgfsetstrokecolor{currentstroke}%
\pgfsetdash{}{0pt}%
\pgfpathmoveto{\pgfqpoint{4.016387in}{1.630057in}}%
\pgfpathcurveto{\pgfqpoint{4.023520in}{1.630057in}}{\pgfqpoint{4.030361in}{1.632891in}}{\pgfqpoint{4.035405in}{1.637935in}}%
\pgfpathcurveto{\pgfqpoint{4.040449in}{1.642978in}}{\pgfqpoint{4.043283in}{1.649820in}}{\pgfqpoint{4.043283in}{1.656953in}}%
\pgfpathcurveto{\pgfqpoint{4.043283in}{1.664086in}}{\pgfqpoint{4.040449in}{1.670927in}}{\pgfqpoint{4.035405in}{1.675971in}}%
\pgfpathcurveto{\pgfqpoint{4.030361in}{1.681015in}}{\pgfqpoint{4.023520in}{1.683849in}}{\pgfqpoint{4.016387in}{1.683849in}}%
\pgfpathcurveto{\pgfqpoint{4.009254in}{1.683849in}}{\pgfqpoint{4.002413in}{1.681015in}}{\pgfqpoint{3.997369in}{1.675971in}}%
\pgfpathcurveto{\pgfqpoint{3.992325in}{1.670927in}}{\pgfqpoint{3.989491in}{1.664086in}}{\pgfqpoint{3.989491in}{1.656953in}}%
\pgfpathcurveto{\pgfqpoint{3.989491in}{1.649820in}}{\pgfqpoint{3.992325in}{1.642978in}}{\pgfqpoint{3.997369in}{1.637935in}}%
\pgfpathcurveto{\pgfqpoint{4.002413in}{1.632891in}}{\pgfqpoint{4.009254in}{1.630057in}}{\pgfqpoint{4.016387in}{1.630057in}}%
\pgfpathclose%
\pgfusepath{stroke,fill}%
\end{pgfscope}%
\begin{pgfscope}%
\pgfpathrectangle{\pgfqpoint{2.867647in}{0.500000in}}{\pgfqpoint{1.764706in}{1.700000in}}%
\pgfusepath{clip}%
\pgfsetbuttcap%
\pgfsetroundjoin%
\definecolor{currentfill}{rgb}{0.967398,0.774513,0.650573}%
\pgfsetfillcolor{currentfill}%
\pgfsetlinewidth{0.311001pt}%
\definecolor{currentstroke}{rgb}{1.000000,1.000000,1.000000}%
\pgfsetstrokecolor{currentstroke}%
\pgfsetdash{}{0pt}%
\pgfpathmoveto{\pgfqpoint{4.023259in}{0.989934in}}%
\pgfpathcurveto{\pgfqpoint{4.030392in}{0.989934in}}{\pgfqpoint{4.037233in}{0.992767in}}{\pgfqpoint{4.042277in}{0.997811in}}%
\pgfpathcurveto{\pgfqpoint{4.047321in}{1.002855in}}{\pgfqpoint{4.050155in}{1.009696in}}{\pgfqpoint{4.050155in}{1.016829in}}%
\pgfpathcurveto{\pgfqpoint{4.050155in}{1.023962in}}{\pgfqpoint{4.047321in}{1.030804in}}{\pgfqpoint{4.042277in}{1.035847in}}%
\pgfpathcurveto{\pgfqpoint{4.037233in}{1.040891in}}{\pgfqpoint{4.030392in}{1.043725in}}{\pgfqpoint{4.023259in}{1.043725in}}%
\pgfpathcurveto{\pgfqpoint{4.016126in}{1.043725in}}{\pgfqpoint{4.009284in}{1.040891in}}{\pgfqpoint{4.004241in}{1.035847in}}%
\pgfpathcurveto{\pgfqpoint{3.999197in}{1.030804in}}{\pgfqpoint{3.996363in}{1.023962in}}{\pgfqpoint{3.996363in}{1.016829in}}%
\pgfpathcurveto{\pgfqpoint{3.996363in}{1.009696in}}{\pgfqpoint{3.999197in}{1.002855in}}{\pgfqpoint{4.004241in}{0.997811in}}%
\pgfpathcurveto{\pgfqpoint{4.009284in}{0.992767in}}{\pgfqpoint{4.016126in}{0.989934in}}{\pgfqpoint{4.023259in}{0.989934in}}%
\pgfpathclose%
\pgfusepath{stroke,fill}%
\end{pgfscope}%
\begin{pgfscope}%
\pgfpathrectangle{\pgfqpoint{2.867647in}{0.500000in}}{\pgfqpoint{1.764706in}{1.700000in}}%
\pgfusepath{clip}%
\pgfsetbuttcap%
\pgfsetroundjoin%
\definecolor{currentfill}{rgb}{0.972726,0.844889,0.754401}%
\pgfsetfillcolor{currentfill}%
\pgfsetlinewidth{0.311001pt}%
\definecolor{currentstroke}{rgb}{1.000000,1.000000,1.000000}%
\pgfsetstrokecolor{currentstroke}%
\pgfsetdash{}{0pt}%
\pgfpathmoveto{\pgfqpoint{4.133322in}{0.990433in}}%
\pgfpathcurveto{\pgfqpoint{4.140455in}{0.990433in}}{\pgfqpoint{4.147297in}{0.993267in}}{\pgfqpoint{4.152341in}{0.998311in}}%
\pgfpathcurveto{\pgfqpoint{4.157384in}{1.003354in}}{\pgfqpoint{4.160218in}{1.010196in}}{\pgfqpoint{4.160218in}{1.017329in}}%
\pgfpathcurveto{\pgfqpoint{4.160218in}{1.024461in}}{\pgfqpoint{4.157384in}{1.031303in}}{\pgfqpoint{4.152341in}{1.036347in}}%
\pgfpathcurveto{\pgfqpoint{4.147297in}{1.041390in}}{\pgfqpoint{4.140455in}{1.044224in}}{\pgfqpoint{4.133322in}{1.044224in}}%
\pgfpathcurveto{\pgfqpoint{4.126190in}{1.044224in}}{\pgfqpoint{4.119348in}{1.041390in}}{\pgfqpoint{4.114304in}{1.036347in}}%
\pgfpathcurveto{\pgfqpoint{4.109261in}{1.031303in}}{\pgfqpoint{4.106427in}{1.024461in}}{\pgfqpoint{4.106427in}{1.017329in}}%
\pgfpathcurveto{\pgfqpoint{4.106427in}{1.010196in}}{\pgfqpoint{4.109261in}{1.003354in}}{\pgfqpoint{4.114304in}{0.998311in}}%
\pgfpathcurveto{\pgfqpoint{4.119348in}{0.993267in}}{\pgfqpoint{4.126190in}{0.990433in}}{\pgfqpoint{4.133322in}{0.990433in}}%
\pgfpathclose%
\pgfusepath{stroke,fill}%
\end{pgfscope}%
\begin{pgfscope}%
\pgfpathrectangle{\pgfqpoint{2.867647in}{0.500000in}}{\pgfqpoint{1.764706in}{1.700000in}}%
\pgfusepath{clip}%
\pgfsetbuttcap%
\pgfsetroundjoin%
\definecolor{currentfill}{rgb}{0.967735,0.780441,0.659127}%
\pgfsetfillcolor{currentfill}%
\pgfsetlinewidth{0.311001pt}%
\definecolor{currentstroke}{rgb}{1.000000,1.000000,1.000000}%
\pgfsetstrokecolor{currentstroke}%
\pgfsetdash{}{0pt}%
\pgfpathmoveto{\pgfqpoint{4.052426in}{1.717865in}}%
\pgfpathcurveto{\pgfqpoint{4.059559in}{1.717865in}}{\pgfqpoint{4.066401in}{1.720698in}}{\pgfqpoint{4.071444in}{1.725742in}}%
\pgfpathcurveto{\pgfqpoint{4.076488in}{1.730786in}}{\pgfqpoint{4.079322in}{1.737627in}}{\pgfqpoint{4.079322in}{1.744760in}}%
\pgfpathcurveto{\pgfqpoint{4.079322in}{1.751893in}}{\pgfqpoint{4.076488in}{1.758735in}}{\pgfqpoint{4.071444in}{1.763778in}}%
\pgfpathcurveto{\pgfqpoint{4.066401in}{1.768822in}}{\pgfqpoint{4.059559in}{1.771656in}}{\pgfqpoint{4.052426in}{1.771656in}}%
\pgfpathcurveto{\pgfqpoint{4.045293in}{1.771656in}}{\pgfqpoint{4.038452in}{1.768822in}}{\pgfqpoint{4.033408in}{1.763778in}}%
\pgfpathcurveto{\pgfqpoint{4.028364in}{1.758735in}}{\pgfqpoint{4.025530in}{1.751893in}}{\pgfqpoint{4.025530in}{1.744760in}}%
\pgfpathcurveto{\pgfqpoint{4.025530in}{1.737627in}}{\pgfqpoint{4.028364in}{1.730786in}}{\pgfqpoint{4.033408in}{1.725742in}}%
\pgfpathcurveto{\pgfqpoint{4.038452in}{1.720698in}}{\pgfqpoint{4.045293in}{1.717865in}}{\pgfqpoint{4.052426in}{1.717865in}}%
\pgfpathclose%
\pgfusepath{stroke,fill}%
\end{pgfscope}%
\begin{pgfscope}%
\pgfpathrectangle{\pgfqpoint{2.867647in}{0.500000in}}{\pgfqpoint{1.764706in}{1.700000in}}%
\pgfusepath{clip}%
\pgfsetbuttcap%
\pgfsetroundjoin%
\definecolor{currentfill}{rgb}{0.976287,0.879862,0.805788}%
\pgfsetfillcolor{currentfill}%
\pgfsetlinewidth{0.311001pt}%
\definecolor{currentstroke}{rgb}{1.000000,1.000000,1.000000}%
\pgfsetstrokecolor{currentstroke}%
\pgfsetdash{}{0pt}%
\pgfpathmoveto{\pgfqpoint{4.131577in}{1.417050in}}%
\pgfpathcurveto{\pgfqpoint{4.138710in}{1.417050in}}{\pgfqpoint{4.145552in}{1.419884in}}{\pgfqpoint{4.150596in}{1.424928in}}%
\pgfpathcurveto{\pgfqpoint{4.155639in}{1.429971in}}{\pgfqpoint{4.158473in}{1.436813in}}{\pgfqpoint{4.158473in}{1.443946in}}%
\pgfpathcurveto{\pgfqpoint{4.158473in}{1.451079in}}{\pgfqpoint{4.155639in}{1.457920in}}{\pgfqpoint{4.150596in}{1.462964in}}%
\pgfpathcurveto{\pgfqpoint{4.145552in}{1.468008in}}{\pgfqpoint{4.138710in}{1.470842in}}{\pgfqpoint{4.131577in}{1.470842in}}%
\pgfpathcurveto{\pgfqpoint{4.124445in}{1.470842in}}{\pgfqpoint{4.117603in}{1.468008in}}{\pgfqpoint{4.112559in}{1.462964in}}%
\pgfpathcurveto{\pgfqpoint{4.107516in}{1.457920in}}{\pgfqpoint{4.104682in}{1.451079in}}{\pgfqpoint{4.104682in}{1.443946in}}%
\pgfpathcurveto{\pgfqpoint{4.104682in}{1.436813in}}{\pgfqpoint{4.107516in}{1.429971in}}{\pgfqpoint{4.112559in}{1.424928in}}%
\pgfpathcurveto{\pgfqpoint{4.117603in}{1.419884in}}{\pgfqpoint{4.124445in}{1.417050in}}{\pgfqpoint{4.131577in}{1.417050in}}%
\pgfpathclose%
\pgfusepath{stroke,fill}%
\end{pgfscope}%
\begin{pgfscope}%
\pgfpathrectangle{\pgfqpoint{2.867647in}{0.500000in}}{\pgfqpoint{1.764706in}{1.700000in}}%
\pgfusepath{clip}%
\pgfsetbuttcap%
\pgfsetroundjoin%
\definecolor{currentfill}{rgb}{0.979124,0.903132,0.839793}%
\pgfsetfillcolor{currentfill}%
\pgfsetlinewidth{0.311001pt}%
\definecolor{currentstroke}{rgb}{1.000000,1.000000,1.000000}%
\pgfsetstrokecolor{currentstroke}%
\pgfsetdash{}{0pt}%
\pgfpathmoveto{\pgfqpoint{4.210953in}{1.429948in}}%
\pgfpathcurveto{\pgfqpoint{4.218086in}{1.429948in}}{\pgfqpoint{4.224927in}{1.432782in}}{\pgfqpoint{4.229971in}{1.437826in}}%
\pgfpathcurveto{\pgfqpoint{4.235015in}{1.442869in}}{\pgfqpoint{4.237849in}{1.449711in}}{\pgfqpoint{4.237849in}{1.456844in}}%
\pgfpathcurveto{\pgfqpoint{4.237849in}{1.463977in}}{\pgfqpoint{4.235015in}{1.470818in}}{\pgfqpoint{4.229971in}{1.475862in}}%
\pgfpathcurveto{\pgfqpoint{4.224927in}{1.480906in}}{\pgfqpoint{4.218086in}{1.483740in}}{\pgfqpoint{4.210953in}{1.483740in}}%
\pgfpathcurveto{\pgfqpoint{4.203820in}{1.483740in}}{\pgfqpoint{4.196979in}{1.480906in}}{\pgfqpoint{4.191935in}{1.475862in}}%
\pgfpathcurveto{\pgfqpoint{4.186891in}{1.470818in}}{\pgfqpoint{4.184057in}{1.463977in}}{\pgfqpoint{4.184057in}{1.456844in}}%
\pgfpathcurveto{\pgfqpoint{4.184057in}{1.449711in}}{\pgfqpoint{4.186891in}{1.442869in}}{\pgfqpoint{4.191935in}{1.437826in}}%
\pgfpathcurveto{\pgfqpoint{4.196979in}{1.432782in}}{\pgfqpoint{4.203820in}{1.429948in}}{\pgfqpoint{4.210953in}{1.429948in}}%
\pgfpathclose%
\pgfusepath{stroke,fill}%
\end{pgfscope}%
\begin{pgfscope}%
\pgfpathrectangle{\pgfqpoint{2.867647in}{0.500000in}}{\pgfqpoint{1.764706in}{1.700000in}}%
\pgfusepath{clip}%
\pgfsetbuttcap%
\pgfsetroundjoin%
\definecolor{currentfill}{rgb}{0.967398,0.774513,0.650573}%
\pgfsetfillcolor{currentfill}%
\pgfsetlinewidth{0.311001pt}%
\definecolor{currentstroke}{rgb}{1.000000,1.000000,1.000000}%
\pgfsetstrokecolor{currentstroke}%
\pgfsetdash{}{0pt}%
\pgfpathmoveto{\pgfqpoint{4.020874in}{1.660409in}}%
\pgfpathcurveto{\pgfqpoint{4.028007in}{1.660409in}}{\pgfqpoint{4.034849in}{1.663243in}}{\pgfqpoint{4.039892in}{1.668286in}}%
\pgfpathcurveto{\pgfqpoint{4.044936in}{1.673330in}}{\pgfqpoint{4.047770in}{1.680172in}}{\pgfqpoint{4.047770in}{1.687305in}}%
\pgfpathcurveto{\pgfqpoint{4.047770in}{1.694437in}}{\pgfqpoint{4.044936in}{1.701279in}}{\pgfqpoint{4.039892in}{1.706323in}}%
\pgfpathcurveto{\pgfqpoint{4.034849in}{1.711366in}}{\pgfqpoint{4.028007in}{1.714200in}}{\pgfqpoint{4.020874in}{1.714200in}}%
\pgfpathcurveto{\pgfqpoint{4.013741in}{1.714200in}}{\pgfqpoint{4.006900in}{1.711366in}}{\pgfqpoint{4.001856in}{1.706323in}}%
\pgfpathcurveto{\pgfqpoint{3.996812in}{1.701279in}}{\pgfqpoint{3.993978in}{1.694437in}}{\pgfqpoint{3.993978in}{1.687305in}}%
\pgfpathcurveto{\pgfqpoint{3.993978in}{1.680172in}}{\pgfqpoint{3.996812in}{1.673330in}}{\pgfqpoint{4.001856in}{1.668286in}}%
\pgfpathcurveto{\pgfqpoint{4.006900in}{1.663243in}}{\pgfqpoint{4.013741in}{1.660409in}}{\pgfqpoint{4.020874in}{1.660409in}}%
\pgfpathclose%
\pgfusepath{stroke,fill}%
\end{pgfscope}%
\begin{pgfscope}%
\pgfpathrectangle{\pgfqpoint{2.867647in}{0.500000in}}{\pgfqpoint{1.764706in}{1.700000in}}%
\pgfusepath{clip}%
\pgfsetbuttcap%
\pgfsetroundjoin%
\definecolor{currentfill}{rgb}{0.970255,0.815666,0.711203}%
\pgfsetfillcolor{currentfill}%
\pgfsetlinewidth{0.311001pt}%
\definecolor{currentstroke}{rgb}{1.000000,1.000000,1.000000}%
\pgfsetstrokecolor{currentstroke}%
\pgfsetdash{}{0pt}%
\pgfpathmoveto{\pgfqpoint{4.165273in}{0.991705in}}%
\pgfpathcurveto{\pgfqpoint{4.172406in}{0.991705in}}{\pgfqpoint{4.179248in}{0.994539in}}{\pgfqpoint{4.184291in}{0.999583in}}%
\pgfpathcurveto{\pgfqpoint{4.189335in}{1.004627in}}{\pgfqpoint{4.192169in}{1.011468in}}{\pgfqpoint{4.192169in}{1.018601in}}%
\pgfpathcurveto{\pgfqpoint{4.192169in}{1.025734in}}{\pgfqpoint{4.189335in}{1.032576in}}{\pgfqpoint{4.184291in}{1.037619in}}%
\pgfpathcurveto{\pgfqpoint{4.179248in}{1.042663in}}{\pgfqpoint{4.172406in}{1.045497in}}{\pgfqpoint{4.165273in}{1.045497in}}%
\pgfpathcurveto{\pgfqpoint{4.158140in}{1.045497in}}{\pgfqpoint{4.151299in}{1.042663in}}{\pgfqpoint{4.146255in}{1.037619in}}%
\pgfpathcurveto{\pgfqpoint{4.141212in}{1.032576in}}{\pgfqpoint{4.138378in}{1.025734in}}{\pgfqpoint{4.138378in}{1.018601in}}%
\pgfpathcurveto{\pgfqpoint{4.138378in}{1.011468in}}{\pgfqpoint{4.141212in}{1.004627in}}{\pgfqpoint{4.146255in}{0.999583in}}%
\pgfpathcurveto{\pgfqpoint{4.151299in}{0.994539in}}{\pgfqpoint{4.158140in}{0.991705in}}{\pgfqpoint{4.165273in}{0.991705in}}%
\pgfpathclose%
\pgfusepath{stroke,fill}%
\end{pgfscope}%
\begin{pgfscope}%
\pgfpathrectangle{\pgfqpoint{2.867647in}{0.500000in}}{\pgfqpoint{1.764706in}{1.700000in}}%
\pgfusepath{clip}%
\pgfsetbuttcap%
\pgfsetroundjoin%
\definecolor{currentfill}{rgb}{0.968931,0.798091,0.685123}%
\pgfsetfillcolor{currentfill}%
\pgfsetlinewidth{0.311001pt}%
\definecolor{currentstroke}{rgb}{1.000000,1.000000,1.000000}%
\pgfsetstrokecolor{currentstroke}%
\pgfsetdash{}{0pt}%
\pgfpathmoveto{\pgfqpoint{4.239015in}{1.550307in}}%
\pgfpathcurveto{\pgfqpoint{4.246147in}{1.550307in}}{\pgfqpoint{4.252989in}{1.553141in}}{\pgfqpoint{4.258033in}{1.558184in}}%
\pgfpathcurveto{\pgfqpoint{4.263076in}{1.563228in}}{\pgfqpoint{4.265910in}{1.570069in}}{\pgfqpoint{4.265910in}{1.577202in}}%
\pgfpathcurveto{\pgfqpoint{4.265910in}{1.584335in}}{\pgfqpoint{4.263076in}{1.591177in}}{\pgfqpoint{4.258033in}{1.596220in}}%
\pgfpathcurveto{\pgfqpoint{4.252989in}{1.601264in}}{\pgfqpoint{4.246147in}{1.604098in}}{\pgfqpoint{4.239015in}{1.604098in}}%
\pgfpathcurveto{\pgfqpoint{4.231882in}{1.604098in}}{\pgfqpoint{4.225040in}{1.601264in}}{\pgfqpoint{4.219996in}{1.596220in}}%
\pgfpathcurveto{\pgfqpoint{4.214953in}{1.591177in}}{\pgfqpoint{4.212119in}{1.584335in}}{\pgfqpoint{4.212119in}{1.577202in}}%
\pgfpathcurveto{\pgfqpoint{4.212119in}{1.570069in}}{\pgfqpoint{4.214953in}{1.563228in}}{\pgfqpoint{4.219996in}{1.558184in}}%
\pgfpathcurveto{\pgfqpoint{4.225040in}{1.553141in}}{\pgfqpoint{4.231882in}{1.550307in}}{\pgfqpoint{4.239015in}{1.550307in}}%
\pgfpathclose%
\pgfusepath{stroke,fill}%
\end{pgfscope}%
\begin{pgfscope}%
\pgfpathrectangle{\pgfqpoint{2.867647in}{0.500000in}}{\pgfqpoint{1.764706in}{1.700000in}}%
\pgfusepath{clip}%
\pgfsetbuttcap%
\pgfsetroundjoin%
\definecolor{currentfill}{rgb}{0.966120,0.744512,0.608720}%
\pgfsetfillcolor{currentfill}%
\pgfsetlinewidth{0.311001pt}%
\definecolor{currentstroke}{rgb}{1.000000,1.000000,1.000000}%
\pgfsetstrokecolor{currentstroke}%
\pgfsetdash{}{0pt}%
\pgfpathmoveto{\pgfqpoint{4.028121in}{1.726294in}}%
\pgfpathcurveto{\pgfqpoint{4.035253in}{1.726294in}}{\pgfqpoint{4.042095in}{1.729128in}}{\pgfqpoint{4.047139in}{1.734172in}}%
\pgfpathcurveto{\pgfqpoint{4.052182in}{1.739216in}}{\pgfqpoint{4.055016in}{1.746057in}}{\pgfqpoint{4.055016in}{1.753190in}}%
\pgfpathcurveto{\pgfqpoint{4.055016in}{1.760323in}}{\pgfqpoint{4.052182in}{1.767165in}}{\pgfqpoint{4.047139in}{1.772208in}}%
\pgfpathcurveto{\pgfqpoint{4.042095in}{1.777252in}}{\pgfqpoint{4.035253in}{1.780086in}}{\pgfqpoint{4.028121in}{1.780086in}}%
\pgfpathcurveto{\pgfqpoint{4.020988in}{1.780086in}}{\pgfqpoint{4.014146in}{1.777252in}}{\pgfqpoint{4.009102in}{1.772208in}}%
\pgfpathcurveto{\pgfqpoint{4.004059in}{1.767165in}}{\pgfqpoint{4.001225in}{1.760323in}}{\pgfqpoint{4.001225in}{1.753190in}}%
\pgfpathcurveto{\pgfqpoint{4.001225in}{1.746057in}}{\pgfqpoint{4.004059in}{1.739216in}}{\pgfqpoint{4.009102in}{1.734172in}}%
\pgfpathcurveto{\pgfqpoint{4.014146in}{1.729128in}}{\pgfqpoint{4.020988in}{1.726294in}}{\pgfqpoint{4.028121in}{1.726294in}}%
\pgfpathclose%
\pgfusepath{stroke,fill}%
\end{pgfscope}%
\begin{pgfscope}%
\pgfpathrectangle{\pgfqpoint{2.867647in}{0.500000in}}{\pgfqpoint{1.764706in}{1.700000in}}%
\pgfusepath{clip}%
\pgfsetbuttcap%
\pgfsetroundjoin%
\definecolor{currentfill}{rgb}{0.980678,0.914765,0.856766}%
\pgfsetfillcolor{currentfill}%
\pgfsetlinewidth{0.311001pt}%
\definecolor{currentstroke}{rgb}{1.000000,1.000000,1.000000}%
\pgfsetstrokecolor{currentstroke}%
\pgfsetdash{}{0pt}%
\pgfpathmoveto{\pgfqpoint{4.209343in}{1.328997in}}%
\pgfpathcurveto{\pgfqpoint{4.216476in}{1.328997in}}{\pgfqpoint{4.223318in}{1.331831in}}{\pgfqpoint{4.228361in}{1.336875in}}%
\pgfpathcurveto{\pgfqpoint{4.233405in}{1.341919in}}{\pgfqpoint{4.236239in}{1.348760in}}{\pgfqpoint{4.236239in}{1.355893in}}%
\pgfpathcurveto{\pgfqpoint{4.236239in}{1.363026in}}{\pgfqpoint{4.233405in}{1.369868in}}{\pgfqpoint{4.228361in}{1.374911in}}%
\pgfpathcurveto{\pgfqpoint{4.223318in}{1.379955in}}{\pgfqpoint{4.216476in}{1.382789in}}{\pgfqpoint{4.209343in}{1.382789in}}%
\pgfpathcurveto{\pgfqpoint{4.202210in}{1.382789in}}{\pgfqpoint{4.195369in}{1.379955in}}{\pgfqpoint{4.190325in}{1.374911in}}%
\pgfpathcurveto{\pgfqpoint{4.185281in}{1.369868in}}{\pgfqpoint{4.182447in}{1.363026in}}{\pgfqpoint{4.182447in}{1.355893in}}%
\pgfpathcurveto{\pgfqpoint{4.182447in}{1.348760in}}{\pgfqpoint{4.185281in}{1.341919in}}{\pgfqpoint{4.190325in}{1.336875in}}%
\pgfpathcurveto{\pgfqpoint{4.195369in}{1.331831in}}{\pgfqpoint{4.202210in}{1.328997in}}{\pgfqpoint{4.209343in}{1.328997in}}%
\pgfpathclose%
\pgfusepath{stroke,fill}%
\end{pgfscope}%
\begin{pgfscope}%
\pgfpathrectangle{\pgfqpoint{2.867647in}{0.500000in}}{\pgfqpoint{1.764706in}{1.700000in}}%
\pgfusepath{clip}%
\pgfsetbuttcap%
\pgfsetroundjoin%
\definecolor{currentfill}{rgb}{0.976961,0.885681,0.814303}%
\pgfsetfillcolor{currentfill}%
\pgfsetlinewidth{0.311001pt}%
\definecolor{currentstroke}{rgb}{1.000000,1.000000,1.000000}%
\pgfsetstrokecolor{currentstroke}%
\pgfsetdash{}{0pt}%
\pgfpathmoveto{\pgfqpoint{4.108641in}{1.044716in}}%
\pgfpathcurveto{\pgfqpoint{4.115774in}{1.044716in}}{\pgfqpoint{4.122616in}{1.047550in}}{\pgfqpoint{4.127659in}{1.052594in}}%
\pgfpathcurveto{\pgfqpoint{4.132703in}{1.057637in}}{\pgfqpoint{4.135537in}{1.064479in}}{\pgfqpoint{4.135537in}{1.071612in}}%
\pgfpathcurveto{\pgfqpoint{4.135537in}{1.078745in}}{\pgfqpoint{4.132703in}{1.085586in}}{\pgfqpoint{4.127659in}{1.090630in}}%
\pgfpathcurveto{\pgfqpoint{4.122616in}{1.095674in}}{\pgfqpoint{4.115774in}{1.098508in}}{\pgfqpoint{4.108641in}{1.098508in}}%
\pgfpathcurveto{\pgfqpoint{4.101508in}{1.098508in}}{\pgfqpoint{4.094667in}{1.095674in}}{\pgfqpoint{4.089623in}{1.090630in}}%
\pgfpathcurveto{\pgfqpoint{4.084579in}{1.085586in}}{\pgfqpoint{4.081745in}{1.078745in}}{\pgfqpoint{4.081745in}{1.071612in}}%
\pgfpathcurveto{\pgfqpoint{4.081745in}{1.064479in}}{\pgfqpoint{4.084579in}{1.057637in}}{\pgfqpoint{4.089623in}{1.052594in}}%
\pgfpathcurveto{\pgfqpoint{4.094667in}{1.047550in}}{\pgfqpoint{4.101508in}{1.044716in}}{\pgfqpoint{4.108641in}{1.044716in}}%
\pgfpathclose%
\pgfusepath{stroke,fill}%
\end{pgfscope}%
\begin{pgfscope}%
\pgfpathrectangle{\pgfqpoint{2.867647in}{0.500000in}}{\pgfqpoint{1.764706in}{1.700000in}}%
\pgfusepath{clip}%
\pgfsetbuttcap%
\pgfsetroundjoin%
\definecolor{currentfill}{rgb}{0.979124,0.903132,0.839793}%
\pgfsetfillcolor{currentfill}%
\pgfsetlinewidth{0.311001pt}%
\definecolor{currentstroke}{rgb}{1.000000,1.000000,1.000000}%
\pgfsetstrokecolor{currentstroke}%
\pgfsetdash{}{0pt}%
\pgfpathmoveto{\pgfqpoint{4.119537in}{1.537848in}}%
\pgfpathcurveto{\pgfqpoint{4.126670in}{1.537848in}}{\pgfqpoint{4.133512in}{1.540682in}}{\pgfqpoint{4.138555in}{1.545726in}}%
\pgfpathcurveto{\pgfqpoint{4.143599in}{1.550770in}}{\pgfqpoint{4.146433in}{1.557611in}}{\pgfqpoint{4.146433in}{1.564744in}}%
\pgfpathcurveto{\pgfqpoint{4.146433in}{1.571877in}}{\pgfqpoint{4.143599in}{1.578719in}}{\pgfqpoint{4.138555in}{1.583762in}}%
\pgfpathcurveto{\pgfqpoint{4.133512in}{1.588806in}}{\pgfqpoint{4.126670in}{1.591640in}}{\pgfqpoint{4.119537in}{1.591640in}}%
\pgfpathcurveto{\pgfqpoint{4.112404in}{1.591640in}}{\pgfqpoint{4.105563in}{1.588806in}}{\pgfqpoint{4.100519in}{1.583762in}}%
\pgfpathcurveto{\pgfqpoint{4.095475in}{1.578719in}}{\pgfqpoint{4.092642in}{1.571877in}}{\pgfqpoint{4.092642in}{1.564744in}}%
\pgfpathcurveto{\pgfqpoint{4.092642in}{1.557611in}}{\pgfqpoint{4.095475in}{1.550770in}}{\pgfqpoint{4.100519in}{1.545726in}}%
\pgfpathcurveto{\pgfqpoint{4.105563in}{1.540682in}}{\pgfqpoint{4.112404in}{1.537848in}}{\pgfqpoint{4.119537in}{1.537848in}}%
\pgfpathclose%
\pgfusepath{stroke,fill}%
\end{pgfscope}%
\begin{pgfscope}%
\pgfpathrectangle{\pgfqpoint{2.867647in}{0.500000in}}{\pgfqpoint{1.764706in}{1.700000in}}%
\pgfusepath{clip}%
\pgfsetbuttcap%
\pgfsetroundjoin%
\definecolor{currentfill}{rgb}{0.977657,0.891500,0.822809}%
\pgfsetfillcolor{currentfill}%
\pgfsetlinewidth{0.311001pt}%
\definecolor{currentstroke}{rgb}{1.000000,1.000000,1.000000}%
\pgfsetstrokecolor{currentstroke}%
\pgfsetdash{}{0pt}%
\pgfpathmoveto{\pgfqpoint{4.148402in}{1.291462in}}%
\pgfpathcurveto{\pgfqpoint{4.155535in}{1.291462in}}{\pgfqpoint{4.162377in}{1.294295in}}{\pgfqpoint{4.167420in}{1.299339in}}%
\pgfpathcurveto{\pgfqpoint{4.172464in}{1.304383in}}{\pgfqpoint{4.175298in}{1.311224in}}{\pgfqpoint{4.175298in}{1.318357in}}%
\pgfpathcurveto{\pgfqpoint{4.175298in}{1.325490in}}{\pgfqpoint{4.172464in}{1.332332in}}{\pgfqpoint{4.167420in}{1.337375in}}%
\pgfpathcurveto{\pgfqpoint{4.162377in}{1.342419in}}{\pgfqpoint{4.155535in}{1.345253in}}{\pgfqpoint{4.148402in}{1.345253in}}%
\pgfpathcurveto{\pgfqpoint{4.141269in}{1.345253in}}{\pgfqpoint{4.134428in}{1.342419in}}{\pgfqpoint{4.129384in}{1.337375in}}%
\pgfpathcurveto{\pgfqpoint{4.124340in}{1.332332in}}{\pgfqpoint{4.121506in}{1.325490in}}{\pgfqpoint{4.121506in}{1.318357in}}%
\pgfpathcurveto{\pgfqpoint{4.121506in}{1.311224in}}{\pgfqpoint{4.124340in}{1.304383in}}{\pgfqpoint{4.129384in}{1.299339in}}%
\pgfpathcurveto{\pgfqpoint{4.134428in}{1.294295in}}{\pgfqpoint{4.141269in}{1.291462in}}{\pgfqpoint{4.148402in}{1.291462in}}%
\pgfpathclose%
\pgfusepath{stroke,fill}%
\end{pgfscope}%
\begin{pgfscope}%
\pgfpathrectangle{\pgfqpoint{2.867647in}{0.500000in}}{\pgfqpoint{1.764706in}{1.700000in}}%
\pgfusepath{clip}%
\pgfsetbuttcap%
\pgfsetroundjoin%
\definecolor{currentfill}{rgb}{0.975644,0.874038,0.797253}%
\pgfsetfillcolor{currentfill}%
\pgfsetlinewidth{0.311001pt}%
\definecolor{currentstroke}{rgb}{1.000000,1.000000,1.000000}%
\pgfsetstrokecolor{currentstroke}%
\pgfsetdash{}{0pt}%
\pgfpathmoveto{\pgfqpoint{4.129674in}{1.402207in}}%
\pgfpathcurveto{\pgfqpoint{4.136807in}{1.402207in}}{\pgfqpoint{4.143649in}{1.405041in}}{\pgfqpoint{4.148692in}{1.410085in}}%
\pgfpathcurveto{\pgfqpoint{4.153736in}{1.415128in}}{\pgfqpoint{4.156570in}{1.421970in}}{\pgfqpoint{4.156570in}{1.429103in}}%
\pgfpathcurveto{\pgfqpoint{4.156570in}{1.436236in}}{\pgfqpoint{4.153736in}{1.443077in}}{\pgfqpoint{4.148692in}{1.448121in}}%
\pgfpathcurveto{\pgfqpoint{4.143649in}{1.453165in}}{\pgfqpoint{4.136807in}{1.455999in}}{\pgfqpoint{4.129674in}{1.455999in}}%
\pgfpathcurveto{\pgfqpoint{4.122541in}{1.455999in}}{\pgfqpoint{4.115700in}{1.453165in}}{\pgfqpoint{4.110656in}{1.448121in}}%
\pgfpathcurveto{\pgfqpoint{4.105612in}{1.443077in}}{\pgfqpoint{4.102779in}{1.436236in}}{\pgfqpoint{4.102779in}{1.429103in}}%
\pgfpathcurveto{\pgfqpoint{4.102779in}{1.421970in}}{\pgfqpoint{4.105612in}{1.415128in}}{\pgfqpoint{4.110656in}{1.410085in}}%
\pgfpathcurveto{\pgfqpoint{4.115700in}{1.405041in}}{\pgfqpoint{4.122541in}{1.402207in}}{\pgfqpoint{4.129674in}{1.402207in}}%
\pgfpathclose%
\pgfusepath{stroke,fill}%
\end{pgfscope}%
\begin{pgfscope}%
\pgfpathrectangle{\pgfqpoint{2.867647in}{0.500000in}}{\pgfqpoint{1.764706in}{1.700000in}}%
\pgfusepath{clip}%
\pgfsetbuttcap%
\pgfsetroundjoin%
\definecolor{currentfill}{rgb}{0.969803,0.809811,0.702523}%
\pgfsetfillcolor{currentfill}%
\pgfsetlinewidth{0.311001pt}%
\definecolor{currentstroke}{rgb}{1.000000,1.000000,1.000000}%
\pgfsetstrokecolor{currentstroke}%
\pgfsetdash{}{0pt}%
\pgfpathmoveto{\pgfqpoint{4.107307in}{0.950615in}}%
\pgfpathcurveto{\pgfqpoint{4.114440in}{0.950615in}}{\pgfqpoint{4.121282in}{0.953449in}}{\pgfqpoint{4.126325in}{0.958493in}}%
\pgfpathcurveto{\pgfqpoint{4.131369in}{0.963536in}}{\pgfqpoint{4.134203in}{0.970378in}}{\pgfqpoint{4.134203in}{0.977511in}}%
\pgfpathcurveto{\pgfqpoint{4.134203in}{0.984644in}}{\pgfqpoint{4.131369in}{0.991485in}}{\pgfqpoint{4.126325in}{0.996529in}}%
\pgfpathcurveto{\pgfqpoint{4.121282in}{1.001573in}}{\pgfqpoint{4.114440in}{1.004407in}}{\pgfqpoint{4.107307in}{1.004407in}}%
\pgfpathcurveto{\pgfqpoint{4.100174in}{1.004407in}}{\pgfqpoint{4.093333in}{1.001573in}}{\pgfqpoint{4.088289in}{0.996529in}}%
\pgfpathcurveto{\pgfqpoint{4.083245in}{0.991485in}}{\pgfqpoint{4.080411in}{0.984644in}}{\pgfqpoint{4.080411in}{0.977511in}}%
\pgfpathcurveto{\pgfqpoint{4.080411in}{0.970378in}}{\pgfqpoint{4.083245in}{0.963536in}}{\pgfqpoint{4.088289in}{0.958493in}}%
\pgfpathcurveto{\pgfqpoint{4.093333in}{0.953449in}}{\pgfqpoint{4.100174in}{0.950615in}}{\pgfqpoint{4.107307in}{0.950615in}}%
\pgfpathclose%
\pgfusepath{stroke,fill}%
\end{pgfscope}%
\begin{pgfscope}%
\pgfpathrectangle{\pgfqpoint{2.867647in}{0.500000in}}{\pgfqpoint{1.764706in}{1.700000in}}%
\pgfusepath{clip}%
\pgfsetbuttcap%
\pgfsetroundjoin%
\definecolor{currentfill}{rgb}{0.975018,0.868213,0.788710}%
\pgfsetfillcolor{currentfill}%
\pgfsetlinewidth{0.311001pt}%
\definecolor{currentstroke}{rgb}{1.000000,1.000000,1.000000}%
\pgfsetstrokecolor{currentstroke}%
\pgfsetdash{}{0pt}%
\pgfpathmoveto{\pgfqpoint{4.089462in}{1.025155in}}%
\pgfpathcurveto{\pgfqpoint{4.096595in}{1.025155in}}{\pgfqpoint{4.103436in}{1.027989in}}{\pgfqpoint{4.108480in}{1.033033in}}%
\pgfpathcurveto{\pgfqpoint{4.113524in}{1.038077in}}{\pgfqpoint{4.116358in}{1.044918in}}{\pgfqpoint{4.116358in}{1.052051in}}%
\pgfpathcurveto{\pgfqpoint{4.116358in}{1.059184in}}{\pgfqpoint{4.113524in}{1.066026in}}{\pgfqpoint{4.108480in}{1.071069in}}%
\pgfpathcurveto{\pgfqpoint{4.103436in}{1.076113in}}{\pgfqpoint{4.096595in}{1.078947in}}{\pgfqpoint{4.089462in}{1.078947in}}%
\pgfpathcurveto{\pgfqpoint{4.082329in}{1.078947in}}{\pgfqpoint{4.075487in}{1.076113in}}{\pgfqpoint{4.070444in}{1.071069in}}%
\pgfpathcurveto{\pgfqpoint{4.065400in}{1.066026in}}{\pgfqpoint{4.062566in}{1.059184in}}{\pgfqpoint{4.062566in}{1.052051in}}%
\pgfpathcurveto{\pgfqpoint{4.062566in}{1.044918in}}{\pgfqpoint{4.065400in}{1.038077in}}{\pgfqpoint{4.070444in}{1.033033in}}%
\pgfpathcurveto{\pgfqpoint{4.075487in}{1.027989in}}{\pgfqpoint{4.082329in}{1.025155in}}{\pgfqpoint{4.089462in}{1.025155in}}%
\pgfpathclose%
\pgfusepath{stroke,fill}%
\end{pgfscope}%
\begin{pgfscope}%
\pgfpathrectangle{\pgfqpoint{2.867647in}{0.500000in}}{\pgfqpoint{1.764706in}{1.700000in}}%
\pgfusepath{clip}%
\pgfsetbuttcap%
\pgfsetroundjoin%
\definecolor{currentfill}{rgb}{0.980678,0.914765,0.856766}%
\pgfsetfillcolor{currentfill}%
\pgfsetlinewidth{0.311001pt}%
\definecolor{currentstroke}{rgb}{1.000000,1.000000,1.000000}%
\pgfsetstrokecolor{currentstroke}%
\pgfsetdash{}{0pt}%
\pgfpathmoveto{\pgfqpoint{4.142892in}{1.540276in}}%
\pgfpathcurveto{\pgfqpoint{4.150025in}{1.540276in}}{\pgfqpoint{4.156867in}{1.543110in}}{\pgfqpoint{4.161911in}{1.548154in}}%
\pgfpathcurveto{\pgfqpoint{4.166954in}{1.553198in}}{\pgfqpoint{4.169788in}{1.560039in}}{\pgfqpoint{4.169788in}{1.567172in}}%
\pgfpathcurveto{\pgfqpoint{4.169788in}{1.574305in}}{\pgfqpoint{4.166954in}{1.581147in}}{\pgfqpoint{4.161911in}{1.586190in}}%
\pgfpathcurveto{\pgfqpoint{4.156867in}{1.591234in}}{\pgfqpoint{4.150025in}{1.594068in}}{\pgfqpoint{4.142892in}{1.594068in}}%
\pgfpathcurveto{\pgfqpoint{4.135760in}{1.594068in}}{\pgfqpoint{4.128918in}{1.591234in}}{\pgfqpoint{4.123874in}{1.586190in}}%
\pgfpathcurveto{\pgfqpoint{4.118831in}{1.581147in}}{\pgfqpoint{4.115997in}{1.574305in}}{\pgfqpoint{4.115997in}{1.567172in}}%
\pgfpathcurveto{\pgfqpoint{4.115997in}{1.560039in}}{\pgfqpoint{4.118831in}{1.553198in}}{\pgfqpoint{4.123874in}{1.548154in}}%
\pgfpathcurveto{\pgfqpoint{4.128918in}{1.543110in}}{\pgfqpoint{4.135760in}{1.540276in}}{\pgfqpoint{4.142892in}{1.540276in}}%
\pgfpathclose%
\pgfusepath{stroke,fill}%
\end{pgfscope}%
\begin{pgfscope}%
\pgfpathrectangle{\pgfqpoint{2.867647in}{0.500000in}}{\pgfqpoint{1.764706in}{1.700000in}}%
\pgfusepath{clip}%
\pgfsetbuttcap%
\pgfsetroundjoin%
\definecolor{currentfill}{rgb}{0.935991,0.337039,0.249722}%
\pgfsetfillcolor{currentfill}%
\pgfsetlinewidth{0.311001pt}%
\definecolor{currentstroke}{rgb}{1.000000,1.000000,1.000000}%
\pgfsetstrokecolor{currentstroke}%
\pgfsetdash{}{0pt}%
\pgfpathmoveto{\pgfqpoint{4.295400in}{1.662546in}}%
\pgfpathcurveto{\pgfqpoint{4.302533in}{1.662546in}}{\pgfqpoint{4.309374in}{1.665380in}}{\pgfqpoint{4.314418in}{1.670424in}}%
\pgfpathcurveto{\pgfqpoint{4.319462in}{1.675467in}}{\pgfqpoint{4.322296in}{1.682309in}}{\pgfqpoint{4.322296in}{1.689442in}}%
\pgfpathcurveto{\pgfqpoint{4.322296in}{1.696575in}}{\pgfqpoint{4.319462in}{1.703416in}}{\pgfqpoint{4.314418in}{1.708460in}}%
\pgfpathcurveto{\pgfqpoint{4.309374in}{1.713504in}}{\pgfqpoint{4.302533in}{1.716338in}}{\pgfqpoint{4.295400in}{1.716338in}}%
\pgfpathcurveto{\pgfqpoint{4.288267in}{1.716338in}}{\pgfqpoint{4.281425in}{1.713504in}}{\pgfqpoint{4.276382in}{1.708460in}}%
\pgfpathcurveto{\pgfqpoint{4.271338in}{1.703416in}}{\pgfqpoint{4.268504in}{1.696575in}}{\pgfqpoint{4.268504in}{1.689442in}}%
\pgfpathcurveto{\pgfqpoint{4.268504in}{1.682309in}}{\pgfqpoint{4.271338in}{1.675467in}}{\pgfqpoint{4.276382in}{1.670424in}}%
\pgfpathcurveto{\pgfqpoint{4.281425in}{1.665380in}}{\pgfqpoint{4.288267in}{1.662546in}}{\pgfqpoint{4.295400in}{1.662546in}}%
\pgfpathclose%
\pgfusepath{stroke,fill}%
\end{pgfscope}%
\begin{pgfscope}%
\pgfpathrectangle{\pgfqpoint{2.867647in}{0.500000in}}{\pgfqpoint{1.764706in}{1.700000in}}%
\pgfusepath{clip}%
\pgfsetbuttcap%
\pgfsetroundjoin%
\definecolor{currentfill}{rgb}{0.973271,0.850724,0.762998}%
\pgfsetfillcolor{currentfill}%
\pgfsetlinewidth{0.311001pt}%
\definecolor{currentstroke}{rgb}{1.000000,1.000000,1.000000}%
\pgfsetstrokecolor{currentstroke}%
\pgfsetdash{}{0pt}%
\pgfpathmoveto{\pgfqpoint{4.204260in}{1.072838in}}%
\pgfpathcurveto{\pgfqpoint{4.211393in}{1.072838in}}{\pgfqpoint{4.218235in}{1.075672in}}{\pgfqpoint{4.223278in}{1.080715in}}%
\pgfpathcurveto{\pgfqpoint{4.228322in}{1.085759in}}{\pgfqpoint{4.231156in}{1.092601in}}{\pgfqpoint{4.231156in}{1.099734in}}%
\pgfpathcurveto{\pgfqpoint{4.231156in}{1.106866in}}{\pgfqpoint{4.228322in}{1.113708in}}{\pgfqpoint{4.223278in}{1.118752in}}%
\pgfpathcurveto{\pgfqpoint{4.218235in}{1.123795in}}{\pgfqpoint{4.211393in}{1.126629in}}{\pgfqpoint{4.204260in}{1.126629in}}%
\pgfpathcurveto{\pgfqpoint{4.197127in}{1.126629in}}{\pgfqpoint{4.190286in}{1.123795in}}{\pgfqpoint{4.185242in}{1.118752in}}%
\pgfpathcurveto{\pgfqpoint{4.180198in}{1.113708in}}{\pgfqpoint{4.177364in}{1.106866in}}{\pgfqpoint{4.177364in}{1.099734in}}%
\pgfpathcurveto{\pgfqpoint{4.177364in}{1.092601in}}{\pgfqpoint{4.180198in}{1.085759in}}{\pgfqpoint{4.185242in}{1.080715in}}%
\pgfpathcurveto{\pgfqpoint{4.190286in}{1.075672in}}{\pgfqpoint{4.197127in}{1.072838in}}{\pgfqpoint{4.204260in}{1.072838in}}%
\pgfpathclose%
\pgfusepath{stroke,fill}%
\end{pgfscope}%
\begin{pgfscope}%
\pgfpathrectangle{\pgfqpoint{2.867647in}{0.500000in}}{\pgfqpoint{1.764706in}{1.700000in}}%
\pgfusepath{clip}%
\pgfsetbuttcap%
\pgfsetroundjoin%
\definecolor{currentfill}{rgb}{0.972201,0.839051,0.745789}%
\pgfsetfillcolor{currentfill}%
\pgfsetlinewidth{0.311001pt}%
\definecolor{currentstroke}{rgb}{1.000000,1.000000,1.000000}%
\pgfsetstrokecolor{currentstroke}%
\pgfsetdash{}{0pt}%
\pgfpathmoveto{\pgfqpoint{4.057558in}{1.597154in}}%
\pgfpathcurveto{\pgfqpoint{4.064691in}{1.597154in}}{\pgfqpoint{4.071533in}{1.599988in}}{\pgfqpoint{4.076577in}{1.605032in}}%
\pgfpathcurveto{\pgfqpoint{4.081620in}{1.610076in}}{\pgfqpoint{4.084454in}{1.616917in}}{\pgfqpoint{4.084454in}{1.624050in}}%
\pgfpathcurveto{\pgfqpoint{4.084454in}{1.631183in}}{\pgfqpoint{4.081620in}{1.638025in}}{\pgfqpoint{4.076577in}{1.643068in}}%
\pgfpathcurveto{\pgfqpoint{4.071533in}{1.648112in}}{\pgfqpoint{4.064691in}{1.650946in}}{\pgfqpoint{4.057558in}{1.650946in}}%
\pgfpathcurveto{\pgfqpoint{4.050426in}{1.650946in}}{\pgfqpoint{4.043584in}{1.648112in}}{\pgfqpoint{4.038540in}{1.643068in}}%
\pgfpathcurveto{\pgfqpoint{4.033497in}{1.638025in}}{\pgfqpoint{4.030663in}{1.631183in}}{\pgfqpoint{4.030663in}{1.624050in}}%
\pgfpathcurveto{\pgfqpoint{4.030663in}{1.616917in}}{\pgfqpoint{4.033497in}{1.610076in}}{\pgfqpoint{4.038540in}{1.605032in}}%
\pgfpathcurveto{\pgfqpoint{4.043584in}{1.599988in}}{\pgfqpoint{4.050426in}{1.597154in}}{\pgfqpoint{4.057558in}{1.597154in}}%
\pgfpathclose%
\pgfusepath{stroke,fill}%
\end{pgfscope}%
\begin{pgfscope}%
\pgfpathrectangle{\pgfqpoint{2.867647in}{0.500000in}}{\pgfqpoint{1.764706in}{1.700000in}}%
\pgfusepath{clip}%
\pgfsetbuttcap%
\pgfsetroundjoin%
\definecolor{currentfill}{rgb}{0.971694,0.833208,0.737161}%
\pgfsetfillcolor{currentfill}%
\pgfsetlinewidth{0.311001pt}%
\definecolor{currentstroke}{rgb}{1.000000,1.000000,1.000000}%
\pgfsetstrokecolor{currentstroke}%
\pgfsetdash{}{0pt}%
\pgfpathmoveto{\pgfqpoint{4.076318in}{0.975899in}}%
\pgfpathcurveto{\pgfqpoint{4.083451in}{0.975899in}}{\pgfqpoint{4.090293in}{0.978733in}}{\pgfqpoint{4.095336in}{0.983777in}}%
\pgfpathcurveto{\pgfqpoint{4.100380in}{0.988821in}}{\pgfqpoint{4.103214in}{0.995662in}}{\pgfqpoint{4.103214in}{1.002795in}}%
\pgfpathcurveto{\pgfqpoint{4.103214in}{1.009928in}}{\pgfqpoint{4.100380in}{1.016770in}}{\pgfqpoint{4.095336in}{1.021813in}}%
\pgfpathcurveto{\pgfqpoint{4.090293in}{1.026857in}}{\pgfqpoint{4.083451in}{1.029691in}}{\pgfqpoint{4.076318in}{1.029691in}}%
\pgfpathcurveto{\pgfqpoint{4.069185in}{1.029691in}}{\pgfqpoint{4.062344in}{1.026857in}}{\pgfqpoint{4.057300in}{1.021813in}}%
\pgfpathcurveto{\pgfqpoint{4.052256in}{1.016770in}}{\pgfqpoint{4.049422in}{1.009928in}}{\pgfqpoint{4.049422in}{1.002795in}}%
\pgfpathcurveto{\pgfqpoint{4.049422in}{0.995662in}}{\pgfqpoint{4.052256in}{0.988821in}}{\pgfqpoint{4.057300in}{0.983777in}}%
\pgfpathcurveto{\pgfqpoint{4.062344in}{0.978733in}}{\pgfqpoint{4.069185in}{0.975899in}}{\pgfqpoint{4.076318in}{0.975899in}}%
\pgfpathclose%
\pgfusepath{stroke,fill}%
\end{pgfscope}%
\begin{pgfscope}%
\pgfpathrectangle{\pgfqpoint{2.867647in}{0.500000in}}{\pgfqpoint{1.764706in}{1.700000in}}%
\pgfusepath{clip}%
\pgfsetbuttcap%
\pgfsetroundjoin%
\definecolor{currentfill}{rgb}{0.968105,0.786346,0.667739}%
\pgfsetfillcolor{currentfill}%
\pgfsetlinewidth{0.311001pt}%
\definecolor{currentstroke}{rgb}{1.000000,1.000000,1.000000}%
\pgfsetstrokecolor{currentstroke}%
\pgfsetdash{}{0pt}%
\pgfpathmoveto{\pgfqpoint{4.108667in}{1.712427in}}%
\pgfpathcurveto{\pgfqpoint{4.115800in}{1.712427in}}{\pgfqpoint{4.122642in}{1.715261in}}{\pgfqpoint{4.127685in}{1.720304in}}%
\pgfpathcurveto{\pgfqpoint{4.132729in}{1.725348in}}{\pgfqpoint{4.135563in}{1.732190in}}{\pgfqpoint{4.135563in}{1.739322in}}%
\pgfpathcurveto{\pgfqpoint{4.135563in}{1.746455in}}{\pgfqpoint{4.132729in}{1.753297in}}{\pgfqpoint{4.127685in}{1.758341in}}%
\pgfpathcurveto{\pgfqpoint{4.122642in}{1.763384in}}{\pgfqpoint{4.115800in}{1.766218in}}{\pgfqpoint{4.108667in}{1.766218in}}%
\pgfpathcurveto{\pgfqpoint{4.101534in}{1.766218in}}{\pgfqpoint{4.094693in}{1.763384in}}{\pgfqpoint{4.089649in}{1.758341in}}%
\pgfpathcurveto{\pgfqpoint{4.084605in}{1.753297in}}{\pgfqpoint{4.081772in}{1.746455in}}{\pgfqpoint{4.081772in}{1.739322in}}%
\pgfpathcurveto{\pgfqpoint{4.081772in}{1.732190in}}{\pgfqpoint{4.084605in}{1.725348in}}{\pgfqpoint{4.089649in}{1.720304in}}%
\pgfpathcurveto{\pgfqpoint{4.094693in}{1.715261in}}{\pgfqpoint{4.101534in}{1.712427in}}{\pgfqpoint{4.108667in}{1.712427in}}%
\pgfpathclose%
\pgfusepath{stroke,fill}%
\end{pgfscope}%
\begin{pgfscope}%
\pgfpathrectangle{\pgfqpoint{2.867647in}{0.500000in}}{\pgfqpoint{1.764706in}{1.700000in}}%
\pgfusepath{clip}%
\pgfsetbuttcap%
\pgfsetroundjoin%
\definecolor{currentfill}{rgb}{0.964306,0.663930,0.507747}%
\pgfsetfillcolor{currentfill}%
\pgfsetlinewidth{0.311001pt}%
\definecolor{currentstroke}{rgb}{1.000000,1.000000,1.000000}%
\pgfsetstrokecolor{currentstroke}%
\pgfsetdash{}{0pt}%
\pgfpathmoveto{\pgfqpoint{4.009309in}{1.770205in}}%
\pgfpathcurveto{\pgfqpoint{4.016442in}{1.770205in}}{\pgfqpoint{4.023284in}{1.773038in}}{\pgfqpoint{4.028328in}{1.778082in}}%
\pgfpathcurveto{\pgfqpoint{4.033371in}{1.783126in}}{\pgfqpoint{4.036205in}{1.789967in}}{\pgfqpoint{4.036205in}{1.797100in}}%
\pgfpathcurveto{\pgfqpoint{4.036205in}{1.804233in}}{\pgfqpoint{4.033371in}{1.811075in}}{\pgfqpoint{4.028328in}{1.816118in}}%
\pgfpathcurveto{\pgfqpoint{4.023284in}{1.821162in}}{\pgfqpoint{4.016442in}{1.823996in}}{\pgfqpoint{4.009309in}{1.823996in}}%
\pgfpathcurveto{\pgfqpoint{4.002177in}{1.823996in}}{\pgfqpoint{3.995335in}{1.821162in}}{\pgfqpoint{3.990291in}{1.816118in}}%
\pgfpathcurveto{\pgfqpoint{3.985248in}{1.811075in}}{\pgfqpoint{3.982414in}{1.804233in}}{\pgfqpoint{3.982414in}{1.797100in}}%
\pgfpathcurveto{\pgfqpoint{3.982414in}{1.789967in}}{\pgfqpoint{3.985248in}{1.783126in}}{\pgfqpoint{3.990291in}{1.778082in}}%
\pgfpathcurveto{\pgfqpoint{3.995335in}{1.773038in}}{\pgfqpoint{4.002177in}{1.770205in}}{\pgfqpoint{4.009309in}{1.770205in}}%
\pgfpathclose%
\pgfusepath{stroke,fill}%
\end{pgfscope}%
\begin{pgfscope}%
\pgfpathrectangle{\pgfqpoint{2.867647in}{0.500000in}}{\pgfqpoint{1.764706in}{1.700000in}}%
\pgfusepath{clip}%
\pgfsetbuttcap%
\pgfsetroundjoin%
\definecolor{currentfill}{rgb}{0.973832,0.856556,0.771584}%
\pgfsetfillcolor{currentfill}%
\pgfsetlinewidth{0.311001pt}%
\definecolor{currentstroke}{rgb}{1.000000,1.000000,1.000000}%
\pgfsetstrokecolor{currentstroke}%
\pgfsetdash{}{0pt}%
\pgfpathmoveto{\pgfqpoint{4.218206in}{1.528568in}}%
\pgfpathcurveto{\pgfqpoint{4.225339in}{1.528568in}}{\pgfqpoint{4.232181in}{1.531402in}}{\pgfqpoint{4.237224in}{1.536445in}}%
\pgfpathcurveto{\pgfqpoint{4.242268in}{1.541489in}}{\pgfqpoint{4.245102in}{1.548331in}}{\pgfqpoint{4.245102in}{1.555463in}}%
\pgfpathcurveto{\pgfqpoint{4.245102in}{1.562596in}}{\pgfqpoint{4.242268in}{1.569438in}}{\pgfqpoint{4.237224in}{1.574481in}}%
\pgfpathcurveto{\pgfqpoint{4.232181in}{1.579525in}}{\pgfqpoint{4.225339in}{1.582359in}}{\pgfqpoint{4.218206in}{1.582359in}}%
\pgfpathcurveto{\pgfqpoint{4.211074in}{1.582359in}}{\pgfqpoint{4.204232in}{1.579525in}}{\pgfqpoint{4.199188in}{1.574481in}}%
\pgfpathcurveto{\pgfqpoint{4.194145in}{1.569438in}}{\pgfqpoint{4.191311in}{1.562596in}}{\pgfqpoint{4.191311in}{1.555463in}}%
\pgfpathcurveto{\pgfqpoint{4.191311in}{1.548331in}}{\pgfqpoint{4.194145in}{1.541489in}}{\pgfqpoint{4.199188in}{1.536445in}}%
\pgfpathcurveto{\pgfqpoint{4.204232in}{1.531402in}}{\pgfqpoint{4.211074in}{1.528568in}}{\pgfqpoint{4.218206in}{1.528568in}}%
\pgfpathclose%
\pgfusepath{stroke,fill}%
\end{pgfscope}%
\begin{pgfscope}%
\pgfpathrectangle{\pgfqpoint{2.867647in}{0.500000in}}{\pgfqpoint{1.764706in}{1.700000in}}%
\pgfusepath{clip}%
\pgfsetbuttcap%
\pgfsetroundjoin%
\definecolor{currentfill}{rgb}{0.979124,0.903132,0.839793}%
\pgfsetfillcolor{currentfill}%
\pgfsetlinewidth{0.311001pt}%
\definecolor{currentstroke}{rgb}{1.000000,1.000000,1.000000}%
\pgfsetstrokecolor{currentstroke}%
\pgfsetdash{}{0pt}%
\pgfpathmoveto{\pgfqpoint{4.157462in}{1.400837in}}%
\pgfpathcurveto{\pgfqpoint{4.164595in}{1.400837in}}{\pgfqpoint{4.171437in}{1.403671in}}{\pgfqpoint{4.176480in}{1.408715in}}%
\pgfpathcurveto{\pgfqpoint{4.181524in}{1.413758in}}{\pgfqpoint{4.184358in}{1.420600in}}{\pgfqpoint{4.184358in}{1.427733in}}%
\pgfpathcurveto{\pgfqpoint{4.184358in}{1.434866in}}{\pgfqpoint{4.181524in}{1.441707in}}{\pgfqpoint{4.176480in}{1.446751in}}%
\pgfpathcurveto{\pgfqpoint{4.171437in}{1.451795in}}{\pgfqpoint{4.164595in}{1.454629in}}{\pgfqpoint{4.157462in}{1.454629in}}%
\pgfpathcurveto{\pgfqpoint{4.150329in}{1.454629in}}{\pgfqpoint{4.143488in}{1.451795in}}{\pgfqpoint{4.138444in}{1.446751in}}%
\pgfpathcurveto{\pgfqpoint{4.133400in}{1.441707in}}{\pgfqpoint{4.130567in}{1.434866in}}{\pgfqpoint{4.130567in}{1.427733in}}%
\pgfpathcurveto{\pgfqpoint{4.130567in}{1.420600in}}{\pgfqpoint{4.133400in}{1.413758in}}{\pgfqpoint{4.138444in}{1.408715in}}%
\pgfpathcurveto{\pgfqpoint{4.143488in}{1.403671in}}{\pgfqpoint{4.150329in}{1.400837in}}{\pgfqpoint{4.157462in}{1.400837in}}%
\pgfpathclose%
\pgfusepath{stroke,fill}%
\end{pgfscope}%
\begin{pgfscope}%
\pgfpathrectangle{\pgfqpoint{2.867647in}{0.500000in}}{\pgfqpoint{1.764706in}{1.700000in}}%
\pgfusepath{clip}%
\pgfsetbuttcap%
\pgfsetroundjoin%
\definecolor{currentfill}{rgb}{0.975018,0.868213,0.788710}%
\pgfsetfillcolor{currentfill}%
\pgfsetlinewidth{0.311001pt}%
\definecolor{currentstroke}{rgb}{1.000000,1.000000,1.000000}%
\pgfsetstrokecolor{currentstroke}%
\pgfsetdash{}{0pt}%
\pgfpathmoveto{\pgfqpoint{4.125742in}{1.406553in}}%
\pgfpathcurveto{\pgfqpoint{4.132875in}{1.406553in}}{\pgfqpoint{4.139716in}{1.409387in}}{\pgfqpoint{4.144760in}{1.414430in}}%
\pgfpathcurveto{\pgfqpoint{4.149804in}{1.419474in}}{\pgfqpoint{4.152638in}{1.426316in}}{\pgfqpoint{4.152638in}{1.433448in}}%
\pgfpathcurveto{\pgfqpoint{4.152638in}{1.440581in}}{\pgfqpoint{4.149804in}{1.447423in}}{\pgfqpoint{4.144760in}{1.452467in}}%
\pgfpathcurveto{\pgfqpoint{4.139716in}{1.457510in}}{\pgfqpoint{4.132875in}{1.460344in}}{\pgfqpoint{4.125742in}{1.460344in}}%
\pgfpathcurveto{\pgfqpoint{4.118609in}{1.460344in}}{\pgfqpoint{4.111767in}{1.457510in}}{\pgfqpoint{4.106724in}{1.452467in}}%
\pgfpathcurveto{\pgfqpoint{4.101680in}{1.447423in}}{\pgfqpoint{4.098846in}{1.440581in}}{\pgfqpoint{4.098846in}{1.433448in}}%
\pgfpathcurveto{\pgfqpoint{4.098846in}{1.426316in}}{\pgfqpoint{4.101680in}{1.419474in}}{\pgfqpoint{4.106724in}{1.414430in}}%
\pgfpathcurveto{\pgfqpoint{4.111767in}{1.409387in}}{\pgfqpoint{4.118609in}{1.406553in}}{\pgfqpoint{4.125742in}{1.406553in}}%
\pgfpathclose%
\pgfusepath{stroke,fill}%
\end{pgfscope}%
\begin{pgfscope}%
\pgfpathrectangle{\pgfqpoint{2.867647in}{0.500000in}}{\pgfqpoint{1.764706in}{1.700000in}}%
\pgfusepath{clip}%
\pgfsetbuttcap%
\pgfsetroundjoin%
\definecolor{currentfill}{rgb}{0.975018,0.868213,0.788710}%
\pgfsetfillcolor{currentfill}%
\pgfsetlinewidth{0.311001pt}%
\definecolor{currentstroke}{rgb}{1.000000,1.000000,1.000000}%
\pgfsetstrokecolor{currentstroke}%
\pgfsetdash{}{0pt}%
\pgfpathmoveto{\pgfqpoint{4.087737in}{1.065311in}}%
\pgfpathcurveto{\pgfqpoint{4.094869in}{1.065311in}}{\pgfqpoint{4.101711in}{1.068145in}}{\pgfqpoint{4.106755in}{1.073189in}}%
\pgfpathcurveto{\pgfqpoint{4.111798in}{1.078232in}}{\pgfqpoint{4.114632in}{1.085074in}}{\pgfqpoint{4.114632in}{1.092207in}}%
\pgfpathcurveto{\pgfqpoint{4.114632in}{1.099340in}}{\pgfqpoint{4.111798in}{1.106181in}}{\pgfqpoint{4.106755in}{1.111225in}}%
\pgfpathcurveto{\pgfqpoint{4.101711in}{1.116269in}}{\pgfqpoint{4.094869in}{1.119103in}}{\pgfqpoint{4.087737in}{1.119103in}}%
\pgfpathcurveto{\pgfqpoint{4.080604in}{1.119103in}}{\pgfqpoint{4.073762in}{1.116269in}}{\pgfqpoint{4.068718in}{1.111225in}}%
\pgfpathcurveto{\pgfqpoint{4.063675in}{1.106181in}}{\pgfqpoint{4.060841in}{1.099340in}}{\pgfqpoint{4.060841in}{1.092207in}}%
\pgfpathcurveto{\pgfqpoint{4.060841in}{1.085074in}}{\pgfqpoint{4.063675in}{1.078232in}}{\pgfqpoint{4.068718in}{1.073189in}}%
\pgfpathcurveto{\pgfqpoint{4.073762in}{1.068145in}}{\pgfqpoint{4.080604in}{1.065311in}}{\pgfqpoint{4.087737in}{1.065311in}}%
\pgfpathclose%
\pgfusepath{stroke,fill}%
\end{pgfscope}%
\begin{pgfscope}%
\pgfpathrectangle{\pgfqpoint{2.867647in}{0.500000in}}{\pgfqpoint{1.764706in}{1.700000in}}%
\pgfusepath{clip}%
\pgfsetbuttcap%
\pgfsetroundjoin%
\definecolor{currentfill}{rgb}{0.966328,0.750560,0.616961}%
\pgfsetfillcolor{currentfill}%
\pgfsetlinewidth{0.311001pt}%
\definecolor{currentstroke}{rgb}{1.000000,1.000000,1.000000}%
\pgfsetstrokecolor{currentstroke}%
\pgfsetdash{}{0pt}%
\pgfpathmoveto{\pgfqpoint{4.092985in}{1.739322in}}%
\pgfpathcurveto{\pgfqpoint{4.100118in}{1.739322in}}{\pgfqpoint{4.106960in}{1.742156in}}{\pgfqpoint{4.112004in}{1.747199in}}%
\pgfpathcurveto{\pgfqpoint{4.117047in}{1.752243in}}{\pgfqpoint{4.119881in}{1.759085in}}{\pgfqpoint{4.119881in}{1.766218in}}%
\pgfpathcurveto{\pgfqpoint{4.119881in}{1.773350in}}{\pgfqpoint{4.117047in}{1.780192in}}{\pgfqpoint{4.112004in}{1.785236in}}%
\pgfpathcurveto{\pgfqpoint{4.106960in}{1.790279in}}{\pgfqpoint{4.100118in}{1.793113in}}{\pgfqpoint{4.092985in}{1.793113in}}%
\pgfpathcurveto{\pgfqpoint{4.085853in}{1.793113in}}{\pgfqpoint{4.079011in}{1.790279in}}{\pgfqpoint{4.073967in}{1.785236in}}%
\pgfpathcurveto{\pgfqpoint{4.068924in}{1.780192in}}{\pgfqpoint{4.066090in}{1.773350in}}{\pgfqpoint{4.066090in}{1.766218in}}%
\pgfpathcurveto{\pgfqpoint{4.066090in}{1.759085in}}{\pgfqpoint{4.068924in}{1.752243in}}{\pgfqpoint{4.073967in}{1.747199in}}%
\pgfpathcurveto{\pgfqpoint{4.079011in}{1.742156in}}{\pgfqpoint{4.085853in}{1.739322in}}{\pgfqpoint{4.092985in}{1.739322in}}%
\pgfpathclose%
\pgfusepath{stroke,fill}%
\end{pgfscope}%
\begin{pgfscope}%
\pgfpathrectangle{\pgfqpoint{2.867647in}{0.500000in}}{\pgfqpoint{1.764706in}{1.700000in}}%
\pgfusepath{clip}%
\pgfsetbuttcap%
\pgfsetroundjoin%
\definecolor{currentfill}{rgb}{0.961734,0.579886,0.418445}%
\pgfsetfillcolor{currentfill}%
\pgfsetlinewidth{0.311001pt}%
\definecolor{currentstroke}{rgb}{1.000000,1.000000,1.000000}%
\pgfsetstrokecolor{currentstroke}%
\pgfsetdash{}{0pt}%
\pgfpathmoveto{\pgfqpoint{4.005571in}{1.138942in}}%
\pgfpathcurveto{\pgfqpoint{4.012704in}{1.138942in}}{\pgfqpoint{4.019546in}{1.141776in}}{\pgfqpoint{4.024589in}{1.146820in}}%
\pgfpathcurveto{\pgfqpoint{4.029633in}{1.151863in}}{\pgfqpoint{4.032467in}{1.158705in}}{\pgfqpoint{4.032467in}{1.165838in}}%
\pgfpathcurveto{\pgfqpoint{4.032467in}{1.172971in}}{\pgfqpoint{4.029633in}{1.179812in}}{\pgfqpoint{4.024589in}{1.184856in}}%
\pgfpathcurveto{\pgfqpoint{4.019546in}{1.189900in}}{\pgfqpoint{4.012704in}{1.192734in}}{\pgfqpoint{4.005571in}{1.192734in}}%
\pgfpathcurveto{\pgfqpoint{3.998438in}{1.192734in}}{\pgfqpoint{3.991597in}{1.189900in}}{\pgfqpoint{3.986553in}{1.184856in}}%
\pgfpathcurveto{\pgfqpoint{3.981509in}{1.179812in}}{\pgfqpoint{3.978675in}{1.172971in}}{\pgfqpoint{3.978675in}{1.165838in}}%
\pgfpathcurveto{\pgfqpoint{3.978675in}{1.158705in}}{\pgfqpoint{3.981509in}{1.151863in}}{\pgfqpoint{3.986553in}{1.146820in}}%
\pgfpathcurveto{\pgfqpoint{3.991597in}{1.141776in}}{\pgfqpoint{3.998438in}{1.138942in}}{\pgfqpoint{4.005571in}{1.138942in}}%
\pgfpathclose%
\pgfusepath{stroke,fill}%
\end{pgfscope}%
\begin{pgfscope}%
\pgfpathrectangle{\pgfqpoint{2.867647in}{0.500000in}}{\pgfqpoint{1.764706in}{1.700000in}}%
\pgfusepath{clip}%
\pgfsetbuttcap%
\pgfsetroundjoin%
\definecolor{currentfill}{rgb}{0.971694,0.833208,0.737161}%
\pgfsetfillcolor{currentfill}%
\pgfsetlinewidth{0.311001pt}%
\definecolor{currentstroke}{rgb}{1.000000,1.000000,1.000000}%
\pgfsetstrokecolor{currentstroke}%
\pgfsetdash{}{0pt}%
\pgfpathmoveto{\pgfqpoint{4.064808in}{1.064654in}}%
\pgfpathcurveto{\pgfqpoint{4.071940in}{1.064654in}}{\pgfqpoint{4.078782in}{1.067488in}}{\pgfqpoint{4.083826in}{1.072532in}}%
\pgfpathcurveto{\pgfqpoint{4.088869in}{1.077576in}}{\pgfqpoint{4.091703in}{1.084417in}}{\pgfqpoint{4.091703in}{1.091550in}}%
\pgfpathcurveto{\pgfqpoint{4.091703in}{1.098683in}}{\pgfqpoint{4.088869in}{1.105525in}}{\pgfqpoint{4.083826in}{1.110568in}}%
\pgfpathcurveto{\pgfqpoint{4.078782in}{1.115612in}}{\pgfqpoint{4.071940in}{1.118446in}}{\pgfqpoint{4.064808in}{1.118446in}}%
\pgfpathcurveto{\pgfqpoint{4.057675in}{1.118446in}}{\pgfqpoint{4.050833in}{1.115612in}}{\pgfqpoint{4.045789in}{1.110568in}}%
\pgfpathcurveto{\pgfqpoint{4.040746in}{1.105525in}}{\pgfqpoint{4.037912in}{1.098683in}}{\pgfqpoint{4.037912in}{1.091550in}}%
\pgfpathcurveto{\pgfqpoint{4.037912in}{1.084417in}}{\pgfqpoint{4.040746in}{1.077576in}}{\pgfqpoint{4.045789in}{1.072532in}}%
\pgfpathcurveto{\pgfqpoint{4.050833in}{1.067488in}}{\pgfqpoint{4.057675in}{1.064654in}}{\pgfqpoint{4.064808in}{1.064654in}}%
\pgfpathclose%
\pgfusepath{stroke,fill}%
\end{pgfscope}%
\begin{pgfscope}%
\pgfpathrectangle{\pgfqpoint{2.867647in}{0.500000in}}{\pgfqpoint{1.764706in}{1.700000in}}%
\pgfusepath{clip}%
\pgfsetbuttcap%
\pgfsetroundjoin%
\definecolor{currentfill}{rgb}{0.973832,0.856556,0.771584}%
\pgfsetfillcolor{currentfill}%
\pgfsetlinewidth{0.311001pt}%
\definecolor{currentstroke}{rgb}{1.000000,1.000000,1.000000}%
\pgfsetstrokecolor{currentstroke}%
\pgfsetdash{}{0pt}%
\pgfpathmoveto{\pgfqpoint{4.156093in}{1.629116in}}%
\pgfpathcurveto{\pgfqpoint{4.163226in}{1.629116in}}{\pgfqpoint{4.170068in}{1.631949in}}{\pgfqpoint{4.175111in}{1.636993in}}%
\pgfpathcurveto{\pgfqpoint{4.180155in}{1.642037in}}{\pgfqpoint{4.182989in}{1.648878in}}{\pgfqpoint{4.182989in}{1.656011in}}%
\pgfpathcurveto{\pgfqpoint{4.182989in}{1.663144in}}{\pgfqpoint{4.180155in}{1.669986in}}{\pgfqpoint{4.175111in}{1.675029in}}%
\pgfpathcurveto{\pgfqpoint{4.170068in}{1.680073in}}{\pgfqpoint{4.163226in}{1.682907in}}{\pgfqpoint{4.156093in}{1.682907in}}%
\pgfpathcurveto{\pgfqpoint{4.148960in}{1.682907in}}{\pgfqpoint{4.142119in}{1.680073in}}{\pgfqpoint{4.137075in}{1.675029in}}%
\pgfpathcurveto{\pgfqpoint{4.132031in}{1.669986in}}{\pgfqpoint{4.129198in}{1.663144in}}{\pgfqpoint{4.129198in}{1.656011in}}%
\pgfpathcurveto{\pgfqpoint{4.129198in}{1.648878in}}{\pgfqpoint{4.132031in}{1.642037in}}{\pgfqpoint{4.137075in}{1.636993in}}%
\pgfpathcurveto{\pgfqpoint{4.142119in}{1.631949in}}{\pgfqpoint{4.148960in}{1.629116in}}{\pgfqpoint{4.156093in}{1.629116in}}%
\pgfpathclose%
\pgfusepath{stroke,fill}%
\end{pgfscope}%
\begin{pgfscope}%
\pgfpathrectangle{\pgfqpoint{2.867647in}{0.500000in}}{\pgfqpoint{1.764706in}{1.700000in}}%
\pgfusepath{clip}%
\pgfsetbuttcap%
\pgfsetroundjoin%
\definecolor{currentfill}{rgb}{0.971694,0.833208,0.737161}%
\pgfsetfillcolor{currentfill}%
\pgfsetlinewidth{0.311001pt}%
\definecolor{currentstroke}{rgb}{1.000000,1.000000,1.000000}%
\pgfsetstrokecolor{currentstroke}%
\pgfsetdash{}{0pt}%
\pgfpathmoveto{\pgfqpoint{4.081393in}{1.125922in}}%
\pgfpathcurveto{\pgfqpoint{4.088526in}{1.125922in}}{\pgfqpoint{4.095367in}{1.128756in}}{\pgfqpoint{4.100411in}{1.133800in}}%
\pgfpathcurveto{\pgfqpoint{4.105455in}{1.138844in}}{\pgfqpoint{4.108288in}{1.145685in}}{\pgfqpoint{4.108288in}{1.152818in}}%
\pgfpathcurveto{\pgfqpoint{4.108288in}{1.159951in}}{\pgfqpoint{4.105455in}{1.166792in}}{\pgfqpoint{4.100411in}{1.171836in}}%
\pgfpathcurveto{\pgfqpoint{4.095367in}{1.176880in}}{\pgfqpoint{4.088526in}{1.179714in}}{\pgfqpoint{4.081393in}{1.179714in}}%
\pgfpathcurveto{\pgfqpoint{4.074260in}{1.179714in}}{\pgfqpoint{4.067418in}{1.176880in}}{\pgfqpoint{4.062375in}{1.171836in}}%
\pgfpathcurveto{\pgfqpoint{4.057331in}{1.166792in}}{\pgfqpoint{4.054497in}{1.159951in}}{\pgfqpoint{4.054497in}{1.152818in}}%
\pgfpathcurveto{\pgfqpoint{4.054497in}{1.145685in}}{\pgfqpoint{4.057331in}{1.138844in}}{\pgfqpoint{4.062375in}{1.133800in}}%
\pgfpathcurveto{\pgfqpoint{4.067418in}{1.128756in}}{\pgfqpoint{4.074260in}{1.125922in}}{\pgfqpoint{4.081393in}{1.125922in}}%
\pgfpathclose%
\pgfusepath{stroke,fill}%
\end{pgfscope}%
\begin{pgfscope}%
\pgfpathrectangle{\pgfqpoint{2.867647in}{0.500000in}}{\pgfqpoint{1.764706in}{1.700000in}}%
\pgfusepath{clip}%
\pgfsetbuttcap%
\pgfsetroundjoin%
\definecolor{currentfill}{rgb}{0.972201,0.839051,0.745789}%
\pgfsetfillcolor{currentfill}%
\pgfsetlinewidth{0.311001pt}%
\definecolor{currentstroke}{rgb}{1.000000,1.000000,1.000000}%
\pgfsetstrokecolor{currentstroke}%
\pgfsetdash{}{0pt}%
\pgfpathmoveto{\pgfqpoint{4.065324in}{0.998847in}}%
\pgfpathcurveto{\pgfqpoint{4.072457in}{0.998847in}}{\pgfqpoint{4.079298in}{1.001681in}}{\pgfqpoint{4.084342in}{1.006724in}}%
\pgfpathcurveto{\pgfqpoint{4.089386in}{1.011768in}}{\pgfqpoint{4.092219in}{1.018610in}}{\pgfqpoint{4.092219in}{1.025743in}}%
\pgfpathcurveto{\pgfqpoint{4.092219in}{1.032875in}}{\pgfqpoint{4.089386in}{1.039717in}}{\pgfqpoint{4.084342in}{1.044761in}}%
\pgfpathcurveto{\pgfqpoint{4.079298in}{1.049804in}}{\pgfqpoint{4.072457in}{1.052638in}}{\pgfqpoint{4.065324in}{1.052638in}}%
\pgfpathcurveto{\pgfqpoint{4.058191in}{1.052638in}}{\pgfqpoint{4.051349in}{1.049804in}}{\pgfqpoint{4.046306in}{1.044761in}}%
\pgfpathcurveto{\pgfqpoint{4.041262in}{1.039717in}}{\pgfqpoint{4.038428in}{1.032875in}}{\pgfqpoint{4.038428in}{1.025743in}}%
\pgfpathcurveto{\pgfqpoint{4.038428in}{1.018610in}}{\pgfqpoint{4.041262in}{1.011768in}}{\pgfqpoint{4.046306in}{1.006724in}}%
\pgfpathcurveto{\pgfqpoint{4.051349in}{1.001681in}}{\pgfqpoint{4.058191in}{0.998847in}}{\pgfqpoint{4.065324in}{0.998847in}}%
\pgfpathclose%
\pgfusepath{stroke,fill}%
\end{pgfscope}%
\begin{pgfscope}%
\pgfpathrectangle{\pgfqpoint{2.867647in}{0.500000in}}{\pgfqpoint{1.764706in}{1.700000in}}%
\pgfusepath{clip}%
\pgfsetbuttcap%
\pgfsetroundjoin%
\definecolor{currentfill}{rgb}{0.978376,0.897317,0.831308}%
\pgfsetfillcolor{currentfill}%
\pgfsetlinewidth{0.311001pt}%
\definecolor{currentstroke}{rgb}{1.000000,1.000000,1.000000}%
\pgfsetstrokecolor{currentstroke}%
\pgfsetdash{}{0pt}%
\pgfpathmoveto{\pgfqpoint{4.209900in}{1.450339in}}%
\pgfpathcurveto{\pgfqpoint{4.217033in}{1.450339in}}{\pgfqpoint{4.223874in}{1.453173in}}{\pgfqpoint{4.228918in}{1.458217in}}%
\pgfpathcurveto{\pgfqpoint{4.233962in}{1.463260in}}{\pgfqpoint{4.236795in}{1.470102in}}{\pgfqpoint{4.236795in}{1.477235in}}%
\pgfpathcurveto{\pgfqpoint{4.236795in}{1.484368in}}{\pgfqpoint{4.233962in}{1.491209in}}{\pgfqpoint{4.228918in}{1.496253in}}%
\pgfpathcurveto{\pgfqpoint{4.223874in}{1.501297in}}{\pgfqpoint{4.217033in}{1.504130in}}{\pgfqpoint{4.209900in}{1.504130in}}%
\pgfpathcurveto{\pgfqpoint{4.202767in}{1.504130in}}{\pgfqpoint{4.195925in}{1.501297in}}{\pgfqpoint{4.190882in}{1.496253in}}%
\pgfpathcurveto{\pgfqpoint{4.185838in}{1.491209in}}{\pgfqpoint{4.183004in}{1.484368in}}{\pgfqpoint{4.183004in}{1.477235in}}%
\pgfpathcurveto{\pgfqpoint{4.183004in}{1.470102in}}{\pgfqpoint{4.185838in}{1.463260in}}{\pgfqpoint{4.190882in}{1.458217in}}%
\pgfpathcurveto{\pgfqpoint{4.195925in}{1.453173in}}{\pgfqpoint{4.202767in}{1.450339in}}{\pgfqpoint{4.209900in}{1.450339in}}%
\pgfpathclose%
\pgfusepath{stroke,fill}%
\end{pgfscope}%
\begin{pgfscope}%
\pgfpathrectangle{\pgfqpoint{2.867647in}{0.500000in}}{\pgfqpoint{1.764706in}{1.700000in}}%
\pgfusepath{clip}%
\pgfsetbuttcap%
\pgfsetroundjoin%
\definecolor{currentfill}{rgb}{0.965928,0.738443,0.600540}%
\pgfsetfillcolor{currentfill}%
\pgfsetlinewidth{0.311001pt}%
\definecolor{currentstroke}{rgb}{1.000000,1.000000,1.000000}%
\pgfsetstrokecolor{currentstroke}%
\pgfsetdash{}{0pt}%
\pgfpathmoveto{\pgfqpoint{4.155241in}{0.941060in}}%
\pgfpathcurveto{\pgfqpoint{4.162374in}{0.941060in}}{\pgfqpoint{4.169216in}{0.943893in}}{\pgfqpoint{4.174259in}{0.948937in}}%
\pgfpathcurveto{\pgfqpoint{4.179303in}{0.953981in}}{\pgfqpoint{4.182137in}{0.960822in}}{\pgfqpoint{4.182137in}{0.967955in}}%
\pgfpathcurveto{\pgfqpoint{4.182137in}{0.975088in}}{\pgfqpoint{4.179303in}{0.981930in}}{\pgfqpoint{4.174259in}{0.986973in}}%
\pgfpathcurveto{\pgfqpoint{4.169216in}{0.992017in}}{\pgfqpoint{4.162374in}{0.994851in}}{\pgfqpoint{4.155241in}{0.994851in}}%
\pgfpathcurveto{\pgfqpoint{4.148108in}{0.994851in}}{\pgfqpoint{4.141267in}{0.992017in}}{\pgfqpoint{4.136223in}{0.986973in}}%
\pgfpathcurveto{\pgfqpoint{4.131179in}{0.981930in}}{\pgfqpoint{4.128345in}{0.975088in}}{\pgfqpoint{4.128345in}{0.967955in}}%
\pgfpathcurveto{\pgfqpoint{4.128345in}{0.960822in}}{\pgfqpoint{4.131179in}{0.953981in}}{\pgfqpoint{4.136223in}{0.948937in}}%
\pgfpathcurveto{\pgfqpoint{4.141267in}{0.943893in}}{\pgfqpoint{4.148108in}{0.941060in}}{\pgfqpoint{4.155241in}{0.941060in}}%
\pgfpathclose%
\pgfusepath{stroke,fill}%
\end{pgfscope}%
\begin{pgfscope}%
\pgfpathrectangle{\pgfqpoint{2.867647in}{0.500000in}}{\pgfqpoint{1.764706in}{1.700000in}}%
\pgfusepath{clip}%
\pgfsetbuttcap%
\pgfsetroundjoin%
\definecolor{currentfill}{rgb}{0.962283,0.593046,0.431453}%
\pgfsetfillcolor{currentfill}%
\pgfsetlinewidth{0.311001pt}%
\definecolor{currentstroke}{rgb}{1.000000,1.000000,1.000000}%
\pgfsetstrokecolor{currentstroke}%
\pgfsetdash{}{0pt}%
\pgfpathmoveto{\pgfqpoint{4.083399in}{0.853882in}}%
\pgfpathcurveto{\pgfqpoint{4.090532in}{0.853882in}}{\pgfqpoint{4.097373in}{0.856716in}}{\pgfqpoint{4.102417in}{0.861759in}}%
\pgfpathcurveto{\pgfqpoint{4.107461in}{0.866803in}}{\pgfqpoint{4.110294in}{0.873645in}}{\pgfqpoint{4.110294in}{0.880778in}}%
\pgfpathcurveto{\pgfqpoint{4.110294in}{0.887910in}}{\pgfqpoint{4.107461in}{0.894752in}}{\pgfqpoint{4.102417in}{0.899796in}}%
\pgfpathcurveto{\pgfqpoint{4.097373in}{0.904839in}}{\pgfqpoint{4.090532in}{0.907673in}}{\pgfqpoint{4.083399in}{0.907673in}}%
\pgfpathcurveto{\pgfqpoint{4.076266in}{0.907673in}}{\pgfqpoint{4.069424in}{0.904839in}}{\pgfqpoint{4.064381in}{0.899796in}}%
\pgfpathcurveto{\pgfqpoint{4.059337in}{0.894752in}}{\pgfqpoint{4.056503in}{0.887910in}}{\pgfqpoint{4.056503in}{0.880778in}}%
\pgfpathcurveto{\pgfqpoint{4.056503in}{0.873645in}}{\pgfqpoint{4.059337in}{0.866803in}}{\pgfqpoint{4.064381in}{0.861759in}}%
\pgfpathcurveto{\pgfqpoint{4.069424in}{0.856716in}}{\pgfqpoint{4.076266in}{0.853882in}}{\pgfqpoint{4.083399in}{0.853882in}}%
\pgfpathclose%
\pgfusepath{stroke,fill}%
\end{pgfscope}%
\begin{pgfscope}%
\pgfpathrectangle{\pgfqpoint{2.867647in}{0.500000in}}{\pgfqpoint{1.764706in}{1.700000in}}%
\pgfusepath{clip}%
\pgfsetbuttcap%
\pgfsetroundjoin%
\definecolor{currentfill}{rgb}{0.972726,0.844889,0.754401}%
\pgfsetfillcolor{currentfill}%
\pgfsetlinewidth{0.311001pt}%
\definecolor{currentstroke}{rgb}{1.000000,1.000000,1.000000}%
\pgfsetstrokecolor{currentstroke}%
\pgfsetdash{}{0pt}%
\pgfpathmoveto{\pgfqpoint{4.120657in}{1.285803in}}%
\pgfpathcurveto{\pgfqpoint{4.127790in}{1.285803in}}{\pgfqpoint{4.134631in}{1.288637in}}{\pgfqpoint{4.139675in}{1.293681in}}%
\pgfpathcurveto{\pgfqpoint{4.144719in}{1.298725in}}{\pgfqpoint{4.147552in}{1.305566in}}{\pgfqpoint{4.147552in}{1.312699in}}%
\pgfpathcurveto{\pgfqpoint{4.147552in}{1.319832in}}{\pgfqpoint{4.144719in}{1.326674in}}{\pgfqpoint{4.139675in}{1.331717in}}%
\pgfpathcurveto{\pgfqpoint{4.134631in}{1.336761in}}{\pgfqpoint{4.127790in}{1.339595in}}{\pgfqpoint{4.120657in}{1.339595in}}%
\pgfpathcurveto{\pgfqpoint{4.113524in}{1.339595in}}{\pgfqpoint{4.106682in}{1.336761in}}{\pgfqpoint{4.101639in}{1.331717in}}%
\pgfpathcurveto{\pgfqpoint{4.096595in}{1.326674in}}{\pgfqpoint{4.093761in}{1.319832in}}{\pgfqpoint{4.093761in}{1.312699in}}%
\pgfpathcurveto{\pgfqpoint{4.093761in}{1.305566in}}{\pgfqpoint{4.096595in}{1.298725in}}{\pgfqpoint{4.101639in}{1.293681in}}%
\pgfpathcurveto{\pgfqpoint{4.106682in}{1.288637in}}{\pgfqpoint{4.113524in}{1.285803in}}{\pgfqpoint{4.120657in}{1.285803in}}%
\pgfpathclose%
\pgfusepath{stroke,fill}%
\end{pgfscope}%
\begin{pgfscope}%
\pgfpathrectangle{\pgfqpoint{2.867647in}{0.500000in}}{\pgfqpoint{1.764706in}{1.700000in}}%
\pgfusepath{clip}%
\pgfsetbuttcap%
\pgfsetroundjoin%
\definecolor{currentfill}{rgb}{0.969803,0.809811,0.702523}%
\pgfsetfillcolor{currentfill}%
\pgfsetlinewidth{0.311001pt}%
\definecolor{currentstroke}{rgb}{1.000000,1.000000,1.000000}%
\pgfsetstrokecolor{currentstroke}%
\pgfsetdash{}{0pt}%
\pgfpathmoveto{\pgfqpoint{4.065737in}{1.109683in}}%
\pgfpathcurveto{\pgfqpoint{4.072870in}{1.109683in}}{\pgfqpoint{4.079711in}{1.112517in}}{\pgfqpoint{4.084755in}{1.117561in}}%
\pgfpathcurveto{\pgfqpoint{4.089799in}{1.122605in}}{\pgfqpoint{4.092633in}{1.129446in}}{\pgfqpoint{4.092633in}{1.136579in}}%
\pgfpathcurveto{\pgfqpoint{4.092633in}{1.143712in}}{\pgfqpoint{4.089799in}{1.150553in}}{\pgfqpoint{4.084755in}{1.155597in}}%
\pgfpathcurveto{\pgfqpoint{4.079711in}{1.160641in}}{\pgfqpoint{4.072870in}{1.163475in}}{\pgfqpoint{4.065737in}{1.163475in}}%
\pgfpathcurveto{\pgfqpoint{4.058604in}{1.163475in}}{\pgfqpoint{4.051762in}{1.160641in}}{\pgfqpoint{4.046719in}{1.155597in}}%
\pgfpathcurveto{\pgfqpoint{4.041675in}{1.150553in}}{\pgfqpoint{4.038841in}{1.143712in}}{\pgfqpoint{4.038841in}{1.136579in}}%
\pgfpathcurveto{\pgfqpoint{4.038841in}{1.129446in}}{\pgfqpoint{4.041675in}{1.122605in}}{\pgfqpoint{4.046719in}{1.117561in}}%
\pgfpathcurveto{\pgfqpoint{4.051762in}{1.112517in}}{\pgfqpoint{4.058604in}{1.109683in}}{\pgfqpoint{4.065737in}{1.109683in}}%
\pgfpathclose%
\pgfusepath{stroke,fill}%
\end{pgfscope}%
\begin{pgfscope}%
\pgfpathrectangle{\pgfqpoint{2.867647in}{0.500000in}}{\pgfqpoint{1.764706in}{1.700000in}}%
\pgfusepath{clip}%
\pgfsetbuttcap%
\pgfsetroundjoin%
\definecolor{currentfill}{rgb}{0.966812,0.762584,0.633643}%
\pgfsetfillcolor{currentfill}%
\pgfsetlinewidth{0.311001pt}%
\definecolor{currentstroke}{rgb}{1.000000,1.000000,1.000000}%
\pgfsetstrokecolor{currentstroke}%
\pgfsetdash{}{0pt}%
\pgfpathmoveto{\pgfqpoint{4.249941in}{1.554532in}}%
\pgfpathcurveto{\pgfqpoint{4.257073in}{1.554532in}}{\pgfqpoint{4.263915in}{1.557366in}}{\pgfqpoint{4.268959in}{1.562409in}}%
\pgfpathcurveto{\pgfqpoint{4.274002in}{1.567453in}}{\pgfqpoint{4.276836in}{1.574295in}}{\pgfqpoint{4.276836in}{1.581427in}}%
\pgfpathcurveto{\pgfqpoint{4.276836in}{1.588560in}}{\pgfqpoint{4.274002in}{1.595402in}}{\pgfqpoint{4.268959in}{1.600446in}}%
\pgfpathcurveto{\pgfqpoint{4.263915in}{1.605489in}}{\pgfqpoint{4.257073in}{1.608323in}}{\pgfqpoint{4.249941in}{1.608323in}}%
\pgfpathcurveto{\pgfqpoint{4.242808in}{1.608323in}}{\pgfqpoint{4.235966in}{1.605489in}}{\pgfqpoint{4.230922in}{1.600446in}}%
\pgfpathcurveto{\pgfqpoint{4.225879in}{1.595402in}}{\pgfqpoint{4.223045in}{1.588560in}}{\pgfqpoint{4.223045in}{1.581427in}}%
\pgfpathcurveto{\pgfqpoint{4.223045in}{1.574295in}}{\pgfqpoint{4.225879in}{1.567453in}}{\pgfqpoint{4.230922in}{1.562409in}}%
\pgfpathcurveto{\pgfqpoint{4.235966in}{1.557366in}}{\pgfqpoint{4.242808in}{1.554532in}}{\pgfqpoint{4.249941in}{1.554532in}}%
\pgfpathclose%
\pgfusepath{stroke,fill}%
\end{pgfscope}%
\begin{pgfscope}%
\pgfpathrectangle{\pgfqpoint{2.867647in}{0.500000in}}{\pgfqpoint{1.764706in}{1.700000in}}%
\pgfusepath{clip}%
\pgfsetbuttcap%
\pgfsetroundjoin%
\definecolor{currentfill}{rgb}{0.964920,0.695342,0.545192}%
\pgfsetfillcolor{currentfill}%
\pgfsetlinewidth{0.311001pt}%
\definecolor{currentstroke}{rgb}{1.000000,1.000000,1.000000}%
\pgfsetstrokecolor{currentstroke}%
\pgfsetdash{}{0pt}%
\pgfpathmoveto{\pgfqpoint{4.072000in}{1.293453in}}%
\pgfpathcurveto{\pgfqpoint{4.079133in}{1.293453in}}{\pgfqpoint{4.085975in}{1.296287in}}{\pgfqpoint{4.091018in}{1.301331in}}%
\pgfpathcurveto{\pgfqpoint{4.096062in}{1.306374in}}{\pgfqpoint{4.098896in}{1.313216in}}{\pgfqpoint{4.098896in}{1.320349in}}%
\pgfpathcurveto{\pgfqpoint{4.098896in}{1.327482in}}{\pgfqpoint{4.096062in}{1.334323in}}{\pgfqpoint{4.091018in}{1.339367in}}%
\pgfpathcurveto{\pgfqpoint{4.085975in}{1.344410in}}{\pgfqpoint{4.079133in}{1.347244in}}{\pgfqpoint{4.072000in}{1.347244in}}%
\pgfpathcurveto{\pgfqpoint{4.064867in}{1.347244in}}{\pgfqpoint{4.058026in}{1.344410in}}{\pgfqpoint{4.052982in}{1.339367in}}%
\pgfpathcurveto{\pgfqpoint{4.047938in}{1.334323in}}{\pgfqpoint{4.045104in}{1.327482in}}{\pgfqpoint{4.045104in}{1.320349in}}%
\pgfpathcurveto{\pgfqpoint{4.045104in}{1.313216in}}{\pgfqpoint{4.047938in}{1.306374in}}{\pgfqpoint{4.052982in}{1.301331in}}%
\pgfpathcurveto{\pgfqpoint{4.058026in}{1.296287in}}{\pgfqpoint{4.064867in}{1.293453in}}{\pgfqpoint{4.072000in}{1.293453in}}%
\pgfpathclose%
\pgfusepath{stroke,fill}%
\end{pgfscope}%
\begin{pgfscope}%
\pgfpathrectangle{\pgfqpoint{2.867647in}{0.500000in}}{\pgfqpoint{1.764706in}{1.700000in}}%
\pgfusepath{clip}%
\pgfsetbuttcap%
\pgfsetroundjoin%
\definecolor{currentfill}{rgb}{0.963884,0.644842,0.486120}%
\pgfsetfillcolor{currentfill}%
\pgfsetlinewidth{0.311001pt}%
\definecolor{currentstroke}{rgb}{1.000000,1.000000,1.000000}%
\pgfsetstrokecolor{currentstroke}%
\pgfsetdash{}{0pt}%
\pgfpathmoveto{\pgfqpoint{4.059288in}{1.342379in}}%
\pgfpathcurveto{\pgfqpoint{4.066421in}{1.342379in}}{\pgfqpoint{4.073263in}{1.345213in}}{\pgfqpoint{4.078306in}{1.350256in}}%
\pgfpathcurveto{\pgfqpoint{4.083350in}{1.355300in}}{\pgfqpoint{4.086184in}{1.362142in}}{\pgfqpoint{4.086184in}{1.369274in}}%
\pgfpathcurveto{\pgfqpoint{4.086184in}{1.376407in}}{\pgfqpoint{4.083350in}{1.383249in}}{\pgfqpoint{4.078306in}{1.388293in}}%
\pgfpathcurveto{\pgfqpoint{4.073263in}{1.393336in}}{\pgfqpoint{4.066421in}{1.396170in}}{\pgfqpoint{4.059288in}{1.396170in}}%
\pgfpathcurveto{\pgfqpoint{4.052155in}{1.396170in}}{\pgfqpoint{4.045314in}{1.393336in}}{\pgfqpoint{4.040270in}{1.388293in}}%
\pgfpathcurveto{\pgfqpoint{4.035226in}{1.383249in}}{\pgfqpoint{4.032392in}{1.376407in}}{\pgfqpoint{4.032392in}{1.369274in}}%
\pgfpathcurveto{\pgfqpoint{4.032392in}{1.362142in}}{\pgfqpoint{4.035226in}{1.355300in}}{\pgfqpoint{4.040270in}{1.350256in}}%
\pgfpathcurveto{\pgfqpoint{4.045314in}{1.345213in}}{\pgfqpoint{4.052155in}{1.342379in}}{\pgfqpoint{4.059288in}{1.342379in}}%
\pgfpathclose%
\pgfusepath{stroke,fill}%
\end{pgfscope}%
\begin{pgfscope}%
\pgfpathrectangle{\pgfqpoint{2.867647in}{0.500000in}}{\pgfqpoint{1.764706in}{1.700000in}}%
\pgfusepath{clip}%
\pgfsetbuttcap%
\pgfsetroundjoin%
\definecolor{currentfill}{rgb}{0.964920,0.695342,0.545192}%
\pgfsetfillcolor{currentfill}%
\pgfsetlinewidth{0.311001pt}%
\definecolor{currentstroke}{rgb}{1.000000,1.000000,1.000000}%
\pgfsetstrokecolor{currentstroke}%
\pgfsetdash{}{0pt}%
\pgfpathmoveto{\pgfqpoint{4.061309in}{1.769049in}}%
\pgfpathcurveto{\pgfqpoint{4.068442in}{1.769049in}}{\pgfqpoint{4.075284in}{1.771883in}}{\pgfqpoint{4.080327in}{1.776927in}}%
\pgfpathcurveto{\pgfqpoint{4.085371in}{1.781970in}}{\pgfqpoint{4.088205in}{1.788812in}}{\pgfqpoint{4.088205in}{1.795945in}}%
\pgfpathcurveto{\pgfqpoint{4.088205in}{1.803078in}}{\pgfqpoint{4.085371in}{1.809919in}}{\pgfqpoint{4.080327in}{1.814963in}}%
\pgfpathcurveto{\pgfqpoint{4.075284in}{1.820007in}}{\pgfqpoint{4.068442in}{1.822841in}}{\pgfqpoint{4.061309in}{1.822841in}}%
\pgfpathcurveto{\pgfqpoint{4.054176in}{1.822841in}}{\pgfqpoint{4.047335in}{1.820007in}}{\pgfqpoint{4.042291in}{1.814963in}}%
\pgfpathcurveto{\pgfqpoint{4.037247in}{1.809919in}}{\pgfqpoint{4.034414in}{1.803078in}}{\pgfqpoint{4.034414in}{1.795945in}}%
\pgfpathcurveto{\pgfqpoint{4.034414in}{1.788812in}}{\pgfqpoint{4.037247in}{1.781970in}}{\pgfqpoint{4.042291in}{1.776927in}}%
\pgfpathcurveto{\pgfqpoint{4.047335in}{1.771883in}}{\pgfqpoint{4.054176in}{1.769049in}}{\pgfqpoint{4.061309in}{1.769049in}}%
\pgfpathclose%
\pgfusepath{stroke,fill}%
\end{pgfscope}%
\begin{pgfscope}%
\pgfpathrectangle{\pgfqpoint{2.867647in}{0.500000in}}{\pgfqpoint{1.764706in}{1.700000in}}%
\pgfusepath{clip}%
\pgfsetbuttcap%
\pgfsetroundjoin%
\definecolor{currentfill}{rgb}{0.942910,0.375495,0.263698}%
\pgfsetfillcolor{currentfill}%
\pgfsetlinewidth{0.311001pt}%
\definecolor{currentstroke}{rgb}{1.000000,1.000000,1.000000}%
\pgfsetstrokecolor{currentstroke}%
\pgfsetdash{}{0pt}%
\pgfpathmoveto{\pgfqpoint{4.299491in}{1.642476in}}%
\pgfpathcurveto{\pgfqpoint{4.306624in}{1.642476in}}{\pgfqpoint{4.313466in}{1.645310in}}{\pgfqpoint{4.318510in}{1.650354in}}%
\pgfpathcurveto{\pgfqpoint{4.323553in}{1.655398in}}{\pgfqpoint{4.326387in}{1.662239in}}{\pgfqpoint{4.326387in}{1.669372in}}%
\pgfpathcurveto{\pgfqpoint{4.326387in}{1.676505in}}{\pgfqpoint{4.323553in}{1.683346in}}{\pgfqpoint{4.318510in}{1.688390in}}%
\pgfpathcurveto{\pgfqpoint{4.313466in}{1.693434in}}{\pgfqpoint{4.306624in}{1.696268in}}{\pgfqpoint{4.299491in}{1.696268in}}%
\pgfpathcurveto{\pgfqpoint{4.292359in}{1.696268in}}{\pgfqpoint{4.285517in}{1.693434in}}{\pgfqpoint{4.280473in}{1.688390in}}%
\pgfpathcurveto{\pgfqpoint{4.275430in}{1.683346in}}{\pgfqpoint{4.272596in}{1.676505in}}{\pgfqpoint{4.272596in}{1.669372in}}%
\pgfpathcurveto{\pgfqpoint{4.272596in}{1.662239in}}{\pgfqpoint{4.275430in}{1.655398in}}{\pgfqpoint{4.280473in}{1.650354in}}%
\pgfpathcurveto{\pgfqpoint{4.285517in}{1.645310in}}{\pgfqpoint{4.292359in}{1.642476in}}{\pgfqpoint{4.299491in}{1.642476in}}%
\pgfpathclose%
\pgfusepath{stroke,fill}%
\end{pgfscope}%
\begin{pgfscope}%
\pgfpathrectangle{\pgfqpoint{2.867647in}{0.500000in}}{\pgfqpoint{1.764706in}{1.700000in}}%
\pgfusepath{clip}%
\pgfsetbuttcap%
\pgfsetroundjoin%
\definecolor{currentfill}{rgb}{0.971694,0.833208,0.737161}%
\pgfsetfillcolor{currentfill}%
\pgfsetlinewidth{0.311001pt}%
\definecolor{currentstroke}{rgb}{1.000000,1.000000,1.000000}%
\pgfsetstrokecolor{currentstroke}%
\pgfsetdash{}{0pt}%
\pgfpathmoveto{\pgfqpoint{4.072856in}{1.097272in}}%
\pgfpathcurveto{\pgfqpoint{4.079989in}{1.097272in}}{\pgfqpoint{4.086831in}{1.100106in}}{\pgfqpoint{4.091874in}{1.105150in}}%
\pgfpathcurveto{\pgfqpoint{4.096918in}{1.110194in}}{\pgfqpoint{4.099752in}{1.117035in}}{\pgfqpoint{4.099752in}{1.124168in}}%
\pgfpathcurveto{\pgfqpoint{4.099752in}{1.131301in}}{\pgfqpoint{4.096918in}{1.138143in}}{\pgfqpoint{4.091874in}{1.143186in}}%
\pgfpathcurveto{\pgfqpoint{4.086831in}{1.148230in}}{\pgfqpoint{4.079989in}{1.151064in}}{\pgfqpoint{4.072856in}{1.151064in}}%
\pgfpathcurveto{\pgfqpoint{4.065723in}{1.151064in}}{\pgfqpoint{4.058882in}{1.148230in}}{\pgfqpoint{4.053838in}{1.143186in}}%
\pgfpathcurveto{\pgfqpoint{4.048794in}{1.138143in}}{\pgfqpoint{4.045961in}{1.131301in}}{\pgfqpoint{4.045961in}{1.124168in}}%
\pgfpathcurveto{\pgfqpoint{4.045961in}{1.117035in}}{\pgfqpoint{4.048794in}{1.110194in}}{\pgfqpoint{4.053838in}{1.105150in}}%
\pgfpathcurveto{\pgfqpoint{4.058882in}{1.100106in}}{\pgfqpoint{4.065723in}{1.097272in}}{\pgfqpoint{4.072856in}{1.097272in}}%
\pgfpathclose%
\pgfusepath{stroke,fill}%
\end{pgfscope}%
\begin{pgfscope}%
\pgfpathrectangle{\pgfqpoint{2.867647in}{0.500000in}}{\pgfqpoint{1.764706in}{1.700000in}}%
\pgfusepath{clip}%
\pgfsetbuttcap%
\pgfsetroundjoin%
\definecolor{currentfill}{rgb}{0.975018,0.868213,0.788710}%
\pgfsetfillcolor{currentfill}%
\pgfsetlinewidth{0.311001pt}%
\definecolor{currentstroke}{rgb}{1.000000,1.000000,1.000000}%
\pgfsetstrokecolor{currentstroke}%
\pgfsetdash{}{0pt}%
\pgfpathmoveto{\pgfqpoint{4.103594in}{1.487428in}}%
\pgfpathcurveto{\pgfqpoint{4.110727in}{1.487428in}}{\pgfqpoint{4.117568in}{1.490262in}}{\pgfqpoint{4.122612in}{1.495305in}}%
\pgfpathcurveto{\pgfqpoint{4.127656in}{1.500349in}}{\pgfqpoint{4.130489in}{1.507191in}}{\pgfqpoint{4.130489in}{1.514323in}}%
\pgfpathcurveto{\pgfqpoint{4.130489in}{1.521456in}}{\pgfqpoint{4.127656in}{1.528298in}}{\pgfqpoint{4.122612in}{1.533342in}}%
\pgfpathcurveto{\pgfqpoint{4.117568in}{1.538385in}}{\pgfqpoint{4.110727in}{1.541219in}}{\pgfqpoint{4.103594in}{1.541219in}}%
\pgfpathcurveto{\pgfqpoint{4.096461in}{1.541219in}}{\pgfqpoint{4.089619in}{1.538385in}}{\pgfqpoint{4.084576in}{1.533342in}}%
\pgfpathcurveto{\pgfqpoint{4.079532in}{1.528298in}}{\pgfqpoint{4.076698in}{1.521456in}}{\pgfqpoint{4.076698in}{1.514323in}}%
\pgfpathcurveto{\pgfqpoint{4.076698in}{1.507191in}}{\pgfqpoint{4.079532in}{1.500349in}}{\pgfqpoint{4.084576in}{1.495305in}}%
\pgfpathcurveto{\pgfqpoint{4.089619in}{1.490262in}}{\pgfqpoint{4.096461in}{1.487428in}}{\pgfqpoint{4.103594in}{1.487428in}}%
\pgfpathclose%
\pgfusepath{stroke,fill}%
\end{pgfscope}%
\begin{pgfscope}%
\pgfpathrectangle{\pgfqpoint{2.867647in}{0.500000in}}{\pgfqpoint{1.764706in}{1.700000in}}%
\pgfusepath{clip}%
\pgfsetbuttcap%
\pgfsetroundjoin%
\definecolor{currentfill}{rgb}{0.980678,0.914765,0.856766}%
\pgfsetfillcolor{currentfill}%
\pgfsetlinewidth{0.311001pt}%
\definecolor{currentstroke}{rgb}{1.000000,1.000000,1.000000}%
\pgfsetstrokecolor{currentstroke}%
\pgfsetdash{}{0pt}%
\pgfpathmoveto{\pgfqpoint{4.153048in}{1.202962in}}%
\pgfpathcurveto{\pgfqpoint{4.160181in}{1.202962in}}{\pgfqpoint{4.167023in}{1.205796in}}{\pgfqpoint{4.172066in}{1.210839in}}%
\pgfpathcurveto{\pgfqpoint{4.177110in}{1.215883in}}{\pgfqpoint{4.179944in}{1.222725in}}{\pgfqpoint{4.179944in}{1.229857in}}%
\pgfpathcurveto{\pgfqpoint{4.179944in}{1.236990in}}{\pgfqpoint{4.177110in}{1.243832in}}{\pgfqpoint{4.172066in}{1.248876in}}%
\pgfpathcurveto{\pgfqpoint{4.167023in}{1.253919in}}{\pgfqpoint{4.160181in}{1.256753in}}{\pgfqpoint{4.153048in}{1.256753in}}%
\pgfpathcurveto{\pgfqpoint{4.145916in}{1.256753in}}{\pgfqpoint{4.139074in}{1.253919in}}{\pgfqpoint{4.134030in}{1.248876in}}%
\pgfpathcurveto{\pgfqpoint{4.128987in}{1.243832in}}{\pgfqpoint{4.126153in}{1.236990in}}{\pgfqpoint{4.126153in}{1.229857in}}%
\pgfpathcurveto{\pgfqpoint{4.126153in}{1.222725in}}{\pgfqpoint{4.128987in}{1.215883in}}{\pgfqpoint{4.134030in}{1.210839in}}%
\pgfpathcurveto{\pgfqpoint{4.139074in}{1.205796in}}{\pgfqpoint{4.145916in}{1.202962in}}{\pgfqpoint{4.153048in}{1.202962in}}%
\pgfpathclose%
\pgfusepath{stroke,fill}%
\end{pgfscope}%
\begin{pgfscope}%
\pgfpathrectangle{\pgfqpoint{2.867647in}{0.500000in}}{\pgfqpoint{1.764706in}{1.700000in}}%
\pgfusepath{clip}%
\pgfsetbuttcap%
\pgfsetroundjoin%
\definecolor{currentfill}{rgb}{0.978376,0.897317,0.831308}%
\pgfsetfillcolor{currentfill}%
\pgfsetlinewidth{0.311001pt}%
\definecolor{currentstroke}{rgb}{1.000000,1.000000,1.000000}%
\pgfsetstrokecolor{currentstroke}%
\pgfsetdash{}{0pt}%
\pgfpathmoveto{\pgfqpoint{4.162846in}{1.080637in}}%
\pgfpathcurveto{\pgfqpoint{4.169978in}{1.080637in}}{\pgfqpoint{4.176820in}{1.083471in}}{\pgfqpoint{4.181864in}{1.088515in}}%
\pgfpathcurveto{\pgfqpoint{4.186907in}{1.093558in}}{\pgfqpoint{4.189741in}{1.100400in}}{\pgfqpoint{4.189741in}{1.107533in}}%
\pgfpathcurveto{\pgfqpoint{4.189741in}{1.114666in}}{\pgfqpoint{4.186907in}{1.121507in}}{\pgfqpoint{4.181864in}{1.126551in}}%
\pgfpathcurveto{\pgfqpoint{4.176820in}{1.131594in}}{\pgfqpoint{4.169978in}{1.134428in}}{\pgfqpoint{4.162846in}{1.134428in}}%
\pgfpathcurveto{\pgfqpoint{4.155713in}{1.134428in}}{\pgfqpoint{4.148871in}{1.131594in}}{\pgfqpoint{4.143827in}{1.126551in}}%
\pgfpathcurveto{\pgfqpoint{4.138784in}{1.121507in}}{\pgfqpoint{4.135950in}{1.114666in}}{\pgfqpoint{4.135950in}{1.107533in}}%
\pgfpathcurveto{\pgfqpoint{4.135950in}{1.100400in}}{\pgfqpoint{4.138784in}{1.093558in}}{\pgfqpoint{4.143827in}{1.088515in}}%
\pgfpathcurveto{\pgfqpoint{4.148871in}{1.083471in}}{\pgfqpoint{4.155713in}{1.080637in}}{\pgfqpoint{4.162846in}{1.080637in}}%
\pgfpathclose%
\pgfusepath{stroke,fill}%
\end{pgfscope}%
\begin{pgfscope}%
\pgfpathrectangle{\pgfqpoint{2.867647in}{0.500000in}}{\pgfqpoint{1.764706in}{1.700000in}}%
\pgfusepath{clip}%
\pgfsetbuttcap%
\pgfsetroundjoin%
\definecolor{currentfill}{rgb}{0.979124,0.903132,0.839793}%
\pgfsetfillcolor{currentfill}%
\pgfsetlinewidth{0.311001pt}%
\definecolor{currentstroke}{rgb}{1.000000,1.000000,1.000000}%
\pgfsetstrokecolor{currentstroke}%
\pgfsetdash{}{0pt}%
\pgfpathmoveto{\pgfqpoint{4.214053in}{1.190156in}}%
\pgfpathcurveto{\pgfqpoint{4.221185in}{1.190156in}}{\pgfqpoint{4.228027in}{1.192989in}}{\pgfqpoint{4.233071in}{1.198033in}}%
\pgfpathcurveto{\pgfqpoint{4.238114in}{1.203077in}}{\pgfqpoint{4.240948in}{1.209918in}}{\pgfqpoint{4.240948in}{1.217051in}}%
\pgfpathcurveto{\pgfqpoint{4.240948in}{1.224184in}}{\pgfqpoint{4.238114in}{1.231026in}}{\pgfqpoint{4.233071in}{1.236069in}}%
\pgfpathcurveto{\pgfqpoint{4.228027in}{1.241113in}}{\pgfqpoint{4.221185in}{1.243947in}}{\pgfqpoint{4.214053in}{1.243947in}}%
\pgfpathcurveto{\pgfqpoint{4.206920in}{1.243947in}}{\pgfqpoint{4.200078in}{1.241113in}}{\pgfqpoint{4.195034in}{1.236069in}}%
\pgfpathcurveto{\pgfqpoint{4.189991in}{1.231026in}}{\pgfqpoint{4.187157in}{1.224184in}}{\pgfqpoint{4.187157in}{1.217051in}}%
\pgfpathcurveto{\pgfqpoint{4.187157in}{1.209918in}}{\pgfqpoint{4.189991in}{1.203077in}}{\pgfqpoint{4.195034in}{1.198033in}}%
\pgfpathcurveto{\pgfqpoint{4.200078in}{1.192989in}}{\pgfqpoint{4.206920in}{1.190156in}}{\pgfqpoint{4.214053in}{1.190156in}}%
\pgfpathclose%
\pgfusepath{stroke,fill}%
\end{pgfscope}%
\begin{pgfscope}%
\pgfpathrectangle{\pgfqpoint{2.867647in}{0.500000in}}{\pgfqpoint{1.764706in}{1.700000in}}%
\pgfusepath{clip}%
\pgfsetbuttcap%
\pgfsetroundjoin%
\definecolor{currentfill}{rgb}{0.968931,0.798091,0.685123}%
\pgfsetfillcolor{currentfill}%
\pgfsetlinewidth{0.311001pt}%
\definecolor{currentstroke}{rgb}{1.000000,1.000000,1.000000}%
\pgfsetstrokecolor{currentstroke}%
\pgfsetdash{}{0pt}%
\pgfpathmoveto{\pgfqpoint{4.049875in}{1.548848in}}%
\pgfpathcurveto{\pgfqpoint{4.057008in}{1.548848in}}{\pgfqpoint{4.063849in}{1.551682in}}{\pgfqpoint{4.068893in}{1.556726in}}%
\pgfpathcurveto{\pgfqpoint{4.073937in}{1.561769in}}{\pgfqpoint{4.076770in}{1.568611in}}{\pgfqpoint{4.076770in}{1.575744in}}%
\pgfpathcurveto{\pgfqpoint{4.076770in}{1.582877in}}{\pgfqpoint{4.073937in}{1.589718in}}{\pgfqpoint{4.068893in}{1.594762in}}%
\pgfpathcurveto{\pgfqpoint{4.063849in}{1.599806in}}{\pgfqpoint{4.057008in}{1.602640in}}{\pgfqpoint{4.049875in}{1.602640in}}%
\pgfpathcurveto{\pgfqpoint{4.042742in}{1.602640in}}{\pgfqpoint{4.035900in}{1.599806in}}{\pgfqpoint{4.030857in}{1.594762in}}%
\pgfpathcurveto{\pgfqpoint{4.025813in}{1.589718in}}{\pgfqpoint{4.022979in}{1.582877in}}{\pgfqpoint{4.022979in}{1.575744in}}%
\pgfpathcurveto{\pgfqpoint{4.022979in}{1.568611in}}{\pgfqpoint{4.025813in}{1.561769in}}{\pgfqpoint{4.030857in}{1.556726in}}%
\pgfpathcurveto{\pgfqpoint{4.035900in}{1.551682in}}{\pgfqpoint{4.042742in}{1.548848in}}{\pgfqpoint{4.049875in}{1.548848in}}%
\pgfpathclose%
\pgfusepath{stroke,fill}%
\end{pgfscope}%
\begin{pgfscope}%
\pgfpathrectangle{\pgfqpoint{2.867647in}{0.500000in}}{\pgfqpoint{1.764706in}{1.700000in}}%
\pgfusepath{clip}%
\pgfsetbuttcap%
\pgfsetroundjoin%
\definecolor{currentfill}{rgb}{0.980678,0.914765,0.856766}%
\pgfsetfillcolor{currentfill}%
\pgfsetlinewidth{0.311001pt}%
\definecolor{currentstroke}{rgb}{1.000000,1.000000,1.000000}%
\pgfsetstrokecolor{currentstroke}%
\pgfsetdash{}{0pt}%
\pgfpathmoveto{\pgfqpoint{4.199489in}{1.394330in}}%
\pgfpathcurveto{\pgfqpoint{4.206621in}{1.394330in}}{\pgfqpoint{4.213463in}{1.397164in}}{\pgfqpoint{4.218507in}{1.402208in}}%
\pgfpathcurveto{\pgfqpoint{4.223550in}{1.407252in}}{\pgfqpoint{4.226384in}{1.414093in}}{\pgfqpoint{4.226384in}{1.421226in}}%
\pgfpathcurveto{\pgfqpoint{4.226384in}{1.428359in}}{\pgfqpoint{4.223550in}{1.435200in}}{\pgfqpoint{4.218507in}{1.440244in}}%
\pgfpathcurveto{\pgfqpoint{4.213463in}{1.445288in}}{\pgfqpoint{4.206621in}{1.448122in}}{\pgfqpoint{4.199489in}{1.448122in}}%
\pgfpathcurveto{\pgfqpoint{4.192356in}{1.448122in}}{\pgfqpoint{4.185514in}{1.445288in}}{\pgfqpoint{4.180470in}{1.440244in}}%
\pgfpathcurveto{\pgfqpoint{4.175427in}{1.435200in}}{\pgfqpoint{4.172593in}{1.428359in}}{\pgfqpoint{4.172593in}{1.421226in}}%
\pgfpathcurveto{\pgfqpoint{4.172593in}{1.414093in}}{\pgfqpoint{4.175427in}{1.407252in}}{\pgfqpoint{4.180470in}{1.402208in}}%
\pgfpathcurveto{\pgfqpoint{4.185514in}{1.397164in}}{\pgfqpoint{4.192356in}{1.394330in}}{\pgfqpoint{4.199489in}{1.394330in}}%
\pgfpathclose%
\pgfusepath{stroke,fill}%
\end{pgfscope}%
\begin{pgfscope}%
\pgfpathrectangle{\pgfqpoint{2.867647in}{0.500000in}}{\pgfqpoint{1.764706in}{1.700000in}}%
\pgfusepath{clip}%
\pgfsetbuttcap%
\pgfsetroundjoin%
\definecolor{currentfill}{rgb}{0.577499,0.110312,0.358417}%
\pgfsetfillcolor{currentfill}%
\pgfsetlinewidth{0.311001pt}%
\definecolor{currentstroke}{rgb}{1.000000,1.000000,1.000000}%
\pgfsetstrokecolor{currentstroke}%
\pgfsetdash{}{0pt}%
\pgfpathmoveto{\pgfqpoint{4.050628in}{0.698879in}}%
\pgfpathcurveto{\pgfqpoint{4.057761in}{0.698879in}}{\pgfqpoint{4.064603in}{0.701713in}}{\pgfqpoint{4.069646in}{0.706756in}}%
\pgfpathcurveto{\pgfqpoint{4.074690in}{0.711800in}}{\pgfqpoint{4.077524in}{0.718642in}}{\pgfqpoint{4.077524in}{0.725774in}}%
\pgfpathcurveto{\pgfqpoint{4.077524in}{0.732907in}}{\pgfqpoint{4.074690in}{0.739749in}}{\pgfqpoint{4.069646in}{0.744793in}}%
\pgfpathcurveto{\pgfqpoint{4.064603in}{0.749836in}}{\pgfqpoint{4.057761in}{0.752670in}}{\pgfqpoint{4.050628in}{0.752670in}}%
\pgfpathcurveto{\pgfqpoint{4.043495in}{0.752670in}}{\pgfqpoint{4.036654in}{0.749836in}}{\pgfqpoint{4.031610in}{0.744793in}}%
\pgfpathcurveto{\pgfqpoint{4.026566in}{0.739749in}}{\pgfqpoint{4.023732in}{0.732907in}}{\pgfqpoint{4.023732in}{0.725774in}}%
\pgfpathcurveto{\pgfqpoint{4.023732in}{0.718642in}}{\pgfqpoint{4.026566in}{0.711800in}}{\pgfqpoint{4.031610in}{0.706756in}}%
\pgfpathcurveto{\pgfqpoint{4.036654in}{0.701713in}}{\pgfqpoint{4.043495in}{0.698879in}}{\pgfqpoint{4.050628in}{0.698879in}}%
\pgfpathclose%
\pgfusepath{stroke,fill}%
\end{pgfscope}%
\begin{pgfscope}%
\pgfpathrectangle{\pgfqpoint{2.867647in}{0.500000in}}{\pgfqpoint{1.764706in}{1.700000in}}%
\pgfusepath{clip}%
\pgfsetbuttcap%
\pgfsetroundjoin%
\definecolor{currentfill}{rgb}{0.977657,0.891500,0.822809}%
\pgfsetfillcolor{currentfill}%
\pgfsetlinewidth{0.311001pt}%
\definecolor{currentstroke}{rgb}{1.000000,1.000000,1.000000}%
\pgfsetstrokecolor{currentstroke}%
\pgfsetdash{}{0pt}%
\pgfpathmoveto{\pgfqpoint{4.208338in}{1.481543in}}%
\pgfpathcurveto{\pgfqpoint{4.215470in}{1.481543in}}{\pgfqpoint{4.222312in}{1.484377in}}{\pgfqpoint{4.227356in}{1.489420in}}%
\pgfpathcurveto{\pgfqpoint{4.232399in}{1.494464in}}{\pgfqpoint{4.235233in}{1.501306in}}{\pgfqpoint{4.235233in}{1.508438in}}%
\pgfpathcurveto{\pgfqpoint{4.235233in}{1.515571in}}{\pgfqpoint{4.232399in}{1.522413in}}{\pgfqpoint{4.227356in}{1.527457in}}%
\pgfpathcurveto{\pgfqpoint{4.222312in}{1.532500in}}{\pgfqpoint{4.215470in}{1.535334in}}{\pgfqpoint{4.208338in}{1.535334in}}%
\pgfpathcurveto{\pgfqpoint{4.201205in}{1.535334in}}{\pgfqpoint{4.194363in}{1.532500in}}{\pgfqpoint{4.189319in}{1.527457in}}%
\pgfpathcurveto{\pgfqpoint{4.184276in}{1.522413in}}{\pgfqpoint{4.181442in}{1.515571in}}{\pgfqpoint{4.181442in}{1.508438in}}%
\pgfpathcurveto{\pgfqpoint{4.181442in}{1.501306in}}{\pgfqpoint{4.184276in}{1.494464in}}{\pgfqpoint{4.189319in}{1.489420in}}%
\pgfpathcurveto{\pgfqpoint{4.194363in}{1.484377in}}{\pgfqpoint{4.201205in}{1.481543in}}{\pgfqpoint{4.208338in}{1.481543in}}%
\pgfpathclose%
\pgfusepath{stroke,fill}%
\end{pgfscope}%
\begin{pgfscope}%
\pgfpathrectangle{\pgfqpoint{2.867647in}{0.500000in}}{\pgfqpoint{1.764706in}{1.700000in}}%
\pgfusepath{clip}%
\pgfsetbuttcap%
\pgfsetroundjoin%
\definecolor{currentfill}{rgb}{0.959229,0.533075,0.374889}%
\pgfsetfillcolor{currentfill}%
\pgfsetlinewidth{0.311001pt}%
\definecolor{currentstroke}{rgb}{1.000000,1.000000,1.000000}%
\pgfsetstrokecolor{currentstroke}%
\pgfsetdash{}{0pt}%
\pgfpathmoveto{\pgfqpoint{4.195325in}{0.904382in}}%
\pgfpathcurveto{\pgfqpoint{4.202458in}{0.904382in}}{\pgfqpoint{4.209299in}{0.907216in}}{\pgfqpoint{4.214343in}{0.912260in}}%
\pgfpathcurveto{\pgfqpoint{4.219387in}{0.917303in}}{\pgfqpoint{4.222221in}{0.924145in}}{\pgfqpoint{4.222221in}{0.931278in}}%
\pgfpathcurveto{\pgfqpoint{4.222221in}{0.938411in}}{\pgfqpoint{4.219387in}{0.945252in}}{\pgfqpoint{4.214343in}{0.950296in}}%
\pgfpathcurveto{\pgfqpoint{4.209299in}{0.955340in}}{\pgfqpoint{4.202458in}{0.958174in}}{\pgfqpoint{4.195325in}{0.958174in}}%
\pgfpathcurveto{\pgfqpoint{4.188192in}{0.958174in}}{\pgfqpoint{4.181350in}{0.955340in}}{\pgfqpoint{4.176307in}{0.950296in}}%
\pgfpathcurveto{\pgfqpoint{4.171263in}{0.945252in}}{\pgfqpoint{4.168429in}{0.938411in}}{\pgfqpoint{4.168429in}{0.931278in}}%
\pgfpathcurveto{\pgfqpoint{4.168429in}{0.924145in}}{\pgfqpoint{4.171263in}{0.917303in}}{\pgfqpoint{4.176307in}{0.912260in}}%
\pgfpathcurveto{\pgfqpoint{4.181350in}{0.907216in}}{\pgfqpoint{4.188192in}{0.904382in}}{\pgfqpoint{4.195325in}{0.904382in}}%
\pgfpathclose%
\pgfusepath{stroke,fill}%
\end{pgfscope}%
\begin{pgfscope}%
\pgfpathrectangle{\pgfqpoint{2.867647in}{0.500000in}}{\pgfqpoint{1.764706in}{1.700000in}}%
\pgfusepath{clip}%
\pgfsetbuttcap%
\pgfsetroundjoin%
\definecolor{currentfill}{rgb}{0.970718,0.821518,0.719872}%
\pgfsetfillcolor{currentfill}%
\pgfsetlinewidth{0.311001pt}%
\definecolor{currentstroke}{rgb}{1.000000,1.000000,1.000000}%
\pgfsetstrokecolor{currentstroke}%
\pgfsetdash{}{0pt}%
\pgfpathmoveto{\pgfqpoint{4.066643in}{1.098172in}}%
\pgfpathcurveto{\pgfqpoint{4.073776in}{1.098172in}}{\pgfqpoint{4.080617in}{1.101006in}}{\pgfqpoint{4.085661in}{1.106050in}}%
\pgfpathcurveto{\pgfqpoint{4.090705in}{1.111093in}}{\pgfqpoint{4.093539in}{1.117935in}}{\pgfqpoint{4.093539in}{1.125068in}}%
\pgfpathcurveto{\pgfqpoint{4.093539in}{1.132201in}}{\pgfqpoint{4.090705in}{1.139042in}}{\pgfqpoint{4.085661in}{1.144086in}}%
\pgfpathcurveto{\pgfqpoint{4.080617in}{1.149130in}}{\pgfqpoint{4.073776in}{1.151964in}}{\pgfqpoint{4.066643in}{1.151964in}}%
\pgfpathcurveto{\pgfqpoint{4.059510in}{1.151964in}}{\pgfqpoint{4.052668in}{1.149130in}}{\pgfqpoint{4.047625in}{1.144086in}}%
\pgfpathcurveto{\pgfqpoint{4.042581in}{1.139042in}}{\pgfqpoint{4.039747in}{1.132201in}}{\pgfqpoint{4.039747in}{1.125068in}}%
\pgfpathcurveto{\pgfqpoint{4.039747in}{1.117935in}}{\pgfqpoint{4.042581in}{1.111093in}}{\pgfqpoint{4.047625in}{1.106050in}}%
\pgfpathcurveto{\pgfqpoint{4.052668in}{1.101006in}}{\pgfqpoint{4.059510in}{1.098172in}}{\pgfqpoint{4.066643in}{1.098172in}}%
\pgfpathclose%
\pgfusepath{stroke,fill}%
\end{pgfscope}%
\begin{pgfscope}%
\pgfpathrectangle{\pgfqpoint{2.867647in}{0.500000in}}{\pgfqpoint{1.764706in}{1.700000in}}%
\pgfusepath{clip}%
\pgfsetbuttcap%
\pgfsetroundjoin%
\definecolor{currentfill}{rgb}{0.965302,0.713942,0.568499}%
\pgfsetfillcolor{currentfill}%
\pgfsetlinewidth{0.311001pt}%
\definecolor{currentstroke}{rgb}{1.000000,1.000000,1.000000}%
\pgfsetstrokecolor{currentstroke}%
\pgfsetdash{}{0pt}%
\pgfpathmoveto{\pgfqpoint{4.042516in}{1.488843in}}%
\pgfpathcurveto{\pgfqpoint{4.049649in}{1.488843in}}{\pgfqpoint{4.056491in}{1.491677in}}{\pgfqpoint{4.061535in}{1.496721in}}%
\pgfpathcurveto{\pgfqpoint{4.066578in}{1.501765in}}{\pgfqpoint{4.069412in}{1.508606in}}{\pgfqpoint{4.069412in}{1.515739in}}%
\pgfpathcurveto{\pgfqpoint{4.069412in}{1.522872in}}{\pgfqpoint{4.066578in}{1.529714in}}{\pgfqpoint{4.061535in}{1.534757in}}%
\pgfpathcurveto{\pgfqpoint{4.056491in}{1.539801in}}{\pgfqpoint{4.049649in}{1.542635in}}{\pgfqpoint{4.042516in}{1.542635in}}%
\pgfpathcurveto{\pgfqpoint{4.035384in}{1.542635in}}{\pgfqpoint{4.028542in}{1.539801in}}{\pgfqpoint{4.023498in}{1.534757in}}%
\pgfpathcurveto{\pgfqpoint{4.018455in}{1.529714in}}{\pgfqpoint{4.015621in}{1.522872in}}{\pgfqpoint{4.015621in}{1.515739in}}%
\pgfpathcurveto{\pgfqpoint{4.015621in}{1.508606in}}{\pgfqpoint{4.018455in}{1.501765in}}{\pgfqpoint{4.023498in}{1.496721in}}%
\pgfpathcurveto{\pgfqpoint{4.028542in}{1.491677in}}{\pgfqpoint{4.035384in}{1.488843in}}{\pgfqpoint{4.042516in}{1.488843in}}%
\pgfpathclose%
\pgfusepath{stroke,fill}%
\end{pgfscope}%
\begin{pgfscope}%
\pgfpathrectangle{\pgfqpoint{2.867647in}{0.500000in}}{\pgfqpoint{1.764706in}{1.700000in}}%
\pgfusepath{clip}%
\pgfsetbuttcap%
\pgfsetroundjoin%
\definecolor{currentfill}{rgb}{0.968105,0.786346,0.667739}%
\pgfsetfillcolor{currentfill}%
\pgfsetlinewidth{0.311001pt}%
\definecolor{currentstroke}{rgb}{1.000000,1.000000,1.000000}%
\pgfsetstrokecolor{currentstroke}%
\pgfsetdash{}{0pt}%
\pgfpathmoveto{\pgfqpoint{4.071307in}{1.164845in}}%
\pgfpathcurveto{\pgfqpoint{4.078440in}{1.164845in}}{\pgfqpoint{4.085282in}{1.167678in}}{\pgfqpoint{4.090325in}{1.172722in}}%
\pgfpathcurveto{\pgfqpoint{4.095369in}{1.177766in}}{\pgfqpoint{4.098203in}{1.184607in}}{\pgfqpoint{4.098203in}{1.191740in}}%
\pgfpathcurveto{\pgfqpoint{4.098203in}{1.198873in}}{\pgfqpoint{4.095369in}{1.205715in}}{\pgfqpoint{4.090325in}{1.210758in}}%
\pgfpathcurveto{\pgfqpoint{4.085282in}{1.215802in}}{\pgfqpoint{4.078440in}{1.218636in}}{\pgfqpoint{4.071307in}{1.218636in}}%
\pgfpathcurveto{\pgfqpoint{4.064174in}{1.218636in}}{\pgfqpoint{4.057333in}{1.215802in}}{\pgfqpoint{4.052289in}{1.210758in}}%
\pgfpathcurveto{\pgfqpoint{4.047245in}{1.205715in}}{\pgfqpoint{4.044412in}{1.198873in}}{\pgfqpoint{4.044412in}{1.191740in}}%
\pgfpathcurveto{\pgfqpoint{4.044412in}{1.184607in}}{\pgfqpoint{4.047245in}{1.177766in}}{\pgfqpoint{4.052289in}{1.172722in}}%
\pgfpathcurveto{\pgfqpoint{4.057333in}{1.167678in}}{\pgfqpoint{4.064174in}{1.164845in}}{\pgfqpoint{4.071307in}{1.164845in}}%
\pgfpathclose%
\pgfusepath{stroke,fill}%
\end{pgfscope}%
\begin{pgfscope}%
\pgfpathrectangle{\pgfqpoint{2.867647in}{0.500000in}}{\pgfqpoint{1.764706in}{1.700000in}}%
\pgfusepath{clip}%
\pgfsetbuttcap%
\pgfsetroundjoin%
\definecolor{currentfill}{rgb}{0.972726,0.844889,0.754401}%
\pgfsetfillcolor{currentfill}%
\pgfsetlinewidth{0.311001pt}%
\definecolor{currentstroke}{rgb}{1.000000,1.000000,1.000000}%
\pgfsetstrokecolor{currentstroke}%
\pgfsetdash{}{0pt}%
\pgfpathmoveto{\pgfqpoint{4.068506in}{1.034124in}}%
\pgfpathcurveto{\pgfqpoint{4.075638in}{1.034124in}}{\pgfqpoint{4.082480in}{1.036958in}}{\pgfqpoint{4.087524in}{1.042002in}}%
\pgfpathcurveto{\pgfqpoint{4.092567in}{1.047045in}}{\pgfqpoint{4.095401in}{1.053887in}}{\pgfqpoint{4.095401in}{1.061020in}}%
\pgfpathcurveto{\pgfqpoint{4.095401in}{1.068153in}}{\pgfqpoint{4.092567in}{1.074994in}}{\pgfqpoint{4.087524in}{1.080038in}}%
\pgfpathcurveto{\pgfqpoint{4.082480in}{1.085082in}}{\pgfqpoint{4.075638in}{1.087916in}}{\pgfqpoint{4.068506in}{1.087916in}}%
\pgfpathcurveto{\pgfqpoint{4.061373in}{1.087916in}}{\pgfqpoint{4.054531in}{1.085082in}}{\pgfqpoint{4.049487in}{1.080038in}}%
\pgfpathcurveto{\pgfqpoint{4.044444in}{1.074994in}}{\pgfqpoint{4.041610in}{1.068153in}}{\pgfqpoint{4.041610in}{1.061020in}}%
\pgfpathcurveto{\pgfqpoint{4.041610in}{1.053887in}}{\pgfqpoint{4.044444in}{1.047045in}}{\pgfqpoint{4.049487in}{1.042002in}}%
\pgfpathcurveto{\pgfqpoint{4.054531in}{1.036958in}}{\pgfqpoint{4.061373in}{1.034124in}}{\pgfqpoint{4.068506in}{1.034124in}}%
\pgfpathclose%
\pgfusepath{stroke,fill}%
\end{pgfscope}%
\begin{pgfscope}%
\pgfpathrectangle{\pgfqpoint{2.867647in}{0.500000in}}{\pgfqpoint{1.764706in}{1.700000in}}%
\pgfusepath{clip}%
\pgfsetbuttcap%
\pgfsetroundjoin%
\definecolor{currentfill}{rgb}{0.979124,0.903132,0.839793}%
\pgfsetfillcolor{currentfill}%
\pgfsetlinewidth{0.311001pt}%
\definecolor{currentstroke}{rgb}{1.000000,1.000000,1.000000}%
\pgfsetstrokecolor{currentstroke}%
\pgfsetdash{}{0pt}%
\pgfpathmoveto{\pgfqpoint{4.155053in}{1.271333in}}%
\pgfpathcurveto{\pgfqpoint{4.162185in}{1.271333in}}{\pgfqpoint{4.169027in}{1.274167in}}{\pgfqpoint{4.174071in}{1.279210in}}%
\pgfpathcurveto{\pgfqpoint{4.179114in}{1.284254in}}{\pgfqpoint{4.181948in}{1.291096in}}{\pgfqpoint{4.181948in}{1.298228in}}%
\pgfpathcurveto{\pgfqpoint{4.181948in}{1.305361in}}{\pgfqpoint{4.179114in}{1.312203in}}{\pgfqpoint{4.174071in}{1.317246in}}%
\pgfpathcurveto{\pgfqpoint{4.169027in}{1.322290in}}{\pgfqpoint{4.162185in}{1.325124in}}{\pgfqpoint{4.155053in}{1.325124in}}%
\pgfpathcurveto{\pgfqpoint{4.147920in}{1.325124in}}{\pgfqpoint{4.141078in}{1.322290in}}{\pgfqpoint{4.136034in}{1.317246in}}%
\pgfpathcurveto{\pgfqpoint{4.130991in}{1.312203in}}{\pgfqpoint{4.128157in}{1.305361in}}{\pgfqpoint{4.128157in}{1.298228in}}%
\pgfpathcurveto{\pgfqpoint{4.128157in}{1.291096in}}{\pgfqpoint{4.130991in}{1.284254in}}{\pgfqpoint{4.136034in}{1.279210in}}%
\pgfpathcurveto{\pgfqpoint{4.141078in}{1.274167in}}{\pgfqpoint{4.147920in}{1.271333in}}{\pgfqpoint{4.155053in}{1.271333in}}%
\pgfpathclose%
\pgfusepath{stroke,fill}%
\end{pgfscope}%
\begin{pgfscope}%
\pgfpathrectangle{\pgfqpoint{2.867647in}{0.500000in}}{\pgfqpoint{1.764706in}{1.700000in}}%
\pgfusepath{clip}%
\pgfsetbuttcap%
\pgfsetroundjoin%
\definecolor{currentfill}{rgb}{0.975644,0.874038,0.797253}%
\pgfsetfillcolor{currentfill}%
\pgfsetlinewidth{0.311001pt}%
\definecolor{currentstroke}{rgb}{1.000000,1.000000,1.000000}%
\pgfsetstrokecolor{currentstroke}%
\pgfsetdash{}{0pt}%
\pgfpathmoveto{\pgfqpoint{4.108539in}{1.486402in}}%
\pgfpathcurveto{\pgfqpoint{4.115672in}{1.486402in}}{\pgfqpoint{4.122514in}{1.489236in}}{\pgfqpoint{4.127557in}{1.494280in}}%
\pgfpathcurveto{\pgfqpoint{4.132601in}{1.499324in}}{\pgfqpoint{4.135435in}{1.506165in}}{\pgfqpoint{4.135435in}{1.513298in}}%
\pgfpathcurveto{\pgfqpoint{4.135435in}{1.520431in}}{\pgfqpoint{4.132601in}{1.527273in}}{\pgfqpoint{4.127557in}{1.532316in}}%
\pgfpathcurveto{\pgfqpoint{4.122514in}{1.537360in}}{\pgfqpoint{4.115672in}{1.540194in}}{\pgfqpoint{4.108539in}{1.540194in}}%
\pgfpathcurveto{\pgfqpoint{4.101406in}{1.540194in}}{\pgfqpoint{4.094565in}{1.537360in}}{\pgfqpoint{4.089521in}{1.532316in}}%
\pgfpathcurveto{\pgfqpoint{4.084477in}{1.527273in}}{\pgfqpoint{4.081643in}{1.520431in}}{\pgfqpoint{4.081643in}{1.513298in}}%
\pgfpathcurveto{\pgfqpoint{4.081643in}{1.506165in}}{\pgfqpoint{4.084477in}{1.499324in}}{\pgfqpoint{4.089521in}{1.494280in}}%
\pgfpathcurveto{\pgfqpoint{4.094565in}{1.489236in}}{\pgfqpoint{4.101406in}{1.486402in}}{\pgfqpoint{4.108539in}{1.486402in}}%
\pgfpathclose%
\pgfusepath{stroke,fill}%
\end{pgfscope}%
\begin{pgfscope}%
\pgfpathrectangle{\pgfqpoint{2.867647in}{0.500000in}}{\pgfqpoint{1.764706in}{1.700000in}}%
\pgfusepath{clip}%
\pgfsetbuttcap%
\pgfsetroundjoin%
\definecolor{currentfill}{rgb}{0.980678,0.914765,0.856766}%
\pgfsetfillcolor{currentfill}%
\pgfsetlinewidth{0.311001pt}%
\definecolor{currentstroke}{rgb}{1.000000,1.000000,1.000000}%
\pgfsetstrokecolor{currentstroke}%
\pgfsetdash{}{0pt}%
\pgfpathmoveto{\pgfqpoint{4.206713in}{1.362794in}}%
\pgfpathcurveto{\pgfqpoint{4.213846in}{1.362794in}}{\pgfqpoint{4.220687in}{1.365628in}}{\pgfqpoint{4.225731in}{1.370672in}}%
\pgfpathcurveto{\pgfqpoint{4.230775in}{1.375715in}}{\pgfqpoint{4.233609in}{1.382557in}}{\pgfqpoint{4.233609in}{1.389690in}}%
\pgfpathcurveto{\pgfqpoint{4.233609in}{1.396823in}}{\pgfqpoint{4.230775in}{1.403664in}}{\pgfqpoint{4.225731in}{1.408708in}}%
\pgfpathcurveto{\pgfqpoint{4.220687in}{1.413752in}}{\pgfqpoint{4.213846in}{1.416585in}}{\pgfqpoint{4.206713in}{1.416585in}}%
\pgfpathcurveto{\pgfqpoint{4.199580in}{1.416585in}}{\pgfqpoint{4.192738in}{1.413752in}}{\pgfqpoint{4.187695in}{1.408708in}}%
\pgfpathcurveto{\pgfqpoint{4.182651in}{1.403664in}}{\pgfqpoint{4.179817in}{1.396823in}}{\pgfqpoint{4.179817in}{1.389690in}}%
\pgfpathcurveto{\pgfqpoint{4.179817in}{1.382557in}}{\pgfqpoint{4.182651in}{1.375715in}}{\pgfqpoint{4.187695in}{1.370672in}}%
\pgfpathcurveto{\pgfqpoint{4.192738in}{1.365628in}}{\pgfqpoint{4.199580in}{1.362794in}}{\pgfqpoint{4.206713in}{1.362794in}}%
\pgfpathclose%
\pgfusepath{stroke,fill}%
\end{pgfscope}%
\begin{pgfscope}%
\pgfpathrectangle{\pgfqpoint{2.867647in}{0.500000in}}{\pgfqpoint{1.764706in}{1.700000in}}%
\pgfusepath{clip}%
\pgfsetbuttcap%
\pgfsetroundjoin%
\definecolor{currentfill}{rgb}{0.970255,0.815666,0.711203}%
\pgfsetfillcolor{currentfill}%
\pgfsetlinewidth{0.311001pt}%
\definecolor{currentstroke}{rgb}{1.000000,1.000000,1.000000}%
\pgfsetstrokecolor{currentstroke}%
\pgfsetdash{}{0pt}%
\pgfpathmoveto{\pgfqpoint{4.268263in}{1.405984in}}%
\pgfpathcurveto{\pgfqpoint{4.275395in}{1.405984in}}{\pgfqpoint{4.282237in}{1.408818in}}{\pgfqpoint{4.287281in}{1.413862in}}%
\pgfpathcurveto{\pgfqpoint{4.292324in}{1.418905in}}{\pgfqpoint{4.295158in}{1.425747in}}{\pgfqpoint{4.295158in}{1.432880in}}%
\pgfpathcurveto{\pgfqpoint{4.295158in}{1.440013in}}{\pgfqpoint{4.292324in}{1.446854in}}{\pgfqpoint{4.287281in}{1.451898in}}%
\pgfpathcurveto{\pgfqpoint{4.282237in}{1.456941in}}{\pgfqpoint{4.275395in}{1.459775in}}{\pgfqpoint{4.268263in}{1.459775in}}%
\pgfpathcurveto{\pgfqpoint{4.261130in}{1.459775in}}{\pgfqpoint{4.254288in}{1.456941in}}{\pgfqpoint{4.249244in}{1.451898in}}%
\pgfpathcurveto{\pgfqpoint{4.244201in}{1.446854in}}{\pgfqpoint{4.241367in}{1.440013in}}{\pgfqpoint{4.241367in}{1.432880in}}%
\pgfpathcurveto{\pgfqpoint{4.241367in}{1.425747in}}{\pgfqpoint{4.244201in}{1.418905in}}{\pgfqpoint{4.249244in}{1.413862in}}%
\pgfpathcurveto{\pgfqpoint{4.254288in}{1.408818in}}{\pgfqpoint{4.261130in}{1.405984in}}{\pgfqpoint{4.268263in}{1.405984in}}%
\pgfpathclose%
\pgfusepath{stroke,fill}%
\end{pgfscope}%
\begin{pgfscope}%
\pgfpathrectangle{\pgfqpoint{2.867647in}{0.500000in}}{\pgfqpoint{1.764706in}{1.700000in}}%
\pgfusepath{clip}%
\pgfsetbuttcap%
\pgfsetroundjoin%
\definecolor{currentfill}{rgb}{0.975644,0.874038,0.797253}%
\pgfsetfillcolor{currentfill}%
\pgfsetlinewidth{0.311001pt}%
\definecolor{currentstroke}{rgb}{1.000000,1.000000,1.000000}%
\pgfsetstrokecolor{currentstroke}%
\pgfsetdash{}{0pt}%
\pgfpathmoveto{\pgfqpoint{4.093890in}{1.047083in}}%
\pgfpathcurveto{\pgfqpoint{4.101023in}{1.047083in}}{\pgfqpoint{4.107865in}{1.049917in}}{\pgfqpoint{4.112908in}{1.054960in}}%
\pgfpathcurveto{\pgfqpoint{4.117952in}{1.060004in}}{\pgfqpoint{4.120786in}{1.066846in}}{\pgfqpoint{4.120786in}{1.073979in}}%
\pgfpathcurveto{\pgfqpoint{4.120786in}{1.081111in}}{\pgfqpoint{4.117952in}{1.087953in}}{\pgfqpoint{4.112908in}{1.092997in}}%
\pgfpathcurveto{\pgfqpoint{4.107865in}{1.098040in}}{\pgfqpoint{4.101023in}{1.100874in}}{\pgfqpoint{4.093890in}{1.100874in}}%
\pgfpathcurveto{\pgfqpoint{4.086757in}{1.100874in}}{\pgfqpoint{4.079916in}{1.098040in}}{\pgfqpoint{4.074872in}{1.092997in}}%
\pgfpathcurveto{\pgfqpoint{4.069828in}{1.087953in}}{\pgfqpoint{4.066994in}{1.081111in}}{\pgfqpoint{4.066994in}{1.073979in}}%
\pgfpathcurveto{\pgfqpoint{4.066994in}{1.066846in}}{\pgfqpoint{4.069828in}{1.060004in}}{\pgfqpoint{4.074872in}{1.054960in}}%
\pgfpathcurveto{\pgfqpoint{4.079916in}{1.049917in}}{\pgfqpoint{4.086757in}{1.047083in}}{\pgfqpoint{4.093890in}{1.047083in}}%
\pgfpathclose%
\pgfusepath{stroke,fill}%
\end{pgfscope}%
\begin{pgfscope}%
\pgfpathrectangle{\pgfqpoint{2.867647in}{0.500000in}}{\pgfqpoint{1.764706in}{1.700000in}}%
\pgfusepath{clip}%
\pgfsetbuttcap%
\pgfsetroundjoin%
\definecolor{currentfill}{rgb}{0.972726,0.844889,0.754401}%
\pgfsetfillcolor{currentfill}%
\pgfsetlinewidth{0.311001pt}%
\definecolor{currentstroke}{rgb}{1.000000,1.000000,1.000000}%
\pgfsetstrokecolor{currentstroke}%
\pgfsetdash{}{0pt}%
\pgfpathmoveto{\pgfqpoint{4.115256in}{0.981960in}}%
\pgfpathcurveto{\pgfqpoint{4.122389in}{0.981960in}}{\pgfqpoint{4.129230in}{0.984794in}}{\pgfqpoint{4.134274in}{0.989838in}}%
\pgfpathcurveto{\pgfqpoint{4.139318in}{0.994882in}}{\pgfqpoint{4.142151in}{1.001723in}}{\pgfqpoint{4.142151in}{1.008856in}}%
\pgfpathcurveto{\pgfqpoint{4.142151in}{1.015989in}}{\pgfqpoint{4.139318in}{1.022831in}}{\pgfqpoint{4.134274in}{1.027874in}}%
\pgfpathcurveto{\pgfqpoint{4.129230in}{1.032918in}}{\pgfqpoint{4.122389in}{1.035752in}}{\pgfqpoint{4.115256in}{1.035752in}}%
\pgfpathcurveto{\pgfqpoint{4.108123in}{1.035752in}}{\pgfqpoint{4.101281in}{1.032918in}}{\pgfqpoint{4.096238in}{1.027874in}}%
\pgfpathcurveto{\pgfqpoint{4.091194in}{1.022831in}}{\pgfqpoint{4.088360in}{1.015989in}}{\pgfqpoint{4.088360in}{1.008856in}}%
\pgfpathcurveto{\pgfqpoint{4.088360in}{1.001723in}}{\pgfqpoint{4.091194in}{0.994882in}}{\pgfqpoint{4.096238in}{0.989838in}}%
\pgfpathcurveto{\pgfqpoint{4.101281in}{0.984794in}}{\pgfqpoint{4.108123in}{0.981960in}}{\pgfqpoint{4.115256in}{0.981960in}}%
\pgfpathclose%
\pgfusepath{stroke,fill}%
\end{pgfscope}%
\begin{pgfscope}%
\pgfpathrectangle{\pgfqpoint{2.867647in}{0.500000in}}{\pgfqpoint{1.764706in}{1.700000in}}%
\pgfusepath{clip}%
\pgfsetbuttcap%
\pgfsetroundjoin%
\definecolor{currentfill}{rgb}{0.967092,0.768560,0.642079}%
\pgfsetfillcolor{currentfill}%
\pgfsetlinewidth{0.311001pt}%
\definecolor{currentstroke}{rgb}{1.000000,1.000000,1.000000}%
\pgfsetstrokecolor{currentstroke}%
\pgfsetdash{}{0pt}%
\pgfpathmoveto{\pgfqpoint{4.095031in}{0.921710in}}%
\pgfpathcurveto{\pgfqpoint{4.102163in}{0.921710in}}{\pgfqpoint{4.109005in}{0.924543in}}{\pgfqpoint{4.114049in}{0.929587in}}%
\pgfpathcurveto{\pgfqpoint{4.119092in}{0.934631in}}{\pgfqpoint{4.121926in}{0.941472in}}{\pgfqpoint{4.121926in}{0.948605in}}%
\pgfpathcurveto{\pgfqpoint{4.121926in}{0.955738in}}{\pgfqpoint{4.119092in}{0.962580in}}{\pgfqpoint{4.114049in}{0.967623in}}%
\pgfpathcurveto{\pgfqpoint{4.109005in}{0.972667in}}{\pgfqpoint{4.102163in}{0.975501in}}{\pgfqpoint{4.095031in}{0.975501in}}%
\pgfpathcurveto{\pgfqpoint{4.087898in}{0.975501in}}{\pgfqpoint{4.081056in}{0.972667in}}{\pgfqpoint{4.076012in}{0.967623in}}%
\pgfpathcurveto{\pgfqpoint{4.070969in}{0.962580in}}{\pgfqpoint{4.068135in}{0.955738in}}{\pgfqpoint{4.068135in}{0.948605in}}%
\pgfpathcurveto{\pgfqpoint{4.068135in}{0.941472in}}{\pgfqpoint{4.070969in}{0.934631in}}{\pgfqpoint{4.076012in}{0.929587in}}%
\pgfpathcurveto{\pgfqpoint{4.081056in}{0.924543in}}{\pgfqpoint{4.087898in}{0.921710in}}{\pgfqpoint{4.095031in}{0.921710in}}%
\pgfpathclose%
\pgfusepath{stroke,fill}%
\end{pgfscope}%
\begin{pgfscope}%
\pgfpathrectangle{\pgfqpoint{2.867647in}{0.500000in}}{\pgfqpoint{1.764706in}{1.700000in}}%
\pgfusepath{clip}%
\pgfsetbuttcap%
\pgfsetroundjoin%
\definecolor{currentfill}{rgb}{0.979124,0.903132,0.839793}%
\pgfsetfillcolor{currentfill}%
\pgfsetlinewidth{0.311001pt}%
\definecolor{currentstroke}{rgb}{1.000000,1.000000,1.000000}%
\pgfsetstrokecolor{currentstroke}%
\pgfsetdash{}{0pt}%
\pgfpathmoveto{\pgfqpoint{4.222953in}{1.251231in}}%
\pgfpathcurveto{\pgfqpoint{4.230086in}{1.251231in}}{\pgfqpoint{4.236927in}{1.254065in}}{\pgfqpoint{4.241971in}{1.259109in}}%
\pgfpathcurveto{\pgfqpoint{4.247015in}{1.264152in}}{\pgfqpoint{4.249849in}{1.270994in}}{\pgfqpoint{4.249849in}{1.278127in}}%
\pgfpathcurveto{\pgfqpoint{4.249849in}{1.285260in}}{\pgfqpoint{4.247015in}{1.292101in}}{\pgfqpoint{4.241971in}{1.297145in}}%
\pgfpathcurveto{\pgfqpoint{4.236927in}{1.302188in}}{\pgfqpoint{4.230086in}{1.305022in}}{\pgfqpoint{4.222953in}{1.305022in}}%
\pgfpathcurveto{\pgfqpoint{4.215820in}{1.305022in}}{\pgfqpoint{4.208979in}{1.302188in}}{\pgfqpoint{4.203935in}{1.297145in}}%
\pgfpathcurveto{\pgfqpoint{4.198891in}{1.292101in}}{\pgfqpoint{4.196057in}{1.285260in}}{\pgfqpoint{4.196057in}{1.278127in}}%
\pgfpathcurveto{\pgfqpoint{4.196057in}{1.270994in}}{\pgfqpoint{4.198891in}{1.264152in}}{\pgfqpoint{4.203935in}{1.259109in}}%
\pgfpathcurveto{\pgfqpoint{4.208979in}{1.254065in}}{\pgfqpoint{4.215820in}{1.251231in}}{\pgfqpoint{4.222953in}{1.251231in}}%
\pgfpathclose%
\pgfusepath{stroke,fill}%
\end{pgfscope}%
\begin{pgfscope}%
\pgfpathrectangle{\pgfqpoint{2.867647in}{0.500000in}}{\pgfqpoint{1.764706in}{1.700000in}}%
\pgfusepath{clip}%
\pgfsetbuttcap%
\pgfsetroundjoin%
\definecolor{currentfill}{rgb}{0.966812,0.762584,0.633643}%
\pgfsetfillcolor{currentfill}%
\pgfsetlinewidth{0.311001pt}%
\definecolor{currentstroke}{rgb}{1.000000,1.000000,1.000000}%
\pgfsetstrokecolor{currentstroke}%
\pgfsetdash{}{0pt}%
\pgfpathmoveto{\pgfqpoint{4.053158in}{1.730320in}}%
\pgfpathcurveto{\pgfqpoint{4.060291in}{1.730320in}}{\pgfqpoint{4.067132in}{1.733154in}}{\pgfqpoint{4.072176in}{1.738198in}}%
\pgfpathcurveto{\pgfqpoint{4.077220in}{1.743241in}}{\pgfqpoint{4.080053in}{1.750083in}}{\pgfqpoint{4.080053in}{1.757216in}}%
\pgfpathcurveto{\pgfqpoint{4.080053in}{1.764349in}}{\pgfqpoint{4.077220in}{1.771190in}}{\pgfqpoint{4.072176in}{1.776234in}}%
\pgfpathcurveto{\pgfqpoint{4.067132in}{1.781278in}}{\pgfqpoint{4.060291in}{1.784112in}}{\pgfqpoint{4.053158in}{1.784112in}}%
\pgfpathcurveto{\pgfqpoint{4.046025in}{1.784112in}}{\pgfqpoint{4.039183in}{1.781278in}}{\pgfqpoint{4.034140in}{1.776234in}}%
\pgfpathcurveto{\pgfqpoint{4.029096in}{1.771190in}}{\pgfqpoint{4.026262in}{1.764349in}}{\pgfqpoint{4.026262in}{1.757216in}}%
\pgfpathcurveto{\pgfqpoint{4.026262in}{1.750083in}}{\pgfqpoint{4.029096in}{1.743241in}}{\pgfqpoint{4.034140in}{1.738198in}}%
\pgfpathcurveto{\pgfqpoint{4.039183in}{1.733154in}}{\pgfqpoint{4.046025in}{1.730320in}}{\pgfqpoint{4.053158in}{1.730320in}}%
\pgfpathclose%
\pgfusepath{stroke,fill}%
\end{pgfscope}%
\begin{pgfscope}%
\pgfpathrectangle{\pgfqpoint{2.867647in}{0.500000in}}{\pgfqpoint{1.764706in}{1.700000in}}%
\pgfusepath{clip}%
\pgfsetbuttcap%
\pgfsetroundjoin%
\definecolor{currentfill}{rgb}{0.974412,0.862387,0.780156}%
\pgfsetfillcolor{currentfill}%
\pgfsetlinewidth{0.311001pt}%
\definecolor{currentstroke}{rgb}{1.000000,1.000000,1.000000}%
\pgfsetstrokecolor{currentstroke}%
\pgfsetdash{}{0pt}%
\pgfpathmoveto{\pgfqpoint{4.099930in}{1.493192in}}%
\pgfpathcurveto{\pgfqpoint{4.107063in}{1.493192in}}{\pgfqpoint{4.113904in}{1.496026in}}{\pgfqpoint{4.118948in}{1.501069in}}%
\pgfpathcurveto{\pgfqpoint{4.123992in}{1.506113in}}{\pgfqpoint{4.126825in}{1.512955in}}{\pgfqpoint{4.126825in}{1.520087in}}%
\pgfpathcurveto{\pgfqpoint{4.126825in}{1.527220in}}{\pgfqpoint{4.123992in}{1.534062in}}{\pgfqpoint{4.118948in}{1.539106in}}%
\pgfpathcurveto{\pgfqpoint{4.113904in}{1.544149in}}{\pgfqpoint{4.107063in}{1.546983in}}{\pgfqpoint{4.099930in}{1.546983in}}%
\pgfpathcurveto{\pgfqpoint{4.092797in}{1.546983in}}{\pgfqpoint{4.085955in}{1.544149in}}{\pgfqpoint{4.080912in}{1.539106in}}%
\pgfpathcurveto{\pgfqpoint{4.075868in}{1.534062in}}{\pgfqpoint{4.073034in}{1.527220in}}{\pgfqpoint{4.073034in}{1.520087in}}%
\pgfpathcurveto{\pgfqpoint{4.073034in}{1.512955in}}{\pgfqpoint{4.075868in}{1.506113in}}{\pgfqpoint{4.080912in}{1.501069in}}%
\pgfpathcurveto{\pgfqpoint{4.085955in}{1.496026in}}{\pgfqpoint{4.092797in}{1.493192in}}{\pgfqpoint{4.099930in}{1.493192in}}%
\pgfpathclose%
\pgfusepath{stroke,fill}%
\end{pgfscope}%
\begin{pgfscope}%
\pgfpathrectangle{\pgfqpoint{2.867647in}{0.500000in}}{\pgfqpoint{1.764706in}{1.700000in}}%
\pgfusepath{clip}%
\pgfsetbuttcap%
\pgfsetroundjoin%
\definecolor{currentfill}{rgb}{0.967735,0.780441,0.659127}%
\pgfsetfillcolor{currentfill}%
\pgfsetlinewidth{0.311001pt}%
\definecolor{currentstroke}{rgb}{1.000000,1.000000,1.000000}%
\pgfsetstrokecolor{currentstroke}%
\pgfsetdash{}{0pt}%
\pgfpathmoveto{\pgfqpoint{4.079964in}{1.199839in}}%
\pgfpathcurveto{\pgfqpoint{4.087097in}{1.199839in}}{\pgfqpoint{4.093938in}{1.202673in}}{\pgfqpoint{4.098982in}{1.207717in}}%
\pgfpathcurveto{\pgfqpoint{4.104026in}{1.212761in}}{\pgfqpoint{4.106860in}{1.219602in}}{\pgfqpoint{4.106860in}{1.226735in}}%
\pgfpathcurveto{\pgfqpoint{4.106860in}{1.233868in}}{\pgfqpoint{4.104026in}{1.240710in}}{\pgfqpoint{4.098982in}{1.245753in}}%
\pgfpathcurveto{\pgfqpoint{4.093938in}{1.250797in}}{\pgfqpoint{4.087097in}{1.253631in}}{\pgfqpoint{4.079964in}{1.253631in}}%
\pgfpathcurveto{\pgfqpoint{4.072831in}{1.253631in}}{\pgfqpoint{4.065990in}{1.250797in}}{\pgfqpoint{4.060946in}{1.245753in}}%
\pgfpathcurveto{\pgfqpoint{4.055902in}{1.240710in}}{\pgfqpoint{4.053068in}{1.233868in}}{\pgfqpoint{4.053068in}{1.226735in}}%
\pgfpathcurveto{\pgfqpoint{4.053068in}{1.219602in}}{\pgfqpoint{4.055902in}{1.212761in}}{\pgfqpoint{4.060946in}{1.207717in}}%
\pgfpathcurveto{\pgfqpoint{4.065990in}{1.202673in}}{\pgfqpoint{4.072831in}{1.199839in}}{\pgfqpoint{4.079964in}{1.199839in}}%
\pgfpathclose%
\pgfusepath{stroke,fill}%
\end{pgfscope}%
\begin{pgfscope}%
\pgfpathrectangle{\pgfqpoint{2.867647in}{0.500000in}}{\pgfqpoint{1.764706in}{1.700000in}}%
\pgfusepath{clip}%
\pgfsetbuttcap%
\pgfsetroundjoin%
\definecolor{currentfill}{rgb}{0.972201,0.839051,0.745789}%
\pgfsetfillcolor{currentfill}%
\pgfsetlinewidth{0.311001pt}%
\definecolor{currentstroke}{rgb}{1.000000,1.000000,1.000000}%
\pgfsetstrokecolor{currentstroke}%
\pgfsetdash{}{0pt}%
\pgfpathmoveto{\pgfqpoint{4.181134in}{1.024014in}}%
\pgfpathcurveto{\pgfqpoint{4.188267in}{1.024014in}}{\pgfqpoint{4.195109in}{1.026848in}}{\pgfqpoint{4.200152in}{1.031892in}}%
\pgfpathcurveto{\pgfqpoint{4.205196in}{1.036936in}}{\pgfqpoint{4.208030in}{1.043777in}}{\pgfqpoint{4.208030in}{1.050910in}}%
\pgfpathcurveto{\pgfqpoint{4.208030in}{1.058043in}}{\pgfqpoint{4.205196in}{1.064885in}}{\pgfqpoint{4.200152in}{1.069928in}}%
\pgfpathcurveto{\pgfqpoint{4.195109in}{1.074972in}}{\pgfqpoint{4.188267in}{1.077806in}}{\pgfqpoint{4.181134in}{1.077806in}}%
\pgfpathcurveto{\pgfqpoint{4.174001in}{1.077806in}}{\pgfqpoint{4.167160in}{1.074972in}}{\pgfqpoint{4.162116in}{1.069928in}}%
\pgfpathcurveto{\pgfqpoint{4.157072in}{1.064885in}}{\pgfqpoint{4.154238in}{1.058043in}}{\pgfqpoint{4.154238in}{1.050910in}}%
\pgfpathcurveto{\pgfqpoint{4.154238in}{1.043777in}}{\pgfqpoint{4.157072in}{1.036936in}}{\pgfqpoint{4.162116in}{1.031892in}}%
\pgfpathcurveto{\pgfqpoint{4.167160in}{1.026848in}}{\pgfqpoint{4.174001in}{1.024014in}}{\pgfqpoint{4.181134in}{1.024014in}}%
\pgfpathclose%
\pgfusepath{stroke,fill}%
\end{pgfscope}%
\begin{pgfscope}%
\pgfpathrectangle{\pgfqpoint{2.867647in}{0.500000in}}{\pgfqpoint{1.764706in}{1.700000in}}%
\pgfusepath{clip}%
\pgfsetbuttcap%
\pgfsetroundjoin%
\definecolor{currentfill}{rgb}{0.952404,0.449449,0.307210}%
\pgfsetfillcolor{currentfill}%
\pgfsetlinewidth{0.311001pt}%
\definecolor{currentstroke}{rgb}{1.000000,1.000000,1.000000}%
\pgfsetstrokecolor{currentstroke}%
\pgfsetdash{}{0pt}%
\pgfpathmoveto{\pgfqpoint{3.962845in}{1.547612in}}%
\pgfpathcurveto{\pgfqpoint{3.969978in}{1.547612in}}{\pgfqpoint{3.976820in}{1.550446in}}{\pgfqpoint{3.981863in}{1.555490in}}%
\pgfpathcurveto{\pgfqpoint{3.986907in}{1.560533in}}{\pgfqpoint{3.989741in}{1.567375in}}{\pgfqpoint{3.989741in}{1.574508in}}%
\pgfpathcurveto{\pgfqpoint{3.989741in}{1.581641in}}{\pgfqpoint{3.986907in}{1.588482in}}{\pgfqpoint{3.981863in}{1.593526in}}%
\pgfpathcurveto{\pgfqpoint{3.976820in}{1.598570in}}{\pgfqpoint{3.969978in}{1.601403in}}{\pgfqpoint{3.962845in}{1.601403in}}%
\pgfpathcurveto{\pgfqpoint{3.955712in}{1.601403in}}{\pgfqpoint{3.948871in}{1.598570in}}{\pgfqpoint{3.943827in}{1.593526in}}%
\pgfpathcurveto{\pgfqpoint{3.938783in}{1.588482in}}{\pgfqpoint{3.935949in}{1.581641in}}{\pgfqpoint{3.935949in}{1.574508in}}%
\pgfpathcurveto{\pgfqpoint{3.935949in}{1.567375in}}{\pgfqpoint{3.938783in}{1.560533in}}{\pgfqpoint{3.943827in}{1.555490in}}%
\pgfpathcurveto{\pgfqpoint{3.948871in}{1.550446in}}{\pgfqpoint{3.955712in}{1.547612in}}{\pgfqpoint{3.962845in}{1.547612in}}%
\pgfpathclose%
\pgfusepath{stroke,fill}%
\end{pgfscope}%
\begin{pgfscope}%
\pgfpathrectangle{\pgfqpoint{2.867647in}{0.500000in}}{\pgfqpoint{1.764706in}{1.700000in}}%
\pgfusepath{clip}%
\pgfsetbuttcap%
\pgfsetroundjoin%
\definecolor{currentfill}{rgb}{0.980678,0.914765,0.856766}%
\pgfsetfillcolor{currentfill}%
\pgfsetlinewidth{0.311001pt}%
\definecolor{currentstroke}{rgb}{1.000000,1.000000,1.000000}%
\pgfsetstrokecolor{currentstroke}%
\pgfsetdash{}{0pt}%
\pgfpathmoveto{\pgfqpoint{4.159748in}{1.476632in}}%
\pgfpathcurveto{\pgfqpoint{4.166881in}{1.476632in}}{\pgfqpoint{4.173723in}{1.479466in}}{\pgfqpoint{4.178767in}{1.484510in}}%
\pgfpathcurveto{\pgfqpoint{4.183810in}{1.489553in}}{\pgfqpoint{4.186644in}{1.496395in}}{\pgfqpoint{4.186644in}{1.503528in}}%
\pgfpathcurveto{\pgfqpoint{4.186644in}{1.510660in}}{\pgfqpoint{4.183810in}{1.517502in}}{\pgfqpoint{4.178767in}{1.522546in}}%
\pgfpathcurveto{\pgfqpoint{4.173723in}{1.527589in}}{\pgfqpoint{4.166881in}{1.530423in}}{\pgfqpoint{4.159748in}{1.530423in}}%
\pgfpathcurveto{\pgfqpoint{4.152616in}{1.530423in}}{\pgfqpoint{4.145774in}{1.527589in}}{\pgfqpoint{4.140730in}{1.522546in}}%
\pgfpathcurveto{\pgfqpoint{4.135687in}{1.517502in}}{\pgfqpoint{4.132853in}{1.510660in}}{\pgfqpoint{4.132853in}{1.503528in}}%
\pgfpathcurveto{\pgfqpoint{4.132853in}{1.496395in}}{\pgfqpoint{4.135687in}{1.489553in}}{\pgfqpoint{4.140730in}{1.484510in}}%
\pgfpathcurveto{\pgfqpoint{4.145774in}{1.479466in}}{\pgfqpoint{4.152616in}{1.476632in}}{\pgfqpoint{4.159748in}{1.476632in}}%
\pgfpathclose%
\pgfusepath{stroke,fill}%
\end{pgfscope}%
\begin{pgfscope}%
\pgfpathrectangle{\pgfqpoint{2.867647in}{0.500000in}}{\pgfqpoint{1.764706in}{1.700000in}}%
\pgfusepath{clip}%
\pgfsetbuttcap%
\pgfsetroundjoin%
\definecolor{currentfill}{rgb}{0.973271,0.850724,0.762998}%
\pgfsetfillcolor{currentfill}%
\pgfsetlinewidth{0.311001pt}%
\definecolor{currentstroke}{rgb}{1.000000,1.000000,1.000000}%
\pgfsetstrokecolor{currentstroke}%
\pgfsetdash{}{0pt}%
\pgfpathmoveto{\pgfqpoint{4.076998in}{1.001594in}}%
\pgfpathcurveto{\pgfqpoint{4.084131in}{1.001594in}}{\pgfqpoint{4.090972in}{1.004428in}}{\pgfqpoint{4.096016in}{1.009472in}}%
\pgfpathcurveto{\pgfqpoint{4.101060in}{1.014516in}}{\pgfqpoint{4.103893in}{1.021357in}}{\pgfqpoint{4.103893in}{1.028490in}}%
\pgfpathcurveto{\pgfqpoint{4.103893in}{1.035623in}}{\pgfqpoint{4.101060in}{1.042465in}}{\pgfqpoint{4.096016in}{1.047508in}}%
\pgfpathcurveto{\pgfqpoint{4.090972in}{1.052552in}}{\pgfqpoint{4.084131in}{1.055386in}}{\pgfqpoint{4.076998in}{1.055386in}}%
\pgfpathcurveto{\pgfqpoint{4.069865in}{1.055386in}}{\pgfqpoint{4.063023in}{1.052552in}}{\pgfqpoint{4.057980in}{1.047508in}}%
\pgfpathcurveto{\pgfqpoint{4.052936in}{1.042465in}}{\pgfqpoint{4.050102in}{1.035623in}}{\pgfqpoint{4.050102in}{1.028490in}}%
\pgfpathcurveto{\pgfqpoint{4.050102in}{1.021357in}}{\pgfqpoint{4.052936in}{1.014516in}}{\pgfqpoint{4.057980in}{1.009472in}}%
\pgfpathcurveto{\pgfqpoint{4.063023in}{1.004428in}}{\pgfqpoint{4.069865in}{1.001594in}}{\pgfqpoint{4.076998in}{1.001594in}}%
\pgfpathclose%
\pgfusepath{stroke,fill}%
\end{pgfscope}%
\begin{pgfscope}%
\pgfpathrectangle{\pgfqpoint{2.867647in}{0.500000in}}{\pgfqpoint{1.764706in}{1.700000in}}%
\pgfusepath{clip}%
\pgfsetbuttcap%
\pgfsetroundjoin%
\definecolor{currentfill}{rgb}{0.967092,0.768560,0.642079}%
\pgfsetfillcolor{currentfill}%
\pgfsetlinewidth{0.311001pt}%
\definecolor{currentstroke}{rgb}{1.000000,1.000000,1.000000}%
\pgfsetstrokecolor{currentstroke}%
\pgfsetdash{}{0pt}%
\pgfpathmoveto{\pgfqpoint{4.086946in}{1.246769in}}%
\pgfpathcurveto{\pgfqpoint{4.094079in}{1.246769in}}{\pgfqpoint{4.100920in}{1.249602in}}{\pgfqpoint{4.105964in}{1.254646in}}%
\pgfpathcurveto{\pgfqpoint{4.111008in}{1.259690in}}{\pgfqpoint{4.113842in}{1.266531in}}{\pgfqpoint{4.113842in}{1.273664in}}%
\pgfpathcurveto{\pgfqpoint{4.113842in}{1.280797in}}{\pgfqpoint{4.111008in}{1.287639in}}{\pgfqpoint{4.105964in}{1.292682in}}%
\pgfpathcurveto{\pgfqpoint{4.100920in}{1.297726in}}{\pgfqpoint{4.094079in}{1.300560in}}{\pgfqpoint{4.086946in}{1.300560in}}%
\pgfpathcurveto{\pgfqpoint{4.079813in}{1.300560in}}{\pgfqpoint{4.072971in}{1.297726in}}{\pgfqpoint{4.067928in}{1.292682in}}%
\pgfpathcurveto{\pgfqpoint{4.062884in}{1.287639in}}{\pgfqpoint{4.060050in}{1.280797in}}{\pgfqpoint{4.060050in}{1.273664in}}%
\pgfpathcurveto{\pgfqpoint{4.060050in}{1.266531in}}{\pgfqpoint{4.062884in}{1.259690in}}{\pgfqpoint{4.067928in}{1.254646in}}%
\pgfpathcurveto{\pgfqpoint{4.072971in}{1.249602in}}{\pgfqpoint{4.079813in}{1.246769in}}{\pgfqpoint{4.086946in}{1.246769in}}%
\pgfpathclose%
\pgfusepath{stroke,fill}%
\end{pgfscope}%
\begin{pgfscope}%
\pgfpathrectangle{\pgfqpoint{2.867647in}{0.500000in}}{\pgfqpoint{1.764706in}{1.700000in}}%
\pgfusepath{clip}%
\pgfsetbuttcap%
\pgfsetroundjoin%
\definecolor{currentfill}{rgb}{0.977657,0.891500,0.822809}%
\pgfsetfillcolor{currentfill}%
\pgfsetlinewidth{0.311001pt}%
\definecolor{currentstroke}{rgb}{1.000000,1.000000,1.000000}%
\pgfsetstrokecolor{currentstroke}%
\pgfsetdash{}{0pt}%
\pgfpathmoveto{\pgfqpoint{4.104668in}{1.564919in}}%
\pgfpathcurveto{\pgfqpoint{4.111801in}{1.564919in}}{\pgfqpoint{4.118643in}{1.567753in}}{\pgfqpoint{4.123686in}{1.572797in}}%
\pgfpathcurveto{\pgfqpoint{4.128730in}{1.577841in}}{\pgfqpoint{4.131564in}{1.584682in}}{\pgfqpoint{4.131564in}{1.591815in}}%
\pgfpathcurveto{\pgfqpoint{4.131564in}{1.598948in}}{\pgfqpoint{4.128730in}{1.605789in}}{\pgfqpoint{4.123686in}{1.610833in}}%
\pgfpathcurveto{\pgfqpoint{4.118643in}{1.615877in}}{\pgfqpoint{4.111801in}{1.618711in}}{\pgfqpoint{4.104668in}{1.618711in}}%
\pgfpathcurveto{\pgfqpoint{4.097535in}{1.618711in}}{\pgfqpoint{4.090694in}{1.615877in}}{\pgfqpoint{4.085650in}{1.610833in}}%
\pgfpathcurveto{\pgfqpoint{4.080606in}{1.605789in}}{\pgfqpoint{4.077772in}{1.598948in}}{\pgfqpoint{4.077772in}{1.591815in}}%
\pgfpathcurveto{\pgfqpoint{4.077772in}{1.584682in}}{\pgfqpoint{4.080606in}{1.577841in}}{\pgfqpoint{4.085650in}{1.572797in}}%
\pgfpathcurveto{\pgfqpoint{4.090694in}{1.567753in}}{\pgfqpoint{4.097535in}{1.564919in}}{\pgfqpoint{4.104668in}{1.564919in}}%
\pgfpathclose%
\pgfusepath{stroke,fill}%
\end{pgfscope}%
\begin{pgfscope}%
\pgfpathrectangle{\pgfqpoint{2.867647in}{0.500000in}}{\pgfqpoint{1.764706in}{1.700000in}}%
\pgfusepath{clip}%
\pgfsetbuttcap%
\pgfsetroundjoin%
\definecolor{currentfill}{rgb}{0.870791,0.179821,0.267974}%
\pgfsetfillcolor{currentfill}%
\pgfsetlinewidth{0.311001pt}%
\definecolor{currentstroke}{rgb}{1.000000,1.000000,1.000000}%
\pgfsetstrokecolor{currentstroke}%
\pgfsetdash{}{0pt}%
\pgfpathmoveto{\pgfqpoint{3.972228in}{1.393495in}}%
\pgfpathcurveto{\pgfqpoint{3.979360in}{1.393495in}}{\pgfqpoint{3.986202in}{1.396329in}}{\pgfqpoint{3.991246in}{1.401373in}}%
\pgfpathcurveto{\pgfqpoint{3.996289in}{1.406417in}}{\pgfqpoint{3.999123in}{1.413258in}}{\pgfqpoint{3.999123in}{1.420391in}}%
\pgfpathcurveto{\pgfqpoint{3.999123in}{1.427524in}}{\pgfqpoint{3.996289in}{1.434366in}}{\pgfqpoint{3.991246in}{1.439409in}}%
\pgfpathcurveto{\pgfqpoint{3.986202in}{1.444453in}}{\pgfqpoint{3.979360in}{1.447287in}}{\pgfqpoint{3.972228in}{1.447287in}}%
\pgfpathcurveto{\pgfqpoint{3.965095in}{1.447287in}}{\pgfqpoint{3.958253in}{1.444453in}}{\pgfqpoint{3.953209in}{1.439409in}}%
\pgfpathcurveto{\pgfqpoint{3.948166in}{1.434366in}}{\pgfqpoint{3.945332in}{1.427524in}}{\pgfqpoint{3.945332in}{1.420391in}}%
\pgfpathcurveto{\pgfqpoint{3.945332in}{1.413258in}}{\pgfqpoint{3.948166in}{1.406417in}}{\pgfqpoint{3.953209in}{1.401373in}}%
\pgfpathcurveto{\pgfqpoint{3.958253in}{1.396329in}}{\pgfqpoint{3.965095in}{1.393495in}}{\pgfqpoint{3.972228in}{1.393495in}}%
\pgfpathclose%
\pgfusepath{stroke,fill}%
\end{pgfscope}%
\begin{pgfscope}%
\pgfpathrectangle{\pgfqpoint{2.867647in}{0.500000in}}{\pgfqpoint{1.764706in}{1.700000in}}%
\pgfusepath{clip}%
\pgfsetbuttcap%
\pgfsetroundjoin%
\definecolor{currentfill}{rgb}{0.972726,0.844889,0.754401}%
\pgfsetfillcolor{currentfill}%
\pgfsetlinewidth{0.311001pt}%
\definecolor{currentstroke}{rgb}{1.000000,1.000000,1.000000}%
\pgfsetstrokecolor{currentstroke}%
\pgfsetdash{}{0pt}%
\pgfpathmoveto{\pgfqpoint{4.117156in}{1.376457in}}%
\pgfpathcurveto{\pgfqpoint{4.124289in}{1.376457in}}{\pgfqpoint{4.131131in}{1.379290in}}{\pgfqpoint{4.136175in}{1.384334in}}%
\pgfpathcurveto{\pgfqpoint{4.141218in}{1.389378in}}{\pgfqpoint{4.144052in}{1.396219in}}{\pgfqpoint{4.144052in}{1.403352in}}%
\pgfpathcurveto{\pgfqpoint{4.144052in}{1.410485in}}{\pgfqpoint{4.141218in}{1.417327in}}{\pgfqpoint{4.136175in}{1.422370in}}%
\pgfpathcurveto{\pgfqpoint{4.131131in}{1.427414in}}{\pgfqpoint{4.124289in}{1.430248in}}{\pgfqpoint{4.117156in}{1.430248in}}%
\pgfpathcurveto{\pgfqpoint{4.110024in}{1.430248in}}{\pgfqpoint{4.103182in}{1.427414in}}{\pgfqpoint{4.098138in}{1.422370in}}%
\pgfpathcurveto{\pgfqpoint{4.093095in}{1.417327in}}{\pgfqpoint{4.090261in}{1.410485in}}{\pgfqpoint{4.090261in}{1.403352in}}%
\pgfpathcurveto{\pgfqpoint{4.090261in}{1.396219in}}{\pgfqpoint{4.093095in}{1.389378in}}{\pgfqpoint{4.098138in}{1.384334in}}%
\pgfpathcurveto{\pgfqpoint{4.103182in}{1.379290in}}{\pgfqpoint{4.110024in}{1.376457in}}{\pgfqpoint{4.117156in}{1.376457in}}%
\pgfpathclose%
\pgfusepath{stroke,fill}%
\end{pgfscope}%
\begin{pgfscope}%
\pgfpathrectangle{\pgfqpoint{2.867647in}{0.500000in}}{\pgfqpoint{1.764706in}{1.700000in}}%
\pgfusepath{clip}%
\pgfsetbuttcap%
\pgfsetroundjoin%
\definecolor{currentfill}{rgb}{0.975644,0.874038,0.797253}%
\pgfsetfillcolor{currentfill}%
\pgfsetlinewidth{0.311001pt}%
\definecolor{currentstroke}{rgb}{1.000000,1.000000,1.000000}%
\pgfsetstrokecolor{currentstroke}%
\pgfsetdash{}{0pt}%
\pgfpathmoveto{\pgfqpoint{4.093102in}{1.067992in}}%
\pgfpathcurveto{\pgfqpoint{4.100235in}{1.067992in}}{\pgfqpoint{4.107076in}{1.070826in}}{\pgfqpoint{4.112120in}{1.075870in}}%
\pgfpathcurveto{\pgfqpoint{4.117164in}{1.080914in}}{\pgfqpoint{4.119998in}{1.087755in}}{\pgfqpoint{4.119998in}{1.094888in}}%
\pgfpathcurveto{\pgfqpoint{4.119998in}{1.102021in}}{\pgfqpoint{4.117164in}{1.108863in}}{\pgfqpoint{4.112120in}{1.113906in}}%
\pgfpathcurveto{\pgfqpoint{4.107076in}{1.118950in}}{\pgfqpoint{4.100235in}{1.121784in}}{\pgfqpoint{4.093102in}{1.121784in}}%
\pgfpathcurveto{\pgfqpoint{4.085969in}{1.121784in}}{\pgfqpoint{4.079127in}{1.118950in}}{\pgfqpoint{4.074084in}{1.113906in}}%
\pgfpathcurveto{\pgfqpoint{4.069040in}{1.108863in}}{\pgfqpoint{4.066206in}{1.102021in}}{\pgfqpoint{4.066206in}{1.094888in}}%
\pgfpathcurveto{\pgfqpoint{4.066206in}{1.087755in}}{\pgfqpoint{4.069040in}{1.080914in}}{\pgfqpoint{4.074084in}{1.075870in}}%
\pgfpathcurveto{\pgfqpoint{4.079127in}{1.070826in}}{\pgfqpoint{4.085969in}{1.067992in}}{\pgfqpoint{4.093102in}{1.067992in}}%
\pgfpathclose%
\pgfusepath{stroke,fill}%
\end{pgfscope}%
\begin{pgfscope}%
\pgfpathrectangle{\pgfqpoint{2.867647in}{0.500000in}}{\pgfqpoint{1.764706in}{1.700000in}}%
\pgfusepath{clip}%
\pgfsetbuttcap%
\pgfsetroundjoin%
\definecolor{currentfill}{rgb}{0.966560,0.756582,0.625273}%
\pgfsetfillcolor{currentfill}%
\pgfsetlinewidth{0.311001pt}%
\definecolor{currentstroke}{rgb}{1.000000,1.000000,1.000000}%
\pgfsetstrokecolor{currentstroke}%
\pgfsetdash{}{0pt}%
\pgfpathmoveto{\pgfqpoint{4.013906in}{0.972035in}}%
\pgfpathcurveto{\pgfqpoint{4.021039in}{0.972035in}}{\pgfqpoint{4.027881in}{0.974869in}}{\pgfqpoint{4.032924in}{0.979913in}}%
\pgfpathcurveto{\pgfqpoint{4.037968in}{0.984956in}}{\pgfqpoint{4.040802in}{0.991798in}}{\pgfqpoint{4.040802in}{0.998931in}}%
\pgfpathcurveto{\pgfqpoint{4.040802in}{1.006064in}}{\pgfqpoint{4.037968in}{1.012905in}}{\pgfqpoint{4.032924in}{1.017949in}}%
\pgfpathcurveto{\pgfqpoint{4.027881in}{1.022993in}}{\pgfqpoint{4.021039in}{1.025826in}}{\pgfqpoint{4.013906in}{1.025826in}}%
\pgfpathcurveto{\pgfqpoint{4.006773in}{1.025826in}}{\pgfqpoint{3.999932in}{1.022993in}}{\pgfqpoint{3.994888in}{1.017949in}}%
\pgfpathcurveto{\pgfqpoint{3.989844in}{1.012905in}}{\pgfqpoint{3.987011in}{1.006064in}}{\pgfqpoint{3.987011in}{0.998931in}}%
\pgfpathcurveto{\pgfqpoint{3.987011in}{0.991798in}}{\pgfqpoint{3.989844in}{0.984956in}}{\pgfqpoint{3.994888in}{0.979913in}}%
\pgfpathcurveto{\pgfqpoint{3.999932in}{0.974869in}}{\pgfqpoint{4.006773in}{0.972035in}}{\pgfqpoint{4.013906in}{0.972035in}}%
\pgfpathclose%
\pgfusepath{stroke,fill}%
\end{pgfscope}%
\begin{pgfscope}%
\pgfpathrectangle{\pgfqpoint{2.867647in}{0.500000in}}{\pgfqpoint{1.764706in}{1.700000in}}%
\pgfusepath{clip}%
\pgfsetbuttcap%
\pgfsetroundjoin%
\definecolor{currentfill}{rgb}{0.972726,0.844889,0.754401}%
\pgfsetfillcolor{currentfill}%
\pgfsetlinewidth{0.311001pt}%
\definecolor{currentstroke}{rgb}{1.000000,1.000000,1.000000}%
\pgfsetstrokecolor{currentstroke}%
\pgfsetdash{}{0pt}%
\pgfpathmoveto{\pgfqpoint{4.261216in}{1.277942in}}%
\pgfpathcurveto{\pgfqpoint{4.268349in}{1.277942in}}{\pgfqpoint{4.275191in}{1.280776in}}{\pgfqpoint{4.280234in}{1.285820in}}%
\pgfpathcurveto{\pgfqpoint{4.285278in}{1.290863in}}{\pgfqpoint{4.288112in}{1.297705in}}{\pgfqpoint{4.288112in}{1.304838in}}%
\pgfpathcurveto{\pgfqpoint{4.288112in}{1.311971in}}{\pgfqpoint{4.285278in}{1.318812in}}{\pgfqpoint{4.280234in}{1.323856in}}%
\pgfpathcurveto{\pgfqpoint{4.275191in}{1.328900in}}{\pgfqpoint{4.268349in}{1.331734in}}{\pgfqpoint{4.261216in}{1.331734in}}%
\pgfpathcurveto{\pgfqpoint{4.254083in}{1.331734in}}{\pgfqpoint{4.247242in}{1.328900in}}{\pgfqpoint{4.242198in}{1.323856in}}%
\pgfpathcurveto{\pgfqpoint{4.237154in}{1.318812in}}{\pgfqpoint{4.234320in}{1.311971in}}{\pgfqpoint{4.234320in}{1.304838in}}%
\pgfpathcurveto{\pgfqpoint{4.234320in}{1.297705in}}{\pgfqpoint{4.237154in}{1.290863in}}{\pgfqpoint{4.242198in}{1.285820in}}%
\pgfpathcurveto{\pgfqpoint{4.247242in}{1.280776in}}{\pgfqpoint{4.254083in}{1.277942in}}{\pgfqpoint{4.261216in}{1.277942in}}%
\pgfpathclose%
\pgfusepath{stroke,fill}%
\end{pgfscope}%
\begin{pgfscope}%
\pgfpathrectangle{\pgfqpoint{2.867647in}{0.500000in}}{\pgfqpoint{1.764706in}{1.700000in}}%
\pgfusepath{clip}%
\pgfsetbuttcap%
\pgfsetroundjoin%
\definecolor{currentfill}{rgb}{0.966328,0.750560,0.616961}%
\pgfsetfillcolor{currentfill}%
\pgfsetlinewidth{0.311001pt}%
\definecolor{currentstroke}{rgb}{1.000000,1.000000,1.000000}%
\pgfsetstrokecolor{currentstroke}%
\pgfsetdash{}{0pt}%
\pgfpathmoveto{\pgfqpoint{4.116410in}{0.923104in}}%
\pgfpathcurveto{\pgfqpoint{4.123543in}{0.923104in}}{\pgfqpoint{4.130385in}{0.925938in}}{\pgfqpoint{4.135429in}{0.930982in}}%
\pgfpathcurveto{\pgfqpoint{4.140472in}{0.936025in}}{\pgfqpoint{4.143306in}{0.942867in}}{\pgfqpoint{4.143306in}{0.950000in}}%
\pgfpathcurveto{\pgfqpoint{4.143306in}{0.957133in}}{\pgfqpoint{4.140472in}{0.963974in}}{\pgfqpoint{4.135429in}{0.969018in}}%
\pgfpathcurveto{\pgfqpoint{4.130385in}{0.974062in}}{\pgfqpoint{4.123543in}{0.976896in}}{\pgfqpoint{4.116410in}{0.976896in}}%
\pgfpathcurveto{\pgfqpoint{4.109278in}{0.976896in}}{\pgfqpoint{4.102436in}{0.974062in}}{\pgfqpoint{4.097392in}{0.969018in}}%
\pgfpathcurveto{\pgfqpoint{4.092349in}{0.963974in}}{\pgfqpoint{4.089515in}{0.957133in}}{\pgfqpoint{4.089515in}{0.950000in}}%
\pgfpathcurveto{\pgfqpoint{4.089515in}{0.942867in}}{\pgfqpoint{4.092349in}{0.936025in}}{\pgfqpoint{4.097392in}{0.930982in}}%
\pgfpathcurveto{\pgfqpoint{4.102436in}{0.925938in}}{\pgfqpoint{4.109278in}{0.923104in}}{\pgfqpoint{4.116410in}{0.923104in}}%
\pgfpathclose%
\pgfusepath{stroke,fill}%
\end{pgfscope}%
\begin{pgfscope}%
\pgfpathrectangle{\pgfqpoint{2.867647in}{0.500000in}}{\pgfqpoint{1.764706in}{1.700000in}}%
\pgfusepath{clip}%
\pgfsetbuttcap%
\pgfsetroundjoin%
\definecolor{currentfill}{rgb}{0.966120,0.744512,0.608720}%
\pgfsetfillcolor{currentfill}%
\pgfsetlinewidth{0.311001pt}%
\definecolor{currentstroke}{rgb}{1.000000,1.000000,1.000000}%
\pgfsetstrokecolor{currentstroke}%
\pgfsetdash{}{0pt}%
\pgfpathmoveto{\pgfqpoint{4.091208in}{0.909692in}}%
\pgfpathcurveto{\pgfqpoint{4.098341in}{0.909692in}}{\pgfqpoint{4.105183in}{0.912526in}}{\pgfqpoint{4.110226in}{0.917570in}}%
\pgfpathcurveto{\pgfqpoint{4.115270in}{0.922613in}}{\pgfqpoint{4.118104in}{0.929455in}}{\pgfqpoint{4.118104in}{0.936588in}}%
\pgfpathcurveto{\pgfqpoint{4.118104in}{0.943721in}}{\pgfqpoint{4.115270in}{0.950562in}}{\pgfqpoint{4.110226in}{0.955606in}}%
\pgfpathcurveto{\pgfqpoint{4.105183in}{0.960650in}}{\pgfqpoint{4.098341in}{0.963484in}}{\pgfqpoint{4.091208in}{0.963484in}}%
\pgfpathcurveto{\pgfqpoint{4.084075in}{0.963484in}}{\pgfqpoint{4.077234in}{0.960650in}}{\pgfqpoint{4.072190in}{0.955606in}}%
\pgfpathcurveto{\pgfqpoint{4.067146in}{0.950562in}}{\pgfqpoint{4.064312in}{0.943721in}}{\pgfqpoint{4.064312in}{0.936588in}}%
\pgfpathcurveto{\pgfqpoint{4.064312in}{0.929455in}}{\pgfqpoint{4.067146in}{0.922613in}}{\pgfqpoint{4.072190in}{0.917570in}}%
\pgfpathcurveto{\pgfqpoint{4.077234in}{0.912526in}}{\pgfqpoint{4.084075in}{0.909692in}}{\pgfqpoint{4.091208in}{0.909692in}}%
\pgfpathclose%
\pgfusepath{stroke,fill}%
\end{pgfscope}%
\begin{pgfscope}%
\pgfpathrectangle{\pgfqpoint{2.867647in}{0.500000in}}{\pgfqpoint{1.764706in}{1.700000in}}%
\pgfusepath{clip}%
\pgfsetbuttcap%
\pgfsetroundjoin%
\definecolor{currentfill}{rgb}{0.970718,0.821518,0.719872}%
\pgfsetfillcolor{currentfill}%
\pgfsetlinewidth{0.311001pt}%
\definecolor{currentstroke}{rgb}{1.000000,1.000000,1.000000}%
\pgfsetstrokecolor{currentstroke}%
\pgfsetdash{}{0pt}%
\pgfpathmoveto{\pgfqpoint{4.051126in}{1.651183in}}%
\pgfpathcurveto{\pgfqpoint{4.058259in}{1.651183in}}{\pgfqpoint{4.065101in}{1.654017in}}{\pgfqpoint{4.070144in}{1.659061in}}%
\pgfpathcurveto{\pgfqpoint{4.075188in}{1.664105in}}{\pgfqpoint{4.078022in}{1.670946in}}{\pgfqpoint{4.078022in}{1.678079in}}%
\pgfpathcurveto{\pgfqpoint{4.078022in}{1.685212in}}{\pgfqpoint{4.075188in}{1.692054in}}{\pgfqpoint{4.070144in}{1.697097in}}%
\pgfpathcurveto{\pgfqpoint{4.065101in}{1.702141in}}{\pgfqpoint{4.058259in}{1.704975in}}{\pgfqpoint{4.051126in}{1.704975in}}%
\pgfpathcurveto{\pgfqpoint{4.043993in}{1.704975in}}{\pgfqpoint{4.037152in}{1.702141in}}{\pgfqpoint{4.032108in}{1.697097in}}%
\pgfpathcurveto{\pgfqpoint{4.027064in}{1.692054in}}{\pgfqpoint{4.024230in}{1.685212in}}{\pgfqpoint{4.024230in}{1.678079in}}%
\pgfpathcurveto{\pgfqpoint{4.024230in}{1.670946in}}{\pgfqpoint{4.027064in}{1.664105in}}{\pgfqpoint{4.032108in}{1.659061in}}%
\pgfpathcurveto{\pgfqpoint{4.037152in}{1.654017in}}{\pgfqpoint{4.043993in}{1.651183in}}{\pgfqpoint{4.051126in}{1.651183in}}%
\pgfpathclose%
\pgfusepath{stroke,fill}%
\end{pgfscope}%
\begin{pgfscope}%
\pgfpathrectangle{\pgfqpoint{2.867647in}{0.500000in}}{\pgfqpoint{1.764706in}{1.700000in}}%
\pgfusepath{clip}%
\pgfsetbuttcap%
\pgfsetroundjoin%
\definecolor{currentfill}{rgb}{0.970255,0.815666,0.711203}%
\pgfsetfillcolor{currentfill}%
\pgfsetlinewidth{0.311001pt}%
\definecolor{currentstroke}{rgb}{1.000000,1.000000,1.000000}%
\pgfsetstrokecolor{currentstroke}%
\pgfsetdash{}{0pt}%
\pgfpathmoveto{\pgfqpoint{4.268731in}{1.392351in}}%
\pgfpathcurveto{\pgfqpoint{4.275864in}{1.392351in}}{\pgfqpoint{4.282705in}{1.395185in}}{\pgfqpoint{4.287749in}{1.400229in}}%
\pgfpathcurveto{\pgfqpoint{4.292793in}{1.405272in}}{\pgfqpoint{4.295627in}{1.412114in}}{\pgfqpoint{4.295627in}{1.419247in}}%
\pgfpathcurveto{\pgfqpoint{4.295627in}{1.426380in}}{\pgfqpoint{4.292793in}{1.433221in}}{\pgfqpoint{4.287749in}{1.438265in}}%
\pgfpathcurveto{\pgfqpoint{4.282705in}{1.443309in}}{\pgfqpoint{4.275864in}{1.446143in}}{\pgfqpoint{4.268731in}{1.446143in}}%
\pgfpathcurveto{\pgfqpoint{4.261598in}{1.446143in}}{\pgfqpoint{4.254757in}{1.443309in}}{\pgfqpoint{4.249713in}{1.438265in}}%
\pgfpathcurveto{\pgfqpoint{4.244669in}{1.433221in}}{\pgfqpoint{4.241835in}{1.426380in}}{\pgfqpoint{4.241835in}{1.419247in}}%
\pgfpathcurveto{\pgfqpoint{4.241835in}{1.412114in}}{\pgfqpoint{4.244669in}{1.405272in}}{\pgfqpoint{4.249713in}{1.400229in}}%
\pgfpathcurveto{\pgfqpoint{4.254757in}{1.395185in}}{\pgfqpoint{4.261598in}{1.392351in}}{\pgfqpoint{4.268731in}{1.392351in}}%
\pgfpathclose%
\pgfusepath{stroke,fill}%
\end{pgfscope}%
\begin{pgfscope}%
\pgfpathrectangle{\pgfqpoint{2.867647in}{0.500000in}}{\pgfqpoint{1.764706in}{1.700000in}}%
\pgfusepath{clip}%
\pgfsetbuttcap%
\pgfsetroundjoin%
\definecolor{currentfill}{rgb}{0.981377,0.920617,0.865369}%
\pgfsetfillcolor{currentfill}%
\pgfsetlinewidth{0.311001pt}%
\definecolor{currentstroke}{rgb}{1.000000,1.000000,1.000000}%
\pgfsetstrokecolor{currentstroke}%
\pgfsetdash{}{0pt}%
\pgfpathmoveto{\pgfqpoint{4.189085in}{1.264972in}}%
\pgfpathcurveto{\pgfqpoint{4.196217in}{1.264972in}}{\pgfqpoint{4.203059in}{1.267806in}}{\pgfqpoint{4.208103in}{1.272850in}}%
\pgfpathcurveto{\pgfqpoint{4.213146in}{1.277893in}}{\pgfqpoint{4.215980in}{1.284735in}}{\pgfqpoint{4.215980in}{1.291868in}}%
\pgfpathcurveto{\pgfqpoint{4.215980in}{1.299001in}}{\pgfqpoint{4.213146in}{1.305842in}}{\pgfqpoint{4.208103in}{1.310886in}}%
\pgfpathcurveto{\pgfqpoint{4.203059in}{1.315930in}}{\pgfqpoint{4.196217in}{1.318764in}}{\pgfqpoint{4.189085in}{1.318764in}}%
\pgfpathcurveto{\pgfqpoint{4.181952in}{1.318764in}}{\pgfqpoint{4.175110in}{1.315930in}}{\pgfqpoint{4.170067in}{1.310886in}}%
\pgfpathcurveto{\pgfqpoint{4.165023in}{1.305842in}}{\pgfqpoint{4.162189in}{1.299001in}}{\pgfqpoint{4.162189in}{1.291868in}}%
\pgfpathcurveto{\pgfqpoint{4.162189in}{1.284735in}}{\pgfqpoint{4.165023in}{1.277893in}}{\pgfqpoint{4.170067in}{1.272850in}}%
\pgfpathcurveto{\pgfqpoint{4.175110in}{1.267806in}}{\pgfqpoint{4.181952in}{1.264972in}}{\pgfqpoint{4.189085in}{1.264972in}}%
\pgfpathclose%
\pgfusepath{stroke,fill}%
\end{pgfscope}%
\begin{pgfscope}%
\pgfpathrectangle{\pgfqpoint{2.867647in}{0.500000in}}{\pgfqpoint{1.764706in}{1.700000in}}%
\pgfusepath{clip}%
\pgfsetbuttcap%
\pgfsetroundjoin%
\definecolor{currentfill}{rgb}{0.823415,0.125353,0.296370}%
\pgfsetfillcolor{currentfill}%
\pgfsetlinewidth{0.311001pt}%
\definecolor{currentstroke}{rgb}{1.000000,1.000000,1.000000}%
\pgfsetstrokecolor{currentstroke}%
\pgfsetdash{}{0pt}%
\pgfpathmoveto{\pgfqpoint{4.342473in}{0.989155in}}%
\pgfpathcurveto{\pgfqpoint{4.349606in}{0.989155in}}{\pgfqpoint{4.356448in}{0.991989in}}{\pgfqpoint{4.361492in}{0.997032in}}%
\pgfpathcurveto{\pgfqpoint{4.366535in}{1.002076in}}{\pgfqpoint{4.369369in}{1.008918in}}{\pgfqpoint{4.369369in}{1.016051in}}%
\pgfpathcurveto{\pgfqpoint{4.369369in}{1.023183in}}{\pgfqpoint{4.366535in}{1.030025in}}{\pgfqpoint{4.361492in}{1.035069in}}%
\pgfpathcurveto{\pgfqpoint{4.356448in}{1.040112in}}{\pgfqpoint{4.349606in}{1.042946in}}{\pgfqpoint{4.342473in}{1.042946in}}%
\pgfpathcurveto{\pgfqpoint{4.335341in}{1.042946in}}{\pgfqpoint{4.328499in}{1.040112in}}{\pgfqpoint{4.323455in}{1.035069in}}%
\pgfpathcurveto{\pgfqpoint{4.318412in}{1.030025in}}{\pgfqpoint{4.315578in}{1.023183in}}{\pgfqpoint{4.315578in}{1.016051in}}%
\pgfpathcurveto{\pgfqpoint{4.315578in}{1.008918in}}{\pgfqpoint{4.318412in}{1.002076in}}{\pgfqpoint{4.323455in}{0.997032in}}%
\pgfpathcurveto{\pgfqpoint{4.328499in}{0.991989in}}{\pgfqpoint{4.335341in}{0.989155in}}{\pgfqpoint{4.342473in}{0.989155in}}%
\pgfpathclose%
\pgfusepath{stroke,fill}%
\end{pgfscope}%
\begin{pgfscope}%
\pgfpathrectangle{\pgfqpoint{2.867647in}{0.500000in}}{\pgfqpoint{1.764706in}{1.700000in}}%
\pgfusepath{clip}%
\pgfsetbuttcap%
\pgfsetroundjoin%
\definecolor{currentfill}{rgb}{0.978376,0.897317,0.831308}%
\pgfsetfillcolor{currentfill}%
\pgfsetlinewidth{0.311001pt}%
\definecolor{currentstroke}{rgb}{1.000000,1.000000,1.000000}%
\pgfsetstrokecolor{currentstroke}%
\pgfsetdash{}{0pt}%
\pgfpathmoveto{\pgfqpoint{4.145590in}{1.254437in}}%
\pgfpathcurveto{\pgfqpoint{4.152722in}{1.254437in}}{\pgfqpoint{4.159564in}{1.257271in}}{\pgfqpoint{4.164608in}{1.262314in}}%
\pgfpathcurveto{\pgfqpoint{4.169651in}{1.267358in}}{\pgfqpoint{4.172485in}{1.274200in}}{\pgfqpoint{4.172485in}{1.281332in}}%
\pgfpathcurveto{\pgfqpoint{4.172485in}{1.288465in}}{\pgfqpoint{4.169651in}{1.295307in}}{\pgfqpoint{4.164608in}{1.300351in}}%
\pgfpathcurveto{\pgfqpoint{4.159564in}{1.305394in}}{\pgfqpoint{4.152722in}{1.308228in}}{\pgfqpoint{4.145590in}{1.308228in}}%
\pgfpathcurveto{\pgfqpoint{4.138457in}{1.308228in}}{\pgfqpoint{4.131615in}{1.305394in}}{\pgfqpoint{4.126571in}{1.300351in}}%
\pgfpathcurveto{\pgfqpoint{4.121528in}{1.295307in}}{\pgfqpoint{4.118694in}{1.288465in}}{\pgfqpoint{4.118694in}{1.281332in}}%
\pgfpathcurveto{\pgfqpoint{4.118694in}{1.274200in}}{\pgfqpoint{4.121528in}{1.267358in}}{\pgfqpoint{4.126571in}{1.262314in}}%
\pgfpathcurveto{\pgfqpoint{4.131615in}{1.257271in}}{\pgfqpoint{4.138457in}{1.254437in}}{\pgfqpoint{4.145590in}{1.254437in}}%
\pgfpathclose%
\pgfusepath{stroke,fill}%
\end{pgfscope}%
\begin{pgfscope}%
\pgfpathrectangle{\pgfqpoint{2.867647in}{0.500000in}}{\pgfqpoint{1.764706in}{1.700000in}}%
\pgfusepath{clip}%
\pgfsetbuttcap%
\pgfsetroundjoin%
\definecolor{currentfill}{rgb}{0.977657,0.891500,0.822809}%
\pgfsetfillcolor{currentfill}%
\pgfsetlinewidth{0.311001pt}%
\definecolor{currentstroke}{rgb}{1.000000,1.000000,1.000000}%
\pgfsetstrokecolor{currentstroke}%
\pgfsetdash{}{0pt}%
\pgfpathmoveto{\pgfqpoint{4.191219in}{1.536971in}}%
\pgfpathcurveto{\pgfqpoint{4.198352in}{1.536971in}}{\pgfqpoint{4.205193in}{1.539804in}}{\pgfqpoint{4.210237in}{1.544848in}}%
\pgfpathcurveto{\pgfqpoint{4.215281in}{1.549892in}}{\pgfqpoint{4.218114in}{1.556733in}}{\pgfqpoint{4.218114in}{1.563866in}}%
\pgfpathcurveto{\pgfqpoint{4.218114in}{1.570999in}}{\pgfqpoint{4.215281in}{1.577841in}}{\pgfqpoint{4.210237in}{1.582884in}}%
\pgfpathcurveto{\pgfqpoint{4.205193in}{1.587928in}}{\pgfqpoint{4.198352in}{1.590762in}}{\pgfqpoint{4.191219in}{1.590762in}}%
\pgfpathcurveto{\pgfqpoint{4.184086in}{1.590762in}}{\pgfqpoint{4.177244in}{1.587928in}}{\pgfqpoint{4.172201in}{1.582884in}}%
\pgfpathcurveto{\pgfqpoint{4.167157in}{1.577841in}}{\pgfqpoint{4.164323in}{1.570999in}}{\pgfqpoint{4.164323in}{1.563866in}}%
\pgfpathcurveto{\pgfqpoint{4.164323in}{1.556733in}}{\pgfqpoint{4.167157in}{1.549892in}}{\pgfqpoint{4.172201in}{1.544848in}}%
\pgfpathcurveto{\pgfqpoint{4.177244in}{1.539804in}}{\pgfqpoint{4.184086in}{1.536971in}}{\pgfqpoint{4.191219in}{1.536971in}}%
\pgfpathclose%
\pgfusepath{stroke,fill}%
\end{pgfscope}%
\begin{pgfscope}%
\pgfpathrectangle{\pgfqpoint{2.867647in}{0.500000in}}{\pgfqpoint{1.764706in}{1.700000in}}%
\pgfusepath{clip}%
\pgfsetbuttcap%
\pgfsetroundjoin%
\definecolor{currentfill}{rgb}{0.978376,0.897317,0.831308}%
\pgfsetfillcolor{currentfill}%
\pgfsetlinewidth{0.311001pt}%
\definecolor{currentstroke}{rgb}{1.000000,1.000000,1.000000}%
\pgfsetstrokecolor{currentstroke}%
\pgfsetdash{}{0pt}%
\pgfpathmoveto{\pgfqpoint{4.215157in}{1.178824in}}%
\pgfpathcurveto{\pgfqpoint{4.222290in}{1.178824in}}{\pgfqpoint{4.229131in}{1.181658in}}{\pgfqpoint{4.234175in}{1.186701in}}%
\pgfpathcurveto{\pgfqpoint{4.239219in}{1.191745in}}{\pgfqpoint{4.242053in}{1.198587in}}{\pgfqpoint{4.242053in}{1.205719in}}%
\pgfpathcurveto{\pgfqpoint{4.242053in}{1.212852in}}{\pgfqpoint{4.239219in}{1.219694in}}{\pgfqpoint{4.234175in}{1.224738in}}%
\pgfpathcurveto{\pgfqpoint{4.229131in}{1.229781in}}{\pgfqpoint{4.222290in}{1.232615in}}{\pgfqpoint{4.215157in}{1.232615in}}%
\pgfpathcurveto{\pgfqpoint{4.208024in}{1.232615in}}{\pgfqpoint{4.201183in}{1.229781in}}{\pgfqpoint{4.196139in}{1.224738in}}%
\pgfpathcurveto{\pgfqpoint{4.191095in}{1.219694in}}{\pgfqpoint{4.188261in}{1.212852in}}{\pgfqpoint{4.188261in}{1.205719in}}%
\pgfpathcurveto{\pgfqpoint{4.188261in}{1.198587in}}{\pgfqpoint{4.191095in}{1.191745in}}{\pgfqpoint{4.196139in}{1.186701in}}%
\pgfpathcurveto{\pgfqpoint{4.201183in}{1.181658in}}{\pgfqpoint{4.208024in}{1.178824in}}{\pgfqpoint{4.215157in}{1.178824in}}%
\pgfpathclose%
\pgfusepath{stroke,fill}%
\end{pgfscope}%
\begin{pgfscope}%
\pgfpathrectangle{\pgfqpoint{2.867647in}{0.500000in}}{\pgfqpoint{1.764706in}{1.700000in}}%
\pgfusepath{clip}%
\pgfsetbuttcap%
\pgfsetroundjoin%
\definecolor{currentfill}{rgb}{0.979891,0.908948,0.848279}%
\pgfsetfillcolor{currentfill}%
\pgfsetlinewidth{0.311001pt}%
\definecolor{currentstroke}{rgb}{1.000000,1.000000,1.000000}%
\pgfsetstrokecolor{currentstroke}%
\pgfsetdash{}{0pt}%
\pgfpathmoveto{\pgfqpoint{4.161409in}{1.400615in}}%
\pgfpathcurveto{\pgfqpoint{4.168542in}{1.400615in}}{\pgfqpoint{4.175384in}{1.403448in}}{\pgfqpoint{4.180427in}{1.408492in}}%
\pgfpathcurveto{\pgfqpoint{4.185471in}{1.413536in}}{\pgfqpoint{4.188305in}{1.420377in}}{\pgfqpoint{4.188305in}{1.427510in}}%
\pgfpathcurveto{\pgfqpoint{4.188305in}{1.434643in}}{\pgfqpoint{4.185471in}{1.441485in}}{\pgfqpoint{4.180427in}{1.446528in}}%
\pgfpathcurveto{\pgfqpoint{4.175384in}{1.451572in}}{\pgfqpoint{4.168542in}{1.454406in}}{\pgfqpoint{4.161409in}{1.454406in}}%
\pgfpathcurveto{\pgfqpoint{4.154276in}{1.454406in}}{\pgfqpoint{4.147435in}{1.451572in}}{\pgfqpoint{4.142391in}{1.446528in}}%
\pgfpathcurveto{\pgfqpoint{4.137347in}{1.441485in}}{\pgfqpoint{4.134513in}{1.434643in}}{\pgfqpoint{4.134513in}{1.427510in}}%
\pgfpathcurveto{\pgfqpoint{4.134513in}{1.420377in}}{\pgfqpoint{4.137347in}{1.413536in}}{\pgfqpoint{4.142391in}{1.408492in}}%
\pgfpathcurveto{\pgfqpoint{4.147435in}{1.403448in}}{\pgfqpoint{4.154276in}{1.400615in}}{\pgfqpoint{4.161409in}{1.400615in}}%
\pgfpathclose%
\pgfusepath{stroke,fill}%
\end{pgfscope}%
\begin{pgfscope}%
\pgfpathrectangle{\pgfqpoint{2.867647in}{0.500000in}}{\pgfqpoint{1.764706in}{1.700000in}}%
\pgfusepath{clip}%
\pgfsetbuttcap%
\pgfsetroundjoin%
\definecolor{currentfill}{rgb}{0.979891,0.908948,0.848279}%
\pgfsetfillcolor{currentfill}%
\pgfsetlinewidth{0.311001pt}%
\definecolor{currentstroke}{rgb}{1.000000,1.000000,1.000000}%
\pgfsetstrokecolor{currentstroke}%
\pgfsetdash{}{0pt}%
\pgfpathmoveto{\pgfqpoint{4.204099in}{1.428364in}}%
\pgfpathcurveto{\pgfqpoint{4.211232in}{1.428364in}}{\pgfqpoint{4.218074in}{1.431198in}}{\pgfqpoint{4.223117in}{1.436242in}}%
\pgfpathcurveto{\pgfqpoint{4.228161in}{1.441286in}}{\pgfqpoint{4.230995in}{1.448127in}}{\pgfqpoint{4.230995in}{1.455260in}}%
\pgfpathcurveto{\pgfqpoint{4.230995in}{1.462393in}}{\pgfqpoint{4.228161in}{1.469235in}}{\pgfqpoint{4.223117in}{1.474278in}}%
\pgfpathcurveto{\pgfqpoint{4.218074in}{1.479322in}}{\pgfqpoint{4.211232in}{1.482156in}}{\pgfqpoint{4.204099in}{1.482156in}}%
\pgfpathcurveto{\pgfqpoint{4.196966in}{1.482156in}}{\pgfqpoint{4.190125in}{1.479322in}}{\pgfqpoint{4.185081in}{1.474278in}}%
\pgfpathcurveto{\pgfqpoint{4.180037in}{1.469235in}}{\pgfqpoint{4.177203in}{1.462393in}}{\pgfqpoint{4.177203in}{1.455260in}}%
\pgfpathcurveto{\pgfqpoint{4.177203in}{1.448127in}}{\pgfqpoint{4.180037in}{1.441286in}}{\pgfqpoint{4.185081in}{1.436242in}}%
\pgfpathcurveto{\pgfqpoint{4.190125in}{1.431198in}}{\pgfqpoint{4.196966in}{1.428364in}}{\pgfqpoint{4.204099in}{1.428364in}}%
\pgfpathclose%
\pgfusepath{stroke,fill}%
\end{pgfscope}%
\begin{pgfscope}%
\pgfpathrectangle{\pgfqpoint{2.867647in}{0.500000in}}{\pgfqpoint{1.764706in}{1.700000in}}%
\pgfusepath{clip}%
\pgfsetbuttcap%
\pgfsetroundjoin%
\definecolor{currentfill}{rgb}{0.964558,0.676556,0.522514}%
\pgfsetfillcolor{currentfill}%
\pgfsetlinewidth{0.311001pt}%
\definecolor{currentstroke}{rgb}{1.000000,1.000000,1.000000}%
\pgfsetstrokecolor{currentstroke}%
\pgfsetdash{}{0pt}%
\pgfpathmoveto{\pgfqpoint{3.970968in}{1.703814in}}%
\pgfpathcurveto{\pgfqpoint{3.978101in}{1.703814in}}{\pgfqpoint{3.984942in}{1.706648in}}{\pgfqpoint{3.989986in}{1.711691in}}%
\pgfpathcurveto{\pgfqpoint{3.995030in}{1.716735in}}{\pgfqpoint{3.997864in}{1.723577in}}{\pgfqpoint{3.997864in}{1.730709in}}%
\pgfpathcurveto{\pgfqpoint{3.997864in}{1.737842in}}{\pgfqpoint{3.995030in}{1.744684in}}{\pgfqpoint{3.989986in}{1.749728in}}%
\pgfpathcurveto{\pgfqpoint{3.984942in}{1.754771in}}{\pgfqpoint{3.978101in}{1.757605in}}{\pgfqpoint{3.970968in}{1.757605in}}%
\pgfpathcurveto{\pgfqpoint{3.963835in}{1.757605in}}{\pgfqpoint{3.956993in}{1.754771in}}{\pgfqpoint{3.951950in}{1.749728in}}%
\pgfpathcurveto{\pgfqpoint{3.946906in}{1.744684in}}{\pgfqpoint{3.944072in}{1.737842in}}{\pgfqpoint{3.944072in}{1.730709in}}%
\pgfpathcurveto{\pgfqpoint{3.944072in}{1.723577in}}{\pgfqpoint{3.946906in}{1.716735in}}{\pgfqpoint{3.951950in}{1.711691in}}%
\pgfpathcurveto{\pgfqpoint{3.956993in}{1.706648in}}{\pgfqpoint{3.963835in}{1.703814in}}{\pgfqpoint{3.970968in}{1.703814in}}%
\pgfpathclose%
\pgfusepath{stroke,fill}%
\end{pgfscope}%
\begin{pgfscope}%
\pgfpathrectangle{\pgfqpoint{2.867647in}{0.500000in}}{\pgfqpoint{1.764706in}{1.700000in}}%
\pgfusepath{clip}%
\pgfsetbuttcap%
\pgfsetroundjoin%
\definecolor{currentfill}{rgb}{0.969803,0.809811,0.702523}%
\pgfsetfillcolor{currentfill}%
\pgfsetlinewidth{0.311001pt}%
\definecolor{currentstroke}{rgb}{1.000000,1.000000,1.000000}%
\pgfsetstrokecolor{currentstroke}%
\pgfsetdash{}{0pt}%
\pgfpathmoveto{\pgfqpoint{4.080279in}{0.949751in}}%
\pgfpathcurveto{\pgfqpoint{4.087411in}{0.949751in}}{\pgfqpoint{4.094253in}{0.952585in}}{\pgfqpoint{4.099297in}{0.957629in}}%
\pgfpathcurveto{\pgfqpoint{4.104340in}{0.962673in}}{\pgfqpoint{4.107174in}{0.969514in}}{\pgfqpoint{4.107174in}{0.976647in}}%
\pgfpathcurveto{\pgfqpoint{4.107174in}{0.983780in}}{\pgfqpoint{4.104340in}{0.990622in}}{\pgfqpoint{4.099297in}{0.995665in}}%
\pgfpathcurveto{\pgfqpoint{4.094253in}{1.000709in}}{\pgfqpoint{4.087411in}{1.003543in}}{\pgfqpoint{4.080279in}{1.003543in}}%
\pgfpathcurveto{\pgfqpoint{4.073146in}{1.003543in}}{\pgfqpoint{4.066304in}{1.000709in}}{\pgfqpoint{4.061261in}{0.995665in}}%
\pgfpathcurveto{\pgfqpoint{4.056217in}{0.990622in}}{\pgfqpoint{4.053383in}{0.983780in}}{\pgfqpoint{4.053383in}{0.976647in}}%
\pgfpathcurveto{\pgfqpoint{4.053383in}{0.969514in}}{\pgfqpoint{4.056217in}{0.962673in}}{\pgfqpoint{4.061261in}{0.957629in}}%
\pgfpathcurveto{\pgfqpoint{4.066304in}{0.952585in}}{\pgfqpoint{4.073146in}{0.949751in}}{\pgfqpoint{4.080279in}{0.949751in}}%
\pgfpathclose%
\pgfusepath{stroke,fill}%
\end{pgfscope}%
\begin{pgfscope}%
\pgfpathrectangle{\pgfqpoint{2.867647in}{0.500000in}}{\pgfqpoint{1.764706in}{1.700000in}}%
\pgfusepath{clip}%
\pgfsetbuttcap%
\pgfsetroundjoin%
\definecolor{currentfill}{rgb}{0.958791,0.526283,0.368909}%
\pgfsetfillcolor{currentfill}%
\pgfsetlinewidth{0.311001pt}%
\definecolor{currentstroke}{rgb}{1.000000,1.000000,1.000000}%
\pgfsetstrokecolor{currentstroke}%
\pgfsetdash{}{0pt}%
\pgfpathmoveto{\pgfqpoint{4.018233in}{1.203469in}}%
\pgfpathcurveto{\pgfqpoint{4.025366in}{1.203469in}}{\pgfqpoint{4.032208in}{1.206303in}}{\pgfqpoint{4.037251in}{1.211347in}}%
\pgfpathcurveto{\pgfqpoint{4.042295in}{1.216390in}}{\pgfqpoint{4.045129in}{1.223232in}}{\pgfqpoint{4.045129in}{1.230365in}}%
\pgfpathcurveto{\pgfqpoint{4.045129in}{1.237498in}}{\pgfqpoint{4.042295in}{1.244339in}}{\pgfqpoint{4.037251in}{1.249383in}}%
\pgfpathcurveto{\pgfqpoint{4.032208in}{1.254427in}}{\pgfqpoint{4.025366in}{1.257260in}}{\pgfqpoint{4.018233in}{1.257260in}}%
\pgfpathcurveto{\pgfqpoint{4.011100in}{1.257260in}}{\pgfqpoint{4.004259in}{1.254427in}}{\pgfqpoint{3.999215in}{1.249383in}}%
\pgfpathcurveto{\pgfqpoint{3.994171in}{1.244339in}}{\pgfqpoint{3.991337in}{1.237498in}}{\pgfqpoint{3.991337in}{1.230365in}}%
\pgfpathcurveto{\pgfqpoint{3.991337in}{1.223232in}}{\pgfqpoint{3.994171in}{1.216390in}}{\pgfqpoint{3.999215in}{1.211347in}}%
\pgfpathcurveto{\pgfqpoint{4.004259in}{1.206303in}}{\pgfqpoint{4.011100in}{1.203469in}}{\pgfqpoint{4.018233in}{1.203469in}}%
\pgfpathclose%
\pgfusepath{stroke,fill}%
\end{pgfscope}%
\begin{pgfscope}%
\pgfpathrectangle{\pgfqpoint{2.867647in}{0.500000in}}{\pgfqpoint{1.764706in}{1.700000in}}%
\pgfusepath{clip}%
\pgfsetbuttcap%
\pgfsetroundjoin%
\definecolor{currentfill}{rgb}{0.972201,0.839051,0.745789}%
\pgfsetfillcolor{currentfill}%
\pgfsetlinewidth{0.311001pt}%
\definecolor{currentstroke}{rgb}{1.000000,1.000000,1.000000}%
\pgfsetstrokecolor{currentstroke}%
\pgfsetdash{}{0pt}%
\pgfpathmoveto{\pgfqpoint{4.073355in}{0.990541in}}%
\pgfpathcurveto{\pgfqpoint{4.080488in}{0.990541in}}{\pgfqpoint{4.087329in}{0.993374in}}{\pgfqpoint{4.092373in}{0.998418in}}%
\pgfpathcurveto{\pgfqpoint{4.097417in}{1.003462in}}{\pgfqpoint{4.100251in}{1.010303in}}{\pgfqpoint{4.100251in}{1.017436in}}%
\pgfpathcurveto{\pgfqpoint{4.100251in}{1.024569in}}{\pgfqpoint{4.097417in}{1.031411in}}{\pgfqpoint{4.092373in}{1.036454in}}%
\pgfpathcurveto{\pgfqpoint{4.087329in}{1.041498in}}{\pgfqpoint{4.080488in}{1.044332in}}{\pgfqpoint{4.073355in}{1.044332in}}%
\pgfpathcurveto{\pgfqpoint{4.066222in}{1.044332in}}{\pgfqpoint{4.059380in}{1.041498in}}{\pgfqpoint{4.054337in}{1.036454in}}%
\pgfpathcurveto{\pgfqpoint{4.049293in}{1.031411in}}{\pgfqpoint{4.046459in}{1.024569in}}{\pgfqpoint{4.046459in}{1.017436in}}%
\pgfpathcurveto{\pgfqpoint{4.046459in}{1.010303in}}{\pgfqpoint{4.049293in}{1.003462in}}{\pgfqpoint{4.054337in}{0.998418in}}%
\pgfpathcurveto{\pgfqpoint{4.059380in}{0.993374in}}{\pgfqpoint{4.066222in}{0.990541in}}{\pgfqpoint{4.073355in}{0.990541in}}%
\pgfpathclose%
\pgfusepath{stroke,fill}%
\end{pgfscope}%
\begin{pgfscope}%
\pgfpathrectangle{\pgfqpoint{2.867647in}{0.500000in}}{\pgfqpoint{1.764706in}{1.700000in}}%
\pgfusepath{clip}%
\pgfsetbuttcap%
\pgfsetroundjoin%
\definecolor{currentfill}{rgb}{0.975644,0.874038,0.797253}%
\pgfsetfillcolor{currentfill}%
\pgfsetlinewidth{0.311001pt}%
\definecolor{currentstroke}{rgb}{1.000000,1.000000,1.000000}%
\pgfsetstrokecolor{currentstroke}%
\pgfsetdash{}{0pt}%
\pgfpathmoveto{\pgfqpoint{4.235976in}{1.193756in}}%
\pgfpathcurveto{\pgfqpoint{4.243108in}{1.193756in}}{\pgfqpoint{4.249950in}{1.196590in}}{\pgfqpoint{4.254994in}{1.201634in}}%
\pgfpathcurveto{\pgfqpoint{4.260037in}{1.206678in}}{\pgfqpoint{4.262871in}{1.213519in}}{\pgfqpoint{4.262871in}{1.220652in}}%
\pgfpathcurveto{\pgfqpoint{4.262871in}{1.227785in}}{\pgfqpoint{4.260037in}{1.234627in}}{\pgfqpoint{4.254994in}{1.239670in}}%
\pgfpathcurveto{\pgfqpoint{4.249950in}{1.244714in}}{\pgfqpoint{4.243108in}{1.247548in}}{\pgfqpoint{4.235976in}{1.247548in}}%
\pgfpathcurveto{\pgfqpoint{4.228843in}{1.247548in}}{\pgfqpoint{4.222001in}{1.244714in}}{\pgfqpoint{4.216958in}{1.239670in}}%
\pgfpathcurveto{\pgfqpoint{4.211914in}{1.234627in}}{\pgfqpoint{4.209080in}{1.227785in}}{\pgfqpoint{4.209080in}{1.220652in}}%
\pgfpathcurveto{\pgfqpoint{4.209080in}{1.213519in}}{\pgfqpoint{4.211914in}{1.206678in}}{\pgfqpoint{4.216958in}{1.201634in}}%
\pgfpathcurveto{\pgfqpoint{4.222001in}{1.196590in}}{\pgfqpoint{4.228843in}{1.193756in}}{\pgfqpoint{4.235976in}{1.193756in}}%
\pgfpathclose%
\pgfusepath{stroke,fill}%
\end{pgfscope}%
\begin{pgfscope}%
\pgfpathrectangle{\pgfqpoint{2.867647in}{0.500000in}}{\pgfqpoint{1.764706in}{1.700000in}}%
\pgfusepath{clip}%
\pgfsetbuttcap%
\pgfsetroundjoin%
\definecolor{currentfill}{rgb}{0.978376,0.897317,0.831308}%
\pgfsetfillcolor{currentfill}%
\pgfsetlinewidth{0.311001pt}%
\definecolor{currentstroke}{rgb}{1.000000,1.000000,1.000000}%
\pgfsetstrokecolor{currentstroke}%
\pgfsetdash{}{0pt}%
\pgfpathmoveto{\pgfqpoint{4.129239in}{1.060411in}}%
\pgfpathcurveto{\pgfqpoint{4.136371in}{1.060411in}}{\pgfqpoint{4.143213in}{1.063245in}}{\pgfqpoint{4.148257in}{1.068288in}}%
\pgfpathcurveto{\pgfqpoint{4.153300in}{1.073332in}}{\pgfqpoint{4.156134in}{1.080174in}}{\pgfqpoint{4.156134in}{1.087307in}}%
\pgfpathcurveto{\pgfqpoint{4.156134in}{1.094439in}}{\pgfqpoint{4.153300in}{1.101281in}}{\pgfqpoint{4.148257in}{1.106325in}}%
\pgfpathcurveto{\pgfqpoint{4.143213in}{1.111368in}}{\pgfqpoint{4.136371in}{1.114202in}}{\pgfqpoint{4.129239in}{1.114202in}}%
\pgfpathcurveto{\pgfqpoint{4.122106in}{1.114202in}}{\pgfqpoint{4.115264in}{1.111368in}}{\pgfqpoint{4.110220in}{1.106325in}}%
\pgfpathcurveto{\pgfqpoint{4.105177in}{1.101281in}}{\pgfqpoint{4.102343in}{1.094439in}}{\pgfqpoint{4.102343in}{1.087307in}}%
\pgfpathcurveto{\pgfqpoint{4.102343in}{1.080174in}}{\pgfqpoint{4.105177in}{1.073332in}}{\pgfqpoint{4.110220in}{1.068288in}}%
\pgfpathcurveto{\pgfqpoint{4.115264in}{1.063245in}}{\pgfqpoint{4.122106in}{1.060411in}}{\pgfqpoint{4.129239in}{1.060411in}}%
\pgfpathclose%
\pgfusepath{stroke,fill}%
\end{pgfscope}%
\begin{pgfscope}%
\pgfpathrectangle{\pgfqpoint{2.867647in}{0.500000in}}{\pgfqpoint{1.764706in}{1.700000in}}%
\pgfusepath{clip}%
\pgfsetbuttcap%
\pgfsetroundjoin%
\definecolor{currentfill}{rgb}{0.975644,0.874038,0.797253}%
\pgfsetfillcolor{currentfill}%
\pgfsetlinewidth{0.311001pt}%
\definecolor{currentstroke}{rgb}{1.000000,1.000000,1.000000}%
\pgfsetstrokecolor{currentstroke}%
\pgfsetdash{}{0pt}%
\pgfpathmoveto{\pgfqpoint{4.243504in}{1.256459in}}%
\pgfpathcurveto{\pgfqpoint{4.250637in}{1.256459in}}{\pgfqpoint{4.257479in}{1.259293in}}{\pgfqpoint{4.262523in}{1.264337in}}%
\pgfpathcurveto{\pgfqpoint{4.267566in}{1.269380in}}{\pgfqpoint{4.270400in}{1.276222in}}{\pgfqpoint{4.270400in}{1.283355in}}%
\pgfpathcurveto{\pgfqpoint{4.270400in}{1.290488in}}{\pgfqpoint{4.267566in}{1.297329in}}{\pgfqpoint{4.262523in}{1.302373in}}%
\pgfpathcurveto{\pgfqpoint{4.257479in}{1.307417in}}{\pgfqpoint{4.250637in}{1.310250in}}{\pgfqpoint{4.243504in}{1.310250in}}%
\pgfpathcurveto{\pgfqpoint{4.236372in}{1.310250in}}{\pgfqpoint{4.229530in}{1.307417in}}{\pgfqpoint{4.224486in}{1.302373in}}%
\pgfpathcurveto{\pgfqpoint{4.219443in}{1.297329in}}{\pgfqpoint{4.216609in}{1.290488in}}{\pgfqpoint{4.216609in}{1.283355in}}%
\pgfpathcurveto{\pgfqpoint{4.216609in}{1.276222in}}{\pgfqpoint{4.219443in}{1.269380in}}{\pgfqpoint{4.224486in}{1.264337in}}%
\pgfpathcurveto{\pgfqpoint{4.229530in}{1.259293in}}{\pgfqpoint{4.236372in}{1.256459in}}{\pgfqpoint{4.243504in}{1.256459in}}%
\pgfpathclose%
\pgfusepath{stroke,fill}%
\end{pgfscope}%
\begin{pgfscope}%
\pgfpathrectangle{\pgfqpoint{2.867647in}{0.500000in}}{\pgfqpoint{1.764706in}{1.700000in}}%
\pgfusepath{clip}%
\pgfsetbuttcap%
\pgfsetroundjoin%
\definecolor{currentfill}{rgb}{0.935991,0.337039,0.249722}%
\pgfsetfillcolor{currentfill}%
\pgfsetlinewidth{0.311001pt}%
\definecolor{currentstroke}{rgb}{1.000000,1.000000,1.000000}%
\pgfsetstrokecolor{currentstroke}%
\pgfsetdash{}{0pt}%
\pgfpathmoveto{\pgfqpoint{4.124137in}{0.809568in}}%
\pgfpathcurveto{\pgfqpoint{4.131270in}{0.809568in}}{\pgfqpoint{4.138112in}{0.812401in}}{\pgfqpoint{4.143156in}{0.817445in}}%
\pgfpathcurveto{\pgfqpoint{4.148199in}{0.822489in}}{\pgfqpoint{4.151033in}{0.829330in}}{\pgfqpoint{4.151033in}{0.836463in}}%
\pgfpathcurveto{\pgfqpoint{4.151033in}{0.843596in}}{\pgfqpoint{4.148199in}{0.850438in}}{\pgfqpoint{4.143156in}{0.855481in}}%
\pgfpathcurveto{\pgfqpoint{4.138112in}{0.860525in}}{\pgfqpoint{4.131270in}{0.863359in}}{\pgfqpoint{4.124137in}{0.863359in}}%
\pgfpathcurveto{\pgfqpoint{4.117005in}{0.863359in}}{\pgfqpoint{4.110163in}{0.860525in}}{\pgfqpoint{4.105119in}{0.855481in}}%
\pgfpathcurveto{\pgfqpoint{4.100076in}{0.850438in}}{\pgfqpoint{4.097242in}{0.843596in}}{\pgfqpoint{4.097242in}{0.836463in}}%
\pgfpathcurveto{\pgfqpoint{4.097242in}{0.829330in}}{\pgfqpoint{4.100076in}{0.822489in}}{\pgfqpoint{4.105119in}{0.817445in}}%
\pgfpathcurveto{\pgfqpoint{4.110163in}{0.812401in}}{\pgfqpoint{4.117005in}{0.809568in}}{\pgfqpoint{4.124137in}{0.809568in}}%
\pgfpathclose%
\pgfusepath{stroke,fill}%
\end{pgfscope}%
\begin{pgfscope}%
\pgfpathrectangle{\pgfqpoint{2.867647in}{0.500000in}}{\pgfqpoint{1.764706in}{1.700000in}}%
\pgfusepath{clip}%
\pgfsetbuttcap%
\pgfsetroundjoin%
\definecolor{currentfill}{rgb}{0.975644,0.874038,0.797253}%
\pgfsetfillcolor{currentfill}%
\pgfsetlinewidth{0.311001pt}%
\definecolor{currentstroke}{rgb}{1.000000,1.000000,1.000000}%
\pgfsetstrokecolor{currentstroke}%
\pgfsetdash{}{0pt}%
\pgfpathmoveto{\pgfqpoint{4.127589in}{1.635360in}}%
\pgfpathcurveto{\pgfqpoint{4.134722in}{1.635360in}}{\pgfqpoint{4.141563in}{1.638194in}}{\pgfqpoint{4.146607in}{1.643238in}}%
\pgfpathcurveto{\pgfqpoint{4.151651in}{1.648281in}}{\pgfqpoint{4.154485in}{1.655123in}}{\pgfqpoint{4.154485in}{1.662256in}}%
\pgfpathcurveto{\pgfqpoint{4.154485in}{1.669389in}}{\pgfqpoint{4.151651in}{1.676230in}}{\pgfqpoint{4.146607in}{1.681274in}}%
\pgfpathcurveto{\pgfqpoint{4.141563in}{1.686318in}}{\pgfqpoint{4.134722in}{1.689151in}}{\pgfqpoint{4.127589in}{1.689151in}}%
\pgfpathcurveto{\pgfqpoint{4.120456in}{1.689151in}}{\pgfqpoint{4.113614in}{1.686318in}}{\pgfqpoint{4.108571in}{1.681274in}}%
\pgfpathcurveto{\pgfqpoint{4.103527in}{1.676230in}}{\pgfqpoint{4.100693in}{1.669389in}}{\pgfqpoint{4.100693in}{1.662256in}}%
\pgfpathcurveto{\pgfqpoint{4.100693in}{1.655123in}}{\pgfqpoint{4.103527in}{1.648281in}}{\pgfqpoint{4.108571in}{1.643238in}}%
\pgfpathcurveto{\pgfqpoint{4.113614in}{1.638194in}}{\pgfqpoint{4.120456in}{1.635360in}}{\pgfqpoint{4.127589in}{1.635360in}}%
\pgfpathclose%
\pgfusepath{stroke,fill}%
\end{pgfscope}%
\begin{pgfscope}%
\pgfpathrectangle{\pgfqpoint{2.867647in}{0.500000in}}{\pgfqpoint{1.764706in}{1.700000in}}%
\pgfusepath{clip}%
\pgfsetbuttcap%
\pgfsetroundjoin%
\definecolor{currentfill}{rgb}{0.979124,0.903132,0.839793}%
\pgfsetfillcolor{currentfill}%
\pgfsetlinewidth{0.311001pt}%
\definecolor{currentstroke}{rgb}{1.000000,1.000000,1.000000}%
\pgfsetstrokecolor{currentstroke}%
\pgfsetdash{}{0pt}%
\pgfpathmoveto{\pgfqpoint{4.213148in}{1.397678in}}%
\pgfpathcurveto{\pgfqpoint{4.220281in}{1.397678in}}{\pgfqpoint{4.227123in}{1.400512in}}{\pgfqpoint{4.232166in}{1.405555in}}%
\pgfpathcurveto{\pgfqpoint{4.237210in}{1.410599in}}{\pgfqpoint{4.240044in}{1.417441in}}{\pgfqpoint{4.240044in}{1.424573in}}%
\pgfpathcurveto{\pgfqpoint{4.240044in}{1.431706in}}{\pgfqpoint{4.237210in}{1.438548in}}{\pgfqpoint{4.232166in}{1.443592in}}%
\pgfpathcurveto{\pgfqpoint{4.227123in}{1.448635in}}{\pgfqpoint{4.220281in}{1.451469in}}{\pgfqpoint{4.213148in}{1.451469in}}%
\pgfpathcurveto{\pgfqpoint{4.206015in}{1.451469in}}{\pgfqpoint{4.199174in}{1.448635in}}{\pgfqpoint{4.194130in}{1.443592in}}%
\pgfpathcurveto{\pgfqpoint{4.189086in}{1.438548in}}{\pgfqpoint{4.186252in}{1.431706in}}{\pgfqpoint{4.186252in}{1.424573in}}%
\pgfpathcurveto{\pgfqpoint{4.186252in}{1.417441in}}{\pgfqpoint{4.189086in}{1.410599in}}{\pgfqpoint{4.194130in}{1.405555in}}%
\pgfpathcurveto{\pgfqpoint{4.199174in}{1.400512in}}{\pgfqpoint{4.206015in}{1.397678in}}{\pgfqpoint{4.213148in}{1.397678in}}%
\pgfpathclose%
\pgfusepath{stroke,fill}%
\end{pgfscope}%
\begin{pgfscope}%
\pgfpathrectangle{\pgfqpoint{2.867647in}{0.500000in}}{\pgfqpoint{1.764706in}{1.700000in}}%
\pgfusepath{clip}%
\pgfsetbuttcap%
\pgfsetroundjoin%
\definecolor{currentfill}{rgb}{0.966560,0.756582,0.625273}%
\pgfsetfillcolor{currentfill}%
\pgfsetlinewidth{0.311001pt}%
\definecolor{currentstroke}{rgb}{1.000000,1.000000,1.000000}%
\pgfsetstrokecolor{currentstroke}%
\pgfsetdash{}{0pt}%
\pgfpathmoveto{\pgfqpoint{4.011541in}{1.676952in}}%
\pgfpathcurveto{\pgfqpoint{4.018674in}{1.676952in}}{\pgfqpoint{4.025516in}{1.679785in}}{\pgfqpoint{4.030559in}{1.684829in}}%
\pgfpathcurveto{\pgfqpoint{4.035603in}{1.689873in}}{\pgfqpoint{4.038437in}{1.696714in}}{\pgfqpoint{4.038437in}{1.703847in}}%
\pgfpathcurveto{\pgfqpoint{4.038437in}{1.710980in}}{\pgfqpoint{4.035603in}{1.717822in}}{\pgfqpoint{4.030559in}{1.722865in}}%
\pgfpathcurveto{\pgfqpoint{4.025516in}{1.727909in}}{\pgfqpoint{4.018674in}{1.730743in}}{\pgfqpoint{4.011541in}{1.730743in}}%
\pgfpathcurveto{\pgfqpoint{4.004408in}{1.730743in}}{\pgfqpoint{3.997567in}{1.727909in}}{\pgfqpoint{3.992523in}{1.722865in}}%
\pgfpathcurveto{\pgfqpoint{3.987479in}{1.717822in}}{\pgfqpoint{3.984645in}{1.710980in}}{\pgfqpoint{3.984645in}{1.703847in}}%
\pgfpathcurveto{\pgfqpoint{3.984645in}{1.696714in}}{\pgfqpoint{3.987479in}{1.689873in}}{\pgfqpoint{3.992523in}{1.684829in}}%
\pgfpathcurveto{\pgfqpoint{3.997567in}{1.679785in}}{\pgfqpoint{4.004408in}{1.676952in}}{\pgfqpoint{4.011541in}{1.676952in}}%
\pgfpathclose%
\pgfusepath{stroke,fill}%
\end{pgfscope}%
\begin{pgfscope}%
\pgfpathrectangle{\pgfqpoint{2.867647in}{0.500000in}}{\pgfqpoint{1.764706in}{1.700000in}}%
\pgfusepath{clip}%
\pgfsetbuttcap%
\pgfsetroundjoin%
\definecolor{currentfill}{rgb}{0.976961,0.885681,0.814303}%
\pgfsetfillcolor{currentfill}%
\pgfsetlinewidth{0.311001pt}%
\definecolor{currentstroke}{rgb}{1.000000,1.000000,1.000000}%
\pgfsetstrokecolor{currentstroke}%
\pgfsetdash{}{0pt}%
\pgfpathmoveto{\pgfqpoint{4.142497in}{1.394908in}}%
\pgfpathcurveto{\pgfqpoint{4.149630in}{1.394908in}}{\pgfqpoint{4.156472in}{1.397742in}}{\pgfqpoint{4.161515in}{1.402786in}}%
\pgfpathcurveto{\pgfqpoint{4.166559in}{1.407829in}}{\pgfqpoint{4.169393in}{1.414671in}}{\pgfqpoint{4.169393in}{1.421804in}}%
\pgfpathcurveto{\pgfqpoint{4.169393in}{1.428937in}}{\pgfqpoint{4.166559in}{1.435778in}}{\pgfqpoint{4.161515in}{1.440822in}}%
\pgfpathcurveto{\pgfqpoint{4.156472in}{1.445866in}}{\pgfqpoint{4.149630in}{1.448699in}}{\pgfqpoint{4.142497in}{1.448699in}}%
\pgfpathcurveto{\pgfqpoint{4.135364in}{1.448699in}}{\pgfqpoint{4.128523in}{1.445866in}}{\pgfqpoint{4.123479in}{1.440822in}}%
\pgfpathcurveto{\pgfqpoint{4.118435in}{1.435778in}}{\pgfqpoint{4.115602in}{1.428937in}}{\pgfqpoint{4.115602in}{1.421804in}}%
\pgfpathcurveto{\pgfqpoint{4.115602in}{1.414671in}}{\pgfqpoint{4.118435in}{1.407829in}}{\pgfqpoint{4.123479in}{1.402786in}}%
\pgfpathcurveto{\pgfqpoint{4.128523in}{1.397742in}}{\pgfqpoint{4.135364in}{1.394908in}}{\pgfqpoint{4.142497in}{1.394908in}}%
\pgfpathclose%
\pgfusepath{stroke,fill}%
\end{pgfscope}%
\begin{pgfscope}%
\pgfpathrectangle{\pgfqpoint{2.867647in}{0.500000in}}{\pgfqpoint{1.764706in}{1.700000in}}%
\pgfusepath{clip}%
\pgfsetbuttcap%
\pgfsetroundjoin%
\definecolor{currentfill}{rgb}{0.964679,0.682838,0.530002}%
\pgfsetfillcolor{currentfill}%
\pgfsetlinewidth{0.311001pt}%
\definecolor{currentstroke}{rgb}{1.000000,1.000000,1.000000}%
\pgfsetstrokecolor{currentstroke}%
\pgfsetdash{}{0pt}%
\pgfpathmoveto{\pgfqpoint{4.101604in}{1.763718in}}%
\pgfpathcurveto{\pgfqpoint{4.108736in}{1.763718in}}{\pgfqpoint{4.115578in}{1.766552in}}{\pgfqpoint{4.120622in}{1.771595in}}%
\pgfpathcurveto{\pgfqpoint{4.125665in}{1.776639in}}{\pgfqpoint{4.128499in}{1.783481in}}{\pgfqpoint{4.128499in}{1.790613in}}%
\pgfpathcurveto{\pgfqpoint{4.128499in}{1.797746in}}{\pgfqpoint{4.125665in}{1.804588in}}{\pgfqpoint{4.120622in}{1.809632in}}%
\pgfpathcurveto{\pgfqpoint{4.115578in}{1.814675in}}{\pgfqpoint{4.108736in}{1.817509in}}{\pgfqpoint{4.101604in}{1.817509in}}%
\pgfpathcurveto{\pgfqpoint{4.094471in}{1.817509in}}{\pgfqpoint{4.087629in}{1.814675in}}{\pgfqpoint{4.082585in}{1.809632in}}%
\pgfpathcurveto{\pgfqpoint{4.077542in}{1.804588in}}{\pgfqpoint{4.074708in}{1.797746in}}{\pgfqpoint{4.074708in}{1.790613in}}%
\pgfpathcurveto{\pgfqpoint{4.074708in}{1.783481in}}{\pgfqpoint{4.077542in}{1.776639in}}{\pgfqpoint{4.082585in}{1.771595in}}%
\pgfpathcurveto{\pgfqpoint{4.087629in}{1.766552in}}{\pgfqpoint{4.094471in}{1.763718in}}{\pgfqpoint{4.101604in}{1.763718in}}%
\pgfpathclose%
\pgfusepath{stroke,fill}%
\end{pgfscope}%
\begin{pgfscope}%
\pgfpathrectangle{\pgfqpoint{2.867647in}{0.500000in}}{\pgfqpoint{1.764706in}{1.700000in}}%
\pgfusepath{clip}%
\pgfsetbuttcap%
\pgfsetroundjoin%
\definecolor{currentfill}{rgb}{0.964679,0.682838,0.530002}%
\pgfsetfillcolor{currentfill}%
\pgfsetlinewidth{0.311001pt}%
\definecolor{currentstroke}{rgb}{1.000000,1.000000,1.000000}%
\pgfsetstrokecolor{currentstroke}%
\pgfsetdash{}{0pt}%
\pgfpathmoveto{\pgfqpoint{3.981742in}{1.645654in}}%
\pgfpathcurveto{\pgfqpoint{3.988875in}{1.645654in}}{\pgfqpoint{3.995717in}{1.648488in}}{\pgfqpoint{4.000761in}{1.653531in}}%
\pgfpathcurveto{\pgfqpoint{4.005804in}{1.658575in}}{\pgfqpoint{4.008638in}{1.665417in}}{\pgfqpoint{4.008638in}{1.672549in}}%
\pgfpathcurveto{\pgfqpoint{4.008638in}{1.679682in}}{\pgfqpoint{4.005804in}{1.686524in}}{\pgfqpoint{4.000761in}{1.691568in}}%
\pgfpathcurveto{\pgfqpoint{3.995717in}{1.696611in}}{\pgfqpoint{3.988875in}{1.699445in}}{\pgfqpoint{3.981742in}{1.699445in}}%
\pgfpathcurveto{\pgfqpoint{3.974610in}{1.699445in}}{\pgfqpoint{3.967768in}{1.696611in}}{\pgfqpoint{3.962724in}{1.691568in}}%
\pgfpathcurveto{\pgfqpoint{3.957681in}{1.686524in}}{\pgfqpoint{3.954847in}{1.679682in}}{\pgfqpoint{3.954847in}{1.672549in}}%
\pgfpathcurveto{\pgfqpoint{3.954847in}{1.665417in}}{\pgfqpoint{3.957681in}{1.658575in}}{\pgfqpoint{3.962724in}{1.653531in}}%
\pgfpathcurveto{\pgfqpoint{3.967768in}{1.648488in}}{\pgfqpoint{3.974610in}{1.645654in}}{\pgfqpoint{3.981742in}{1.645654in}}%
\pgfpathclose%
\pgfusepath{stroke,fill}%
\end{pgfscope}%
\begin{pgfscope}%
\pgfpathrectangle{\pgfqpoint{2.867647in}{0.500000in}}{\pgfqpoint{1.764706in}{1.700000in}}%
\pgfusepath{clip}%
\pgfsetbuttcap%
\pgfsetroundjoin%
\definecolor{currentfill}{rgb}{0.779326,0.096348,0.318766}%
\pgfsetfillcolor{currentfill}%
\pgfsetlinewidth{0.311001pt}%
\definecolor{currentstroke}{rgb}{1.000000,1.000000,1.000000}%
\pgfsetstrokecolor{currentstroke}%
\pgfsetdash{}{0pt}%
\pgfpathmoveto{\pgfqpoint{3.742199in}{1.833808in}}%
\pgfpathcurveto{\pgfqpoint{3.749332in}{1.833808in}}{\pgfqpoint{3.756174in}{1.836642in}}{\pgfqpoint{3.761217in}{1.841685in}}%
\pgfpathcurveto{\pgfqpoint{3.766261in}{1.846729in}}{\pgfqpoint{3.769095in}{1.853571in}}{\pgfqpoint{3.769095in}{1.860703in}}%
\pgfpathcurveto{\pgfqpoint{3.769095in}{1.867836in}}{\pgfqpoint{3.766261in}{1.874678in}}{\pgfqpoint{3.761217in}{1.879721in}}%
\pgfpathcurveto{\pgfqpoint{3.756174in}{1.884765in}}{\pgfqpoint{3.749332in}{1.887599in}}{\pgfqpoint{3.742199in}{1.887599in}}%
\pgfpathcurveto{\pgfqpoint{3.735066in}{1.887599in}}{\pgfqpoint{3.728225in}{1.884765in}}{\pgfqpoint{3.723181in}{1.879721in}}%
\pgfpathcurveto{\pgfqpoint{3.718137in}{1.874678in}}{\pgfqpoint{3.715303in}{1.867836in}}{\pgfqpoint{3.715303in}{1.860703in}}%
\pgfpathcurveto{\pgfqpoint{3.715303in}{1.853571in}}{\pgfqpoint{3.718137in}{1.846729in}}{\pgfqpoint{3.723181in}{1.841685in}}%
\pgfpathcurveto{\pgfqpoint{3.728225in}{1.836642in}}{\pgfqpoint{3.735066in}{1.833808in}}{\pgfqpoint{3.742199in}{1.833808in}}%
\pgfpathclose%
\pgfusepath{stroke,fill}%
\end{pgfscope}%
\begin{pgfscope}%
\pgfpathrectangle{\pgfqpoint{2.867647in}{0.500000in}}{\pgfqpoint{1.764706in}{1.700000in}}%
\pgfusepath{clip}%
\pgfsetbuttcap%
\pgfsetroundjoin%
\definecolor{currentfill}{rgb}{0.971694,0.833208,0.737161}%
\pgfsetfillcolor{currentfill}%
\pgfsetlinewidth{0.311001pt}%
\definecolor{currentstroke}{rgb}{1.000000,1.000000,1.000000}%
\pgfsetstrokecolor{currentstroke}%
\pgfsetdash{}{0pt}%
\pgfpathmoveto{\pgfqpoint{4.087564in}{0.968894in}}%
\pgfpathcurveto{\pgfqpoint{4.094697in}{0.968894in}}{\pgfqpoint{4.101539in}{0.971728in}}{\pgfqpoint{4.106583in}{0.976772in}}%
\pgfpathcurveto{\pgfqpoint{4.111626in}{0.981816in}}{\pgfqpoint{4.114460in}{0.988657in}}{\pgfqpoint{4.114460in}{0.995790in}}%
\pgfpathcurveto{\pgfqpoint{4.114460in}{1.002923in}}{\pgfqpoint{4.111626in}{1.009765in}}{\pgfqpoint{4.106583in}{1.014808in}}%
\pgfpathcurveto{\pgfqpoint{4.101539in}{1.019852in}}{\pgfqpoint{4.094697in}{1.022686in}}{\pgfqpoint{4.087564in}{1.022686in}}%
\pgfpathcurveto{\pgfqpoint{4.080432in}{1.022686in}}{\pgfqpoint{4.073590in}{1.019852in}}{\pgfqpoint{4.068546in}{1.014808in}}%
\pgfpathcurveto{\pgfqpoint{4.063503in}{1.009765in}}{\pgfqpoint{4.060669in}{1.002923in}}{\pgfqpoint{4.060669in}{0.995790in}}%
\pgfpathcurveto{\pgfqpoint{4.060669in}{0.988657in}}{\pgfqpoint{4.063503in}{0.981816in}}{\pgfqpoint{4.068546in}{0.976772in}}%
\pgfpathcurveto{\pgfqpoint{4.073590in}{0.971728in}}{\pgfqpoint{4.080432in}{0.968894in}}{\pgfqpoint{4.087564in}{0.968894in}}%
\pgfpathclose%
\pgfusepath{stroke,fill}%
\end{pgfscope}%
\begin{pgfscope}%
\pgfpathrectangle{\pgfqpoint{2.867647in}{0.500000in}}{\pgfqpoint{1.764706in}{1.700000in}}%
\pgfusepath{clip}%
\pgfsetbuttcap%
\pgfsetroundjoin%
\definecolor{currentfill}{rgb}{0.961734,0.579886,0.418445}%
\pgfsetfillcolor{currentfill}%
\pgfsetlinewidth{0.311001pt}%
\definecolor{currentstroke}{rgb}{1.000000,1.000000,1.000000}%
\pgfsetstrokecolor{currentstroke}%
\pgfsetdash{}{0pt}%
\pgfpathmoveto{\pgfqpoint{4.240546in}{0.973709in}}%
\pgfpathcurveto{\pgfqpoint{4.247679in}{0.973709in}}{\pgfqpoint{4.254520in}{0.976543in}}{\pgfqpoint{4.259564in}{0.981586in}}%
\pgfpathcurveto{\pgfqpoint{4.264608in}{0.986630in}}{\pgfqpoint{4.267442in}{0.993472in}}{\pgfqpoint{4.267442in}{1.000605in}}%
\pgfpathcurveto{\pgfqpoint{4.267442in}{1.007737in}}{\pgfqpoint{4.264608in}{1.014579in}}{\pgfqpoint{4.259564in}{1.019623in}}%
\pgfpathcurveto{\pgfqpoint{4.254520in}{1.024666in}}{\pgfqpoint{4.247679in}{1.027500in}}{\pgfqpoint{4.240546in}{1.027500in}}%
\pgfpathcurveto{\pgfqpoint{4.233413in}{1.027500in}}{\pgfqpoint{4.226571in}{1.024666in}}{\pgfqpoint{4.221528in}{1.019623in}}%
\pgfpathcurveto{\pgfqpoint{4.216484in}{1.014579in}}{\pgfqpoint{4.213650in}{1.007737in}}{\pgfqpoint{4.213650in}{1.000605in}}%
\pgfpathcurveto{\pgfqpoint{4.213650in}{0.993472in}}{\pgfqpoint{4.216484in}{0.986630in}}{\pgfqpoint{4.221528in}{0.981586in}}%
\pgfpathcurveto{\pgfqpoint{4.226571in}{0.976543in}}{\pgfqpoint{4.233413in}{0.973709in}}{\pgfqpoint{4.240546in}{0.973709in}}%
\pgfpathclose%
\pgfusepath{stroke,fill}%
\end{pgfscope}%
\begin{pgfscope}%
\pgfpathrectangle{\pgfqpoint{2.867647in}{0.500000in}}{\pgfqpoint{1.764706in}{1.700000in}}%
\pgfusepath{clip}%
\pgfsetbuttcap%
\pgfsetroundjoin%
\definecolor{currentfill}{rgb}{0.972726,0.844889,0.754401}%
\pgfsetfillcolor{currentfill}%
\pgfsetlinewidth{0.311001pt}%
\definecolor{currentstroke}{rgb}{1.000000,1.000000,1.000000}%
\pgfsetstrokecolor{currentstroke}%
\pgfsetdash{}{0pt}%
\pgfpathmoveto{\pgfqpoint{4.066199in}{1.033560in}}%
\pgfpathcurveto{\pgfqpoint{4.073332in}{1.033560in}}{\pgfqpoint{4.080173in}{1.036394in}}{\pgfqpoint{4.085217in}{1.041437in}}%
\pgfpathcurveto{\pgfqpoint{4.090261in}{1.046481in}}{\pgfqpoint{4.093094in}{1.053323in}}{\pgfqpoint{4.093094in}{1.060456in}}%
\pgfpathcurveto{\pgfqpoint{4.093094in}{1.067588in}}{\pgfqpoint{4.090261in}{1.074430in}}{\pgfqpoint{4.085217in}{1.079474in}}%
\pgfpathcurveto{\pgfqpoint{4.080173in}{1.084517in}}{\pgfqpoint{4.073332in}{1.087351in}}{\pgfqpoint{4.066199in}{1.087351in}}%
\pgfpathcurveto{\pgfqpoint{4.059066in}{1.087351in}}{\pgfqpoint{4.052224in}{1.084517in}}{\pgfqpoint{4.047181in}{1.079474in}}%
\pgfpathcurveto{\pgfqpoint{4.042137in}{1.074430in}}{\pgfqpoint{4.039303in}{1.067588in}}{\pgfqpoint{4.039303in}{1.060456in}}%
\pgfpathcurveto{\pgfqpoint{4.039303in}{1.053323in}}{\pgfqpoint{4.042137in}{1.046481in}}{\pgfqpoint{4.047181in}{1.041437in}}%
\pgfpathcurveto{\pgfqpoint{4.052224in}{1.036394in}}{\pgfqpoint{4.059066in}{1.033560in}}{\pgfqpoint{4.066199in}{1.033560in}}%
\pgfpathclose%
\pgfusepath{stroke,fill}%
\end{pgfscope}%
\begin{pgfscope}%
\pgfpathrectangle{\pgfqpoint{2.867647in}{0.500000in}}{\pgfqpoint{1.764706in}{1.700000in}}%
\pgfusepath{clip}%
\pgfsetbuttcap%
\pgfsetroundjoin%
\definecolor{currentfill}{rgb}{0.978376,0.897317,0.831308}%
\pgfsetfillcolor{currentfill}%
\pgfsetlinewidth{0.311001pt}%
\definecolor{currentstroke}{rgb}{1.000000,1.000000,1.000000}%
\pgfsetstrokecolor{currentstroke}%
\pgfsetdash{}{0pt}%
\pgfpathmoveto{\pgfqpoint{4.221589in}{1.211142in}}%
\pgfpathcurveto{\pgfqpoint{4.228722in}{1.211142in}}{\pgfqpoint{4.235564in}{1.213976in}}{\pgfqpoint{4.240607in}{1.219019in}}%
\pgfpathcurveto{\pgfqpoint{4.245651in}{1.224063in}}{\pgfqpoint{4.248485in}{1.230905in}}{\pgfqpoint{4.248485in}{1.238038in}}%
\pgfpathcurveto{\pgfqpoint{4.248485in}{1.245170in}}{\pgfqpoint{4.245651in}{1.252012in}}{\pgfqpoint{4.240607in}{1.257056in}}%
\pgfpathcurveto{\pgfqpoint{4.235564in}{1.262099in}}{\pgfqpoint{4.228722in}{1.264933in}}{\pgfqpoint{4.221589in}{1.264933in}}%
\pgfpathcurveto{\pgfqpoint{4.214456in}{1.264933in}}{\pgfqpoint{4.207615in}{1.262099in}}{\pgfqpoint{4.202571in}{1.257056in}}%
\pgfpathcurveto{\pgfqpoint{4.197527in}{1.252012in}}{\pgfqpoint{4.194693in}{1.245170in}}{\pgfqpoint{4.194693in}{1.238038in}}%
\pgfpathcurveto{\pgfqpoint{4.194693in}{1.230905in}}{\pgfqpoint{4.197527in}{1.224063in}}{\pgfqpoint{4.202571in}{1.219019in}}%
\pgfpathcurveto{\pgfqpoint{4.207615in}{1.213976in}}{\pgfqpoint{4.214456in}{1.211142in}}{\pgfqpoint{4.221589in}{1.211142in}}%
\pgfpathclose%
\pgfusepath{stroke,fill}%
\end{pgfscope}%
\begin{pgfscope}%
\pgfpathrectangle{\pgfqpoint{2.867647in}{0.500000in}}{\pgfqpoint{1.764706in}{1.700000in}}%
\pgfusepath{clip}%
\pgfsetbuttcap%
\pgfsetroundjoin%
\definecolor{currentfill}{rgb}{0.964433,0.670254,0.515093}%
\pgfsetfillcolor{currentfill}%
\pgfsetlinewidth{0.311001pt}%
\definecolor{currentstroke}{rgb}{1.000000,1.000000,1.000000}%
\pgfsetstrokecolor{currentstroke}%
\pgfsetdash{}{0pt}%
\pgfpathmoveto{\pgfqpoint{4.312910in}{1.394270in}}%
\pgfpathcurveto{\pgfqpoint{4.320043in}{1.394270in}}{\pgfqpoint{4.326885in}{1.397104in}}{\pgfqpoint{4.331929in}{1.402148in}}%
\pgfpathcurveto{\pgfqpoint{4.336972in}{1.407192in}}{\pgfqpoint{4.339806in}{1.414033in}}{\pgfqpoint{4.339806in}{1.421166in}}%
\pgfpathcurveto{\pgfqpoint{4.339806in}{1.428299in}}{\pgfqpoint{4.336972in}{1.435141in}}{\pgfqpoint{4.331929in}{1.440184in}}%
\pgfpathcurveto{\pgfqpoint{4.326885in}{1.445228in}}{\pgfqpoint{4.320043in}{1.448062in}}{\pgfqpoint{4.312910in}{1.448062in}}%
\pgfpathcurveto{\pgfqpoint{4.305778in}{1.448062in}}{\pgfqpoint{4.298936in}{1.445228in}}{\pgfqpoint{4.293892in}{1.440184in}}%
\pgfpathcurveto{\pgfqpoint{4.288849in}{1.435141in}}{\pgfqpoint{4.286015in}{1.428299in}}{\pgfqpoint{4.286015in}{1.421166in}}%
\pgfpathcurveto{\pgfqpoint{4.286015in}{1.414033in}}{\pgfqpoint{4.288849in}{1.407192in}}{\pgfqpoint{4.293892in}{1.402148in}}%
\pgfpathcurveto{\pgfqpoint{4.298936in}{1.397104in}}{\pgfqpoint{4.305778in}{1.394270in}}{\pgfqpoint{4.312910in}{1.394270in}}%
\pgfpathclose%
\pgfusepath{stroke,fill}%
\end{pgfscope}%
\begin{pgfscope}%
\pgfpathrectangle{\pgfqpoint{2.867647in}{0.500000in}}{\pgfqpoint{1.764706in}{1.700000in}}%
\pgfusepath{clip}%
\pgfsetbuttcap%
\pgfsetroundjoin%
\definecolor{currentfill}{rgb}{0.967735,0.780441,0.659127}%
\pgfsetfillcolor{currentfill}%
\pgfsetlinewidth{0.311001pt}%
\definecolor{currentstroke}{rgb}{1.000000,1.000000,1.000000}%
\pgfsetstrokecolor{currentstroke}%
\pgfsetdash{}{0pt}%
\pgfpathmoveto{\pgfqpoint{4.274905in}{1.438525in}}%
\pgfpathcurveto{\pgfqpoint{4.282038in}{1.438525in}}{\pgfqpoint{4.288879in}{1.441359in}}{\pgfqpoint{4.293923in}{1.446403in}}%
\pgfpathcurveto{\pgfqpoint{4.298967in}{1.451446in}}{\pgfqpoint{4.301801in}{1.458288in}}{\pgfqpoint{4.301801in}{1.465421in}}%
\pgfpathcurveto{\pgfqpoint{4.301801in}{1.472554in}}{\pgfqpoint{4.298967in}{1.479395in}}{\pgfqpoint{4.293923in}{1.484439in}}%
\pgfpathcurveto{\pgfqpoint{4.288879in}{1.489482in}}{\pgfqpoint{4.282038in}{1.492316in}}{\pgfqpoint{4.274905in}{1.492316in}}%
\pgfpathcurveto{\pgfqpoint{4.267772in}{1.492316in}}{\pgfqpoint{4.260931in}{1.489482in}}{\pgfqpoint{4.255887in}{1.484439in}}%
\pgfpathcurveto{\pgfqpoint{4.250843in}{1.479395in}}{\pgfqpoint{4.248009in}{1.472554in}}{\pgfqpoint{4.248009in}{1.465421in}}%
\pgfpathcurveto{\pgfqpoint{4.248009in}{1.458288in}}{\pgfqpoint{4.250843in}{1.451446in}}{\pgfqpoint{4.255887in}{1.446403in}}%
\pgfpathcurveto{\pgfqpoint{4.260931in}{1.441359in}}{\pgfqpoint{4.267772in}{1.438525in}}{\pgfqpoint{4.274905in}{1.438525in}}%
\pgfpathclose%
\pgfusepath{stroke,fill}%
\end{pgfscope}%
\begin{pgfscope}%
\pgfpathrectangle{\pgfqpoint{2.867647in}{0.500000in}}{\pgfqpoint{1.764706in}{1.700000in}}%
\pgfusepath{clip}%
\pgfsetbuttcap%
\pgfsetroundjoin%
\definecolor{currentfill}{rgb}{0.980678,0.914765,0.856766}%
\pgfsetfillcolor{currentfill}%
\pgfsetlinewidth{0.311001pt}%
\definecolor{currentstroke}{rgb}{1.000000,1.000000,1.000000}%
\pgfsetstrokecolor{currentstroke}%
\pgfsetdash{}{0pt}%
\pgfpathmoveto{\pgfqpoint{4.149229in}{1.168841in}}%
\pgfpathcurveto{\pgfqpoint{4.156362in}{1.168841in}}{\pgfqpoint{4.163203in}{1.171675in}}{\pgfqpoint{4.168247in}{1.176718in}}%
\pgfpathcurveto{\pgfqpoint{4.173291in}{1.181762in}}{\pgfqpoint{4.176125in}{1.188604in}}{\pgfqpoint{4.176125in}{1.195737in}}%
\pgfpathcurveto{\pgfqpoint{4.176125in}{1.202869in}}{\pgfqpoint{4.173291in}{1.209711in}}{\pgfqpoint{4.168247in}{1.214755in}}%
\pgfpathcurveto{\pgfqpoint{4.163203in}{1.219798in}}{\pgfqpoint{4.156362in}{1.222632in}}{\pgfqpoint{4.149229in}{1.222632in}}%
\pgfpathcurveto{\pgfqpoint{4.142096in}{1.222632in}}{\pgfqpoint{4.135254in}{1.219798in}}{\pgfqpoint{4.130211in}{1.214755in}}%
\pgfpathcurveto{\pgfqpoint{4.125167in}{1.209711in}}{\pgfqpoint{4.122333in}{1.202869in}}{\pgfqpoint{4.122333in}{1.195737in}}%
\pgfpathcurveto{\pgfqpoint{4.122333in}{1.188604in}}{\pgfqpoint{4.125167in}{1.181762in}}{\pgfqpoint{4.130211in}{1.176718in}}%
\pgfpathcurveto{\pgfqpoint{4.135254in}{1.171675in}}{\pgfqpoint{4.142096in}{1.168841in}}{\pgfqpoint{4.149229in}{1.168841in}}%
\pgfpathclose%
\pgfusepath{stroke,fill}%
\end{pgfscope}%
\begin{pgfscope}%
\pgfpathrectangle{\pgfqpoint{2.867647in}{0.500000in}}{\pgfqpoint{1.764706in}{1.700000in}}%
\pgfusepath{clip}%
\pgfsetbuttcap%
\pgfsetroundjoin%
\definecolor{currentfill}{rgb}{0.973832,0.856556,0.771584}%
\pgfsetfillcolor{currentfill}%
\pgfsetlinewidth{0.311001pt}%
\definecolor{currentstroke}{rgb}{1.000000,1.000000,1.000000}%
\pgfsetstrokecolor{currentstroke}%
\pgfsetdash{}{0pt}%
\pgfpathmoveto{\pgfqpoint{4.179508in}{1.603553in}}%
\pgfpathcurveto{\pgfqpoint{4.186641in}{1.603553in}}{\pgfqpoint{4.193483in}{1.606387in}}{\pgfqpoint{4.198526in}{1.611431in}}%
\pgfpathcurveto{\pgfqpoint{4.203570in}{1.616475in}}{\pgfqpoint{4.206404in}{1.623316in}}{\pgfqpoint{4.206404in}{1.630449in}}%
\pgfpathcurveto{\pgfqpoint{4.206404in}{1.637582in}}{\pgfqpoint{4.203570in}{1.644424in}}{\pgfqpoint{4.198526in}{1.649467in}}%
\pgfpathcurveto{\pgfqpoint{4.193483in}{1.654511in}}{\pgfqpoint{4.186641in}{1.657345in}}{\pgfqpoint{4.179508in}{1.657345in}}%
\pgfpathcurveto{\pgfqpoint{4.172375in}{1.657345in}}{\pgfqpoint{4.165534in}{1.654511in}}{\pgfqpoint{4.160490in}{1.649467in}}%
\pgfpathcurveto{\pgfqpoint{4.155446in}{1.644424in}}{\pgfqpoint{4.152612in}{1.637582in}}{\pgfqpoint{4.152612in}{1.630449in}}%
\pgfpathcurveto{\pgfqpoint{4.152612in}{1.623316in}}{\pgfqpoint{4.155446in}{1.616475in}}{\pgfqpoint{4.160490in}{1.611431in}}%
\pgfpathcurveto{\pgfqpoint{4.165534in}{1.606387in}}{\pgfqpoint{4.172375in}{1.603553in}}{\pgfqpoint{4.179508in}{1.603553in}}%
\pgfpathclose%
\pgfusepath{stroke,fill}%
\end{pgfscope}%
\begin{pgfscope}%
\pgfpathrectangle{\pgfqpoint{2.867647in}{0.500000in}}{\pgfqpoint{1.764706in}{1.700000in}}%
\pgfusepath{clip}%
\pgfsetbuttcap%
\pgfsetroundjoin%
\definecolor{currentfill}{rgb}{0.964679,0.682838,0.530002}%
\pgfsetfillcolor{currentfill}%
\pgfsetlinewidth{0.311001pt}%
\definecolor{currentstroke}{rgb}{1.000000,1.000000,1.000000}%
\pgfsetstrokecolor{currentstroke}%
\pgfsetdash{}{0pt}%
\pgfpathmoveto{\pgfqpoint{4.126404in}{0.899171in}}%
\pgfpathcurveto{\pgfqpoint{4.133537in}{0.899171in}}{\pgfqpoint{4.140378in}{0.902005in}}{\pgfqpoint{4.145422in}{0.907049in}}%
\pgfpathcurveto{\pgfqpoint{4.150466in}{0.912092in}}{\pgfqpoint{4.153299in}{0.918934in}}{\pgfqpoint{4.153299in}{0.926067in}}%
\pgfpathcurveto{\pgfqpoint{4.153299in}{0.933200in}}{\pgfqpoint{4.150466in}{0.940041in}}{\pgfqpoint{4.145422in}{0.945085in}}%
\pgfpathcurveto{\pgfqpoint{4.140378in}{0.950129in}}{\pgfqpoint{4.133537in}{0.952962in}}{\pgfqpoint{4.126404in}{0.952962in}}%
\pgfpathcurveto{\pgfqpoint{4.119271in}{0.952962in}}{\pgfqpoint{4.112429in}{0.950129in}}{\pgfqpoint{4.107386in}{0.945085in}}%
\pgfpathcurveto{\pgfqpoint{4.102342in}{0.940041in}}{\pgfqpoint{4.099508in}{0.933200in}}{\pgfqpoint{4.099508in}{0.926067in}}%
\pgfpathcurveto{\pgfqpoint{4.099508in}{0.918934in}}{\pgfqpoint{4.102342in}{0.912092in}}{\pgfqpoint{4.107386in}{0.907049in}}%
\pgfpathcurveto{\pgfqpoint{4.112429in}{0.902005in}}{\pgfqpoint{4.119271in}{0.899171in}}{\pgfqpoint{4.126404in}{0.899171in}}%
\pgfpathclose%
\pgfusepath{stroke,fill}%
\end{pgfscope}%
\begin{pgfscope}%
\pgfpathrectangle{\pgfqpoint{2.867647in}{0.500000in}}{\pgfqpoint{1.764706in}{1.700000in}}%
\pgfusepath{clip}%
\pgfsetbuttcap%
\pgfsetroundjoin%
\definecolor{currentfill}{rgb}{0.979124,0.903132,0.839793}%
\pgfsetfillcolor{currentfill}%
\pgfsetlinewidth{0.311001pt}%
\definecolor{currentstroke}{rgb}{1.000000,1.000000,1.000000}%
\pgfsetstrokecolor{currentstroke}%
\pgfsetdash{}{0pt}%
\pgfpathmoveto{\pgfqpoint{4.209891in}{1.185966in}}%
\pgfpathcurveto{\pgfqpoint{4.217024in}{1.185966in}}{\pgfqpoint{4.223866in}{1.188800in}}{\pgfqpoint{4.228909in}{1.193844in}}%
\pgfpathcurveto{\pgfqpoint{4.233953in}{1.198887in}}{\pgfqpoint{4.236787in}{1.205729in}}{\pgfqpoint{4.236787in}{1.212862in}}%
\pgfpathcurveto{\pgfqpoint{4.236787in}{1.219995in}}{\pgfqpoint{4.233953in}{1.226836in}}{\pgfqpoint{4.228909in}{1.231880in}}%
\pgfpathcurveto{\pgfqpoint{4.223866in}{1.236924in}}{\pgfqpoint{4.217024in}{1.239758in}}{\pgfqpoint{4.209891in}{1.239758in}}%
\pgfpathcurveto{\pgfqpoint{4.202759in}{1.239758in}}{\pgfqpoint{4.195917in}{1.236924in}}{\pgfqpoint{4.190873in}{1.231880in}}%
\pgfpathcurveto{\pgfqpoint{4.185830in}{1.226836in}}{\pgfqpoint{4.182996in}{1.219995in}}{\pgfqpoint{4.182996in}{1.212862in}}%
\pgfpathcurveto{\pgfqpoint{4.182996in}{1.205729in}}{\pgfqpoint{4.185830in}{1.198887in}}{\pgfqpoint{4.190873in}{1.193844in}}%
\pgfpathcurveto{\pgfqpoint{4.195917in}{1.188800in}}{\pgfqpoint{4.202759in}{1.185966in}}{\pgfqpoint{4.209891in}{1.185966in}}%
\pgfpathclose%
\pgfusepath{stroke,fill}%
\end{pgfscope}%
\begin{pgfscope}%
\pgfpathrectangle{\pgfqpoint{2.867647in}{0.500000in}}{\pgfqpoint{1.764706in}{1.700000in}}%
\pgfusepath{clip}%
\pgfsetbuttcap%
\pgfsetroundjoin%
\definecolor{currentfill}{rgb}{0.973832,0.856556,0.771584}%
\pgfsetfillcolor{currentfill}%
\pgfsetlinewidth{0.311001pt}%
\definecolor{currentstroke}{rgb}{1.000000,1.000000,1.000000}%
\pgfsetstrokecolor{currentstroke}%
\pgfsetdash{}{0pt}%
\pgfpathmoveto{\pgfqpoint{4.225697in}{1.507508in}}%
\pgfpathcurveto{\pgfqpoint{4.232830in}{1.507508in}}{\pgfqpoint{4.239671in}{1.510342in}}{\pgfqpoint{4.244715in}{1.515386in}}%
\pgfpathcurveto{\pgfqpoint{4.249759in}{1.520430in}}{\pgfqpoint{4.252593in}{1.527271in}}{\pgfqpoint{4.252593in}{1.534404in}}%
\pgfpathcurveto{\pgfqpoint{4.252593in}{1.541537in}}{\pgfqpoint{4.249759in}{1.548379in}}{\pgfqpoint{4.244715in}{1.553422in}}%
\pgfpathcurveto{\pgfqpoint{4.239671in}{1.558466in}}{\pgfqpoint{4.232830in}{1.561300in}}{\pgfqpoint{4.225697in}{1.561300in}}%
\pgfpathcurveto{\pgfqpoint{4.218564in}{1.561300in}}{\pgfqpoint{4.211722in}{1.558466in}}{\pgfqpoint{4.206679in}{1.553422in}}%
\pgfpathcurveto{\pgfqpoint{4.201635in}{1.548379in}}{\pgfqpoint{4.198801in}{1.541537in}}{\pgfqpoint{4.198801in}{1.534404in}}%
\pgfpathcurveto{\pgfqpoint{4.198801in}{1.527271in}}{\pgfqpoint{4.201635in}{1.520430in}}{\pgfqpoint{4.206679in}{1.515386in}}%
\pgfpathcurveto{\pgfqpoint{4.211722in}{1.510342in}}{\pgfqpoint{4.218564in}{1.507508in}}{\pgfqpoint{4.225697in}{1.507508in}}%
\pgfpathclose%
\pgfusepath{stroke,fill}%
\end{pgfscope}%
\begin{pgfscope}%
\pgfpathrectangle{\pgfqpoint{2.867647in}{0.500000in}}{\pgfqpoint{1.764706in}{1.700000in}}%
\pgfusepath{clip}%
\pgfsetbuttcap%
\pgfsetroundjoin%
\definecolor{currentfill}{rgb}{0.977657,0.891500,0.822809}%
\pgfsetfillcolor{currentfill}%
\pgfsetlinewidth{0.311001pt}%
\definecolor{currentstroke}{rgb}{1.000000,1.000000,1.000000}%
\pgfsetstrokecolor{currentstroke}%
\pgfsetdash{}{0pt}%
\pgfpathmoveto{\pgfqpoint{4.222120in}{1.428624in}}%
\pgfpathcurveto{\pgfqpoint{4.229253in}{1.428624in}}{\pgfqpoint{4.236094in}{1.431458in}}{\pgfqpoint{4.241138in}{1.436501in}}%
\pgfpathcurveto{\pgfqpoint{4.246182in}{1.441545in}}{\pgfqpoint{4.249016in}{1.448387in}}{\pgfqpoint{4.249016in}{1.455520in}}%
\pgfpathcurveto{\pgfqpoint{4.249016in}{1.462652in}}{\pgfqpoint{4.246182in}{1.469494in}}{\pgfqpoint{4.241138in}{1.474538in}}%
\pgfpathcurveto{\pgfqpoint{4.236094in}{1.479581in}}{\pgfqpoint{4.229253in}{1.482415in}}{\pgfqpoint{4.222120in}{1.482415in}}%
\pgfpathcurveto{\pgfqpoint{4.214987in}{1.482415in}}{\pgfqpoint{4.208145in}{1.479581in}}{\pgfqpoint{4.203102in}{1.474538in}}%
\pgfpathcurveto{\pgfqpoint{4.198058in}{1.469494in}}{\pgfqpoint{4.195224in}{1.462652in}}{\pgfqpoint{4.195224in}{1.455520in}}%
\pgfpathcurveto{\pgfqpoint{4.195224in}{1.448387in}}{\pgfqpoint{4.198058in}{1.441545in}}{\pgfqpoint{4.203102in}{1.436501in}}%
\pgfpathcurveto{\pgfqpoint{4.208145in}{1.431458in}}{\pgfqpoint{4.214987in}{1.428624in}}{\pgfqpoint{4.222120in}{1.428624in}}%
\pgfpathclose%
\pgfusepath{stroke,fill}%
\end{pgfscope}%
\begin{pgfscope}%
\pgfpathrectangle{\pgfqpoint{2.867647in}{0.500000in}}{\pgfqpoint{1.764706in}{1.700000in}}%
\pgfusepath{clip}%
\pgfsetbuttcap%
\pgfsetroundjoin%
\definecolor{currentfill}{rgb}{0.981377,0.920617,0.865369}%
\pgfsetfillcolor{currentfill}%
\pgfsetlinewidth{0.311001pt}%
\definecolor{currentstroke}{rgb}{1.000000,1.000000,1.000000}%
\pgfsetstrokecolor{currentstroke}%
\pgfsetdash{}{0pt}%
\pgfpathmoveto{\pgfqpoint{4.206263in}{1.296539in}}%
\pgfpathcurveto{\pgfqpoint{4.213396in}{1.296539in}}{\pgfqpoint{4.220237in}{1.299373in}}{\pgfqpoint{4.225281in}{1.304417in}}%
\pgfpathcurveto{\pgfqpoint{4.230325in}{1.309460in}}{\pgfqpoint{4.233159in}{1.316302in}}{\pgfqpoint{4.233159in}{1.323435in}}%
\pgfpathcurveto{\pgfqpoint{4.233159in}{1.330568in}}{\pgfqpoint{4.230325in}{1.337409in}}{\pgfqpoint{4.225281in}{1.342453in}}%
\pgfpathcurveto{\pgfqpoint{4.220237in}{1.347497in}}{\pgfqpoint{4.213396in}{1.350331in}}{\pgfqpoint{4.206263in}{1.350331in}}%
\pgfpathcurveto{\pgfqpoint{4.199130in}{1.350331in}}{\pgfqpoint{4.192288in}{1.347497in}}{\pgfqpoint{4.187245in}{1.342453in}}%
\pgfpathcurveto{\pgfqpoint{4.182201in}{1.337409in}}{\pgfqpoint{4.179367in}{1.330568in}}{\pgfqpoint{4.179367in}{1.323435in}}%
\pgfpathcurveto{\pgfqpoint{4.179367in}{1.316302in}}{\pgfqpoint{4.182201in}{1.309460in}}{\pgfqpoint{4.187245in}{1.304417in}}%
\pgfpathcurveto{\pgfqpoint{4.192288in}{1.299373in}}{\pgfqpoint{4.199130in}{1.296539in}}{\pgfqpoint{4.206263in}{1.296539in}}%
\pgfpathclose%
\pgfusepath{stroke,fill}%
\end{pgfscope}%
\begin{pgfscope}%
\pgfpathrectangle{\pgfqpoint{2.867647in}{0.500000in}}{\pgfqpoint{1.764706in}{1.700000in}}%
\pgfusepath{clip}%
\pgfsetbuttcap%
\pgfsetroundjoin%
\definecolor{currentfill}{rgb}{0.961115,0.566634,0.405693}%
\pgfsetfillcolor{currentfill}%
\pgfsetlinewidth{0.311001pt}%
\definecolor{currentstroke}{rgb}{1.000000,1.000000,1.000000}%
\pgfsetstrokecolor{currentstroke}%
\pgfsetdash{}{0pt}%
\pgfpathmoveto{\pgfqpoint{3.974698in}{1.801224in}}%
\pgfpathcurveto{\pgfqpoint{3.981831in}{1.801224in}}{\pgfqpoint{3.988673in}{1.804058in}}{\pgfqpoint{3.993716in}{1.809102in}}%
\pgfpathcurveto{\pgfqpoint{3.998760in}{1.814145in}}{\pgfqpoint{4.001594in}{1.820987in}}{\pgfqpoint{4.001594in}{1.828120in}}%
\pgfpathcurveto{\pgfqpoint{4.001594in}{1.835253in}}{\pgfqpoint{3.998760in}{1.842094in}}{\pgfqpoint{3.993716in}{1.847138in}}%
\pgfpathcurveto{\pgfqpoint{3.988673in}{1.852182in}}{\pgfqpoint{3.981831in}{1.855015in}}{\pgfqpoint{3.974698in}{1.855015in}}%
\pgfpathcurveto{\pgfqpoint{3.967565in}{1.855015in}}{\pgfqpoint{3.960724in}{1.852182in}}{\pgfqpoint{3.955680in}{1.847138in}}%
\pgfpathcurveto{\pgfqpoint{3.950636in}{1.842094in}}{\pgfqpoint{3.947803in}{1.835253in}}{\pgfqpoint{3.947803in}{1.828120in}}%
\pgfpathcurveto{\pgfqpoint{3.947803in}{1.820987in}}{\pgfqpoint{3.950636in}{1.814145in}}{\pgfqpoint{3.955680in}{1.809102in}}%
\pgfpathcurveto{\pgfqpoint{3.960724in}{1.804058in}}{\pgfqpoint{3.967565in}{1.801224in}}{\pgfqpoint{3.974698in}{1.801224in}}%
\pgfpathclose%
\pgfusepath{stroke,fill}%
\end{pgfscope}%
\begin{pgfscope}%
\pgfpathrectangle{\pgfqpoint{2.867647in}{0.500000in}}{\pgfqpoint{1.764706in}{1.700000in}}%
\pgfusepath{clip}%
\pgfsetbuttcap%
\pgfsetroundjoin%
\definecolor{currentfill}{rgb}{0.973832,0.856556,0.771584}%
\pgfsetfillcolor{currentfill}%
\pgfsetlinewidth{0.311001pt}%
\definecolor{currentstroke}{rgb}{1.000000,1.000000,1.000000}%
\pgfsetstrokecolor{currentstroke}%
\pgfsetdash{}{0pt}%
\pgfpathmoveto{\pgfqpoint{4.244999in}{1.435196in}}%
\pgfpathcurveto{\pgfqpoint{4.252132in}{1.435196in}}{\pgfqpoint{4.258973in}{1.438030in}}{\pgfqpoint{4.264017in}{1.443074in}}%
\pgfpathcurveto{\pgfqpoint{4.269060in}{1.448117in}}{\pgfqpoint{4.271894in}{1.454959in}}{\pgfqpoint{4.271894in}{1.462092in}}%
\pgfpathcurveto{\pgfqpoint{4.271894in}{1.469225in}}{\pgfqpoint{4.269060in}{1.476066in}}{\pgfqpoint{4.264017in}{1.481110in}}%
\pgfpathcurveto{\pgfqpoint{4.258973in}{1.486154in}}{\pgfqpoint{4.252132in}{1.488988in}}{\pgfqpoint{4.244999in}{1.488988in}}%
\pgfpathcurveto{\pgfqpoint{4.237866in}{1.488988in}}{\pgfqpoint{4.231024in}{1.486154in}}{\pgfqpoint{4.225981in}{1.481110in}}%
\pgfpathcurveto{\pgfqpoint{4.220937in}{1.476066in}}{\pgfqpoint{4.218103in}{1.469225in}}{\pgfqpoint{4.218103in}{1.462092in}}%
\pgfpathcurveto{\pgfqpoint{4.218103in}{1.454959in}}{\pgfqpoint{4.220937in}{1.448117in}}{\pgfqpoint{4.225981in}{1.443074in}}%
\pgfpathcurveto{\pgfqpoint{4.231024in}{1.438030in}}{\pgfqpoint{4.237866in}{1.435196in}}{\pgfqpoint{4.244999in}{1.435196in}}%
\pgfpathclose%
\pgfusepath{stroke,fill}%
\end{pgfscope}%
\begin{pgfscope}%
\pgfpathrectangle{\pgfqpoint{2.867647in}{0.500000in}}{\pgfqpoint{1.764706in}{1.700000in}}%
\pgfusepath{clip}%
\pgfsetbuttcap%
\pgfsetroundjoin%
\definecolor{currentfill}{rgb}{0.964920,0.695342,0.545192}%
\pgfsetfillcolor{currentfill}%
\pgfsetlinewidth{0.311001pt}%
\definecolor{currentstroke}{rgb}{1.000000,1.000000,1.000000}%
\pgfsetstrokecolor{currentstroke}%
\pgfsetdash{}{0pt}%
\pgfpathmoveto{\pgfqpoint{4.044540in}{1.467590in}}%
\pgfpathcurveto{\pgfqpoint{4.051673in}{1.467590in}}{\pgfqpoint{4.058515in}{1.470424in}}{\pgfqpoint{4.063558in}{1.475468in}}%
\pgfpathcurveto{\pgfqpoint{4.068602in}{1.480511in}}{\pgfqpoint{4.071436in}{1.487353in}}{\pgfqpoint{4.071436in}{1.494486in}}%
\pgfpathcurveto{\pgfqpoint{4.071436in}{1.501619in}}{\pgfqpoint{4.068602in}{1.508460in}}{\pgfqpoint{4.063558in}{1.513504in}}%
\pgfpathcurveto{\pgfqpoint{4.058515in}{1.518548in}}{\pgfqpoint{4.051673in}{1.521382in}}{\pgfqpoint{4.044540in}{1.521382in}}%
\pgfpathcurveto{\pgfqpoint{4.037407in}{1.521382in}}{\pgfqpoint{4.030566in}{1.518548in}}{\pgfqpoint{4.025522in}{1.513504in}}%
\pgfpathcurveto{\pgfqpoint{4.020478in}{1.508460in}}{\pgfqpoint{4.017644in}{1.501619in}}{\pgfqpoint{4.017644in}{1.494486in}}%
\pgfpathcurveto{\pgfqpoint{4.017644in}{1.487353in}}{\pgfqpoint{4.020478in}{1.480511in}}{\pgfqpoint{4.025522in}{1.475468in}}%
\pgfpathcurveto{\pgfqpoint{4.030566in}{1.470424in}}{\pgfqpoint{4.037407in}{1.467590in}}{\pgfqpoint{4.044540in}{1.467590in}}%
\pgfpathclose%
\pgfusepath{stroke,fill}%
\end{pgfscope}%
\begin{pgfscope}%
\pgfpathrectangle{\pgfqpoint{2.867647in}{0.500000in}}{\pgfqpoint{1.764706in}{1.700000in}}%
\pgfusepath{clip}%
\pgfsetbuttcap%
\pgfsetroundjoin%
\definecolor{currentfill}{rgb}{0.980678,0.914765,0.856766}%
\pgfsetfillcolor{currentfill}%
\pgfsetlinewidth{0.311001pt}%
\definecolor{currentstroke}{rgb}{1.000000,1.000000,1.000000}%
\pgfsetstrokecolor{currentstroke}%
\pgfsetdash{}{0pt}%
\pgfpathmoveto{\pgfqpoint{4.157293in}{1.202995in}}%
\pgfpathcurveto{\pgfqpoint{4.164426in}{1.202995in}}{\pgfqpoint{4.171267in}{1.205829in}}{\pgfqpoint{4.176311in}{1.210873in}}%
\pgfpathcurveto{\pgfqpoint{4.181355in}{1.215916in}}{\pgfqpoint{4.184189in}{1.222758in}}{\pgfqpoint{4.184189in}{1.229891in}}%
\pgfpathcurveto{\pgfqpoint{4.184189in}{1.237024in}}{\pgfqpoint{4.181355in}{1.243865in}}{\pgfqpoint{4.176311in}{1.248909in}}%
\pgfpathcurveto{\pgfqpoint{4.171267in}{1.253953in}}{\pgfqpoint{4.164426in}{1.256787in}}{\pgfqpoint{4.157293in}{1.256787in}}%
\pgfpathcurveto{\pgfqpoint{4.150160in}{1.256787in}}{\pgfqpoint{4.143318in}{1.253953in}}{\pgfqpoint{4.138275in}{1.248909in}}%
\pgfpathcurveto{\pgfqpoint{4.133231in}{1.243865in}}{\pgfqpoint{4.130397in}{1.237024in}}{\pgfqpoint{4.130397in}{1.229891in}}%
\pgfpathcurveto{\pgfqpoint{4.130397in}{1.222758in}}{\pgfqpoint{4.133231in}{1.215916in}}{\pgfqpoint{4.138275in}{1.210873in}}%
\pgfpathcurveto{\pgfqpoint{4.143318in}{1.205829in}}{\pgfqpoint{4.150160in}{1.202995in}}{\pgfqpoint{4.157293in}{1.202995in}}%
\pgfpathclose%
\pgfusepath{stroke,fill}%
\end{pgfscope}%
\begin{pgfscope}%
\pgfpathrectangle{\pgfqpoint{2.867647in}{0.500000in}}{\pgfqpoint{1.764706in}{1.700000in}}%
\pgfusepath{clip}%
\pgfsetbuttcap%
\pgfsetroundjoin%
\definecolor{currentfill}{rgb}{0.973832,0.856556,0.771584}%
\pgfsetfillcolor{currentfill}%
\pgfsetlinewidth{0.311001pt}%
\definecolor{currentstroke}{rgb}{1.000000,1.000000,1.000000}%
\pgfsetstrokecolor{currentstroke}%
\pgfsetdash{}{0pt}%
\pgfpathmoveto{\pgfqpoint{4.136924in}{1.001976in}}%
\pgfpathcurveto{\pgfqpoint{4.144057in}{1.001976in}}{\pgfqpoint{4.150898in}{1.004810in}}{\pgfqpoint{4.155942in}{1.009854in}}%
\pgfpathcurveto{\pgfqpoint{4.160986in}{1.014897in}}{\pgfqpoint{4.163819in}{1.021739in}}{\pgfqpoint{4.163819in}{1.028872in}}%
\pgfpathcurveto{\pgfqpoint{4.163819in}{1.036005in}}{\pgfqpoint{4.160986in}{1.042846in}}{\pgfqpoint{4.155942in}{1.047890in}}%
\pgfpathcurveto{\pgfqpoint{4.150898in}{1.052934in}}{\pgfqpoint{4.144057in}{1.055768in}}{\pgfqpoint{4.136924in}{1.055768in}}%
\pgfpathcurveto{\pgfqpoint{4.129791in}{1.055768in}}{\pgfqpoint{4.122949in}{1.052934in}}{\pgfqpoint{4.117906in}{1.047890in}}%
\pgfpathcurveto{\pgfqpoint{4.112862in}{1.042846in}}{\pgfqpoint{4.110028in}{1.036005in}}{\pgfqpoint{4.110028in}{1.028872in}}%
\pgfpathcurveto{\pgfqpoint{4.110028in}{1.021739in}}{\pgfqpoint{4.112862in}{1.014897in}}{\pgfqpoint{4.117906in}{1.009854in}}%
\pgfpathcurveto{\pgfqpoint{4.122949in}{1.004810in}}{\pgfqpoint{4.129791in}{1.001976in}}{\pgfqpoint{4.136924in}{1.001976in}}%
\pgfpathclose%
\pgfusepath{stroke,fill}%
\end{pgfscope}%
\begin{pgfscope}%
\pgfpathrectangle{\pgfqpoint{2.867647in}{0.500000in}}{\pgfqpoint{1.764706in}{1.700000in}}%
\pgfusepath{clip}%
\pgfsetbuttcap%
\pgfsetroundjoin%
\definecolor{currentfill}{rgb}{0.981377,0.920617,0.865369}%
\pgfsetfillcolor{currentfill}%
\pgfsetlinewidth{0.311001pt}%
\definecolor{currentstroke}{rgb}{1.000000,1.000000,1.000000}%
\pgfsetstrokecolor{currentstroke}%
\pgfsetdash{}{0pt}%
\pgfpathmoveto{\pgfqpoint{4.193659in}{1.321933in}}%
\pgfpathcurveto{\pgfqpoint{4.200791in}{1.321933in}}{\pgfqpoint{4.207633in}{1.324767in}}{\pgfqpoint{4.212677in}{1.329811in}}%
\pgfpathcurveto{\pgfqpoint{4.217720in}{1.334855in}}{\pgfqpoint{4.220554in}{1.341696in}}{\pgfqpoint{4.220554in}{1.348829in}}%
\pgfpathcurveto{\pgfqpoint{4.220554in}{1.355962in}}{\pgfqpoint{4.217720in}{1.362803in}}{\pgfqpoint{4.212677in}{1.367847in}}%
\pgfpathcurveto{\pgfqpoint{4.207633in}{1.372891in}}{\pgfqpoint{4.200791in}{1.375725in}}{\pgfqpoint{4.193659in}{1.375725in}}%
\pgfpathcurveto{\pgfqpoint{4.186526in}{1.375725in}}{\pgfqpoint{4.179684in}{1.372891in}}{\pgfqpoint{4.174641in}{1.367847in}}%
\pgfpathcurveto{\pgfqpoint{4.169597in}{1.362803in}}{\pgfqpoint{4.166763in}{1.355962in}}{\pgfqpoint{4.166763in}{1.348829in}}%
\pgfpathcurveto{\pgfqpoint{4.166763in}{1.341696in}}{\pgfqpoint{4.169597in}{1.334855in}}{\pgfqpoint{4.174641in}{1.329811in}}%
\pgfpathcurveto{\pgfqpoint{4.179684in}{1.324767in}}{\pgfqpoint{4.186526in}{1.321933in}}{\pgfqpoint{4.193659in}{1.321933in}}%
\pgfpathclose%
\pgfusepath{stroke,fill}%
\end{pgfscope}%
\begin{pgfscope}%
\pgfpathrectangle{\pgfqpoint{2.867647in}{0.500000in}}{\pgfqpoint{1.764706in}{1.700000in}}%
\pgfusepath{clip}%
\pgfsetbuttcap%
\pgfsetroundjoin%
\definecolor{currentfill}{rgb}{0.965169,0.707764,0.560659}%
\pgfsetfillcolor{currentfill}%
\pgfsetlinewidth{0.311001pt}%
\definecolor{currentstroke}{rgb}{1.000000,1.000000,1.000000}%
\pgfsetstrokecolor{currentstroke}%
\pgfsetdash{}{0pt}%
\pgfpathmoveto{\pgfqpoint{4.003814in}{0.916716in}}%
\pgfpathcurveto{\pgfqpoint{4.010947in}{0.916716in}}{\pgfqpoint{4.017788in}{0.919550in}}{\pgfqpoint{4.022832in}{0.924594in}}%
\pgfpathcurveto{\pgfqpoint{4.027876in}{0.929637in}}{\pgfqpoint{4.030709in}{0.936479in}}{\pgfqpoint{4.030709in}{0.943612in}}%
\pgfpathcurveto{\pgfqpoint{4.030709in}{0.950745in}}{\pgfqpoint{4.027876in}{0.957586in}}{\pgfqpoint{4.022832in}{0.962630in}}%
\pgfpathcurveto{\pgfqpoint{4.017788in}{0.967674in}}{\pgfqpoint{4.010947in}{0.970508in}}{\pgfqpoint{4.003814in}{0.970508in}}%
\pgfpathcurveto{\pgfqpoint{3.996681in}{0.970508in}}{\pgfqpoint{3.989839in}{0.967674in}}{\pgfqpoint{3.984796in}{0.962630in}}%
\pgfpathcurveto{\pgfqpoint{3.979752in}{0.957586in}}{\pgfqpoint{3.976918in}{0.950745in}}{\pgfqpoint{3.976918in}{0.943612in}}%
\pgfpathcurveto{\pgfqpoint{3.976918in}{0.936479in}}{\pgfqpoint{3.979752in}{0.929637in}}{\pgfqpoint{3.984796in}{0.924594in}}%
\pgfpathcurveto{\pgfqpoint{3.989839in}{0.919550in}}{\pgfqpoint{3.996681in}{0.916716in}}{\pgfqpoint{4.003814in}{0.916716in}}%
\pgfpathclose%
\pgfusepath{stroke,fill}%
\end{pgfscope}%
\begin{pgfscope}%
\pgfpathrectangle{\pgfqpoint{2.867647in}{0.500000in}}{\pgfqpoint{1.764706in}{1.700000in}}%
\pgfusepath{clip}%
\pgfsetbuttcap%
\pgfsetroundjoin%
\definecolor{currentfill}{rgb}{0.960421,0.553286,0.393191}%
\pgfsetfillcolor{currentfill}%
\pgfsetlinewidth{0.311001pt}%
\definecolor{currentstroke}{rgb}{1.000000,1.000000,1.000000}%
\pgfsetstrokecolor{currentstroke}%
\pgfsetdash{}{0pt}%
\pgfpathmoveto{\pgfqpoint{4.172141in}{1.752993in}}%
\pgfpathcurveto{\pgfqpoint{4.179274in}{1.752993in}}{\pgfqpoint{4.186116in}{1.755826in}}{\pgfqpoint{4.191160in}{1.760870in}}%
\pgfpathcurveto{\pgfqpoint{4.196203in}{1.765914in}}{\pgfqpoint{4.199037in}{1.772755in}}{\pgfqpoint{4.199037in}{1.779888in}}%
\pgfpathcurveto{\pgfqpoint{4.199037in}{1.787021in}}{\pgfqpoint{4.196203in}{1.793863in}}{\pgfqpoint{4.191160in}{1.798906in}}%
\pgfpathcurveto{\pgfqpoint{4.186116in}{1.803950in}}{\pgfqpoint{4.179274in}{1.806784in}}{\pgfqpoint{4.172141in}{1.806784in}}%
\pgfpathcurveto{\pgfqpoint{4.165009in}{1.806784in}}{\pgfqpoint{4.158167in}{1.803950in}}{\pgfqpoint{4.153123in}{1.798906in}}%
\pgfpathcurveto{\pgfqpoint{4.148080in}{1.793863in}}{\pgfqpoint{4.145246in}{1.787021in}}{\pgfqpoint{4.145246in}{1.779888in}}%
\pgfpathcurveto{\pgfqpoint{4.145246in}{1.772755in}}{\pgfqpoint{4.148080in}{1.765914in}}{\pgfqpoint{4.153123in}{1.760870in}}%
\pgfpathcurveto{\pgfqpoint{4.158167in}{1.755826in}}{\pgfqpoint{4.165009in}{1.752993in}}{\pgfqpoint{4.172141in}{1.752993in}}%
\pgfpathclose%
\pgfusepath{stroke,fill}%
\end{pgfscope}%
\begin{pgfscope}%
\pgfpathrectangle{\pgfqpoint{2.867647in}{0.500000in}}{\pgfqpoint{1.764706in}{1.700000in}}%
\pgfusepath{clip}%
\pgfsetbuttcap%
\pgfsetroundjoin%
\definecolor{currentfill}{rgb}{0.976287,0.879862,0.805788}%
\pgfsetfillcolor{currentfill}%
\pgfsetlinewidth{0.311001pt}%
\definecolor{currentstroke}{rgb}{1.000000,1.000000,1.000000}%
\pgfsetstrokecolor{currentstroke}%
\pgfsetdash{}{0pt}%
\pgfpathmoveto{\pgfqpoint{4.239693in}{1.252594in}}%
\pgfpathcurveto{\pgfqpoint{4.246826in}{1.252594in}}{\pgfqpoint{4.253667in}{1.255428in}}{\pgfqpoint{4.258711in}{1.260472in}}%
\pgfpathcurveto{\pgfqpoint{4.263755in}{1.265515in}}{\pgfqpoint{4.266589in}{1.272357in}}{\pgfqpoint{4.266589in}{1.279490in}}%
\pgfpathcurveto{\pgfqpoint{4.266589in}{1.286623in}}{\pgfqpoint{4.263755in}{1.293464in}}{\pgfqpoint{4.258711in}{1.298508in}}%
\pgfpathcurveto{\pgfqpoint{4.253667in}{1.303552in}}{\pgfqpoint{4.246826in}{1.306386in}}{\pgfqpoint{4.239693in}{1.306386in}}%
\pgfpathcurveto{\pgfqpoint{4.232560in}{1.306386in}}{\pgfqpoint{4.225719in}{1.303552in}}{\pgfqpoint{4.220675in}{1.298508in}}%
\pgfpathcurveto{\pgfqpoint{4.215631in}{1.293464in}}{\pgfqpoint{4.212797in}{1.286623in}}{\pgfqpoint{4.212797in}{1.279490in}}%
\pgfpathcurveto{\pgfqpoint{4.212797in}{1.272357in}}{\pgfqpoint{4.215631in}{1.265515in}}{\pgfqpoint{4.220675in}{1.260472in}}%
\pgfpathcurveto{\pgfqpoint{4.225719in}{1.255428in}}{\pgfqpoint{4.232560in}{1.252594in}}{\pgfqpoint{4.239693in}{1.252594in}}%
\pgfpathclose%
\pgfusepath{stroke,fill}%
\end{pgfscope}%
\begin{pgfscope}%
\pgfpathrectangle{\pgfqpoint{2.867647in}{0.500000in}}{\pgfqpoint{1.764706in}{1.700000in}}%
\pgfusepath{clip}%
\pgfsetbuttcap%
\pgfsetroundjoin%
\definecolor{currentfill}{rgb}{0.978376,0.897317,0.831308}%
\pgfsetfillcolor{currentfill}%
\pgfsetlinewidth{0.311001pt}%
\definecolor{currentstroke}{rgb}{1.000000,1.000000,1.000000}%
\pgfsetstrokecolor{currentstroke}%
\pgfsetdash{}{0pt}%
\pgfpathmoveto{\pgfqpoint{4.150054in}{1.293925in}}%
\pgfpathcurveto{\pgfqpoint{4.157187in}{1.293925in}}{\pgfqpoint{4.164029in}{1.296759in}}{\pgfqpoint{4.169072in}{1.301802in}}%
\pgfpathcurveto{\pgfqpoint{4.174116in}{1.306846in}}{\pgfqpoint{4.176950in}{1.313688in}}{\pgfqpoint{4.176950in}{1.320820in}}%
\pgfpathcurveto{\pgfqpoint{4.176950in}{1.327953in}}{\pgfqpoint{4.174116in}{1.334795in}}{\pgfqpoint{4.169072in}{1.339838in}}%
\pgfpathcurveto{\pgfqpoint{4.164029in}{1.344882in}}{\pgfqpoint{4.157187in}{1.347716in}}{\pgfqpoint{4.150054in}{1.347716in}}%
\pgfpathcurveto{\pgfqpoint{4.142921in}{1.347716in}}{\pgfqpoint{4.136080in}{1.344882in}}{\pgfqpoint{4.131036in}{1.339838in}}%
\pgfpathcurveto{\pgfqpoint{4.125992in}{1.334795in}}{\pgfqpoint{4.123158in}{1.327953in}}{\pgfqpoint{4.123158in}{1.320820in}}%
\pgfpathcurveto{\pgfqpoint{4.123158in}{1.313688in}}{\pgfqpoint{4.125992in}{1.306846in}}{\pgfqpoint{4.131036in}{1.301802in}}%
\pgfpathcurveto{\pgfqpoint{4.136080in}{1.296759in}}{\pgfqpoint{4.142921in}{1.293925in}}{\pgfqpoint{4.150054in}{1.293925in}}%
\pgfpathclose%
\pgfusepath{stroke,fill}%
\end{pgfscope}%
\begin{pgfscope}%
\pgfpathrectangle{\pgfqpoint{2.867647in}{0.500000in}}{\pgfqpoint{1.764706in}{1.700000in}}%
\pgfusepath{clip}%
\pgfsetbuttcap%
\pgfsetroundjoin%
\definecolor{currentfill}{rgb}{0.947270,0.405591,0.279023}%
\pgfsetfillcolor{currentfill}%
\pgfsetlinewidth{0.311001pt}%
\definecolor{currentstroke}{rgb}{1.000000,1.000000,1.000000}%
\pgfsetstrokecolor{currentstroke}%
\pgfsetdash{}{0pt}%
\pgfpathmoveto{\pgfqpoint{3.884279in}{0.948368in}}%
\pgfpathcurveto{\pgfqpoint{3.891411in}{0.948368in}}{\pgfqpoint{3.898253in}{0.951202in}}{\pgfqpoint{3.903297in}{0.956246in}}%
\pgfpathcurveto{\pgfqpoint{3.908340in}{0.961290in}}{\pgfqpoint{3.911174in}{0.968131in}}{\pgfqpoint{3.911174in}{0.975264in}}%
\pgfpathcurveto{\pgfqpoint{3.911174in}{0.982397in}}{\pgfqpoint{3.908340in}{0.989239in}}{\pgfqpoint{3.903297in}{0.994282in}}%
\pgfpathcurveto{\pgfqpoint{3.898253in}{0.999326in}}{\pgfqpoint{3.891411in}{1.002160in}}{\pgfqpoint{3.884279in}{1.002160in}}%
\pgfpathcurveto{\pgfqpoint{3.877146in}{1.002160in}}{\pgfqpoint{3.870304in}{0.999326in}}{\pgfqpoint{3.865260in}{0.994282in}}%
\pgfpathcurveto{\pgfqpoint{3.860217in}{0.989239in}}{\pgfqpoint{3.857383in}{0.982397in}}{\pgfqpoint{3.857383in}{0.975264in}}%
\pgfpathcurveto{\pgfqpoint{3.857383in}{0.968131in}}{\pgfqpoint{3.860217in}{0.961290in}}{\pgfqpoint{3.865260in}{0.956246in}}%
\pgfpathcurveto{\pgfqpoint{3.870304in}{0.951202in}}{\pgfqpoint{3.877146in}{0.948368in}}{\pgfqpoint{3.884279in}{0.948368in}}%
\pgfpathclose%
\pgfusepath{stroke,fill}%
\end{pgfscope}%
\begin{pgfscope}%
\pgfpathrectangle{\pgfqpoint{2.867647in}{0.500000in}}{\pgfqpoint{1.764706in}{1.700000in}}%
\pgfusepath{clip}%
\pgfsetbuttcap%
\pgfsetroundjoin%
\definecolor{currentfill}{rgb}{0.971202,0.827364,0.728520}%
\pgfsetfillcolor{currentfill}%
\pgfsetlinewidth{0.311001pt}%
\definecolor{currentstroke}{rgb}{1.000000,1.000000,1.000000}%
\pgfsetstrokecolor{currentstroke}%
\pgfsetdash{}{0pt}%
\pgfpathmoveto{\pgfqpoint{4.073923in}{1.683102in}}%
\pgfpathcurveto{\pgfqpoint{4.081056in}{1.683102in}}{\pgfqpoint{4.087898in}{1.685935in}}{\pgfqpoint{4.092941in}{1.690979in}}%
\pgfpathcurveto{\pgfqpoint{4.097985in}{1.696023in}}{\pgfqpoint{4.100819in}{1.702864in}}{\pgfqpoint{4.100819in}{1.709997in}}%
\pgfpathcurveto{\pgfqpoint{4.100819in}{1.717130in}}{\pgfqpoint{4.097985in}{1.723972in}}{\pgfqpoint{4.092941in}{1.729015in}}%
\pgfpathcurveto{\pgfqpoint{4.087898in}{1.734059in}}{\pgfqpoint{4.081056in}{1.736893in}}{\pgfqpoint{4.073923in}{1.736893in}}%
\pgfpathcurveto{\pgfqpoint{4.066790in}{1.736893in}}{\pgfqpoint{4.059949in}{1.734059in}}{\pgfqpoint{4.054905in}{1.729015in}}%
\pgfpathcurveto{\pgfqpoint{4.049861in}{1.723972in}}{\pgfqpoint{4.047028in}{1.717130in}}{\pgfqpoint{4.047028in}{1.709997in}}%
\pgfpathcurveto{\pgfqpoint{4.047028in}{1.702864in}}{\pgfqpoint{4.049861in}{1.696023in}}{\pgfqpoint{4.054905in}{1.690979in}}%
\pgfpathcurveto{\pgfqpoint{4.059949in}{1.685935in}}{\pgfqpoint{4.066790in}{1.683102in}}{\pgfqpoint{4.073923in}{1.683102in}}%
\pgfpathclose%
\pgfusepath{stroke,fill}%
\end{pgfscope}%
\begin{pgfscope}%
\pgfpathrectangle{\pgfqpoint{2.867647in}{0.500000in}}{\pgfqpoint{1.764706in}{1.700000in}}%
\pgfusepath{clip}%
\pgfsetbuttcap%
\pgfsetroundjoin%
\definecolor{currentfill}{rgb}{0.980678,0.914765,0.856766}%
\pgfsetfillcolor{currentfill}%
\pgfsetlinewidth{0.311001pt}%
\definecolor{currentstroke}{rgb}{1.000000,1.000000,1.000000}%
\pgfsetstrokecolor{currentstroke}%
\pgfsetdash{}{0pt}%
\pgfpathmoveto{\pgfqpoint{4.174426in}{1.361781in}}%
\pgfpathcurveto{\pgfqpoint{4.181559in}{1.361781in}}{\pgfqpoint{4.188401in}{1.364615in}}{\pgfqpoint{4.193444in}{1.369659in}}%
\pgfpathcurveto{\pgfqpoint{4.198488in}{1.374702in}}{\pgfqpoint{4.201322in}{1.381544in}}{\pgfqpoint{4.201322in}{1.388677in}}%
\pgfpathcurveto{\pgfqpoint{4.201322in}{1.395810in}}{\pgfqpoint{4.198488in}{1.402651in}}{\pgfqpoint{4.193444in}{1.407695in}}%
\pgfpathcurveto{\pgfqpoint{4.188401in}{1.412739in}}{\pgfqpoint{4.181559in}{1.415572in}}{\pgfqpoint{4.174426in}{1.415572in}}%
\pgfpathcurveto{\pgfqpoint{4.167293in}{1.415572in}}{\pgfqpoint{4.160452in}{1.412739in}}{\pgfqpoint{4.155408in}{1.407695in}}%
\pgfpathcurveto{\pgfqpoint{4.150364in}{1.402651in}}{\pgfqpoint{4.147531in}{1.395810in}}{\pgfqpoint{4.147531in}{1.388677in}}%
\pgfpathcurveto{\pgfqpoint{4.147531in}{1.381544in}}{\pgfqpoint{4.150364in}{1.374702in}}{\pgfqpoint{4.155408in}{1.369659in}}%
\pgfpathcurveto{\pgfqpoint{4.160452in}{1.364615in}}{\pgfqpoint{4.167293in}{1.361781in}}{\pgfqpoint{4.174426in}{1.361781in}}%
\pgfpathclose%
\pgfusepath{stroke,fill}%
\end{pgfscope}%
\begin{pgfscope}%
\pgfpathrectangle{\pgfqpoint{2.867647in}{0.500000in}}{\pgfqpoint{1.764706in}{1.700000in}}%
\pgfusepath{clip}%
\pgfsetbuttcap%
\pgfsetroundjoin%
\definecolor{currentfill}{rgb}{0.979124,0.903132,0.839793}%
\pgfsetfillcolor{currentfill}%
\pgfsetlinewidth{0.311001pt}%
\definecolor{currentstroke}{rgb}{1.000000,1.000000,1.000000}%
\pgfsetstrokecolor{currentstroke}%
\pgfsetdash{}{0pt}%
\pgfpathmoveto{\pgfqpoint{4.155907in}{1.285453in}}%
\pgfpathcurveto{\pgfqpoint{4.163040in}{1.285453in}}{\pgfqpoint{4.169882in}{1.288286in}}{\pgfqpoint{4.174925in}{1.293330in}}%
\pgfpathcurveto{\pgfqpoint{4.179969in}{1.298374in}}{\pgfqpoint{4.182803in}{1.305215in}}{\pgfqpoint{4.182803in}{1.312348in}}%
\pgfpathcurveto{\pgfqpoint{4.182803in}{1.319481in}}{\pgfqpoint{4.179969in}{1.326323in}}{\pgfqpoint{4.174925in}{1.331366in}}%
\pgfpathcurveto{\pgfqpoint{4.169882in}{1.336410in}}{\pgfqpoint{4.163040in}{1.339244in}}{\pgfqpoint{4.155907in}{1.339244in}}%
\pgfpathcurveto{\pgfqpoint{4.148774in}{1.339244in}}{\pgfqpoint{4.141933in}{1.336410in}}{\pgfqpoint{4.136889in}{1.331366in}}%
\pgfpathcurveto{\pgfqpoint{4.131845in}{1.326323in}}{\pgfqpoint{4.129011in}{1.319481in}}{\pgfqpoint{4.129011in}{1.312348in}}%
\pgfpathcurveto{\pgfqpoint{4.129011in}{1.305215in}}{\pgfqpoint{4.131845in}{1.298374in}}{\pgfqpoint{4.136889in}{1.293330in}}%
\pgfpathcurveto{\pgfqpoint{4.141933in}{1.288286in}}{\pgfqpoint{4.148774in}{1.285453in}}{\pgfqpoint{4.155907in}{1.285453in}}%
\pgfpathclose%
\pgfusepath{stroke,fill}%
\end{pgfscope}%
\begin{pgfscope}%
\pgfpathrectangle{\pgfqpoint{2.867647in}{0.500000in}}{\pgfqpoint{1.764706in}{1.700000in}}%
\pgfusepath{clip}%
\pgfsetbuttcap%
\pgfsetroundjoin%
\definecolor{currentfill}{rgb}{0.971694,0.833208,0.737161}%
\pgfsetfillcolor{currentfill}%
\pgfsetlinewidth{0.311001pt}%
\definecolor{currentstroke}{rgb}{1.000000,1.000000,1.000000}%
\pgfsetstrokecolor{currentstroke}%
\pgfsetdash{}{0pt}%
\pgfpathmoveto{\pgfqpoint{4.059605in}{1.577803in}}%
\pgfpathcurveto{\pgfqpoint{4.066738in}{1.577803in}}{\pgfqpoint{4.073580in}{1.580637in}}{\pgfqpoint{4.078623in}{1.585680in}}%
\pgfpathcurveto{\pgfqpoint{4.083667in}{1.590724in}}{\pgfqpoint{4.086501in}{1.597566in}}{\pgfqpoint{4.086501in}{1.604699in}}%
\pgfpathcurveto{\pgfqpoint{4.086501in}{1.611831in}}{\pgfqpoint{4.083667in}{1.618673in}}{\pgfqpoint{4.078623in}{1.623717in}}%
\pgfpathcurveto{\pgfqpoint{4.073580in}{1.628760in}}{\pgfqpoint{4.066738in}{1.631594in}}{\pgfqpoint{4.059605in}{1.631594in}}%
\pgfpathcurveto{\pgfqpoint{4.052472in}{1.631594in}}{\pgfqpoint{4.045631in}{1.628760in}}{\pgfqpoint{4.040587in}{1.623717in}}%
\pgfpathcurveto{\pgfqpoint{4.035543in}{1.618673in}}{\pgfqpoint{4.032709in}{1.611831in}}{\pgfqpoint{4.032709in}{1.604699in}}%
\pgfpathcurveto{\pgfqpoint{4.032709in}{1.597566in}}{\pgfqpoint{4.035543in}{1.590724in}}{\pgfqpoint{4.040587in}{1.585680in}}%
\pgfpathcurveto{\pgfqpoint{4.045631in}{1.580637in}}{\pgfqpoint{4.052472in}{1.577803in}}{\pgfqpoint{4.059605in}{1.577803in}}%
\pgfpathclose%
\pgfusepath{stroke,fill}%
\end{pgfscope}%
\begin{pgfscope}%
\pgfpathrectangle{\pgfqpoint{2.867647in}{0.500000in}}{\pgfqpoint{1.764706in}{1.700000in}}%
\pgfusepath{clip}%
\pgfsetbuttcap%
\pgfsetroundjoin%
\definecolor{currentfill}{rgb}{0.964433,0.670254,0.515093}%
\pgfsetfillcolor{currentfill}%
\pgfsetlinewidth{0.311001pt}%
\definecolor{currentstroke}{rgb}{1.000000,1.000000,1.000000}%
\pgfsetstrokecolor{currentstroke}%
\pgfsetdash{}{0pt}%
\pgfpathmoveto{\pgfqpoint{3.994234in}{1.758466in}}%
\pgfpathcurveto{\pgfqpoint{4.001367in}{1.758466in}}{\pgfqpoint{4.008209in}{1.761300in}}{\pgfqpoint{4.013252in}{1.766343in}}%
\pgfpathcurveto{\pgfqpoint{4.018296in}{1.771387in}}{\pgfqpoint{4.021130in}{1.778229in}}{\pgfqpoint{4.021130in}{1.785361in}}%
\pgfpathcurveto{\pgfqpoint{4.021130in}{1.792494in}}{\pgfqpoint{4.018296in}{1.799336in}}{\pgfqpoint{4.013252in}{1.804380in}}%
\pgfpathcurveto{\pgfqpoint{4.008209in}{1.809423in}}{\pgfqpoint{4.001367in}{1.812257in}}{\pgfqpoint{3.994234in}{1.812257in}}%
\pgfpathcurveto{\pgfqpoint{3.987101in}{1.812257in}}{\pgfqpoint{3.980260in}{1.809423in}}{\pgfqpoint{3.975216in}{1.804380in}}%
\pgfpathcurveto{\pgfqpoint{3.970172in}{1.799336in}}{\pgfqpoint{3.967339in}{1.792494in}}{\pgfqpoint{3.967339in}{1.785361in}}%
\pgfpathcurveto{\pgfqpoint{3.967339in}{1.778229in}}{\pgfqpoint{3.970172in}{1.771387in}}{\pgfqpoint{3.975216in}{1.766343in}}%
\pgfpathcurveto{\pgfqpoint{3.980260in}{1.761300in}}{\pgfqpoint{3.987101in}{1.758466in}}{\pgfqpoint{3.994234in}{1.758466in}}%
\pgfpathclose%
\pgfusepath{stroke,fill}%
\end{pgfscope}%
\begin{pgfscope}%
\pgfpathrectangle{\pgfqpoint{2.867647in}{0.500000in}}{\pgfqpoint{1.764706in}{1.700000in}}%
\pgfusepath{clip}%
\pgfsetbuttcap%
\pgfsetroundjoin%
\definecolor{currentfill}{rgb}{0.967398,0.774513,0.650573}%
\pgfsetfillcolor{currentfill}%
\pgfsetlinewidth{0.311001pt}%
\definecolor{currentstroke}{rgb}{1.000000,1.000000,1.000000}%
\pgfsetstrokecolor{currentstroke}%
\pgfsetdash{}{0pt}%
\pgfpathmoveto{\pgfqpoint{4.023402in}{0.982605in}}%
\pgfpathcurveto{\pgfqpoint{4.030535in}{0.982605in}}{\pgfqpoint{4.037377in}{0.985439in}}{\pgfqpoint{4.042420in}{0.990482in}}%
\pgfpathcurveto{\pgfqpoint{4.047464in}{0.995526in}}{\pgfqpoint{4.050298in}{1.002368in}}{\pgfqpoint{4.050298in}{1.009500in}}%
\pgfpathcurveto{\pgfqpoint{4.050298in}{1.016633in}}{\pgfqpoint{4.047464in}{1.023475in}}{\pgfqpoint{4.042420in}{1.028519in}}%
\pgfpathcurveto{\pgfqpoint{4.037377in}{1.033562in}}{\pgfqpoint{4.030535in}{1.036396in}}{\pgfqpoint{4.023402in}{1.036396in}}%
\pgfpathcurveto{\pgfqpoint{4.016269in}{1.036396in}}{\pgfqpoint{4.009428in}{1.033562in}}{\pgfqpoint{4.004384in}{1.028519in}}%
\pgfpathcurveto{\pgfqpoint{3.999340in}{1.023475in}}{\pgfqpoint{3.996507in}{1.016633in}}{\pgfqpoint{3.996507in}{1.009500in}}%
\pgfpathcurveto{\pgfqpoint{3.996507in}{1.002368in}}{\pgfqpoint{3.999340in}{0.995526in}}{\pgfqpoint{4.004384in}{0.990482in}}%
\pgfpathcurveto{\pgfqpoint{4.009428in}{0.985439in}}{\pgfqpoint{4.016269in}{0.982605in}}{\pgfqpoint{4.023402in}{0.982605in}}%
\pgfpathclose%
\pgfusepath{stroke,fill}%
\end{pgfscope}%
\begin{pgfscope}%
\pgfpathrectangle{\pgfqpoint{2.867647in}{0.500000in}}{\pgfqpoint{1.764706in}{1.700000in}}%
\pgfusepath{clip}%
\pgfsetbuttcap%
\pgfsetroundjoin%
\definecolor{currentfill}{rgb}{0.970255,0.815666,0.711203}%
\pgfsetfillcolor{currentfill}%
\pgfsetlinewidth{0.311001pt}%
\definecolor{currentstroke}{rgb}{1.000000,1.000000,1.000000}%
\pgfsetstrokecolor{currentstroke}%
\pgfsetdash{}{0pt}%
\pgfpathmoveto{\pgfqpoint{4.093152in}{1.201536in}}%
\pgfpathcurveto{\pgfqpoint{4.100285in}{1.201536in}}{\pgfqpoint{4.107126in}{1.204370in}}{\pgfqpoint{4.112170in}{1.209414in}}%
\pgfpathcurveto{\pgfqpoint{4.117214in}{1.214457in}}{\pgfqpoint{4.120048in}{1.221299in}}{\pgfqpoint{4.120048in}{1.228432in}}%
\pgfpathcurveto{\pgfqpoint{4.120048in}{1.235565in}}{\pgfqpoint{4.117214in}{1.242406in}}{\pgfqpoint{4.112170in}{1.247450in}}%
\pgfpathcurveto{\pgfqpoint{4.107126in}{1.252493in}}{\pgfqpoint{4.100285in}{1.255327in}}{\pgfqpoint{4.093152in}{1.255327in}}%
\pgfpathcurveto{\pgfqpoint{4.086019in}{1.255327in}}{\pgfqpoint{4.079177in}{1.252493in}}{\pgfqpoint{4.074134in}{1.247450in}}%
\pgfpathcurveto{\pgfqpoint{4.069090in}{1.242406in}}{\pgfqpoint{4.066256in}{1.235565in}}{\pgfqpoint{4.066256in}{1.228432in}}%
\pgfpathcurveto{\pgfqpoint{4.066256in}{1.221299in}}{\pgfqpoint{4.069090in}{1.214457in}}{\pgfqpoint{4.074134in}{1.209414in}}%
\pgfpathcurveto{\pgfqpoint{4.079177in}{1.204370in}}{\pgfqpoint{4.086019in}{1.201536in}}{\pgfqpoint{4.093152in}{1.201536in}}%
\pgfpathclose%
\pgfusepath{stroke,fill}%
\end{pgfscope}%
\begin{pgfscope}%
\pgfpathrectangle{\pgfqpoint{2.867647in}{0.500000in}}{\pgfqpoint{1.764706in}{1.700000in}}%
\pgfusepath{clip}%
\pgfsetbuttcap%
\pgfsetroundjoin%
\definecolor{currentfill}{rgb}{0.969359,0.803954,0.693832}%
\pgfsetfillcolor{currentfill}%
\pgfsetlinewidth{0.311001pt}%
\definecolor{currentstroke}{rgb}{1.000000,1.000000,1.000000}%
\pgfsetstrokecolor{currentstroke}%
\pgfsetdash{}{0pt}%
\pgfpathmoveto{\pgfqpoint{4.229223in}{1.564556in}}%
\pgfpathcurveto{\pgfqpoint{4.236356in}{1.564556in}}{\pgfqpoint{4.243197in}{1.567390in}}{\pgfqpoint{4.248241in}{1.572434in}}%
\pgfpathcurveto{\pgfqpoint{4.253285in}{1.577478in}}{\pgfqpoint{4.256118in}{1.584319in}}{\pgfqpoint{4.256118in}{1.591452in}}%
\pgfpathcurveto{\pgfqpoint{4.256118in}{1.598585in}}{\pgfqpoint{4.253285in}{1.605427in}}{\pgfqpoint{4.248241in}{1.610470in}}%
\pgfpathcurveto{\pgfqpoint{4.243197in}{1.615514in}}{\pgfqpoint{4.236356in}{1.618348in}}{\pgfqpoint{4.229223in}{1.618348in}}%
\pgfpathcurveto{\pgfqpoint{4.222090in}{1.618348in}}{\pgfqpoint{4.215248in}{1.615514in}}{\pgfqpoint{4.210205in}{1.610470in}}%
\pgfpathcurveto{\pgfqpoint{4.205161in}{1.605427in}}{\pgfqpoint{4.202327in}{1.598585in}}{\pgfqpoint{4.202327in}{1.591452in}}%
\pgfpathcurveto{\pgfqpoint{4.202327in}{1.584319in}}{\pgfqpoint{4.205161in}{1.577478in}}{\pgfqpoint{4.210205in}{1.572434in}}%
\pgfpathcurveto{\pgfqpoint{4.215248in}{1.567390in}}{\pgfqpoint{4.222090in}{1.564556in}}{\pgfqpoint{4.229223in}{1.564556in}}%
\pgfpathclose%
\pgfusepath{stroke,fill}%
\end{pgfscope}%
\begin{pgfscope}%
\pgfpathrectangle{\pgfqpoint{2.867647in}{0.500000in}}{\pgfqpoint{1.764706in}{1.700000in}}%
\pgfusepath{clip}%
\pgfsetbuttcap%
\pgfsetroundjoin%
\definecolor{currentfill}{rgb}{0.974412,0.862387,0.780156}%
\pgfsetfillcolor{currentfill}%
\pgfsetlinewidth{0.311001pt}%
\definecolor{currentstroke}{rgb}{1.000000,1.000000,1.000000}%
\pgfsetstrokecolor{currentstroke}%
\pgfsetdash{}{0pt}%
\pgfpathmoveto{\pgfqpoint{4.122006in}{1.414443in}}%
\pgfpathcurveto{\pgfqpoint{4.129139in}{1.414443in}}{\pgfqpoint{4.135981in}{1.417277in}}{\pgfqpoint{4.141024in}{1.422320in}}%
\pgfpathcurveto{\pgfqpoint{4.146068in}{1.427364in}}{\pgfqpoint{4.148902in}{1.434206in}}{\pgfqpoint{4.148902in}{1.441338in}}%
\pgfpathcurveto{\pgfqpoint{4.148902in}{1.448471in}}{\pgfqpoint{4.146068in}{1.455313in}}{\pgfqpoint{4.141024in}{1.460357in}}%
\pgfpathcurveto{\pgfqpoint{4.135981in}{1.465400in}}{\pgfqpoint{4.129139in}{1.468234in}}{\pgfqpoint{4.122006in}{1.468234in}}%
\pgfpathcurveto{\pgfqpoint{4.114874in}{1.468234in}}{\pgfqpoint{4.108032in}{1.465400in}}{\pgfqpoint{4.102988in}{1.460357in}}%
\pgfpathcurveto{\pgfqpoint{4.097945in}{1.455313in}}{\pgfqpoint{4.095111in}{1.448471in}}{\pgfqpoint{4.095111in}{1.441338in}}%
\pgfpathcurveto{\pgfqpoint{4.095111in}{1.434206in}}{\pgfqpoint{4.097945in}{1.427364in}}{\pgfqpoint{4.102988in}{1.422320in}}%
\pgfpathcurveto{\pgfqpoint{4.108032in}{1.417277in}}{\pgfqpoint{4.114874in}{1.414443in}}{\pgfqpoint{4.122006in}{1.414443in}}%
\pgfpathclose%
\pgfusepath{stroke,fill}%
\end{pgfscope}%
\begin{pgfscope}%
\pgfpathrectangle{\pgfqpoint{2.867647in}{0.500000in}}{\pgfqpoint{1.764706in}{1.700000in}}%
\pgfusepath{clip}%
\pgfsetbuttcap%
\pgfsetroundjoin%
\definecolor{currentfill}{rgb}{0.969359,0.803954,0.693832}%
\pgfsetfillcolor{currentfill}%
\pgfsetlinewidth{0.311001pt}%
\definecolor{currentstroke}{rgb}{1.000000,1.000000,1.000000}%
\pgfsetstrokecolor{currentstroke}%
\pgfsetdash{}{0pt}%
\pgfpathmoveto{\pgfqpoint{4.143054in}{0.966770in}}%
\pgfpathcurveto{\pgfqpoint{4.150186in}{0.966770in}}{\pgfqpoint{4.157028in}{0.969604in}}{\pgfqpoint{4.162072in}{0.974648in}}%
\pgfpathcurveto{\pgfqpoint{4.167115in}{0.979692in}}{\pgfqpoint{4.169949in}{0.986533in}}{\pgfqpoint{4.169949in}{0.993666in}}%
\pgfpathcurveto{\pgfqpoint{4.169949in}{1.000799in}}{\pgfqpoint{4.167115in}{1.007641in}}{\pgfqpoint{4.162072in}{1.012684in}}%
\pgfpathcurveto{\pgfqpoint{4.157028in}{1.017728in}}{\pgfqpoint{4.150186in}{1.020562in}}{\pgfqpoint{4.143054in}{1.020562in}}%
\pgfpathcurveto{\pgfqpoint{4.135921in}{1.020562in}}{\pgfqpoint{4.129079in}{1.017728in}}{\pgfqpoint{4.124035in}{1.012684in}}%
\pgfpathcurveto{\pgfqpoint{4.118992in}{1.007641in}}{\pgfqpoint{4.116158in}{1.000799in}}{\pgfqpoint{4.116158in}{0.993666in}}%
\pgfpathcurveto{\pgfqpoint{4.116158in}{0.986533in}}{\pgfqpoint{4.118992in}{0.979692in}}{\pgfqpoint{4.124035in}{0.974648in}}%
\pgfpathcurveto{\pgfqpoint{4.129079in}{0.969604in}}{\pgfqpoint{4.135921in}{0.966770in}}{\pgfqpoint{4.143054in}{0.966770in}}%
\pgfpathclose%
\pgfusepath{stroke,fill}%
\end{pgfscope}%
\begin{pgfscope}%
\pgfpathrectangle{\pgfqpoint{2.867647in}{0.500000in}}{\pgfqpoint{1.764706in}{1.700000in}}%
\pgfusepath{clip}%
\pgfsetbuttcap%
\pgfsetroundjoin%
\definecolor{currentfill}{rgb}{0.964558,0.676556,0.522514}%
\pgfsetfillcolor{currentfill}%
\pgfsetlinewidth{0.311001pt}%
\definecolor{currentstroke}{rgb}{1.000000,1.000000,1.000000}%
\pgfsetstrokecolor{currentstroke}%
\pgfsetdash{}{0pt}%
\pgfpathmoveto{\pgfqpoint{4.307372in}{1.416677in}}%
\pgfpathcurveto{\pgfqpoint{4.314505in}{1.416677in}}{\pgfqpoint{4.321347in}{1.419510in}}{\pgfqpoint{4.326390in}{1.424554in}}%
\pgfpathcurveto{\pgfqpoint{4.331434in}{1.429598in}}{\pgfqpoint{4.334268in}{1.436439in}}{\pgfqpoint{4.334268in}{1.443572in}}%
\pgfpathcurveto{\pgfqpoint{4.334268in}{1.450705in}}{\pgfqpoint{4.331434in}{1.457547in}}{\pgfqpoint{4.326390in}{1.462590in}}%
\pgfpathcurveto{\pgfqpoint{4.321347in}{1.467634in}}{\pgfqpoint{4.314505in}{1.470468in}}{\pgfqpoint{4.307372in}{1.470468in}}%
\pgfpathcurveto{\pgfqpoint{4.300240in}{1.470468in}}{\pgfqpoint{4.293398in}{1.467634in}}{\pgfqpoint{4.288354in}{1.462590in}}%
\pgfpathcurveto{\pgfqpoint{4.283311in}{1.457547in}}{\pgfqpoint{4.280477in}{1.450705in}}{\pgfqpoint{4.280477in}{1.443572in}}%
\pgfpathcurveto{\pgfqpoint{4.280477in}{1.436439in}}{\pgfqpoint{4.283311in}{1.429598in}}{\pgfqpoint{4.288354in}{1.424554in}}%
\pgfpathcurveto{\pgfqpoint{4.293398in}{1.419510in}}{\pgfqpoint{4.300240in}{1.416677in}}{\pgfqpoint{4.307372in}{1.416677in}}%
\pgfpathclose%
\pgfusepath{stroke,fill}%
\end{pgfscope}%
\begin{pgfscope}%
\pgfpathrectangle{\pgfqpoint{2.867647in}{0.500000in}}{\pgfqpoint{1.764706in}{1.700000in}}%
\pgfusepath{clip}%
\pgfsetbuttcap%
\pgfsetroundjoin%
\definecolor{currentfill}{rgb}{0.956817,0.498820,0.345554}%
\pgfsetfillcolor{currentfill}%
\pgfsetlinewidth{0.311001pt}%
\definecolor{currentstroke}{rgb}{1.000000,1.000000,1.000000}%
\pgfsetstrokecolor{currentstroke}%
\pgfsetdash{}{0pt}%
\pgfpathmoveto{\pgfqpoint{4.351019in}{1.279593in}}%
\pgfpathcurveto{\pgfqpoint{4.358152in}{1.279593in}}{\pgfqpoint{4.364994in}{1.282427in}}{\pgfqpoint{4.370037in}{1.287471in}}%
\pgfpathcurveto{\pgfqpoint{4.375081in}{1.292515in}}{\pgfqpoint{4.377915in}{1.299356in}}{\pgfqpoint{4.377915in}{1.306489in}}%
\pgfpathcurveto{\pgfqpoint{4.377915in}{1.313622in}}{\pgfqpoint{4.375081in}{1.320463in}}{\pgfqpoint{4.370037in}{1.325507in}}%
\pgfpathcurveto{\pgfqpoint{4.364994in}{1.330551in}}{\pgfqpoint{4.358152in}{1.333385in}}{\pgfqpoint{4.351019in}{1.333385in}}%
\pgfpathcurveto{\pgfqpoint{4.343886in}{1.333385in}}{\pgfqpoint{4.337045in}{1.330551in}}{\pgfqpoint{4.332001in}{1.325507in}}%
\pgfpathcurveto{\pgfqpoint{4.326957in}{1.320463in}}{\pgfqpoint{4.324124in}{1.313622in}}{\pgfqpoint{4.324124in}{1.306489in}}%
\pgfpathcurveto{\pgfqpoint{4.324124in}{1.299356in}}{\pgfqpoint{4.326957in}{1.292515in}}{\pgfqpoint{4.332001in}{1.287471in}}%
\pgfpathcurveto{\pgfqpoint{4.337045in}{1.282427in}}{\pgfqpoint{4.343886in}{1.279593in}}{\pgfqpoint{4.351019in}{1.279593in}}%
\pgfpathclose%
\pgfusepath{stroke,fill}%
\end{pgfscope}%
\begin{pgfscope}%
\pgfpathrectangle{\pgfqpoint{2.867647in}{0.500000in}}{\pgfqpoint{1.764706in}{1.700000in}}%
\pgfusepath{clip}%
\pgfsetbuttcap%
\pgfsetroundjoin%
\definecolor{currentfill}{rgb}{0.975644,0.874038,0.797253}%
\pgfsetfillcolor{currentfill}%
\pgfsetlinewidth{0.311001pt}%
\definecolor{currentstroke}{rgb}{1.000000,1.000000,1.000000}%
\pgfsetstrokecolor{currentstroke}%
\pgfsetdash{}{0pt}%
\pgfpathmoveto{\pgfqpoint{4.086388in}{1.619905in}}%
\pgfpathcurveto{\pgfqpoint{4.093520in}{1.619905in}}{\pgfqpoint{4.100362in}{1.622739in}}{\pgfqpoint{4.105406in}{1.627782in}}%
\pgfpathcurveto{\pgfqpoint{4.110449in}{1.632826in}}{\pgfqpoint{4.113283in}{1.639668in}}{\pgfqpoint{4.113283in}{1.646801in}}%
\pgfpathcurveto{\pgfqpoint{4.113283in}{1.653933in}}{\pgfqpoint{4.110449in}{1.660775in}}{\pgfqpoint{4.105406in}{1.665819in}}%
\pgfpathcurveto{\pgfqpoint{4.100362in}{1.670862in}}{\pgfqpoint{4.093520in}{1.673696in}}{\pgfqpoint{4.086388in}{1.673696in}}%
\pgfpathcurveto{\pgfqpoint{4.079255in}{1.673696in}}{\pgfqpoint{4.072413in}{1.670862in}}{\pgfqpoint{4.067369in}{1.665819in}}%
\pgfpathcurveto{\pgfqpoint{4.062326in}{1.660775in}}{\pgfqpoint{4.059492in}{1.653933in}}{\pgfqpoint{4.059492in}{1.646801in}}%
\pgfpathcurveto{\pgfqpoint{4.059492in}{1.639668in}}{\pgfqpoint{4.062326in}{1.632826in}}{\pgfqpoint{4.067369in}{1.627782in}}%
\pgfpathcurveto{\pgfqpoint{4.072413in}{1.622739in}}{\pgfqpoint{4.079255in}{1.619905in}}{\pgfqpoint{4.086388in}{1.619905in}}%
\pgfpathclose%
\pgfusepath{stroke,fill}%
\end{pgfscope}%
\begin{pgfscope}%
\pgfpathrectangle{\pgfqpoint{2.867647in}{0.500000in}}{\pgfqpoint{1.764706in}{1.700000in}}%
\pgfusepath{clip}%
\pgfsetbuttcap%
\pgfsetroundjoin%
\definecolor{currentfill}{rgb}{0.964173,0.657587,0.500469}%
\pgfsetfillcolor{currentfill}%
\pgfsetlinewidth{0.311001pt}%
\definecolor{currentstroke}{rgb}{1.000000,1.000000,1.000000}%
\pgfsetstrokecolor{currentstroke}%
\pgfsetdash{}{0pt}%
\pgfpathmoveto{\pgfqpoint{4.130589in}{0.893094in}}%
\pgfpathcurveto{\pgfqpoint{4.137722in}{0.893094in}}{\pgfqpoint{4.144564in}{0.895928in}}{\pgfqpoint{4.149607in}{0.900972in}}%
\pgfpathcurveto{\pgfqpoint{4.154651in}{0.906016in}}{\pgfqpoint{4.157485in}{0.912857in}}{\pgfqpoint{4.157485in}{0.919990in}}%
\pgfpathcurveto{\pgfqpoint{4.157485in}{0.927123in}}{\pgfqpoint{4.154651in}{0.933965in}}{\pgfqpoint{4.149607in}{0.939008in}}%
\pgfpathcurveto{\pgfqpoint{4.144564in}{0.944052in}}{\pgfqpoint{4.137722in}{0.946886in}}{\pgfqpoint{4.130589in}{0.946886in}}%
\pgfpathcurveto{\pgfqpoint{4.123456in}{0.946886in}}{\pgfqpoint{4.116615in}{0.944052in}}{\pgfqpoint{4.111571in}{0.939008in}}%
\pgfpathcurveto{\pgfqpoint{4.106527in}{0.933965in}}{\pgfqpoint{4.103693in}{0.927123in}}{\pgfqpoint{4.103693in}{0.919990in}}%
\pgfpathcurveto{\pgfqpoint{4.103693in}{0.912857in}}{\pgfqpoint{4.106527in}{0.906016in}}{\pgfqpoint{4.111571in}{0.900972in}}%
\pgfpathcurveto{\pgfqpoint{4.116615in}{0.895928in}}{\pgfqpoint{4.123456in}{0.893094in}}{\pgfqpoint{4.130589in}{0.893094in}}%
\pgfpathclose%
\pgfusepath{stroke,fill}%
\end{pgfscope}%
\begin{pgfscope}%
\pgfpathrectangle{\pgfqpoint{2.867647in}{0.500000in}}{\pgfqpoint{1.764706in}{1.700000in}}%
\pgfusepath{clip}%
\pgfsetbuttcap%
\pgfsetroundjoin%
\definecolor{currentfill}{rgb}{0.955103,0.477872,0.328626}%
\pgfsetfillcolor{currentfill}%
\pgfsetlinewidth{0.311001pt}%
\definecolor{currentstroke}{rgb}{1.000000,1.000000,1.000000}%
\pgfsetstrokecolor{currentstroke}%
\pgfsetdash{}{0pt}%
\pgfpathmoveto{\pgfqpoint{3.904781in}{0.915653in}}%
\pgfpathcurveto{\pgfqpoint{3.911913in}{0.915653in}}{\pgfqpoint{3.918755in}{0.918487in}}{\pgfqpoint{3.923799in}{0.923530in}}%
\pgfpathcurveto{\pgfqpoint{3.928842in}{0.928574in}}{\pgfqpoint{3.931676in}{0.935416in}}{\pgfqpoint{3.931676in}{0.942548in}}%
\pgfpathcurveto{\pgfqpoint{3.931676in}{0.949681in}}{\pgfqpoint{3.928842in}{0.956523in}}{\pgfqpoint{3.923799in}{0.961567in}}%
\pgfpathcurveto{\pgfqpoint{3.918755in}{0.966610in}}{\pgfqpoint{3.911913in}{0.969444in}}{\pgfqpoint{3.904781in}{0.969444in}}%
\pgfpathcurveto{\pgfqpoint{3.897648in}{0.969444in}}{\pgfqpoint{3.890806in}{0.966610in}}{\pgfqpoint{3.885762in}{0.961567in}}%
\pgfpathcurveto{\pgfqpoint{3.880719in}{0.956523in}}{\pgfqpoint{3.877885in}{0.949681in}}{\pgfqpoint{3.877885in}{0.942548in}}%
\pgfpathcurveto{\pgfqpoint{3.877885in}{0.935416in}}{\pgfqpoint{3.880719in}{0.928574in}}{\pgfqpoint{3.885762in}{0.923530in}}%
\pgfpathcurveto{\pgfqpoint{3.890806in}{0.918487in}}{\pgfqpoint{3.897648in}{0.915653in}}{\pgfqpoint{3.904781in}{0.915653in}}%
\pgfpathclose%
\pgfusepath{stroke,fill}%
\end{pgfscope}%
\begin{pgfscope}%
\pgfpathrectangle{\pgfqpoint{2.867647in}{0.500000in}}{\pgfqpoint{1.764706in}{1.700000in}}%
\pgfusepath{clip}%
\pgfsetbuttcap%
\pgfsetroundjoin%
\definecolor{currentfill}{rgb}{0.973832,0.856556,0.771584}%
\pgfsetfillcolor{currentfill}%
\pgfsetlinewidth{0.311001pt}%
\definecolor{currentstroke}{rgb}{1.000000,1.000000,1.000000}%
\pgfsetstrokecolor{currentstroke}%
\pgfsetdash{}{0pt}%
\pgfpathmoveto{\pgfqpoint{4.220390in}{1.109562in}}%
\pgfpathcurveto{\pgfqpoint{4.227523in}{1.109562in}}{\pgfqpoint{4.234365in}{1.112396in}}{\pgfqpoint{4.239408in}{1.117440in}}%
\pgfpathcurveto{\pgfqpoint{4.244452in}{1.122483in}}{\pgfqpoint{4.247286in}{1.129325in}}{\pgfqpoint{4.247286in}{1.136458in}}%
\pgfpathcurveto{\pgfqpoint{4.247286in}{1.143591in}}{\pgfqpoint{4.244452in}{1.150432in}}{\pgfqpoint{4.239408in}{1.155476in}}%
\pgfpathcurveto{\pgfqpoint{4.234365in}{1.160520in}}{\pgfqpoint{4.227523in}{1.163353in}}{\pgfqpoint{4.220390in}{1.163353in}}%
\pgfpathcurveto{\pgfqpoint{4.213257in}{1.163353in}}{\pgfqpoint{4.206416in}{1.160520in}}{\pgfqpoint{4.201372in}{1.155476in}}%
\pgfpathcurveto{\pgfqpoint{4.196328in}{1.150432in}}{\pgfqpoint{4.193494in}{1.143591in}}{\pgfqpoint{4.193494in}{1.136458in}}%
\pgfpathcurveto{\pgfqpoint{4.193494in}{1.129325in}}{\pgfqpoint{4.196328in}{1.122483in}}{\pgfqpoint{4.201372in}{1.117440in}}%
\pgfpathcurveto{\pgfqpoint{4.206416in}{1.112396in}}{\pgfqpoint{4.213257in}{1.109562in}}{\pgfqpoint{4.220390in}{1.109562in}}%
\pgfpathclose%
\pgfusepath{stroke,fill}%
\end{pgfscope}%
\begin{pgfscope}%
\pgfpathrectangle{\pgfqpoint{2.867647in}{0.500000in}}{\pgfqpoint{1.764706in}{1.700000in}}%
\pgfusepath{clip}%
\pgfsetbuttcap%
\pgfsetroundjoin%
\definecolor{currentfill}{rgb}{0.917171,0.267738,0.242941}%
\pgfsetfillcolor{currentfill}%
\pgfsetlinewidth{0.311001pt}%
\definecolor{currentstroke}{rgb}{1.000000,1.000000,1.000000}%
\pgfsetstrokecolor{currentstroke}%
\pgfsetdash{}{0pt}%
\pgfpathmoveto{\pgfqpoint{3.917424in}{1.089366in}}%
\pgfpathcurveto{\pgfqpoint{3.924557in}{1.089366in}}{\pgfqpoint{3.931398in}{1.092200in}}{\pgfqpoint{3.936442in}{1.097243in}}%
\pgfpathcurveto{\pgfqpoint{3.941486in}{1.102287in}}{\pgfqpoint{3.944320in}{1.109129in}}{\pgfqpoint{3.944320in}{1.116261in}}%
\pgfpathcurveto{\pgfqpoint{3.944320in}{1.123394in}}{\pgfqpoint{3.941486in}{1.130236in}}{\pgfqpoint{3.936442in}{1.135280in}}%
\pgfpathcurveto{\pgfqpoint{3.931398in}{1.140323in}}{\pgfqpoint{3.924557in}{1.143157in}}{\pgfqpoint{3.917424in}{1.143157in}}%
\pgfpathcurveto{\pgfqpoint{3.910291in}{1.143157in}}{\pgfqpoint{3.903449in}{1.140323in}}{\pgfqpoint{3.898406in}{1.135280in}}%
\pgfpathcurveto{\pgfqpoint{3.893362in}{1.130236in}}{\pgfqpoint{3.890528in}{1.123394in}}{\pgfqpoint{3.890528in}{1.116261in}}%
\pgfpathcurveto{\pgfqpoint{3.890528in}{1.109129in}}{\pgfqpoint{3.893362in}{1.102287in}}{\pgfqpoint{3.898406in}{1.097243in}}%
\pgfpathcurveto{\pgfqpoint{3.903449in}{1.092200in}}{\pgfqpoint{3.910291in}{1.089366in}}{\pgfqpoint{3.917424in}{1.089366in}}%
\pgfpathclose%
\pgfusepath{stroke,fill}%
\end{pgfscope}%
\begin{pgfscope}%
\pgfpathrectangle{\pgfqpoint{2.867647in}{0.500000in}}{\pgfqpoint{1.764706in}{1.700000in}}%
\pgfusepath{clip}%
\pgfsetbuttcap%
\pgfsetroundjoin%
\definecolor{currentfill}{rgb}{0.976961,0.885681,0.814303}%
\pgfsetfillcolor{currentfill}%
\pgfsetlinewidth{0.311001pt}%
\definecolor{currentstroke}{rgb}{1.000000,1.000000,1.000000}%
\pgfsetstrokecolor{currentstroke}%
\pgfsetdash{}{0pt}%
\pgfpathmoveto{\pgfqpoint{4.135229in}{1.039912in}}%
\pgfpathcurveto{\pgfqpoint{4.142362in}{1.039912in}}{\pgfqpoint{4.149204in}{1.042746in}}{\pgfqpoint{4.154247in}{1.047790in}}%
\pgfpathcurveto{\pgfqpoint{4.159291in}{1.052834in}}{\pgfqpoint{4.162125in}{1.059675in}}{\pgfqpoint{4.162125in}{1.066808in}}%
\pgfpathcurveto{\pgfqpoint{4.162125in}{1.073941in}}{\pgfqpoint{4.159291in}{1.080782in}}{\pgfqpoint{4.154247in}{1.085826in}}%
\pgfpathcurveto{\pgfqpoint{4.149204in}{1.090870in}}{\pgfqpoint{4.142362in}{1.093704in}}{\pgfqpoint{4.135229in}{1.093704in}}%
\pgfpathcurveto{\pgfqpoint{4.128097in}{1.093704in}}{\pgfqpoint{4.121255in}{1.090870in}}{\pgfqpoint{4.116211in}{1.085826in}}%
\pgfpathcurveto{\pgfqpoint{4.111168in}{1.080782in}}{\pgfqpoint{4.108334in}{1.073941in}}{\pgfqpoint{4.108334in}{1.066808in}}%
\pgfpathcurveto{\pgfqpoint{4.108334in}{1.059675in}}{\pgfqpoint{4.111168in}{1.052834in}}{\pgfqpoint{4.116211in}{1.047790in}}%
\pgfpathcurveto{\pgfqpoint{4.121255in}{1.042746in}}{\pgfqpoint{4.128097in}{1.039912in}}{\pgfqpoint{4.135229in}{1.039912in}}%
\pgfpathclose%
\pgfusepath{stroke,fill}%
\end{pgfscope}%
\begin{pgfscope}%
\pgfpathrectangle{\pgfqpoint{2.867647in}{0.500000in}}{\pgfqpoint{1.764706in}{1.700000in}}%
\pgfusepath{clip}%
\pgfsetbuttcap%
\pgfsetroundjoin%
\definecolor{currentfill}{rgb}{0.979891,0.908948,0.848279}%
\pgfsetfillcolor{currentfill}%
\pgfsetlinewidth{0.311001pt}%
\definecolor{currentstroke}{rgb}{1.000000,1.000000,1.000000}%
\pgfsetstrokecolor{currentstroke}%
\pgfsetdash{}{0pt}%
\pgfpathmoveto{\pgfqpoint{4.160039in}{1.306083in}}%
\pgfpathcurveto{\pgfqpoint{4.167172in}{1.306083in}}{\pgfqpoint{4.174014in}{1.308917in}}{\pgfqpoint{4.179057in}{1.313960in}}%
\pgfpathcurveto{\pgfqpoint{4.184101in}{1.319004in}}{\pgfqpoint{4.186935in}{1.325846in}}{\pgfqpoint{4.186935in}{1.332978in}}%
\pgfpathcurveto{\pgfqpoint{4.186935in}{1.340111in}}{\pgfqpoint{4.184101in}{1.346953in}}{\pgfqpoint{4.179057in}{1.351997in}}%
\pgfpathcurveto{\pgfqpoint{4.174014in}{1.357040in}}{\pgfqpoint{4.167172in}{1.359874in}}{\pgfqpoint{4.160039in}{1.359874in}}%
\pgfpathcurveto{\pgfqpoint{4.152906in}{1.359874in}}{\pgfqpoint{4.146065in}{1.357040in}}{\pgfqpoint{4.141021in}{1.351997in}}%
\pgfpathcurveto{\pgfqpoint{4.135977in}{1.346953in}}{\pgfqpoint{4.133144in}{1.340111in}}{\pgfqpoint{4.133144in}{1.332978in}}%
\pgfpathcurveto{\pgfqpoint{4.133144in}{1.325846in}}{\pgfqpoint{4.135977in}{1.319004in}}{\pgfqpoint{4.141021in}{1.313960in}}%
\pgfpathcurveto{\pgfqpoint{4.146065in}{1.308917in}}{\pgfqpoint{4.152906in}{1.306083in}}{\pgfqpoint{4.160039in}{1.306083in}}%
\pgfpathclose%
\pgfusepath{stroke,fill}%
\end{pgfscope}%
\begin{pgfscope}%
\pgfpathrectangle{\pgfqpoint{2.867647in}{0.500000in}}{\pgfqpoint{1.764706in}{1.700000in}}%
\pgfusepath{clip}%
\pgfsetbuttcap%
\pgfsetroundjoin%
\definecolor{currentfill}{rgb}{0.969803,0.809811,0.702523}%
\pgfsetfillcolor{currentfill}%
\pgfsetlinewidth{0.311001pt}%
\definecolor{currentstroke}{rgb}{1.000000,1.000000,1.000000}%
\pgfsetstrokecolor{currentstroke}%
\pgfsetdash{}{0pt}%
\pgfpathmoveto{\pgfqpoint{4.070003in}{1.127392in}}%
\pgfpathcurveto{\pgfqpoint{4.077136in}{1.127392in}}{\pgfqpoint{4.083978in}{1.130226in}}{\pgfqpoint{4.089021in}{1.135270in}}%
\pgfpathcurveto{\pgfqpoint{4.094065in}{1.140314in}}{\pgfqpoint{4.096899in}{1.147155in}}{\pgfqpoint{4.096899in}{1.154288in}}%
\pgfpathcurveto{\pgfqpoint{4.096899in}{1.161421in}}{\pgfqpoint{4.094065in}{1.168263in}}{\pgfqpoint{4.089021in}{1.173306in}}%
\pgfpathcurveto{\pgfqpoint{4.083978in}{1.178350in}}{\pgfqpoint{4.077136in}{1.181184in}}{\pgfqpoint{4.070003in}{1.181184in}}%
\pgfpathcurveto{\pgfqpoint{4.062870in}{1.181184in}}{\pgfqpoint{4.056029in}{1.178350in}}{\pgfqpoint{4.050985in}{1.173306in}}%
\pgfpathcurveto{\pgfqpoint{4.045941in}{1.168263in}}{\pgfqpoint{4.043107in}{1.161421in}}{\pgfqpoint{4.043107in}{1.154288in}}%
\pgfpathcurveto{\pgfqpoint{4.043107in}{1.147155in}}{\pgfqpoint{4.045941in}{1.140314in}}{\pgfqpoint{4.050985in}{1.135270in}}%
\pgfpathcurveto{\pgfqpoint{4.056029in}{1.130226in}}{\pgfqpoint{4.062870in}{1.127392in}}{\pgfqpoint{4.070003in}{1.127392in}}%
\pgfpathclose%
\pgfusepath{stroke,fill}%
\end{pgfscope}%
\begin{pgfscope}%
\pgfpathrectangle{\pgfqpoint{2.867647in}{0.500000in}}{\pgfqpoint{1.764706in}{1.700000in}}%
\pgfusepath{clip}%
\pgfsetbuttcap%
\pgfsetroundjoin%
\definecolor{currentfill}{rgb}{0.981377,0.920617,0.865369}%
\pgfsetfillcolor{currentfill}%
\pgfsetlinewidth{0.311001pt}%
\definecolor{currentstroke}{rgb}{1.000000,1.000000,1.000000}%
\pgfsetstrokecolor{currentstroke}%
\pgfsetdash{}{0pt}%
\pgfpathmoveto{\pgfqpoint{4.184847in}{1.317950in}}%
\pgfpathcurveto{\pgfqpoint{4.191980in}{1.317950in}}{\pgfqpoint{4.198822in}{1.320784in}}{\pgfqpoint{4.203865in}{1.325828in}}%
\pgfpathcurveto{\pgfqpoint{4.208909in}{1.330871in}}{\pgfqpoint{4.211743in}{1.337713in}}{\pgfqpoint{4.211743in}{1.344846in}}%
\pgfpathcurveto{\pgfqpoint{4.211743in}{1.351979in}}{\pgfqpoint{4.208909in}{1.358820in}}{\pgfqpoint{4.203865in}{1.363864in}}%
\pgfpathcurveto{\pgfqpoint{4.198822in}{1.368908in}}{\pgfqpoint{4.191980in}{1.371741in}}{\pgfqpoint{4.184847in}{1.371741in}}%
\pgfpathcurveto{\pgfqpoint{4.177715in}{1.371741in}}{\pgfqpoint{4.170873in}{1.368908in}}{\pgfqpoint{4.165829in}{1.363864in}}%
\pgfpathcurveto{\pgfqpoint{4.160786in}{1.358820in}}{\pgfqpoint{4.157952in}{1.351979in}}{\pgfqpoint{4.157952in}{1.344846in}}%
\pgfpathcurveto{\pgfqpoint{4.157952in}{1.337713in}}{\pgfqpoint{4.160786in}{1.330871in}}{\pgfqpoint{4.165829in}{1.325828in}}%
\pgfpathcurveto{\pgfqpoint{4.170873in}{1.320784in}}{\pgfqpoint{4.177715in}{1.317950in}}{\pgfqpoint{4.184847in}{1.317950in}}%
\pgfpathclose%
\pgfusepath{stroke,fill}%
\end{pgfscope}%
\begin{pgfscope}%
\pgfpathrectangle{\pgfqpoint{2.867647in}{0.500000in}}{\pgfqpoint{1.764706in}{1.700000in}}%
\pgfusepath{clip}%
\pgfsetbuttcap%
\pgfsetroundjoin%
\definecolor{currentfill}{rgb}{0.978376,0.897317,0.831308}%
\pgfsetfillcolor{currentfill}%
\pgfsetlinewidth{0.311001pt}%
\definecolor{currentstroke}{rgb}{1.000000,1.000000,1.000000}%
\pgfsetstrokecolor{currentstroke}%
\pgfsetdash{}{0pt}%
\pgfpathmoveto{\pgfqpoint{4.198638in}{1.492357in}}%
\pgfpathcurveto{\pgfqpoint{4.205771in}{1.492357in}}{\pgfqpoint{4.212613in}{1.495191in}}{\pgfqpoint{4.217656in}{1.500235in}}%
\pgfpathcurveto{\pgfqpoint{4.222700in}{1.505278in}}{\pgfqpoint{4.225534in}{1.512120in}}{\pgfqpoint{4.225534in}{1.519253in}}%
\pgfpathcurveto{\pgfqpoint{4.225534in}{1.526386in}}{\pgfqpoint{4.222700in}{1.533227in}}{\pgfqpoint{4.217656in}{1.538271in}}%
\pgfpathcurveto{\pgfqpoint{4.212613in}{1.543315in}}{\pgfqpoint{4.205771in}{1.546148in}}{\pgfqpoint{4.198638in}{1.546148in}}%
\pgfpathcurveto{\pgfqpoint{4.191505in}{1.546148in}}{\pgfqpoint{4.184664in}{1.543315in}}{\pgfqpoint{4.179620in}{1.538271in}}%
\pgfpathcurveto{\pgfqpoint{4.174577in}{1.533227in}}{\pgfqpoint{4.171743in}{1.526386in}}{\pgfqpoint{4.171743in}{1.519253in}}%
\pgfpathcurveto{\pgfqpoint{4.171743in}{1.512120in}}{\pgfqpoint{4.174577in}{1.505278in}}{\pgfqpoint{4.179620in}{1.500235in}}%
\pgfpathcurveto{\pgfqpoint{4.184664in}{1.495191in}}{\pgfqpoint{4.191505in}{1.492357in}}{\pgfqpoint{4.198638in}{1.492357in}}%
\pgfpathclose%
\pgfusepath{stroke,fill}%
\end{pgfscope}%
\begin{pgfscope}%
\pgfpathrectangle{\pgfqpoint{2.867647in}{0.500000in}}{\pgfqpoint{1.764706in}{1.700000in}}%
\pgfusepath{clip}%
\pgfsetbuttcap%
\pgfsetroundjoin%
\definecolor{currentfill}{rgb}{0.979891,0.908948,0.848279}%
\pgfsetfillcolor{currentfill}%
\pgfsetlinewidth{0.311001pt}%
\definecolor{currentstroke}{rgb}{1.000000,1.000000,1.000000}%
\pgfsetstrokecolor{currentstroke}%
\pgfsetdash{}{0pt}%
\pgfpathmoveto{\pgfqpoint{4.155349in}{1.453939in}}%
\pgfpathcurveto{\pgfqpoint{4.162482in}{1.453939in}}{\pgfqpoint{4.169324in}{1.456773in}}{\pgfqpoint{4.174368in}{1.461816in}}%
\pgfpathcurveto{\pgfqpoint{4.179411in}{1.466860in}}{\pgfqpoint{4.182245in}{1.473702in}}{\pgfqpoint{4.182245in}{1.480835in}}%
\pgfpathcurveto{\pgfqpoint{4.182245in}{1.487967in}}{\pgfqpoint{4.179411in}{1.494809in}}{\pgfqpoint{4.174368in}{1.499853in}}%
\pgfpathcurveto{\pgfqpoint{4.169324in}{1.504896in}}{\pgfqpoint{4.162482in}{1.507730in}}{\pgfqpoint{4.155349in}{1.507730in}}%
\pgfpathcurveto{\pgfqpoint{4.148217in}{1.507730in}}{\pgfqpoint{4.141375in}{1.504896in}}{\pgfqpoint{4.136331in}{1.499853in}}%
\pgfpathcurveto{\pgfqpoint{4.131288in}{1.494809in}}{\pgfqpoint{4.128454in}{1.487967in}}{\pgfqpoint{4.128454in}{1.480835in}}%
\pgfpathcurveto{\pgfqpoint{4.128454in}{1.473702in}}{\pgfqpoint{4.131288in}{1.466860in}}{\pgfqpoint{4.136331in}{1.461816in}}%
\pgfpathcurveto{\pgfqpoint{4.141375in}{1.456773in}}{\pgfqpoint{4.148217in}{1.453939in}}{\pgfqpoint{4.155349in}{1.453939in}}%
\pgfpathclose%
\pgfusepath{stroke,fill}%
\end{pgfscope}%
\begin{pgfscope}%
\pgfpathrectangle{\pgfqpoint{2.867647in}{0.500000in}}{\pgfqpoint{1.764706in}{1.700000in}}%
\pgfusepath{clip}%
\pgfsetbuttcap%
\pgfsetroundjoin%
\definecolor{currentfill}{rgb}{0.973271,0.850724,0.762998}%
\pgfsetfillcolor{currentfill}%
\pgfsetlinewidth{0.311001pt}%
\definecolor{currentstroke}{rgb}{1.000000,1.000000,1.000000}%
\pgfsetstrokecolor{currentstroke}%
\pgfsetdash{}{0pt}%
\pgfpathmoveto{\pgfqpoint{4.182857in}{1.040671in}}%
\pgfpathcurveto{\pgfqpoint{4.189989in}{1.040671in}}{\pgfqpoint{4.196831in}{1.043505in}}{\pgfqpoint{4.201875in}{1.048549in}}%
\pgfpathcurveto{\pgfqpoint{4.206918in}{1.053593in}}{\pgfqpoint{4.209752in}{1.060434in}}{\pgfqpoint{4.209752in}{1.067567in}}%
\pgfpathcurveto{\pgfqpoint{4.209752in}{1.074700in}}{\pgfqpoint{4.206918in}{1.081542in}}{\pgfqpoint{4.201875in}{1.086585in}}%
\pgfpathcurveto{\pgfqpoint{4.196831in}{1.091629in}}{\pgfqpoint{4.189989in}{1.094463in}}{\pgfqpoint{4.182857in}{1.094463in}}%
\pgfpathcurveto{\pgfqpoint{4.175724in}{1.094463in}}{\pgfqpoint{4.168882in}{1.091629in}}{\pgfqpoint{4.163838in}{1.086585in}}%
\pgfpathcurveto{\pgfqpoint{4.158795in}{1.081542in}}{\pgfqpoint{4.155961in}{1.074700in}}{\pgfqpoint{4.155961in}{1.067567in}}%
\pgfpathcurveto{\pgfqpoint{4.155961in}{1.060434in}}{\pgfqpoint{4.158795in}{1.053593in}}{\pgfqpoint{4.163838in}{1.048549in}}%
\pgfpathcurveto{\pgfqpoint{4.168882in}{1.043505in}}{\pgfqpoint{4.175724in}{1.040671in}}{\pgfqpoint{4.182857in}{1.040671in}}%
\pgfpathclose%
\pgfusepath{stroke,fill}%
\end{pgfscope}%
\begin{pgfscope}%
\pgfpathrectangle{\pgfqpoint{2.867647in}{0.500000in}}{\pgfqpoint{1.764706in}{1.700000in}}%
\pgfusepath{clip}%
\pgfsetbuttcap%
\pgfsetroundjoin%
\definecolor{currentfill}{rgb}{0.981377,0.920617,0.865369}%
\pgfsetfillcolor{currentfill}%
\pgfsetlinewidth{0.311001pt}%
\definecolor{currentstroke}{rgb}{1.000000,1.000000,1.000000}%
\pgfsetstrokecolor{currentstroke}%
\pgfsetdash{}{0pt}%
\pgfpathmoveto{\pgfqpoint{4.168852in}{1.199299in}}%
\pgfpathcurveto{\pgfqpoint{4.175985in}{1.199299in}}{\pgfqpoint{4.182826in}{1.202133in}}{\pgfqpoint{4.187870in}{1.207177in}}%
\pgfpathcurveto{\pgfqpoint{4.192914in}{1.212221in}}{\pgfqpoint{4.195748in}{1.219062in}}{\pgfqpoint{4.195748in}{1.226195in}}%
\pgfpathcurveto{\pgfqpoint{4.195748in}{1.233328in}}{\pgfqpoint{4.192914in}{1.240170in}}{\pgfqpoint{4.187870in}{1.245213in}}%
\pgfpathcurveto{\pgfqpoint{4.182826in}{1.250257in}}{\pgfqpoint{4.175985in}{1.253091in}}{\pgfqpoint{4.168852in}{1.253091in}}%
\pgfpathcurveto{\pgfqpoint{4.161719in}{1.253091in}}{\pgfqpoint{4.154877in}{1.250257in}}{\pgfqpoint{4.149834in}{1.245213in}}%
\pgfpathcurveto{\pgfqpoint{4.144790in}{1.240170in}}{\pgfqpoint{4.141956in}{1.233328in}}{\pgfqpoint{4.141956in}{1.226195in}}%
\pgfpathcurveto{\pgfqpoint{4.141956in}{1.219062in}}{\pgfqpoint{4.144790in}{1.212221in}}{\pgfqpoint{4.149834in}{1.207177in}}%
\pgfpathcurveto{\pgfqpoint{4.154877in}{1.202133in}}{\pgfqpoint{4.161719in}{1.199299in}}{\pgfqpoint{4.168852in}{1.199299in}}%
\pgfpathclose%
\pgfusepath{stroke,fill}%
\end{pgfscope}%
\begin{pgfscope}%
\pgfpathrectangle{\pgfqpoint{2.867647in}{0.500000in}}{\pgfqpoint{1.764706in}{1.700000in}}%
\pgfusepath{clip}%
\pgfsetbuttcap%
\pgfsetroundjoin%
\definecolor{currentfill}{rgb}{0.971202,0.827364,0.728520}%
\pgfsetfillcolor{currentfill}%
\pgfsetlinewidth{0.311001pt}%
\definecolor{currentstroke}{rgb}{1.000000,1.000000,1.000000}%
\pgfsetstrokecolor{currentstroke}%
\pgfsetdash{}{0pt}%
\pgfpathmoveto{\pgfqpoint{4.084292in}{0.961172in}}%
\pgfpathcurveto{\pgfqpoint{4.091425in}{0.961172in}}{\pgfqpoint{4.098267in}{0.964006in}}{\pgfqpoint{4.103310in}{0.969050in}}%
\pgfpathcurveto{\pgfqpoint{4.108354in}{0.974093in}}{\pgfqpoint{4.111188in}{0.980935in}}{\pgfqpoint{4.111188in}{0.988068in}}%
\pgfpathcurveto{\pgfqpoint{4.111188in}{0.995201in}}{\pgfqpoint{4.108354in}{1.002042in}}{\pgfqpoint{4.103310in}{1.007086in}}%
\pgfpathcurveto{\pgfqpoint{4.098267in}{1.012130in}}{\pgfqpoint{4.091425in}{1.014964in}}{\pgfqpoint{4.084292in}{1.014964in}}%
\pgfpathcurveto{\pgfqpoint{4.077159in}{1.014964in}}{\pgfqpoint{4.070318in}{1.012130in}}{\pgfqpoint{4.065274in}{1.007086in}}%
\pgfpathcurveto{\pgfqpoint{4.060230in}{1.002042in}}{\pgfqpoint{4.057397in}{0.995201in}}{\pgfqpoint{4.057397in}{0.988068in}}%
\pgfpathcurveto{\pgfqpoint{4.057397in}{0.980935in}}{\pgfqpoint{4.060230in}{0.974093in}}{\pgfqpoint{4.065274in}{0.969050in}}%
\pgfpathcurveto{\pgfqpoint{4.070318in}{0.964006in}}{\pgfqpoint{4.077159in}{0.961172in}}{\pgfqpoint{4.084292in}{0.961172in}}%
\pgfpathclose%
\pgfusepath{stroke,fill}%
\end{pgfscope}%
\begin{pgfscope}%
\pgfpathrectangle{\pgfqpoint{2.867647in}{0.500000in}}{\pgfqpoint{1.764706in}{1.700000in}}%
\pgfusepath{clip}%
\pgfsetbuttcap%
\pgfsetroundjoin%
\definecolor{currentfill}{rgb}{0.802060,0.108583,0.307830}%
\pgfsetfillcolor{currentfill}%
\pgfsetlinewidth{0.311001pt}%
\definecolor{currentstroke}{rgb}{1.000000,1.000000,1.000000}%
\pgfsetstrokecolor{currentstroke}%
\pgfsetdash{}{0pt}%
\pgfpathmoveto{\pgfqpoint{3.784646in}{0.829579in}}%
\pgfpathcurveto{\pgfqpoint{3.791779in}{0.829579in}}{\pgfqpoint{3.798620in}{0.832413in}}{\pgfqpoint{3.803664in}{0.837456in}}%
\pgfpathcurveto{\pgfqpoint{3.808708in}{0.842500in}}{\pgfqpoint{3.811542in}{0.849342in}}{\pgfqpoint{3.811542in}{0.856474in}}%
\pgfpathcurveto{\pgfqpoint{3.811542in}{0.863607in}}{\pgfqpoint{3.808708in}{0.870449in}}{\pgfqpoint{3.803664in}{0.875493in}}%
\pgfpathcurveto{\pgfqpoint{3.798620in}{0.880536in}}{\pgfqpoint{3.791779in}{0.883370in}}{\pgfqpoint{3.784646in}{0.883370in}}%
\pgfpathcurveto{\pgfqpoint{3.777513in}{0.883370in}}{\pgfqpoint{3.770672in}{0.880536in}}{\pgfqpoint{3.765628in}{0.875493in}}%
\pgfpathcurveto{\pgfqpoint{3.760584in}{0.870449in}}{\pgfqpoint{3.757750in}{0.863607in}}{\pgfqpoint{3.757750in}{0.856474in}}%
\pgfpathcurveto{\pgfqpoint{3.757750in}{0.849342in}}{\pgfqpoint{3.760584in}{0.842500in}}{\pgfqpoint{3.765628in}{0.837456in}}%
\pgfpathcurveto{\pgfqpoint{3.770672in}{0.832413in}}{\pgfqpoint{3.777513in}{0.829579in}}{\pgfqpoint{3.784646in}{0.829579in}}%
\pgfpathclose%
\pgfusepath{stroke,fill}%
\end{pgfscope}%
\begin{pgfscope}%
\pgfpathrectangle{\pgfqpoint{2.867647in}{0.500000in}}{\pgfqpoint{1.764706in}{1.700000in}}%
\pgfusepath{clip}%
\pgfsetbuttcap%
\pgfsetroundjoin%
\definecolor{currentfill}{rgb}{0.964558,0.676556,0.522514}%
\pgfsetfillcolor{currentfill}%
\pgfsetlinewidth{0.311001pt}%
\definecolor{currentstroke}{rgb}{1.000000,1.000000,1.000000}%
\pgfsetstrokecolor{currentstroke}%
\pgfsetdash{}{0pt}%
\pgfpathmoveto{\pgfqpoint{4.315269in}{1.283141in}}%
\pgfpathcurveto{\pgfqpoint{4.322402in}{1.283141in}}{\pgfqpoint{4.329244in}{1.285975in}}{\pgfqpoint{4.334287in}{1.291018in}}%
\pgfpathcurveto{\pgfqpoint{4.339331in}{1.296062in}}{\pgfqpoint{4.342165in}{1.302904in}}{\pgfqpoint{4.342165in}{1.310036in}}%
\pgfpathcurveto{\pgfqpoint{4.342165in}{1.317169in}}{\pgfqpoint{4.339331in}{1.324011in}}{\pgfqpoint{4.334287in}{1.329055in}}%
\pgfpathcurveto{\pgfqpoint{4.329244in}{1.334098in}}{\pgfqpoint{4.322402in}{1.336932in}}{\pgfqpoint{4.315269in}{1.336932in}}%
\pgfpathcurveto{\pgfqpoint{4.308136in}{1.336932in}}{\pgfqpoint{4.301295in}{1.334098in}}{\pgfqpoint{4.296251in}{1.329055in}}%
\pgfpathcurveto{\pgfqpoint{4.291207in}{1.324011in}}{\pgfqpoint{4.288373in}{1.317169in}}{\pgfqpoint{4.288373in}{1.310036in}}%
\pgfpathcurveto{\pgfqpoint{4.288373in}{1.302904in}}{\pgfqpoint{4.291207in}{1.296062in}}{\pgfqpoint{4.296251in}{1.291018in}}%
\pgfpathcurveto{\pgfqpoint{4.301295in}{1.285975in}}{\pgfqpoint{4.308136in}{1.283141in}}{\pgfqpoint{4.315269in}{1.283141in}}%
\pgfpathclose%
\pgfusepath{stroke,fill}%
\end{pgfscope}%
\begin{pgfscope}%
\pgfpathrectangle{\pgfqpoint{2.867647in}{0.500000in}}{\pgfqpoint{1.764706in}{1.700000in}}%
\pgfusepath{clip}%
\pgfsetbuttcap%
\pgfsetroundjoin%
\definecolor{currentfill}{rgb}{0.981377,0.920617,0.865369}%
\pgfsetfillcolor{currentfill}%
\pgfsetlinewidth{0.311001pt}%
\definecolor{currentstroke}{rgb}{1.000000,1.000000,1.000000}%
\pgfsetstrokecolor{currentstroke}%
\pgfsetdash{}{0pt}%
\pgfpathmoveto{\pgfqpoint{4.191071in}{1.262022in}}%
\pgfpathcurveto{\pgfqpoint{4.198204in}{1.262022in}}{\pgfqpoint{4.205046in}{1.264856in}}{\pgfqpoint{4.210089in}{1.269899in}}%
\pgfpathcurveto{\pgfqpoint{4.215133in}{1.274943in}}{\pgfqpoint{4.217967in}{1.281785in}}{\pgfqpoint{4.217967in}{1.288918in}}%
\pgfpathcurveto{\pgfqpoint{4.217967in}{1.296050in}}{\pgfqpoint{4.215133in}{1.302892in}}{\pgfqpoint{4.210089in}{1.307936in}}%
\pgfpathcurveto{\pgfqpoint{4.205046in}{1.312979in}}{\pgfqpoint{4.198204in}{1.315813in}}{\pgfqpoint{4.191071in}{1.315813in}}%
\pgfpathcurveto{\pgfqpoint{4.183939in}{1.315813in}}{\pgfqpoint{4.177097in}{1.312979in}}{\pgfqpoint{4.172053in}{1.307936in}}%
\pgfpathcurveto{\pgfqpoint{4.167010in}{1.302892in}}{\pgfqpoint{4.164176in}{1.296050in}}{\pgfqpoint{4.164176in}{1.288918in}}%
\pgfpathcurveto{\pgfqpoint{4.164176in}{1.281785in}}{\pgfqpoint{4.167010in}{1.274943in}}{\pgfqpoint{4.172053in}{1.269899in}}%
\pgfpathcurveto{\pgfqpoint{4.177097in}{1.264856in}}{\pgfqpoint{4.183939in}{1.262022in}}{\pgfqpoint{4.191071in}{1.262022in}}%
\pgfpathclose%
\pgfusepath{stroke,fill}%
\end{pgfscope}%
\begin{pgfscope}%
\pgfpathrectangle{\pgfqpoint{2.867647in}{0.500000in}}{\pgfqpoint{1.764706in}{1.700000in}}%
\pgfusepath{clip}%
\pgfsetbuttcap%
\pgfsetroundjoin%
\definecolor{currentfill}{rgb}{0.971694,0.833208,0.737161}%
\pgfsetfillcolor{currentfill}%
\pgfsetlinewidth{0.311001pt}%
\definecolor{currentstroke}{rgb}{1.000000,1.000000,1.000000}%
\pgfsetstrokecolor{currentstroke}%
\pgfsetdash{}{0pt}%
\pgfpathmoveto{\pgfqpoint{4.149484in}{1.657039in}}%
\pgfpathcurveto{\pgfqpoint{4.156617in}{1.657039in}}{\pgfqpoint{4.163458in}{1.659873in}}{\pgfqpoint{4.168502in}{1.664916in}}%
\pgfpathcurveto{\pgfqpoint{4.173546in}{1.669960in}}{\pgfqpoint{4.176379in}{1.676802in}}{\pgfqpoint{4.176379in}{1.683935in}}%
\pgfpathcurveto{\pgfqpoint{4.176379in}{1.691067in}}{\pgfqpoint{4.173546in}{1.697909in}}{\pgfqpoint{4.168502in}{1.702953in}}%
\pgfpathcurveto{\pgfqpoint{4.163458in}{1.707996in}}{\pgfqpoint{4.156617in}{1.710830in}}{\pgfqpoint{4.149484in}{1.710830in}}%
\pgfpathcurveto{\pgfqpoint{4.142351in}{1.710830in}}{\pgfqpoint{4.135509in}{1.707996in}}{\pgfqpoint{4.130466in}{1.702953in}}%
\pgfpathcurveto{\pgfqpoint{4.125422in}{1.697909in}}{\pgfqpoint{4.122588in}{1.691067in}}{\pgfqpoint{4.122588in}{1.683935in}}%
\pgfpathcurveto{\pgfqpoint{4.122588in}{1.676802in}}{\pgfqpoint{4.125422in}{1.669960in}}{\pgfqpoint{4.130466in}{1.664916in}}%
\pgfpathcurveto{\pgfqpoint{4.135509in}{1.659873in}}{\pgfqpoint{4.142351in}{1.657039in}}{\pgfqpoint{4.149484in}{1.657039in}}%
\pgfpathclose%
\pgfusepath{stroke,fill}%
\end{pgfscope}%
\begin{pgfscope}%
\pgfpathrectangle{\pgfqpoint{2.867647in}{0.500000in}}{\pgfqpoint{1.764706in}{1.700000in}}%
\pgfusepath{clip}%
\pgfsetbuttcap%
\pgfsetroundjoin%
\definecolor{currentfill}{rgb}{0.917171,0.267738,0.242941}%
\pgfsetfillcolor{currentfill}%
\pgfsetlinewidth{0.311001pt}%
\definecolor{currentstroke}{rgb}{1.000000,1.000000,1.000000}%
\pgfsetstrokecolor{currentstroke}%
\pgfsetdash{}{0pt}%
\pgfpathmoveto{\pgfqpoint{4.376327in}{1.206988in}}%
\pgfpathcurveto{\pgfqpoint{4.383460in}{1.206988in}}{\pgfqpoint{4.390302in}{1.209822in}}{\pgfqpoint{4.395345in}{1.214865in}}%
\pgfpathcurveto{\pgfqpoint{4.400389in}{1.219909in}}{\pgfqpoint{4.403223in}{1.226751in}}{\pgfqpoint{4.403223in}{1.233883in}}%
\pgfpathcurveto{\pgfqpoint{4.403223in}{1.241016in}}{\pgfqpoint{4.400389in}{1.247858in}}{\pgfqpoint{4.395345in}{1.252902in}}%
\pgfpathcurveto{\pgfqpoint{4.390302in}{1.257945in}}{\pgfqpoint{4.383460in}{1.260779in}}{\pgfqpoint{4.376327in}{1.260779in}}%
\pgfpathcurveto{\pgfqpoint{4.369194in}{1.260779in}}{\pgfqpoint{4.362353in}{1.257945in}}{\pgfqpoint{4.357309in}{1.252902in}}%
\pgfpathcurveto{\pgfqpoint{4.352265in}{1.247858in}}{\pgfqpoint{4.349431in}{1.241016in}}{\pgfqpoint{4.349431in}{1.233883in}}%
\pgfpathcurveto{\pgfqpoint{4.349431in}{1.226751in}}{\pgfqpoint{4.352265in}{1.219909in}}{\pgfqpoint{4.357309in}{1.214865in}}%
\pgfpathcurveto{\pgfqpoint{4.362353in}{1.209822in}}{\pgfqpoint{4.369194in}{1.206988in}}{\pgfqpoint{4.376327in}{1.206988in}}%
\pgfpathclose%
\pgfusepath{stroke,fill}%
\end{pgfscope}%
\begin{pgfscope}%
\pgfpathrectangle{\pgfqpoint{2.867647in}{0.500000in}}{\pgfqpoint{1.764706in}{1.700000in}}%
\pgfusepath{clip}%
\pgfsetbuttcap%
\pgfsetroundjoin%
\definecolor{currentfill}{rgb}{0.963559,0.632016,0.472047}%
\pgfsetfillcolor{currentfill}%
\pgfsetlinewidth{0.311001pt}%
\definecolor{currentstroke}{rgb}{1.000000,1.000000,1.000000}%
\pgfsetstrokecolor{currentstroke}%
\pgfsetdash{}{0pt}%
\pgfpathmoveto{\pgfqpoint{4.260877in}{1.615112in}}%
\pgfpathcurveto{\pgfqpoint{4.268010in}{1.615112in}}{\pgfqpoint{4.274851in}{1.617946in}}{\pgfqpoint{4.279895in}{1.622989in}}%
\pgfpathcurveto{\pgfqpoint{4.284939in}{1.628033in}}{\pgfqpoint{4.287773in}{1.634875in}}{\pgfqpoint{4.287773in}{1.642007in}}%
\pgfpathcurveto{\pgfqpoint{4.287773in}{1.649140in}}{\pgfqpoint{4.284939in}{1.655982in}}{\pgfqpoint{4.279895in}{1.661026in}}%
\pgfpathcurveto{\pgfqpoint{4.274851in}{1.666069in}}{\pgfqpoint{4.268010in}{1.668903in}}{\pgfqpoint{4.260877in}{1.668903in}}%
\pgfpathcurveto{\pgfqpoint{4.253744in}{1.668903in}}{\pgfqpoint{4.246903in}{1.666069in}}{\pgfqpoint{4.241859in}{1.661026in}}%
\pgfpathcurveto{\pgfqpoint{4.236815in}{1.655982in}}{\pgfqpoint{4.233981in}{1.649140in}}{\pgfqpoint{4.233981in}{1.642007in}}%
\pgfpathcurveto{\pgfqpoint{4.233981in}{1.634875in}}{\pgfqpoint{4.236815in}{1.628033in}}{\pgfqpoint{4.241859in}{1.622989in}}%
\pgfpathcurveto{\pgfqpoint{4.246903in}{1.617946in}}{\pgfqpoint{4.253744in}{1.615112in}}{\pgfqpoint{4.260877in}{1.615112in}}%
\pgfpathclose%
\pgfusepath{stroke,fill}%
\end{pgfscope}%
\begin{pgfscope}%
\pgfpathrectangle{\pgfqpoint{2.867647in}{0.500000in}}{\pgfqpoint{1.764706in}{1.700000in}}%
\pgfusepath{clip}%
\pgfsetbuttcap%
\pgfsetroundjoin%
\definecolor{currentfill}{rgb}{0.980678,0.914765,0.856766}%
\pgfsetfillcolor{currentfill}%
\pgfsetlinewidth{0.311001pt}%
\definecolor{currentstroke}{rgb}{1.000000,1.000000,1.000000}%
\pgfsetstrokecolor{currentstroke}%
\pgfsetdash{}{0pt}%
\pgfpathmoveto{\pgfqpoint{4.179745in}{1.407527in}}%
\pgfpathcurveto{\pgfqpoint{4.186878in}{1.407527in}}{\pgfqpoint{4.193719in}{1.410361in}}{\pgfqpoint{4.198763in}{1.415405in}}%
\pgfpathcurveto{\pgfqpoint{4.203807in}{1.420449in}}{\pgfqpoint{4.206641in}{1.427290in}}{\pgfqpoint{4.206641in}{1.434423in}}%
\pgfpathcurveto{\pgfqpoint{4.206641in}{1.441556in}}{\pgfqpoint{4.203807in}{1.448398in}}{\pgfqpoint{4.198763in}{1.453441in}}%
\pgfpathcurveto{\pgfqpoint{4.193719in}{1.458485in}}{\pgfqpoint{4.186878in}{1.461319in}}{\pgfqpoint{4.179745in}{1.461319in}}%
\pgfpathcurveto{\pgfqpoint{4.172612in}{1.461319in}}{\pgfqpoint{4.165771in}{1.458485in}}{\pgfqpoint{4.160727in}{1.453441in}}%
\pgfpathcurveto{\pgfqpoint{4.155683in}{1.448398in}}{\pgfqpoint{4.152849in}{1.441556in}}{\pgfqpoint{4.152849in}{1.434423in}}%
\pgfpathcurveto{\pgfqpoint{4.152849in}{1.427290in}}{\pgfqpoint{4.155683in}{1.420449in}}{\pgfqpoint{4.160727in}{1.415405in}}%
\pgfpathcurveto{\pgfqpoint{4.165771in}{1.410361in}}{\pgfqpoint{4.172612in}{1.407527in}}{\pgfqpoint{4.179745in}{1.407527in}}%
\pgfpathclose%
\pgfusepath{stroke,fill}%
\end{pgfscope}%
\begin{pgfscope}%
\pgfpathrectangle{\pgfqpoint{2.867647in}{0.500000in}}{\pgfqpoint{1.764706in}{1.700000in}}%
\pgfusepath{clip}%
\pgfsetbuttcap%
\pgfsetroundjoin%
\definecolor{currentfill}{rgb}{0.971694,0.833208,0.737161}%
\pgfsetfillcolor{currentfill}%
\pgfsetlinewidth{0.311001pt}%
\definecolor{currentstroke}{rgb}{1.000000,1.000000,1.000000}%
\pgfsetstrokecolor{currentstroke}%
\pgfsetdash{}{0pt}%
\pgfpathmoveto{\pgfqpoint{4.264800in}{1.363856in}}%
\pgfpathcurveto{\pgfqpoint{4.271933in}{1.363856in}}{\pgfqpoint{4.278774in}{1.366690in}}{\pgfqpoint{4.283818in}{1.371734in}}%
\pgfpathcurveto{\pgfqpoint{4.288862in}{1.376777in}}{\pgfqpoint{4.291695in}{1.383619in}}{\pgfqpoint{4.291695in}{1.390752in}}%
\pgfpathcurveto{\pgfqpoint{4.291695in}{1.397885in}}{\pgfqpoint{4.288862in}{1.404726in}}{\pgfqpoint{4.283818in}{1.409770in}}%
\pgfpathcurveto{\pgfqpoint{4.278774in}{1.414814in}}{\pgfqpoint{4.271933in}{1.417648in}}{\pgfqpoint{4.264800in}{1.417648in}}%
\pgfpathcurveto{\pgfqpoint{4.257667in}{1.417648in}}{\pgfqpoint{4.250825in}{1.414814in}}{\pgfqpoint{4.245782in}{1.409770in}}%
\pgfpathcurveto{\pgfqpoint{4.240738in}{1.404726in}}{\pgfqpoint{4.237904in}{1.397885in}}{\pgfqpoint{4.237904in}{1.390752in}}%
\pgfpathcurveto{\pgfqpoint{4.237904in}{1.383619in}}{\pgfqpoint{4.240738in}{1.376777in}}{\pgfqpoint{4.245782in}{1.371734in}}%
\pgfpathcurveto{\pgfqpoint{4.250825in}{1.366690in}}{\pgfqpoint{4.257667in}{1.363856in}}{\pgfqpoint{4.264800in}{1.363856in}}%
\pgfpathclose%
\pgfusepath{stroke,fill}%
\end{pgfscope}%
\begin{pgfscope}%
\pgfpathrectangle{\pgfqpoint{2.867647in}{0.500000in}}{\pgfqpoint{1.764706in}{1.700000in}}%
\pgfusepath{clip}%
\pgfsetbuttcap%
\pgfsetroundjoin%
\definecolor{currentfill}{rgb}{0.972201,0.839051,0.745789}%
\pgfsetfillcolor{currentfill}%
\pgfsetlinewidth{0.311001pt}%
\definecolor{currentstroke}{rgb}{1.000000,1.000000,1.000000}%
\pgfsetstrokecolor{currentstroke}%
\pgfsetdash{}{0pt}%
\pgfpathmoveto{\pgfqpoint{4.257238in}{1.404854in}}%
\pgfpathcurveto{\pgfqpoint{4.264370in}{1.404854in}}{\pgfqpoint{4.271212in}{1.407688in}}{\pgfqpoint{4.276256in}{1.412732in}}%
\pgfpathcurveto{\pgfqpoint{4.281299in}{1.417775in}}{\pgfqpoint{4.284133in}{1.424617in}}{\pgfqpoint{4.284133in}{1.431750in}}%
\pgfpathcurveto{\pgfqpoint{4.284133in}{1.438883in}}{\pgfqpoint{4.281299in}{1.445724in}}{\pgfqpoint{4.276256in}{1.450768in}}%
\pgfpathcurveto{\pgfqpoint{4.271212in}{1.455812in}}{\pgfqpoint{4.264370in}{1.458646in}}{\pgfqpoint{4.257238in}{1.458646in}}%
\pgfpathcurveto{\pgfqpoint{4.250105in}{1.458646in}}{\pgfqpoint{4.243263in}{1.455812in}}{\pgfqpoint{4.238220in}{1.450768in}}%
\pgfpathcurveto{\pgfqpoint{4.233176in}{1.445724in}}{\pgfqpoint{4.230342in}{1.438883in}}{\pgfqpoint{4.230342in}{1.431750in}}%
\pgfpathcurveto{\pgfqpoint{4.230342in}{1.424617in}}{\pgfqpoint{4.233176in}{1.417775in}}{\pgfqpoint{4.238220in}{1.412732in}}%
\pgfpathcurveto{\pgfqpoint{4.243263in}{1.407688in}}{\pgfqpoint{4.250105in}{1.404854in}}{\pgfqpoint{4.257238in}{1.404854in}}%
\pgfpathclose%
\pgfusepath{stroke,fill}%
\end{pgfscope}%
\begin{pgfscope}%
\pgfpathrectangle{\pgfqpoint{2.867647in}{0.500000in}}{\pgfqpoint{1.764706in}{1.700000in}}%
\pgfusepath{clip}%
\pgfsetbuttcap%
\pgfsetroundjoin%
\definecolor{currentfill}{rgb}{0.958791,0.526283,0.368909}%
\pgfsetfillcolor{currentfill}%
\pgfsetlinewidth{0.311001pt}%
\definecolor{currentstroke}{rgb}{1.000000,1.000000,1.000000}%
\pgfsetstrokecolor{currentstroke}%
\pgfsetdash{}{0pt}%
\pgfpathmoveto{\pgfqpoint{3.952601in}{1.046319in}}%
\pgfpathcurveto{\pgfqpoint{3.959734in}{1.046319in}}{\pgfqpoint{3.966575in}{1.049153in}}{\pgfqpoint{3.971619in}{1.054196in}}%
\pgfpathcurveto{\pgfqpoint{3.976663in}{1.059240in}}{\pgfqpoint{3.979496in}{1.066082in}}{\pgfqpoint{3.979496in}{1.073215in}}%
\pgfpathcurveto{\pgfqpoint{3.979496in}{1.080347in}}{\pgfqpoint{3.976663in}{1.087189in}}{\pgfqpoint{3.971619in}{1.092233in}}%
\pgfpathcurveto{\pgfqpoint{3.966575in}{1.097276in}}{\pgfqpoint{3.959734in}{1.100110in}}{\pgfqpoint{3.952601in}{1.100110in}}%
\pgfpathcurveto{\pgfqpoint{3.945468in}{1.100110in}}{\pgfqpoint{3.938626in}{1.097276in}}{\pgfqpoint{3.933583in}{1.092233in}}%
\pgfpathcurveto{\pgfqpoint{3.928539in}{1.087189in}}{\pgfqpoint{3.925705in}{1.080347in}}{\pgfqpoint{3.925705in}{1.073215in}}%
\pgfpathcurveto{\pgfqpoint{3.925705in}{1.066082in}}{\pgfqpoint{3.928539in}{1.059240in}}{\pgfqpoint{3.933583in}{1.054196in}}%
\pgfpathcurveto{\pgfqpoint{3.938626in}{1.049153in}}{\pgfqpoint{3.945468in}{1.046319in}}{\pgfqpoint{3.952601in}{1.046319in}}%
\pgfpathclose%
\pgfusepath{stroke,fill}%
\end{pgfscope}%
\begin{pgfscope}%
\pgfpathrectangle{\pgfqpoint{2.867647in}{0.500000in}}{\pgfqpoint{1.764706in}{1.700000in}}%
\pgfusepath{clip}%
\pgfsetbuttcap%
\pgfsetroundjoin%
\definecolor{currentfill}{rgb}{0.964558,0.676556,0.522514}%
\pgfsetfillcolor{currentfill}%
\pgfsetlinewidth{0.311001pt}%
\definecolor{currentstroke}{rgb}{1.000000,1.000000,1.000000}%
\pgfsetstrokecolor{currentstroke}%
\pgfsetdash{}{0pt}%
\pgfpathmoveto{\pgfqpoint{3.969597in}{1.694527in}}%
\pgfpathcurveto{\pgfqpoint{3.976730in}{1.694527in}}{\pgfqpoint{3.983572in}{1.697361in}}{\pgfqpoint{3.988615in}{1.702404in}}%
\pgfpathcurveto{\pgfqpoint{3.993659in}{1.707448in}}{\pgfqpoint{3.996493in}{1.714290in}}{\pgfqpoint{3.996493in}{1.721422in}}%
\pgfpathcurveto{\pgfqpoint{3.996493in}{1.728555in}}{\pgfqpoint{3.993659in}{1.735397in}}{\pgfqpoint{3.988615in}{1.740441in}}%
\pgfpathcurveto{\pgfqpoint{3.983572in}{1.745484in}}{\pgfqpoint{3.976730in}{1.748318in}}{\pgfqpoint{3.969597in}{1.748318in}}%
\pgfpathcurveto{\pgfqpoint{3.962464in}{1.748318in}}{\pgfqpoint{3.955623in}{1.745484in}}{\pgfqpoint{3.950579in}{1.740441in}}%
\pgfpathcurveto{\pgfqpoint{3.945535in}{1.735397in}}{\pgfqpoint{3.942702in}{1.728555in}}{\pgfqpoint{3.942702in}{1.721422in}}%
\pgfpathcurveto{\pgfqpoint{3.942702in}{1.714290in}}{\pgfqpoint{3.945535in}{1.707448in}}{\pgfqpoint{3.950579in}{1.702404in}}%
\pgfpathcurveto{\pgfqpoint{3.955623in}{1.697361in}}{\pgfqpoint{3.962464in}{1.694527in}}{\pgfqpoint{3.969597in}{1.694527in}}%
\pgfpathclose%
\pgfusepath{stroke,fill}%
\end{pgfscope}%
\begin{pgfscope}%
\pgfpathrectangle{\pgfqpoint{2.867647in}{0.500000in}}{\pgfqpoint{1.764706in}{1.700000in}}%
\pgfusepath{clip}%
\pgfsetbuttcap%
\pgfsetroundjoin%
\definecolor{currentfill}{rgb}{0.964920,0.695342,0.545192}%
\pgfsetfillcolor{currentfill}%
\pgfsetlinewidth{0.311001pt}%
\definecolor{currentstroke}{rgb}{1.000000,1.000000,1.000000}%
\pgfsetstrokecolor{currentstroke}%
\pgfsetdash{}{0pt}%
\pgfpathmoveto{\pgfqpoint{4.194721in}{1.680863in}}%
\pgfpathcurveto{\pgfqpoint{4.201854in}{1.680863in}}{\pgfqpoint{4.208696in}{1.683697in}}{\pgfqpoint{4.213740in}{1.688740in}}%
\pgfpathcurveto{\pgfqpoint{4.218783in}{1.693784in}}{\pgfqpoint{4.221617in}{1.700626in}}{\pgfqpoint{4.221617in}{1.707758in}}%
\pgfpathcurveto{\pgfqpoint{4.221617in}{1.714891in}}{\pgfqpoint{4.218783in}{1.721733in}}{\pgfqpoint{4.213740in}{1.726776in}}%
\pgfpathcurveto{\pgfqpoint{4.208696in}{1.731820in}}{\pgfqpoint{4.201854in}{1.734654in}}{\pgfqpoint{4.194721in}{1.734654in}}%
\pgfpathcurveto{\pgfqpoint{4.187589in}{1.734654in}}{\pgfqpoint{4.180747in}{1.731820in}}{\pgfqpoint{4.175703in}{1.726776in}}%
\pgfpathcurveto{\pgfqpoint{4.170660in}{1.721733in}}{\pgfqpoint{4.167826in}{1.714891in}}{\pgfqpoint{4.167826in}{1.707758in}}%
\pgfpathcurveto{\pgfqpoint{4.167826in}{1.700626in}}{\pgfqpoint{4.170660in}{1.693784in}}{\pgfqpoint{4.175703in}{1.688740in}}%
\pgfpathcurveto{\pgfqpoint{4.180747in}{1.683697in}}{\pgfqpoint{4.187589in}{1.680863in}}{\pgfqpoint{4.194721in}{1.680863in}}%
\pgfpathclose%
\pgfusepath{stroke,fill}%
\end{pgfscope}%
\begin{pgfscope}%
\pgfpathrectangle{\pgfqpoint{2.867647in}{0.500000in}}{\pgfqpoint{1.764706in}{1.700000in}}%
\pgfusepath{clip}%
\pgfsetbuttcap%
\pgfsetroundjoin%
\definecolor{currentfill}{rgb}{0.971202,0.827364,0.728520}%
\pgfsetfillcolor{currentfill}%
\pgfsetlinewidth{0.311001pt}%
\definecolor{currentstroke}{rgb}{1.000000,1.000000,1.000000}%
\pgfsetstrokecolor{currentstroke}%
\pgfsetdash{}{0pt}%
\pgfpathmoveto{\pgfqpoint{4.139822in}{1.665403in}}%
\pgfpathcurveto{\pgfqpoint{4.146955in}{1.665403in}}{\pgfqpoint{4.153797in}{1.668237in}}{\pgfqpoint{4.158841in}{1.673280in}}%
\pgfpathcurveto{\pgfqpoint{4.163884in}{1.678324in}}{\pgfqpoint{4.166718in}{1.685166in}}{\pgfqpoint{4.166718in}{1.692299in}}%
\pgfpathcurveto{\pgfqpoint{4.166718in}{1.699431in}}{\pgfqpoint{4.163884in}{1.706273in}}{\pgfqpoint{4.158841in}{1.711317in}}%
\pgfpathcurveto{\pgfqpoint{4.153797in}{1.716360in}}{\pgfqpoint{4.146955in}{1.719194in}}{\pgfqpoint{4.139822in}{1.719194in}}%
\pgfpathcurveto{\pgfqpoint{4.132690in}{1.719194in}}{\pgfqpoint{4.125848in}{1.716360in}}{\pgfqpoint{4.120804in}{1.711317in}}%
\pgfpathcurveto{\pgfqpoint{4.115761in}{1.706273in}}{\pgfqpoint{4.112927in}{1.699431in}}{\pgfqpoint{4.112927in}{1.692299in}}%
\pgfpathcurveto{\pgfqpoint{4.112927in}{1.685166in}}{\pgfqpoint{4.115761in}{1.678324in}}{\pgfqpoint{4.120804in}{1.673280in}}%
\pgfpathcurveto{\pgfqpoint{4.125848in}{1.668237in}}{\pgfqpoint{4.132690in}{1.665403in}}{\pgfqpoint{4.139822in}{1.665403in}}%
\pgfpathclose%
\pgfusepath{stroke,fill}%
\end{pgfscope}%
\begin{pgfscope}%
\pgfpathrectangle{\pgfqpoint{2.867647in}{0.500000in}}{\pgfqpoint{1.764706in}{1.700000in}}%
\pgfusepath{clip}%
\pgfsetbuttcap%
\pgfsetroundjoin%
\definecolor{currentfill}{rgb}{0.955697,0.484891,0.334214}%
\pgfsetfillcolor{currentfill}%
\pgfsetlinewidth{0.311001pt}%
\definecolor{currentstroke}{rgb}{1.000000,1.000000,1.000000}%
\pgfsetstrokecolor{currentstroke}%
\pgfsetdash{}{0pt}%
\pgfpathmoveto{\pgfqpoint{4.337394in}{1.459539in}}%
\pgfpathcurveto{\pgfqpoint{4.344527in}{1.459539in}}{\pgfqpoint{4.351368in}{1.462372in}}{\pgfqpoint{4.356412in}{1.467416in}}%
\pgfpathcurveto{\pgfqpoint{4.361456in}{1.472460in}}{\pgfqpoint{4.364289in}{1.479301in}}{\pgfqpoint{4.364289in}{1.486434in}}%
\pgfpathcurveto{\pgfqpoint{4.364289in}{1.493567in}}{\pgfqpoint{4.361456in}{1.500409in}}{\pgfqpoint{4.356412in}{1.505452in}}%
\pgfpathcurveto{\pgfqpoint{4.351368in}{1.510496in}}{\pgfqpoint{4.344527in}{1.513330in}}{\pgfqpoint{4.337394in}{1.513330in}}%
\pgfpathcurveto{\pgfqpoint{4.330261in}{1.513330in}}{\pgfqpoint{4.323419in}{1.510496in}}{\pgfqpoint{4.318376in}{1.505452in}}%
\pgfpathcurveto{\pgfqpoint{4.313332in}{1.500409in}}{\pgfqpoint{4.310498in}{1.493567in}}{\pgfqpoint{4.310498in}{1.486434in}}%
\pgfpathcurveto{\pgfqpoint{4.310498in}{1.479301in}}{\pgfqpoint{4.313332in}{1.472460in}}{\pgfqpoint{4.318376in}{1.467416in}}%
\pgfpathcurveto{\pgfqpoint{4.323419in}{1.462372in}}{\pgfqpoint{4.330261in}{1.459539in}}{\pgfqpoint{4.337394in}{1.459539in}}%
\pgfpathclose%
\pgfusepath{stroke,fill}%
\end{pgfscope}%
\begin{pgfscope}%
\pgfpathrectangle{\pgfqpoint{2.867647in}{0.500000in}}{\pgfqpoint{1.764706in}{1.700000in}}%
\pgfusepath{clip}%
\pgfsetbuttcap%
\pgfsetroundjoin%
\definecolor{currentfill}{rgb}{0.980678,0.914765,0.856766}%
\pgfsetfillcolor{currentfill}%
\pgfsetlinewidth{0.311001pt}%
\definecolor{currentstroke}{rgb}{1.000000,1.000000,1.000000}%
\pgfsetstrokecolor{currentstroke}%
\pgfsetdash{}{0pt}%
\pgfpathmoveto{\pgfqpoint{4.172245in}{1.419615in}}%
\pgfpathcurveto{\pgfqpoint{4.179378in}{1.419615in}}{\pgfqpoint{4.186220in}{1.422449in}}{\pgfqpoint{4.191263in}{1.427492in}}%
\pgfpathcurveto{\pgfqpoint{4.196307in}{1.432536in}}{\pgfqpoint{4.199141in}{1.439378in}}{\pgfqpoint{4.199141in}{1.446510in}}%
\pgfpathcurveto{\pgfqpoint{4.199141in}{1.453643in}}{\pgfqpoint{4.196307in}{1.460485in}}{\pgfqpoint{4.191263in}{1.465529in}}%
\pgfpathcurveto{\pgfqpoint{4.186220in}{1.470572in}}{\pgfqpoint{4.179378in}{1.473406in}}{\pgfqpoint{4.172245in}{1.473406in}}%
\pgfpathcurveto{\pgfqpoint{4.165112in}{1.473406in}}{\pgfqpoint{4.158271in}{1.470572in}}{\pgfqpoint{4.153227in}{1.465529in}}%
\pgfpathcurveto{\pgfqpoint{4.148183in}{1.460485in}}{\pgfqpoint{4.145350in}{1.453643in}}{\pgfqpoint{4.145350in}{1.446510in}}%
\pgfpathcurveto{\pgfqpoint{4.145350in}{1.439378in}}{\pgfqpoint{4.148183in}{1.432536in}}{\pgfqpoint{4.153227in}{1.427492in}}%
\pgfpathcurveto{\pgfqpoint{4.158271in}{1.422449in}}{\pgfqpoint{4.165112in}{1.419615in}}{\pgfqpoint{4.172245in}{1.419615in}}%
\pgfpathclose%
\pgfusepath{stroke,fill}%
\end{pgfscope}%
\begin{pgfscope}%
\pgfpathrectangle{\pgfqpoint{2.867647in}{0.500000in}}{\pgfqpoint{1.764706in}{1.700000in}}%
\pgfusepath{clip}%
\pgfsetbuttcap%
\pgfsetroundjoin%
\definecolor{currentfill}{rgb}{0.974412,0.862387,0.780156}%
\pgfsetfillcolor{currentfill}%
\pgfsetlinewidth{0.311001pt}%
\definecolor{currentstroke}{rgb}{1.000000,1.000000,1.000000}%
\pgfsetstrokecolor{currentstroke}%
\pgfsetdash{}{0pt}%
\pgfpathmoveto{\pgfqpoint{4.176746in}{1.043929in}}%
\pgfpathcurveto{\pgfqpoint{4.183878in}{1.043929in}}{\pgfqpoint{4.190720in}{1.046762in}}{\pgfqpoint{4.195764in}{1.051806in}}%
\pgfpathcurveto{\pgfqpoint{4.200807in}{1.056850in}}{\pgfqpoint{4.203641in}{1.063691in}}{\pgfqpoint{4.203641in}{1.070824in}}%
\pgfpathcurveto{\pgfqpoint{4.203641in}{1.077957in}}{\pgfqpoint{4.200807in}{1.084799in}}{\pgfqpoint{4.195764in}{1.089842in}}%
\pgfpathcurveto{\pgfqpoint{4.190720in}{1.094886in}}{\pgfqpoint{4.183878in}{1.097720in}}{\pgfqpoint{4.176746in}{1.097720in}}%
\pgfpathcurveto{\pgfqpoint{4.169613in}{1.097720in}}{\pgfqpoint{4.162771in}{1.094886in}}{\pgfqpoint{4.157727in}{1.089842in}}%
\pgfpathcurveto{\pgfqpoint{4.152684in}{1.084799in}}{\pgfqpoint{4.149850in}{1.077957in}}{\pgfqpoint{4.149850in}{1.070824in}}%
\pgfpathcurveto{\pgfqpoint{4.149850in}{1.063691in}}{\pgfqpoint{4.152684in}{1.056850in}}{\pgfqpoint{4.157727in}{1.051806in}}%
\pgfpathcurveto{\pgfqpoint{4.162771in}{1.046762in}}{\pgfqpoint{4.169613in}{1.043929in}}{\pgfqpoint{4.176746in}{1.043929in}}%
\pgfpathclose%
\pgfusepath{stroke,fill}%
\end{pgfscope}%
\begin{pgfscope}%
\pgfpathrectangle{\pgfqpoint{2.867647in}{0.500000in}}{\pgfqpoint{1.764706in}{1.700000in}}%
\pgfusepath{clip}%
\pgfsetbuttcap%
\pgfsetroundjoin%
\definecolor{currentfill}{rgb}{0.964173,0.657587,0.500469}%
\pgfsetfillcolor{currentfill}%
\pgfsetlinewidth{0.311001pt}%
\definecolor{currentstroke}{rgb}{1.000000,1.000000,1.000000}%
\pgfsetstrokecolor{currentstroke}%
\pgfsetdash{}{0pt}%
\pgfpathmoveto{\pgfqpoint{4.233445in}{1.648552in}}%
\pgfpathcurveto{\pgfqpoint{4.240578in}{1.648552in}}{\pgfqpoint{4.247420in}{1.651386in}}{\pgfqpoint{4.252464in}{1.656430in}}%
\pgfpathcurveto{\pgfqpoint{4.257507in}{1.661473in}}{\pgfqpoint{4.260341in}{1.668315in}}{\pgfqpoint{4.260341in}{1.675448in}}%
\pgfpathcurveto{\pgfqpoint{4.260341in}{1.682581in}}{\pgfqpoint{4.257507in}{1.689422in}}{\pgfqpoint{4.252464in}{1.694466in}}%
\pgfpathcurveto{\pgfqpoint{4.247420in}{1.699510in}}{\pgfqpoint{4.240578in}{1.702344in}}{\pgfqpoint{4.233445in}{1.702344in}}%
\pgfpathcurveto{\pgfqpoint{4.226313in}{1.702344in}}{\pgfqpoint{4.219471in}{1.699510in}}{\pgfqpoint{4.214427in}{1.694466in}}%
\pgfpathcurveto{\pgfqpoint{4.209384in}{1.689422in}}{\pgfqpoint{4.206550in}{1.682581in}}{\pgfqpoint{4.206550in}{1.675448in}}%
\pgfpathcurveto{\pgfqpoint{4.206550in}{1.668315in}}{\pgfqpoint{4.209384in}{1.661473in}}{\pgfqpoint{4.214427in}{1.656430in}}%
\pgfpathcurveto{\pgfqpoint{4.219471in}{1.651386in}}{\pgfqpoint{4.226313in}{1.648552in}}{\pgfqpoint{4.233445in}{1.648552in}}%
\pgfpathclose%
\pgfusepath{stroke,fill}%
\end{pgfscope}%
\begin{pgfscope}%
\pgfpathrectangle{\pgfqpoint{2.867647in}{0.500000in}}{\pgfqpoint{1.764706in}{1.700000in}}%
\pgfusepath{clip}%
\pgfsetbuttcap%
\pgfsetroundjoin%
\definecolor{currentfill}{rgb}{0.755358,0.089198,0.328762}%
\pgfsetfillcolor{currentfill}%
\pgfsetlinewidth{0.311001pt}%
\definecolor{currentstroke}{rgb}{1.000000,1.000000,1.000000}%
\pgfsetstrokecolor{currentstroke}%
\pgfsetdash{}{0pt}%
\pgfpathmoveto{\pgfqpoint{4.071627in}{0.733998in}}%
\pgfpathcurveto{\pgfqpoint{4.078759in}{0.733998in}}{\pgfqpoint{4.085601in}{0.736832in}}{\pgfqpoint{4.090645in}{0.741875in}}%
\pgfpathcurveto{\pgfqpoint{4.095688in}{0.746919in}}{\pgfqpoint{4.098522in}{0.753761in}}{\pgfqpoint{4.098522in}{0.760893in}}%
\pgfpathcurveto{\pgfqpoint{4.098522in}{0.768026in}}{\pgfqpoint{4.095688in}{0.774868in}}{\pgfqpoint{4.090645in}{0.779912in}}%
\pgfpathcurveto{\pgfqpoint{4.085601in}{0.784955in}}{\pgfqpoint{4.078759in}{0.787789in}}{\pgfqpoint{4.071627in}{0.787789in}}%
\pgfpathcurveto{\pgfqpoint{4.064494in}{0.787789in}}{\pgfqpoint{4.057652in}{0.784955in}}{\pgfqpoint{4.052608in}{0.779912in}}%
\pgfpathcurveto{\pgfqpoint{4.047565in}{0.774868in}}{\pgfqpoint{4.044731in}{0.768026in}}{\pgfqpoint{4.044731in}{0.760893in}}%
\pgfpathcurveto{\pgfqpoint{4.044731in}{0.753761in}}{\pgfqpoint{4.047565in}{0.746919in}}{\pgfqpoint{4.052608in}{0.741875in}}%
\pgfpathcurveto{\pgfqpoint{4.057652in}{0.736832in}}{\pgfqpoint{4.064494in}{0.733998in}}{\pgfqpoint{4.071627in}{0.733998in}}%
\pgfpathclose%
\pgfusepath{stroke,fill}%
\end{pgfscope}%
\begin{pgfscope}%
\pgfpathrectangle{\pgfqpoint{2.867647in}{0.500000in}}{\pgfqpoint{1.764706in}{1.700000in}}%
\pgfusepath{clip}%
\pgfsetbuttcap%
\pgfsetroundjoin%
\definecolor{currentfill}{rgb}{0.962765,0.606121,0.444717}%
\pgfsetfillcolor{currentfill}%
\pgfsetlinewidth{0.311001pt}%
\definecolor{currentstroke}{rgb}{1.000000,1.000000,1.000000}%
\pgfsetstrokecolor{currentstroke}%
\pgfsetdash{}{0pt}%
\pgfpathmoveto{\pgfqpoint{3.951761in}{1.761382in}}%
\pgfpathcurveto{\pgfqpoint{3.958894in}{1.761382in}}{\pgfqpoint{3.965736in}{1.764216in}}{\pgfqpoint{3.970780in}{1.769260in}}%
\pgfpathcurveto{\pgfqpoint{3.975823in}{1.774304in}}{\pgfqpoint{3.978657in}{1.781145in}}{\pgfqpoint{3.978657in}{1.788278in}}%
\pgfpathcurveto{\pgfqpoint{3.978657in}{1.795411in}}{\pgfqpoint{3.975823in}{1.802253in}}{\pgfqpoint{3.970780in}{1.807296in}}%
\pgfpathcurveto{\pgfqpoint{3.965736in}{1.812340in}}{\pgfqpoint{3.958894in}{1.815174in}}{\pgfqpoint{3.951761in}{1.815174in}}%
\pgfpathcurveto{\pgfqpoint{3.944629in}{1.815174in}}{\pgfqpoint{3.937787in}{1.812340in}}{\pgfqpoint{3.932743in}{1.807296in}}%
\pgfpathcurveto{\pgfqpoint{3.927700in}{1.802253in}}{\pgfqpoint{3.924866in}{1.795411in}}{\pgfqpoint{3.924866in}{1.788278in}}%
\pgfpathcurveto{\pgfqpoint{3.924866in}{1.781145in}}{\pgfqpoint{3.927700in}{1.774304in}}{\pgfqpoint{3.932743in}{1.769260in}}%
\pgfpathcurveto{\pgfqpoint{3.937787in}{1.764216in}}{\pgfqpoint{3.944629in}{1.761382in}}{\pgfqpoint{3.951761in}{1.761382in}}%
\pgfpathclose%
\pgfusepath{stroke,fill}%
\end{pgfscope}%
\begin{pgfscope}%
\pgfpathrectangle{\pgfqpoint{2.867647in}{0.500000in}}{\pgfqpoint{1.764706in}{1.700000in}}%
\pgfusepath{clip}%
\pgfsetbuttcap%
\pgfsetroundjoin%
\definecolor{currentfill}{rgb}{0.966812,0.762584,0.633643}%
\pgfsetfillcolor{currentfill}%
\pgfsetlinewidth{0.311001pt}%
\definecolor{currentstroke}{rgb}{1.000000,1.000000,1.000000}%
\pgfsetstrokecolor{currentstroke}%
\pgfsetdash{}{0pt}%
\pgfpathmoveto{\pgfqpoint{4.018809in}{1.689991in}}%
\pgfpathcurveto{\pgfqpoint{4.025942in}{1.689991in}}{\pgfqpoint{4.032783in}{1.692825in}}{\pgfqpoint{4.037827in}{1.697868in}}%
\pgfpathcurveto{\pgfqpoint{4.042871in}{1.702912in}}{\pgfqpoint{4.045705in}{1.709754in}}{\pgfqpoint{4.045705in}{1.716887in}}%
\pgfpathcurveto{\pgfqpoint{4.045705in}{1.724019in}}{\pgfqpoint{4.042871in}{1.730861in}}{\pgfqpoint{4.037827in}{1.735905in}}%
\pgfpathcurveto{\pgfqpoint{4.032783in}{1.740948in}}{\pgfqpoint{4.025942in}{1.743782in}}{\pgfqpoint{4.018809in}{1.743782in}}%
\pgfpathcurveto{\pgfqpoint{4.011676in}{1.743782in}}{\pgfqpoint{4.004834in}{1.740948in}}{\pgfqpoint{3.999791in}{1.735905in}}%
\pgfpathcurveto{\pgfqpoint{3.994747in}{1.730861in}}{\pgfqpoint{3.991913in}{1.724019in}}{\pgfqpoint{3.991913in}{1.716887in}}%
\pgfpathcurveto{\pgfqpoint{3.991913in}{1.709754in}}{\pgfqpoint{3.994747in}{1.702912in}}{\pgfqpoint{3.999791in}{1.697868in}}%
\pgfpathcurveto{\pgfqpoint{4.004834in}{1.692825in}}{\pgfqpoint{4.011676in}{1.689991in}}{\pgfqpoint{4.018809in}{1.689991in}}%
\pgfpathclose%
\pgfusepath{stroke,fill}%
\end{pgfscope}%
\begin{pgfscope}%
\pgfpathrectangle{\pgfqpoint{2.867647in}{0.500000in}}{\pgfqpoint{1.764706in}{1.700000in}}%
\pgfusepath{clip}%
\pgfsetbuttcap%
\pgfsetroundjoin%
\definecolor{currentfill}{rgb}{0.965440,0.720101,0.576404}%
\pgfsetfillcolor{currentfill}%
\pgfsetlinewidth{0.311001pt}%
\definecolor{currentstroke}{rgb}{1.000000,1.000000,1.000000}%
\pgfsetstrokecolor{currentstroke}%
\pgfsetdash{}{0pt}%
\pgfpathmoveto{\pgfqpoint{3.998025in}{0.946395in}}%
\pgfpathcurveto{\pgfqpoint{4.005158in}{0.946395in}}{\pgfqpoint{4.011999in}{0.949229in}}{\pgfqpoint{4.017043in}{0.954273in}}%
\pgfpathcurveto{\pgfqpoint{4.022087in}{0.959316in}}{\pgfqpoint{4.024921in}{0.966158in}}{\pgfqpoint{4.024921in}{0.973291in}}%
\pgfpathcurveto{\pgfqpoint{4.024921in}{0.980424in}}{\pgfqpoint{4.022087in}{0.987265in}}{\pgfqpoint{4.017043in}{0.992309in}}%
\pgfpathcurveto{\pgfqpoint{4.011999in}{0.997353in}}{\pgfqpoint{4.005158in}{1.000187in}}{\pgfqpoint{3.998025in}{1.000187in}}%
\pgfpathcurveto{\pgfqpoint{3.990892in}{1.000187in}}{\pgfqpoint{3.984051in}{0.997353in}}{\pgfqpoint{3.979007in}{0.992309in}}%
\pgfpathcurveto{\pgfqpoint{3.973963in}{0.987265in}}{\pgfqpoint{3.971129in}{0.980424in}}{\pgfqpoint{3.971129in}{0.973291in}}%
\pgfpathcurveto{\pgfqpoint{3.971129in}{0.966158in}}{\pgfqpoint{3.973963in}{0.959316in}}{\pgfqpoint{3.979007in}{0.954273in}}%
\pgfpathcurveto{\pgfqpoint{3.984051in}{0.949229in}}{\pgfqpoint{3.990892in}{0.946395in}}{\pgfqpoint{3.998025in}{0.946395in}}%
\pgfpathclose%
\pgfusepath{stroke,fill}%
\end{pgfscope}%
\begin{pgfscope}%
\pgfpathrectangle{\pgfqpoint{2.867647in}{0.500000in}}{\pgfqpoint{1.764706in}{1.700000in}}%
\pgfusepath{clip}%
\pgfsetbuttcap%
\pgfsetroundjoin%
\definecolor{currentfill}{rgb}{0.973832,0.856556,0.771584}%
\pgfsetfillcolor{currentfill}%
\pgfsetlinewidth{0.311001pt}%
\definecolor{currentstroke}{rgb}{1.000000,1.000000,1.000000}%
\pgfsetstrokecolor{currentstroke}%
\pgfsetdash{}{0pt}%
\pgfpathmoveto{\pgfqpoint{4.161809in}{1.018363in}}%
\pgfpathcurveto{\pgfqpoint{4.168942in}{1.018363in}}{\pgfqpoint{4.175784in}{1.021196in}}{\pgfqpoint{4.180828in}{1.026240in}}%
\pgfpathcurveto{\pgfqpoint{4.185871in}{1.031284in}}{\pgfqpoint{4.188705in}{1.038125in}}{\pgfqpoint{4.188705in}{1.045258in}}%
\pgfpathcurveto{\pgfqpoint{4.188705in}{1.052391in}}{\pgfqpoint{4.185871in}{1.059233in}}{\pgfqpoint{4.180828in}{1.064276in}}%
\pgfpathcurveto{\pgfqpoint{4.175784in}{1.069320in}}{\pgfqpoint{4.168942in}{1.072154in}}{\pgfqpoint{4.161809in}{1.072154in}}%
\pgfpathcurveto{\pgfqpoint{4.154677in}{1.072154in}}{\pgfqpoint{4.147835in}{1.069320in}}{\pgfqpoint{4.142791in}{1.064276in}}%
\pgfpathcurveto{\pgfqpoint{4.137748in}{1.059233in}}{\pgfqpoint{4.134914in}{1.052391in}}{\pgfqpoint{4.134914in}{1.045258in}}%
\pgfpathcurveto{\pgfqpoint{4.134914in}{1.038125in}}{\pgfqpoint{4.137748in}{1.031284in}}{\pgfqpoint{4.142791in}{1.026240in}}%
\pgfpathcurveto{\pgfqpoint{4.147835in}{1.021196in}}{\pgfqpoint{4.154677in}{1.018363in}}{\pgfqpoint{4.161809in}{1.018363in}}%
\pgfpathclose%
\pgfusepath{stroke,fill}%
\end{pgfscope}%
\begin{pgfscope}%
\pgfpathrectangle{\pgfqpoint{2.867647in}{0.500000in}}{\pgfqpoint{1.764706in}{1.700000in}}%
\pgfusepath{clip}%
\pgfsetbuttcap%
\pgfsetroundjoin%
\definecolor{currentfill}{rgb}{0.973832,0.856556,0.771584}%
\pgfsetfillcolor{currentfill}%
\pgfsetlinewidth{0.311001pt}%
\definecolor{currentstroke}{rgb}{1.000000,1.000000,1.000000}%
\pgfsetstrokecolor{currentstroke}%
\pgfsetdash{}{0pt}%
\pgfpathmoveto{\pgfqpoint{4.126545in}{1.328710in}}%
\pgfpathcurveto{\pgfqpoint{4.133678in}{1.328710in}}{\pgfqpoint{4.140520in}{1.331544in}}{\pgfqpoint{4.145564in}{1.336588in}}%
\pgfpathcurveto{\pgfqpoint{4.150607in}{1.341631in}}{\pgfqpoint{4.153441in}{1.348473in}}{\pgfqpoint{4.153441in}{1.355606in}}%
\pgfpathcurveto{\pgfqpoint{4.153441in}{1.362739in}}{\pgfqpoint{4.150607in}{1.369580in}}{\pgfqpoint{4.145564in}{1.374624in}}%
\pgfpathcurveto{\pgfqpoint{4.140520in}{1.379668in}}{\pgfqpoint{4.133678in}{1.382501in}}{\pgfqpoint{4.126545in}{1.382501in}}%
\pgfpathcurveto{\pgfqpoint{4.119413in}{1.382501in}}{\pgfqpoint{4.112571in}{1.379668in}}{\pgfqpoint{4.107527in}{1.374624in}}%
\pgfpathcurveto{\pgfqpoint{4.102484in}{1.369580in}}{\pgfqpoint{4.099650in}{1.362739in}}{\pgfqpoint{4.099650in}{1.355606in}}%
\pgfpathcurveto{\pgfqpoint{4.099650in}{1.348473in}}{\pgfqpoint{4.102484in}{1.341631in}}{\pgfqpoint{4.107527in}{1.336588in}}%
\pgfpathcurveto{\pgfqpoint{4.112571in}{1.331544in}}{\pgfqpoint{4.119413in}{1.328710in}}{\pgfqpoint{4.126545in}{1.328710in}}%
\pgfpathclose%
\pgfusepath{stroke,fill}%
\end{pgfscope}%
\begin{pgfscope}%
\pgfpathrectangle{\pgfqpoint{2.867647in}{0.500000in}}{\pgfqpoint{1.764706in}{1.700000in}}%
\pgfusepath{clip}%
\pgfsetbuttcap%
\pgfsetroundjoin%
\definecolor{currentfill}{rgb}{0.922239,0.282873,0.242296}%
\pgfsetfillcolor{currentfill}%
\pgfsetlinewidth{0.311001pt}%
\definecolor{currentstroke}{rgb}{1.000000,1.000000,1.000000}%
\pgfsetstrokecolor{currentstroke}%
\pgfsetdash{}{0pt}%
\pgfpathmoveto{\pgfqpoint{4.271514in}{1.715759in}}%
\pgfpathcurveto{\pgfqpoint{4.278646in}{1.715759in}}{\pgfqpoint{4.285488in}{1.718593in}}{\pgfqpoint{4.290532in}{1.723637in}}%
\pgfpathcurveto{\pgfqpoint{4.295575in}{1.728680in}}{\pgfqpoint{4.298409in}{1.735522in}}{\pgfqpoint{4.298409in}{1.742655in}}%
\pgfpathcurveto{\pgfqpoint{4.298409in}{1.749788in}}{\pgfqpoint{4.295575in}{1.756629in}}{\pgfqpoint{4.290532in}{1.761673in}}%
\pgfpathcurveto{\pgfqpoint{4.285488in}{1.766717in}}{\pgfqpoint{4.278646in}{1.769550in}}{\pgfqpoint{4.271514in}{1.769550in}}%
\pgfpathcurveto{\pgfqpoint{4.264381in}{1.769550in}}{\pgfqpoint{4.257539in}{1.766717in}}{\pgfqpoint{4.252495in}{1.761673in}}%
\pgfpathcurveto{\pgfqpoint{4.247452in}{1.756629in}}{\pgfqpoint{4.244618in}{1.749788in}}{\pgfqpoint{4.244618in}{1.742655in}}%
\pgfpathcurveto{\pgfqpoint{4.244618in}{1.735522in}}{\pgfqpoint{4.247452in}{1.728680in}}{\pgfqpoint{4.252495in}{1.723637in}}%
\pgfpathcurveto{\pgfqpoint{4.257539in}{1.718593in}}{\pgfqpoint{4.264381in}{1.715759in}}{\pgfqpoint{4.271514in}{1.715759in}}%
\pgfpathclose%
\pgfusepath{stroke,fill}%
\end{pgfscope}%
\begin{pgfscope}%
\pgfpathrectangle{\pgfqpoint{2.867647in}{0.500000in}}{\pgfqpoint{1.764706in}{1.700000in}}%
\pgfusepath{clip}%
\pgfsetbuttcap%
\pgfsetroundjoin%
\definecolor{currentfill}{rgb}{0.972201,0.839051,0.745789}%
\pgfsetfillcolor{currentfill}%
\pgfsetlinewidth{0.311001pt}%
\definecolor{currentstroke}{rgb}{1.000000,1.000000,1.000000}%
\pgfsetstrokecolor{currentstroke}%
\pgfsetdash{}{0pt}%
\pgfpathmoveto{\pgfqpoint{4.235260in}{1.506340in}}%
\pgfpathcurveto{\pgfqpoint{4.242393in}{1.506340in}}{\pgfqpoint{4.249234in}{1.509174in}}{\pgfqpoint{4.254278in}{1.514217in}}%
\pgfpathcurveto{\pgfqpoint{4.259322in}{1.519261in}}{\pgfqpoint{4.262156in}{1.526103in}}{\pgfqpoint{4.262156in}{1.533236in}}%
\pgfpathcurveto{\pgfqpoint{4.262156in}{1.540368in}}{\pgfqpoint{4.259322in}{1.547210in}}{\pgfqpoint{4.254278in}{1.552254in}}%
\pgfpathcurveto{\pgfqpoint{4.249234in}{1.557297in}}{\pgfqpoint{4.242393in}{1.560131in}}{\pgfqpoint{4.235260in}{1.560131in}}%
\pgfpathcurveto{\pgfqpoint{4.228127in}{1.560131in}}{\pgfqpoint{4.221285in}{1.557297in}}{\pgfqpoint{4.216242in}{1.552254in}}%
\pgfpathcurveto{\pgfqpoint{4.211198in}{1.547210in}}{\pgfqpoint{4.208364in}{1.540368in}}{\pgfqpoint{4.208364in}{1.533236in}}%
\pgfpathcurveto{\pgfqpoint{4.208364in}{1.526103in}}{\pgfqpoint{4.211198in}{1.519261in}}{\pgfqpoint{4.216242in}{1.514217in}}%
\pgfpathcurveto{\pgfqpoint{4.221285in}{1.509174in}}{\pgfqpoint{4.228127in}{1.506340in}}{\pgfqpoint{4.235260in}{1.506340in}}%
\pgfpathclose%
\pgfusepath{stroke,fill}%
\end{pgfscope}%
\begin{pgfscope}%
\pgfpathrectangle{\pgfqpoint{2.867647in}{0.500000in}}{\pgfqpoint{1.764706in}{1.700000in}}%
\pgfusepath{clip}%
\pgfsetbuttcap%
\pgfsetroundjoin%
\definecolor{currentfill}{rgb}{0.977657,0.891500,0.822809}%
\pgfsetfillcolor{currentfill}%
\pgfsetlinewidth{0.311001pt}%
\definecolor{currentstroke}{rgb}{1.000000,1.000000,1.000000}%
\pgfsetstrokecolor{currentstroke}%
\pgfsetdash{}{0pt}%
\pgfpathmoveto{\pgfqpoint{4.145852in}{1.282356in}}%
\pgfpathcurveto{\pgfqpoint{4.152984in}{1.282356in}}{\pgfqpoint{4.159826in}{1.285190in}}{\pgfqpoint{4.164870in}{1.290234in}}%
\pgfpathcurveto{\pgfqpoint{4.169913in}{1.295277in}}{\pgfqpoint{4.172747in}{1.302119in}}{\pgfqpoint{4.172747in}{1.309252in}}%
\pgfpathcurveto{\pgfqpoint{4.172747in}{1.316385in}}{\pgfqpoint{4.169913in}{1.323226in}}{\pgfqpoint{4.164870in}{1.328270in}}%
\pgfpathcurveto{\pgfqpoint{4.159826in}{1.333314in}}{\pgfqpoint{4.152984in}{1.336147in}}{\pgfqpoint{4.145852in}{1.336147in}}%
\pgfpathcurveto{\pgfqpoint{4.138719in}{1.336147in}}{\pgfqpoint{4.131877in}{1.333314in}}{\pgfqpoint{4.126833in}{1.328270in}}%
\pgfpathcurveto{\pgfqpoint{4.121790in}{1.323226in}}{\pgfqpoint{4.118956in}{1.316385in}}{\pgfqpoint{4.118956in}{1.309252in}}%
\pgfpathcurveto{\pgfqpoint{4.118956in}{1.302119in}}{\pgfqpoint{4.121790in}{1.295277in}}{\pgfqpoint{4.126833in}{1.290234in}}%
\pgfpathcurveto{\pgfqpoint{4.131877in}{1.285190in}}{\pgfqpoint{4.138719in}{1.282356in}}{\pgfqpoint{4.145852in}{1.282356in}}%
\pgfpathclose%
\pgfusepath{stroke,fill}%
\end{pgfscope}%
\begin{pgfscope}%
\pgfpathrectangle{\pgfqpoint{2.867647in}{0.500000in}}{\pgfqpoint{1.764706in}{1.700000in}}%
\pgfusepath{clip}%
\pgfsetbuttcap%
\pgfsetroundjoin%
\definecolor{currentfill}{rgb}{0.979891,0.908948,0.848279}%
\pgfsetfillcolor{currentfill}%
\pgfsetlinewidth{0.311001pt}%
\definecolor{currentstroke}{rgb}{1.000000,1.000000,1.000000}%
\pgfsetstrokecolor{currentstroke}%
\pgfsetdash{}{0pt}%
\pgfpathmoveto{\pgfqpoint{4.133670in}{1.528455in}}%
\pgfpathcurveto{\pgfqpoint{4.140803in}{1.528455in}}{\pgfqpoint{4.147644in}{1.531289in}}{\pgfqpoint{4.152688in}{1.536332in}}%
\pgfpathcurveto{\pgfqpoint{4.157732in}{1.541376in}}{\pgfqpoint{4.160565in}{1.548218in}}{\pgfqpoint{4.160565in}{1.555350in}}%
\pgfpathcurveto{\pgfqpoint{4.160565in}{1.562483in}}{\pgfqpoint{4.157732in}{1.569325in}}{\pgfqpoint{4.152688in}{1.574369in}}%
\pgfpathcurveto{\pgfqpoint{4.147644in}{1.579412in}}{\pgfqpoint{4.140803in}{1.582246in}}{\pgfqpoint{4.133670in}{1.582246in}}%
\pgfpathcurveto{\pgfqpoint{4.126537in}{1.582246in}}{\pgfqpoint{4.119695in}{1.579412in}}{\pgfqpoint{4.114652in}{1.574369in}}%
\pgfpathcurveto{\pgfqpoint{4.109608in}{1.569325in}}{\pgfqpoint{4.106774in}{1.562483in}}{\pgfqpoint{4.106774in}{1.555350in}}%
\pgfpathcurveto{\pgfqpoint{4.106774in}{1.548218in}}{\pgfqpoint{4.109608in}{1.541376in}}{\pgfqpoint{4.114652in}{1.536332in}}%
\pgfpathcurveto{\pgfqpoint{4.119695in}{1.531289in}}{\pgfqpoint{4.126537in}{1.528455in}}{\pgfqpoint{4.133670in}{1.528455in}}%
\pgfpathclose%
\pgfusepath{stroke,fill}%
\end{pgfscope}%
\begin{pgfscope}%
\pgfpathrectangle{\pgfqpoint{2.867647in}{0.500000in}}{\pgfqpoint{1.764706in}{1.700000in}}%
\pgfusepath{clip}%
\pgfsetbuttcap%
\pgfsetroundjoin%
\definecolor{currentfill}{rgb}{0.961115,0.566634,0.405693}%
\pgfsetfillcolor{currentfill}%
\pgfsetlinewidth{0.311001pt}%
\definecolor{currentstroke}{rgb}{1.000000,1.000000,1.000000}%
\pgfsetstrokecolor{currentstroke}%
\pgfsetdash{}{0pt}%
\pgfpathmoveto{\pgfqpoint{4.107492in}{0.854700in}}%
\pgfpathcurveto{\pgfqpoint{4.114625in}{0.854700in}}{\pgfqpoint{4.121466in}{0.857534in}}{\pgfqpoint{4.126510in}{0.862577in}}%
\pgfpathcurveto{\pgfqpoint{4.131553in}{0.867621in}}{\pgfqpoint{4.134387in}{0.874463in}}{\pgfqpoint{4.134387in}{0.881595in}}%
\pgfpathcurveto{\pgfqpoint{4.134387in}{0.888728in}}{\pgfqpoint{4.131553in}{0.895570in}}{\pgfqpoint{4.126510in}{0.900614in}}%
\pgfpathcurveto{\pgfqpoint{4.121466in}{0.905657in}}{\pgfqpoint{4.114625in}{0.908491in}}{\pgfqpoint{4.107492in}{0.908491in}}%
\pgfpathcurveto{\pgfqpoint{4.100359in}{0.908491in}}{\pgfqpoint{4.093517in}{0.905657in}}{\pgfqpoint{4.088474in}{0.900614in}}%
\pgfpathcurveto{\pgfqpoint{4.083430in}{0.895570in}}{\pgfqpoint{4.080596in}{0.888728in}}{\pgfqpoint{4.080596in}{0.881595in}}%
\pgfpathcurveto{\pgfqpoint{4.080596in}{0.874463in}}{\pgfqpoint{4.083430in}{0.867621in}}{\pgfqpoint{4.088474in}{0.862577in}}%
\pgfpathcurveto{\pgfqpoint{4.093517in}{0.857534in}}{\pgfqpoint{4.100359in}{0.854700in}}{\pgfqpoint{4.107492in}{0.854700in}}%
\pgfpathclose%
\pgfusepath{stroke,fill}%
\end{pgfscope}%
\begin{pgfscope}%
\pgfpathrectangle{\pgfqpoint{2.867647in}{0.500000in}}{\pgfqpoint{1.764706in}{1.700000in}}%
\pgfusepath{clip}%
\pgfsetbuttcap%
\pgfsetroundjoin%
\definecolor{currentfill}{rgb}{0.972201,0.839051,0.745789}%
\pgfsetfillcolor{currentfill}%
\pgfsetlinewidth{0.311001pt}%
\definecolor{currentstroke}{rgb}{1.000000,1.000000,1.000000}%
\pgfsetstrokecolor{currentstroke}%
\pgfsetdash{}{0pt}%
\pgfpathmoveto{\pgfqpoint{4.123756in}{0.976560in}}%
\pgfpathcurveto{\pgfqpoint{4.130889in}{0.976560in}}{\pgfqpoint{4.137730in}{0.979394in}}{\pgfqpoint{4.142774in}{0.984438in}}%
\pgfpathcurveto{\pgfqpoint{4.147818in}{0.989482in}}{\pgfqpoint{4.150652in}{0.996323in}}{\pgfqpoint{4.150652in}{1.003456in}}%
\pgfpathcurveto{\pgfqpoint{4.150652in}{1.010589in}}{\pgfqpoint{4.147818in}{1.017430in}}{\pgfqpoint{4.142774in}{1.022474in}}%
\pgfpathcurveto{\pgfqpoint{4.137730in}{1.027518in}}{\pgfqpoint{4.130889in}{1.030352in}}{\pgfqpoint{4.123756in}{1.030352in}}%
\pgfpathcurveto{\pgfqpoint{4.116623in}{1.030352in}}{\pgfqpoint{4.109781in}{1.027518in}}{\pgfqpoint{4.104738in}{1.022474in}}%
\pgfpathcurveto{\pgfqpoint{4.099694in}{1.017430in}}{\pgfqpoint{4.096860in}{1.010589in}}{\pgfqpoint{4.096860in}{1.003456in}}%
\pgfpathcurveto{\pgfqpoint{4.096860in}{0.996323in}}{\pgfqpoint{4.099694in}{0.989482in}}{\pgfqpoint{4.104738in}{0.984438in}}%
\pgfpathcurveto{\pgfqpoint{4.109781in}{0.979394in}}{\pgfqpoint{4.116623in}{0.976560in}}{\pgfqpoint{4.123756in}{0.976560in}}%
\pgfpathclose%
\pgfusepath{stroke,fill}%
\end{pgfscope}%
\begin{pgfscope}%
\pgfpathrectangle{\pgfqpoint{2.867647in}{0.500000in}}{\pgfqpoint{1.764706in}{1.700000in}}%
\pgfusepath{clip}%
\pgfsetbuttcap%
\pgfsetroundjoin%
\definecolor{currentfill}{rgb}{0.964306,0.663930,0.507747}%
\pgfsetfillcolor{currentfill}%
\pgfsetlinewidth{0.311001pt}%
\definecolor{currentstroke}{rgb}{1.000000,1.000000,1.000000}%
\pgfsetstrokecolor{currentstroke}%
\pgfsetdash{}{0pt}%
\pgfpathmoveto{\pgfqpoint{4.036761in}{1.465484in}}%
\pgfpathcurveto{\pgfqpoint{4.043894in}{1.465484in}}{\pgfqpoint{4.050736in}{1.468318in}}{\pgfqpoint{4.055779in}{1.473361in}}%
\pgfpathcurveto{\pgfqpoint{4.060823in}{1.478405in}}{\pgfqpoint{4.063657in}{1.485247in}}{\pgfqpoint{4.063657in}{1.492379in}}%
\pgfpathcurveto{\pgfqpoint{4.063657in}{1.499512in}}{\pgfqpoint{4.060823in}{1.506354in}}{\pgfqpoint{4.055779in}{1.511397in}}%
\pgfpathcurveto{\pgfqpoint{4.050736in}{1.516441in}}{\pgfqpoint{4.043894in}{1.519275in}}{\pgfqpoint{4.036761in}{1.519275in}}%
\pgfpathcurveto{\pgfqpoint{4.029628in}{1.519275in}}{\pgfqpoint{4.022787in}{1.516441in}}{\pgfqpoint{4.017743in}{1.511397in}}%
\pgfpathcurveto{\pgfqpoint{4.012699in}{1.506354in}}{\pgfqpoint{4.009865in}{1.499512in}}{\pgfqpoint{4.009865in}{1.492379in}}%
\pgfpathcurveto{\pgfqpoint{4.009865in}{1.485247in}}{\pgfqpoint{4.012699in}{1.478405in}}{\pgfqpoint{4.017743in}{1.473361in}}%
\pgfpathcurveto{\pgfqpoint{4.022787in}{1.468318in}}{\pgfqpoint{4.029628in}{1.465484in}}{\pgfqpoint{4.036761in}{1.465484in}}%
\pgfpathclose%
\pgfusepath{stroke,fill}%
\end{pgfscope}%
\begin{pgfscope}%
\pgfpathrectangle{\pgfqpoint{2.867647in}{0.500000in}}{\pgfqpoint{1.764706in}{1.700000in}}%
\pgfusepath{clip}%
\pgfsetbuttcap%
\pgfsetroundjoin%
\definecolor{currentfill}{rgb}{0.965928,0.738443,0.600540}%
\pgfsetfillcolor{currentfill}%
\pgfsetlinewidth{0.311001pt}%
\definecolor{currentstroke}{rgb}{1.000000,1.000000,1.000000}%
\pgfsetstrokecolor{currentstroke}%
\pgfsetdash{}{0pt}%
\pgfpathmoveto{\pgfqpoint{4.066560in}{1.749331in}}%
\pgfpathcurveto{\pgfqpoint{4.073693in}{1.749331in}}{\pgfqpoint{4.080535in}{1.752165in}}{\pgfqpoint{4.085578in}{1.757208in}}%
\pgfpathcurveto{\pgfqpoint{4.090622in}{1.762252in}}{\pgfqpoint{4.093456in}{1.769094in}}{\pgfqpoint{4.093456in}{1.776227in}}%
\pgfpathcurveto{\pgfqpoint{4.093456in}{1.783359in}}{\pgfqpoint{4.090622in}{1.790201in}}{\pgfqpoint{4.085578in}{1.795245in}}%
\pgfpathcurveto{\pgfqpoint{4.080535in}{1.800288in}}{\pgfqpoint{4.073693in}{1.803122in}}{\pgfqpoint{4.066560in}{1.803122in}}%
\pgfpathcurveto{\pgfqpoint{4.059427in}{1.803122in}}{\pgfqpoint{4.052586in}{1.800288in}}{\pgfqpoint{4.047542in}{1.795245in}}%
\pgfpathcurveto{\pgfqpoint{4.042498in}{1.790201in}}{\pgfqpoint{4.039664in}{1.783359in}}{\pgfqpoint{4.039664in}{1.776227in}}%
\pgfpathcurveto{\pgfqpoint{4.039664in}{1.769094in}}{\pgfqpoint{4.042498in}{1.762252in}}{\pgfqpoint{4.047542in}{1.757208in}}%
\pgfpathcurveto{\pgfqpoint{4.052586in}{1.752165in}}{\pgfqpoint{4.059427in}{1.749331in}}{\pgfqpoint{4.066560in}{1.749331in}}%
\pgfpathclose%
\pgfusepath{stroke,fill}%
\end{pgfscope}%
\begin{pgfscope}%
\pgfpathrectangle{\pgfqpoint{2.867647in}{0.500000in}}{\pgfqpoint{1.764706in}{1.700000in}}%
\pgfusepath{clip}%
\pgfsetbuttcap%
\pgfsetroundjoin%
\definecolor{currentfill}{rgb}{0.981377,0.920617,0.865369}%
\pgfsetfillcolor{currentfill}%
\pgfsetlinewidth{0.311001pt}%
\definecolor{currentstroke}{rgb}{1.000000,1.000000,1.000000}%
\pgfsetstrokecolor{currentstroke}%
\pgfsetdash{}{0pt}%
\pgfpathmoveto{\pgfqpoint{4.185975in}{1.309917in}}%
\pgfpathcurveto{\pgfqpoint{4.193107in}{1.309917in}}{\pgfqpoint{4.199949in}{1.312751in}}{\pgfqpoint{4.204993in}{1.317795in}}%
\pgfpathcurveto{\pgfqpoint{4.210036in}{1.322839in}}{\pgfqpoint{4.212870in}{1.329680in}}{\pgfqpoint{4.212870in}{1.336813in}}%
\pgfpathcurveto{\pgfqpoint{4.212870in}{1.343946in}}{\pgfqpoint{4.210036in}{1.350788in}}{\pgfqpoint{4.204993in}{1.355831in}}%
\pgfpathcurveto{\pgfqpoint{4.199949in}{1.360875in}}{\pgfqpoint{4.193107in}{1.363709in}}{\pgfqpoint{4.185975in}{1.363709in}}%
\pgfpathcurveto{\pgfqpoint{4.178842in}{1.363709in}}{\pgfqpoint{4.172000in}{1.360875in}}{\pgfqpoint{4.166957in}{1.355831in}}%
\pgfpathcurveto{\pgfqpoint{4.161913in}{1.350788in}}{\pgfqpoint{4.159079in}{1.343946in}}{\pgfqpoint{4.159079in}{1.336813in}}%
\pgfpathcurveto{\pgfqpoint{4.159079in}{1.329680in}}{\pgfqpoint{4.161913in}{1.322839in}}{\pgfqpoint{4.166957in}{1.317795in}}%
\pgfpathcurveto{\pgfqpoint{4.172000in}{1.312751in}}{\pgfqpoint{4.178842in}{1.309917in}}{\pgfqpoint{4.185975in}{1.309917in}}%
\pgfpathclose%
\pgfusepath{stroke,fill}%
\end{pgfscope}%
\begin{pgfscope}%
\pgfpathrectangle{\pgfqpoint{2.867647in}{0.500000in}}{\pgfqpoint{1.764706in}{1.700000in}}%
\pgfusepath{clip}%
\pgfsetbuttcap%
\pgfsetroundjoin%
\definecolor{currentfill}{rgb}{0.964679,0.682838,0.530002}%
\pgfsetfillcolor{currentfill}%
\pgfsetlinewidth{0.311001pt}%
\definecolor{currentstroke}{rgb}{1.000000,1.000000,1.000000}%
\pgfsetstrokecolor{currentstroke}%
\pgfsetdash{}{0pt}%
\pgfpathmoveto{\pgfqpoint{4.044865in}{1.455836in}}%
\pgfpathcurveto{\pgfqpoint{4.051998in}{1.455836in}}{\pgfqpoint{4.058840in}{1.458670in}}{\pgfqpoint{4.063883in}{1.463713in}}%
\pgfpathcurveto{\pgfqpoint{4.068927in}{1.468757in}}{\pgfqpoint{4.071761in}{1.475599in}}{\pgfqpoint{4.071761in}{1.482731in}}%
\pgfpathcurveto{\pgfqpoint{4.071761in}{1.489864in}}{\pgfqpoint{4.068927in}{1.496706in}}{\pgfqpoint{4.063883in}{1.501750in}}%
\pgfpathcurveto{\pgfqpoint{4.058840in}{1.506793in}}{\pgfqpoint{4.051998in}{1.509627in}}{\pgfqpoint{4.044865in}{1.509627in}}%
\pgfpathcurveto{\pgfqpoint{4.037732in}{1.509627in}}{\pgfqpoint{4.030891in}{1.506793in}}{\pgfqpoint{4.025847in}{1.501750in}}%
\pgfpathcurveto{\pgfqpoint{4.020803in}{1.496706in}}{\pgfqpoint{4.017969in}{1.489864in}}{\pgfqpoint{4.017969in}{1.482731in}}%
\pgfpathcurveto{\pgfqpoint{4.017969in}{1.475599in}}{\pgfqpoint{4.020803in}{1.468757in}}{\pgfqpoint{4.025847in}{1.463713in}}%
\pgfpathcurveto{\pgfqpoint{4.030891in}{1.458670in}}{\pgfqpoint{4.037732in}{1.455836in}}{\pgfqpoint{4.044865in}{1.455836in}}%
\pgfpathclose%
\pgfusepath{stroke,fill}%
\end{pgfscope}%
\begin{pgfscope}%
\pgfpathrectangle{\pgfqpoint{2.867647in}{0.500000in}}{\pgfqpoint{1.764706in}{1.700000in}}%
\pgfusepath{clip}%
\pgfsetbuttcap%
\pgfsetroundjoin%
\definecolor{currentfill}{rgb}{0.979124,0.903132,0.839793}%
\pgfsetfillcolor{currentfill}%
\pgfsetlinewidth{0.311001pt}%
\definecolor{currentstroke}{rgb}{1.000000,1.000000,1.000000}%
\pgfsetstrokecolor{currentstroke}%
\pgfsetdash{}{0pt}%
\pgfpathmoveto{\pgfqpoint{4.159342in}{1.390539in}}%
\pgfpathcurveto{\pgfqpoint{4.166475in}{1.390539in}}{\pgfqpoint{4.173316in}{1.393373in}}{\pgfqpoint{4.178360in}{1.398417in}}%
\pgfpathcurveto{\pgfqpoint{4.183404in}{1.403461in}}{\pgfqpoint{4.186238in}{1.410302in}}{\pgfqpoint{4.186238in}{1.417435in}}%
\pgfpathcurveto{\pgfqpoint{4.186238in}{1.424568in}}{\pgfqpoint{4.183404in}{1.431410in}}{\pgfqpoint{4.178360in}{1.436453in}}%
\pgfpathcurveto{\pgfqpoint{4.173316in}{1.441497in}}{\pgfqpoint{4.166475in}{1.444331in}}{\pgfqpoint{4.159342in}{1.444331in}}%
\pgfpathcurveto{\pgfqpoint{4.152209in}{1.444331in}}{\pgfqpoint{4.145368in}{1.441497in}}{\pgfqpoint{4.140324in}{1.436453in}}%
\pgfpathcurveto{\pgfqpoint{4.135280in}{1.431410in}}{\pgfqpoint{4.132446in}{1.424568in}}{\pgfqpoint{4.132446in}{1.417435in}}%
\pgfpathcurveto{\pgfqpoint{4.132446in}{1.410302in}}{\pgfqpoint{4.135280in}{1.403461in}}{\pgfqpoint{4.140324in}{1.398417in}}%
\pgfpathcurveto{\pgfqpoint{4.145368in}{1.393373in}}{\pgfqpoint{4.152209in}{1.390539in}}{\pgfqpoint{4.159342in}{1.390539in}}%
\pgfpathclose%
\pgfusepath{stroke,fill}%
\end{pgfscope}%
\begin{pgfscope}%
\pgfpathrectangle{\pgfqpoint{2.867647in}{0.500000in}}{\pgfqpoint{1.764706in}{1.700000in}}%
\pgfusepath{clip}%
\pgfsetbuttcap%
\pgfsetroundjoin%
\definecolor{currentfill}{rgb}{0.978376,0.897317,0.831308}%
\pgfsetfillcolor{currentfill}%
\pgfsetlinewidth{0.311001pt}%
\definecolor{currentstroke}{rgb}{1.000000,1.000000,1.000000}%
\pgfsetstrokecolor{currentstroke}%
\pgfsetdash{}{0pt}%
\pgfpathmoveto{\pgfqpoint{4.227380in}{1.251188in}}%
\pgfpathcurveto{\pgfqpoint{4.234513in}{1.251188in}}{\pgfqpoint{4.241354in}{1.254022in}}{\pgfqpoint{4.246398in}{1.259066in}}%
\pgfpathcurveto{\pgfqpoint{4.251442in}{1.264109in}}{\pgfqpoint{4.254275in}{1.270951in}}{\pgfqpoint{4.254275in}{1.278084in}}%
\pgfpathcurveto{\pgfqpoint{4.254275in}{1.285217in}}{\pgfqpoint{4.251442in}{1.292058in}}{\pgfqpoint{4.246398in}{1.297102in}}%
\pgfpathcurveto{\pgfqpoint{4.241354in}{1.302146in}}{\pgfqpoint{4.234513in}{1.304979in}}{\pgfqpoint{4.227380in}{1.304979in}}%
\pgfpathcurveto{\pgfqpoint{4.220247in}{1.304979in}}{\pgfqpoint{4.213405in}{1.302146in}}{\pgfqpoint{4.208362in}{1.297102in}}%
\pgfpathcurveto{\pgfqpoint{4.203318in}{1.292058in}}{\pgfqpoint{4.200484in}{1.285217in}}{\pgfqpoint{4.200484in}{1.278084in}}%
\pgfpathcurveto{\pgfqpoint{4.200484in}{1.270951in}}{\pgfqpoint{4.203318in}{1.264109in}}{\pgfqpoint{4.208362in}{1.259066in}}%
\pgfpathcurveto{\pgfqpoint{4.213405in}{1.254022in}}{\pgfqpoint{4.220247in}{1.251188in}}{\pgfqpoint{4.227380in}{1.251188in}}%
\pgfpathclose%
\pgfusepath{stroke,fill}%
\end{pgfscope}%
\begin{pgfscope}%
\pgfpathrectangle{\pgfqpoint{2.867647in}{0.500000in}}{\pgfqpoint{1.764706in}{1.700000in}}%
\pgfusepath{clip}%
\pgfsetbuttcap%
\pgfsetroundjoin%
\definecolor{currentfill}{rgb}{0.979124,0.903132,0.839793}%
\pgfsetfillcolor{currentfill}%
\pgfsetlinewidth{0.311001pt}%
\definecolor{currentstroke}{rgb}{1.000000,1.000000,1.000000}%
\pgfsetstrokecolor{currentstroke}%
\pgfsetdash{}{0pt}%
\pgfpathmoveto{\pgfqpoint{4.156037in}{1.308824in}}%
\pgfpathcurveto{\pgfqpoint{4.163170in}{1.308824in}}{\pgfqpoint{4.170011in}{1.311658in}}{\pgfqpoint{4.175055in}{1.316701in}}%
\pgfpathcurveto{\pgfqpoint{4.180099in}{1.321745in}}{\pgfqpoint{4.182933in}{1.328587in}}{\pgfqpoint{4.182933in}{1.335719in}}%
\pgfpathcurveto{\pgfqpoint{4.182933in}{1.342852in}}{\pgfqpoint{4.180099in}{1.349694in}}{\pgfqpoint{4.175055in}{1.354738in}}%
\pgfpathcurveto{\pgfqpoint{4.170011in}{1.359781in}}{\pgfqpoint{4.163170in}{1.362615in}}{\pgfqpoint{4.156037in}{1.362615in}}%
\pgfpathcurveto{\pgfqpoint{4.148904in}{1.362615in}}{\pgfqpoint{4.142062in}{1.359781in}}{\pgfqpoint{4.137019in}{1.354738in}}%
\pgfpathcurveto{\pgfqpoint{4.131975in}{1.349694in}}{\pgfqpoint{4.129141in}{1.342852in}}{\pgfqpoint{4.129141in}{1.335719in}}%
\pgfpathcurveto{\pgfqpoint{4.129141in}{1.328587in}}{\pgfqpoint{4.131975in}{1.321745in}}{\pgfqpoint{4.137019in}{1.316701in}}%
\pgfpathcurveto{\pgfqpoint{4.142062in}{1.311658in}}{\pgfqpoint{4.148904in}{1.308824in}}{\pgfqpoint{4.156037in}{1.308824in}}%
\pgfpathclose%
\pgfusepath{stroke,fill}%
\end{pgfscope}%
\begin{pgfscope}%
\pgfpathrectangle{\pgfqpoint{2.867647in}{0.500000in}}{\pgfqpoint{1.764706in}{1.700000in}}%
\pgfusepath{clip}%
\pgfsetbuttcap%
\pgfsetroundjoin%
\definecolor{currentfill}{rgb}{0.966328,0.750560,0.616961}%
\pgfsetfillcolor{currentfill}%
\pgfsetlinewidth{0.311001pt}%
\definecolor{currentstroke}{rgb}{1.000000,1.000000,1.000000}%
\pgfsetstrokecolor{currentstroke}%
\pgfsetdash{}{0pt}%
\pgfpathmoveto{\pgfqpoint{4.099300in}{0.915167in}}%
\pgfpathcurveto{\pgfqpoint{4.106433in}{0.915167in}}{\pgfqpoint{4.113274in}{0.918001in}}{\pgfqpoint{4.118318in}{0.923044in}}%
\pgfpathcurveto{\pgfqpoint{4.123362in}{0.928088in}}{\pgfqpoint{4.126196in}{0.934930in}}{\pgfqpoint{4.126196in}{0.942063in}}%
\pgfpathcurveto{\pgfqpoint{4.126196in}{0.949195in}}{\pgfqpoint{4.123362in}{0.956037in}}{\pgfqpoint{4.118318in}{0.961081in}}%
\pgfpathcurveto{\pgfqpoint{4.113274in}{0.966124in}}{\pgfqpoint{4.106433in}{0.968958in}}{\pgfqpoint{4.099300in}{0.968958in}}%
\pgfpathcurveto{\pgfqpoint{4.092167in}{0.968958in}}{\pgfqpoint{4.085325in}{0.966124in}}{\pgfqpoint{4.080282in}{0.961081in}}%
\pgfpathcurveto{\pgfqpoint{4.075238in}{0.956037in}}{\pgfqpoint{4.072404in}{0.949195in}}{\pgfqpoint{4.072404in}{0.942063in}}%
\pgfpathcurveto{\pgfqpoint{4.072404in}{0.934930in}}{\pgfqpoint{4.075238in}{0.928088in}}{\pgfqpoint{4.080282in}{0.923044in}}%
\pgfpathcurveto{\pgfqpoint{4.085325in}{0.918001in}}{\pgfqpoint{4.092167in}{0.915167in}}{\pgfqpoint{4.099300in}{0.915167in}}%
\pgfpathclose%
\pgfusepath{stroke,fill}%
\end{pgfscope}%
\begin{pgfscope}%
\pgfpathrectangle{\pgfqpoint{2.867647in}{0.500000in}}{\pgfqpoint{1.764706in}{1.700000in}}%
\pgfusepath{clip}%
\pgfsetbuttcap%
\pgfsetroundjoin%
\definecolor{currentfill}{rgb}{0.970255,0.815666,0.711203}%
\pgfsetfillcolor{currentfill}%
\pgfsetlinewidth{0.311001pt}%
\definecolor{currentstroke}{rgb}{1.000000,1.000000,1.000000}%
\pgfsetstrokecolor{currentstroke}%
\pgfsetdash{}{0pt}%
\pgfpathmoveto{\pgfqpoint{4.214057in}{1.585399in}}%
\pgfpathcurveto{\pgfqpoint{4.221190in}{1.585399in}}{\pgfqpoint{4.228031in}{1.588233in}}{\pgfqpoint{4.233075in}{1.593277in}}%
\pgfpathcurveto{\pgfqpoint{4.238119in}{1.598320in}}{\pgfqpoint{4.240952in}{1.605162in}}{\pgfqpoint{4.240952in}{1.612295in}}%
\pgfpathcurveto{\pgfqpoint{4.240952in}{1.619428in}}{\pgfqpoint{4.238119in}{1.626269in}}{\pgfqpoint{4.233075in}{1.631313in}}%
\pgfpathcurveto{\pgfqpoint{4.228031in}{1.636356in}}{\pgfqpoint{4.221190in}{1.639190in}}{\pgfqpoint{4.214057in}{1.639190in}}%
\pgfpathcurveto{\pgfqpoint{4.206924in}{1.639190in}}{\pgfqpoint{4.200082in}{1.636356in}}{\pgfqpoint{4.195039in}{1.631313in}}%
\pgfpathcurveto{\pgfqpoint{4.189995in}{1.626269in}}{\pgfqpoint{4.187161in}{1.619428in}}{\pgfqpoint{4.187161in}{1.612295in}}%
\pgfpathcurveto{\pgfqpoint{4.187161in}{1.605162in}}{\pgfqpoint{4.189995in}{1.598320in}}{\pgfqpoint{4.195039in}{1.593277in}}%
\pgfpathcurveto{\pgfqpoint{4.200082in}{1.588233in}}{\pgfqpoint{4.206924in}{1.585399in}}{\pgfqpoint{4.214057in}{1.585399in}}%
\pgfpathclose%
\pgfusepath{stroke,fill}%
\end{pgfscope}%
\begin{pgfscope}%
\pgfpathrectangle{\pgfqpoint{2.867647in}{0.500000in}}{\pgfqpoint{1.764706in}{1.700000in}}%
\pgfusepath{clip}%
\pgfsetbuttcap%
\pgfsetroundjoin%
\definecolor{currentfill}{rgb}{0.971694,0.833208,0.737161}%
\pgfsetfillcolor{currentfill}%
\pgfsetlinewidth{0.311001pt}%
\definecolor{currentstroke}{rgb}{1.000000,1.000000,1.000000}%
\pgfsetstrokecolor{currentstroke}%
\pgfsetdash{}{0pt}%
\pgfpathmoveto{\pgfqpoint{4.210876in}{1.062133in}}%
\pgfpathcurveto{\pgfqpoint{4.218009in}{1.062133in}}{\pgfqpoint{4.224851in}{1.064967in}}{\pgfqpoint{4.229894in}{1.070011in}}%
\pgfpathcurveto{\pgfqpoint{4.234938in}{1.075055in}}{\pgfqpoint{4.237772in}{1.081896in}}{\pgfqpoint{4.237772in}{1.089029in}}%
\pgfpathcurveto{\pgfqpoint{4.237772in}{1.096162in}}{\pgfqpoint{4.234938in}{1.103004in}}{\pgfqpoint{4.229894in}{1.108047in}}%
\pgfpathcurveto{\pgfqpoint{4.224851in}{1.113091in}}{\pgfqpoint{4.218009in}{1.115925in}}{\pgfqpoint{4.210876in}{1.115925in}}%
\pgfpathcurveto{\pgfqpoint{4.203743in}{1.115925in}}{\pgfqpoint{4.196902in}{1.113091in}}{\pgfqpoint{4.191858in}{1.108047in}}%
\pgfpathcurveto{\pgfqpoint{4.186814in}{1.103004in}}{\pgfqpoint{4.183980in}{1.096162in}}{\pgfqpoint{4.183980in}{1.089029in}}%
\pgfpathcurveto{\pgfqpoint{4.183980in}{1.081896in}}{\pgfqpoint{4.186814in}{1.075055in}}{\pgfqpoint{4.191858in}{1.070011in}}%
\pgfpathcurveto{\pgfqpoint{4.196902in}{1.064967in}}{\pgfqpoint{4.203743in}{1.062133in}}{\pgfqpoint{4.210876in}{1.062133in}}%
\pgfpathclose%
\pgfusepath{stroke,fill}%
\end{pgfscope}%
\begin{pgfscope}%
\pgfpathrectangle{\pgfqpoint{2.867647in}{0.500000in}}{\pgfqpoint{1.764706in}{1.700000in}}%
\pgfusepath{clip}%
\pgfsetbuttcap%
\pgfsetroundjoin%
\definecolor{currentfill}{rgb}{0.976961,0.885681,0.814303}%
\pgfsetfillcolor{currentfill}%
\pgfsetlinewidth{0.311001pt}%
\definecolor{currentstroke}{rgb}{1.000000,1.000000,1.000000}%
\pgfsetstrokecolor{currentstroke}%
\pgfsetdash{}{0pt}%
\pgfpathmoveto{\pgfqpoint{4.125639in}{1.452057in}}%
\pgfpathcurveto{\pgfqpoint{4.132771in}{1.452057in}}{\pgfqpoint{4.139613in}{1.454890in}}{\pgfqpoint{4.144657in}{1.459934in}}%
\pgfpathcurveto{\pgfqpoint{4.149700in}{1.464978in}}{\pgfqpoint{4.152534in}{1.471819in}}{\pgfqpoint{4.152534in}{1.478952in}}%
\pgfpathcurveto{\pgfqpoint{4.152534in}{1.486085in}}{\pgfqpoint{4.149700in}{1.492927in}}{\pgfqpoint{4.144657in}{1.497970in}}%
\pgfpathcurveto{\pgfqpoint{4.139613in}{1.503014in}}{\pgfqpoint{4.132771in}{1.505848in}}{\pgfqpoint{4.125639in}{1.505848in}}%
\pgfpathcurveto{\pgfqpoint{4.118506in}{1.505848in}}{\pgfqpoint{4.111664in}{1.503014in}}{\pgfqpoint{4.106620in}{1.497970in}}%
\pgfpathcurveto{\pgfqpoint{4.101577in}{1.492927in}}{\pgfqpoint{4.098743in}{1.486085in}}{\pgfqpoint{4.098743in}{1.478952in}}%
\pgfpathcurveto{\pgfqpoint{4.098743in}{1.471819in}}{\pgfqpoint{4.101577in}{1.464978in}}{\pgfqpoint{4.106620in}{1.459934in}}%
\pgfpathcurveto{\pgfqpoint{4.111664in}{1.454890in}}{\pgfqpoint{4.118506in}{1.452057in}}{\pgfqpoint{4.125639in}{1.452057in}}%
\pgfpathclose%
\pgfusepath{stroke,fill}%
\end{pgfscope}%
\begin{pgfscope}%
\pgfpathrectangle{\pgfqpoint{2.867647in}{0.500000in}}{\pgfqpoint{1.764706in}{1.700000in}}%
\pgfusepath{clip}%
\pgfsetbuttcap%
\pgfsetroundjoin%
\definecolor{currentfill}{rgb}{0.981377,0.920617,0.865369}%
\pgfsetfillcolor{currentfill}%
\pgfsetlinewidth{0.311001pt}%
\definecolor{currentstroke}{rgb}{1.000000,1.000000,1.000000}%
\pgfsetstrokecolor{currentstroke}%
\pgfsetdash{}{0pt}%
\pgfpathmoveto{\pgfqpoint{4.197247in}{1.307224in}}%
\pgfpathcurveto{\pgfqpoint{4.204380in}{1.307224in}}{\pgfqpoint{4.211221in}{1.310058in}}{\pgfqpoint{4.216265in}{1.315101in}}%
\pgfpathcurveto{\pgfqpoint{4.221308in}{1.320145in}}{\pgfqpoint{4.224142in}{1.326987in}}{\pgfqpoint{4.224142in}{1.334119in}}%
\pgfpathcurveto{\pgfqpoint{4.224142in}{1.341252in}}{\pgfqpoint{4.221308in}{1.348094in}}{\pgfqpoint{4.216265in}{1.353137in}}%
\pgfpathcurveto{\pgfqpoint{4.211221in}{1.358181in}}{\pgfqpoint{4.204380in}{1.361015in}}{\pgfqpoint{4.197247in}{1.361015in}}%
\pgfpathcurveto{\pgfqpoint{4.190114in}{1.361015in}}{\pgfqpoint{4.183272in}{1.358181in}}{\pgfqpoint{4.178229in}{1.353137in}}%
\pgfpathcurveto{\pgfqpoint{4.173185in}{1.348094in}}{\pgfqpoint{4.170351in}{1.341252in}}{\pgfqpoint{4.170351in}{1.334119in}}%
\pgfpathcurveto{\pgfqpoint{4.170351in}{1.326987in}}{\pgfqpoint{4.173185in}{1.320145in}}{\pgfqpoint{4.178229in}{1.315101in}}%
\pgfpathcurveto{\pgfqpoint{4.183272in}{1.310058in}}{\pgfqpoint{4.190114in}{1.307224in}}{\pgfqpoint{4.197247in}{1.307224in}}%
\pgfpathclose%
\pgfusepath{stroke,fill}%
\end{pgfscope}%
\begin{pgfscope}%
\pgfpathrectangle{\pgfqpoint{2.867647in}{0.500000in}}{\pgfqpoint{1.764706in}{1.700000in}}%
\pgfusepath{clip}%
\pgfsetbuttcap%
\pgfsetroundjoin%
\definecolor{currentfill}{rgb}{0.975018,0.868213,0.788710}%
\pgfsetfillcolor{currentfill}%
\pgfsetlinewidth{0.311001pt}%
\definecolor{currentstroke}{rgb}{1.000000,1.000000,1.000000}%
\pgfsetstrokecolor{currentstroke}%
\pgfsetdash{}{0pt}%
\pgfpathmoveto{\pgfqpoint{4.130329in}{1.394176in}}%
\pgfpathcurveto{\pgfqpoint{4.137462in}{1.394176in}}{\pgfqpoint{4.144304in}{1.397010in}}{\pgfqpoint{4.149347in}{1.402054in}}%
\pgfpathcurveto{\pgfqpoint{4.154391in}{1.407097in}}{\pgfqpoint{4.157225in}{1.413939in}}{\pgfqpoint{4.157225in}{1.421072in}}%
\pgfpathcurveto{\pgfqpoint{4.157225in}{1.428204in}}{\pgfqpoint{4.154391in}{1.435046in}}{\pgfqpoint{4.149347in}{1.440090in}}%
\pgfpathcurveto{\pgfqpoint{4.144304in}{1.445133in}}{\pgfqpoint{4.137462in}{1.447967in}}{\pgfqpoint{4.130329in}{1.447967in}}%
\pgfpathcurveto{\pgfqpoint{4.123196in}{1.447967in}}{\pgfqpoint{4.116355in}{1.445133in}}{\pgfqpoint{4.111311in}{1.440090in}}%
\pgfpathcurveto{\pgfqpoint{4.106267in}{1.435046in}}{\pgfqpoint{4.103434in}{1.428204in}}{\pgfqpoint{4.103434in}{1.421072in}}%
\pgfpathcurveto{\pgfqpoint{4.103434in}{1.413939in}}{\pgfqpoint{4.106267in}{1.407097in}}{\pgfqpoint{4.111311in}{1.402054in}}%
\pgfpathcurveto{\pgfqpoint{4.116355in}{1.397010in}}{\pgfqpoint{4.123196in}{1.394176in}}{\pgfqpoint{4.130329in}{1.394176in}}%
\pgfpathclose%
\pgfusepath{stroke,fill}%
\end{pgfscope}%
\begin{pgfscope}%
\pgfpathrectangle{\pgfqpoint{2.867647in}{0.500000in}}{\pgfqpoint{1.764706in}{1.700000in}}%
\pgfusepath{clip}%
\pgfsetbuttcap%
\pgfsetroundjoin%
\definecolor{currentfill}{rgb}{0.964920,0.695342,0.545192}%
\pgfsetfillcolor{currentfill}%
\pgfsetlinewidth{0.311001pt}%
\definecolor{currentstroke}{rgb}{1.000000,1.000000,1.000000}%
\pgfsetstrokecolor{currentstroke}%
\pgfsetdash{}{0pt}%
\pgfpathmoveto{\pgfqpoint{4.300015in}{1.439557in}}%
\pgfpathcurveto{\pgfqpoint{4.307148in}{1.439557in}}{\pgfqpoint{4.313989in}{1.442391in}}{\pgfqpoint{4.319033in}{1.447434in}}%
\pgfpathcurveto{\pgfqpoint{4.324077in}{1.452478in}}{\pgfqpoint{4.326910in}{1.459320in}}{\pgfqpoint{4.326910in}{1.466452in}}%
\pgfpathcurveto{\pgfqpoint{4.326910in}{1.473585in}}{\pgfqpoint{4.324077in}{1.480427in}}{\pgfqpoint{4.319033in}{1.485471in}}%
\pgfpathcurveto{\pgfqpoint{4.313989in}{1.490514in}}{\pgfqpoint{4.307148in}{1.493348in}}{\pgfqpoint{4.300015in}{1.493348in}}%
\pgfpathcurveto{\pgfqpoint{4.292882in}{1.493348in}}{\pgfqpoint{4.286040in}{1.490514in}}{\pgfqpoint{4.280997in}{1.485471in}}%
\pgfpathcurveto{\pgfqpoint{4.275953in}{1.480427in}}{\pgfqpoint{4.273119in}{1.473585in}}{\pgfqpoint{4.273119in}{1.466452in}}%
\pgfpathcurveto{\pgfqpoint{4.273119in}{1.459320in}}{\pgfqpoint{4.275953in}{1.452478in}}{\pgfqpoint{4.280997in}{1.447434in}}%
\pgfpathcurveto{\pgfqpoint{4.286040in}{1.442391in}}{\pgfqpoint{4.292882in}{1.439557in}}{\pgfqpoint{4.300015in}{1.439557in}}%
\pgfpathclose%
\pgfusepath{stroke,fill}%
\end{pgfscope}%
\begin{pgfscope}%
\pgfpathrectangle{\pgfqpoint{2.867647in}{0.500000in}}{\pgfqpoint{1.764706in}{1.700000in}}%
\pgfusepath{clip}%
\pgfsetbuttcap%
\pgfsetroundjoin%
\definecolor{currentfill}{rgb}{0.978376,0.897317,0.831308}%
\pgfsetfillcolor{currentfill}%
\pgfsetlinewidth{0.311001pt}%
\definecolor{currentstroke}{rgb}{1.000000,1.000000,1.000000}%
\pgfsetstrokecolor{currentstroke}%
\pgfsetdash{}{0pt}%
\pgfpathmoveto{\pgfqpoint{4.231145in}{1.281824in}}%
\pgfpathcurveto{\pgfqpoint{4.238278in}{1.281824in}}{\pgfqpoint{4.245120in}{1.284658in}}{\pgfqpoint{4.250163in}{1.289702in}}%
\pgfpathcurveto{\pgfqpoint{4.255207in}{1.294745in}}{\pgfqpoint{4.258041in}{1.301587in}}{\pgfqpoint{4.258041in}{1.308720in}}%
\pgfpathcurveto{\pgfqpoint{4.258041in}{1.315853in}}{\pgfqpoint{4.255207in}{1.322694in}}{\pgfqpoint{4.250163in}{1.327738in}}%
\pgfpathcurveto{\pgfqpoint{4.245120in}{1.332782in}}{\pgfqpoint{4.238278in}{1.335616in}}{\pgfqpoint{4.231145in}{1.335616in}}%
\pgfpathcurveto{\pgfqpoint{4.224013in}{1.335616in}}{\pgfqpoint{4.217171in}{1.332782in}}{\pgfqpoint{4.212127in}{1.327738in}}%
\pgfpathcurveto{\pgfqpoint{4.207084in}{1.322694in}}{\pgfqpoint{4.204250in}{1.315853in}}{\pgfqpoint{4.204250in}{1.308720in}}%
\pgfpathcurveto{\pgfqpoint{4.204250in}{1.301587in}}{\pgfqpoint{4.207084in}{1.294745in}}{\pgfqpoint{4.212127in}{1.289702in}}%
\pgfpathcurveto{\pgfqpoint{4.217171in}{1.284658in}}{\pgfqpoint{4.224013in}{1.281824in}}{\pgfqpoint{4.231145in}{1.281824in}}%
\pgfpathclose%
\pgfusepath{stroke,fill}%
\end{pgfscope}%
\begin{pgfscope}%
\pgfpathrectangle{\pgfqpoint{2.867647in}{0.500000in}}{\pgfqpoint{1.764706in}{1.700000in}}%
\pgfusepath{clip}%
\pgfsetbuttcap%
\pgfsetroundjoin%
\definecolor{currentfill}{rgb}{0.967398,0.774513,0.650573}%
\pgfsetfillcolor{currentfill}%
\pgfsetlinewidth{0.311001pt}%
\definecolor{currentstroke}{rgb}{1.000000,1.000000,1.000000}%
\pgfsetstrokecolor{currentstroke}%
\pgfsetdash{}{0pt}%
\pgfpathmoveto{\pgfqpoint{4.212152in}{1.020480in}}%
\pgfpathcurveto{\pgfqpoint{4.219285in}{1.020480in}}{\pgfqpoint{4.226126in}{1.023314in}}{\pgfqpoint{4.231170in}{1.028357in}}%
\pgfpathcurveto{\pgfqpoint{4.236214in}{1.033401in}}{\pgfqpoint{4.239047in}{1.040243in}}{\pgfqpoint{4.239047in}{1.047376in}}%
\pgfpathcurveto{\pgfqpoint{4.239047in}{1.054508in}}{\pgfqpoint{4.236214in}{1.061350in}}{\pgfqpoint{4.231170in}{1.066394in}}%
\pgfpathcurveto{\pgfqpoint{4.226126in}{1.071437in}}{\pgfqpoint{4.219285in}{1.074271in}}{\pgfqpoint{4.212152in}{1.074271in}}%
\pgfpathcurveto{\pgfqpoint{4.205019in}{1.074271in}}{\pgfqpoint{4.198177in}{1.071437in}}{\pgfqpoint{4.193134in}{1.066394in}}%
\pgfpathcurveto{\pgfqpoint{4.188090in}{1.061350in}}{\pgfqpoint{4.185256in}{1.054508in}}{\pgfqpoint{4.185256in}{1.047376in}}%
\pgfpathcurveto{\pgfqpoint{4.185256in}{1.040243in}}{\pgfqpoint{4.188090in}{1.033401in}}{\pgfqpoint{4.193134in}{1.028357in}}%
\pgfpathcurveto{\pgfqpoint{4.198177in}{1.023314in}}{\pgfqpoint{4.205019in}{1.020480in}}{\pgfqpoint{4.212152in}{1.020480in}}%
\pgfpathclose%
\pgfusepath{stroke,fill}%
\end{pgfscope}%
\begin{pgfscope}%
\pgfpathrectangle{\pgfqpoint{2.867647in}{0.500000in}}{\pgfqpoint{1.764706in}{1.700000in}}%
\pgfusepath{clip}%
\pgfsetbuttcap%
\pgfsetroundjoin%
\definecolor{currentfill}{rgb}{0.970718,0.821518,0.719872}%
\pgfsetfillcolor{currentfill}%
\pgfsetlinewidth{0.311001pt}%
\definecolor{currentstroke}{rgb}{1.000000,1.000000,1.000000}%
\pgfsetstrokecolor{currentstroke}%
\pgfsetdash{}{0pt}%
\pgfpathmoveto{\pgfqpoint{4.152688in}{1.660313in}}%
\pgfpathcurveto{\pgfqpoint{4.159821in}{1.660313in}}{\pgfqpoint{4.166663in}{1.663147in}}{\pgfqpoint{4.171706in}{1.668191in}}%
\pgfpathcurveto{\pgfqpoint{4.176750in}{1.673234in}}{\pgfqpoint{4.179584in}{1.680076in}}{\pgfqpoint{4.179584in}{1.687209in}}%
\pgfpathcurveto{\pgfqpoint{4.179584in}{1.694342in}}{\pgfqpoint{4.176750in}{1.701183in}}{\pgfqpoint{4.171706in}{1.706227in}}%
\pgfpathcurveto{\pgfqpoint{4.166663in}{1.711271in}}{\pgfqpoint{4.159821in}{1.714105in}}{\pgfqpoint{4.152688in}{1.714105in}}%
\pgfpathcurveto{\pgfqpoint{4.145555in}{1.714105in}}{\pgfqpoint{4.138714in}{1.711271in}}{\pgfqpoint{4.133670in}{1.706227in}}%
\pgfpathcurveto{\pgfqpoint{4.128626in}{1.701183in}}{\pgfqpoint{4.125792in}{1.694342in}}{\pgfqpoint{4.125792in}{1.687209in}}%
\pgfpathcurveto{\pgfqpoint{4.125792in}{1.680076in}}{\pgfqpoint{4.128626in}{1.673234in}}{\pgfqpoint{4.133670in}{1.668191in}}%
\pgfpathcurveto{\pgfqpoint{4.138714in}{1.663147in}}{\pgfqpoint{4.145555in}{1.660313in}}{\pgfqpoint{4.152688in}{1.660313in}}%
\pgfpathclose%
\pgfusepath{stroke,fill}%
\end{pgfscope}%
\begin{pgfscope}%
\pgfpathrectangle{\pgfqpoint{2.867647in}{0.500000in}}{\pgfqpoint{1.764706in}{1.700000in}}%
\pgfusepath{clip}%
\pgfsetbuttcap%
\pgfsetroundjoin%
\definecolor{currentfill}{rgb}{0.967735,0.780441,0.659127}%
\pgfsetfillcolor{currentfill}%
\pgfsetlinewidth{0.311001pt}%
\definecolor{currentstroke}{rgb}{1.000000,1.000000,1.000000}%
\pgfsetstrokecolor{currentstroke}%
\pgfsetdash{}{0pt}%
\pgfpathmoveto{\pgfqpoint{4.091695in}{1.382268in}}%
\pgfpathcurveto{\pgfqpoint{4.098828in}{1.382268in}}{\pgfqpoint{4.105670in}{1.385102in}}{\pgfqpoint{4.110713in}{1.390146in}}%
\pgfpathcurveto{\pgfqpoint{4.115757in}{1.395189in}}{\pgfqpoint{4.118591in}{1.402031in}}{\pgfqpoint{4.118591in}{1.409164in}}%
\pgfpathcurveto{\pgfqpoint{4.118591in}{1.416297in}}{\pgfqpoint{4.115757in}{1.423138in}}{\pgfqpoint{4.110713in}{1.428182in}}%
\pgfpathcurveto{\pgfqpoint{4.105670in}{1.433226in}}{\pgfqpoint{4.098828in}{1.436060in}}{\pgfqpoint{4.091695in}{1.436060in}}%
\pgfpathcurveto{\pgfqpoint{4.084563in}{1.436060in}}{\pgfqpoint{4.077721in}{1.433226in}}{\pgfqpoint{4.072677in}{1.428182in}}%
\pgfpathcurveto{\pgfqpoint{4.067634in}{1.423138in}}{\pgfqpoint{4.064800in}{1.416297in}}{\pgfqpoint{4.064800in}{1.409164in}}%
\pgfpathcurveto{\pgfqpoint{4.064800in}{1.402031in}}{\pgfqpoint{4.067634in}{1.395189in}}{\pgfqpoint{4.072677in}{1.390146in}}%
\pgfpathcurveto{\pgfqpoint{4.077721in}{1.385102in}}{\pgfqpoint{4.084563in}{1.382268in}}{\pgfqpoint{4.091695in}{1.382268in}}%
\pgfpathclose%
\pgfusepath{stroke,fill}%
\end{pgfscope}%
\begin{pgfscope}%
\pgfpathrectangle{\pgfqpoint{2.867647in}{0.500000in}}{\pgfqpoint{1.764706in}{1.700000in}}%
\pgfusepath{clip}%
\pgfsetbuttcap%
\pgfsetroundjoin%
\definecolor{currentfill}{rgb}{0.962532,0.599594,0.438051}%
\pgfsetfillcolor{currentfill}%
\pgfsetlinewidth{0.311001pt}%
\definecolor{currentstroke}{rgb}{1.000000,1.000000,1.000000}%
\pgfsetstrokecolor{currentstroke}%
\pgfsetdash{}{0pt}%
\pgfpathmoveto{\pgfqpoint{4.028379in}{0.849951in}}%
\pgfpathcurveto{\pgfqpoint{4.035511in}{0.849951in}}{\pgfqpoint{4.042353in}{0.852785in}}{\pgfqpoint{4.047397in}{0.857829in}}%
\pgfpathcurveto{\pgfqpoint{4.052440in}{0.862873in}}{\pgfqpoint{4.055274in}{0.869714in}}{\pgfqpoint{4.055274in}{0.876847in}}%
\pgfpathcurveto{\pgfqpoint{4.055274in}{0.883980in}}{\pgfqpoint{4.052440in}{0.890822in}}{\pgfqpoint{4.047397in}{0.895865in}}%
\pgfpathcurveto{\pgfqpoint{4.042353in}{0.900909in}}{\pgfqpoint{4.035511in}{0.903743in}}{\pgfqpoint{4.028379in}{0.903743in}}%
\pgfpathcurveto{\pgfqpoint{4.021246in}{0.903743in}}{\pgfqpoint{4.014404in}{0.900909in}}{\pgfqpoint{4.009360in}{0.895865in}}%
\pgfpathcurveto{\pgfqpoint{4.004317in}{0.890822in}}{\pgfqpoint{4.001483in}{0.883980in}}{\pgfqpoint{4.001483in}{0.876847in}}%
\pgfpathcurveto{\pgfqpoint{4.001483in}{0.869714in}}{\pgfqpoint{4.004317in}{0.862873in}}{\pgfqpoint{4.009360in}{0.857829in}}%
\pgfpathcurveto{\pgfqpoint{4.014404in}{0.852785in}}{\pgfqpoint{4.021246in}{0.849951in}}{\pgfqpoint{4.028379in}{0.849951in}}%
\pgfpathclose%
\pgfusepath{stroke,fill}%
\end{pgfscope}%
\begin{pgfscope}%
\pgfpathrectangle{\pgfqpoint{2.867647in}{0.500000in}}{\pgfqpoint{1.764706in}{1.700000in}}%
\pgfusepath{clip}%
\pgfsetbuttcap%
\pgfsetroundjoin%
\definecolor{currentfill}{rgb}{0.979124,0.903132,0.839793}%
\pgfsetfillcolor{currentfill}%
\pgfsetlinewidth{0.311001pt}%
\definecolor{currentstroke}{rgb}{1.000000,1.000000,1.000000}%
\pgfsetstrokecolor{currentstroke}%
\pgfsetdash{}{0pt}%
\pgfpathmoveto{\pgfqpoint{4.156541in}{1.380666in}}%
\pgfpathcurveto{\pgfqpoint{4.163674in}{1.380666in}}{\pgfqpoint{4.170516in}{1.383500in}}{\pgfqpoint{4.175559in}{1.388544in}}%
\pgfpathcurveto{\pgfqpoint{4.180603in}{1.393587in}}{\pgfqpoint{4.183437in}{1.400429in}}{\pgfqpoint{4.183437in}{1.407562in}}%
\pgfpathcurveto{\pgfqpoint{4.183437in}{1.414695in}}{\pgfqpoint{4.180603in}{1.421536in}}{\pgfqpoint{4.175559in}{1.426580in}}%
\pgfpathcurveto{\pgfqpoint{4.170516in}{1.431623in}}{\pgfqpoint{4.163674in}{1.434457in}}{\pgfqpoint{4.156541in}{1.434457in}}%
\pgfpathcurveto{\pgfqpoint{4.149408in}{1.434457in}}{\pgfqpoint{4.142567in}{1.431623in}}{\pgfqpoint{4.137523in}{1.426580in}}%
\pgfpathcurveto{\pgfqpoint{4.132479in}{1.421536in}}{\pgfqpoint{4.129646in}{1.414695in}}{\pgfqpoint{4.129646in}{1.407562in}}%
\pgfpathcurveto{\pgfqpoint{4.129646in}{1.400429in}}{\pgfqpoint{4.132479in}{1.393587in}}{\pgfqpoint{4.137523in}{1.388544in}}%
\pgfpathcurveto{\pgfqpoint{4.142567in}{1.383500in}}{\pgfqpoint{4.149408in}{1.380666in}}{\pgfqpoint{4.156541in}{1.380666in}}%
\pgfpathclose%
\pgfusepath{stroke,fill}%
\end{pgfscope}%
\begin{pgfscope}%
\pgfpathrectangle{\pgfqpoint{2.867647in}{0.500000in}}{\pgfqpoint{1.764706in}{1.700000in}}%
\pgfusepath{clip}%
\pgfsetbuttcap%
\pgfsetroundjoin%
\definecolor{currentfill}{rgb}{0.978376,0.897317,0.831308}%
\pgfsetfillcolor{currentfill}%
\pgfsetlinewidth{0.311001pt}%
\definecolor{currentstroke}{rgb}{1.000000,1.000000,1.000000}%
\pgfsetstrokecolor{currentstroke}%
\pgfsetdash{}{0pt}%
\pgfpathmoveto{\pgfqpoint{4.232167in}{1.326399in}}%
\pgfpathcurveto{\pgfqpoint{4.239300in}{1.326399in}}{\pgfqpoint{4.246141in}{1.329233in}}{\pgfqpoint{4.251185in}{1.334277in}}%
\pgfpathcurveto{\pgfqpoint{4.256228in}{1.339321in}}{\pgfqpoint{4.259062in}{1.346162in}}{\pgfqpoint{4.259062in}{1.353295in}}%
\pgfpathcurveto{\pgfqpoint{4.259062in}{1.360428in}}{\pgfqpoint{4.256228in}{1.367270in}}{\pgfqpoint{4.251185in}{1.372313in}}%
\pgfpathcurveto{\pgfqpoint{4.246141in}{1.377357in}}{\pgfqpoint{4.239300in}{1.380191in}}{\pgfqpoint{4.232167in}{1.380191in}}%
\pgfpathcurveto{\pgfqpoint{4.225034in}{1.380191in}}{\pgfqpoint{4.218192in}{1.377357in}}{\pgfqpoint{4.213149in}{1.372313in}}%
\pgfpathcurveto{\pgfqpoint{4.208105in}{1.367270in}}{\pgfqpoint{4.205271in}{1.360428in}}{\pgfqpoint{4.205271in}{1.353295in}}%
\pgfpathcurveto{\pgfqpoint{4.205271in}{1.346162in}}{\pgfqpoint{4.208105in}{1.339321in}}{\pgfqpoint{4.213149in}{1.334277in}}%
\pgfpathcurveto{\pgfqpoint{4.218192in}{1.329233in}}{\pgfqpoint{4.225034in}{1.326399in}}{\pgfqpoint{4.232167in}{1.326399in}}%
\pgfpathclose%
\pgfusepath{stroke,fill}%
\end{pgfscope}%
\begin{pgfscope}%
\pgfpathrectangle{\pgfqpoint{2.867647in}{0.500000in}}{\pgfqpoint{1.764706in}{1.700000in}}%
\pgfusepath{clip}%
\pgfsetbuttcap%
\pgfsetroundjoin%
\definecolor{currentfill}{rgb}{0.977657,0.891500,0.822809}%
\pgfsetfillcolor{currentfill}%
\pgfsetlinewidth{0.311001pt}%
\definecolor{currentstroke}{rgb}{1.000000,1.000000,1.000000}%
\pgfsetstrokecolor{currentstroke}%
\pgfsetdash{}{0pt}%
\pgfpathmoveto{\pgfqpoint{4.213210in}{1.464894in}}%
\pgfpathcurveto{\pgfqpoint{4.220343in}{1.464894in}}{\pgfqpoint{4.227185in}{1.467728in}}{\pgfqpoint{4.232228in}{1.472772in}}%
\pgfpathcurveto{\pgfqpoint{4.237272in}{1.477816in}}{\pgfqpoint{4.240106in}{1.484657in}}{\pgfqpoint{4.240106in}{1.491790in}}%
\pgfpathcurveto{\pgfqpoint{4.240106in}{1.498923in}}{\pgfqpoint{4.237272in}{1.505765in}}{\pgfqpoint{4.232228in}{1.510808in}}%
\pgfpathcurveto{\pgfqpoint{4.227185in}{1.515852in}}{\pgfqpoint{4.220343in}{1.518686in}}{\pgfqpoint{4.213210in}{1.518686in}}%
\pgfpathcurveto{\pgfqpoint{4.206077in}{1.518686in}}{\pgfqpoint{4.199236in}{1.515852in}}{\pgfqpoint{4.194192in}{1.510808in}}%
\pgfpathcurveto{\pgfqpoint{4.189148in}{1.505765in}}{\pgfqpoint{4.186315in}{1.498923in}}{\pgfqpoint{4.186315in}{1.491790in}}%
\pgfpathcurveto{\pgfqpoint{4.186315in}{1.484657in}}{\pgfqpoint{4.189148in}{1.477816in}}{\pgfqpoint{4.194192in}{1.472772in}}%
\pgfpathcurveto{\pgfqpoint{4.199236in}{1.467728in}}{\pgfqpoint{4.206077in}{1.464894in}}{\pgfqpoint{4.213210in}{1.464894in}}%
\pgfpathclose%
\pgfusepath{stroke,fill}%
\end{pgfscope}%
\begin{pgfscope}%
\pgfpathrectangle{\pgfqpoint{2.867647in}{0.500000in}}{\pgfqpoint{1.764706in}{1.700000in}}%
\pgfusepath{clip}%
\pgfsetbuttcap%
\pgfsetroundjoin%
\definecolor{currentfill}{rgb}{0.965753,0.732351,0.592427}%
\pgfsetfillcolor{currentfill}%
\pgfsetlinewidth{0.311001pt}%
\definecolor{currentstroke}{rgb}{1.000000,1.000000,1.000000}%
\pgfsetstrokecolor{currentstroke}%
\pgfsetdash{}{0pt}%
\pgfpathmoveto{\pgfqpoint{4.091370in}{1.747215in}}%
\pgfpathcurveto{\pgfqpoint{4.098503in}{1.747215in}}{\pgfqpoint{4.105345in}{1.750049in}}{\pgfqpoint{4.110388in}{1.755092in}}%
\pgfpathcurveto{\pgfqpoint{4.115432in}{1.760136in}}{\pgfqpoint{4.118266in}{1.766978in}}{\pgfqpoint{4.118266in}{1.774111in}}%
\pgfpathcurveto{\pgfqpoint{4.118266in}{1.781243in}}{\pgfqpoint{4.115432in}{1.788085in}}{\pgfqpoint{4.110388in}{1.793129in}}%
\pgfpathcurveto{\pgfqpoint{4.105345in}{1.798172in}}{\pgfqpoint{4.098503in}{1.801006in}}{\pgfqpoint{4.091370in}{1.801006in}}%
\pgfpathcurveto{\pgfqpoint{4.084237in}{1.801006in}}{\pgfqpoint{4.077396in}{1.798172in}}{\pgfqpoint{4.072352in}{1.793129in}}%
\pgfpathcurveto{\pgfqpoint{4.067308in}{1.788085in}}{\pgfqpoint{4.064475in}{1.781243in}}{\pgfqpoint{4.064475in}{1.774111in}}%
\pgfpathcurveto{\pgfqpoint{4.064475in}{1.766978in}}{\pgfqpoint{4.067308in}{1.760136in}}{\pgfqpoint{4.072352in}{1.755092in}}%
\pgfpathcurveto{\pgfqpoint{4.077396in}{1.750049in}}{\pgfqpoint{4.084237in}{1.747215in}}{\pgfqpoint{4.091370in}{1.747215in}}%
\pgfpathclose%
\pgfusepath{stroke,fill}%
\end{pgfscope}%
\begin{pgfscope}%
\pgfpathrectangle{\pgfqpoint{2.867647in}{0.500000in}}{\pgfqpoint{1.764706in}{1.700000in}}%
\pgfusepath{clip}%
\pgfsetbuttcap%
\pgfsetroundjoin%
\definecolor{currentfill}{rgb}{0.973271,0.850724,0.762998}%
\pgfsetfillcolor{currentfill}%
\pgfsetlinewidth{0.311001pt}%
\definecolor{currentstroke}{rgb}{1.000000,1.000000,1.000000}%
\pgfsetstrokecolor{currentstroke}%
\pgfsetdash{}{0pt}%
\pgfpathmoveto{\pgfqpoint{4.107294in}{1.195631in}}%
\pgfpathcurveto{\pgfqpoint{4.114427in}{1.195631in}}{\pgfqpoint{4.121269in}{1.198465in}}{\pgfqpoint{4.126313in}{1.203509in}}%
\pgfpathcurveto{\pgfqpoint{4.131356in}{1.208553in}}{\pgfqpoint{4.134190in}{1.215394in}}{\pgfqpoint{4.134190in}{1.222527in}}%
\pgfpathcurveto{\pgfqpoint{4.134190in}{1.229660in}}{\pgfqpoint{4.131356in}{1.236501in}}{\pgfqpoint{4.126313in}{1.241545in}}%
\pgfpathcurveto{\pgfqpoint{4.121269in}{1.246589in}}{\pgfqpoint{4.114427in}{1.249423in}}{\pgfqpoint{4.107294in}{1.249423in}}%
\pgfpathcurveto{\pgfqpoint{4.100162in}{1.249423in}}{\pgfqpoint{4.093320in}{1.246589in}}{\pgfqpoint{4.088276in}{1.241545in}}%
\pgfpathcurveto{\pgfqpoint{4.083233in}{1.236501in}}{\pgfqpoint{4.080399in}{1.229660in}}{\pgfqpoint{4.080399in}{1.222527in}}%
\pgfpathcurveto{\pgfqpoint{4.080399in}{1.215394in}}{\pgfqpoint{4.083233in}{1.208553in}}{\pgfqpoint{4.088276in}{1.203509in}}%
\pgfpathcurveto{\pgfqpoint{4.093320in}{1.198465in}}{\pgfqpoint{4.100162in}{1.195631in}}{\pgfqpoint{4.107294in}{1.195631in}}%
\pgfpathclose%
\pgfusepath{stroke,fill}%
\end{pgfscope}%
\begin{pgfscope}%
\pgfpathrectangle{\pgfqpoint{2.867647in}{0.500000in}}{\pgfqpoint{1.764706in}{1.700000in}}%
\pgfusepath{clip}%
\pgfsetbuttcap%
\pgfsetroundjoin%
\definecolor{currentfill}{rgb}{0.957848,0.512613,0.357119}%
\pgfsetfillcolor{currentfill}%
\pgfsetlinewidth{0.311001pt}%
\definecolor{currentstroke}{rgb}{1.000000,1.000000,1.000000}%
\pgfsetstrokecolor{currentstroke}%
\pgfsetdash{}{0pt}%
\pgfpathmoveto{\pgfqpoint{3.928293in}{0.880092in}}%
\pgfpathcurveto{\pgfqpoint{3.935426in}{0.880092in}}{\pgfqpoint{3.942267in}{0.882926in}}{\pgfqpoint{3.947311in}{0.887969in}}%
\pgfpathcurveto{\pgfqpoint{3.952355in}{0.893013in}}{\pgfqpoint{3.955188in}{0.899855in}}{\pgfqpoint{3.955188in}{0.906987in}}%
\pgfpathcurveto{\pgfqpoint{3.955188in}{0.914120in}}{\pgfqpoint{3.952355in}{0.920962in}}{\pgfqpoint{3.947311in}{0.926006in}}%
\pgfpathcurveto{\pgfqpoint{3.942267in}{0.931049in}}{\pgfqpoint{3.935426in}{0.933883in}}{\pgfqpoint{3.928293in}{0.933883in}}%
\pgfpathcurveto{\pgfqpoint{3.921160in}{0.933883in}}{\pgfqpoint{3.914318in}{0.931049in}}{\pgfqpoint{3.909275in}{0.926006in}}%
\pgfpathcurveto{\pgfqpoint{3.904231in}{0.920962in}}{\pgfqpoint{3.901397in}{0.914120in}}{\pgfqpoint{3.901397in}{0.906987in}}%
\pgfpathcurveto{\pgfqpoint{3.901397in}{0.899855in}}{\pgfqpoint{3.904231in}{0.893013in}}{\pgfqpoint{3.909275in}{0.887969in}}%
\pgfpathcurveto{\pgfqpoint{3.914318in}{0.882926in}}{\pgfqpoint{3.921160in}{0.880092in}}{\pgfqpoint{3.928293in}{0.880092in}}%
\pgfpathclose%
\pgfusepath{stroke,fill}%
\end{pgfscope}%
\begin{pgfscope}%
\pgfpathrectangle{\pgfqpoint{2.867647in}{0.500000in}}{\pgfqpoint{1.764706in}{1.700000in}}%
\pgfusepath{clip}%
\pgfsetbuttcap%
\pgfsetroundjoin%
\definecolor{currentfill}{rgb}{0.972201,0.839051,0.745789}%
\pgfsetfillcolor{currentfill}%
\pgfsetlinewidth{0.311001pt}%
\definecolor{currentstroke}{rgb}{1.000000,1.000000,1.000000}%
\pgfsetstrokecolor{currentstroke}%
\pgfsetdash{}{0pt}%
\pgfpathmoveto{\pgfqpoint{4.261661in}{1.368470in}}%
\pgfpathcurveto{\pgfqpoint{4.268794in}{1.368470in}}{\pgfqpoint{4.275636in}{1.371304in}}{\pgfqpoint{4.280679in}{1.376348in}}%
\pgfpathcurveto{\pgfqpoint{4.285723in}{1.381391in}}{\pgfqpoint{4.288557in}{1.388233in}}{\pgfqpoint{4.288557in}{1.395366in}}%
\pgfpathcurveto{\pgfqpoint{4.288557in}{1.402499in}}{\pgfqpoint{4.285723in}{1.409340in}}{\pgfqpoint{4.280679in}{1.414384in}}%
\pgfpathcurveto{\pgfqpoint{4.275636in}{1.419428in}}{\pgfqpoint{4.268794in}{1.422262in}}{\pgfqpoint{4.261661in}{1.422262in}}%
\pgfpathcurveto{\pgfqpoint{4.254528in}{1.422262in}}{\pgfqpoint{4.247687in}{1.419428in}}{\pgfqpoint{4.242643in}{1.414384in}}%
\pgfpathcurveto{\pgfqpoint{4.237599in}{1.409340in}}{\pgfqpoint{4.234766in}{1.402499in}}{\pgfqpoint{4.234766in}{1.395366in}}%
\pgfpathcurveto{\pgfqpoint{4.234766in}{1.388233in}}{\pgfqpoint{4.237599in}{1.381391in}}{\pgfqpoint{4.242643in}{1.376348in}}%
\pgfpathcurveto{\pgfqpoint{4.247687in}{1.371304in}}{\pgfqpoint{4.254528in}{1.368470in}}{\pgfqpoint{4.261661in}{1.368470in}}%
\pgfpathclose%
\pgfusepath{stroke,fill}%
\end{pgfscope}%
\begin{pgfscope}%
\pgfpathrectangle{\pgfqpoint{2.867647in}{0.500000in}}{\pgfqpoint{1.764706in}{1.700000in}}%
\pgfusepath{clip}%
\pgfsetbuttcap%
\pgfsetroundjoin%
\definecolor{currentfill}{rgb}{0.963884,0.644842,0.486120}%
\pgfsetfillcolor{currentfill}%
\pgfsetlinewidth{0.311001pt}%
\definecolor{currentstroke}{rgb}{1.000000,1.000000,1.000000}%
\pgfsetstrokecolor{currentstroke}%
\pgfsetdash{}{0pt}%
\pgfpathmoveto{\pgfqpoint{4.173121in}{1.722381in}}%
\pgfpathcurveto{\pgfqpoint{4.180254in}{1.722381in}}{\pgfqpoint{4.187095in}{1.725215in}}{\pgfqpoint{4.192139in}{1.730259in}}%
\pgfpathcurveto{\pgfqpoint{4.197183in}{1.735303in}}{\pgfqpoint{4.200017in}{1.742144in}}{\pgfqpoint{4.200017in}{1.749277in}}%
\pgfpathcurveto{\pgfqpoint{4.200017in}{1.756410in}}{\pgfqpoint{4.197183in}{1.763252in}}{\pgfqpoint{4.192139in}{1.768295in}}%
\pgfpathcurveto{\pgfqpoint{4.187095in}{1.773339in}}{\pgfqpoint{4.180254in}{1.776173in}}{\pgfqpoint{4.173121in}{1.776173in}}%
\pgfpathcurveto{\pgfqpoint{4.165988in}{1.776173in}}{\pgfqpoint{4.159147in}{1.773339in}}{\pgfqpoint{4.154103in}{1.768295in}}%
\pgfpathcurveto{\pgfqpoint{4.149059in}{1.763252in}}{\pgfqpoint{4.146225in}{1.756410in}}{\pgfqpoint{4.146225in}{1.749277in}}%
\pgfpathcurveto{\pgfqpoint{4.146225in}{1.742144in}}{\pgfqpoint{4.149059in}{1.735303in}}{\pgfqpoint{4.154103in}{1.730259in}}%
\pgfpathcurveto{\pgfqpoint{4.159147in}{1.725215in}}{\pgfqpoint{4.165988in}{1.722381in}}{\pgfqpoint{4.173121in}{1.722381in}}%
\pgfpathclose%
\pgfusepath{stroke,fill}%
\end{pgfscope}%
\begin{pgfscope}%
\pgfpathrectangle{\pgfqpoint{2.867647in}{0.500000in}}{\pgfqpoint{1.764706in}{1.700000in}}%
\pgfusepath{clip}%
\pgfsetbuttcap%
\pgfsetroundjoin%
\definecolor{currentfill}{rgb}{0.973271,0.850724,0.762998}%
\pgfsetfillcolor{currentfill}%
\pgfsetlinewidth{0.311001pt}%
\definecolor{currentstroke}{rgb}{1.000000,1.000000,1.000000}%
\pgfsetstrokecolor{currentstroke}%
\pgfsetdash{}{0pt}%
\pgfpathmoveto{\pgfqpoint{4.247291in}{1.188303in}}%
\pgfpathcurveto{\pgfqpoint{4.254424in}{1.188303in}}{\pgfqpoint{4.261265in}{1.191137in}}{\pgfqpoint{4.266309in}{1.196180in}}%
\pgfpathcurveto{\pgfqpoint{4.271353in}{1.201224in}}{\pgfqpoint{4.274187in}{1.208066in}}{\pgfqpoint{4.274187in}{1.215198in}}%
\pgfpathcurveto{\pgfqpoint{4.274187in}{1.222331in}}{\pgfqpoint{4.271353in}{1.229173in}}{\pgfqpoint{4.266309in}{1.234216in}}%
\pgfpathcurveto{\pgfqpoint{4.261265in}{1.239260in}}{\pgfqpoint{4.254424in}{1.242094in}}{\pgfqpoint{4.247291in}{1.242094in}}%
\pgfpathcurveto{\pgfqpoint{4.240158in}{1.242094in}}{\pgfqpoint{4.233316in}{1.239260in}}{\pgfqpoint{4.228273in}{1.234216in}}%
\pgfpathcurveto{\pgfqpoint{4.223229in}{1.229173in}}{\pgfqpoint{4.220395in}{1.222331in}}{\pgfqpoint{4.220395in}{1.215198in}}%
\pgfpathcurveto{\pgfqpoint{4.220395in}{1.208066in}}{\pgfqpoint{4.223229in}{1.201224in}}{\pgfqpoint{4.228273in}{1.196180in}}%
\pgfpathcurveto{\pgfqpoint{4.233316in}{1.191137in}}{\pgfqpoint{4.240158in}{1.188303in}}{\pgfqpoint{4.247291in}{1.188303in}}%
\pgfpathclose%
\pgfusepath{stroke,fill}%
\end{pgfscope}%
\begin{pgfscope}%
\pgfpathrectangle{\pgfqpoint{2.867647in}{0.500000in}}{\pgfqpoint{1.764706in}{1.700000in}}%
\pgfusepath{clip}%
\pgfsetbuttcap%
\pgfsetroundjoin%
\definecolor{currentfill}{rgb}{0.980678,0.914765,0.856766}%
\pgfsetfillcolor{currentfill}%
\pgfsetlinewidth{0.311001pt}%
\definecolor{currentstroke}{rgb}{1.000000,1.000000,1.000000}%
\pgfsetstrokecolor{currentstroke}%
\pgfsetdash{}{0pt}%
\pgfpathmoveto{\pgfqpoint{4.149562in}{1.502985in}}%
\pgfpathcurveto{\pgfqpoint{4.156695in}{1.502985in}}{\pgfqpoint{4.163537in}{1.505819in}}{\pgfqpoint{4.168581in}{1.510863in}}%
\pgfpathcurveto{\pgfqpoint{4.173624in}{1.515906in}}{\pgfqpoint{4.176458in}{1.522748in}}{\pgfqpoint{4.176458in}{1.529881in}}%
\pgfpathcurveto{\pgfqpoint{4.176458in}{1.537014in}}{\pgfqpoint{4.173624in}{1.543855in}}{\pgfqpoint{4.168581in}{1.548899in}}%
\pgfpathcurveto{\pgfqpoint{4.163537in}{1.553943in}}{\pgfqpoint{4.156695in}{1.556776in}}{\pgfqpoint{4.149562in}{1.556776in}}%
\pgfpathcurveto{\pgfqpoint{4.142430in}{1.556776in}}{\pgfqpoint{4.135588in}{1.553943in}}{\pgfqpoint{4.130544in}{1.548899in}}%
\pgfpathcurveto{\pgfqpoint{4.125501in}{1.543855in}}{\pgfqpoint{4.122667in}{1.537014in}}{\pgfqpoint{4.122667in}{1.529881in}}%
\pgfpathcurveto{\pgfqpoint{4.122667in}{1.522748in}}{\pgfqpoint{4.125501in}{1.515906in}}{\pgfqpoint{4.130544in}{1.510863in}}%
\pgfpathcurveto{\pgfqpoint{4.135588in}{1.505819in}}{\pgfqpoint{4.142430in}{1.502985in}}{\pgfqpoint{4.149562in}{1.502985in}}%
\pgfpathclose%
\pgfusepath{stroke,fill}%
\end{pgfscope}%
\begin{pgfscope}%
\pgfpathrectangle{\pgfqpoint{2.867647in}{0.500000in}}{\pgfqpoint{1.764706in}{1.700000in}}%
\pgfusepath{clip}%
\pgfsetbuttcap%
\pgfsetroundjoin%
\definecolor{currentfill}{rgb}{0.965169,0.707764,0.560659}%
\pgfsetfillcolor{currentfill}%
\pgfsetlinewidth{0.311001pt}%
\definecolor{currentstroke}{rgb}{1.000000,1.000000,1.000000}%
\pgfsetstrokecolor{currentstroke}%
\pgfsetdash{}{0pt}%
\pgfpathmoveto{\pgfqpoint{4.046238in}{1.158721in}}%
\pgfpathcurveto{\pgfqpoint{4.053371in}{1.158721in}}{\pgfqpoint{4.060213in}{1.161555in}}{\pgfqpoint{4.065256in}{1.166599in}}%
\pgfpathcurveto{\pgfqpoint{4.070300in}{1.171642in}}{\pgfqpoint{4.073134in}{1.178484in}}{\pgfqpoint{4.073134in}{1.185617in}}%
\pgfpathcurveto{\pgfqpoint{4.073134in}{1.192750in}}{\pgfqpoint{4.070300in}{1.199591in}}{\pgfqpoint{4.065256in}{1.204635in}}%
\pgfpathcurveto{\pgfqpoint{4.060213in}{1.209679in}}{\pgfqpoint{4.053371in}{1.212513in}}{\pgfqpoint{4.046238in}{1.212513in}}%
\pgfpathcurveto{\pgfqpoint{4.039105in}{1.212513in}}{\pgfqpoint{4.032264in}{1.209679in}}{\pgfqpoint{4.027220in}{1.204635in}}%
\pgfpathcurveto{\pgfqpoint{4.022176in}{1.199591in}}{\pgfqpoint{4.019342in}{1.192750in}}{\pgfqpoint{4.019342in}{1.185617in}}%
\pgfpathcurveto{\pgfqpoint{4.019342in}{1.178484in}}{\pgfqpoint{4.022176in}{1.171642in}}{\pgfqpoint{4.027220in}{1.166599in}}%
\pgfpathcurveto{\pgfqpoint{4.032264in}{1.161555in}}{\pgfqpoint{4.039105in}{1.158721in}}{\pgfqpoint{4.046238in}{1.158721in}}%
\pgfpathclose%
\pgfusepath{stroke,fill}%
\end{pgfscope}%
\begin{pgfscope}%
\pgfpathrectangle{\pgfqpoint{2.867647in}{0.500000in}}{\pgfqpoint{1.764706in}{1.700000in}}%
\pgfusepath{clip}%
\pgfsetbuttcap%
\pgfsetroundjoin%
\definecolor{currentfill}{rgb}{0.971202,0.827364,0.728520}%
\pgfsetfillcolor{currentfill}%
\pgfsetlinewidth{0.311001pt}%
\definecolor{currentstroke}{rgb}{1.000000,1.000000,1.000000}%
\pgfsetstrokecolor{currentstroke}%
\pgfsetdash{}{0pt}%
\pgfpathmoveto{\pgfqpoint{4.266482in}{1.380059in}}%
\pgfpathcurveto{\pgfqpoint{4.273615in}{1.380059in}}{\pgfqpoint{4.280457in}{1.382893in}}{\pgfqpoint{4.285500in}{1.387936in}}%
\pgfpathcurveto{\pgfqpoint{4.290544in}{1.392980in}}{\pgfqpoint{4.293378in}{1.399822in}}{\pgfqpoint{4.293378in}{1.406954in}}%
\pgfpathcurveto{\pgfqpoint{4.293378in}{1.414087in}}{\pgfqpoint{4.290544in}{1.420929in}}{\pgfqpoint{4.285500in}{1.425972in}}%
\pgfpathcurveto{\pgfqpoint{4.280457in}{1.431016in}}{\pgfqpoint{4.273615in}{1.433850in}}{\pgfqpoint{4.266482in}{1.433850in}}%
\pgfpathcurveto{\pgfqpoint{4.259350in}{1.433850in}}{\pgfqpoint{4.252508in}{1.431016in}}{\pgfqpoint{4.247464in}{1.425972in}}%
\pgfpathcurveto{\pgfqpoint{4.242421in}{1.420929in}}{\pgfqpoint{4.239587in}{1.414087in}}{\pgfqpoint{4.239587in}{1.406954in}}%
\pgfpathcurveto{\pgfqpoint{4.239587in}{1.399822in}}{\pgfqpoint{4.242421in}{1.392980in}}{\pgfqpoint{4.247464in}{1.387936in}}%
\pgfpathcurveto{\pgfqpoint{4.252508in}{1.382893in}}{\pgfqpoint{4.259350in}{1.380059in}}{\pgfqpoint{4.266482in}{1.380059in}}%
\pgfpathclose%
\pgfusepath{stroke,fill}%
\end{pgfscope}%
\begin{pgfscope}%
\pgfpathrectangle{\pgfqpoint{2.867647in}{0.500000in}}{\pgfqpoint{1.764706in}{1.700000in}}%
\pgfusepath{clip}%
\pgfsetbuttcap%
\pgfsetroundjoin%
\definecolor{currentfill}{rgb}{0.964173,0.657587,0.500469}%
\pgfsetfillcolor{currentfill}%
\pgfsetlinewidth{0.311001pt}%
\definecolor{currentstroke}{rgb}{1.000000,1.000000,1.000000}%
\pgfsetstrokecolor{currentstroke}%
\pgfsetdash{}{0pt}%
\pgfpathmoveto{\pgfqpoint{3.969466in}{0.938538in}}%
\pgfpathcurveto{\pgfqpoint{3.976598in}{0.938538in}}{\pgfqpoint{3.983440in}{0.941371in}}{\pgfqpoint{3.988484in}{0.946415in}}%
\pgfpathcurveto{\pgfqpoint{3.993527in}{0.951459in}}{\pgfqpoint{3.996361in}{0.958300in}}{\pgfqpoint{3.996361in}{0.965433in}}%
\pgfpathcurveto{\pgfqpoint{3.996361in}{0.972566in}}{\pgfqpoint{3.993527in}{0.979408in}}{\pgfqpoint{3.988484in}{0.984451in}}%
\pgfpathcurveto{\pgfqpoint{3.983440in}{0.989495in}}{\pgfqpoint{3.976598in}{0.992329in}}{\pgfqpoint{3.969466in}{0.992329in}}%
\pgfpathcurveto{\pgfqpoint{3.962333in}{0.992329in}}{\pgfqpoint{3.955491in}{0.989495in}}{\pgfqpoint{3.950448in}{0.984451in}}%
\pgfpathcurveto{\pgfqpoint{3.945404in}{0.979408in}}{\pgfqpoint{3.942570in}{0.972566in}}{\pgfqpoint{3.942570in}{0.965433in}}%
\pgfpathcurveto{\pgfqpoint{3.942570in}{0.958300in}}{\pgfqpoint{3.945404in}{0.951459in}}{\pgfqpoint{3.950448in}{0.946415in}}%
\pgfpathcurveto{\pgfqpoint{3.955491in}{0.941371in}}{\pgfqpoint{3.962333in}{0.938538in}}{\pgfqpoint{3.969466in}{0.938538in}}%
\pgfpathclose%
\pgfusepath{stroke,fill}%
\end{pgfscope}%
\begin{pgfscope}%
\pgfpathrectangle{\pgfqpoint{2.867647in}{0.500000in}}{\pgfqpoint{1.764706in}{1.700000in}}%
\pgfusepath{clip}%
\pgfsetbuttcap%
\pgfsetroundjoin%
\definecolor{currentfill}{rgb}{0.975018,0.868213,0.788710}%
\pgfsetfillcolor{currentfill}%
\pgfsetlinewidth{0.311001pt}%
\definecolor{currentstroke}{rgb}{1.000000,1.000000,1.000000}%
\pgfsetstrokecolor{currentstroke}%
\pgfsetdash{}{0pt}%
\pgfpathmoveto{\pgfqpoint{4.139870in}{1.014696in}}%
\pgfpathcurveto{\pgfqpoint{4.147002in}{1.014696in}}{\pgfqpoint{4.153844in}{1.017530in}}{\pgfqpoint{4.158888in}{1.022574in}}%
\pgfpathcurveto{\pgfqpoint{4.163931in}{1.027617in}}{\pgfqpoint{4.166765in}{1.034459in}}{\pgfqpoint{4.166765in}{1.041592in}}%
\pgfpathcurveto{\pgfqpoint{4.166765in}{1.048725in}}{\pgfqpoint{4.163931in}{1.055566in}}{\pgfqpoint{4.158888in}{1.060610in}}%
\pgfpathcurveto{\pgfqpoint{4.153844in}{1.065654in}}{\pgfqpoint{4.147002in}{1.068488in}}{\pgfqpoint{4.139870in}{1.068488in}}%
\pgfpathcurveto{\pgfqpoint{4.132737in}{1.068488in}}{\pgfqpoint{4.125895in}{1.065654in}}{\pgfqpoint{4.120851in}{1.060610in}}%
\pgfpathcurveto{\pgfqpoint{4.115808in}{1.055566in}}{\pgfqpoint{4.112974in}{1.048725in}}{\pgfqpoint{4.112974in}{1.041592in}}%
\pgfpathcurveto{\pgfqpoint{4.112974in}{1.034459in}}{\pgfqpoint{4.115808in}{1.027617in}}{\pgfqpoint{4.120851in}{1.022574in}}%
\pgfpathcurveto{\pgfqpoint{4.125895in}{1.017530in}}{\pgfqpoint{4.132737in}{1.014696in}}{\pgfqpoint{4.139870in}{1.014696in}}%
\pgfpathclose%
\pgfusepath{stroke,fill}%
\end{pgfscope}%
\begin{pgfscope}%
\pgfpathrectangle{\pgfqpoint{2.867647in}{0.500000in}}{\pgfqpoint{1.764706in}{1.700000in}}%
\pgfusepath{clip}%
\pgfsetbuttcap%
\pgfsetroundjoin%
\definecolor{currentfill}{rgb}{0.981377,0.920617,0.865369}%
\pgfsetfillcolor{currentfill}%
\pgfsetlinewidth{0.311001pt}%
\definecolor{currentstroke}{rgb}{1.000000,1.000000,1.000000}%
\pgfsetstrokecolor{currentstroke}%
\pgfsetdash{}{0pt}%
\pgfpathmoveto{\pgfqpoint{4.196479in}{1.222755in}}%
\pgfpathcurveto{\pgfqpoint{4.203611in}{1.222755in}}{\pgfqpoint{4.210453in}{1.225589in}}{\pgfqpoint{4.215497in}{1.230632in}}%
\pgfpathcurveto{\pgfqpoint{4.220540in}{1.235676in}}{\pgfqpoint{4.223374in}{1.242518in}}{\pgfqpoint{4.223374in}{1.249651in}}%
\pgfpathcurveto{\pgfqpoint{4.223374in}{1.256783in}}{\pgfqpoint{4.220540in}{1.263625in}}{\pgfqpoint{4.215497in}{1.268669in}}%
\pgfpathcurveto{\pgfqpoint{4.210453in}{1.273712in}}{\pgfqpoint{4.203611in}{1.276546in}}{\pgfqpoint{4.196479in}{1.276546in}}%
\pgfpathcurveto{\pgfqpoint{4.189346in}{1.276546in}}{\pgfqpoint{4.182504in}{1.273712in}}{\pgfqpoint{4.177460in}{1.268669in}}%
\pgfpathcurveto{\pgfqpoint{4.172417in}{1.263625in}}{\pgfqpoint{4.169583in}{1.256783in}}{\pgfqpoint{4.169583in}{1.249651in}}%
\pgfpathcurveto{\pgfqpoint{4.169583in}{1.242518in}}{\pgfqpoint{4.172417in}{1.235676in}}{\pgfqpoint{4.177460in}{1.230632in}}%
\pgfpathcurveto{\pgfqpoint{4.182504in}{1.225589in}}{\pgfqpoint{4.189346in}{1.222755in}}{\pgfqpoint{4.196479in}{1.222755in}}%
\pgfpathclose%
\pgfusepath{stroke,fill}%
\end{pgfscope}%
\begin{pgfscope}%
\pgfpathrectangle{\pgfqpoint{2.867647in}{0.500000in}}{\pgfqpoint{1.764706in}{1.700000in}}%
\pgfusepath{clip}%
\pgfsetbuttcap%
\pgfsetroundjoin%
\definecolor{currentfill}{rgb}{0.966560,0.756582,0.625273}%
\pgfsetfillcolor{currentfill}%
\pgfsetlinewidth{0.311001pt}%
\definecolor{currentstroke}{rgb}{1.000000,1.000000,1.000000}%
\pgfsetstrokecolor{currentstroke}%
\pgfsetdash{}{0pt}%
\pgfpathmoveto{\pgfqpoint{4.203345in}{0.999219in}}%
\pgfpathcurveto{\pgfqpoint{4.210477in}{0.999219in}}{\pgfqpoint{4.217319in}{1.002053in}}{\pgfqpoint{4.222363in}{1.007097in}}%
\pgfpathcurveto{\pgfqpoint{4.227406in}{1.012140in}}{\pgfqpoint{4.230240in}{1.018982in}}{\pgfqpoint{4.230240in}{1.026115in}}%
\pgfpathcurveto{\pgfqpoint{4.230240in}{1.033248in}}{\pgfqpoint{4.227406in}{1.040089in}}{\pgfqpoint{4.222363in}{1.045133in}}%
\pgfpathcurveto{\pgfqpoint{4.217319in}{1.050177in}}{\pgfqpoint{4.210477in}{1.053010in}}{\pgfqpoint{4.203345in}{1.053010in}}%
\pgfpathcurveto{\pgfqpoint{4.196212in}{1.053010in}}{\pgfqpoint{4.189370in}{1.050177in}}{\pgfqpoint{4.184327in}{1.045133in}}%
\pgfpathcurveto{\pgfqpoint{4.179283in}{1.040089in}}{\pgfqpoint{4.176449in}{1.033248in}}{\pgfqpoint{4.176449in}{1.026115in}}%
\pgfpathcurveto{\pgfqpoint{4.176449in}{1.018982in}}{\pgfqpoint{4.179283in}{1.012140in}}{\pgfqpoint{4.184327in}{1.007097in}}%
\pgfpathcurveto{\pgfqpoint{4.189370in}{1.002053in}}{\pgfqpoint{4.196212in}{0.999219in}}{\pgfqpoint{4.203345in}{0.999219in}}%
\pgfpathclose%
\pgfusepath{stroke,fill}%
\end{pgfscope}%
\begin{pgfscope}%
\pgfpathrectangle{\pgfqpoint{2.867647in}{0.500000in}}{\pgfqpoint{1.764706in}{1.700000in}}%
\pgfusepath{clip}%
\pgfsetbuttcap%
\pgfsetroundjoin%
\definecolor{currentfill}{rgb}{0.976287,0.879862,0.805788}%
\pgfsetfillcolor{currentfill}%
\pgfsetlinewidth{0.311001pt}%
\definecolor{currentstroke}{rgb}{1.000000,1.000000,1.000000}%
\pgfsetstrokecolor{currentstroke}%
\pgfsetdash{}{0pt}%
\pgfpathmoveto{\pgfqpoint{4.122437in}{1.187050in}}%
\pgfpathcurveto{\pgfqpoint{4.129569in}{1.187050in}}{\pgfqpoint{4.136411in}{1.189884in}}{\pgfqpoint{4.141455in}{1.194927in}}%
\pgfpathcurveto{\pgfqpoint{4.146498in}{1.199971in}}{\pgfqpoint{4.149332in}{1.206813in}}{\pgfqpoint{4.149332in}{1.213946in}}%
\pgfpathcurveto{\pgfqpoint{4.149332in}{1.221078in}}{\pgfqpoint{4.146498in}{1.227920in}}{\pgfqpoint{4.141455in}{1.232964in}}%
\pgfpathcurveto{\pgfqpoint{4.136411in}{1.238007in}}{\pgfqpoint{4.129569in}{1.240841in}}{\pgfqpoint{4.122437in}{1.240841in}}%
\pgfpathcurveto{\pgfqpoint{4.115304in}{1.240841in}}{\pgfqpoint{4.108462in}{1.238007in}}{\pgfqpoint{4.103418in}{1.232964in}}%
\pgfpathcurveto{\pgfqpoint{4.098375in}{1.227920in}}{\pgfqpoint{4.095541in}{1.221078in}}{\pgfqpoint{4.095541in}{1.213946in}}%
\pgfpathcurveto{\pgfqpoint{4.095541in}{1.206813in}}{\pgfqpoint{4.098375in}{1.199971in}}{\pgfqpoint{4.103418in}{1.194927in}}%
\pgfpathcurveto{\pgfqpoint{4.108462in}{1.189884in}}{\pgfqpoint{4.115304in}{1.187050in}}{\pgfqpoint{4.122437in}{1.187050in}}%
\pgfpathclose%
\pgfusepath{stroke,fill}%
\end{pgfscope}%
\begin{pgfscope}%
\pgfpathrectangle{\pgfqpoint{2.867647in}{0.500000in}}{\pgfqpoint{1.764706in}{1.700000in}}%
\pgfusepath{clip}%
\pgfsetbuttcap%
\pgfsetroundjoin%
\definecolor{currentfill}{rgb}{0.972201,0.839051,0.745789}%
\pgfsetfillcolor{currentfill}%
\pgfsetlinewidth{0.311001pt}%
\definecolor{currentstroke}{rgb}{1.000000,1.000000,1.000000}%
\pgfsetstrokecolor{currentstroke}%
\pgfsetdash{}{0pt}%
\pgfpathmoveto{\pgfqpoint{4.247542in}{1.172497in}}%
\pgfpathcurveto{\pgfqpoint{4.254675in}{1.172497in}}{\pgfqpoint{4.261517in}{1.175331in}}{\pgfqpoint{4.266560in}{1.180375in}}%
\pgfpathcurveto{\pgfqpoint{4.271604in}{1.185418in}}{\pgfqpoint{4.274438in}{1.192260in}}{\pgfqpoint{4.274438in}{1.199393in}}%
\pgfpathcurveto{\pgfqpoint{4.274438in}{1.206526in}}{\pgfqpoint{4.271604in}{1.213367in}}{\pgfqpoint{4.266560in}{1.218411in}}%
\pgfpathcurveto{\pgfqpoint{4.261517in}{1.223455in}}{\pgfqpoint{4.254675in}{1.226288in}}{\pgfqpoint{4.247542in}{1.226288in}}%
\pgfpathcurveto{\pgfqpoint{4.240409in}{1.226288in}}{\pgfqpoint{4.233568in}{1.223455in}}{\pgfqpoint{4.228524in}{1.218411in}}%
\pgfpathcurveto{\pgfqpoint{4.223480in}{1.213367in}}{\pgfqpoint{4.220647in}{1.206526in}}{\pgfqpoint{4.220647in}{1.199393in}}%
\pgfpathcurveto{\pgfqpoint{4.220647in}{1.192260in}}{\pgfqpoint{4.223480in}{1.185418in}}{\pgfqpoint{4.228524in}{1.180375in}}%
\pgfpathcurveto{\pgfqpoint{4.233568in}{1.175331in}}{\pgfqpoint{4.240409in}{1.172497in}}{\pgfqpoint{4.247542in}{1.172497in}}%
\pgfpathclose%
\pgfusepath{stroke,fill}%
\end{pgfscope}%
\begin{pgfscope}%
\pgfpathrectangle{\pgfqpoint{2.867647in}{0.500000in}}{\pgfqpoint{1.764706in}{1.700000in}}%
\pgfusepath{clip}%
\pgfsetbuttcap%
\pgfsetroundjoin%
\definecolor{currentfill}{rgb}{0.963884,0.644842,0.486120}%
\pgfsetfillcolor{currentfill}%
\pgfsetlinewidth{0.311001pt}%
\definecolor{currentstroke}{rgb}{1.000000,1.000000,1.000000}%
\pgfsetstrokecolor{currentstroke}%
\pgfsetdash{}{0pt}%
\pgfpathmoveto{\pgfqpoint{4.276537in}{1.569855in}}%
\pgfpathcurveto{\pgfqpoint{4.283669in}{1.569855in}}{\pgfqpoint{4.290511in}{1.572689in}}{\pgfqpoint{4.295555in}{1.577733in}}%
\pgfpathcurveto{\pgfqpoint{4.300598in}{1.582777in}}{\pgfqpoint{4.303432in}{1.589618in}}{\pgfqpoint{4.303432in}{1.596751in}}%
\pgfpathcurveto{\pgfqpoint{4.303432in}{1.603884in}}{\pgfqpoint{4.300598in}{1.610725in}}{\pgfqpoint{4.295555in}{1.615769in}}%
\pgfpathcurveto{\pgfqpoint{4.290511in}{1.620813in}}{\pgfqpoint{4.283669in}{1.623647in}}{\pgfqpoint{4.276537in}{1.623647in}}%
\pgfpathcurveto{\pgfqpoint{4.269404in}{1.623647in}}{\pgfqpoint{4.262562in}{1.620813in}}{\pgfqpoint{4.257518in}{1.615769in}}%
\pgfpathcurveto{\pgfqpoint{4.252475in}{1.610725in}}{\pgfqpoint{4.249641in}{1.603884in}}{\pgfqpoint{4.249641in}{1.596751in}}%
\pgfpathcurveto{\pgfqpoint{4.249641in}{1.589618in}}{\pgfqpoint{4.252475in}{1.582777in}}{\pgfqpoint{4.257518in}{1.577733in}}%
\pgfpathcurveto{\pgfqpoint{4.262562in}{1.572689in}}{\pgfqpoint{4.269404in}{1.569855in}}{\pgfqpoint{4.276537in}{1.569855in}}%
\pgfpathclose%
\pgfusepath{stroke,fill}%
\end{pgfscope}%
\begin{pgfscope}%
\pgfpathrectangle{\pgfqpoint{2.867647in}{0.500000in}}{\pgfqpoint{1.764706in}{1.700000in}}%
\pgfusepath{clip}%
\pgfsetbuttcap%
\pgfsetroundjoin%
\definecolor{currentfill}{rgb}{0.971694,0.833208,0.737161}%
\pgfsetfillcolor{currentfill}%
\pgfsetlinewidth{0.311001pt}%
\definecolor{currentstroke}{rgb}{1.000000,1.000000,1.000000}%
\pgfsetstrokecolor{currentstroke}%
\pgfsetdash{}{0pt}%
\pgfpathmoveto{\pgfqpoint{4.166965in}{1.006178in}}%
\pgfpathcurveto{\pgfqpoint{4.174097in}{1.006178in}}{\pgfqpoint{4.180939in}{1.009012in}}{\pgfqpoint{4.185983in}{1.014055in}}%
\pgfpathcurveto{\pgfqpoint{4.191026in}{1.019099in}}{\pgfqpoint{4.193860in}{1.025941in}}{\pgfqpoint{4.193860in}{1.033074in}}%
\pgfpathcurveto{\pgfqpoint{4.193860in}{1.040206in}}{\pgfqpoint{4.191026in}{1.047048in}}{\pgfqpoint{4.185983in}{1.052092in}}%
\pgfpathcurveto{\pgfqpoint{4.180939in}{1.057135in}}{\pgfqpoint{4.174097in}{1.059969in}}{\pgfqpoint{4.166965in}{1.059969in}}%
\pgfpathcurveto{\pgfqpoint{4.159832in}{1.059969in}}{\pgfqpoint{4.152990in}{1.057135in}}{\pgfqpoint{4.147946in}{1.052092in}}%
\pgfpathcurveto{\pgfqpoint{4.142903in}{1.047048in}}{\pgfqpoint{4.140069in}{1.040206in}}{\pgfqpoint{4.140069in}{1.033074in}}%
\pgfpathcurveto{\pgfqpoint{4.140069in}{1.025941in}}{\pgfqpoint{4.142903in}{1.019099in}}{\pgfqpoint{4.147946in}{1.014055in}}%
\pgfpathcurveto{\pgfqpoint{4.152990in}{1.009012in}}{\pgfqpoint{4.159832in}{1.006178in}}{\pgfqpoint{4.166965in}{1.006178in}}%
\pgfpathclose%
\pgfusepath{stroke,fill}%
\end{pgfscope}%
\begin{pgfscope}%
\pgfpathrectangle{\pgfqpoint{2.867647in}{0.500000in}}{\pgfqpoint{1.764706in}{1.700000in}}%
\pgfusepath{clip}%
\pgfsetbuttcap%
\pgfsetroundjoin%
\definecolor{currentfill}{rgb}{0.976961,0.885681,0.814303}%
\pgfsetfillcolor{currentfill}%
\pgfsetlinewidth{0.311001pt}%
\definecolor{currentstroke}{rgb}{1.000000,1.000000,1.000000}%
\pgfsetstrokecolor{currentstroke}%
\pgfsetdash{}{0pt}%
\pgfpathmoveto{\pgfqpoint{4.110857in}{1.106580in}}%
\pgfpathcurveto{\pgfqpoint{4.117990in}{1.106580in}}{\pgfqpoint{4.124831in}{1.109414in}}{\pgfqpoint{4.129875in}{1.114457in}}%
\pgfpathcurveto{\pgfqpoint{4.134919in}{1.119501in}}{\pgfqpoint{4.137753in}{1.126343in}}{\pgfqpoint{4.137753in}{1.133475in}}%
\pgfpathcurveto{\pgfqpoint{4.137753in}{1.140608in}}{\pgfqpoint{4.134919in}{1.147450in}}{\pgfqpoint{4.129875in}{1.152494in}}%
\pgfpathcurveto{\pgfqpoint{4.124831in}{1.157537in}}{\pgfqpoint{4.117990in}{1.160371in}}{\pgfqpoint{4.110857in}{1.160371in}}%
\pgfpathcurveto{\pgfqpoint{4.103724in}{1.160371in}}{\pgfqpoint{4.096882in}{1.157537in}}{\pgfqpoint{4.091839in}{1.152494in}}%
\pgfpathcurveto{\pgfqpoint{4.086795in}{1.147450in}}{\pgfqpoint{4.083961in}{1.140608in}}{\pgfqpoint{4.083961in}{1.133475in}}%
\pgfpathcurveto{\pgfqpoint{4.083961in}{1.126343in}}{\pgfqpoint{4.086795in}{1.119501in}}{\pgfqpoint{4.091839in}{1.114457in}}%
\pgfpathcurveto{\pgfqpoint{4.096882in}{1.109414in}}{\pgfqpoint{4.103724in}{1.106580in}}{\pgfqpoint{4.110857in}{1.106580in}}%
\pgfpathclose%
\pgfusepath{stroke,fill}%
\end{pgfscope}%
\begin{pgfscope}%
\pgfpathrectangle{\pgfqpoint{2.867647in}{0.500000in}}{\pgfqpoint{1.764706in}{1.700000in}}%
\pgfusepath{clip}%
\pgfsetbuttcap%
\pgfsetroundjoin%
\definecolor{currentfill}{rgb}{0.965302,0.713942,0.568499}%
\pgfsetfillcolor{currentfill}%
\pgfsetlinewidth{0.311001pt}%
\definecolor{currentstroke}{rgb}{1.000000,1.000000,1.000000}%
\pgfsetstrokecolor{currentstroke}%
\pgfsetdash{}{0pt}%
\pgfpathmoveto{\pgfqpoint{4.302269in}{1.390142in}}%
\pgfpathcurveto{\pgfqpoint{4.309402in}{1.390142in}}{\pgfqpoint{4.316243in}{1.392976in}}{\pgfqpoint{4.321287in}{1.398020in}}%
\pgfpathcurveto{\pgfqpoint{4.326331in}{1.403063in}}{\pgfqpoint{4.329164in}{1.409905in}}{\pgfqpoint{4.329164in}{1.417038in}}%
\pgfpathcurveto{\pgfqpoint{4.329164in}{1.424171in}}{\pgfqpoint{4.326331in}{1.431012in}}{\pgfqpoint{4.321287in}{1.436056in}}%
\pgfpathcurveto{\pgfqpoint{4.316243in}{1.441100in}}{\pgfqpoint{4.309402in}{1.443934in}}{\pgfqpoint{4.302269in}{1.443934in}}%
\pgfpathcurveto{\pgfqpoint{4.295136in}{1.443934in}}{\pgfqpoint{4.288294in}{1.441100in}}{\pgfqpoint{4.283251in}{1.436056in}}%
\pgfpathcurveto{\pgfqpoint{4.278207in}{1.431012in}}{\pgfqpoint{4.275373in}{1.424171in}}{\pgfqpoint{4.275373in}{1.417038in}}%
\pgfpathcurveto{\pgfqpoint{4.275373in}{1.409905in}}{\pgfqpoint{4.278207in}{1.403063in}}{\pgfqpoint{4.283251in}{1.398020in}}%
\pgfpathcurveto{\pgfqpoint{4.288294in}{1.392976in}}{\pgfqpoint{4.295136in}{1.390142in}}{\pgfqpoint{4.302269in}{1.390142in}}%
\pgfpathclose%
\pgfusepath{stroke,fill}%
\end{pgfscope}%
\begin{pgfscope}%
\pgfpathrectangle{\pgfqpoint{2.867647in}{0.500000in}}{\pgfqpoint{1.764706in}{1.700000in}}%
\pgfusepath{clip}%
\pgfsetbuttcap%
\pgfsetroundjoin%
\definecolor{currentfill}{rgb}{0.971202,0.827364,0.728520}%
\pgfsetfillcolor{currentfill}%
\pgfsetlinewidth{0.311001pt}%
\definecolor{currentstroke}{rgb}{1.000000,1.000000,1.000000}%
\pgfsetstrokecolor{currentstroke}%
\pgfsetdash{}{0pt}%
\pgfpathmoveto{\pgfqpoint{4.262454in}{1.407588in}}%
\pgfpathcurveto{\pgfqpoint{4.269587in}{1.407588in}}{\pgfqpoint{4.276429in}{1.410422in}}{\pgfqpoint{4.281472in}{1.415465in}}%
\pgfpathcurveto{\pgfqpoint{4.286516in}{1.420509in}}{\pgfqpoint{4.289350in}{1.427351in}}{\pgfqpoint{4.289350in}{1.434484in}}%
\pgfpathcurveto{\pgfqpoint{4.289350in}{1.441616in}}{\pgfqpoint{4.286516in}{1.448458in}}{\pgfqpoint{4.281472in}{1.453502in}}%
\pgfpathcurveto{\pgfqpoint{4.276429in}{1.458545in}}{\pgfqpoint{4.269587in}{1.461379in}}{\pgfqpoint{4.262454in}{1.461379in}}%
\pgfpathcurveto{\pgfqpoint{4.255321in}{1.461379in}}{\pgfqpoint{4.248480in}{1.458545in}}{\pgfqpoint{4.243436in}{1.453502in}}%
\pgfpathcurveto{\pgfqpoint{4.238392in}{1.448458in}}{\pgfqpoint{4.235559in}{1.441616in}}{\pgfqpoint{4.235559in}{1.434484in}}%
\pgfpathcurveto{\pgfqpoint{4.235559in}{1.427351in}}{\pgfqpoint{4.238392in}{1.420509in}}{\pgfqpoint{4.243436in}{1.415465in}}%
\pgfpathcurveto{\pgfqpoint{4.248480in}{1.410422in}}{\pgfqpoint{4.255321in}{1.407588in}}{\pgfqpoint{4.262454in}{1.407588in}}%
\pgfpathclose%
\pgfusepath{stroke,fill}%
\end{pgfscope}%
\begin{pgfscope}%
\pgfpathrectangle{\pgfqpoint{2.867647in}{0.500000in}}{\pgfqpoint{1.764706in}{1.700000in}}%
\pgfusepath{clip}%
\pgfsetbuttcap%
\pgfsetroundjoin%
\definecolor{currentfill}{rgb}{0.964799,0.689101,0.537560}%
\pgfsetfillcolor{currentfill}%
\pgfsetlinewidth{0.311001pt}%
\definecolor{currentstroke}{rgb}{1.000000,1.000000,1.000000}%
\pgfsetstrokecolor{currentstroke}%
\pgfsetdash{}{0pt}%
\pgfpathmoveto{\pgfqpoint{4.010716in}{1.561718in}}%
\pgfpathcurveto{\pgfqpoint{4.017849in}{1.561718in}}{\pgfqpoint{4.024691in}{1.564552in}}{\pgfqpoint{4.029734in}{1.569596in}}%
\pgfpathcurveto{\pgfqpoint{4.034778in}{1.574639in}}{\pgfqpoint{4.037612in}{1.581481in}}{\pgfqpoint{4.037612in}{1.588614in}}%
\pgfpathcurveto{\pgfqpoint{4.037612in}{1.595747in}}{\pgfqpoint{4.034778in}{1.602588in}}{\pgfqpoint{4.029734in}{1.607632in}}%
\pgfpathcurveto{\pgfqpoint{4.024691in}{1.612676in}}{\pgfqpoint{4.017849in}{1.615509in}}{\pgfqpoint{4.010716in}{1.615509in}}%
\pgfpathcurveto{\pgfqpoint{4.003583in}{1.615509in}}{\pgfqpoint{3.996742in}{1.612676in}}{\pgfqpoint{3.991698in}{1.607632in}}%
\pgfpathcurveto{\pgfqpoint{3.986655in}{1.602588in}}{\pgfqpoint{3.983821in}{1.595747in}}{\pgfqpoint{3.983821in}{1.588614in}}%
\pgfpathcurveto{\pgfqpoint{3.983821in}{1.581481in}}{\pgfqpoint{3.986655in}{1.574639in}}{\pgfqpoint{3.991698in}{1.569596in}}%
\pgfpathcurveto{\pgfqpoint{3.996742in}{1.564552in}}{\pgfqpoint{4.003583in}{1.561718in}}{\pgfqpoint{4.010716in}{1.561718in}}%
\pgfpathclose%
\pgfusepath{stroke,fill}%
\end{pgfscope}%
\begin{pgfscope}%
\pgfpathrectangle{\pgfqpoint{2.867647in}{0.500000in}}{\pgfqpoint{1.764706in}{1.700000in}}%
\pgfusepath{clip}%
\pgfsetbuttcap%
\pgfsetroundjoin%
\definecolor{currentfill}{rgb}{0.980678,0.914765,0.856766}%
\pgfsetfillcolor{currentfill}%
\pgfsetlinewidth{0.311001pt}%
\definecolor{currentstroke}{rgb}{1.000000,1.000000,1.000000}%
\pgfsetstrokecolor{currentstroke}%
\pgfsetdash{}{0pt}%
\pgfpathmoveto{\pgfqpoint{4.152716in}{1.143263in}}%
\pgfpathcurveto{\pgfqpoint{4.159848in}{1.143263in}}{\pgfqpoint{4.166690in}{1.146096in}}{\pgfqpoint{4.171734in}{1.151140in}}%
\pgfpathcurveto{\pgfqpoint{4.176777in}{1.156184in}}{\pgfqpoint{4.179611in}{1.163025in}}{\pgfqpoint{4.179611in}{1.170158in}}%
\pgfpathcurveto{\pgfqpoint{4.179611in}{1.177291in}}{\pgfqpoint{4.176777in}{1.184133in}}{\pgfqpoint{4.171734in}{1.189176in}}%
\pgfpathcurveto{\pgfqpoint{4.166690in}{1.194220in}}{\pgfqpoint{4.159848in}{1.197054in}}{\pgfqpoint{4.152716in}{1.197054in}}%
\pgfpathcurveto{\pgfqpoint{4.145583in}{1.197054in}}{\pgfqpoint{4.138741in}{1.194220in}}{\pgfqpoint{4.133697in}{1.189176in}}%
\pgfpathcurveto{\pgfqpoint{4.128654in}{1.184133in}}{\pgfqpoint{4.125820in}{1.177291in}}{\pgfqpoint{4.125820in}{1.170158in}}%
\pgfpathcurveto{\pgfqpoint{4.125820in}{1.163025in}}{\pgfqpoint{4.128654in}{1.156184in}}{\pgfqpoint{4.133697in}{1.151140in}}%
\pgfpathcurveto{\pgfqpoint{4.138741in}{1.146096in}}{\pgfqpoint{4.145583in}{1.143263in}}{\pgfqpoint{4.152716in}{1.143263in}}%
\pgfpathclose%
\pgfusepath{stroke,fill}%
\end{pgfscope}%
\begin{pgfscope}%
\pgfpathrectangle{\pgfqpoint{2.867647in}{0.500000in}}{\pgfqpoint{1.764706in}{1.700000in}}%
\pgfusepath{clip}%
\pgfsetbuttcap%
\pgfsetroundjoin%
\definecolor{currentfill}{rgb}{0.977657,0.891500,0.822809}%
\pgfsetfillcolor{currentfill}%
\pgfsetlinewidth{0.311001pt}%
\definecolor{currentstroke}{rgb}{1.000000,1.000000,1.000000}%
\pgfsetstrokecolor{currentstroke}%
\pgfsetdash{}{0pt}%
\pgfpathmoveto{\pgfqpoint{4.172047in}{1.073339in}}%
\pgfpathcurveto{\pgfqpoint{4.179180in}{1.073339in}}{\pgfqpoint{4.186022in}{1.076173in}}{\pgfqpoint{4.191065in}{1.081216in}}%
\pgfpathcurveto{\pgfqpoint{4.196109in}{1.086260in}}{\pgfqpoint{4.198943in}{1.093102in}}{\pgfqpoint{4.198943in}{1.100235in}}%
\pgfpathcurveto{\pgfqpoint{4.198943in}{1.107367in}}{\pgfqpoint{4.196109in}{1.114209in}}{\pgfqpoint{4.191065in}{1.119253in}}%
\pgfpathcurveto{\pgfqpoint{4.186022in}{1.124296in}}{\pgfqpoint{4.179180in}{1.127130in}}{\pgfqpoint{4.172047in}{1.127130in}}%
\pgfpathcurveto{\pgfqpoint{4.164914in}{1.127130in}}{\pgfqpoint{4.158073in}{1.124296in}}{\pgfqpoint{4.153029in}{1.119253in}}%
\pgfpathcurveto{\pgfqpoint{4.147985in}{1.114209in}}{\pgfqpoint{4.145151in}{1.107367in}}{\pgfqpoint{4.145151in}{1.100235in}}%
\pgfpathcurveto{\pgfqpoint{4.145151in}{1.093102in}}{\pgfqpoint{4.147985in}{1.086260in}}{\pgfqpoint{4.153029in}{1.081216in}}%
\pgfpathcurveto{\pgfqpoint{4.158073in}{1.076173in}}{\pgfqpoint{4.164914in}{1.073339in}}{\pgfqpoint{4.172047in}{1.073339in}}%
\pgfpathclose%
\pgfusepath{stroke,fill}%
\end{pgfscope}%
\begin{pgfscope}%
\pgfpathrectangle{\pgfqpoint{2.867647in}{0.500000in}}{\pgfqpoint{1.764706in}{1.700000in}}%
\pgfusepath{clip}%
\pgfsetbuttcap%
\pgfsetroundjoin%
\definecolor{currentfill}{rgb}{0.966328,0.750560,0.616961}%
\pgfsetfillcolor{currentfill}%
\pgfsetlinewidth{0.311001pt}%
\definecolor{currentstroke}{rgb}{1.000000,1.000000,1.000000}%
\pgfsetstrokecolor{currentstroke}%
\pgfsetdash{}{0pt}%
\pgfpathmoveto{\pgfqpoint{4.263398in}{1.523593in}}%
\pgfpathcurveto{\pgfqpoint{4.270531in}{1.523593in}}{\pgfqpoint{4.277372in}{1.526427in}}{\pgfqpoint{4.282416in}{1.531470in}}%
\pgfpathcurveto{\pgfqpoint{4.287460in}{1.536514in}}{\pgfqpoint{4.290294in}{1.543356in}}{\pgfqpoint{4.290294in}{1.550488in}}%
\pgfpathcurveto{\pgfqpoint{4.290294in}{1.557621in}}{\pgfqpoint{4.287460in}{1.564463in}}{\pgfqpoint{4.282416in}{1.569507in}}%
\pgfpathcurveto{\pgfqpoint{4.277372in}{1.574550in}}{\pgfqpoint{4.270531in}{1.577384in}}{\pgfqpoint{4.263398in}{1.577384in}}%
\pgfpathcurveto{\pgfqpoint{4.256265in}{1.577384in}}{\pgfqpoint{4.249423in}{1.574550in}}{\pgfqpoint{4.244380in}{1.569507in}}%
\pgfpathcurveto{\pgfqpoint{4.239336in}{1.564463in}}{\pgfqpoint{4.236502in}{1.557621in}}{\pgfqpoint{4.236502in}{1.550488in}}%
\pgfpathcurveto{\pgfqpoint{4.236502in}{1.543356in}}{\pgfqpoint{4.239336in}{1.536514in}}{\pgfqpoint{4.244380in}{1.531470in}}%
\pgfpathcurveto{\pgfqpoint{4.249423in}{1.526427in}}{\pgfqpoint{4.256265in}{1.523593in}}{\pgfqpoint{4.263398in}{1.523593in}}%
\pgfpathclose%
\pgfusepath{stroke,fill}%
\end{pgfscope}%
\begin{pgfscope}%
\pgfpathrectangle{\pgfqpoint{2.867647in}{0.500000in}}{\pgfqpoint{1.764706in}{1.700000in}}%
\pgfusepath{clip}%
\pgfsetbuttcap%
\pgfsetroundjoin%
\definecolor{currentfill}{rgb}{0.973271,0.850724,0.762998}%
\pgfsetfillcolor{currentfill}%
\pgfsetlinewidth{0.311001pt}%
\definecolor{currentstroke}{rgb}{1.000000,1.000000,1.000000}%
\pgfsetstrokecolor{currentstroke}%
\pgfsetdash{}{0pt}%
\pgfpathmoveto{\pgfqpoint{4.101472in}{1.669493in}}%
\pgfpathcurveto{\pgfqpoint{4.108604in}{1.669493in}}{\pgfqpoint{4.115446in}{1.672327in}}{\pgfqpoint{4.120490in}{1.677371in}}%
\pgfpathcurveto{\pgfqpoint{4.125533in}{1.682414in}}{\pgfqpoint{4.128367in}{1.689256in}}{\pgfqpoint{4.128367in}{1.696389in}}%
\pgfpathcurveto{\pgfqpoint{4.128367in}{1.703522in}}{\pgfqpoint{4.125533in}{1.710363in}}{\pgfqpoint{4.120490in}{1.715407in}}%
\pgfpathcurveto{\pgfqpoint{4.115446in}{1.720451in}}{\pgfqpoint{4.108604in}{1.723285in}}{\pgfqpoint{4.101472in}{1.723285in}}%
\pgfpathcurveto{\pgfqpoint{4.094339in}{1.723285in}}{\pgfqpoint{4.087497in}{1.720451in}}{\pgfqpoint{4.082453in}{1.715407in}}%
\pgfpathcurveto{\pgfqpoint{4.077410in}{1.710363in}}{\pgfqpoint{4.074576in}{1.703522in}}{\pgfqpoint{4.074576in}{1.696389in}}%
\pgfpathcurveto{\pgfqpoint{4.074576in}{1.689256in}}{\pgfqpoint{4.077410in}{1.682414in}}{\pgfqpoint{4.082453in}{1.677371in}}%
\pgfpathcurveto{\pgfqpoint{4.087497in}{1.672327in}}{\pgfqpoint{4.094339in}{1.669493in}}{\pgfqpoint{4.101472in}{1.669493in}}%
\pgfpathclose%
\pgfusepath{stroke,fill}%
\end{pgfscope}%
\begin{pgfscope}%
\pgfpathrectangle{\pgfqpoint{2.867647in}{0.500000in}}{\pgfqpoint{1.764706in}{1.700000in}}%
\pgfusepath{clip}%
\pgfsetbuttcap%
\pgfsetroundjoin%
\definecolor{currentfill}{rgb}{0.979891,0.908948,0.848279}%
\pgfsetfillcolor{currentfill}%
\pgfsetlinewidth{0.311001pt}%
\definecolor{currentstroke}{rgb}{1.000000,1.000000,1.000000}%
\pgfsetstrokecolor{currentstroke}%
\pgfsetdash{}{0pt}%
\pgfpathmoveto{\pgfqpoint{4.192150in}{1.147171in}}%
\pgfpathcurveto{\pgfqpoint{4.199283in}{1.147171in}}{\pgfqpoint{4.206125in}{1.150004in}}{\pgfqpoint{4.211169in}{1.155048in}}%
\pgfpathcurveto{\pgfqpoint{4.216212in}{1.160092in}}{\pgfqpoint{4.219046in}{1.166933in}}{\pgfqpoint{4.219046in}{1.174066in}}%
\pgfpathcurveto{\pgfqpoint{4.219046in}{1.181199in}}{\pgfqpoint{4.216212in}{1.188041in}}{\pgfqpoint{4.211169in}{1.193084in}}%
\pgfpathcurveto{\pgfqpoint{4.206125in}{1.198128in}}{\pgfqpoint{4.199283in}{1.200962in}}{\pgfqpoint{4.192150in}{1.200962in}}%
\pgfpathcurveto{\pgfqpoint{4.185018in}{1.200962in}}{\pgfqpoint{4.178176in}{1.198128in}}{\pgfqpoint{4.173132in}{1.193084in}}%
\pgfpathcurveto{\pgfqpoint{4.168089in}{1.188041in}}{\pgfqpoint{4.165255in}{1.181199in}}{\pgfqpoint{4.165255in}{1.174066in}}%
\pgfpathcurveto{\pgfqpoint{4.165255in}{1.166933in}}{\pgfqpoint{4.168089in}{1.160092in}}{\pgfqpoint{4.173132in}{1.155048in}}%
\pgfpathcurveto{\pgfqpoint{4.178176in}{1.150004in}}{\pgfqpoint{4.185018in}{1.147171in}}{\pgfqpoint{4.192150in}{1.147171in}}%
\pgfpathclose%
\pgfusepath{stroke,fill}%
\end{pgfscope}%
\begin{pgfscope}%
\pgfpathrectangle{\pgfqpoint{2.867647in}{0.500000in}}{\pgfqpoint{1.764706in}{1.700000in}}%
\pgfusepath{clip}%
\pgfsetbuttcap%
\pgfsetroundjoin%
\definecolor{currentfill}{rgb}{0.965302,0.713942,0.568499}%
\pgfsetfillcolor{currentfill}%
\pgfsetlinewidth{0.311001pt}%
\definecolor{currentstroke}{rgb}{1.000000,1.000000,1.000000}%
\pgfsetstrokecolor{currentstroke}%
\pgfsetdash{}{0pt}%
\pgfpathmoveto{\pgfqpoint{4.039843in}{0.893418in}}%
\pgfpathcurveto{\pgfqpoint{4.046975in}{0.893418in}}{\pgfqpoint{4.053817in}{0.896252in}}{\pgfqpoint{4.058861in}{0.901295in}}%
\pgfpathcurveto{\pgfqpoint{4.063904in}{0.906339in}}{\pgfqpoint{4.066738in}{0.913181in}}{\pgfqpoint{4.066738in}{0.920313in}}%
\pgfpathcurveto{\pgfqpoint{4.066738in}{0.927446in}}{\pgfqpoint{4.063904in}{0.934288in}}{\pgfqpoint{4.058861in}{0.939332in}}%
\pgfpathcurveto{\pgfqpoint{4.053817in}{0.944375in}}{\pgfqpoint{4.046975in}{0.947209in}}{\pgfqpoint{4.039843in}{0.947209in}}%
\pgfpathcurveto{\pgfqpoint{4.032710in}{0.947209in}}{\pgfqpoint{4.025868in}{0.944375in}}{\pgfqpoint{4.020824in}{0.939332in}}%
\pgfpathcurveto{\pgfqpoint{4.015781in}{0.934288in}}{\pgfqpoint{4.012947in}{0.927446in}}{\pgfqpoint{4.012947in}{0.920313in}}%
\pgfpathcurveto{\pgfqpoint{4.012947in}{0.913181in}}{\pgfqpoint{4.015781in}{0.906339in}}{\pgfqpoint{4.020824in}{0.901295in}}%
\pgfpathcurveto{\pgfqpoint{4.025868in}{0.896252in}}{\pgfqpoint{4.032710in}{0.893418in}}{\pgfqpoint{4.039843in}{0.893418in}}%
\pgfpathclose%
\pgfusepath{stroke,fill}%
\end{pgfscope}%
\begin{pgfscope}%
\pgfpathrectangle{\pgfqpoint{2.867647in}{0.500000in}}{\pgfqpoint{1.764706in}{1.700000in}}%
\pgfusepath{clip}%
\pgfsetbuttcap%
\pgfsetroundjoin%
\definecolor{currentfill}{rgb}{0.964799,0.689101,0.537560}%
\pgfsetfillcolor{currentfill}%
\pgfsetlinewidth{0.311001pt}%
\definecolor{currentstroke}{rgb}{1.000000,1.000000,1.000000}%
\pgfsetstrokecolor{currentstroke}%
\pgfsetdash{}{0pt}%
\pgfpathmoveto{\pgfqpoint{4.050362in}{1.190917in}}%
\pgfpathcurveto{\pgfqpoint{4.057495in}{1.190917in}}{\pgfqpoint{4.064337in}{1.193751in}}{\pgfqpoint{4.069381in}{1.198794in}}%
\pgfpathcurveto{\pgfqpoint{4.074424in}{1.203838in}}{\pgfqpoint{4.077258in}{1.210680in}}{\pgfqpoint{4.077258in}{1.217813in}}%
\pgfpathcurveto{\pgfqpoint{4.077258in}{1.224945in}}{\pgfqpoint{4.074424in}{1.231787in}}{\pgfqpoint{4.069381in}{1.236831in}}%
\pgfpathcurveto{\pgfqpoint{4.064337in}{1.241874in}}{\pgfqpoint{4.057495in}{1.244708in}}{\pgfqpoint{4.050362in}{1.244708in}}%
\pgfpathcurveto{\pgfqpoint{4.043230in}{1.244708in}}{\pgfqpoint{4.036388in}{1.241874in}}{\pgfqpoint{4.031344in}{1.236831in}}%
\pgfpathcurveto{\pgfqpoint{4.026301in}{1.231787in}}{\pgfqpoint{4.023467in}{1.224945in}}{\pgfqpoint{4.023467in}{1.217813in}}%
\pgfpathcurveto{\pgfqpoint{4.023467in}{1.210680in}}{\pgfqpoint{4.026301in}{1.203838in}}{\pgfqpoint{4.031344in}{1.198794in}}%
\pgfpathcurveto{\pgfqpoint{4.036388in}{1.193751in}}{\pgfqpoint{4.043230in}{1.190917in}}{\pgfqpoint{4.050362in}{1.190917in}}%
\pgfpathclose%
\pgfusepath{stroke,fill}%
\end{pgfscope}%
\begin{pgfscope}%
\pgfpathrectangle{\pgfqpoint{2.867647in}{0.500000in}}{\pgfqpoint{1.764706in}{1.700000in}}%
\pgfusepath{clip}%
\pgfsetbuttcap%
\pgfsetroundjoin%
\definecolor{currentfill}{rgb}{0.971694,0.833208,0.737161}%
\pgfsetfillcolor{currentfill}%
\pgfsetlinewidth{0.311001pt}%
\definecolor{currentstroke}{rgb}{1.000000,1.000000,1.000000}%
\pgfsetstrokecolor{currentstroke}%
\pgfsetdash{}{0pt}%
\pgfpathmoveto{\pgfqpoint{4.092600in}{1.176473in}}%
\pgfpathcurveto{\pgfqpoint{4.099732in}{1.176473in}}{\pgfqpoint{4.106574in}{1.179307in}}{\pgfqpoint{4.111618in}{1.184350in}}%
\pgfpathcurveto{\pgfqpoint{4.116661in}{1.189394in}}{\pgfqpoint{4.119495in}{1.196236in}}{\pgfqpoint{4.119495in}{1.203369in}}%
\pgfpathcurveto{\pgfqpoint{4.119495in}{1.210501in}}{\pgfqpoint{4.116661in}{1.217343in}}{\pgfqpoint{4.111618in}{1.222387in}}%
\pgfpathcurveto{\pgfqpoint{4.106574in}{1.227430in}}{\pgfqpoint{4.099732in}{1.230264in}}{\pgfqpoint{4.092600in}{1.230264in}}%
\pgfpathcurveto{\pgfqpoint{4.085467in}{1.230264in}}{\pgfqpoint{4.078625in}{1.227430in}}{\pgfqpoint{4.073581in}{1.222387in}}%
\pgfpathcurveto{\pgfqpoint{4.068538in}{1.217343in}}{\pgfqpoint{4.065704in}{1.210501in}}{\pgfqpoint{4.065704in}{1.203369in}}%
\pgfpathcurveto{\pgfqpoint{4.065704in}{1.196236in}}{\pgfqpoint{4.068538in}{1.189394in}}{\pgfqpoint{4.073581in}{1.184350in}}%
\pgfpathcurveto{\pgfqpoint{4.078625in}{1.179307in}}{\pgfqpoint{4.085467in}{1.176473in}}{\pgfqpoint{4.092600in}{1.176473in}}%
\pgfpathclose%
\pgfusepath{stroke,fill}%
\end{pgfscope}%
\begin{pgfscope}%
\pgfpathrectangle{\pgfqpoint{2.867647in}{0.500000in}}{\pgfqpoint{1.764706in}{1.700000in}}%
\pgfusepath{clip}%
\pgfsetbuttcap%
\pgfsetroundjoin%
\definecolor{currentfill}{rgb}{0.964558,0.676556,0.522514}%
\pgfsetfillcolor{currentfill}%
\pgfsetlinewidth{0.311001pt}%
\definecolor{currentstroke}{rgb}{1.000000,1.000000,1.000000}%
\pgfsetstrokecolor{currentstroke}%
\pgfsetdash{}{0pt}%
\pgfpathmoveto{\pgfqpoint{4.246098in}{1.022961in}}%
\pgfpathcurveto{\pgfqpoint{4.253231in}{1.022961in}}{\pgfqpoint{4.260072in}{1.025795in}}{\pgfqpoint{4.265116in}{1.030838in}}%
\pgfpathcurveto{\pgfqpoint{4.270160in}{1.035882in}}{\pgfqpoint{4.272994in}{1.042724in}}{\pgfqpoint{4.272994in}{1.049857in}}%
\pgfpathcurveto{\pgfqpoint{4.272994in}{1.056989in}}{\pgfqpoint{4.270160in}{1.063831in}}{\pgfqpoint{4.265116in}{1.068875in}}%
\pgfpathcurveto{\pgfqpoint{4.260072in}{1.073918in}}{\pgfqpoint{4.253231in}{1.076752in}}{\pgfqpoint{4.246098in}{1.076752in}}%
\pgfpathcurveto{\pgfqpoint{4.238965in}{1.076752in}}{\pgfqpoint{4.232123in}{1.073918in}}{\pgfqpoint{4.227080in}{1.068875in}}%
\pgfpathcurveto{\pgfqpoint{4.222036in}{1.063831in}}{\pgfqpoint{4.219202in}{1.056989in}}{\pgfqpoint{4.219202in}{1.049857in}}%
\pgfpathcurveto{\pgfqpoint{4.219202in}{1.042724in}}{\pgfqpoint{4.222036in}{1.035882in}}{\pgfqpoint{4.227080in}{1.030838in}}%
\pgfpathcurveto{\pgfqpoint{4.232123in}{1.025795in}}{\pgfqpoint{4.238965in}{1.022961in}}{\pgfqpoint{4.246098in}{1.022961in}}%
\pgfpathclose%
\pgfusepath{stroke,fill}%
\end{pgfscope}%
\begin{pgfscope}%
\pgfpathrectangle{\pgfqpoint{2.867647in}{0.500000in}}{\pgfqpoint{1.764706in}{1.700000in}}%
\pgfusepath{clip}%
\pgfsetbuttcap%
\pgfsetroundjoin%
\definecolor{currentfill}{rgb}{0.981377,0.920617,0.865369}%
\pgfsetfillcolor{currentfill}%
\pgfsetlinewidth{0.311001pt}%
\definecolor{currentstroke}{rgb}{1.000000,1.000000,1.000000}%
\pgfsetstrokecolor{currentstroke}%
\pgfsetdash{}{0pt}%
\pgfpathmoveto{\pgfqpoint{4.196002in}{1.296957in}}%
\pgfpathcurveto{\pgfqpoint{4.203135in}{1.296957in}}{\pgfqpoint{4.209977in}{1.299791in}}{\pgfqpoint{4.215020in}{1.304835in}}%
\pgfpathcurveto{\pgfqpoint{4.220064in}{1.309878in}}{\pgfqpoint{4.222898in}{1.316720in}}{\pgfqpoint{4.222898in}{1.323853in}}%
\pgfpathcurveto{\pgfqpoint{4.222898in}{1.330986in}}{\pgfqpoint{4.220064in}{1.337827in}}{\pgfqpoint{4.215020in}{1.342871in}}%
\pgfpathcurveto{\pgfqpoint{4.209977in}{1.347915in}}{\pgfqpoint{4.203135in}{1.350749in}}{\pgfqpoint{4.196002in}{1.350749in}}%
\pgfpathcurveto{\pgfqpoint{4.188869in}{1.350749in}}{\pgfqpoint{4.182028in}{1.347915in}}{\pgfqpoint{4.176984in}{1.342871in}}%
\pgfpathcurveto{\pgfqpoint{4.171940in}{1.337827in}}{\pgfqpoint{4.169106in}{1.330986in}}{\pgfqpoint{4.169106in}{1.323853in}}%
\pgfpathcurveto{\pgfqpoint{4.169106in}{1.316720in}}{\pgfqpoint{4.171940in}{1.309878in}}{\pgfqpoint{4.176984in}{1.304835in}}%
\pgfpathcurveto{\pgfqpoint{4.182028in}{1.299791in}}{\pgfqpoint{4.188869in}{1.296957in}}{\pgfqpoint{4.196002in}{1.296957in}}%
\pgfpathclose%
\pgfusepath{stroke,fill}%
\end{pgfscope}%
\begin{pgfscope}%
\pgfpathrectangle{\pgfqpoint{2.867647in}{0.500000in}}{\pgfqpoint{1.764706in}{1.700000in}}%
\pgfusepath{clip}%
\pgfsetbuttcap%
\pgfsetroundjoin%
\definecolor{currentfill}{rgb}{0.967735,0.780441,0.659127}%
\pgfsetfillcolor{currentfill}%
\pgfsetlinewidth{0.311001pt}%
\definecolor{currentstroke}{rgb}{1.000000,1.000000,1.000000}%
\pgfsetstrokecolor{currentstroke}%
\pgfsetdash{}{0pt}%
\pgfpathmoveto{\pgfqpoint{4.255853in}{1.118592in}}%
\pgfpathcurveto{\pgfqpoint{4.262986in}{1.118592in}}{\pgfqpoint{4.269827in}{1.121426in}}{\pgfqpoint{4.274871in}{1.126469in}}%
\pgfpathcurveto{\pgfqpoint{4.279915in}{1.131513in}}{\pgfqpoint{4.282749in}{1.138355in}}{\pgfqpoint{4.282749in}{1.145488in}}%
\pgfpathcurveto{\pgfqpoint{4.282749in}{1.152620in}}{\pgfqpoint{4.279915in}{1.159462in}}{\pgfqpoint{4.274871in}{1.164506in}}%
\pgfpathcurveto{\pgfqpoint{4.269827in}{1.169549in}}{\pgfqpoint{4.262986in}{1.172383in}}{\pgfqpoint{4.255853in}{1.172383in}}%
\pgfpathcurveto{\pgfqpoint{4.248720in}{1.172383in}}{\pgfqpoint{4.241879in}{1.169549in}}{\pgfqpoint{4.236835in}{1.164506in}}%
\pgfpathcurveto{\pgfqpoint{4.231791in}{1.159462in}}{\pgfqpoint{4.228957in}{1.152620in}}{\pgfqpoint{4.228957in}{1.145488in}}%
\pgfpathcurveto{\pgfqpoint{4.228957in}{1.138355in}}{\pgfqpoint{4.231791in}{1.131513in}}{\pgfqpoint{4.236835in}{1.126469in}}%
\pgfpathcurveto{\pgfqpoint{4.241879in}{1.121426in}}{\pgfqpoint{4.248720in}{1.118592in}}{\pgfqpoint{4.255853in}{1.118592in}}%
\pgfpathclose%
\pgfusepath{stroke,fill}%
\end{pgfscope}%
\begin{pgfscope}%
\pgfpathrectangle{\pgfqpoint{2.867647in}{0.500000in}}{\pgfqpoint{1.764706in}{1.700000in}}%
\pgfusepath{clip}%
\pgfsetbuttcap%
\pgfsetroundjoin%
\definecolor{currentfill}{rgb}{0.977657,0.891500,0.822809}%
\pgfsetfillcolor{currentfill}%
\pgfsetlinewidth{0.311001pt}%
\definecolor{currentstroke}{rgb}{1.000000,1.000000,1.000000}%
\pgfsetstrokecolor{currentstroke}%
\pgfsetdash{}{0pt}%
\pgfpathmoveto{\pgfqpoint{4.129462in}{1.052563in}}%
\pgfpathcurveto{\pgfqpoint{4.136594in}{1.052563in}}{\pgfqpoint{4.143436in}{1.055397in}}{\pgfqpoint{4.148480in}{1.060441in}}%
\pgfpathcurveto{\pgfqpoint{4.153523in}{1.065484in}}{\pgfqpoint{4.156357in}{1.072326in}}{\pgfqpoint{4.156357in}{1.079459in}}%
\pgfpathcurveto{\pgfqpoint{4.156357in}{1.086591in}}{\pgfqpoint{4.153523in}{1.093433in}}{\pgfqpoint{4.148480in}{1.098477in}}%
\pgfpathcurveto{\pgfqpoint{4.143436in}{1.103520in}}{\pgfqpoint{4.136594in}{1.106354in}}{\pgfqpoint{4.129462in}{1.106354in}}%
\pgfpathcurveto{\pgfqpoint{4.122329in}{1.106354in}}{\pgfqpoint{4.115487in}{1.103520in}}{\pgfqpoint{4.110443in}{1.098477in}}%
\pgfpathcurveto{\pgfqpoint{4.105400in}{1.093433in}}{\pgfqpoint{4.102566in}{1.086591in}}{\pgfqpoint{4.102566in}{1.079459in}}%
\pgfpathcurveto{\pgfqpoint{4.102566in}{1.072326in}}{\pgfqpoint{4.105400in}{1.065484in}}{\pgfqpoint{4.110443in}{1.060441in}}%
\pgfpathcurveto{\pgfqpoint{4.115487in}{1.055397in}}{\pgfqpoint{4.122329in}{1.052563in}}{\pgfqpoint{4.129462in}{1.052563in}}%
\pgfpathclose%
\pgfusepath{stroke,fill}%
\end{pgfscope}%
\begin{pgfscope}%
\pgfpathrectangle{\pgfqpoint{2.867647in}{0.500000in}}{\pgfqpoint{1.764706in}{1.700000in}}%
\pgfusepath{clip}%
\pgfsetbuttcap%
\pgfsetroundjoin%
\definecolor{currentfill}{rgb}{0.950851,0.435000,0.297228}%
\pgfsetfillcolor{currentfill}%
\pgfsetlinewidth{0.311001pt}%
\definecolor{currentstroke}{rgb}{1.000000,1.000000,1.000000}%
\pgfsetstrokecolor{currentstroke}%
\pgfsetdash{}{0pt}%
\pgfpathmoveto{\pgfqpoint{4.320728in}{1.559062in}}%
\pgfpathcurveto{\pgfqpoint{4.327861in}{1.559062in}}{\pgfqpoint{4.334703in}{1.561896in}}{\pgfqpoint{4.339746in}{1.566940in}}%
\pgfpathcurveto{\pgfqpoint{4.344790in}{1.571984in}}{\pgfqpoint{4.347624in}{1.578825in}}{\pgfqpoint{4.347624in}{1.585958in}}%
\pgfpathcurveto{\pgfqpoint{4.347624in}{1.593091in}}{\pgfqpoint{4.344790in}{1.599932in}}{\pgfqpoint{4.339746in}{1.604976in}}%
\pgfpathcurveto{\pgfqpoint{4.334703in}{1.610020in}}{\pgfqpoint{4.327861in}{1.612854in}}{\pgfqpoint{4.320728in}{1.612854in}}%
\pgfpathcurveto{\pgfqpoint{4.313595in}{1.612854in}}{\pgfqpoint{4.306754in}{1.610020in}}{\pgfqpoint{4.301710in}{1.604976in}}%
\pgfpathcurveto{\pgfqpoint{4.296666in}{1.599932in}}{\pgfqpoint{4.293832in}{1.593091in}}{\pgfqpoint{4.293832in}{1.585958in}}%
\pgfpathcurveto{\pgfqpoint{4.293832in}{1.578825in}}{\pgfqpoint{4.296666in}{1.571984in}}{\pgfqpoint{4.301710in}{1.566940in}}%
\pgfpathcurveto{\pgfqpoint{4.306754in}{1.561896in}}{\pgfqpoint{4.313595in}{1.559062in}}{\pgfqpoint{4.320728in}{1.559062in}}%
\pgfpathclose%
\pgfusepath{stroke,fill}%
\end{pgfscope}%
\begin{pgfscope}%
\pgfpathrectangle{\pgfqpoint{2.867647in}{0.500000in}}{\pgfqpoint{1.764706in}{1.700000in}}%
\pgfusepath{clip}%
\pgfsetbuttcap%
\pgfsetroundjoin%
\definecolor{currentfill}{rgb}{0.970718,0.821518,0.719872}%
\pgfsetfillcolor{currentfill}%
\pgfsetlinewidth{0.311001pt}%
\definecolor{currentstroke}{rgb}{1.000000,1.000000,1.000000}%
\pgfsetstrokecolor{currentstroke}%
\pgfsetdash{}{0pt}%
\pgfpathmoveto{\pgfqpoint{4.085809in}{1.693097in}}%
\pgfpathcurveto{\pgfqpoint{4.092942in}{1.693097in}}{\pgfqpoint{4.099783in}{1.695931in}}{\pgfqpoint{4.104827in}{1.700975in}}%
\pgfpathcurveto{\pgfqpoint{4.109871in}{1.706019in}}{\pgfqpoint{4.112705in}{1.712860in}}{\pgfqpoint{4.112705in}{1.719993in}}%
\pgfpathcurveto{\pgfqpoint{4.112705in}{1.727126in}}{\pgfqpoint{4.109871in}{1.733968in}}{\pgfqpoint{4.104827in}{1.739011in}}%
\pgfpathcurveto{\pgfqpoint{4.099783in}{1.744055in}}{\pgfqpoint{4.092942in}{1.746889in}}{\pgfqpoint{4.085809in}{1.746889in}}%
\pgfpathcurveto{\pgfqpoint{4.078676in}{1.746889in}}{\pgfqpoint{4.071834in}{1.744055in}}{\pgfqpoint{4.066791in}{1.739011in}}%
\pgfpathcurveto{\pgfqpoint{4.061747in}{1.733968in}}{\pgfqpoint{4.058913in}{1.727126in}}{\pgfqpoint{4.058913in}{1.719993in}}%
\pgfpathcurveto{\pgfqpoint{4.058913in}{1.712860in}}{\pgfqpoint{4.061747in}{1.706019in}}{\pgfqpoint{4.066791in}{1.700975in}}%
\pgfpathcurveto{\pgfqpoint{4.071834in}{1.695931in}}{\pgfqpoint{4.078676in}{1.693097in}}{\pgfqpoint{4.085809in}{1.693097in}}%
\pgfpathclose%
\pgfusepath{stroke,fill}%
\end{pgfscope}%
\begin{pgfscope}%
\pgfpathrectangle{\pgfqpoint{2.867647in}{0.500000in}}{\pgfqpoint{1.764706in}{1.700000in}}%
\pgfusepath{clip}%
\pgfsetbuttcap%
\pgfsetroundjoin%
\definecolor{currentfill}{rgb}{0.977657,0.891500,0.822809}%
\pgfsetfillcolor{currentfill}%
\pgfsetlinewidth{0.311001pt}%
\definecolor{currentstroke}{rgb}{1.000000,1.000000,1.000000}%
\pgfsetstrokecolor{currentstroke}%
\pgfsetdash{}{0pt}%
\pgfpathmoveto{\pgfqpoint{4.148922in}{1.351852in}}%
\pgfpathcurveto{\pgfqpoint{4.156055in}{1.351852in}}{\pgfqpoint{4.162897in}{1.354685in}}{\pgfqpoint{4.167940in}{1.359729in}}%
\pgfpathcurveto{\pgfqpoint{4.172984in}{1.364773in}}{\pgfqpoint{4.175818in}{1.371614in}}{\pgfqpoint{4.175818in}{1.378747in}}%
\pgfpathcurveto{\pgfqpoint{4.175818in}{1.385880in}}{\pgfqpoint{4.172984in}{1.392722in}}{\pgfqpoint{4.167940in}{1.397765in}}%
\pgfpathcurveto{\pgfqpoint{4.162897in}{1.402809in}}{\pgfqpoint{4.156055in}{1.405643in}}{\pgfqpoint{4.148922in}{1.405643in}}%
\pgfpathcurveto{\pgfqpoint{4.141789in}{1.405643in}}{\pgfqpoint{4.134948in}{1.402809in}}{\pgfqpoint{4.129904in}{1.397765in}}%
\pgfpathcurveto{\pgfqpoint{4.124860in}{1.392722in}}{\pgfqpoint{4.122027in}{1.385880in}}{\pgfqpoint{4.122027in}{1.378747in}}%
\pgfpathcurveto{\pgfqpoint{4.122027in}{1.371614in}}{\pgfqpoint{4.124860in}{1.364773in}}{\pgfqpoint{4.129904in}{1.359729in}}%
\pgfpathcurveto{\pgfqpoint{4.134948in}{1.354685in}}{\pgfqpoint{4.141789in}{1.351852in}}{\pgfqpoint{4.148922in}{1.351852in}}%
\pgfpathclose%
\pgfusepath{stroke,fill}%
\end{pgfscope}%
\begin{pgfscope}%
\pgfpathrectangle{\pgfqpoint{2.867647in}{0.500000in}}{\pgfqpoint{1.764706in}{1.700000in}}%
\pgfusepath{clip}%
\pgfsetbuttcap%
\pgfsetroundjoin%
\definecolor{currentfill}{rgb}{0.961734,0.579886,0.418445}%
\pgfsetfillcolor{currentfill}%
\pgfsetlinewidth{0.311001pt}%
\definecolor{currentstroke}{rgb}{1.000000,1.000000,1.000000}%
\pgfsetstrokecolor{currentstroke}%
\pgfsetdash{}{0pt}%
\pgfpathmoveto{\pgfqpoint{3.946470in}{0.982317in}}%
\pgfpathcurveto{\pgfqpoint{3.953602in}{0.982317in}}{\pgfqpoint{3.960444in}{0.985151in}}{\pgfqpoint{3.965488in}{0.990194in}}%
\pgfpathcurveto{\pgfqpoint{3.970531in}{0.995238in}}{\pgfqpoint{3.973365in}{1.002080in}}{\pgfqpoint{3.973365in}{1.009212in}}%
\pgfpathcurveto{\pgfqpoint{3.973365in}{1.016345in}}{\pgfqpoint{3.970531in}{1.023187in}}{\pgfqpoint{3.965488in}{1.028230in}}%
\pgfpathcurveto{\pgfqpoint{3.960444in}{1.033274in}}{\pgfqpoint{3.953602in}{1.036108in}}{\pgfqpoint{3.946470in}{1.036108in}}%
\pgfpathcurveto{\pgfqpoint{3.939337in}{1.036108in}}{\pgfqpoint{3.932495in}{1.033274in}}{\pgfqpoint{3.927451in}{1.028230in}}%
\pgfpathcurveto{\pgfqpoint{3.922408in}{1.023187in}}{\pgfqpoint{3.919574in}{1.016345in}}{\pgfqpoint{3.919574in}{1.009212in}}%
\pgfpathcurveto{\pgfqpoint{3.919574in}{1.002080in}}{\pgfqpoint{3.922408in}{0.995238in}}{\pgfqpoint{3.927451in}{0.990194in}}%
\pgfpathcurveto{\pgfqpoint{3.932495in}{0.985151in}}{\pgfqpoint{3.939337in}{0.982317in}}{\pgfqpoint{3.946470in}{0.982317in}}%
\pgfpathclose%
\pgfusepath{stroke,fill}%
\end{pgfscope}%
\begin{pgfscope}%
\pgfpathrectangle{\pgfqpoint{2.867647in}{0.500000in}}{\pgfqpoint{1.764706in}{1.700000in}}%
\pgfusepath{clip}%
\pgfsetbuttcap%
\pgfsetroundjoin%
\definecolor{currentfill}{rgb}{0.964032,0.651225,0.493258}%
\pgfsetfillcolor{currentfill}%
\pgfsetlinewidth{0.311001pt}%
\definecolor{currentstroke}{rgb}{1.000000,1.000000,1.000000}%
\pgfsetstrokecolor{currentstroke}%
\pgfsetdash{}{0pt}%
\pgfpathmoveto{\pgfqpoint{3.984839in}{1.034979in}}%
\pgfpathcurveto{\pgfqpoint{3.991972in}{1.034979in}}{\pgfqpoint{3.998813in}{1.037813in}}{\pgfqpoint{4.003857in}{1.042856in}}%
\pgfpathcurveto{\pgfqpoint{4.008901in}{1.047900in}}{\pgfqpoint{4.011735in}{1.054742in}}{\pgfqpoint{4.011735in}{1.061874in}}%
\pgfpathcurveto{\pgfqpoint{4.011735in}{1.069007in}}{\pgfqpoint{4.008901in}{1.075849in}}{\pgfqpoint{4.003857in}{1.080892in}}%
\pgfpathcurveto{\pgfqpoint{3.998813in}{1.085936in}}{\pgfqpoint{3.991972in}{1.088770in}}{\pgfqpoint{3.984839in}{1.088770in}}%
\pgfpathcurveto{\pgfqpoint{3.977706in}{1.088770in}}{\pgfqpoint{3.970864in}{1.085936in}}{\pgfqpoint{3.965821in}{1.080892in}}%
\pgfpathcurveto{\pgfqpoint{3.960777in}{1.075849in}}{\pgfqpoint{3.957943in}{1.069007in}}{\pgfqpoint{3.957943in}{1.061874in}}%
\pgfpathcurveto{\pgfqpoint{3.957943in}{1.054742in}}{\pgfqpoint{3.960777in}{1.047900in}}{\pgfqpoint{3.965821in}{1.042856in}}%
\pgfpathcurveto{\pgfqpoint{3.970864in}{1.037813in}}{\pgfqpoint{3.977706in}{1.034979in}}{\pgfqpoint{3.984839in}{1.034979in}}%
\pgfpathclose%
\pgfusepath{stroke,fill}%
\end{pgfscope}%
\begin{pgfscope}%
\pgfpathrectangle{\pgfqpoint{2.867647in}{0.500000in}}{\pgfqpoint{1.764706in}{1.700000in}}%
\pgfusepath{clip}%
\pgfsetbuttcap%
\pgfsetroundjoin%
\definecolor{currentfill}{rgb}{0.964306,0.663930,0.507747}%
\pgfsetfillcolor{currentfill}%
\pgfsetlinewidth{0.311001pt}%
\definecolor{currentstroke}{rgb}{1.000000,1.000000,1.000000}%
\pgfsetstrokecolor{currentstroke}%
\pgfsetdash{}{0pt}%
\pgfpathmoveto{\pgfqpoint{4.053651in}{1.224396in}}%
\pgfpathcurveto{\pgfqpoint{4.060784in}{1.224396in}}{\pgfqpoint{4.067626in}{1.227230in}}{\pgfqpoint{4.072670in}{1.232274in}}%
\pgfpathcurveto{\pgfqpoint{4.077713in}{1.237318in}}{\pgfqpoint{4.080547in}{1.244159in}}{\pgfqpoint{4.080547in}{1.251292in}}%
\pgfpathcurveto{\pgfqpoint{4.080547in}{1.258425in}}{\pgfqpoint{4.077713in}{1.265267in}}{\pgfqpoint{4.072670in}{1.270310in}}%
\pgfpathcurveto{\pgfqpoint{4.067626in}{1.275354in}}{\pgfqpoint{4.060784in}{1.278188in}}{\pgfqpoint{4.053651in}{1.278188in}}%
\pgfpathcurveto{\pgfqpoint{4.046519in}{1.278188in}}{\pgfqpoint{4.039677in}{1.275354in}}{\pgfqpoint{4.034633in}{1.270310in}}%
\pgfpathcurveto{\pgfqpoint{4.029590in}{1.265267in}}{\pgfqpoint{4.026756in}{1.258425in}}{\pgfqpoint{4.026756in}{1.251292in}}%
\pgfpathcurveto{\pgfqpoint{4.026756in}{1.244159in}}{\pgfqpoint{4.029590in}{1.237318in}}{\pgfqpoint{4.034633in}{1.232274in}}%
\pgfpathcurveto{\pgfqpoint{4.039677in}{1.227230in}}{\pgfqpoint{4.046519in}{1.224396in}}{\pgfqpoint{4.053651in}{1.224396in}}%
\pgfpathclose%
\pgfusepath{stroke,fill}%
\end{pgfscope}%
\begin{pgfscope}%
\pgfpathrectangle{\pgfqpoint{2.867647in}{0.500000in}}{\pgfqpoint{1.764706in}{1.700000in}}%
\pgfusepath{clip}%
\pgfsetbuttcap%
\pgfsetroundjoin%
\definecolor{currentfill}{rgb}{0.968509,0.792226,0.676405}%
\pgfsetfillcolor{currentfill}%
\pgfsetlinewidth{0.311001pt}%
\definecolor{currentstroke}{rgb}{1.000000,1.000000,1.000000}%
\pgfsetstrokecolor{currentstroke}%
\pgfsetdash{}{0pt}%
\pgfpathmoveto{\pgfqpoint{4.050791in}{1.089248in}}%
\pgfpathcurveto{\pgfqpoint{4.057924in}{1.089248in}}{\pgfqpoint{4.064765in}{1.092081in}}{\pgfqpoint{4.069809in}{1.097125in}}%
\pgfpathcurveto{\pgfqpoint{4.074853in}{1.102169in}}{\pgfqpoint{4.077686in}{1.109010in}}{\pgfqpoint{4.077686in}{1.116143in}}%
\pgfpathcurveto{\pgfqpoint{4.077686in}{1.123276in}}{\pgfqpoint{4.074853in}{1.130118in}}{\pgfqpoint{4.069809in}{1.135161in}}%
\pgfpathcurveto{\pgfqpoint{4.064765in}{1.140205in}}{\pgfqpoint{4.057924in}{1.143039in}}{\pgfqpoint{4.050791in}{1.143039in}}%
\pgfpathcurveto{\pgfqpoint{4.043658in}{1.143039in}}{\pgfqpoint{4.036816in}{1.140205in}}{\pgfqpoint{4.031773in}{1.135161in}}%
\pgfpathcurveto{\pgfqpoint{4.026729in}{1.130118in}}{\pgfqpoint{4.023895in}{1.123276in}}{\pgfqpoint{4.023895in}{1.116143in}}%
\pgfpathcurveto{\pgfqpoint{4.023895in}{1.109010in}}{\pgfqpoint{4.026729in}{1.102169in}}{\pgfqpoint{4.031773in}{1.097125in}}%
\pgfpathcurveto{\pgfqpoint{4.036816in}{1.092081in}}{\pgfqpoint{4.043658in}{1.089248in}}{\pgfqpoint{4.050791in}{1.089248in}}%
\pgfpathclose%
\pgfusepath{stroke,fill}%
\end{pgfscope}%
\begin{pgfscope}%
\pgfpathrectangle{\pgfqpoint{2.867647in}{0.500000in}}{\pgfqpoint{1.764706in}{1.700000in}}%
\pgfusepath{clip}%
\pgfsetbuttcap%
\pgfsetroundjoin%
\definecolor{currentfill}{rgb}{0.978376,0.897317,0.831308}%
\pgfsetfillcolor{currentfill}%
\pgfsetlinewidth{0.311001pt}%
\definecolor{currentstroke}{rgb}{1.000000,1.000000,1.000000}%
\pgfsetstrokecolor{currentstroke}%
\pgfsetdash{}{0pt}%
\pgfpathmoveto{\pgfqpoint{4.199832in}{1.498406in}}%
\pgfpathcurveto{\pgfqpoint{4.206965in}{1.498406in}}{\pgfqpoint{4.213806in}{1.501240in}}{\pgfqpoint{4.218850in}{1.506283in}}%
\pgfpathcurveto{\pgfqpoint{4.223894in}{1.511327in}}{\pgfqpoint{4.226727in}{1.518169in}}{\pgfqpoint{4.226727in}{1.525301in}}%
\pgfpathcurveto{\pgfqpoint{4.226727in}{1.532434in}}{\pgfqpoint{4.223894in}{1.539276in}}{\pgfqpoint{4.218850in}{1.544320in}}%
\pgfpathcurveto{\pgfqpoint{4.213806in}{1.549363in}}{\pgfqpoint{4.206965in}{1.552197in}}{\pgfqpoint{4.199832in}{1.552197in}}%
\pgfpathcurveto{\pgfqpoint{4.192699in}{1.552197in}}{\pgfqpoint{4.185857in}{1.549363in}}{\pgfqpoint{4.180814in}{1.544320in}}%
\pgfpathcurveto{\pgfqpoint{4.175770in}{1.539276in}}{\pgfqpoint{4.172936in}{1.532434in}}{\pgfqpoint{4.172936in}{1.525301in}}%
\pgfpathcurveto{\pgfqpoint{4.172936in}{1.518169in}}{\pgfqpoint{4.175770in}{1.511327in}}{\pgfqpoint{4.180814in}{1.506283in}}%
\pgfpathcurveto{\pgfqpoint{4.185857in}{1.501240in}}{\pgfqpoint{4.192699in}{1.498406in}}{\pgfqpoint{4.199832in}{1.498406in}}%
\pgfpathclose%
\pgfusepath{stroke,fill}%
\end{pgfscope}%
\begin{pgfscope}%
\pgfpathrectangle{\pgfqpoint{2.867647in}{0.500000in}}{\pgfqpoint{1.764706in}{1.700000in}}%
\pgfusepath{clip}%
\pgfsetbuttcap%
\pgfsetroundjoin%
\definecolor{currentfill}{rgb}{0.976287,0.879862,0.805788}%
\pgfsetfillcolor{currentfill}%
\pgfsetlinewidth{0.311001pt}%
\definecolor{currentstroke}{rgb}{1.000000,1.000000,1.000000}%
\pgfsetstrokecolor{currentstroke}%
\pgfsetdash{}{0pt}%
\pgfpathmoveto{\pgfqpoint{4.244560in}{1.312865in}}%
\pgfpathcurveto{\pgfqpoint{4.251692in}{1.312865in}}{\pgfqpoint{4.258534in}{1.315699in}}{\pgfqpoint{4.263578in}{1.320742in}}%
\pgfpathcurveto{\pgfqpoint{4.268621in}{1.325786in}}{\pgfqpoint{4.271455in}{1.332628in}}{\pgfqpoint{4.271455in}{1.339760in}}%
\pgfpathcurveto{\pgfqpoint{4.271455in}{1.346893in}}{\pgfqpoint{4.268621in}{1.353735in}}{\pgfqpoint{4.263578in}{1.358778in}}%
\pgfpathcurveto{\pgfqpoint{4.258534in}{1.363822in}}{\pgfqpoint{4.251692in}{1.366656in}}{\pgfqpoint{4.244560in}{1.366656in}}%
\pgfpathcurveto{\pgfqpoint{4.237427in}{1.366656in}}{\pgfqpoint{4.230585in}{1.363822in}}{\pgfqpoint{4.225541in}{1.358778in}}%
\pgfpathcurveto{\pgfqpoint{4.220498in}{1.353735in}}{\pgfqpoint{4.217664in}{1.346893in}}{\pgfqpoint{4.217664in}{1.339760in}}%
\pgfpathcurveto{\pgfqpoint{4.217664in}{1.332628in}}{\pgfqpoint{4.220498in}{1.325786in}}{\pgfqpoint{4.225541in}{1.320742in}}%
\pgfpathcurveto{\pgfqpoint{4.230585in}{1.315699in}}{\pgfqpoint{4.237427in}{1.312865in}}{\pgfqpoint{4.244560in}{1.312865in}}%
\pgfpathclose%
\pgfusepath{stroke,fill}%
\end{pgfscope}%
\begin{pgfscope}%
\pgfpathrectangle{\pgfqpoint{2.867647in}{0.500000in}}{\pgfqpoint{1.764706in}{1.700000in}}%
\pgfusepath{clip}%
\pgfsetbuttcap%
\pgfsetroundjoin%
\definecolor{currentfill}{rgb}{0.968105,0.786346,0.667739}%
\pgfsetfillcolor{currentfill}%
\pgfsetlinewidth{0.311001pt}%
\definecolor{currentstroke}{rgb}{1.000000,1.000000,1.000000}%
\pgfsetstrokecolor{currentstroke}%
\pgfsetdash{}{0pt}%
\pgfpathmoveto{\pgfqpoint{4.284889in}{1.332265in}}%
\pgfpathcurveto{\pgfqpoint{4.292022in}{1.332265in}}{\pgfqpoint{4.298864in}{1.335099in}}{\pgfqpoint{4.303907in}{1.340143in}}%
\pgfpathcurveto{\pgfqpoint{4.308951in}{1.345186in}}{\pgfqpoint{4.311785in}{1.352028in}}{\pgfqpoint{4.311785in}{1.359161in}}%
\pgfpathcurveto{\pgfqpoint{4.311785in}{1.366294in}}{\pgfqpoint{4.308951in}{1.373135in}}{\pgfqpoint{4.303907in}{1.378179in}}%
\pgfpathcurveto{\pgfqpoint{4.298864in}{1.383223in}}{\pgfqpoint{4.292022in}{1.386057in}}{\pgfqpoint{4.284889in}{1.386057in}}%
\pgfpathcurveto{\pgfqpoint{4.277756in}{1.386057in}}{\pgfqpoint{4.270915in}{1.383223in}}{\pgfqpoint{4.265871in}{1.378179in}}%
\pgfpathcurveto{\pgfqpoint{4.260827in}{1.373135in}}{\pgfqpoint{4.257994in}{1.366294in}}{\pgfqpoint{4.257994in}{1.359161in}}%
\pgfpathcurveto{\pgfqpoint{4.257994in}{1.352028in}}{\pgfqpoint{4.260827in}{1.345186in}}{\pgfqpoint{4.265871in}{1.340143in}}%
\pgfpathcurveto{\pgfqpoint{4.270915in}{1.335099in}}{\pgfqpoint{4.277756in}{1.332265in}}{\pgfqpoint{4.284889in}{1.332265in}}%
\pgfpathclose%
\pgfusepath{stroke,fill}%
\end{pgfscope}%
\begin{pgfscope}%
\pgfpathrectangle{\pgfqpoint{2.867647in}{0.500000in}}{\pgfqpoint{1.764706in}{1.700000in}}%
\pgfusepath{clip}%
\pgfsetbuttcap%
\pgfsetroundjoin%
\definecolor{currentfill}{rgb}{0.977657,0.891500,0.822809}%
\pgfsetfillcolor{currentfill}%
\pgfsetlinewidth{0.311001pt}%
\definecolor{currentstroke}{rgb}{1.000000,1.000000,1.000000}%
\pgfsetstrokecolor{currentstroke}%
\pgfsetdash{}{0pt}%
\pgfpathmoveto{\pgfqpoint{4.112889in}{1.608743in}}%
\pgfpathcurveto{\pgfqpoint{4.120021in}{1.608743in}}{\pgfqpoint{4.126863in}{1.611577in}}{\pgfqpoint{4.131907in}{1.616621in}}%
\pgfpathcurveto{\pgfqpoint{4.136950in}{1.621665in}}{\pgfqpoint{4.139784in}{1.628506in}}{\pgfqpoint{4.139784in}{1.635639in}}%
\pgfpathcurveto{\pgfqpoint{4.139784in}{1.642772in}}{\pgfqpoint{4.136950in}{1.649614in}}{\pgfqpoint{4.131907in}{1.654657in}}%
\pgfpathcurveto{\pgfqpoint{4.126863in}{1.659701in}}{\pgfqpoint{4.120021in}{1.662535in}}{\pgfqpoint{4.112889in}{1.662535in}}%
\pgfpathcurveto{\pgfqpoint{4.105756in}{1.662535in}}{\pgfqpoint{4.098914in}{1.659701in}}{\pgfqpoint{4.093870in}{1.654657in}}%
\pgfpathcurveto{\pgfqpoint{4.088827in}{1.649614in}}{\pgfqpoint{4.085993in}{1.642772in}}{\pgfqpoint{4.085993in}{1.635639in}}%
\pgfpathcurveto{\pgfqpoint{4.085993in}{1.628506in}}{\pgfqpoint{4.088827in}{1.621665in}}{\pgfqpoint{4.093870in}{1.616621in}}%
\pgfpathcurveto{\pgfqpoint{4.098914in}{1.611577in}}{\pgfqpoint{4.105756in}{1.608743in}}{\pgfqpoint{4.112889in}{1.608743in}}%
\pgfpathclose%
\pgfusepath{stroke,fill}%
\end{pgfscope}%
\begin{pgfscope}%
\pgfpathrectangle{\pgfqpoint{2.867647in}{0.500000in}}{\pgfqpoint{1.764706in}{1.700000in}}%
\pgfusepath{clip}%
\pgfsetbuttcap%
\pgfsetroundjoin%
\definecolor{currentfill}{rgb}{0.971202,0.827364,0.728520}%
\pgfsetfillcolor{currentfill}%
\pgfsetlinewidth{0.311001pt}%
\definecolor{currentstroke}{rgb}{1.000000,1.000000,1.000000}%
\pgfsetstrokecolor{currentstroke}%
\pgfsetdash{}{0pt}%
\pgfpathmoveto{\pgfqpoint{4.257422in}{1.186263in}}%
\pgfpathcurveto{\pgfqpoint{4.264554in}{1.186263in}}{\pgfqpoint{4.271396in}{1.189097in}}{\pgfqpoint{4.276440in}{1.194141in}}%
\pgfpathcurveto{\pgfqpoint{4.281483in}{1.199185in}}{\pgfqpoint{4.284317in}{1.206026in}}{\pgfqpoint{4.284317in}{1.213159in}}%
\pgfpathcurveto{\pgfqpoint{4.284317in}{1.220292in}}{\pgfqpoint{4.281483in}{1.227133in}}{\pgfqpoint{4.276440in}{1.232177in}}%
\pgfpathcurveto{\pgfqpoint{4.271396in}{1.237221in}}{\pgfqpoint{4.264554in}{1.240055in}}{\pgfqpoint{4.257422in}{1.240055in}}%
\pgfpathcurveto{\pgfqpoint{4.250289in}{1.240055in}}{\pgfqpoint{4.243447in}{1.237221in}}{\pgfqpoint{4.238404in}{1.232177in}}%
\pgfpathcurveto{\pgfqpoint{4.233360in}{1.227133in}}{\pgfqpoint{4.230526in}{1.220292in}}{\pgfqpoint{4.230526in}{1.213159in}}%
\pgfpathcurveto{\pgfqpoint{4.230526in}{1.206026in}}{\pgfqpoint{4.233360in}{1.199185in}}{\pgfqpoint{4.238404in}{1.194141in}}%
\pgfpathcurveto{\pgfqpoint{4.243447in}{1.189097in}}{\pgfqpoint{4.250289in}{1.186263in}}{\pgfqpoint{4.257422in}{1.186263in}}%
\pgfpathclose%
\pgfusepath{stroke,fill}%
\end{pgfscope}%
\begin{pgfscope}%
\pgfpathrectangle{\pgfqpoint{2.867647in}{0.500000in}}{\pgfqpoint{1.764706in}{1.700000in}}%
\pgfusepath{clip}%
\pgfsetbuttcap%
\pgfsetroundjoin%
\definecolor{currentfill}{rgb}{0.976287,0.879862,0.805788}%
\pgfsetfillcolor{currentfill}%
\pgfsetlinewidth{0.311001pt}%
\definecolor{currentstroke}{rgb}{1.000000,1.000000,1.000000}%
\pgfsetstrokecolor{currentstroke}%
\pgfsetdash{}{0pt}%
\pgfpathmoveto{\pgfqpoint{4.113635in}{1.036328in}}%
\pgfpathcurveto{\pgfqpoint{4.120768in}{1.036328in}}{\pgfqpoint{4.127609in}{1.039162in}}{\pgfqpoint{4.132653in}{1.044206in}}%
\pgfpathcurveto{\pgfqpoint{4.137697in}{1.049249in}}{\pgfqpoint{4.140531in}{1.056091in}}{\pgfqpoint{4.140531in}{1.063224in}}%
\pgfpathcurveto{\pgfqpoint{4.140531in}{1.070357in}}{\pgfqpoint{4.137697in}{1.077198in}}{\pgfqpoint{4.132653in}{1.082242in}}%
\pgfpathcurveto{\pgfqpoint{4.127609in}{1.087286in}}{\pgfqpoint{4.120768in}{1.090119in}}{\pgfqpoint{4.113635in}{1.090119in}}%
\pgfpathcurveto{\pgfqpoint{4.106502in}{1.090119in}}{\pgfqpoint{4.099660in}{1.087286in}}{\pgfqpoint{4.094617in}{1.082242in}}%
\pgfpathcurveto{\pgfqpoint{4.089573in}{1.077198in}}{\pgfqpoint{4.086739in}{1.070357in}}{\pgfqpoint{4.086739in}{1.063224in}}%
\pgfpathcurveto{\pgfqpoint{4.086739in}{1.056091in}}{\pgfqpoint{4.089573in}{1.049249in}}{\pgfqpoint{4.094617in}{1.044206in}}%
\pgfpathcurveto{\pgfqpoint{4.099660in}{1.039162in}}{\pgfqpoint{4.106502in}{1.036328in}}{\pgfqpoint{4.113635in}{1.036328in}}%
\pgfpathclose%
\pgfusepath{stroke,fill}%
\end{pgfscope}%
\begin{pgfscope}%
\pgfpathrectangle{\pgfqpoint{2.867647in}{0.500000in}}{\pgfqpoint{1.764706in}{1.700000in}}%
\pgfusepath{clip}%
\pgfsetbuttcap%
\pgfsetroundjoin%
\definecolor{currentfill}{rgb}{0.977657,0.891500,0.822809}%
\pgfsetfillcolor{currentfill}%
\pgfsetlinewidth{0.311001pt}%
\definecolor{currentstroke}{rgb}{1.000000,1.000000,1.000000}%
\pgfsetstrokecolor{currentstroke}%
\pgfsetdash{}{0pt}%
\pgfpathmoveto{\pgfqpoint{4.170515in}{1.077781in}}%
\pgfpathcurveto{\pgfqpoint{4.177648in}{1.077781in}}{\pgfqpoint{4.184489in}{1.080615in}}{\pgfqpoint{4.189533in}{1.085659in}}%
\pgfpathcurveto{\pgfqpoint{4.194577in}{1.090702in}}{\pgfqpoint{4.197411in}{1.097544in}}{\pgfqpoint{4.197411in}{1.104677in}}%
\pgfpathcurveto{\pgfqpoint{4.197411in}{1.111810in}}{\pgfqpoint{4.194577in}{1.118651in}}{\pgfqpoint{4.189533in}{1.123695in}}%
\pgfpathcurveto{\pgfqpoint{4.184489in}{1.128739in}}{\pgfqpoint{4.177648in}{1.131573in}}{\pgfqpoint{4.170515in}{1.131573in}}%
\pgfpathcurveto{\pgfqpoint{4.163382in}{1.131573in}}{\pgfqpoint{4.156540in}{1.128739in}}{\pgfqpoint{4.151497in}{1.123695in}}%
\pgfpathcurveto{\pgfqpoint{4.146453in}{1.118651in}}{\pgfqpoint{4.143619in}{1.111810in}}{\pgfqpoint{4.143619in}{1.104677in}}%
\pgfpathcurveto{\pgfqpoint{4.143619in}{1.097544in}}{\pgfqpoint{4.146453in}{1.090702in}}{\pgfqpoint{4.151497in}{1.085659in}}%
\pgfpathcurveto{\pgfqpoint{4.156540in}{1.080615in}}{\pgfqpoint{4.163382in}{1.077781in}}{\pgfqpoint{4.170515in}{1.077781in}}%
\pgfpathclose%
\pgfusepath{stroke,fill}%
\end{pgfscope}%
\begin{pgfscope}%
\pgfpathrectangle{\pgfqpoint{2.867647in}{0.500000in}}{\pgfqpoint{1.764706in}{1.700000in}}%
\pgfusepath{clip}%
\pgfsetbuttcap%
\pgfsetroundjoin%
\definecolor{currentfill}{rgb}{0.980678,0.914765,0.856766}%
\pgfsetfillcolor{currentfill}%
\pgfsetlinewidth{0.311001pt}%
\definecolor{currentstroke}{rgb}{1.000000,1.000000,1.000000}%
\pgfsetstrokecolor{currentstroke}%
\pgfsetdash{}{0pt}%
\pgfpathmoveto{\pgfqpoint{4.188431in}{1.405042in}}%
\pgfpathcurveto{\pgfqpoint{4.195564in}{1.405042in}}{\pgfqpoint{4.202406in}{1.407876in}}{\pgfqpoint{4.207449in}{1.412920in}}%
\pgfpathcurveto{\pgfqpoint{4.212493in}{1.417963in}}{\pgfqpoint{4.215327in}{1.424805in}}{\pgfqpoint{4.215327in}{1.431938in}}%
\pgfpathcurveto{\pgfqpoint{4.215327in}{1.439071in}}{\pgfqpoint{4.212493in}{1.445912in}}{\pgfqpoint{4.207449in}{1.450956in}}%
\pgfpathcurveto{\pgfqpoint{4.202406in}{1.456000in}}{\pgfqpoint{4.195564in}{1.458833in}}{\pgfqpoint{4.188431in}{1.458833in}}%
\pgfpathcurveto{\pgfqpoint{4.181298in}{1.458833in}}{\pgfqpoint{4.174457in}{1.456000in}}{\pgfqpoint{4.169413in}{1.450956in}}%
\pgfpathcurveto{\pgfqpoint{4.164369in}{1.445912in}}{\pgfqpoint{4.161535in}{1.439071in}}{\pgfqpoint{4.161535in}{1.431938in}}%
\pgfpathcurveto{\pgfqpoint{4.161535in}{1.424805in}}{\pgfqpoint{4.164369in}{1.417963in}}{\pgfqpoint{4.169413in}{1.412920in}}%
\pgfpathcurveto{\pgfqpoint{4.174457in}{1.407876in}}{\pgfqpoint{4.181298in}{1.405042in}}{\pgfqpoint{4.188431in}{1.405042in}}%
\pgfpathclose%
\pgfusepath{stroke,fill}%
\end{pgfscope}%
\begin{pgfscope}%
\pgfpathrectangle{\pgfqpoint{2.867647in}{0.500000in}}{\pgfqpoint{1.764706in}{1.700000in}}%
\pgfusepath{clip}%
\pgfsetbuttcap%
\pgfsetroundjoin%
\definecolor{currentfill}{rgb}{0.964173,0.657587,0.500469}%
\pgfsetfillcolor{currentfill}%
\pgfsetlinewidth{0.311001pt}%
\definecolor{currentstroke}{rgb}{1.000000,1.000000,1.000000}%
\pgfsetstrokecolor{currentstroke}%
\pgfsetdash{}{0pt}%
\pgfpathmoveto{\pgfqpoint{4.319803in}{1.332957in}}%
\pgfpathcurveto{\pgfqpoint{4.326936in}{1.332957in}}{\pgfqpoint{4.333778in}{1.335791in}}{\pgfqpoint{4.338822in}{1.340835in}}%
\pgfpathcurveto{\pgfqpoint{4.343865in}{1.345879in}}{\pgfqpoint{4.346699in}{1.352720in}}{\pgfqpoint{4.346699in}{1.359853in}}%
\pgfpathcurveto{\pgfqpoint{4.346699in}{1.366986in}}{\pgfqpoint{4.343865in}{1.373828in}}{\pgfqpoint{4.338822in}{1.378871in}}%
\pgfpathcurveto{\pgfqpoint{4.333778in}{1.383915in}}{\pgfqpoint{4.326936in}{1.386749in}}{\pgfqpoint{4.319803in}{1.386749in}}%
\pgfpathcurveto{\pgfqpoint{4.312671in}{1.386749in}}{\pgfqpoint{4.305829in}{1.383915in}}{\pgfqpoint{4.300785in}{1.378871in}}%
\pgfpathcurveto{\pgfqpoint{4.295742in}{1.373828in}}{\pgfqpoint{4.292908in}{1.366986in}}{\pgfqpoint{4.292908in}{1.359853in}}%
\pgfpathcurveto{\pgfqpoint{4.292908in}{1.352720in}}{\pgfqpoint{4.295742in}{1.345879in}}{\pgfqpoint{4.300785in}{1.340835in}}%
\pgfpathcurveto{\pgfqpoint{4.305829in}{1.335791in}}{\pgfqpoint{4.312671in}{1.332957in}}{\pgfqpoint{4.319803in}{1.332957in}}%
\pgfpathclose%
\pgfusepath{stroke,fill}%
\end{pgfscope}%
\begin{pgfscope}%
\pgfpathrectangle{\pgfqpoint{2.867647in}{0.500000in}}{\pgfqpoint{1.764706in}{1.700000in}}%
\pgfusepath{clip}%
\pgfsetbuttcap%
\pgfsetroundjoin%
\definecolor{currentfill}{rgb}{0.969803,0.809811,0.702523}%
\pgfsetfillcolor{currentfill}%
\pgfsetlinewidth{0.311001pt}%
\definecolor{currentstroke}{rgb}{1.000000,1.000000,1.000000}%
\pgfsetstrokecolor{currentstroke}%
\pgfsetdash{}{0pt}%
\pgfpathmoveto{\pgfqpoint{4.162861in}{0.985784in}}%
\pgfpathcurveto{\pgfqpoint{4.169994in}{0.985784in}}{\pgfqpoint{4.176836in}{0.988618in}}{\pgfqpoint{4.181879in}{0.993662in}}%
\pgfpathcurveto{\pgfqpoint{4.186923in}{0.998705in}}{\pgfqpoint{4.189757in}{1.005547in}}{\pgfqpoint{4.189757in}{1.012680in}}%
\pgfpathcurveto{\pgfqpoint{4.189757in}{1.019813in}}{\pgfqpoint{4.186923in}{1.026654in}}{\pgfqpoint{4.181879in}{1.031698in}}%
\pgfpathcurveto{\pgfqpoint{4.176836in}{1.036742in}}{\pgfqpoint{4.169994in}{1.039576in}}{\pgfqpoint{4.162861in}{1.039576in}}%
\pgfpathcurveto{\pgfqpoint{4.155728in}{1.039576in}}{\pgfqpoint{4.148887in}{1.036742in}}{\pgfqpoint{4.143843in}{1.031698in}}%
\pgfpathcurveto{\pgfqpoint{4.138799in}{1.026654in}}{\pgfqpoint{4.135965in}{1.019813in}}{\pgfqpoint{4.135965in}{1.012680in}}%
\pgfpathcurveto{\pgfqpoint{4.135965in}{1.005547in}}{\pgfqpoint{4.138799in}{0.998705in}}{\pgfqpoint{4.143843in}{0.993662in}}%
\pgfpathcurveto{\pgfqpoint{4.148887in}{0.988618in}}{\pgfqpoint{4.155728in}{0.985784in}}{\pgfqpoint{4.162861in}{0.985784in}}%
\pgfpathclose%
\pgfusepath{stroke,fill}%
\end{pgfscope}%
\begin{pgfscope}%
\pgfpathrectangle{\pgfqpoint{2.867647in}{0.500000in}}{\pgfqpoint{1.764706in}{1.700000in}}%
\pgfusepath{clip}%
\pgfsetbuttcap%
\pgfsetroundjoin%
\definecolor{currentfill}{rgb}{0.980678,0.914765,0.856766}%
\pgfsetfillcolor{currentfill}%
\pgfsetlinewidth{0.311001pt}%
\definecolor{currentstroke}{rgb}{1.000000,1.000000,1.000000}%
\pgfsetstrokecolor{currentstroke}%
\pgfsetdash{}{0pt}%
\pgfpathmoveto{\pgfqpoint{4.205082in}{1.359838in}}%
\pgfpathcurveto{\pgfqpoint{4.212214in}{1.359838in}}{\pgfqpoint{4.219056in}{1.362672in}}{\pgfqpoint{4.224100in}{1.367715in}}%
\pgfpathcurveto{\pgfqpoint{4.229143in}{1.372759in}}{\pgfqpoint{4.231977in}{1.379601in}}{\pgfqpoint{4.231977in}{1.386733in}}%
\pgfpathcurveto{\pgfqpoint{4.231977in}{1.393866in}}{\pgfqpoint{4.229143in}{1.400708in}}{\pgfqpoint{4.224100in}{1.405752in}}%
\pgfpathcurveto{\pgfqpoint{4.219056in}{1.410795in}}{\pgfqpoint{4.212214in}{1.413629in}}{\pgfqpoint{4.205082in}{1.413629in}}%
\pgfpathcurveto{\pgfqpoint{4.197949in}{1.413629in}}{\pgfqpoint{4.191107in}{1.410795in}}{\pgfqpoint{4.186063in}{1.405752in}}%
\pgfpathcurveto{\pgfqpoint{4.181020in}{1.400708in}}{\pgfqpoint{4.178186in}{1.393866in}}{\pgfqpoint{4.178186in}{1.386733in}}%
\pgfpathcurveto{\pgfqpoint{4.178186in}{1.379601in}}{\pgfqpoint{4.181020in}{1.372759in}}{\pgfqpoint{4.186063in}{1.367715in}}%
\pgfpathcurveto{\pgfqpoint{4.191107in}{1.362672in}}{\pgfqpoint{4.197949in}{1.359838in}}{\pgfqpoint{4.205082in}{1.359838in}}%
\pgfpathclose%
\pgfusepath{stroke,fill}%
\end{pgfscope}%
\begin{pgfscope}%
\pgfpathrectangle{\pgfqpoint{2.867647in}{0.500000in}}{\pgfqpoint{1.764706in}{1.700000in}}%
\pgfusepath{clip}%
\pgfsetbuttcap%
\pgfsetroundjoin%
\definecolor{currentfill}{rgb}{0.969359,0.803954,0.693832}%
\pgfsetfillcolor{currentfill}%
\pgfsetlinewidth{0.311001pt}%
\definecolor{currentstroke}{rgb}{1.000000,1.000000,1.000000}%
\pgfsetstrokecolor{currentstroke}%
\pgfsetdash{}{0pt}%
\pgfpathmoveto{\pgfqpoint{4.263014in}{1.171866in}}%
\pgfpathcurveto{\pgfqpoint{4.270147in}{1.171866in}}{\pgfqpoint{4.276988in}{1.174700in}}{\pgfqpoint{4.282032in}{1.179744in}}%
\pgfpathcurveto{\pgfqpoint{4.287076in}{1.184787in}}{\pgfqpoint{4.289909in}{1.191629in}}{\pgfqpoint{4.289909in}{1.198762in}}%
\pgfpathcurveto{\pgfqpoint{4.289909in}{1.205895in}}{\pgfqpoint{4.287076in}{1.212736in}}{\pgfqpoint{4.282032in}{1.217780in}}%
\pgfpathcurveto{\pgfqpoint{4.276988in}{1.222824in}}{\pgfqpoint{4.270147in}{1.225657in}}{\pgfqpoint{4.263014in}{1.225657in}}%
\pgfpathcurveto{\pgfqpoint{4.255881in}{1.225657in}}{\pgfqpoint{4.249039in}{1.222824in}}{\pgfqpoint{4.243996in}{1.217780in}}%
\pgfpathcurveto{\pgfqpoint{4.238952in}{1.212736in}}{\pgfqpoint{4.236118in}{1.205895in}}{\pgfqpoint{4.236118in}{1.198762in}}%
\pgfpathcurveto{\pgfqpoint{4.236118in}{1.191629in}}{\pgfqpoint{4.238952in}{1.184787in}}{\pgfqpoint{4.243996in}{1.179744in}}%
\pgfpathcurveto{\pgfqpoint{4.249039in}{1.174700in}}{\pgfqpoint{4.255881in}{1.171866in}}{\pgfqpoint{4.263014in}{1.171866in}}%
\pgfpathclose%
\pgfusepath{stroke,fill}%
\end{pgfscope}%
\begin{pgfscope}%
\pgfpathrectangle{\pgfqpoint{2.867647in}{0.500000in}}{\pgfqpoint{1.764706in}{1.700000in}}%
\pgfusepath{clip}%
\pgfsetbuttcap%
\pgfsetroundjoin%
\definecolor{currentfill}{rgb}{0.981377,0.920617,0.865369}%
\pgfsetfillcolor{currentfill}%
\pgfsetlinewidth{0.311001pt}%
\definecolor{currentstroke}{rgb}{1.000000,1.000000,1.000000}%
\pgfsetstrokecolor{currentstroke}%
\pgfsetdash{}{0pt}%
\pgfpathmoveto{\pgfqpoint{4.178293in}{1.204845in}}%
\pgfpathcurveto{\pgfqpoint{4.185426in}{1.204845in}}{\pgfqpoint{4.192267in}{1.207679in}}{\pgfqpoint{4.197311in}{1.212723in}}%
\pgfpathcurveto{\pgfqpoint{4.202355in}{1.217766in}}{\pgfqpoint{4.205188in}{1.224608in}}{\pgfqpoint{4.205188in}{1.231741in}}%
\pgfpathcurveto{\pgfqpoint{4.205188in}{1.238874in}}{\pgfqpoint{4.202355in}{1.245715in}}{\pgfqpoint{4.197311in}{1.250759in}}%
\pgfpathcurveto{\pgfqpoint{4.192267in}{1.255803in}}{\pgfqpoint{4.185426in}{1.258636in}}{\pgfqpoint{4.178293in}{1.258636in}}%
\pgfpathcurveto{\pgfqpoint{4.171160in}{1.258636in}}{\pgfqpoint{4.164318in}{1.255803in}}{\pgfqpoint{4.159275in}{1.250759in}}%
\pgfpathcurveto{\pgfqpoint{4.154231in}{1.245715in}}{\pgfqpoint{4.151397in}{1.238874in}}{\pgfqpoint{4.151397in}{1.231741in}}%
\pgfpathcurveto{\pgfqpoint{4.151397in}{1.224608in}}{\pgfqpoint{4.154231in}{1.217766in}}{\pgfqpoint{4.159275in}{1.212723in}}%
\pgfpathcurveto{\pgfqpoint{4.164318in}{1.207679in}}{\pgfqpoint{4.171160in}{1.204845in}}{\pgfqpoint{4.178293in}{1.204845in}}%
\pgfpathclose%
\pgfusepath{stroke,fill}%
\end{pgfscope}%
\begin{pgfscope}%
\pgfpathrectangle{\pgfqpoint{2.867647in}{0.500000in}}{\pgfqpoint{1.764706in}{1.700000in}}%
\pgfusepath{clip}%
\pgfsetbuttcap%
\pgfsetroundjoin%
\definecolor{currentfill}{rgb}{0.975644,0.874038,0.797253}%
\pgfsetfillcolor{currentfill}%
\pgfsetlinewidth{0.311001pt}%
\definecolor{currentstroke}{rgb}{1.000000,1.000000,1.000000}%
\pgfsetstrokecolor{currentstroke}%
\pgfsetdash{}{0pt}%
\pgfpathmoveto{\pgfqpoint{4.238897in}{1.208159in}}%
\pgfpathcurveto{\pgfqpoint{4.246030in}{1.208159in}}{\pgfqpoint{4.252871in}{1.210993in}}{\pgfqpoint{4.257915in}{1.216037in}}%
\pgfpathcurveto{\pgfqpoint{4.262959in}{1.221081in}}{\pgfqpoint{4.265792in}{1.227922in}}{\pgfqpoint{4.265792in}{1.235055in}}%
\pgfpathcurveto{\pgfqpoint{4.265792in}{1.242188in}}{\pgfqpoint{4.262959in}{1.249030in}}{\pgfqpoint{4.257915in}{1.254073in}}%
\pgfpathcurveto{\pgfqpoint{4.252871in}{1.259117in}}{\pgfqpoint{4.246030in}{1.261951in}}{\pgfqpoint{4.238897in}{1.261951in}}%
\pgfpathcurveto{\pgfqpoint{4.231764in}{1.261951in}}{\pgfqpoint{4.224922in}{1.259117in}}{\pgfqpoint{4.219879in}{1.254073in}}%
\pgfpathcurveto{\pgfqpoint{4.214835in}{1.249030in}}{\pgfqpoint{4.212001in}{1.242188in}}{\pgfqpoint{4.212001in}{1.235055in}}%
\pgfpathcurveto{\pgfqpoint{4.212001in}{1.227922in}}{\pgfqpoint{4.214835in}{1.221081in}}{\pgfqpoint{4.219879in}{1.216037in}}%
\pgfpathcurveto{\pgfqpoint{4.224922in}{1.210993in}}{\pgfqpoint{4.231764in}{1.208159in}}{\pgfqpoint{4.238897in}{1.208159in}}%
\pgfpathclose%
\pgfusepath{stroke,fill}%
\end{pgfscope}%
\begin{pgfscope}%
\pgfpathrectangle{\pgfqpoint{2.867647in}{0.500000in}}{\pgfqpoint{1.764706in}{1.700000in}}%
\pgfusepath{clip}%
\pgfsetbuttcap%
\pgfsetroundjoin%
\definecolor{currentfill}{rgb}{0.975644,0.874038,0.797253}%
\pgfsetfillcolor{currentfill}%
\pgfsetlinewidth{0.311001pt}%
\definecolor{currentstroke}{rgb}{1.000000,1.000000,1.000000}%
\pgfsetstrokecolor{currentstroke}%
\pgfsetdash{}{0pt}%
\pgfpathmoveto{\pgfqpoint{4.098849in}{1.029236in}}%
\pgfpathcurveto{\pgfqpoint{4.105982in}{1.029236in}}{\pgfqpoint{4.112824in}{1.032070in}}{\pgfqpoint{4.117867in}{1.037114in}}%
\pgfpathcurveto{\pgfqpoint{4.122911in}{1.042157in}}{\pgfqpoint{4.125745in}{1.048999in}}{\pgfqpoint{4.125745in}{1.056132in}}%
\pgfpathcurveto{\pgfqpoint{4.125745in}{1.063265in}}{\pgfqpoint{4.122911in}{1.070106in}}{\pgfqpoint{4.117867in}{1.075150in}}%
\pgfpathcurveto{\pgfqpoint{4.112824in}{1.080194in}}{\pgfqpoint{4.105982in}{1.083028in}}{\pgfqpoint{4.098849in}{1.083028in}}%
\pgfpathcurveto{\pgfqpoint{4.091716in}{1.083028in}}{\pgfqpoint{4.084875in}{1.080194in}}{\pgfqpoint{4.079831in}{1.075150in}}%
\pgfpathcurveto{\pgfqpoint{4.074787in}{1.070106in}}{\pgfqpoint{4.071953in}{1.063265in}}{\pgfqpoint{4.071953in}{1.056132in}}%
\pgfpathcurveto{\pgfqpoint{4.071953in}{1.048999in}}{\pgfqpoint{4.074787in}{1.042157in}}{\pgfqpoint{4.079831in}{1.037114in}}%
\pgfpathcurveto{\pgfqpoint{4.084875in}{1.032070in}}{\pgfqpoint{4.091716in}{1.029236in}}{\pgfqpoint{4.098849in}{1.029236in}}%
\pgfpathclose%
\pgfusepath{stroke,fill}%
\end{pgfscope}%
\begin{pgfscope}%
\pgfpathrectangle{\pgfqpoint{2.867647in}{0.500000in}}{\pgfqpoint{1.764706in}{1.700000in}}%
\pgfusepath{clip}%
\pgfsetbuttcap%
\pgfsetroundjoin%
\definecolor{currentfill}{rgb}{0.968105,0.786346,0.667739}%
\pgfsetfillcolor{currentfill}%
\pgfsetlinewidth{0.311001pt}%
\definecolor{currentstroke}{rgb}{1.000000,1.000000,1.000000}%
\pgfsetstrokecolor{currentstroke}%
\pgfsetdash{}{0pt}%
\pgfpathmoveto{\pgfqpoint{4.096421in}{1.352980in}}%
\pgfpathcurveto{\pgfqpoint{4.103554in}{1.352980in}}{\pgfqpoint{4.110396in}{1.355814in}}{\pgfqpoint{4.115440in}{1.360858in}}%
\pgfpathcurveto{\pgfqpoint{4.120483in}{1.365902in}}{\pgfqpoint{4.123317in}{1.372743in}}{\pgfqpoint{4.123317in}{1.379876in}}%
\pgfpathcurveto{\pgfqpoint{4.123317in}{1.387009in}}{\pgfqpoint{4.120483in}{1.393851in}}{\pgfqpoint{4.115440in}{1.398894in}}%
\pgfpathcurveto{\pgfqpoint{4.110396in}{1.403938in}}{\pgfqpoint{4.103554in}{1.406772in}}{\pgfqpoint{4.096421in}{1.406772in}}%
\pgfpathcurveto{\pgfqpoint{4.089289in}{1.406772in}}{\pgfqpoint{4.082447in}{1.403938in}}{\pgfqpoint{4.077403in}{1.398894in}}%
\pgfpathcurveto{\pgfqpoint{4.072360in}{1.393851in}}{\pgfqpoint{4.069526in}{1.387009in}}{\pgfqpoint{4.069526in}{1.379876in}}%
\pgfpathcurveto{\pgfqpoint{4.069526in}{1.372743in}}{\pgfqpoint{4.072360in}{1.365902in}}{\pgfqpoint{4.077403in}{1.360858in}}%
\pgfpathcurveto{\pgfqpoint{4.082447in}{1.355814in}}{\pgfqpoint{4.089289in}{1.352980in}}{\pgfqpoint{4.096421in}{1.352980in}}%
\pgfpathclose%
\pgfusepath{stroke,fill}%
\end{pgfscope}%
\begin{pgfscope}%
\pgfpathrectangle{\pgfqpoint{2.867647in}{0.500000in}}{\pgfqpoint{1.764706in}{1.700000in}}%
\pgfusepath{clip}%
\pgfsetbuttcap%
\pgfsetroundjoin%
\definecolor{currentfill}{rgb}{0.974412,0.862387,0.780156}%
\pgfsetfillcolor{currentfill}%
\pgfsetlinewidth{0.311001pt}%
\definecolor{currentstroke}{rgb}{1.000000,1.000000,1.000000}%
\pgfsetstrokecolor{currentstroke}%
\pgfsetdash{}{0pt}%
\pgfpathmoveto{\pgfqpoint{4.165293in}{1.613791in}}%
\pgfpathcurveto{\pgfqpoint{4.172426in}{1.613791in}}{\pgfqpoint{4.179267in}{1.616625in}}{\pgfqpoint{4.184311in}{1.621668in}}%
\pgfpathcurveto{\pgfqpoint{4.189355in}{1.626712in}}{\pgfqpoint{4.192189in}{1.633554in}}{\pgfqpoint{4.192189in}{1.640686in}}%
\pgfpathcurveto{\pgfqpoint{4.192189in}{1.647819in}}{\pgfqpoint{4.189355in}{1.654661in}}{\pgfqpoint{4.184311in}{1.659705in}}%
\pgfpathcurveto{\pgfqpoint{4.179267in}{1.664748in}}{\pgfqpoint{4.172426in}{1.667582in}}{\pgfqpoint{4.165293in}{1.667582in}}%
\pgfpathcurveto{\pgfqpoint{4.158160in}{1.667582in}}{\pgfqpoint{4.151318in}{1.664748in}}{\pgfqpoint{4.146275in}{1.659705in}}%
\pgfpathcurveto{\pgfqpoint{4.141231in}{1.654661in}}{\pgfqpoint{4.138397in}{1.647819in}}{\pgfqpoint{4.138397in}{1.640686in}}%
\pgfpathcurveto{\pgfqpoint{4.138397in}{1.633554in}}{\pgfqpoint{4.141231in}{1.626712in}}{\pgfqpoint{4.146275in}{1.621668in}}%
\pgfpathcurveto{\pgfqpoint{4.151318in}{1.616625in}}{\pgfqpoint{4.158160in}{1.613791in}}{\pgfqpoint{4.165293in}{1.613791in}}%
\pgfpathclose%
\pgfusepath{stroke,fill}%
\end{pgfscope}%
\begin{pgfscope}%
\pgfpathrectangle{\pgfqpoint{2.867647in}{0.500000in}}{\pgfqpoint{1.764706in}{1.700000in}}%
\pgfusepath{clip}%
\pgfsetbuttcap%
\pgfsetroundjoin%
\definecolor{currentfill}{rgb}{0.975644,0.874038,0.797253}%
\pgfsetfillcolor{currentfill}%
\pgfsetlinewidth{0.311001pt}%
\definecolor{currentstroke}{rgb}{1.000000,1.000000,1.000000}%
\pgfsetstrokecolor{currentstroke}%
\pgfsetdash{}{0pt}%
\pgfpathmoveto{\pgfqpoint{4.089696in}{1.626164in}}%
\pgfpathcurveto{\pgfqpoint{4.096829in}{1.626164in}}{\pgfqpoint{4.103670in}{1.628998in}}{\pgfqpoint{4.108714in}{1.634042in}}%
\pgfpathcurveto{\pgfqpoint{4.113758in}{1.639085in}}{\pgfqpoint{4.116592in}{1.645927in}}{\pgfqpoint{4.116592in}{1.653060in}}%
\pgfpathcurveto{\pgfqpoint{4.116592in}{1.660193in}}{\pgfqpoint{4.113758in}{1.667034in}}{\pgfqpoint{4.108714in}{1.672078in}}%
\pgfpathcurveto{\pgfqpoint{4.103670in}{1.677122in}}{\pgfqpoint{4.096829in}{1.679956in}}{\pgfqpoint{4.089696in}{1.679956in}}%
\pgfpathcurveto{\pgfqpoint{4.082563in}{1.679956in}}{\pgfqpoint{4.075721in}{1.677122in}}{\pgfqpoint{4.070678in}{1.672078in}}%
\pgfpathcurveto{\pgfqpoint{4.065634in}{1.667034in}}{\pgfqpoint{4.062800in}{1.660193in}}{\pgfqpoint{4.062800in}{1.653060in}}%
\pgfpathcurveto{\pgfqpoint{4.062800in}{1.645927in}}{\pgfqpoint{4.065634in}{1.639085in}}{\pgfqpoint{4.070678in}{1.634042in}}%
\pgfpathcurveto{\pgfqpoint{4.075721in}{1.628998in}}{\pgfqpoint{4.082563in}{1.626164in}}{\pgfqpoint{4.089696in}{1.626164in}}%
\pgfpathclose%
\pgfusepath{stroke,fill}%
\end{pgfscope}%
\begin{pgfscope}%
\pgfpathrectangle{\pgfqpoint{2.867647in}{0.500000in}}{\pgfqpoint{1.764706in}{1.700000in}}%
\pgfusepath{clip}%
\pgfsetbuttcap%
\pgfsetroundjoin%
\definecolor{currentfill}{rgb}{0.966812,0.762584,0.633643}%
\pgfsetfillcolor{currentfill}%
\pgfsetlinewidth{0.311001pt}%
\definecolor{currentstroke}{rgb}{1.000000,1.000000,1.000000}%
\pgfsetstrokecolor{currentstroke}%
\pgfsetdash{}{0pt}%
\pgfpathmoveto{\pgfqpoint{4.190141in}{1.655936in}}%
\pgfpathcurveto{\pgfqpoint{4.197274in}{1.655936in}}{\pgfqpoint{4.204116in}{1.658770in}}{\pgfqpoint{4.209159in}{1.663813in}}%
\pgfpathcurveto{\pgfqpoint{4.214203in}{1.668857in}}{\pgfqpoint{4.217037in}{1.675699in}}{\pgfqpoint{4.217037in}{1.682832in}}%
\pgfpathcurveto{\pgfqpoint{4.217037in}{1.689964in}}{\pgfqpoint{4.214203in}{1.696806in}}{\pgfqpoint{4.209159in}{1.701850in}}%
\pgfpathcurveto{\pgfqpoint{4.204116in}{1.706893in}}{\pgfqpoint{4.197274in}{1.709727in}}{\pgfqpoint{4.190141in}{1.709727in}}%
\pgfpathcurveto{\pgfqpoint{4.183008in}{1.709727in}}{\pgfqpoint{4.176167in}{1.706893in}}{\pgfqpoint{4.171123in}{1.701850in}}%
\pgfpathcurveto{\pgfqpoint{4.166079in}{1.696806in}}{\pgfqpoint{4.163245in}{1.689964in}}{\pgfqpoint{4.163245in}{1.682832in}}%
\pgfpathcurveto{\pgfqpoint{4.163245in}{1.675699in}}{\pgfqpoint{4.166079in}{1.668857in}}{\pgfqpoint{4.171123in}{1.663813in}}%
\pgfpathcurveto{\pgfqpoint{4.176167in}{1.658770in}}{\pgfqpoint{4.183008in}{1.655936in}}{\pgfqpoint{4.190141in}{1.655936in}}%
\pgfpathclose%
\pgfusepath{stroke,fill}%
\end{pgfscope}%
\begin{pgfscope}%
\pgfpathrectangle{\pgfqpoint{2.867647in}{0.500000in}}{\pgfqpoint{1.764706in}{1.700000in}}%
\pgfusepath{clip}%
\pgfsetbuttcap%
\pgfsetroundjoin%
\definecolor{currentfill}{rgb}{0.967735,0.780441,0.659127}%
\pgfsetfillcolor{currentfill}%
\pgfsetlinewidth{0.311001pt}%
\definecolor{currentstroke}{rgb}{1.000000,1.000000,1.000000}%
\pgfsetstrokecolor{currentstroke}%
\pgfsetdash{}{0pt}%
\pgfpathmoveto{\pgfqpoint{4.271006in}{1.462509in}}%
\pgfpathcurveto{\pgfqpoint{4.278139in}{1.462509in}}{\pgfqpoint{4.284981in}{1.465343in}}{\pgfqpoint{4.290024in}{1.470387in}}%
\pgfpathcurveto{\pgfqpoint{4.295068in}{1.475430in}}{\pgfqpoint{4.297902in}{1.482272in}}{\pgfqpoint{4.297902in}{1.489405in}}%
\pgfpathcurveto{\pgfqpoint{4.297902in}{1.496538in}}{\pgfqpoint{4.295068in}{1.503379in}}{\pgfqpoint{4.290024in}{1.508423in}}%
\pgfpathcurveto{\pgfqpoint{4.284981in}{1.513467in}}{\pgfqpoint{4.278139in}{1.516300in}}{\pgfqpoint{4.271006in}{1.516300in}}%
\pgfpathcurveto{\pgfqpoint{4.263873in}{1.516300in}}{\pgfqpoint{4.257032in}{1.513467in}}{\pgfqpoint{4.251988in}{1.508423in}}%
\pgfpathcurveto{\pgfqpoint{4.246944in}{1.503379in}}{\pgfqpoint{4.244111in}{1.496538in}}{\pgfqpoint{4.244111in}{1.489405in}}%
\pgfpathcurveto{\pgfqpoint{4.244111in}{1.482272in}}{\pgfqpoint{4.246944in}{1.475430in}}{\pgfqpoint{4.251988in}{1.470387in}}%
\pgfpathcurveto{\pgfqpoint{4.257032in}{1.465343in}}{\pgfqpoint{4.263873in}{1.462509in}}{\pgfqpoint{4.271006in}{1.462509in}}%
\pgfpathclose%
\pgfusepath{stroke,fill}%
\end{pgfscope}%
\begin{pgfscope}%
\pgfpathrectangle{\pgfqpoint{2.867647in}{0.500000in}}{\pgfqpoint{1.764706in}{1.700000in}}%
\pgfusepath{clip}%
\pgfsetbuttcap%
\pgfsetroundjoin%
\definecolor{currentfill}{rgb}{0.922239,0.282873,0.242296}%
\pgfsetfillcolor{currentfill}%
\pgfsetlinewidth{0.311001pt}%
\definecolor{currentstroke}{rgb}{1.000000,1.000000,1.000000}%
\pgfsetstrokecolor{currentstroke}%
\pgfsetdash{}{0pt}%
\pgfpathmoveto{\pgfqpoint{3.993646in}{1.369410in}}%
\pgfpathcurveto{\pgfqpoint{4.000778in}{1.369410in}}{\pgfqpoint{4.007620in}{1.372244in}}{\pgfqpoint{4.012664in}{1.377288in}}%
\pgfpathcurveto{\pgfqpoint{4.017707in}{1.382332in}}{\pgfqpoint{4.020541in}{1.389173in}}{\pgfqpoint{4.020541in}{1.396306in}}%
\pgfpathcurveto{\pgfqpoint{4.020541in}{1.403439in}}{\pgfqpoint{4.017707in}{1.410281in}}{\pgfqpoint{4.012664in}{1.415324in}}%
\pgfpathcurveto{\pgfqpoint{4.007620in}{1.420368in}}{\pgfqpoint{4.000778in}{1.423202in}}{\pgfqpoint{3.993646in}{1.423202in}}%
\pgfpathcurveto{\pgfqpoint{3.986513in}{1.423202in}}{\pgfqpoint{3.979671in}{1.420368in}}{\pgfqpoint{3.974627in}{1.415324in}}%
\pgfpathcurveto{\pgfqpoint{3.969584in}{1.410281in}}{\pgfqpoint{3.966750in}{1.403439in}}{\pgfqpoint{3.966750in}{1.396306in}}%
\pgfpathcurveto{\pgfqpoint{3.966750in}{1.389173in}}{\pgfqpoint{3.969584in}{1.382332in}}{\pgfqpoint{3.974627in}{1.377288in}}%
\pgfpathcurveto{\pgfqpoint{3.979671in}{1.372244in}}{\pgfqpoint{3.986513in}{1.369410in}}{\pgfqpoint{3.993646in}{1.369410in}}%
\pgfpathclose%
\pgfusepath{stroke,fill}%
\end{pgfscope}%
\begin{pgfscope}%
\pgfpathrectangle{\pgfqpoint{2.867647in}{0.500000in}}{\pgfqpoint{1.764706in}{1.700000in}}%
\pgfusepath{clip}%
\pgfsetbuttcap%
\pgfsetroundjoin%
\definecolor{currentfill}{rgb}{0.972726,0.844889,0.754401}%
\pgfsetfillcolor{currentfill}%
\pgfsetlinewidth{0.311001pt}%
\definecolor{currentstroke}{rgb}{1.000000,1.000000,1.000000}%
\pgfsetstrokecolor{currentstroke}%
\pgfsetdash{}{0pt}%
\pgfpathmoveto{\pgfqpoint{4.073799in}{1.072522in}}%
\pgfpathcurveto{\pgfqpoint{4.080932in}{1.072522in}}{\pgfqpoint{4.087774in}{1.075356in}}{\pgfqpoint{4.092817in}{1.080399in}}%
\pgfpathcurveto{\pgfqpoint{4.097861in}{1.085443in}}{\pgfqpoint{4.100695in}{1.092285in}}{\pgfqpoint{4.100695in}{1.099417in}}%
\pgfpathcurveto{\pgfqpoint{4.100695in}{1.106550in}}{\pgfqpoint{4.097861in}{1.113392in}}{\pgfqpoint{4.092817in}{1.118436in}}%
\pgfpathcurveto{\pgfqpoint{4.087774in}{1.123479in}}{\pgfqpoint{4.080932in}{1.126313in}}{\pgfqpoint{4.073799in}{1.126313in}}%
\pgfpathcurveto{\pgfqpoint{4.066666in}{1.126313in}}{\pgfqpoint{4.059825in}{1.123479in}}{\pgfqpoint{4.054781in}{1.118436in}}%
\pgfpathcurveto{\pgfqpoint{4.049738in}{1.113392in}}{\pgfqpoint{4.046904in}{1.106550in}}{\pgfqpoint{4.046904in}{1.099417in}}%
\pgfpathcurveto{\pgfqpoint{4.046904in}{1.092285in}}{\pgfqpoint{4.049738in}{1.085443in}}{\pgfqpoint{4.054781in}{1.080399in}}%
\pgfpathcurveto{\pgfqpoint{4.059825in}{1.075356in}}{\pgfqpoint{4.066666in}{1.072522in}}{\pgfqpoint{4.073799in}{1.072522in}}%
\pgfpathclose%
\pgfusepath{stroke,fill}%
\end{pgfscope}%
\begin{pgfscope}%
\pgfpathrectangle{\pgfqpoint{2.867647in}{0.500000in}}{\pgfqpoint{1.764706in}{1.700000in}}%
\pgfusepath{clip}%
\pgfsetbuttcap%
\pgfsetroundjoin%
\definecolor{currentfill}{rgb}{0.975644,0.874038,0.797253}%
\pgfsetfillcolor{currentfill}%
\pgfsetlinewidth{0.311001pt}%
\definecolor{currentstroke}{rgb}{1.000000,1.000000,1.000000}%
\pgfsetstrokecolor{currentstroke}%
\pgfsetdash{}{0pt}%
\pgfpathmoveto{\pgfqpoint{4.147266in}{1.027787in}}%
\pgfpathcurveto{\pgfqpoint{4.154399in}{1.027787in}}{\pgfqpoint{4.161240in}{1.030621in}}{\pgfqpoint{4.166284in}{1.035665in}}%
\pgfpathcurveto{\pgfqpoint{4.171328in}{1.040709in}}{\pgfqpoint{4.174162in}{1.047550in}}{\pgfqpoint{4.174162in}{1.054683in}}%
\pgfpathcurveto{\pgfqpoint{4.174162in}{1.061816in}}{\pgfqpoint{4.171328in}{1.068658in}}{\pgfqpoint{4.166284in}{1.073701in}}%
\pgfpathcurveto{\pgfqpoint{4.161240in}{1.078745in}}{\pgfqpoint{4.154399in}{1.081579in}}{\pgfqpoint{4.147266in}{1.081579in}}%
\pgfpathcurveto{\pgfqpoint{4.140133in}{1.081579in}}{\pgfqpoint{4.133291in}{1.078745in}}{\pgfqpoint{4.128248in}{1.073701in}}%
\pgfpathcurveto{\pgfqpoint{4.123204in}{1.068658in}}{\pgfqpoint{4.120370in}{1.061816in}}{\pgfqpoint{4.120370in}{1.054683in}}%
\pgfpathcurveto{\pgfqpoint{4.120370in}{1.047550in}}{\pgfqpoint{4.123204in}{1.040709in}}{\pgfqpoint{4.128248in}{1.035665in}}%
\pgfpathcurveto{\pgfqpoint{4.133291in}{1.030621in}}{\pgfqpoint{4.140133in}{1.027787in}}{\pgfqpoint{4.147266in}{1.027787in}}%
\pgfpathclose%
\pgfusepath{stroke,fill}%
\end{pgfscope}%
\begin{pgfscope}%
\pgfpathrectangle{\pgfqpoint{2.867647in}{0.500000in}}{\pgfqpoint{1.764706in}{1.700000in}}%
\pgfusepath{clip}%
\pgfsetbuttcap%
\pgfsetroundjoin%
\definecolor{currentfill}{rgb}{0.980678,0.914765,0.856766}%
\pgfsetfillcolor{currentfill}%
\pgfsetlinewidth{0.311001pt}%
\definecolor{currentstroke}{rgb}{1.000000,1.000000,1.000000}%
\pgfsetstrokecolor{currentstroke}%
\pgfsetdash{}{0pt}%
\pgfpathmoveto{\pgfqpoint{4.197312in}{1.373542in}}%
\pgfpathcurveto{\pgfqpoint{4.204445in}{1.373542in}}{\pgfqpoint{4.211286in}{1.376376in}}{\pgfqpoint{4.216330in}{1.381420in}}%
\pgfpathcurveto{\pgfqpoint{4.221374in}{1.386463in}}{\pgfqpoint{4.224208in}{1.393305in}}{\pgfqpoint{4.224208in}{1.400438in}}%
\pgfpathcurveto{\pgfqpoint{4.224208in}{1.407571in}}{\pgfqpoint{4.221374in}{1.414412in}}{\pgfqpoint{4.216330in}{1.419456in}}%
\pgfpathcurveto{\pgfqpoint{4.211286in}{1.424500in}}{\pgfqpoint{4.204445in}{1.427333in}}{\pgfqpoint{4.197312in}{1.427333in}}%
\pgfpathcurveto{\pgfqpoint{4.190179in}{1.427333in}}{\pgfqpoint{4.183337in}{1.424500in}}{\pgfqpoint{4.178294in}{1.419456in}}%
\pgfpathcurveto{\pgfqpoint{4.173250in}{1.414412in}}{\pgfqpoint{4.170416in}{1.407571in}}{\pgfqpoint{4.170416in}{1.400438in}}%
\pgfpathcurveto{\pgfqpoint{4.170416in}{1.393305in}}{\pgfqpoint{4.173250in}{1.386463in}}{\pgfqpoint{4.178294in}{1.381420in}}%
\pgfpathcurveto{\pgfqpoint{4.183337in}{1.376376in}}{\pgfqpoint{4.190179in}{1.373542in}}{\pgfqpoint{4.197312in}{1.373542in}}%
\pgfpathclose%
\pgfusepath{stroke,fill}%
\end{pgfscope}%
\begin{pgfscope}%
\pgfpathrectangle{\pgfqpoint{2.867647in}{0.500000in}}{\pgfqpoint{1.764706in}{1.700000in}}%
\pgfusepath{clip}%
\pgfsetbuttcap%
\pgfsetroundjoin%
\definecolor{currentfill}{rgb}{0.962532,0.599594,0.438051}%
\pgfsetfillcolor{currentfill}%
\pgfsetlinewidth{0.311001pt}%
\definecolor{currentstroke}{rgb}{1.000000,1.000000,1.000000}%
\pgfsetstrokecolor{currentstroke}%
\pgfsetdash{}{0pt}%
\pgfpathmoveto{\pgfqpoint{3.970933in}{0.876920in}}%
\pgfpathcurveto{\pgfqpoint{3.978066in}{0.876920in}}{\pgfqpoint{3.984907in}{0.879754in}}{\pgfqpoint{3.989951in}{0.884797in}}%
\pgfpathcurveto{\pgfqpoint{3.994995in}{0.889841in}}{\pgfqpoint{3.997829in}{0.896683in}}{\pgfqpoint{3.997829in}{0.903816in}}%
\pgfpathcurveto{\pgfqpoint{3.997829in}{0.910948in}}{\pgfqpoint{3.994995in}{0.917790in}}{\pgfqpoint{3.989951in}{0.922834in}}%
\pgfpathcurveto{\pgfqpoint{3.984907in}{0.927877in}}{\pgfqpoint{3.978066in}{0.930711in}}{\pgfqpoint{3.970933in}{0.930711in}}%
\pgfpathcurveto{\pgfqpoint{3.963800in}{0.930711in}}{\pgfqpoint{3.956958in}{0.927877in}}{\pgfqpoint{3.951915in}{0.922834in}}%
\pgfpathcurveto{\pgfqpoint{3.946871in}{0.917790in}}{\pgfqpoint{3.944037in}{0.910948in}}{\pgfqpoint{3.944037in}{0.903816in}}%
\pgfpathcurveto{\pgfqpoint{3.944037in}{0.896683in}}{\pgfqpoint{3.946871in}{0.889841in}}{\pgfqpoint{3.951915in}{0.884797in}}%
\pgfpathcurveto{\pgfqpoint{3.956958in}{0.879754in}}{\pgfqpoint{3.963800in}{0.876920in}}{\pgfqpoint{3.970933in}{0.876920in}}%
\pgfpathclose%
\pgfusepath{stroke,fill}%
\end{pgfscope}%
\begin{pgfscope}%
\pgfpathrectangle{\pgfqpoint{2.867647in}{0.500000in}}{\pgfqpoint{1.764706in}{1.700000in}}%
\pgfusepath{clip}%
\pgfsetbuttcap%
\pgfsetroundjoin%
\definecolor{currentfill}{rgb}{0.980678,0.914765,0.856766}%
\pgfsetfillcolor{currentfill}%
\pgfsetlinewidth{0.311001pt}%
\definecolor{currentstroke}{rgb}{1.000000,1.000000,1.000000}%
\pgfsetstrokecolor{currentstroke}%
\pgfsetdash{}{0pt}%
\pgfpathmoveto{\pgfqpoint{4.211116in}{1.311828in}}%
\pgfpathcurveto{\pgfqpoint{4.218249in}{1.311828in}}{\pgfqpoint{4.225090in}{1.314662in}}{\pgfqpoint{4.230134in}{1.319706in}}%
\pgfpathcurveto{\pgfqpoint{4.235178in}{1.324750in}}{\pgfqpoint{4.238012in}{1.331591in}}{\pgfqpoint{4.238012in}{1.338724in}}%
\pgfpathcurveto{\pgfqpoint{4.238012in}{1.345857in}}{\pgfqpoint{4.235178in}{1.352699in}}{\pgfqpoint{4.230134in}{1.357742in}}%
\pgfpathcurveto{\pgfqpoint{4.225090in}{1.362786in}}{\pgfqpoint{4.218249in}{1.365620in}}{\pgfqpoint{4.211116in}{1.365620in}}%
\pgfpathcurveto{\pgfqpoint{4.203983in}{1.365620in}}{\pgfqpoint{4.197141in}{1.362786in}}{\pgfqpoint{4.192098in}{1.357742in}}%
\pgfpathcurveto{\pgfqpoint{4.187054in}{1.352699in}}{\pgfqpoint{4.184220in}{1.345857in}}{\pgfqpoint{4.184220in}{1.338724in}}%
\pgfpathcurveto{\pgfqpoint{4.184220in}{1.331591in}}{\pgfqpoint{4.187054in}{1.324750in}}{\pgfqpoint{4.192098in}{1.319706in}}%
\pgfpathcurveto{\pgfqpoint{4.197141in}{1.314662in}}{\pgfqpoint{4.203983in}{1.311828in}}{\pgfqpoint{4.211116in}{1.311828in}}%
\pgfpathclose%
\pgfusepath{stroke,fill}%
\end{pgfscope}%
\begin{pgfscope}%
\pgfpathrectangle{\pgfqpoint{2.867647in}{0.500000in}}{\pgfqpoint{1.764706in}{1.700000in}}%
\pgfusepath{clip}%
\pgfsetbuttcap%
\pgfsetroundjoin%
\definecolor{currentfill}{rgb}{0.976961,0.885681,0.814303}%
\pgfsetfillcolor{currentfill}%
\pgfsetlinewidth{0.311001pt}%
\definecolor{currentstroke}{rgb}{1.000000,1.000000,1.000000}%
\pgfsetstrokecolor{currentstroke}%
\pgfsetdash{}{0pt}%
\pgfpathmoveto{\pgfqpoint{4.112596in}{1.114717in}}%
\pgfpathcurveto{\pgfqpoint{4.119729in}{1.114717in}}{\pgfqpoint{4.126570in}{1.117551in}}{\pgfqpoint{4.131614in}{1.122594in}}%
\pgfpathcurveto{\pgfqpoint{4.136658in}{1.127638in}}{\pgfqpoint{4.139492in}{1.134480in}}{\pgfqpoint{4.139492in}{1.141612in}}%
\pgfpathcurveto{\pgfqpoint{4.139492in}{1.148745in}}{\pgfqpoint{4.136658in}{1.155587in}}{\pgfqpoint{4.131614in}{1.160631in}}%
\pgfpathcurveto{\pgfqpoint{4.126570in}{1.165674in}}{\pgfqpoint{4.119729in}{1.168508in}}{\pgfqpoint{4.112596in}{1.168508in}}%
\pgfpathcurveto{\pgfqpoint{4.105463in}{1.168508in}}{\pgfqpoint{4.098622in}{1.165674in}}{\pgfqpoint{4.093578in}{1.160631in}}%
\pgfpathcurveto{\pgfqpoint{4.088534in}{1.155587in}}{\pgfqpoint{4.085700in}{1.148745in}}{\pgfqpoint{4.085700in}{1.141612in}}%
\pgfpathcurveto{\pgfqpoint{4.085700in}{1.134480in}}{\pgfqpoint{4.088534in}{1.127638in}}{\pgfqpoint{4.093578in}{1.122594in}}%
\pgfpathcurveto{\pgfqpoint{4.098622in}{1.117551in}}{\pgfqpoint{4.105463in}{1.114717in}}{\pgfqpoint{4.112596in}{1.114717in}}%
\pgfpathclose%
\pgfusepath{stroke,fill}%
\end{pgfscope}%
\begin{pgfscope}%
\pgfpathrectangle{\pgfqpoint{2.867647in}{0.500000in}}{\pgfqpoint{1.764706in}{1.700000in}}%
\pgfusepath{clip}%
\pgfsetbuttcap%
\pgfsetroundjoin%
\definecolor{currentfill}{rgb}{0.684863,0.090856,0.349141}%
\pgfsetfillcolor{currentfill}%
\pgfsetlinewidth{0.311001pt}%
\definecolor{currentstroke}{rgb}{1.000000,1.000000,1.000000}%
\pgfsetstrokecolor{currentstroke}%
\pgfsetdash{}{0pt}%
\pgfpathmoveto{\pgfqpoint{3.694404in}{1.833568in}}%
\pgfpathcurveto{\pgfqpoint{3.701537in}{1.833568in}}{\pgfqpoint{3.708378in}{1.836402in}}{\pgfqpoint{3.713422in}{1.841446in}}%
\pgfpathcurveto{\pgfqpoint{3.718466in}{1.846489in}}{\pgfqpoint{3.721299in}{1.853331in}}{\pgfqpoint{3.721299in}{1.860464in}}%
\pgfpathcurveto{\pgfqpoint{3.721299in}{1.867597in}}{\pgfqpoint{3.718466in}{1.874438in}}{\pgfqpoint{3.713422in}{1.879482in}}%
\pgfpathcurveto{\pgfqpoint{3.708378in}{1.884526in}}{\pgfqpoint{3.701537in}{1.887360in}}{\pgfqpoint{3.694404in}{1.887360in}}%
\pgfpathcurveto{\pgfqpoint{3.687271in}{1.887360in}}{\pgfqpoint{3.680429in}{1.884526in}}{\pgfqpoint{3.675386in}{1.879482in}}%
\pgfpathcurveto{\pgfqpoint{3.670342in}{1.874438in}}{\pgfqpoint{3.667508in}{1.867597in}}{\pgfqpoint{3.667508in}{1.860464in}}%
\pgfpathcurveto{\pgfqpoint{3.667508in}{1.853331in}}{\pgfqpoint{3.670342in}{1.846489in}}{\pgfqpoint{3.675386in}{1.841446in}}%
\pgfpathcurveto{\pgfqpoint{3.680429in}{1.836402in}}{\pgfqpoint{3.687271in}{1.833568in}}{\pgfqpoint{3.694404in}{1.833568in}}%
\pgfpathclose%
\pgfusepath{stroke,fill}%
\end{pgfscope}%
\begin{pgfscope}%
\pgfpathrectangle{\pgfqpoint{2.867647in}{0.500000in}}{\pgfqpoint{1.764706in}{1.700000in}}%
\pgfusepath{clip}%
\pgfsetbuttcap%
\pgfsetroundjoin%
\definecolor{currentfill}{rgb}{0.981377,0.920617,0.865369}%
\pgfsetfillcolor{currentfill}%
\pgfsetlinewidth{0.311001pt}%
\definecolor{currentstroke}{rgb}{1.000000,1.000000,1.000000}%
\pgfsetstrokecolor{currentstroke}%
\pgfsetdash{}{0pt}%
\pgfpathmoveto{\pgfqpoint{4.165721in}{1.218152in}}%
\pgfpathcurveto{\pgfqpoint{4.172853in}{1.218152in}}{\pgfqpoint{4.179695in}{1.220986in}}{\pgfqpoint{4.184739in}{1.226029in}}%
\pgfpathcurveto{\pgfqpoint{4.189782in}{1.231073in}}{\pgfqpoint{4.192616in}{1.237915in}}{\pgfqpoint{4.192616in}{1.245048in}}%
\pgfpathcurveto{\pgfqpoint{4.192616in}{1.252180in}}{\pgfqpoint{4.189782in}{1.259022in}}{\pgfqpoint{4.184739in}{1.264066in}}%
\pgfpathcurveto{\pgfqpoint{4.179695in}{1.269109in}}{\pgfqpoint{4.172853in}{1.271943in}}{\pgfqpoint{4.165721in}{1.271943in}}%
\pgfpathcurveto{\pgfqpoint{4.158588in}{1.271943in}}{\pgfqpoint{4.151746in}{1.269109in}}{\pgfqpoint{4.146702in}{1.264066in}}%
\pgfpathcurveto{\pgfqpoint{4.141659in}{1.259022in}}{\pgfqpoint{4.138825in}{1.252180in}}{\pgfqpoint{4.138825in}{1.245048in}}%
\pgfpathcurveto{\pgfqpoint{4.138825in}{1.237915in}}{\pgfqpoint{4.141659in}{1.231073in}}{\pgfqpoint{4.146702in}{1.226029in}}%
\pgfpathcurveto{\pgfqpoint{4.151746in}{1.220986in}}{\pgfqpoint{4.158588in}{1.218152in}}{\pgfqpoint{4.165721in}{1.218152in}}%
\pgfpathclose%
\pgfusepath{stroke,fill}%
\end{pgfscope}%
\begin{pgfscope}%
\pgfpathrectangle{\pgfqpoint{2.867647in}{0.500000in}}{\pgfqpoint{1.764706in}{1.700000in}}%
\pgfusepath{clip}%
\pgfsetbuttcap%
\pgfsetroundjoin%
\definecolor{currentfill}{rgb}{0.978376,0.897317,0.831308}%
\pgfsetfillcolor{currentfill}%
\pgfsetlinewidth{0.311001pt}%
\definecolor{currentstroke}{rgb}{1.000000,1.000000,1.000000}%
\pgfsetstrokecolor{currentstroke}%
\pgfsetdash{}{0pt}%
\pgfpathmoveto{\pgfqpoint{4.150115in}{1.388768in}}%
\pgfpathcurveto{\pgfqpoint{4.157248in}{1.388768in}}{\pgfqpoint{4.164089in}{1.391602in}}{\pgfqpoint{4.169133in}{1.396646in}}%
\pgfpathcurveto{\pgfqpoint{4.174177in}{1.401690in}}{\pgfqpoint{4.177011in}{1.408531in}}{\pgfqpoint{4.177011in}{1.415664in}}%
\pgfpathcurveto{\pgfqpoint{4.177011in}{1.422797in}}{\pgfqpoint{4.174177in}{1.429639in}}{\pgfqpoint{4.169133in}{1.434682in}}%
\pgfpathcurveto{\pgfqpoint{4.164089in}{1.439726in}}{\pgfqpoint{4.157248in}{1.442560in}}{\pgfqpoint{4.150115in}{1.442560in}}%
\pgfpathcurveto{\pgfqpoint{4.142982in}{1.442560in}}{\pgfqpoint{4.136140in}{1.439726in}}{\pgfqpoint{4.131097in}{1.434682in}}%
\pgfpathcurveto{\pgfqpoint{4.126053in}{1.429639in}}{\pgfqpoint{4.123219in}{1.422797in}}{\pgfqpoint{4.123219in}{1.415664in}}%
\pgfpathcurveto{\pgfqpoint{4.123219in}{1.408531in}}{\pgfqpoint{4.126053in}{1.401690in}}{\pgfqpoint{4.131097in}{1.396646in}}%
\pgfpathcurveto{\pgfqpoint{4.136140in}{1.391602in}}{\pgfqpoint{4.142982in}{1.388768in}}{\pgfqpoint{4.150115in}{1.388768in}}%
\pgfpathclose%
\pgfusepath{stroke,fill}%
\end{pgfscope}%
\begin{pgfscope}%
\pgfpathrectangle{\pgfqpoint{2.867647in}{0.500000in}}{\pgfqpoint{1.764706in}{1.700000in}}%
\pgfusepath{clip}%
\pgfsetbuttcap%
\pgfsetroundjoin%
\definecolor{currentfill}{rgb}{0.973271,0.850724,0.762998}%
\pgfsetfillcolor{currentfill}%
\pgfsetlinewidth{0.311001pt}%
\definecolor{currentstroke}{rgb}{1.000000,1.000000,1.000000}%
\pgfsetstrokecolor{currentstroke}%
\pgfsetdash{}{0pt}%
\pgfpathmoveto{\pgfqpoint{4.066113in}{1.584790in}}%
\pgfpathcurveto{\pgfqpoint{4.073246in}{1.584790in}}{\pgfqpoint{4.080088in}{1.587624in}}{\pgfqpoint{4.085131in}{1.592668in}}%
\pgfpathcurveto{\pgfqpoint{4.090175in}{1.597711in}}{\pgfqpoint{4.093009in}{1.604553in}}{\pgfqpoint{4.093009in}{1.611686in}}%
\pgfpathcurveto{\pgfqpoint{4.093009in}{1.618819in}}{\pgfqpoint{4.090175in}{1.625660in}}{\pgfqpoint{4.085131in}{1.630704in}}%
\pgfpathcurveto{\pgfqpoint{4.080088in}{1.635748in}}{\pgfqpoint{4.073246in}{1.638582in}}{\pgfqpoint{4.066113in}{1.638582in}}%
\pgfpathcurveto{\pgfqpoint{4.058980in}{1.638582in}}{\pgfqpoint{4.052139in}{1.635748in}}{\pgfqpoint{4.047095in}{1.630704in}}%
\pgfpathcurveto{\pgfqpoint{4.042051in}{1.625660in}}{\pgfqpoint{4.039217in}{1.618819in}}{\pgfqpoint{4.039217in}{1.611686in}}%
\pgfpathcurveto{\pgfqpoint{4.039217in}{1.604553in}}{\pgfqpoint{4.042051in}{1.597711in}}{\pgfqpoint{4.047095in}{1.592668in}}%
\pgfpathcurveto{\pgfqpoint{4.052139in}{1.587624in}}{\pgfqpoint{4.058980in}{1.584790in}}{\pgfqpoint{4.066113in}{1.584790in}}%
\pgfpathclose%
\pgfusepath{stroke,fill}%
\end{pgfscope}%
\begin{pgfscope}%
\pgfpathrectangle{\pgfqpoint{2.867647in}{0.500000in}}{\pgfqpoint{1.764706in}{1.700000in}}%
\pgfusepath{clip}%
\pgfsetbuttcap%
\pgfsetroundjoin%
\definecolor{currentfill}{rgb}{0.977657,0.891500,0.822809}%
\pgfsetfillcolor{currentfill}%
\pgfsetlinewidth{0.311001pt}%
\definecolor{currentstroke}{rgb}{1.000000,1.000000,1.000000}%
\pgfsetstrokecolor{currentstroke}%
\pgfsetdash{}{0pt}%
\pgfpathmoveto{\pgfqpoint{4.114743in}{1.604254in}}%
\pgfpathcurveto{\pgfqpoint{4.121876in}{1.604254in}}{\pgfqpoint{4.128718in}{1.607088in}}{\pgfqpoint{4.133761in}{1.612132in}}%
\pgfpathcurveto{\pgfqpoint{4.138805in}{1.617176in}}{\pgfqpoint{4.141639in}{1.624017in}}{\pgfqpoint{4.141639in}{1.631150in}}%
\pgfpathcurveto{\pgfqpoint{4.141639in}{1.638283in}}{\pgfqpoint{4.138805in}{1.645125in}}{\pgfqpoint{4.133761in}{1.650168in}}%
\pgfpathcurveto{\pgfqpoint{4.128718in}{1.655212in}}{\pgfqpoint{4.121876in}{1.658046in}}{\pgfqpoint{4.114743in}{1.658046in}}%
\pgfpathcurveto{\pgfqpoint{4.107610in}{1.658046in}}{\pgfqpoint{4.100769in}{1.655212in}}{\pgfqpoint{4.095725in}{1.650168in}}%
\pgfpathcurveto{\pgfqpoint{4.090681in}{1.645125in}}{\pgfqpoint{4.087848in}{1.638283in}}{\pgfqpoint{4.087848in}{1.631150in}}%
\pgfpathcurveto{\pgfqpoint{4.087848in}{1.624017in}}{\pgfqpoint{4.090681in}{1.617176in}}{\pgfqpoint{4.095725in}{1.612132in}}%
\pgfpathcurveto{\pgfqpoint{4.100769in}{1.607088in}}{\pgfqpoint{4.107610in}{1.604254in}}{\pgfqpoint{4.114743in}{1.604254in}}%
\pgfpathclose%
\pgfusepath{stroke,fill}%
\end{pgfscope}%
\begin{pgfscope}%
\pgfpathrectangle{\pgfqpoint{2.867647in}{0.500000in}}{\pgfqpoint{1.764706in}{1.700000in}}%
\pgfusepath{clip}%
\pgfsetbuttcap%
\pgfsetroundjoin%
\definecolor{currentfill}{rgb}{0.979891,0.908948,0.848279}%
\pgfsetfillcolor{currentfill}%
\pgfsetlinewidth{0.311001pt}%
\definecolor{currentstroke}{rgb}{1.000000,1.000000,1.000000}%
\pgfsetstrokecolor{currentstroke}%
\pgfsetdash{}{0pt}%
\pgfpathmoveto{\pgfqpoint{4.150254in}{1.198870in}}%
\pgfpathcurveto{\pgfqpoint{4.157387in}{1.198870in}}{\pgfqpoint{4.164229in}{1.201704in}}{\pgfqpoint{4.169272in}{1.206748in}}%
\pgfpathcurveto{\pgfqpoint{4.174316in}{1.211791in}}{\pgfqpoint{4.177150in}{1.218633in}}{\pgfqpoint{4.177150in}{1.225766in}}%
\pgfpathcurveto{\pgfqpoint{4.177150in}{1.232899in}}{\pgfqpoint{4.174316in}{1.239740in}}{\pgfqpoint{4.169272in}{1.244784in}}%
\pgfpathcurveto{\pgfqpoint{4.164229in}{1.249828in}}{\pgfqpoint{4.157387in}{1.252662in}}{\pgfqpoint{4.150254in}{1.252662in}}%
\pgfpathcurveto{\pgfqpoint{4.143121in}{1.252662in}}{\pgfqpoint{4.136280in}{1.249828in}}{\pgfqpoint{4.131236in}{1.244784in}}%
\pgfpathcurveto{\pgfqpoint{4.126192in}{1.239740in}}{\pgfqpoint{4.123358in}{1.232899in}}{\pgfqpoint{4.123358in}{1.225766in}}%
\pgfpathcurveto{\pgfqpoint{4.123358in}{1.218633in}}{\pgfqpoint{4.126192in}{1.211791in}}{\pgfqpoint{4.131236in}{1.206748in}}%
\pgfpathcurveto{\pgfqpoint{4.136280in}{1.201704in}}{\pgfqpoint{4.143121in}{1.198870in}}{\pgfqpoint{4.150254in}{1.198870in}}%
\pgfpathclose%
\pgfusepath{stroke,fill}%
\end{pgfscope}%
\begin{pgfscope}%
\pgfpathrectangle{\pgfqpoint{2.867647in}{0.500000in}}{\pgfqpoint{1.764706in}{1.700000in}}%
\pgfusepath{clip}%
\pgfsetbuttcap%
\pgfsetroundjoin%
\definecolor{currentfill}{rgb}{0.960421,0.553286,0.393191}%
\pgfsetfillcolor{currentfill}%
\pgfsetlinewidth{0.311001pt}%
\definecolor{currentstroke}{rgb}{1.000000,1.000000,1.000000}%
\pgfsetstrokecolor{currentstroke}%
\pgfsetdash{}{0pt}%
\pgfpathmoveto{\pgfqpoint{3.964198in}{0.857922in}}%
\pgfpathcurveto{\pgfqpoint{3.971331in}{0.857922in}}{\pgfqpoint{3.978173in}{0.860756in}}{\pgfqpoint{3.983216in}{0.865800in}}%
\pgfpathcurveto{\pgfqpoint{3.988260in}{0.870843in}}{\pgfqpoint{3.991094in}{0.877685in}}{\pgfqpoint{3.991094in}{0.884818in}}%
\pgfpathcurveto{\pgfqpoint{3.991094in}{0.891951in}}{\pgfqpoint{3.988260in}{0.898792in}}{\pgfqpoint{3.983216in}{0.903836in}}%
\pgfpathcurveto{\pgfqpoint{3.978173in}{0.908880in}}{\pgfqpoint{3.971331in}{0.911713in}}{\pgfqpoint{3.964198in}{0.911713in}}%
\pgfpathcurveto{\pgfqpoint{3.957065in}{0.911713in}}{\pgfqpoint{3.950224in}{0.908880in}}{\pgfqpoint{3.945180in}{0.903836in}}%
\pgfpathcurveto{\pgfqpoint{3.940136in}{0.898792in}}{\pgfqpoint{3.937302in}{0.891951in}}{\pgfqpoint{3.937302in}{0.884818in}}%
\pgfpathcurveto{\pgfqpoint{3.937302in}{0.877685in}}{\pgfqpoint{3.940136in}{0.870843in}}{\pgfqpoint{3.945180in}{0.865800in}}%
\pgfpathcurveto{\pgfqpoint{3.950224in}{0.860756in}}{\pgfqpoint{3.957065in}{0.857922in}}{\pgfqpoint{3.964198in}{0.857922in}}%
\pgfpathclose%
\pgfusepath{stroke,fill}%
\end{pgfscope}%
\begin{pgfscope}%
\pgfpathrectangle{\pgfqpoint{2.867647in}{0.500000in}}{\pgfqpoint{1.764706in}{1.700000in}}%
\pgfusepath{clip}%
\pgfsetbuttcap%
\pgfsetroundjoin%
\definecolor{currentfill}{rgb}{0.937528,0.344792,0.251999}%
\pgfsetfillcolor{currentfill}%
\pgfsetlinewidth{0.311001pt}%
\definecolor{currentstroke}{rgb}{1.000000,1.000000,1.000000}%
\pgfsetstrokecolor{currentstroke}%
\pgfsetdash{}{0pt}%
\pgfpathmoveto{\pgfqpoint{4.005398in}{1.305683in}}%
\pgfpathcurveto{\pgfqpoint{4.012530in}{1.305683in}}{\pgfqpoint{4.019372in}{1.308517in}}{\pgfqpoint{4.024416in}{1.313561in}}%
\pgfpathcurveto{\pgfqpoint{4.029459in}{1.318604in}}{\pgfqpoint{4.032293in}{1.325446in}}{\pgfqpoint{4.032293in}{1.332579in}}%
\pgfpathcurveto{\pgfqpoint{4.032293in}{1.339712in}}{\pgfqpoint{4.029459in}{1.346553in}}{\pgfqpoint{4.024416in}{1.351597in}}%
\pgfpathcurveto{\pgfqpoint{4.019372in}{1.356641in}}{\pgfqpoint{4.012530in}{1.359474in}}{\pgfqpoint{4.005398in}{1.359474in}}%
\pgfpathcurveto{\pgfqpoint{3.998265in}{1.359474in}}{\pgfqpoint{3.991423in}{1.356641in}}{\pgfqpoint{3.986380in}{1.351597in}}%
\pgfpathcurveto{\pgfqpoint{3.981336in}{1.346553in}}{\pgfqpoint{3.978502in}{1.339712in}}{\pgfqpoint{3.978502in}{1.332579in}}%
\pgfpathcurveto{\pgfqpoint{3.978502in}{1.325446in}}{\pgfqpoint{3.981336in}{1.318604in}}{\pgfqpoint{3.986380in}{1.313561in}}%
\pgfpathcurveto{\pgfqpoint{3.991423in}{1.308517in}}{\pgfqpoint{3.998265in}{1.305683in}}{\pgfqpoint{4.005398in}{1.305683in}}%
\pgfpathclose%
\pgfusepath{stroke,fill}%
\end{pgfscope}%
\begin{pgfscope}%
\pgfpathrectangle{\pgfqpoint{2.867647in}{0.500000in}}{\pgfqpoint{1.764706in}{1.700000in}}%
\pgfusepath{clip}%
\pgfsetbuttcap%
\pgfsetroundjoin%
\definecolor{currentfill}{rgb}{0.965440,0.720101,0.576404}%
\pgfsetfillcolor{currentfill}%
\pgfsetlinewidth{0.311001pt}%
\definecolor{currentstroke}{rgb}{1.000000,1.000000,1.000000}%
\pgfsetstrokecolor{currentstroke}%
\pgfsetdash{}{0pt}%
\pgfpathmoveto{\pgfqpoint{4.002013in}{1.018067in}}%
\pgfpathcurveto{\pgfqpoint{4.009146in}{1.018067in}}{\pgfqpoint{4.015987in}{1.020900in}}{\pgfqpoint{4.021031in}{1.025944in}}%
\pgfpathcurveto{\pgfqpoint{4.026075in}{1.030988in}}{\pgfqpoint{4.028909in}{1.037829in}}{\pgfqpoint{4.028909in}{1.044962in}}%
\pgfpathcurveto{\pgfqpoint{4.028909in}{1.052095in}}{\pgfqpoint{4.026075in}{1.058937in}}{\pgfqpoint{4.021031in}{1.063980in}}%
\pgfpathcurveto{\pgfqpoint{4.015987in}{1.069024in}}{\pgfqpoint{4.009146in}{1.071858in}}{\pgfqpoint{4.002013in}{1.071858in}}%
\pgfpathcurveto{\pgfqpoint{3.994880in}{1.071858in}}{\pgfqpoint{3.988038in}{1.069024in}}{\pgfqpoint{3.982995in}{1.063980in}}%
\pgfpathcurveto{\pgfqpoint{3.977951in}{1.058937in}}{\pgfqpoint{3.975117in}{1.052095in}}{\pgfqpoint{3.975117in}{1.044962in}}%
\pgfpathcurveto{\pgfqpoint{3.975117in}{1.037829in}}{\pgfqpoint{3.977951in}{1.030988in}}{\pgfqpoint{3.982995in}{1.025944in}}%
\pgfpathcurveto{\pgfqpoint{3.988038in}{1.020900in}}{\pgfqpoint{3.994880in}{1.018067in}}{\pgfqpoint{4.002013in}{1.018067in}}%
\pgfpathclose%
\pgfusepath{stroke,fill}%
\end{pgfscope}%
\begin{pgfscope}%
\pgfpathrectangle{\pgfqpoint{2.867647in}{0.500000in}}{\pgfqpoint{1.764706in}{1.700000in}}%
\pgfusepath{clip}%
\pgfsetbuttcap%
\pgfsetroundjoin%
\definecolor{currentfill}{rgb}{0.978376,0.897317,0.831308}%
\pgfsetfillcolor{currentfill}%
\pgfsetlinewidth{0.311001pt}%
\definecolor{currentstroke}{rgb}{1.000000,1.000000,1.000000}%
\pgfsetstrokecolor{currentstroke}%
\pgfsetdash{}{0pt}%
\pgfpathmoveto{\pgfqpoint{4.144825in}{1.236857in}}%
\pgfpathcurveto{\pgfqpoint{4.151958in}{1.236857in}}{\pgfqpoint{4.158800in}{1.239691in}}{\pgfqpoint{4.163843in}{1.244735in}}%
\pgfpathcurveto{\pgfqpoint{4.168887in}{1.249778in}}{\pgfqpoint{4.171721in}{1.256620in}}{\pgfqpoint{4.171721in}{1.263753in}}%
\pgfpathcurveto{\pgfqpoint{4.171721in}{1.270886in}}{\pgfqpoint{4.168887in}{1.277727in}}{\pgfqpoint{4.163843in}{1.282771in}}%
\pgfpathcurveto{\pgfqpoint{4.158800in}{1.287815in}}{\pgfqpoint{4.151958in}{1.290648in}}{\pgfqpoint{4.144825in}{1.290648in}}%
\pgfpathcurveto{\pgfqpoint{4.137692in}{1.290648in}}{\pgfqpoint{4.130851in}{1.287815in}}{\pgfqpoint{4.125807in}{1.282771in}}%
\pgfpathcurveto{\pgfqpoint{4.120763in}{1.277727in}}{\pgfqpoint{4.117929in}{1.270886in}}{\pgfqpoint{4.117929in}{1.263753in}}%
\pgfpathcurveto{\pgfqpoint{4.117929in}{1.256620in}}{\pgfqpoint{4.120763in}{1.249778in}}{\pgfqpoint{4.125807in}{1.244735in}}%
\pgfpathcurveto{\pgfqpoint{4.130851in}{1.239691in}}{\pgfqpoint{4.137692in}{1.236857in}}{\pgfqpoint{4.144825in}{1.236857in}}%
\pgfpathclose%
\pgfusepath{stroke,fill}%
\end{pgfscope}%
\begin{pgfscope}%
\pgfpathrectangle{\pgfqpoint{2.867647in}{0.500000in}}{\pgfqpoint{1.764706in}{1.700000in}}%
\pgfusepath{clip}%
\pgfsetbuttcap%
\pgfsetroundjoin%
\definecolor{currentfill}{rgb}{0.967092,0.768560,0.642079}%
\pgfsetfillcolor{currentfill}%
\pgfsetlinewidth{0.311001pt}%
\definecolor{currentstroke}{rgb}{1.000000,1.000000,1.000000}%
\pgfsetstrokecolor{currentstroke}%
\pgfsetdash{}{0pt}%
\pgfpathmoveto{\pgfqpoint{4.053041in}{1.122003in}}%
\pgfpathcurveto{\pgfqpoint{4.060173in}{1.122003in}}{\pgfqpoint{4.067015in}{1.124837in}}{\pgfqpoint{4.072059in}{1.129881in}}%
\pgfpathcurveto{\pgfqpoint{4.077102in}{1.134925in}}{\pgfqpoint{4.079936in}{1.141766in}}{\pgfqpoint{4.079936in}{1.148899in}}%
\pgfpathcurveto{\pgfqpoint{4.079936in}{1.156032in}}{\pgfqpoint{4.077102in}{1.162873in}}{\pgfqpoint{4.072059in}{1.167917in}}%
\pgfpathcurveto{\pgfqpoint{4.067015in}{1.172961in}}{\pgfqpoint{4.060173in}{1.175795in}}{\pgfqpoint{4.053041in}{1.175795in}}%
\pgfpathcurveto{\pgfqpoint{4.045908in}{1.175795in}}{\pgfqpoint{4.039066in}{1.172961in}}{\pgfqpoint{4.034022in}{1.167917in}}%
\pgfpathcurveto{\pgfqpoint{4.028979in}{1.162873in}}{\pgfqpoint{4.026145in}{1.156032in}}{\pgfqpoint{4.026145in}{1.148899in}}%
\pgfpathcurveto{\pgfqpoint{4.026145in}{1.141766in}}{\pgfqpoint{4.028979in}{1.134925in}}{\pgfqpoint{4.034022in}{1.129881in}}%
\pgfpathcurveto{\pgfqpoint{4.039066in}{1.124837in}}{\pgfqpoint{4.045908in}{1.122003in}}{\pgfqpoint{4.053041in}{1.122003in}}%
\pgfpathclose%
\pgfusepath{stroke,fill}%
\end{pgfscope}%
\begin{pgfscope}%
\pgfpathrectangle{\pgfqpoint{2.867647in}{0.500000in}}{\pgfqpoint{1.764706in}{1.700000in}}%
\pgfusepath{clip}%
\pgfsetbuttcap%
\pgfsetroundjoin%
\definecolor{currentfill}{rgb}{0.960778,0.559972,0.399412}%
\pgfsetfillcolor{currentfill}%
\pgfsetlinewidth{0.311001pt}%
\definecolor{currentstroke}{rgb}{1.000000,1.000000,1.000000}%
\pgfsetstrokecolor{currentstroke}%
\pgfsetdash{}{0pt}%
\pgfpathmoveto{\pgfqpoint{3.918027in}{1.756127in}}%
\pgfpathcurveto{\pgfqpoint{3.925160in}{1.756127in}}{\pgfqpoint{3.932001in}{1.758960in}}{\pgfqpoint{3.937045in}{1.764004in}}%
\pgfpathcurveto{\pgfqpoint{3.942089in}{1.769048in}}{\pgfqpoint{3.944923in}{1.775889in}}{\pgfqpoint{3.944923in}{1.783022in}}%
\pgfpathcurveto{\pgfqpoint{3.944923in}{1.790155in}}{\pgfqpoint{3.942089in}{1.796997in}}{\pgfqpoint{3.937045in}{1.802040in}}%
\pgfpathcurveto{\pgfqpoint{3.932001in}{1.807084in}}{\pgfqpoint{3.925160in}{1.809918in}}{\pgfqpoint{3.918027in}{1.809918in}}%
\pgfpathcurveto{\pgfqpoint{3.910894in}{1.809918in}}{\pgfqpoint{3.904052in}{1.807084in}}{\pgfqpoint{3.899009in}{1.802040in}}%
\pgfpathcurveto{\pgfqpoint{3.893965in}{1.796997in}}{\pgfqpoint{3.891131in}{1.790155in}}{\pgfqpoint{3.891131in}{1.783022in}}%
\pgfpathcurveto{\pgfqpoint{3.891131in}{1.775889in}}{\pgfqpoint{3.893965in}{1.769048in}}{\pgfqpoint{3.899009in}{1.764004in}}%
\pgfpathcurveto{\pgfqpoint{3.904052in}{1.758960in}}{\pgfqpoint{3.910894in}{1.756127in}}{\pgfqpoint{3.918027in}{1.756127in}}%
\pgfpathclose%
\pgfusepath{stroke,fill}%
\end{pgfscope}%
\begin{pgfscope}%
\pgfpathrectangle{\pgfqpoint{2.867647in}{0.500000in}}{\pgfqpoint{1.764706in}{1.700000in}}%
\pgfusepath{clip}%
\pgfsetbuttcap%
\pgfsetroundjoin%
\definecolor{currentfill}{rgb}{0.966812,0.762584,0.633643}%
\pgfsetfillcolor{currentfill}%
\pgfsetlinewidth{0.311001pt}%
\definecolor{currentstroke}{rgb}{1.000000,1.000000,1.000000}%
\pgfsetstrokecolor{currentstroke}%
\pgfsetdash{}{0pt}%
\pgfpathmoveto{\pgfqpoint{4.015617in}{0.975703in}}%
\pgfpathcurveto{\pgfqpoint{4.022749in}{0.975703in}}{\pgfqpoint{4.029591in}{0.978537in}}{\pgfqpoint{4.034635in}{0.983581in}}%
\pgfpathcurveto{\pgfqpoint{4.039678in}{0.988624in}}{\pgfqpoint{4.042512in}{0.995466in}}{\pgfqpoint{4.042512in}{1.002599in}}%
\pgfpathcurveto{\pgfqpoint{4.042512in}{1.009732in}}{\pgfqpoint{4.039678in}{1.016573in}}{\pgfqpoint{4.034635in}{1.021617in}}%
\pgfpathcurveto{\pgfqpoint{4.029591in}{1.026661in}}{\pgfqpoint{4.022749in}{1.029494in}}{\pgfqpoint{4.015617in}{1.029494in}}%
\pgfpathcurveto{\pgfqpoint{4.008484in}{1.029494in}}{\pgfqpoint{4.001642in}{1.026661in}}{\pgfqpoint{3.996598in}{1.021617in}}%
\pgfpathcurveto{\pgfqpoint{3.991555in}{1.016573in}}{\pgfqpoint{3.988721in}{1.009732in}}{\pgfqpoint{3.988721in}{1.002599in}}%
\pgfpathcurveto{\pgfqpoint{3.988721in}{0.995466in}}{\pgfqpoint{3.991555in}{0.988624in}}{\pgfqpoint{3.996598in}{0.983581in}}%
\pgfpathcurveto{\pgfqpoint{4.001642in}{0.978537in}}{\pgfqpoint{4.008484in}{0.975703in}}{\pgfqpoint{4.015617in}{0.975703in}}%
\pgfpathclose%
\pgfusepath{stroke,fill}%
\end{pgfscope}%
\begin{pgfscope}%
\pgfpathrectangle{\pgfqpoint{2.867647in}{0.500000in}}{\pgfqpoint{1.764706in}{1.700000in}}%
\pgfusepath{clip}%
\pgfsetbuttcap%
\pgfsetroundjoin%
\definecolor{currentfill}{rgb}{0.975018,0.868213,0.788710}%
\pgfsetfillcolor{currentfill}%
\pgfsetlinewidth{0.311001pt}%
\definecolor{currentstroke}{rgb}{1.000000,1.000000,1.000000}%
\pgfsetstrokecolor{currentstroke}%
\pgfsetdash{}{0pt}%
\pgfpathmoveto{\pgfqpoint{4.110397in}{1.175912in}}%
\pgfpathcurveto{\pgfqpoint{4.117530in}{1.175912in}}{\pgfqpoint{4.124371in}{1.178746in}}{\pgfqpoint{4.129415in}{1.183790in}}%
\pgfpathcurveto{\pgfqpoint{4.134459in}{1.188834in}}{\pgfqpoint{4.137293in}{1.195675in}}{\pgfqpoint{4.137293in}{1.202808in}}%
\pgfpathcurveto{\pgfqpoint{4.137293in}{1.209941in}}{\pgfqpoint{4.134459in}{1.216782in}}{\pgfqpoint{4.129415in}{1.221826in}}%
\pgfpathcurveto{\pgfqpoint{4.124371in}{1.226870in}}{\pgfqpoint{4.117530in}{1.229704in}}{\pgfqpoint{4.110397in}{1.229704in}}%
\pgfpathcurveto{\pgfqpoint{4.103264in}{1.229704in}}{\pgfqpoint{4.096422in}{1.226870in}}{\pgfqpoint{4.091379in}{1.221826in}}%
\pgfpathcurveto{\pgfqpoint{4.086335in}{1.216782in}}{\pgfqpoint{4.083501in}{1.209941in}}{\pgfqpoint{4.083501in}{1.202808in}}%
\pgfpathcurveto{\pgfqpoint{4.083501in}{1.195675in}}{\pgfqpoint{4.086335in}{1.188834in}}{\pgfqpoint{4.091379in}{1.183790in}}%
\pgfpathcurveto{\pgfqpoint{4.096422in}{1.178746in}}{\pgfqpoint{4.103264in}{1.175912in}}{\pgfqpoint{4.110397in}{1.175912in}}%
\pgfpathclose%
\pgfusepath{stroke,fill}%
\end{pgfscope}%
\begin{pgfscope}%
\pgfpathrectangle{\pgfqpoint{2.867647in}{0.500000in}}{\pgfqpoint{1.764706in}{1.700000in}}%
\pgfusepath{clip}%
\pgfsetbuttcap%
\pgfsetroundjoin%
\definecolor{currentfill}{rgb}{0.973832,0.856556,0.771584}%
\pgfsetfillcolor{currentfill}%
\pgfsetlinewidth{0.311001pt}%
\definecolor{currentstroke}{rgb}{1.000000,1.000000,1.000000}%
\pgfsetstrokecolor{currentstroke}%
\pgfsetdash{}{0pt}%
\pgfpathmoveto{\pgfqpoint{4.126003in}{1.341436in}}%
\pgfpathcurveto{\pgfqpoint{4.133136in}{1.341436in}}{\pgfqpoint{4.139978in}{1.344270in}}{\pgfqpoint{4.145021in}{1.349314in}}%
\pgfpathcurveto{\pgfqpoint{4.150065in}{1.354358in}}{\pgfqpoint{4.152899in}{1.361199in}}{\pgfqpoint{4.152899in}{1.368332in}}%
\pgfpathcurveto{\pgfqpoint{4.152899in}{1.375465in}}{\pgfqpoint{4.150065in}{1.382307in}}{\pgfqpoint{4.145021in}{1.387350in}}%
\pgfpathcurveto{\pgfqpoint{4.139978in}{1.392394in}}{\pgfqpoint{4.133136in}{1.395228in}}{\pgfqpoint{4.126003in}{1.395228in}}%
\pgfpathcurveto{\pgfqpoint{4.118870in}{1.395228in}}{\pgfqpoint{4.112029in}{1.392394in}}{\pgfqpoint{4.106985in}{1.387350in}}%
\pgfpathcurveto{\pgfqpoint{4.101941in}{1.382307in}}{\pgfqpoint{4.099108in}{1.375465in}}{\pgfqpoint{4.099108in}{1.368332in}}%
\pgfpathcurveto{\pgfqpoint{4.099108in}{1.361199in}}{\pgfqpoint{4.101941in}{1.354358in}}{\pgfqpoint{4.106985in}{1.349314in}}%
\pgfpathcurveto{\pgfqpoint{4.112029in}{1.344270in}}{\pgfqpoint{4.118870in}{1.341436in}}{\pgfqpoint{4.126003in}{1.341436in}}%
\pgfpathclose%
\pgfusepath{stroke,fill}%
\end{pgfscope}%
\begin{pgfscope}%
\pgfpathrectangle{\pgfqpoint{2.867647in}{0.500000in}}{\pgfqpoint{1.764706in}{1.700000in}}%
\pgfusepath{clip}%
\pgfsetbuttcap%
\pgfsetroundjoin%
\definecolor{currentfill}{rgb}{0.978376,0.897317,0.831308}%
\pgfsetfillcolor{currentfill}%
\pgfsetlinewidth{0.311001pt}%
\definecolor{currentstroke}{rgb}{1.000000,1.000000,1.000000}%
\pgfsetstrokecolor{currentstroke}%
\pgfsetdash{}{0pt}%
\pgfpathmoveto{\pgfqpoint{4.138775in}{1.459274in}}%
\pgfpathcurveto{\pgfqpoint{4.145908in}{1.459274in}}{\pgfqpoint{4.152750in}{1.462108in}}{\pgfqpoint{4.157793in}{1.467151in}}%
\pgfpathcurveto{\pgfqpoint{4.162837in}{1.472195in}}{\pgfqpoint{4.165671in}{1.479037in}}{\pgfqpoint{4.165671in}{1.486170in}}%
\pgfpathcurveto{\pgfqpoint{4.165671in}{1.493302in}}{\pgfqpoint{4.162837in}{1.500144in}}{\pgfqpoint{4.157793in}{1.505188in}}%
\pgfpathcurveto{\pgfqpoint{4.152750in}{1.510231in}}{\pgfqpoint{4.145908in}{1.513065in}}{\pgfqpoint{4.138775in}{1.513065in}}%
\pgfpathcurveto{\pgfqpoint{4.131642in}{1.513065in}}{\pgfqpoint{4.124801in}{1.510231in}}{\pgfqpoint{4.119757in}{1.505188in}}%
\pgfpathcurveto{\pgfqpoint{4.114713in}{1.500144in}}{\pgfqpoint{4.111879in}{1.493302in}}{\pgfqpoint{4.111879in}{1.486170in}}%
\pgfpathcurveto{\pgfqpoint{4.111879in}{1.479037in}}{\pgfqpoint{4.114713in}{1.472195in}}{\pgfqpoint{4.119757in}{1.467151in}}%
\pgfpathcurveto{\pgfqpoint{4.124801in}{1.462108in}}{\pgfqpoint{4.131642in}{1.459274in}}{\pgfqpoint{4.138775in}{1.459274in}}%
\pgfpathclose%
\pgfusepath{stroke,fill}%
\end{pgfscope}%
\begin{pgfscope}%
\pgfpathrectangle{\pgfqpoint{2.867647in}{0.500000in}}{\pgfqpoint{1.764706in}{1.700000in}}%
\pgfusepath{clip}%
\pgfsetbuttcap%
\pgfsetroundjoin%
\definecolor{currentfill}{rgb}{0.965042,0.701564,0.552889}%
\pgfsetfillcolor{currentfill}%
\pgfsetlinewidth{0.311001pt}%
\definecolor{currentstroke}{rgb}{1.000000,1.000000,1.000000}%
\pgfsetstrokecolor{currentstroke}%
\pgfsetdash{}{0pt}%
\pgfpathmoveto{\pgfqpoint{4.260570in}{1.576734in}}%
\pgfpathcurveto{\pgfqpoint{4.267702in}{1.576734in}}{\pgfqpoint{4.274544in}{1.579568in}}{\pgfqpoint{4.279588in}{1.584612in}}%
\pgfpathcurveto{\pgfqpoint{4.284631in}{1.589656in}}{\pgfqpoint{4.287465in}{1.596497in}}{\pgfqpoint{4.287465in}{1.603630in}}%
\pgfpathcurveto{\pgfqpoint{4.287465in}{1.610763in}}{\pgfqpoint{4.284631in}{1.617605in}}{\pgfqpoint{4.279588in}{1.622648in}}%
\pgfpathcurveto{\pgfqpoint{4.274544in}{1.627692in}}{\pgfqpoint{4.267702in}{1.630526in}}{\pgfqpoint{4.260570in}{1.630526in}}%
\pgfpathcurveto{\pgfqpoint{4.253437in}{1.630526in}}{\pgfqpoint{4.246595in}{1.627692in}}{\pgfqpoint{4.241551in}{1.622648in}}%
\pgfpathcurveto{\pgfqpoint{4.236508in}{1.617605in}}{\pgfqpoint{4.233674in}{1.610763in}}{\pgfqpoint{4.233674in}{1.603630in}}%
\pgfpathcurveto{\pgfqpoint{4.233674in}{1.596497in}}{\pgfqpoint{4.236508in}{1.589656in}}{\pgfqpoint{4.241551in}{1.584612in}}%
\pgfpathcurveto{\pgfqpoint{4.246595in}{1.579568in}}{\pgfqpoint{4.253437in}{1.576734in}}{\pgfqpoint{4.260570in}{1.576734in}}%
\pgfpathclose%
\pgfusepath{stroke,fill}%
\end{pgfscope}%
\begin{pgfscope}%
\pgfpathrectangle{\pgfqpoint{2.867647in}{0.500000in}}{\pgfqpoint{1.764706in}{1.700000in}}%
\pgfusepath{clip}%
\pgfsetbuttcap%
\pgfsetroundjoin%
\definecolor{currentfill}{rgb}{0.974412,0.862387,0.780156}%
\pgfsetfillcolor{currentfill}%
\pgfsetlinewidth{0.311001pt}%
\definecolor{currentstroke}{rgb}{1.000000,1.000000,1.000000}%
\pgfsetstrokecolor{currentstroke}%
\pgfsetdash{}{0pt}%
\pgfpathmoveto{\pgfqpoint{4.247405in}{1.381985in}}%
\pgfpathcurveto{\pgfqpoint{4.254538in}{1.381985in}}{\pgfqpoint{4.261379in}{1.384819in}}{\pgfqpoint{4.266423in}{1.389863in}}%
\pgfpathcurveto{\pgfqpoint{4.271467in}{1.394906in}}{\pgfqpoint{4.274301in}{1.401748in}}{\pgfqpoint{4.274301in}{1.408881in}}%
\pgfpathcurveto{\pgfqpoint{4.274301in}{1.416014in}}{\pgfqpoint{4.271467in}{1.422855in}}{\pgfqpoint{4.266423in}{1.427899in}}%
\pgfpathcurveto{\pgfqpoint{4.261379in}{1.432943in}}{\pgfqpoint{4.254538in}{1.435777in}}{\pgfqpoint{4.247405in}{1.435777in}}%
\pgfpathcurveto{\pgfqpoint{4.240272in}{1.435777in}}{\pgfqpoint{4.233430in}{1.432943in}}{\pgfqpoint{4.228387in}{1.427899in}}%
\pgfpathcurveto{\pgfqpoint{4.223343in}{1.422855in}}{\pgfqpoint{4.220509in}{1.416014in}}{\pgfqpoint{4.220509in}{1.408881in}}%
\pgfpathcurveto{\pgfqpoint{4.220509in}{1.401748in}}{\pgfqpoint{4.223343in}{1.394906in}}{\pgfqpoint{4.228387in}{1.389863in}}%
\pgfpathcurveto{\pgfqpoint{4.233430in}{1.384819in}}{\pgfqpoint{4.240272in}{1.381985in}}{\pgfqpoint{4.247405in}{1.381985in}}%
\pgfpathclose%
\pgfusepath{stroke,fill}%
\end{pgfscope}%
\begin{pgfscope}%
\pgfpathrectangle{\pgfqpoint{2.867647in}{0.500000in}}{\pgfqpoint{1.764706in}{1.700000in}}%
\pgfusepath{clip}%
\pgfsetbuttcap%
\pgfsetroundjoin%
\definecolor{currentfill}{rgb}{0.972201,0.839051,0.745789}%
\pgfsetfillcolor{currentfill}%
\pgfsetlinewidth{0.311001pt}%
\definecolor{currentstroke}{rgb}{1.000000,1.000000,1.000000}%
\pgfsetstrokecolor{currentstroke}%
\pgfsetdash{}{0pt}%
\pgfpathmoveto{\pgfqpoint{4.066163in}{0.992489in}}%
\pgfpathcurveto{\pgfqpoint{4.073296in}{0.992489in}}{\pgfqpoint{4.080137in}{0.995322in}}{\pgfqpoint{4.085181in}{1.000366in}}%
\pgfpathcurveto{\pgfqpoint{4.090225in}{1.005410in}}{\pgfqpoint{4.093059in}{1.012251in}}{\pgfqpoint{4.093059in}{1.019384in}}%
\pgfpathcurveto{\pgfqpoint{4.093059in}{1.026517in}}{\pgfqpoint{4.090225in}{1.033359in}}{\pgfqpoint{4.085181in}{1.038402in}}%
\pgfpathcurveto{\pgfqpoint{4.080137in}{1.043446in}}{\pgfqpoint{4.073296in}{1.046280in}}{\pgfqpoint{4.066163in}{1.046280in}}%
\pgfpathcurveto{\pgfqpoint{4.059030in}{1.046280in}}{\pgfqpoint{4.052188in}{1.043446in}}{\pgfqpoint{4.047145in}{1.038402in}}%
\pgfpathcurveto{\pgfqpoint{4.042101in}{1.033359in}}{\pgfqpoint{4.039267in}{1.026517in}}{\pgfqpoint{4.039267in}{1.019384in}}%
\pgfpathcurveto{\pgfqpoint{4.039267in}{1.012251in}}{\pgfqpoint{4.042101in}{1.005410in}}{\pgfqpoint{4.047145in}{1.000366in}}%
\pgfpathcurveto{\pgfqpoint{4.052188in}{0.995322in}}{\pgfqpoint{4.059030in}{0.992489in}}{\pgfqpoint{4.066163in}{0.992489in}}%
\pgfpathclose%
\pgfusepath{stroke,fill}%
\end{pgfscope}%
\begin{pgfscope}%
\pgfpathrectangle{\pgfqpoint{2.867647in}{0.500000in}}{\pgfqpoint{1.764706in}{1.700000in}}%
\pgfusepath{clip}%
\pgfsetbuttcap%
\pgfsetroundjoin%
\definecolor{currentfill}{rgb}{0.973832,0.856556,0.771584}%
\pgfsetfillcolor{currentfill}%
\pgfsetlinewidth{0.311001pt}%
\definecolor{currentstroke}{rgb}{1.000000,1.000000,1.000000}%
\pgfsetstrokecolor{currentstroke}%
\pgfsetdash{}{0pt}%
\pgfpathmoveto{\pgfqpoint{4.078019in}{1.552959in}}%
\pgfpathcurveto{\pgfqpoint{4.085152in}{1.552959in}}{\pgfqpoint{4.091994in}{1.555793in}}{\pgfqpoint{4.097037in}{1.560837in}}%
\pgfpathcurveto{\pgfqpoint{4.102081in}{1.565880in}}{\pgfqpoint{4.104915in}{1.572722in}}{\pgfqpoint{4.104915in}{1.579855in}}%
\pgfpathcurveto{\pgfqpoint{4.104915in}{1.586988in}}{\pgfqpoint{4.102081in}{1.593829in}}{\pgfqpoint{4.097037in}{1.598873in}}%
\pgfpathcurveto{\pgfqpoint{4.091994in}{1.603916in}}{\pgfqpoint{4.085152in}{1.606750in}}{\pgfqpoint{4.078019in}{1.606750in}}%
\pgfpathcurveto{\pgfqpoint{4.070886in}{1.606750in}}{\pgfqpoint{4.064045in}{1.603916in}}{\pgfqpoint{4.059001in}{1.598873in}}%
\pgfpathcurveto{\pgfqpoint{4.053957in}{1.593829in}}{\pgfqpoint{4.051124in}{1.586988in}}{\pgfqpoint{4.051124in}{1.579855in}}%
\pgfpathcurveto{\pgfqpoint{4.051124in}{1.572722in}}{\pgfqpoint{4.053957in}{1.565880in}}{\pgfqpoint{4.059001in}{1.560837in}}%
\pgfpathcurveto{\pgfqpoint{4.064045in}{1.555793in}}{\pgfqpoint{4.070886in}{1.552959in}}{\pgfqpoint{4.078019in}{1.552959in}}%
\pgfpathclose%
\pgfusepath{stroke,fill}%
\end{pgfscope}%
\begin{pgfscope}%
\pgfpathrectangle{\pgfqpoint{2.867647in}{0.500000in}}{\pgfqpoint{1.764706in}{1.700000in}}%
\pgfusepath{clip}%
\pgfsetbuttcap%
\pgfsetroundjoin%
\definecolor{currentfill}{rgb}{0.972726,0.844889,0.754401}%
\pgfsetfillcolor{currentfill}%
\pgfsetlinewidth{0.311001pt}%
\definecolor{currentstroke}{rgb}{1.000000,1.000000,1.000000}%
\pgfsetstrokecolor{currentstroke}%
\pgfsetdash{}{0pt}%
\pgfpathmoveto{\pgfqpoint{4.104777in}{0.982096in}}%
\pgfpathcurveto{\pgfqpoint{4.111910in}{0.982096in}}{\pgfqpoint{4.118751in}{0.984930in}}{\pgfqpoint{4.123795in}{0.989974in}}%
\pgfpathcurveto{\pgfqpoint{4.128839in}{0.995018in}}{\pgfqpoint{4.131672in}{1.001859in}}{\pgfqpoint{4.131672in}{1.008992in}}%
\pgfpathcurveto{\pgfqpoint{4.131672in}{1.016125in}}{\pgfqpoint{4.128839in}{1.022967in}}{\pgfqpoint{4.123795in}{1.028010in}}%
\pgfpathcurveto{\pgfqpoint{4.118751in}{1.033054in}}{\pgfqpoint{4.111910in}{1.035888in}}{\pgfqpoint{4.104777in}{1.035888in}}%
\pgfpathcurveto{\pgfqpoint{4.097644in}{1.035888in}}{\pgfqpoint{4.090802in}{1.033054in}}{\pgfqpoint{4.085759in}{1.028010in}}%
\pgfpathcurveto{\pgfqpoint{4.080715in}{1.022967in}}{\pgfqpoint{4.077881in}{1.016125in}}{\pgfqpoint{4.077881in}{1.008992in}}%
\pgfpathcurveto{\pgfqpoint{4.077881in}{1.001859in}}{\pgfqpoint{4.080715in}{0.995018in}}{\pgfqpoint{4.085759in}{0.989974in}}%
\pgfpathcurveto{\pgfqpoint{4.090802in}{0.984930in}}{\pgfqpoint{4.097644in}{0.982096in}}{\pgfqpoint{4.104777in}{0.982096in}}%
\pgfpathclose%
\pgfusepath{stroke,fill}%
\end{pgfscope}%
\begin{pgfscope}%
\pgfpathrectangle{\pgfqpoint{2.867647in}{0.500000in}}{\pgfqpoint{1.764706in}{1.700000in}}%
\pgfusepath{clip}%
\pgfsetbuttcap%
\pgfsetroundjoin%
\definecolor{currentfill}{rgb}{0.980678,0.914765,0.856766}%
\pgfsetfillcolor{currentfill}%
\pgfsetlinewidth{0.311001pt}%
\definecolor{currentstroke}{rgb}{1.000000,1.000000,1.000000}%
\pgfsetstrokecolor{currentstroke}%
\pgfsetdash{}{0pt}%
\pgfpathmoveto{\pgfqpoint{4.173155in}{1.368611in}}%
\pgfpathcurveto{\pgfqpoint{4.180288in}{1.368611in}}{\pgfqpoint{4.187129in}{1.371445in}}{\pgfqpoint{4.192173in}{1.376489in}}%
\pgfpathcurveto{\pgfqpoint{4.197217in}{1.381533in}}{\pgfqpoint{4.200051in}{1.388374in}}{\pgfqpoint{4.200051in}{1.395507in}}%
\pgfpathcurveto{\pgfqpoint{4.200051in}{1.402640in}}{\pgfqpoint{4.197217in}{1.409482in}}{\pgfqpoint{4.192173in}{1.414525in}}%
\pgfpathcurveto{\pgfqpoint{4.187129in}{1.419569in}}{\pgfqpoint{4.180288in}{1.422403in}}{\pgfqpoint{4.173155in}{1.422403in}}%
\pgfpathcurveto{\pgfqpoint{4.166022in}{1.422403in}}{\pgfqpoint{4.159180in}{1.419569in}}{\pgfqpoint{4.154137in}{1.414525in}}%
\pgfpathcurveto{\pgfqpoint{4.149093in}{1.409482in}}{\pgfqpoint{4.146259in}{1.402640in}}{\pgfqpoint{4.146259in}{1.395507in}}%
\pgfpathcurveto{\pgfqpoint{4.146259in}{1.388374in}}{\pgfqpoint{4.149093in}{1.381533in}}{\pgfqpoint{4.154137in}{1.376489in}}%
\pgfpathcurveto{\pgfqpoint{4.159180in}{1.371445in}}{\pgfqpoint{4.166022in}{1.368611in}}{\pgfqpoint{4.173155in}{1.368611in}}%
\pgfpathclose%
\pgfusepath{stroke,fill}%
\end{pgfscope}%
\begin{pgfscope}%
\pgfpathrectangle{\pgfqpoint{2.867647in}{0.500000in}}{\pgfqpoint{1.764706in}{1.700000in}}%
\pgfusepath{clip}%
\pgfsetbuttcap%
\pgfsetroundjoin%
\definecolor{currentfill}{rgb}{0.977657,0.891500,0.822809}%
\pgfsetfillcolor{currentfill}%
\pgfsetlinewidth{0.311001pt}%
\definecolor{currentstroke}{rgb}{1.000000,1.000000,1.000000}%
\pgfsetstrokecolor{currentstroke}%
\pgfsetdash{}{0pt}%
\pgfpathmoveto{\pgfqpoint{4.144211in}{1.267345in}}%
\pgfpathcurveto{\pgfqpoint{4.151344in}{1.267345in}}{\pgfqpoint{4.158186in}{1.270179in}}{\pgfqpoint{4.163229in}{1.275222in}}%
\pgfpathcurveto{\pgfqpoint{4.168273in}{1.280266in}}{\pgfqpoint{4.171107in}{1.287108in}}{\pgfqpoint{4.171107in}{1.294241in}}%
\pgfpathcurveto{\pgfqpoint{4.171107in}{1.301373in}}{\pgfqpoint{4.168273in}{1.308215in}}{\pgfqpoint{4.163229in}{1.313259in}}%
\pgfpathcurveto{\pgfqpoint{4.158186in}{1.318302in}}{\pgfqpoint{4.151344in}{1.321136in}}{\pgfqpoint{4.144211in}{1.321136in}}%
\pgfpathcurveto{\pgfqpoint{4.137078in}{1.321136in}}{\pgfqpoint{4.130237in}{1.318302in}}{\pgfqpoint{4.125193in}{1.313259in}}%
\pgfpathcurveto{\pgfqpoint{4.120149in}{1.308215in}}{\pgfqpoint{4.117315in}{1.301373in}}{\pgfqpoint{4.117315in}{1.294241in}}%
\pgfpathcurveto{\pgfqpoint{4.117315in}{1.287108in}}{\pgfqpoint{4.120149in}{1.280266in}}{\pgfqpoint{4.125193in}{1.275222in}}%
\pgfpathcurveto{\pgfqpoint{4.130237in}{1.270179in}}{\pgfqpoint{4.137078in}{1.267345in}}{\pgfqpoint{4.144211in}{1.267345in}}%
\pgfpathclose%
\pgfusepath{stroke,fill}%
\end{pgfscope}%
\begin{pgfscope}%
\pgfpathrectangle{\pgfqpoint{2.867647in}{0.500000in}}{\pgfqpoint{1.764706in}{1.700000in}}%
\pgfusepath{clip}%
\pgfsetbuttcap%
\pgfsetroundjoin%
\definecolor{currentfill}{rgb}{0.978376,0.897317,0.831308}%
\pgfsetfillcolor{currentfill}%
\pgfsetlinewidth{0.311001pt}%
\definecolor{currentstroke}{rgb}{1.000000,1.000000,1.000000}%
\pgfsetstrokecolor{currentstroke}%
\pgfsetdash{}{0pt}%
\pgfpathmoveto{\pgfqpoint{4.155521in}{1.357300in}}%
\pgfpathcurveto{\pgfqpoint{4.162654in}{1.357300in}}{\pgfqpoint{4.169496in}{1.360134in}}{\pgfqpoint{4.174539in}{1.365177in}}%
\pgfpathcurveto{\pgfqpoint{4.179583in}{1.370221in}}{\pgfqpoint{4.182417in}{1.377063in}}{\pgfqpoint{4.182417in}{1.384195in}}%
\pgfpathcurveto{\pgfqpoint{4.182417in}{1.391328in}}{\pgfqpoint{4.179583in}{1.398170in}}{\pgfqpoint{4.174539in}{1.403214in}}%
\pgfpathcurveto{\pgfqpoint{4.169496in}{1.408257in}}{\pgfqpoint{4.162654in}{1.411091in}}{\pgfqpoint{4.155521in}{1.411091in}}%
\pgfpathcurveto{\pgfqpoint{4.148388in}{1.411091in}}{\pgfqpoint{4.141547in}{1.408257in}}{\pgfqpoint{4.136503in}{1.403214in}}%
\pgfpathcurveto{\pgfqpoint{4.131459in}{1.398170in}}{\pgfqpoint{4.128625in}{1.391328in}}{\pgfqpoint{4.128625in}{1.384195in}}%
\pgfpathcurveto{\pgfqpoint{4.128625in}{1.377063in}}{\pgfqpoint{4.131459in}{1.370221in}}{\pgfqpoint{4.136503in}{1.365177in}}%
\pgfpathcurveto{\pgfqpoint{4.141547in}{1.360134in}}{\pgfqpoint{4.148388in}{1.357300in}}{\pgfqpoint{4.155521in}{1.357300in}}%
\pgfpathclose%
\pgfusepath{stroke,fill}%
\end{pgfscope}%
\begin{pgfscope}%
\pgfpathrectangle{\pgfqpoint{2.867647in}{0.500000in}}{\pgfqpoint{1.764706in}{1.700000in}}%
\pgfusepath{clip}%
\pgfsetbuttcap%
\pgfsetroundjoin%
\definecolor{currentfill}{rgb}{0.971202,0.827364,0.728520}%
\pgfsetfillcolor{currentfill}%
\pgfsetlinewidth{0.311001pt}%
\definecolor{currentstroke}{rgb}{1.000000,1.000000,1.000000}%
\pgfsetstrokecolor{currentstroke}%
\pgfsetdash{}{0pt}%
\pgfpathmoveto{\pgfqpoint{4.185304in}{1.623241in}}%
\pgfpathcurveto{\pgfqpoint{4.192437in}{1.623241in}}{\pgfqpoint{4.199279in}{1.626075in}}{\pgfqpoint{4.204322in}{1.631119in}}%
\pgfpathcurveto{\pgfqpoint{4.209366in}{1.636163in}}{\pgfqpoint{4.212200in}{1.643004in}}{\pgfqpoint{4.212200in}{1.650137in}}%
\pgfpathcurveto{\pgfqpoint{4.212200in}{1.657270in}}{\pgfqpoint{4.209366in}{1.664112in}}{\pgfqpoint{4.204322in}{1.669155in}}%
\pgfpathcurveto{\pgfqpoint{4.199279in}{1.674199in}}{\pgfqpoint{4.192437in}{1.677033in}}{\pgfqpoint{4.185304in}{1.677033in}}%
\pgfpathcurveto{\pgfqpoint{4.178171in}{1.677033in}}{\pgfqpoint{4.171330in}{1.674199in}}{\pgfqpoint{4.166286in}{1.669155in}}%
\pgfpathcurveto{\pgfqpoint{4.161242in}{1.664112in}}{\pgfqpoint{4.158409in}{1.657270in}}{\pgfqpoint{4.158409in}{1.650137in}}%
\pgfpathcurveto{\pgfqpoint{4.158409in}{1.643004in}}{\pgfqpoint{4.161242in}{1.636163in}}{\pgfqpoint{4.166286in}{1.631119in}}%
\pgfpathcurveto{\pgfqpoint{4.171330in}{1.626075in}}{\pgfqpoint{4.178171in}{1.623241in}}{\pgfqpoint{4.185304in}{1.623241in}}%
\pgfpathclose%
\pgfusepath{stroke,fill}%
\end{pgfscope}%
\begin{pgfscope}%
\pgfpathrectangle{\pgfqpoint{2.867647in}{0.500000in}}{\pgfqpoint{1.764706in}{1.700000in}}%
\pgfusepath{clip}%
\pgfsetbuttcap%
\pgfsetroundjoin%
\definecolor{currentfill}{rgb}{0.975018,0.868213,0.788710}%
\pgfsetfillcolor{currentfill}%
\pgfsetlinewidth{0.311001pt}%
\definecolor{currentstroke}{rgb}{1.000000,1.000000,1.000000}%
\pgfsetstrokecolor{currentstroke}%
\pgfsetdash{}{0pt}%
\pgfpathmoveto{\pgfqpoint{4.239345in}{1.430638in}}%
\pgfpathcurveto{\pgfqpoint{4.246478in}{1.430638in}}{\pgfqpoint{4.253320in}{1.433472in}}{\pgfqpoint{4.258363in}{1.438516in}}%
\pgfpathcurveto{\pgfqpoint{4.263407in}{1.443559in}}{\pgfqpoint{4.266241in}{1.450401in}}{\pgfqpoint{4.266241in}{1.457534in}}%
\pgfpathcurveto{\pgfqpoint{4.266241in}{1.464667in}}{\pgfqpoint{4.263407in}{1.471508in}}{\pgfqpoint{4.258363in}{1.476552in}}%
\pgfpathcurveto{\pgfqpoint{4.253320in}{1.481596in}}{\pgfqpoint{4.246478in}{1.484429in}}{\pgfqpoint{4.239345in}{1.484429in}}%
\pgfpathcurveto{\pgfqpoint{4.232212in}{1.484429in}}{\pgfqpoint{4.225371in}{1.481596in}}{\pgfqpoint{4.220327in}{1.476552in}}%
\pgfpathcurveto{\pgfqpoint{4.215283in}{1.471508in}}{\pgfqpoint{4.212449in}{1.464667in}}{\pgfqpoint{4.212449in}{1.457534in}}%
\pgfpathcurveto{\pgfqpoint{4.212449in}{1.450401in}}{\pgfqpoint{4.215283in}{1.443559in}}{\pgfqpoint{4.220327in}{1.438516in}}%
\pgfpathcurveto{\pgfqpoint{4.225371in}{1.433472in}}{\pgfqpoint{4.232212in}{1.430638in}}{\pgfqpoint{4.239345in}{1.430638in}}%
\pgfpathclose%
\pgfusepath{stroke,fill}%
\end{pgfscope}%
\begin{pgfscope}%
\pgfpathrectangle{\pgfqpoint{2.867647in}{0.500000in}}{\pgfqpoint{1.764706in}{1.700000in}}%
\pgfusepath{clip}%
\pgfsetbuttcap%
\pgfsetroundjoin%
\definecolor{currentfill}{rgb}{0.979891,0.908948,0.848279}%
\pgfsetfillcolor{currentfill}%
\pgfsetlinewidth{0.311001pt}%
\definecolor{currentstroke}{rgb}{1.000000,1.000000,1.000000}%
\pgfsetstrokecolor{currentstroke}%
\pgfsetdash{}{0pt}%
\pgfpathmoveto{\pgfqpoint{4.204142in}{1.398927in}}%
\pgfpathcurveto{\pgfqpoint{4.211274in}{1.398927in}}{\pgfqpoint{4.218116in}{1.401761in}}{\pgfqpoint{4.223160in}{1.406804in}}%
\pgfpathcurveto{\pgfqpoint{4.228203in}{1.411848in}}{\pgfqpoint{4.231037in}{1.418690in}}{\pgfqpoint{4.231037in}{1.425822in}}%
\pgfpathcurveto{\pgfqpoint{4.231037in}{1.432955in}}{\pgfqpoint{4.228203in}{1.439797in}}{\pgfqpoint{4.223160in}{1.444841in}}%
\pgfpathcurveto{\pgfqpoint{4.218116in}{1.449884in}}{\pgfqpoint{4.211274in}{1.452718in}}{\pgfqpoint{4.204142in}{1.452718in}}%
\pgfpathcurveto{\pgfqpoint{4.197009in}{1.452718in}}{\pgfqpoint{4.190167in}{1.449884in}}{\pgfqpoint{4.185123in}{1.444841in}}%
\pgfpathcurveto{\pgfqpoint{4.180080in}{1.439797in}}{\pgfqpoint{4.177246in}{1.432955in}}{\pgfqpoint{4.177246in}{1.425822in}}%
\pgfpathcurveto{\pgfqpoint{4.177246in}{1.418690in}}{\pgfqpoint{4.180080in}{1.411848in}}{\pgfqpoint{4.185123in}{1.406804in}}%
\pgfpathcurveto{\pgfqpoint{4.190167in}{1.401761in}}{\pgfqpoint{4.197009in}{1.398927in}}{\pgfqpoint{4.204142in}{1.398927in}}%
\pgfpathclose%
\pgfusepath{stroke,fill}%
\end{pgfscope}%
\begin{pgfscope}%
\pgfpathrectangle{\pgfqpoint{2.867647in}{0.500000in}}{\pgfqpoint{1.764706in}{1.700000in}}%
\pgfusepath{clip}%
\pgfsetbuttcap%
\pgfsetroundjoin%
\definecolor{currentfill}{rgb}{0.979891,0.908948,0.848279}%
\pgfsetfillcolor{currentfill}%
\pgfsetlinewidth{0.311001pt}%
\definecolor{currentstroke}{rgb}{1.000000,1.000000,1.000000}%
\pgfsetstrokecolor{currentstroke}%
\pgfsetdash{}{0pt}%
\pgfpathmoveto{\pgfqpoint{4.142222in}{1.179839in}}%
\pgfpathcurveto{\pgfqpoint{4.149355in}{1.179839in}}{\pgfqpoint{4.156197in}{1.182673in}}{\pgfqpoint{4.161241in}{1.187716in}}%
\pgfpathcurveto{\pgfqpoint{4.166284in}{1.192760in}}{\pgfqpoint{4.169118in}{1.199602in}}{\pgfqpoint{4.169118in}{1.206735in}}%
\pgfpathcurveto{\pgfqpoint{4.169118in}{1.213867in}}{\pgfqpoint{4.166284in}{1.220709in}}{\pgfqpoint{4.161241in}{1.225753in}}%
\pgfpathcurveto{\pgfqpoint{4.156197in}{1.230796in}}{\pgfqpoint{4.149355in}{1.233630in}}{\pgfqpoint{4.142222in}{1.233630in}}%
\pgfpathcurveto{\pgfqpoint{4.135090in}{1.233630in}}{\pgfqpoint{4.128248in}{1.230796in}}{\pgfqpoint{4.123204in}{1.225753in}}%
\pgfpathcurveto{\pgfqpoint{4.118161in}{1.220709in}}{\pgfqpoint{4.115327in}{1.213867in}}{\pgfqpoint{4.115327in}{1.206735in}}%
\pgfpathcurveto{\pgfqpoint{4.115327in}{1.199602in}}{\pgfqpoint{4.118161in}{1.192760in}}{\pgfqpoint{4.123204in}{1.187716in}}%
\pgfpathcurveto{\pgfqpoint{4.128248in}{1.182673in}}{\pgfqpoint{4.135090in}{1.179839in}}{\pgfqpoint{4.142222in}{1.179839in}}%
\pgfpathclose%
\pgfusepath{stroke,fill}%
\end{pgfscope}%
\begin{pgfscope}%
\pgfpathrectangle{\pgfqpoint{2.867647in}{0.500000in}}{\pgfqpoint{1.764706in}{1.700000in}}%
\pgfusepath{clip}%
\pgfsetbuttcap%
\pgfsetroundjoin%
\definecolor{currentfill}{rgb}{0.970255,0.815666,0.711203}%
\pgfsetfillcolor{currentfill}%
\pgfsetlinewidth{0.311001pt}%
\definecolor{currentstroke}{rgb}{1.000000,1.000000,1.000000}%
\pgfsetstrokecolor{currentstroke}%
\pgfsetdash{}{0pt}%
\pgfpathmoveto{\pgfqpoint{4.044825in}{1.649246in}}%
\pgfpathcurveto{\pgfqpoint{4.051958in}{1.649246in}}{\pgfqpoint{4.058799in}{1.652080in}}{\pgfqpoint{4.063843in}{1.657124in}}%
\pgfpathcurveto{\pgfqpoint{4.068887in}{1.662167in}}{\pgfqpoint{4.071721in}{1.669009in}}{\pgfqpoint{4.071721in}{1.676142in}}%
\pgfpathcurveto{\pgfqpoint{4.071721in}{1.683275in}}{\pgfqpoint{4.068887in}{1.690116in}}{\pgfqpoint{4.063843in}{1.695160in}}%
\pgfpathcurveto{\pgfqpoint{4.058799in}{1.700204in}}{\pgfqpoint{4.051958in}{1.703037in}}{\pgfqpoint{4.044825in}{1.703037in}}%
\pgfpathcurveto{\pgfqpoint{4.037692in}{1.703037in}}{\pgfqpoint{4.030850in}{1.700204in}}{\pgfqpoint{4.025807in}{1.695160in}}%
\pgfpathcurveto{\pgfqpoint{4.020763in}{1.690116in}}{\pgfqpoint{4.017929in}{1.683275in}}{\pgfqpoint{4.017929in}{1.676142in}}%
\pgfpathcurveto{\pgfqpoint{4.017929in}{1.669009in}}{\pgfqpoint{4.020763in}{1.662167in}}{\pgfqpoint{4.025807in}{1.657124in}}%
\pgfpathcurveto{\pgfqpoint{4.030850in}{1.652080in}}{\pgfqpoint{4.037692in}{1.649246in}}{\pgfqpoint{4.044825in}{1.649246in}}%
\pgfpathclose%
\pgfusepath{stroke,fill}%
\end{pgfscope}%
\begin{pgfscope}%
\pgfpathrectangle{\pgfqpoint{2.867647in}{0.500000in}}{\pgfqpoint{1.764706in}{1.700000in}}%
\pgfusepath{clip}%
\pgfsetbuttcap%
\pgfsetroundjoin%
\definecolor{currentfill}{rgb}{0.964032,0.651225,0.493258}%
\pgfsetfillcolor{currentfill}%
\pgfsetlinewidth{0.311001pt}%
\definecolor{currentstroke}{rgb}{1.000000,1.000000,1.000000}%
\pgfsetstrokecolor{currentstroke}%
\pgfsetdash{}{0pt}%
\pgfpathmoveto{\pgfqpoint{4.022065in}{1.134598in}}%
\pgfpathcurveto{\pgfqpoint{4.029198in}{1.134598in}}{\pgfqpoint{4.036040in}{1.137432in}}{\pgfqpoint{4.041083in}{1.142475in}}%
\pgfpathcurveto{\pgfqpoint{4.046127in}{1.147519in}}{\pgfqpoint{4.048961in}{1.154361in}}{\pgfqpoint{4.048961in}{1.161493in}}%
\pgfpathcurveto{\pgfqpoint{4.048961in}{1.168626in}}{\pgfqpoint{4.046127in}{1.175468in}}{\pgfqpoint{4.041083in}{1.180512in}}%
\pgfpathcurveto{\pgfqpoint{4.036040in}{1.185555in}}{\pgfqpoint{4.029198in}{1.188389in}}{\pgfqpoint{4.022065in}{1.188389in}}%
\pgfpathcurveto{\pgfqpoint{4.014932in}{1.188389in}}{\pgfqpoint{4.008091in}{1.185555in}}{\pgfqpoint{4.003047in}{1.180512in}}%
\pgfpathcurveto{\pgfqpoint{3.998003in}{1.175468in}}{\pgfqpoint{3.995169in}{1.168626in}}{\pgfqpoint{3.995169in}{1.161493in}}%
\pgfpathcurveto{\pgfqpoint{3.995169in}{1.154361in}}{\pgfqpoint{3.998003in}{1.147519in}}{\pgfqpoint{4.003047in}{1.142475in}}%
\pgfpathcurveto{\pgfqpoint{4.008091in}{1.137432in}}{\pgfqpoint{4.014932in}{1.134598in}}{\pgfqpoint{4.022065in}{1.134598in}}%
\pgfpathclose%
\pgfusepath{stroke,fill}%
\end{pgfscope}%
\begin{pgfscope}%
\pgfpathrectangle{\pgfqpoint{2.867647in}{0.500000in}}{\pgfqpoint{1.764706in}{1.700000in}}%
\pgfusepath{clip}%
\pgfsetbuttcap%
\pgfsetroundjoin%
\definecolor{currentfill}{rgb}{0.976287,0.879862,0.805788}%
\pgfsetfillcolor{currentfill}%
\pgfsetlinewidth{0.311001pt}%
\definecolor{currentstroke}{rgb}{1.000000,1.000000,1.000000}%
\pgfsetstrokecolor{currentstroke}%
\pgfsetdash{}{0pt}%
\pgfpathmoveto{\pgfqpoint{4.194087in}{1.546638in}}%
\pgfpathcurveto{\pgfqpoint{4.201219in}{1.546638in}}{\pgfqpoint{4.208061in}{1.549472in}}{\pgfqpoint{4.213105in}{1.554515in}}%
\pgfpathcurveto{\pgfqpoint{4.218148in}{1.559559in}}{\pgfqpoint{4.220982in}{1.566401in}}{\pgfqpoint{4.220982in}{1.573534in}}%
\pgfpathcurveto{\pgfqpoint{4.220982in}{1.580666in}}{\pgfqpoint{4.218148in}{1.587508in}}{\pgfqpoint{4.213105in}{1.592552in}}%
\pgfpathcurveto{\pgfqpoint{4.208061in}{1.597595in}}{\pgfqpoint{4.201219in}{1.600429in}}{\pgfqpoint{4.194087in}{1.600429in}}%
\pgfpathcurveto{\pgfqpoint{4.186954in}{1.600429in}}{\pgfqpoint{4.180112in}{1.597595in}}{\pgfqpoint{4.175069in}{1.592552in}}%
\pgfpathcurveto{\pgfqpoint{4.170025in}{1.587508in}}{\pgfqpoint{4.167191in}{1.580666in}}{\pgfqpoint{4.167191in}{1.573534in}}%
\pgfpathcurveto{\pgfqpoint{4.167191in}{1.566401in}}{\pgfqpoint{4.170025in}{1.559559in}}{\pgfqpoint{4.175069in}{1.554515in}}%
\pgfpathcurveto{\pgfqpoint{4.180112in}{1.549472in}}{\pgfqpoint{4.186954in}{1.546638in}}{\pgfqpoint{4.194087in}{1.546638in}}%
\pgfpathclose%
\pgfusepath{stroke,fill}%
\end{pgfscope}%
\begin{pgfscope}%
\pgfpathrectangle{\pgfqpoint{2.867647in}{0.500000in}}{\pgfqpoint{1.764706in}{1.700000in}}%
\pgfusepath{clip}%
\pgfsetbuttcap%
\pgfsetroundjoin%
\definecolor{currentfill}{rgb}{0.966328,0.750560,0.616961}%
\pgfsetfillcolor{currentfill}%
\pgfsetlinewidth{0.311001pt}%
\definecolor{currentstroke}{rgb}{1.000000,1.000000,1.000000}%
\pgfsetstrokecolor{currentstroke}%
\pgfsetdash{}{0pt}%
\pgfpathmoveto{\pgfqpoint{4.074169in}{1.419261in}}%
\pgfpathcurveto{\pgfqpoint{4.081302in}{1.419261in}}{\pgfqpoint{4.088144in}{1.422095in}}{\pgfqpoint{4.093187in}{1.427138in}}%
\pgfpathcurveto{\pgfqpoint{4.098231in}{1.432182in}}{\pgfqpoint{4.101065in}{1.439024in}}{\pgfqpoint{4.101065in}{1.446157in}}%
\pgfpathcurveto{\pgfqpoint{4.101065in}{1.453289in}}{\pgfqpoint{4.098231in}{1.460131in}}{\pgfqpoint{4.093187in}{1.465175in}}%
\pgfpathcurveto{\pgfqpoint{4.088144in}{1.470218in}}{\pgfqpoint{4.081302in}{1.473052in}}{\pgfqpoint{4.074169in}{1.473052in}}%
\pgfpathcurveto{\pgfqpoint{4.067036in}{1.473052in}}{\pgfqpoint{4.060195in}{1.470218in}}{\pgfqpoint{4.055151in}{1.465175in}}%
\pgfpathcurveto{\pgfqpoint{4.050107in}{1.460131in}}{\pgfqpoint{4.047273in}{1.453289in}}{\pgfqpoint{4.047273in}{1.446157in}}%
\pgfpathcurveto{\pgfqpoint{4.047273in}{1.439024in}}{\pgfqpoint{4.050107in}{1.432182in}}{\pgfqpoint{4.055151in}{1.427138in}}%
\pgfpathcurveto{\pgfqpoint{4.060195in}{1.422095in}}{\pgfqpoint{4.067036in}{1.419261in}}{\pgfqpoint{4.074169in}{1.419261in}}%
\pgfpathclose%
\pgfusepath{stroke,fill}%
\end{pgfscope}%
\begin{pgfscope}%
\pgfpathrectangle{\pgfqpoint{2.867647in}{0.500000in}}{\pgfqpoint{1.764706in}{1.700000in}}%
\pgfusepath{clip}%
\pgfsetbuttcap%
\pgfsetroundjoin%
\definecolor{currentfill}{rgb}{0.969803,0.809811,0.702523}%
\pgfsetfillcolor{currentfill}%
\pgfsetlinewidth{0.311001pt}%
\definecolor{currentstroke}{rgb}{1.000000,1.000000,1.000000}%
\pgfsetstrokecolor{currentstroke}%
\pgfsetdash{}{0pt}%
\pgfpathmoveto{\pgfqpoint{4.047562in}{1.046258in}}%
\pgfpathcurveto{\pgfqpoint{4.054695in}{1.046258in}}{\pgfqpoint{4.061537in}{1.049092in}}{\pgfqpoint{4.066580in}{1.054136in}}%
\pgfpathcurveto{\pgfqpoint{4.071624in}{1.059179in}}{\pgfqpoint{4.074458in}{1.066021in}}{\pgfqpoint{4.074458in}{1.073154in}}%
\pgfpathcurveto{\pgfqpoint{4.074458in}{1.080287in}}{\pgfqpoint{4.071624in}{1.087128in}}{\pgfqpoint{4.066580in}{1.092172in}}%
\pgfpathcurveto{\pgfqpoint{4.061537in}{1.097216in}}{\pgfqpoint{4.054695in}{1.100050in}}{\pgfqpoint{4.047562in}{1.100050in}}%
\pgfpathcurveto{\pgfqpoint{4.040429in}{1.100050in}}{\pgfqpoint{4.033588in}{1.097216in}}{\pgfqpoint{4.028544in}{1.092172in}}%
\pgfpathcurveto{\pgfqpoint{4.023500in}{1.087128in}}{\pgfqpoint{4.020666in}{1.080287in}}{\pgfqpoint{4.020666in}{1.073154in}}%
\pgfpathcurveto{\pgfqpoint{4.020666in}{1.066021in}}{\pgfqpoint{4.023500in}{1.059179in}}{\pgfqpoint{4.028544in}{1.054136in}}%
\pgfpathcurveto{\pgfqpoint{4.033588in}{1.049092in}}{\pgfqpoint{4.040429in}{1.046258in}}{\pgfqpoint{4.047562in}{1.046258in}}%
\pgfpathclose%
\pgfusepath{stroke,fill}%
\end{pgfscope}%
\begin{pgfscope}%
\pgfpathrectangle{\pgfqpoint{2.867647in}{0.500000in}}{\pgfqpoint{1.764706in}{1.700000in}}%
\pgfusepath{clip}%
\pgfsetbuttcap%
\pgfsetroundjoin%
\definecolor{currentfill}{rgb}{0.976961,0.885681,0.814303}%
\pgfsetfillcolor{currentfill}%
\pgfsetlinewidth{0.311001pt}%
\definecolor{currentstroke}{rgb}{1.000000,1.000000,1.000000}%
\pgfsetstrokecolor{currentstroke}%
\pgfsetdash{}{0pt}%
\pgfpathmoveto{\pgfqpoint{4.138852in}{1.612513in}}%
\pgfpathcurveto{\pgfqpoint{4.145985in}{1.612513in}}{\pgfqpoint{4.152826in}{1.615347in}}{\pgfqpoint{4.157870in}{1.620390in}}%
\pgfpathcurveto{\pgfqpoint{4.162914in}{1.625434in}}{\pgfqpoint{4.165748in}{1.632276in}}{\pgfqpoint{4.165748in}{1.639408in}}%
\pgfpathcurveto{\pgfqpoint{4.165748in}{1.646541in}}{\pgfqpoint{4.162914in}{1.653383in}}{\pgfqpoint{4.157870in}{1.658426in}}%
\pgfpathcurveto{\pgfqpoint{4.152826in}{1.663470in}}{\pgfqpoint{4.145985in}{1.666304in}}{\pgfqpoint{4.138852in}{1.666304in}}%
\pgfpathcurveto{\pgfqpoint{4.131719in}{1.666304in}}{\pgfqpoint{4.124877in}{1.663470in}}{\pgfqpoint{4.119834in}{1.658426in}}%
\pgfpathcurveto{\pgfqpoint{4.114790in}{1.653383in}}{\pgfqpoint{4.111956in}{1.646541in}}{\pgfqpoint{4.111956in}{1.639408in}}%
\pgfpathcurveto{\pgfqpoint{4.111956in}{1.632276in}}{\pgfqpoint{4.114790in}{1.625434in}}{\pgfqpoint{4.119834in}{1.620390in}}%
\pgfpathcurveto{\pgfqpoint{4.124877in}{1.615347in}}{\pgfqpoint{4.131719in}{1.612513in}}{\pgfqpoint{4.138852in}{1.612513in}}%
\pgfpathclose%
\pgfusepath{stroke,fill}%
\end{pgfscope}%
\begin{pgfscope}%
\pgfpathrectangle{\pgfqpoint{2.867647in}{0.500000in}}{\pgfqpoint{1.764706in}{1.700000in}}%
\pgfusepath{clip}%
\pgfsetbuttcap%
\pgfsetroundjoin%
\definecolor{currentfill}{rgb}{0.976287,0.879862,0.805788}%
\pgfsetfillcolor{currentfill}%
\pgfsetlinewidth{0.311001pt}%
\definecolor{currentstroke}{rgb}{1.000000,1.000000,1.000000}%
\pgfsetstrokecolor{currentstroke}%
\pgfsetdash{}{0pt}%
\pgfpathmoveto{\pgfqpoint{4.095651in}{1.622429in}}%
\pgfpathcurveto{\pgfqpoint{4.102784in}{1.622429in}}{\pgfqpoint{4.109625in}{1.625263in}}{\pgfqpoint{4.114669in}{1.630307in}}%
\pgfpathcurveto{\pgfqpoint{4.119712in}{1.635351in}}{\pgfqpoint{4.122546in}{1.642192in}}{\pgfqpoint{4.122546in}{1.649325in}}%
\pgfpathcurveto{\pgfqpoint{4.122546in}{1.656458in}}{\pgfqpoint{4.119712in}{1.663300in}}{\pgfqpoint{4.114669in}{1.668343in}}%
\pgfpathcurveto{\pgfqpoint{4.109625in}{1.673387in}}{\pgfqpoint{4.102784in}{1.676221in}}{\pgfqpoint{4.095651in}{1.676221in}}%
\pgfpathcurveto{\pgfqpoint{4.088518in}{1.676221in}}{\pgfqpoint{4.081676in}{1.673387in}}{\pgfqpoint{4.076633in}{1.668343in}}%
\pgfpathcurveto{\pgfqpoint{4.071589in}{1.663300in}}{\pgfqpoint{4.068755in}{1.656458in}}{\pgfqpoint{4.068755in}{1.649325in}}%
\pgfpathcurveto{\pgfqpoint{4.068755in}{1.642192in}}{\pgfqpoint{4.071589in}{1.635351in}}{\pgfqpoint{4.076633in}{1.630307in}}%
\pgfpathcurveto{\pgfqpoint{4.081676in}{1.625263in}}{\pgfqpoint{4.088518in}{1.622429in}}{\pgfqpoint{4.095651in}{1.622429in}}%
\pgfpathclose%
\pgfusepath{stroke,fill}%
\end{pgfscope}%
\begin{pgfscope}%
\pgfpathrectangle{\pgfqpoint{2.867647in}{0.500000in}}{\pgfqpoint{1.764706in}{1.700000in}}%
\pgfusepath{clip}%
\pgfsetbuttcap%
\pgfsetroundjoin%
\definecolor{currentfill}{rgb}{0.965753,0.732351,0.592427}%
\pgfsetfillcolor{currentfill}%
\pgfsetlinewidth{0.311001pt}%
\definecolor{currentstroke}{rgb}{1.000000,1.000000,1.000000}%
\pgfsetstrokecolor{currentstroke}%
\pgfsetdash{}{0pt}%
\pgfpathmoveto{\pgfqpoint{4.246590in}{1.057566in}}%
\pgfpathcurveto{\pgfqpoint{4.253723in}{1.057566in}}{\pgfqpoint{4.260565in}{1.060399in}}{\pgfqpoint{4.265609in}{1.065443in}}%
\pgfpathcurveto{\pgfqpoint{4.270652in}{1.070487in}}{\pgfqpoint{4.273486in}{1.077328in}}{\pgfqpoint{4.273486in}{1.084461in}}%
\pgfpathcurveto{\pgfqpoint{4.273486in}{1.091594in}}{\pgfqpoint{4.270652in}{1.098436in}}{\pgfqpoint{4.265609in}{1.103479in}}%
\pgfpathcurveto{\pgfqpoint{4.260565in}{1.108523in}}{\pgfqpoint{4.253723in}{1.111357in}}{\pgfqpoint{4.246590in}{1.111357in}}%
\pgfpathcurveto{\pgfqpoint{4.239458in}{1.111357in}}{\pgfqpoint{4.232616in}{1.108523in}}{\pgfqpoint{4.227572in}{1.103479in}}%
\pgfpathcurveto{\pgfqpoint{4.222529in}{1.098436in}}{\pgfqpoint{4.219695in}{1.091594in}}{\pgfqpoint{4.219695in}{1.084461in}}%
\pgfpathcurveto{\pgfqpoint{4.219695in}{1.077328in}}{\pgfqpoint{4.222529in}{1.070487in}}{\pgfqpoint{4.227572in}{1.065443in}}%
\pgfpathcurveto{\pgfqpoint{4.232616in}{1.060399in}}{\pgfqpoint{4.239458in}{1.057566in}}{\pgfqpoint{4.246590in}{1.057566in}}%
\pgfpathclose%
\pgfusepath{stroke,fill}%
\end{pgfscope}%
\begin{pgfscope}%
\pgfpathrectangle{\pgfqpoint{2.867647in}{0.500000in}}{\pgfqpoint{1.764706in}{1.700000in}}%
\pgfusepath{clip}%
\pgfsetbuttcap%
\pgfsetroundjoin%
\definecolor{currentfill}{rgb}{0.971694,0.833208,0.737161}%
\pgfsetfillcolor{currentfill}%
\pgfsetlinewidth{0.311001pt}%
\definecolor{currentstroke}{rgb}{1.000000,1.000000,1.000000}%
\pgfsetstrokecolor{currentstroke}%
\pgfsetdash{}{0pt}%
\pgfpathmoveto{\pgfqpoint{4.085686in}{1.484124in}}%
\pgfpathcurveto{\pgfqpoint{4.092818in}{1.484124in}}{\pgfqpoint{4.099660in}{1.486958in}}{\pgfqpoint{4.104704in}{1.492002in}}%
\pgfpathcurveto{\pgfqpoint{4.109747in}{1.497045in}}{\pgfqpoint{4.112581in}{1.503887in}}{\pgfqpoint{4.112581in}{1.511020in}}%
\pgfpathcurveto{\pgfqpoint{4.112581in}{1.518152in}}{\pgfqpoint{4.109747in}{1.524994in}}{\pgfqpoint{4.104704in}{1.530038in}}%
\pgfpathcurveto{\pgfqpoint{4.099660in}{1.535081in}}{\pgfqpoint{4.092818in}{1.537915in}}{\pgfqpoint{4.085686in}{1.537915in}}%
\pgfpathcurveto{\pgfqpoint{4.078553in}{1.537915in}}{\pgfqpoint{4.071711in}{1.535081in}}{\pgfqpoint{4.066667in}{1.530038in}}%
\pgfpathcurveto{\pgfqpoint{4.061624in}{1.524994in}}{\pgfqpoint{4.058790in}{1.518152in}}{\pgfqpoint{4.058790in}{1.511020in}}%
\pgfpathcurveto{\pgfqpoint{4.058790in}{1.503887in}}{\pgfqpoint{4.061624in}{1.497045in}}{\pgfqpoint{4.066667in}{1.492002in}}%
\pgfpathcurveto{\pgfqpoint{4.071711in}{1.486958in}}{\pgfqpoint{4.078553in}{1.484124in}}{\pgfqpoint{4.085686in}{1.484124in}}%
\pgfpathclose%
\pgfusepath{stroke,fill}%
\end{pgfscope}%
\begin{pgfscope}%
\pgfpathrectangle{\pgfqpoint{2.867647in}{0.500000in}}{\pgfqpoint{1.764706in}{1.700000in}}%
\pgfusepath{clip}%
\pgfsetbuttcap%
\pgfsetroundjoin%
\definecolor{currentfill}{rgb}{0.972726,0.844889,0.754401}%
\pgfsetfillcolor{currentfill}%
\pgfsetlinewidth{0.311001pt}%
\definecolor{currentstroke}{rgb}{1.000000,1.000000,1.000000}%
\pgfsetstrokecolor{currentstroke}%
\pgfsetdash{}{0pt}%
\pgfpathmoveto{\pgfqpoint{4.200934in}{1.060767in}}%
\pgfpathcurveto{\pgfqpoint{4.208067in}{1.060767in}}{\pgfqpoint{4.214909in}{1.063601in}}{\pgfqpoint{4.219952in}{1.068645in}}%
\pgfpathcurveto{\pgfqpoint{4.224996in}{1.073688in}}{\pgfqpoint{4.227830in}{1.080530in}}{\pgfqpoint{4.227830in}{1.087663in}}%
\pgfpathcurveto{\pgfqpoint{4.227830in}{1.094796in}}{\pgfqpoint{4.224996in}{1.101637in}}{\pgfqpoint{4.219952in}{1.106681in}}%
\pgfpathcurveto{\pgfqpoint{4.214909in}{1.111725in}}{\pgfqpoint{4.208067in}{1.114559in}}{\pgfqpoint{4.200934in}{1.114559in}}%
\pgfpathcurveto{\pgfqpoint{4.193802in}{1.114559in}}{\pgfqpoint{4.186960in}{1.111725in}}{\pgfqpoint{4.181916in}{1.106681in}}%
\pgfpathcurveto{\pgfqpoint{4.176873in}{1.101637in}}{\pgfqpoint{4.174039in}{1.094796in}}{\pgfqpoint{4.174039in}{1.087663in}}%
\pgfpathcurveto{\pgfqpoint{4.174039in}{1.080530in}}{\pgfqpoint{4.176873in}{1.073688in}}{\pgfqpoint{4.181916in}{1.068645in}}%
\pgfpathcurveto{\pgfqpoint{4.186960in}{1.063601in}}{\pgfqpoint{4.193802in}{1.060767in}}{\pgfqpoint{4.200934in}{1.060767in}}%
\pgfpathclose%
\pgfusepath{stroke,fill}%
\end{pgfscope}%
\begin{pgfscope}%
\pgfpathrectangle{\pgfqpoint{2.867647in}{0.500000in}}{\pgfqpoint{1.764706in}{1.700000in}}%
\pgfusepath{clip}%
\pgfsetbuttcap%
\pgfsetroundjoin%
\definecolor{currentfill}{rgb}{0.979891,0.908948,0.848279}%
\pgfsetfillcolor{currentfill}%
\pgfsetlinewidth{0.311001pt}%
\definecolor{currentstroke}{rgb}{1.000000,1.000000,1.000000}%
\pgfsetstrokecolor{currentstroke}%
\pgfsetdash{}{0pt}%
\pgfpathmoveto{\pgfqpoint{4.212763in}{1.210845in}}%
\pgfpathcurveto{\pgfqpoint{4.219896in}{1.210845in}}{\pgfqpoint{4.226737in}{1.213679in}}{\pgfqpoint{4.231781in}{1.218723in}}%
\pgfpathcurveto{\pgfqpoint{4.236825in}{1.223766in}}{\pgfqpoint{4.239659in}{1.230608in}}{\pgfqpoint{4.239659in}{1.237741in}}%
\pgfpathcurveto{\pgfqpoint{4.239659in}{1.244874in}}{\pgfqpoint{4.236825in}{1.251715in}}{\pgfqpoint{4.231781in}{1.256759in}}%
\pgfpathcurveto{\pgfqpoint{4.226737in}{1.261803in}}{\pgfqpoint{4.219896in}{1.264637in}}{\pgfqpoint{4.212763in}{1.264637in}}%
\pgfpathcurveto{\pgfqpoint{4.205630in}{1.264637in}}{\pgfqpoint{4.198788in}{1.261803in}}{\pgfqpoint{4.193745in}{1.256759in}}%
\pgfpathcurveto{\pgfqpoint{4.188701in}{1.251715in}}{\pgfqpoint{4.185867in}{1.244874in}}{\pgfqpoint{4.185867in}{1.237741in}}%
\pgfpathcurveto{\pgfqpoint{4.185867in}{1.230608in}}{\pgfqpoint{4.188701in}{1.223766in}}{\pgfqpoint{4.193745in}{1.218723in}}%
\pgfpathcurveto{\pgfqpoint{4.198788in}{1.213679in}}{\pgfqpoint{4.205630in}{1.210845in}}{\pgfqpoint{4.212763in}{1.210845in}}%
\pgfpathclose%
\pgfusepath{stroke,fill}%
\end{pgfscope}%
\begin{pgfscope}%
\pgfpathrectangle{\pgfqpoint{2.867647in}{0.500000in}}{\pgfqpoint{1.764706in}{1.700000in}}%
\pgfusepath{clip}%
\pgfsetbuttcap%
\pgfsetroundjoin%
\definecolor{currentfill}{rgb}{0.979891,0.908948,0.848279}%
\pgfsetfillcolor{currentfill}%
\pgfsetlinewidth{0.311001pt}%
\definecolor{currentstroke}{rgb}{1.000000,1.000000,1.000000}%
\pgfsetstrokecolor{currentstroke}%
\pgfsetdash{}{0pt}%
\pgfpathmoveto{\pgfqpoint{4.153855in}{1.453668in}}%
\pgfpathcurveto{\pgfqpoint{4.160987in}{1.453668in}}{\pgfqpoint{4.167829in}{1.456502in}}{\pgfqpoint{4.172873in}{1.461546in}}%
\pgfpathcurveto{\pgfqpoint{4.177916in}{1.466589in}}{\pgfqpoint{4.180750in}{1.473431in}}{\pgfqpoint{4.180750in}{1.480564in}}%
\pgfpathcurveto{\pgfqpoint{4.180750in}{1.487697in}}{\pgfqpoint{4.177916in}{1.494538in}}{\pgfqpoint{4.172873in}{1.499582in}}%
\pgfpathcurveto{\pgfqpoint{4.167829in}{1.504626in}}{\pgfqpoint{4.160987in}{1.507460in}}{\pgfqpoint{4.153855in}{1.507460in}}%
\pgfpathcurveto{\pgfqpoint{4.146722in}{1.507460in}}{\pgfqpoint{4.139880in}{1.504626in}}{\pgfqpoint{4.134836in}{1.499582in}}%
\pgfpathcurveto{\pgfqpoint{4.129793in}{1.494538in}}{\pgfqpoint{4.126959in}{1.487697in}}{\pgfqpoint{4.126959in}{1.480564in}}%
\pgfpathcurveto{\pgfqpoint{4.126959in}{1.473431in}}{\pgfqpoint{4.129793in}{1.466589in}}{\pgfqpoint{4.134836in}{1.461546in}}%
\pgfpathcurveto{\pgfqpoint{4.139880in}{1.456502in}}{\pgfqpoint{4.146722in}{1.453668in}}{\pgfqpoint{4.153855in}{1.453668in}}%
\pgfpathclose%
\pgfusepath{stroke,fill}%
\end{pgfscope}%
\begin{pgfscope}%
\pgfpathrectangle{\pgfqpoint{2.867647in}{0.500000in}}{\pgfqpoint{1.764706in}{1.700000in}}%
\pgfusepath{clip}%
\pgfsetbuttcap%
\pgfsetroundjoin%
\definecolor{currentfill}{rgb}{0.969359,0.803954,0.693832}%
\pgfsetfillcolor{currentfill}%
\pgfsetlinewidth{0.311001pt}%
\definecolor{currentstroke}{rgb}{1.000000,1.000000,1.000000}%
\pgfsetstrokecolor{currentstroke}%
\pgfsetdash{}{0pt}%
\pgfpathmoveto{\pgfqpoint{4.104240in}{1.349633in}}%
\pgfpathcurveto{\pgfqpoint{4.111373in}{1.349633in}}{\pgfqpoint{4.118215in}{1.352467in}}{\pgfqpoint{4.123259in}{1.357511in}}%
\pgfpathcurveto{\pgfqpoint{4.128302in}{1.362555in}}{\pgfqpoint{4.131136in}{1.369396in}}{\pgfqpoint{4.131136in}{1.376529in}}%
\pgfpathcurveto{\pgfqpoint{4.131136in}{1.383662in}}{\pgfqpoint{4.128302in}{1.390503in}}{\pgfqpoint{4.123259in}{1.395547in}}%
\pgfpathcurveto{\pgfqpoint{4.118215in}{1.400591in}}{\pgfqpoint{4.111373in}{1.403425in}}{\pgfqpoint{4.104240in}{1.403425in}}%
\pgfpathcurveto{\pgfqpoint{4.097108in}{1.403425in}}{\pgfqpoint{4.090266in}{1.400591in}}{\pgfqpoint{4.085222in}{1.395547in}}%
\pgfpathcurveto{\pgfqpoint{4.080179in}{1.390503in}}{\pgfqpoint{4.077345in}{1.383662in}}{\pgfqpoint{4.077345in}{1.376529in}}%
\pgfpathcurveto{\pgfqpoint{4.077345in}{1.369396in}}{\pgfqpoint{4.080179in}{1.362555in}}{\pgfqpoint{4.085222in}{1.357511in}}%
\pgfpathcurveto{\pgfqpoint{4.090266in}{1.352467in}}{\pgfqpoint{4.097108in}{1.349633in}}{\pgfqpoint{4.104240in}{1.349633in}}%
\pgfpathclose%
\pgfusepath{stroke,fill}%
\end{pgfscope}%
\begin{pgfscope}%
\pgfpathrectangle{\pgfqpoint{2.867647in}{0.500000in}}{\pgfqpoint{1.764706in}{1.700000in}}%
\pgfusepath{clip}%
\pgfsetbuttcap%
\pgfsetroundjoin%
\definecolor{currentfill}{rgb}{0.965753,0.732351,0.592427}%
\pgfsetfillcolor{currentfill}%
\pgfsetlinewidth{0.311001pt}%
\definecolor{currentstroke}{rgb}{1.000000,1.000000,1.000000}%
\pgfsetstrokecolor{currentstroke}%
\pgfsetdash{}{0pt}%
\pgfpathmoveto{\pgfqpoint{4.242491in}{1.047965in}}%
\pgfpathcurveto{\pgfqpoint{4.249624in}{1.047965in}}{\pgfqpoint{4.256465in}{1.050798in}}{\pgfqpoint{4.261509in}{1.055842in}}%
\pgfpathcurveto{\pgfqpoint{4.266553in}{1.060886in}}{\pgfqpoint{4.269387in}{1.067727in}}{\pgfqpoint{4.269387in}{1.074860in}}%
\pgfpathcurveto{\pgfqpoint{4.269387in}{1.081993in}}{\pgfqpoint{4.266553in}{1.088835in}}{\pgfqpoint{4.261509in}{1.093878in}}%
\pgfpathcurveto{\pgfqpoint{4.256465in}{1.098922in}}{\pgfqpoint{4.249624in}{1.101756in}}{\pgfqpoint{4.242491in}{1.101756in}}%
\pgfpathcurveto{\pgfqpoint{4.235358in}{1.101756in}}{\pgfqpoint{4.228517in}{1.098922in}}{\pgfqpoint{4.223473in}{1.093878in}}%
\pgfpathcurveto{\pgfqpoint{4.218429in}{1.088835in}}{\pgfqpoint{4.215595in}{1.081993in}}{\pgfqpoint{4.215595in}{1.074860in}}%
\pgfpathcurveto{\pgfqpoint{4.215595in}{1.067727in}}{\pgfqpoint{4.218429in}{1.060886in}}{\pgfqpoint{4.223473in}{1.055842in}}%
\pgfpathcurveto{\pgfqpoint{4.228517in}{1.050798in}}{\pgfqpoint{4.235358in}{1.047965in}}{\pgfqpoint{4.242491in}{1.047965in}}%
\pgfpathclose%
\pgfusepath{stroke,fill}%
\end{pgfscope}%
\begin{pgfscope}%
\pgfpathrectangle{\pgfqpoint{2.867647in}{0.500000in}}{\pgfqpoint{1.764706in}{1.700000in}}%
\pgfusepath{clip}%
\pgfsetbuttcap%
\pgfsetroundjoin%
\definecolor{currentfill}{rgb}{0.976961,0.885681,0.814303}%
\pgfsetfillcolor{currentfill}%
\pgfsetlinewidth{0.311001pt}%
\definecolor{currentstroke}{rgb}{1.000000,1.000000,1.000000}%
\pgfsetstrokecolor{currentstroke}%
\pgfsetdash{}{0pt}%
\pgfpathmoveto{\pgfqpoint{4.143092in}{1.318728in}}%
\pgfpathcurveto{\pgfqpoint{4.150225in}{1.318728in}}{\pgfqpoint{4.157066in}{1.321562in}}{\pgfqpoint{4.162110in}{1.326606in}}%
\pgfpathcurveto{\pgfqpoint{4.167154in}{1.331649in}}{\pgfqpoint{4.169987in}{1.338491in}}{\pgfqpoint{4.169987in}{1.345624in}}%
\pgfpathcurveto{\pgfqpoint{4.169987in}{1.352757in}}{\pgfqpoint{4.167154in}{1.359598in}}{\pgfqpoint{4.162110in}{1.364642in}}%
\pgfpathcurveto{\pgfqpoint{4.157066in}{1.369686in}}{\pgfqpoint{4.150225in}{1.372520in}}{\pgfqpoint{4.143092in}{1.372520in}}%
\pgfpathcurveto{\pgfqpoint{4.135959in}{1.372520in}}{\pgfqpoint{4.129117in}{1.369686in}}{\pgfqpoint{4.124074in}{1.364642in}}%
\pgfpathcurveto{\pgfqpoint{4.119030in}{1.359598in}}{\pgfqpoint{4.116196in}{1.352757in}}{\pgfqpoint{4.116196in}{1.345624in}}%
\pgfpathcurveto{\pgfqpoint{4.116196in}{1.338491in}}{\pgfqpoint{4.119030in}{1.331649in}}{\pgfqpoint{4.124074in}{1.326606in}}%
\pgfpathcurveto{\pgfqpoint{4.129117in}{1.321562in}}{\pgfqpoint{4.135959in}{1.318728in}}{\pgfqpoint{4.143092in}{1.318728in}}%
\pgfpathclose%
\pgfusepath{stroke,fill}%
\end{pgfscope}%
\begin{pgfscope}%
\pgfpathrectangle{\pgfqpoint{2.867647in}{0.500000in}}{\pgfqpoint{1.764706in}{1.700000in}}%
\pgfusepath{clip}%
\pgfsetbuttcap%
\pgfsetroundjoin%
\definecolor{currentfill}{rgb}{0.966120,0.744512,0.608720}%
\pgfsetfillcolor{currentfill}%
\pgfsetlinewidth{0.311001pt}%
\definecolor{currentstroke}{rgb}{1.000000,1.000000,1.000000}%
\pgfsetstrokecolor{currentstroke}%
\pgfsetdash{}{0pt}%
\pgfpathmoveto{\pgfqpoint{4.056381in}{0.909177in}}%
\pgfpathcurveto{\pgfqpoint{4.063514in}{0.909177in}}{\pgfqpoint{4.070355in}{0.912011in}}{\pgfqpoint{4.075399in}{0.917055in}}%
\pgfpathcurveto{\pgfqpoint{4.080443in}{0.922098in}}{\pgfqpoint{4.083276in}{0.928940in}}{\pgfqpoint{4.083276in}{0.936073in}}%
\pgfpathcurveto{\pgfqpoint{4.083276in}{0.943206in}}{\pgfqpoint{4.080443in}{0.950047in}}{\pgfqpoint{4.075399in}{0.955091in}}%
\pgfpathcurveto{\pgfqpoint{4.070355in}{0.960135in}}{\pgfqpoint{4.063514in}{0.962969in}}{\pgfqpoint{4.056381in}{0.962969in}}%
\pgfpathcurveto{\pgfqpoint{4.049248in}{0.962969in}}{\pgfqpoint{4.042406in}{0.960135in}}{\pgfqpoint{4.037363in}{0.955091in}}%
\pgfpathcurveto{\pgfqpoint{4.032319in}{0.950047in}}{\pgfqpoint{4.029485in}{0.943206in}}{\pgfqpoint{4.029485in}{0.936073in}}%
\pgfpathcurveto{\pgfqpoint{4.029485in}{0.928940in}}{\pgfqpoint{4.032319in}{0.922098in}}{\pgfqpoint{4.037363in}{0.917055in}}%
\pgfpathcurveto{\pgfqpoint{4.042406in}{0.912011in}}{\pgfqpoint{4.049248in}{0.909177in}}{\pgfqpoint{4.056381in}{0.909177in}}%
\pgfpathclose%
\pgfusepath{stroke,fill}%
\end{pgfscope}%
\begin{pgfscope}%
\pgfpathrectangle{\pgfqpoint{2.867647in}{0.500000in}}{\pgfqpoint{1.764706in}{1.700000in}}%
\pgfusepath{clip}%
\pgfsetbuttcap%
\pgfsetroundjoin%
\definecolor{currentfill}{rgb}{0.894903,0.217856,0.253144}%
\pgfsetfillcolor{currentfill}%
\pgfsetlinewidth{0.311001pt}%
\definecolor{currentstroke}{rgb}{1.000000,1.000000,1.000000}%
\pgfsetstrokecolor{currentstroke}%
\pgfsetdash{}{0pt}%
\pgfpathmoveto{\pgfqpoint{3.762871in}{1.778546in}}%
\pgfpathcurveto{\pgfqpoint{3.770004in}{1.778546in}}{\pgfqpoint{3.776846in}{1.781380in}}{\pgfqpoint{3.781890in}{1.786423in}}%
\pgfpathcurveto{\pgfqpoint{3.786933in}{1.791467in}}{\pgfqpoint{3.789767in}{1.798309in}}{\pgfqpoint{3.789767in}{1.805441in}}%
\pgfpathcurveto{\pgfqpoint{3.789767in}{1.812574in}}{\pgfqpoint{3.786933in}{1.819416in}}{\pgfqpoint{3.781890in}{1.824460in}}%
\pgfpathcurveto{\pgfqpoint{3.776846in}{1.829503in}}{\pgfqpoint{3.770004in}{1.832337in}}{\pgfqpoint{3.762871in}{1.832337in}}%
\pgfpathcurveto{\pgfqpoint{3.755739in}{1.832337in}}{\pgfqpoint{3.748897in}{1.829503in}}{\pgfqpoint{3.743853in}{1.824460in}}%
\pgfpathcurveto{\pgfqpoint{3.738810in}{1.819416in}}{\pgfqpoint{3.735976in}{1.812574in}}{\pgfqpoint{3.735976in}{1.805441in}}%
\pgfpathcurveto{\pgfqpoint{3.735976in}{1.798309in}}{\pgfqpoint{3.738810in}{1.791467in}}{\pgfqpoint{3.743853in}{1.786423in}}%
\pgfpathcurveto{\pgfqpoint{3.748897in}{1.781380in}}{\pgfqpoint{3.755739in}{1.778546in}}{\pgfqpoint{3.762871in}{1.778546in}}%
\pgfpathclose%
\pgfusepath{stroke,fill}%
\end{pgfscope}%
\begin{pgfscope}%
\pgfpathrectangle{\pgfqpoint{2.867647in}{0.500000in}}{\pgfqpoint{1.764706in}{1.700000in}}%
\pgfusepath{clip}%
\pgfsetbuttcap%
\pgfsetroundjoin%
\definecolor{currentfill}{rgb}{0.966120,0.744512,0.608720}%
\pgfsetfillcolor{currentfill}%
\pgfsetlinewidth{0.311001pt}%
\definecolor{currentstroke}{rgb}{1.000000,1.000000,1.000000}%
\pgfsetstrokecolor{currentstroke}%
\pgfsetdash{}{0pt}%
\pgfpathmoveto{\pgfqpoint{4.021282in}{1.720324in}}%
\pgfpathcurveto{\pgfqpoint{4.028415in}{1.720324in}}{\pgfqpoint{4.035257in}{1.723157in}}{\pgfqpoint{4.040300in}{1.728201in}}%
\pgfpathcurveto{\pgfqpoint{4.045344in}{1.733245in}}{\pgfqpoint{4.048178in}{1.740086in}}{\pgfqpoint{4.048178in}{1.747219in}}%
\pgfpathcurveto{\pgfqpoint{4.048178in}{1.754352in}}{\pgfqpoint{4.045344in}{1.761194in}}{\pgfqpoint{4.040300in}{1.766237in}}%
\pgfpathcurveto{\pgfqpoint{4.035257in}{1.771281in}}{\pgfqpoint{4.028415in}{1.774115in}}{\pgfqpoint{4.021282in}{1.774115in}}%
\pgfpathcurveto{\pgfqpoint{4.014149in}{1.774115in}}{\pgfqpoint{4.007308in}{1.771281in}}{\pgfqpoint{4.002264in}{1.766237in}}%
\pgfpathcurveto{\pgfqpoint{3.997220in}{1.761194in}}{\pgfqpoint{3.994386in}{1.754352in}}{\pgfqpoint{3.994386in}{1.747219in}}%
\pgfpathcurveto{\pgfqpoint{3.994386in}{1.740086in}}{\pgfqpoint{3.997220in}{1.733245in}}{\pgfqpoint{4.002264in}{1.728201in}}%
\pgfpathcurveto{\pgfqpoint{4.007308in}{1.723157in}}{\pgfqpoint{4.014149in}{1.720324in}}{\pgfqpoint{4.021282in}{1.720324in}}%
\pgfpathclose%
\pgfusepath{stroke,fill}%
\end{pgfscope}%
\begin{pgfscope}%
\pgfpathrectangle{\pgfqpoint{2.867647in}{0.500000in}}{\pgfqpoint{1.764706in}{1.700000in}}%
\pgfusepath{clip}%
\pgfsetbuttcap%
\pgfsetroundjoin%
\definecolor{currentfill}{rgb}{0.965042,0.701564,0.552889}%
\pgfsetfillcolor{currentfill}%
\pgfsetlinewidth{0.311001pt}%
\definecolor{currentstroke}{rgb}{1.000000,1.000000,1.000000}%
\pgfsetstrokecolor{currentstroke}%
\pgfsetdash{}{0pt}%
\pgfpathmoveto{\pgfqpoint{4.230850in}{1.010230in}}%
\pgfpathcurveto{\pgfqpoint{4.237983in}{1.010230in}}{\pgfqpoint{4.244824in}{1.013064in}}{\pgfqpoint{4.249868in}{1.018108in}}%
\pgfpathcurveto{\pgfqpoint{4.254912in}{1.023151in}}{\pgfqpoint{4.257745in}{1.029993in}}{\pgfqpoint{4.257745in}{1.037126in}}%
\pgfpathcurveto{\pgfqpoint{4.257745in}{1.044259in}}{\pgfqpoint{4.254912in}{1.051100in}}{\pgfqpoint{4.249868in}{1.056144in}}%
\pgfpathcurveto{\pgfqpoint{4.244824in}{1.061187in}}{\pgfqpoint{4.237983in}{1.064021in}}{\pgfqpoint{4.230850in}{1.064021in}}%
\pgfpathcurveto{\pgfqpoint{4.223717in}{1.064021in}}{\pgfqpoint{4.216875in}{1.061187in}}{\pgfqpoint{4.211832in}{1.056144in}}%
\pgfpathcurveto{\pgfqpoint{4.206788in}{1.051100in}}{\pgfqpoint{4.203954in}{1.044259in}}{\pgfqpoint{4.203954in}{1.037126in}}%
\pgfpathcurveto{\pgfqpoint{4.203954in}{1.029993in}}{\pgfqpoint{4.206788in}{1.023151in}}{\pgfqpoint{4.211832in}{1.018108in}}%
\pgfpathcurveto{\pgfqpoint{4.216875in}{1.013064in}}{\pgfqpoint{4.223717in}{1.010230in}}{\pgfqpoint{4.230850in}{1.010230in}}%
\pgfpathclose%
\pgfusepath{stroke,fill}%
\end{pgfscope}%
\begin{pgfscope}%
\pgfpathrectangle{\pgfqpoint{2.867647in}{0.500000in}}{\pgfqpoint{1.764706in}{1.700000in}}%
\pgfusepath{clip}%
\pgfsetbuttcap%
\pgfsetroundjoin%
\definecolor{currentfill}{rgb}{0.976961,0.885681,0.814303}%
\pgfsetfillcolor{currentfill}%
\pgfsetlinewidth{0.311001pt}%
\definecolor{currentstroke}{rgb}{1.000000,1.000000,1.000000}%
\pgfsetstrokecolor{currentstroke}%
\pgfsetdash{}{0pt}%
\pgfpathmoveto{\pgfqpoint{4.144826in}{1.333771in}}%
\pgfpathcurveto{\pgfqpoint{4.151958in}{1.333771in}}{\pgfqpoint{4.158800in}{1.336605in}}{\pgfqpoint{4.163844in}{1.341649in}}%
\pgfpathcurveto{\pgfqpoint{4.168887in}{1.346692in}}{\pgfqpoint{4.171721in}{1.353534in}}{\pgfqpoint{4.171721in}{1.360667in}}%
\pgfpathcurveto{\pgfqpoint{4.171721in}{1.367800in}}{\pgfqpoint{4.168887in}{1.374641in}}{\pgfqpoint{4.163844in}{1.379685in}}%
\pgfpathcurveto{\pgfqpoint{4.158800in}{1.384729in}}{\pgfqpoint{4.151958in}{1.387562in}}{\pgfqpoint{4.144826in}{1.387562in}}%
\pgfpathcurveto{\pgfqpoint{4.137693in}{1.387562in}}{\pgfqpoint{4.130851in}{1.384729in}}{\pgfqpoint{4.125807in}{1.379685in}}%
\pgfpathcurveto{\pgfqpoint{4.120764in}{1.374641in}}{\pgfqpoint{4.117930in}{1.367800in}}{\pgfqpoint{4.117930in}{1.360667in}}%
\pgfpathcurveto{\pgfqpoint{4.117930in}{1.353534in}}{\pgfqpoint{4.120764in}{1.346692in}}{\pgfqpoint{4.125807in}{1.341649in}}%
\pgfpathcurveto{\pgfqpoint{4.130851in}{1.336605in}}{\pgfqpoint{4.137693in}{1.333771in}}{\pgfqpoint{4.144826in}{1.333771in}}%
\pgfpathclose%
\pgfusepath{stroke,fill}%
\end{pgfscope}%
\begin{pgfscope}%
\pgfpathrectangle{\pgfqpoint{2.867647in}{0.500000in}}{\pgfqpoint{1.764706in}{1.700000in}}%
\pgfusepath{clip}%
\pgfsetbuttcap%
\pgfsetroundjoin%
\definecolor{currentfill}{rgb}{0.962283,0.593046,0.431453}%
\pgfsetfillcolor{currentfill}%
\pgfsetlinewidth{0.311001pt}%
\definecolor{currentstroke}{rgb}{1.000000,1.000000,1.000000}%
\pgfsetstrokecolor{currentstroke}%
\pgfsetdash{}{0pt}%
\pgfpathmoveto{\pgfqpoint{4.166609in}{1.747669in}}%
\pgfpathcurveto{\pgfqpoint{4.173742in}{1.747669in}}{\pgfqpoint{4.180584in}{1.750503in}}{\pgfqpoint{4.185627in}{1.755547in}}%
\pgfpathcurveto{\pgfqpoint{4.190671in}{1.760591in}}{\pgfqpoint{4.193505in}{1.767432in}}{\pgfqpoint{4.193505in}{1.774565in}}%
\pgfpathcurveto{\pgfqpoint{4.193505in}{1.781698in}}{\pgfqpoint{4.190671in}{1.788540in}}{\pgfqpoint{4.185627in}{1.793583in}}%
\pgfpathcurveto{\pgfqpoint{4.180584in}{1.798627in}}{\pgfqpoint{4.173742in}{1.801461in}}{\pgfqpoint{4.166609in}{1.801461in}}%
\pgfpathcurveto{\pgfqpoint{4.159476in}{1.801461in}}{\pgfqpoint{4.152635in}{1.798627in}}{\pgfqpoint{4.147591in}{1.793583in}}%
\pgfpathcurveto{\pgfqpoint{4.142547in}{1.788540in}}{\pgfqpoint{4.139713in}{1.781698in}}{\pgfqpoint{4.139713in}{1.774565in}}%
\pgfpathcurveto{\pgfqpoint{4.139713in}{1.767432in}}{\pgfqpoint{4.142547in}{1.760591in}}{\pgfqpoint{4.147591in}{1.755547in}}%
\pgfpathcurveto{\pgfqpoint{4.152635in}{1.750503in}}{\pgfqpoint{4.159476in}{1.747669in}}{\pgfqpoint{4.166609in}{1.747669in}}%
\pgfpathclose%
\pgfusepath{stroke,fill}%
\end{pgfscope}%
\begin{pgfscope}%
\pgfpathrectangle{\pgfqpoint{2.867647in}{0.500000in}}{\pgfqpoint{1.764706in}{1.700000in}}%
\pgfusepath{clip}%
\pgfsetbuttcap%
\pgfsetroundjoin%
\definecolor{currentfill}{rgb}{0.905301,0.238545,0.247481}%
\pgfsetfillcolor{currentfill}%
\pgfsetlinewidth{0.311001pt}%
\definecolor{currentstroke}{rgb}{1.000000,1.000000,1.000000}%
\pgfsetstrokecolor{currentstroke}%
\pgfsetdash{}{0pt}%
\pgfpathmoveto{\pgfqpoint{3.882550in}{1.039157in}}%
\pgfpathcurveto{\pgfqpoint{3.889682in}{1.039157in}}{\pgfqpoint{3.896524in}{1.041991in}}{\pgfqpoint{3.901568in}{1.047035in}}%
\pgfpathcurveto{\pgfqpoint{3.906611in}{1.052078in}}{\pgfqpoint{3.909445in}{1.058920in}}{\pgfqpoint{3.909445in}{1.066053in}}%
\pgfpathcurveto{\pgfqpoint{3.909445in}{1.073186in}}{\pgfqpoint{3.906611in}{1.080027in}}{\pgfqpoint{3.901568in}{1.085071in}}%
\pgfpathcurveto{\pgfqpoint{3.896524in}{1.090115in}}{\pgfqpoint{3.889682in}{1.092949in}}{\pgfqpoint{3.882550in}{1.092949in}}%
\pgfpathcurveto{\pgfqpoint{3.875417in}{1.092949in}}{\pgfqpoint{3.868575in}{1.090115in}}{\pgfqpoint{3.863531in}{1.085071in}}%
\pgfpathcurveto{\pgfqpoint{3.858488in}{1.080027in}}{\pgfqpoint{3.855654in}{1.073186in}}{\pgfqpoint{3.855654in}{1.066053in}}%
\pgfpathcurveto{\pgfqpoint{3.855654in}{1.058920in}}{\pgfqpoint{3.858488in}{1.052078in}}{\pgfqpoint{3.863531in}{1.047035in}}%
\pgfpathcurveto{\pgfqpoint{3.868575in}{1.041991in}}{\pgfqpoint{3.875417in}{1.039157in}}{\pgfqpoint{3.882550in}{1.039157in}}%
\pgfpathclose%
\pgfusepath{stroke,fill}%
\end{pgfscope}%
\begin{pgfscope}%
\pgfpathrectangle{\pgfqpoint{2.867647in}{0.500000in}}{\pgfqpoint{1.764706in}{1.700000in}}%
\pgfusepath{clip}%
\pgfsetbuttcap%
\pgfsetroundjoin%
\definecolor{currentfill}{rgb}{0.980678,0.914765,0.856766}%
\pgfsetfillcolor{currentfill}%
\pgfsetlinewidth{0.311001pt}%
\definecolor{currentstroke}{rgb}{1.000000,1.000000,1.000000}%
\pgfsetstrokecolor{currentstroke}%
\pgfsetdash{}{0pt}%
\pgfpathmoveto{\pgfqpoint{4.173083in}{1.437238in}}%
\pgfpathcurveto{\pgfqpoint{4.180215in}{1.437238in}}{\pgfqpoint{4.187057in}{1.440072in}}{\pgfqpoint{4.192101in}{1.445115in}}%
\pgfpathcurveto{\pgfqpoint{4.197144in}{1.450159in}}{\pgfqpoint{4.199978in}{1.457001in}}{\pgfqpoint{4.199978in}{1.464133in}}%
\pgfpathcurveto{\pgfqpoint{4.199978in}{1.471266in}}{\pgfqpoint{4.197144in}{1.478108in}}{\pgfqpoint{4.192101in}{1.483151in}}%
\pgfpathcurveto{\pgfqpoint{4.187057in}{1.488195in}}{\pgfqpoint{4.180215in}{1.491029in}}{\pgfqpoint{4.173083in}{1.491029in}}%
\pgfpathcurveto{\pgfqpoint{4.165950in}{1.491029in}}{\pgfqpoint{4.159108in}{1.488195in}}{\pgfqpoint{4.154064in}{1.483151in}}%
\pgfpathcurveto{\pgfqpoint{4.149021in}{1.478108in}}{\pgfqpoint{4.146187in}{1.471266in}}{\pgfqpoint{4.146187in}{1.464133in}}%
\pgfpathcurveto{\pgfqpoint{4.146187in}{1.457001in}}{\pgfqpoint{4.149021in}{1.450159in}}{\pgfqpoint{4.154064in}{1.445115in}}%
\pgfpathcurveto{\pgfqpoint{4.159108in}{1.440072in}}{\pgfqpoint{4.165950in}{1.437238in}}{\pgfqpoint{4.173083in}{1.437238in}}%
\pgfpathclose%
\pgfusepath{stroke,fill}%
\end{pgfscope}%
\begin{pgfscope}%
\pgfpathrectangle{\pgfqpoint{2.867647in}{0.500000in}}{\pgfqpoint{1.764706in}{1.700000in}}%
\pgfusepath{clip}%
\pgfsetbuttcap%
\pgfsetroundjoin%
\definecolor{currentfill}{rgb}{0.979124,0.903132,0.839793}%
\pgfsetfillcolor{currentfill}%
\pgfsetlinewidth{0.311001pt}%
\definecolor{currentstroke}{rgb}{1.000000,1.000000,1.000000}%
\pgfsetstrokecolor{currentstroke}%
\pgfsetdash{}{0pt}%
\pgfpathmoveto{\pgfqpoint{4.142209in}{1.205379in}}%
\pgfpathcurveto{\pgfqpoint{4.149342in}{1.205379in}}{\pgfqpoint{4.156183in}{1.208213in}}{\pgfqpoint{4.161227in}{1.213256in}}%
\pgfpathcurveto{\pgfqpoint{4.166271in}{1.218300in}}{\pgfqpoint{4.169105in}{1.225141in}}{\pgfqpoint{4.169105in}{1.232274in}}%
\pgfpathcurveto{\pgfqpoint{4.169105in}{1.239407in}}{\pgfqpoint{4.166271in}{1.246249in}}{\pgfqpoint{4.161227in}{1.251292in}}%
\pgfpathcurveto{\pgfqpoint{4.156183in}{1.256336in}}{\pgfqpoint{4.149342in}{1.259170in}}{\pgfqpoint{4.142209in}{1.259170in}}%
\pgfpathcurveto{\pgfqpoint{4.135076in}{1.259170in}}{\pgfqpoint{4.128234in}{1.256336in}}{\pgfqpoint{4.123191in}{1.251292in}}%
\pgfpathcurveto{\pgfqpoint{4.118147in}{1.246249in}}{\pgfqpoint{4.115313in}{1.239407in}}{\pgfqpoint{4.115313in}{1.232274in}}%
\pgfpathcurveto{\pgfqpoint{4.115313in}{1.225141in}}{\pgfqpoint{4.118147in}{1.218300in}}{\pgfqpoint{4.123191in}{1.213256in}}%
\pgfpathcurveto{\pgfqpoint{4.128234in}{1.208213in}}{\pgfqpoint{4.135076in}{1.205379in}}{\pgfqpoint{4.142209in}{1.205379in}}%
\pgfpathclose%
\pgfusepath{stroke,fill}%
\end{pgfscope}%
\begin{pgfscope}%
\pgfpathrectangle{\pgfqpoint{2.867647in}{0.500000in}}{\pgfqpoint{1.764706in}{1.700000in}}%
\pgfusepath{clip}%
\pgfsetbuttcap%
\pgfsetroundjoin%
\definecolor{currentfill}{rgb}{0.967092,0.768560,0.642079}%
\pgfsetfillcolor{currentfill}%
\pgfsetlinewidth{0.311001pt}%
\definecolor{currentstroke}{rgb}{1.000000,1.000000,1.000000}%
\pgfsetstrokecolor{currentstroke}%
\pgfsetdash{}{0pt}%
\pgfpathmoveto{\pgfqpoint{4.039232in}{1.080661in}}%
\pgfpathcurveto{\pgfqpoint{4.046364in}{1.080661in}}{\pgfqpoint{4.053206in}{1.083495in}}{\pgfqpoint{4.058250in}{1.088538in}}%
\pgfpathcurveto{\pgfqpoint{4.063293in}{1.093582in}}{\pgfqpoint{4.066127in}{1.100424in}}{\pgfqpoint{4.066127in}{1.107557in}}%
\pgfpathcurveto{\pgfqpoint{4.066127in}{1.114689in}}{\pgfqpoint{4.063293in}{1.121531in}}{\pgfqpoint{4.058250in}{1.126575in}}%
\pgfpathcurveto{\pgfqpoint{4.053206in}{1.131618in}}{\pgfqpoint{4.046364in}{1.134452in}}{\pgfqpoint{4.039232in}{1.134452in}}%
\pgfpathcurveto{\pgfqpoint{4.032099in}{1.134452in}}{\pgfqpoint{4.025257in}{1.131618in}}{\pgfqpoint{4.020213in}{1.126575in}}%
\pgfpathcurveto{\pgfqpoint{4.015170in}{1.121531in}}{\pgfqpoint{4.012336in}{1.114689in}}{\pgfqpoint{4.012336in}{1.107557in}}%
\pgfpathcurveto{\pgfqpoint{4.012336in}{1.100424in}}{\pgfqpoint{4.015170in}{1.093582in}}{\pgfqpoint{4.020213in}{1.088538in}}%
\pgfpathcurveto{\pgfqpoint{4.025257in}{1.083495in}}{\pgfqpoint{4.032099in}{1.080661in}}{\pgfqpoint{4.039232in}{1.080661in}}%
\pgfpathclose%
\pgfusepath{stroke,fill}%
\end{pgfscope}%
\begin{pgfscope}%
\pgfpathrectangle{\pgfqpoint{2.867647in}{0.500000in}}{\pgfqpoint{1.764706in}{1.700000in}}%
\pgfusepath{clip}%
\pgfsetbuttcap%
\pgfsetroundjoin%
\definecolor{currentfill}{rgb}{0.970718,0.821518,0.719872}%
\pgfsetfillcolor{currentfill}%
\pgfsetlinewidth{0.311001pt}%
\definecolor{currentstroke}{rgb}{1.000000,1.000000,1.000000}%
\pgfsetstrokecolor{currentstroke}%
\pgfsetdash{}{0pt}%
\pgfpathmoveto{\pgfqpoint{4.068978in}{1.517106in}}%
\pgfpathcurveto{\pgfqpoint{4.076111in}{1.517106in}}{\pgfqpoint{4.082953in}{1.519940in}}{\pgfqpoint{4.087996in}{1.524984in}}%
\pgfpathcurveto{\pgfqpoint{4.093040in}{1.530027in}}{\pgfqpoint{4.095874in}{1.536869in}}{\pgfqpoint{4.095874in}{1.544002in}}%
\pgfpathcurveto{\pgfqpoint{4.095874in}{1.551135in}}{\pgfqpoint{4.093040in}{1.557976in}}{\pgfqpoint{4.087996in}{1.563020in}}%
\pgfpathcurveto{\pgfqpoint{4.082953in}{1.568064in}}{\pgfqpoint{4.076111in}{1.570897in}}{\pgfqpoint{4.068978in}{1.570897in}}%
\pgfpathcurveto{\pgfqpoint{4.061845in}{1.570897in}}{\pgfqpoint{4.055004in}{1.568064in}}{\pgfqpoint{4.049960in}{1.563020in}}%
\pgfpathcurveto{\pgfqpoint{4.044916in}{1.557976in}}{\pgfqpoint{4.042082in}{1.551135in}}{\pgfqpoint{4.042082in}{1.544002in}}%
\pgfpathcurveto{\pgfqpoint{4.042082in}{1.536869in}}{\pgfqpoint{4.044916in}{1.530027in}}{\pgfqpoint{4.049960in}{1.524984in}}%
\pgfpathcurveto{\pgfqpoint{4.055004in}{1.519940in}}{\pgfqpoint{4.061845in}{1.517106in}}{\pgfqpoint{4.068978in}{1.517106in}}%
\pgfpathclose%
\pgfusepath{stroke,fill}%
\end{pgfscope}%
\begin{pgfscope}%
\pgfpathrectangle{\pgfqpoint{2.867647in}{0.500000in}}{\pgfqpoint{1.764706in}{1.700000in}}%
\pgfusepath{clip}%
\pgfsetbuttcap%
\pgfsetroundjoin%
\definecolor{currentfill}{rgb}{0.979891,0.908948,0.848279}%
\pgfsetfillcolor{currentfill}%
\pgfsetlinewidth{0.311001pt}%
\definecolor{currentstroke}{rgb}{1.000000,1.000000,1.000000}%
\pgfsetstrokecolor{currentstroke}%
\pgfsetdash{}{0pt}%
\pgfpathmoveto{\pgfqpoint{4.162157in}{1.113177in}}%
\pgfpathcurveto{\pgfqpoint{4.169290in}{1.113177in}}{\pgfqpoint{4.176132in}{1.116011in}}{\pgfqpoint{4.181175in}{1.121054in}}%
\pgfpathcurveto{\pgfqpoint{4.186219in}{1.126098in}}{\pgfqpoint{4.189053in}{1.132940in}}{\pgfqpoint{4.189053in}{1.140073in}}%
\pgfpathcurveto{\pgfqpoint{4.189053in}{1.147205in}}{\pgfqpoint{4.186219in}{1.154047in}}{\pgfqpoint{4.181175in}{1.159091in}}%
\pgfpathcurveto{\pgfqpoint{4.176132in}{1.164134in}}{\pgfqpoint{4.169290in}{1.166968in}}{\pgfqpoint{4.162157in}{1.166968in}}%
\pgfpathcurveto{\pgfqpoint{4.155025in}{1.166968in}}{\pgfqpoint{4.148183in}{1.164134in}}{\pgfqpoint{4.143139in}{1.159091in}}%
\pgfpathcurveto{\pgfqpoint{4.138096in}{1.154047in}}{\pgfqpoint{4.135262in}{1.147205in}}{\pgfqpoint{4.135262in}{1.140073in}}%
\pgfpathcurveto{\pgfqpoint{4.135262in}{1.132940in}}{\pgfqpoint{4.138096in}{1.126098in}}{\pgfqpoint{4.143139in}{1.121054in}}%
\pgfpathcurveto{\pgfqpoint{4.148183in}{1.116011in}}{\pgfqpoint{4.155025in}{1.113177in}}{\pgfqpoint{4.162157in}{1.113177in}}%
\pgfpathclose%
\pgfusepath{stroke,fill}%
\end{pgfscope}%
\begin{pgfscope}%
\pgfpathrectangle{\pgfqpoint{2.867647in}{0.500000in}}{\pgfqpoint{1.764706in}{1.700000in}}%
\pgfusepath{clip}%
\pgfsetbuttcap%
\pgfsetroundjoin%
\definecolor{currentfill}{rgb}{0.960778,0.559972,0.399412}%
\pgfsetfillcolor{currentfill}%
\pgfsetlinewidth{0.311001pt}%
\definecolor{currentstroke}{rgb}{1.000000,1.000000,1.000000}%
\pgfsetstrokecolor{currentstroke}%
\pgfsetdash{}{0pt}%
\pgfpathmoveto{\pgfqpoint{3.928198in}{1.666901in}}%
\pgfpathcurveto{\pgfqpoint{3.935331in}{1.666901in}}{\pgfqpoint{3.942173in}{1.669735in}}{\pgfqpoint{3.947216in}{1.674779in}}%
\pgfpathcurveto{\pgfqpoint{3.952260in}{1.679823in}}{\pgfqpoint{3.955094in}{1.686664in}}{\pgfqpoint{3.955094in}{1.693797in}}%
\pgfpathcurveto{\pgfqpoint{3.955094in}{1.700930in}}{\pgfqpoint{3.952260in}{1.707772in}}{\pgfqpoint{3.947216in}{1.712815in}}%
\pgfpathcurveto{\pgfqpoint{3.942173in}{1.717859in}}{\pgfqpoint{3.935331in}{1.720693in}}{\pgfqpoint{3.928198in}{1.720693in}}%
\pgfpathcurveto{\pgfqpoint{3.921065in}{1.720693in}}{\pgfqpoint{3.914224in}{1.717859in}}{\pgfqpoint{3.909180in}{1.712815in}}%
\pgfpathcurveto{\pgfqpoint{3.904136in}{1.707772in}}{\pgfqpoint{3.901302in}{1.700930in}}{\pgfqpoint{3.901302in}{1.693797in}}%
\pgfpathcurveto{\pgfqpoint{3.901302in}{1.686664in}}{\pgfqpoint{3.904136in}{1.679823in}}{\pgfqpoint{3.909180in}{1.674779in}}%
\pgfpathcurveto{\pgfqpoint{3.914224in}{1.669735in}}{\pgfqpoint{3.921065in}{1.666901in}}{\pgfqpoint{3.928198in}{1.666901in}}%
\pgfpathclose%
\pgfusepath{stroke,fill}%
\end{pgfscope}%
\begin{pgfscope}%
\pgfpathrectangle{\pgfqpoint{2.867647in}{0.500000in}}{\pgfqpoint{1.764706in}{1.700000in}}%
\pgfusepath{clip}%
\pgfsetbuttcap%
\pgfsetroundjoin%
\definecolor{currentfill}{rgb}{0.965592,0.726236,0.584384}%
\pgfsetfillcolor{currentfill}%
\pgfsetlinewidth{0.311001pt}%
\definecolor{currentstroke}{rgb}{1.000000,1.000000,1.000000}%
\pgfsetstrokecolor{currentstroke}%
\pgfsetdash{}{0pt}%
\pgfpathmoveto{\pgfqpoint{4.232371in}{1.026656in}}%
\pgfpathcurveto{\pgfqpoint{4.239504in}{1.026656in}}{\pgfqpoint{4.246346in}{1.029490in}}{\pgfqpoint{4.251389in}{1.034534in}}%
\pgfpathcurveto{\pgfqpoint{4.256433in}{1.039577in}}{\pgfqpoint{4.259267in}{1.046419in}}{\pgfqpoint{4.259267in}{1.053552in}}%
\pgfpathcurveto{\pgfqpoint{4.259267in}{1.060684in}}{\pgfqpoint{4.256433in}{1.067526in}}{\pgfqpoint{4.251389in}{1.072570in}}%
\pgfpathcurveto{\pgfqpoint{4.246346in}{1.077613in}}{\pgfqpoint{4.239504in}{1.080447in}}{\pgfqpoint{4.232371in}{1.080447in}}%
\pgfpathcurveto{\pgfqpoint{4.225238in}{1.080447in}}{\pgfqpoint{4.218397in}{1.077613in}}{\pgfqpoint{4.213353in}{1.072570in}}%
\pgfpathcurveto{\pgfqpoint{4.208309in}{1.067526in}}{\pgfqpoint{4.205475in}{1.060684in}}{\pgfqpoint{4.205475in}{1.053552in}}%
\pgfpathcurveto{\pgfqpoint{4.205475in}{1.046419in}}{\pgfqpoint{4.208309in}{1.039577in}}{\pgfqpoint{4.213353in}{1.034534in}}%
\pgfpathcurveto{\pgfqpoint{4.218397in}{1.029490in}}{\pgfqpoint{4.225238in}{1.026656in}}{\pgfqpoint{4.232371in}{1.026656in}}%
\pgfpathclose%
\pgfusepath{stroke,fill}%
\end{pgfscope}%
\begin{pgfscope}%
\pgfpathrectangle{\pgfqpoint{2.867647in}{0.500000in}}{\pgfqpoint{1.764706in}{1.700000in}}%
\pgfusepath{clip}%
\pgfsetbuttcap%
\pgfsetroundjoin%
\definecolor{currentfill}{rgb}{0.973271,0.850724,0.762998}%
\pgfsetfillcolor{currentfill}%
\pgfsetlinewidth{0.311001pt}%
\definecolor{currentstroke}{rgb}{1.000000,1.000000,1.000000}%
\pgfsetstrokecolor{currentstroke}%
\pgfsetdash{}{0pt}%
\pgfpathmoveto{\pgfqpoint{4.208002in}{1.074680in}}%
\pgfpathcurveto{\pgfqpoint{4.215135in}{1.074680in}}{\pgfqpoint{4.221977in}{1.077514in}}{\pgfqpoint{4.227021in}{1.082557in}}%
\pgfpathcurveto{\pgfqpoint{4.232064in}{1.087601in}}{\pgfqpoint{4.234898in}{1.094443in}}{\pgfqpoint{4.234898in}{1.101576in}}%
\pgfpathcurveto{\pgfqpoint{4.234898in}{1.108708in}}{\pgfqpoint{4.232064in}{1.115550in}}{\pgfqpoint{4.227021in}{1.120594in}}%
\pgfpathcurveto{\pgfqpoint{4.221977in}{1.125637in}}{\pgfqpoint{4.215135in}{1.128471in}}{\pgfqpoint{4.208002in}{1.128471in}}%
\pgfpathcurveto{\pgfqpoint{4.200870in}{1.128471in}}{\pgfqpoint{4.194028in}{1.125637in}}{\pgfqpoint{4.188984in}{1.120594in}}%
\pgfpathcurveto{\pgfqpoint{4.183941in}{1.115550in}}{\pgfqpoint{4.181107in}{1.108708in}}{\pgfqpoint{4.181107in}{1.101576in}}%
\pgfpathcurveto{\pgfqpoint{4.181107in}{1.094443in}}{\pgfqpoint{4.183941in}{1.087601in}}{\pgfqpoint{4.188984in}{1.082557in}}%
\pgfpathcurveto{\pgfqpoint{4.194028in}{1.077514in}}{\pgfqpoint{4.200870in}{1.074680in}}{\pgfqpoint{4.208002in}{1.074680in}}%
\pgfpathclose%
\pgfusepath{stroke,fill}%
\end{pgfscope}%
\begin{pgfscope}%
\pgfpathrectangle{\pgfqpoint{2.867647in}{0.500000in}}{\pgfqpoint{1.764706in}{1.700000in}}%
\pgfusepath{clip}%
\pgfsetbuttcap%
\pgfsetroundjoin%
\definecolor{currentfill}{rgb}{0.976961,0.885681,0.814303}%
\pgfsetfillcolor{currentfill}%
\pgfsetlinewidth{0.311001pt}%
\definecolor{currentstroke}{rgb}{1.000000,1.000000,1.000000}%
\pgfsetstrokecolor{currentstroke}%
\pgfsetdash{}{0pt}%
\pgfpathmoveto{\pgfqpoint{4.121043in}{1.042158in}}%
\pgfpathcurveto{\pgfqpoint{4.128176in}{1.042158in}}{\pgfqpoint{4.135018in}{1.044992in}}{\pgfqpoint{4.140061in}{1.050036in}}%
\pgfpathcurveto{\pgfqpoint{4.145105in}{1.055079in}}{\pgfqpoint{4.147939in}{1.061921in}}{\pgfqpoint{4.147939in}{1.069054in}}%
\pgfpathcurveto{\pgfqpoint{4.147939in}{1.076187in}}{\pgfqpoint{4.145105in}{1.083028in}}{\pgfqpoint{4.140061in}{1.088072in}}%
\pgfpathcurveto{\pgfqpoint{4.135018in}{1.093116in}}{\pgfqpoint{4.128176in}{1.095950in}}{\pgfqpoint{4.121043in}{1.095950in}}%
\pgfpathcurveto{\pgfqpoint{4.113910in}{1.095950in}}{\pgfqpoint{4.107069in}{1.093116in}}{\pgfqpoint{4.102025in}{1.088072in}}%
\pgfpathcurveto{\pgfqpoint{4.096981in}{1.083028in}}{\pgfqpoint{4.094148in}{1.076187in}}{\pgfqpoint{4.094148in}{1.069054in}}%
\pgfpathcurveto{\pgfqpoint{4.094148in}{1.061921in}}{\pgfqpoint{4.096981in}{1.055079in}}{\pgfqpoint{4.102025in}{1.050036in}}%
\pgfpathcurveto{\pgfqpoint{4.107069in}{1.044992in}}{\pgfqpoint{4.113910in}{1.042158in}}{\pgfqpoint{4.121043in}{1.042158in}}%
\pgfpathclose%
\pgfusepath{stroke,fill}%
\end{pgfscope}%
\begin{pgfscope}%
\pgfpathrectangle{\pgfqpoint{2.867647in}{0.500000in}}{\pgfqpoint{1.764706in}{1.700000in}}%
\pgfusepath{clip}%
\pgfsetbuttcap%
\pgfsetroundjoin%
\definecolor{currentfill}{rgb}{0.597702,0.106938,0.358380}%
\pgfsetfillcolor{currentfill}%
\pgfsetlinewidth{0.311001pt}%
\definecolor{currentstroke}{rgb}{1.000000,1.000000,1.000000}%
\pgfsetstrokecolor{currentstroke}%
\pgfsetdash{}{0pt}%
\pgfpathmoveto{\pgfqpoint{3.925434in}{1.260134in}}%
\pgfpathcurveto{\pgfqpoint{3.932566in}{1.260134in}}{\pgfqpoint{3.939408in}{1.262968in}}{\pgfqpoint{3.944452in}{1.268011in}}%
\pgfpathcurveto{\pgfqpoint{3.949495in}{1.273055in}}{\pgfqpoint{3.952329in}{1.279897in}}{\pgfqpoint{3.952329in}{1.287029in}}%
\pgfpathcurveto{\pgfqpoint{3.952329in}{1.294162in}}{\pgfqpoint{3.949495in}{1.301004in}}{\pgfqpoint{3.944452in}{1.306048in}}%
\pgfpathcurveto{\pgfqpoint{3.939408in}{1.311091in}}{\pgfqpoint{3.932566in}{1.313925in}}{\pgfqpoint{3.925434in}{1.313925in}}%
\pgfpathcurveto{\pgfqpoint{3.918301in}{1.313925in}}{\pgfqpoint{3.911459in}{1.311091in}}{\pgfqpoint{3.906415in}{1.306048in}}%
\pgfpathcurveto{\pgfqpoint{3.901372in}{1.301004in}}{\pgfqpoint{3.898538in}{1.294162in}}{\pgfqpoint{3.898538in}{1.287029in}}%
\pgfpathcurveto{\pgfqpoint{3.898538in}{1.279897in}}{\pgfqpoint{3.901372in}{1.273055in}}{\pgfqpoint{3.906415in}{1.268011in}}%
\pgfpathcurveto{\pgfqpoint{3.911459in}{1.262968in}}{\pgfqpoint{3.918301in}{1.260134in}}{\pgfqpoint{3.925434in}{1.260134in}}%
\pgfpathclose%
\pgfusepath{stroke,fill}%
\end{pgfscope}%
\begin{pgfscope}%
\pgfpathrectangle{\pgfqpoint{2.867647in}{0.500000in}}{\pgfqpoint{1.764706in}{1.700000in}}%
\pgfusepath{clip}%
\pgfsetbuttcap%
\pgfsetroundjoin%
\definecolor{currentfill}{rgb}{0.968931,0.798091,0.685123}%
\pgfsetfillcolor{currentfill}%
\pgfsetlinewidth{0.311001pt}%
\definecolor{currentstroke}{rgb}{1.000000,1.000000,1.000000}%
\pgfsetstrokecolor{currentstroke}%
\pgfsetdash{}{0pt}%
\pgfpathmoveto{\pgfqpoint{4.044886in}{1.564810in}}%
\pgfpathcurveto{\pgfqpoint{4.052019in}{1.564810in}}{\pgfqpoint{4.058860in}{1.567644in}}{\pgfqpoint{4.063904in}{1.572688in}}%
\pgfpathcurveto{\pgfqpoint{4.068948in}{1.577732in}}{\pgfqpoint{4.071782in}{1.584573in}}{\pgfqpoint{4.071782in}{1.591706in}}%
\pgfpathcurveto{\pgfqpoint{4.071782in}{1.598839in}}{\pgfqpoint{4.068948in}{1.605680in}}{\pgfqpoint{4.063904in}{1.610724in}}%
\pgfpathcurveto{\pgfqpoint{4.058860in}{1.615768in}}{\pgfqpoint{4.052019in}{1.618602in}}{\pgfqpoint{4.044886in}{1.618602in}}%
\pgfpathcurveto{\pgfqpoint{4.037753in}{1.618602in}}{\pgfqpoint{4.030911in}{1.615768in}}{\pgfqpoint{4.025868in}{1.610724in}}%
\pgfpathcurveto{\pgfqpoint{4.020824in}{1.605680in}}{\pgfqpoint{4.017990in}{1.598839in}}{\pgfqpoint{4.017990in}{1.591706in}}%
\pgfpathcurveto{\pgfqpoint{4.017990in}{1.584573in}}{\pgfqpoint{4.020824in}{1.577732in}}{\pgfqpoint{4.025868in}{1.572688in}}%
\pgfpathcurveto{\pgfqpoint{4.030911in}{1.567644in}}{\pgfqpoint{4.037753in}{1.564810in}}{\pgfqpoint{4.044886in}{1.564810in}}%
\pgfpathclose%
\pgfusepath{stroke,fill}%
\end{pgfscope}%
\begin{pgfscope}%
\pgfpathrectangle{\pgfqpoint{2.867647in}{0.500000in}}{\pgfqpoint{1.764706in}{1.700000in}}%
\pgfusepath{clip}%
\pgfsetbuttcap%
\pgfsetroundjoin%
\definecolor{currentfill}{rgb}{0.980678,0.914765,0.856766}%
\pgfsetfillcolor{currentfill}%
\pgfsetlinewidth{0.311001pt}%
\definecolor{currentstroke}{rgb}{1.000000,1.000000,1.000000}%
\pgfsetstrokecolor{currentstroke}%
\pgfsetdash{}{0pt}%
\pgfpathmoveto{\pgfqpoint{4.173889in}{1.332639in}}%
\pgfpathcurveto{\pgfqpoint{4.181022in}{1.332639in}}{\pgfqpoint{4.187864in}{1.335473in}}{\pgfqpoint{4.192908in}{1.340517in}}%
\pgfpathcurveto{\pgfqpoint{4.197951in}{1.345561in}}{\pgfqpoint{4.200785in}{1.352402in}}{\pgfqpoint{4.200785in}{1.359535in}}%
\pgfpathcurveto{\pgfqpoint{4.200785in}{1.366668in}}{\pgfqpoint{4.197951in}{1.373510in}}{\pgfqpoint{4.192908in}{1.378553in}}%
\pgfpathcurveto{\pgfqpoint{4.187864in}{1.383597in}}{\pgfqpoint{4.181022in}{1.386431in}}{\pgfqpoint{4.173889in}{1.386431in}}%
\pgfpathcurveto{\pgfqpoint{4.166757in}{1.386431in}}{\pgfqpoint{4.159915in}{1.383597in}}{\pgfqpoint{4.154871in}{1.378553in}}%
\pgfpathcurveto{\pgfqpoint{4.149828in}{1.373510in}}{\pgfqpoint{4.146994in}{1.366668in}}{\pgfqpoint{4.146994in}{1.359535in}}%
\pgfpathcurveto{\pgfqpoint{4.146994in}{1.352402in}}{\pgfqpoint{4.149828in}{1.345561in}}{\pgfqpoint{4.154871in}{1.340517in}}%
\pgfpathcurveto{\pgfqpoint{4.159915in}{1.335473in}}{\pgfqpoint{4.166757in}{1.332639in}}{\pgfqpoint{4.173889in}{1.332639in}}%
\pgfpathclose%
\pgfusepath{stroke,fill}%
\end{pgfscope}%
\begin{pgfscope}%
\pgfpathrectangle{\pgfqpoint{2.867647in}{0.500000in}}{\pgfqpoint{1.764706in}{1.700000in}}%
\pgfusepath{clip}%
\pgfsetbuttcap%
\pgfsetroundjoin%
\definecolor{currentfill}{rgb}{0.975018,0.868213,0.788710}%
\pgfsetfillcolor{currentfill}%
\pgfsetlinewidth{0.311001pt}%
\definecolor{currentstroke}{rgb}{1.000000,1.000000,1.000000}%
\pgfsetstrokecolor{currentstroke}%
\pgfsetdash{}{0pt}%
\pgfpathmoveto{\pgfqpoint{4.232987in}{1.449079in}}%
\pgfpathcurveto{\pgfqpoint{4.240120in}{1.449079in}}{\pgfqpoint{4.246962in}{1.451913in}}{\pgfqpoint{4.252005in}{1.456957in}}%
\pgfpathcurveto{\pgfqpoint{4.257049in}{1.462000in}}{\pgfqpoint{4.259883in}{1.468842in}}{\pgfqpoint{4.259883in}{1.475975in}}%
\pgfpathcurveto{\pgfqpoint{4.259883in}{1.483108in}}{\pgfqpoint{4.257049in}{1.489949in}}{\pgfqpoint{4.252005in}{1.494993in}}%
\pgfpathcurveto{\pgfqpoint{4.246962in}{1.500037in}}{\pgfqpoint{4.240120in}{1.502871in}}{\pgfqpoint{4.232987in}{1.502871in}}%
\pgfpathcurveto{\pgfqpoint{4.225854in}{1.502871in}}{\pgfqpoint{4.219013in}{1.500037in}}{\pgfqpoint{4.213969in}{1.494993in}}%
\pgfpathcurveto{\pgfqpoint{4.208925in}{1.489949in}}{\pgfqpoint{4.206091in}{1.483108in}}{\pgfqpoint{4.206091in}{1.475975in}}%
\pgfpathcurveto{\pgfqpoint{4.206091in}{1.468842in}}{\pgfqpoint{4.208925in}{1.462000in}}{\pgfqpoint{4.213969in}{1.456957in}}%
\pgfpathcurveto{\pgfqpoint{4.219013in}{1.451913in}}{\pgfqpoint{4.225854in}{1.449079in}}{\pgfqpoint{4.232987in}{1.449079in}}%
\pgfpathclose%
\pgfusepath{stroke,fill}%
\end{pgfscope}%
\begin{pgfscope}%
\pgfpathrectangle{\pgfqpoint{2.867647in}{0.500000in}}{\pgfqpoint{1.764706in}{1.700000in}}%
\pgfusepath{clip}%
\pgfsetbuttcap%
\pgfsetroundjoin%
\definecolor{currentfill}{rgb}{0.981377,0.920617,0.865369}%
\pgfsetfillcolor{currentfill}%
\pgfsetlinewidth{0.311001pt}%
\definecolor{currentstroke}{rgb}{1.000000,1.000000,1.000000}%
\pgfsetstrokecolor{currentstroke}%
\pgfsetdash{}{0pt}%
\pgfpathmoveto{\pgfqpoint{4.174691in}{1.174563in}}%
\pgfpathcurveto{\pgfqpoint{4.181824in}{1.174563in}}{\pgfqpoint{4.188666in}{1.177397in}}{\pgfqpoint{4.193709in}{1.182440in}}%
\pgfpathcurveto{\pgfqpoint{4.198753in}{1.187484in}}{\pgfqpoint{4.201587in}{1.194326in}}{\pgfqpoint{4.201587in}{1.201459in}}%
\pgfpathcurveto{\pgfqpoint{4.201587in}{1.208591in}}{\pgfqpoint{4.198753in}{1.215433in}}{\pgfqpoint{4.193709in}{1.220477in}}%
\pgfpathcurveto{\pgfqpoint{4.188666in}{1.225520in}}{\pgfqpoint{4.181824in}{1.228354in}}{\pgfqpoint{4.174691in}{1.228354in}}%
\pgfpathcurveto{\pgfqpoint{4.167559in}{1.228354in}}{\pgfqpoint{4.160717in}{1.225520in}}{\pgfqpoint{4.155673in}{1.220477in}}%
\pgfpathcurveto{\pgfqpoint{4.150630in}{1.215433in}}{\pgfqpoint{4.147796in}{1.208591in}}{\pgfqpoint{4.147796in}{1.201459in}}%
\pgfpathcurveto{\pgfqpoint{4.147796in}{1.194326in}}{\pgfqpoint{4.150630in}{1.187484in}}{\pgfqpoint{4.155673in}{1.182440in}}%
\pgfpathcurveto{\pgfqpoint{4.160717in}{1.177397in}}{\pgfqpoint{4.167559in}{1.174563in}}{\pgfqpoint{4.174691in}{1.174563in}}%
\pgfpathclose%
\pgfusepath{stroke,fill}%
\end{pgfscope}%
\begin{pgfscope}%
\pgfpathrectangle{\pgfqpoint{2.867647in}{0.500000in}}{\pgfqpoint{1.764706in}{1.700000in}}%
\pgfusepath{clip}%
\pgfsetbuttcap%
\pgfsetroundjoin%
\definecolor{currentfill}{rgb}{0.965440,0.720101,0.576404}%
\pgfsetfillcolor{currentfill}%
\pgfsetlinewidth{0.311001pt}%
\definecolor{currentstroke}{rgb}{1.000000,1.000000,1.000000}%
\pgfsetstrokecolor{currentstroke}%
\pgfsetdash{}{0pt}%
\pgfpathmoveto{\pgfqpoint{4.210412in}{1.652817in}}%
\pgfpathcurveto{\pgfqpoint{4.217545in}{1.652817in}}{\pgfqpoint{4.224386in}{1.655650in}}{\pgfqpoint{4.229430in}{1.660694in}}%
\pgfpathcurveto{\pgfqpoint{4.234474in}{1.665738in}}{\pgfqpoint{4.237307in}{1.672579in}}{\pgfqpoint{4.237307in}{1.679712in}}%
\pgfpathcurveto{\pgfqpoint{4.237307in}{1.686845in}}{\pgfqpoint{4.234474in}{1.693687in}}{\pgfqpoint{4.229430in}{1.698730in}}%
\pgfpathcurveto{\pgfqpoint{4.224386in}{1.703774in}}{\pgfqpoint{4.217545in}{1.706608in}}{\pgfqpoint{4.210412in}{1.706608in}}%
\pgfpathcurveto{\pgfqpoint{4.203279in}{1.706608in}}{\pgfqpoint{4.196437in}{1.703774in}}{\pgfqpoint{4.191394in}{1.698730in}}%
\pgfpathcurveto{\pgfqpoint{4.186350in}{1.693687in}}{\pgfqpoint{4.183516in}{1.686845in}}{\pgfqpoint{4.183516in}{1.679712in}}%
\pgfpathcurveto{\pgfqpoint{4.183516in}{1.672579in}}{\pgfqpoint{4.186350in}{1.665738in}}{\pgfqpoint{4.191394in}{1.660694in}}%
\pgfpathcurveto{\pgfqpoint{4.196437in}{1.655650in}}{\pgfqpoint{4.203279in}{1.652817in}}{\pgfqpoint{4.210412in}{1.652817in}}%
\pgfpathclose%
\pgfusepath{stroke,fill}%
\end{pgfscope}%
\begin{pgfscope}%
\pgfpathrectangle{\pgfqpoint{2.867647in}{0.500000in}}{\pgfqpoint{1.764706in}{1.700000in}}%
\pgfusepath{clip}%
\pgfsetbuttcap%
\pgfsetroundjoin%
\definecolor{currentfill}{rgb}{0.976287,0.879862,0.805788}%
\pgfsetfillcolor{currentfill}%
\pgfsetlinewidth{0.311001pt}%
\definecolor{currentstroke}{rgb}{1.000000,1.000000,1.000000}%
\pgfsetstrokecolor{currentstroke}%
\pgfsetdash{}{0pt}%
\pgfpathmoveto{\pgfqpoint{4.222784in}{1.470232in}}%
\pgfpathcurveto{\pgfqpoint{4.229916in}{1.470232in}}{\pgfqpoint{4.236758in}{1.473066in}}{\pgfqpoint{4.241802in}{1.478110in}}%
\pgfpathcurveto{\pgfqpoint{4.246845in}{1.483154in}}{\pgfqpoint{4.249679in}{1.489995in}}{\pgfqpoint{4.249679in}{1.497128in}}%
\pgfpathcurveto{\pgfqpoint{4.249679in}{1.504261in}}{\pgfqpoint{4.246845in}{1.511103in}}{\pgfqpoint{4.241802in}{1.516146in}}%
\pgfpathcurveto{\pgfqpoint{4.236758in}{1.521190in}}{\pgfqpoint{4.229916in}{1.524024in}}{\pgfqpoint{4.222784in}{1.524024in}}%
\pgfpathcurveto{\pgfqpoint{4.215651in}{1.524024in}}{\pgfqpoint{4.208809in}{1.521190in}}{\pgfqpoint{4.203765in}{1.516146in}}%
\pgfpathcurveto{\pgfqpoint{4.198722in}{1.511103in}}{\pgfqpoint{4.195888in}{1.504261in}}{\pgfqpoint{4.195888in}{1.497128in}}%
\pgfpathcurveto{\pgfqpoint{4.195888in}{1.489995in}}{\pgfqpoint{4.198722in}{1.483154in}}{\pgfqpoint{4.203765in}{1.478110in}}%
\pgfpathcurveto{\pgfqpoint{4.208809in}{1.473066in}}{\pgfqpoint{4.215651in}{1.470232in}}{\pgfqpoint{4.222784in}{1.470232in}}%
\pgfpathclose%
\pgfusepath{stroke,fill}%
\end{pgfscope}%
\begin{pgfscope}%
\pgfpathrectangle{\pgfqpoint{2.867647in}{0.500000in}}{\pgfqpoint{1.764706in}{1.700000in}}%
\pgfusepath{clip}%
\pgfsetbuttcap%
\pgfsetroundjoin%
\definecolor{currentfill}{rgb}{0.972726,0.844889,0.754401}%
\pgfsetfillcolor{currentfill}%
\pgfsetlinewidth{0.311001pt}%
\definecolor{currentstroke}{rgb}{1.000000,1.000000,1.000000}%
\pgfsetstrokecolor{currentstroke}%
\pgfsetdash{}{0pt}%
\pgfpathmoveto{\pgfqpoint{4.077959in}{0.994469in}}%
\pgfpathcurveto{\pgfqpoint{4.085092in}{0.994469in}}{\pgfqpoint{4.091933in}{0.997303in}}{\pgfqpoint{4.096977in}{1.002346in}}%
\pgfpathcurveto{\pgfqpoint{4.102021in}{1.007390in}}{\pgfqpoint{4.104854in}{1.014232in}}{\pgfqpoint{4.104854in}{1.021365in}}%
\pgfpathcurveto{\pgfqpoint{4.104854in}{1.028497in}}{\pgfqpoint{4.102021in}{1.035339in}}{\pgfqpoint{4.096977in}{1.040383in}}%
\pgfpathcurveto{\pgfqpoint{4.091933in}{1.045426in}}{\pgfqpoint{4.085092in}{1.048260in}}{\pgfqpoint{4.077959in}{1.048260in}}%
\pgfpathcurveto{\pgfqpoint{4.070826in}{1.048260in}}{\pgfqpoint{4.063984in}{1.045426in}}{\pgfqpoint{4.058941in}{1.040383in}}%
\pgfpathcurveto{\pgfqpoint{4.053897in}{1.035339in}}{\pgfqpoint{4.051063in}{1.028497in}}{\pgfqpoint{4.051063in}{1.021365in}}%
\pgfpathcurveto{\pgfqpoint{4.051063in}{1.014232in}}{\pgfqpoint{4.053897in}{1.007390in}}{\pgfqpoint{4.058941in}{1.002346in}}%
\pgfpathcurveto{\pgfqpoint{4.063984in}{0.997303in}}{\pgfqpoint{4.070826in}{0.994469in}}{\pgfqpoint{4.077959in}{0.994469in}}%
\pgfpathclose%
\pgfusepath{stroke,fill}%
\end{pgfscope}%
\begin{pgfscope}%
\pgfpathrectangle{\pgfqpoint{2.867647in}{0.500000in}}{\pgfqpoint{1.764706in}{1.700000in}}%
\pgfusepath{clip}%
\pgfsetbuttcap%
\pgfsetroundjoin%
\definecolor{currentfill}{rgb}{0.973271,0.850724,0.762998}%
\pgfsetfillcolor{currentfill}%
\pgfsetlinewidth{0.311001pt}%
\definecolor{currentstroke}{rgb}{1.000000,1.000000,1.000000}%
\pgfsetstrokecolor{currentstroke}%
\pgfsetdash{}{0pt}%
\pgfpathmoveto{\pgfqpoint{4.257933in}{1.322900in}}%
\pgfpathcurveto{\pgfqpoint{4.265066in}{1.322900in}}{\pgfqpoint{4.271908in}{1.325734in}}{\pgfqpoint{4.276952in}{1.330777in}}%
\pgfpathcurveto{\pgfqpoint{4.281995in}{1.335821in}}{\pgfqpoint{4.284829in}{1.342663in}}{\pgfqpoint{4.284829in}{1.349796in}}%
\pgfpathcurveto{\pgfqpoint{4.284829in}{1.356928in}}{\pgfqpoint{4.281995in}{1.363770in}}{\pgfqpoint{4.276952in}{1.368814in}}%
\pgfpathcurveto{\pgfqpoint{4.271908in}{1.373857in}}{\pgfqpoint{4.265066in}{1.376691in}}{\pgfqpoint{4.257933in}{1.376691in}}%
\pgfpathcurveto{\pgfqpoint{4.250801in}{1.376691in}}{\pgfqpoint{4.243959in}{1.373857in}}{\pgfqpoint{4.238915in}{1.368814in}}%
\pgfpathcurveto{\pgfqpoint{4.233872in}{1.363770in}}{\pgfqpoint{4.231038in}{1.356928in}}{\pgfqpoint{4.231038in}{1.349796in}}%
\pgfpathcurveto{\pgfqpoint{4.231038in}{1.342663in}}{\pgfqpoint{4.233872in}{1.335821in}}{\pgfqpoint{4.238915in}{1.330777in}}%
\pgfpathcurveto{\pgfqpoint{4.243959in}{1.325734in}}{\pgfqpoint{4.250801in}{1.322900in}}{\pgfqpoint{4.257933in}{1.322900in}}%
\pgfpathclose%
\pgfusepath{stroke,fill}%
\end{pgfscope}%
\begin{pgfscope}%
\pgfpathrectangle{\pgfqpoint{2.867647in}{0.500000in}}{\pgfqpoint{1.764706in}{1.700000in}}%
\pgfusepath{clip}%
\pgfsetbuttcap%
\pgfsetroundjoin%
\definecolor{currentfill}{rgb}{0.976961,0.885681,0.814303}%
\pgfsetfillcolor{currentfill}%
\pgfsetlinewidth{0.311001pt}%
\definecolor{currentstroke}{rgb}{1.000000,1.000000,1.000000}%
\pgfsetstrokecolor{currentstroke}%
\pgfsetdash{}{0pt}%
\pgfpathmoveto{\pgfqpoint{4.098225in}{1.588798in}}%
\pgfpathcurveto{\pgfqpoint{4.105358in}{1.588798in}}{\pgfqpoint{4.112199in}{1.591632in}}{\pgfqpoint{4.117243in}{1.596676in}}%
\pgfpathcurveto{\pgfqpoint{4.122287in}{1.601719in}}{\pgfqpoint{4.125121in}{1.608561in}}{\pgfqpoint{4.125121in}{1.615694in}}%
\pgfpathcurveto{\pgfqpoint{4.125121in}{1.622827in}}{\pgfqpoint{4.122287in}{1.629668in}}{\pgfqpoint{4.117243in}{1.634712in}}%
\pgfpathcurveto{\pgfqpoint{4.112199in}{1.639756in}}{\pgfqpoint{4.105358in}{1.642589in}}{\pgfqpoint{4.098225in}{1.642589in}}%
\pgfpathcurveto{\pgfqpoint{4.091092in}{1.642589in}}{\pgfqpoint{4.084250in}{1.639756in}}{\pgfqpoint{4.079207in}{1.634712in}}%
\pgfpathcurveto{\pgfqpoint{4.074163in}{1.629668in}}{\pgfqpoint{4.071329in}{1.622827in}}{\pgfqpoint{4.071329in}{1.615694in}}%
\pgfpathcurveto{\pgfqpoint{4.071329in}{1.608561in}}{\pgfqpoint{4.074163in}{1.601719in}}{\pgfqpoint{4.079207in}{1.596676in}}%
\pgfpathcurveto{\pgfqpoint{4.084250in}{1.591632in}}{\pgfqpoint{4.091092in}{1.588798in}}{\pgfqpoint{4.098225in}{1.588798in}}%
\pgfpathclose%
\pgfusepath{stroke,fill}%
\end{pgfscope}%
\begin{pgfscope}%
\pgfpathrectangle{\pgfqpoint{2.867647in}{0.500000in}}{\pgfqpoint{1.764706in}{1.700000in}}%
\pgfusepath{clip}%
\pgfsetbuttcap%
\pgfsetroundjoin%
\definecolor{currentfill}{rgb}{0.976961,0.885681,0.814303}%
\pgfsetfillcolor{currentfill}%
\pgfsetlinewidth{0.311001pt}%
\definecolor{currentstroke}{rgb}{1.000000,1.000000,1.000000}%
\pgfsetstrokecolor{currentstroke}%
\pgfsetdash{}{0pt}%
\pgfpathmoveto{\pgfqpoint{4.144203in}{1.344816in}}%
\pgfpathcurveto{\pgfqpoint{4.151336in}{1.344816in}}{\pgfqpoint{4.158177in}{1.347649in}}{\pgfqpoint{4.163221in}{1.352693in}}%
\pgfpathcurveto{\pgfqpoint{4.168265in}{1.357737in}}{\pgfqpoint{4.171099in}{1.364578in}}{\pgfqpoint{4.171099in}{1.371711in}}%
\pgfpathcurveto{\pgfqpoint{4.171099in}{1.378844in}}{\pgfqpoint{4.168265in}{1.385686in}}{\pgfqpoint{4.163221in}{1.390729in}}%
\pgfpathcurveto{\pgfqpoint{4.158177in}{1.395773in}}{\pgfqpoint{4.151336in}{1.398607in}}{\pgfqpoint{4.144203in}{1.398607in}}%
\pgfpathcurveto{\pgfqpoint{4.137070in}{1.398607in}}{\pgfqpoint{4.130228in}{1.395773in}}{\pgfqpoint{4.125185in}{1.390729in}}%
\pgfpathcurveto{\pgfqpoint{4.120141in}{1.385686in}}{\pgfqpoint{4.117307in}{1.378844in}}{\pgfqpoint{4.117307in}{1.371711in}}%
\pgfpathcurveto{\pgfqpoint{4.117307in}{1.364578in}}{\pgfqpoint{4.120141in}{1.357737in}}{\pgfqpoint{4.125185in}{1.352693in}}%
\pgfpathcurveto{\pgfqpoint{4.130228in}{1.347649in}}{\pgfqpoint{4.137070in}{1.344816in}}{\pgfqpoint{4.144203in}{1.344816in}}%
\pgfpathclose%
\pgfusepath{stroke,fill}%
\end{pgfscope}%
\begin{pgfscope}%
\pgfpathrectangle{\pgfqpoint{2.867647in}{0.500000in}}{\pgfqpoint{1.764706in}{1.700000in}}%
\pgfusepath{clip}%
\pgfsetbuttcap%
\pgfsetroundjoin%
\definecolor{currentfill}{rgb}{0.969803,0.809811,0.702523}%
\pgfsetfillcolor{currentfill}%
\pgfsetlinewidth{0.311001pt}%
\definecolor{currentstroke}{rgb}{1.000000,1.000000,1.000000}%
\pgfsetstrokecolor{currentstroke}%
\pgfsetdash{}{0pt}%
\pgfpathmoveto{\pgfqpoint{4.273889in}{1.377297in}}%
\pgfpathcurveto{\pgfqpoint{4.281022in}{1.377297in}}{\pgfqpoint{4.287863in}{1.380131in}}{\pgfqpoint{4.292907in}{1.385174in}}%
\pgfpathcurveto{\pgfqpoint{4.297951in}{1.390218in}}{\pgfqpoint{4.300785in}{1.397060in}}{\pgfqpoint{4.300785in}{1.404192in}}%
\pgfpathcurveto{\pgfqpoint{4.300785in}{1.411325in}}{\pgfqpoint{4.297951in}{1.418167in}}{\pgfqpoint{4.292907in}{1.423211in}}%
\pgfpathcurveto{\pgfqpoint{4.287863in}{1.428254in}}{\pgfqpoint{4.281022in}{1.431088in}}{\pgfqpoint{4.273889in}{1.431088in}}%
\pgfpathcurveto{\pgfqpoint{4.266756in}{1.431088in}}{\pgfqpoint{4.259914in}{1.428254in}}{\pgfqpoint{4.254871in}{1.423211in}}%
\pgfpathcurveto{\pgfqpoint{4.249827in}{1.418167in}}{\pgfqpoint{4.246993in}{1.411325in}}{\pgfqpoint{4.246993in}{1.404192in}}%
\pgfpathcurveto{\pgfqpoint{4.246993in}{1.397060in}}{\pgfqpoint{4.249827in}{1.390218in}}{\pgfqpoint{4.254871in}{1.385174in}}%
\pgfpathcurveto{\pgfqpoint{4.259914in}{1.380131in}}{\pgfqpoint{4.266756in}{1.377297in}}{\pgfqpoint{4.273889in}{1.377297in}}%
\pgfpathclose%
\pgfusepath{stroke,fill}%
\end{pgfscope}%
\begin{pgfscope}%
\pgfpathrectangle{\pgfqpoint{2.867647in}{0.500000in}}{\pgfqpoint{1.764706in}{1.700000in}}%
\pgfusepath{clip}%
\pgfsetbuttcap%
\pgfsetroundjoin%
\definecolor{currentfill}{rgb}{0.980678,0.914765,0.856766}%
\pgfsetfillcolor{currentfill}%
\pgfsetlinewidth{0.311001pt}%
\definecolor{currentstroke}{rgb}{1.000000,1.000000,1.000000}%
\pgfsetstrokecolor{currentstroke}%
\pgfsetdash{}{0pt}%
\pgfpathmoveto{\pgfqpoint{4.213006in}{1.280212in}}%
\pgfpathcurveto{\pgfqpoint{4.220139in}{1.280212in}}{\pgfqpoint{4.226980in}{1.283046in}}{\pgfqpoint{4.232024in}{1.288089in}}%
\pgfpathcurveto{\pgfqpoint{4.237068in}{1.293133in}}{\pgfqpoint{4.239901in}{1.299974in}}{\pgfqpoint{4.239901in}{1.307107in}}%
\pgfpathcurveto{\pgfqpoint{4.239901in}{1.314240in}}{\pgfqpoint{4.237068in}{1.321082in}}{\pgfqpoint{4.232024in}{1.326125in}}%
\pgfpathcurveto{\pgfqpoint{4.226980in}{1.331169in}}{\pgfqpoint{4.220139in}{1.334003in}}{\pgfqpoint{4.213006in}{1.334003in}}%
\pgfpathcurveto{\pgfqpoint{4.205873in}{1.334003in}}{\pgfqpoint{4.199031in}{1.331169in}}{\pgfqpoint{4.193988in}{1.326125in}}%
\pgfpathcurveto{\pgfqpoint{4.188944in}{1.321082in}}{\pgfqpoint{4.186110in}{1.314240in}}{\pgfqpoint{4.186110in}{1.307107in}}%
\pgfpathcurveto{\pgfqpoint{4.186110in}{1.299974in}}{\pgfqpoint{4.188944in}{1.293133in}}{\pgfqpoint{4.193988in}{1.288089in}}%
\pgfpathcurveto{\pgfqpoint{4.199031in}{1.283046in}}{\pgfqpoint{4.205873in}{1.280212in}}{\pgfqpoint{4.213006in}{1.280212in}}%
\pgfpathclose%
\pgfusepath{stroke,fill}%
\end{pgfscope}%
\begin{pgfscope}%
\pgfpathrectangle{\pgfqpoint{2.867647in}{0.500000in}}{\pgfqpoint{1.764706in}{1.700000in}}%
\pgfusepath{clip}%
\pgfsetbuttcap%
\pgfsetroundjoin%
\definecolor{currentfill}{rgb}{0.979891,0.908948,0.848279}%
\pgfsetfillcolor{currentfill}%
\pgfsetlinewidth{0.311001pt}%
\definecolor{currentstroke}{rgb}{1.000000,1.000000,1.000000}%
\pgfsetstrokecolor{currentstroke}%
\pgfsetdash{}{0pt}%
\pgfpathmoveto{\pgfqpoint{4.196201in}{1.158650in}}%
\pgfpathcurveto{\pgfqpoint{4.203333in}{1.158650in}}{\pgfqpoint{4.210175in}{1.161484in}}{\pgfqpoint{4.215219in}{1.166528in}}%
\pgfpathcurveto{\pgfqpoint{4.220262in}{1.171571in}}{\pgfqpoint{4.223096in}{1.178413in}}{\pgfqpoint{4.223096in}{1.185546in}}%
\pgfpathcurveto{\pgfqpoint{4.223096in}{1.192678in}}{\pgfqpoint{4.220262in}{1.199520in}}{\pgfqpoint{4.215219in}{1.204564in}}%
\pgfpathcurveto{\pgfqpoint{4.210175in}{1.209607in}}{\pgfqpoint{4.203333in}{1.212441in}}{\pgfqpoint{4.196201in}{1.212441in}}%
\pgfpathcurveto{\pgfqpoint{4.189068in}{1.212441in}}{\pgfqpoint{4.182226in}{1.209607in}}{\pgfqpoint{4.177182in}{1.204564in}}%
\pgfpathcurveto{\pgfqpoint{4.172139in}{1.199520in}}{\pgfqpoint{4.169305in}{1.192678in}}{\pgfqpoint{4.169305in}{1.185546in}}%
\pgfpathcurveto{\pgfqpoint{4.169305in}{1.178413in}}{\pgfqpoint{4.172139in}{1.171571in}}{\pgfqpoint{4.177182in}{1.166528in}}%
\pgfpathcurveto{\pgfqpoint{4.182226in}{1.161484in}}{\pgfqpoint{4.189068in}{1.158650in}}{\pgfqpoint{4.196201in}{1.158650in}}%
\pgfpathclose%
\pgfusepath{stroke,fill}%
\end{pgfscope}%
\begin{pgfscope}%
\pgfpathrectangle{\pgfqpoint{2.867647in}{0.500000in}}{\pgfqpoint{1.764706in}{1.700000in}}%
\pgfusepath{clip}%
\pgfsetbuttcap%
\pgfsetroundjoin%
\definecolor{currentfill}{rgb}{0.981377,0.920617,0.865369}%
\pgfsetfillcolor{currentfill}%
\pgfsetlinewidth{0.311001pt}%
\definecolor{currentstroke}{rgb}{1.000000,1.000000,1.000000}%
\pgfsetstrokecolor{currentstroke}%
\pgfsetdash{}{0pt}%
\pgfpathmoveto{\pgfqpoint{4.167283in}{1.254800in}}%
\pgfpathcurveto{\pgfqpoint{4.174416in}{1.254800in}}{\pgfqpoint{4.181258in}{1.257634in}}{\pgfqpoint{4.186301in}{1.262677in}}%
\pgfpathcurveto{\pgfqpoint{4.191345in}{1.267721in}}{\pgfqpoint{4.194179in}{1.274563in}}{\pgfqpoint{4.194179in}{1.281696in}}%
\pgfpathcurveto{\pgfqpoint{4.194179in}{1.288828in}}{\pgfqpoint{4.191345in}{1.295670in}}{\pgfqpoint{4.186301in}{1.300714in}}%
\pgfpathcurveto{\pgfqpoint{4.181258in}{1.305757in}}{\pgfqpoint{4.174416in}{1.308591in}}{\pgfqpoint{4.167283in}{1.308591in}}%
\pgfpathcurveto{\pgfqpoint{4.160150in}{1.308591in}}{\pgfqpoint{4.153309in}{1.305757in}}{\pgfqpoint{4.148265in}{1.300714in}}%
\pgfpathcurveto{\pgfqpoint{4.143221in}{1.295670in}}{\pgfqpoint{4.140388in}{1.288828in}}{\pgfqpoint{4.140388in}{1.281696in}}%
\pgfpathcurveto{\pgfqpoint{4.140388in}{1.274563in}}{\pgfqpoint{4.143221in}{1.267721in}}{\pgfqpoint{4.148265in}{1.262677in}}%
\pgfpathcurveto{\pgfqpoint{4.153309in}{1.257634in}}{\pgfqpoint{4.160150in}{1.254800in}}{\pgfqpoint{4.167283in}{1.254800in}}%
\pgfpathclose%
\pgfusepath{stroke,fill}%
\end{pgfscope}%
\begin{pgfscope}%
\pgfpathrectangle{\pgfqpoint{2.867647in}{0.500000in}}{\pgfqpoint{1.764706in}{1.700000in}}%
\pgfusepath{clip}%
\pgfsetbuttcap%
\pgfsetroundjoin%
\definecolor{currentfill}{rgb}{0.979891,0.908948,0.848279}%
\pgfsetfillcolor{currentfill}%
\pgfsetlinewidth{0.311001pt}%
\definecolor{currentstroke}{rgb}{1.000000,1.000000,1.000000}%
\pgfsetstrokecolor{currentstroke}%
\pgfsetdash{}{0pt}%
\pgfpathmoveto{\pgfqpoint{4.213375in}{1.382576in}}%
\pgfpathcurveto{\pgfqpoint{4.220507in}{1.382576in}}{\pgfqpoint{4.227349in}{1.385410in}}{\pgfqpoint{4.232393in}{1.390453in}}%
\pgfpathcurveto{\pgfqpoint{4.237436in}{1.395497in}}{\pgfqpoint{4.240270in}{1.402339in}}{\pgfqpoint{4.240270in}{1.409472in}}%
\pgfpathcurveto{\pgfqpoint{4.240270in}{1.416604in}}{\pgfqpoint{4.237436in}{1.423446in}}{\pgfqpoint{4.232393in}{1.428490in}}%
\pgfpathcurveto{\pgfqpoint{4.227349in}{1.433533in}}{\pgfqpoint{4.220507in}{1.436367in}}{\pgfqpoint{4.213375in}{1.436367in}}%
\pgfpathcurveto{\pgfqpoint{4.206242in}{1.436367in}}{\pgfqpoint{4.199400in}{1.433533in}}{\pgfqpoint{4.194356in}{1.428490in}}%
\pgfpathcurveto{\pgfqpoint{4.189313in}{1.423446in}}{\pgfqpoint{4.186479in}{1.416604in}}{\pgfqpoint{4.186479in}{1.409472in}}%
\pgfpathcurveto{\pgfqpoint{4.186479in}{1.402339in}}{\pgfqpoint{4.189313in}{1.395497in}}{\pgfqpoint{4.194356in}{1.390453in}}%
\pgfpathcurveto{\pgfqpoint{4.199400in}{1.385410in}}{\pgfqpoint{4.206242in}{1.382576in}}{\pgfqpoint{4.213375in}{1.382576in}}%
\pgfpathclose%
\pgfusepath{stroke,fill}%
\end{pgfscope}%
\begin{pgfscope}%
\pgfpathrectangle{\pgfqpoint{2.867647in}{0.500000in}}{\pgfqpoint{1.764706in}{1.700000in}}%
\pgfusepath{clip}%
\pgfsetbuttcap%
\pgfsetroundjoin%
\definecolor{currentfill}{rgb}{0.969359,0.803954,0.693832}%
\pgfsetfillcolor{currentfill}%
\pgfsetlinewidth{0.311001pt}%
\definecolor{currentstroke}{rgb}{1.000000,1.000000,1.000000}%
\pgfsetstrokecolor{currentstroke}%
\pgfsetdash{}{0pt}%
\pgfpathmoveto{\pgfqpoint{4.097175in}{0.942970in}}%
\pgfpathcurveto{\pgfqpoint{4.104308in}{0.942970in}}{\pgfqpoint{4.111150in}{0.945804in}}{\pgfqpoint{4.116193in}{0.950847in}}%
\pgfpathcurveto{\pgfqpoint{4.121237in}{0.955891in}}{\pgfqpoint{4.124071in}{0.962733in}}{\pgfqpoint{4.124071in}{0.969865in}}%
\pgfpathcurveto{\pgfqpoint{4.124071in}{0.976998in}}{\pgfqpoint{4.121237in}{0.983840in}}{\pgfqpoint{4.116193in}{0.988884in}}%
\pgfpathcurveto{\pgfqpoint{4.111150in}{0.993927in}}{\pgfqpoint{4.104308in}{0.996761in}}{\pgfqpoint{4.097175in}{0.996761in}}%
\pgfpathcurveto{\pgfqpoint{4.090042in}{0.996761in}}{\pgfqpoint{4.083201in}{0.993927in}}{\pgfqpoint{4.078157in}{0.988884in}}%
\pgfpathcurveto{\pgfqpoint{4.073113in}{0.983840in}}{\pgfqpoint{4.070279in}{0.976998in}}{\pgfqpoint{4.070279in}{0.969865in}}%
\pgfpathcurveto{\pgfqpoint{4.070279in}{0.962733in}}{\pgfqpoint{4.073113in}{0.955891in}}{\pgfqpoint{4.078157in}{0.950847in}}%
\pgfpathcurveto{\pgfqpoint{4.083201in}{0.945804in}}{\pgfqpoint{4.090042in}{0.942970in}}{\pgfqpoint{4.097175in}{0.942970in}}%
\pgfpathclose%
\pgfusepath{stroke,fill}%
\end{pgfscope}%
\begin{pgfscope}%
\pgfpathrectangle{\pgfqpoint{2.867647in}{0.500000in}}{\pgfqpoint{1.764706in}{1.700000in}}%
\pgfusepath{clip}%
\pgfsetbuttcap%
\pgfsetroundjoin%
\definecolor{currentfill}{rgb}{0.964558,0.676556,0.522514}%
\pgfsetfillcolor{currentfill}%
\pgfsetlinewidth{0.311001pt}%
\definecolor{currentstroke}{rgb}{1.000000,1.000000,1.000000}%
\pgfsetstrokecolor{currentstroke}%
\pgfsetdash{}{0pt}%
\pgfpathmoveto{\pgfqpoint{4.152246in}{1.731982in}}%
\pgfpathcurveto{\pgfqpoint{4.159378in}{1.731982in}}{\pgfqpoint{4.166220in}{1.734815in}}{\pgfqpoint{4.171264in}{1.739859in}}%
\pgfpathcurveto{\pgfqpoint{4.176307in}{1.744903in}}{\pgfqpoint{4.179141in}{1.751744in}}{\pgfqpoint{4.179141in}{1.758877in}}%
\pgfpathcurveto{\pgfqpoint{4.179141in}{1.766010in}}{\pgfqpoint{4.176307in}{1.772852in}}{\pgfqpoint{4.171264in}{1.777895in}}%
\pgfpathcurveto{\pgfqpoint{4.166220in}{1.782939in}}{\pgfqpoint{4.159378in}{1.785773in}}{\pgfqpoint{4.152246in}{1.785773in}}%
\pgfpathcurveto{\pgfqpoint{4.145113in}{1.785773in}}{\pgfqpoint{4.138271in}{1.782939in}}{\pgfqpoint{4.133227in}{1.777895in}}%
\pgfpathcurveto{\pgfqpoint{4.128184in}{1.772852in}}{\pgfqpoint{4.125350in}{1.766010in}}{\pgfqpoint{4.125350in}{1.758877in}}%
\pgfpathcurveto{\pgfqpoint{4.125350in}{1.751744in}}{\pgfqpoint{4.128184in}{1.744903in}}{\pgfqpoint{4.133227in}{1.739859in}}%
\pgfpathcurveto{\pgfqpoint{4.138271in}{1.734815in}}{\pgfqpoint{4.145113in}{1.731982in}}{\pgfqpoint{4.152246in}{1.731982in}}%
\pgfpathclose%
\pgfusepath{stroke,fill}%
\end{pgfscope}%
\begin{pgfscope}%
\pgfpathrectangle{\pgfqpoint{2.867647in}{0.500000in}}{\pgfqpoint{1.764706in}{1.700000in}}%
\pgfusepath{clip}%
\pgfsetbuttcap%
\pgfsetroundjoin%
\definecolor{currentfill}{rgb}{0.975018,0.868213,0.788710}%
\pgfsetfillcolor{currentfill}%
\pgfsetlinewidth{0.311001pt}%
\definecolor{currentstroke}{rgb}{1.000000,1.000000,1.000000}%
\pgfsetstrokecolor{currentstroke}%
\pgfsetdash{}{0pt}%
\pgfpathmoveto{\pgfqpoint{4.222439in}{1.498332in}}%
\pgfpathcurveto{\pgfqpoint{4.229571in}{1.498332in}}{\pgfqpoint{4.236413in}{1.501166in}}{\pgfqpoint{4.241457in}{1.506209in}}%
\pgfpathcurveto{\pgfqpoint{4.246500in}{1.511253in}}{\pgfqpoint{4.249334in}{1.518095in}}{\pgfqpoint{4.249334in}{1.525227in}}%
\pgfpathcurveto{\pgfqpoint{4.249334in}{1.532360in}}{\pgfqpoint{4.246500in}{1.539202in}}{\pgfqpoint{4.241457in}{1.544245in}}%
\pgfpathcurveto{\pgfqpoint{4.236413in}{1.549289in}}{\pgfqpoint{4.229571in}{1.552123in}}{\pgfqpoint{4.222439in}{1.552123in}}%
\pgfpathcurveto{\pgfqpoint{4.215306in}{1.552123in}}{\pgfqpoint{4.208464in}{1.549289in}}{\pgfqpoint{4.203421in}{1.544245in}}%
\pgfpathcurveto{\pgfqpoint{4.198377in}{1.539202in}}{\pgfqpoint{4.195543in}{1.532360in}}{\pgfqpoint{4.195543in}{1.525227in}}%
\pgfpathcurveto{\pgfqpoint{4.195543in}{1.518095in}}{\pgfqpoint{4.198377in}{1.511253in}}{\pgfqpoint{4.203421in}{1.506209in}}%
\pgfpathcurveto{\pgfqpoint{4.208464in}{1.501166in}}{\pgfqpoint{4.215306in}{1.498332in}}{\pgfqpoint{4.222439in}{1.498332in}}%
\pgfpathclose%
\pgfusepath{stroke,fill}%
\end{pgfscope}%
\begin{pgfscope}%
\pgfpathrectangle{\pgfqpoint{2.867647in}{0.500000in}}{\pgfqpoint{1.764706in}{1.700000in}}%
\pgfusepath{clip}%
\pgfsetbuttcap%
\pgfsetroundjoin%
\definecolor{currentfill}{rgb}{0.967092,0.768560,0.642079}%
\pgfsetfillcolor{currentfill}%
\pgfsetlinewidth{0.311001pt}%
\definecolor{currentstroke}{rgb}{1.000000,1.000000,1.000000}%
\pgfsetstrokecolor{currentstroke}%
\pgfsetdash{}{0pt}%
\pgfpathmoveto{\pgfqpoint{4.028756in}{1.706446in}}%
\pgfpathcurveto{\pgfqpoint{4.035889in}{1.706446in}}{\pgfqpoint{4.042731in}{1.709280in}}{\pgfqpoint{4.047774in}{1.714324in}}%
\pgfpathcurveto{\pgfqpoint{4.052818in}{1.719368in}}{\pgfqpoint{4.055652in}{1.726209in}}{\pgfqpoint{4.055652in}{1.733342in}}%
\pgfpathcurveto{\pgfqpoint{4.055652in}{1.740475in}}{\pgfqpoint{4.052818in}{1.747317in}}{\pgfqpoint{4.047774in}{1.752360in}}%
\pgfpathcurveto{\pgfqpoint{4.042731in}{1.757404in}}{\pgfqpoint{4.035889in}{1.760238in}}{\pgfqpoint{4.028756in}{1.760238in}}%
\pgfpathcurveto{\pgfqpoint{4.021624in}{1.760238in}}{\pgfqpoint{4.014782in}{1.757404in}}{\pgfqpoint{4.009738in}{1.752360in}}%
\pgfpathcurveto{\pgfqpoint{4.004695in}{1.747317in}}{\pgfqpoint{4.001861in}{1.740475in}}{\pgfqpoint{4.001861in}{1.733342in}}%
\pgfpathcurveto{\pgfqpoint{4.001861in}{1.726209in}}{\pgfqpoint{4.004695in}{1.719368in}}{\pgfqpoint{4.009738in}{1.714324in}}%
\pgfpathcurveto{\pgfqpoint{4.014782in}{1.709280in}}{\pgfqpoint{4.021624in}{1.706446in}}{\pgfqpoint{4.028756in}{1.706446in}}%
\pgfpathclose%
\pgfusepath{stroke,fill}%
\end{pgfscope}%
\begin{pgfscope}%
\pgfpathrectangle{\pgfqpoint{2.867647in}{0.500000in}}{\pgfqpoint{1.764706in}{1.700000in}}%
\pgfusepath{clip}%
\pgfsetbuttcap%
\pgfsetroundjoin%
\definecolor{currentfill}{rgb}{0.968931,0.798091,0.685123}%
\pgfsetfillcolor{currentfill}%
\pgfsetlinewidth{0.311001pt}%
\definecolor{currentstroke}{rgb}{1.000000,1.000000,1.000000}%
\pgfsetstrokecolor{currentstroke}%
\pgfsetdash{}{0pt}%
\pgfpathmoveto{\pgfqpoint{4.208858in}{1.033452in}}%
\pgfpathcurveto{\pgfqpoint{4.215991in}{1.033452in}}{\pgfqpoint{4.222832in}{1.036286in}}{\pgfqpoint{4.227876in}{1.041330in}}%
\pgfpathcurveto{\pgfqpoint{4.232920in}{1.046373in}}{\pgfqpoint{4.235754in}{1.053215in}}{\pgfqpoint{4.235754in}{1.060348in}}%
\pgfpathcurveto{\pgfqpoint{4.235754in}{1.067481in}}{\pgfqpoint{4.232920in}{1.074322in}}{\pgfqpoint{4.227876in}{1.079366in}}%
\pgfpathcurveto{\pgfqpoint{4.222832in}{1.084409in}}{\pgfqpoint{4.215991in}{1.087243in}}{\pgfqpoint{4.208858in}{1.087243in}}%
\pgfpathcurveto{\pgfqpoint{4.201725in}{1.087243in}}{\pgfqpoint{4.194883in}{1.084409in}}{\pgfqpoint{4.189840in}{1.079366in}}%
\pgfpathcurveto{\pgfqpoint{4.184796in}{1.074322in}}{\pgfqpoint{4.181962in}{1.067481in}}{\pgfqpoint{4.181962in}{1.060348in}}%
\pgfpathcurveto{\pgfqpoint{4.181962in}{1.053215in}}{\pgfqpoint{4.184796in}{1.046373in}}{\pgfqpoint{4.189840in}{1.041330in}}%
\pgfpathcurveto{\pgfqpoint{4.194883in}{1.036286in}}{\pgfqpoint{4.201725in}{1.033452in}}{\pgfqpoint{4.208858in}{1.033452in}}%
\pgfpathclose%
\pgfusepath{stroke,fill}%
\end{pgfscope}%
\begin{pgfscope}%
\pgfpathrectangle{\pgfqpoint{2.867647in}{0.500000in}}{\pgfqpoint{1.764706in}{1.700000in}}%
\pgfusepath{clip}%
\pgfsetbuttcap%
\pgfsetroundjoin%
\definecolor{currentfill}{rgb}{0.974412,0.862387,0.780156}%
\pgfsetfillcolor{currentfill}%
\pgfsetlinewidth{0.311001pt}%
\definecolor{currentstroke}{rgb}{1.000000,1.000000,1.000000}%
\pgfsetstrokecolor{currentstroke}%
\pgfsetdash{}{0pt}%
\pgfpathmoveto{\pgfqpoint{4.100934in}{1.000500in}}%
\pgfpathcurveto{\pgfqpoint{4.108067in}{1.000500in}}{\pgfqpoint{4.114909in}{1.003333in}}{\pgfqpoint{4.119953in}{1.008377in}}%
\pgfpathcurveto{\pgfqpoint{4.124996in}{1.013421in}}{\pgfqpoint{4.127830in}{1.020262in}}{\pgfqpoint{4.127830in}{1.027395in}}%
\pgfpathcurveto{\pgfqpoint{4.127830in}{1.034528in}}{\pgfqpoint{4.124996in}{1.041370in}}{\pgfqpoint{4.119953in}{1.046413in}}%
\pgfpathcurveto{\pgfqpoint{4.114909in}{1.051457in}}{\pgfqpoint{4.108067in}{1.054291in}}{\pgfqpoint{4.100934in}{1.054291in}}%
\pgfpathcurveto{\pgfqpoint{4.093802in}{1.054291in}}{\pgfqpoint{4.086960in}{1.051457in}}{\pgfqpoint{4.081916in}{1.046413in}}%
\pgfpathcurveto{\pgfqpoint{4.076873in}{1.041370in}}{\pgfqpoint{4.074039in}{1.034528in}}{\pgfqpoint{4.074039in}{1.027395in}}%
\pgfpathcurveto{\pgfqpoint{4.074039in}{1.020262in}}{\pgfqpoint{4.076873in}{1.013421in}}{\pgfqpoint{4.081916in}{1.008377in}}%
\pgfpathcurveto{\pgfqpoint{4.086960in}{1.003333in}}{\pgfqpoint{4.093802in}{1.000500in}}{\pgfqpoint{4.100934in}{1.000500in}}%
\pgfpathclose%
\pgfusepath{stroke,fill}%
\end{pgfscope}%
\begin{pgfscope}%
\pgfpathrectangle{\pgfqpoint{2.867647in}{0.500000in}}{\pgfqpoint{1.764706in}{1.700000in}}%
\pgfusepath{clip}%
\pgfsetbuttcap%
\pgfsetroundjoin%
\definecolor{currentfill}{rgb}{0.973832,0.856556,0.771584}%
\pgfsetfillcolor{currentfill}%
\pgfsetlinewidth{0.311001pt}%
\definecolor{currentstroke}{rgb}{1.000000,1.000000,1.000000}%
\pgfsetstrokecolor{currentstroke}%
\pgfsetdash{}{0pt}%
\pgfpathmoveto{\pgfqpoint{4.255023in}{1.332173in}}%
\pgfpathcurveto{\pgfqpoint{4.262156in}{1.332173in}}{\pgfqpoint{4.268998in}{1.335007in}}{\pgfqpoint{4.274041in}{1.340050in}}%
\pgfpathcurveto{\pgfqpoint{4.279085in}{1.345094in}}{\pgfqpoint{4.281919in}{1.351936in}}{\pgfqpoint{4.281919in}{1.359068in}}%
\pgfpathcurveto{\pgfqpoint{4.281919in}{1.366201in}}{\pgfqpoint{4.279085in}{1.373043in}}{\pgfqpoint{4.274041in}{1.378087in}}%
\pgfpathcurveto{\pgfqpoint{4.268998in}{1.383130in}}{\pgfqpoint{4.262156in}{1.385964in}}{\pgfqpoint{4.255023in}{1.385964in}}%
\pgfpathcurveto{\pgfqpoint{4.247890in}{1.385964in}}{\pgfqpoint{4.241049in}{1.383130in}}{\pgfqpoint{4.236005in}{1.378087in}}%
\pgfpathcurveto{\pgfqpoint{4.230961in}{1.373043in}}{\pgfqpoint{4.228127in}{1.366201in}}{\pgfqpoint{4.228127in}{1.359068in}}%
\pgfpathcurveto{\pgfqpoint{4.228127in}{1.351936in}}{\pgfqpoint{4.230961in}{1.345094in}}{\pgfqpoint{4.236005in}{1.340050in}}%
\pgfpathcurveto{\pgfqpoint{4.241049in}{1.335007in}}{\pgfqpoint{4.247890in}{1.332173in}}{\pgfqpoint{4.255023in}{1.332173in}}%
\pgfpathclose%
\pgfusepath{stroke,fill}%
\end{pgfscope}%
\begin{pgfscope}%
\pgfpathrectangle{\pgfqpoint{2.867647in}{0.500000in}}{\pgfqpoint{1.764706in}{1.700000in}}%
\pgfusepath{clip}%
\pgfsetbuttcap%
\pgfsetroundjoin%
\definecolor{currentfill}{rgb}{0.981377,0.920617,0.865369}%
\pgfsetfillcolor{currentfill}%
\pgfsetlinewidth{0.311001pt}%
\definecolor{currentstroke}{rgb}{1.000000,1.000000,1.000000}%
\pgfsetstrokecolor{currentstroke}%
\pgfsetdash{}{0pt}%
\pgfpathmoveto{\pgfqpoint{4.200604in}{1.306394in}}%
\pgfpathcurveto{\pgfqpoint{4.207737in}{1.306394in}}{\pgfqpoint{4.214579in}{1.309228in}}{\pgfqpoint{4.219622in}{1.314271in}}%
\pgfpathcurveto{\pgfqpoint{4.224666in}{1.319315in}}{\pgfqpoint{4.227500in}{1.326156in}}{\pgfqpoint{4.227500in}{1.333289in}}%
\pgfpathcurveto{\pgfqpoint{4.227500in}{1.340422in}}{\pgfqpoint{4.224666in}{1.347264in}}{\pgfqpoint{4.219622in}{1.352307in}}%
\pgfpathcurveto{\pgfqpoint{4.214579in}{1.357351in}}{\pgfqpoint{4.207737in}{1.360185in}}{\pgfqpoint{4.200604in}{1.360185in}}%
\pgfpathcurveto{\pgfqpoint{4.193471in}{1.360185in}}{\pgfqpoint{4.186630in}{1.357351in}}{\pgfqpoint{4.181586in}{1.352307in}}%
\pgfpathcurveto{\pgfqpoint{4.176542in}{1.347264in}}{\pgfqpoint{4.173709in}{1.340422in}}{\pgfqpoint{4.173709in}{1.333289in}}%
\pgfpathcurveto{\pgfqpoint{4.173709in}{1.326156in}}{\pgfqpoint{4.176542in}{1.319315in}}{\pgfqpoint{4.181586in}{1.314271in}}%
\pgfpathcurveto{\pgfqpoint{4.186630in}{1.309228in}}{\pgfqpoint{4.193471in}{1.306394in}}{\pgfqpoint{4.200604in}{1.306394in}}%
\pgfpathclose%
\pgfusepath{stroke,fill}%
\end{pgfscope}%
\begin{pgfscope}%
\pgfpathrectangle{\pgfqpoint{2.867647in}{0.500000in}}{\pgfqpoint{1.764706in}{1.700000in}}%
\pgfusepath{clip}%
\pgfsetbuttcap%
\pgfsetroundjoin%
\definecolor{currentfill}{rgb}{0.953816,0.463738,0.317699}%
\pgfsetfillcolor{currentfill}%
\pgfsetlinewidth{0.311001pt}%
\definecolor{currentstroke}{rgb}{1.000000,1.000000,1.000000}%
\pgfsetstrokecolor{currentstroke}%
\pgfsetdash{}{0pt}%
\pgfpathmoveto{\pgfqpoint{4.135877in}{1.809094in}}%
\pgfpathcurveto{\pgfqpoint{4.143010in}{1.809094in}}{\pgfqpoint{4.149852in}{1.811928in}}{\pgfqpoint{4.154895in}{1.816971in}}%
\pgfpathcurveto{\pgfqpoint{4.159939in}{1.822015in}}{\pgfqpoint{4.162773in}{1.828857in}}{\pgfqpoint{4.162773in}{1.835989in}}%
\pgfpathcurveto{\pgfqpoint{4.162773in}{1.843122in}}{\pgfqpoint{4.159939in}{1.849964in}}{\pgfqpoint{4.154895in}{1.855008in}}%
\pgfpathcurveto{\pgfqpoint{4.149852in}{1.860051in}}{\pgfqpoint{4.143010in}{1.862885in}}{\pgfqpoint{4.135877in}{1.862885in}}%
\pgfpathcurveto{\pgfqpoint{4.128744in}{1.862885in}}{\pgfqpoint{4.121903in}{1.860051in}}{\pgfqpoint{4.116859in}{1.855008in}}%
\pgfpathcurveto{\pgfqpoint{4.111815in}{1.849964in}}{\pgfqpoint{4.108981in}{1.843122in}}{\pgfqpoint{4.108981in}{1.835989in}}%
\pgfpathcurveto{\pgfqpoint{4.108981in}{1.828857in}}{\pgfqpoint{4.111815in}{1.822015in}}{\pgfqpoint{4.116859in}{1.816971in}}%
\pgfpathcurveto{\pgfqpoint{4.121903in}{1.811928in}}{\pgfqpoint{4.128744in}{1.809094in}}{\pgfqpoint{4.135877in}{1.809094in}}%
\pgfpathclose%
\pgfusepath{stroke,fill}%
\end{pgfscope}%
\begin{pgfscope}%
\pgfpathrectangle{\pgfqpoint{2.867647in}{0.500000in}}{\pgfqpoint{1.764706in}{1.700000in}}%
\pgfusepath{clip}%
\pgfsetbuttcap%
\pgfsetroundjoin%
\definecolor{currentfill}{rgb}{0.969359,0.803954,0.693832}%
\pgfsetfillcolor{currentfill}%
\pgfsetlinewidth{0.311001pt}%
\definecolor{currentstroke}{rgb}{1.000000,1.000000,1.000000}%
\pgfsetstrokecolor{currentstroke}%
\pgfsetdash{}{0pt}%
\pgfpathmoveto{\pgfqpoint{4.102360in}{1.366266in}}%
\pgfpathcurveto{\pgfqpoint{4.109493in}{1.366266in}}{\pgfqpoint{4.116334in}{1.369099in}}{\pgfqpoint{4.121378in}{1.374143in}}%
\pgfpathcurveto{\pgfqpoint{4.126422in}{1.379187in}}{\pgfqpoint{4.129256in}{1.386028in}}{\pgfqpoint{4.129256in}{1.393161in}}%
\pgfpathcurveto{\pgfqpoint{4.129256in}{1.400294in}}{\pgfqpoint{4.126422in}{1.407136in}}{\pgfqpoint{4.121378in}{1.412179in}}%
\pgfpathcurveto{\pgfqpoint{4.116334in}{1.417223in}}{\pgfqpoint{4.109493in}{1.420057in}}{\pgfqpoint{4.102360in}{1.420057in}}%
\pgfpathcurveto{\pgfqpoint{4.095227in}{1.420057in}}{\pgfqpoint{4.088385in}{1.417223in}}{\pgfqpoint{4.083342in}{1.412179in}}%
\pgfpathcurveto{\pgfqpoint{4.078298in}{1.407136in}}{\pgfqpoint{4.075464in}{1.400294in}}{\pgfqpoint{4.075464in}{1.393161in}}%
\pgfpathcurveto{\pgfqpoint{4.075464in}{1.386028in}}{\pgfqpoint{4.078298in}{1.379187in}}{\pgfqpoint{4.083342in}{1.374143in}}%
\pgfpathcurveto{\pgfqpoint{4.088385in}{1.369099in}}{\pgfqpoint{4.095227in}{1.366266in}}{\pgfqpoint{4.102360in}{1.366266in}}%
\pgfpathclose%
\pgfusepath{stroke,fill}%
\end{pgfscope}%
\begin{pgfscope}%
\pgfpathrectangle{\pgfqpoint{2.867647in}{0.500000in}}{\pgfqpoint{1.764706in}{1.700000in}}%
\pgfusepath{clip}%
\pgfsetbuttcap%
\pgfsetroundjoin%
\definecolor{currentfill}{rgb}{0.956817,0.498820,0.345554}%
\pgfsetfillcolor{currentfill}%
\pgfsetlinewidth{0.311001pt}%
\definecolor{currentstroke}{rgb}{1.000000,1.000000,1.000000}%
\pgfsetstrokecolor{currentstroke}%
\pgfsetdash{}{0pt}%
\pgfpathmoveto{\pgfqpoint{3.983429in}{0.827239in}}%
\pgfpathcurveto{\pgfqpoint{3.990562in}{0.827239in}}{\pgfqpoint{3.997404in}{0.830072in}}{\pgfqpoint{4.002448in}{0.835116in}}%
\pgfpathcurveto{\pgfqpoint{4.007491in}{0.840160in}}{\pgfqpoint{4.010325in}{0.847001in}}{\pgfqpoint{4.010325in}{0.854134in}}%
\pgfpathcurveto{\pgfqpoint{4.010325in}{0.861267in}}{\pgfqpoint{4.007491in}{0.868109in}}{\pgfqpoint{4.002448in}{0.873152in}}%
\pgfpathcurveto{\pgfqpoint{3.997404in}{0.878196in}}{\pgfqpoint{3.990562in}{0.881030in}}{\pgfqpoint{3.983429in}{0.881030in}}%
\pgfpathcurveto{\pgfqpoint{3.976297in}{0.881030in}}{\pgfqpoint{3.969455in}{0.878196in}}{\pgfqpoint{3.964411in}{0.873152in}}%
\pgfpathcurveto{\pgfqpoint{3.959368in}{0.868109in}}{\pgfqpoint{3.956534in}{0.861267in}}{\pgfqpoint{3.956534in}{0.854134in}}%
\pgfpathcurveto{\pgfqpoint{3.956534in}{0.847001in}}{\pgfqpoint{3.959368in}{0.840160in}}{\pgfqpoint{3.964411in}{0.835116in}}%
\pgfpathcurveto{\pgfqpoint{3.969455in}{0.830072in}}{\pgfqpoint{3.976297in}{0.827239in}}{\pgfqpoint{3.983429in}{0.827239in}}%
\pgfpathclose%
\pgfusepath{stroke,fill}%
\end{pgfscope}%
\begin{pgfscope}%
\pgfpathrectangle{\pgfqpoint{2.867647in}{0.500000in}}{\pgfqpoint{1.764706in}{1.700000in}}%
\pgfusepath{clip}%
\pgfsetbuttcap%
\pgfsetroundjoin%
\definecolor{currentfill}{rgb}{0.922239,0.282873,0.242296}%
\pgfsetfillcolor{currentfill}%
\pgfsetlinewidth{0.311001pt}%
\definecolor{currentstroke}{rgb}{1.000000,1.000000,1.000000}%
\pgfsetstrokecolor{currentstroke}%
\pgfsetdash{}{0pt}%
\pgfpathmoveto{\pgfqpoint{3.866072in}{1.840712in}}%
\pgfpathcurveto{\pgfqpoint{3.873204in}{1.840712in}}{\pgfqpoint{3.880046in}{1.843546in}}{\pgfqpoint{3.885090in}{1.848589in}}%
\pgfpathcurveto{\pgfqpoint{3.890133in}{1.853633in}}{\pgfqpoint{3.892967in}{1.860475in}}{\pgfqpoint{3.892967in}{1.867608in}}%
\pgfpathcurveto{\pgfqpoint{3.892967in}{1.874740in}}{\pgfqpoint{3.890133in}{1.881582in}}{\pgfqpoint{3.885090in}{1.886626in}}%
\pgfpathcurveto{\pgfqpoint{3.880046in}{1.891669in}}{\pgfqpoint{3.873204in}{1.894503in}}{\pgfqpoint{3.866072in}{1.894503in}}%
\pgfpathcurveto{\pgfqpoint{3.858939in}{1.894503in}}{\pgfqpoint{3.852097in}{1.891669in}}{\pgfqpoint{3.847054in}{1.886626in}}%
\pgfpathcurveto{\pgfqpoint{3.842010in}{1.881582in}}{\pgfqpoint{3.839176in}{1.874740in}}{\pgfqpoint{3.839176in}{1.867608in}}%
\pgfpathcurveto{\pgfqpoint{3.839176in}{1.860475in}}{\pgfqpoint{3.842010in}{1.853633in}}{\pgfqpoint{3.847054in}{1.848589in}}%
\pgfpathcurveto{\pgfqpoint{3.852097in}{1.843546in}}{\pgfqpoint{3.858939in}{1.840712in}}{\pgfqpoint{3.866072in}{1.840712in}}%
\pgfpathclose%
\pgfusepath{stroke,fill}%
\end{pgfscope}%
\begin{pgfscope}%
\pgfpathrectangle{\pgfqpoint{2.867647in}{0.500000in}}{\pgfqpoint{1.764706in}{1.700000in}}%
\pgfusepath{clip}%
\pgfsetbuttcap%
\pgfsetroundjoin%
\definecolor{currentfill}{rgb}{0.969803,0.809811,0.702523}%
\pgfsetfillcolor{currentfill}%
\pgfsetlinewidth{0.311001pt}%
\definecolor{currentstroke}{rgb}{1.000000,1.000000,1.000000}%
\pgfsetstrokecolor{currentstroke}%
\pgfsetdash{}{0pt}%
\pgfpathmoveto{\pgfqpoint{4.275138in}{1.282795in}}%
\pgfpathcurveto{\pgfqpoint{4.282271in}{1.282795in}}{\pgfqpoint{4.289113in}{1.285629in}}{\pgfqpoint{4.294156in}{1.290673in}}%
\pgfpathcurveto{\pgfqpoint{4.299200in}{1.295717in}}{\pgfqpoint{4.302034in}{1.302558in}}{\pgfqpoint{4.302034in}{1.309691in}}%
\pgfpathcurveto{\pgfqpoint{4.302034in}{1.316824in}}{\pgfqpoint{4.299200in}{1.323665in}}{\pgfqpoint{4.294156in}{1.328709in}}%
\pgfpathcurveto{\pgfqpoint{4.289113in}{1.333753in}}{\pgfqpoint{4.282271in}{1.336587in}}{\pgfqpoint{4.275138in}{1.336587in}}%
\pgfpathcurveto{\pgfqpoint{4.268005in}{1.336587in}}{\pgfqpoint{4.261164in}{1.333753in}}{\pgfqpoint{4.256120in}{1.328709in}}%
\pgfpathcurveto{\pgfqpoint{4.251076in}{1.323665in}}{\pgfqpoint{4.248242in}{1.316824in}}{\pgfqpoint{4.248242in}{1.309691in}}%
\pgfpathcurveto{\pgfqpoint{4.248242in}{1.302558in}}{\pgfqpoint{4.251076in}{1.295717in}}{\pgfqpoint{4.256120in}{1.290673in}}%
\pgfpathcurveto{\pgfqpoint{4.261164in}{1.285629in}}{\pgfqpoint{4.268005in}{1.282795in}}{\pgfqpoint{4.275138in}{1.282795in}}%
\pgfpathclose%
\pgfusepath{stroke,fill}%
\end{pgfscope}%
\begin{pgfscope}%
\pgfpathrectangle{\pgfqpoint{2.867647in}{0.500000in}}{\pgfqpoint{1.764706in}{1.700000in}}%
\pgfusepath{clip}%
\pgfsetbuttcap%
\pgfsetroundjoin%
\definecolor{currentfill}{rgb}{0.971202,0.827364,0.728520}%
\pgfsetfillcolor{currentfill}%
\pgfsetlinewidth{0.311001pt}%
\definecolor{currentstroke}{rgb}{1.000000,1.000000,1.000000}%
\pgfsetstrokecolor{currentstroke}%
\pgfsetdash{}{0pt}%
\pgfpathmoveto{\pgfqpoint{4.063720in}{1.545579in}}%
\pgfpathcurveto{\pgfqpoint{4.070853in}{1.545579in}}{\pgfqpoint{4.077694in}{1.548412in}}{\pgfqpoint{4.082738in}{1.553456in}}%
\pgfpathcurveto{\pgfqpoint{4.087782in}{1.558500in}}{\pgfqpoint{4.090615in}{1.565341in}}{\pgfqpoint{4.090615in}{1.572474in}}%
\pgfpathcurveto{\pgfqpoint{4.090615in}{1.579607in}}{\pgfqpoint{4.087782in}{1.586449in}}{\pgfqpoint{4.082738in}{1.591492in}}%
\pgfpathcurveto{\pgfqpoint{4.077694in}{1.596536in}}{\pgfqpoint{4.070853in}{1.599370in}}{\pgfqpoint{4.063720in}{1.599370in}}%
\pgfpathcurveto{\pgfqpoint{4.056587in}{1.599370in}}{\pgfqpoint{4.049745in}{1.596536in}}{\pgfqpoint{4.044702in}{1.591492in}}%
\pgfpathcurveto{\pgfqpoint{4.039658in}{1.586449in}}{\pgfqpoint{4.036824in}{1.579607in}}{\pgfqpoint{4.036824in}{1.572474in}}%
\pgfpathcurveto{\pgfqpoint{4.036824in}{1.565341in}}{\pgfqpoint{4.039658in}{1.558500in}}{\pgfqpoint{4.044702in}{1.553456in}}%
\pgfpathcurveto{\pgfqpoint{4.049745in}{1.548412in}}{\pgfqpoint{4.056587in}{1.545579in}}{\pgfqpoint{4.063720in}{1.545579in}}%
\pgfpathclose%
\pgfusepath{stroke,fill}%
\end{pgfscope}%
\begin{pgfscope}%
\pgfpathrectangle{\pgfqpoint{2.867647in}{0.500000in}}{\pgfqpoint{1.764706in}{1.700000in}}%
\pgfusepath{clip}%
\pgfsetbuttcap%
\pgfsetroundjoin%
\definecolor{currentfill}{rgb}{0.972726,0.844889,0.754401}%
\pgfsetfillcolor{currentfill}%
\pgfsetlinewidth{0.311001pt}%
\definecolor{currentstroke}{rgb}{1.000000,1.000000,1.000000}%
\pgfsetstrokecolor{currentstroke}%
\pgfsetdash{}{0pt}%
\pgfpathmoveto{\pgfqpoint{4.259578in}{1.364757in}}%
\pgfpathcurveto{\pgfqpoint{4.266711in}{1.364757in}}{\pgfqpoint{4.273552in}{1.367590in}}{\pgfqpoint{4.278596in}{1.372634in}}%
\pgfpathcurveto{\pgfqpoint{4.283640in}{1.377678in}}{\pgfqpoint{4.286474in}{1.384519in}}{\pgfqpoint{4.286474in}{1.391652in}}%
\pgfpathcurveto{\pgfqpoint{4.286474in}{1.398785in}}{\pgfqpoint{4.283640in}{1.405627in}}{\pgfqpoint{4.278596in}{1.410670in}}%
\pgfpathcurveto{\pgfqpoint{4.273552in}{1.415714in}}{\pgfqpoint{4.266711in}{1.418548in}}{\pgfqpoint{4.259578in}{1.418548in}}%
\pgfpathcurveto{\pgfqpoint{4.252445in}{1.418548in}}{\pgfqpoint{4.245603in}{1.415714in}}{\pgfqpoint{4.240560in}{1.410670in}}%
\pgfpathcurveto{\pgfqpoint{4.235516in}{1.405627in}}{\pgfqpoint{4.232682in}{1.398785in}}{\pgfqpoint{4.232682in}{1.391652in}}%
\pgfpathcurveto{\pgfqpoint{4.232682in}{1.384519in}}{\pgfqpoint{4.235516in}{1.377678in}}{\pgfqpoint{4.240560in}{1.372634in}}%
\pgfpathcurveto{\pgfqpoint{4.245603in}{1.367590in}}{\pgfqpoint{4.252445in}{1.364757in}}{\pgfqpoint{4.259578in}{1.364757in}}%
\pgfpathclose%
\pgfusepath{stroke,fill}%
\end{pgfscope}%
\begin{pgfscope}%
\pgfpathrectangle{\pgfqpoint{2.867647in}{0.500000in}}{\pgfqpoint{1.764706in}{1.700000in}}%
\pgfusepath{clip}%
\pgfsetbuttcap%
\pgfsetroundjoin%
\definecolor{currentfill}{rgb}{0.973832,0.856556,0.771584}%
\pgfsetfillcolor{currentfill}%
\pgfsetlinewidth{0.311001pt}%
\definecolor{currentstroke}{rgb}{1.000000,1.000000,1.000000}%
\pgfsetstrokecolor{currentstroke}%
\pgfsetdash{}{0pt}%
\pgfpathmoveto{\pgfqpoint{4.254449in}{1.279760in}}%
\pgfpathcurveto{\pgfqpoint{4.261582in}{1.279760in}}{\pgfqpoint{4.268424in}{1.282594in}}{\pgfqpoint{4.273467in}{1.287638in}}%
\pgfpathcurveto{\pgfqpoint{4.278511in}{1.292681in}}{\pgfqpoint{4.281345in}{1.299523in}}{\pgfqpoint{4.281345in}{1.306656in}}%
\pgfpathcurveto{\pgfqpoint{4.281345in}{1.313789in}}{\pgfqpoint{4.278511in}{1.320630in}}{\pgfqpoint{4.273467in}{1.325674in}}%
\pgfpathcurveto{\pgfqpoint{4.268424in}{1.330718in}}{\pgfqpoint{4.261582in}{1.333552in}}{\pgfqpoint{4.254449in}{1.333552in}}%
\pgfpathcurveto{\pgfqpoint{4.247316in}{1.333552in}}{\pgfqpoint{4.240475in}{1.330718in}}{\pgfqpoint{4.235431in}{1.325674in}}%
\pgfpathcurveto{\pgfqpoint{4.230387in}{1.320630in}}{\pgfqpoint{4.227554in}{1.313789in}}{\pgfqpoint{4.227554in}{1.306656in}}%
\pgfpathcurveto{\pgfqpoint{4.227554in}{1.299523in}}{\pgfqpoint{4.230387in}{1.292681in}}{\pgfqpoint{4.235431in}{1.287638in}}%
\pgfpathcurveto{\pgfqpoint{4.240475in}{1.282594in}}{\pgfqpoint{4.247316in}{1.279760in}}{\pgfqpoint{4.254449in}{1.279760in}}%
\pgfpathclose%
\pgfusepath{stroke,fill}%
\end{pgfscope}%
\begin{pgfscope}%
\pgfpathrectangle{\pgfqpoint{2.867647in}{0.500000in}}{\pgfqpoint{1.764706in}{1.700000in}}%
\pgfusepath{clip}%
\pgfsetbuttcap%
\pgfsetroundjoin%
\definecolor{currentfill}{rgb}{0.979124,0.903132,0.839793}%
\pgfsetfillcolor{currentfill}%
\pgfsetlinewidth{0.311001pt}%
\definecolor{currentstroke}{rgb}{1.000000,1.000000,1.000000}%
\pgfsetstrokecolor{currentstroke}%
\pgfsetdash{}{0pt}%
\pgfpathmoveto{\pgfqpoint{4.122020in}{1.588268in}}%
\pgfpathcurveto{\pgfqpoint{4.129153in}{1.588268in}}{\pgfqpoint{4.135995in}{1.591102in}}{\pgfqpoint{4.141038in}{1.596146in}}%
\pgfpathcurveto{\pgfqpoint{4.146082in}{1.601190in}}{\pgfqpoint{4.148916in}{1.608031in}}{\pgfqpoint{4.148916in}{1.615164in}}%
\pgfpathcurveto{\pgfqpoint{4.148916in}{1.622297in}}{\pgfqpoint{4.146082in}{1.629139in}}{\pgfqpoint{4.141038in}{1.634182in}}%
\pgfpathcurveto{\pgfqpoint{4.135995in}{1.639226in}}{\pgfqpoint{4.129153in}{1.642060in}}{\pgfqpoint{4.122020in}{1.642060in}}%
\pgfpathcurveto{\pgfqpoint{4.114887in}{1.642060in}}{\pgfqpoint{4.108046in}{1.639226in}}{\pgfqpoint{4.103002in}{1.634182in}}%
\pgfpathcurveto{\pgfqpoint{4.097958in}{1.629139in}}{\pgfqpoint{4.095124in}{1.622297in}}{\pgfqpoint{4.095124in}{1.615164in}}%
\pgfpathcurveto{\pgfqpoint{4.095124in}{1.608031in}}{\pgfqpoint{4.097958in}{1.601190in}}{\pgfqpoint{4.103002in}{1.596146in}}%
\pgfpathcurveto{\pgfqpoint{4.108046in}{1.591102in}}{\pgfqpoint{4.114887in}{1.588268in}}{\pgfqpoint{4.122020in}{1.588268in}}%
\pgfpathclose%
\pgfusepath{stroke,fill}%
\end{pgfscope}%
\begin{pgfscope}%
\pgfpathrectangle{\pgfqpoint{2.867647in}{0.500000in}}{\pgfqpoint{1.764706in}{1.700000in}}%
\pgfusepath{clip}%
\pgfsetbuttcap%
\pgfsetroundjoin%
\definecolor{currentfill}{rgb}{0.964679,0.682838,0.530002}%
\pgfsetfillcolor{currentfill}%
\pgfsetlinewidth{0.311001pt}%
\definecolor{currentstroke}{rgb}{1.000000,1.000000,1.000000}%
\pgfsetstrokecolor{currentstroke}%
\pgfsetdash{}{0pt}%
\pgfpathmoveto{\pgfqpoint{4.078503in}{0.881391in}}%
\pgfpathcurveto{\pgfqpoint{4.085636in}{0.881391in}}{\pgfqpoint{4.092478in}{0.884225in}}{\pgfqpoint{4.097521in}{0.889269in}}%
\pgfpathcurveto{\pgfqpoint{4.102565in}{0.894313in}}{\pgfqpoint{4.105399in}{0.901154in}}{\pgfqpoint{4.105399in}{0.908287in}}%
\pgfpathcurveto{\pgfqpoint{4.105399in}{0.915420in}}{\pgfqpoint{4.102565in}{0.922262in}}{\pgfqpoint{4.097521in}{0.927305in}}%
\pgfpathcurveto{\pgfqpoint{4.092478in}{0.932349in}}{\pgfqpoint{4.085636in}{0.935183in}}{\pgfqpoint{4.078503in}{0.935183in}}%
\pgfpathcurveto{\pgfqpoint{4.071370in}{0.935183in}}{\pgfqpoint{4.064529in}{0.932349in}}{\pgfqpoint{4.059485in}{0.927305in}}%
\pgfpathcurveto{\pgfqpoint{4.054441in}{0.922262in}}{\pgfqpoint{4.051607in}{0.915420in}}{\pgfqpoint{4.051607in}{0.908287in}}%
\pgfpathcurveto{\pgfqpoint{4.051607in}{0.901154in}}{\pgfqpoint{4.054441in}{0.894313in}}{\pgfqpoint{4.059485in}{0.889269in}}%
\pgfpathcurveto{\pgfqpoint{4.064529in}{0.884225in}}{\pgfqpoint{4.071370in}{0.881391in}}{\pgfqpoint{4.078503in}{0.881391in}}%
\pgfpathclose%
\pgfusepath{stroke,fill}%
\end{pgfscope}%
\begin{pgfscope}%
\pgfpathrectangle{\pgfqpoint{2.867647in}{0.500000in}}{\pgfqpoint{1.764706in}{1.700000in}}%
\pgfusepath{clip}%
\pgfsetbuttcap%
\pgfsetroundjoin%
\definecolor{currentfill}{rgb}{0.973832,0.856556,0.771584}%
\pgfsetfillcolor{currentfill}%
\pgfsetlinewidth{0.311001pt}%
\definecolor{currentstroke}{rgb}{1.000000,1.000000,1.000000}%
\pgfsetstrokecolor{currentstroke}%
\pgfsetdash{}{0pt}%
\pgfpathmoveto{\pgfqpoint{4.247910in}{1.412859in}}%
\pgfpathcurveto{\pgfqpoint{4.255043in}{1.412859in}}{\pgfqpoint{4.261884in}{1.415693in}}{\pgfqpoint{4.266928in}{1.420737in}}%
\pgfpathcurveto{\pgfqpoint{4.271972in}{1.425781in}}{\pgfqpoint{4.274805in}{1.432622in}}{\pgfqpoint{4.274805in}{1.439755in}}%
\pgfpathcurveto{\pgfqpoint{4.274805in}{1.446888in}}{\pgfqpoint{4.271972in}{1.453730in}}{\pgfqpoint{4.266928in}{1.458773in}}%
\pgfpathcurveto{\pgfqpoint{4.261884in}{1.463817in}}{\pgfqpoint{4.255043in}{1.466651in}}{\pgfqpoint{4.247910in}{1.466651in}}%
\pgfpathcurveto{\pgfqpoint{4.240777in}{1.466651in}}{\pgfqpoint{4.233935in}{1.463817in}}{\pgfqpoint{4.228892in}{1.458773in}}%
\pgfpathcurveto{\pgfqpoint{4.223848in}{1.453730in}}{\pgfqpoint{4.221014in}{1.446888in}}{\pgfqpoint{4.221014in}{1.439755in}}%
\pgfpathcurveto{\pgfqpoint{4.221014in}{1.432622in}}{\pgfqpoint{4.223848in}{1.425781in}}{\pgfqpoint{4.228892in}{1.420737in}}%
\pgfpathcurveto{\pgfqpoint{4.233935in}{1.415693in}}{\pgfqpoint{4.240777in}{1.412859in}}{\pgfqpoint{4.247910in}{1.412859in}}%
\pgfpathclose%
\pgfusepath{stroke,fill}%
\end{pgfscope}%
\begin{pgfscope}%
\pgfpathrectangle{\pgfqpoint{2.867647in}{0.500000in}}{\pgfqpoint{1.764706in}{1.700000in}}%
\pgfusepath{clip}%
\pgfsetbuttcap%
\pgfsetroundjoin%
\definecolor{currentfill}{rgb}{0.972726,0.844889,0.754401}%
\pgfsetfillcolor{currentfill}%
\pgfsetlinewidth{0.311001pt}%
\definecolor{currentstroke}{rgb}{1.000000,1.000000,1.000000}%
\pgfsetstrokecolor{currentstroke}%
\pgfsetdash{}{0pt}%
\pgfpathmoveto{\pgfqpoint{4.120681in}{1.306621in}}%
\pgfpathcurveto{\pgfqpoint{4.127814in}{1.306621in}}{\pgfqpoint{4.134655in}{1.309455in}}{\pgfqpoint{4.139699in}{1.314499in}}%
\pgfpathcurveto{\pgfqpoint{4.144743in}{1.319543in}}{\pgfqpoint{4.147576in}{1.326384in}}{\pgfqpoint{4.147576in}{1.333517in}}%
\pgfpathcurveto{\pgfqpoint{4.147576in}{1.340650in}}{\pgfqpoint{4.144743in}{1.347492in}}{\pgfqpoint{4.139699in}{1.352535in}}%
\pgfpathcurveto{\pgfqpoint{4.134655in}{1.357579in}}{\pgfqpoint{4.127814in}{1.360413in}}{\pgfqpoint{4.120681in}{1.360413in}}%
\pgfpathcurveto{\pgfqpoint{4.113548in}{1.360413in}}{\pgfqpoint{4.106706in}{1.357579in}}{\pgfqpoint{4.101663in}{1.352535in}}%
\pgfpathcurveto{\pgfqpoint{4.096619in}{1.347492in}}{\pgfqpoint{4.093785in}{1.340650in}}{\pgfqpoint{4.093785in}{1.333517in}}%
\pgfpathcurveto{\pgfqpoint{4.093785in}{1.326384in}}{\pgfqpoint{4.096619in}{1.319543in}}{\pgfqpoint{4.101663in}{1.314499in}}%
\pgfpathcurveto{\pgfqpoint{4.106706in}{1.309455in}}{\pgfqpoint{4.113548in}{1.306621in}}{\pgfqpoint{4.120681in}{1.306621in}}%
\pgfpathclose%
\pgfusepath{stroke,fill}%
\end{pgfscope}%
\begin{pgfscope}%
\pgfpathrectangle{\pgfqpoint{2.867647in}{0.500000in}}{\pgfqpoint{1.764706in}{1.700000in}}%
\pgfusepath{clip}%
\pgfsetbuttcap%
\pgfsetroundjoin%
\definecolor{currentfill}{rgb}{0.965042,0.701564,0.552889}%
\pgfsetfillcolor{currentfill}%
\pgfsetlinewidth{0.311001pt}%
\definecolor{currentstroke}{rgb}{1.000000,1.000000,1.000000}%
\pgfsetstrokecolor{currentstroke}%
\pgfsetdash{}{0pt}%
\pgfpathmoveto{\pgfqpoint{4.102006in}{1.757909in}}%
\pgfpathcurveto{\pgfqpoint{4.109139in}{1.757909in}}{\pgfqpoint{4.115981in}{1.760743in}}{\pgfqpoint{4.121024in}{1.765786in}}%
\pgfpathcurveto{\pgfqpoint{4.126068in}{1.770830in}}{\pgfqpoint{4.128902in}{1.777672in}}{\pgfqpoint{4.128902in}{1.784804in}}%
\pgfpathcurveto{\pgfqpoint{4.128902in}{1.791937in}}{\pgfqpoint{4.126068in}{1.798779in}}{\pgfqpoint{4.121024in}{1.803823in}}%
\pgfpathcurveto{\pgfqpoint{4.115981in}{1.808866in}}{\pgfqpoint{4.109139in}{1.811700in}}{\pgfqpoint{4.102006in}{1.811700in}}%
\pgfpathcurveto{\pgfqpoint{4.094873in}{1.811700in}}{\pgfqpoint{4.088032in}{1.808866in}}{\pgfqpoint{4.082988in}{1.803823in}}%
\pgfpathcurveto{\pgfqpoint{4.077944in}{1.798779in}}{\pgfqpoint{4.075110in}{1.791937in}}{\pgfqpoint{4.075110in}{1.784804in}}%
\pgfpathcurveto{\pgfqpoint{4.075110in}{1.777672in}}{\pgfqpoint{4.077944in}{1.770830in}}{\pgfqpoint{4.082988in}{1.765786in}}%
\pgfpathcurveto{\pgfqpoint{4.088032in}{1.760743in}}{\pgfqpoint{4.094873in}{1.757909in}}{\pgfqpoint{4.102006in}{1.757909in}}%
\pgfpathclose%
\pgfusepath{stroke,fill}%
\end{pgfscope}%
\begin{pgfscope}%
\pgfpathrectangle{\pgfqpoint{2.867647in}{0.500000in}}{\pgfqpoint{1.764706in}{1.700000in}}%
\pgfusepath{clip}%
\pgfsetbuttcap%
\pgfsetroundjoin%
\definecolor{currentfill}{rgb}{0.979124,0.903132,0.839793}%
\pgfsetfillcolor{currentfill}%
\pgfsetlinewidth{0.311001pt}%
\definecolor{currentstroke}{rgb}{1.000000,1.000000,1.000000}%
\pgfsetstrokecolor{currentstroke}%
\pgfsetdash{}{0pt}%
\pgfpathmoveto{\pgfqpoint{4.128045in}{1.100925in}}%
\pgfpathcurveto{\pgfqpoint{4.135178in}{1.100925in}}{\pgfqpoint{4.142019in}{1.103759in}}{\pgfqpoint{4.147063in}{1.108803in}}%
\pgfpathcurveto{\pgfqpoint{4.152107in}{1.113846in}}{\pgfqpoint{4.154941in}{1.120688in}}{\pgfqpoint{4.154941in}{1.127821in}}%
\pgfpathcurveto{\pgfqpoint{4.154941in}{1.134954in}}{\pgfqpoint{4.152107in}{1.141795in}}{\pgfqpoint{4.147063in}{1.146839in}}%
\pgfpathcurveto{\pgfqpoint{4.142019in}{1.151883in}}{\pgfqpoint{4.135178in}{1.154717in}}{\pgfqpoint{4.128045in}{1.154717in}}%
\pgfpathcurveto{\pgfqpoint{4.120912in}{1.154717in}}{\pgfqpoint{4.114070in}{1.151883in}}{\pgfqpoint{4.109027in}{1.146839in}}%
\pgfpathcurveto{\pgfqpoint{4.103983in}{1.141795in}}{\pgfqpoint{4.101149in}{1.134954in}}{\pgfqpoint{4.101149in}{1.127821in}}%
\pgfpathcurveto{\pgfqpoint{4.101149in}{1.120688in}}{\pgfqpoint{4.103983in}{1.113846in}}{\pgfqpoint{4.109027in}{1.108803in}}%
\pgfpathcurveto{\pgfqpoint{4.114070in}{1.103759in}}{\pgfqpoint{4.120912in}{1.100925in}}{\pgfqpoint{4.128045in}{1.100925in}}%
\pgfpathclose%
\pgfusepath{stroke,fill}%
\end{pgfscope}%
\begin{pgfscope}%
\pgfpathrectangle{\pgfqpoint{2.867647in}{0.500000in}}{\pgfqpoint{1.764706in}{1.700000in}}%
\pgfusepath{clip}%
\pgfsetbuttcap%
\pgfsetroundjoin%
\definecolor{currentfill}{rgb}{0.971202,0.827364,0.728520}%
\pgfsetfillcolor{currentfill}%
\pgfsetlinewidth{0.311001pt}%
\definecolor{currentstroke}{rgb}{1.000000,1.000000,1.000000}%
\pgfsetstrokecolor{currentstroke}%
\pgfsetdash{}{0pt}%
\pgfpathmoveto{\pgfqpoint{4.131518in}{0.972915in}}%
\pgfpathcurveto{\pgfqpoint{4.138651in}{0.972915in}}{\pgfqpoint{4.145492in}{0.975749in}}{\pgfqpoint{4.150536in}{0.980793in}}%
\pgfpathcurveto{\pgfqpoint{4.155580in}{0.985837in}}{\pgfqpoint{4.158414in}{0.992678in}}{\pgfqpoint{4.158414in}{0.999811in}}%
\pgfpathcurveto{\pgfqpoint{4.158414in}{1.006944in}}{\pgfqpoint{4.155580in}{1.013786in}}{\pgfqpoint{4.150536in}{1.018829in}}%
\pgfpathcurveto{\pgfqpoint{4.145492in}{1.023873in}}{\pgfqpoint{4.138651in}{1.026707in}}{\pgfqpoint{4.131518in}{1.026707in}}%
\pgfpathcurveto{\pgfqpoint{4.124385in}{1.026707in}}{\pgfqpoint{4.117543in}{1.023873in}}{\pgfqpoint{4.112500in}{1.018829in}}%
\pgfpathcurveto{\pgfqpoint{4.107456in}{1.013786in}}{\pgfqpoint{4.104622in}{1.006944in}}{\pgfqpoint{4.104622in}{0.999811in}}%
\pgfpathcurveto{\pgfqpoint{4.104622in}{0.992678in}}{\pgfqpoint{4.107456in}{0.985837in}}{\pgfqpoint{4.112500in}{0.980793in}}%
\pgfpathcurveto{\pgfqpoint{4.117543in}{0.975749in}}{\pgfqpoint{4.124385in}{0.972915in}}{\pgfqpoint{4.131518in}{0.972915in}}%
\pgfpathclose%
\pgfusepath{stroke,fill}%
\end{pgfscope}%
\begin{pgfscope}%
\pgfpathrectangle{\pgfqpoint{2.867647in}{0.500000in}}{\pgfqpoint{1.764706in}{1.700000in}}%
\pgfusepath{clip}%
\pgfsetbuttcap%
\pgfsetroundjoin%
\definecolor{currentfill}{rgb}{0.978376,0.897317,0.831308}%
\pgfsetfillcolor{currentfill}%
\pgfsetlinewidth{0.311001pt}%
\definecolor{currentstroke}{rgb}{1.000000,1.000000,1.000000}%
\pgfsetstrokecolor{currentstroke}%
\pgfsetdash{}{0pt}%
\pgfpathmoveto{\pgfqpoint{4.146518in}{1.066452in}}%
\pgfpathcurveto{\pgfqpoint{4.153651in}{1.066452in}}{\pgfqpoint{4.160492in}{1.069286in}}{\pgfqpoint{4.165536in}{1.074329in}}%
\pgfpathcurveto{\pgfqpoint{4.170580in}{1.079373in}}{\pgfqpoint{4.173414in}{1.086215in}}{\pgfqpoint{4.173414in}{1.093348in}}%
\pgfpathcurveto{\pgfqpoint{4.173414in}{1.100480in}}{\pgfqpoint{4.170580in}{1.107322in}}{\pgfqpoint{4.165536in}{1.112366in}}%
\pgfpathcurveto{\pgfqpoint{4.160492in}{1.117409in}}{\pgfqpoint{4.153651in}{1.120243in}}{\pgfqpoint{4.146518in}{1.120243in}}%
\pgfpathcurveto{\pgfqpoint{4.139385in}{1.120243in}}{\pgfqpoint{4.132544in}{1.117409in}}{\pgfqpoint{4.127500in}{1.112366in}}%
\pgfpathcurveto{\pgfqpoint{4.122456in}{1.107322in}}{\pgfqpoint{4.119622in}{1.100480in}}{\pgfqpoint{4.119622in}{1.093348in}}%
\pgfpathcurveto{\pgfqpoint{4.119622in}{1.086215in}}{\pgfqpoint{4.122456in}{1.079373in}}{\pgfqpoint{4.127500in}{1.074329in}}%
\pgfpathcurveto{\pgfqpoint{4.132544in}{1.069286in}}{\pgfqpoint{4.139385in}{1.066452in}}{\pgfqpoint{4.146518in}{1.066452in}}%
\pgfpathclose%
\pgfusepath{stroke,fill}%
\end{pgfscope}%
\begin{pgfscope}%
\pgfpathrectangle{\pgfqpoint{2.867647in}{0.500000in}}{\pgfqpoint{1.764706in}{1.700000in}}%
\pgfusepath{clip}%
\pgfsetbuttcap%
\pgfsetroundjoin%
\definecolor{currentfill}{rgb}{0.962018,0.586477,0.424918}%
\pgfsetfillcolor{currentfill}%
\pgfsetlinewidth{0.311001pt}%
\definecolor{currentstroke}{rgb}{1.000000,1.000000,1.000000}%
\pgfsetstrokecolor{currentstroke}%
\pgfsetdash{}{0pt}%
\pgfpathmoveto{\pgfqpoint{4.270020in}{1.619920in}}%
\pgfpathcurveto{\pgfqpoint{4.277152in}{1.619920in}}{\pgfqpoint{4.283994in}{1.622754in}}{\pgfqpoint{4.289038in}{1.627797in}}%
\pgfpathcurveto{\pgfqpoint{4.294081in}{1.632841in}}{\pgfqpoint{4.296915in}{1.639683in}}{\pgfqpoint{4.296915in}{1.646816in}}%
\pgfpathcurveto{\pgfqpoint{4.296915in}{1.653948in}}{\pgfqpoint{4.294081in}{1.660790in}}{\pgfqpoint{4.289038in}{1.665834in}}%
\pgfpathcurveto{\pgfqpoint{4.283994in}{1.670877in}}{\pgfqpoint{4.277152in}{1.673711in}}{\pgfqpoint{4.270020in}{1.673711in}}%
\pgfpathcurveto{\pgfqpoint{4.262887in}{1.673711in}}{\pgfqpoint{4.256045in}{1.670877in}}{\pgfqpoint{4.251001in}{1.665834in}}%
\pgfpathcurveto{\pgfqpoint{4.245958in}{1.660790in}}{\pgfqpoint{4.243124in}{1.653948in}}{\pgfqpoint{4.243124in}{1.646816in}}%
\pgfpathcurveto{\pgfqpoint{4.243124in}{1.639683in}}{\pgfqpoint{4.245958in}{1.632841in}}{\pgfqpoint{4.251001in}{1.627797in}}%
\pgfpathcurveto{\pgfqpoint{4.256045in}{1.622754in}}{\pgfqpoint{4.262887in}{1.619920in}}{\pgfqpoint{4.270020in}{1.619920in}}%
\pgfpathclose%
\pgfusepath{stroke,fill}%
\end{pgfscope}%
\begin{pgfscope}%
\pgfpathrectangle{\pgfqpoint{2.867647in}{0.500000in}}{\pgfqpoint{1.764706in}{1.700000in}}%
\pgfusepath{clip}%
\pgfsetbuttcap%
\pgfsetroundjoin%
\definecolor{currentfill}{rgb}{0.960043,0.546576,0.387029}%
\pgfsetfillcolor{currentfill}%
\pgfsetlinewidth{0.311001pt}%
\definecolor{currentstroke}{rgb}{1.000000,1.000000,1.000000}%
\pgfsetstrokecolor{currentstroke}%
\pgfsetdash{}{0pt}%
\pgfpathmoveto{\pgfqpoint{3.928087in}{0.947136in}}%
\pgfpathcurveto{\pgfqpoint{3.935220in}{0.947136in}}{\pgfqpoint{3.942062in}{0.949970in}}{\pgfqpoint{3.947106in}{0.955013in}}%
\pgfpathcurveto{\pgfqpoint{3.952149in}{0.960057in}}{\pgfqpoint{3.954983in}{0.966899in}}{\pgfqpoint{3.954983in}{0.974032in}}%
\pgfpathcurveto{\pgfqpoint{3.954983in}{0.981164in}}{\pgfqpoint{3.952149in}{0.988006in}}{\pgfqpoint{3.947106in}{0.993050in}}%
\pgfpathcurveto{\pgfqpoint{3.942062in}{0.998093in}}{\pgfqpoint{3.935220in}{1.000927in}}{\pgfqpoint{3.928087in}{1.000927in}}%
\pgfpathcurveto{\pgfqpoint{3.920955in}{1.000927in}}{\pgfqpoint{3.914113in}{0.998093in}}{\pgfqpoint{3.909069in}{0.993050in}}%
\pgfpathcurveto{\pgfqpoint{3.904026in}{0.988006in}}{\pgfqpoint{3.901192in}{0.981164in}}{\pgfqpoint{3.901192in}{0.974032in}}%
\pgfpathcurveto{\pgfqpoint{3.901192in}{0.966899in}}{\pgfqpoint{3.904026in}{0.960057in}}{\pgfqpoint{3.909069in}{0.955013in}}%
\pgfpathcurveto{\pgfqpoint{3.914113in}{0.949970in}}{\pgfqpoint{3.920955in}{0.947136in}}{\pgfqpoint{3.928087in}{0.947136in}}%
\pgfpathclose%
\pgfusepath{stroke,fill}%
\end{pgfscope}%
\begin{pgfscope}%
\pgfpathrectangle{\pgfqpoint{2.867647in}{0.500000in}}{\pgfqpoint{1.764706in}{1.700000in}}%
\pgfusepath{clip}%
\pgfsetbuttcap%
\pgfsetroundjoin%
\definecolor{currentfill}{rgb}{0.979891,0.908948,0.848279}%
\pgfsetfillcolor{currentfill}%
\pgfsetlinewidth{0.311001pt}%
\definecolor{currentstroke}{rgb}{1.000000,1.000000,1.000000}%
\pgfsetstrokecolor{currentstroke}%
\pgfsetdash{}{0pt}%
\pgfpathmoveto{\pgfqpoint{4.148279in}{1.561820in}}%
\pgfpathcurveto{\pgfqpoint{4.155412in}{1.561820in}}{\pgfqpoint{4.162254in}{1.564654in}}{\pgfqpoint{4.167297in}{1.569697in}}%
\pgfpathcurveto{\pgfqpoint{4.172341in}{1.574741in}}{\pgfqpoint{4.175175in}{1.581583in}}{\pgfqpoint{4.175175in}{1.588715in}}%
\pgfpathcurveto{\pgfqpoint{4.175175in}{1.595848in}}{\pgfqpoint{4.172341in}{1.602690in}}{\pgfqpoint{4.167297in}{1.607734in}}%
\pgfpathcurveto{\pgfqpoint{4.162254in}{1.612777in}}{\pgfqpoint{4.155412in}{1.615611in}}{\pgfqpoint{4.148279in}{1.615611in}}%
\pgfpathcurveto{\pgfqpoint{4.141146in}{1.615611in}}{\pgfqpoint{4.134305in}{1.612777in}}{\pgfqpoint{4.129261in}{1.607734in}}%
\pgfpathcurveto{\pgfqpoint{4.124217in}{1.602690in}}{\pgfqpoint{4.121383in}{1.595848in}}{\pgfqpoint{4.121383in}{1.588715in}}%
\pgfpathcurveto{\pgfqpoint{4.121383in}{1.581583in}}{\pgfqpoint{4.124217in}{1.574741in}}{\pgfqpoint{4.129261in}{1.569697in}}%
\pgfpathcurveto{\pgfqpoint{4.134305in}{1.564654in}}{\pgfqpoint{4.141146in}{1.561820in}}{\pgfqpoint{4.148279in}{1.561820in}}%
\pgfpathclose%
\pgfusepath{stroke,fill}%
\end{pgfscope}%
\begin{pgfscope}%
\pgfpathrectangle{\pgfqpoint{2.867647in}{0.500000in}}{\pgfqpoint{1.764706in}{1.700000in}}%
\pgfusepath{clip}%
\pgfsetbuttcap%
\pgfsetroundjoin%
\definecolor{currentfill}{rgb}{0.967735,0.780441,0.659127}%
\pgfsetfillcolor{currentfill}%
\pgfsetlinewidth{0.311001pt}%
\definecolor{currentstroke}{rgb}{1.000000,1.000000,1.000000}%
\pgfsetstrokecolor{currentstroke}%
\pgfsetdash{}{0pt}%
\pgfpathmoveto{\pgfqpoint{4.171356in}{0.975996in}}%
\pgfpathcurveto{\pgfqpoint{4.178489in}{0.975996in}}{\pgfqpoint{4.185330in}{0.978830in}}{\pgfqpoint{4.190374in}{0.983874in}}%
\pgfpathcurveto{\pgfqpoint{4.195418in}{0.988918in}}{\pgfqpoint{4.198251in}{0.995759in}}{\pgfqpoint{4.198251in}{1.002892in}}%
\pgfpathcurveto{\pgfqpoint{4.198251in}{1.010025in}}{\pgfqpoint{4.195418in}{1.016866in}}{\pgfqpoint{4.190374in}{1.021910in}}%
\pgfpathcurveto{\pgfqpoint{4.185330in}{1.026954in}}{\pgfqpoint{4.178489in}{1.029788in}}{\pgfqpoint{4.171356in}{1.029788in}}%
\pgfpathcurveto{\pgfqpoint{4.164223in}{1.029788in}}{\pgfqpoint{4.157381in}{1.026954in}}{\pgfqpoint{4.152338in}{1.021910in}}%
\pgfpathcurveto{\pgfqpoint{4.147294in}{1.016866in}}{\pgfqpoint{4.144460in}{1.010025in}}{\pgfqpoint{4.144460in}{1.002892in}}%
\pgfpathcurveto{\pgfqpoint{4.144460in}{0.995759in}}{\pgfqpoint{4.147294in}{0.988918in}}{\pgfqpoint{4.152338in}{0.983874in}}%
\pgfpathcurveto{\pgfqpoint{4.157381in}{0.978830in}}{\pgfqpoint{4.164223in}{0.975996in}}{\pgfqpoint{4.171356in}{0.975996in}}%
\pgfpathclose%
\pgfusepath{stroke,fill}%
\end{pgfscope}%
\begin{pgfscope}%
\pgfpathrectangle{\pgfqpoint{2.867647in}{0.500000in}}{\pgfqpoint{1.764706in}{1.700000in}}%
\pgfusepath{clip}%
\pgfsetbuttcap%
\pgfsetroundjoin%
\definecolor{currentfill}{rgb}{0.979124,0.903132,0.839793}%
\pgfsetfillcolor{currentfill}%
\pgfsetlinewidth{0.311001pt}%
\definecolor{currentstroke}{rgb}{1.000000,1.000000,1.000000}%
\pgfsetstrokecolor{currentstroke}%
\pgfsetdash{}{0pt}%
\pgfpathmoveto{\pgfqpoint{4.223413in}{1.333485in}}%
\pgfpathcurveto{\pgfqpoint{4.230546in}{1.333485in}}{\pgfqpoint{4.237388in}{1.336319in}}{\pgfqpoint{4.242432in}{1.341362in}}%
\pgfpathcurveto{\pgfqpoint{4.247475in}{1.346406in}}{\pgfqpoint{4.250309in}{1.353248in}}{\pgfqpoint{4.250309in}{1.360380in}}%
\pgfpathcurveto{\pgfqpoint{4.250309in}{1.367513in}}{\pgfqpoint{4.247475in}{1.374355in}}{\pgfqpoint{4.242432in}{1.379398in}}%
\pgfpathcurveto{\pgfqpoint{4.237388in}{1.384442in}}{\pgfqpoint{4.230546in}{1.387276in}}{\pgfqpoint{4.223413in}{1.387276in}}%
\pgfpathcurveto{\pgfqpoint{4.216281in}{1.387276in}}{\pgfqpoint{4.209439in}{1.384442in}}{\pgfqpoint{4.204395in}{1.379398in}}%
\pgfpathcurveto{\pgfqpoint{4.199352in}{1.374355in}}{\pgfqpoint{4.196518in}{1.367513in}}{\pgfqpoint{4.196518in}{1.360380in}}%
\pgfpathcurveto{\pgfqpoint{4.196518in}{1.353248in}}{\pgfqpoint{4.199352in}{1.346406in}}{\pgfqpoint{4.204395in}{1.341362in}}%
\pgfpathcurveto{\pgfqpoint{4.209439in}{1.336319in}}{\pgfqpoint{4.216281in}{1.333485in}}{\pgfqpoint{4.223413in}{1.333485in}}%
\pgfpathclose%
\pgfusepath{stroke,fill}%
\end{pgfscope}%
\begin{pgfscope}%
\pgfpathrectangle{\pgfqpoint{2.867647in}{0.500000in}}{\pgfqpoint{1.764706in}{1.700000in}}%
\pgfusepath{clip}%
\pgfsetbuttcap%
\pgfsetroundjoin%
\definecolor{currentfill}{rgb}{0.964799,0.689101,0.537560}%
\pgfsetfillcolor{currentfill}%
\pgfsetlinewidth{0.311001pt}%
\definecolor{currentstroke}{rgb}{1.000000,1.000000,1.000000}%
\pgfsetstrokecolor{currentstroke}%
\pgfsetdash{}{0pt}%
\pgfpathmoveto{\pgfqpoint{3.975067in}{1.682862in}}%
\pgfpathcurveto{\pgfqpoint{3.982200in}{1.682862in}}{\pgfqpoint{3.989042in}{1.685696in}}{\pgfqpoint{3.994086in}{1.690740in}}%
\pgfpathcurveto{\pgfqpoint{3.999129in}{1.695783in}}{\pgfqpoint{4.001963in}{1.702625in}}{\pgfqpoint{4.001963in}{1.709758in}}%
\pgfpathcurveto{\pgfqpoint{4.001963in}{1.716891in}}{\pgfqpoint{3.999129in}{1.723732in}}{\pgfqpoint{3.994086in}{1.728776in}}%
\pgfpathcurveto{\pgfqpoint{3.989042in}{1.733820in}}{\pgfqpoint{3.982200in}{1.736654in}}{\pgfqpoint{3.975067in}{1.736654in}}%
\pgfpathcurveto{\pgfqpoint{3.967935in}{1.736654in}}{\pgfqpoint{3.961093in}{1.733820in}}{\pgfqpoint{3.956049in}{1.728776in}}%
\pgfpathcurveto{\pgfqpoint{3.951006in}{1.723732in}}{\pgfqpoint{3.948172in}{1.716891in}}{\pgfqpoint{3.948172in}{1.709758in}}%
\pgfpathcurveto{\pgfqpoint{3.948172in}{1.702625in}}{\pgfqpoint{3.951006in}{1.695783in}}{\pgfqpoint{3.956049in}{1.690740in}}%
\pgfpathcurveto{\pgfqpoint{3.961093in}{1.685696in}}{\pgfqpoint{3.967935in}{1.682862in}}{\pgfqpoint{3.975067in}{1.682862in}}%
\pgfpathclose%
\pgfusepath{stroke,fill}%
\end{pgfscope}%
\begin{pgfscope}%
\pgfpathrectangle{\pgfqpoint{2.867647in}{0.500000in}}{\pgfqpoint{1.764706in}{1.700000in}}%
\pgfusepath{clip}%
\pgfsetbuttcap%
\pgfsetroundjoin%
\definecolor{currentfill}{rgb}{0.964032,0.651225,0.493258}%
\pgfsetfillcolor{currentfill}%
\pgfsetlinewidth{0.311001pt}%
\definecolor{currentstroke}{rgb}{1.000000,1.000000,1.000000}%
\pgfsetstrokecolor{currentstroke}%
\pgfsetdash{}{0pt}%
\pgfpathmoveto{\pgfqpoint{4.320748in}{1.353040in}}%
\pgfpathcurveto{\pgfqpoint{4.327881in}{1.353040in}}{\pgfqpoint{4.334722in}{1.355874in}}{\pgfqpoint{4.339766in}{1.360918in}}%
\pgfpathcurveto{\pgfqpoint{4.344810in}{1.365961in}}{\pgfqpoint{4.347643in}{1.372803in}}{\pgfqpoint{4.347643in}{1.379936in}}%
\pgfpathcurveto{\pgfqpoint{4.347643in}{1.387069in}}{\pgfqpoint{4.344810in}{1.393910in}}{\pgfqpoint{4.339766in}{1.398954in}}%
\pgfpathcurveto{\pgfqpoint{4.334722in}{1.403998in}}{\pgfqpoint{4.327881in}{1.406831in}}{\pgfqpoint{4.320748in}{1.406831in}}%
\pgfpathcurveto{\pgfqpoint{4.313615in}{1.406831in}}{\pgfqpoint{4.306773in}{1.403998in}}{\pgfqpoint{4.301730in}{1.398954in}}%
\pgfpathcurveto{\pgfqpoint{4.296686in}{1.393910in}}{\pgfqpoint{4.293852in}{1.387069in}}{\pgfqpoint{4.293852in}{1.379936in}}%
\pgfpathcurveto{\pgfqpoint{4.293852in}{1.372803in}}{\pgfqpoint{4.296686in}{1.365961in}}{\pgfqpoint{4.301730in}{1.360918in}}%
\pgfpathcurveto{\pgfqpoint{4.306773in}{1.355874in}}{\pgfqpoint{4.313615in}{1.353040in}}{\pgfqpoint{4.320748in}{1.353040in}}%
\pgfpathclose%
\pgfusepath{stroke,fill}%
\end{pgfscope}%
\begin{pgfscope}%
\pgfpathrectangle{\pgfqpoint{2.867647in}{0.500000in}}{\pgfqpoint{1.764706in}{1.700000in}}%
\pgfusepath{clip}%
\pgfsetbuttcap%
\pgfsetroundjoin%
\definecolor{currentfill}{rgb}{0.972201,0.839051,0.745789}%
\pgfsetfillcolor{currentfill}%
\pgfsetlinewidth{0.311001pt}%
\definecolor{currentstroke}{rgb}{1.000000,1.000000,1.000000}%
\pgfsetstrokecolor{currentstroke}%
\pgfsetdash{}{0pt}%
\pgfpathmoveto{\pgfqpoint{4.222084in}{1.094640in}}%
\pgfpathcurveto{\pgfqpoint{4.229217in}{1.094640in}}{\pgfqpoint{4.236059in}{1.097474in}}{\pgfqpoint{4.241102in}{1.102518in}}%
\pgfpathcurveto{\pgfqpoint{4.246146in}{1.107561in}}{\pgfqpoint{4.248980in}{1.114403in}}{\pgfqpoint{4.248980in}{1.121536in}}%
\pgfpathcurveto{\pgfqpoint{4.248980in}{1.128669in}}{\pgfqpoint{4.246146in}{1.135510in}}{\pgfqpoint{4.241102in}{1.140554in}}%
\pgfpathcurveto{\pgfqpoint{4.236059in}{1.145598in}}{\pgfqpoint{4.229217in}{1.148432in}}{\pgfqpoint{4.222084in}{1.148432in}}%
\pgfpathcurveto{\pgfqpoint{4.214952in}{1.148432in}}{\pgfqpoint{4.208110in}{1.145598in}}{\pgfqpoint{4.203066in}{1.140554in}}%
\pgfpathcurveto{\pgfqpoint{4.198023in}{1.135510in}}{\pgfqpoint{4.195189in}{1.128669in}}{\pgfqpoint{4.195189in}{1.121536in}}%
\pgfpathcurveto{\pgfqpoint{4.195189in}{1.114403in}}{\pgfqpoint{4.198023in}{1.107561in}}{\pgfqpoint{4.203066in}{1.102518in}}%
\pgfpathcurveto{\pgfqpoint{4.208110in}{1.097474in}}{\pgfqpoint{4.214952in}{1.094640in}}{\pgfqpoint{4.222084in}{1.094640in}}%
\pgfpathclose%
\pgfusepath{stroke,fill}%
\end{pgfscope}%
\begin{pgfscope}%
\pgfpathrectangle{\pgfqpoint{2.867647in}{0.500000in}}{\pgfqpoint{1.764706in}{1.700000in}}%
\pgfusepath{clip}%
\pgfsetbuttcap%
\pgfsetroundjoin%
\definecolor{currentfill}{rgb}{0.980678,0.914765,0.856766}%
\pgfsetfillcolor{currentfill}%
\pgfsetlinewidth{0.311001pt}%
\definecolor{currentstroke}{rgb}{1.000000,1.000000,1.000000}%
\pgfsetstrokecolor{currentstroke}%
\pgfsetdash{}{0pt}%
\pgfpathmoveto{\pgfqpoint{4.189036in}{1.449123in}}%
\pgfpathcurveto{\pgfqpoint{4.196169in}{1.449123in}}{\pgfqpoint{4.203010in}{1.451957in}}{\pgfqpoint{4.208054in}{1.457001in}}%
\pgfpathcurveto{\pgfqpoint{4.213098in}{1.462045in}}{\pgfqpoint{4.215932in}{1.468886in}}{\pgfqpoint{4.215932in}{1.476019in}}%
\pgfpathcurveto{\pgfqpoint{4.215932in}{1.483152in}}{\pgfqpoint{4.213098in}{1.489994in}}{\pgfqpoint{4.208054in}{1.495037in}}%
\pgfpathcurveto{\pgfqpoint{4.203010in}{1.500081in}}{\pgfqpoint{4.196169in}{1.502915in}}{\pgfqpoint{4.189036in}{1.502915in}}%
\pgfpathcurveto{\pgfqpoint{4.181903in}{1.502915in}}{\pgfqpoint{4.175061in}{1.500081in}}{\pgfqpoint{4.170018in}{1.495037in}}%
\pgfpathcurveto{\pgfqpoint{4.164974in}{1.489994in}}{\pgfqpoint{4.162140in}{1.483152in}}{\pgfqpoint{4.162140in}{1.476019in}}%
\pgfpathcurveto{\pgfqpoint{4.162140in}{1.468886in}}{\pgfqpoint{4.164974in}{1.462045in}}{\pgfqpoint{4.170018in}{1.457001in}}%
\pgfpathcurveto{\pgfqpoint{4.175061in}{1.451957in}}{\pgfqpoint{4.181903in}{1.449123in}}{\pgfqpoint{4.189036in}{1.449123in}}%
\pgfpathclose%
\pgfusepath{stroke,fill}%
\end{pgfscope}%
\begin{pgfscope}%
\pgfpathrectangle{\pgfqpoint{2.867647in}{0.500000in}}{\pgfqpoint{1.764706in}{1.700000in}}%
\pgfusepath{clip}%
\pgfsetbuttcap%
\pgfsetroundjoin%
\definecolor{currentfill}{rgb}{0.978376,0.897317,0.831308}%
\pgfsetfillcolor{currentfill}%
\pgfsetlinewidth{0.311001pt}%
\definecolor{currentstroke}{rgb}{1.000000,1.000000,1.000000}%
\pgfsetstrokecolor{currentstroke}%
\pgfsetdash{}{0pt}%
\pgfpathmoveto{\pgfqpoint{4.114129in}{1.533854in}}%
\pgfpathcurveto{\pgfqpoint{4.121262in}{1.533854in}}{\pgfqpoint{4.128103in}{1.536688in}}{\pgfqpoint{4.133147in}{1.541732in}}%
\pgfpathcurveto{\pgfqpoint{4.138191in}{1.546775in}}{\pgfqpoint{4.141025in}{1.553617in}}{\pgfqpoint{4.141025in}{1.560750in}}%
\pgfpathcurveto{\pgfqpoint{4.141025in}{1.567883in}}{\pgfqpoint{4.138191in}{1.574724in}}{\pgfqpoint{4.133147in}{1.579768in}}%
\pgfpathcurveto{\pgfqpoint{4.128103in}{1.584812in}}{\pgfqpoint{4.121262in}{1.587646in}}{\pgfqpoint{4.114129in}{1.587646in}}%
\pgfpathcurveto{\pgfqpoint{4.106996in}{1.587646in}}{\pgfqpoint{4.100154in}{1.584812in}}{\pgfqpoint{4.095111in}{1.579768in}}%
\pgfpathcurveto{\pgfqpoint{4.090067in}{1.574724in}}{\pgfqpoint{4.087233in}{1.567883in}}{\pgfqpoint{4.087233in}{1.560750in}}%
\pgfpathcurveto{\pgfqpoint{4.087233in}{1.553617in}}{\pgfqpoint{4.090067in}{1.546775in}}{\pgfqpoint{4.095111in}{1.541732in}}%
\pgfpathcurveto{\pgfqpoint{4.100154in}{1.536688in}}{\pgfqpoint{4.106996in}{1.533854in}}{\pgfqpoint{4.114129in}{1.533854in}}%
\pgfpathclose%
\pgfusepath{stroke,fill}%
\end{pgfscope}%
\begin{pgfscope}%
\pgfpathrectangle{\pgfqpoint{2.867647in}{0.500000in}}{\pgfqpoint{1.764706in}{1.700000in}}%
\pgfusepath{clip}%
\pgfsetbuttcap%
\pgfsetroundjoin%
\definecolor{currentfill}{rgb}{0.971202,0.827364,0.728520}%
\pgfsetfillcolor{currentfill}%
\pgfsetlinewidth{0.311001pt}%
\definecolor{currentstroke}{rgb}{1.000000,1.000000,1.000000}%
\pgfsetstrokecolor{currentstroke}%
\pgfsetdash{}{0pt}%
\pgfpathmoveto{\pgfqpoint{4.056903in}{1.002409in}}%
\pgfpathcurveto{\pgfqpoint{4.064036in}{1.002409in}}{\pgfqpoint{4.070877in}{1.005243in}}{\pgfqpoint{4.075921in}{1.010287in}}%
\pgfpathcurveto{\pgfqpoint{4.080965in}{1.015330in}}{\pgfqpoint{4.083798in}{1.022172in}}{\pgfqpoint{4.083798in}{1.029305in}}%
\pgfpathcurveto{\pgfqpoint{4.083798in}{1.036438in}}{\pgfqpoint{4.080965in}{1.043279in}}{\pgfqpoint{4.075921in}{1.048323in}}%
\pgfpathcurveto{\pgfqpoint{4.070877in}{1.053367in}}{\pgfqpoint{4.064036in}{1.056200in}}{\pgfqpoint{4.056903in}{1.056200in}}%
\pgfpathcurveto{\pgfqpoint{4.049770in}{1.056200in}}{\pgfqpoint{4.042928in}{1.053367in}}{\pgfqpoint{4.037885in}{1.048323in}}%
\pgfpathcurveto{\pgfqpoint{4.032841in}{1.043279in}}{\pgfqpoint{4.030007in}{1.036438in}}{\pgfqpoint{4.030007in}{1.029305in}}%
\pgfpathcurveto{\pgfqpoint{4.030007in}{1.022172in}}{\pgfqpoint{4.032841in}{1.015330in}}{\pgfqpoint{4.037885in}{1.010287in}}%
\pgfpathcurveto{\pgfqpoint{4.042928in}{1.005243in}}{\pgfqpoint{4.049770in}{1.002409in}}{\pgfqpoint{4.056903in}{1.002409in}}%
\pgfpathclose%
\pgfusepath{stroke,fill}%
\end{pgfscope}%
\begin{pgfscope}%
\pgfpathrectangle{\pgfqpoint{2.867647in}{0.500000in}}{\pgfqpoint{1.764706in}{1.700000in}}%
\pgfusepath{clip}%
\pgfsetbuttcap%
\pgfsetroundjoin%
\definecolor{currentfill}{rgb}{0.967735,0.780441,0.659127}%
\pgfsetfillcolor{currentfill}%
\pgfsetlinewidth{0.311001pt}%
\definecolor{currentstroke}{rgb}{1.000000,1.000000,1.000000}%
\pgfsetstrokecolor{currentstroke}%
\pgfsetdash{}{0pt}%
\pgfpathmoveto{\pgfqpoint{4.262298in}{1.493270in}}%
\pgfpathcurveto{\pgfqpoint{4.269431in}{1.493270in}}{\pgfqpoint{4.276273in}{1.496104in}}{\pgfqpoint{4.281316in}{1.501147in}}%
\pgfpathcurveto{\pgfqpoint{4.286360in}{1.506191in}}{\pgfqpoint{4.289194in}{1.513033in}}{\pgfqpoint{4.289194in}{1.520165in}}%
\pgfpathcurveto{\pgfqpoint{4.289194in}{1.527298in}}{\pgfqpoint{4.286360in}{1.534140in}}{\pgfqpoint{4.281316in}{1.539184in}}%
\pgfpathcurveto{\pgfqpoint{4.276273in}{1.544227in}}{\pgfqpoint{4.269431in}{1.547061in}}{\pgfqpoint{4.262298in}{1.547061in}}%
\pgfpathcurveto{\pgfqpoint{4.255165in}{1.547061in}}{\pgfqpoint{4.248324in}{1.544227in}}{\pgfqpoint{4.243280in}{1.539184in}}%
\pgfpathcurveto{\pgfqpoint{4.238236in}{1.534140in}}{\pgfqpoint{4.235402in}{1.527298in}}{\pgfqpoint{4.235402in}{1.520165in}}%
\pgfpathcurveto{\pgfqpoint{4.235402in}{1.513033in}}{\pgfqpoint{4.238236in}{1.506191in}}{\pgfqpoint{4.243280in}{1.501147in}}%
\pgfpathcurveto{\pgfqpoint{4.248324in}{1.496104in}}{\pgfqpoint{4.255165in}{1.493270in}}{\pgfqpoint{4.262298in}{1.493270in}}%
\pgfpathclose%
\pgfusepath{stroke,fill}%
\end{pgfscope}%
\begin{pgfscope}%
\pgfpathrectangle{\pgfqpoint{2.867647in}{0.500000in}}{\pgfqpoint{1.764706in}{1.700000in}}%
\pgfusepath{clip}%
\pgfsetbuttcap%
\pgfsetroundjoin%
\definecolor{currentfill}{rgb}{0.975018,0.868213,0.788710}%
\pgfsetfillcolor{currentfill}%
\pgfsetlinewidth{0.311001pt}%
\definecolor{currentstroke}{rgb}{1.000000,1.000000,1.000000}%
\pgfsetstrokecolor{currentstroke}%
\pgfsetdash{}{0pt}%
\pgfpathmoveto{\pgfqpoint{4.146768in}{1.017138in}}%
\pgfpathcurveto{\pgfqpoint{4.153900in}{1.017138in}}{\pgfqpoint{4.160742in}{1.019972in}}{\pgfqpoint{4.165786in}{1.025015in}}%
\pgfpathcurveto{\pgfqpoint{4.170829in}{1.030059in}}{\pgfqpoint{4.173663in}{1.036901in}}{\pgfqpoint{4.173663in}{1.044033in}}%
\pgfpathcurveto{\pgfqpoint{4.173663in}{1.051166in}}{\pgfqpoint{4.170829in}{1.058008in}}{\pgfqpoint{4.165786in}{1.063052in}}%
\pgfpathcurveto{\pgfqpoint{4.160742in}{1.068095in}}{\pgfqpoint{4.153900in}{1.070929in}}{\pgfqpoint{4.146768in}{1.070929in}}%
\pgfpathcurveto{\pgfqpoint{4.139635in}{1.070929in}}{\pgfqpoint{4.132793in}{1.068095in}}{\pgfqpoint{4.127749in}{1.063052in}}%
\pgfpathcurveto{\pgfqpoint{4.122706in}{1.058008in}}{\pgfqpoint{4.119872in}{1.051166in}}{\pgfqpoint{4.119872in}{1.044033in}}%
\pgfpathcurveto{\pgfqpoint{4.119872in}{1.036901in}}{\pgfqpoint{4.122706in}{1.030059in}}{\pgfqpoint{4.127749in}{1.025015in}}%
\pgfpathcurveto{\pgfqpoint{4.132793in}{1.019972in}}{\pgfqpoint{4.139635in}{1.017138in}}{\pgfqpoint{4.146768in}{1.017138in}}%
\pgfpathclose%
\pgfusepath{stroke,fill}%
\end{pgfscope}%
\begin{pgfscope}%
\pgfpathrectangle{\pgfqpoint{2.867647in}{0.500000in}}{\pgfqpoint{1.764706in}{1.700000in}}%
\pgfusepath{clip}%
\pgfsetbuttcap%
\pgfsetroundjoin%
\definecolor{currentfill}{rgb}{0.965753,0.732351,0.592427}%
\pgfsetfillcolor{currentfill}%
\pgfsetlinewidth{0.311001pt}%
\definecolor{currentstroke}{rgb}{1.000000,1.000000,1.000000}%
\pgfsetstrokecolor{currentstroke}%
\pgfsetdash{}{0pt}%
\pgfpathmoveto{\pgfqpoint{4.049119in}{1.749661in}}%
\pgfpathcurveto{\pgfqpoint{4.056252in}{1.749661in}}{\pgfqpoint{4.063093in}{1.752495in}}{\pgfqpoint{4.068137in}{1.757539in}}%
\pgfpathcurveto{\pgfqpoint{4.073181in}{1.762583in}}{\pgfqpoint{4.076015in}{1.769424in}}{\pgfqpoint{4.076015in}{1.776557in}}%
\pgfpathcurveto{\pgfqpoint{4.076015in}{1.783690in}}{\pgfqpoint{4.073181in}{1.790532in}}{\pgfqpoint{4.068137in}{1.795575in}}%
\pgfpathcurveto{\pgfqpoint{4.063093in}{1.800619in}}{\pgfqpoint{4.056252in}{1.803453in}}{\pgfqpoint{4.049119in}{1.803453in}}%
\pgfpathcurveto{\pgfqpoint{4.041986in}{1.803453in}}{\pgfqpoint{4.035144in}{1.800619in}}{\pgfqpoint{4.030101in}{1.795575in}}%
\pgfpathcurveto{\pgfqpoint{4.025057in}{1.790532in}}{\pgfqpoint{4.022223in}{1.783690in}}{\pgfqpoint{4.022223in}{1.776557in}}%
\pgfpathcurveto{\pgfqpoint{4.022223in}{1.769424in}}{\pgfqpoint{4.025057in}{1.762583in}}{\pgfqpoint{4.030101in}{1.757539in}}%
\pgfpathcurveto{\pgfqpoint{4.035144in}{1.752495in}}{\pgfqpoint{4.041986in}{1.749661in}}{\pgfqpoint{4.049119in}{1.749661in}}%
\pgfpathclose%
\pgfusepath{stroke,fill}%
\end{pgfscope}%
\begin{pgfscope}%
\pgfpathrectangle{\pgfqpoint{2.867647in}{0.500000in}}{\pgfqpoint{1.764706in}{1.700000in}}%
\pgfusepath{clip}%
\pgfsetbuttcap%
\pgfsetroundjoin%
\definecolor{currentfill}{rgb}{0.961433,0.573272,0.412036}%
\pgfsetfillcolor{currentfill}%
\pgfsetlinewidth{0.311001pt}%
\definecolor{currentstroke}{rgb}{1.000000,1.000000,1.000000}%
\pgfsetstrokecolor{currentstroke}%
\pgfsetdash{}{0pt}%
\pgfpathmoveto{\pgfqpoint{4.071212in}{1.811973in}}%
\pgfpathcurveto{\pgfqpoint{4.078345in}{1.811973in}}{\pgfqpoint{4.085187in}{1.814806in}}{\pgfqpoint{4.090230in}{1.819850in}}%
\pgfpathcurveto{\pgfqpoint{4.095274in}{1.824894in}}{\pgfqpoint{4.098108in}{1.831735in}}{\pgfqpoint{4.098108in}{1.838868in}}%
\pgfpathcurveto{\pgfqpoint{4.098108in}{1.846001in}}{\pgfqpoint{4.095274in}{1.852843in}}{\pgfqpoint{4.090230in}{1.857886in}}%
\pgfpathcurveto{\pgfqpoint{4.085187in}{1.862930in}}{\pgfqpoint{4.078345in}{1.865764in}}{\pgfqpoint{4.071212in}{1.865764in}}%
\pgfpathcurveto{\pgfqpoint{4.064079in}{1.865764in}}{\pgfqpoint{4.057238in}{1.862930in}}{\pgfqpoint{4.052194in}{1.857886in}}%
\pgfpathcurveto{\pgfqpoint{4.047150in}{1.852843in}}{\pgfqpoint{4.044316in}{1.846001in}}{\pgfqpoint{4.044316in}{1.838868in}}%
\pgfpathcurveto{\pgfqpoint{4.044316in}{1.831735in}}{\pgfqpoint{4.047150in}{1.824894in}}{\pgfqpoint{4.052194in}{1.819850in}}%
\pgfpathcurveto{\pgfqpoint{4.057238in}{1.814806in}}{\pgfqpoint{4.064079in}{1.811973in}}{\pgfqpoint{4.071212in}{1.811973in}}%
\pgfpathclose%
\pgfusepath{stroke,fill}%
\end{pgfscope}%
\begin{pgfscope}%
\pgfpathrectangle{\pgfqpoint{2.867647in}{0.500000in}}{\pgfqpoint{1.764706in}{1.700000in}}%
\pgfusepath{clip}%
\pgfsetbuttcap%
\pgfsetroundjoin%
\definecolor{currentfill}{rgb}{0.965042,0.701564,0.552889}%
\pgfsetfillcolor{currentfill}%
\pgfsetlinewidth{0.311001pt}%
\definecolor{currentstroke}{rgb}{1.000000,1.000000,1.000000}%
\pgfsetstrokecolor{currentstroke}%
\pgfsetdash{}{0pt}%
\pgfpathmoveto{\pgfqpoint{4.117009in}{0.901688in}}%
\pgfpathcurveto{\pgfqpoint{4.124142in}{0.901688in}}{\pgfqpoint{4.130983in}{0.904521in}}{\pgfqpoint{4.136027in}{0.909565in}}%
\pgfpathcurveto{\pgfqpoint{4.141071in}{0.914609in}}{\pgfqpoint{4.143905in}{0.921450in}}{\pgfqpoint{4.143905in}{0.928583in}}%
\pgfpathcurveto{\pgfqpoint{4.143905in}{0.935716in}}{\pgfqpoint{4.141071in}{0.942558in}}{\pgfqpoint{4.136027in}{0.947601in}}%
\pgfpathcurveto{\pgfqpoint{4.130983in}{0.952645in}}{\pgfqpoint{4.124142in}{0.955479in}}{\pgfqpoint{4.117009in}{0.955479in}}%
\pgfpathcurveto{\pgfqpoint{4.109876in}{0.955479in}}{\pgfqpoint{4.103034in}{0.952645in}}{\pgfqpoint{4.097991in}{0.947601in}}%
\pgfpathcurveto{\pgfqpoint{4.092947in}{0.942558in}}{\pgfqpoint{4.090113in}{0.935716in}}{\pgfqpoint{4.090113in}{0.928583in}}%
\pgfpathcurveto{\pgfqpoint{4.090113in}{0.921450in}}{\pgfqpoint{4.092947in}{0.914609in}}{\pgfqpoint{4.097991in}{0.909565in}}%
\pgfpathcurveto{\pgfqpoint{4.103034in}{0.904521in}}{\pgfqpoint{4.109876in}{0.901688in}}{\pgfqpoint{4.117009in}{0.901688in}}%
\pgfpathclose%
\pgfusepath{stroke,fill}%
\end{pgfscope}%
\begin{pgfscope}%
\pgfpathrectangle{\pgfqpoint{2.867647in}{0.500000in}}{\pgfqpoint{1.764706in}{1.700000in}}%
\pgfusepath{clip}%
\pgfsetbuttcap%
\pgfsetroundjoin%
\definecolor{currentfill}{rgb}{0.976287,0.879862,0.805788}%
\pgfsetfillcolor{currentfill}%
\pgfsetlinewidth{0.311001pt}%
\definecolor{currentstroke}{rgb}{1.000000,1.000000,1.000000}%
\pgfsetstrokecolor{currentstroke}%
\pgfsetdash{}{0pt}%
\pgfpathmoveto{\pgfqpoint{4.202790in}{1.110558in}}%
\pgfpathcurveto{\pgfqpoint{4.209922in}{1.110558in}}{\pgfqpoint{4.216764in}{1.113392in}}{\pgfqpoint{4.221808in}{1.118436in}}%
\pgfpathcurveto{\pgfqpoint{4.226851in}{1.123479in}}{\pgfqpoint{4.229685in}{1.130321in}}{\pgfqpoint{4.229685in}{1.137454in}}%
\pgfpathcurveto{\pgfqpoint{4.229685in}{1.144587in}}{\pgfqpoint{4.226851in}{1.151428in}}{\pgfqpoint{4.221808in}{1.156472in}}%
\pgfpathcurveto{\pgfqpoint{4.216764in}{1.161516in}}{\pgfqpoint{4.209922in}{1.164350in}}{\pgfqpoint{4.202790in}{1.164350in}}%
\pgfpathcurveto{\pgfqpoint{4.195657in}{1.164350in}}{\pgfqpoint{4.188815in}{1.161516in}}{\pgfqpoint{4.183771in}{1.156472in}}%
\pgfpathcurveto{\pgfqpoint{4.178728in}{1.151428in}}{\pgfqpoint{4.175894in}{1.144587in}}{\pgfqpoint{4.175894in}{1.137454in}}%
\pgfpathcurveto{\pgfqpoint{4.175894in}{1.130321in}}{\pgfqpoint{4.178728in}{1.123479in}}{\pgfqpoint{4.183771in}{1.118436in}}%
\pgfpathcurveto{\pgfqpoint{4.188815in}{1.113392in}}{\pgfqpoint{4.195657in}{1.110558in}}{\pgfqpoint{4.202790in}{1.110558in}}%
\pgfpathclose%
\pgfusepath{stroke,fill}%
\end{pgfscope}%
\begin{pgfscope}%
\pgfpathrectangle{\pgfqpoint{2.867647in}{0.500000in}}{\pgfqpoint{1.764706in}{1.700000in}}%
\pgfusepath{clip}%
\pgfsetbuttcap%
\pgfsetroundjoin%
\definecolor{currentfill}{rgb}{0.977657,0.891500,0.822809}%
\pgfsetfillcolor{currentfill}%
\pgfsetlinewidth{0.311001pt}%
\definecolor{currentstroke}{rgb}{1.000000,1.000000,1.000000}%
\pgfsetstrokecolor{currentstroke}%
\pgfsetdash{}{0pt}%
\pgfpathmoveto{\pgfqpoint{4.231431in}{1.252143in}}%
\pgfpathcurveto{\pgfqpoint{4.238564in}{1.252143in}}{\pgfqpoint{4.245405in}{1.254976in}}{\pgfqpoint{4.250449in}{1.260020in}}%
\pgfpathcurveto{\pgfqpoint{4.255493in}{1.265064in}}{\pgfqpoint{4.258327in}{1.271905in}}{\pgfqpoint{4.258327in}{1.279038in}}%
\pgfpathcurveto{\pgfqpoint{4.258327in}{1.286171in}}{\pgfqpoint{4.255493in}{1.293013in}}{\pgfqpoint{4.250449in}{1.298056in}}%
\pgfpathcurveto{\pgfqpoint{4.245405in}{1.303100in}}{\pgfqpoint{4.238564in}{1.305934in}}{\pgfqpoint{4.231431in}{1.305934in}}%
\pgfpathcurveto{\pgfqpoint{4.224298in}{1.305934in}}{\pgfqpoint{4.217456in}{1.303100in}}{\pgfqpoint{4.212413in}{1.298056in}}%
\pgfpathcurveto{\pgfqpoint{4.207369in}{1.293013in}}{\pgfqpoint{4.204535in}{1.286171in}}{\pgfqpoint{4.204535in}{1.279038in}}%
\pgfpathcurveto{\pgfqpoint{4.204535in}{1.271905in}}{\pgfqpoint{4.207369in}{1.265064in}}{\pgfqpoint{4.212413in}{1.260020in}}%
\pgfpathcurveto{\pgfqpoint{4.217456in}{1.254976in}}{\pgfqpoint{4.224298in}{1.252143in}}{\pgfqpoint{4.231431in}{1.252143in}}%
\pgfpathclose%
\pgfusepath{stroke,fill}%
\end{pgfscope}%
\begin{pgfscope}%
\pgfpathrectangle{\pgfqpoint{2.867647in}{0.500000in}}{\pgfqpoint{1.764706in}{1.700000in}}%
\pgfusepath{clip}%
\pgfsetbuttcap%
\pgfsetroundjoin%
\definecolor{currentfill}{rgb}{0.972201,0.839051,0.745789}%
\pgfsetfillcolor{currentfill}%
\pgfsetlinewidth{0.311001pt}%
\definecolor{currentstroke}{rgb}{1.000000,1.000000,1.000000}%
\pgfsetstrokecolor{currentstroke}%
\pgfsetdash{}{0pt}%
\pgfpathmoveto{\pgfqpoint{4.247375in}{1.456809in}}%
\pgfpathcurveto{\pgfqpoint{4.254508in}{1.456809in}}{\pgfqpoint{4.261349in}{1.459643in}}{\pgfqpoint{4.266393in}{1.464687in}}%
\pgfpathcurveto{\pgfqpoint{4.271437in}{1.469730in}}{\pgfqpoint{4.274270in}{1.476572in}}{\pgfqpoint{4.274270in}{1.483705in}}%
\pgfpathcurveto{\pgfqpoint{4.274270in}{1.490838in}}{\pgfqpoint{4.271437in}{1.497679in}}{\pgfqpoint{4.266393in}{1.502723in}}%
\pgfpathcurveto{\pgfqpoint{4.261349in}{1.507767in}}{\pgfqpoint{4.254508in}{1.510600in}}{\pgfqpoint{4.247375in}{1.510600in}}%
\pgfpathcurveto{\pgfqpoint{4.240242in}{1.510600in}}{\pgfqpoint{4.233400in}{1.507767in}}{\pgfqpoint{4.228357in}{1.502723in}}%
\pgfpathcurveto{\pgfqpoint{4.223313in}{1.497679in}}{\pgfqpoint{4.220479in}{1.490838in}}{\pgfqpoint{4.220479in}{1.483705in}}%
\pgfpathcurveto{\pgfqpoint{4.220479in}{1.476572in}}{\pgfqpoint{4.223313in}{1.469730in}}{\pgfqpoint{4.228357in}{1.464687in}}%
\pgfpathcurveto{\pgfqpoint{4.233400in}{1.459643in}}{\pgfqpoint{4.240242in}{1.456809in}}{\pgfqpoint{4.247375in}{1.456809in}}%
\pgfpathclose%
\pgfusepath{stroke,fill}%
\end{pgfscope}%
\begin{pgfscope}%
\pgfpathrectangle{\pgfqpoint{2.867647in}{0.500000in}}{\pgfqpoint{1.764706in}{1.700000in}}%
\pgfusepath{clip}%
\pgfsetbuttcap%
\pgfsetroundjoin%
\definecolor{currentfill}{rgb}{0.968509,0.792226,0.676405}%
\pgfsetfillcolor{currentfill}%
\pgfsetlinewidth{0.311001pt}%
\definecolor{currentstroke}{rgb}{1.000000,1.000000,1.000000}%
\pgfsetstrokecolor{currentstroke}%
\pgfsetdash{}{0pt}%
\pgfpathmoveto{\pgfqpoint{4.038339in}{0.961645in}}%
\pgfpathcurveto{\pgfqpoint{4.045472in}{0.961645in}}{\pgfqpoint{4.052313in}{0.964479in}}{\pgfqpoint{4.057357in}{0.969523in}}%
\pgfpathcurveto{\pgfqpoint{4.062401in}{0.974566in}}{\pgfqpoint{4.065235in}{0.981408in}}{\pgfqpoint{4.065235in}{0.988541in}}%
\pgfpathcurveto{\pgfqpoint{4.065235in}{0.995674in}}{\pgfqpoint{4.062401in}{1.002515in}}{\pgfqpoint{4.057357in}{1.007559in}}%
\pgfpathcurveto{\pgfqpoint{4.052313in}{1.012603in}}{\pgfqpoint{4.045472in}{1.015436in}}{\pgfqpoint{4.038339in}{1.015436in}}%
\pgfpathcurveto{\pgfqpoint{4.031206in}{1.015436in}}{\pgfqpoint{4.024364in}{1.012603in}}{\pgfqpoint{4.019321in}{1.007559in}}%
\pgfpathcurveto{\pgfqpoint{4.014277in}{1.002515in}}{\pgfqpoint{4.011443in}{0.995674in}}{\pgfqpoint{4.011443in}{0.988541in}}%
\pgfpathcurveto{\pgfqpoint{4.011443in}{0.981408in}}{\pgfqpoint{4.014277in}{0.974566in}}{\pgfqpoint{4.019321in}{0.969523in}}%
\pgfpathcurveto{\pgfqpoint{4.024364in}{0.964479in}}{\pgfqpoint{4.031206in}{0.961645in}}{\pgfqpoint{4.038339in}{0.961645in}}%
\pgfpathclose%
\pgfusepath{stroke,fill}%
\end{pgfscope}%
\begin{pgfscope}%
\pgfpathrectangle{\pgfqpoint{2.867647in}{0.500000in}}{\pgfqpoint{1.764706in}{1.700000in}}%
\pgfusepath{clip}%
\pgfsetbuttcap%
\pgfsetroundjoin%
\definecolor{currentfill}{rgb}{0.958791,0.526283,0.368909}%
\pgfsetfillcolor{currentfill}%
\pgfsetlinewidth{0.311001pt}%
\definecolor{currentstroke}{rgb}{1.000000,1.000000,1.000000}%
\pgfsetstrokecolor{currentstroke}%
\pgfsetdash{}{0pt}%
\pgfpathmoveto{\pgfqpoint{4.231454in}{0.942599in}}%
\pgfpathcurveto{\pgfqpoint{4.238587in}{0.942599in}}{\pgfqpoint{4.245429in}{0.945432in}}{\pgfqpoint{4.250473in}{0.950476in}}%
\pgfpathcurveto{\pgfqpoint{4.255516in}{0.955520in}}{\pgfqpoint{4.258350in}{0.962361in}}{\pgfqpoint{4.258350in}{0.969494in}}%
\pgfpathcurveto{\pgfqpoint{4.258350in}{0.976627in}}{\pgfqpoint{4.255516in}{0.983469in}}{\pgfqpoint{4.250473in}{0.988512in}}%
\pgfpathcurveto{\pgfqpoint{4.245429in}{0.993556in}}{\pgfqpoint{4.238587in}{0.996390in}}{\pgfqpoint{4.231454in}{0.996390in}}%
\pgfpathcurveto{\pgfqpoint{4.224322in}{0.996390in}}{\pgfqpoint{4.217480in}{0.993556in}}{\pgfqpoint{4.212436in}{0.988512in}}%
\pgfpathcurveto{\pgfqpoint{4.207393in}{0.983469in}}{\pgfqpoint{4.204559in}{0.976627in}}{\pgfqpoint{4.204559in}{0.969494in}}%
\pgfpathcurveto{\pgfqpoint{4.204559in}{0.962361in}}{\pgfqpoint{4.207393in}{0.955520in}}{\pgfqpoint{4.212436in}{0.950476in}}%
\pgfpathcurveto{\pgfqpoint{4.217480in}{0.945432in}}{\pgfqpoint{4.224322in}{0.942599in}}{\pgfqpoint{4.231454in}{0.942599in}}%
\pgfpathclose%
\pgfusepath{stroke,fill}%
\end{pgfscope}%
\begin{pgfscope}%
\pgfpathrectangle{\pgfqpoint{2.867647in}{0.500000in}}{\pgfqpoint{1.764706in}{1.700000in}}%
\pgfusepath{clip}%
\pgfsetbuttcap%
\pgfsetroundjoin%
\definecolor{currentfill}{rgb}{0.979124,0.903132,0.839793}%
\pgfsetfillcolor{currentfill}%
\pgfsetlinewidth{0.311001pt}%
\definecolor{currentstroke}{rgb}{1.000000,1.000000,1.000000}%
\pgfsetstrokecolor{currentstroke}%
\pgfsetdash{}{0pt}%
\pgfpathmoveto{\pgfqpoint{4.127252in}{1.109447in}}%
\pgfpathcurveto{\pgfqpoint{4.134385in}{1.109447in}}{\pgfqpoint{4.141227in}{1.112281in}}{\pgfqpoint{4.146270in}{1.117325in}}%
\pgfpathcurveto{\pgfqpoint{4.151314in}{1.122368in}}{\pgfqpoint{4.154148in}{1.129210in}}{\pgfqpoint{4.154148in}{1.136343in}}%
\pgfpathcurveto{\pgfqpoint{4.154148in}{1.143476in}}{\pgfqpoint{4.151314in}{1.150317in}}{\pgfqpoint{4.146270in}{1.155361in}}%
\pgfpathcurveto{\pgfqpoint{4.141227in}{1.160405in}}{\pgfqpoint{4.134385in}{1.163239in}}{\pgfqpoint{4.127252in}{1.163239in}}%
\pgfpathcurveto{\pgfqpoint{4.120119in}{1.163239in}}{\pgfqpoint{4.113278in}{1.160405in}}{\pgfqpoint{4.108234in}{1.155361in}}%
\pgfpathcurveto{\pgfqpoint{4.103190in}{1.150317in}}{\pgfqpoint{4.100356in}{1.143476in}}{\pgfqpoint{4.100356in}{1.136343in}}%
\pgfpathcurveto{\pgfqpoint{4.100356in}{1.129210in}}{\pgfqpoint{4.103190in}{1.122368in}}{\pgfqpoint{4.108234in}{1.117325in}}%
\pgfpathcurveto{\pgfqpoint{4.113278in}{1.112281in}}{\pgfqpoint{4.120119in}{1.109447in}}{\pgfqpoint{4.127252in}{1.109447in}}%
\pgfpathclose%
\pgfusepath{stroke,fill}%
\end{pgfscope}%
\begin{pgfscope}%
\pgfpathrectangle{\pgfqpoint{2.867647in}{0.500000in}}{\pgfqpoint{1.764706in}{1.700000in}}%
\pgfusepath{clip}%
\pgfsetbuttcap%
\pgfsetroundjoin%
\definecolor{currentfill}{rgb}{0.973832,0.856556,0.771584}%
\pgfsetfillcolor{currentfill}%
\pgfsetlinewidth{0.311001pt}%
\definecolor{currentstroke}{rgb}{1.000000,1.000000,1.000000}%
\pgfsetstrokecolor{currentstroke}%
\pgfsetdash{}{0pt}%
\pgfpathmoveto{\pgfqpoint{4.253253in}{1.259046in}}%
\pgfpathcurveto{\pgfqpoint{4.260386in}{1.259046in}}{\pgfqpoint{4.267228in}{1.261880in}}{\pgfqpoint{4.272272in}{1.266924in}}%
\pgfpathcurveto{\pgfqpoint{4.277315in}{1.271968in}}{\pgfqpoint{4.280149in}{1.278809in}}{\pgfqpoint{4.280149in}{1.285942in}}%
\pgfpathcurveto{\pgfqpoint{4.280149in}{1.293075in}}{\pgfqpoint{4.277315in}{1.299917in}}{\pgfqpoint{4.272272in}{1.304960in}}%
\pgfpathcurveto{\pgfqpoint{4.267228in}{1.310004in}}{\pgfqpoint{4.260386in}{1.312838in}}{\pgfqpoint{4.253253in}{1.312838in}}%
\pgfpathcurveto{\pgfqpoint{4.246121in}{1.312838in}}{\pgfqpoint{4.239279in}{1.310004in}}{\pgfqpoint{4.234235in}{1.304960in}}%
\pgfpathcurveto{\pgfqpoint{4.229192in}{1.299917in}}{\pgfqpoint{4.226358in}{1.293075in}}{\pgfqpoint{4.226358in}{1.285942in}}%
\pgfpathcurveto{\pgfqpoint{4.226358in}{1.278809in}}{\pgfqpoint{4.229192in}{1.271968in}}{\pgfqpoint{4.234235in}{1.266924in}}%
\pgfpathcurveto{\pgfqpoint{4.239279in}{1.261880in}}{\pgfqpoint{4.246121in}{1.259046in}}{\pgfqpoint{4.253253in}{1.259046in}}%
\pgfpathclose%
\pgfusepath{stroke,fill}%
\end{pgfscope}%
\begin{pgfscope}%
\pgfpathrectangle{\pgfqpoint{2.867647in}{0.500000in}}{\pgfqpoint{1.764706in}{1.700000in}}%
\pgfusepath{clip}%
\pgfsetbuttcap%
\pgfsetroundjoin%
\definecolor{currentfill}{rgb}{0.979891,0.908948,0.848279}%
\pgfsetfillcolor{currentfill}%
\pgfsetlinewidth{0.311001pt}%
\definecolor{currentstroke}{rgb}{1.000000,1.000000,1.000000}%
\pgfsetstrokecolor{currentstroke}%
\pgfsetdash{}{0pt}%
\pgfpathmoveto{\pgfqpoint{4.166860in}{1.359381in}}%
\pgfpathcurveto{\pgfqpoint{4.173992in}{1.359381in}}{\pgfqpoint{4.180834in}{1.362215in}}{\pgfqpoint{4.185878in}{1.367259in}}%
\pgfpathcurveto{\pgfqpoint{4.190921in}{1.372302in}}{\pgfqpoint{4.193755in}{1.379144in}}{\pgfqpoint{4.193755in}{1.386277in}}%
\pgfpathcurveto{\pgfqpoint{4.193755in}{1.393410in}}{\pgfqpoint{4.190921in}{1.400251in}}{\pgfqpoint{4.185878in}{1.405295in}}%
\pgfpathcurveto{\pgfqpoint{4.180834in}{1.410339in}}{\pgfqpoint{4.173992in}{1.413172in}}{\pgfqpoint{4.166860in}{1.413172in}}%
\pgfpathcurveto{\pgfqpoint{4.159727in}{1.413172in}}{\pgfqpoint{4.152885in}{1.410339in}}{\pgfqpoint{4.147841in}{1.405295in}}%
\pgfpathcurveto{\pgfqpoint{4.142798in}{1.400251in}}{\pgfqpoint{4.139964in}{1.393410in}}{\pgfqpoint{4.139964in}{1.386277in}}%
\pgfpathcurveto{\pgfqpoint{4.139964in}{1.379144in}}{\pgfqpoint{4.142798in}{1.372302in}}{\pgfqpoint{4.147841in}{1.367259in}}%
\pgfpathcurveto{\pgfqpoint{4.152885in}{1.362215in}}{\pgfqpoint{4.159727in}{1.359381in}}{\pgfqpoint{4.166860in}{1.359381in}}%
\pgfpathclose%
\pgfusepath{stroke,fill}%
\end{pgfscope}%
\begin{pgfscope}%
\pgfpathrectangle{\pgfqpoint{2.867647in}{0.500000in}}{\pgfqpoint{1.764706in}{1.700000in}}%
\pgfusepath{clip}%
\pgfsetbuttcap%
\pgfsetroundjoin%
\definecolor{currentfill}{rgb}{0.977657,0.891500,0.822809}%
\pgfsetfillcolor{currentfill}%
\pgfsetlinewidth{0.311001pt}%
\definecolor{currentstroke}{rgb}{1.000000,1.000000,1.000000}%
\pgfsetstrokecolor{currentstroke}%
\pgfsetdash{}{0pt}%
\pgfpathmoveto{\pgfqpoint{4.113788in}{1.616319in}}%
\pgfpathcurveto{\pgfqpoint{4.120921in}{1.616319in}}{\pgfqpoint{4.127762in}{1.619153in}}{\pgfqpoint{4.132806in}{1.624197in}}%
\pgfpathcurveto{\pgfqpoint{4.137850in}{1.629241in}}{\pgfqpoint{4.140684in}{1.636082in}}{\pgfqpoint{4.140684in}{1.643215in}}%
\pgfpathcurveto{\pgfqpoint{4.140684in}{1.650348in}}{\pgfqpoint{4.137850in}{1.657190in}}{\pgfqpoint{4.132806in}{1.662233in}}%
\pgfpathcurveto{\pgfqpoint{4.127762in}{1.667277in}}{\pgfqpoint{4.120921in}{1.670111in}}{\pgfqpoint{4.113788in}{1.670111in}}%
\pgfpathcurveto{\pgfqpoint{4.106655in}{1.670111in}}{\pgfqpoint{4.099813in}{1.667277in}}{\pgfqpoint{4.094770in}{1.662233in}}%
\pgfpathcurveto{\pgfqpoint{4.089726in}{1.657190in}}{\pgfqpoint{4.086892in}{1.650348in}}{\pgfqpoint{4.086892in}{1.643215in}}%
\pgfpathcurveto{\pgfqpoint{4.086892in}{1.636082in}}{\pgfqpoint{4.089726in}{1.629241in}}{\pgfqpoint{4.094770in}{1.624197in}}%
\pgfpathcurveto{\pgfqpoint{4.099813in}{1.619153in}}{\pgfqpoint{4.106655in}{1.616319in}}{\pgfqpoint{4.113788in}{1.616319in}}%
\pgfpathclose%
\pgfusepath{stroke,fill}%
\end{pgfscope}%
\begin{pgfscope}%
\pgfpathrectangle{\pgfqpoint{2.867647in}{0.500000in}}{\pgfqpoint{1.764706in}{1.700000in}}%
\pgfusepath{clip}%
\pgfsetbuttcap%
\pgfsetroundjoin%
\definecolor{currentfill}{rgb}{0.963884,0.644842,0.486120}%
\pgfsetfillcolor{currentfill}%
\pgfsetlinewidth{0.311001pt}%
\definecolor{currentstroke}{rgb}{1.000000,1.000000,1.000000}%
\pgfsetstrokecolor{currentstroke}%
\pgfsetdash{}{0pt}%
\pgfpathmoveto{\pgfqpoint{4.266455in}{1.597564in}}%
\pgfpathcurveto{\pgfqpoint{4.273588in}{1.597564in}}{\pgfqpoint{4.280430in}{1.600398in}}{\pgfqpoint{4.285473in}{1.605441in}}%
\pgfpathcurveto{\pgfqpoint{4.290517in}{1.610485in}}{\pgfqpoint{4.293351in}{1.617327in}}{\pgfqpoint{4.293351in}{1.624459in}}%
\pgfpathcurveto{\pgfqpoint{4.293351in}{1.631592in}}{\pgfqpoint{4.290517in}{1.638434in}}{\pgfqpoint{4.285473in}{1.643478in}}%
\pgfpathcurveto{\pgfqpoint{4.280430in}{1.648521in}}{\pgfqpoint{4.273588in}{1.651355in}}{\pgfqpoint{4.266455in}{1.651355in}}%
\pgfpathcurveto{\pgfqpoint{4.259322in}{1.651355in}}{\pgfqpoint{4.252481in}{1.648521in}}{\pgfqpoint{4.247437in}{1.643478in}}%
\pgfpathcurveto{\pgfqpoint{4.242393in}{1.638434in}}{\pgfqpoint{4.239559in}{1.631592in}}{\pgfqpoint{4.239559in}{1.624459in}}%
\pgfpathcurveto{\pgfqpoint{4.239559in}{1.617327in}}{\pgfqpoint{4.242393in}{1.610485in}}{\pgfqpoint{4.247437in}{1.605441in}}%
\pgfpathcurveto{\pgfqpoint{4.252481in}{1.600398in}}{\pgfqpoint{4.259322in}{1.597564in}}{\pgfqpoint{4.266455in}{1.597564in}}%
\pgfpathclose%
\pgfusepath{stroke,fill}%
\end{pgfscope}%
\begin{pgfscope}%
\pgfpathrectangle{\pgfqpoint{2.867647in}{0.500000in}}{\pgfqpoint{1.764706in}{1.700000in}}%
\pgfusepath{clip}%
\pgfsetbuttcap%
\pgfsetroundjoin%
\definecolor{currentfill}{rgb}{0.970255,0.815666,0.711203}%
\pgfsetfillcolor{currentfill}%
\pgfsetlinewidth{0.311001pt}%
\definecolor{currentstroke}{rgb}{1.000000,1.000000,1.000000}%
\pgfsetstrokecolor{currentstroke}%
\pgfsetdash{}{0pt}%
\pgfpathmoveto{\pgfqpoint{4.178280in}{1.002855in}}%
\pgfpathcurveto{\pgfqpoint{4.185413in}{1.002855in}}{\pgfqpoint{4.192255in}{1.005689in}}{\pgfqpoint{4.197298in}{1.010733in}}%
\pgfpathcurveto{\pgfqpoint{4.202342in}{1.015777in}}{\pgfqpoint{4.205176in}{1.022618in}}{\pgfqpoint{4.205176in}{1.029751in}}%
\pgfpathcurveto{\pgfqpoint{4.205176in}{1.036884in}}{\pgfqpoint{4.202342in}{1.043726in}}{\pgfqpoint{4.197298in}{1.048769in}}%
\pgfpathcurveto{\pgfqpoint{4.192255in}{1.053813in}}{\pgfqpoint{4.185413in}{1.056647in}}{\pgfqpoint{4.178280in}{1.056647in}}%
\pgfpathcurveto{\pgfqpoint{4.171147in}{1.056647in}}{\pgfqpoint{4.164306in}{1.053813in}}{\pgfqpoint{4.159262in}{1.048769in}}%
\pgfpathcurveto{\pgfqpoint{4.154218in}{1.043726in}}{\pgfqpoint{4.151384in}{1.036884in}}{\pgfqpoint{4.151384in}{1.029751in}}%
\pgfpathcurveto{\pgfqpoint{4.151384in}{1.022618in}}{\pgfqpoint{4.154218in}{1.015777in}}{\pgfqpoint{4.159262in}{1.010733in}}%
\pgfpathcurveto{\pgfqpoint{4.164306in}{1.005689in}}{\pgfqpoint{4.171147in}{1.002855in}}{\pgfqpoint{4.178280in}{1.002855in}}%
\pgfpathclose%
\pgfusepath{stroke,fill}%
\end{pgfscope}%
\begin{pgfscope}%
\pgfpathrectangle{\pgfqpoint{2.867647in}{0.500000in}}{\pgfqpoint{1.764706in}{1.700000in}}%
\pgfusepath{clip}%
\pgfsetbuttcap%
\pgfsetroundjoin%
\definecolor{currentfill}{rgb}{0.937528,0.344792,0.251999}%
\pgfsetfillcolor{currentfill}%
\pgfsetlinewidth{0.311001pt}%
\definecolor{currentstroke}{rgb}{1.000000,1.000000,1.000000}%
\pgfsetstrokecolor{currentstroke}%
\pgfsetdash{}{0pt}%
\pgfpathmoveto{\pgfqpoint{4.367117in}{1.212378in}}%
\pgfpathcurveto{\pgfqpoint{4.374250in}{1.212378in}}{\pgfqpoint{4.381092in}{1.215212in}}{\pgfqpoint{4.386136in}{1.220256in}}%
\pgfpathcurveto{\pgfqpoint{4.391179in}{1.225299in}}{\pgfqpoint{4.394013in}{1.232141in}}{\pgfqpoint{4.394013in}{1.239274in}}%
\pgfpathcurveto{\pgfqpoint{4.394013in}{1.246407in}}{\pgfqpoint{4.391179in}{1.253248in}}{\pgfqpoint{4.386136in}{1.258292in}}%
\pgfpathcurveto{\pgfqpoint{4.381092in}{1.263336in}}{\pgfqpoint{4.374250in}{1.266169in}}{\pgfqpoint{4.367117in}{1.266169in}}%
\pgfpathcurveto{\pgfqpoint{4.359985in}{1.266169in}}{\pgfqpoint{4.353143in}{1.263336in}}{\pgfqpoint{4.348099in}{1.258292in}}%
\pgfpathcurveto{\pgfqpoint{4.343056in}{1.253248in}}{\pgfqpoint{4.340222in}{1.246407in}}{\pgfqpoint{4.340222in}{1.239274in}}%
\pgfpathcurveto{\pgfqpoint{4.340222in}{1.232141in}}{\pgfqpoint{4.343056in}{1.225299in}}{\pgfqpoint{4.348099in}{1.220256in}}%
\pgfpathcurveto{\pgfqpoint{4.353143in}{1.215212in}}{\pgfqpoint{4.359985in}{1.212378in}}{\pgfqpoint{4.367117in}{1.212378in}}%
\pgfpathclose%
\pgfusepath{stroke,fill}%
\end{pgfscope}%
\begin{pgfscope}%
\pgfpathrectangle{\pgfqpoint{2.867647in}{0.500000in}}{\pgfqpoint{1.764706in}{1.700000in}}%
\pgfusepath{clip}%
\pgfsetbuttcap%
\pgfsetroundjoin%
\definecolor{currentfill}{rgb}{0.956817,0.498820,0.345554}%
\pgfsetfillcolor{currentfill}%
\pgfsetlinewidth{0.311001pt}%
\definecolor{currentstroke}{rgb}{1.000000,1.000000,1.000000}%
\pgfsetstrokecolor{currentstroke}%
\pgfsetdash{}{0pt}%
\pgfpathmoveto{\pgfqpoint{3.879895in}{1.722965in}}%
\pgfpathcurveto{\pgfqpoint{3.887028in}{1.722965in}}{\pgfqpoint{3.893869in}{1.725799in}}{\pgfqpoint{3.898913in}{1.730843in}}%
\pgfpathcurveto{\pgfqpoint{3.903957in}{1.735886in}}{\pgfqpoint{3.906790in}{1.742728in}}{\pgfqpoint{3.906790in}{1.749861in}}%
\pgfpathcurveto{\pgfqpoint{3.906790in}{1.756994in}}{\pgfqpoint{3.903957in}{1.763835in}}{\pgfqpoint{3.898913in}{1.768879in}}%
\pgfpathcurveto{\pgfqpoint{3.893869in}{1.773923in}}{\pgfqpoint{3.887028in}{1.776757in}}{\pgfqpoint{3.879895in}{1.776757in}}%
\pgfpathcurveto{\pgfqpoint{3.872762in}{1.776757in}}{\pgfqpoint{3.865920in}{1.773923in}}{\pgfqpoint{3.860877in}{1.768879in}}%
\pgfpathcurveto{\pgfqpoint{3.855833in}{1.763835in}}{\pgfqpoint{3.852999in}{1.756994in}}{\pgfqpoint{3.852999in}{1.749861in}}%
\pgfpathcurveto{\pgfqpoint{3.852999in}{1.742728in}}{\pgfqpoint{3.855833in}{1.735886in}}{\pgfqpoint{3.860877in}{1.730843in}}%
\pgfpathcurveto{\pgfqpoint{3.865920in}{1.725799in}}{\pgfqpoint{3.872762in}{1.722965in}}{\pgfqpoint{3.879895in}{1.722965in}}%
\pgfpathclose%
\pgfusepath{stroke,fill}%
\end{pgfscope}%
\begin{pgfscope}%
\pgfpathrectangle{\pgfqpoint{2.867647in}{0.500000in}}{\pgfqpoint{1.764706in}{1.700000in}}%
\pgfusepath{clip}%
\pgfsetbuttcap%
\pgfsetroundjoin%
\definecolor{currentfill}{rgb}{0.879259,0.192033,0.262681}%
\pgfsetfillcolor{currentfill}%
\pgfsetlinewidth{0.311001pt}%
\definecolor{currentstroke}{rgb}{1.000000,1.000000,1.000000}%
\pgfsetstrokecolor{currentstroke}%
\pgfsetdash{}{0pt}%
\pgfpathmoveto{\pgfqpoint{3.831575in}{0.837018in}}%
\pgfpathcurveto{\pgfqpoint{3.838708in}{0.837018in}}{\pgfqpoint{3.845550in}{0.839852in}}{\pgfqpoint{3.850594in}{0.844896in}}%
\pgfpathcurveto{\pgfqpoint{3.855637in}{0.849939in}}{\pgfqpoint{3.858471in}{0.856781in}}{\pgfqpoint{3.858471in}{0.863914in}}%
\pgfpathcurveto{\pgfqpoint{3.858471in}{0.871047in}}{\pgfqpoint{3.855637in}{0.877888in}}{\pgfqpoint{3.850594in}{0.882932in}}%
\pgfpathcurveto{\pgfqpoint{3.845550in}{0.887976in}}{\pgfqpoint{3.838708in}{0.890809in}}{\pgfqpoint{3.831575in}{0.890809in}}%
\pgfpathcurveto{\pgfqpoint{3.824443in}{0.890809in}}{\pgfqpoint{3.817601in}{0.887976in}}{\pgfqpoint{3.812557in}{0.882932in}}%
\pgfpathcurveto{\pgfqpoint{3.807514in}{0.877888in}}{\pgfqpoint{3.804680in}{0.871047in}}{\pgfqpoint{3.804680in}{0.863914in}}%
\pgfpathcurveto{\pgfqpoint{3.804680in}{0.856781in}}{\pgfqpoint{3.807514in}{0.849939in}}{\pgfqpoint{3.812557in}{0.844896in}}%
\pgfpathcurveto{\pgfqpoint{3.817601in}{0.839852in}}{\pgfqpoint{3.824443in}{0.837018in}}{\pgfqpoint{3.831575in}{0.837018in}}%
\pgfpathclose%
\pgfusepath{stroke,fill}%
\end{pgfscope}%
\begin{pgfscope}%
\pgfpathrectangle{\pgfqpoint{2.867647in}{0.500000in}}{\pgfqpoint{1.764706in}{1.700000in}}%
\pgfusepath{clip}%
\pgfsetbuttcap%
\pgfsetroundjoin%
\definecolor{currentfill}{rgb}{0.971694,0.833208,0.737161}%
\pgfsetfillcolor{currentfill}%
\pgfsetlinewidth{0.311001pt}%
\definecolor{currentstroke}{rgb}{1.000000,1.000000,1.000000}%
\pgfsetstrokecolor{currentstroke}%
\pgfsetdash{}{0pt}%
\pgfpathmoveto{\pgfqpoint{4.228489in}{1.536542in}}%
\pgfpathcurveto{\pgfqpoint{4.235621in}{1.536542in}}{\pgfqpoint{4.242463in}{1.539376in}}{\pgfqpoint{4.247507in}{1.544420in}}%
\pgfpathcurveto{\pgfqpoint{4.252550in}{1.549463in}}{\pgfqpoint{4.255384in}{1.556305in}}{\pgfqpoint{4.255384in}{1.563438in}}%
\pgfpathcurveto{\pgfqpoint{4.255384in}{1.570571in}}{\pgfqpoint{4.252550in}{1.577412in}}{\pgfqpoint{4.247507in}{1.582456in}}%
\pgfpathcurveto{\pgfqpoint{4.242463in}{1.587500in}}{\pgfqpoint{4.235621in}{1.590334in}}{\pgfqpoint{4.228489in}{1.590334in}}%
\pgfpathcurveto{\pgfqpoint{4.221356in}{1.590334in}}{\pgfqpoint{4.214514in}{1.587500in}}{\pgfqpoint{4.209470in}{1.582456in}}%
\pgfpathcurveto{\pgfqpoint{4.204427in}{1.577412in}}{\pgfqpoint{4.201593in}{1.570571in}}{\pgfqpoint{4.201593in}{1.563438in}}%
\pgfpathcurveto{\pgfqpoint{4.201593in}{1.556305in}}{\pgfqpoint{4.204427in}{1.549463in}}{\pgfqpoint{4.209470in}{1.544420in}}%
\pgfpathcurveto{\pgfqpoint{4.214514in}{1.539376in}}{\pgfqpoint{4.221356in}{1.536542in}}{\pgfqpoint{4.228489in}{1.536542in}}%
\pgfpathclose%
\pgfusepath{stroke,fill}%
\end{pgfscope}%
\begin{pgfscope}%
\pgfpathrectangle{\pgfqpoint{2.867647in}{0.500000in}}{\pgfqpoint{1.764706in}{1.700000in}}%
\pgfusepath{clip}%
\pgfsetbuttcap%
\pgfsetroundjoin%
\definecolor{currentfill}{rgb}{0.964920,0.695342,0.545192}%
\pgfsetfillcolor{currentfill}%
\pgfsetlinewidth{0.311001pt}%
\definecolor{currentstroke}{rgb}{1.000000,1.000000,1.000000}%
\pgfsetstrokecolor{currentstroke}%
\pgfsetdash{}{0pt}%
\pgfpathmoveto{\pgfqpoint{3.993399in}{1.017181in}}%
\pgfpathcurveto{\pgfqpoint{4.000531in}{1.017181in}}{\pgfqpoint{4.007373in}{1.020015in}}{\pgfqpoint{4.012417in}{1.025058in}}%
\pgfpathcurveto{\pgfqpoint{4.017460in}{1.030102in}}{\pgfqpoint{4.020294in}{1.036944in}}{\pgfqpoint{4.020294in}{1.044077in}}%
\pgfpathcurveto{\pgfqpoint{4.020294in}{1.051209in}}{\pgfqpoint{4.017460in}{1.058051in}}{\pgfqpoint{4.012417in}{1.063095in}}%
\pgfpathcurveto{\pgfqpoint{4.007373in}{1.068138in}}{\pgfqpoint{4.000531in}{1.070972in}}{\pgfqpoint{3.993399in}{1.070972in}}%
\pgfpathcurveto{\pgfqpoint{3.986266in}{1.070972in}}{\pgfqpoint{3.979424in}{1.068138in}}{\pgfqpoint{3.974380in}{1.063095in}}%
\pgfpathcurveto{\pgfqpoint{3.969337in}{1.058051in}}{\pgfqpoint{3.966503in}{1.051209in}}{\pgfqpoint{3.966503in}{1.044077in}}%
\pgfpathcurveto{\pgfqpoint{3.966503in}{1.036944in}}{\pgfqpoint{3.969337in}{1.030102in}}{\pgfqpoint{3.974380in}{1.025058in}}%
\pgfpathcurveto{\pgfqpoint{3.979424in}{1.020015in}}{\pgfqpoint{3.986266in}{1.017181in}}{\pgfqpoint{3.993399in}{1.017181in}}%
\pgfpathclose%
\pgfusepath{stroke,fill}%
\end{pgfscope}%
\begin{pgfscope}%
\pgfpathrectangle{\pgfqpoint{2.867647in}{0.500000in}}{\pgfqpoint{1.764706in}{1.700000in}}%
\pgfusepath{clip}%
\pgfsetbuttcap%
\pgfsetroundjoin%
\definecolor{currentfill}{rgb}{0.981377,0.920617,0.865369}%
\pgfsetfillcolor{currentfill}%
\pgfsetlinewidth{0.311001pt}%
\definecolor{currentstroke}{rgb}{1.000000,1.000000,1.000000}%
\pgfsetstrokecolor{currentstroke}%
\pgfsetdash{}{0pt}%
\pgfpathmoveto{\pgfqpoint{4.181101in}{1.271425in}}%
\pgfpathcurveto{\pgfqpoint{4.188234in}{1.271425in}}{\pgfqpoint{4.195076in}{1.274259in}}{\pgfqpoint{4.200119in}{1.279302in}}%
\pgfpathcurveto{\pgfqpoint{4.205163in}{1.284346in}}{\pgfqpoint{4.207997in}{1.291188in}}{\pgfqpoint{4.207997in}{1.298320in}}%
\pgfpathcurveto{\pgfqpoint{4.207997in}{1.305453in}}{\pgfqpoint{4.205163in}{1.312295in}}{\pgfqpoint{4.200119in}{1.317338in}}%
\pgfpathcurveto{\pgfqpoint{4.195076in}{1.322382in}}{\pgfqpoint{4.188234in}{1.325216in}}{\pgfqpoint{4.181101in}{1.325216in}}%
\pgfpathcurveto{\pgfqpoint{4.173968in}{1.325216in}}{\pgfqpoint{4.167127in}{1.322382in}}{\pgfqpoint{4.162083in}{1.317338in}}%
\pgfpathcurveto{\pgfqpoint{4.157039in}{1.312295in}}{\pgfqpoint{4.154206in}{1.305453in}}{\pgfqpoint{4.154206in}{1.298320in}}%
\pgfpathcurveto{\pgfqpoint{4.154206in}{1.291188in}}{\pgfqpoint{4.157039in}{1.284346in}}{\pgfqpoint{4.162083in}{1.279302in}}%
\pgfpathcurveto{\pgfqpoint{4.167127in}{1.274259in}}{\pgfqpoint{4.173968in}{1.271425in}}{\pgfqpoint{4.181101in}{1.271425in}}%
\pgfpathclose%
\pgfusepath{stroke,fill}%
\end{pgfscope}%
\begin{pgfscope}%
\pgfpathrectangle{\pgfqpoint{2.867647in}{0.500000in}}{\pgfqpoint{1.764706in}{1.700000in}}%
\pgfusepath{clip}%
\pgfsetbuttcap%
\pgfsetroundjoin%
\definecolor{currentfill}{rgb}{0.972726,0.844889,0.754401}%
\pgfsetfillcolor{currentfill}%
\pgfsetlinewidth{0.311001pt}%
\definecolor{currentstroke}{rgb}{1.000000,1.000000,1.000000}%
\pgfsetstrokecolor{currentstroke}%
\pgfsetdash{}{0pt}%
\pgfpathmoveto{\pgfqpoint{4.242470in}{1.158899in}}%
\pgfpathcurveto{\pgfqpoint{4.249603in}{1.158899in}}{\pgfqpoint{4.256445in}{1.161733in}}{\pgfqpoint{4.261488in}{1.166777in}}%
\pgfpathcurveto{\pgfqpoint{4.266532in}{1.171820in}}{\pgfqpoint{4.269366in}{1.178662in}}{\pgfqpoint{4.269366in}{1.185795in}}%
\pgfpathcurveto{\pgfqpoint{4.269366in}{1.192928in}}{\pgfqpoint{4.266532in}{1.199769in}}{\pgfqpoint{4.261488in}{1.204813in}}%
\pgfpathcurveto{\pgfqpoint{4.256445in}{1.209857in}}{\pgfqpoint{4.249603in}{1.212690in}}{\pgfqpoint{4.242470in}{1.212690in}}%
\pgfpathcurveto{\pgfqpoint{4.235337in}{1.212690in}}{\pgfqpoint{4.228496in}{1.209857in}}{\pgfqpoint{4.223452in}{1.204813in}}%
\pgfpathcurveto{\pgfqpoint{4.218408in}{1.199769in}}{\pgfqpoint{4.215575in}{1.192928in}}{\pgfqpoint{4.215575in}{1.185795in}}%
\pgfpathcurveto{\pgfqpoint{4.215575in}{1.178662in}}{\pgfqpoint{4.218408in}{1.171820in}}{\pgfqpoint{4.223452in}{1.166777in}}%
\pgfpathcurveto{\pgfqpoint{4.228496in}{1.161733in}}{\pgfqpoint{4.235337in}{1.158899in}}{\pgfqpoint{4.242470in}{1.158899in}}%
\pgfpathclose%
\pgfusepath{stroke,fill}%
\end{pgfscope}%
\begin{pgfscope}%
\pgfpathrectangle{\pgfqpoint{2.867647in}{0.500000in}}{\pgfqpoint{1.764706in}{1.700000in}}%
\pgfusepath{clip}%
\pgfsetbuttcap%
\pgfsetroundjoin%
\definecolor{currentfill}{rgb}{0.974412,0.862387,0.780156}%
\pgfsetfillcolor{currentfill}%
\pgfsetlinewidth{0.311001pt}%
\definecolor{currentstroke}{rgb}{1.000000,1.000000,1.000000}%
\pgfsetstrokecolor{currentstroke}%
\pgfsetdash{}{0pt}%
\pgfpathmoveto{\pgfqpoint{4.090335in}{1.645502in}}%
\pgfpathcurveto{\pgfqpoint{4.097467in}{1.645502in}}{\pgfqpoint{4.104309in}{1.648336in}}{\pgfqpoint{4.109353in}{1.653379in}}%
\pgfpathcurveto{\pgfqpoint{4.114396in}{1.658423in}}{\pgfqpoint{4.117230in}{1.665265in}}{\pgfqpoint{4.117230in}{1.672398in}}%
\pgfpathcurveto{\pgfqpoint{4.117230in}{1.679530in}}{\pgfqpoint{4.114396in}{1.686372in}}{\pgfqpoint{4.109353in}{1.691416in}}%
\pgfpathcurveto{\pgfqpoint{4.104309in}{1.696459in}}{\pgfqpoint{4.097467in}{1.699293in}}{\pgfqpoint{4.090335in}{1.699293in}}%
\pgfpathcurveto{\pgfqpoint{4.083202in}{1.699293in}}{\pgfqpoint{4.076360in}{1.696459in}}{\pgfqpoint{4.071316in}{1.691416in}}%
\pgfpathcurveto{\pgfqpoint{4.066273in}{1.686372in}}{\pgfqpoint{4.063439in}{1.679530in}}{\pgfqpoint{4.063439in}{1.672398in}}%
\pgfpathcurveto{\pgfqpoint{4.063439in}{1.665265in}}{\pgfqpoint{4.066273in}{1.658423in}}{\pgfqpoint{4.071316in}{1.653379in}}%
\pgfpathcurveto{\pgfqpoint{4.076360in}{1.648336in}}{\pgfqpoint{4.083202in}{1.645502in}}{\pgfqpoint{4.090335in}{1.645502in}}%
\pgfpathclose%
\pgfusepath{stroke,fill}%
\end{pgfscope}%
\begin{pgfscope}%
\pgfpathrectangle{\pgfqpoint{2.867647in}{0.500000in}}{\pgfqpoint{1.764706in}{1.700000in}}%
\pgfusepath{clip}%
\pgfsetbuttcap%
\pgfsetroundjoin%
\definecolor{currentfill}{rgb}{0.966812,0.762584,0.633643}%
\pgfsetfillcolor{currentfill}%
\pgfsetlinewidth{0.311001pt}%
\definecolor{currentstroke}{rgb}{1.000000,1.000000,1.000000}%
\pgfsetstrokecolor{currentstroke}%
\pgfsetdash{}{0pt}%
\pgfpathmoveto{\pgfqpoint{4.019932in}{1.695981in}}%
\pgfpathcurveto{\pgfqpoint{4.027064in}{1.695981in}}{\pgfqpoint{4.033906in}{1.698815in}}{\pgfqpoint{4.038950in}{1.703859in}}%
\pgfpathcurveto{\pgfqpoint{4.043993in}{1.708903in}}{\pgfqpoint{4.046827in}{1.715744in}}{\pgfqpoint{4.046827in}{1.722877in}}%
\pgfpathcurveto{\pgfqpoint{4.046827in}{1.730010in}}{\pgfqpoint{4.043993in}{1.736852in}}{\pgfqpoint{4.038950in}{1.741895in}}%
\pgfpathcurveto{\pgfqpoint{4.033906in}{1.746939in}}{\pgfqpoint{4.027064in}{1.749773in}}{\pgfqpoint{4.019932in}{1.749773in}}%
\pgfpathcurveto{\pgfqpoint{4.012799in}{1.749773in}}{\pgfqpoint{4.005957in}{1.746939in}}{\pgfqpoint{4.000913in}{1.741895in}}%
\pgfpathcurveto{\pgfqpoint{3.995870in}{1.736852in}}{\pgfqpoint{3.993036in}{1.730010in}}{\pgfqpoint{3.993036in}{1.722877in}}%
\pgfpathcurveto{\pgfqpoint{3.993036in}{1.715744in}}{\pgfqpoint{3.995870in}{1.708903in}}{\pgfqpoint{4.000913in}{1.703859in}}%
\pgfpathcurveto{\pgfqpoint{4.005957in}{1.698815in}}{\pgfqpoint{4.012799in}{1.695981in}}{\pgfqpoint{4.019932in}{1.695981in}}%
\pgfpathclose%
\pgfusepath{stroke,fill}%
\end{pgfscope}%
\begin{pgfscope}%
\pgfpathrectangle{\pgfqpoint{2.867647in}{0.500000in}}{\pgfqpoint{1.764706in}{1.700000in}}%
\pgfusepath{clip}%
\pgfsetbuttcap%
\pgfsetroundjoin%
\definecolor{currentfill}{rgb}{0.970718,0.821518,0.719872}%
\pgfsetfillcolor{currentfill}%
\pgfsetlinewidth{0.311001pt}%
\definecolor{currentstroke}{rgb}{1.000000,1.000000,1.000000}%
\pgfsetstrokecolor{currentstroke}%
\pgfsetdash{}{0pt}%
\pgfpathmoveto{\pgfqpoint{4.079348in}{1.144919in}}%
\pgfpathcurveto{\pgfqpoint{4.086481in}{1.144919in}}{\pgfqpoint{4.093322in}{1.147753in}}{\pgfqpoint{4.098366in}{1.152797in}}%
\pgfpathcurveto{\pgfqpoint{4.103410in}{1.157841in}}{\pgfqpoint{4.106244in}{1.164682in}}{\pgfqpoint{4.106244in}{1.171815in}}%
\pgfpathcurveto{\pgfqpoint{4.106244in}{1.178948in}}{\pgfqpoint{4.103410in}{1.185790in}}{\pgfqpoint{4.098366in}{1.190833in}}%
\pgfpathcurveto{\pgfqpoint{4.093322in}{1.195877in}}{\pgfqpoint{4.086481in}{1.198711in}}{\pgfqpoint{4.079348in}{1.198711in}}%
\pgfpathcurveto{\pgfqpoint{4.072215in}{1.198711in}}{\pgfqpoint{4.065373in}{1.195877in}}{\pgfqpoint{4.060330in}{1.190833in}}%
\pgfpathcurveto{\pgfqpoint{4.055286in}{1.185790in}}{\pgfqpoint{4.052452in}{1.178948in}}{\pgfqpoint{4.052452in}{1.171815in}}%
\pgfpathcurveto{\pgfqpoint{4.052452in}{1.164682in}}{\pgfqpoint{4.055286in}{1.157841in}}{\pgfqpoint{4.060330in}{1.152797in}}%
\pgfpathcurveto{\pgfqpoint{4.065373in}{1.147753in}}{\pgfqpoint{4.072215in}{1.144919in}}{\pgfqpoint{4.079348in}{1.144919in}}%
\pgfpathclose%
\pgfusepath{stroke,fill}%
\end{pgfscope}%
\begin{pgfscope}%
\pgfpathrectangle{\pgfqpoint{2.867647in}{0.500000in}}{\pgfqpoint{1.764706in}{1.700000in}}%
\pgfusepath{clip}%
\pgfsetbuttcap%
\pgfsetroundjoin%
\definecolor{currentfill}{rgb}{0.981377,0.920617,0.865369}%
\pgfsetfillcolor{currentfill}%
\pgfsetlinewidth{0.311001pt}%
\definecolor{currentstroke}{rgb}{1.000000,1.000000,1.000000}%
\pgfsetstrokecolor{currentstroke}%
\pgfsetdash{}{0pt}%
\pgfpathmoveto{\pgfqpoint{4.185525in}{1.178827in}}%
\pgfpathcurveto{\pgfqpoint{4.192658in}{1.178827in}}{\pgfqpoint{4.199499in}{1.181660in}}{\pgfqpoint{4.204543in}{1.186704in}}%
\pgfpathcurveto{\pgfqpoint{4.209587in}{1.191748in}}{\pgfqpoint{4.212421in}{1.198589in}}{\pgfqpoint{4.212421in}{1.205722in}}%
\pgfpathcurveto{\pgfqpoint{4.212421in}{1.212855in}}{\pgfqpoint{4.209587in}{1.219697in}}{\pgfqpoint{4.204543in}{1.224740in}}%
\pgfpathcurveto{\pgfqpoint{4.199499in}{1.229784in}}{\pgfqpoint{4.192658in}{1.232618in}}{\pgfqpoint{4.185525in}{1.232618in}}%
\pgfpathcurveto{\pgfqpoint{4.178392in}{1.232618in}}{\pgfqpoint{4.171550in}{1.229784in}}{\pgfqpoint{4.166507in}{1.224740in}}%
\pgfpathcurveto{\pgfqpoint{4.161463in}{1.219697in}}{\pgfqpoint{4.158629in}{1.212855in}}{\pgfqpoint{4.158629in}{1.205722in}}%
\pgfpathcurveto{\pgfqpoint{4.158629in}{1.198589in}}{\pgfqpoint{4.161463in}{1.191748in}}{\pgfqpoint{4.166507in}{1.186704in}}%
\pgfpathcurveto{\pgfqpoint{4.171550in}{1.181660in}}{\pgfqpoint{4.178392in}{1.178827in}}{\pgfqpoint{4.185525in}{1.178827in}}%
\pgfpathclose%
\pgfusepath{stroke,fill}%
\end{pgfscope}%
\begin{pgfscope}%
\pgfpathrectangle{\pgfqpoint{2.867647in}{0.500000in}}{\pgfqpoint{1.764706in}{1.700000in}}%
\pgfusepath{clip}%
\pgfsetbuttcap%
\pgfsetroundjoin%
\definecolor{currentfill}{rgb}{0.981377,0.920617,0.865369}%
\pgfsetfillcolor{currentfill}%
\pgfsetlinewidth{0.311001pt}%
\definecolor{currentstroke}{rgb}{1.000000,1.000000,1.000000}%
\pgfsetstrokecolor{currentstroke}%
\pgfsetdash{}{0pt}%
\pgfpathmoveto{\pgfqpoint{4.198107in}{1.322131in}}%
\pgfpathcurveto{\pgfqpoint{4.205240in}{1.322131in}}{\pgfqpoint{4.212082in}{1.324965in}}{\pgfqpoint{4.217125in}{1.330009in}}%
\pgfpathcurveto{\pgfqpoint{4.222169in}{1.335053in}}{\pgfqpoint{4.225003in}{1.341894in}}{\pgfqpoint{4.225003in}{1.349027in}}%
\pgfpathcurveto{\pgfqpoint{4.225003in}{1.356160in}}{\pgfqpoint{4.222169in}{1.363002in}}{\pgfqpoint{4.217125in}{1.368045in}}%
\pgfpathcurveto{\pgfqpoint{4.212082in}{1.373089in}}{\pgfqpoint{4.205240in}{1.375923in}}{\pgfqpoint{4.198107in}{1.375923in}}%
\pgfpathcurveto{\pgfqpoint{4.190974in}{1.375923in}}{\pgfqpoint{4.184133in}{1.373089in}}{\pgfqpoint{4.179089in}{1.368045in}}%
\pgfpathcurveto{\pgfqpoint{4.174045in}{1.363002in}}{\pgfqpoint{4.171212in}{1.356160in}}{\pgfqpoint{4.171212in}{1.349027in}}%
\pgfpathcurveto{\pgfqpoint{4.171212in}{1.341894in}}{\pgfqpoint{4.174045in}{1.335053in}}{\pgfqpoint{4.179089in}{1.330009in}}%
\pgfpathcurveto{\pgfqpoint{4.184133in}{1.324965in}}{\pgfqpoint{4.190974in}{1.322131in}}{\pgfqpoint{4.198107in}{1.322131in}}%
\pgfpathclose%
\pgfusepath{stroke,fill}%
\end{pgfscope}%
\begin{pgfscope}%
\pgfpathrectangle{\pgfqpoint{2.867647in}{0.500000in}}{\pgfqpoint{1.764706in}{1.700000in}}%
\pgfusepath{clip}%
\pgfsetbuttcap%
\pgfsetroundjoin%
\definecolor{currentfill}{rgb}{0.972726,0.844889,0.754401}%
\pgfsetfillcolor{currentfill}%
\pgfsetlinewidth{0.311001pt}%
\definecolor{currentstroke}{rgb}{1.000000,1.000000,1.000000}%
\pgfsetstrokecolor{currentstroke}%
\pgfsetdash{}{0pt}%
\pgfpathmoveto{\pgfqpoint{4.069274in}{1.014366in}}%
\pgfpathcurveto{\pgfqpoint{4.076407in}{1.014366in}}{\pgfqpoint{4.083248in}{1.017200in}}{\pgfqpoint{4.088292in}{1.022244in}}%
\pgfpathcurveto{\pgfqpoint{4.093335in}{1.027288in}}{\pgfqpoint{4.096169in}{1.034129in}}{\pgfqpoint{4.096169in}{1.041262in}}%
\pgfpathcurveto{\pgfqpoint{4.096169in}{1.048395in}}{\pgfqpoint{4.093335in}{1.055236in}}{\pgfqpoint{4.088292in}{1.060280in}}%
\pgfpathcurveto{\pgfqpoint{4.083248in}{1.065324in}}{\pgfqpoint{4.076407in}{1.068158in}}{\pgfqpoint{4.069274in}{1.068158in}}%
\pgfpathcurveto{\pgfqpoint{4.062141in}{1.068158in}}{\pgfqpoint{4.055299in}{1.065324in}}{\pgfqpoint{4.050256in}{1.060280in}}%
\pgfpathcurveto{\pgfqpoint{4.045212in}{1.055236in}}{\pgfqpoint{4.042378in}{1.048395in}}{\pgfqpoint{4.042378in}{1.041262in}}%
\pgfpathcurveto{\pgfqpoint{4.042378in}{1.034129in}}{\pgfqpoint{4.045212in}{1.027288in}}{\pgfqpoint{4.050256in}{1.022244in}}%
\pgfpathcurveto{\pgfqpoint{4.055299in}{1.017200in}}{\pgfqpoint{4.062141in}{1.014366in}}{\pgfqpoint{4.069274in}{1.014366in}}%
\pgfpathclose%
\pgfusepath{stroke,fill}%
\end{pgfscope}%
\begin{pgfscope}%
\pgfpathrectangle{\pgfqpoint{2.867647in}{0.500000in}}{\pgfqpoint{1.764706in}{1.700000in}}%
\pgfusepath{clip}%
\pgfsetbuttcap%
\pgfsetroundjoin%
\definecolor{currentfill}{rgb}{0.970255,0.815666,0.711203}%
\pgfsetfillcolor{currentfill}%
\pgfsetlinewidth{0.311001pt}%
\definecolor{currentstroke}{rgb}{1.000000,1.000000,1.000000}%
\pgfsetstrokecolor{currentstroke}%
\pgfsetdash{}{0pt}%
\pgfpathmoveto{\pgfqpoint{4.076951in}{1.486931in}}%
\pgfpathcurveto{\pgfqpoint{4.084084in}{1.486931in}}{\pgfqpoint{4.090925in}{1.489765in}}{\pgfqpoint{4.095969in}{1.494808in}}%
\pgfpathcurveto{\pgfqpoint{4.101013in}{1.499852in}}{\pgfqpoint{4.103847in}{1.506694in}}{\pgfqpoint{4.103847in}{1.513826in}}%
\pgfpathcurveto{\pgfqpoint{4.103847in}{1.520959in}}{\pgfqpoint{4.101013in}{1.527801in}}{\pgfqpoint{4.095969in}{1.532845in}}%
\pgfpathcurveto{\pgfqpoint{4.090925in}{1.537888in}}{\pgfqpoint{4.084084in}{1.540722in}}{\pgfqpoint{4.076951in}{1.540722in}}%
\pgfpathcurveto{\pgfqpoint{4.069818in}{1.540722in}}{\pgfqpoint{4.062976in}{1.537888in}}{\pgfqpoint{4.057933in}{1.532845in}}%
\pgfpathcurveto{\pgfqpoint{4.052889in}{1.527801in}}{\pgfqpoint{4.050055in}{1.520959in}}{\pgfqpoint{4.050055in}{1.513826in}}%
\pgfpathcurveto{\pgfqpoint{4.050055in}{1.506694in}}{\pgfqpoint{4.052889in}{1.499852in}}{\pgfqpoint{4.057933in}{1.494808in}}%
\pgfpathcurveto{\pgfqpoint{4.062976in}{1.489765in}}{\pgfqpoint{4.069818in}{1.486931in}}{\pgfqpoint{4.076951in}{1.486931in}}%
\pgfpathclose%
\pgfusepath{stroke,fill}%
\end{pgfscope}%
\begin{pgfscope}%
\pgfpathrectangle{\pgfqpoint{2.867647in}{0.500000in}}{\pgfqpoint{1.764706in}{1.700000in}}%
\pgfusepath{clip}%
\pgfsetbuttcap%
\pgfsetroundjoin%
\definecolor{currentfill}{rgb}{0.979124,0.903132,0.839793}%
\pgfsetfillcolor{currentfill}%
\pgfsetlinewidth{0.311001pt}%
\definecolor{currentstroke}{rgb}{1.000000,1.000000,1.000000}%
\pgfsetstrokecolor{currentstroke}%
\pgfsetdash{}{0pt}%
\pgfpathmoveto{\pgfqpoint{4.169820in}{1.552512in}}%
\pgfpathcurveto{\pgfqpoint{4.176953in}{1.552512in}}{\pgfqpoint{4.183795in}{1.555345in}}{\pgfqpoint{4.188838in}{1.560389in}}%
\pgfpathcurveto{\pgfqpoint{4.193882in}{1.565433in}}{\pgfqpoint{4.196716in}{1.572274in}}{\pgfqpoint{4.196716in}{1.579407in}}%
\pgfpathcurveto{\pgfqpoint{4.196716in}{1.586540in}}{\pgfqpoint{4.193882in}{1.593382in}}{\pgfqpoint{4.188838in}{1.598425in}}%
\pgfpathcurveto{\pgfqpoint{4.183795in}{1.603469in}}{\pgfqpoint{4.176953in}{1.606303in}}{\pgfqpoint{4.169820in}{1.606303in}}%
\pgfpathcurveto{\pgfqpoint{4.162687in}{1.606303in}}{\pgfqpoint{4.155846in}{1.603469in}}{\pgfqpoint{4.150802in}{1.598425in}}%
\pgfpathcurveto{\pgfqpoint{4.145758in}{1.593382in}}{\pgfqpoint{4.142925in}{1.586540in}}{\pgfqpoint{4.142925in}{1.579407in}}%
\pgfpathcurveto{\pgfqpoint{4.142925in}{1.572274in}}{\pgfqpoint{4.145758in}{1.565433in}}{\pgfqpoint{4.150802in}{1.560389in}}%
\pgfpathcurveto{\pgfqpoint{4.155846in}{1.555345in}}{\pgfqpoint{4.162687in}{1.552512in}}{\pgfqpoint{4.169820in}{1.552512in}}%
\pgfpathclose%
\pgfusepath{stroke,fill}%
\end{pgfscope}%
\begin{pgfscope}%
\pgfpathrectangle{\pgfqpoint{2.867647in}{0.500000in}}{\pgfqpoint{1.764706in}{1.700000in}}%
\pgfusepath{clip}%
\pgfsetbuttcap%
\pgfsetroundjoin%
\definecolor{currentfill}{rgb}{0.950017,0.427714,0.292447}%
\pgfsetfillcolor{currentfill}%
\pgfsetlinewidth{0.311001pt}%
\definecolor{currentstroke}{rgb}{1.000000,1.000000,1.000000}%
\pgfsetstrokecolor{currentstroke}%
\pgfsetdash{}{0pt}%
\pgfpathmoveto{\pgfqpoint{3.982507in}{1.849752in}}%
\pgfpathcurveto{\pgfqpoint{3.989640in}{1.849752in}}{\pgfqpoint{3.996482in}{1.852586in}}{\pgfqpoint{4.001525in}{1.857630in}}%
\pgfpathcurveto{\pgfqpoint{4.006569in}{1.862674in}}{\pgfqpoint{4.009403in}{1.869515in}}{\pgfqpoint{4.009403in}{1.876648in}}%
\pgfpathcurveto{\pgfqpoint{4.009403in}{1.883781in}}{\pgfqpoint{4.006569in}{1.890623in}}{\pgfqpoint{4.001525in}{1.895666in}}%
\pgfpathcurveto{\pgfqpoint{3.996482in}{1.900710in}}{\pgfqpoint{3.989640in}{1.903544in}}{\pgfqpoint{3.982507in}{1.903544in}}%
\pgfpathcurveto{\pgfqpoint{3.975375in}{1.903544in}}{\pgfqpoint{3.968533in}{1.900710in}}{\pgfqpoint{3.963489in}{1.895666in}}%
\pgfpathcurveto{\pgfqpoint{3.958446in}{1.890623in}}{\pgfqpoint{3.955612in}{1.883781in}}{\pgfqpoint{3.955612in}{1.876648in}}%
\pgfpathcurveto{\pgfqpoint{3.955612in}{1.869515in}}{\pgfqpoint{3.958446in}{1.862674in}}{\pgfqpoint{3.963489in}{1.857630in}}%
\pgfpathcurveto{\pgfqpoint{3.968533in}{1.852586in}}{\pgfqpoint{3.975375in}{1.849752in}}{\pgfqpoint{3.982507in}{1.849752in}}%
\pgfpathclose%
\pgfusepath{stroke,fill}%
\end{pgfscope}%
\begin{pgfscope}%
\pgfpathrectangle{\pgfqpoint{2.867647in}{0.500000in}}{\pgfqpoint{1.764706in}{1.700000in}}%
\pgfusepath{clip}%
\pgfsetbuttcap%
\pgfsetroundjoin%
\definecolor{currentfill}{rgb}{0.974412,0.862387,0.780156}%
\pgfsetfillcolor{currentfill}%
\pgfsetlinewidth{0.311001pt}%
\definecolor{currentstroke}{rgb}{1.000000,1.000000,1.000000}%
\pgfsetstrokecolor{currentstroke}%
\pgfsetdash{}{0pt}%
\pgfpathmoveto{\pgfqpoint{4.114370in}{1.206901in}}%
\pgfpathcurveto{\pgfqpoint{4.121502in}{1.206901in}}{\pgfqpoint{4.128344in}{1.209735in}}{\pgfqpoint{4.133388in}{1.214779in}}%
\pgfpathcurveto{\pgfqpoint{4.138431in}{1.219823in}}{\pgfqpoint{4.141265in}{1.226664in}}{\pgfqpoint{4.141265in}{1.233797in}}%
\pgfpathcurveto{\pgfqpoint{4.141265in}{1.240930in}}{\pgfqpoint{4.138431in}{1.247772in}}{\pgfqpoint{4.133388in}{1.252815in}}%
\pgfpathcurveto{\pgfqpoint{4.128344in}{1.257859in}}{\pgfqpoint{4.121502in}{1.260693in}}{\pgfqpoint{4.114370in}{1.260693in}}%
\pgfpathcurveto{\pgfqpoint{4.107237in}{1.260693in}}{\pgfqpoint{4.100395in}{1.257859in}}{\pgfqpoint{4.095351in}{1.252815in}}%
\pgfpathcurveto{\pgfqpoint{4.090308in}{1.247772in}}{\pgfqpoint{4.087474in}{1.240930in}}{\pgfqpoint{4.087474in}{1.233797in}}%
\pgfpathcurveto{\pgfqpoint{4.087474in}{1.226664in}}{\pgfqpoint{4.090308in}{1.219823in}}{\pgfqpoint{4.095351in}{1.214779in}}%
\pgfpathcurveto{\pgfqpoint{4.100395in}{1.209735in}}{\pgfqpoint{4.107237in}{1.206901in}}{\pgfqpoint{4.114370in}{1.206901in}}%
\pgfpathclose%
\pgfusepath{stroke,fill}%
\end{pgfscope}%
\begin{pgfscope}%
\pgfpathrectangle{\pgfqpoint{2.867647in}{0.500000in}}{\pgfqpoint{1.764706in}{1.700000in}}%
\pgfusepath{clip}%
\pgfsetbuttcap%
\pgfsetroundjoin%
\definecolor{currentfill}{rgb}{0.979891,0.908948,0.848279}%
\pgfsetfillcolor{currentfill}%
\pgfsetlinewidth{0.311001pt}%
\definecolor{currentstroke}{rgb}{1.000000,1.000000,1.000000}%
\pgfsetstrokecolor{currentstroke}%
\pgfsetdash{}{0pt}%
\pgfpathmoveto{\pgfqpoint{4.212510in}{1.357633in}}%
\pgfpathcurveto{\pgfqpoint{4.219643in}{1.357633in}}{\pgfqpoint{4.226484in}{1.360467in}}{\pgfqpoint{4.231528in}{1.365510in}}%
\pgfpathcurveto{\pgfqpoint{4.236572in}{1.370554in}}{\pgfqpoint{4.239406in}{1.377396in}}{\pgfqpoint{4.239406in}{1.384528in}}%
\pgfpathcurveto{\pgfqpoint{4.239406in}{1.391661in}}{\pgfqpoint{4.236572in}{1.398503in}}{\pgfqpoint{4.231528in}{1.403546in}}%
\pgfpathcurveto{\pgfqpoint{4.226484in}{1.408590in}}{\pgfqpoint{4.219643in}{1.411424in}}{\pgfqpoint{4.212510in}{1.411424in}}%
\pgfpathcurveto{\pgfqpoint{4.205377in}{1.411424in}}{\pgfqpoint{4.198535in}{1.408590in}}{\pgfqpoint{4.193492in}{1.403546in}}%
\pgfpathcurveto{\pgfqpoint{4.188448in}{1.398503in}}{\pgfqpoint{4.185614in}{1.391661in}}{\pgfqpoint{4.185614in}{1.384528in}}%
\pgfpathcurveto{\pgfqpoint{4.185614in}{1.377396in}}{\pgfqpoint{4.188448in}{1.370554in}}{\pgfqpoint{4.193492in}{1.365510in}}%
\pgfpathcurveto{\pgfqpoint{4.198535in}{1.360467in}}{\pgfqpoint{4.205377in}{1.357633in}}{\pgfqpoint{4.212510in}{1.357633in}}%
\pgfpathclose%
\pgfusepath{stroke,fill}%
\end{pgfscope}%
\begin{pgfscope}%
\pgfpathrectangle{\pgfqpoint{2.867647in}{0.500000in}}{\pgfqpoint{1.764706in}{1.700000in}}%
\pgfusepath{clip}%
\pgfsetbuttcap%
\pgfsetroundjoin%
\definecolor{currentfill}{rgb}{0.964032,0.651225,0.493258}%
\pgfsetfillcolor{currentfill}%
\pgfsetlinewidth{0.311001pt}%
\definecolor{currentstroke}{rgb}{1.000000,1.000000,1.000000}%
\pgfsetstrokecolor{currentstroke}%
\pgfsetdash{}{0pt}%
\pgfpathmoveto{\pgfqpoint{3.968255in}{0.961912in}}%
\pgfpathcurveto{\pgfqpoint{3.975388in}{0.961912in}}{\pgfqpoint{3.982230in}{0.964746in}}{\pgfqpoint{3.987273in}{0.969789in}}%
\pgfpathcurveto{\pgfqpoint{3.992317in}{0.974833in}}{\pgfqpoint{3.995151in}{0.981675in}}{\pgfqpoint{3.995151in}{0.988807in}}%
\pgfpathcurveto{\pgfqpoint{3.995151in}{0.995940in}}{\pgfqpoint{3.992317in}{1.002782in}}{\pgfqpoint{3.987273in}{1.007826in}}%
\pgfpathcurveto{\pgfqpoint{3.982230in}{1.012869in}}{\pgfqpoint{3.975388in}{1.015703in}}{\pgfqpoint{3.968255in}{1.015703in}}%
\pgfpathcurveto{\pgfqpoint{3.961122in}{1.015703in}}{\pgfqpoint{3.954281in}{1.012869in}}{\pgfqpoint{3.949237in}{1.007826in}}%
\pgfpathcurveto{\pgfqpoint{3.944194in}{1.002782in}}{\pgfqpoint{3.941360in}{0.995940in}}{\pgfqpoint{3.941360in}{0.988807in}}%
\pgfpathcurveto{\pgfqpoint{3.941360in}{0.981675in}}{\pgfqpoint{3.944194in}{0.974833in}}{\pgfqpoint{3.949237in}{0.969789in}}%
\pgfpathcurveto{\pgfqpoint{3.954281in}{0.964746in}}{\pgfqpoint{3.961122in}{0.961912in}}{\pgfqpoint{3.968255in}{0.961912in}}%
\pgfpathclose%
\pgfusepath{stroke,fill}%
\end{pgfscope}%
\begin{pgfscope}%
\pgfpathrectangle{\pgfqpoint{2.867647in}{0.500000in}}{\pgfqpoint{1.764706in}{1.700000in}}%
\pgfusepath{clip}%
\pgfsetbuttcap%
\pgfsetroundjoin%
\definecolor{currentfill}{rgb}{0.965928,0.738443,0.600540}%
\pgfsetfillcolor{currentfill}%
\pgfsetlinewidth{0.311001pt}%
\definecolor{currentstroke}{rgb}{1.000000,1.000000,1.000000}%
\pgfsetstrokecolor{currentstroke}%
\pgfsetdash{}{0pt}%
\pgfpathmoveto{\pgfqpoint{4.082954in}{0.905022in}}%
\pgfpathcurveto{\pgfqpoint{4.090087in}{0.905022in}}{\pgfqpoint{4.096929in}{0.907856in}}{\pgfqpoint{4.101973in}{0.912900in}}%
\pgfpathcurveto{\pgfqpoint{4.107016in}{0.917944in}}{\pgfqpoint{4.109850in}{0.924785in}}{\pgfqpoint{4.109850in}{0.931918in}}%
\pgfpathcurveto{\pgfqpoint{4.109850in}{0.939051in}}{\pgfqpoint{4.107016in}{0.945893in}}{\pgfqpoint{4.101973in}{0.950936in}}%
\pgfpathcurveto{\pgfqpoint{4.096929in}{0.955980in}}{\pgfqpoint{4.090087in}{0.958814in}}{\pgfqpoint{4.082954in}{0.958814in}}%
\pgfpathcurveto{\pgfqpoint{4.075822in}{0.958814in}}{\pgfqpoint{4.068980in}{0.955980in}}{\pgfqpoint{4.063936in}{0.950936in}}%
\pgfpathcurveto{\pgfqpoint{4.058893in}{0.945893in}}{\pgfqpoint{4.056059in}{0.939051in}}{\pgfqpoint{4.056059in}{0.931918in}}%
\pgfpathcurveto{\pgfqpoint{4.056059in}{0.924785in}}{\pgfqpoint{4.058893in}{0.917944in}}{\pgfqpoint{4.063936in}{0.912900in}}%
\pgfpathcurveto{\pgfqpoint{4.068980in}{0.907856in}}{\pgfqpoint{4.075822in}{0.905022in}}{\pgfqpoint{4.082954in}{0.905022in}}%
\pgfpathclose%
\pgfusepath{stroke,fill}%
\end{pgfscope}%
\begin{pgfscope}%
\pgfpathrectangle{\pgfqpoint{2.867647in}{0.500000in}}{\pgfqpoint{1.764706in}{1.700000in}}%
\pgfusepath{clip}%
\pgfsetbuttcap%
\pgfsetroundjoin%
\definecolor{currentfill}{rgb}{0.966120,0.744512,0.608720}%
\pgfsetfillcolor{currentfill}%
\pgfsetlinewidth{0.311001pt}%
\definecolor{currentstroke}{rgb}{1.000000,1.000000,1.000000}%
\pgfsetstrokecolor{currentstroke}%
\pgfsetdash{}{0pt}%
\pgfpathmoveto{\pgfqpoint{4.274847in}{1.491528in}}%
\pgfpathcurveto{\pgfqpoint{4.281980in}{1.491528in}}{\pgfqpoint{4.288822in}{1.494362in}}{\pgfqpoint{4.293865in}{1.499405in}}%
\pgfpathcurveto{\pgfqpoint{4.298909in}{1.504449in}}{\pgfqpoint{4.301743in}{1.511291in}}{\pgfqpoint{4.301743in}{1.518423in}}%
\pgfpathcurveto{\pgfqpoint{4.301743in}{1.525556in}}{\pgfqpoint{4.298909in}{1.532398in}}{\pgfqpoint{4.293865in}{1.537442in}}%
\pgfpathcurveto{\pgfqpoint{4.288822in}{1.542485in}}{\pgfqpoint{4.281980in}{1.545319in}}{\pgfqpoint{4.274847in}{1.545319in}}%
\pgfpathcurveto{\pgfqpoint{4.267715in}{1.545319in}}{\pgfqpoint{4.260873in}{1.542485in}}{\pgfqpoint{4.255829in}{1.537442in}}%
\pgfpathcurveto{\pgfqpoint{4.250786in}{1.532398in}}{\pgfqpoint{4.247952in}{1.525556in}}{\pgfqpoint{4.247952in}{1.518423in}}%
\pgfpathcurveto{\pgfqpoint{4.247952in}{1.511291in}}{\pgfqpoint{4.250786in}{1.504449in}}{\pgfqpoint{4.255829in}{1.499405in}}%
\pgfpathcurveto{\pgfqpoint{4.260873in}{1.494362in}}{\pgfqpoint{4.267715in}{1.491528in}}{\pgfqpoint{4.274847in}{1.491528in}}%
\pgfpathclose%
\pgfusepath{stroke,fill}%
\end{pgfscope}%
\begin{pgfscope}%
\pgfpathrectangle{\pgfqpoint{2.867647in}{0.500000in}}{\pgfqpoint{1.764706in}{1.700000in}}%
\pgfusepath{clip}%
\pgfsetbuttcap%
\pgfsetroundjoin%
\definecolor{currentfill}{rgb}{0.971694,0.833208,0.737161}%
\pgfsetfillcolor{currentfill}%
\pgfsetlinewidth{0.311001pt}%
\definecolor{currentstroke}{rgb}{1.000000,1.000000,1.000000}%
\pgfsetstrokecolor{currentstroke}%
\pgfsetdash{}{0pt}%
\pgfpathmoveto{\pgfqpoint{4.068398in}{1.670130in}}%
\pgfpathcurveto{\pgfqpoint{4.075531in}{1.670130in}}{\pgfqpoint{4.082372in}{1.672964in}}{\pgfqpoint{4.087416in}{1.678008in}}%
\pgfpathcurveto{\pgfqpoint{4.092460in}{1.683052in}}{\pgfqpoint{4.095294in}{1.689893in}}{\pgfqpoint{4.095294in}{1.697026in}}%
\pgfpathcurveto{\pgfqpoint{4.095294in}{1.704159in}}{\pgfqpoint{4.092460in}{1.711001in}}{\pgfqpoint{4.087416in}{1.716044in}}%
\pgfpathcurveto{\pgfqpoint{4.082372in}{1.721088in}}{\pgfqpoint{4.075531in}{1.723922in}}{\pgfqpoint{4.068398in}{1.723922in}}%
\pgfpathcurveto{\pgfqpoint{4.061265in}{1.723922in}}{\pgfqpoint{4.054424in}{1.721088in}}{\pgfqpoint{4.049380in}{1.716044in}}%
\pgfpathcurveto{\pgfqpoint{4.044336in}{1.711001in}}{\pgfqpoint{4.041502in}{1.704159in}}{\pgfqpoint{4.041502in}{1.697026in}}%
\pgfpathcurveto{\pgfqpoint{4.041502in}{1.689893in}}{\pgfqpoint{4.044336in}{1.683052in}}{\pgfqpoint{4.049380in}{1.678008in}}%
\pgfpathcurveto{\pgfqpoint{4.054424in}{1.672964in}}{\pgfqpoint{4.061265in}{1.670130in}}{\pgfqpoint{4.068398in}{1.670130in}}%
\pgfpathclose%
\pgfusepath{stroke,fill}%
\end{pgfscope}%
\begin{pgfscope}%
\pgfpathrectangle{\pgfqpoint{2.867647in}{0.500000in}}{\pgfqpoint{1.764706in}{1.700000in}}%
\pgfusepath{clip}%
\pgfsetbuttcap%
\pgfsetroundjoin%
\definecolor{currentfill}{rgb}{0.981377,0.920617,0.865369}%
\pgfsetfillcolor{currentfill}%
\pgfsetlinewidth{0.311001pt}%
\definecolor{currentstroke}{rgb}{1.000000,1.000000,1.000000}%
\pgfsetstrokecolor{currentstroke}%
\pgfsetdash{}{0pt}%
\pgfpathmoveto{\pgfqpoint{4.187321in}{1.186011in}}%
\pgfpathcurveto{\pgfqpoint{4.194453in}{1.186011in}}{\pgfqpoint{4.201295in}{1.188845in}}{\pgfqpoint{4.206339in}{1.193888in}}%
\pgfpathcurveto{\pgfqpoint{4.211382in}{1.198932in}}{\pgfqpoint{4.214216in}{1.205774in}}{\pgfqpoint{4.214216in}{1.212906in}}%
\pgfpathcurveto{\pgfqpoint{4.214216in}{1.220039in}}{\pgfqpoint{4.211382in}{1.226881in}}{\pgfqpoint{4.206339in}{1.231925in}}%
\pgfpathcurveto{\pgfqpoint{4.201295in}{1.236968in}}{\pgfqpoint{4.194453in}{1.239802in}}{\pgfqpoint{4.187321in}{1.239802in}}%
\pgfpathcurveto{\pgfqpoint{4.180188in}{1.239802in}}{\pgfqpoint{4.173346in}{1.236968in}}{\pgfqpoint{4.168302in}{1.231925in}}%
\pgfpathcurveto{\pgfqpoint{4.163259in}{1.226881in}}{\pgfqpoint{4.160425in}{1.220039in}}{\pgfqpoint{4.160425in}{1.212906in}}%
\pgfpathcurveto{\pgfqpoint{4.160425in}{1.205774in}}{\pgfqpoint{4.163259in}{1.198932in}}{\pgfqpoint{4.168302in}{1.193888in}}%
\pgfpathcurveto{\pgfqpoint{4.173346in}{1.188845in}}{\pgfqpoint{4.180188in}{1.186011in}}{\pgfqpoint{4.187321in}{1.186011in}}%
\pgfpathclose%
\pgfusepath{stroke,fill}%
\end{pgfscope}%
\begin{pgfscope}%
\pgfpathrectangle{\pgfqpoint{2.867647in}{0.500000in}}{\pgfqpoint{1.764706in}{1.700000in}}%
\pgfusepath{clip}%
\pgfsetbuttcap%
\pgfsetroundjoin%
\definecolor{currentfill}{rgb}{0.971202,0.827364,0.728520}%
\pgfsetfillcolor{currentfill}%
\pgfsetlinewidth{0.311001pt}%
\definecolor{currentstroke}{rgb}{1.000000,1.000000,1.000000}%
\pgfsetstrokecolor{currentstroke}%
\pgfsetdash{}{0pt}%
\pgfpathmoveto{\pgfqpoint{4.083866in}{1.149193in}}%
\pgfpathcurveto{\pgfqpoint{4.090999in}{1.149193in}}{\pgfqpoint{4.097841in}{1.152027in}}{\pgfqpoint{4.102884in}{1.157070in}}%
\pgfpathcurveto{\pgfqpoint{4.107928in}{1.162114in}}{\pgfqpoint{4.110762in}{1.168956in}}{\pgfqpoint{4.110762in}{1.176088in}}%
\pgfpathcurveto{\pgfqpoint{4.110762in}{1.183221in}}{\pgfqpoint{4.107928in}{1.190063in}}{\pgfqpoint{4.102884in}{1.195107in}}%
\pgfpathcurveto{\pgfqpoint{4.097841in}{1.200150in}}{\pgfqpoint{4.090999in}{1.202984in}}{\pgfqpoint{4.083866in}{1.202984in}}%
\pgfpathcurveto{\pgfqpoint{4.076733in}{1.202984in}}{\pgfqpoint{4.069892in}{1.200150in}}{\pgfqpoint{4.064848in}{1.195107in}}%
\pgfpathcurveto{\pgfqpoint{4.059804in}{1.190063in}}{\pgfqpoint{4.056971in}{1.183221in}}{\pgfqpoint{4.056971in}{1.176088in}}%
\pgfpathcurveto{\pgfqpoint{4.056971in}{1.168956in}}{\pgfqpoint{4.059804in}{1.162114in}}{\pgfqpoint{4.064848in}{1.157070in}}%
\pgfpathcurveto{\pgfqpoint{4.069892in}{1.152027in}}{\pgfqpoint{4.076733in}{1.149193in}}{\pgfqpoint{4.083866in}{1.149193in}}%
\pgfpathclose%
\pgfusepath{stroke,fill}%
\end{pgfscope}%
\begin{pgfscope}%
\pgfpathrectangle{\pgfqpoint{2.867647in}{0.500000in}}{\pgfqpoint{1.764706in}{1.700000in}}%
\pgfusepath{clip}%
\pgfsetbuttcap%
\pgfsetroundjoin%
\definecolor{currentfill}{rgb}{0.974412,0.862387,0.780156}%
\pgfsetfillcolor{currentfill}%
\pgfsetlinewidth{0.311001pt}%
\definecolor{currentstroke}{rgb}{1.000000,1.000000,1.000000}%
\pgfsetstrokecolor{currentstroke}%
\pgfsetdash{}{0pt}%
\pgfpathmoveto{\pgfqpoint{4.126988in}{1.266687in}}%
\pgfpathcurveto{\pgfqpoint{4.134120in}{1.266687in}}{\pgfqpoint{4.140962in}{1.269521in}}{\pgfqpoint{4.146006in}{1.274565in}}%
\pgfpathcurveto{\pgfqpoint{4.151049in}{1.279609in}}{\pgfqpoint{4.153883in}{1.286450in}}{\pgfqpoint{4.153883in}{1.293583in}}%
\pgfpathcurveto{\pgfqpoint{4.153883in}{1.300716in}}{\pgfqpoint{4.151049in}{1.307557in}}{\pgfqpoint{4.146006in}{1.312601in}}%
\pgfpathcurveto{\pgfqpoint{4.140962in}{1.317645in}}{\pgfqpoint{4.134120in}{1.320479in}}{\pgfqpoint{4.126988in}{1.320479in}}%
\pgfpathcurveto{\pgfqpoint{4.119855in}{1.320479in}}{\pgfqpoint{4.113013in}{1.317645in}}{\pgfqpoint{4.107969in}{1.312601in}}%
\pgfpathcurveto{\pgfqpoint{4.102926in}{1.307557in}}{\pgfqpoint{4.100092in}{1.300716in}}{\pgfqpoint{4.100092in}{1.293583in}}%
\pgfpathcurveto{\pgfqpoint{4.100092in}{1.286450in}}{\pgfqpoint{4.102926in}{1.279609in}}{\pgfqpoint{4.107969in}{1.274565in}}%
\pgfpathcurveto{\pgfqpoint{4.113013in}{1.269521in}}{\pgfqpoint{4.119855in}{1.266687in}}{\pgfqpoint{4.126988in}{1.266687in}}%
\pgfpathclose%
\pgfusepath{stroke,fill}%
\end{pgfscope}%
\begin{pgfscope}%
\pgfpathrectangle{\pgfqpoint{2.867647in}{0.500000in}}{\pgfqpoint{1.764706in}{1.700000in}}%
\pgfusepath{clip}%
\pgfsetbuttcap%
\pgfsetroundjoin%
\definecolor{currentfill}{rgb}{0.979124,0.903132,0.839793}%
\pgfsetfillcolor{currentfill}%
\pgfsetlinewidth{0.311001pt}%
\definecolor{currentstroke}{rgb}{1.000000,1.000000,1.000000}%
\pgfsetstrokecolor{currentstroke}%
\pgfsetdash{}{0pt}%
\pgfpathmoveto{\pgfqpoint{4.132302in}{1.080271in}}%
\pgfpathcurveto{\pgfqpoint{4.139435in}{1.080271in}}{\pgfqpoint{4.146277in}{1.083105in}}{\pgfqpoint{4.151320in}{1.088149in}}%
\pgfpathcurveto{\pgfqpoint{4.156364in}{1.093193in}}{\pgfqpoint{4.159198in}{1.100034in}}{\pgfqpoint{4.159198in}{1.107167in}}%
\pgfpathcurveto{\pgfqpoint{4.159198in}{1.114300in}}{\pgfqpoint{4.156364in}{1.121142in}}{\pgfqpoint{4.151320in}{1.126185in}}%
\pgfpathcurveto{\pgfqpoint{4.146277in}{1.131229in}}{\pgfqpoint{4.139435in}{1.134063in}}{\pgfqpoint{4.132302in}{1.134063in}}%
\pgfpathcurveto{\pgfqpoint{4.125169in}{1.134063in}}{\pgfqpoint{4.118328in}{1.131229in}}{\pgfqpoint{4.113284in}{1.126185in}}%
\pgfpathcurveto{\pgfqpoint{4.108240in}{1.121142in}}{\pgfqpoint{4.105406in}{1.114300in}}{\pgfqpoint{4.105406in}{1.107167in}}%
\pgfpathcurveto{\pgfqpoint{4.105406in}{1.100034in}}{\pgfqpoint{4.108240in}{1.093193in}}{\pgfqpoint{4.113284in}{1.088149in}}%
\pgfpathcurveto{\pgfqpoint{4.118328in}{1.083105in}}{\pgfqpoint{4.125169in}{1.080271in}}{\pgfqpoint{4.132302in}{1.080271in}}%
\pgfpathclose%
\pgfusepath{stroke,fill}%
\end{pgfscope}%
\begin{pgfscope}%
\pgfpathrectangle{\pgfqpoint{2.867647in}{0.500000in}}{\pgfqpoint{1.764706in}{1.700000in}}%
\pgfusepath{clip}%
\pgfsetbuttcap%
\pgfsetroundjoin%
\definecolor{currentfill}{rgb}{0.970255,0.815666,0.711203}%
\pgfsetfillcolor{currentfill}%
\pgfsetlinewidth{0.311001pt}%
\definecolor{currentstroke}{rgb}{1.000000,1.000000,1.000000}%
\pgfsetstrokecolor{currentstroke}%
\pgfsetdash{}{0pt}%
\pgfpathmoveto{\pgfqpoint{4.099048in}{1.410353in}}%
\pgfpathcurveto{\pgfqpoint{4.106181in}{1.410353in}}{\pgfqpoint{4.113023in}{1.413187in}}{\pgfqpoint{4.118066in}{1.418230in}}%
\pgfpathcurveto{\pgfqpoint{4.123110in}{1.423274in}}{\pgfqpoint{4.125944in}{1.430116in}}{\pgfqpoint{4.125944in}{1.437248in}}%
\pgfpathcurveto{\pgfqpoint{4.125944in}{1.444381in}}{\pgfqpoint{4.123110in}{1.451223in}}{\pgfqpoint{4.118066in}{1.456266in}}%
\pgfpathcurveto{\pgfqpoint{4.113023in}{1.461310in}}{\pgfqpoint{4.106181in}{1.464144in}}{\pgfqpoint{4.099048in}{1.464144in}}%
\pgfpathcurveto{\pgfqpoint{4.091915in}{1.464144in}}{\pgfqpoint{4.085074in}{1.461310in}}{\pgfqpoint{4.080030in}{1.456266in}}%
\pgfpathcurveto{\pgfqpoint{4.074986in}{1.451223in}}{\pgfqpoint{4.072152in}{1.444381in}}{\pgfqpoint{4.072152in}{1.437248in}}%
\pgfpathcurveto{\pgfqpoint{4.072152in}{1.430116in}}{\pgfqpoint{4.074986in}{1.423274in}}{\pgfqpoint{4.080030in}{1.418230in}}%
\pgfpathcurveto{\pgfqpoint{4.085074in}{1.413187in}}{\pgfqpoint{4.091915in}{1.410353in}}{\pgfqpoint{4.099048in}{1.410353in}}%
\pgfpathclose%
\pgfusepath{stroke,fill}%
\end{pgfscope}%
\begin{pgfscope}%
\pgfpathrectangle{\pgfqpoint{2.867647in}{0.500000in}}{\pgfqpoint{1.764706in}{1.700000in}}%
\pgfusepath{clip}%
\pgfsetbuttcap%
\pgfsetroundjoin%
\definecolor{currentfill}{rgb}{0.947270,0.405591,0.279023}%
\pgfsetfillcolor{currentfill}%
\pgfsetlinewidth{0.311001pt}%
\definecolor{currentstroke}{rgb}{1.000000,1.000000,1.000000}%
\pgfsetstrokecolor{currentstroke}%
\pgfsetdash{}{0pt}%
\pgfpathmoveto{\pgfqpoint{3.843153in}{1.695418in}}%
\pgfpathcurveto{\pgfqpoint{3.850286in}{1.695418in}}{\pgfqpoint{3.857128in}{1.698252in}}{\pgfqpoint{3.862171in}{1.703296in}}%
\pgfpathcurveto{\pgfqpoint{3.867215in}{1.708339in}}{\pgfqpoint{3.870049in}{1.715181in}}{\pgfqpoint{3.870049in}{1.722314in}}%
\pgfpathcurveto{\pgfqpoint{3.870049in}{1.729447in}}{\pgfqpoint{3.867215in}{1.736288in}}{\pgfqpoint{3.862171in}{1.741332in}}%
\pgfpathcurveto{\pgfqpoint{3.857128in}{1.746376in}}{\pgfqpoint{3.850286in}{1.749209in}}{\pgfqpoint{3.843153in}{1.749209in}}%
\pgfpathcurveto{\pgfqpoint{3.836020in}{1.749209in}}{\pgfqpoint{3.829179in}{1.746376in}}{\pgfqpoint{3.824135in}{1.741332in}}%
\pgfpathcurveto{\pgfqpoint{3.819091in}{1.736288in}}{\pgfqpoint{3.816257in}{1.729447in}}{\pgfqpoint{3.816257in}{1.722314in}}%
\pgfpathcurveto{\pgfqpoint{3.816257in}{1.715181in}}{\pgfqpoint{3.819091in}{1.708339in}}{\pgfqpoint{3.824135in}{1.703296in}}%
\pgfpathcurveto{\pgfqpoint{3.829179in}{1.698252in}}{\pgfqpoint{3.836020in}{1.695418in}}{\pgfqpoint{3.843153in}{1.695418in}}%
\pgfpathclose%
\pgfusepath{stroke,fill}%
\end{pgfscope}%
\begin{pgfscope}%
\pgfpathrectangle{\pgfqpoint{2.867647in}{0.500000in}}{\pgfqpoint{1.764706in}{1.700000in}}%
\pgfusepath{clip}%
\pgfsetbuttcap%
\pgfsetroundjoin%
\definecolor{currentfill}{rgb}{0.979891,0.908948,0.848279}%
\pgfsetfillcolor{currentfill}%
\pgfsetlinewidth{0.311001pt}%
\definecolor{currentstroke}{rgb}{1.000000,1.000000,1.000000}%
\pgfsetstrokecolor{currentstroke}%
\pgfsetdash{}{0pt}%
\pgfpathmoveto{\pgfqpoint{4.176193in}{1.124580in}}%
\pgfpathcurveto{\pgfqpoint{4.183326in}{1.124580in}}{\pgfqpoint{4.190167in}{1.127414in}}{\pgfqpoint{4.195211in}{1.132458in}}%
\pgfpathcurveto{\pgfqpoint{4.200255in}{1.137501in}}{\pgfqpoint{4.203088in}{1.144343in}}{\pgfqpoint{4.203088in}{1.151476in}}%
\pgfpathcurveto{\pgfqpoint{4.203088in}{1.158609in}}{\pgfqpoint{4.200255in}{1.165450in}}{\pgfqpoint{4.195211in}{1.170494in}}%
\pgfpathcurveto{\pgfqpoint{4.190167in}{1.175538in}}{\pgfqpoint{4.183326in}{1.178372in}}{\pgfqpoint{4.176193in}{1.178372in}}%
\pgfpathcurveto{\pgfqpoint{4.169060in}{1.178372in}}{\pgfqpoint{4.162218in}{1.175538in}}{\pgfqpoint{4.157175in}{1.170494in}}%
\pgfpathcurveto{\pgfqpoint{4.152131in}{1.165450in}}{\pgfqpoint{4.149297in}{1.158609in}}{\pgfqpoint{4.149297in}{1.151476in}}%
\pgfpathcurveto{\pgfqpoint{4.149297in}{1.144343in}}{\pgfqpoint{4.152131in}{1.137501in}}{\pgfqpoint{4.157175in}{1.132458in}}%
\pgfpathcurveto{\pgfqpoint{4.162218in}{1.127414in}}{\pgfqpoint{4.169060in}{1.124580in}}{\pgfqpoint{4.176193in}{1.124580in}}%
\pgfpathclose%
\pgfusepath{stroke,fill}%
\end{pgfscope}%
\begin{pgfscope}%
\pgfpathrectangle{\pgfqpoint{2.867647in}{0.500000in}}{\pgfqpoint{1.764706in}{1.700000in}}%
\pgfusepath{clip}%
\pgfsetbuttcap%
\pgfsetroundjoin%
\definecolor{currentfill}{rgb}{0.962985,0.612625,0.451451}%
\pgfsetfillcolor{currentfill}%
\pgfsetlinewidth{0.311001pt}%
\definecolor{currentstroke}{rgb}{1.000000,1.000000,1.000000}%
\pgfsetstrokecolor{currentstroke}%
\pgfsetdash{}{0pt}%
\pgfpathmoveto{\pgfqpoint{3.982214in}{0.873152in}}%
\pgfpathcurveto{\pgfqpoint{3.989347in}{0.873152in}}{\pgfqpoint{3.996189in}{0.875986in}}{\pgfqpoint{4.001232in}{0.881030in}}%
\pgfpathcurveto{\pgfqpoint{4.006276in}{0.886073in}}{\pgfqpoint{4.009110in}{0.892915in}}{\pgfqpoint{4.009110in}{0.900048in}}%
\pgfpathcurveto{\pgfqpoint{4.009110in}{0.907181in}}{\pgfqpoint{4.006276in}{0.914022in}}{\pgfqpoint{4.001232in}{0.919066in}}%
\pgfpathcurveto{\pgfqpoint{3.996189in}{0.924110in}}{\pgfqpoint{3.989347in}{0.926944in}}{\pgfqpoint{3.982214in}{0.926944in}}%
\pgfpathcurveto{\pgfqpoint{3.975081in}{0.926944in}}{\pgfqpoint{3.968240in}{0.924110in}}{\pgfqpoint{3.963196in}{0.919066in}}%
\pgfpathcurveto{\pgfqpoint{3.958152in}{0.914022in}}{\pgfqpoint{3.955318in}{0.907181in}}{\pgfqpoint{3.955318in}{0.900048in}}%
\pgfpathcurveto{\pgfqpoint{3.955318in}{0.892915in}}{\pgfqpoint{3.958152in}{0.886073in}}{\pgfqpoint{3.963196in}{0.881030in}}%
\pgfpathcurveto{\pgfqpoint{3.968240in}{0.875986in}}{\pgfqpoint{3.975081in}{0.873152in}}{\pgfqpoint{3.982214in}{0.873152in}}%
\pgfpathclose%
\pgfusepath{stroke,fill}%
\end{pgfscope}%
\begin{pgfscope}%
\pgfpathrectangle{\pgfqpoint{2.867647in}{0.500000in}}{\pgfqpoint{1.764706in}{1.700000in}}%
\pgfusepath{clip}%
\pgfsetbuttcap%
\pgfsetroundjoin%
\definecolor{currentfill}{rgb}{0.970718,0.821518,0.719872}%
\pgfsetfillcolor{currentfill}%
\pgfsetlinewidth{0.311001pt}%
\definecolor{currentstroke}{rgb}{1.000000,1.000000,1.000000}%
\pgfsetstrokecolor{currentstroke}%
\pgfsetdash{}{0pt}%
\pgfpathmoveto{\pgfqpoint{4.087355in}{1.461236in}}%
\pgfpathcurveto{\pgfqpoint{4.094488in}{1.461236in}}{\pgfqpoint{4.101329in}{1.464070in}}{\pgfqpoint{4.106373in}{1.469113in}}%
\pgfpathcurveto{\pgfqpoint{4.111417in}{1.474157in}}{\pgfqpoint{4.114251in}{1.480999in}}{\pgfqpoint{4.114251in}{1.488132in}}%
\pgfpathcurveto{\pgfqpoint{4.114251in}{1.495264in}}{\pgfqpoint{4.111417in}{1.502106in}}{\pgfqpoint{4.106373in}{1.507150in}}%
\pgfpathcurveto{\pgfqpoint{4.101329in}{1.512193in}}{\pgfqpoint{4.094488in}{1.515027in}}{\pgfqpoint{4.087355in}{1.515027in}}%
\pgfpathcurveto{\pgfqpoint{4.080222in}{1.515027in}}{\pgfqpoint{4.073380in}{1.512193in}}{\pgfqpoint{4.068337in}{1.507150in}}%
\pgfpathcurveto{\pgfqpoint{4.063293in}{1.502106in}}{\pgfqpoint{4.060459in}{1.495264in}}{\pgfqpoint{4.060459in}{1.488132in}}%
\pgfpathcurveto{\pgfqpoint{4.060459in}{1.480999in}}{\pgfqpoint{4.063293in}{1.474157in}}{\pgfqpoint{4.068337in}{1.469113in}}%
\pgfpathcurveto{\pgfqpoint{4.073380in}{1.464070in}}{\pgfqpoint{4.080222in}{1.461236in}}{\pgfqpoint{4.087355in}{1.461236in}}%
\pgfpathclose%
\pgfusepath{stroke,fill}%
\end{pgfscope}%
\begin{pgfscope}%
\pgfpathrectangle{\pgfqpoint{2.867647in}{0.500000in}}{\pgfqpoint{1.764706in}{1.700000in}}%
\pgfusepath{clip}%
\pgfsetbuttcap%
\pgfsetroundjoin%
\definecolor{currentfill}{rgb}{0.973271,0.850724,0.762998}%
\pgfsetfillcolor{currentfill}%
\pgfsetlinewidth{0.311001pt}%
\definecolor{currentstroke}{rgb}{1.000000,1.000000,1.000000}%
\pgfsetstrokecolor{currentstroke}%
\pgfsetdash{}{0pt}%
\pgfpathmoveto{\pgfqpoint{4.165437in}{1.020391in}}%
\pgfpathcurveto{\pgfqpoint{4.172570in}{1.020391in}}{\pgfqpoint{4.179411in}{1.023224in}}{\pgfqpoint{4.184455in}{1.028268in}}%
\pgfpathcurveto{\pgfqpoint{4.189499in}{1.033312in}}{\pgfqpoint{4.192333in}{1.040153in}}{\pgfqpoint{4.192333in}{1.047286in}}%
\pgfpathcurveto{\pgfqpoint{4.192333in}{1.054419in}}{\pgfqpoint{4.189499in}{1.061261in}}{\pgfqpoint{4.184455in}{1.066304in}}%
\pgfpathcurveto{\pgfqpoint{4.179411in}{1.071348in}}{\pgfqpoint{4.172570in}{1.074182in}}{\pgfqpoint{4.165437in}{1.074182in}}%
\pgfpathcurveto{\pgfqpoint{4.158304in}{1.074182in}}{\pgfqpoint{4.151462in}{1.071348in}}{\pgfqpoint{4.146419in}{1.066304in}}%
\pgfpathcurveto{\pgfqpoint{4.141375in}{1.061261in}}{\pgfqpoint{4.138541in}{1.054419in}}{\pgfqpoint{4.138541in}{1.047286in}}%
\pgfpathcurveto{\pgfqpoint{4.138541in}{1.040153in}}{\pgfqpoint{4.141375in}{1.033312in}}{\pgfqpoint{4.146419in}{1.028268in}}%
\pgfpathcurveto{\pgfqpoint{4.151462in}{1.023224in}}{\pgfqpoint{4.158304in}{1.020391in}}{\pgfqpoint{4.165437in}{1.020391in}}%
\pgfpathclose%
\pgfusepath{stroke,fill}%
\end{pgfscope}%
\begin{pgfscope}%
\pgfpathrectangle{\pgfqpoint{2.867647in}{0.500000in}}{\pgfqpoint{1.764706in}{1.700000in}}%
\pgfusepath{clip}%
\pgfsetbuttcap%
\pgfsetroundjoin%
\definecolor{currentfill}{rgb}{0.981377,0.920617,0.865369}%
\pgfsetfillcolor{currentfill}%
\pgfsetlinewidth{0.311001pt}%
\definecolor{currentstroke}{rgb}{1.000000,1.000000,1.000000}%
\pgfsetstrokecolor{currentstroke}%
\pgfsetdash{}{0pt}%
\pgfpathmoveto{\pgfqpoint{4.181803in}{1.257197in}}%
\pgfpathcurveto{\pgfqpoint{4.188936in}{1.257197in}}{\pgfqpoint{4.195778in}{1.260031in}}{\pgfqpoint{4.200821in}{1.265075in}}%
\pgfpathcurveto{\pgfqpoint{4.205865in}{1.270119in}}{\pgfqpoint{4.208699in}{1.276960in}}{\pgfqpoint{4.208699in}{1.284093in}}%
\pgfpathcurveto{\pgfqpoint{4.208699in}{1.291226in}}{\pgfqpoint{4.205865in}{1.298067in}}{\pgfqpoint{4.200821in}{1.303111in}}%
\pgfpathcurveto{\pgfqpoint{4.195778in}{1.308155in}}{\pgfqpoint{4.188936in}{1.310989in}}{\pgfqpoint{4.181803in}{1.310989in}}%
\pgfpathcurveto{\pgfqpoint{4.174670in}{1.310989in}}{\pgfqpoint{4.167829in}{1.308155in}}{\pgfqpoint{4.162785in}{1.303111in}}%
\pgfpathcurveto{\pgfqpoint{4.157741in}{1.298067in}}{\pgfqpoint{4.154908in}{1.291226in}}{\pgfqpoint{4.154908in}{1.284093in}}%
\pgfpathcurveto{\pgfqpoint{4.154908in}{1.276960in}}{\pgfqpoint{4.157741in}{1.270119in}}{\pgfqpoint{4.162785in}{1.265075in}}%
\pgfpathcurveto{\pgfqpoint{4.167829in}{1.260031in}}{\pgfqpoint{4.174670in}{1.257197in}}{\pgfqpoint{4.181803in}{1.257197in}}%
\pgfpathclose%
\pgfusepath{stroke,fill}%
\end{pgfscope}%
\begin{pgfscope}%
\pgfpathrectangle{\pgfqpoint{2.867647in}{0.500000in}}{\pgfqpoint{1.764706in}{1.700000in}}%
\pgfusepath{clip}%
\pgfsetbuttcap%
\pgfsetroundjoin%
\definecolor{currentfill}{rgb}{0.977657,0.891500,0.822809}%
\pgfsetfillcolor{currentfill}%
\pgfsetlinewidth{0.311001pt}%
\definecolor{currentstroke}{rgb}{1.000000,1.000000,1.000000}%
\pgfsetstrokecolor{currentstroke}%
\pgfsetdash{}{0pt}%
\pgfpathmoveto{\pgfqpoint{4.226191in}{1.406555in}}%
\pgfpathcurveto{\pgfqpoint{4.233324in}{1.406555in}}{\pgfqpoint{4.240166in}{1.409389in}}{\pgfqpoint{4.245209in}{1.414433in}}%
\pgfpathcurveto{\pgfqpoint{4.250253in}{1.419477in}}{\pgfqpoint{4.253087in}{1.426318in}}{\pgfqpoint{4.253087in}{1.433451in}}%
\pgfpathcurveto{\pgfqpoint{4.253087in}{1.440584in}}{\pgfqpoint{4.250253in}{1.447425in}}{\pgfqpoint{4.245209in}{1.452469in}}%
\pgfpathcurveto{\pgfqpoint{4.240166in}{1.457513in}}{\pgfqpoint{4.233324in}{1.460347in}}{\pgfqpoint{4.226191in}{1.460347in}}%
\pgfpathcurveto{\pgfqpoint{4.219059in}{1.460347in}}{\pgfqpoint{4.212217in}{1.457513in}}{\pgfqpoint{4.207173in}{1.452469in}}%
\pgfpathcurveto{\pgfqpoint{4.202130in}{1.447425in}}{\pgfqpoint{4.199296in}{1.440584in}}{\pgfqpoint{4.199296in}{1.433451in}}%
\pgfpathcurveto{\pgfqpoint{4.199296in}{1.426318in}}{\pgfqpoint{4.202130in}{1.419477in}}{\pgfqpoint{4.207173in}{1.414433in}}%
\pgfpathcurveto{\pgfqpoint{4.212217in}{1.409389in}}{\pgfqpoint{4.219059in}{1.406555in}}{\pgfqpoint{4.226191in}{1.406555in}}%
\pgfpathclose%
\pgfusepath{stroke,fill}%
\end{pgfscope}%
\begin{pgfscope}%
\pgfpathrectangle{\pgfqpoint{2.867647in}{0.500000in}}{\pgfqpoint{1.764706in}{1.700000in}}%
\pgfusepath{clip}%
\pgfsetbuttcap%
\pgfsetroundjoin%
\definecolor{currentfill}{rgb}{0.951650,0.442241,0.302145}%
\pgfsetfillcolor{currentfill}%
\pgfsetlinewidth{0.311001pt}%
\definecolor{currentstroke}{rgb}{1.000000,1.000000,1.000000}%
\pgfsetstrokecolor{currentstroke}%
\pgfsetdash{}{0pt}%
\pgfpathmoveto{\pgfqpoint{3.850337in}{1.739075in}}%
\pgfpathcurveto{\pgfqpoint{3.857470in}{1.739075in}}{\pgfqpoint{3.864312in}{1.741908in}}{\pgfqpoint{3.869355in}{1.746952in}}%
\pgfpathcurveto{\pgfqpoint{3.874399in}{1.751996in}}{\pgfqpoint{3.877233in}{1.758837in}}{\pgfqpoint{3.877233in}{1.765970in}}%
\pgfpathcurveto{\pgfqpoint{3.877233in}{1.773103in}}{\pgfqpoint{3.874399in}{1.779945in}}{\pgfqpoint{3.869355in}{1.784988in}}%
\pgfpathcurveto{\pgfqpoint{3.864312in}{1.790032in}}{\pgfqpoint{3.857470in}{1.792866in}}{\pgfqpoint{3.850337in}{1.792866in}}%
\pgfpathcurveto{\pgfqpoint{3.843204in}{1.792866in}}{\pgfqpoint{3.836363in}{1.790032in}}{\pgfqpoint{3.831319in}{1.784988in}}%
\pgfpathcurveto{\pgfqpoint{3.826275in}{1.779945in}}{\pgfqpoint{3.823442in}{1.773103in}}{\pgfqpoint{3.823442in}{1.765970in}}%
\pgfpathcurveto{\pgfqpoint{3.823442in}{1.758837in}}{\pgfqpoint{3.826275in}{1.751996in}}{\pgfqpoint{3.831319in}{1.746952in}}%
\pgfpathcurveto{\pgfqpoint{3.836363in}{1.741908in}}{\pgfqpoint{3.843204in}{1.739075in}}{\pgfqpoint{3.850337in}{1.739075in}}%
\pgfpathclose%
\pgfusepath{stroke,fill}%
\end{pgfscope}%
\begin{pgfscope}%
\pgfpathrectangle{\pgfqpoint{2.867647in}{0.500000in}}{\pgfqpoint{1.764706in}{1.700000in}}%
\pgfusepath{clip}%
\pgfsetbuttcap%
\pgfsetroundjoin%
\definecolor{currentfill}{rgb}{0.966560,0.756582,0.625273}%
\pgfsetfillcolor{currentfill}%
\pgfsetlinewidth{0.311001pt}%
\definecolor{currentstroke}{rgb}{1.000000,1.000000,1.000000}%
\pgfsetstrokecolor{currentstroke}%
\pgfsetdash{}{0pt}%
\pgfpathmoveto{\pgfqpoint{4.020508in}{1.600919in}}%
\pgfpathcurveto{\pgfqpoint{4.027641in}{1.600919in}}{\pgfqpoint{4.034483in}{1.603752in}}{\pgfqpoint{4.039526in}{1.608796in}}%
\pgfpathcurveto{\pgfqpoint{4.044570in}{1.613840in}}{\pgfqpoint{4.047404in}{1.620681in}}{\pgfqpoint{4.047404in}{1.627814in}}%
\pgfpathcurveto{\pgfqpoint{4.047404in}{1.634947in}}{\pgfqpoint{4.044570in}{1.641789in}}{\pgfqpoint{4.039526in}{1.646832in}}%
\pgfpathcurveto{\pgfqpoint{4.034483in}{1.651876in}}{\pgfqpoint{4.027641in}{1.654710in}}{\pgfqpoint{4.020508in}{1.654710in}}%
\pgfpathcurveto{\pgfqpoint{4.013375in}{1.654710in}}{\pgfqpoint{4.006534in}{1.651876in}}{\pgfqpoint{4.001490in}{1.646832in}}%
\pgfpathcurveto{\pgfqpoint{3.996446in}{1.641789in}}{\pgfqpoint{3.993613in}{1.634947in}}{\pgfqpoint{3.993613in}{1.627814in}}%
\pgfpathcurveto{\pgfqpoint{3.993613in}{1.620681in}}{\pgfqpoint{3.996446in}{1.613840in}}{\pgfqpoint{4.001490in}{1.608796in}}%
\pgfpathcurveto{\pgfqpoint{4.006534in}{1.603752in}}{\pgfqpoint{4.013375in}{1.600919in}}{\pgfqpoint{4.020508in}{1.600919in}}%
\pgfpathclose%
\pgfusepath{stroke,fill}%
\end{pgfscope}%
\begin{pgfscope}%
\pgfpathrectangle{\pgfqpoint{2.867647in}{0.500000in}}{\pgfqpoint{1.764706in}{1.700000in}}%
\pgfusepath{clip}%
\pgfsetbuttcap%
\pgfsetroundjoin%
\definecolor{currentfill}{rgb}{0.970718,0.821518,0.719872}%
\pgfsetfillcolor{currentfill}%
\pgfsetlinewidth{0.311001pt}%
\definecolor{currentstroke}{rgb}{1.000000,1.000000,1.000000}%
\pgfsetstrokecolor{currentstroke}%
\pgfsetdash{}{0pt}%
\pgfpathmoveto{\pgfqpoint{4.109150in}{1.276517in}}%
\pgfpathcurveto{\pgfqpoint{4.116283in}{1.276517in}}{\pgfqpoint{4.123125in}{1.279350in}}{\pgfqpoint{4.128168in}{1.284394in}}%
\pgfpathcurveto{\pgfqpoint{4.133212in}{1.289438in}}{\pgfqpoint{4.136046in}{1.296279in}}{\pgfqpoint{4.136046in}{1.303412in}}%
\pgfpathcurveto{\pgfqpoint{4.136046in}{1.310545in}}{\pgfqpoint{4.133212in}{1.317387in}}{\pgfqpoint{4.128168in}{1.322430in}}%
\pgfpathcurveto{\pgfqpoint{4.123125in}{1.327474in}}{\pgfqpoint{4.116283in}{1.330308in}}{\pgfqpoint{4.109150in}{1.330308in}}%
\pgfpathcurveto{\pgfqpoint{4.102017in}{1.330308in}}{\pgfqpoint{4.095176in}{1.327474in}}{\pgfqpoint{4.090132in}{1.322430in}}%
\pgfpathcurveto{\pgfqpoint{4.085088in}{1.317387in}}{\pgfqpoint{4.082254in}{1.310545in}}{\pgfqpoint{4.082254in}{1.303412in}}%
\pgfpathcurveto{\pgfqpoint{4.082254in}{1.296279in}}{\pgfqpoint{4.085088in}{1.289438in}}{\pgfqpoint{4.090132in}{1.284394in}}%
\pgfpathcurveto{\pgfqpoint{4.095176in}{1.279350in}}{\pgfqpoint{4.102017in}{1.276517in}}{\pgfqpoint{4.109150in}{1.276517in}}%
\pgfpathclose%
\pgfusepath{stroke,fill}%
\end{pgfscope}%
\begin{pgfscope}%
\pgfpathrectangle{\pgfqpoint{2.867647in}{0.500000in}}{\pgfqpoint{1.764706in}{1.700000in}}%
\pgfusepath{clip}%
\pgfsetbuttcap%
\pgfsetroundjoin%
\definecolor{currentfill}{rgb}{0.965753,0.732351,0.592427}%
\pgfsetfillcolor{currentfill}%
\pgfsetlinewidth{0.311001pt}%
\definecolor{currentstroke}{rgb}{1.000000,1.000000,1.000000}%
\pgfsetstrokecolor{currentstroke}%
\pgfsetdash{}{0pt}%
\pgfpathmoveto{\pgfqpoint{4.291599in}{1.426035in}}%
\pgfpathcurveto{\pgfqpoint{4.298732in}{1.426035in}}{\pgfqpoint{4.305574in}{1.428869in}}{\pgfqpoint{4.310617in}{1.433913in}}%
\pgfpathcurveto{\pgfqpoint{4.315661in}{1.438957in}}{\pgfqpoint{4.318495in}{1.445798in}}{\pgfqpoint{4.318495in}{1.452931in}}%
\pgfpathcurveto{\pgfqpoint{4.318495in}{1.460064in}}{\pgfqpoint{4.315661in}{1.466905in}}{\pgfqpoint{4.310617in}{1.471949in}}%
\pgfpathcurveto{\pgfqpoint{4.305574in}{1.476993in}}{\pgfqpoint{4.298732in}{1.479827in}}{\pgfqpoint{4.291599in}{1.479827in}}%
\pgfpathcurveto{\pgfqpoint{4.284466in}{1.479827in}}{\pgfqpoint{4.277625in}{1.476993in}}{\pgfqpoint{4.272581in}{1.471949in}}%
\pgfpathcurveto{\pgfqpoint{4.267537in}{1.466905in}}{\pgfqpoint{4.264704in}{1.460064in}}{\pgfqpoint{4.264704in}{1.452931in}}%
\pgfpathcurveto{\pgfqpoint{4.264704in}{1.445798in}}{\pgfqpoint{4.267537in}{1.438957in}}{\pgfqpoint{4.272581in}{1.433913in}}%
\pgfpathcurveto{\pgfqpoint{4.277625in}{1.428869in}}{\pgfqpoint{4.284466in}{1.426035in}}{\pgfqpoint{4.291599in}{1.426035in}}%
\pgfpathclose%
\pgfusepath{stroke,fill}%
\end{pgfscope}%
\begin{pgfscope}%
\pgfpathrectangle{\pgfqpoint{2.867647in}{0.500000in}}{\pgfqpoint{1.764706in}{1.700000in}}%
\pgfusepath{clip}%
\pgfsetbuttcap%
\pgfsetroundjoin%
\definecolor{currentfill}{rgb}{0.965592,0.726236,0.584384}%
\pgfsetfillcolor{currentfill}%
\pgfsetlinewidth{0.311001pt}%
\definecolor{currentstroke}{rgb}{1.000000,1.000000,1.000000}%
\pgfsetstrokecolor{currentstroke}%
\pgfsetdash{}{0pt}%
\pgfpathmoveto{\pgfqpoint{4.179106in}{0.957048in}}%
\pgfpathcurveto{\pgfqpoint{4.186238in}{0.957048in}}{\pgfqpoint{4.193080in}{0.959882in}}{\pgfqpoint{4.198124in}{0.964926in}}%
\pgfpathcurveto{\pgfqpoint{4.203167in}{0.969969in}}{\pgfqpoint{4.206001in}{0.976811in}}{\pgfqpoint{4.206001in}{0.983944in}}%
\pgfpathcurveto{\pgfqpoint{4.206001in}{0.991077in}}{\pgfqpoint{4.203167in}{0.997918in}}{\pgfqpoint{4.198124in}{1.002962in}}%
\pgfpathcurveto{\pgfqpoint{4.193080in}{1.008006in}}{\pgfqpoint{4.186238in}{1.010840in}}{\pgfqpoint{4.179106in}{1.010840in}}%
\pgfpathcurveto{\pgfqpoint{4.171973in}{1.010840in}}{\pgfqpoint{4.165131in}{1.008006in}}{\pgfqpoint{4.160087in}{1.002962in}}%
\pgfpathcurveto{\pgfqpoint{4.155044in}{0.997918in}}{\pgfqpoint{4.152210in}{0.991077in}}{\pgfqpoint{4.152210in}{0.983944in}}%
\pgfpathcurveto{\pgfqpoint{4.152210in}{0.976811in}}{\pgfqpoint{4.155044in}{0.969969in}}{\pgfqpoint{4.160087in}{0.964926in}}%
\pgfpathcurveto{\pgfqpoint{4.165131in}{0.959882in}}{\pgfqpoint{4.171973in}{0.957048in}}{\pgfqpoint{4.179106in}{0.957048in}}%
\pgfpathclose%
\pgfusepath{stroke,fill}%
\end{pgfscope}%
\begin{pgfscope}%
\pgfpathrectangle{\pgfqpoint{2.867647in}{0.500000in}}{\pgfqpoint{1.764706in}{1.700000in}}%
\pgfusepath{clip}%
\pgfsetbuttcap%
\pgfsetroundjoin%
\definecolor{currentfill}{rgb}{0.975644,0.874038,0.797253}%
\pgfsetfillcolor{currentfill}%
\pgfsetlinewidth{0.311001pt}%
\definecolor{currentstroke}{rgb}{1.000000,1.000000,1.000000}%
\pgfsetstrokecolor{currentstroke}%
\pgfsetdash{}{0pt}%
\pgfpathmoveto{\pgfqpoint{4.237106in}{1.418487in}}%
\pgfpathcurveto{\pgfqpoint{4.244238in}{1.418487in}}{\pgfqpoint{4.251080in}{1.421321in}}{\pgfqpoint{4.256124in}{1.426365in}}%
\pgfpathcurveto{\pgfqpoint{4.261167in}{1.431408in}}{\pgfqpoint{4.264001in}{1.438250in}}{\pgfqpoint{4.264001in}{1.445383in}}%
\pgfpathcurveto{\pgfqpoint{4.264001in}{1.452516in}}{\pgfqpoint{4.261167in}{1.459357in}}{\pgfqpoint{4.256124in}{1.464401in}}%
\pgfpathcurveto{\pgfqpoint{4.251080in}{1.469445in}}{\pgfqpoint{4.244238in}{1.472279in}}{\pgfqpoint{4.237106in}{1.472279in}}%
\pgfpathcurveto{\pgfqpoint{4.229973in}{1.472279in}}{\pgfqpoint{4.223131in}{1.469445in}}{\pgfqpoint{4.218087in}{1.464401in}}%
\pgfpathcurveto{\pgfqpoint{4.213044in}{1.459357in}}{\pgfqpoint{4.210210in}{1.452516in}}{\pgfqpoint{4.210210in}{1.445383in}}%
\pgfpathcurveto{\pgfqpoint{4.210210in}{1.438250in}}{\pgfqpoint{4.213044in}{1.431408in}}{\pgfqpoint{4.218087in}{1.426365in}}%
\pgfpathcurveto{\pgfqpoint{4.223131in}{1.421321in}}{\pgfqpoint{4.229973in}{1.418487in}}{\pgfqpoint{4.237106in}{1.418487in}}%
\pgfpathclose%
\pgfusepath{stroke,fill}%
\end{pgfscope}%
\begin{pgfscope}%
\pgfpathrectangle{\pgfqpoint{2.867647in}{0.500000in}}{\pgfqpoint{1.764706in}{1.700000in}}%
\pgfusepath{clip}%
\pgfsetbuttcap%
\pgfsetroundjoin%
\definecolor{currentfill}{rgb}{0.964032,0.651225,0.493258}%
\pgfsetfillcolor{currentfill}%
\pgfsetlinewidth{0.311001pt}%
\definecolor{currentstroke}{rgb}{1.000000,1.000000,1.000000}%
\pgfsetstrokecolor{currentstroke}%
\pgfsetdash{}{0pt}%
\pgfpathmoveto{\pgfqpoint{4.321988in}{1.307489in}}%
\pgfpathcurveto{\pgfqpoint{4.329121in}{1.307489in}}{\pgfqpoint{4.335963in}{1.310323in}}{\pgfqpoint{4.341006in}{1.315367in}}%
\pgfpathcurveto{\pgfqpoint{4.346050in}{1.320410in}}{\pgfqpoint{4.348884in}{1.327252in}}{\pgfqpoint{4.348884in}{1.334385in}}%
\pgfpathcurveto{\pgfqpoint{4.348884in}{1.341518in}}{\pgfqpoint{4.346050in}{1.348359in}}{\pgfqpoint{4.341006in}{1.353403in}}%
\pgfpathcurveto{\pgfqpoint{4.335963in}{1.358447in}}{\pgfqpoint{4.329121in}{1.361281in}}{\pgfqpoint{4.321988in}{1.361281in}}%
\pgfpathcurveto{\pgfqpoint{4.314855in}{1.361281in}}{\pgfqpoint{4.308014in}{1.358447in}}{\pgfqpoint{4.302970in}{1.353403in}}%
\pgfpathcurveto{\pgfqpoint{4.297926in}{1.348359in}}{\pgfqpoint{4.295092in}{1.341518in}}{\pgfqpoint{4.295092in}{1.334385in}}%
\pgfpathcurveto{\pgfqpoint{4.295092in}{1.327252in}}{\pgfqpoint{4.297926in}{1.320410in}}{\pgfqpoint{4.302970in}{1.315367in}}%
\pgfpathcurveto{\pgfqpoint{4.308014in}{1.310323in}}{\pgfqpoint{4.314855in}{1.307489in}}{\pgfqpoint{4.321988in}{1.307489in}}%
\pgfpathclose%
\pgfusepath{stroke,fill}%
\end{pgfscope}%
\begin{pgfscope}%
\pgfpathrectangle{\pgfqpoint{2.867647in}{0.500000in}}{\pgfqpoint{1.764706in}{1.700000in}}%
\pgfusepath{clip}%
\pgfsetbuttcap%
\pgfsetroundjoin%
\definecolor{currentfill}{rgb}{0.979891,0.908948,0.848279}%
\pgfsetfillcolor{currentfill}%
\pgfsetlinewidth{0.311001pt}%
\definecolor{currentstroke}{rgb}{1.000000,1.000000,1.000000}%
\pgfsetstrokecolor{currentstroke}%
\pgfsetdash{}{0pt}%
\pgfpathmoveto{\pgfqpoint{4.146353in}{1.457233in}}%
\pgfpathcurveto{\pgfqpoint{4.153486in}{1.457233in}}{\pgfqpoint{4.160327in}{1.460067in}}{\pgfqpoint{4.165371in}{1.465110in}}%
\pgfpathcurveto{\pgfqpoint{4.170415in}{1.470154in}}{\pgfqpoint{4.173249in}{1.476996in}}{\pgfqpoint{4.173249in}{1.484129in}}%
\pgfpathcurveto{\pgfqpoint{4.173249in}{1.491261in}}{\pgfqpoint{4.170415in}{1.498103in}}{\pgfqpoint{4.165371in}{1.503147in}}%
\pgfpathcurveto{\pgfqpoint{4.160327in}{1.508190in}}{\pgfqpoint{4.153486in}{1.511024in}}{\pgfqpoint{4.146353in}{1.511024in}}%
\pgfpathcurveto{\pgfqpoint{4.139220in}{1.511024in}}{\pgfqpoint{4.132379in}{1.508190in}}{\pgfqpoint{4.127335in}{1.503147in}}%
\pgfpathcurveto{\pgfqpoint{4.122291in}{1.498103in}}{\pgfqpoint{4.119457in}{1.491261in}}{\pgfqpoint{4.119457in}{1.484129in}}%
\pgfpathcurveto{\pgfqpoint{4.119457in}{1.476996in}}{\pgfqpoint{4.122291in}{1.470154in}}{\pgfqpoint{4.127335in}{1.465110in}}%
\pgfpathcurveto{\pgfqpoint{4.132379in}{1.460067in}}{\pgfqpoint{4.139220in}{1.457233in}}{\pgfqpoint{4.146353in}{1.457233in}}%
\pgfpathclose%
\pgfusepath{stroke,fill}%
\end{pgfscope}%
\begin{pgfscope}%
\pgfpathrectangle{\pgfqpoint{2.867647in}{0.500000in}}{\pgfqpoint{1.764706in}{1.700000in}}%
\pgfusepath{clip}%
\pgfsetbuttcap%
\pgfsetroundjoin%
\definecolor{currentfill}{rgb}{0.975018,0.868213,0.788710}%
\pgfsetfillcolor{currentfill}%
\pgfsetlinewidth{0.311001pt}%
\definecolor{currentstroke}{rgb}{1.000000,1.000000,1.000000}%
\pgfsetstrokecolor{currentstroke}%
\pgfsetdash{}{0pt}%
\pgfpathmoveto{\pgfqpoint{4.117489in}{1.647461in}}%
\pgfpathcurveto{\pgfqpoint{4.124622in}{1.647461in}}{\pgfqpoint{4.131463in}{1.650295in}}{\pgfqpoint{4.136507in}{1.655339in}}%
\pgfpathcurveto{\pgfqpoint{4.141551in}{1.660382in}}{\pgfqpoint{4.144384in}{1.667224in}}{\pgfqpoint{4.144384in}{1.674357in}}%
\pgfpathcurveto{\pgfqpoint{4.144384in}{1.681490in}}{\pgfqpoint{4.141551in}{1.688331in}}{\pgfqpoint{4.136507in}{1.693375in}}%
\pgfpathcurveto{\pgfqpoint{4.131463in}{1.698419in}}{\pgfqpoint{4.124622in}{1.701253in}}{\pgfqpoint{4.117489in}{1.701253in}}%
\pgfpathcurveto{\pgfqpoint{4.110356in}{1.701253in}}{\pgfqpoint{4.103514in}{1.698419in}}{\pgfqpoint{4.098471in}{1.693375in}}%
\pgfpathcurveto{\pgfqpoint{4.093427in}{1.688331in}}{\pgfqpoint{4.090593in}{1.681490in}}{\pgfqpoint{4.090593in}{1.674357in}}%
\pgfpathcurveto{\pgfqpoint{4.090593in}{1.667224in}}{\pgfqpoint{4.093427in}{1.660382in}}{\pgfqpoint{4.098471in}{1.655339in}}%
\pgfpathcurveto{\pgfqpoint{4.103514in}{1.650295in}}{\pgfqpoint{4.110356in}{1.647461in}}{\pgfqpoint{4.117489in}{1.647461in}}%
\pgfpathclose%
\pgfusepath{stroke,fill}%
\end{pgfscope}%
\begin{pgfscope}%
\pgfpathrectangle{\pgfqpoint{2.867647in}{0.500000in}}{\pgfqpoint{1.764706in}{1.700000in}}%
\pgfusepath{clip}%
\pgfsetbuttcap%
\pgfsetroundjoin%
\definecolor{currentfill}{rgb}{0.970718,0.821518,0.719872}%
\pgfsetfillcolor{currentfill}%
\pgfsetlinewidth{0.311001pt}%
\definecolor{currentstroke}{rgb}{1.000000,1.000000,1.000000}%
\pgfsetstrokecolor{currentstroke}%
\pgfsetdash{}{0pt}%
\pgfpathmoveto{\pgfqpoint{4.257085in}{1.172210in}}%
\pgfpathcurveto{\pgfqpoint{4.264218in}{1.172210in}}{\pgfqpoint{4.271059in}{1.175044in}}{\pgfqpoint{4.276103in}{1.180088in}}%
\pgfpathcurveto{\pgfqpoint{4.281147in}{1.185131in}}{\pgfqpoint{4.283980in}{1.191973in}}{\pgfqpoint{4.283980in}{1.199106in}}%
\pgfpathcurveto{\pgfqpoint{4.283980in}{1.206239in}}{\pgfqpoint{4.281147in}{1.213080in}}{\pgfqpoint{4.276103in}{1.218124in}}%
\pgfpathcurveto{\pgfqpoint{4.271059in}{1.223168in}}{\pgfqpoint{4.264218in}{1.226002in}}{\pgfqpoint{4.257085in}{1.226002in}}%
\pgfpathcurveto{\pgfqpoint{4.249952in}{1.226002in}}{\pgfqpoint{4.243110in}{1.223168in}}{\pgfqpoint{4.238067in}{1.218124in}}%
\pgfpathcurveto{\pgfqpoint{4.233023in}{1.213080in}}{\pgfqpoint{4.230189in}{1.206239in}}{\pgfqpoint{4.230189in}{1.199106in}}%
\pgfpathcurveto{\pgfqpoint{4.230189in}{1.191973in}}{\pgfqpoint{4.233023in}{1.185131in}}{\pgfqpoint{4.238067in}{1.180088in}}%
\pgfpathcurveto{\pgfqpoint{4.243110in}{1.175044in}}{\pgfqpoint{4.249952in}{1.172210in}}{\pgfqpoint{4.257085in}{1.172210in}}%
\pgfpathclose%
\pgfusepath{stroke,fill}%
\end{pgfscope}%
\begin{pgfscope}%
\pgfpathrectangle{\pgfqpoint{2.867647in}{0.500000in}}{\pgfqpoint{1.764706in}{1.700000in}}%
\pgfusepath{clip}%
\pgfsetbuttcap%
\pgfsetroundjoin%
\definecolor{currentfill}{rgb}{0.968509,0.792226,0.676405}%
\pgfsetfillcolor{currentfill}%
\pgfsetlinewidth{0.311001pt}%
\definecolor{currentstroke}{rgb}{1.000000,1.000000,1.000000}%
\pgfsetstrokecolor{currentstroke}%
\pgfsetdash{}{0pt}%
\pgfpathmoveto{\pgfqpoint{4.260680in}{1.490406in}}%
\pgfpathcurveto{\pgfqpoint{4.267813in}{1.490406in}}{\pgfqpoint{4.274655in}{1.493240in}}{\pgfqpoint{4.279698in}{1.498284in}}%
\pgfpathcurveto{\pgfqpoint{4.284742in}{1.503327in}}{\pgfqpoint{4.287576in}{1.510169in}}{\pgfqpoint{4.287576in}{1.517302in}}%
\pgfpathcurveto{\pgfqpoint{4.287576in}{1.524435in}}{\pgfqpoint{4.284742in}{1.531276in}}{\pgfqpoint{4.279698in}{1.536320in}}%
\pgfpathcurveto{\pgfqpoint{4.274655in}{1.541364in}}{\pgfqpoint{4.267813in}{1.544197in}}{\pgfqpoint{4.260680in}{1.544197in}}%
\pgfpathcurveto{\pgfqpoint{4.253547in}{1.544197in}}{\pgfqpoint{4.246706in}{1.541364in}}{\pgfqpoint{4.241662in}{1.536320in}}%
\pgfpathcurveto{\pgfqpoint{4.236618in}{1.531276in}}{\pgfqpoint{4.233784in}{1.524435in}}{\pgfqpoint{4.233784in}{1.517302in}}%
\pgfpathcurveto{\pgfqpoint{4.233784in}{1.510169in}}{\pgfqpoint{4.236618in}{1.503327in}}{\pgfqpoint{4.241662in}{1.498284in}}%
\pgfpathcurveto{\pgfqpoint{4.246706in}{1.493240in}}{\pgfqpoint{4.253547in}{1.490406in}}{\pgfqpoint{4.260680in}{1.490406in}}%
\pgfpathclose%
\pgfusepath{stroke,fill}%
\end{pgfscope}%
\begin{pgfscope}%
\pgfpathrectangle{\pgfqpoint{2.867647in}{0.500000in}}{\pgfqpoint{1.764706in}{1.700000in}}%
\pgfusepath{clip}%
\pgfsetbuttcap%
\pgfsetroundjoin%
\definecolor{currentfill}{rgb}{0.968931,0.798091,0.685123}%
\pgfsetfillcolor{currentfill}%
\pgfsetlinewidth{0.311001pt}%
\definecolor{currentstroke}{rgb}{1.000000,1.000000,1.000000}%
\pgfsetstrokecolor{currentstroke}%
\pgfsetdash{}{0pt}%
\pgfpathmoveto{\pgfqpoint{4.117794in}{1.701274in}}%
\pgfpathcurveto{\pgfqpoint{4.124926in}{1.701274in}}{\pgfqpoint{4.131768in}{1.704108in}}{\pgfqpoint{4.136812in}{1.709151in}}%
\pgfpathcurveto{\pgfqpoint{4.141855in}{1.714195in}}{\pgfqpoint{4.144689in}{1.721037in}}{\pgfqpoint{4.144689in}{1.728170in}}%
\pgfpathcurveto{\pgfqpoint{4.144689in}{1.735302in}}{\pgfqpoint{4.141855in}{1.742144in}}{\pgfqpoint{4.136812in}{1.747188in}}%
\pgfpathcurveto{\pgfqpoint{4.131768in}{1.752231in}}{\pgfqpoint{4.124926in}{1.755065in}}{\pgfqpoint{4.117794in}{1.755065in}}%
\pgfpathcurveto{\pgfqpoint{4.110661in}{1.755065in}}{\pgfqpoint{4.103819in}{1.752231in}}{\pgfqpoint{4.098775in}{1.747188in}}%
\pgfpathcurveto{\pgfqpoint{4.093732in}{1.742144in}}{\pgfqpoint{4.090898in}{1.735302in}}{\pgfqpoint{4.090898in}{1.728170in}}%
\pgfpathcurveto{\pgfqpoint{4.090898in}{1.721037in}}{\pgfqpoint{4.093732in}{1.714195in}}{\pgfqpoint{4.098775in}{1.709151in}}%
\pgfpathcurveto{\pgfqpoint{4.103819in}{1.704108in}}{\pgfqpoint{4.110661in}{1.701274in}}{\pgfqpoint{4.117794in}{1.701274in}}%
\pgfpathclose%
\pgfusepath{stroke,fill}%
\end{pgfscope}%
\begin{pgfscope}%
\pgfpathrectangle{\pgfqpoint{2.867647in}{0.500000in}}{\pgfqpoint{1.764706in}{1.700000in}}%
\pgfusepath{clip}%
\pgfsetbuttcap%
\pgfsetroundjoin%
\definecolor{currentfill}{rgb}{0.976287,0.879862,0.805788}%
\pgfsetfillcolor{currentfill}%
\pgfsetlinewidth{0.311001pt}%
\definecolor{currentstroke}{rgb}{1.000000,1.000000,1.000000}%
\pgfsetstrokecolor{currentstroke}%
\pgfsetdash{}{0pt}%
\pgfpathmoveto{\pgfqpoint{4.140822in}{1.339397in}}%
\pgfpathcurveto{\pgfqpoint{4.147955in}{1.339397in}}{\pgfqpoint{4.154796in}{1.342231in}}{\pgfqpoint{4.159840in}{1.347274in}}%
\pgfpathcurveto{\pgfqpoint{4.164884in}{1.352318in}}{\pgfqpoint{4.167718in}{1.359160in}}{\pgfqpoint{4.167718in}{1.366293in}}%
\pgfpathcurveto{\pgfqpoint{4.167718in}{1.373425in}}{\pgfqpoint{4.164884in}{1.380267in}}{\pgfqpoint{4.159840in}{1.385311in}}%
\pgfpathcurveto{\pgfqpoint{4.154796in}{1.390354in}}{\pgfqpoint{4.147955in}{1.393188in}}{\pgfqpoint{4.140822in}{1.393188in}}%
\pgfpathcurveto{\pgfqpoint{4.133689in}{1.393188in}}{\pgfqpoint{4.126847in}{1.390354in}}{\pgfqpoint{4.121804in}{1.385311in}}%
\pgfpathcurveto{\pgfqpoint{4.116760in}{1.380267in}}{\pgfqpoint{4.113926in}{1.373425in}}{\pgfqpoint{4.113926in}{1.366293in}}%
\pgfpathcurveto{\pgfqpoint{4.113926in}{1.359160in}}{\pgfqpoint{4.116760in}{1.352318in}}{\pgfqpoint{4.121804in}{1.347274in}}%
\pgfpathcurveto{\pgfqpoint{4.126847in}{1.342231in}}{\pgfqpoint{4.133689in}{1.339397in}}{\pgfqpoint{4.140822in}{1.339397in}}%
\pgfpathclose%
\pgfusepath{stroke,fill}%
\end{pgfscope}%
\begin{pgfscope}%
\pgfpathrectangle{\pgfqpoint{2.867647in}{0.500000in}}{\pgfqpoint{1.764706in}{1.700000in}}%
\pgfusepath{clip}%
\pgfsetbuttcap%
\pgfsetroundjoin%
\definecolor{currentfill}{rgb}{0.974412,0.862387,0.780156}%
\pgfsetfillcolor{currentfill}%
\pgfsetlinewidth{0.311001pt}%
\definecolor{currentstroke}{rgb}{1.000000,1.000000,1.000000}%
\pgfsetstrokecolor{currentstroke}%
\pgfsetdash{}{0pt}%
\pgfpathmoveto{\pgfqpoint{4.115740in}{1.437731in}}%
\pgfpathcurveto{\pgfqpoint{4.122873in}{1.437731in}}{\pgfqpoint{4.129714in}{1.440565in}}{\pgfqpoint{4.134758in}{1.445608in}}%
\pgfpathcurveto{\pgfqpoint{4.139802in}{1.450652in}}{\pgfqpoint{4.142636in}{1.457494in}}{\pgfqpoint{4.142636in}{1.464626in}}%
\pgfpathcurveto{\pgfqpoint{4.142636in}{1.471759in}}{\pgfqpoint{4.139802in}{1.478601in}}{\pgfqpoint{4.134758in}{1.483645in}}%
\pgfpathcurveto{\pgfqpoint{4.129714in}{1.488688in}}{\pgfqpoint{4.122873in}{1.491522in}}{\pgfqpoint{4.115740in}{1.491522in}}%
\pgfpathcurveto{\pgfqpoint{4.108607in}{1.491522in}}{\pgfqpoint{4.101765in}{1.488688in}}{\pgfqpoint{4.096722in}{1.483645in}}%
\pgfpathcurveto{\pgfqpoint{4.091678in}{1.478601in}}{\pgfqpoint{4.088844in}{1.471759in}}{\pgfqpoint{4.088844in}{1.464626in}}%
\pgfpathcurveto{\pgfqpoint{4.088844in}{1.457494in}}{\pgfqpoint{4.091678in}{1.450652in}}{\pgfqpoint{4.096722in}{1.445608in}}%
\pgfpathcurveto{\pgfqpoint{4.101765in}{1.440565in}}{\pgfqpoint{4.108607in}{1.437731in}}{\pgfqpoint{4.115740in}{1.437731in}}%
\pgfpathclose%
\pgfusepath{stroke,fill}%
\end{pgfscope}%
\begin{pgfscope}%
\pgfpathrectangle{\pgfqpoint{2.867647in}{0.500000in}}{\pgfqpoint{1.764706in}{1.700000in}}%
\pgfusepath{clip}%
\pgfsetbuttcap%
\pgfsetroundjoin%
\definecolor{currentfill}{rgb}{0.981377,0.920617,0.865369}%
\pgfsetfillcolor{currentfill}%
\pgfsetlinewidth{0.311001pt}%
\definecolor{currentstroke}{rgb}{1.000000,1.000000,1.000000}%
\pgfsetstrokecolor{currentstroke}%
\pgfsetdash{}{0pt}%
\pgfpathmoveto{\pgfqpoint{4.180538in}{1.357857in}}%
\pgfpathcurveto{\pgfqpoint{4.187671in}{1.357857in}}{\pgfqpoint{4.194513in}{1.360691in}}{\pgfqpoint{4.199557in}{1.365735in}}%
\pgfpathcurveto{\pgfqpoint{4.204600in}{1.370778in}}{\pgfqpoint{4.207434in}{1.377620in}}{\pgfqpoint{4.207434in}{1.384753in}}%
\pgfpathcurveto{\pgfqpoint{4.207434in}{1.391886in}}{\pgfqpoint{4.204600in}{1.398727in}}{\pgfqpoint{4.199557in}{1.403771in}}%
\pgfpathcurveto{\pgfqpoint{4.194513in}{1.408815in}}{\pgfqpoint{4.187671in}{1.411649in}}{\pgfqpoint{4.180538in}{1.411649in}}%
\pgfpathcurveto{\pgfqpoint{4.173406in}{1.411649in}}{\pgfqpoint{4.166564in}{1.408815in}}{\pgfqpoint{4.161520in}{1.403771in}}%
\pgfpathcurveto{\pgfqpoint{4.156477in}{1.398727in}}{\pgfqpoint{4.153643in}{1.391886in}}{\pgfqpoint{4.153643in}{1.384753in}}%
\pgfpathcurveto{\pgfqpoint{4.153643in}{1.377620in}}{\pgfqpoint{4.156477in}{1.370778in}}{\pgfqpoint{4.161520in}{1.365735in}}%
\pgfpathcurveto{\pgfqpoint{4.166564in}{1.360691in}}{\pgfqpoint{4.173406in}{1.357857in}}{\pgfqpoint{4.180538in}{1.357857in}}%
\pgfpathclose%
\pgfusepath{stroke,fill}%
\end{pgfscope}%
\begin{pgfscope}%
\pgfpathrectangle{\pgfqpoint{2.867647in}{0.500000in}}{\pgfqpoint{1.764706in}{1.700000in}}%
\pgfusepath{clip}%
\pgfsetbuttcap%
\pgfsetroundjoin%
\definecolor{currentfill}{rgb}{0.962985,0.612625,0.451451}%
\pgfsetfillcolor{currentfill}%
\pgfsetlinewidth{0.311001pt}%
\definecolor{currentstroke}{rgb}{1.000000,1.000000,1.000000}%
\pgfsetstrokecolor{currentstroke}%
\pgfsetdash{}{0pt}%
\pgfpathmoveto{\pgfqpoint{3.983658in}{0.870628in}}%
\pgfpathcurveto{\pgfqpoint{3.990791in}{0.870628in}}{\pgfqpoint{3.997633in}{0.873462in}}{\pgfqpoint{4.002676in}{0.878506in}}%
\pgfpathcurveto{\pgfqpoint{4.007720in}{0.883549in}}{\pgfqpoint{4.010554in}{0.890391in}}{\pgfqpoint{4.010554in}{0.897524in}}%
\pgfpathcurveto{\pgfqpoint{4.010554in}{0.904657in}}{\pgfqpoint{4.007720in}{0.911498in}}{\pgfqpoint{4.002676in}{0.916542in}}%
\pgfpathcurveto{\pgfqpoint{3.997633in}{0.921585in}}{\pgfqpoint{3.990791in}{0.924419in}}{\pgfqpoint{3.983658in}{0.924419in}}%
\pgfpathcurveto{\pgfqpoint{3.976525in}{0.924419in}}{\pgfqpoint{3.969684in}{0.921585in}}{\pgfqpoint{3.964640in}{0.916542in}}%
\pgfpathcurveto{\pgfqpoint{3.959596in}{0.911498in}}{\pgfqpoint{3.956762in}{0.904657in}}{\pgfqpoint{3.956762in}{0.897524in}}%
\pgfpathcurveto{\pgfqpoint{3.956762in}{0.890391in}}{\pgfqpoint{3.959596in}{0.883549in}}{\pgfqpoint{3.964640in}{0.878506in}}%
\pgfpathcurveto{\pgfqpoint{3.969684in}{0.873462in}}{\pgfqpoint{3.976525in}{0.870628in}}{\pgfqpoint{3.983658in}{0.870628in}}%
\pgfpathclose%
\pgfusepath{stroke,fill}%
\end{pgfscope}%
\begin{pgfscope}%
\pgfpathrectangle{\pgfqpoint{2.867647in}{0.500000in}}{\pgfqpoint{1.764706in}{1.700000in}}%
\pgfusepath{clip}%
\pgfsetbuttcap%
\pgfsetroundjoin%
\definecolor{currentfill}{rgb}{0.975018,0.868213,0.788710}%
\pgfsetfillcolor{currentfill}%
\pgfsetlinewidth{0.311001pt}%
\definecolor{currentstroke}{rgb}{1.000000,1.000000,1.000000}%
\pgfsetstrokecolor{currentstroke}%
\pgfsetdash{}{0pt}%
\pgfpathmoveto{\pgfqpoint{4.228238in}{1.471087in}}%
\pgfpathcurveto{\pgfqpoint{4.235371in}{1.471087in}}{\pgfqpoint{4.242213in}{1.473921in}}{\pgfqpoint{4.247256in}{1.478965in}}%
\pgfpathcurveto{\pgfqpoint{4.252300in}{1.484008in}}{\pgfqpoint{4.255134in}{1.490850in}}{\pgfqpoint{4.255134in}{1.497983in}}%
\pgfpathcurveto{\pgfqpoint{4.255134in}{1.505116in}}{\pgfqpoint{4.252300in}{1.511957in}}{\pgfqpoint{4.247256in}{1.517001in}}%
\pgfpathcurveto{\pgfqpoint{4.242213in}{1.522045in}}{\pgfqpoint{4.235371in}{1.524879in}}{\pgfqpoint{4.228238in}{1.524879in}}%
\pgfpathcurveto{\pgfqpoint{4.221105in}{1.524879in}}{\pgfqpoint{4.214264in}{1.522045in}}{\pgfqpoint{4.209220in}{1.517001in}}%
\pgfpathcurveto{\pgfqpoint{4.204176in}{1.511957in}}{\pgfqpoint{4.201343in}{1.505116in}}{\pgfqpoint{4.201343in}{1.497983in}}%
\pgfpathcurveto{\pgfqpoint{4.201343in}{1.490850in}}{\pgfqpoint{4.204176in}{1.484008in}}{\pgfqpoint{4.209220in}{1.478965in}}%
\pgfpathcurveto{\pgfqpoint{4.214264in}{1.473921in}}{\pgfqpoint{4.221105in}{1.471087in}}{\pgfqpoint{4.228238in}{1.471087in}}%
\pgfpathclose%
\pgfusepath{stroke,fill}%
\end{pgfscope}%
\begin{pgfscope}%
\pgfpathrectangle{\pgfqpoint{2.867647in}{0.500000in}}{\pgfqpoint{1.764706in}{1.700000in}}%
\pgfusepath{clip}%
\pgfsetbuttcap%
\pgfsetroundjoin%
\definecolor{currentfill}{rgb}{0.970718,0.821518,0.719872}%
\pgfsetfillcolor{currentfill}%
\pgfsetlinewidth{0.311001pt}%
\definecolor{currentstroke}{rgb}{1.000000,1.000000,1.000000}%
\pgfsetstrokecolor{currentstroke}%
\pgfsetdash{}{0pt}%
\pgfpathmoveto{\pgfqpoint{4.122107in}{0.964969in}}%
\pgfpathcurveto{\pgfqpoint{4.129240in}{0.964969in}}{\pgfqpoint{4.136082in}{0.967803in}}{\pgfqpoint{4.141125in}{0.972846in}}%
\pgfpathcurveto{\pgfqpoint{4.146169in}{0.977890in}}{\pgfqpoint{4.149003in}{0.984732in}}{\pgfqpoint{4.149003in}{0.991864in}}%
\pgfpathcurveto{\pgfqpoint{4.149003in}{0.998997in}}{\pgfqpoint{4.146169in}{1.005839in}}{\pgfqpoint{4.141125in}{1.010883in}}%
\pgfpathcurveto{\pgfqpoint{4.136082in}{1.015926in}}{\pgfqpoint{4.129240in}{1.018760in}}{\pgfqpoint{4.122107in}{1.018760in}}%
\pgfpathcurveto{\pgfqpoint{4.114974in}{1.018760in}}{\pgfqpoint{4.108133in}{1.015926in}}{\pgfqpoint{4.103089in}{1.010883in}}%
\pgfpathcurveto{\pgfqpoint{4.098045in}{1.005839in}}{\pgfqpoint{4.095212in}{0.998997in}}{\pgfqpoint{4.095212in}{0.991864in}}%
\pgfpathcurveto{\pgfqpoint{4.095212in}{0.984732in}}{\pgfqpoint{4.098045in}{0.977890in}}{\pgfqpoint{4.103089in}{0.972846in}}%
\pgfpathcurveto{\pgfqpoint{4.108133in}{0.967803in}}{\pgfqpoint{4.114974in}{0.964969in}}{\pgfqpoint{4.122107in}{0.964969in}}%
\pgfpathclose%
\pgfusepath{stroke,fill}%
\end{pgfscope}%
\begin{pgfscope}%
\pgfpathrectangle{\pgfqpoint{2.867647in}{0.500000in}}{\pgfqpoint{1.764706in}{1.700000in}}%
\pgfusepath{clip}%
\pgfsetbuttcap%
\pgfsetroundjoin%
\definecolor{currentfill}{rgb}{0.977657,0.891500,0.822809}%
\pgfsetfillcolor{currentfill}%
\pgfsetlinewidth{0.311001pt}%
\definecolor{currentstroke}{rgb}{1.000000,1.000000,1.000000}%
\pgfsetstrokecolor{currentstroke}%
\pgfsetdash{}{0pt}%
\pgfpathmoveto{\pgfqpoint{4.125397in}{1.486177in}}%
\pgfpathcurveto{\pgfqpoint{4.132530in}{1.486177in}}{\pgfqpoint{4.139372in}{1.489010in}}{\pgfqpoint{4.144415in}{1.494054in}}%
\pgfpathcurveto{\pgfqpoint{4.149459in}{1.499098in}}{\pgfqpoint{4.152293in}{1.505939in}}{\pgfqpoint{4.152293in}{1.513072in}}%
\pgfpathcurveto{\pgfqpoint{4.152293in}{1.520205in}}{\pgfqpoint{4.149459in}{1.527047in}}{\pgfqpoint{4.144415in}{1.532090in}}%
\pgfpathcurveto{\pgfqpoint{4.139372in}{1.537134in}}{\pgfqpoint{4.132530in}{1.539968in}}{\pgfqpoint{4.125397in}{1.539968in}}%
\pgfpathcurveto{\pgfqpoint{4.118264in}{1.539968in}}{\pgfqpoint{4.111423in}{1.537134in}}{\pgfqpoint{4.106379in}{1.532090in}}%
\pgfpathcurveto{\pgfqpoint{4.101336in}{1.527047in}}{\pgfqpoint{4.098502in}{1.520205in}}{\pgfqpoint{4.098502in}{1.513072in}}%
\pgfpathcurveto{\pgfqpoint{4.098502in}{1.505939in}}{\pgfqpoint{4.101336in}{1.499098in}}{\pgfqpoint{4.106379in}{1.494054in}}%
\pgfpathcurveto{\pgfqpoint{4.111423in}{1.489010in}}{\pgfqpoint{4.118264in}{1.486177in}}{\pgfqpoint{4.125397in}{1.486177in}}%
\pgfpathclose%
\pgfusepath{stroke,fill}%
\end{pgfscope}%
\begin{pgfscope}%
\pgfpathrectangle{\pgfqpoint{2.867647in}{0.500000in}}{\pgfqpoint{1.764706in}{1.700000in}}%
\pgfusepath{clip}%
\pgfsetbuttcap%
\pgfsetroundjoin%
\definecolor{currentfill}{rgb}{0.977657,0.891500,0.822809}%
\pgfsetfillcolor{currentfill}%
\pgfsetlinewidth{0.311001pt}%
\definecolor{currentstroke}{rgb}{1.000000,1.000000,1.000000}%
\pgfsetstrokecolor{currentstroke}%
\pgfsetdash{}{0pt}%
\pgfpathmoveto{\pgfqpoint{4.105753in}{1.586282in}}%
\pgfpathcurveto{\pgfqpoint{4.112886in}{1.586282in}}{\pgfqpoint{4.119727in}{1.589116in}}{\pgfqpoint{4.124771in}{1.594160in}}%
\pgfpathcurveto{\pgfqpoint{4.129815in}{1.599203in}}{\pgfqpoint{4.132649in}{1.606045in}}{\pgfqpoint{4.132649in}{1.613178in}}%
\pgfpathcurveto{\pgfqpoint{4.132649in}{1.620311in}}{\pgfqpoint{4.129815in}{1.627152in}}{\pgfqpoint{4.124771in}{1.632196in}}%
\pgfpathcurveto{\pgfqpoint{4.119727in}{1.637240in}}{\pgfqpoint{4.112886in}{1.640073in}}{\pgfqpoint{4.105753in}{1.640073in}}%
\pgfpathcurveto{\pgfqpoint{4.098620in}{1.640073in}}{\pgfqpoint{4.091779in}{1.637240in}}{\pgfqpoint{4.086735in}{1.632196in}}%
\pgfpathcurveto{\pgfqpoint{4.081691in}{1.627152in}}{\pgfqpoint{4.078857in}{1.620311in}}{\pgfqpoint{4.078857in}{1.613178in}}%
\pgfpathcurveto{\pgfqpoint{4.078857in}{1.606045in}}{\pgfqpoint{4.081691in}{1.599203in}}{\pgfqpoint{4.086735in}{1.594160in}}%
\pgfpathcurveto{\pgfqpoint{4.091779in}{1.589116in}}{\pgfqpoint{4.098620in}{1.586282in}}{\pgfqpoint{4.105753in}{1.586282in}}%
\pgfpathclose%
\pgfusepath{stroke,fill}%
\end{pgfscope}%
\begin{pgfscope}%
\pgfpathrectangle{\pgfqpoint{2.867647in}{0.500000in}}{\pgfqpoint{1.764706in}{1.700000in}}%
\pgfusepath{clip}%
\pgfsetbuttcap%
\pgfsetroundjoin%
\definecolor{currentfill}{rgb}{0.976287,0.879862,0.805788}%
\pgfsetfillcolor{currentfill}%
\pgfsetlinewidth{0.311001pt}%
\definecolor{currentstroke}{rgb}{1.000000,1.000000,1.000000}%
\pgfsetstrokecolor{currentstroke}%
\pgfsetdash{}{0pt}%
\pgfpathmoveto{\pgfqpoint{4.136919in}{1.389781in}}%
\pgfpathcurveto{\pgfqpoint{4.144052in}{1.389781in}}{\pgfqpoint{4.150894in}{1.392614in}}{\pgfqpoint{4.155937in}{1.397658in}}%
\pgfpathcurveto{\pgfqpoint{4.160981in}{1.402702in}}{\pgfqpoint{4.163815in}{1.409543in}}{\pgfqpoint{4.163815in}{1.416676in}}%
\pgfpathcurveto{\pgfqpoint{4.163815in}{1.423809in}}{\pgfqpoint{4.160981in}{1.430651in}}{\pgfqpoint{4.155937in}{1.435694in}}%
\pgfpathcurveto{\pgfqpoint{4.150894in}{1.440738in}}{\pgfqpoint{4.144052in}{1.443572in}}{\pgfqpoint{4.136919in}{1.443572in}}%
\pgfpathcurveto{\pgfqpoint{4.129786in}{1.443572in}}{\pgfqpoint{4.122945in}{1.440738in}}{\pgfqpoint{4.117901in}{1.435694in}}%
\pgfpathcurveto{\pgfqpoint{4.112857in}{1.430651in}}{\pgfqpoint{4.110024in}{1.423809in}}{\pgfqpoint{4.110024in}{1.416676in}}%
\pgfpathcurveto{\pgfqpoint{4.110024in}{1.409543in}}{\pgfqpoint{4.112857in}{1.402702in}}{\pgfqpoint{4.117901in}{1.397658in}}%
\pgfpathcurveto{\pgfqpoint{4.122945in}{1.392614in}}{\pgfqpoint{4.129786in}{1.389781in}}{\pgfqpoint{4.136919in}{1.389781in}}%
\pgfpathclose%
\pgfusepath{stroke,fill}%
\end{pgfscope}%
\begin{pgfscope}%
\pgfpathrectangle{\pgfqpoint{2.867647in}{0.500000in}}{\pgfqpoint{1.764706in}{1.700000in}}%
\pgfusepath{clip}%
\pgfsetbuttcap%
\pgfsetroundjoin%
\definecolor{currentfill}{rgb}{0.976287,0.879862,0.805788}%
\pgfsetfillcolor{currentfill}%
\pgfsetlinewidth{0.311001pt}%
\definecolor{currentstroke}{rgb}{1.000000,1.000000,1.000000}%
\pgfsetstrokecolor{currentstroke}%
\pgfsetdash{}{0pt}%
\pgfpathmoveto{\pgfqpoint{4.201232in}{1.532183in}}%
\pgfpathcurveto{\pgfqpoint{4.208365in}{1.532183in}}{\pgfqpoint{4.215207in}{1.535017in}}{\pgfqpoint{4.220250in}{1.540060in}}%
\pgfpathcurveto{\pgfqpoint{4.225294in}{1.545104in}}{\pgfqpoint{4.228128in}{1.551946in}}{\pgfqpoint{4.228128in}{1.559078in}}%
\pgfpathcurveto{\pgfqpoint{4.228128in}{1.566211in}}{\pgfqpoint{4.225294in}{1.573053in}}{\pgfqpoint{4.220250in}{1.578097in}}%
\pgfpathcurveto{\pgfqpoint{4.215207in}{1.583140in}}{\pgfqpoint{4.208365in}{1.585974in}}{\pgfqpoint{4.201232in}{1.585974in}}%
\pgfpathcurveto{\pgfqpoint{4.194099in}{1.585974in}}{\pgfqpoint{4.187258in}{1.583140in}}{\pgfqpoint{4.182214in}{1.578097in}}%
\pgfpathcurveto{\pgfqpoint{4.177170in}{1.573053in}}{\pgfqpoint{4.174336in}{1.566211in}}{\pgfqpoint{4.174336in}{1.559078in}}%
\pgfpathcurveto{\pgfqpoint{4.174336in}{1.551946in}}{\pgfqpoint{4.177170in}{1.545104in}}{\pgfqpoint{4.182214in}{1.540060in}}%
\pgfpathcurveto{\pgfqpoint{4.187258in}{1.535017in}}{\pgfqpoint{4.194099in}{1.532183in}}{\pgfqpoint{4.201232in}{1.532183in}}%
\pgfpathclose%
\pgfusepath{stroke,fill}%
\end{pgfscope}%
\begin{pgfscope}%
\pgfpathrectangle{\pgfqpoint{2.867647in}{0.500000in}}{\pgfqpoint{1.764706in}{1.700000in}}%
\pgfusepath{clip}%
\pgfsetbuttcap%
\pgfsetroundjoin%
\definecolor{currentfill}{rgb}{0.961433,0.573272,0.412036}%
\pgfsetfillcolor{currentfill}%
\pgfsetlinewidth{0.311001pt}%
\definecolor{currentstroke}{rgb}{1.000000,1.000000,1.000000}%
\pgfsetstrokecolor{currentstroke}%
\pgfsetdash{}{0pt}%
\pgfpathmoveto{\pgfqpoint{4.155944in}{1.762476in}}%
\pgfpathcurveto{\pgfqpoint{4.163077in}{1.762476in}}{\pgfqpoint{4.169918in}{1.765310in}}{\pgfqpoint{4.174962in}{1.770354in}}%
\pgfpathcurveto{\pgfqpoint{4.180006in}{1.775398in}}{\pgfqpoint{4.182840in}{1.782239in}}{\pgfqpoint{4.182840in}{1.789372in}}%
\pgfpathcurveto{\pgfqpoint{4.182840in}{1.796505in}}{\pgfqpoint{4.180006in}{1.803347in}}{\pgfqpoint{4.174962in}{1.808390in}}%
\pgfpathcurveto{\pgfqpoint{4.169918in}{1.813434in}}{\pgfqpoint{4.163077in}{1.816268in}}{\pgfqpoint{4.155944in}{1.816268in}}%
\pgfpathcurveto{\pgfqpoint{4.148811in}{1.816268in}}{\pgfqpoint{4.141969in}{1.813434in}}{\pgfqpoint{4.136926in}{1.808390in}}%
\pgfpathcurveto{\pgfqpoint{4.131882in}{1.803347in}}{\pgfqpoint{4.129048in}{1.796505in}}{\pgfqpoint{4.129048in}{1.789372in}}%
\pgfpathcurveto{\pgfqpoint{4.129048in}{1.782239in}}{\pgfqpoint{4.131882in}{1.775398in}}{\pgfqpoint{4.136926in}{1.770354in}}%
\pgfpathcurveto{\pgfqpoint{4.141969in}{1.765310in}}{\pgfqpoint{4.148811in}{1.762476in}}{\pgfqpoint{4.155944in}{1.762476in}}%
\pgfpathclose%
\pgfusepath{stroke,fill}%
\end{pgfscope}%
\begin{pgfscope}%
\pgfpathrectangle{\pgfqpoint{2.867647in}{0.500000in}}{\pgfqpoint{1.764706in}{1.700000in}}%
\pgfusepath{clip}%
\pgfsetbuttcap%
\pgfsetroundjoin%
\definecolor{currentfill}{rgb}{0.957344,0.505732,0.351309}%
\pgfsetfillcolor{currentfill}%
\pgfsetlinewidth{0.311001pt}%
\definecolor{currentstroke}{rgb}{1.000000,1.000000,1.000000}%
\pgfsetstrokecolor{currentstroke}%
\pgfsetdash{}{0pt}%
\pgfpathmoveto{\pgfqpoint{4.336845in}{1.176773in}}%
\pgfpathcurveto{\pgfqpoint{4.343977in}{1.176773in}}{\pgfqpoint{4.350819in}{1.179607in}}{\pgfqpoint{4.355863in}{1.184651in}}%
\pgfpathcurveto{\pgfqpoint{4.360906in}{1.189695in}}{\pgfqpoint{4.363740in}{1.196536in}}{\pgfqpoint{4.363740in}{1.203669in}}%
\pgfpathcurveto{\pgfqpoint{4.363740in}{1.210802in}}{\pgfqpoint{4.360906in}{1.217643in}}{\pgfqpoint{4.355863in}{1.222687in}}%
\pgfpathcurveto{\pgfqpoint{4.350819in}{1.227731in}}{\pgfqpoint{4.343977in}{1.230565in}}{\pgfqpoint{4.336845in}{1.230565in}}%
\pgfpathcurveto{\pgfqpoint{4.329712in}{1.230565in}}{\pgfqpoint{4.322870in}{1.227731in}}{\pgfqpoint{4.317826in}{1.222687in}}%
\pgfpathcurveto{\pgfqpoint{4.312783in}{1.217643in}}{\pgfqpoint{4.309949in}{1.210802in}}{\pgfqpoint{4.309949in}{1.203669in}}%
\pgfpathcurveto{\pgfqpoint{4.309949in}{1.196536in}}{\pgfqpoint{4.312783in}{1.189695in}}{\pgfqpoint{4.317826in}{1.184651in}}%
\pgfpathcurveto{\pgfqpoint{4.322870in}{1.179607in}}{\pgfqpoint{4.329712in}{1.176773in}}{\pgfqpoint{4.336845in}{1.176773in}}%
\pgfpathclose%
\pgfusepath{stroke,fill}%
\end{pgfscope}%
\begin{pgfscope}%
\pgfpathrectangle{\pgfqpoint{2.867647in}{0.500000in}}{\pgfqpoint{1.764706in}{1.700000in}}%
\pgfusepath{clip}%
\pgfsetbuttcap%
\pgfsetroundjoin%
\definecolor{currentfill}{rgb}{0.966560,0.756582,0.625273}%
\pgfsetfillcolor{currentfill}%
\pgfsetlinewidth{0.311001pt}%
\definecolor{currentstroke}{rgb}{1.000000,1.000000,1.000000}%
\pgfsetstrokecolor{currentstroke}%
\pgfsetdash{}{0pt}%
\pgfpathmoveto{\pgfqpoint{4.118377in}{1.722733in}}%
\pgfpathcurveto{\pgfqpoint{4.125510in}{1.722733in}}{\pgfqpoint{4.132352in}{1.725567in}}{\pgfqpoint{4.137395in}{1.730610in}}%
\pgfpathcurveto{\pgfqpoint{4.142439in}{1.735654in}}{\pgfqpoint{4.145273in}{1.742496in}}{\pgfqpoint{4.145273in}{1.749628in}}%
\pgfpathcurveto{\pgfqpoint{4.145273in}{1.756761in}}{\pgfqpoint{4.142439in}{1.763603in}}{\pgfqpoint{4.137395in}{1.768646in}}%
\pgfpathcurveto{\pgfqpoint{4.132352in}{1.773690in}}{\pgfqpoint{4.125510in}{1.776524in}}{\pgfqpoint{4.118377in}{1.776524in}}%
\pgfpathcurveto{\pgfqpoint{4.111244in}{1.776524in}}{\pgfqpoint{4.104403in}{1.773690in}}{\pgfqpoint{4.099359in}{1.768646in}}%
\pgfpathcurveto{\pgfqpoint{4.094315in}{1.763603in}}{\pgfqpoint{4.091481in}{1.756761in}}{\pgfqpoint{4.091481in}{1.749628in}}%
\pgfpathcurveto{\pgfqpoint{4.091481in}{1.742496in}}{\pgfqpoint{4.094315in}{1.735654in}}{\pgfqpoint{4.099359in}{1.730610in}}%
\pgfpathcurveto{\pgfqpoint{4.104403in}{1.725567in}}{\pgfqpoint{4.111244in}{1.722733in}}{\pgfqpoint{4.118377in}{1.722733in}}%
\pgfpathclose%
\pgfusepath{stroke,fill}%
\end{pgfscope}%
\begin{pgfscope}%
\pgfpathrectangle{\pgfqpoint{2.867647in}{0.500000in}}{\pgfqpoint{1.764706in}{1.700000in}}%
\pgfusepath{clip}%
\pgfsetbuttcap%
\pgfsetroundjoin%
\definecolor{currentfill}{rgb}{0.976961,0.885681,0.814303}%
\pgfsetfillcolor{currentfill}%
\pgfsetlinewidth{0.311001pt}%
\definecolor{currentstroke}{rgb}{1.000000,1.000000,1.000000}%
\pgfsetstrokecolor{currentstroke}%
\pgfsetdash{}{0pt}%
\pgfpathmoveto{\pgfqpoint{4.216046in}{1.469806in}}%
\pgfpathcurveto{\pgfqpoint{4.223179in}{1.469806in}}{\pgfqpoint{4.230020in}{1.472640in}}{\pgfqpoint{4.235064in}{1.477683in}}%
\pgfpathcurveto{\pgfqpoint{4.240108in}{1.482727in}}{\pgfqpoint{4.242942in}{1.489569in}}{\pgfqpoint{4.242942in}{1.496702in}}%
\pgfpathcurveto{\pgfqpoint{4.242942in}{1.503834in}}{\pgfqpoint{4.240108in}{1.510676in}}{\pgfqpoint{4.235064in}{1.515720in}}%
\pgfpathcurveto{\pgfqpoint{4.230020in}{1.520763in}}{\pgfqpoint{4.223179in}{1.523597in}}{\pgfqpoint{4.216046in}{1.523597in}}%
\pgfpathcurveto{\pgfqpoint{4.208913in}{1.523597in}}{\pgfqpoint{4.202071in}{1.520763in}}{\pgfqpoint{4.197028in}{1.515720in}}%
\pgfpathcurveto{\pgfqpoint{4.191984in}{1.510676in}}{\pgfqpoint{4.189150in}{1.503834in}}{\pgfqpoint{4.189150in}{1.496702in}}%
\pgfpathcurveto{\pgfqpoint{4.189150in}{1.489569in}}{\pgfqpoint{4.191984in}{1.482727in}}{\pgfqpoint{4.197028in}{1.477683in}}%
\pgfpathcurveto{\pgfqpoint{4.202071in}{1.472640in}}{\pgfqpoint{4.208913in}{1.469806in}}{\pgfqpoint{4.216046in}{1.469806in}}%
\pgfpathclose%
\pgfusepath{stroke,fill}%
\end{pgfscope}%
\begin{pgfscope}%
\pgfpathrectangle{\pgfqpoint{2.867647in}{0.500000in}}{\pgfqpoint{1.764706in}{1.700000in}}%
\pgfusepath{clip}%
\pgfsetbuttcap%
\pgfsetroundjoin%
\definecolor{currentfill}{rgb}{0.965753,0.732351,0.592427}%
\pgfsetfillcolor{currentfill}%
\pgfsetlinewidth{0.311001pt}%
\definecolor{currentstroke}{rgb}{1.000000,1.000000,1.000000}%
\pgfsetstrokecolor{currentstroke}%
\pgfsetdash{}{0pt}%
\pgfpathmoveto{\pgfqpoint{4.047241in}{0.901372in}}%
\pgfpathcurveto{\pgfqpoint{4.054374in}{0.901372in}}{\pgfqpoint{4.061216in}{0.904206in}}{\pgfqpoint{4.066259in}{0.909250in}}%
\pgfpathcurveto{\pgfqpoint{4.071303in}{0.914293in}}{\pgfqpoint{4.074137in}{0.921135in}}{\pgfqpoint{4.074137in}{0.928268in}}%
\pgfpathcurveto{\pgfqpoint{4.074137in}{0.935400in}}{\pgfqpoint{4.071303in}{0.942242in}}{\pgfqpoint{4.066259in}{0.947286in}}%
\pgfpathcurveto{\pgfqpoint{4.061216in}{0.952329in}}{\pgfqpoint{4.054374in}{0.955163in}}{\pgfqpoint{4.047241in}{0.955163in}}%
\pgfpathcurveto{\pgfqpoint{4.040108in}{0.955163in}}{\pgfqpoint{4.033267in}{0.952329in}}{\pgfqpoint{4.028223in}{0.947286in}}%
\pgfpathcurveto{\pgfqpoint{4.023179in}{0.942242in}}{\pgfqpoint{4.020345in}{0.935400in}}{\pgfqpoint{4.020345in}{0.928268in}}%
\pgfpathcurveto{\pgfqpoint{4.020345in}{0.921135in}}{\pgfqpoint{4.023179in}{0.914293in}}{\pgfqpoint{4.028223in}{0.909250in}}%
\pgfpathcurveto{\pgfqpoint{4.033267in}{0.904206in}}{\pgfqpoint{4.040108in}{0.901372in}}{\pgfqpoint{4.047241in}{0.901372in}}%
\pgfpathclose%
\pgfusepath{stroke,fill}%
\end{pgfscope}%
\begin{pgfscope}%
\pgfpathrectangle{\pgfqpoint{2.867647in}{0.500000in}}{\pgfqpoint{1.764706in}{1.700000in}}%
\pgfusepath{clip}%
\pgfsetbuttcap%
\pgfsetroundjoin%
\definecolor{currentfill}{rgb}{0.976961,0.885681,0.814303}%
\pgfsetfillcolor{currentfill}%
\pgfsetlinewidth{0.311001pt}%
\definecolor{currentstroke}{rgb}{1.000000,1.000000,1.000000}%
\pgfsetstrokecolor{currentstroke}%
\pgfsetdash{}{0pt}%
\pgfpathmoveto{\pgfqpoint{4.234635in}{1.242154in}}%
\pgfpathcurveto{\pgfqpoint{4.241768in}{1.242154in}}{\pgfqpoint{4.248609in}{1.244988in}}{\pgfqpoint{4.253653in}{1.250031in}}%
\pgfpathcurveto{\pgfqpoint{4.258697in}{1.255075in}}{\pgfqpoint{4.261531in}{1.261917in}}{\pgfqpoint{4.261531in}{1.269050in}}%
\pgfpathcurveto{\pgfqpoint{4.261531in}{1.276182in}}{\pgfqpoint{4.258697in}{1.283024in}}{\pgfqpoint{4.253653in}{1.288068in}}%
\pgfpathcurveto{\pgfqpoint{4.248609in}{1.293111in}}{\pgfqpoint{4.241768in}{1.295945in}}{\pgfqpoint{4.234635in}{1.295945in}}%
\pgfpathcurveto{\pgfqpoint{4.227502in}{1.295945in}}{\pgfqpoint{4.220660in}{1.293111in}}{\pgfqpoint{4.215617in}{1.288068in}}%
\pgfpathcurveto{\pgfqpoint{4.210573in}{1.283024in}}{\pgfqpoint{4.207739in}{1.276182in}}{\pgfqpoint{4.207739in}{1.269050in}}%
\pgfpathcurveto{\pgfqpoint{4.207739in}{1.261917in}}{\pgfqpoint{4.210573in}{1.255075in}}{\pgfqpoint{4.215617in}{1.250031in}}%
\pgfpathcurveto{\pgfqpoint{4.220660in}{1.244988in}}{\pgfqpoint{4.227502in}{1.242154in}}{\pgfqpoint{4.234635in}{1.242154in}}%
\pgfpathclose%
\pgfusepath{stroke,fill}%
\end{pgfscope}%
\begin{pgfscope}%
\pgfpathrectangle{\pgfqpoint{2.867647in}{0.500000in}}{\pgfqpoint{1.764706in}{1.700000in}}%
\pgfusepath{clip}%
\pgfsetbuttcap%
\pgfsetroundjoin%
\definecolor{currentfill}{rgb}{0.967735,0.780441,0.659127}%
\pgfsetfillcolor{currentfill}%
\pgfsetlinewidth{0.311001pt}%
\definecolor{currentstroke}{rgb}{1.000000,1.000000,1.000000}%
\pgfsetstrokecolor{currentstroke}%
\pgfsetdash{}{0pt}%
\pgfpathmoveto{\pgfqpoint{4.062803in}{0.927541in}}%
\pgfpathcurveto{\pgfqpoint{4.069936in}{0.927541in}}{\pgfqpoint{4.076777in}{0.930375in}}{\pgfqpoint{4.081821in}{0.935418in}}%
\pgfpathcurveto{\pgfqpoint{4.086865in}{0.940462in}}{\pgfqpoint{4.089699in}{0.947304in}}{\pgfqpoint{4.089699in}{0.954437in}}%
\pgfpathcurveto{\pgfqpoint{4.089699in}{0.961569in}}{\pgfqpoint{4.086865in}{0.968411in}}{\pgfqpoint{4.081821in}{0.973455in}}%
\pgfpathcurveto{\pgfqpoint{4.076777in}{0.978498in}}{\pgfqpoint{4.069936in}{0.981332in}}{\pgfqpoint{4.062803in}{0.981332in}}%
\pgfpathcurveto{\pgfqpoint{4.055670in}{0.981332in}}{\pgfqpoint{4.048828in}{0.978498in}}{\pgfqpoint{4.043785in}{0.973455in}}%
\pgfpathcurveto{\pgfqpoint{4.038741in}{0.968411in}}{\pgfqpoint{4.035907in}{0.961569in}}{\pgfqpoint{4.035907in}{0.954437in}}%
\pgfpathcurveto{\pgfqpoint{4.035907in}{0.947304in}}{\pgfqpoint{4.038741in}{0.940462in}}{\pgfqpoint{4.043785in}{0.935418in}}%
\pgfpathcurveto{\pgfqpoint{4.048828in}{0.930375in}}{\pgfqpoint{4.055670in}{0.927541in}}{\pgfqpoint{4.062803in}{0.927541in}}%
\pgfpathclose%
\pgfusepath{stroke,fill}%
\end{pgfscope}%
\begin{pgfscope}%
\pgfpathrectangle{\pgfqpoint{2.867647in}{0.500000in}}{\pgfqpoint{1.764706in}{1.700000in}}%
\pgfusepath{clip}%
\pgfsetbuttcap%
\pgfsetroundjoin%
\definecolor{currentfill}{rgb}{0.973271,0.850724,0.762998}%
\pgfsetfillcolor{currentfill}%
\pgfsetlinewidth{0.311001pt}%
\definecolor{currentstroke}{rgb}{1.000000,1.000000,1.000000}%
\pgfsetstrokecolor{currentstroke}%
\pgfsetdash{}{0pt}%
\pgfpathmoveto{\pgfqpoint{4.120048in}{1.379030in}}%
\pgfpathcurveto{\pgfqpoint{4.127181in}{1.379030in}}{\pgfqpoint{4.134023in}{1.381864in}}{\pgfqpoint{4.139066in}{1.386908in}}%
\pgfpathcurveto{\pgfqpoint{4.144110in}{1.391952in}}{\pgfqpoint{4.146944in}{1.398793in}}{\pgfqpoint{4.146944in}{1.405926in}}%
\pgfpathcurveto{\pgfqpoint{4.146944in}{1.413059in}}{\pgfqpoint{4.144110in}{1.419901in}}{\pgfqpoint{4.139066in}{1.424944in}}%
\pgfpathcurveto{\pgfqpoint{4.134023in}{1.429988in}}{\pgfqpoint{4.127181in}{1.432822in}}{\pgfqpoint{4.120048in}{1.432822in}}%
\pgfpathcurveto{\pgfqpoint{4.112915in}{1.432822in}}{\pgfqpoint{4.106074in}{1.429988in}}{\pgfqpoint{4.101030in}{1.424944in}}%
\pgfpathcurveto{\pgfqpoint{4.095986in}{1.419901in}}{\pgfqpoint{4.093153in}{1.413059in}}{\pgfqpoint{4.093153in}{1.405926in}}%
\pgfpathcurveto{\pgfqpoint{4.093153in}{1.398793in}}{\pgfqpoint{4.095986in}{1.391952in}}{\pgfqpoint{4.101030in}{1.386908in}}%
\pgfpathcurveto{\pgfqpoint{4.106074in}{1.381864in}}{\pgfqpoint{4.112915in}{1.379030in}}{\pgfqpoint{4.120048in}{1.379030in}}%
\pgfpathclose%
\pgfusepath{stroke,fill}%
\end{pgfscope}%
\begin{pgfscope}%
\pgfpathrectangle{\pgfqpoint{2.867647in}{0.500000in}}{\pgfqpoint{1.764706in}{1.700000in}}%
\pgfusepath{clip}%
\pgfsetbuttcap%
\pgfsetroundjoin%
\definecolor{currentfill}{rgb}{0.969359,0.803954,0.693832}%
\pgfsetfillcolor{currentfill}%
\pgfsetlinewidth{0.311001pt}%
\definecolor{currentstroke}{rgb}{1.000000,1.000000,1.000000}%
\pgfsetstrokecolor{currentstroke}%
\pgfsetdash{}{0pt}%
\pgfpathmoveto{\pgfqpoint{4.088540in}{1.705783in}}%
\pgfpathcurveto{\pgfqpoint{4.095673in}{1.705783in}}{\pgfqpoint{4.102515in}{1.708617in}}{\pgfqpoint{4.107559in}{1.713660in}}%
\pgfpathcurveto{\pgfqpoint{4.112602in}{1.718704in}}{\pgfqpoint{4.115436in}{1.725546in}}{\pgfqpoint{4.115436in}{1.732679in}}%
\pgfpathcurveto{\pgfqpoint{4.115436in}{1.739811in}}{\pgfqpoint{4.112602in}{1.746653in}}{\pgfqpoint{4.107559in}{1.751697in}}%
\pgfpathcurveto{\pgfqpoint{4.102515in}{1.756740in}}{\pgfqpoint{4.095673in}{1.759574in}}{\pgfqpoint{4.088540in}{1.759574in}}%
\pgfpathcurveto{\pgfqpoint{4.081408in}{1.759574in}}{\pgfqpoint{4.074566in}{1.756740in}}{\pgfqpoint{4.069522in}{1.751697in}}%
\pgfpathcurveto{\pgfqpoint{4.064479in}{1.746653in}}{\pgfqpoint{4.061645in}{1.739811in}}{\pgfqpoint{4.061645in}{1.732679in}}%
\pgfpathcurveto{\pgfqpoint{4.061645in}{1.725546in}}{\pgfqpoint{4.064479in}{1.718704in}}{\pgfqpoint{4.069522in}{1.713660in}}%
\pgfpathcurveto{\pgfqpoint{4.074566in}{1.708617in}}{\pgfqpoint{4.081408in}{1.705783in}}{\pgfqpoint{4.088540in}{1.705783in}}%
\pgfpathclose%
\pgfusepath{stroke,fill}%
\end{pgfscope}%
\begin{pgfscope}%
\pgfpathrectangle{\pgfqpoint{2.867647in}{0.500000in}}{\pgfqpoint{1.764706in}{1.700000in}}%
\pgfusepath{clip}%
\pgfsetbuttcap%
\pgfsetroundjoin%
\definecolor{currentfill}{rgb}{0.962018,0.586477,0.424918}%
\pgfsetfillcolor{currentfill}%
\pgfsetlinewidth{0.311001pt}%
\definecolor{currentstroke}{rgb}{1.000000,1.000000,1.000000}%
\pgfsetstrokecolor{currentstroke}%
\pgfsetdash{}{0pt}%
\pgfpathmoveto{\pgfqpoint{4.046700in}{1.347925in}}%
\pgfpathcurveto{\pgfqpoint{4.053833in}{1.347925in}}{\pgfqpoint{4.060674in}{1.350759in}}{\pgfqpoint{4.065718in}{1.355803in}}%
\pgfpathcurveto{\pgfqpoint{4.070762in}{1.360847in}}{\pgfqpoint{4.073596in}{1.367688in}}{\pgfqpoint{4.073596in}{1.374821in}}%
\pgfpathcurveto{\pgfqpoint{4.073596in}{1.381954in}}{\pgfqpoint{4.070762in}{1.388796in}}{\pgfqpoint{4.065718in}{1.393839in}}%
\pgfpathcurveto{\pgfqpoint{4.060674in}{1.398883in}}{\pgfqpoint{4.053833in}{1.401717in}}{\pgfqpoint{4.046700in}{1.401717in}}%
\pgfpathcurveto{\pgfqpoint{4.039567in}{1.401717in}}{\pgfqpoint{4.032725in}{1.398883in}}{\pgfqpoint{4.027682in}{1.393839in}}%
\pgfpathcurveto{\pgfqpoint{4.022638in}{1.388796in}}{\pgfqpoint{4.019804in}{1.381954in}}{\pgfqpoint{4.019804in}{1.374821in}}%
\pgfpathcurveto{\pgfqpoint{4.019804in}{1.367688in}}{\pgfqpoint{4.022638in}{1.360847in}}{\pgfqpoint{4.027682in}{1.355803in}}%
\pgfpathcurveto{\pgfqpoint{4.032725in}{1.350759in}}{\pgfqpoint{4.039567in}{1.347925in}}{\pgfqpoint{4.046700in}{1.347925in}}%
\pgfpathclose%
\pgfusepath{stroke,fill}%
\end{pgfscope}%
\begin{pgfscope}%
\pgfpathrectangle{\pgfqpoint{2.867647in}{0.500000in}}{\pgfqpoint{1.764706in}{1.700000in}}%
\pgfusepath{clip}%
\pgfsetbuttcap%
\pgfsetroundjoin%
\definecolor{currentfill}{rgb}{0.963379,0.625574,0.465113}%
\pgfsetfillcolor{currentfill}%
\pgfsetlinewidth{0.311001pt}%
\definecolor{currentstroke}{rgb}{1.000000,1.000000,1.000000}%
\pgfsetstrokecolor{currentstroke}%
\pgfsetdash{}{0pt}%
\pgfpathmoveto{\pgfqpoint{4.042942in}{0.857272in}}%
\pgfpathcurveto{\pgfqpoint{4.050075in}{0.857272in}}{\pgfqpoint{4.056917in}{0.860106in}}{\pgfqpoint{4.061960in}{0.865149in}}%
\pgfpathcurveto{\pgfqpoint{4.067004in}{0.870193in}}{\pgfqpoint{4.069838in}{0.877035in}}{\pgfqpoint{4.069838in}{0.884168in}}%
\pgfpathcurveto{\pgfqpoint{4.069838in}{0.891300in}}{\pgfqpoint{4.067004in}{0.898142in}}{\pgfqpoint{4.061960in}{0.903186in}}%
\pgfpathcurveto{\pgfqpoint{4.056917in}{0.908229in}}{\pgfqpoint{4.050075in}{0.911063in}}{\pgfqpoint{4.042942in}{0.911063in}}%
\pgfpathcurveto{\pgfqpoint{4.035809in}{0.911063in}}{\pgfqpoint{4.028968in}{0.908229in}}{\pgfqpoint{4.023924in}{0.903186in}}%
\pgfpathcurveto{\pgfqpoint{4.018880in}{0.898142in}}{\pgfqpoint{4.016046in}{0.891300in}}{\pgfqpoint{4.016046in}{0.884168in}}%
\pgfpathcurveto{\pgfqpoint{4.016046in}{0.877035in}}{\pgfqpoint{4.018880in}{0.870193in}}{\pgfqpoint{4.023924in}{0.865149in}}%
\pgfpathcurveto{\pgfqpoint{4.028968in}{0.860106in}}{\pgfqpoint{4.035809in}{0.857272in}}{\pgfqpoint{4.042942in}{0.857272in}}%
\pgfpathclose%
\pgfusepath{stroke,fill}%
\end{pgfscope}%
\begin{pgfscope}%
\pgfpathrectangle{\pgfqpoint{2.867647in}{0.500000in}}{\pgfqpoint{1.764706in}{1.700000in}}%
\pgfusepath{clip}%
\pgfsetbuttcap%
\pgfsetroundjoin%
\definecolor{currentfill}{rgb}{0.981377,0.920617,0.865369}%
\pgfsetfillcolor{currentfill}%
\pgfsetlinewidth{0.311001pt}%
\definecolor{currentstroke}{rgb}{1.000000,1.000000,1.000000}%
\pgfsetstrokecolor{currentstroke}%
\pgfsetdash{}{0pt}%
\pgfpathmoveto{\pgfqpoint{4.158516in}{1.183116in}}%
\pgfpathcurveto{\pgfqpoint{4.165649in}{1.183116in}}{\pgfqpoint{4.172490in}{1.185949in}}{\pgfqpoint{4.177534in}{1.190993in}}%
\pgfpathcurveto{\pgfqpoint{4.182578in}{1.196037in}}{\pgfqpoint{4.185412in}{1.202878in}}{\pgfqpoint{4.185412in}{1.210011in}}%
\pgfpathcurveto{\pgfqpoint{4.185412in}{1.217144in}}{\pgfqpoint{4.182578in}{1.223986in}}{\pgfqpoint{4.177534in}{1.229029in}}%
\pgfpathcurveto{\pgfqpoint{4.172490in}{1.234073in}}{\pgfqpoint{4.165649in}{1.236907in}}{\pgfqpoint{4.158516in}{1.236907in}}%
\pgfpathcurveto{\pgfqpoint{4.151383in}{1.236907in}}{\pgfqpoint{4.144541in}{1.234073in}}{\pgfqpoint{4.139498in}{1.229029in}}%
\pgfpathcurveto{\pgfqpoint{4.134454in}{1.223986in}}{\pgfqpoint{4.131620in}{1.217144in}}{\pgfqpoint{4.131620in}{1.210011in}}%
\pgfpathcurveto{\pgfqpoint{4.131620in}{1.202878in}}{\pgfqpoint{4.134454in}{1.196037in}}{\pgfqpoint{4.139498in}{1.190993in}}%
\pgfpathcurveto{\pgfqpoint{4.144541in}{1.185949in}}{\pgfqpoint{4.151383in}{1.183116in}}{\pgfqpoint{4.158516in}{1.183116in}}%
\pgfpathclose%
\pgfusepath{stroke,fill}%
\end{pgfscope}%
\begin{pgfscope}%
\pgfpathrectangle{\pgfqpoint{2.867647in}{0.500000in}}{\pgfqpoint{1.764706in}{1.700000in}}%
\pgfusepath{clip}%
\pgfsetbuttcap%
\pgfsetroundjoin%
\definecolor{currentfill}{rgb}{0.966812,0.762584,0.633643}%
\pgfsetfillcolor{currentfill}%
\pgfsetlinewidth{0.311001pt}%
\definecolor{currentstroke}{rgb}{1.000000,1.000000,1.000000}%
\pgfsetstrokecolor{currentstroke}%
\pgfsetdash{}{0pt}%
\pgfpathmoveto{\pgfqpoint{4.218113in}{1.024227in}}%
\pgfpathcurveto{\pgfqpoint{4.225246in}{1.024227in}}{\pgfqpoint{4.232088in}{1.027060in}}{\pgfqpoint{4.237131in}{1.032104in}}%
\pgfpathcurveto{\pgfqpoint{4.242175in}{1.037148in}}{\pgfqpoint{4.245009in}{1.043989in}}{\pgfqpoint{4.245009in}{1.051122in}}%
\pgfpathcurveto{\pgfqpoint{4.245009in}{1.058255in}}{\pgfqpoint{4.242175in}{1.065097in}}{\pgfqpoint{4.237131in}{1.070140in}}%
\pgfpathcurveto{\pgfqpoint{4.232088in}{1.075184in}}{\pgfqpoint{4.225246in}{1.078018in}}{\pgfqpoint{4.218113in}{1.078018in}}%
\pgfpathcurveto{\pgfqpoint{4.210980in}{1.078018in}}{\pgfqpoint{4.204139in}{1.075184in}}{\pgfqpoint{4.199095in}{1.070140in}}%
\pgfpathcurveto{\pgfqpoint{4.194051in}{1.065097in}}{\pgfqpoint{4.191217in}{1.058255in}}{\pgfqpoint{4.191217in}{1.051122in}}%
\pgfpathcurveto{\pgfqpoint{4.191217in}{1.043989in}}{\pgfqpoint{4.194051in}{1.037148in}}{\pgfqpoint{4.199095in}{1.032104in}}%
\pgfpathcurveto{\pgfqpoint{4.204139in}{1.027060in}}{\pgfqpoint{4.210980in}{1.024227in}}{\pgfqpoint{4.218113in}{1.024227in}}%
\pgfpathclose%
\pgfusepath{stroke,fill}%
\end{pgfscope}%
\begin{pgfscope}%
\pgfpathrectangle{\pgfqpoint{2.867647in}{0.500000in}}{\pgfqpoint{1.764706in}{1.700000in}}%
\pgfusepath{clip}%
\pgfsetbuttcap%
\pgfsetroundjoin%
\definecolor{currentfill}{rgb}{0.698038,0.088972,0.346299}%
\pgfsetfillcolor{currentfill}%
\pgfsetlinewidth{0.311001pt}%
\definecolor{currentstroke}{rgb}{1.000000,1.000000,1.000000}%
\pgfsetstrokecolor{currentstroke}%
\pgfsetdash{}{0pt}%
\pgfpathmoveto{\pgfqpoint{3.917185in}{0.723352in}}%
\pgfpathcurveto{\pgfqpoint{3.924318in}{0.723352in}}{\pgfqpoint{3.931160in}{0.726186in}}{\pgfqpoint{3.936203in}{0.731229in}}%
\pgfpathcurveto{\pgfqpoint{3.941247in}{0.736273in}}{\pgfqpoint{3.944081in}{0.743115in}}{\pgfqpoint{3.944081in}{0.750247in}}%
\pgfpathcurveto{\pgfqpoint{3.944081in}{0.757380in}}{\pgfqpoint{3.941247in}{0.764222in}}{\pgfqpoint{3.936203in}{0.769266in}}%
\pgfpathcurveto{\pgfqpoint{3.931160in}{0.774309in}}{\pgfqpoint{3.924318in}{0.777143in}}{\pgfqpoint{3.917185in}{0.777143in}}%
\pgfpathcurveto{\pgfqpoint{3.910052in}{0.777143in}}{\pgfqpoint{3.903211in}{0.774309in}}{\pgfqpoint{3.898167in}{0.769266in}}%
\pgfpathcurveto{\pgfqpoint{3.893123in}{0.764222in}}{\pgfqpoint{3.890289in}{0.757380in}}{\pgfqpoint{3.890289in}{0.750247in}}%
\pgfpathcurveto{\pgfqpoint{3.890289in}{0.743115in}}{\pgfqpoint{3.893123in}{0.736273in}}{\pgfqpoint{3.898167in}{0.731229in}}%
\pgfpathcurveto{\pgfqpoint{3.903211in}{0.726186in}}{\pgfqpoint{3.910052in}{0.723352in}}{\pgfqpoint{3.917185in}{0.723352in}}%
\pgfpathclose%
\pgfusepath{stroke,fill}%
\end{pgfscope}%
\begin{pgfscope}%
\pgfpathrectangle{\pgfqpoint{2.867647in}{0.500000in}}{\pgfqpoint{1.764706in}{1.700000in}}%
\pgfusepath{clip}%
\pgfsetbuttcap%
\pgfsetroundjoin%
\definecolor{currentfill}{rgb}{0.975018,0.868213,0.788710}%
\pgfsetfillcolor{currentfill}%
\pgfsetlinewidth{0.311001pt}%
\definecolor{currentstroke}{rgb}{1.000000,1.000000,1.000000}%
\pgfsetstrokecolor{currentstroke}%
\pgfsetdash{}{0pt}%
\pgfpathmoveto{\pgfqpoint{4.160374in}{1.612242in}}%
\pgfpathcurveto{\pgfqpoint{4.167507in}{1.612242in}}{\pgfqpoint{4.174348in}{1.615076in}}{\pgfqpoint{4.179392in}{1.620119in}}%
\pgfpathcurveto{\pgfqpoint{4.184436in}{1.625163in}}{\pgfqpoint{4.187269in}{1.632005in}}{\pgfqpoint{4.187269in}{1.639137in}}%
\pgfpathcurveto{\pgfqpoint{4.187269in}{1.646270in}}{\pgfqpoint{4.184436in}{1.653112in}}{\pgfqpoint{4.179392in}{1.658156in}}%
\pgfpathcurveto{\pgfqpoint{4.174348in}{1.663199in}}{\pgfqpoint{4.167507in}{1.666033in}}{\pgfqpoint{4.160374in}{1.666033in}}%
\pgfpathcurveto{\pgfqpoint{4.153241in}{1.666033in}}{\pgfqpoint{4.146399in}{1.663199in}}{\pgfqpoint{4.141356in}{1.658156in}}%
\pgfpathcurveto{\pgfqpoint{4.136312in}{1.653112in}}{\pgfqpoint{4.133478in}{1.646270in}}{\pgfqpoint{4.133478in}{1.639137in}}%
\pgfpathcurveto{\pgfqpoint{4.133478in}{1.632005in}}{\pgfqpoint{4.136312in}{1.625163in}}{\pgfqpoint{4.141356in}{1.620119in}}%
\pgfpathcurveto{\pgfqpoint{4.146399in}{1.615076in}}{\pgfqpoint{4.153241in}{1.612242in}}{\pgfqpoint{4.160374in}{1.612242in}}%
\pgfpathclose%
\pgfusepath{stroke,fill}%
\end{pgfscope}%
\begin{pgfscope}%
\pgfpathrectangle{\pgfqpoint{2.867647in}{0.500000in}}{\pgfqpoint{1.764706in}{1.700000in}}%
\pgfusepath{clip}%
\pgfsetbuttcap%
\pgfsetroundjoin%
\definecolor{currentfill}{rgb}{0.951650,0.442241,0.302145}%
\pgfsetfillcolor{currentfill}%
\pgfsetlinewidth{0.311001pt}%
\definecolor{currentstroke}{rgb}{1.000000,1.000000,1.000000}%
\pgfsetstrokecolor{currentstroke}%
\pgfsetdash{}{0pt}%
\pgfpathmoveto{\pgfqpoint{3.937822in}{1.060974in}}%
\pgfpathcurveto{\pgfqpoint{3.944955in}{1.060974in}}{\pgfqpoint{3.951797in}{1.063807in}}{\pgfqpoint{3.956840in}{1.068851in}}%
\pgfpathcurveto{\pgfqpoint{3.961884in}{1.073895in}}{\pgfqpoint{3.964718in}{1.080736in}}{\pgfqpoint{3.964718in}{1.087869in}}%
\pgfpathcurveto{\pgfqpoint{3.964718in}{1.095002in}}{\pgfqpoint{3.961884in}{1.101844in}}{\pgfqpoint{3.956840in}{1.106887in}}%
\pgfpathcurveto{\pgfqpoint{3.951797in}{1.111931in}}{\pgfqpoint{3.944955in}{1.114765in}}{\pgfqpoint{3.937822in}{1.114765in}}%
\pgfpathcurveto{\pgfqpoint{3.930689in}{1.114765in}}{\pgfqpoint{3.923848in}{1.111931in}}{\pgfqpoint{3.918804in}{1.106887in}}%
\pgfpathcurveto{\pgfqpoint{3.913760in}{1.101844in}}{\pgfqpoint{3.910927in}{1.095002in}}{\pgfqpoint{3.910927in}{1.087869in}}%
\pgfpathcurveto{\pgfqpoint{3.910927in}{1.080736in}}{\pgfqpoint{3.913760in}{1.073895in}}{\pgfqpoint{3.918804in}{1.068851in}}%
\pgfpathcurveto{\pgfqpoint{3.923848in}{1.063807in}}{\pgfqpoint{3.930689in}{1.060974in}}{\pgfqpoint{3.937822in}{1.060974in}}%
\pgfpathclose%
\pgfusepath{stroke,fill}%
\end{pgfscope}%
\begin{pgfscope}%
\pgfpathrectangle{\pgfqpoint{2.867647in}{0.500000in}}{\pgfqpoint{1.764706in}{1.700000in}}%
\pgfusepath{clip}%
\pgfsetbuttcap%
\pgfsetroundjoin%
\definecolor{currentfill}{rgb}{0.964799,0.689101,0.537560}%
\pgfsetfillcolor{currentfill}%
\pgfsetlinewidth{0.311001pt}%
\definecolor{currentstroke}{rgb}{1.000000,1.000000,1.000000}%
\pgfsetstrokecolor{currentstroke}%
\pgfsetdash{}{0pt}%
\pgfpathmoveto{\pgfqpoint{4.019325in}{1.538266in}}%
\pgfpathcurveto{\pgfqpoint{4.026458in}{1.538266in}}{\pgfqpoint{4.033299in}{1.541100in}}{\pgfqpoint{4.038343in}{1.546144in}}%
\pgfpathcurveto{\pgfqpoint{4.043387in}{1.551187in}}{\pgfqpoint{4.046220in}{1.558029in}}{\pgfqpoint{4.046220in}{1.565162in}}%
\pgfpathcurveto{\pgfqpoint{4.046220in}{1.572295in}}{\pgfqpoint{4.043387in}{1.579136in}}{\pgfqpoint{4.038343in}{1.584180in}}%
\pgfpathcurveto{\pgfqpoint{4.033299in}{1.589224in}}{\pgfqpoint{4.026458in}{1.592057in}}{\pgfqpoint{4.019325in}{1.592057in}}%
\pgfpathcurveto{\pgfqpoint{4.012192in}{1.592057in}}{\pgfqpoint{4.005350in}{1.589224in}}{\pgfqpoint{4.000307in}{1.584180in}}%
\pgfpathcurveto{\pgfqpoint{3.995263in}{1.579136in}}{\pgfqpoint{3.992429in}{1.572295in}}{\pgfqpoint{3.992429in}{1.565162in}}%
\pgfpathcurveto{\pgfqpoint{3.992429in}{1.558029in}}{\pgfqpoint{3.995263in}{1.551187in}}{\pgfqpoint{4.000307in}{1.546144in}}%
\pgfpathcurveto{\pgfqpoint{4.005350in}{1.541100in}}{\pgfqpoint{4.012192in}{1.538266in}}{\pgfqpoint{4.019325in}{1.538266in}}%
\pgfpathclose%
\pgfusepath{stroke,fill}%
\end{pgfscope}%
\begin{pgfscope}%
\pgfpathrectangle{\pgfqpoint{2.867647in}{0.500000in}}{\pgfqpoint{1.764706in}{1.700000in}}%
\pgfusepath{clip}%
\pgfsetbuttcap%
\pgfsetroundjoin%
\definecolor{currentfill}{rgb}{0.974412,0.862387,0.780156}%
\pgfsetfillcolor{currentfill}%
\pgfsetlinewidth{0.311001pt}%
\definecolor{currentstroke}{rgb}{1.000000,1.000000,1.000000}%
\pgfsetstrokecolor{currentstroke}%
\pgfsetdash{}{0pt}%
\pgfpathmoveto{\pgfqpoint{4.105278in}{1.004093in}}%
\pgfpathcurveto{\pgfqpoint{4.112411in}{1.004093in}}{\pgfqpoint{4.119252in}{1.006927in}}{\pgfqpoint{4.124296in}{1.011971in}}%
\pgfpathcurveto{\pgfqpoint{4.129340in}{1.017014in}}{\pgfqpoint{4.132173in}{1.023856in}}{\pgfqpoint{4.132173in}{1.030989in}}%
\pgfpathcurveto{\pgfqpoint{4.132173in}{1.038122in}}{\pgfqpoint{4.129340in}{1.044963in}}{\pgfqpoint{4.124296in}{1.050007in}}%
\pgfpathcurveto{\pgfqpoint{4.119252in}{1.055051in}}{\pgfqpoint{4.112411in}{1.057885in}}{\pgfqpoint{4.105278in}{1.057885in}}%
\pgfpathcurveto{\pgfqpoint{4.098145in}{1.057885in}}{\pgfqpoint{4.091303in}{1.055051in}}{\pgfqpoint{4.086260in}{1.050007in}}%
\pgfpathcurveto{\pgfqpoint{4.081216in}{1.044963in}}{\pgfqpoint{4.078382in}{1.038122in}}{\pgfqpoint{4.078382in}{1.030989in}}%
\pgfpathcurveto{\pgfqpoint{4.078382in}{1.023856in}}{\pgfqpoint{4.081216in}{1.017014in}}{\pgfqpoint{4.086260in}{1.011971in}}%
\pgfpathcurveto{\pgfqpoint{4.091303in}{1.006927in}}{\pgfqpoint{4.098145in}{1.004093in}}{\pgfqpoint{4.105278in}{1.004093in}}%
\pgfpathclose%
\pgfusepath{stroke,fill}%
\end{pgfscope}%
\begin{pgfscope}%
\pgfpathrectangle{\pgfqpoint{2.867647in}{0.500000in}}{\pgfqpoint{1.764706in}{1.700000in}}%
\pgfusepath{clip}%
\pgfsetbuttcap%
\pgfsetroundjoin%
\definecolor{currentfill}{rgb}{0.971202,0.827364,0.728520}%
\pgfsetfillcolor{currentfill}%
\pgfsetlinewidth{0.311001pt}%
\definecolor{currentstroke}{rgb}{1.000000,1.000000,1.000000}%
\pgfsetstrokecolor{currentstroke}%
\pgfsetdash{}{0pt}%
\pgfpathmoveto{\pgfqpoint{4.209982in}{1.581774in}}%
\pgfpathcurveto{\pgfqpoint{4.217114in}{1.581774in}}{\pgfqpoint{4.223956in}{1.584608in}}{\pgfqpoint{4.229000in}{1.589652in}}%
\pgfpathcurveto{\pgfqpoint{4.234043in}{1.594695in}}{\pgfqpoint{4.236877in}{1.601537in}}{\pgfqpoint{4.236877in}{1.608670in}}%
\pgfpathcurveto{\pgfqpoint{4.236877in}{1.615803in}}{\pgfqpoint{4.234043in}{1.622644in}}{\pgfqpoint{4.229000in}{1.627688in}}%
\pgfpathcurveto{\pgfqpoint{4.223956in}{1.632732in}}{\pgfqpoint{4.217114in}{1.635565in}}{\pgfqpoint{4.209982in}{1.635565in}}%
\pgfpathcurveto{\pgfqpoint{4.202849in}{1.635565in}}{\pgfqpoint{4.196007in}{1.632732in}}{\pgfqpoint{4.190963in}{1.627688in}}%
\pgfpathcurveto{\pgfqpoint{4.185920in}{1.622644in}}{\pgfqpoint{4.183086in}{1.615803in}}{\pgfqpoint{4.183086in}{1.608670in}}%
\pgfpathcurveto{\pgfqpoint{4.183086in}{1.601537in}}{\pgfqpoint{4.185920in}{1.594695in}}{\pgfqpoint{4.190963in}{1.589652in}}%
\pgfpathcurveto{\pgfqpoint{4.196007in}{1.584608in}}{\pgfqpoint{4.202849in}{1.581774in}}{\pgfqpoint{4.209982in}{1.581774in}}%
\pgfpathclose%
\pgfusepath{stroke,fill}%
\end{pgfscope}%
\begin{pgfscope}%
\pgfpathrectangle{\pgfqpoint{2.867647in}{0.500000in}}{\pgfqpoint{1.764706in}{1.700000in}}%
\pgfusepath{clip}%
\pgfsetbuttcap%
\pgfsetroundjoin%
\definecolor{currentfill}{rgb}{0.977657,0.891500,0.822809}%
\pgfsetfillcolor{currentfill}%
\pgfsetlinewidth{0.311001pt}%
\definecolor{currentstroke}{rgb}{1.000000,1.000000,1.000000}%
\pgfsetstrokecolor{currentstroke}%
\pgfsetdash{}{0pt}%
\pgfpathmoveto{\pgfqpoint{4.235213in}{1.329042in}}%
\pgfpathcurveto{\pgfqpoint{4.242345in}{1.329042in}}{\pgfqpoint{4.249187in}{1.331876in}}{\pgfqpoint{4.254231in}{1.336920in}}%
\pgfpathcurveto{\pgfqpoint{4.259274in}{1.341964in}}{\pgfqpoint{4.262108in}{1.348805in}}{\pgfqpoint{4.262108in}{1.355938in}}%
\pgfpathcurveto{\pgfqpoint{4.262108in}{1.363071in}}{\pgfqpoint{4.259274in}{1.369912in}}{\pgfqpoint{4.254231in}{1.374956in}}%
\pgfpathcurveto{\pgfqpoint{4.249187in}{1.380000in}}{\pgfqpoint{4.242345in}{1.382834in}}{\pgfqpoint{4.235213in}{1.382834in}}%
\pgfpathcurveto{\pgfqpoint{4.228080in}{1.382834in}}{\pgfqpoint{4.221238in}{1.380000in}}{\pgfqpoint{4.216195in}{1.374956in}}%
\pgfpathcurveto{\pgfqpoint{4.211151in}{1.369912in}}{\pgfqpoint{4.208317in}{1.363071in}}{\pgfqpoint{4.208317in}{1.355938in}}%
\pgfpathcurveto{\pgfqpoint{4.208317in}{1.348805in}}{\pgfqpoint{4.211151in}{1.341964in}}{\pgfqpoint{4.216195in}{1.336920in}}%
\pgfpathcurveto{\pgfqpoint{4.221238in}{1.331876in}}{\pgfqpoint{4.228080in}{1.329042in}}{\pgfqpoint{4.235213in}{1.329042in}}%
\pgfpathclose%
\pgfusepath{stroke,fill}%
\end{pgfscope}%
\begin{pgfscope}%
\pgfpathrectangle{\pgfqpoint{2.867647in}{0.500000in}}{\pgfqpoint{1.764706in}{1.700000in}}%
\pgfusepath{clip}%
\pgfsetbuttcap%
\pgfsetroundjoin%
\definecolor{currentfill}{rgb}{0.981377,0.920617,0.865369}%
\pgfsetfillcolor{currentfill}%
\pgfsetlinewidth{0.311001pt}%
\definecolor{currentstroke}{rgb}{1.000000,1.000000,1.000000}%
\pgfsetstrokecolor{currentstroke}%
\pgfsetdash{}{0pt}%
\pgfpathmoveto{\pgfqpoint{4.187032in}{1.187194in}}%
\pgfpathcurveto{\pgfqpoint{4.194165in}{1.187194in}}{\pgfqpoint{4.201007in}{1.190028in}}{\pgfqpoint{4.206050in}{1.195072in}}%
\pgfpathcurveto{\pgfqpoint{4.211094in}{1.200115in}}{\pgfqpoint{4.213928in}{1.206957in}}{\pgfqpoint{4.213928in}{1.214090in}}%
\pgfpathcurveto{\pgfqpoint{4.213928in}{1.221222in}}{\pgfqpoint{4.211094in}{1.228064in}}{\pgfqpoint{4.206050in}{1.233108in}}%
\pgfpathcurveto{\pgfqpoint{4.201007in}{1.238151in}}{\pgfqpoint{4.194165in}{1.240985in}}{\pgfqpoint{4.187032in}{1.240985in}}%
\pgfpathcurveto{\pgfqpoint{4.179899in}{1.240985in}}{\pgfqpoint{4.173058in}{1.238151in}}{\pgfqpoint{4.168014in}{1.233108in}}%
\pgfpathcurveto{\pgfqpoint{4.162970in}{1.228064in}}{\pgfqpoint{4.160136in}{1.221222in}}{\pgfqpoint{4.160136in}{1.214090in}}%
\pgfpathcurveto{\pgfqpoint{4.160136in}{1.206957in}}{\pgfqpoint{4.162970in}{1.200115in}}{\pgfqpoint{4.168014in}{1.195072in}}%
\pgfpathcurveto{\pgfqpoint{4.173058in}{1.190028in}}{\pgfqpoint{4.179899in}{1.187194in}}{\pgfqpoint{4.187032in}{1.187194in}}%
\pgfpathclose%
\pgfusepath{stroke,fill}%
\end{pgfscope}%
\begin{pgfscope}%
\pgfpathrectangle{\pgfqpoint{2.867647in}{0.500000in}}{\pgfqpoint{1.764706in}{1.700000in}}%
\pgfusepath{clip}%
\pgfsetbuttcap%
\pgfsetroundjoin%
\definecolor{currentfill}{rgb}{0.971202,0.827364,0.728520}%
\pgfsetfillcolor{currentfill}%
\pgfsetlinewidth{0.311001pt}%
\definecolor{currentstroke}{rgb}{1.000000,1.000000,1.000000}%
\pgfsetstrokecolor{currentstroke}%
\pgfsetdash{}{0pt}%
\pgfpathmoveto{\pgfqpoint{4.053672in}{1.653352in}}%
\pgfpathcurveto{\pgfqpoint{4.060805in}{1.653352in}}{\pgfqpoint{4.067647in}{1.656185in}}{\pgfqpoint{4.072690in}{1.661229in}}%
\pgfpathcurveto{\pgfqpoint{4.077734in}{1.666273in}}{\pgfqpoint{4.080568in}{1.673114in}}{\pgfqpoint{4.080568in}{1.680247in}}%
\pgfpathcurveto{\pgfqpoint{4.080568in}{1.687380in}}{\pgfqpoint{4.077734in}{1.694222in}}{\pgfqpoint{4.072690in}{1.699265in}}%
\pgfpathcurveto{\pgfqpoint{4.067647in}{1.704309in}}{\pgfqpoint{4.060805in}{1.707143in}}{\pgfqpoint{4.053672in}{1.707143in}}%
\pgfpathcurveto{\pgfqpoint{4.046539in}{1.707143in}}{\pgfqpoint{4.039698in}{1.704309in}}{\pgfqpoint{4.034654in}{1.699265in}}%
\pgfpathcurveto{\pgfqpoint{4.029610in}{1.694222in}}{\pgfqpoint{4.026777in}{1.687380in}}{\pgfqpoint{4.026777in}{1.680247in}}%
\pgfpathcurveto{\pgfqpoint{4.026777in}{1.673114in}}{\pgfqpoint{4.029610in}{1.666273in}}{\pgfqpoint{4.034654in}{1.661229in}}%
\pgfpathcurveto{\pgfqpoint{4.039698in}{1.656185in}}{\pgfqpoint{4.046539in}{1.653352in}}{\pgfqpoint{4.053672in}{1.653352in}}%
\pgfpathclose%
\pgfusepath{stroke,fill}%
\end{pgfscope}%
\begin{pgfscope}%
\pgfpathrectangle{\pgfqpoint{2.867647in}{0.500000in}}{\pgfqpoint{1.764706in}{1.700000in}}%
\pgfusepath{clip}%
\pgfsetbuttcap%
\pgfsetroundjoin%
\definecolor{currentfill}{rgb}{0.974412,0.862387,0.780156}%
\pgfsetfillcolor{currentfill}%
\pgfsetlinewidth{0.311001pt}%
\definecolor{currentstroke}{rgb}{1.000000,1.000000,1.000000}%
\pgfsetstrokecolor{currentstroke}%
\pgfsetdash{}{0pt}%
\pgfpathmoveto{\pgfqpoint{4.124512in}{1.390234in}}%
\pgfpathcurveto{\pgfqpoint{4.131645in}{1.390234in}}{\pgfqpoint{4.138487in}{1.393067in}}{\pgfqpoint{4.143530in}{1.398111in}}%
\pgfpathcurveto{\pgfqpoint{4.148574in}{1.403155in}}{\pgfqpoint{4.151408in}{1.409996in}}{\pgfqpoint{4.151408in}{1.417129in}}%
\pgfpathcurveto{\pgfqpoint{4.151408in}{1.424262in}}{\pgfqpoint{4.148574in}{1.431104in}}{\pgfqpoint{4.143530in}{1.436147in}}%
\pgfpathcurveto{\pgfqpoint{4.138487in}{1.441191in}}{\pgfqpoint{4.131645in}{1.444025in}}{\pgfqpoint{4.124512in}{1.444025in}}%
\pgfpathcurveto{\pgfqpoint{4.117379in}{1.444025in}}{\pgfqpoint{4.110538in}{1.441191in}}{\pgfqpoint{4.105494in}{1.436147in}}%
\pgfpathcurveto{\pgfqpoint{4.100450in}{1.431104in}}{\pgfqpoint{4.097617in}{1.424262in}}{\pgfqpoint{4.097617in}{1.417129in}}%
\pgfpathcurveto{\pgfqpoint{4.097617in}{1.409996in}}{\pgfqpoint{4.100450in}{1.403155in}}{\pgfqpoint{4.105494in}{1.398111in}}%
\pgfpathcurveto{\pgfqpoint{4.110538in}{1.393067in}}{\pgfqpoint{4.117379in}{1.390234in}}{\pgfqpoint{4.124512in}{1.390234in}}%
\pgfpathclose%
\pgfusepath{stroke,fill}%
\end{pgfscope}%
\begin{pgfscope}%
\pgfpathrectangle{\pgfqpoint{2.867647in}{0.500000in}}{\pgfqpoint{1.764706in}{1.700000in}}%
\pgfusepath{clip}%
\pgfsetbuttcap%
\pgfsetroundjoin%
\definecolor{currentfill}{rgb}{0.964173,0.657587,0.500469}%
\pgfsetfillcolor{currentfill}%
\pgfsetlinewidth{0.311001pt}%
\definecolor{currentstroke}{rgb}{1.000000,1.000000,1.000000}%
\pgfsetstrokecolor{currentstroke}%
\pgfsetdash{}{0pt}%
\pgfpathmoveto{\pgfqpoint{4.307722in}{1.186381in}}%
\pgfpathcurveto{\pgfqpoint{4.314854in}{1.186381in}}{\pgfqpoint{4.321696in}{1.189215in}}{\pgfqpoint{4.326740in}{1.194258in}}%
\pgfpathcurveto{\pgfqpoint{4.331783in}{1.199302in}}{\pgfqpoint{4.334617in}{1.206144in}}{\pgfqpoint{4.334617in}{1.213276in}}%
\pgfpathcurveto{\pgfqpoint{4.334617in}{1.220409in}}{\pgfqpoint{4.331783in}{1.227251in}}{\pgfqpoint{4.326740in}{1.232295in}}%
\pgfpathcurveto{\pgfqpoint{4.321696in}{1.237338in}}{\pgfqpoint{4.314854in}{1.240172in}}{\pgfqpoint{4.307722in}{1.240172in}}%
\pgfpathcurveto{\pgfqpoint{4.300589in}{1.240172in}}{\pgfqpoint{4.293747in}{1.237338in}}{\pgfqpoint{4.288703in}{1.232295in}}%
\pgfpathcurveto{\pgfqpoint{4.283660in}{1.227251in}}{\pgfqpoint{4.280826in}{1.220409in}}{\pgfqpoint{4.280826in}{1.213276in}}%
\pgfpathcurveto{\pgfqpoint{4.280826in}{1.206144in}}{\pgfqpoint{4.283660in}{1.199302in}}{\pgfqpoint{4.288703in}{1.194258in}}%
\pgfpathcurveto{\pgfqpoint{4.293747in}{1.189215in}}{\pgfqpoint{4.300589in}{1.186381in}}{\pgfqpoint{4.307722in}{1.186381in}}%
\pgfpathclose%
\pgfusepath{stroke,fill}%
\end{pgfscope}%
\begin{pgfscope}%
\pgfpathrectangle{\pgfqpoint{2.867647in}{0.500000in}}{\pgfqpoint{1.764706in}{1.700000in}}%
\pgfusepath{clip}%
\pgfsetbuttcap%
\pgfsetroundjoin%
\definecolor{currentfill}{rgb}{0.979891,0.908948,0.848279}%
\pgfsetfillcolor{currentfill}%
\pgfsetlinewidth{0.311001pt}%
\definecolor{currentstroke}{rgb}{1.000000,1.000000,1.000000}%
\pgfsetstrokecolor{currentstroke}%
\pgfsetdash{}{0pt}%
\pgfpathmoveto{\pgfqpoint{4.208899in}{1.387267in}}%
\pgfpathcurveto{\pgfqpoint{4.216032in}{1.387267in}}{\pgfqpoint{4.222873in}{1.390101in}}{\pgfqpoint{4.227917in}{1.395145in}}%
\pgfpathcurveto{\pgfqpoint{4.232961in}{1.400188in}}{\pgfqpoint{4.235795in}{1.407030in}}{\pgfqpoint{4.235795in}{1.414163in}}%
\pgfpathcurveto{\pgfqpoint{4.235795in}{1.421296in}}{\pgfqpoint{4.232961in}{1.428137in}}{\pgfqpoint{4.227917in}{1.433181in}}%
\pgfpathcurveto{\pgfqpoint{4.222873in}{1.438225in}}{\pgfqpoint{4.216032in}{1.441058in}}{\pgfqpoint{4.208899in}{1.441058in}}%
\pgfpathcurveto{\pgfqpoint{4.201766in}{1.441058in}}{\pgfqpoint{4.194924in}{1.438225in}}{\pgfqpoint{4.189881in}{1.433181in}}%
\pgfpathcurveto{\pgfqpoint{4.184837in}{1.428137in}}{\pgfqpoint{4.182003in}{1.421296in}}{\pgfqpoint{4.182003in}{1.414163in}}%
\pgfpathcurveto{\pgfqpoint{4.182003in}{1.407030in}}{\pgfqpoint{4.184837in}{1.400188in}}{\pgfqpoint{4.189881in}{1.395145in}}%
\pgfpathcurveto{\pgfqpoint{4.194924in}{1.390101in}}{\pgfqpoint{4.201766in}{1.387267in}}{\pgfqpoint{4.208899in}{1.387267in}}%
\pgfpathclose%
\pgfusepath{stroke,fill}%
\end{pgfscope}%
\begin{pgfscope}%
\pgfpathrectangle{\pgfqpoint{2.867647in}{0.500000in}}{\pgfqpoint{1.764706in}{1.700000in}}%
\pgfusepath{clip}%
\pgfsetbuttcap%
\pgfsetroundjoin%
\definecolor{currentfill}{rgb}{0.975644,0.874038,0.797253}%
\pgfsetfillcolor{currentfill}%
\pgfsetlinewidth{0.311001pt}%
\definecolor{currentstroke}{rgb}{1.000000,1.000000,1.000000}%
\pgfsetstrokecolor{currentstroke}%
\pgfsetdash{}{0pt}%
\pgfpathmoveto{\pgfqpoint{4.090193in}{1.559655in}}%
\pgfpathcurveto{\pgfqpoint{4.097326in}{1.559655in}}{\pgfqpoint{4.104167in}{1.562489in}}{\pgfqpoint{4.109211in}{1.567533in}}%
\pgfpathcurveto{\pgfqpoint{4.114255in}{1.572577in}}{\pgfqpoint{4.117089in}{1.579418in}}{\pgfqpoint{4.117089in}{1.586551in}}%
\pgfpathcurveto{\pgfqpoint{4.117089in}{1.593684in}}{\pgfqpoint{4.114255in}{1.600526in}}{\pgfqpoint{4.109211in}{1.605569in}}%
\pgfpathcurveto{\pgfqpoint{4.104167in}{1.610613in}}{\pgfqpoint{4.097326in}{1.613447in}}{\pgfqpoint{4.090193in}{1.613447in}}%
\pgfpathcurveto{\pgfqpoint{4.083060in}{1.613447in}}{\pgfqpoint{4.076218in}{1.610613in}}{\pgfqpoint{4.071175in}{1.605569in}}%
\pgfpathcurveto{\pgfqpoint{4.066131in}{1.600526in}}{\pgfqpoint{4.063297in}{1.593684in}}{\pgfqpoint{4.063297in}{1.586551in}}%
\pgfpathcurveto{\pgfqpoint{4.063297in}{1.579418in}}{\pgfqpoint{4.066131in}{1.572577in}}{\pgfqpoint{4.071175in}{1.567533in}}%
\pgfpathcurveto{\pgfqpoint{4.076218in}{1.562489in}}{\pgfqpoint{4.083060in}{1.559655in}}{\pgfqpoint{4.090193in}{1.559655in}}%
\pgfpathclose%
\pgfusepath{stroke,fill}%
\end{pgfscope}%
\begin{pgfscope}%
\pgfpathrectangle{\pgfqpoint{2.867647in}{0.500000in}}{\pgfqpoint{1.764706in}{1.700000in}}%
\pgfusepath{clip}%
\pgfsetbuttcap%
\pgfsetroundjoin%
\definecolor{currentfill}{rgb}{0.972201,0.839051,0.745789}%
\pgfsetfillcolor{currentfill}%
\pgfsetlinewidth{0.311001pt}%
\definecolor{currentstroke}{rgb}{1.000000,1.000000,1.000000}%
\pgfsetstrokecolor{currentstroke}%
\pgfsetdash{}{0pt}%
\pgfpathmoveto{\pgfqpoint{4.106305in}{1.212891in}}%
\pgfpathcurveto{\pgfqpoint{4.113438in}{1.212891in}}{\pgfqpoint{4.120280in}{1.215724in}}{\pgfqpoint{4.125324in}{1.220768in}}%
\pgfpathcurveto{\pgfqpoint{4.130367in}{1.225812in}}{\pgfqpoint{4.133201in}{1.232653in}}{\pgfqpoint{4.133201in}{1.239786in}}%
\pgfpathcurveto{\pgfqpoint{4.133201in}{1.246919in}}{\pgfqpoint{4.130367in}{1.253761in}}{\pgfqpoint{4.125324in}{1.258804in}}%
\pgfpathcurveto{\pgfqpoint{4.120280in}{1.263848in}}{\pgfqpoint{4.113438in}{1.266682in}}{\pgfqpoint{4.106305in}{1.266682in}}%
\pgfpathcurveto{\pgfqpoint{4.099173in}{1.266682in}}{\pgfqpoint{4.092331in}{1.263848in}}{\pgfqpoint{4.087287in}{1.258804in}}%
\pgfpathcurveto{\pgfqpoint{4.082244in}{1.253761in}}{\pgfqpoint{4.079410in}{1.246919in}}{\pgfqpoint{4.079410in}{1.239786in}}%
\pgfpathcurveto{\pgfqpoint{4.079410in}{1.232653in}}{\pgfqpoint{4.082244in}{1.225812in}}{\pgfqpoint{4.087287in}{1.220768in}}%
\pgfpathcurveto{\pgfqpoint{4.092331in}{1.215724in}}{\pgfqpoint{4.099173in}{1.212891in}}{\pgfqpoint{4.106305in}{1.212891in}}%
\pgfpathclose%
\pgfusepath{stroke,fill}%
\end{pgfscope}%
\begin{pgfscope}%
\pgfpathrectangle{\pgfqpoint{2.867647in}{0.500000in}}{\pgfqpoint{1.764706in}{1.700000in}}%
\pgfusepath{clip}%
\pgfsetbuttcap%
\pgfsetroundjoin%
\definecolor{currentfill}{rgb}{0.964920,0.695342,0.545192}%
\pgfsetfillcolor{currentfill}%
\pgfsetlinewidth{0.311001pt}%
\definecolor{currentstroke}{rgb}{1.000000,1.000000,1.000000}%
\pgfsetstrokecolor{currentstroke}%
\pgfsetdash{}{0pt}%
\pgfpathmoveto{\pgfqpoint{4.071280in}{1.343840in}}%
\pgfpathcurveto{\pgfqpoint{4.078412in}{1.343840in}}{\pgfqpoint{4.085254in}{1.346674in}}{\pgfqpoint{4.090298in}{1.351718in}}%
\pgfpathcurveto{\pgfqpoint{4.095341in}{1.356762in}}{\pgfqpoint{4.098175in}{1.363603in}}{\pgfqpoint{4.098175in}{1.370736in}}%
\pgfpathcurveto{\pgfqpoint{4.098175in}{1.377869in}}{\pgfqpoint{4.095341in}{1.384711in}}{\pgfqpoint{4.090298in}{1.389754in}}%
\pgfpathcurveto{\pgfqpoint{4.085254in}{1.394798in}}{\pgfqpoint{4.078412in}{1.397632in}}{\pgfqpoint{4.071280in}{1.397632in}}%
\pgfpathcurveto{\pgfqpoint{4.064147in}{1.397632in}}{\pgfqpoint{4.057305in}{1.394798in}}{\pgfqpoint{4.052261in}{1.389754in}}%
\pgfpathcurveto{\pgfqpoint{4.047218in}{1.384711in}}{\pgfqpoint{4.044384in}{1.377869in}}{\pgfqpoint{4.044384in}{1.370736in}}%
\pgfpathcurveto{\pgfqpoint{4.044384in}{1.363603in}}{\pgfqpoint{4.047218in}{1.356762in}}{\pgfqpoint{4.052261in}{1.351718in}}%
\pgfpathcurveto{\pgfqpoint{4.057305in}{1.346674in}}{\pgfqpoint{4.064147in}{1.343840in}}{\pgfqpoint{4.071280in}{1.343840in}}%
\pgfpathclose%
\pgfusepath{stroke,fill}%
\end{pgfscope}%
\begin{pgfscope}%
\pgfpathrectangle{\pgfqpoint{2.867647in}{0.500000in}}{\pgfqpoint{1.764706in}{1.700000in}}%
\pgfusepath{clip}%
\pgfsetbuttcap%
\pgfsetroundjoin%
\definecolor{currentfill}{rgb}{0.979891,0.908948,0.848279}%
\pgfsetfillcolor{currentfill}%
\pgfsetlinewidth{0.311001pt}%
\definecolor{currentstroke}{rgb}{1.000000,1.000000,1.000000}%
\pgfsetstrokecolor{currentstroke}%
\pgfsetdash{}{0pt}%
\pgfpathmoveto{\pgfqpoint{4.147965in}{1.193964in}}%
\pgfpathcurveto{\pgfqpoint{4.155098in}{1.193964in}}{\pgfqpoint{4.161940in}{1.196798in}}{\pgfqpoint{4.166983in}{1.201842in}}%
\pgfpathcurveto{\pgfqpoint{4.172027in}{1.206886in}}{\pgfqpoint{4.174861in}{1.213727in}}{\pgfqpoint{4.174861in}{1.220860in}}%
\pgfpathcurveto{\pgfqpoint{4.174861in}{1.227993in}}{\pgfqpoint{4.172027in}{1.234835in}}{\pgfqpoint{4.166983in}{1.239878in}}%
\pgfpathcurveto{\pgfqpoint{4.161940in}{1.244922in}}{\pgfqpoint{4.155098in}{1.247756in}}{\pgfqpoint{4.147965in}{1.247756in}}%
\pgfpathcurveto{\pgfqpoint{4.140833in}{1.247756in}}{\pgfqpoint{4.133991in}{1.244922in}}{\pgfqpoint{4.128947in}{1.239878in}}%
\pgfpathcurveto{\pgfqpoint{4.123904in}{1.234835in}}{\pgfqpoint{4.121070in}{1.227993in}}{\pgfqpoint{4.121070in}{1.220860in}}%
\pgfpathcurveto{\pgfqpoint{4.121070in}{1.213727in}}{\pgfqpoint{4.123904in}{1.206886in}}{\pgfqpoint{4.128947in}{1.201842in}}%
\pgfpathcurveto{\pgfqpoint{4.133991in}{1.196798in}}{\pgfqpoint{4.140833in}{1.193964in}}{\pgfqpoint{4.147965in}{1.193964in}}%
\pgfpathclose%
\pgfusepath{stroke,fill}%
\end{pgfscope}%
\begin{pgfscope}%
\pgfpathrectangle{\pgfqpoint{2.867647in}{0.500000in}}{\pgfqpoint{1.764706in}{1.700000in}}%
\pgfusepath{clip}%
\pgfsetbuttcap%
\pgfsetroundjoin%
\definecolor{currentfill}{rgb}{0.960043,0.546576,0.387029}%
\pgfsetfillcolor{currentfill}%
\pgfsetlinewidth{0.311001pt}%
\definecolor{currentstroke}{rgb}{1.000000,1.000000,1.000000}%
\pgfsetstrokecolor{currentstroke}%
\pgfsetdash{}{0pt}%
\pgfpathmoveto{\pgfqpoint{4.308553in}{1.099610in}}%
\pgfpathcurveto{\pgfqpoint{4.315685in}{1.099610in}}{\pgfqpoint{4.322527in}{1.102444in}}{\pgfqpoint{4.327571in}{1.107488in}}%
\pgfpathcurveto{\pgfqpoint{4.332614in}{1.112531in}}{\pgfqpoint{4.335448in}{1.119373in}}{\pgfqpoint{4.335448in}{1.126506in}}%
\pgfpathcurveto{\pgfqpoint{4.335448in}{1.133639in}}{\pgfqpoint{4.332614in}{1.140480in}}{\pgfqpoint{4.327571in}{1.145524in}}%
\pgfpathcurveto{\pgfqpoint{4.322527in}{1.150568in}}{\pgfqpoint{4.315685in}{1.153402in}}{\pgfqpoint{4.308553in}{1.153402in}}%
\pgfpathcurveto{\pgfqpoint{4.301420in}{1.153402in}}{\pgfqpoint{4.294578in}{1.150568in}}{\pgfqpoint{4.289534in}{1.145524in}}%
\pgfpathcurveto{\pgfqpoint{4.284491in}{1.140480in}}{\pgfqpoint{4.281657in}{1.133639in}}{\pgfqpoint{4.281657in}{1.126506in}}%
\pgfpathcurveto{\pgfqpoint{4.281657in}{1.119373in}}{\pgfqpoint{4.284491in}{1.112531in}}{\pgfqpoint{4.289534in}{1.107488in}}%
\pgfpathcurveto{\pgfqpoint{4.294578in}{1.102444in}}{\pgfqpoint{4.301420in}{1.099610in}}{\pgfqpoint{4.308553in}{1.099610in}}%
\pgfpathclose%
\pgfusepath{stroke,fill}%
\end{pgfscope}%
\begin{pgfscope}%
\pgfpathrectangle{\pgfqpoint{2.867647in}{0.500000in}}{\pgfqpoint{1.764706in}{1.700000in}}%
\pgfusepath{clip}%
\pgfsetbuttcap%
\pgfsetroundjoin%
\definecolor{currentfill}{rgb}{0.979124,0.903132,0.839793}%
\pgfsetfillcolor{currentfill}%
\pgfsetlinewidth{0.311001pt}%
\definecolor{currentstroke}{rgb}{1.000000,1.000000,1.000000}%
\pgfsetstrokecolor{currentstroke}%
\pgfsetdash{}{0pt}%
\pgfpathmoveto{\pgfqpoint{4.227222in}{1.302836in}}%
\pgfpathcurveto{\pgfqpoint{4.234355in}{1.302836in}}{\pgfqpoint{4.241197in}{1.305670in}}{\pgfqpoint{4.246241in}{1.310713in}}%
\pgfpathcurveto{\pgfqpoint{4.251284in}{1.315757in}}{\pgfqpoint{4.254118in}{1.322599in}}{\pgfqpoint{4.254118in}{1.329731in}}%
\pgfpathcurveto{\pgfqpoint{4.254118in}{1.336864in}}{\pgfqpoint{4.251284in}{1.343706in}}{\pgfqpoint{4.246241in}{1.348749in}}%
\pgfpathcurveto{\pgfqpoint{4.241197in}{1.353793in}}{\pgfqpoint{4.234355in}{1.356627in}}{\pgfqpoint{4.227222in}{1.356627in}}%
\pgfpathcurveto{\pgfqpoint{4.220090in}{1.356627in}}{\pgfqpoint{4.213248in}{1.353793in}}{\pgfqpoint{4.208204in}{1.348749in}}%
\pgfpathcurveto{\pgfqpoint{4.203161in}{1.343706in}}{\pgfqpoint{4.200327in}{1.336864in}}{\pgfqpoint{4.200327in}{1.329731in}}%
\pgfpathcurveto{\pgfqpoint{4.200327in}{1.322599in}}{\pgfqpoint{4.203161in}{1.315757in}}{\pgfqpoint{4.208204in}{1.310713in}}%
\pgfpathcurveto{\pgfqpoint{4.213248in}{1.305670in}}{\pgfqpoint{4.220090in}{1.302836in}}{\pgfqpoint{4.227222in}{1.302836in}}%
\pgfpathclose%
\pgfusepath{stroke,fill}%
\end{pgfscope}%
\begin{pgfscope}%
\pgfpathrectangle{\pgfqpoint{2.867647in}{0.500000in}}{\pgfqpoint{1.764706in}{1.700000in}}%
\pgfusepath{clip}%
\pgfsetbuttcap%
\pgfsetroundjoin%
\definecolor{currentfill}{rgb}{0.981377,0.920617,0.865369}%
\pgfsetfillcolor{currentfill}%
\pgfsetlinewidth{0.311001pt}%
\definecolor{currentstroke}{rgb}{1.000000,1.000000,1.000000}%
\pgfsetstrokecolor{currentstroke}%
\pgfsetdash{}{0pt}%
\pgfpathmoveto{\pgfqpoint{4.181800in}{1.285264in}}%
\pgfpathcurveto{\pgfqpoint{4.188933in}{1.285264in}}{\pgfqpoint{4.195775in}{1.288098in}}{\pgfqpoint{4.200818in}{1.293142in}}%
\pgfpathcurveto{\pgfqpoint{4.205862in}{1.298185in}}{\pgfqpoint{4.208696in}{1.305027in}}{\pgfqpoint{4.208696in}{1.312160in}}%
\pgfpathcurveto{\pgfqpoint{4.208696in}{1.319293in}}{\pgfqpoint{4.205862in}{1.326134in}}{\pgfqpoint{4.200818in}{1.331178in}}%
\pgfpathcurveto{\pgfqpoint{4.195775in}{1.336222in}}{\pgfqpoint{4.188933in}{1.339056in}}{\pgfqpoint{4.181800in}{1.339056in}}%
\pgfpathcurveto{\pgfqpoint{4.174667in}{1.339056in}}{\pgfqpoint{4.167826in}{1.336222in}}{\pgfqpoint{4.162782in}{1.331178in}}%
\pgfpathcurveto{\pgfqpoint{4.157738in}{1.326134in}}{\pgfqpoint{4.154904in}{1.319293in}}{\pgfqpoint{4.154904in}{1.312160in}}%
\pgfpathcurveto{\pgfqpoint{4.154904in}{1.305027in}}{\pgfqpoint{4.157738in}{1.298185in}}{\pgfqpoint{4.162782in}{1.293142in}}%
\pgfpathcurveto{\pgfqpoint{4.167826in}{1.288098in}}{\pgfqpoint{4.174667in}{1.285264in}}{\pgfqpoint{4.181800in}{1.285264in}}%
\pgfpathclose%
\pgfusepath{stroke,fill}%
\end{pgfscope}%
\begin{pgfscope}%
\pgfpathrectangle{\pgfqpoint{2.867647in}{0.500000in}}{\pgfqpoint{1.764706in}{1.700000in}}%
\pgfusepath{clip}%
\pgfsetbuttcap%
\pgfsetroundjoin%
\definecolor{currentfill}{rgb}{0.968105,0.786346,0.667739}%
\pgfsetfillcolor{currentfill}%
\pgfsetlinewidth{0.311001pt}%
\definecolor{currentstroke}{rgb}{1.000000,1.000000,1.000000}%
\pgfsetstrokecolor{currentstroke}%
\pgfsetdash{}{0pt}%
\pgfpathmoveto{\pgfqpoint{4.030246in}{1.614249in}}%
\pgfpathcurveto{\pgfqpoint{4.037379in}{1.614249in}}{\pgfqpoint{4.044220in}{1.617083in}}{\pgfqpoint{4.049264in}{1.622127in}}%
\pgfpathcurveto{\pgfqpoint{4.054308in}{1.627170in}}{\pgfqpoint{4.057141in}{1.634012in}}{\pgfqpoint{4.057141in}{1.641145in}}%
\pgfpathcurveto{\pgfqpoint{4.057141in}{1.648278in}}{\pgfqpoint{4.054308in}{1.655119in}}{\pgfqpoint{4.049264in}{1.660163in}}%
\pgfpathcurveto{\pgfqpoint{4.044220in}{1.665207in}}{\pgfqpoint{4.037379in}{1.668041in}}{\pgfqpoint{4.030246in}{1.668041in}}%
\pgfpathcurveto{\pgfqpoint{4.023113in}{1.668041in}}{\pgfqpoint{4.016271in}{1.665207in}}{\pgfqpoint{4.011228in}{1.660163in}}%
\pgfpathcurveto{\pgfqpoint{4.006184in}{1.655119in}}{\pgfqpoint{4.003350in}{1.648278in}}{\pgfqpoint{4.003350in}{1.641145in}}%
\pgfpathcurveto{\pgfqpoint{4.003350in}{1.634012in}}{\pgfqpoint{4.006184in}{1.627170in}}{\pgfqpoint{4.011228in}{1.622127in}}%
\pgfpathcurveto{\pgfqpoint{4.016271in}{1.617083in}}{\pgfqpoint{4.023113in}{1.614249in}}{\pgfqpoint{4.030246in}{1.614249in}}%
\pgfpathclose%
\pgfusepath{stroke,fill}%
\end{pgfscope}%
\begin{pgfscope}%
\pgfpathrectangle{\pgfqpoint{2.867647in}{0.500000in}}{\pgfqpoint{1.764706in}{1.700000in}}%
\pgfusepath{clip}%
\pgfsetbuttcap%
\pgfsetroundjoin%
\definecolor{currentfill}{rgb}{0.946260,0.398132,0.274897}%
\pgfsetfillcolor{currentfill}%
\pgfsetlinewidth{0.311001pt}%
\definecolor{currentstroke}{rgb}{1.000000,1.000000,1.000000}%
\pgfsetstrokecolor{currentstroke}%
\pgfsetdash{}{0pt}%
\pgfpathmoveto{\pgfqpoint{3.879677in}{0.934242in}}%
\pgfpathcurveto{\pgfqpoint{3.886810in}{0.934242in}}{\pgfqpoint{3.893651in}{0.937076in}}{\pgfqpoint{3.898695in}{0.942120in}}%
\pgfpathcurveto{\pgfqpoint{3.903739in}{0.947164in}}{\pgfqpoint{3.906572in}{0.954005in}}{\pgfqpoint{3.906572in}{0.961138in}}%
\pgfpathcurveto{\pgfqpoint{3.906572in}{0.968271in}}{\pgfqpoint{3.903739in}{0.975113in}}{\pgfqpoint{3.898695in}{0.980156in}}%
\pgfpathcurveto{\pgfqpoint{3.893651in}{0.985200in}}{\pgfqpoint{3.886810in}{0.988034in}}{\pgfqpoint{3.879677in}{0.988034in}}%
\pgfpathcurveto{\pgfqpoint{3.872544in}{0.988034in}}{\pgfqpoint{3.865702in}{0.985200in}}{\pgfqpoint{3.860659in}{0.980156in}}%
\pgfpathcurveto{\pgfqpoint{3.855615in}{0.975113in}}{\pgfqpoint{3.852781in}{0.968271in}}{\pgfqpoint{3.852781in}{0.961138in}}%
\pgfpathcurveto{\pgfqpoint{3.852781in}{0.954005in}}{\pgfqpoint{3.855615in}{0.947164in}}{\pgfqpoint{3.860659in}{0.942120in}}%
\pgfpathcurveto{\pgfqpoint{3.865702in}{0.937076in}}{\pgfqpoint{3.872544in}{0.934242in}}{\pgfqpoint{3.879677in}{0.934242in}}%
\pgfpathclose%
\pgfusepath{stroke,fill}%
\end{pgfscope}%
\begin{pgfscope}%
\pgfpathrectangle{\pgfqpoint{2.867647in}{0.500000in}}{\pgfqpoint{1.764706in}{1.700000in}}%
\pgfusepath{clip}%
\pgfsetbuttcap%
\pgfsetroundjoin%
\definecolor{currentfill}{rgb}{0.973832,0.856556,0.771584}%
\pgfsetfillcolor{currentfill}%
\pgfsetlinewidth{0.311001pt}%
\definecolor{currentstroke}{rgb}{1.000000,1.000000,1.000000}%
\pgfsetstrokecolor{currentstroke}%
\pgfsetdash{}{0pt}%
\pgfpathmoveto{\pgfqpoint{4.125207in}{1.286213in}}%
\pgfpathcurveto{\pgfqpoint{4.132339in}{1.286213in}}{\pgfqpoint{4.139181in}{1.289047in}}{\pgfqpoint{4.144225in}{1.294090in}}%
\pgfpathcurveto{\pgfqpoint{4.149268in}{1.299134in}}{\pgfqpoint{4.152102in}{1.305976in}}{\pgfqpoint{4.152102in}{1.313109in}}%
\pgfpathcurveto{\pgfqpoint{4.152102in}{1.320241in}}{\pgfqpoint{4.149268in}{1.327083in}}{\pgfqpoint{4.144225in}{1.332127in}}%
\pgfpathcurveto{\pgfqpoint{4.139181in}{1.337170in}}{\pgfqpoint{4.132339in}{1.340004in}}{\pgfqpoint{4.125207in}{1.340004in}}%
\pgfpathcurveto{\pgfqpoint{4.118074in}{1.340004in}}{\pgfqpoint{4.111232in}{1.337170in}}{\pgfqpoint{4.106188in}{1.332127in}}%
\pgfpathcurveto{\pgfqpoint{4.101145in}{1.327083in}}{\pgfqpoint{4.098311in}{1.320241in}}{\pgfqpoint{4.098311in}{1.313109in}}%
\pgfpathcurveto{\pgfqpoint{4.098311in}{1.305976in}}{\pgfqpoint{4.101145in}{1.299134in}}{\pgfqpoint{4.106188in}{1.294090in}}%
\pgfpathcurveto{\pgfqpoint{4.111232in}{1.289047in}}{\pgfqpoint{4.118074in}{1.286213in}}{\pgfqpoint{4.125207in}{1.286213in}}%
\pgfpathclose%
\pgfusepath{stroke,fill}%
\end{pgfscope}%
\begin{pgfscope}%
\pgfpathrectangle{\pgfqpoint{2.867647in}{0.500000in}}{\pgfqpoint{1.764706in}{1.700000in}}%
\pgfusepath{clip}%
\pgfsetbuttcap%
\pgfsetroundjoin%
\definecolor{currentfill}{rgb}{0.959645,0.539840,0.380928}%
\pgfsetfillcolor{currentfill}%
\pgfsetlinewidth{0.311001pt}%
\definecolor{currentstroke}{rgb}{1.000000,1.000000,1.000000}%
\pgfsetstrokecolor{currentstroke}%
\pgfsetdash{}{0pt}%
\pgfpathmoveto{\pgfqpoint{4.024529in}{1.216757in}}%
\pgfpathcurveto{\pgfqpoint{4.031662in}{1.216757in}}{\pgfqpoint{4.038503in}{1.219591in}}{\pgfqpoint{4.043547in}{1.224634in}}%
\pgfpathcurveto{\pgfqpoint{4.048591in}{1.229678in}}{\pgfqpoint{4.051425in}{1.236520in}}{\pgfqpoint{4.051425in}{1.243653in}}%
\pgfpathcurveto{\pgfqpoint{4.051425in}{1.250785in}}{\pgfqpoint{4.048591in}{1.257627in}}{\pgfqpoint{4.043547in}{1.262671in}}%
\pgfpathcurveto{\pgfqpoint{4.038503in}{1.267714in}}{\pgfqpoint{4.031662in}{1.270548in}}{\pgfqpoint{4.024529in}{1.270548in}}%
\pgfpathcurveto{\pgfqpoint{4.017396in}{1.270548in}}{\pgfqpoint{4.010554in}{1.267714in}}{\pgfqpoint{4.005511in}{1.262671in}}%
\pgfpathcurveto{\pgfqpoint{4.000467in}{1.257627in}}{\pgfqpoint{3.997633in}{1.250785in}}{\pgfqpoint{3.997633in}{1.243653in}}%
\pgfpathcurveto{\pgfqpoint{3.997633in}{1.236520in}}{\pgfqpoint{4.000467in}{1.229678in}}{\pgfqpoint{4.005511in}{1.224634in}}%
\pgfpathcurveto{\pgfqpoint{4.010554in}{1.219591in}}{\pgfqpoint{4.017396in}{1.216757in}}{\pgfqpoint{4.024529in}{1.216757in}}%
\pgfpathclose%
\pgfusepath{stroke,fill}%
\end{pgfscope}%
\begin{pgfscope}%
\pgfpathrectangle{\pgfqpoint{2.867647in}{0.500000in}}{\pgfqpoint{1.764706in}{1.700000in}}%
\pgfusepath{clip}%
\pgfsetbuttcap%
\pgfsetroundjoin%
\definecolor{currentfill}{rgb}{0.965302,0.713942,0.568499}%
\pgfsetfillcolor{currentfill}%
\pgfsetlinewidth{0.311001pt}%
\definecolor{currentstroke}{rgb}{1.000000,1.000000,1.000000}%
\pgfsetstrokecolor{currentstroke}%
\pgfsetdash{}{0pt}%
\pgfpathmoveto{\pgfqpoint{4.091484in}{0.897594in}}%
\pgfpathcurveto{\pgfqpoint{4.098617in}{0.897594in}}{\pgfqpoint{4.105459in}{0.900428in}}{\pgfqpoint{4.110502in}{0.905472in}}%
\pgfpathcurveto{\pgfqpoint{4.115546in}{0.910515in}}{\pgfqpoint{4.118380in}{0.917357in}}{\pgfqpoint{4.118380in}{0.924490in}}%
\pgfpathcurveto{\pgfqpoint{4.118380in}{0.931623in}}{\pgfqpoint{4.115546in}{0.938464in}}{\pgfqpoint{4.110502in}{0.943508in}}%
\pgfpathcurveto{\pgfqpoint{4.105459in}{0.948552in}}{\pgfqpoint{4.098617in}{0.951386in}}{\pgfqpoint{4.091484in}{0.951386in}}%
\pgfpathcurveto{\pgfqpoint{4.084351in}{0.951386in}}{\pgfqpoint{4.077510in}{0.948552in}}{\pgfqpoint{4.072466in}{0.943508in}}%
\pgfpathcurveto{\pgfqpoint{4.067422in}{0.938464in}}{\pgfqpoint{4.064589in}{0.931623in}}{\pgfqpoint{4.064589in}{0.924490in}}%
\pgfpathcurveto{\pgfqpoint{4.064589in}{0.917357in}}{\pgfqpoint{4.067422in}{0.910515in}}{\pgfqpoint{4.072466in}{0.905472in}}%
\pgfpathcurveto{\pgfqpoint{4.077510in}{0.900428in}}{\pgfqpoint{4.084351in}{0.897594in}}{\pgfqpoint{4.091484in}{0.897594in}}%
\pgfpathclose%
\pgfusepath{stroke,fill}%
\end{pgfscope}%
\begin{pgfscope}%
\pgfpathrectangle{\pgfqpoint{2.867647in}{0.500000in}}{\pgfqpoint{1.764706in}{1.700000in}}%
\pgfusepath{clip}%
\pgfsetbuttcap%
\pgfsetroundjoin%
\definecolor{currentfill}{rgb}{0.980678,0.914765,0.856766}%
\pgfsetfillcolor{currentfill}%
\pgfsetlinewidth{0.311001pt}%
\definecolor{currentstroke}{rgb}{1.000000,1.000000,1.000000}%
\pgfsetstrokecolor{currentstroke}%
\pgfsetdash{}{0pt}%
\pgfpathmoveto{\pgfqpoint{4.185669in}{1.410184in}}%
\pgfpathcurveto{\pgfqpoint{4.192801in}{1.410184in}}{\pgfqpoint{4.199643in}{1.413018in}}{\pgfqpoint{4.204687in}{1.418061in}}%
\pgfpathcurveto{\pgfqpoint{4.209730in}{1.423105in}}{\pgfqpoint{4.212564in}{1.429947in}}{\pgfqpoint{4.212564in}{1.437080in}}%
\pgfpathcurveto{\pgfqpoint{4.212564in}{1.444212in}}{\pgfqpoint{4.209730in}{1.451054in}}{\pgfqpoint{4.204687in}{1.456098in}}%
\pgfpathcurveto{\pgfqpoint{4.199643in}{1.461141in}}{\pgfqpoint{4.192801in}{1.463975in}}{\pgfqpoint{4.185669in}{1.463975in}}%
\pgfpathcurveto{\pgfqpoint{4.178536in}{1.463975in}}{\pgfqpoint{4.171694in}{1.461141in}}{\pgfqpoint{4.166650in}{1.456098in}}%
\pgfpathcurveto{\pgfqpoint{4.161607in}{1.451054in}}{\pgfqpoint{4.158773in}{1.444212in}}{\pgfqpoint{4.158773in}{1.437080in}}%
\pgfpathcurveto{\pgfqpoint{4.158773in}{1.429947in}}{\pgfqpoint{4.161607in}{1.423105in}}{\pgfqpoint{4.166650in}{1.418061in}}%
\pgfpathcurveto{\pgfqpoint{4.171694in}{1.413018in}}{\pgfqpoint{4.178536in}{1.410184in}}{\pgfqpoint{4.185669in}{1.410184in}}%
\pgfpathclose%
\pgfusepath{stroke,fill}%
\end{pgfscope}%
\begin{pgfscope}%
\pgfpathrectangle{\pgfqpoint{2.867647in}{0.500000in}}{\pgfqpoint{1.764706in}{1.700000in}}%
\pgfusepath{clip}%
\pgfsetbuttcap%
\pgfsetroundjoin%
\definecolor{currentfill}{rgb}{0.963728,0.638439,0.479050}%
\pgfsetfillcolor{currentfill}%
\pgfsetlinewidth{0.311001pt}%
\definecolor{currentstroke}{rgb}{1.000000,1.000000,1.000000}%
\pgfsetstrokecolor{currentstroke}%
\pgfsetdash{}{0pt}%
\pgfpathmoveto{\pgfqpoint{4.198970in}{0.942693in}}%
\pgfpathcurveto{\pgfqpoint{4.206102in}{0.942693in}}{\pgfqpoint{4.212944in}{0.945526in}}{\pgfqpoint{4.217988in}{0.950570in}}%
\pgfpathcurveto{\pgfqpoint{4.223031in}{0.955614in}}{\pgfqpoint{4.225865in}{0.962455in}}{\pgfqpoint{4.225865in}{0.969588in}}%
\pgfpathcurveto{\pgfqpoint{4.225865in}{0.976721in}}{\pgfqpoint{4.223031in}{0.983563in}}{\pgfqpoint{4.217988in}{0.988606in}}%
\pgfpathcurveto{\pgfqpoint{4.212944in}{0.993650in}}{\pgfqpoint{4.206102in}{0.996484in}}{\pgfqpoint{4.198970in}{0.996484in}}%
\pgfpathcurveto{\pgfqpoint{4.191837in}{0.996484in}}{\pgfqpoint{4.184995in}{0.993650in}}{\pgfqpoint{4.179952in}{0.988606in}}%
\pgfpathcurveto{\pgfqpoint{4.174908in}{0.983563in}}{\pgfqpoint{4.172074in}{0.976721in}}{\pgfqpoint{4.172074in}{0.969588in}}%
\pgfpathcurveto{\pgfqpoint{4.172074in}{0.962455in}}{\pgfqpoint{4.174908in}{0.955614in}}{\pgfqpoint{4.179952in}{0.950570in}}%
\pgfpathcurveto{\pgfqpoint{4.184995in}{0.945526in}}{\pgfqpoint{4.191837in}{0.942693in}}{\pgfqpoint{4.198970in}{0.942693in}}%
\pgfpathclose%
\pgfusepath{stroke,fill}%
\end{pgfscope}%
\begin{pgfscope}%
\pgfpathrectangle{\pgfqpoint{2.867647in}{0.500000in}}{\pgfqpoint{1.764706in}{1.700000in}}%
\pgfusepath{clip}%
\pgfsetbuttcap%
\pgfsetroundjoin%
\definecolor{currentfill}{rgb}{0.981377,0.920617,0.865369}%
\pgfsetfillcolor{currentfill}%
\pgfsetlinewidth{0.311001pt}%
\definecolor{currentstroke}{rgb}{1.000000,1.000000,1.000000}%
\pgfsetstrokecolor{currentstroke}%
\pgfsetdash{}{0pt}%
\pgfpathmoveto{\pgfqpoint{4.170014in}{1.258471in}}%
\pgfpathcurveto{\pgfqpoint{4.177147in}{1.258471in}}{\pgfqpoint{4.183989in}{1.261305in}}{\pgfqpoint{4.189032in}{1.266349in}}%
\pgfpathcurveto{\pgfqpoint{4.194076in}{1.271393in}}{\pgfqpoint{4.196910in}{1.278234in}}{\pgfqpoint{4.196910in}{1.285367in}}%
\pgfpathcurveto{\pgfqpoint{4.196910in}{1.292500in}}{\pgfqpoint{4.194076in}{1.299342in}}{\pgfqpoint{4.189032in}{1.304385in}}%
\pgfpathcurveto{\pgfqpoint{4.183989in}{1.309429in}}{\pgfqpoint{4.177147in}{1.312263in}}{\pgfqpoint{4.170014in}{1.312263in}}%
\pgfpathcurveto{\pgfqpoint{4.162881in}{1.312263in}}{\pgfqpoint{4.156040in}{1.309429in}}{\pgfqpoint{4.150996in}{1.304385in}}%
\pgfpathcurveto{\pgfqpoint{4.145952in}{1.299342in}}{\pgfqpoint{4.143118in}{1.292500in}}{\pgfqpoint{4.143118in}{1.285367in}}%
\pgfpathcurveto{\pgfqpoint{4.143118in}{1.278234in}}{\pgfqpoint{4.145952in}{1.271393in}}{\pgfqpoint{4.150996in}{1.266349in}}%
\pgfpathcurveto{\pgfqpoint{4.156040in}{1.261305in}}{\pgfqpoint{4.162881in}{1.258471in}}{\pgfqpoint{4.170014in}{1.258471in}}%
\pgfpathclose%
\pgfusepath{stroke,fill}%
\end{pgfscope}%
\begin{pgfscope}%
\pgfpathrectangle{\pgfqpoint{2.867647in}{0.500000in}}{\pgfqpoint{1.764706in}{1.700000in}}%
\pgfusepath{clip}%
\pgfsetbuttcap%
\pgfsetroundjoin%
\definecolor{currentfill}{rgb}{0.976961,0.885681,0.814303}%
\pgfsetfillcolor{currentfill}%
\pgfsetlinewidth{0.311001pt}%
\definecolor{currentstroke}{rgb}{1.000000,1.000000,1.000000}%
\pgfsetstrokecolor{currentstroke}%
\pgfsetdash{}{0pt}%
\pgfpathmoveto{\pgfqpoint{4.114301in}{1.122090in}}%
\pgfpathcurveto{\pgfqpoint{4.121433in}{1.122090in}}{\pgfqpoint{4.128275in}{1.124923in}}{\pgfqpoint{4.133319in}{1.129967in}}%
\pgfpathcurveto{\pgfqpoint{4.138362in}{1.135011in}}{\pgfqpoint{4.141196in}{1.141852in}}{\pgfqpoint{4.141196in}{1.148985in}}%
\pgfpathcurveto{\pgfqpoint{4.141196in}{1.156118in}}{\pgfqpoint{4.138362in}{1.162960in}}{\pgfqpoint{4.133319in}{1.168003in}}%
\pgfpathcurveto{\pgfqpoint{4.128275in}{1.173047in}}{\pgfqpoint{4.121433in}{1.175881in}}{\pgfqpoint{4.114301in}{1.175881in}}%
\pgfpathcurveto{\pgfqpoint{4.107168in}{1.175881in}}{\pgfqpoint{4.100326in}{1.173047in}}{\pgfqpoint{4.095283in}{1.168003in}}%
\pgfpathcurveto{\pgfqpoint{4.090239in}{1.162960in}}{\pgfqpoint{4.087405in}{1.156118in}}{\pgfqpoint{4.087405in}{1.148985in}}%
\pgfpathcurveto{\pgfqpoint{4.087405in}{1.141852in}}{\pgfqpoint{4.090239in}{1.135011in}}{\pgfqpoint{4.095283in}{1.129967in}}%
\pgfpathcurveto{\pgfqpoint{4.100326in}{1.124923in}}{\pgfqpoint{4.107168in}{1.122090in}}{\pgfqpoint{4.114301in}{1.122090in}}%
\pgfpathclose%
\pgfusepath{stroke,fill}%
\end{pgfscope}%
\begin{pgfscope}%
\pgfpathrectangle{\pgfqpoint{2.867647in}{0.500000in}}{\pgfqpoint{1.764706in}{1.700000in}}%
\pgfusepath{clip}%
\pgfsetbuttcap%
\pgfsetroundjoin%
\definecolor{currentfill}{rgb}{0.966120,0.744512,0.608720}%
\pgfsetfillcolor{currentfill}%
\pgfsetlinewidth{0.311001pt}%
\definecolor{currentstroke}{rgb}{1.000000,1.000000,1.000000}%
\pgfsetstrokecolor{currentstroke}%
\pgfsetdash{}{0pt}%
\pgfpathmoveto{\pgfqpoint{4.151229in}{0.940338in}}%
\pgfpathcurveto{\pgfqpoint{4.158361in}{0.940338in}}{\pgfqpoint{4.165203in}{0.943172in}}{\pgfqpoint{4.170247in}{0.948216in}}%
\pgfpathcurveto{\pgfqpoint{4.175290in}{0.953260in}}{\pgfqpoint{4.178124in}{0.960101in}}{\pgfqpoint{4.178124in}{0.967234in}}%
\pgfpathcurveto{\pgfqpoint{4.178124in}{0.974367in}}{\pgfqpoint{4.175290in}{0.981208in}}{\pgfqpoint{4.170247in}{0.986252in}}%
\pgfpathcurveto{\pgfqpoint{4.165203in}{0.991296in}}{\pgfqpoint{4.158361in}{0.994130in}}{\pgfqpoint{4.151229in}{0.994130in}}%
\pgfpathcurveto{\pgfqpoint{4.144096in}{0.994130in}}{\pgfqpoint{4.137254in}{0.991296in}}{\pgfqpoint{4.132210in}{0.986252in}}%
\pgfpathcurveto{\pgfqpoint{4.127167in}{0.981208in}}{\pgfqpoint{4.124333in}{0.974367in}}{\pgfqpoint{4.124333in}{0.967234in}}%
\pgfpathcurveto{\pgfqpoint{4.124333in}{0.960101in}}{\pgfqpoint{4.127167in}{0.953260in}}{\pgfqpoint{4.132210in}{0.948216in}}%
\pgfpathcurveto{\pgfqpoint{4.137254in}{0.943172in}}{\pgfqpoint{4.144096in}{0.940338in}}{\pgfqpoint{4.151229in}{0.940338in}}%
\pgfpathclose%
\pgfusepath{stroke,fill}%
\end{pgfscope}%
\begin{pgfscope}%
\pgfpathrectangle{\pgfqpoint{2.867647in}{0.500000in}}{\pgfqpoint{1.764706in}{1.700000in}}%
\pgfusepath{clip}%
\pgfsetbuttcap%
\pgfsetroundjoin%
\definecolor{currentfill}{rgb}{0.966328,0.750560,0.616961}%
\pgfsetfillcolor{currentfill}%
\pgfsetlinewidth{0.311001pt}%
\definecolor{currentstroke}{rgb}{1.000000,1.000000,1.000000}%
\pgfsetstrokecolor{currentstroke}%
\pgfsetdash{}{0pt}%
\pgfpathmoveto{\pgfqpoint{4.230755in}{1.036954in}}%
\pgfpathcurveto{\pgfqpoint{4.237888in}{1.036954in}}{\pgfqpoint{4.244730in}{1.039788in}}{\pgfqpoint{4.249774in}{1.044832in}}%
\pgfpathcurveto{\pgfqpoint{4.254817in}{1.049876in}}{\pgfqpoint{4.257651in}{1.056717in}}{\pgfqpoint{4.257651in}{1.063850in}}%
\pgfpathcurveto{\pgfqpoint{4.257651in}{1.070983in}}{\pgfqpoint{4.254817in}{1.077825in}}{\pgfqpoint{4.249774in}{1.082868in}}%
\pgfpathcurveto{\pgfqpoint{4.244730in}{1.087912in}}{\pgfqpoint{4.237888in}{1.090746in}}{\pgfqpoint{4.230755in}{1.090746in}}%
\pgfpathcurveto{\pgfqpoint{4.223623in}{1.090746in}}{\pgfqpoint{4.216781in}{1.087912in}}{\pgfqpoint{4.211737in}{1.082868in}}%
\pgfpathcurveto{\pgfqpoint{4.206694in}{1.077825in}}{\pgfqpoint{4.203860in}{1.070983in}}{\pgfqpoint{4.203860in}{1.063850in}}%
\pgfpathcurveto{\pgfqpoint{4.203860in}{1.056717in}}{\pgfqpoint{4.206694in}{1.049876in}}{\pgfqpoint{4.211737in}{1.044832in}}%
\pgfpathcurveto{\pgfqpoint{4.216781in}{1.039788in}}{\pgfqpoint{4.223623in}{1.036954in}}{\pgfqpoint{4.230755in}{1.036954in}}%
\pgfpathclose%
\pgfusepath{stroke,fill}%
\end{pgfscope}%
\begin{pgfscope}%
\pgfpathrectangle{\pgfqpoint{2.867647in}{0.500000in}}{\pgfqpoint{1.764706in}{1.700000in}}%
\pgfusepath{clip}%
\pgfsetbuttcap%
\pgfsetroundjoin%
\definecolor{currentfill}{rgb}{0.981377,0.920617,0.865369}%
\pgfsetfillcolor{currentfill}%
\pgfsetlinewidth{0.311001pt}%
\definecolor{currentstroke}{rgb}{1.000000,1.000000,1.000000}%
\pgfsetstrokecolor{currentstroke}%
\pgfsetdash{}{0pt}%
\pgfpathmoveto{\pgfqpoint{4.201593in}{1.255883in}}%
\pgfpathcurveto{\pgfqpoint{4.208726in}{1.255883in}}{\pgfqpoint{4.215568in}{1.258717in}}{\pgfqpoint{4.220611in}{1.263761in}}%
\pgfpathcurveto{\pgfqpoint{4.225655in}{1.268804in}}{\pgfqpoint{4.228489in}{1.275646in}}{\pgfqpoint{4.228489in}{1.282779in}}%
\pgfpathcurveto{\pgfqpoint{4.228489in}{1.289912in}}{\pgfqpoint{4.225655in}{1.296753in}}{\pgfqpoint{4.220611in}{1.301797in}}%
\pgfpathcurveto{\pgfqpoint{4.215568in}{1.306841in}}{\pgfqpoint{4.208726in}{1.309675in}}{\pgfqpoint{4.201593in}{1.309675in}}%
\pgfpathcurveto{\pgfqpoint{4.194461in}{1.309675in}}{\pgfqpoint{4.187619in}{1.306841in}}{\pgfqpoint{4.182575in}{1.301797in}}%
\pgfpathcurveto{\pgfqpoint{4.177532in}{1.296753in}}{\pgfqpoint{4.174698in}{1.289912in}}{\pgfqpoint{4.174698in}{1.282779in}}%
\pgfpathcurveto{\pgfqpoint{4.174698in}{1.275646in}}{\pgfqpoint{4.177532in}{1.268804in}}{\pgfqpoint{4.182575in}{1.263761in}}%
\pgfpathcurveto{\pgfqpoint{4.187619in}{1.258717in}}{\pgfqpoint{4.194461in}{1.255883in}}{\pgfqpoint{4.201593in}{1.255883in}}%
\pgfpathclose%
\pgfusepath{stroke,fill}%
\end{pgfscope}%
\begin{pgfscope}%
\pgfpathrectangle{\pgfqpoint{2.867647in}{0.500000in}}{\pgfqpoint{1.764706in}{1.700000in}}%
\pgfusepath{clip}%
\pgfsetbuttcap%
\pgfsetroundjoin%
\definecolor{currentfill}{rgb}{0.968105,0.786346,0.667739}%
\pgfsetfillcolor{currentfill}%
\pgfsetlinewidth{0.311001pt}%
\definecolor{currentstroke}{rgb}{1.000000,1.000000,1.000000}%
\pgfsetstrokecolor{currentstroke}%
\pgfsetdash{}{0pt}%
\pgfpathmoveto{\pgfqpoint{4.108127in}{0.937763in}}%
\pgfpathcurveto{\pgfqpoint{4.115260in}{0.937763in}}{\pgfqpoint{4.122101in}{0.940597in}}{\pgfqpoint{4.127145in}{0.945641in}}%
\pgfpathcurveto{\pgfqpoint{4.132189in}{0.950684in}}{\pgfqpoint{4.135023in}{0.957526in}}{\pgfqpoint{4.135023in}{0.964659in}}%
\pgfpathcurveto{\pgfqpoint{4.135023in}{0.971792in}}{\pgfqpoint{4.132189in}{0.978633in}}{\pgfqpoint{4.127145in}{0.983677in}}%
\pgfpathcurveto{\pgfqpoint{4.122101in}{0.988721in}}{\pgfqpoint{4.115260in}{0.991555in}}{\pgfqpoint{4.108127in}{0.991555in}}%
\pgfpathcurveto{\pgfqpoint{4.100994in}{0.991555in}}{\pgfqpoint{4.094153in}{0.988721in}}{\pgfqpoint{4.089109in}{0.983677in}}%
\pgfpathcurveto{\pgfqpoint{4.084065in}{0.978633in}}{\pgfqpoint{4.081231in}{0.971792in}}{\pgfqpoint{4.081231in}{0.964659in}}%
\pgfpathcurveto{\pgfqpoint{4.081231in}{0.957526in}}{\pgfqpoint{4.084065in}{0.950684in}}{\pgfqpoint{4.089109in}{0.945641in}}%
\pgfpathcurveto{\pgfqpoint{4.094153in}{0.940597in}}{\pgfqpoint{4.100994in}{0.937763in}}{\pgfqpoint{4.108127in}{0.937763in}}%
\pgfpathclose%
\pgfusepath{stroke,fill}%
\end{pgfscope}%
\begin{pgfscope}%
\pgfpathrectangle{\pgfqpoint{2.867647in}{0.500000in}}{\pgfqpoint{1.764706in}{1.700000in}}%
\pgfusepath{clip}%
\pgfsetbuttcap%
\pgfsetroundjoin%
\definecolor{currentfill}{rgb}{0.965753,0.732351,0.592427}%
\pgfsetfillcolor{currentfill}%
\pgfsetlinewidth{0.311001pt}%
\definecolor{currentstroke}{rgb}{1.000000,1.000000,1.000000}%
\pgfsetstrokecolor{currentstroke}%
\pgfsetdash{}{0pt}%
\pgfpathmoveto{\pgfqpoint{4.008066in}{0.935544in}}%
\pgfpathcurveto{\pgfqpoint{4.015198in}{0.935544in}}{\pgfqpoint{4.022040in}{0.938378in}}{\pgfqpoint{4.027084in}{0.943422in}}%
\pgfpathcurveto{\pgfqpoint{4.032127in}{0.948465in}}{\pgfqpoint{4.034961in}{0.955307in}}{\pgfqpoint{4.034961in}{0.962440in}}%
\pgfpathcurveto{\pgfqpoint{4.034961in}{0.969573in}}{\pgfqpoint{4.032127in}{0.976414in}}{\pgfqpoint{4.027084in}{0.981458in}}%
\pgfpathcurveto{\pgfqpoint{4.022040in}{0.986502in}}{\pgfqpoint{4.015198in}{0.989336in}}{\pgfqpoint{4.008066in}{0.989336in}}%
\pgfpathcurveto{\pgfqpoint{4.000933in}{0.989336in}}{\pgfqpoint{3.994091in}{0.986502in}}{\pgfqpoint{3.989047in}{0.981458in}}%
\pgfpathcurveto{\pgfqpoint{3.984004in}{0.976414in}}{\pgfqpoint{3.981170in}{0.969573in}}{\pgfqpoint{3.981170in}{0.962440in}}%
\pgfpathcurveto{\pgfqpoint{3.981170in}{0.955307in}}{\pgfqpoint{3.984004in}{0.948465in}}{\pgfqpoint{3.989047in}{0.943422in}}%
\pgfpathcurveto{\pgfqpoint{3.994091in}{0.938378in}}{\pgfqpoint{4.000933in}{0.935544in}}{\pgfqpoint{4.008066in}{0.935544in}}%
\pgfpathclose%
\pgfusepath{stroke,fill}%
\end{pgfscope}%
\begin{pgfscope}%
\pgfpathrectangle{\pgfqpoint{2.867647in}{0.500000in}}{\pgfqpoint{1.764706in}{1.700000in}}%
\pgfusepath{clip}%
\pgfsetbuttcap%
\pgfsetroundjoin%
\definecolor{currentfill}{rgb}{0.530589,0.116624,0.355860}%
\pgfsetfillcolor{currentfill}%
\pgfsetlinewidth{0.311001pt}%
\definecolor{currentstroke}{rgb}{1.000000,1.000000,1.000000}%
\pgfsetstrokecolor{currentstroke}%
\pgfsetdash{}{0pt}%
\pgfpathmoveto{\pgfqpoint{3.891880in}{1.458336in}}%
\pgfpathcurveto{\pgfqpoint{3.899013in}{1.458336in}}{\pgfqpoint{3.905855in}{1.461170in}}{\pgfqpoint{3.910898in}{1.466214in}}%
\pgfpathcurveto{\pgfqpoint{3.915942in}{1.471257in}}{\pgfqpoint{3.918776in}{1.478099in}}{\pgfqpoint{3.918776in}{1.485232in}}%
\pgfpathcurveto{\pgfqpoint{3.918776in}{1.492365in}}{\pgfqpoint{3.915942in}{1.499206in}}{\pgfqpoint{3.910898in}{1.504250in}}%
\pgfpathcurveto{\pgfqpoint{3.905855in}{1.509294in}}{\pgfqpoint{3.899013in}{1.512127in}}{\pgfqpoint{3.891880in}{1.512127in}}%
\pgfpathcurveto{\pgfqpoint{3.884747in}{1.512127in}}{\pgfqpoint{3.877906in}{1.509294in}}{\pgfqpoint{3.872862in}{1.504250in}}%
\pgfpathcurveto{\pgfqpoint{3.867818in}{1.499206in}}{\pgfqpoint{3.864984in}{1.492365in}}{\pgfqpoint{3.864984in}{1.485232in}}%
\pgfpathcurveto{\pgfqpoint{3.864984in}{1.478099in}}{\pgfqpoint{3.867818in}{1.471257in}}{\pgfqpoint{3.872862in}{1.466214in}}%
\pgfpathcurveto{\pgfqpoint{3.877906in}{1.461170in}}{\pgfqpoint{3.884747in}{1.458336in}}{\pgfqpoint{3.891880in}{1.458336in}}%
\pgfpathclose%
\pgfusepath{stroke,fill}%
\end{pgfscope}%
\begin{pgfscope}%
\pgfpathrectangle{\pgfqpoint{2.867647in}{0.500000in}}{\pgfqpoint{1.764706in}{1.700000in}}%
\pgfusepath{clip}%
\pgfsetbuttcap%
\pgfsetroundjoin%
\definecolor{currentfill}{rgb}{0.979891,0.908948,0.848279}%
\pgfsetfillcolor{currentfill}%
\pgfsetlinewidth{0.311001pt}%
\definecolor{currentstroke}{rgb}{1.000000,1.000000,1.000000}%
\pgfsetstrokecolor{currentstroke}%
\pgfsetdash{}{0pt}%
\pgfpathmoveto{\pgfqpoint{4.198923in}{1.179104in}}%
\pgfpathcurveto{\pgfqpoint{4.206055in}{1.179104in}}{\pgfqpoint{4.212897in}{1.181938in}}{\pgfqpoint{4.217941in}{1.186982in}}%
\pgfpathcurveto{\pgfqpoint{4.222984in}{1.192026in}}{\pgfqpoint{4.225818in}{1.198867in}}{\pgfqpoint{4.225818in}{1.206000in}}%
\pgfpathcurveto{\pgfqpoint{4.225818in}{1.213133in}}{\pgfqpoint{4.222984in}{1.219975in}}{\pgfqpoint{4.217941in}{1.225018in}}%
\pgfpathcurveto{\pgfqpoint{4.212897in}{1.230062in}}{\pgfqpoint{4.206055in}{1.232896in}}{\pgfqpoint{4.198923in}{1.232896in}}%
\pgfpathcurveto{\pgfqpoint{4.191790in}{1.232896in}}{\pgfqpoint{4.184948in}{1.230062in}}{\pgfqpoint{4.179904in}{1.225018in}}%
\pgfpathcurveto{\pgfqpoint{4.174861in}{1.219975in}}{\pgfqpoint{4.172027in}{1.213133in}}{\pgfqpoint{4.172027in}{1.206000in}}%
\pgfpathcurveto{\pgfqpoint{4.172027in}{1.198867in}}{\pgfqpoint{4.174861in}{1.192026in}}{\pgfqpoint{4.179904in}{1.186982in}}%
\pgfpathcurveto{\pgfqpoint{4.184948in}{1.181938in}}{\pgfqpoint{4.191790in}{1.179104in}}{\pgfqpoint{4.198923in}{1.179104in}}%
\pgfpathclose%
\pgfusepath{stroke,fill}%
\end{pgfscope}%
\begin{pgfscope}%
\pgfpathrectangle{\pgfqpoint{2.867647in}{0.500000in}}{\pgfqpoint{1.764706in}{1.700000in}}%
\pgfusepath{clip}%
\pgfsetbuttcap%
\pgfsetroundjoin%
\definecolor{currentfill}{rgb}{0.953126,0.456614,0.312398}%
\pgfsetfillcolor{currentfill}%
\pgfsetlinewidth{0.311001pt}%
\definecolor{currentstroke}{rgb}{1.000000,1.000000,1.000000}%
\pgfsetstrokecolor{currentstroke}%
\pgfsetdash{}{0pt}%
\pgfpathmoveto{\pgfqpoint{4.340889in}{1.466027in}}%
\pgfpathcurveto{\pgfqpoint{4.348022in}{1.466027in}}{\pgfqpoint{4.354864in}{1.468861in}}{\pgfqpoint{4.359907in}{1.473904in}}%
\pgfpathcurveto{\pgfqpoint{4.364951in}{1.478948in}}{\pgfqpoint{4.367785in}{1.485790in}}{\pgfqpoint{4.367785in}{1.492922in}}%
\pgfpathcurveto{\pgfqpoint{4.367785in}{1.500055in}}{\pgfqpoint{4.364951in}{1.506897in}}{\pgfqpoint{4.359907in}{1.511941in}}%
\pgfpathcurveto{\pgfqpoint{4.354864in}{1.516984in}}{\pgfqpoint{4.348022in}{1.519818in}}{\pgfqpoint{4.340889in}{1.519818in}}%
\pgfpathcurveto{\pgfqpoint{4.333756in}{1.519818in}}{\pgfqpoint{4.326915in}{1.516984in}}{\pgfqpoint{4.321871in}{1.511941in}}%
\pgfpathcurveto{\pgfqpoint{4.316827in}{1.506897in}}{\pgfqpoint{4.313993in}{1.500055in}}{\pgfqpoint{4.313993in}{1.492922in}}%
\pgfpathcurveto{\pgfqpoint{4.313993in}{1.485790in}}{\pgfqpoint{4.316827in}{1.478948in}}{\pgfqpoint{4.321871in}{1.473904in}}%
\pgfpathcurveto{\pgfqpoint{4.326915in}{1.468861in}}{\pgfqpoint{4.333756in}{1.466027in}}{\pgfqpoint{4.340889in}{1.466027in}}%
\pgfpathclose%
\pgfusepath{stroke,fill}%
\end{pgfscope}%
\begin{pgfscope}%
\pgfpathrectangle{\pgfqpoint{2.867647in}{0.500000in}}{\pgfqpoint{1.764706in}{1.700000in}}%
\pgfusepath{clip}%
\pgfsetbuttcap%
\pgfsetroundjoin%
\definecolor{currentfill}{rgb}{0.980678,0.914765,0.856766}%
\pgfsetfillcolor{currentfill}%
\pgfsetlinewidth{0.311001pt}%
\definecolor{currentstroke}{rgb}{1.000000,1.000000,1.000000}%
\pgfsetstrokecolor{currentstroke}%
\pgfsetdash{}{0pt}%
\pgfpathmoveto{\pgfqpoint{4.175164in}{1.327115in}}%
\pgfpathcurveto{\pgfqpoint{4.182297in}{1.327115in}}{\pgfqpoint{4.189139in}{1.329949in}}{\pgfqpoint{4.194182in}{1.334992in}}%
\pgfpathcurveto{\pgfqpoint{4.199226in}{1.340036in}}{\pgfqpoint{4.202060in}{1.346878in}}{\pgfqpoint{4.202060in}{1.354010in}}%
\pgfpathcurveto{\pgfqpoint{4.202060in}{1.361143in}}{\pgfqpoint{4.199226in}{1.367985in}}{\pgfqpoint{4.194182in}{1.373029in}}%
\pgfpathcurveto{\pgfqpoint{4.189139in}{1.378072in}}{\pgfqpoint{4.182297in}{1.380906in}}{\pgfqpoint{4.175164in}{1.380906in}}%
\pgfpathcurveto{\pgfqpoint{4.168031in}{1.380906in}}{\pgfqpoint{4.161190in}{1.378072in}}{\pgfqpoint{4.156146in}{1.373029in}}%
\pgfpathcurveto{\pgfqpoint{4.151102in}{1.367985in}}{\pgfqpoint{4.148268in}{1.361143in}}{\pgfqpoint{4.148268in}{1.354010in}}%
\pgfpathcurveto{\pgfqpoint{4.148268in}{1.346878in}}{\pgfqpoint{4.151102in}{1.340036in}}{\pgfqpoint{4.156146in}{1.334992in}}%
\pgfpathcurveto{\pgfqpoint{4.161190in}{1.329949in}}{\pgfqpoint{4.168031in}{1.327115in}}{\pgfqpoint{4.175164in}{1.327115in}}%
\pgfpathclose%
\pgfusepath{stroke,fill}%
\end{pgfscope}%
\begin{pgfscope}%
\pgfpathrectangle{\pgfqpoint{2.867647in}{0.500000in}}{\pgfqpoint{1.764706in}{1.700000in}}%
\pgfusepath{clip}%
\pgfsetbuttcap%
\pgfsetroundjoin%
\definecolor{currentfill}{rgb}{0.852817,0.156578,0.279098}%
\pgfsetfillcolor{currentfill}%
\pgfsetlinewidth{0.311001pt}%
\definecolor{currentstroke}{rgb}{1.000000,1.000000,1.000000}%
\pgfsetstrokecolor{currentstroke}%
\pgfsetdash{}{0pt}%
\pgfpathmoveto{\pgfqpoint{4.396033in}{1.399019in}}%
\pgfpathcurveto{\pgfqpoint{4.403166in}{1.399019in}}{\pgfqpoint{4.410008in}{1.401853in}}{\pgfqpoint{4.415052in}{1.406896in}}%
\pgfpathcurveto{\pgfqpoint{4.420095in}{1.411940in}}{\pgfqpoint{4.422929in}{1.418782in}}{\pgfqpoint{4.422929in}{1.425915in}}%
\pgfpathcurveto{\pgfqpoint{4.422929in}{1.433047in}}{\pgfqpoint{4.420095in}{1.439889in}}{\pgfqpoint{4.415052in}{1.444933in}}%
\pgfpathcurveto{\pgfqpoint{4.410008in}{1.449976in}}{\pgfqpoint{4.403166in}{1.452810in}}{\pgfqpoint{4.396033in}{1.452810in}}%
\pgfpathcurveto{\pgfqpoint{4.388901in}{1.452810in}}{\pgfqpoint{4.382059in}{1.449976in}}{\pgfqpoint{4.377015in}{1.444933in}}%
\pgfpathcurveto{\pgfqpoint{4.371972in}{1.439889in}}{\pgfqpoint{4.369138in}{1.433047in}}{\pgfqpoint{4.369138in}{1.425915in}}%
\pgfpathcurveto{\pgfqpoint{4.369138in}{1.418782in}}{\pgfqpoint{4.371972in}{1.411940in}}{\pgfqpoint{4.377015in}{1.406896in}}%
\pgfpathcurveto{\pgfqpoint{4.382059in}{1.401853in}}{\pgfqpoint{4.388901in}{1.399019in}}{\pgfqpoint{4.396033in}{1.399019in}}%
\pgfpathclose%
\pgfusepath{stroke,fill}%
\end{pgfscope}%
\begin{pgfscope}%
\pgfpathrectangle{\pgfqpoint{2.867647in}{0.500000in}}{\pgfqpoint{1.764706in}{1.700000in}}%
\pgfusepath{clip}%
\pgfsetbuttcap%
\pgfsetroundjoin%
\definecolor{currentfill}{rgb}{0.979124,0.903132,0.839793}%
\pgfsetfillcolor{currentfill}%
\pgfsetlinewidth{0.311001pt}%
\definecolor{currentstroke}{rgb}{1.000000,1.000000,1.000000}%
\pgfsetstrokecolor{currentstroke}%
\pgfsetdash{}{0pt}%
\pgfpathmoveto{\pgfqpoint{4.143945in}{1.453702in}}%
\pgfpathcurveto{\pgfqpoint{4.151078in}{1.453702in}}{\pgfqpoint{4.157920in}{1.456536in}}{\pgfqpoint{4.162963in}{1.461580in}}%
\pgfpathcurveto{\pgfqpoint{4.168007in}{1.466623in}}{\pgfqpoint{4.170841in}{1.473465in}}{\pgfqpoint{4.170841in}{1.480598in}}%
\pgfpathcurveto{\pgfqpoint{4.170841in}{1.487731in}}{\pgfqpoint{4.168007in}{1.494572in}}{\pgfqpoint{4.162963in}{1.499616in}}%
\pgfpathcurveto{\pgfqpoint{4.157920in}{1.504660in}}{\pgfqpoint{4.151078in}{1.507494in}}{\pgfqpoint{4.143945in}{1.507494in}}%
\pgfpathcurveto{\pgfqpoint{4.136812in}{1.507494in}}{\pgfqpoint{4.129971in}{1.504660in}}{\pgfqpoint{4.124927in}{1.499616in}}%
\pgfpathcurveto{\pgfqpoint{4.119883in}{1.494572in}}{\pgfqpoint{4.117049in}{1.487731in}}{\pgfqpoint{4.117049in}{1.480598in}}%
\pgfpathcurveto{\pgfqpoint{4.117049in}{1.473465in}}{\pgfqpoint{4.119883in}{1.466623in}}{\pgfqpoint{4.124927in}{1.461580in}}%
\pgfpathcurveto{\pgfqpoint{4.129971in}{1.456536in}}{\pgfqpoint{4.136812in}{1.453702in}}{\pgfqpoint{4.143945in}{1.453702in}}%
\pgfpathclose%
\pgfusepath{stroke,fill}%
\end{pgfscope}%
\begin{pgfscope}%
\pgfpathrectangle{\pgfqpoint{2.867647in}{0.500000in}}{\pgfqpoint{1.764706in}{1.700000in}}%
\pgfusepath{clip}%
\pgfsetbuttcap%
\pgfsetroundjoin%
\definecolor{currentfill}{rgb}{0.972201,0.839051,0.745789}%
\pgfsetfillcolor{currentfill}%
\pgfsetlinewidth{0.311001pt}%
\definecolor{currentstroke}{rgb}{1.000000,1.000000,1.000000}%
\pgfsetstrokecolor{currentstroke}%
\pgfsetdash{}{0pt}%
\pgfpathmoveto{\pgfqpoint{4.058098in}{1.619329in}}%
\pgfpathcurveto{\pgfqpoint{4.065231in}{1.619329in}}{\pgfqpoint{4.072072in}{1.622163in}}{\pgfqpoint{4.077116in}{1.627207in}}%
\pgfpathcurveto{\pgfqpoint{4.082160in}{1.632250in}}{\pgfqpoint{4.084994in}{1.639092in}}{\pgfqpoint{4.084994in}{1.646225in}}%
\pgfpathcurveto{\pgfqpoint{4.084994in}{1.653358in}}{\pgfqpoint{4.082160in}{1.660199in}}{\pgfqpoint{4.077116in}{1.665243in}}%
\pgfpathcurveto{\pgfqpoint{4.072072in}{1.670287in}}{\pgfqpoint{4.065231in}{1.673120in}}{\pgfqpoint{4.058098in}{1.673120in}}%
\pgfpathcurveto{\pgfqpoint{4.050965in}{1.673120in}}{\pgfqpoint{4.044123in}{1.670287in}}{\pgfqpoint{4.039080in}{1.665243in}}%
\pgfpathcurveto{\pgfqpoint{4.034036in}{1.660199in}}{\pgfqpoint{4.031202in}{1.653358in}}{\pgfqpoint{4.031202in}{1.646225in}}%
\pgfpathcurveto{\pgfqpoint{4.031202in}{1.639092in}}{\pgfqpoint{4.034036in}{1.632250in}}{\pgfqpoint{4.039080in}{1.627207in}}%
\pgfpathcurveto{\pgfqpoint{4.044123in}{1.622163in}}{\pgfqpoint{4.050965in}{1.619329in}}{\pgfqpoint{4.058098in}{1.619329in}}%
\pgfpathclose%
\pgfusepath{stroke,fill}%
\end{pgfscope}%
\begin{pgfscope}%
\pgfpathrectangle{\pgfqpoint{2.867647in}{0.500000in}}{\pgfqpoint{1.764706in}{1.700000in}}%
\pgfusepath{clip}%
\pgfsetbuttcap%
\pgfsetroundjoin%
\definecolor{currentfill}{rgb}{0.949145,0.420383,0.287810}%
\pgfsetfillcolor{currentfill}%
\pgfsetlinewidth{0.311001pt}%
\definecolor{currentstroke}{rgb}{1.000000,1.000000,1.000000}%
\pgfsetstrokecolor{currentstroke}%
\pgfsetdash{}{0pt}%
\pgfpathmoveto{\pgfqpoint{3.846397in}{1.761280in}}%
\pgfpathcurveto{\pgfqpoint{3.853529in}{1.761280in}}{\pgfqpoint{3.860371in}{1.764114in}}{\pgfqpoint{3.865415in}{1.769158in}}%
\pgfpathcurveto{\pgfqpoint{3.870458in}{1.774201in}}{\pgfqpoint{3.873292in}{1.781043in}}{\pgfqpoint{3.873292in}{1.788176in}}%
\pgfpathcurveto{\pgfqpoint{3.873292in}{1.795309in}}{\pgfqpoint{3.870458in}{1.802150in}}{\pgfqpoint{3.865415in}{1.807194in}}%
\pgfpathcurveto{\pgfqpoint{3.860371in}{1.812238in}}{\pgfqpoint{3.853529in}{1.815071in}}{\pgfqpoint{3.846397in}{1.815071in}}%
\pgfpathcurveto{\pgfqpoint{3.839264in}{1.815071in}}{\pgfqpoint{3.832422in}{1.812238in}}{\pgfqpoint{3.827379in}{1.807194in}}%
\pgfpathcurveto{\pgfqpoint{3.822335in}{1.802150in}}{\pgfqpoint{3.819501in}{1.795309in}}{\pgfqpoint{3.819501in}{1.788176in}}%
\pgfpathcurveto{\pgfqpoint{3.819501in}{1.781043in}}{\pgfqpoint{3.822335in}{1.774201in}}{\pgfqpoint{3.827379in}{1.769158in}}%
\pgfpathcurveto{\pgfqpoint{3.832422in}{1.764114in}}{\pgfqpoint{3.839264in}{1.761280in}}{\pgfqpoint{3.846397in}{1.761280in}}%
\pgfpathclose%
\pgfusepath{stroke,fill}%
\end{pgfscope}%
\begin{pgfscope}%
\pgfpathrectangle{\pgfqpoint{2.867647in}{0.500000in}}{\pgfqpoint{1.764706in}{1.700000in}}%
\pgfusepath{clip}%
\pgfsetbuttcap%
\pgfsetroundjoin%
\definecolor{currentfill}{rgb}{0.979124,0.903132,0.839793}%
\pgfsetfillcolor{currentfill}%
\pgfsetlinewidth{0.311001pt}%
\definecolor{currentstroke}{rgb}{1.000000,1.000000,1.000000}%
\pgfsetstrokecolor{currentstroke}%
\pgfsetdash{}{0pt}%
\pgfpathmoveto{\pgfqpoint{4.222920in}{1.305899in}}%
\pgfpathcurveto{\pgfqpoint{4.230053in}{1.305899in}}{\pgfqpoint{4.236895in}{1.308733in}}{\pgfqpoint{4.241939in}{1.313777in}}%
\pgfpathcurveto{\pgfqpoint{4.246982in}{1.318820in}}{\pgfqpoint{4.249816in}{1.325662in}}{\pgfqpoint{4.249816in}{1.332795in}}%
\pgfpathcurveto{\pgfqpoint{4.249816in}{1.339928in}}{\pgfqpoint{4.246982in}{1.346769in}}{\pgfqpoint{4.241939in}{1.351813in}}%
\pgfpathcurveto{\pgfqpoint{4.236895in}{1.356857in}}{\pgfqpoint{4.230053in}{1.359691in}}{\pgfqpoint{4.222920in}{1.359691in}}%
\pgfpathcurveto{\pgfqpoint{4.215788in}{1.359691in}}{\pgfqpoint{4.208946in}{1.356857in}}{\pgfqpoint{4.203902in}{1.351813in}}%
\pgfpathcurveto{\pgfqpoint{4.198859in}{1.346769in}}{\pgfqpoint{4.196025in}{1.339928in}}{\pgfqpoint{4.196025in}{1.332795in}}%
\pgfpathcurveto{\pgfqpoint{4.196025in}{1.325662in}}{\pgfqpoint{4.198859in}{1.318820in}}{\pgfqpoint{4.203902in}{1.313777in}}%
\pgfpathcurveto{\pgfqpoint{4.208946in}{1.308733in}}{\pgfqpoint{4.215788in}{1.305899in}}{\pgfqpoint{4.222920in}{1.305899in}}%
\pgfpathclose%
\pgfusepath{stroke,fill}%
\end{pgfscope}%
\begin{pgfscope}%
\pgfpathrectangle{\pgfqpoint{2.867647in}{0.500000in}}{\pgfqpoint{1.764706in}{1.700000in}}%
\pgfusepath{clip}%
\pgfsetbuttcap%
\pgfsetroundjoin%
\definecolor{currentfill}{rgb}{0.966560,0.756582,0.625273}%
\pgfsetfillcolor{currentfill}%
\pgfsetlinewidth{0.311001pt}%
\definecolor{currentstroke}{rgb}{1.000000,1.000000,1.000000}%
\pgfsetstrokecolor{currentstroke}%
\pgfsetdash{}{0pt}%
\pgfpathmoveto{\pgfqpoint{4.062391in}{1.466740in}}%
\pgfpathcurveto{\pgfqpoint{4.069524in}{1.466740in}}{\pgfqpoint{4.076365in}{1.469574in}}{\pgfqpoint{4.081409in}{1.474617in}}%
\pgfpathcurveto{\pgfqpoint{4.086452in}{1.479661in}}{\pgfqpoint{4.089286in}{1.486503in}}{\pgfqpoint{4.089286in}{1.493635in}}%
\pgfpathcurveto{\pgfqpoint{4.089286in}{1.500768in}}{\pgfqpoint{4.086452in}{1.507610in}}{\pgfqpoint{4.081409in}{1.512654in}}%
\pgfpathcurveto{\pgfqpoint{4.076365in}{1.517697in}}{\pgfqpoint{4.069524in}{1.520531in}}{\pgfqpoint{4.062391in}{1.520531in}}%
\pgfpathcurveto{\pgfqpoint{4.055258in}{1.520531in}}{\pgfqpoint{4.048416in}{1.517697in}}{\pgfqpoint{4.043373in}{1.512654in}}%
\pgfpathcurveto{\pgfqpoint{4.038329in}{1.507610in}}{\pgfqpoint{4.035495in}{1.500768in}}{\pgfqpoint{4.035495in}{1.493635in}}%
\pgfpathcurveto{\pgfqpoint{4.035495in}{1.486503in}}{\pgfqpoint{4.038329in}{1.479661in}}{\pgfqpoint{4.043373in}{1.474617in}}%
\pgfpathcurveto{\pgfqpoint{4.048416in}{1.469574in}}{\pgfqpoint{4.055258in}{1.466740in}}{\pgfqpoint{4.062391in}{1.466740in}}%
\pgfpathclose%
\pgfusepath{stroke,fill}%
\end{pgfscope}%
\begin{pgfscope}%
\pgfpathrectangle{\pgfqpoint{2.867647in}{0.500000in}}{\pgfqpoint{1.764706in}{1.700000in}}%
\pgfusepath{clip}%
\pgfsetbuttcap%
\pgfsetroundjoin%
\definecolor{currentfill}{rgb}{0.964558,0.676556,0.522514}%
\pgfsetfillcolor{currentfill}%
\pgfsetlinewidth{0.311001pt}%
\definecolor{currentstroke}{rgb}{1.000000,1.000000,1.000000}%
\pgfsetstrokecolor{currentstroke}%
\pgfsetdash{}{0pt}%
\pgfpathmoveto{\pgfqpoint{4.217889in}{1.662026in}}%
\pgfpathcurveto{\pgfqpoint{4.225022in}{1.662026in}}{\pgfqpoint{4.231864in}{1.664860in}}{\pgfqpoint{4.236908in}{1.669903in}}%
\pgfpathcurveto{\pgfqpoint{4.241951in}{1.674947in}}{\pgfqpoint{4.244785in}{1.681789in}}{\pgfqpoint{4.244785in}{1.688921in}}%
\pgfpathcurveto{\pgfqpoint{4.244785in}{1.696054in}}{\pgfqpoint{4.241951in}{1.702896in}}{\pgfqpoint{4.236908in}{1.707940in}}%
\pgfpathcurveto{\pgfqpoint{4.231864in}{1.712983in}}{\pgfqpoint{4.225022in}{1.715817in}}{\pgfqpoint{4.217889in}{1.715817in}}%
\pgfpathcurveto{\pgfqpoint{4.210757in}{1.715817in}}{\pgfqpoint{4.203915in}{1.712983in}}{\pgfqpoint{4.198871in}{1.707940in}}%
\pgfpathcurveto{\pgfqpoint{4.193828in}{1.702896in}}{\pgfqpoint{4.190994in}{1.696054in}}{\pgfqpoint{4.190994in}{1.688921in}}%
\pgfpathcurveto{\pgfqpoint{4.190994in}{1.681789in}}{\pgfqpoint{4.193828in}{1.674947in}}{\pgfqpoint{4.198871in}{1.669903in}}%
\pgfpathcurveto{\pgfqpoint{4.203915in}{1.664860in}}{\pgfqpoint{4.210757in}{1.662026in}}{\pgfqpoint{4.217889in}{1.662026in}}%
\pgfpathclose%
\pgfusepath{stroke,fill}%
\end{pgfscope}%
\begin{pgfscope}%
\pgfpathrectangle{\pgfqpoint{2.867647in}{0.500000in}}{\pgfqpoint{1.764706in}{1.700000in}}%
\pgfusepath{clip}%
\pgfsetbuttcap%
\pgfsetroundjoin%
\definecolor{currentfill}{rgb}{0.980678,0.914765,0.856766}%
\pgfsetfillcolor{currentfill}%
\pgfsetlinewidth{0.311001pt}%
\definecolor{currentstroke}{rgb}{1.000000,1.000000,1.000000}%
\pgfsetstrokecolor{currentstroke}%
\pgfsetdash{}{0pt}%
\pgfpathmoveto{\pgfqpoint{4.187577in}{1.156317in}}%
\pgfpathcurveto{\pgfqpoint{4.194710in}{1.156317in}}{\pgfqpoint{4.201552in}{1.159151in}}{\pgfqpoint{4.206596in}{1.164195in}}%
\pgfpathcurveto{\pgfqpoint{4.211639in}{1.169239in}}{\pgfqpoint{4.214473in}{1.176080in}}{\pgfqpoint{4.214473in}{1.183213in}}%
\pgfpathcurveto{\pgfqpoint{4.214473in}{1.190346in}}{\pgfqpoint{4.211639in}{1.197187in}}{\pgfqpoint{4.206596in}{1.202231in}}%
\pgfpathcurveto{\pgfqpoint{4.201552in}{1.207275in}}{\pgfqpoint{4.194710in}{1.210109in}}{\pgfqpoint{4.187577in}{1.210109in}}%
\pgfpathcurveto{\pgfqpoint{4.180445in}{1.210109in}}{\pgfqpoint{4.173603in}{1.207275in}}{\pgfqpoint{4.168559in}{1.202231in}}%
\pgfpathcurveto{\pgfqpoint{4.163516in}{1.197187in}}{\pgfqpoint{4.160682in}{1.190346in}}{\pgfqpoint{4.160682in}{1.183213in}}%
\pgfpathcurveto{\pgfqpoint{4.160682in}{1.176080in}}{\pgfqpoint{4.163516in}{1.169239in}}{\pgfqpoint{4.168559in}{1.164195in}}%
\pgfpathcurveto{\pgfqpoint{4.173603in}{1.159151in}}{\pgfqpoint{4.180445in}{1.156317in}}{\pgfqpoint{4.187577in}{1.156317in}}%
\pgfpathclose%
\pgfusepath{stroke,fill}%
\end{pgfscope}%
\begin{pgfscope}%
\pgfpathrectangle{\pgfqpoint{2.867647in}{0.500000in}}{\pgfqpoint{1.764706in}{1.700000in}}%
\pgfusepath{clip}%
\pgfsetbuttcap%
\pgfsetroundjoin%
\definecolor{currentfill}{rgb}{0.976961,0.885681,0.814303}%
\pgfsetfillcolor{currentfill}%
\pgfsetlinewidth{0.311001pt}%
\definecolor{currentstroke}{rgb}{1.000000,1.000000,1.000000}%
\pgfsetstrokecolor{currentstroke}%
\pgfsetdash{}{0pt}%
\pgfpathmoveto{\pgfqpoint{4.096646in}{1.561519in}}%
\pgfpathcurveto{\pgfqpoint{4.103779in}{1.561519in}}{\pgfqpoint{4.110620in}{1.564353in}}{\pgfqpoint{4.115664in}{1.569396in}}%
\pgfpathcurveto{\pgfqpoint{4.120708in}{1.574440in}}{\pgfqpoint{4.123542in}{1.581282in}}{\pgfqpoint{4.123542in}{1.588414in}}%
\pgfpathcurveto{\pgfqpoint{4.123542in}{1.595547in}}{\pgfqpoint{4.120708in}{1.602389in}}{\pgfqpoint{4.115664in}{1.607433in}}%
\pgfpathcurveto{\pgfqpoint{4.110620in}{1.612476in}}{\pgfqpoint{4.103779in}{1.615310in}}{\pgfqpoint{4.096646in}{1.615310in}}%
\pgfpathcurveto{\pgfqpoint{4.089513in}{1.615310in}}{\pgfqpoint{4.082671in}{1.612476in}}{\pgfqpoint{4.077628in}{1.607433in}}%
\pgfpathcurveto{\pgfqpoint{4.072584in}{1.602389in}}{\pgfqpoint{4.069750in}{1.595547in}}{\pgfqpoint{4.069750in}{1.588414in}}%
\pgfpathcurveto{\pgfqpoint{4.069750in}{1.581282in}}{\pgfqpoint{4.072584in}{1.574440in}}{\pgfqpoint{4.077628in}{1.569396in}}%
\pgfpathcurveto{\pgfqpoint{4.082671in}{1.564353in}}{\pgfqpoint{4.089513in}{1.561519in}}{\pgfqpoint{4.096646in}{1.561519in}}%
\pgfpathclose%
\pgfusepath{stroke,fill}%
\end{pgfscope}%
\begin{pgfscope}%
\pgfpathrectangle{\pgfqpoint{2.867647in}{0.500000in}}{\pgfqpoint{1.764706in}{1.700000in}}%
\pgfusepath{clip}%
\pgfsetbuttcap%
\pgfsetroundjoin%
\definecolor{currentfill}{rgb}{0.980678,0.914765,0.856766}%
\pgfsetfillcolor{currentfill}%
\pgfsetlinewidth{0.311001pt}%
\definecolor{currentstroke}{rgb}{1.000000,1.000000,1.000000}%
\pgfsetstrokecolor{currentstroke}%
\pgfsetdash{}{0pt}%
\pgfpathmoveto{\pgfqpoint{4.171913in}{1.444166in}}%
\pgfpathcurveto{\pgfqpoint{4.179046in}{1.444166in}}{\pgfqpoint{4.185888in}{1.447000in}}{\pgfqpoint{4.190931in}{1.452044in}}%
\pgfpathcurveto{\pgfqpoint{4.195975in}{1.457087in}}{\pgfqpoint{4.198809in}{1.463929in}}{\pgfqpoint{4.198809in}{1.471062in}}%
\pgfpathcurveto{\pgfqpoint{4.198809in}{1.478195in}}{\pgfqpoint{4.195975in}{1.485036in}}{\pgfqpoint{4.190931in}{1.490080in}}%
\pgfpathcurveto{\pgfqpoint{4.185888in}{1.495124in}}{\pgfqpoint{4.179046in}{1.497957in}}{\pgfqpoint{4.171913in}{1.497957in}}%
\pgfpathcurveto{\pgfqpoint{4.164780in}{1.497957in}}{\pgfqpoint{4.157939in}{1.495124in}}{\pgfqpoint{4.152895in}{1.490080in}}%
\pgfpathcurveto{\pgfqpoint{4.147851in}{1.485036in}}{\pgfqpoint{4.145017in}{1.478195in}}{\pgfqpoint{4.145017in}{1.471062in}}%
\pgfpathcurveto{\pgfqpoint{4.145017in}{1.463929in}}{\pgfqpoint{4.147851in}{1.457087in}}{\pgfqpoint{4.152895in}{1.452044in}}%
\pgfpathcurveto{\pgfqpoint{4.157939in}{1.447000in}}{\pgfqpoint{4.164780in}{1.444166in}}{\pgfqpoint{4.171913in}{1.444166in}}%
\pgfpathclose%
\pgfusepath{stroke,fill}%
\end{pgfscope}%
\begin{pgfscope}%
\pgfpathrectangle{\pgfqpoint{2.867647in}{0.500000in}}{\pgfqpoint{1.764706in}{1.700000in}}%
\pgfusepath{clip}%
\pgfsetbuttcap%
\pgfsetroundjoin%
\definecolor{currentfill}{rgb}{0.967092,0.768560,0.642079}%
\pgfsetfillcolor{currentfill}%
\pgfsetlinewidth{0.311001pt}%
\definecolor{currentstroke}{rgb}{1.000000,1.000000,1.000000}%
\pgfsetstrokecolor{currentstroke}%
\pgfsetdash{}{0pt}%
\pgfpathmoveto{\pgfqpoint{4.021814in}{1.609391in}}%
\pgfpathcurveto{\pgfqpoint{4.028947in}{1.609391in}}{\pgfqpoint{4.035788in}{1.612225in}}{\pgfqpoint{4.040832in}{1.617269in}}%
\pgfpathcurveto{\pgfqpoint{4.045876in}{1.622313in}}{\pgfqpoint{4.048710in}{1.629154in}}{\pgfqpoint{4.048710in}{1.636287in}}%
\pgfpathcurveto{\pgfqpoint{4.048710in}{1.643420in}}{\pgfqpoint{4.045876in}{1.650262in}}{\pgfqpoint{4.040832in}{1.655305in}}%
\pgfpathcurveto{\pgfqpoint{4.035788in}{1.660349in}}{\pgfqpoint{4.028947in}{1.663183in}}{\pgfqpoint{4.021814in}{1.663183in}}%
\pgfpathcurveto{\pgfqpoint{4.014681in}{1.663183in}}{\pgfqpoint{4.007840in}{1.660349in}}{\pgfqpoint{4.002796in}{1.655305in}}%
\pgfpathcurveto{\pgfqpoint{3.997752in}{1.650262in}}{\pgfqpoint{3.994918in}{1.643420in}}{\pgfqpoint{3.994918in}{1.636287in}}%
\pgfpathcurveto{\pgfqpoint{3.994918in}{1.629154in}}{\pgfqpoint{3.997752in}{1.622313in}}{\pgfqpoint{4.002796in}{1.617269in}}%
\pgfpathcurveto{\pgfqpoint{4.007840in}{1.612225in}}{\pgfqpoint{4.014681in}{1.609391in}}{\pgfqpoint{4.021814in}{1.609391in}}%
\pgfpathclose%
\pgfusepath{stroke,fill}%
\end{pgfscope}%
\begin{pgfscope}%
\pgfpathrectangle{\pgfqpoint{2.867647in}{0.500000in}}{\pgfqpoint{1.764706in}{1.700000in}}%
\pgfusepath{clip}%
\pgfsetbuttcap%
\pgfsetroundjoin%
\definecolor{currentfill}{rgb}{0.979124,0.903132,0.839793}%
\pgfsetfillcolor{currentfill}%
\pgfsetlinewidth{0.311001pt}%
\definecolor{currentstroke}{rgb}{1.000000,1.000000,1.000000}%
\pgfsetstrokecolor{currentstroke}%
\pgfsetdash{}{0pt}%
\pgfpathmoveto{\pgfqpoint{4.216593in}{1.214092in}}%
\pgfpathcurveto{\pgfqpoint{4.223726in}{1.214092in}}{\pgfqpoint{4.230568in}{1.216925in}}{\pgfqpoint{4.235612in}{1.221969in}}%
\pgfpathcurveto{\pgfqpoint{4.240655in}{1.227013in}}{\pgfqpoint{4.243489in}{1.233854in}}{\pgfqpoint{4.243489in}{1.240987in}}%
\pgfpathcurveto{\pgfqpoint{4.243489in}{1.248120in}}{\pgfqpoint{4.240655in}{1.254962in}}{\pgfqpoint{4.235612in}{1.260005in}}%
\pgfpathcurveto{\pgfqpoint{4.230568in}{1.265049in}}{\pgfqpoint{4.223726in}{1.267883in}}{\pgfqpoint{4.216593in}{1.267883in}}%
\pgfpathcurveto{\pgfqpoint{4.209461in}{1.267883in}}{\pgfqpoint{4.202619in}{1.265049in}}{\pgfqpoint{4.197575in}{1.260005in}}%
\pgfpathcurveto{\pgfqpoint{4.192532in}{1.254962in}}{\pgfqpoint{4.189698in}{1.248120in}}{\pgfqpoint{4.189698in}{1.240987in}}%
\pgfpathcurveto{\pgfqpoint{4.189698in}{1.233854in}}{\pgfqpoint{4.192532in}{1.227013in}}{\pgfqpoint{4.197575in}{1.221969in}}%
\pgfpathcurveto{\pgfqpoint{4.202619in}{1.216925in}}{\pgfqpoint{4.209461in}{1.214092in}}{\pgfqpoint{4.216593in}{1.214092in}}%
\pgfpathclose%
\pgfusepath{stroke,fill}%
\end{pgfscope}%
\begin{pgfscope}%
\pgfpathrectangle{\pgfqpoint{2.867647in}{0.500000in}}{\pgfqpoint{1.764706in}{1.700000in}}%
\pgfusepath{clip}%
\pgfsetbuttcap%
\pgfsetroundjoin%
\definecolor{currentfill}{rgb}{0.537262,0.115874,0.356429}%
\pgfsetfillcolor{currentfill}%
\pgfsetlinewidth{0.311001pt}%
\definecolor{currentstroke}{rgb}{1.000000,1.000000,1.000000}%
\pgfsetstrokecolor{currentstroke}%
\pgfsetdash{}{0pt}%
\pgfpathmoveto{\pgfqpoint{3.715290in}{0.925127in}}%
\pgfpathcurveto{\pgfqpoint{3.722422in}{0.925127in}}{\pgfqpoint{3.729264in}{0.927961in}}{\pgfqpoint{3.734308in}{0.933004in}}%
\pgfpathcurveto{\pgfqpoint{3.739351in}{0.938048in}}{\pgfqpoint{3.742185in}{0.944890in}}{\pgfqpoint{3.742185in}{0.952023in}}%
\pgfpathcurveto{\pgfqpoint{3.742185in}{0.959155in}}{\pgfqpoint{3.739351in}{0.965997in}}{\pgfqpoint{3.734308in}{0.971041in}}%
\pgfpathcurveto{\pgfqpoint{3.729264in}{0.976084in}}{\pgfqpoint{3.722422in}{0.978918in}}{\pgfqpoint{3.715290in}{0.978918in}}%
\pgfpathcurveto{\pgfqpoint{3.708157in}{0.978918in}}{\pgfqpoint{3.701315in}{0.976084in}}{\pgfqpoint{3.696271in}{0.971041in}}%
\pgfpathcurveto{\pgfqpoint{3.691228in}{0.965997in}}{\pgfqpoint{3.688394in}{0.959155in}}{\pgfqpoint{3.688394in}{0.952023in}}%
\pgfpathcurveto{\pgfqpoint{3.688394in}{0.944890in}}{\pgfqpoint{3.691228in}{0.938048in}}{\pgfqpoint{3.696271in}{0.933004in}}%
\pgfpathcurveto{\pgfqpoint{3.701315in}{0.927961in}}{\pgfqpoint{3.708157in}{0.925127in}}{\pgfqpoint{3.715290in}{0.925127in}}%
\pgfpathclose%
\pgfusepath{stroke,fill}%
\end{pgfscope}%
\begin{pgfscope}%
\pgfpathrectangle{\pgfqpoint{2.867647in}{0.500000in}}{\pgfqpoint{1.764706in}{1.700000in}}%
\pgfusepath{clip}%
\pgfsetbuttcap%
\pgfsetroundjoin%
\definecolor{currentfill}{rgb}{0.866416,0.173878,0.270708}%
\pgfsetfillcolor{currentfill}%
\pgfsetlinewidth{0.311001pt}%
\definecolor{currentstroke}{rgb}{1.000000,1.000000,1.000000}%
\pgfsetstrokecolor{currentstroke}%
\pgfsetdash{}{0pt}%
\pgfpathmoveto{\pgfqpoint{3.750691in}{1.793138in}}%
\pgfpathcurveto{\pgfqpoint{3.757824in}{1.793138in}}{\pgfqpoint{3.764665in}{1.795972in}}{\pgfqpoint{3.769709in}{1.801015in}}%
\pgfpathcurveto{\pgfqpoint{3.774753in}{1.806059in}}{\pgfqpoint{3.777587in}{1.812901in}}{\pgfqpoint{3.777587in}{1.820033in}}%
\pgfpathcurveto{\pgfqpoint{3.777587in}{1.827166in}}{\pgfqpoint{3.774753in}{1.834008in}}{\pgfqpoint{3.769709in}{1.839052in}}%
\pgfpathcurveto{\pgfqpoint{3.764665in}{1.844095in}}{\pgfqpoint{3.757824in}{1.846929in}}{\pgfqpoint{3.750691in}{1.846929in}}%
\pgfpathcurveto{\pgfqpoint{3.743558in}{1.846929in}}{\pgfqpoint{3.736716in}{1.844095in}}{\pgfqpoint{3.731673in}{1.839052in}}%
\pgfpathcurveto{\pgfqpoint{3.726629in}{1.834008in}}{\pgfqpoint{3.723795in}{1.827166in}}{\pgfqpoint{3.723795in}{1.820033in}}%
\pgfpathcurveto{\pgfqpoint{3.723795in}{1.812901in}}{\pgfqpoint{3.726629in}{1.806059in}}{\pgfqpoint{3.731673in}{1.801015in}}%
\pgfpathcurveto{\pgfqpoint{3.736716in}{1.795972in}}{\pgfqpoint{3.743558in}{1.793138in}}{\pgfqpoint{3.750691in}{1.793138in}}%
\pgfpathclose%
\pgfusepath{stroke,fill}%
\end{pgfscope}%
\begin{pgfscope}%
\pgfpathrectangle{\pgfqpoint{2.867647in}{0.500000in}}{\pgfqpoint{1.764706in}{1.700000in}}%
\pgfusepath{clip}%
\pgfsetbuttcap%
\pgfsetroundjoin%
\definecolor{currentfill}{rgb}{0.963190,0.619109,0.458249}%
\pgfsetfillcolor{currentfill}%
\pgfsetlinewidth{0.311001pt}%
\definecolor{currentstroke}{rgb}{1.000000,1.000000,1.000000}%
\pgfsetstrokecolor{currentstroke}%
\pgfsetdash{}{0pt}%
\pgfpathmoveto{\pgfqpoint{3.950944in}{1.746019in}}%
\pgfpathcurveto{\pgfqpoint{3.958077in}{1.746019in}}{\pgfqpoint{3.964919in}{1.748853in}}{\pgfqpoint{3.969962in}{1.753897in}}%
\pgfpathcurveto{\pgfqpoint{3.975006in}{1.758941in}}{\pgfqpoint{3.977840in}{1.765782in}}{\pgfqpoint{3.977840in}{1.772915in}}%
\pgfpathcurveto{\pgfqpoint{3.977840in}{1.780048in}}{\pgfqpoint{3.975006in}{1.786889in}}{\pgfqpoint{3.969962in}{1.791933in}}%
\pgfpathcurveto{\pgfqpoint{3.964919in}{1.796977in}}{\pgfqpoint{3.958077in}{1.799811in}}{\pgfqpoint{3.950944in}{1.799811in}}%
\pgfpathcurveto{\pgfqpoint{3.943811in}{1.799811in}}{\pgfqpoint{3.936970in}{1.796977in}}{\pgfqpoint{3.931926in}{1.791933in}}%
\pgfpathcurveto{\pgfqpoint{3.926882in}{1.786889in}}{\pgfqpoint{3.924048in}{1.780048in}}{\pgfqpoint{3.924048in}{1.772915in}}%
\pgfpathcurveto{\pgfqpoint{3.924048in}{1.765782in}}{\pgfqpoint{3.926882in}{1.758941in}}{\pgfqpoint{3.931926in}{1.753897in}}%
\pgfpathcurveto{\pgfqpoint{3.936970in}{1.748853in}}{\pgfqpoint{3.943811in}{1.746019in}}{\pgfqpoint{3.950944in}{1.746019in}}%
\pgfpathclose%
\pgfusepath{stroke,fill}%
\end{pgfscope}%
\begin{pgfscope}%
\pgfpathrectangle{\pgfqpoint{2.867647in}{0.500000in}}{\pgfqpoint{1.764706in}{1.700000in}}%
\pgfusepath{clip}%
\pgfsetbuttcap%
\pgfsetroundjoin%
\definecolor{currentfill}{rgb}{0.966560,0.756582,0.625273}%
\pgfsetfillcolor{currentfill}%
\pgfsetlinewidth{0.311001pt}%
\definecolor{currentstroke}{rgb}{1.000000,1.000000,1.000000}%
\pgfsetstrokecolor{currentstroke}%
\pgfsetdash{}{0pt}%
\pgfpathmoveto{\pgfqpoint{4.088130in}{1.348567in}}%
\pgfpathcurveto{\pgfqpoint{4.095263in}{1.348567in}}{\pgfqpoint{4.102105in}{1.351401in}}{\pgfqpoint{4.107148in}{1.356444in}}%
\pgfpathcurveto{\pgfqpoint{4.112192in}{1.361488in}}{\pgfqpoint{4.115026in}{1.368330in}}{\pgfqpoint{4.115026in}{1.375462in}}%
\pgfpathcurveto{\pgfqpoint{4.115026in}{1.382595in}}{\pgfqpoint{4.112192in}{1.389437in}}{\pgfqpoint{4.107148in}{1.394481in}}%
\pgfpathcurveto{\pgfqpoint{4.102105in}{1.399524in}}{\pgfqpoint{4.095263in}{1.402358in}}{\pgfqpoint{4.088130in}{1.402358in}}%
\pgfpathcurveto{\pgfqpoint{4.080997in}{1.402358in}}{\pgfqpoint{4.074156in}{1.399524in}}{\pgfqpoint{4.069112in}{1.394481in}}%
\pgfpathcurveto{\pgfqpoint{4.064068in}{1.389437in}}{\pgfqpoint{4.061235in}{1.382595in}}{\pgfqpoint{4.061235in}{1.375462in}}%
\pgfpathcurveto{\pgfqpoint{4.061235in}{1.368330in}}{\pgfqpoint{4.064068in}{1.361488in}}{\pgfqpoint{4.069112in}{1.356444in}}%
\pgfpathcurveto{\pgfqpoint{4.074156in}{1.351401in}}{\pgfqpoint{4.080997in}{1.348567in}}{\pgfqpoint{4.088130in}{1.348567in}}%
\pgfpathclose%
\pgfusepath{stroke,fill}%
\end{pgfscope}%
\begin{pgfscope}%
\pgfpathrectangle{\pgfqpoint{2.867647in}{0.500000in}}{\pgfqpoint{1.764706in}{1.700000in}}%
\pgfusepath{clip}%
\pgfsetbuttcap%
\pgfsetroundjoin%
\definecolor{currentfill}{rgb}{0.964306,0.663930,0.507747}%
\pgfsetfillcolor{currentfill}%
\pgfsetlinewidth{0.311001pt}%
\definecolor{currentstroke}{rgb}{1.000000,1.000000,1.000000}%
\pgfsetstrokecolor{currentstroke}%
\pgfsetdash{}{0pt}%
\pgfpathmoveto{\pgfqpoint{4.298409in}{1.482742in}}%
\pgfpathcurveto{\pgfqpoint{4.305541in}{1.482742in}}{\pgfqpoint{4.312383in}{1.485576in}}{\pgfqpoint{4.317427in}{1.490620in}}%
\pgfpathcurveto{\pgfqpoint{4.322470in}{1.495664in}}{\pgfqpoint{4.325304in}{1.502505in}}{\pgfqpoint{4.325304in}{1.509638in}}%
\pgfpathcurveto{\pgfqpoint{4.325304in}{1.516771in}}{\pgfqpoint{4.322470in}{1.523613in}}{\pgfqpoint{4.317427in}{1.528656in}}%
\pgfpathcurveto{\pgfqpoint{4.312383in}{1.533700in}}{\pgfqpoint{4.305541in}{1.536534in}}{\pgfqpoint{4.298409in}{1.536534in}}%
\pgfpathcurveto{\pgfqpoint{4.291276in}{1.536534in}}{\pgfqpoint{4.284434in}{1.533700in}}{\pgfqpoint{4.279390in}{1.528656in}}%
\pgfpathcurveto{\pgfqpoint{4.274347in}{1.523613in}}{\pgfqpoint{4.271513in}{1.516771in}}{\pgfqpoint{4.271513in}{1.509638in}}%
\pgfpathcurveto{\pgfqpoint{4.271513in}{1.502505in}}{\pgfqpoint{4.274347in}{1.495664in}}{\pgfqpoint{4.279390in}{1.490620in}}%
\pgfpathcurveto{\pgfqpoint{4.284434in}{1.485576in}}{\pgfqpoint{4.291276in}{1.482742in}}{\pgfqpoint{4.298409in}{1.482742in}}%
\pgfpathclose%
\pgfusepath{stroke,fill}%
\end{pgfscope}%
\begin{pgfscope}%
\pgfpathrectangle{\pgfqpoint{2.867647in}{0.500000in}}{\pgfqpoint{1.764706in}{1.700000in}}%
\pgfusepath{clip}%
\pgfsetbuttcap%
\pgfsetroundjoin%
\definecolor{currentfill}{rgb}{0.963728,0.638439,0.479050}%
\pgfsetfillcolor{currentfill}%
\pgfsetlinewidth{0.311001pt}%
\definecolor{currentstroke}{rgb}{1.000000,1.000000,1.000000}%
\pgfsetstrokecolor{currentstroke}%
\pgfsetdash{}{0pt}%
\pgfpathmoveto{\pgfqpoint{4.053061in}{1.250566in}}%
\pgfpathcurveto{\pgfqpoint{4.060193in}{1.250566in}}{\pgfqpoint{4.067035in}{1.253400in}}{\pgfqpoint{4.072079in}{1.258444in}}%
\pgfpathcurveto{\pgfqpoint{4.077122in}{1.263488in}}{\pgfqpoint{4.079956in}{1.270329in}}{\pgfqpoint{4.079956in}{1.277462in}}%
\pgfpathcurveto{\pgfqpoint{4.079956in}{1.284595in}}{\pgfqpoint{4.077122in}{1.291436in}}{\pgfqpoint{4.072079in}{1.296480in}}%
\pgfpathcurveto{\pgfqpoint{4.067035in}{1.301524in}}{\pgfqpoint{4.060193in}{1.304358in}}{\pgfqpoint{4.053061in}{1.304358in}}%
\pgfpathcurveto{\pgfqpoint{4.045928in}{1.304358in}}{\pgfqpoint{4.039086in}{1.301524in}}{\pgfqpoint{4.034043in}{1.296480in}}%
\pgfpathcurveto{\pgfqpoint{4.028999in}{1.291436in}}{\pgfqpoint{4.026165in}{1.284595in}}{\pgfqpoint{4.026165in}{1.277462in}}%
\pgfpathcurveto{\pgfqpoint{4.026165in}{1.270329in}}{\pgfqpoint{4.028999in}{1.263488in}}{\pgfqpoint{4.034043in}{1.258444in}}%
\pgfpathcurveto{\pgfqpoint{4.039086in}{1.253400in}}{\pgfqpoint{4.045928in}{1.250566in}}{\pgfqpoint{4.053061in}{1.250566in}}%
\pgfpathclose%
\pgfusepath{stroke,fill}%
\end{pgfscope}%
\begin{pgfscope}%
\pgfpathrectangle{\pgfqpoint{2.867647in}{0.500000in}}{\pgfqpoint{1.764706in}{1.700000in}}%
\pgfusepath{clip}%
\pgfsetbuttcap%
\pgfsetroundjoin%
\definecolor{currentfill}{rgb}{0.974412,0.862387,0.780156}%
\pgfsetfillcolor{currentfill}%
\pgfsetlinewidth{0.311001pt}%
\definecolor{currentstroke}{rgb}{1.000000,1.000000,1.000000}%
\pgfsetstrokecolor{currentstroke}%
\pgfsetdash{}{0pt}%
\pgfpathmoveto{\pgfqpoint{4.126834in}{1.005021in}}%
\pgfpathcurveto{\pgfqpoint{4.133967in}{1.005021in}}{\pgfqpoint{4.140809in}{1.007855in}}{\pgfqpoint{4.145852in}{1.012899in}}%
\pgfpathcurveto{\pgfqpoint{4.150896in}{1.017942in}}{\pgfqpoint{4.153730in}{1.024784in}}{\pgfqpoint{4.153730in}{1.031917in}}%
\pgfpathcurveto{\pgfqpoint{4.153730in}{1.039050in}}{\pgfqpoint{4.150896in}{1.045891in}}{\pgfqpoint{4.145852in}{1.050935in}}%
\pgfpathcurveto{\pgfqpoint{4.140809in}{1.055979in}}{\pgfqpoint{4.133967in}{1.058813in}}{\pgfqpoint{4.126834in}{1.058813in}}%
\pgfpathcurveto{\pgfqpoint{4.119701in}{1.058813in}}{\pgfqpoint{4.112860in}{1.055979in}}{\pgfqpoint{4.107816in}{1.050935in}}%
\pgfpathcurveto{\pgfqpoint{4.102772in}{1.045891in}}{\pgfqpoint{4.099938in}{1.039050in}}{\pgfqpoint{4.099938in}{1.031917in}}%
\pgfpathcurveto{\pgfqpoint{4.099938in}{1.024784in}}{\pgfqpoint{4.102772in}{1.017942in}}{\pgfqpoint{4.107816in}{1.012899in}}%
\pgfpathcurveto{\pgfqpoint{4.112860in}{1.007855in}}{\pgfqpoint{4.119701in}{1.005021in}}{\pgfqpoint{4.126834in}{1.005021in}}%
\pgfpathclose%
\pgfusepath{stroke,fill}%
\end{pgfscope}%
\begin{pgfscope}%
\pgfpathrectangle{\pgfqpoint{2.867647in}{0.500000in}}{\pgfqpoint{1.764706in}{1.700000in}}%
\pgfusepath{clip}%
\pgfsetbuttcap%
\pgfsetroundjoin%
\definecolor{currentfill}{rgb}{0.843354,0.145567,0.284808}%
\pgfsetfillcolor{currentfill}%
\pgfsetlinewidth{0.311001pt}%
\definecolor{currentstroke}{rgb}{1.000000,1.000000,1.000000}%
\pgfsetstrokecolor{currentstroke}%
\pgfsetdash{}{0pt}%
\pgfpathmoveto{\pgfqpoint{4.374691in}{1.104554in}}%
\pgfpathcurveto{\pgfqpoint{4.381824in}{1.104554in}}{\pgfqpoint{4.388665in}{1.107388in}}{\pgfqpoint{4.393709in}{1.112431in}}%
\pgfpathcurveto{\pgfqpoint{4.398753in}{1.117475in}}{\pgfqpoint{4.401587in}{1.124317in}}{\pgfqpoint{4.401587in}{1.131450in}}%
\pgfpathcurveto{\pgfqpoint{4.401587in}{1.138582in}}{\pgfqpoint{4.398753in}{1.145424in}}{\pgfqpoint{4.393709in}{1.150468in}}%
\pgfpathcurveto{\pgfqpoint{4.388665in}{1.155511in}}{\pgfqpoint{4.381824in}{1.158345in}}{\pgfqpoint{4.374691in}{1.158345in}}%
\pgfpathcurveto{\pgfqpoint{4.367558in}{1.158345in}}{\pgfqpoint{4.360717in}{1.155511in}}{\pgfqpoint{4.355673in}{1.150468in}}%
\pgfpathcurveto{\pgfqpoint{4.350629in}{1.145424in}}{\pgfqpoint{4.347795in}{1.138582in}}{\pgfqpoint{4.347795in}{1.131450in}}%
\pgfpathcurveto{\pgfqpoint{4.347795in}{1.124317in}}{\pgfqpoint{4.350629in}{1.117475in}}{\pgfqpoint{4.355673in}{1.112431in}}%
\pgfpathcurveto{\pgfqpoint{4.360717in}{1.107388in}}{\pgfqpoint{4.367558in}{1.104554in}}{\pgfqpoint{4.374691in}{1.104554in}}%
\pgfpathclose%
\pgfusepath{stroke,fill}%
\end{pgfscope}%
\begin{pgfscope}%
\pgfpathrectangle{\pgfqpoint{2.867647in}{0.500000in}}{\pgfqpoint{1.764706in}{1.700000in}}%
\pgfusepath{clip}%
\pgfsetbuttcap%
\pgfsetroundjoin%
\definecolor{currentfill}{rgb}{0.961433,0.573272,0.412036}%
\pgfsetfillcolor{currentfill}%
\pgfsetlinewidth{0.311001pt}%
\definecolor{currentstroke}{rgb}{1.000000,1.000000,1.000000}%
\pgfsetstrokecolor{currentstroke}%
\pgfsetdash{}{0pt}%
\pgfpathmoveto{\pgfqpoint{3.968071in}{1.597157in}}%
\pgfpathcurveto{\pgfqpoint{3.975203in}{1.597157in}}{\pgfqpoint{3.982045in}{1.599991in}}{\pgfqpoint{3.987089in}{1.605035in}}%
\pgfpathcurveto{\pgfqpoint{3.992132in}{1.610079in}}{\pgfqpoint{3.994966in}{1.616920in}}{\pgfqpoint{3.994966in}{1.624053in}}%
\pgfpathcurveto{\pgfqpoint{3.994966in}{1.631186in}}{\pgfqpoint{3.992132in}{1.638028in}}{\pgfqpoint{3.987089in}{1.643071in}}%
\pgfpathcurveto{\pgfqpoint{3.982045in}{1.648115in}}{\pgfqpoint{3.975203in}{1.650949in}}{\pgfqpoint{3.968071in}{1.650949in}}%
\pgfpathcurveto{\pgfqpoint{3.960938in}{1.650949in}}{\pgfqpoint{3.954096in}{1.648115in}}{\pgfqpoint{3.949052in}{1.643071in}}%
\pgfpathcurveto{\pgfqpoint{3.944009in}{1.638028in}}{\pgfqpoint{3.941175in}{1.631186in}}{\pgfqpoint{3.941175in}{1.624053in}}%
\pgfpathcurveto{\pgfqpoint{3.941175in}{1.616920in}}{\pgfqpoint{3.944009in}{1.610079in}}{\pgfqpoint{3.949052in}{1.605035in}}%
\pgfpathcurveto{\pgfqpoint{3.954096in}{1.599991in}}{\pgfqpoint{3.960938in}{1.597157in}}{\pgfqpoint{3.968071in}{1.597157in}}%
\pgfpathclose%
\pgfusepath{stroke,fill}%
\end{pgfscope}%
\begin{pgfscope}%
\pgfpathrectangle{\pgfqpoint{2.867647in}{0.500000in}}{\pgfqpoint{1.764706in}{1.700000in}}%
\pgfusepath{clip}%
\pgfsetbuttcap%
\pgfsetroundjoin%
\definecolor{currentfill}{rgb}{0.973271,0.850724,0.762998}%
\pgfsetfillcolor{currentfill}%
\pgfsetlinewidth{0.311001pt}%
\definecolor{currentstroke}{rgb}{1.000000,1.000000,1.000000}%
\pgfsetstrokecolor{currentstroke}%
\pgfsetdash{}{0pt}%
\pgfpathmoveto{\pgfqpoint{4.074718in}{1.056816in}}%
\pgfpathcurveto{\pgfqpoint{4.081851in}{1.056816in}}{\pgfqpoint{4.088693in}{1.059650in}}{\pgfqpoint{4.093736in}{1.064694in}}%
\pgfpathcurveto{\pgfqpoint{4.098780in}{1.069738in}}{\pgfqpoint{4.101614in}{1.076579in}}{\pgfqpoint{4.101614in}{1.083712in}}%
\pgfpathcurveto{\pgfqpoint{4.101614in}{1.090845in}}{\pgfqpoint{4.098780in}{1.097687in}}{\pgfqpoint{4.093736in}{1.102730in}}%
\pgfpathcurveto{\pgfqpoint{4.088693in}{1.107774in}}{\pgfqpoint{4.081851in}{1.110608in}}{\pgfqpoint{4.074718in}{1.110608in}}%
\pgfpathcurveto{\pgfqpoint{4.067585in}{1.110608in}}{\pgfqpoint{4.060744in}{1.107774in}}{\pgfqpoint{4.055700in}{1.102730in}}%
\pgfpathcurveto{\pgfqpoint{4.050656in}{1.097687in}}{\pgfqpoint{4.047822in}{1.090845in}}{\pgfqpoint{4.047822in}{1.083712in}}%
\pgfpathcurveto{\pgfqpoint{4.047822in}{1.076579in}}{\pgfqpoint{4.050656in}{1.069738in}}{\pgfqpoint{4.055700in}{1.064694in}}%
\pgfpathcurveto{\pgfqpoint{4.060744in}{1.059650in}}{\pgfqpoint{4.067585in}{1.056816in}}{\pgfqpoint{4.074718in}{1.056816in}}%
\pgfpathclose%
\pgfusepath{stroke,fill}%
\end{pgfscope}%
\begin{pgfscope}%
\pgfpathrectangle{\pgfqpoint{2.867647in}{0.500000in}}{\pgfqpoint{1.764706in}{1.700000in}}%
\pgfusepath{clip}%
\pgfsetbuttcap%
\pgfsetroundjoin%
\definecolor{currentfill}{rgb}{0.956268,0.491874,0.339856}%
\pgfsetfillcolor{currentfill}%
\pgfsetlinewidth{0.311001pt}%
\definecolor{currentstroke}{rgb}{1.000000,1.000000,1.000000}%
\pgfsetstrokecolor{currentstroke}%
\pgfsetdash{}{0pt}%
\pgfpathmoveto{\pgfqpoint{4.014007in}{1.838089in}}%
\pgfpathcurveto{\pgfqpoint{4.021140in}{1.838089in}}{\pgfqpoint{4.027981in}{1.840923in}}{\pgfqpoint{4.033025in}{1.845966in}}%
\pgfpathcurveto{\pgfqpoint{4.038069in}{1.851010in}}{\pgfqpoint{4.040902in}{1.857852in}}{\pgfqpoint{4.040902in}{1.864984in}}%
\pgfpathcurveto{\pgfqpoint{4.040902in}{1.872117in}}{\pgfqpoint{4.038069in}{1.878959in}}{\pgfqpoint{4.033025in}{1.884003in}}%
\pgfpathcurveto{\pgfqpoint{4.027981in}{1.889046in}}{\pgfqpoint{4.021140in}{1.891880in}}{\pgfqpoint{4.014007in}{1.891880in}}%
\pgfpathcurveto{\pgfqpoint{4.006874in}{1.891880in}}{\pgfqpoint{4.000032in}{1.889046in}}{\pgfqpoint{3.994989in}{1.884003in}}%
\pgfpathcurveto{\pgfqpoint{3.989945in}{1.878959in}}{\pgfqpoint{3.987111in}{1.872117in}}{\pgfqpoint{3.987111in}{1.864984in}}%
\pgfpathcurveto{\pgfqpoint{3.987111in}{1.857852in}}{\pgfqpoint{3.989945in}{1.851010in}}{\pgfqpoint{3.994989in}{1.845966in}}%
\pgfpathcurveto{\pgfqpoint{4.000032in}{1.840923in}}{\pgfqpoint{4.006874in}{1.838089in}}{\pgfqpoint{4.014007in}{1.838089in}}%
\pgfpathclose%
\pgfusepath{stroke,fill}%
\end{pgfscope}%
\begin{pgfscope}%
\pgfpathrectangle{\pgfqpoint{2.867647in}{0.500000in}}{\pgfqpoint{1.764706in}{1.700000in}}%
\pgfusepath{clip}%
\pgfsetbuttcap%
\pgfsetroundjoin%
\definecolor{currentfill}{rgb}{0.970718,0.821518,0.719872}%
\pgfsetfillcolor{currentfill}%
\pgfsetlinewidth{0.311001pt}%
\definecolor{currentstroke}{rgb}{1.000000,1.000000,1.000000}%
\pgfsetstrokecolor{currentstroke}%
\pgfsetdash{}{0pt}%
\pgfpathmoveto{\pgfqpoint{4.049184in}{1.005776in}}%
\pgfpathcurveto{\pgfqpoint{4.056316in}{1.005776in}}{\pgfqpoint{4.063158in}{1.008610in}}{\pgfqpoint{4.068202in}{1.013653in}}%
\pgfpathcurveto{\pgfqpoint{4.073245in}{1.018697in}}{\pgfqpoint{4.076079in}{1.025539in}}{\pgfqpoint{4.076079in}{1.032671in}}%
\pgfpathcurveto{\pgfqpoint{4.076079in}{1.039804in}}{\pgfqpoint{4.073245in}{1.046646in}}{\pgfqpoint{4.068202in}{1.051690in}}%
\pgfpathcurveto{\pgfqpoint{4.063158in}{1.056733in}}{\pgfqpoint{4.056316in}{1.059567in}}{\pgfqpoint{4.049184in}{1.059567in}}%
\pgfpathcurveto{\pgfqpoint{4.042051in}{1.059567in}}{\pgfqpoint{4.035209in}{1.056733in}}{\pgfqpoint{4.030165in}{1.051690in}}%
\pgfpathcurveto{\pgfqpoint{4.025122in}{1.046646in}}{\pgfqpoint{4.022288in}{1.039804in}}{\pgfqpoint{4.022288in}{1.032671in}}%
\pgfpathcurveto{\pgfqpoint{4.022288in}{1.025539in}}{\pgfqpoint{4.025122in}{1.018697in}}{\pgfqpoint{4.030165in}{1.013653in}}%
\pgfpathcurveto{\pgfqpoint{4.035209in}{1.008610in}}{\pgfqpoint{4.042051in}{1.005776in}}{\pgfqpoint{4.049184in}{1.005776in}}%
\pgfpathclose%
\pgfusepath{stroke,fill}%
\end{pgfscope}%
\begin{pgfscope}%
\pgfpathrectangle{\pgfqpoint{2.867647in}{0.500000in}}{\pgfqpoint{1.764706in}{1.700000in}}%
\pgfusepath{clip}%
\pgfsetbuttcap%
\pgfsetroundjoin%
\definecolor{currentfill}{rgb}{0.970718,0.821518,0.719872}%
\pgfsetfillcolor{currentfill}%
\pgfsetlinewidth{0.311001pt}%
\definecolor{currentstroke}{rgb}{1.000000,1.000000,1.000000}%
\pgfsetstrokecolor{currentstroke}%
\pgfsetdash{}{0pt}%
\pgfpathmoveto{\pgfqpoint{4.054893in}{1.051115in}}%
\pgfpathcurveto{\pgfqpoint{4.062026in}{1.051115in}}{\pgfqpoint{4.068867in}{1.053949in}}{\pgfqpoint{4.073911in}{1.058992in}}%
\pgfpathcurveto{\pgfqpoint{4.078955in}{1.064036in}}{\pgfqpoint{4.081788in}{1.070878in}}{\pgfqpoint{4.081788in}{1.078010in}}%
\pgfpathcurveto{\pgfqpoint{4.081788in}{1.085143in}}{\pgfqpoint{4.078955in}{1.091985in}}{\pgfqpoint{4.073911in}{1.097029in}}%
\pgfpathcurveto{\pgfqpoint{4.068867in}{1.102072in}}{\pgfqpoint{4.062026in}{1.104906in}}{\pgfqpoint{4.054893in}{1.104906in}}%
\pgfpathcurveto{\pgfqpoint{4.047760in}{1.104906in}}{\pgfqpoint{4.040918in}{1.102072in}}{\pgfqpoint{4.035875in}{1.097029in}}%
\pgfpathcurveto{\pgfqpoint{4.030831in}{1.091985in}}{\pgfqpoint{4.027997in}{1.085143in}}{\pgfqpoint{4.027997in}{1.078010in}}%
\pgfpathcurveto{\pgfqpoint{4.027997in}{1.070878in}}{\pgfqpoint{4.030831in}{1.064036in}}{\pgfqpoint{4.035875in}{1.058992in}}%
\pgfpathcurveto{\pgfqpoint{4.040918in}{1.053949in}}{\pgfqpoint{4.047760in}{1.051115in}}{\pgfqpoint{4.054893in}{1.051115in}}%
\pgfpathclose%
\pgfusepath{stroke,fill}%
\end{pgfscope}%
\begin{pgfscope}%
\pgfpathrectangle{\pgfqpoint{2.867647in}{0.500000in}}{\pgfqpoint{1.764706in}{1.700000in}}%
\pgfusepath{clip}%
\pgfsetbuttcap%
\pgfsetroundjoin%
\definecolor{currentfill}{rgb}{0.956817,0.498820,0.345554}%
\pgfsetfillcolor{currentfill}%
\pgfsetlinewidth{0.311001pt}%
\definecolor{currentstroke}{rgb}{1.000000,1.000000,1.000000}%
\pgfsetstrokecolor{currentstroke}%
\pgfsetdash{}{0pt}%
\pgfpathmoveto{\pgfqpoint{4.051393in}{0.822189in}}%
\pgfpathcurveto{\pgfqpoint{4.058526in}{0.822189in}}{\pgfqpoint{4.065368in}{0.825023in}}{\pgfqpoint{4.070411in}{0.830066in}}%
\pgfpathcurveto{\pgfqpoint{4.075455in}{0.835110in}}{\pgfqpoint{4.078289in}{0.841952in}}{\pgfqpoint{4.078289in}{0.849084in}}%
\pgfpathcurveto{\pgfqpoint{4.078289in}{0.856217in}}{\pgfqpoint{4.075455in}{0.863059in}}{\pgfqpoint{4.070411in}{0.868103in}}%
\pgfpathcurveto{\pgfqpoint{4.065368in}{0.873146in}}{\pgfqpoint{4.058526in}{0.875980in}}{\pgfqpoint{4.051393in}{0.875980in}}%
\pgfpathcurveto{\pgfqpoint{4.044260in}{0.875980in}}{\pgfqpoint{4.037419in}{0.873146in}}{\pgfqpoint{4.032375in}{0.868103in}}%
\pgfpathcurveto{\pgfqpoint{4.027331in}{0.863059in}}{\pgfqpoint{4.024497in}{0.856217in}}{\pgfqpoint{4.024497in}{0.849084in}}%
\pgfpathcurveto{\pgfqpoint{4.024497in}{0.841952in}}{\pgfqpoint{4.027331in}{0.835110in}}{\pgfqpoint{4.032375in}{0.830066in}}%
\pgfpathcurveto{\pgfqpoint{4.037419in}{0.825023in}}{\pgfqpoint{4.044260in}{0.822189in}}{\pgfqpoint{4.051393in}{0.822189in}}%
\pgfpathclose%
\pgfusepath{stroke,fill}%
\end{pgfscope}%
\begin{pgfscope}%
\pgfpathrectangle{\pgfqpoint{2.867647in}{0.500000in}}{\pgfqpoint{1.764706in}{1.700000in}}%
\pgfusepath{clip}%
\pgfsetbuttcap%
\pgfsetroundjoin%
\definecolor{currentfill}{rgb}{0.979124,0.903132,0.839793}%
\pgfsetfillcolor{currentfill}%
\pgfsetlinewidth{0.311001pt}%
\definecolor{currentstroke}{rgb}{1.000000,1.000000,1.000000}%
\pgfsetstrokecolor{currentstroke}%
\pgfsetdash{}{0pt}%
\pgfpathmoveto{\pgfqpoint{4.136090in}{1.480074in}}%
\pgfpathcurveto{\pgfqpoint{4.143223in}{1.480074in}}{\pgfqpoint{4.150065in}{1.482908in}}{\pgfqpoint{4.155108in}{1.487952in}}%
\pgfpathcurveto{\pgfqpoint{4.160152in}{1.492995in}}{\pgfqpoint{4.162986in}{1.499837in}}{\pgfqpoint{4.162986in}{1.506970in}}%
\pgfpathcurveto{\pgfqpoint{4.162986in}{1.514103in}}{\pgfqpoint{4.160152in}{1.520944in}}{\pgfqpoint{4.155108in}{1.525988in}}%
\pgfpathcurveto{\pgfqpoint{4.150065in}{1.531032in}}{\pgfqpoint{4.143223in}{1.533865in}}{\pgfqpoint{4.136090in}{1.533865in}}%
\pgfpathcurveto{\pgfqpoint{4.128957in}{1.533865in}}{\pgfqpoint{4.122116in}{1.531032in}}{\pgfqpoint{4.117072in}{1.525988in}}%
\pgfpathcurveto{\pgfqpoint{4.112028in}{1.520944in}}{\pgfqpoint{4.109194in}{1.514103in}}{\pgfqpoint{4.109194in}{1.506970in}}%
\pgfpathcurveto{\pgfqpoint{4.109194in}{1.499837in}}{\pgfqpoint{4.112028in}{1.492995in}}{\pgfqpoint{4.117072in}{1.487952in}}%
\pgfpathcurveto{\pgfqpoint{4.122116in}{1.482908in}}{\pgfqpoint{4.128957in}{1.480074in}}{\pgfqpoint{4.136090in}{1.480074in}}%
\pgfpathclose%
\pgfusepath{stroke,fill}%
\end{pgfscope}%
\begin{pgfscope}%
\pgfpathrectangle{\pgfqpoint{2.867647in}{0.500000in}}{\pgfqpoint{1.764706in}{1.700000in}}%
\pgfusepath{clip}%
\pgfsetbuttcap%
\pgfsetroundjoin%
\definecolor{currentfill}{rgb}{0.972726,0.844889,0.754401}%
\pgfsetfillcolor{currentfill}%
\pgfsetlinewidth{0.311001pt}%
\definecolor{currentstroke}{rgb}{1.000000,1.000000,1.000000}%
\pgfsetstrokecolor{currentstroke}%
\pgfsetdash{}{0pt}%
\pgfpathmoveto{\pgfqpoint{4.075926in}{0.995279in}}%
\pgfpathcurveto{\pgfqpoint{4.083059in}{0.995279in}}{\pgfqpoint{4.089901in}{0.998113in}}{\pgfqpoint{4.094944in}{1.003156in}}%
\pgfpathcurveto{\pgfqpoint{4.099988in}{1.008200in}}{\pgfqpoint{4.102822in}{1.015042in}}{\pgfqpoint{4.102822in}{1.022175in}}%
\pgfpathcurveto{\pgfqpoint{4.102822in}{1.029307in}}{\pgfqpoint{4.099988in}{1.036149in}}{\pgfqpoint{4.094944in}{1.041193in}}%
\pgfpathcurveto{\pgfqpoint{4.089901in}{1.046236in}}{\pgfqpoint{4.083059in}{1.049070in}}{\pgfqpoint{4.075926in}{1.049070in}}%
\pgfpathcurveto{\pgfqpoint{4.068793in}{1.049070in}}{\pgfqpoint{4.061952in}{1.046236in}}{\pgfqpoint{4.056908in}{1.041193in}}%
\pgfpathcurveto{\pgfqpoint{4.051864in}{1.036149in}}{\pgfqpoint{4.049030in}{1.029307in}}{\pgfqpoint{4.049030in}{1.022175in}}%
\pgfpathcurveto{\pgfqpoint{4.049030in}{1.015042in}}{\pgfqpoint{4.051864in}{1.008200in}}{\pgfqpoint{4.056908in}{1.003156in}}%
\pgfpathcurveto{\pgfqpoint{4.061952in}{0.998113in}}{\pgfqpoint{4.068793in}{0.995279in}}{\pgfqpoint{4.075926in}{0.995279in}}%
\pgfpathclose%
\pgfusepath{stroke,fill}%
\end{pgfscope}%
\begin{pgfscope}%
\pgfpathrectangle{\pgfqpoint{2.867647in}{0.500000in}}{\pgfqpoint{1.764706in}{1.700000in}}%
\pgfusepath{clip}%
\pgfsetbuttcap%
\pgfsetroundjoin%
\definecolor{currentfill}{rgb}{0.964173,0.657587,0.500469}%
\pgfsetfillcolor{currentfill}%
\pgfsetlinewidth{0.311001pt}%
\definecolor{currentstroke}{rgb}{1.000000,1.000000,1.000000}%
\pgfsetstrokecolor{currentstroke}%
\pgfsetdash{}{0pt}%
\pgfpathmoveto{\pgfqpoint{4.027664in}{1.488417in}}%
\pgfpathcurveto{\pgfqpoint{4.034797in}{1.488417in}}{\pgfqpoint{4.041639in}{1.491251in}}{\pgfqpoint{4.046683in}{1.496294in}}%
\pgfpathcurveto{\pgfqpoint{4.051726in}{1.501338in}}{\pgfqpoint{4.054560in}{1.508180in}}{\pgfqpoint{4.054560in}{1.515313in}}%
\pgfpathcurveto{\pgfqpoint{4.054560in}{1.522445in}}{\pgfqpoint{4.051726in}{1.529287in}}{\pgfqpoint{4.046683in}{1.534331in}}%
\pgfpathcurveto{\pgfqpoint{4.041639in}{1.539374in}}{\pgfqpoint{4.034797in}{1.542208in}}{\pgfqpoint{4.027664in}{1.542208in}}%
\pgfpathcurveto{\pgfqpoint{4.020532in}{1.542208in}}{\pgfqpoint{4.013690in}{1.539374in}}{\pgfqpoint{4.008646in}{1.534331in}}%
\pgfpathcurveto{\pgfqpoint{4.003603in}{1.529287in}}{\pgfqpoint{4.000769in}{1.522445in}}{\pgfqpoint{4.000769in}{1.515313in}}%
\pgfpathcurveto{\pgfqpoint{4.000769in}{1.508180in}}{\pgfqpoint{4.003603in}{1.501338in}}{\pgfqpoint{4.008646in}{1.496294in}}%
\pgfpathcurveto{\pgfqpoint{4.013690in}{1.491251in}}{\pgfqpoint{4.020532in}{1.488417in}}{\pgfqpoint{4.027664in}{1.488417in}}%
\pgfpathclose%
\pgfusepath{stroke,fill}%
\end{pgfscope}%
\begin{pgfscope}%
\pgfpathrectangle{\pgfqpoint{2.867647in}{0.500000in}}{\pgfqpoint{1.764706in}{1.700000in}}%
\pgfusepath{clip}%
\pgfsetbuttcap%
\pgfsetroundjoin%
\definecolor{currentfill}{rgb}{0.981377,0.920617,0.865369}%
\pgfsetfillcolor{currentfill}%
\pgfsetlinewidth{0.311001pt}%
\definecolor{currentstroke}{rgb}{1.000000,1.000000,1.000000}%
\pgfsetstrokecolor{currentstroke}%
\pgfsetdash{}{0pt}%
\pgfpathmoveto{\pgfqpoint{4.161183in}{1.202262in}}%
\pgfpathcurveto{\pgfqpoint{4.168316in}{1.202262in}}{\pgfqpoint{4.175157in}{1.205095in}}{\pgfqpoint{4.180201in}{1.210139in}}%
\pgfpathcurveto{\pgfqpoint{4.185245in}{1.215183in}}{\pgfqpoint{4.188079in}{1.222024in}}{\pgfqpoint{4.188079in}{1.229157in}}%
\pgfpathcurveto{\pgfqpoint{4.188079in}{1.236290in}}{\pgfqpoint{4.185245in}{1.243132in}}{\pgfqpoint{4.180201in}{1.248175in}}%
\pgfpathcurveto{\pgfqpoint{4.175157in}{1.253219in}}{\pgfqpoint{4.168316in}{1.256053in}}{\pgfqpoint{4.161183in}{1.256053in}}%
\pgfpathcurveto{\pgfqpoint{4.154050in}{1.256053in}}{\pgfqpoint{4.147208in}{1.253219in}}{\pgfqpoint{4.142165in}{1.248175in}}%
\pgfpathcurveto{\pgfqpoint{4.137121in}{1.243132in}}{\pgfqpoint{4.134287in}{1.236290in}}{\pgfqpoint{4.134287in}{1.229157in}}%
\pgfpathcurveto{\pgfqpoint{4.134287in}{1.222024in}}{\pgfqpoint{4.137121in}{1.215183in}}{\pgfqpoint{4.142165in}{1.210139in}}%
\pgfpathcurveto{\pgfqpoint{4.147208in}{1.205095in}}{\pgfqpoint{4.154050in}{1.202262in}}{\pgfqpoint{4.161183in}{1.202262in}}%
\pgfpathclose%
\pgfusepath{stroke,fill}%
\end{pgfscope}%
\begin{pgfscope}%
\pgfpathrectangle{\pgfqpoint{2.867647in}{0.500000in}}{\pgfqpoint{1.764706in}{1.700000in}}%
\pgfusepath{clip}%
\pgfsetbuttcap%
\pgfsetroundjoin%
\definecolor{currentfill}{rgb}{0.010608,0.018082,0.100187}%
\pgfsetfillcolor{currentfill}%
\pgfsetlinewidth{0.311001pt}%
\definecolor{currentstroke}{rgb}{1.000000,1.000000,1.000000}%
\pgfsetstrokecolor{currentstroke}%
\pgfsetdash{}{0pt}%
\pgfpathmoveto{\pgfqpoint{4.521085in}{1.056396in}}%
\pgfpathcurveto{\pgfqpoint{4.528218in}{1.056396in}}{\pgfqpoint{4.535059in}{1.059230in}}{\pgfqpoint{4.540103in}{1.064274in}}%
\pgfpathcurveto{\pgfqpoint{4.545147in}{1.069317in}}{\pgfqpoint{4.547981in}{1.076159in}}{\pgfqpoint{4.547981in}{1.083292in}}%
\pgfpathcurveto{\pgfqpoint{4.547981in}{1.090425in}}{\pgfqpoint{4.545147in}{1.097266in}}{\pgfqpoint{4.540103in}{1.102310in}}%
\pgfpathcurveto{\pgfqpoint{4.535059in}{1.107354in}}{\pgfqpoint{4.528218in}{1.110188in}}{\pgfqpoint{4.521085in}{1.110188in}}%
\pgfpathcurveto{\pgfqpoint{4.513952in}{1.110188in}}{\pgfqpoint{4.507110in}{1.107354in}}{\pgfqpoint{4.502067in}{1.102310in}}%
\pgfpathcurveto{\pgfqpoint{4.497023in}{1.097266in}}{\pgfqpoint{4.494189in}{1.090425in}}{\pgfqpoint{4.494189in}{1.083292in}}%
\pgfpathcurveto{\pgfqpoint{4.494189in}{1.076159in}}{\pgfqpoint{4.497023in}{1.069317in}}{\pgfqpoint{4.502067in}{1.064274in}}%
\pgfpathcurveto{\pgfqpoint{4.507110in}{1.059230in}}{\pgfqpoint{4.513952in}{1.056396in}}{\pgfqpoint{4.521085in}{1.056396in}}%
\pgfpathclose%
\pgfusepath{stroke,fill}%
\end{pgfscope}%
\begin{pgfscope}%
\pgfpathrectangle{\pgfqpoint{2.867647in}{0.500000in}}{\pgfqpoint{1.764706in}{1.700000in}}%
\pgfusepath{clip}%
\pgfsetbuttcap%
\pgfsetroundjoin%
\definecolor{currentfill}{rgb}{0.970255,0.815666,0.711203}%
\pgfsetfillcolor{currentfill}%
\pgfsetlinewidth{0.311001pt}%
\definecolor{currentstroke}{rgb}{1.000000,1.000000,1.000000}%
\pgfsetstrokecolor{currentstroke}%
\pgfsetdash{}{0pt}%
\pgfpathmoveto{\pgfqpoint{4.088576in}{0.950829in}}%
\pgfpathcurveto{\pgfqpoint{4.095709in}{0.950829in}}{\pgfqpoint{4.102551in}{0.953663in}}{\pgfqpoint{4.107594in}{0.958707in}}%
\pgfpathcurveto{\pgfqpoint{4.112638in}{0.963750in}}{\pgfqpoint{4.115472in}{0.970592in}}{\pgfqpoint{4.115472in}{0.977725in}}%
\pgfpathcurveto{\pgfqpoint{4.115472in}{0.984857in}}{\pgfqpoint{4.112638in}{0.991699in}}{\pgfqpoint{4.107594in}{0.996743in}}%
\pgfpathcurveto{\pgfqpoint{4.102551in}{1.001786in}}{\pgfqpoint{4.095709in}{1.004620in}}{\pgfqpoint{4.088576in}{1.004620in}}%
\pgfpathcurveto{\pgfqpoint{4.081443in}{1.004620in}}{\pgfqpoint{4.074602in}{1.001786in}}{\pgfqpoint{4.069558in}{0.996743in}}%
\pgfpathcurveto{\pgfqpoint{4.064514in}{0.991699in}}{\pgfqpoint{4.061680in}{0.984857in}}{\pgfqpoint{4.061680in}{0.977725in}}%
\pgfpathcurveto{\pgfqpoint{4.061680in}{0.970592in}}{\pgfqpoint{4.064514in}{0.963750in}}{\pgfqpoint{4.069558in}{0.958707in}}%
\pgfpathcurveto{\pgfqpoint{4.074602in}{0.953663in}}{\pgfqpoint{4.081443in}{0.950829in}}{\pgfqpoint{4.088576in}{0.950829in}}%
\pgfpathclose%
\pgfusepath{stroke,fill}%
\end{pgfscope}%
\begin{pgfscope}%
\pgfpathrectangle{\pgfqpoint{2.867647in}{0.500000in}}{\pgfqpoint{1.764706in}{1.700000in}}%
\pgfusepath{clip}%
\pgfsetbuttcap%
\pgfsetroundjoin%
\definecolor{currentfill}{rgb}{0.972201,0.839051,0.745789}%
\pgfsetfillcolor{currentfill}%
\pgfsetlinewidth{0.311001pt}%
\definecolor{currentstroke}{rgb}{1.000000,1.000000,1.000000}%
\pgfsetstrokecolor{currentstroke}%
\pgfsetdash{}{0pt}%
\pgfpathmoveto{\pgfqpoint{4.255464in}{1.207282in}}%
\pgfpathcurveto{\pgfqpoint{4.262597in}{1.207282in}}{\pgfqpoint{4.269438in}{1.210116in}}{\pgfqpoint{4.274482in}{1.215160in}}%
\pgfpathcurveto{\pgfqpoint{4.279526in}{1.220203in}}{\pgfqpoint{4.282360in}{1.227045in}}{\pgfqpoint{4.282360in}{1.234178in}}%
\pgfpathcurveto{\pgfqpoint{4.282360in}{1.241311in}}{\pgfqpoint{4.279526in}{1.248152in}}{\pgfqpoint{4.274482in}{1.253196in}}%
\pgfpathcurveto{\pgfqpoint{4.269438in}{1.258240in}}{\pgfqpoint{4.262597in}{1.261074in}}{\pgfqpoint{4.255464in}{1.261074in}}%
\pgfpathcurveto{\pgfqpoint{4.248331in}{1.261074in}}{\pgfqpoint{4.241489in}{1.258240in}}{\pgfqpoint{4.236446in}{1.253196in}}%
\pgfpathcurveto{\pgfqpoint{4.231402in}{1.248152in}}{\pgfqpoint{4.228568in}{1.241311in}}{\pgfqpoint{4.228568in}{1.234178in}}%
\pgfpathcurveto{\pgfqpoint{4.228568in}{1.227045in}}{\pgfqpoint{4.231402in}{1.220203in}}{\pgfqpoint{4.236446in}{1.215160in}}%
\pgfpathcurveto{\pgfqpoint{4.241489in}{1.210116in}}{\pgfqpoint{4.248331in}{1.207282in}}{\pgfqpoint{4.255464in}{1.207282in}}%
\pgfpathclose%
\pgfusepath{stroke,fill}%
\end{pgfscope}%
\begin{pgfscope}%
\pgfpathrectangle{\pgfqpoint{2.867647in}{0.500000in}}{\pgfqpoint{1.764706in}{1.700000in}}%
\pgfusepath{clip}%
\pgfsetbuttcap%
\pgfsetroundjoin%
\definecolor{currentfill}{rgb}{0.972726,0.844889,0.754401}%
\pgfsetfillcolor{currentfill}%
\pgfsetlinewidth{0.311001pt}%
\definecolor{currentstroke}{rgb}{1.000000,1.000000,1.000000}%
\pgfsetstrokecolor{currentstroke}%
\pgfsetdash{}{0pt}%
\pgfpathmoveto{\pgfqpoint{4.140811in}{0.994383in}}%
\pgfpathcurveto{\pgfqpoint{4.147944in}{0.994383in}}{\pgfqpoint{4.154786in}{0.997217in}}{\pgfqpoint{4.159830in}{1.002261in}}%
\pgfpathcurveto{\pgfqpoint{4.164873in}{1.007305in}}{\pgfqpoint{4.167707in}{1.014146in}}{\pgfqpoint{4.167707in}{1.021279in}}%
\pgfpathcurveto{\pgfqpoint{4.167707in}{1.028412in}}{\pgfqpoint{4.164873in}{1.035254in}}{\pgfqpoint{4.159830in}{1.040297in}}%
\pgfpathcurveto{\pgfqpoint{4.154786in}{1.045341in}}{\pgfqpoint{4.147944in}{1.048175in}}{\pgfqpoint{4.140811in}{1.048175in}}%
\pgfpathcurveto{\pgfqpoint{4.133679in}{1.048175in}}{\pgfqpoint{4.126837in}{1.045341in}}{\pgfqpoint{4.121793in}{1.040297in}}%
\pgfpathcurveto{\pgfqpoint{4.116750in}{1.035254in}}{\pgfqpoint{4.113916in}{1.028412in}}{\pgfqpoint{4.113916in}{1.021279in}}%
\pgfpathcurveto{\pgfqpoint{4.113916in}{1.014146in}}{\pgfqpoint{4.116750in}{1.007305in}}{\pgfqpoint{4.121793in}{1.002261in}}%
\pgfpathcurveto{\pgfqpoint{4.126837in}{0.997217in}}{\pgfqpoint{4.133679in}{0.994383in}}{\pgfqpoint{4.140811in}{0.994383in}}%
\pgfpathclose%
\pgfusepath{stroke,fill}%
\end{pgfscope}%
\begin{pgfscope}%
\pgfpathrectangle{\pgfqpoint{2.867647in}{0.500000in}}{\pgfqpoint{1.764706in}{1.700000in}}%
\pgfusepath{clip}%
\pgfsetbuttcap%
\pgfsetroundjoin%
\definecolor{currentfill}{rgb}{0.981377,0.920617,0.865369}%
\pgfsetfillcolor{currentfill}%
\pgfsetlinewidth{0.311001pt}%
\definecolor{currentstroke}{rgb}{1.000000,1.000000,1.000000}%
\pgfsetstrokecolor{currentstroke}%
\pgfsetdash{}{0pt}%
\pgfpathmoveto{\pgfqpoint{4.170932in}{1.242152in}}%
\pgfpathcurveto{\pgfqpoint{4.178065in}{1.242152in}}{\pgfqpoint{4.184907in}{1.244986in}}{\pgfqpoint{4.189950in}{1.250030in}}%
\pgfpathcurveto{\pgfqpoint{4.194994in}{1.255074in}}{\pgfqpoint{4.197828in}{1.261915in}}{\pgfqpoint{4.197828in}{1.269048in}}%
\pgfpathcurveto{\pgfqpoint{4.197828in}{1.276181in}}{\pgfqpoint{4.194994in}{1.283022in}}{\pgfqpoint{4.189950in}{1.288066in}}%
\pgfpathcurveto{\pgfqpoint{4.184907in}{1.293110in}}{\pgfqpoint{4.178065in}{1.295944in}}{\pgfqpoint{4.170932in}{1.295944in}}%
\pgfpathcurveto{\pgfqpoint{4.163799in}{1.295944in}}{\pgfqpoint{4.156958in}{1.293110in}}{\pgfqpoint{4.151914in}{1.288066in}}%
\pgfpathcurveto{\pgfqpoint{4.146870in}{1.283022in}}{\pgfqpoint{4.144036in}{1.276181in}}{\pgfqpoint{4.144036in}{1.269048in}}%
\pgfpathcurveto{\pgfqpoint{4.144036in}{1.261915in}}{\pgfqpoint{4.146870in}{1.255074in}}{\pgfqpoint{4.151914in}{1.250030in}}%
\pgfpathcurveto{\pgfqpoint{4.156958in}{1.244986in}}{\pgfqpoint{4.163799in}{1.242152in}}{\pgfqpoint{4.170932in}{1.242152in}}%
\pgfpathclose%
\pgfusepath{stroke,fill}%
\end{pgfscope}%
\begin{pgfscope}%
\pgfpathrectangle{\pgfqpoint{2.867647in}{0.500000in}}{\pgfqpoint{1.764706in}{1.700000in}}%
\pgfusepath{clip}%
\pgfsetbuttcap%
\pgfsetroundjoin%
\definecolor{currentfill}{rgb}{0.970718,0.821518,0.719872}%
\pgfsetfillcolor{currentfill}%
\pgfsetlinewidth{0.311001pt}%
\definecolor{currentstroke}{rgb}{1.000000,1.000000,1.000000}%
\pgfsetstrokecolor{currentstroke}%
\pgfsetdash{}{0pt}%
\pgfpathmoveto{\pgfqpoint{4.270565in}{1.283195in}}%
\pgfpathcurveto{\pgfqpoint{4.277698in}{1.283195in}}{\pgfqpoint{4.284540in}{1.286029in}}{\pgfqpoint{4.289583in}{1.291072in}}%
\pgfpathcurveto{\pgfqpoint{4.294627in}{1.296116in}}{\pgfqpoint{4.297461in}{1.302958in}}{\pgfqpoint{4.297461in}{1.310090in}}%
\pgfpathcurveto{\pgfqpoint{4.297461in}{1.317223in}}{\pgfqpoint{4.294627in}{1.324065in}}{\pgfqpoint{4.289583in}{1.329109in}}%
\pgfpathcurveto{\pgfqpoint{4.284540in}{1.334152in}}{\pgfqpoint{4.277698in}{1.336986in}}{\pgfqpoint{4.270565in}{1.336986in}}%
\pgfpathcurveto{\pgfqpoint{4.263432in}{1.336986in}}{\pgfqpoint{4.256591in}{1.334152in}}{\pgfqpoint{4.251547in}{1.329109in}}%
\pgfpathcurveto{\pgfqpoint{4.246503in}{1.324065in}}{\pgfqpoint{4.243669in}{1.317223in}}{\pgfqpoint{4.243669in}{1.310090in}}%
\pgfpathcurveto{\pgfqpoint{4.243669in}{1.302958in}}{\pgfqpoint{4.246503in}{1.296116in}}{\pgfqpoint{4.251547in}{1.291072in}}%
\pgfpathcurveto{\pgfqpoint{4.256591in}{1.286029in}}{\pgfqpoint{4.263432in}{1.283195in}}{\pgfqpoint{4.270565in}{1.283195in}}%
\pgfpathclose%
\pgfusepath{stroke,fill}%
\end{pgfscope}%
\begin{pgfscope}%
\pgfpathrectangle{\pgfqpoint{2.867647in}{0.500000in}}{\pgfqpoint{1.764706in}{1.700000in}}%
\pgfusepath{clip}%
\pgfsetbuttcap%
\pgfsetroundjoin%
\definecolor{currentfill}{rgb}{0.970255,0.815666,0.711203}%
\pgfsetfillcolor{currentfill}%
\pgfsetlinewidth{0.311001pt}%
\definecolor{currentstroke}{rgb}{1.000000,1.000000,1.000000}%
\pgfsetstrokecolor{currentstroke}%
\pgfsetdash{}{0pt}%
\pgfpathmoveto{\pgfqpoint{4.157730in}{1.660675in}}%
\pgfpathcurveto{\pgfqpoint{4.164863in}{1.660675in}}{\pgfqpoint{4.171705in}{1.663509in}}{\pgfqpoint{4.176749in}{1.668552in}}%
\pgfpathcurveto{\pgfqpoint{4.181792in}{1.673596in}}{\pgfqpoint{4.184626in}{1.680438in}}{\pgfqpoint{4.184626in}{1.687570in}}%
\pgfpathcurveto{\pgfqpoint{4.184626in}{1.694703in}}{\pgfqpoint{4.181792in}{1.701545in}}{\pgfqpoint{4.176749in}{1.706589in}}%
\pgfpathcurveto{\pgfqpoint{4.171705in}{1.711632in}}{\pgfqpoint{4.164863in}{1.714466in}}{\pgfqpoint{4.157730in}{1.714466in}}%
\pgfpathcurveto{\pgfqpoint{4.150598in}{1.714466in}}{\pgfqpoint{4.143756in}{1.711632in}}{\pgfqpoint{4.138712in}{1.706589in}}%
\pgfpathcurveto{\pgfqpoint{4.133669in}{1.701545in}}{\pgfqpoint{4.130835in}{1.694703in}}{\pgfqpoint{4.130835in}{1.687570in}}%
\pgfpathcurveto{\pgfqpoint{4.130835in}{1.680438in}}{\pgfqpoint{4.133669in}{1.673596in}}{\pgfqpoint{4.138712in}{1.668552in}}%
\pgfpathcurveto{\pgfqpoint{4.143756in}{1.663509in}}{\pgfqpoint{4.150598in}{1.660675in}}{\pgfqpoint{4.157730in}{1.660675in}}%
\pgfpathclose%
\pgfusepath{stroke,fill}%
\end{pgfscope}%
\begin{pgfscope}%
\pgfpathrectangle{\pgfqpoint{2.867647in}{0.500000in}}{\pgfqpoint{1.764706in}{1.700000in}}%
\pgfusepath{clip}%
\pgfsetbuttcap%
\pgfsetroundjoin%
\definecolor{currentfill}{rgb}{0.975644,0.874038,0.797253}%
\pgfsetfillcolor{currentfill}%
\pgfsetlinewidth{0.311001pt}%
\definecolor{currentstroke}{rgb}{1.000000,1.000000,1.000000}%
\pgfsetstrokecolor{currentstroke}%
\pgfsetdash{}{0pt}%
\pgfpathmoveto{\pgfqpoint{4.093156in}{1.066586in}}%
\pgfpathcurveto{\pgfqpoint{4.100289in}{1.066586in}}{\pgfqpoint{4.107130in}{1.069420in}}{\pgfqpoint{4.112174in}{1.074464in}}%
\pgfpathcurveto{\pgfqpoint{4.117218in}{1.079508in}}{\pgfqpoint{4.120052in}{1.086349in}}{\pgfqpoint{4.120052in}{1.093482in}}%
\pgfpathcurveto{\pgfqpoint{4.120052in}{1.100615in}}{\pgfqpoint{4.117218in}{1.107457in}}{\pgfqpoint{4.112174in}{1.112500in}}%
\pgfpathcurveto{\pgfqpoint{4.107130in}{1.117544in}}{\pgfqpoint{4.100289in}{1.120378in}}{\pgfqpoint{4.093156in}{1.120378in}}%
\pgfpathcurveto{\pgfqpoint{4.086023in}{1.120378in}}{\pgfqpoint{4.079181in}{1.117544in}}{\pgfqpoint{4.074138in}{1.112500in}}%
\pgfpathcurveto{\pgfqpoint{4.069094in}{1.107457in}}{\pgfqpoint{4.066260in}{1.100615in}}{\pgfqpoint{4.066260in}{1.093482in}}%
\pgfpathcurveto{\pgfqpoint{4.066260in}{1.086349in}}{\pgfqpoint{4.069094in}{1.079508in}}{\pgfqpoint{4.074138in}{1.074464in}}%
\pgfpathcurveto{\pgfqpoint{4.079181in}{1.069420in}}{\pgfqpoint{4.086023in}{1.066586in}}{\pgfqpoint{4.093156in}{1.066586in}}%
\pgfpathclose%
\pgfusepath{stroke,fill}%
\end{pgfscope}%
\begin{pgfscope}%
\pgfpathrectangle{\pgfqpoint{2.867647in}{0.500000in}}{\pgfqpoint{1.764706in}{1.700000in}}%
\pgfusepath{clip}%
\pgfsetbuttcap%
\pgfsetroundjoin%
\definecolor{currentfill}{rgb}{0.919781,0.275262,0.242460}%
\pgfsetfillcolor{currentfill}%
\pgfsetlinewidth{0.311001pt}%
\definecolor{currentstroke}{rgb}{1.000000,1.000000,1.000000}%
\pgfsetstrokecolor{currentstroke}%
\pgfsetdash{}{0pt}%
\pgfpathmoveto{\pgfqpoint{3.861147in}{1.630404in}}%
\pgfpathcurveto{\pgfqpoint{3.868280in}{1.630404in}}{\pgfqpoint{3.875122in}{1.633238in}}{\pgfqpoint{3.880166in}{1.638281in}}%
\pgfpathcurveto{\pgfqpoint{3.885209in}{1.643325in}}{\pgfqpoint{3.888043in}{1.650167in}}{\pgfqpoint{3.888043in}{1.657299in}}%
\pgfpathcurveto{\pgfqpoint{3.888043in}{1.664432in}}{\pgfqpoint{3.885209in}{1.671274in}}{\pgfqpoint{3.880166in}{1.676318in}}%
\pgfpathcurveto{\pgfqpoint{3.875122in}{1.681361in}}{\pgfqpoint{3.868280in}{1.684195in}}{\pgfqpoint{3.861147in}{1.684195in}}%
\pgfpathcurveto{\pgfqpoint{3.854015in}{1.684195in}}{\pgfqpoint{3.847173in}{1.681361in}}{\pgfqpoint{3.842129in}{1.676318in}}%
\pgfpathcurveto{\pgfqpoint{3.837086in}{1.671274in}}{\pgfqpoint{3.834252in}{1.664432in}}{\pgfqpoint{3.834252in}{1.657299in}}%
\pgfpathcurveto{\pgfqpoint{3.834252in}{1.650167in}}{\pgfqpoint{3.837086in}{1.643325in}}{\pgfqpoint{3.842129in}{1.638281in}}%
\pgfpathcurveto{\pgfqpoint{3.847173in}{1.633238in}}{\pgfqpoint{3.854015in}{1.630404in}}{\pgfqpoint{3.861147in}{1.630404in}}%
\pgfpathclose%
\pgfusepath{stroke,fill}%
\end{pgfscope}%
\begin{pgfscope}%
\pgfpathrectangle{\pgfqpoint{2.867647in}{0.500000in}}{\pgfqpoint{1.764706in}{1.700000in}}%
\pgfusepath{clip}%
\pgfsetbuttcap%
\pgfsetroundjoin%
\definecolor{currentfill}{rgb}{0.979891,0.908948,0.848279}%
\pgfsetfillcolor{currentfill}%
\pgfsetlinewidth{0.311001pt}%
\definecolor{currentstroke}{rgb}{1.000000,1.000000,1.000000}%
\pgfsetstrokecolor{currentstroke}%
\pgfsetdash{}{0pt}%
\pgfpathmoveto{\pgfqpoint{4.217445in}{1.305316in}}%
\pgfpathcurveto{\pgfqpoint{4.224577in}{1.305316in}}{\pgfqpoint{4.231419in}{1.308150in}}{\pgfqpoint{4.236463in}{1.313194in}}%
\pgfpathcurveto{\pgfqpoint{4.241506in}{1.318237in}}{\pgfqpoint{4.244340in}{1.325079in}}{\pgfqpoint{4.244340in}{1.332212in}}%
\pgfpathcurveto{\pgfqpoint{4.244340in}{1.339344in}}{\pgfqpoint{4.241506in}{1.346186in}}{\pgfqpoint{4.236463in}{1.351230in}}%
\pgfpathcurveto{\pgfqpoint{4.231419in}{1.356273in}}{\pgfqpoint{4.224577in}{1.359107in}}{\pgfqpoint{4.217445in}{1.359107in}}%
\pgfpathcurveto{\pgfqpoint{4.210312in}{1.359107in}}{\pgfqpoint{4.203470in}{1.356273in}}{\pgfqpoint{4.198427in}{1.351230in}}%
\pgfpathcurveto{\pgfqpoint{4.193383in}{1.346186in}}{\pgfqpoint{4.190549in}{1.339344in}}{\pgfqpoint{4.190549in}{1.332212in}}%
\pgfpathcurveto{\pgfqpoint{4.190549in}{1.325079in}}{\pgfqpoint{4.193383in}{1.318237in}}{\pgfqpoint{4.198427in}{1.313194in}}%
\pgfpathcurveto{\pgfqpoint{4.203470in}{1.308150in}}{\pgfqpoint{4.210312in}{1.305316in}}{\pgfqpoint{4.217445in}{1.305316in}}%
\pgfpathclose%
\pgfusepath{stroke,fill}%
\end{pgfscope}%
\begin{pgfscope}%
\pgfpathrectangle{\pgfqpoint{2.867647in}{0.500000in}}{\pgfqpoint{1.764706in}{1.700000in}}%
\pgfusepath{clip}%
\pgfsetbuttcap%
\pgfsetroundjoin%
\definecolor{currentfill}{rgb}{0.979891,0.908948,0.848279}%
\pgfsetfillcolor{currentfill}%
\pgfsetlinewidth{0.311001pt}%
\definecolor{currentstroke}{rgb}{1.000000,1.000000,1.000000}%
\pgfsetstrokecolor{currentstroke}%
\pgfsetdash{}{0pt}%
\pgfpathmoveto{\pgfqpoint{4.163197in}{1.412404in}}%
\pgfpathcurveto{\pgfqpoint{4.170330in}{1.412404in}}{\pgfqpoint{4.177172in}{1.415238in}}{\pgfqpoint{4.182215in}{1.420282in}}%
\pgfpathcurveto{\pgfqpoint{4.187259in}{1.425325in}}{\pgfqpoint{4.190093in}{1.432167in}}{\pgfqpoint{4.190093in}{1.439300in}}%
\pgfpathcurveto{\pgfqpoint{4.190093in}{1.446433in}}{\pgfqpoint{4.187259in}{1.453274in}}{\pgfqpoint{4.182215in}{1.458318in}}%
\pgfpathcurveto{\pgfqpoint{4.177172in}{1.463362in}}{\pgfqpoint{4.170330in}{1.466196in}}{\pgfqpoint{4.163197in}{1.466196in}}%
\pgfpathcurveto{\pgfqpoint{4.156064in}{1.466196in}}{\pgfqpoint{4.149223in}{1.463362in}}{\pgfqpoint{4.144179in}{1.458318in}}%
\pgfpathcurveto{\pgfqpoint{4.139135in}{1.453274in}}{\pgfqpoint{4.136302in}{1.446433in}}{\pgfqpoint{4.136302in}{1.439300in}}%
\pgfpathcurveto{\pgfqpoint{4.136302in}{1.432167in}}{\pgfqpoint{4.139135in}{1.425325in}}{\pgfqpoint{4.144179in}{1.420282in}}%
\pgfpathcurveto{\pgfqpoint{4.149223in}{1.415238in}}{\pgfqpoint{4.156064in}{1.412404in}}{\pgfqpoint{4.163197in}{1.412404in}}%
\pgfpathclose%
\pgfusepath{stroke,fill}%
\end{pgfscope}%
\begin{pgfscope}%
\pgfpathrectangle{\pgfqpoint{2.867647in}{0.500000in}}{\pgfqpoint{1.764706in}{1.700000in}}%
\pgfusepath{clip}%
\pgfsetbuttcap%
\pgfsetroundjoin%
\definecolor{currentfill}{rgb}{0.962283,0.593046,0.431453}%
\pgfsetfillcolor{currentfill}%
\pgfsetlinewidth{0.311001pt}%
\definecolor{currentstroke}{rgb}{1.000000,1.000000,1.000000}%
\pgfsetstrokecolor{currentstroke}%
\pgfsetdash{}{0pt}%
\pgfpathmoveto{\pgfqpoint{3.957380in}{1.775049in}}%
\pgfpathcurveto{\pgfqpoint{3.964513in}{1.775049in}}{\pgfqpoint{3.971354in}{1.777883in}}{\pgfqpoint{3.976398in}{1.782926in}}%
\pgfpathcurveto{\pgfqpoint{3.981442in}{1.787970in}}{\pgfqpoint{3.984275in}{1.794812in}}{\pgfqpoint{3.984275in}{1.801944in}}%
\pgfpathcurveto{\pgfqpoint{3.984275in}{1.809077in}}{\pgfqpoint{3.981442in}{1.815919in}}{\pgfqpoint{3.976398in}{1.820963in}}%
\pgfpathcurveto{\pgfqpoint{3.971354in}{1.826006in}}{\pgfqpoint{3.964513in}{1.828840in}}{\pgfqpoint{3.957380in}{1.828840in}}%
\pgfpathcurveto{\pgfqpoint{3.950247in}{1.828840in}}{\pgfqpoint{3.943405in}{1.826006in}}{\pgfqpoint{3.938362in}{1.820963in}}%
\pgfpathcurveto{\pgfqpoint{3.933318in}{1.815919in}}{\pgfqpoint{3.930484in}{1.809077in}}{\pgfqpoint{3.930484in}{1.801944in}}%
\pgfpathcurveto{\pgfqpoint{3.930484in}{1.794812in}}{\pgfqpoint{3.933318in}{1.787970in}}{\pgfqpoint{3.938362in}{1.782926in}}%
\pgfpathcurveto{\pgfqpoint{3.943405in}{1.777883in}}{\pgfqpoint{3.950247in}{1.775049in}}{\pgfqpoint{3.957380in}{1.775049in}}%
\pgfpathclose%
\pgfusepath{stroke,fill}%
\end{pgfscope}%
\begin{pgfscope}%
\pgfpathrectangle{\pgfqpoint{2.867647in}{0.500000in}}{\pgfqpoint{1.764706in}{1.700000in}}%
\pgfusepath{clip}%
\pgfsetbuttcap%
\pgfsetroundjoin%
\definecolor{currentfill}{rgb}{0.973271,0.850724,0.762998}%
\pgfsetfillcolor{currentfill}%
\pgfsetlinewidth{0.311001pt}%
\definecolor{currentstroke}{rgb}{1.000000,1.000000,1.000000}%
\pgfsetstrokecolor{currentstroke}%
\pgfsetdash{}{0pt}%
\pgfpathmoveto{\pgfqpoint{4.120681in}{1.387500in}}%
\pgfpathcurveto{\pgfqpoint{4.127814in}{1.387500in}}{\pgfqpoint{4.134656in}{1.390334in}}{\pgfqpoint{4.139700in}{1.395378in}}%
\pgfpathcurveto{\pgfqpoint{4.144743in}{1.400421in}}{\pgfqpoint{4.147577in}{1.407263in}}{\pgfqpoint{4.147577in}{1.414396in}}%
\pgfpathcurveto{\pgfqpoint{4.147577in}{1.421529in}}{\pgfqpoint{4.144743in}{1.428370in}}{\pgfqpoint{4.139700in}{1.433414in}}%
\pgfpathcurveto{\pgfqpoint{4.134656in}{1.438458in}}{\pgfqpoint{4.127814in}{1.441292in}}{\pgfqpoint{4.120681in}{1.441292in}}%
\pgfpathcurveto{\pgfqpoint{4.113549in}{1.441292in}}{\pgfqpoint{4.106707in}{1.438458in}}{\pgfqpoint{4.101663in}{1.433414in}}%
\pgfpathcurveto{\pgfqpoint{4.096620in}{1.428370in}}{\pgfqpoint{4.093786in}{1.421529in}}{\pgfqpoint{4.093786in}{1.414396in}}%
\pgfpathcurveto{\pgfqpoint{4.093786in}{1.407263in}}{\pgfqpoint{4.096620in}{1.400421in}}{\pgfqpoint{4.101663in}{1.395378in}}%
\pgfpathcurveto{\pgfqpoint{4.106707in}{1.390334in}}{\pgfqpoint{4.113549in}{1.387500in}}{\pgfqpoint{4.120681in}{1.387500in}}%
\pgfpathclose%
\pgfusepath{stroke,fill}%
\end{pgfscope}%
\begin{pgfscope}%
\pgfpathrectangle{\pgfqpoint{2.867647in}{0.500000in}}{\pgfqpoint{1.764706in}{1.700000in}}%
\pgfusepath{clip}%
\pgfsetbuttcap%
\pgfsetroundjoin%
\definecolor{currentfill}{rgb}{0.978376,0.897317,0.831308}%
\pgfsetfillcolor{currentfill}%
\pgfsetlinewidth{0.311001pt}%
\definecolor{currentstroke}{rgb}{1.000000,1.000000,1.000000}%
\pgfsetstrokecolor{currentstroke}%
\pgfsetdash{}{0pt}%
\pgfpathmoveto{\pgfqpoint{4.113277in}{1.557138in}}%
\pgfpathcurveto{\pgfqpoint{4.120410in}{1.557138in}}{\pgfqpoint{4.127252in}{1.559971in}}{\pgfqpoint{4.132296in}{1.565015in}}%
\pgfpathcurveto{\pgfqpoint{4.137339in}{1.570059in}}{\pgfqpoint{4.140173in}{1.576900in}}{\pgfqpoint{4.140173in}{1.584033in}}%
\pgfpathcurveto{\pgfqpoint{4.140173in}{1.591166in}}{\pgfqpoint{4.137339in}{1.598008in}}{\pgfqpoint{4.132296in}{1.603051in}}%
\pgfpathcurveto{\pgfqpoint{4.127252in}{1.608095in}}{\pgfqpoint{4.120410in}{1.610929in}}{\pgfqpoint{4.113277in}{1.610929in}}%
\pgfpathcurveto{\pgfqpoint{4.106145in}{1.610929in}}{\pgfqpoint{4.099303in}{1.608095in}}{\pgfqpoint{4.094259in}{1.603051in}}%
\pgfpathcurveto{\pgfqpoint{4.089216in}{1.598008in}}{\pgfqpoint{4.086382in}{1.591166in}}{\pgfqpoint{4.086382in}{1.584033in}}%
\pgfpathcurveto{\pgfqpoint{4.086382in}{1.576900in}}{\pgfqpoint{4.089216in}{1.570059in}}{\pgfqpoint{4.094259in}{1.565015in}}%
\pgfpathcurveto{\pgfqpoint{4.099303in}{1.559971in}}{\pgfqpoint{4.106145in}{1.557138in}}{\pgfqpoint{4.113277in}{1.557138in}}%
\pgfpathclose%
\pgfusepath{stroke,fill}%
\end{pgfscope}%
\begin{pgfscope}%
\pgfpathrectangle{\pgfqpoint{2.867647in}{0.500000in}}{\pgfqpoint{1.764706in}{1.700000in}}%
\pgfusepath{clip}%
\pgfsetbuttcap%
\pgfsetroundjoin%
\definecolor{currentfill}{rgb}{0.980678,0.914765,0.856766}%
\pgfsetfillcolor{currentfill}%
\pgfsetlinewidth{0.311001pt}%
\definecolor{currentstroke}{rgb}{1.000000,1.000000,1.000000}%
\pgfsetstrokecolor{currentstroke}%
\pgfsetdash{}{0pt}%
\pgfpathmoveto{\pgfqpoint{4.151933in}{1.492077in}}%
\pgfpathcurveto{\pgfqpoint{4.159065in}{1.492077in}}{\pgfqpoint{4.165907in}{1.494911in}}{\pgfqpoint{4.170951in}{1.499954in}}%
\pgfpathcurveto{\pgfqpoint{4.175994in}{1.504998in}}{\pgfqpoint{4.178828in}{1.511840in}}{\pgfqpoint{4.178828in}{1.518972in}}%
\pgfpathcurveto{\pgfqpoint{4.178828in}{1.526105in}}{\pgfqpoint{4.175994in}{1.532947in}}{\pgfqpoint{4.170951in}{1.537991in}}%
\pgfpathcurveto{\pgfqpoint{4.165907in}{1.543034in}}{\pgfqpoint{4.159065in}{1.545868in}}{\pgfqpoint{4.151933in}{1.545868in}}%
\pgfpathcurveto{\pgfqpoint{4.144800in}{1.545868in}}{\pgfqpoint{4.137958in}{1.543034in}}{\pgfqpoint{4.132915in}{1.537991in}}%
\pgfpathcurveto{\pgfqpoint{4.127871in}{1.532947in}}{\pgfqpoint{4.125037in}{1.526105in}}{\pgfqpoint{4.125037in}{1.518972in}}%
\pgfpathcurveto{\pgfqpoint{4.125037in}{1.511840in}}{\pgfqpoint{4.127871in}{1.504998in}}{\pgfqpoint{4.132915in}{1.499954in}}%
\pgfpathcurveto{\pgfqpoint{4.137958in}{1.494911in}}{\pgfqpoint{4.144800in}{1.492077in}}{\pgfqpoint{4.151933in}{1.492077in}}%
\pgfpathclose%
\pgfusepath{stroke,fill}%
\end{pgfscope}%
\begin{pgfscope}%
\pgfpathrectangle{\pgfqpoint{2.867647in}{0.500000in}}{\pgfqpoint{1.764706in}{1.700000in}}%
\pgfusepath{clip}%
\pgfsetbuttcap%
\pgfsetroundjoin%
\definecolor{currentfill}{rgb}{0.976287,0.879862,0.805788}%
\pgfsetfillcolor{currentfill}%
\pgfsetlinewidth{0.311001pt}%
\definecolor{currentstroke}{rgb}{1.000000,1.000000,1.000000}%
\pgfsetstrokecolor{currentstroke}%
\pgfsetdash{}{0pt}%
\pgfpathmoveto{\pgfqpoint{4.142897in}{1.618249in}}%
\pgfpathcurveto{\pgfqpoint{4.150029in}{1.618249in}}{\pgfqpoint{4.156871in}{1.621083in}}{\pgfqpoint{4.161915in}{1.626127in}}%
\pgfpathcurveto{\pgfqpoint{4.166958in}{1.631171in}}{\pgfqpoint{4.169792in}{1.638012in}}{\pgfqpoint{4.169792in}{1.645145in}}%
\pgfpathcurveto{\pgfqpoint{4.169792in}{1.652278in}}{\pgfqpoint{4.166958in}{1.659120in}}{\pgfqpoint{4.161915in}{1.664163in}}%
\pgfpathcurveto{\pgfqpoint{4.156871in}{1.669207in}}{\pgfqpoint{4.150029in}{1.672041in}}{\pgfqpoint{4.142897in}{1.672041in}}%
\pgfpathcurveto{\pgfqpoint{4.135764in}{1.672041in}}{\pgfqpoint{4.128922in}{1.669207in}}{\pgfqpoint{4.123878in}{1.664163in}}%
\pgfpathcurveto{\pgfqpoint{4.118835in}{1.659120in}}{\pgfqpoint{4.116001in}{1.652278in}}{\pgfqpoint{4.116001in}{1.645145in}}%
\pgfpathcurveto{\pgfqpoint{4.116001in}{1.638012in}}{\pgfqpoint{4.118835in}{1.631171in}}{\pgfqpoint{4.123878in}{1.626127in}}%
\pgfpathcurveto{\pgfqpoint{4.128922in}{1.621083in}}{\pgfqpoint{4.135764in}{1.618249in}}{\pgfqpoint{4.142897in}{1.618249in}}%
\pgfpathclose%
\pgfusepath{stroke,fill}%
\end{pgfscope}%
\begin{pgfscope}%
\pgfpathrectangle{\pgfqpoint{2.867647in}{0.500000in}}{\pgfqpoint{1.764706in}{1.700000in}}%
\pgfusepath{clip}%
\pgfsetbuttcap%
\pgfsetroundjoin%
\definecolor{currentfill}{rgb}{0.976961,0.885681,0.814303}%
\pgfsetfillcolor{currentfill}%
\pgfsetlinewidth{0.311001pt}%
\definecolor{currentstroke}{rgb}{1.000000,1.000000,1.000000}%
\pgfsetstrokecolor{currentstroke}%
\pgfsetdash{}{0pt}%
\pgfpathmoveto{\pgfqpoint{4.228125in}{1.205517in}}%
\pgfpathcurveto{\pgfqpoint{4.235257in}{1.205517in}}{\pgfqpoint{4.242099in}{1.208351in}}{\pgfqpoint{4.247143in}{1.213395in}}%
\pgfpathcurveto{\pgfqpoint{4.252186in}{1.218439in}}{\pgfqpoint{4.255020in}{1.225280in}}{\pgfqpoint{4.255020in}{1.232413in}}%
\pgfpathcurveto{\pgfqpoint{4.255020in}{1.239546in}}{\pgfqpoint{4.252186in}{1.246388in}}{\pgfqpoint{4.247143in}{1.251431in}}%
\pgfpathcurveto{\pgfqpoint{4.242099in}{1.256475in}}{\pgfqpoint{4.235257in}{1.259309in}}{\pgfqpoint{4.228125in}{1.259309in}}%
\pgfpathcurveto{\pgfqpoint{4.220992in}{1.259309in}}{\pgfqpoint{4.214150in}{1.256475in}}{\pgfqpoint{4.209106in}{1.251431in}}%
\pgfpathcurveto{\pgfqpoint{4.204063in}{1.246388in}}{\pgfqpoint{4.201229in}{1.239546in}}{\pgfqpoint{4.201229in}{1.232413in}}%
\pgfpathcurveto{\pgfqpoint{4.201229in}{1.225280in}}{\pgfqpoint{4.204063in}{1.218439in}}{\pgfqpoint{4.209106in}{1.213395in}}%
\pgfpathcurveto{\pgfqpoint{4.214150in}{1.208351in}}{\pgfqpoint{4.220992in}{1.205517in}}{\pgfqpoint{4.228125in}{1.205517in}}%
\pgfpathclose%
\pgfusepath{stroke,fill}%
\end{pgfscope}%
\begin{pgfscope}%
\pgfpathrectangle{\pgfqpoint{2.867647in}{0.500000in}}{\pgfqpoint{1.764706in}{1.700000in}}%
\pgfusepath{clip}%
\pgfsetbuttcap%
\pgfsetroundjoin%
\definecolor{currentfill}{rgb}{0.972726,0.844889,0.754401}%
\pgfsetfillcolor{currentfill}%
\pgfsetlinewidth{0.311001pt}%
\definecolor{currentstroke}{rgb}{1.000000,1.000000,1.000000}%
\pgfsetstrokecolor{currentstroke}%
\pgfsetdash{}{0pt}%
\pgfpathmoveto{\pgfqpoint{4.181314in}{1.030151in}}%
\pgfpathcurveto{\pgfqpoint{4.188447in}{1.030151in}}{\pgfqpoint{4.195289in}{1.032985in}}{\pgfqpoint{4.200332in}{1.038029in}}%
\pgfpathcurveto{\pgfqpoint{4.205376in}{1.043072in}}{\pgfqpoint{4.208210in}{1.049914in}}{\pgfqpoint{4.208210in}{1.057047in}}%
\pgfpathcurveto{\pgfqpoint{4.208210in}{1.064180in}}{\pgfqpoint{4.205376in}{1.071021in}}{\pgfqpoint{4.200332in}{1.076065in}}%
\pgfpathcurveto{\pgfqpoint{4.195289in}{1.081109in}}{\pgfqpoint{4.188447in}{1.083943in}}{\pgfqpoint{4.181314in}{1.083943in}}%
\pgfpathcurveto{\pgfqpoint{4.174181in}{1.083943in}}{\pgfqpoint{4.167340in}{1.081109in}}{\pgfqpoint{4.162296in}{1.076065in}}%
\pgfpathcurveto{\pgfqpoint{4.157252in}{1.071021in}}{\pgfqpoint{4.154418in}{1.064180in}}{\pgfqpoint{4.154418in}{1.057047in}}%
\pgfpathcurveto{\pgfqpoint{4.154418in}{1.049914in}}{\pgfqpoint{4.157252in}{1.043072in}}{\pgfqpoint{4.162296in}{1.038029in}}%
\pgfpathcurveto{\pgfqpoint{4.167340in}{1.032985in}}{\pgfqpoint{4.174181in}{1.030151in}}{\pgfqpoint{4.181314in}{1.030151in}}%
\pgfpathclose%
\pgfusepath{stroke,fill}%
\end{pgfscope}%
\begin{pgfscope}%
\pgfpathrectangle{\pgfqpoint{2.867647in}{0.500000in}}{\pgfqpoint{1.764706in}{1.700000in}}%
\pgfusepath{clip}%
\pgfsetbuttcap%
\pgfsetroundjoin%
\definecolor{currentfill}{rgb}{0.969803,0.809811,0.702523}%
\pgfsetfillcolor{currentfill}%
\pgfsetlinewidth{0.311001pt}%
\definecolor{currentstroke}{rgb}{1.000000,1.000000,1.000000}%
\pgfsetstrokecolor{currentstroke}%
\pgfsetdash{}{0pt}%
\pgfpathmoveto{\pgfqpoint{4.098957in}{1.701238in}}%
\pgfpathcurveto{\pgfqpoint{4.106089in}{1.701238in}}{\pgfqpoint{4.112931in}{1.704071in}}{\pgfqpoint{4.117975in}{1.709115in}}%
\pgfpathcurveto{\pgfqpoint{4.123018in}{1.714159in}}{\pgfqpoint{4.125852in}{1.721000in}}{\pgfqpoint{4.125852in}{1.728133in}}%
\pgfpathcurveto{\pgfqpoint{4.125852in}{1.735266in}}{\pgfqpoint{4.123018in}{1.742108in}}{\pgfqpoint{4.117975in}{1.747151in}}%
\pgfpathcurveto{\pgfqpoint{4.112931in}{1.752195in}}{\pgfqpoint{4.106089in}{1.755029in}}{\pgfqpoint{4.098957in}{1.755029in}}%
\pgfpathcurveto{\pgfqpoint{4.091824in}{1.755029in}}{\pgfqpoint{4.084982in}{1.752195in}}{\pgfqpoint{4.079938in}{1.747151in}}%
\pgfpathcurveto{\pgfqpoint{4.074895in}{1.742108in}}{\pgfqpoint{4.072061in}{1.735266in}}{\pgfqpoint{4.072061in}{1.728133in}}%
\pgfpathcurveto{\pgfqpoint{4.072061in}{1.721000in}}{\pgfqpoint{4.074895in}{1.714159in}}{\pgfqpoint{4.079938in}{1.709115in}}%
\pgfpathcurveto{\pgfqpoint{4.084982in}{1.704071in}}{\pgfqpoint{4.091824in}{1.701238in}}{\pgfqpoint{4.098957in}{1.701238in}}%
\pgfpathclose%
\pgfusepath{stroke,fill}%
\end{pgfscope}%
\begin{pgfscope}%
\pgfpathrectangle{\pgfqpoint{2.867647in}{0.500000in}}{\pgfqpoint{1.764706in}{1.700000in}}%
\pgfusepath{clip}%
\pgfsetbuttcap%
\pgfsetroundjoin%
\definecolor{currentfill}{rgb}{0.967735,0.780441,0.659127}%
\pgfsetfillcolor{currentfill}%
\pgfsetlinewidth{0.311001pt}%
\definecolor{currentstroke}{rgb}{1.000000,1.000000,1.000000}%
\pgfsetstrokecolor{currentstroke}%
\pgfsetdash{}{0pt}%
\pgfpathmoveto{\pgfqpoint{4.198468in}{1.634926in}}%
\pgfpathcurveto{\pgfqpoint{4.205601in}{1.634926in}}{\pgfqpoint{4.212442in}{1.637760in}}{\pgfqpoint{4.217486in}{1.642804in}}%
\pgfpathcurveto{\pgfqpoint{4.222530in}{1.647847in}}{\pgfqpoint{4.225364in}{1.654689in}}{\pgfqpoint{4.225364in}{1.661822in}}%
\pgfpathcurveto{\pgfqpoint{4.225364in}{1.668955in}}{\pgfqpoint{4.222530in}{1.675796in}}{\pgfqpoint{4.217486in}{1.680840in}}%
\pgfpathcurveto{\pgfqpoint{4.212442in}{1.685884in}}{\pgfqpoint{4.205601in}{1.688717in}}{\pgfqpoint{4.198468in}{1.688717in}}%
\pgfpathcurveto{\pgfqpoint{4.191335in}{1.688717in}}{\pgfqpoint{4.184494in}{1.685884in}}{\pgfqpoint{4.179450in}{1.680840in}}%
\pgfpathcurveto{\pgfqpoint{4.174406in}{1.675796in}}{\pgfqpoint{4.171572in}{1.668955in}}{\pgfqpoint{4.171572in}{1.661822in}}%
\pgfpathcurveto{\pgfqpoint{4.171572in}{1.654689in}}{\pgfqpoint{4.174406in}{1.647847in}}{\pgfqpoint{4.179450in}{1.642804in}}%
\pgfpathcurveto{\pgfqpoint{4.184494in}{1.637760in}}{\pgfqpoint{4.191335in}{1.634926in}}{\pgfqpoint{4.198468in}{1.634926in}}%
\pgfpathclose%
\pgfusepath{stroke,fill}%
\end{pgfscope}%
\begin{pgfscope}%
\pgfpathrectangle{\pgfqpoint{2.867647in}{0.500000in}}{\pgfqpoint{1.764706in}{1.700000in}}%
\pgfusepath{clip}%
\pgfsetbuttcap%
\pgfsetroundjoin%
\definecolor{currentfill}{rgb}{0.966560,0.756582,0.625273}%
\pgfsetfillcolor{currentfill}%
\pgfsetlinewidth{0.311001pt}%
\definecolor{currentstroke}{rgb}{1.000000,1.000000,1.000000}%
\pgfsetstrokecolor{currentstroke}%
\pgfsetdash{}{0pt}%
\pgfpathmoveto{\pgfqpoint{4.237587in}{1.055619in}}%
\pgfpathcurveto{\pgfqpoint{4.244720in}{1.055619in}}{\pgfqpoint{4.251561in}{1.058452in}}{\pgfqpoint{4.256605in}{1.063496in}}%
\pgfpathcurveto{\pgfqpoint{4.261649in}{1.068540in}}{\pgfqpoint{4.264483in}{1.075381in}}{\pgfqpoint{4.264483in}{1.082514in}}%
\pgfpathcurveto{\pgfqpoint{4.264483in}{1.089647in}}{\pgfqpoint{4.261649in}{1.096489in}}{\pgfqpoint{4.256605in}{1.101532in}}%
\pgfpathcurveto{\pgfqpoint{4.251561in}{1.106576in}}{\pgfqpoint{4.244720in}{1.109410in}}{\pgfqpoint{4.237587in}{1.109410in}}%
\pgfpathcurveto{\pgfqpoint{4.230454in}{1.109410in}}{\pgfqpoint{4.223612in}{1.106576in}}{\pgfqpoint{4.218569in}{1.101532in}}%
\pgfpathcurveto{\pgfqpoint{4.213525in}{1.096489in}}{\pgfqpoint{4.210691in}{1.089647in}}{\pgfqpoint{4.210691in}{1.082514in}}%
\pgfpathcurveto{\pgfqpoint{4.210691in}{1.075381in}}{\pgfqpoint{4.213525in}{1.068540in}}{\pgfqpoint{4.218569in}{1.063496in}}%
\pgfpathcurveto{\pgfqpoint{4.223612in}{1.058452in}}{\pgfqpoint{4.230454in}{1.055619in}}{\pgfqpoint{4.237587in}{1.055619in}}%
\pgfpathclose%
\pgfusepath{stroke,fill}%
\end{pgfscope}%
\begin{pgfscope}%
\pgfpathrectangle{\pgfqpoint{2.867647in}{0.500000in}}{\pgfqpoint{1.764706in}{1.700000in}}%
\pgfusepath{clip}%
\pgfsetbuttcap%
\pgfsetroundjoin%
\definecolor{currentfill}{rgb}{0.972201,0.839051,0.745789}%
\pgfsetfillcolor{currentfill}%
\pgfsetlinewidth{0.311001pt}%
\definecolor{currentstroke}{rgb}{1.000000,1.000000,1.000000}%
\pgfsetstrokecolor{currentstroke}%
\pgfsetdash{}{0pt}%
\pgfpathmoveto{\pgfqpoint{4.060710in}{1.585260in}}%
\pgfpathcurveto{\pgfqpoint{4.067843in}{1.585260in}}{\pgfqpoint{4.074684in}{1.588093in}}{\pgfqpoint{4.079728in}{1.593137in}}%
\pgfpathcurveto{\pgfqpoint{4.084772in}{1.598181in}}{\pgfqpoint{4.087606in}{1.605022in}}{\pgfqpoint{4.087606in}{1.612155in}}%
\pgfpathcurveto{\pgfqpoint{4.087606in}{1.619288in}}{\pgfqpoint{4.084772in}{1.626130in}}{\pgfqpoint{4.079728in}{1.631173in}}%
\pgfpathcurveto{\pgfqpoint{4.074684in}{1.636217in}}{\pgfqpoint{4.067843in}{1.639051in}}{\pgfqpoint{4.060710in}{1.639051in}}%
\pgfpathcurveto{\pgfqpoint{4.053577in}{1.639051in}}{\pgfqpoint{4.046735in}{1.636217in}}{\pgfqpoint{4.041692in}{1.631173in}}%
\pgfpathcurveto{\pgfqpoint{4.036648in}{1.626130in}}{\pgfqpoint{4.033814in}{1.619288in}}{\pgfqpoint{4.033814in}{1.612155in}}%
\pgfpathcurveto{\pgfqpoint{4.033814in}{1.605022in}}{\pgfqpoint{4.036648in}{1.598181in}}{\pgfqpoint{4.041692in}{1.593137in}}%
\pgfpathcurveto{\pgfqpoint{4.046735in}{1.588093in}}{\pgfqpoint{4.053577in}{1.585260in}}{\pgfqpoint{4.060710in}{1.585260in}}%
\pgfpathclose%
\pgfusepath{stroke,fill}%
\end{pgfscope}%
\begin{pgfscope}%
\pgfpathrectangle{\pgfqpoint{2.867647in}{0.500000in}}{\pgfqpoint{1.764706in}{1.700000in}}%
\pgfusepath{clip}%
\pgfsetbuttcap%
\pgfsetroundjoin%
\definecolor{currentfill}{rgb}{0.956268,0.491874,0.339856}%
\pgfsetfillcolor{currentfill}%
\pgfsetlinewidth{0.311001pt}%
\definecolor{currentstroke}{rgb}{1.000000,1.000000,1.000000}%
\pgfsetstrokecolor{currentstroke}%
\pgfsetdash{}{0pt}%
\pgfpathmoveto{\pgfqpoint{3.985804in}{1.509976in}}%
\pgfpathcurveto{\pgfqpoint{3.992936in}{1.509976in}}{\pgfqpoint{3.999778in}{1.512810in}}{\pgfqpoint{4.004822in}{1.517854in}}%
\pgfpathcurveto{\pgfqpoint{4.009865in}{1.522897in}}{\pgfqpoint{4.012699in}{1.529739in}}{\pgfqpoint{4.012699in}{1.536872in}}%
\pgfpathcurveto{\pgfqpoint{4.012699in}{1.544005in}}{\pgfqpoint{4.009865in}{1.550846in}}{\pgfqpoint{4.004822in}{1.555890in}}%
\pgfpathcurveto{\pgfqpoint{3.999778in}{1.560933in}}{\pgfqpoint{3.992936in}{1.563767in}}{\pgfqpoint{3.985804in}{1.563767in}}%
\pgfpathcurveto{\pgfqpoint{3.978671in}{1.563767in}}{\pgfqpoint{3.971829in}{1.560933in}}{\pgfqpoint{3.966785in}{1.555890in}}%
\pgfpathcurveto{\pgfqpoint{3.961742in}{1.550846in}}{\pgfqpoint{3.958908in}{1.544005in}}{\pgfqpoint{3.958908in}{1.536872in}}%
\pgfpathcurveto{\pgfqpoint{3.958908in}{1.529739in}}{\pgfqpoint{3.961742in}{1.522897in}}{\pgfqpoint{3.966785in}{1.517854in}}%
\pgfpathcurveto{\pgfqpoint{3.971829in}{1.512810in}}{\pgfqpoint{3.978671in}{1.509976in}}{\pgfqpoint{3.985804in}{1.509976in}}%
\pgfpathclose%
\pgfusepath{stroke,fill}%
\end{pgfscope}%
\begin{pgfscope}%
\pgfpathrectangle{\pgfqpoint{2.867647in}{0.500000in}}{\pgfqpoint{1.764706in}{1.700000in}}%
\pgfusepath{clip}%
\pgfsetbuttcap%
\pgfsetroundjoin%
\definecolor{currentfill}{rgb}{0.961115,0.566634,0.405693}%
\pgfsetfillcolor{currentfill}%
\pgfsetlinewidth{0.311001pt}%
\definecolor{currentstroke}{rgb}{1.000000,1.000000,1.000000}%
\pgfsetstrokecolor{currentstroke}%
\pgfsetdash{}{0pt}%
\pgfpathmoveto{\pgfqpoint{3.922497in}{1.683310in}}%
\pgfpathcurveto{\pgfqpoint{3.929630in}{1.683310in}}{\pgfqpoint{3.936472in}{1.686144in}}{\pgfqpoint{3.941515in}{1.691188in}}%
\pgfpathcurveto{\pgfqpoint{3.946559in}{1.696231in}}{\pgfqpoint{3.949393in}{1.703073in}}{\pgfqpoint{3.949393in}{1.710206in}}%
\pgfpathcurveto{\pgfqpoint{3.949393in}{1.717339in}}{\pgfqpoint{3.946559in}{1.724180in}}{\pgfqpoint{3.941515in}{1.729224in}}%
\pgfpathcurveto{\pgfqpoint{3.936472in}{1.734268in}}{\pgfqpoint{3.929630in}{1.737101in}}{\pgfqpoint{3.922497in}{1.737101in}}%
\pgfpathcurveto{\pgfqpoint{3.915364in}{1.737101in}}{\pgfqpoint{3.908523in}{1.734268in}}{\pgfqpoint{3.903479in}{1.729224in}}%
\pgfpathcurveto{\pgfqpoint{3.898435in}{1.724180in}}{\pgfqpoint{3.895601in}{1.717339in}}{\pgfqpoint{3.895601in}{1.710206in}}%
\pgfpathcurveto{\pgfqpoint{3.895601in}{1.703073in}}{\pgfqpoint{3.898435in}{1.696231in}}{\pgfqpoint{3.903479in}{1.691188in}}%
\pgfpathcurveto{\pgfqpoint{3.908523in}{1.686144in}}{\pgfqpoint{3.915364in}{1.683310in}}{\pgfqpoint{3.922497in}{1.683310in}}%
\pgfpathclose%
\pgfusepath{stroke,fill}%
\end{pgfscope}%
\begin{pgfscope}%
\pgfpathrectangle{\pgfqpoint{2.867647in}{0.500000in}}{\pgfqpoint{1.764706in}{1.700000in}}%
\pgfusepath{clip}%
\pgfsetbuttcap%
\pgfsetroundjoin%
\definecolor{currentfill}{rgb}{0.980678,0.914765,0.856766}%
\pgfsetfillcolor{currentfill}%
\pgfsetlinewidth{0.311001pt}%
\definecolor{currentstroke}{rgb}{1.000000,1.000000,1.000000}%
\pgfsetstrokecolor{currentstroke}%
\pgfsetdash{}{0pt}%
\pgfpathmoveto{\pgfqpoint{4.153676in}{1.157848in}}%
\pgfpathcurveto{\pgfqpoint{4.160809in}{1.157848in}}{\pgfqpoint{4.167651in}{1.160682in}}{\pgfqpoint{4.172694in}{1.165725in}}%
\pgfpathcurveto{\pgfqpoint{4.177738in}{1.170769in}}{\pgfqpoint{4.180572in}{1.177611in}}{\pgfqpoint{4.180572in}{1.184743in}}%
\pgfpathcurveto{\pgfqpoint{4.180572in}{1.191876in}}{\pgfqpoint{4.177738in}{1.198718in}}{\pgfqpoint{4.172694in}{1.203762in}}%
\pgfpathcurveto{\pgfqpoint{4.167651in}{1.208805in}}{\pgfqpoint{4.160809in}{1.211639in}}{\pgfqpoint{4.153676in}{1.211639in}}%
\pgfpathcurveto{\pgfqpoint{4.146543in}{1.211639in}}{\pgfqpoint{4.139702in}{1.208805in}}{\pgfqpoint{4.134658in}{1.203762in}}%
\pgfpathcurveto{\pgfqpoint{4.129614in}{1.198718in}}{\pgfqpoint{4.126780in}{1.191876in}}{\pgfqpoint{4.126780in}{1.184743in}}%
\pgfpathcurveto{\pgfqpoint{4.126780in}{1.177611in}}{\pgfqpoint{4.129614in}{1.170769in}}{\pgfqpoint{4.134658in}{1.165725in}}%
\pgfpathcurveto{\pgfqpoint{4.139702in}{1.160682in}}{\pgfqpoint{4.146543in}{1.157848in}}{\pgfqpoint{4.153676in}{1.157848in}}%
\pgfpathclose%
\pgfusepath{stroke,fill}%
\end{pgfscope}%
\begin{pgfscope}%
\pgfpathrectangle{\pgfqpoint{2.867647in}{0.500000in}}{\pgfqpoint{1.764706in}{1.700000in}}%
\pgfusepath{clip}%
\pgfsetbuttcap%
\pgfsetroundjoin%
\definecolor{currentfill}{rgb}{0.968105,0.786346,0.667739}%
\pgfsetfillcolor{currentfill}%
\pgfsetlinewidth{0.311001pt}%
\definecolor{currentstroke}{rgb}{1.000000,1.000000,1.000000}%
\pgfsetstrokecolor{currentstroke}%
\pgfsetdash{}{0pt}%
\pgfpathmoveto{\pgfqpoint{4.107957in}{0.935577in}}%
\pgfpathcurveto{\pgfqpoint{4.115090in}{0.935577in}}{\pgfqpoint{4.121932in}{0.938411in}}{\pgfqpoint{4.126975in}{0.943455in}}%
\pgfpathcurveto{\pgfqpoint{4.132019in}{0.948498in}}{\pgfqpoint{4.134853in}{0.955340in}}{\pgfqpoint{4.134853in}{0.962473in}}%
\pgfpathcurveto{\pgfqpoint{4.134853in}{0.969606in}}{\pgfqpoint{4.132019in}{0.976447in}}{\pgfqpoint{4.126975in}{0.981491in}}%
\pgfpathcurveto{\pgfqpoint{4.121932in}{0.986535in}}{\pgfqpoint{4.115090in}{0.989368in}}{\pgfqpoint{4.107957in}{0.989368in}}%
\pgfpathcurveto{\pgfqpoint{4.100824in}{0.989368in}}{\pgfqpoint{4.093983in}{0.986535in}}{\pgfqpoint{4.088939in}{0.981491in}}%
\pgfpathcurveto{\pgfqpoint{4.083895in}{0.976447in}}{\pgfqpoint{4.081062in}{0.969606in}}{\pgfqpoint{4.081062in}{0.962473in}}%
\pgfpathcurveto{\pgfqpoint{4.081062in}{0.955340in}}{\pgfqpoint{4.083895in}{0.948498in}}{\pgfqpoint{4.088939in}{0.943455in}}%
\pgfpathcurveto{\pgfqpoint{4.093983in}{0.938411in}}{\pgfqpoint{4.100824in}{0.935577in}}{\pgfqpoint{4.107957in}{0.935577in}}%
\pgfpathclose%
\pgfusepath{stroke,fill}%
\end{pgfscope}%
\begin{pgfscope}%
\pgfpathrectangle{\pgfqpoint{2.867647in}{0.500000in}}{\pgfqpoint{1.764706in}{1.700000in}}%
\pgfusepath{clip}%
\pgfsetbuttcap%
\pgfsetroundjoin%
\definecolor{currentfill}{rgb}{0.955103,0.477872,0.328626}%
\pgfsetfillcolor{currentfill}%
\pgfsetlinewidth{0.311001pt}%
\definecolor{currentstroke}{rgb}{1.000000,1.000000,1.000000}%
\pgfsetstrokecolor{currentstroke}%
\pgfsetdash{}{0pt}%
\pgfpathmoveto{\pgfqpoint{4.355439in}{1.287859in}}%
\pgfpathcurveto{\pgfqpoint{4.362572in}{1.287859in}}{\pgfqpoint{4.369414in}{1.290693in}}{\pgfqpoint{4.374457in}{1.295736in}}%
\pgfpathcurveto{\pgfqpoint{4.379501in}{1.300780in}}{\pgfqpoint{4.382335in}{1.307622in}}{\pgfqpoint{4.382335in}{1.314755in}}%
\pgfpathcurveto{\pgfqpoint{4.382335in}{1.321887in}}{\pgfqpoint{4.379501in}{1.328729in}}{\pgfqpoint{4.374457in}{1.333773in}}%
\pgfpathcurveto{\pgfqpoint{4.369414in}{1.338816in}}{\pgfqpoint{4.362572in}{1.341650in}}{\pgfqpoint{4.355439in}{1.341650in}}%
\pgfpathcurveto{\pgfqpoint{4.348306in}{1.341650in}}{\pgfqpoint{4.341465in}{1.338816in}}{\pgfqpoint{4.336421in}{1.333773in}}%
\pgfpathcurveto{\pgfqpoint{4.331378in}{1.328729in}}{\pgfqpoint{4.328544in}{1.321887in}}{\pgfqpoint{4.328544in}{1.314755in}}%
\pgfpathcurveto{\pgfqpoint{4.328544in}{1.307622in}}{\pgfqpoint{4.331378in}{1.300780in}}{\pgfqpoint{4.336421in}{1.295736in}}%
\pgfpathcurveto{\pgfqpoint{4.341465in}{1.290693in}}{\pgfqpoint{4.348306in}{1.287859in}}{\pgfqpoint{4.355439in}{1.287859in}}%
\pgfpathclose%
\pgfusepath{stroke,fill}%
\end{pgfscope}%
\begin{pgfscope}%
\pgfpathrectangle{\pgfqpoint{2.867647in}{0.500000in}}{\pgfqpoint{1.764706in}{1.700000in}}%
\pgfusepath{clip}%
\pgfsetbuttcap%
\pgfsetroundjoin%
\definecolor{currentfill}{rgb}{0.928830,0.305981,0.243623}%
\pgfsetfillcolor{currentfill}%
\pgfsetlinewidth{0.311001pt}%
\definecolor{currentstroke}{rgb}{1.000000,1.000000,1.000000}%
\pgfsetstrokecolor{currentstroke}%
\pgfsetdash{}{0pt}%
\pgfpathmoveto{\pgfqpoint{4.356598in}{1.490981in}}%
\pgfpathcurveto{\pgfqpoint{4.363731in}{1.490981in}}{\pgfqpoint{4.370572in}{1.493815in}}{\pgfqpoint{4.375616in}{1.498858in}}%
\pgfpathcurveto{\pgfqpoint{4.380660in}{1.503902in}}{\pgfqpoint{4.383493in}{1.510744in}}{\pgfqpoint{4.383493in}{1.517876in}}%
\pgfpathcurveto{\pgfqpoint{4.383493in}{1.525009in}}{\pgfqpoint{4.380660in}{1.531851in}}{\pgfqpoint{4.375616in}{1.536894in}}%
\pgfpathcurveto{\pgfqpoint{4.370572in}{1.541938in}}{\pgfqpoint{4.363731in}{1.544772in}}{\pgfqpoint{4.356598in}{1.544772in}}%
\pgfpathcurveto{\pgfqpoint{4.349465in}{1.544772in}}{\pgfqpoint{4.342623in}{1.541938in}}{\pgfqpoint{4.337580in}{1.536894in}}%
\pgfpathcurveto{\pgfqpoint{4.332536in}{1.531851in}}{\pgfqpoint{4.329702in}{1.525009in}}{\pgfqpoint{4.329702in}{1.517876in}}%
\pgfpathcurveto{\pgfqpoint{4.329702in}{1.510744in}}{\pgfqpoint{4.332536in}{1.503902in}}{\pgfqpoint{4.337580in}{1.498858in}}%
\pgfpathcurveto{\pgfqpoint{4.342623in}{1.493815in}}{\pgfqpoint{4.349465in}{1.490981in}}{\pgfqpoint{4.356598in}{1.490981in}}%
\pgfpathclose%
\pgfusepath{stroke,fill}%
\end{pgfscope}%
\begin{pgfscope}%
\pgfpathrectangle{\pgfqpoint{2.867647in}{0.500000in}}{\pgfqpoint{1.764706in}{1.700000in}}%
\pgfusepath{clip}%
\pgfsetbuttcap%
\pgfsetroundjoin%
\definecolor{currentfill}{rgb}{0.973832,0.856556,0.771584}%
\pgfsetfillcolor{currentfill}%
\pgfsetlinewidth{0.311001pt}%
\definecolor{currentstroke}{rgb}{1.000000,1.000000,1.000000}%
\pgfsetstrokecolor{currentstroke}%
\pgfsetdash{}{0pt}%
\pgfpathmoveto{\pgfqpoint{4.248830in}{1.411660in}}%
\pgfpathcurveto{\pgfqpoint{4.255963in}{1.411660in}}{\pgfqpoint{4.262804in}{1.414494in}}{\pgfqpoint{4.267848in}{1.419538in}}%
\pgfpathcurveto{\pgfqpoint{4.272892in}{1.424582in}}{\pgfqpoint{4.275725in}{1.431423in}}{\pgfqpoint{4.275725in}{1.438556in}}%
\pgfpathcurveto{\pgfqpoint{4.275725in}{1.445689in}}{\pgfqpoint{4.272892in}{1.452531in}}{\pgfqpoint{4.267848in}{1.457574in}}%
\pgfpathcurveto{\pgfqpoint{4.262804in}{1.462618in}}{\pgfqpoint{4.255963in}{1.465452in}}{\pgfqpoint{4.248830in}{1.465452in}}%
\pgfpathcurveto{\pgfqpoint{4.241697in}{1.465452in}}{\pgfqpoint{4.234855in}{1.462618in}}{\pgfqpoint{4.229812in}{1.457574in}}%
\pgfpathcurveto{\pgfqpoint{4.224768in}{1.452531in}}{\pgfqpoint{4.221934in}{1.445689in}}{\pgfqpoint{4.221934in}{1.438556in}}%
\pgfpathcurveto{\pgfqpoint{4.221934in}{1.431423in}}{\pgfqpoint{4.224768in}{1.424582in}}{\pgfqpoint{4.229812in}{1.419538in}}%
\pgfpathcurveto{\pgfqpoint{4.234855in}{1.414494in}}{\pgfqpoint{4.241697in}{1.411660in}}{\pgfqpoint{4.248830in}{1.411660in}}%
\pgfpathclose%
\pgfusepath{stroke,fill}%
\end{pgfscope}%
\begin{pgfscope}%
\pgfpathrectangle{\pgfqpoint{2.867647in}{0.500000in}}{\pgfqpoint{1.764706in}{1.700000in}}%
\pgfusepath{clip}%
\pgfsetbuttcap%
\pgfsetroundjoin%
\definecolor{currentfill}{rgb}{0.964920,0.695342,0.545192}%
\pgfsetfillcolor{currentfill}%
\pgfsetlinewidth{0.311001pt}%
\definecolor{currentstroke}{rgb}{1.000000,1.000000,1.000000}%
\pgfsetstrokecolor{currentstroke}%
\pgfsetdash{}{0pt}%
\pgfpathmoveto{\pgfqpoint{3.989269in}{1.638053in}}%
\pgfpathcurveto{\pgfqpoint{3.996402in}{1.638053in}}{\pgfqpoint{4.003244in}{1.640887in}}{\pgfqpoint{4.008287in}{1.645931in}}%
\pgfpathcurveto{\pgfqpoint{4.013331in}{1.650975in}}{\pgfqpoint{4.016165in}{1.657816in}}{\pgfqpoint{4.016165in}{1.664949in}}%
\pgfpathcurveto{\pgfqpoint{4.016165in}{1.672082in}}{\pgfqpoint{4.013331in}{1.678923in}}{\pgfqpoint{4.008287in}{1.683967in}}%
\pgfpathcurveto{\pgfqpoint{4.003244in}{1.689011in}}{\pgfqpoint{3.996402in}{1.691845in}}{\pgfqpoint{3.989269in}{1.691845in}}%
\pgfpathcurveto{\pgfqpoint{3.982136in}{1.691845in}}{\pgfqpoint{3.975295in}{1.689011in}}{\pgfqpoint{3.970251in}{1.683967in}}%
\pgfpathcurveto{\pgfqpoint{3.965207in}{1.678923in}}{\pgfqpoint{3.962373in}{1.672082in}}{\pgfqpoint{3.962373in}{1.664949in}}%
\pgfpathcurveto{\pgfqpoint{3.962373in}{1.657816in}}{\pgfqpoint{3.965207in}{1.650975in}}{\pgfqpoint{3.970251in}{1.645931in}}%
\pgfpathcurveto{\pgfqpoint{3.975295in}{1.640887in}}{\pgfqpoint{3.982136in}{1.638053in}}{\pgfqpoint{3.989269in}{1.638053in}}%
\pgfpathclose%
\pgfusepath{stroke,fill}%
\end{pgfscope}%
\begin{pgfscope}%
\pgfpathrectangle{\pgfqpoint{2.867647in}{0.500000in}}{\pgfqpoint{1.764706in}{1.700000in}}%
\pgfusepath{clip}%
\pgfsetbuttcap%
\pgfsetroundjoin%
\definecolor{currentfill}{rgb}{0.968509,0.792226,0.676405}%
\pgfsetfillcolor{currentfill}%
\pgfsetlinewidth{0.311001pt}%
\definecolor{currentstroke}{rgb}{1.000000,1.000000,1.000000}%
\pgfsetstrokecolor{currentstroke}%
\pgfsetdash{}{0pt}%
\pgfpathmoveto{\pgfqpoint{4.061003in}{1.708787in}}%
\pgfpathcurveto{\pgfqpoint{4.068136in}{1.708787in}}{\pgfqpoint{4.074978in}{1.711621in}}{\pgfqpoint{4.080021in}{1.716665in}}%
\pgfpathcurveto{\pgfqpoint{4.085065in}{1.721708in}}{\pgfqpoint{4.087899in}{1.728550in}}{\pgfqpoint{4.087899in}{1.735683in}}%
\pgfpathcurveto{\pgfqpoint{4.087899in}{1.742816in}}{\pgfqpoint{4.085065in}{1.749657in}}{\pgfqpoint{4.080021in}{1.754701in}}%
\pgfpathcurveto{\pgfqpoint{4.074978in}{1.759745in}}{\pgfqpoint{4.068136in}{1.762579in}}{\pgfqpoint{4.061003in}{1.762579in}}%
\pgfpathcurveto{\pgfqpoint{4.053870in}{1.762579in}}{\pgfqpoint{4.047029in}{1.759745in}}{\pgfqpoint{4.041985in}{1.754701in}}%
\pgfpathcurveto{\pgfqpoint{4.036941in}{1.749657in}}{\pgfqpoint{4.034107in}{1.742816in}}{\pgfqpoint{4.034107in}{1.735683in}}%
\pgfpathcurveto{\pgfqpoint{4.034107in}{1.728550in}}{\pgfqpoint{4.036941in}{1.721708in}}{\pgfqpoint{4.041985in}{1.716665in}}%
\pgfpathcurveto{\pgfqpoint{4.047029in}{1.711621in}}{\pgfqpoint{4.053870in}{1.708787in}}{\pgfqpoint{4.061003in}{1.708787in}}%
\pgfpathclose%
\pgfusepath{stroke,fill}%
\end{pgfscope}%
\begin{pgfscope}%
\pgfpathrectangle{\pgfqpoint{2.867647in}{0.500000in}}{\pgfqpoint{1.764706in}{1.700000in}}%
\pgfusepath{clip}%
\pgfsetbuttcap%
\pgfsetroundjoin%
\definecolor{currentfill}{rgb}{0.975018,0.868213,0.788710}%
\pgfsetfillcolor{currentfill}%
\pgfsetlinewidth{0.311001pt}%
\definecolor{currentstroke}{rgb}{1.000000,1.000000,1.000000}%
\pgfsetstrokecolor{currentstroke}%
\pgfsetdash{}{0pt}%
\pgfpathmoveto{\pgfqpoint{4.091204in}{1.062283in}}%
\pgfpathcurveto{\pgfqpoint{4.098336in}{1.062283in}}{\pgfqpoint{4.105178in}{1.065117in}}{\pgfqpoint{4.110222in}{1.070160in}}%
\pgfpathcurveto{\pgfqpoint{4.115265in}{1.075204in}}{\pgfqpoint{4.118099in}{1.082046in}}{\pgfqpoint{4.118099in}{1.089178in}}%
\pgfpathcurveto{\pgfqpoint{4.118099in}{1.096311in}}{\pgfqpoint{4.115265in}{1.103153in}}{\pgfqpoint{4.110222in}{1.108197in}}%
\pgfpathcurveto{\pgfqpoint{4.105178in}{1.113240in}}{\pgfqpoint{4.098336in}{1.116074in}}{\pgfqpoint{4.091204in}{1.116074in}}%
\pgfpathcurveto{\pgfqpoint{4.084071in}{1.116074in}}{\pgfqpoint{4.077229in}{1.113240in}}{\pgfqpoint{4.072185in}{1.108197in}}%
\pgfpathcurveto{\pgfqpoint{4.067142in}{1.103153in}}{\pgfqpoint{4.064308in}{1.096311in}}{\pgfqpoint{4.064308in}{1.089178in}}%
\pgfpathcurveto{\pgfqpoint{4.064308in}{1.082046in}}{\pgfqpoint{4.067142in}{1.075204in}}{\pgfqpoint{4.072185in}{1.070160in}}%
\pgfpathcurveto{\pgfqpoint{4.077229in}{1.065117in}}{\pgfqpoint{4.084071in}{1.062283in}}{\pgfqpoint{4.091204in}{1.062283in}}%
\pgfpathclose%
\pgfusepath{stroke,fill}%
\end{pgfscope}%
\begin{pgfscope}%
\pgfpathrectangle{\pgfqpoint{2.867647in}{0.500000in}}{\pgfqpoint{1.764706in}{1.700000in}}%
\pgfusepath{clip}%
\pgfsetbuttcap%
\pgfsetroundjoin%
\definecolor{currentfill}{rgb}{0.963728,0.638439,0.479050}%
\pgfsetfillcolor{currentfill}%
\pgfsetlinewidth{0.311001pt}%
\definecolor{currentstroke}{rgb}{1.000000,1.000000,1.000000}%
\pgfsetstrokecolor{currentstroke}%
\pgfsetdash{}{0pt}%
\pgfpathmoveto{\pgfqpoint{3.971424in}{1.760361in}}%
\pgfpathcurveto{\pgfqpoint{3.978557in}{1.760361in}}{\pgfqpoint{3.985398in}{1.763195in}}{\pgfqpoint{3.990442in}{1.768239in}}%
\pgfpathcurveto{\pgfqpoint{3.995486in}{1.773283in}}{\pgfqpoint{3.998320in}{1.780124in}}{\pgfqpoint{3.998320in}{1.787257in}}%
\pgfpathcurveto{\pgfqpoint{3.998320in}{1.794390in}}{\pgfqpoint{3.995486in}{1.801232in}}{\pgfqpoint{3.990442in}{1.806275in}}%
\pgfpathcurveto{\pgfqpoint{3.985398in}{1.811319in}}{\pgfqpoint{3.978557in}{1.814153in}}{\pgfqpoint{3.971424in}{1.814153in}}%
\pgfpathcurveto{\pgfqpoint{3.964291in}{1.814153in}}{\pgfqpoint{3.957449in}{1.811319in}}{\pgfqpoint{3.952406in}{1.806275in}}%
\pgfpathcurveto{\pgfqpoint{3.947362in}{1.801232in}}{\pgfqpoint{3.944528in}{1.794390in}}{\pgfqpoint{3.944528in}{1.787257in}}%
\pgfpathcurveto{\pgfqpoint{3.944528in}{1.780124in}}{\pgfqpoint{3.947362in}{1.773283in}}{\pgfqpoint{3.952406in}{1.768239in}}%
\pgfpathcurveto{\pgfqpoint{3.957449in}{1.763195in}}{\pgfqpoint{3.964291in}{1.760361in}}{\pgfqpoint{3.971424in}{1.760361in}}%
\pgfpathclose%
\pgfusepath{stroke,fill}%
\end{pgfscope}%
\begin{pgfscope}%
\pgfpathrectangle{\pgfqpoint{2.867647in}{0.500000in}}{\pgfqpoint{1.764706in}{1.700000in}}%
\pgfusepath{clip}%
\pgfsetbuttcap%
\pgfsetroundjoin%
\definecolor{currentfill}{rgb}{0.979124,0.903132,0.839793}%
\pgfsetfillcolor{currentfill}%
\pgfsetlinewidth{0.311001pt}%
\definecolor{currentstroke}{rgb}{1.000000,1.000000,1.000000}%
\pgfsetstrokecolor{currentstroke}%
\pgfsetdash{}{0pt}%
\pgfpathmoveto{\pgfqpoint{4.128099in}{1.135630in}}%
\pgfpathcurveto{\pgfqpoint{4.135232in}{1.135630in}}{\pgfqpoint{4.142073in}{1.138464in}}{\pgfqpoint{4.147117in}{1.143507in}}%
\pgfpathcurveto{\pgfqpoint{4.152161in}{1.148551in}}{\pgfqpoint{4.154995in}{1.155393in}}{\pgfqpoint{4.154995in}{1.162525in}}%
\pgfpathcurveto{\pgfqpoint{4.154995in}{1.169658in}}{\pgfqpoint{4.152161in}{1.176500in}}{\pgfqpoint{4.147117in}{1.181544in}}%
\pgfpathcurveto{\pgfqpoint{4.142073in}{1.186587in}}{\pgfqpoint{4.135232in}{1.189421in}}{\pgfqpoint{4.128099in}{1.189421in}}%
\pgfpathcurveto{\pgfqpoint{4.120966in}{1.189421in}}{\pgfqpoint{4.114124in}{1.186587in}}{\pgfqpoint{4.109081in}{1.181544in}}%
\pgfpathcurveto{\pgfqpoint{4.104037in}{1.176500in}}{\pgfqpoint{4.101203in}{1.169658in}}{\pgfqpoint{4.101203in}{1.162525in}}%
\pgfpathcurveto{\pgfqpoint{4.101203in}{1.155393in}}{\pgfqpoint{4.104037in}{1.148551in}}{\pgfqpoint{4.109081in}{1.143507in}}%
\pgfpathcurveto{\pgfqpoint{4.114124in}{1.138464in}}{\pgfqpoint{4.120966in}{1.135630in}}{\pgfqpoint{4.128099in}{1.135630in}}%
\pgfpathclose%
\pgfusepath{stroke,fill}%
\end{pgfscope}%
\begin{pgfscope}%
\pgfpathrectangle{\pgfqpoint{2.867647in}{0.500000in}}{\pgfqpoint{1.764706in}{1.700000in}}%
\pgfusepath{clip}%
\pgfsetbuttcap%
\pgfsetroundjoin%
\definecolor{currentfill}{rgb}{0.977657,0.891500,0.822809}%
\pgfsetfillcolor{currentfill}%
\pgfsetlinewidth{0.311001pt}%
\definecolor{currentstroke}{rgb}{1.000000,1.000000,1.000000}%
\pgfsetstrokecolor{currentstroke}%
\pgfsetdash{}{0pt}%
\pgfpathmoveto{\pgfqpoint{4.123760in}{1.046357in}}%
\pgfpathcurveto{\pgfqpoint{4.130893in}{1.046357in}}{\pgfqpoint{4.137734in}{1.049190in}}{\pgfqpoint{4.142778in}{1.054234in}}%
\pgfpathcurveto{\pgfqpoint{4.147822in}{1.059278in}}{\pgfqpoint{4.150655in}{1.066119in}}{\pgfqpoint{4.150655in}{1.073252in}}%
\pgfpathcurveto{\pgfqpoint{4.150655in}{1.080385in}}{\pgfqpoint{4.147822in}{1.087227in}}{\pgfqpoint{4.142778in}{1.092270in}}%
\pgfpathcurveto{\pgfqpoint{4.137734in}{1.097314in}}{\pgfqpoint{4.130893in}{1.100148in}}{\pgfqpoint{4.123760in}{1.100148in}}%
\pgfpathcurveto{\pgfqpoint{4.116627in}{1.100148in}}{\pgfqpoint{4.109785in}{1.097314in}}{\pgfqpoint{4.104742in}{1.092270in}}%
\pgfpathcurveto{\pgfqpoint{4.099698in}{1.087227in}}{\pgfqpoint{4.096864in}{1.080385in}}{\pgfqpoint{4.096864in}{1.073252in}}%
\pgfpathcurveto{\pgfqpoint{4.096864in}{1.066119in}}{\pgfqpoint{4.099698in}{1.059278in}}{\pgfqpoint{4.104742in}{1.054234in}}%
\pgfpathcurveto{\pgfqpoint{4.109785in}{1.049190in}}{\pgfqpoint{4.116627in}{1.046357in}}{\pgfqpoint{4.123760in}{1.046357in}}%
\pgfpathclose%
\pgfusepath{stroke,fill}%
\end{pgfscope}%
\begin{pgfscope}%
\pgfpathrectangle{\pgfqpoint{2.867647in}{0.500000in}}{\pgfqpoint{1.764706in}{1.700000in}}%
\pgfusepath{clip}%
\pgfsetbuttcap%
\pgfsetroundjoin%
\definecolor{currentfill}{rgb}{0.975018,0.868213,0.788710}%
\pgfsetfillcolor{currentfill}%
\pgfsetlinewidth{0.311001pt}%
\definecolor{currentstroke}{rgb}{1.000000,1.000000,1.000000}%
\pgfsetstrokecolor{currentstroke}%
\pgfsetdash{}{0pt}%
\pgfpathmoveto{\pgfqpoint{4.128682in}{1.253741in}}%
\pgfpathcurveto{\pgfqpoint{4.135815in}{1.253741in}}{\pgfqpoint{4.142656in}{1.256575in}}{\pgfqpoint{4.147700in}{1.261619in}}%
\pgfpathcurveto{\pgfqpoint{4.152744in}{1.266663in}}{\pgfqpoint{4.155578in}{1.273504in}}{\pgfqpoint{4.155578in}{1.280637in}}%
\pgfpathcurveto{\pgfqpoint{4.155578in}{1.287770in}}{\pgfqpoint{4.152744in}{1.294612in}}{\pgfqpoint{4.147700in}{1.299655in}}%
\pgfpathcurveto{\pgfqpoint{4.142656in}{1.304699in}}{\pgfqpoint{4.135815in}{1.307533in}}{\pgfqpoint{4.128682in}{1.307533in}}%
\pgfpathcurveto{\pgfqpoint{4.121549in}{1.307533in}}{\pgfqpoint{4.114707in}{1.304699in}}{\pgfqpoint{4.109664in}{1.299655in}}%
\pgfpathcurveto{\pgfqpoint{4.104620in}{1.294612in}}{\pgfqpoint{4.101786in}{1.287770in}}{\pgfqpoint{4.101786in}{1.280637in}}%
\pgfpathcurveto{\pgfqpoint{4.101786in}{1.273504in}}{\pgfqpoint{4.104620in}{1.266663in}}{\pgfqpoint{4.109664in}{1.261619in}}%
\pgfpathcurveto{\pgfqpoint{4.114707in}{1.256575in}}{\pgfqpoint{4.121549in}{1.253741in}}{\pgfqpoint{4.128682in}{1.253741in}}%
\pgfpathclose%
\pgfusepath{stroke,fill}%
\end{pgfscope}%
\begin{pgfscope}%
\pgfpathrectangle{\pgfqpoint{2.867647in}{0.500000in}}{\pgfqpoint{1.764706in}{1.700000in}}%
\pgfusepath{clip}%
\pgfsetbuttcap%
\pgfsetroundjoin%
\definecolor{currentfill}{rgb}{0.977657,0.891500,0.822809}%
\pgfsetfillcolor{currentfill}%
\pgfsetlinewidth{0.311001pt}%
\definecolor{currentstroke}{rgb}{1.000000,1.000000,1.000000}%
\pgfsetstrokecolor{currentstroke}%
\pgfsetdash{}{0pt}%
\pgfpathmoveto{\pgfqpoint{4.206646in}{1.496691in}}%
\pgfpathcurveto{\pgfqpoint{4.213778in}{1.496691in}}{\pgfqpoint{4.220620in}{1.499525in}}{\pgfqpoint{4.225664in}{1.504568in}}%
\pgfpathcurveto{\pgfqpoint{4.230707in}{1.509612in}}{\pgfqpoint{4.233541in}{1.516454in}}{\pgfqpoint{4.233541in}{1.523586in}}%
\pgfpathcurveto{\pgfqpoint{4.233541in}{1.530719in}}{\pgfqpoint{4.230707in}{1.537561in}}{\pgfqpoint{4.225664in}{1.542604in}}%
\pgfpathcurveto{\pgfqpoint{4.220620in}{1.547648in}}{\pgfqpoint{4.213778in}{1.550482in}}{\pgfqpoint{4.206646in}{1.550482in}}%
\pgfpathcurveto{\pgfqpoint{4.199513in}{1.550482in}}{\pgfqpoint{4.192671in}{1.547648in}}{\pgfqpoint{4.187627in}{1.542604in}}%
\pgfpathcurveto{\pgfqpoint{4.182584in}{1.537561in}}{\pgfqpoint{4.179750in}{1.530719in}}{\pgfqpoint{4.179750in}{1.523586in}}%
\pgfpathcurveto{\pgfqpoint{4.179750in}{1.516454in}}{\pgfqpoint{4.182584in}{1.509612in}}{\pgfqpoint{4.187627in}{1.504568in}}%
\pgfpathcurveto{\pgfqpoint{4.192671in}{1.499525in}}{\pgfqpoint{4.199513in}{1.496691in}}{\pgfqpoint{4.206646in}{1.496691in}}%
\pgfpathclose%
\pgfusepath{stroke,fill}%
\end{pgfscope}%
\begin{pgfscope}%
\pgfpathrectangle{\pgfqpoint{2.867647in}{0.500000in}}{\pgfqpoint{1.764706in}{1.700000in}}%
\pgfusepath{clip}%
\pgfsetbuttcap%
\pgfsetroundjoin%
\definecolor{currentfill}{rgb}{0.975018,0.868213,0.788710}%
\pgfsetfillcolor{currentfill}%
\pgfsetlinewidth{0.311001pt}%
\definecolor{currentstroke}{rgb}{1.000000,1.000000,1.000000}%
\pgfsetstrokecolor{currentstroke}%
\pgfsetdash{}{0pt}%
\pgfpathmoveto{\pgfqpoint{4.124253in}{1.229271in}}%
\pgfpathcurveto{\pgfqpoint{4.131386in}{1.229271in}}{\pgfqpoint{4.138228in}{1.232105in}}{\pgfqpoint{4.143271in}{1.237148in}}%
\pgfpathcurveto{\pgfqpoint{4.148315in}{1.242192in}}{\pgfqpoint{4.151149in}{1.249034in}}{\pgfqpoint{4.151149in}{1.256166in}}%
\pgfpathcurveto{\pgfqpoint{4.151149in}{1.263299in}}{\pgfqpoint{4.148315in}{1.270141in}}{\pgfqpoint{4.143271in}{1.275185in}}%
\pgfpathcurveto{\pgfqpoint{4.138228in}{1.280228in}}{\pgfqpoint{4.131386in}{1.283062in}}{\pgfqpoint{4.124253in}{1.283062in}}%
\pgfpathcurveto{\pgfqpoint{4.117120in}{1.283062in}}{\pgfqpoint{4.110279in}{1.280228in}}{\pgfqpoint{4.105235in}{1.275185in}}%
\pgfpathcurveto{\pgfqpoint{4.100191in}{1.270141in}}{\pgfqpoint{4.097358in}{1.263299in}}{\pgfqpoint{4.097358in}{1.256166in}}%
\pgfpathcurveto{\pgfqpoint{4.097358in}{1.249034in}}{\pgfqpoint{4.100191in}{1.242192in}}{\pgfqpoint{4.105235in}{1.237148in}}%
\pgfpathcurveto{\pgfqpoint{4.110279in}{1.232105in}}{\pgfqpoint{4.117120in}{1.229271in}}{\pgfqpoint{4.124253in}{1.229271in}}%
\pgfpathclose%
\pgfusepath{stroke,fill}%
\end{pgfscope}%
\begin{pgfscope}%
\pgfpathrectangle{\pgfqpoint{2.867647in}{0.500000in}}{\pgfqpoint{1.764706in}{1.700000in}}%
\pgfusepath{clip}%
\pgfsetbuttcap%
\pgfsetroundjoin%
\definecolor{currentfill}{rgb}{0.974412,0.862387,0.780156}%
\pgfsetfillcolor{currentfill}%
\pgfsetlinewidth{0.311001pt}%
\definecolor{currentstroke}{rgb}{1.000000,1.000000,1.000000}%
\pgfsetstrokecolor{currentstroke}%
\pgfsetdash{}{0pt}%
\pgfpathmoveto{\pgfqpoint{4.223033in}{1.506294in}}%
\pgfpathcurveto{\pgfqpoint{4.230166in}{1.506294in}}{\pgfqpoint{4.237008in}{1.509128in}}{\pgfqpoint{4.242051in}{1.514171in}}%
\pgfpathcurveto{\pgfqpoint{4.247095in}{1.519215in}}{\pgfqpoint{4.249929in}{1.526057in}}{\pgfqpoint{4.249929in}{1.533189in}}%
\pgfpathcurveto{\pgfqpoint{4.249929in}{1.540322in}}{\pgfqpoint{4.247095in}{1.547164in}}{\pgfqpoint{4.242051in}{1.552208in}}%
\pgfpathcurveto{\pgfqpoint{4.237008in}{1.557251in}}{\pgfqpoint{4.230166in}{1.560085in}}{\pgfqpoint{4.223033in}{1.560085in}}%
\pgfpathcurveto{\pgfqpoint{4.215900in}{1.560085in}}{\pgfqpoint{4.209059in}{1.557251in}}{\pgfqpoint{4.204015in}{1.552208in}}%
\pgfpathcurveto{\pgfqpoint{4.198971in}{1.547164in}}{\pgfqpoint{4.196137in}{1.540322in}}{\pgfqpoint{4.196137in}{1.533189in}}%
\pgfpathcurveto{\pgfqpoint{4.196137in}{1.526057in}}{\pgfqpoint{4.198971in}{1.519215in}}{\pgfqpoint{4.204015in}{1.514171in}}%
\pgfpathcurveto{\pgfqpoint{4.209059in}{1.509128in}}{\pgfqpoint{4.215900in}{1.506294in}}{\pgfqpoint{4.223033in}{1.506294in}}%
\pgfpathclose%
\pgfusepath{stroke,fill}%
\end{pgfscope}%
\begin{pgfscope}%
\pgfpathrectangle{\pgfqpoint{2.867647in}{0.500000in}}{\pgfqpoint{1.764706in}{1.700000in}}%
\pgfusepath{clip}%
\pgfsetbuttcap%
\pgfsetroundjoin%
\definecolor{currentfill}{rgb}{0.975018,0.868213,0.788710}%
\pgfsetfillcolor{currentfill}%
\pgfsetlinewidth{0.311001pt}%
\definecolor{currentstroke}{rgb}{1.000000,1.000000,1.000000}%
\pgfsetstrokecolor{currentstroke}%
\pgfsetdash{}{0pt}%
\pgfpathmoveto{\pgfqpoint{4.106161in}{1.008862in}}%
\pgfpathcurveto{\pgfqpoint{4.113294in}{1.008862in}}{\pgfqpoint{4.120135in}{1.011696in}}{\pgfqpoint{4.125179in}{1.016740in}}%
\pgfpathcurveto{\pgfqpoint{4.130223in}{1.021784in}}{\pgfqpoint{4.133057in}{1.028625in}}{\pgfqpoint{4.133057in}{1.035758in}}%
\pgfpathcurveto{\pgfqpoint{4.133057in}{1.042891in}}{\pgfqpoint{4.130223in}{1.049733in}}{\pgfqpoint{4.125179in}{1.054776in}}%
\pgfpathcurveto{\pgfqpoint{4.120135in}{1.059820in}}{\pgfqpoint{4.113294in}{1.062654in}}{\pgfqpoint{4.106161in}{1.062654in}}%
\pgfpathcurveto{\pgfqpoint{4.099028in}{1.062654in}}{\pgfqpoint{4.092186in}{1.059820in}}{\pgfqpoint{4.087143in}{1.054776in}}%
\pgfpathcurveto{\pgfqpoint{4.082099in}{1.049733in}}{\pgfqpoint{4.079265in}{1.042891in}}{\pgfqpoint{4.079265in}{1.035758in}}%
\pgfpathcurveto{\pgfqpoint{4.079265in}{1.028625in}}{\pgfqpoint{4.082099in}{1.021784in}}{\pgfqpoint{4.087143in}{1.016740in}}%
\pgfpathcurveto{\pgfqpoint{4.092186in}{1.011696in}}{\pgfqpoint{4.099028in}{1.008862in}}{\pgfqpoint{4.106161in}{1.008862in}}%
\pgfpathclose%
\pgfusepath{stroke,fill}%
\end{pgfscope}%
\begin{pgfscope}%
\pgfpathrectangle{\pgfqpoint{2.867647in}{0.500000in}}{\pgfqpoint{1.764706in}{1.700000in}}%
\pgfusepath{clip}%
\pgfsetbuttcap%
\pgfsetroundjoin%
\definecolor{currentfill}{rgb}{0.945204,0.390623,0.270949}%
\pgfsetfillcolor{currentfill}%
\pgfsetlinewidth{0.311001pt}%
\definecolor{currentstroke}{rgb}{1.000000,1.000000,1.000000}%
\pgfsetstrokecolor{currentstroke}%
\pgfsetdash{}{0pt}%
\pgfpathmoveto{\pgfqpoint{3.880428in}{0.892684in}}%
\pgfpathcurveto{\pgfqpoint{3.887561in}{0.892684in}}{\pgfqpoint{3.894403in}{0.895518in}}{\pgfqpoint{3.899446in}{0.900562in}}%
\pgfpathcurveto{\pgfqpoint{3.904490in}{0.905605in}}{\pgfqpoint{3.907324in}{0.912447in}}{\pgfqpoint{3.907324in}{0.919580in}}%
\pgfpathcurveto{\pgfqpoint{3.907324in}{0.926713in}}{\pgfqpoint{3.904490in}{0.933554in}}{\pgfqpoint{3.899446in}{0.938598in}}%
\pgfpathcurveto{\pgfqpoint{3.894403in}{0.943642in}}{\pgfqpoint{3.887561in}{0.946476in}}{\pgfqpoint{3.880428in}{0.946476in}}%
\pgfpathcurveto{\pgfqpoint{3.873295in}{0.946476in}}{\pgfqpoint{3.866454in}{0.943642in}}{\pgfqpoint{3.861410in}{0.938598in}}%
\pgfpathcurveto{\pgfqpoint{3.856366in}{0.933554in}}{\pgfqpoint{3.853532in}{0.926713in}}{\pgfqpoint{3.853532in}{0.919580in}}%
\pgfpathcurveto{\pgfqpoint{3.853532in}{0.912447in}}{\pgfqpoint{3.856366in}{0.905605in}}{\pgfqpoint{3.861410in}{0.900562in}}%
\pgfpathcurveto{\pgfqpoint{3.866454in}{0.895518in}}{\pgfqpoint{3.873295in}{0.892684in}}{\pgfqpoint{3.880428in}{0.892684in}}%
\pgfpathclose%
\pgfusepath{stroke,fill}%
\end{pgfscope}%
\begin{pgfscope}%
\pgfpathrectangle{\pgfqpoint{2.867647in}{0.500000in}}{\pgfqpoint{1.764706in}{1.700000in}}%
\pgfusepath{clip}%
\pgfsetbuttcap%
\pgfsetroundjoin%
\definecolor{currentfill}{rgb}{0.981377,0.920617,0.865369}%
\pgfsetfillcolor{currentfill}%
\pgfsetlinewidth{0.311001pt}%
\definecolor{currentstroke}{rgb}{1.000000,1.000000,1.000000}%
\pgfsetstrokecolor{currentstroke}%
\pgfsetdash{}{0pt}%
\pgfpathmoveto{\pgfqpoint{4.162350in}{1.184192in}}%
\pgfpathcurveto{\pgfqpoint{4.169482in}{1.184192in}}{\pgfqpoint{4.176324in}{1.187026in}}{\pgfqpoint{4.181368in}{1.192070in}}%
\pgfpathcurveto{\pgfqpoint{4.186411in}{1.197114in}}{\pgfqpoint{4.189245in}{1.203955in}}{\pgfqpoint{4.189245in}{1.211088in}}%
\pgfpathcurveto{\pgfqpoint{4.189245in}{1.218221in}}{\pgfqpoint{4.186411in}{1.225063in}}{\pgfqpoint{4.181368in}{1.230106in}}%
\pgfpathcurveto{\pgfqpoint{4.176324in}{1.235150in}}{\pgfqpoint{4.169482in}{1.237984in}}{\pgfqpoint{4.162350in}{1.237984in}}%
\pgfpathcurveto{\pgfqpoint{4.155217in}{1.237984in}}{\pgfqpoint{4.148375in}{1.235150in}}{\pgfqpoint{4.143331in}{1.230106in}}%
\pgfpathcurveto{\pgfqpoint{4.138288in}{1.225063in}}{\pgfqpoint{4.135454in}{1.218221in}}{\pgfqpoint{4.135454in}{1.211088in}}%
\pgfpathcurveto{\pgfqpoint{4.135454in}{1.203955in}}{\pgfqpoint{4.138288in}{1.197114in}}{\pgfqpoint{4.143331in}{1.192070in}}%
\pgfpathcurveto{\pgfqpoint{4.148375in}{1.187026in}}{\pgfqpoint{4.155217in}{1.184192in}}{\pgfqpoint{4.162350in}{1.184192in}}%
\pgfpathclose%
\pgfusepath{stroke,fill}%
\end{pgfscope}%
\begin{pgfscope}%
\pgfpathrectangle{\pgfqpoint{2.867647in}{0.500000in}}{\pgfqpoint{1.764706in}{1.700000in}}%
\pgfusepath{clip}%
\pgfsetbuttcap%
\pgfsetroundjoin%
\definecolor{currentfill}{rgb}{0.979124,0.903132,0.839793}%
\pgfsetfillcolor{currentfill}%
\pgfsetlinewidth{0.311001pt}%
\definecolor{currentstroke}{rgb}{1.000000,1.000000,1.000000}%
\pgfsetstrokecolor{currentstroke}%
\pgfsetdash{}{0pt}%
\pgfpathmoveto{\pgfqpoint{4.158827in}{1.323275in}}%
\pgfpathcurveto{\pgfqpoint{4.165959in}{1.323275in}}{\pgfqpoint{4.172801in}{1.326109in}}{\pgfqpoint{4.177845in}{1.331153in}}%
\pgfpathcurveto{\pgfqpoint{4.182888in}{1.336197in}}{\pgfqpoint{4.185722in}{1.343038in}}{\pgfqpoint{4.185722in}{1.350171in}}%
\pgfpathcurveto{\pgfqpoint{4.185722in}{1.357304in}}{\pgfqpoint{4.182888in}{1.364146in}}{\pgfqpoint{4.177845in}{1.369189in}}%
\pgfpathcurveto{\pgfqpoint{4.172801in}{1.374233in}}{\pgfqpoint{4.165959in}{1.377067in}}{\pgfqpoint{4.158827in}{1.377067in}}%
\pgfpathcurveto{\pgfqpoint{4.151694in}{1.377067in}}{\pgfqpoint{4.144852in}{1.374233in}}{\pgfqpoint{4.139808in}{1.369189in}}%
\pgfpathcurveto{\pgfqpoint{4.134765in}{1.364146in}}{\pgfqpoint{4.131931in}{1.357304in}}{\pgfqpoint{4.131931in}{1.350171in}}%
\pgfpathcurveto{\pgfqpoint{4.131931in}{1.343038in}}{\pgfqpoint{4.134765in}{1.336197in}}{\pgfqpoint{4.139808in}{1.331153in}}%
\pgfpathcurveto{\pgfqpoint{4.144852in}{1.326109in}}{\pgfqpoint{4.151694in}{1.323275in}}{\pgfqpoint{4.158827in}{1.323275in}}%
\pgfpathclose%
\pgfusepath{stroke,fill}%
\end{pgfscope}%
\begin{pgfscope}%
\pgfpathrectangle{\pgfqpoint{2.867647in}{0.500000in}}{\pgfqpoint{1.764706in}{1.700000in}}%
\pgfusepath{clip}%
\pgfsetbuttcap%
\pgfsetroundjoin%
\definecolor{currentfill}{rgb}{0.970255,0.815666,0.711203}%
\pgfsetfillcolor{currentfill}%
\pgfsetlinewidth{0.311001pt}%
\definecolor{currentstroke}{rgb}{1.000000,1.000000,1.000000}%
\pgfsetstrokecolor{currentstroke}%
\pgfsetdash{}{0pt}%
\pgfpathmoveto{\pgfqpoint{4.052824in}{0.971803in}}%
\pgfpathcurveto{\pgfqpoint{4.059957in}{0.971803in}}{\pgfqpoint{4.066799in}{0.974637in}}{\pgfqpoint{4.071842in}{0.979681in}}%
\pgfpathcurveto{\pgfqpoint{4.076886in}{0.984724in}}{\pgfqpoint{4.079720in}{0.991566in}}{\pgfqpoint{4.079720in}{0.998699in}}%
\pgfpathcurveto{\pgfqpoint{4.079720in}{1.005832in}}{\pgfqpoint{4.076886in}{1.012673in}}{\pgfqpoint{4.071842in}{1.017717in}}%
\pgfpathcurveto{\pgfqpoint{4.066799in}{1.022761in}}{\pgfqpoint{4.059957in}{1.025595in}}{\pgfqpoint{4.052824in}{1.025595in}}%
\pgfpathcurveto{\pgfqpoint{4.045691in}{1.025595in}}{\pgfqpoint{4.038850in}{1.022761in}}{\pgfqpoint{4.033806in}{1.017717in}}%
\pgfpathcurveto{\pgfqpoint{4.028762in}{1.012673in}}{\pgfqpoint{4.025928in}{1.005832in}}{\pgfqpoint{4.025928in}{0.998699in}}%
\pgfpathcurveto{\pgfqpoint{4.025928in}{0.991566in}}{\pgfqpoint{4.028762in}{0.984724in}}{\pgfqpoint{4.033806in}{0.979681in}}%
\pgfpathcurveto{\pgfqpoint{4.038850in}{0.974637in}}{\pgfqpoint{4.045691in}{0.971803in}}{\pgfqpoint{4.052824in}{0.971803in}}%
\pgfpathclose%
\pgfusepath{stroke,fill}%
\end{pgfscope}%
\begin{pgfscope}%
\pgfpathrectangle{\pgfqpoint{2.867647in}{0.500000in}}{\pgfqpoint{1.764706in}{1.700000in}}%
\pgfusepath{clip}%
\pgfsetbuttcap%
\pgfsetroundjoin%
\definecolor{currentfill}{rgb}{0.971694,0.833208,0.737161}%
\pgfsetfillcolor{currentfill}%
\pgfsetlinewidth{0.311001pt}%
\definecolor{currentstroke}{rgb}{1.000000,1.000000,1.000000}%
\pgfsetstrokecolor{currentstroke}%
\pgfsetdash{}{0pt}%
\pgfpathmoveto{\pgfqpoint{4.099117in}{1.436741in}}%
\pgfpathcurveto{\pgfqpoint{4.106250in}{1.436741in}}{\pgfqpoint{4.113091in}{1.439574in}}{\pgfqpoint{4.118135in}{1.444618in}}%
\pgfpathcurveto{\pgfqpoint{4.123179in}{1.449662in}}{\pgfqpoint{4.126013in}{1.456503in}}{\pgfqpoint{4.126013in}{1.463636in}}%
\pgfpathcurveto{\pgfqpoint{4.126013in}{1.470769in}}{\pgfqpoint{4.123179in}{1.477611in}}{\pgfqpoint{4.118135in}{1.482654in}}%
\pgfpathcurveto{\pgfqpoint{4.113091in}{1.487698in}}{\pgfqpoint{4.106250in}{1.490532in}}{\pgfqpoint{4.099117in}{1.490532in}}%
\pgfpathcurveto{\pgfqpoint{4.091984in}{1.490532in}}{\pgfqpoint{4.085142in}{1.487698in}}{\pgfqpoint{4.080099in}{1.482654in}}%
\pgfpathcurveto{\pgfqpoint{4.075055in}{1.477611in}}{\pgfqpoint{4.072221in}{1.470769in}}{\pgfqpoint{4.072221in}{1.463636in}}%
\pgfpathcurveto{\pgfqpoint{4.072221in}{1.456503in}}{\pgfqpoint{4.075055in}{1.449662in}}{\pgfqpoint{4.080099in}{1.444618in}}%
\pgfpathcurveto{\pgfqpoint{4.085142in}{1.439574in}}{\pgfqpoint{4.091984in}{1.436741in}}{\pgfqpoint{4.099117in}{1.436741in}}%
\pgfpathclose%
\pgfusepath{stroke,fill}%
\end{pgfscope}%
\begin{pgfscope}%
\pgfpathrectangle{\pgfqpoint{2.867647in}{0.500000in}}{\pgfqpoint{1.764706in}{1.700000in}}%
\pgfusepath{clip}%
\pgfsetbuttcap%
\pgfsetroundjoin%
\definecolor{currentfill}{rgb}{0.978376,0.897317,0.831308}%
\pgfsetfillcolor{currentfill}%
\pgfsetlinewidth{0.311001pt}%
\definecolor{currentstroke}{rgb}{1.000000,1.000000,1.000000}%
\pgfsetstrokecolor{currentstroke}%
\pgfsetdash{}{0pt}%
\pgfpathmoveto{\pgfqpoint{4.153123in}{1.073337in}}%
\pgfpathcurveto{\pgfqpoint{4.160256in}{1.073337in}}{\pgfqpoint{4.167097in}{1.076171in}}{\pgfqpoint{4.172141in}{1.081214in}}%
\pgfpathcurveto{\pgfqpoint{4.177185in}{1.086258in}}{\pgfqpoint{4.180019in}{1.093100in}}{\pgfqpoint{4.180019in}{1.100233in}}%
\pgfpathcurveto{\pgfqpoint{4.180019in}{1.107365in}}{\pgfqpoint{4.177185in}{1.114207in}}{\pgfqpoint{4.172141in}{1.119251in}}%
\pgfpathcurveto{\pgfqpoint{4.167097in}{1.124294in}}{\pgfqpoint{4.160256in}{1.127128in}}{\pgfqpoint{4.153123in}{1.127128in}}%
\pgfpathcurveto{\pgfqpoint{4.145990in}{1.127128in}}{\pgfqpoint{4.139148in}{1.124294in}}{\pgfqpoint{4.134105in}{1.119251in}}%
\pgfpathcurveto{\pgfqpoint{4.129061in}{1.114207in}}{\pgfqpoint{4.126227in}{1.107365in}}{\pgfqpoint{4.126227in}{1.100233in}}%
\pgfpathcurveto{\pgfqpoint{4.126227in}{1.093100in}}{\pgfqpoint{4.129061in}{1.086258in}}{\pgfqpoint{4.134105in}{1.081214in}}%
\pgfpathcurveto{\pgfqpoint{4.139148in}{1.076171in}}{\pgfqpoint{4.145990in}{1.073337in}}{\pgfqpoint{4.153123in}{1.073337in}}%
\pgfpathclose%
\pgfusepath{stroke,fill}%
\end{pgfscope}%
\begin{pgfscope}%
\pgfpathrectangle{\pgfqpoint{2.867647in}{0.500000in}}{\pgfqpoint{1.764706in}{1.700000in}}%
\pgfusepath{clip}%
\pgfsetbuttcap%
\pgfsetroundjoin%
\definecolor{currentfill}{rgb}{0.964306,0.663930,0.507747}%
\pgfsetfillcolor{currentfill}%
\pgfsetlinewidth{0.311001pt}%
\definecolor{currentstroke}{rgb}{1.000000,1.000000,1.000000}%
\pgfsetstrokecolor{currentstroke}%
\pgfsetdash{}{0pt}%
\pgfpathmoveto{\pgfqpoint{3.972920in}{0.979045in}}%
\pgfpathcurveto{\pgfqpoint{3.980053in}{0.979045in}}{\pgfqpoint{3.986894in}{0.981879in}}{\pgfqpoint{3.991938in}{0.986923in}}%
\pgfpathcurveto{\pgfqpoint{3.996982in}{0.991966in}}{\pgfqpoint{3.999815in}{0.998808in}}{\pgfqpoint{3.999815in}{1.005941in}}%
\pgfpathcurveto{\pgfqpoint{3.999815in}{1.013074in}}{\pgfqpoint{3.996982in}{1.019915in}}{\pgfqpoint{3.991938in}{1.024959in}}%
\pgfpathcurveto{\pgfqpoint{3.986894in}{1.030003in}}{\pgfqpoint{3.980053in}{1.032837in}}{\pgfqpoint{3.972920in}{1.032837in}}%
\pgfpathcurveto{\pgfqpoint{3.965787in}{1.032837in}}{\pgfqpoint{3.958945in}{1.030003in}}{\pgfqpoint{3.953902in}{1.024959in}}%
\pgfpathcurveto{\pgfqpoint{3.948858in}{1.019915in}}{\pgfqpoint{3.946024in}{1.013074in}}{\pgfqpoint{3.946024in}{1.005941in}}%
\pgfpathcurveto{\pgfqpoint{3.946024in}{0.998808in}}{\pgfqpoint{3.948858in}{0.991966in}}{\pgfqpoint{3.953902in}{0.986923in}}%
\pgfpathcurveto{\pgfqpoint{3.958945in}{0.981879in}}{\pgfqpoint{3.965787in}{0.979045in}}{\pgfqpoint{3.972920in}{0.979045in}}%
\pgfpathclose%
\pgfusepath{stroke,fill}%
\end{pgfscope}%
\begin{pgfscope}%
\pgfpathrectangle{\pgfqpoint{2.867647in}{0.500000in}}{\pgfqpoint{1.764706in}{1.700000in}}%
\pgfusepath{clip}%
\pgfsetbuttcap%
\pgfsetroundjoin%
\definecolor{currentfill}{rgb}{0.965592,0.726236,0.584384}%
\pgfsetfillcolor{currentfill}%
\pgfsetlinewidth{0.311001pt}%
\definecolor{currentstroke}{rgb}{1.000000,1.000000,1.000000}%
\pgfsetstrokecolor{currentstroke}%
\pgfsetdash{}{0pt}%
\pgfpathmoveto{\pgfqpoint{4.076525in}{1.266583in}}%
\pgfpathcurveto{\pgfqpoint{4.083658in}{1.266583in}}{\pgfqpoint{4.090499in}{1.269417in}}{\pgfqpoint{4.095543in}{1.274460in}}%
\pgfpathcurveto{\pgfqpoint{4.100587in}{1.279504in}}{\pgfqpoint{4.103421in}{1.286346in}}{\pgfqpoint{4.103421in}{1.293479in}}%
\pgfpathcurveto{\pgfqpoint{4.103421in}{1.300611in}}{\pgfqpoint{4.100587in}{1.307453in}}{\pgfqpoint{4.095543in}{1.312497in}}%
\pgfpathcurveto{\pgfqpoint{4.090499in}{1.317540in}}{\pgfqpoint{4.083658in}{1.320374in}}{\pgfqpoint{4.076525in}{1.320374in}}%
\pgfpathcurveto{\pgfqpoint{4.069392in}{1.320374in}}{\pgfqpoint{4.062550in}{1.317540in}}{\pgfqpoint{4.057507in}{1.312497in}}%
\pgfpathcurveto{\pgfqpoint{4.052463in}{1.307453in}}{\pgfqpoint{4.049629in}{1.300611in}}{\pgfqpoint{4.049629in}{1.293479in}}%
\pgfpathcurveto{\pgfqpoint{4.049629in}{1.286346in}}{\pgfqpoint{4.052463in}{1.279504in}}{\pgfqpoint{4.057507in}{1.274460in}}%
\pgfpathcurveto{\pgfqpoint{4.062550in}{1.269417in}}{\pgfqpoint{4.069392in}{1.266583in}}{\pgfqpoint{4.076525in}{1.266583in}}%
\pgfpathclose%
\pgfusepath{stroke,fill}%
\end{pgfscope}%
\begin{pgfscope}%
\pgfpathrectangle{\pgfqpoint{2.867647in}{0.500000in}}{\pgfqpoint{1.764706in}{1.700000in}}%
\pgfusepath{clip}%
\pgfsetbuttcap%
\pgfsetroundjoin%
\definecolor{currentfill}{rgb}{0.980678,0.914765,0.856766}%
\pgfsetfillcolor{currentfill}%
\pgfsetlinewidth{0.311001pt}%
\definecolor{currentstroke}{rgb}{1.000000,1.000000,1.000000}%
\pgfsetstrokecolor{currentstroke}%
\pgfsetdash{}{0pt}%
\pgfpathmoveto{\pgfqpoint{4.177275in}{1.152435in}}%
\pgfpathcurveto{\pgfqpoint{4.184408in}{1.152435in}}{\pgfqpoint{4.191249in}{1.155269in}}{\pgfqpoint{4.196293in}{1.160313in}}%
\pgfpathcurveto{\pgfqpoint{4.201337in}{1.165357in}}{\pgfqpoint{4.204171in}{1.172198in}}{\pgfqpoint{4.204171in}{1.179331in}}%
\pgfpathcurveto{\pgfqpoint{4.204171in}{1.186464in}}{\pgfqpoint{4.201337in}{1.193305in}}{\pgfqpoint{4.196293in}{1.198349in}}%
\pgfpathcurveto{\pgfqpoint{4.191249in}{1.203393in}}{\pgfqpoint{4.184408in}{1.206227in}}{\pgfqpoint{4.177275in}{1.206227in}}%
\pgfpathcurveto{\pgfqpoint{4.170142in}{1.206227in}}{\pgfqpoint{4.163301in}{1.203393in}}{\pgfqpoint{4.158257in}{1.198349in}}%
\pgfpathcurveto{\pgfqpoint{4.153213in}{1.193305in}}{\pgfqpoint{4.150379in}{1.186464in}}{\pgfqpoint{4.150379in}{1.179331in}}%
\pgfpathcurveto{\pgfqpoint{4.150379in}{1.172198in}}{\pgfqpoint{4.153213in}{1.165357in}}{\pgfqpoint{4.158257in}{1.160313in}}%
\pgfpathcurveto{\pgfqpoint{4.163301in}{1.155269in}}{\pgfqpoint{4.170142in}{1.152435in}}{\pgfqpoint{4.177275in}{1.152435in}}%
\pgfpathclose%
\pgfusepath{stroke,fill}%
\end{pgfscope}%
\begin{pgfscope}%
\pgfpathrectangle{\pgfqpoint{2.867647in}{0.500000in}}{\pgfqpoint{1.764706in}{1.700000in}}%
\pgfusepath{clip}%
\pgfsetbuttcap%
\pgfsetroundjoin%
\definecolor{currentfill}{rgb}{0.964679,0.682838,0.530002}%
\pgfsetfillcolor{currentfill}%
\pgfsetlinewidth{0.311001pt}%
\definecolor{currentstroke}{rgb}{1.000000,1.000000,1.000000}%
\pgfsetstrokecolor{currentstroke}%
\pgfsetdash{}{0pt}%
\pgfpathmoveto{\pgfqpoint{4.119709in}{1.755763in}}%
\pgfpathcurveto{\pgfqpoint{4.126842in}{1.755763in}}{\pgfqpoint{4.133684in}{1.758597in}}{\pgfqpoint{4.138727in}{1.763640in}}%
\pgfpathcurveto{\pgfqpoint{4.143771in}{1.768684in}}{\pgfqpoint{4.146605in}{1.775526in}}{\pgfqpoint{4.146605in}{1.782658in}}%
\pgfpathcurveto{\pgfqpoint{4.146605in}{1.789791in}}{\pgfqpoint{4.143771in}{1.796633in}}{\pgfqpoint{4.138727in}{1.801677in}}%
\pgfpathcurveto{\pgfqpoint{4.133684in}{1.806720in}}{\pgfqpoint{4.126842in}{1.809554in}}{\pgfqpoint{4.119709in}{1.809554in}}%
\pgfpathcurveto{\pgfqpoint{4.112576in}{1.809554in}}{\pgfqpoint{4.105735in}{1.806720in}}{\pgfqpoint{4.100691in}{1.801677in}}%
\pgfpathcurveto{\pgfqpoint{4.095647in}{1.796633in}}{\pgfqpoint{4.092813in}{1.789791in}}{\pgfqpoint{4.092813in}{1.782658in}}%
\pgfpathcurveto{\pgfqpoint{4.092813in}{1.775526in}}{\pgfqpoint{4.095647in}{1.768684in}}{\pgfqpoint{4.100691in}{1.763640in}}%
\pgfpathcurveto{\pgfqpoint{4.105735in}{1.758597in}}{\pgfqpoint{4.112576in}{1.755763in}}{\pgfqpoint{4.119709in}{1.755763in}}%
\pgfpathclose%
\pgfusepath{stroke,fill}%
\end{pgfscope}%
\begin{pgfscope}%
\pgfpathrectangle{\pgfqpoint{2.867647in}{0.500000in}}{\pgfqpoint{1.764706in}{1.700000in}}%
\pgfusepath{clip}%
\pgfsetbuttcap%
\pgfsetroundjoin%
\definecolor{currentfill}{rgb}{0.980678,0.914765,0.856766}%
\pgfsetfillcolor{currentfill}%
\pgfsetlinewidth{0.311001pt}%
\definecolor{currentstroke}{rgb}{1.000000,1.000000,1.000000}%
\pgfsetstrokecolor{currentstroke}%
\pgfsetdash{}{0pt}%
\pgfpathmoveto{\pgfqpoint{4.159043in}{1.510954in}}%
\pgfpathcurveto{\pgfqpoint{4.166176in}{1.510954in}}{\pgfqpoint{4.173018in}{1.513787in}}{\pgfqpoint{4.178062in}{1.518831in}}%
\pgfpathcurveto{\pgfqpoint{4.183105in}{1.523875in}}{\pgfqpoint{4.185939in}{1.530716in}}{\pgfqpoint{4.185939in}{1.537849in}}%
\pgfpathcurveto{\pgfqpoint{4.185939in}{1.544982in}}{\pgfqpoint{4.183105in}{1.551824in}}{\pgfqpoint{4.178062in}{1.556867in}}%
\pgfpathcurveto{\pgfqpoint{4.173018in}{1.561911in}}{\pgfqpoint{4.166176in}{1.564745in}}{\pgfqpoint{4.159043in}{1.564745in}}%
\pgfpathcurveto{\pgfqpoint{4.151911in}{1.564745in}}{\pgfqpoint{4.145069in}{1.561911in}}{\pgfqpoint{4.140025in}{1.556867in}}%
\pgfpathcurveto{\pgfqpoint{4.134982in}{1.551824in}}{\pgfqpoint{4.132148in}{1.544982in}}{\pgfqpoint{4.132148in}{1.537849in}}%
\pgfpathcurveto{\pgfqpoint{4.132148in}{1.530716in}}{\pgfqpoint{4.134982in}{1.523875in}}{\pgfqpoint{4.140025in}{1.518831in}}%
\pgfpathcurveto{\pgfqpoint{4.145069in}{1.513787in}}{\pgfqpoint{4.151911in}{1.510954in}}{\pgfqpoint{4.159043in}{1.510954in}}%
\pgfpathclose%
\pgfusepath{stroke,fill}%
\end{pgfscope}%
\begin{pgfscope}%
\pgfpathrectangle{\pgfqpoint{2.867647in}{0.500000in}}{\pgfqpoint{1.764706in}{1.700000in}}%
\pgfusepath{clip}%
\pgfsetbuttcap%
\pgfsetroundjoin%
\definecolor{currentfill}{rgb}{0.971694,0.833208,0.737161}%
\pgfsetfillcolor{currentfill}%
\pgfsetlinewidth{0.311001pt}%
\definecolor{currentstroke}{rgb}{1.000000,1.000000,1.000000}%
\pgfsetstrokecolor{currentstroke}%
\pgfsetdash{}{0pt}%
\pgfpathmoveto{\pgfqpoint{4.057441in}{1.023740in}}%
\pgfpathcurveto{\pgfqpoint{4.064573in}{1.023740in}}{\pgfqpoint{4.071415in}{1.026574in}}{\pgfqpoint{4.076459in}{1.031617in}}%
\pgfpathcurveto{\pgfqpoint{4.081502in}{1.036661in}}{\pgfqpoint{4.084336in}{1.043503in}}{\pgfqpoint{4.084336in}{1.050636in}}%
\pgfpathcurveto{\pgfqpoint{4.084336in}{1.057768in}}{\pgfqpoint{4.081502in}{1.064610in}}{\pgfqpoint{4.076459in}{1.069654in}}%
\pgfpathcurveto{\pgfqpoint{4.071415in}{1.074697in}}{\pgfqpoint{4.064573in}{1.077531in}}{\pgfqpoint{4.057441in}{1.077531in}}%
\pgfpathcurveto{\pgfqpoint{4.050308in}{1.077531in}}{\pgfqpoint{4.043466in}{1.074697in}}{\pgfqpoint{4.038422in}{1.069654in}}%
\pgfpathcurveto{\pgfqpoint{4.033379in}{1.064610in}}{\pgfqpoint{4.030545in}{1.057768in}}{\pgfqpoint{4.030545in}{1.050636in}}%
\pgfpathcurveto{\pgfqpoint{4.030545in}{1.043503in}}{\pgfqpoint{4.033379in}{1.036661in}}{\pgfqpoint{4.038422in}{1.031617in}}%
\pgfpathcurveto{\pgfqpoint{4.043466in}{1.026574in}}{\pgfqpoint{4.050308in}{1.023740in}}{\pgfqpoint{4.057441in}{1.023740in}}%
\pgfpathclose%
\pgfusepath{stroke,fill}%
\end{pgfscope}%
\begin{pgfscope}%
\pgfpathrectangle{\pgfqpoint{2.867647in}{0.500000in}}{\pgfqpoint{1.764706in}{1.700000in}}%
\pgfusepath{clip}%
\pgfsetbuttcap%
\pgfsetroundjoin%
\definecolor{currentfill}{rgb}{0.975018,0.868213,0.788710}%
\pgfsetfillcolor{currentfill}%
\pgfsetlinewidth{0.311001pt}%
\definecolor{currentstroke}{rgb}{1.000000,1.000000,1.000000}%
\pgfsetstrokecolor{currentstroke}%
\pgfsetdash{}{0pt}%
\pgfpathmoveto{\pgfqpoint{4.183319in}{1.055656in}}%
\pgfpathcurveto{\pgfqpoint{4.190451in}{1.055656in}}{\pgfqpoint{4.197293in}{1.058490in}}{\pgfqpoint{4.202337in}{1.063533in}}%
\pgfpathcurveto{\pgfqpoint{4.207380in}{1.068577in}}{\pgfqpoint{4.210214in}{1.075419in}}{\pgfqpoint{4.210214in}{1.082552in}}%
\pgfpathcurveto{\pgfqpoint{4.210214in}{1.089684in}}{\pgfqpoint{4.207380in}{1.096526in}}{\pgfqpoint{4.202337in}{1.101570in}}%
\pgfpathcurveto{\pgfqpoint{4.197293in}{1.106613in}}{\pgfqpoint{4.190451in}{1.109447in}}{\pgfqpoint{4.183319in}{1.109447in}}%
\pgfpathcurveto{\pgfqpoint{4.176186in}{1.109447in}}{\pgfqpoint{4.169344in}{1.106613in}}{\pgfqpoint{4.164300in}{1.101570in}}%
\pgfpathcurveto{\pgfqpoint{4.159257in}{1.096526in}}{\pgfqpoint{4.156423in}{1.089684in}}{\pgfqpoint{4.156423in}{1.082552in}}%
\pgfpathcurveto{\pgfqpoint{4.156423in}{1.075419in}}{\pgfqpoint{4.159257in}{1.068577in}}{\pgfqpoint{4.164300in}{1.063533in}}%
\pgfpathcurveto{\pgfqpoint{4.169344in}{1.058490in}}{\pgfqpoint{4.176186in}{1.055656in}}{\pgfqpoint{4.183319in}{1.055656in}}%
\pgfpathclose%
\pgfusepath{stroke,fill}%
\end{pgfscope}%
\begin{pgfscope}%
\pgfpathrectangle{\pgfqpoint{2.867647in}{0.500000in}}{\pgfqpoint{1.764706in}{1.700000in}}%
\pgfusepath{clip}%
\pgfsetbuttcap%
\pgfsetroundjoin%
\definecolor{currentfill}{rgb}{0.979891,0.908948,0.848279}%
\pgfsetfillcolor{currentfill}%
\pgfsetlinewidth{0.311001pt}%
\definecolor{currentstroke}{rgb}{1.000000,1.000000,1.000000}%
\pgfsetstrokecolor{currentstroke}%
\pgfsetdash{}{0pt}%
\pgfpathmoveto{\pgfqpoint{4.145051in}{1.553348in}}%
\pgfpathcurveto{\pgfqpoint{4.152184in}{1.553348in}}{\pgfqpoint{4.159026in}{1.556182in}}{\pgfqpoint{4.164069in}{1.561226in}}%
\pgfpathcurveto{\pgfqpoint{4.169113in}{1.566270in}}{\pgfqpoint{4.171947in}{1.573111in}}{\pgfqpoint{4.171947in}{1.580244in}}%
\pgfpathcurveto{\pgfqpoint{4.171947in}{1.587377in}}{\pgfqpoint{4.169113in}{1.594219in}}{\pgfqpoint{4.164069in}{1.599262in}}%
\pgfpathcurveto{\pgfqpoint{4.159026in}{1.604306in}}{\pgfqpoint{4.152184in}{1.607140in}}{\pgfqpoint{4.145051in}{1.607140in}}%
\pgfpathcurveto{\pgfqpoint{4.137918in}{1.607140in}}{\pgfqpoint{4.131077in}{1.604306in}}{\pgfqpoint{4.126033in}{1.599262in}}%
\pgfpathcurveto{\pgfqpoint{4.120989in}{1.594219in}}{\pgfqpoint{4.118156in}{1.587377in}}{\pgfqpoint{4.118156in}{1.580244in}}%
\pgfpathcurveto{\pgfqpoint{4.118156in}{1.573111in}}{\pgfqpoint{4.120989in}{1.566270in}}{\pgfqpoint{4.126033in}{1.561226in}}%
\pgfpathcurveto{\pgfqpoint{4.131077in}{1.556182in}}{\pgfqpoint{4.137918in}{1.553348in}}{\pgfqpoint{4.145051in}{1.553348in}}%
\pgfpathclose%
\pgfusepath{stroke,fill}%
\end{pgfscope}%
\begin{pgfscope}%
\pgfpathrectangle{\pgfqpoint{2.867647in}{0.500000in}}{\pgfqpoint{1.764706in}{1.700000in}}%
\pgfusepath{clip}%
\pgfsetbuttcap%
\pgfsetroundjoin%
\definecolor{currentfill}{rgb}{0.979124,0.903132,0.839793}%
\pgfsetfillcolor{currentfill}%
\pgfsetlinewidth{0.311001pt}%
\definecolor{currentstroke}{rgb}{1.000000,1.000000,1.000000}%
\pgfsetstrokecolor{currentstroke}%
\pgfsetdash{}{0pt}%
\pgfpathmoveto{\pgfqpoint{4.157276in}{1.082136in}}%
\pgfpathcurveto{\pgfqpoint{4.164409in}{1.082136in}}{\pgfqpoint{4.171251in}{1.084970in}}{\pgfqpoint{4.176295in}{1.090013in}}%
\pgfpathcurveto{\pgfqpoint{4.181338in}{1.095057in}}{\pgfqpoint{4.184172in}{1.101899in}}{\pgfqpoint{4.184172in}{1.109032in}}%
\pgfpathcurveto{\pgfqpoint{4.184172in}{1.116164in}}{\pgfqpoint{4.181338in}{1.123006in}}{\pgfqpoint{4.176295in}{1.128050in}}%
\pgfpathcurveto{\pgfqpoint{4.171251in}{1.133093in}}{\pgfqpoint{4.164409in}{1.135927in}}{\pgfqpoint{4.157276in}{1.135927in}}%
\pgfpathcurveto{\pgfqpoint{4.150144in}{1.135927in}}{\pgfqpoint{4.143302in}{1.133093in}}{\pgfqpoint{4.138258in}{1.128050in}}%
\pgfpathcurveto{\pgfqpoint{4.133215in}{1.123006in}}{\pgfqpoint{4.130381in}{1.116164in}}{\pgfqpoint{4.130381in}{1.109032in}}%
\pgfpathcurveto{\pgfqpoint{4.130381in}{1.101899in}}{\pgfqpoint{4.133215in}{1.095057in}}{\pgfqpoint{4.138258in}{1.090013in}}%
\pgfpathcurveto{\pgfqpoint{4.143302in}{1.084970in}}{\pgfqpoint{4.150144in}{1.082136in}}{\pgfqpoint{4.157276in}{1.082136in}}%
\pgfpathclose%
\pgfusepath{stroke,fill}%
\end{pgfscope}%
\begin{pgfscope}%
\pgfpathrectangle{\pgfqpoint{2.867647in}{0.500000in}}{\pgfqpoint{1.764706in}{1.700000in}}%
\pgfusepath{clip}%
\pgfsetbuttcap%
\pgfsetroundjoin%
\definecolor{currentfill}{rgb}{0.972201,0.839051,0.745789}%
\pgfsetfillcolor{currentfill}%
\pgfsetlinewidth{0.311001pt}%
\definecolor{currentstroke}{rgb}{1.000000,1.000000,1.000000}%
\pgfsetstrokecolor{currentstroke}%
\pgfsetdash{}{0pt}%
\pgfpathmoveto{\pgfqpoint{4.118432in}{1.322010in}}%
\pgfpathcurveto{\pgfqpoint{4.125564in}{1.322010in}}{\pgfqpoint{4.132406in}{1.324844in}}{\pgfqpoint{4.137450in}{1.329887in}}%
\pgfpathcurveto{\pgfqpoint{4.142493in}{1.334931in}}{\pgfqpoint{4.145327in}{1.341773in}}{\pgfqpoint{4.145327in}{1.348905in}}%
\pgfpathcurveto{\pgfqpoint{4.145327in}{1.356038in}}{\pgfqpoint{4.142493in}{1.362880in}}{\pgfqpoint{4.137450in}{1.367924in}}%
\pgfpathcurveto{\pgfqpoint{4.132406in}{1.372967in}}{\pgfqpoint{4.125564in}{1.375801in}}{\pgfqpoint{4.118432in}{1.375801in}}%
\pgfpathcurveto{\pgfqpoint{4.111299in}{1.375801in}}{\pgfqpoint{4.104457in}{1.372967in}}{\pgfqpoint{4.099414in}{1.367924in}}%
\pgfpathcurveto{\pgfqpoint{4.094370in}{1.362880in}}{\pgfqpoint{4.091536in}{1.356038in}}{\pgfqpoint{4.091536in}{1.348905in}}%
\pgfpathcurveto{\pgfqpoint{4.091536in}{1.341773in}}{\pgfqpoint{4.094370in}{1.334931in}}{\pgfqpoint{4.099414in}{1.329887in}}%
\pgfpathcurveto{\pgfqpoint{4.104457in}{1.324844in}}{\pgfqpoint{4.111299in}{1.322010in}}{\pgfqpoint{4.118432in}{1.322010in}}%
\pgfpathclose%
\pgfusepath{stroke,fill}%
\end{pgfscope}%
\begin{pgfscope}%
\pgfpathrectangle{\pgfqpoint{2.867647in}{0.500000in}}{\pgfqpoint{1.764706in}{1.700000in}}%
\pgfusepath{clip}%
\pgfsetbuttcap%
\pgfsetroundjoin%
\definecolor{currentfill}{rgb}{0.980678,0.914765,0.856766}%
\pgfsetfillcolor{currentfill}%
\pgfsetlinewidth{0.311001pt}%
\definecolor{currentstroke}{rgb}{1.000000,1.000000,1.000000}%
\pgfsetstrokecolor{currentstroke}%
\pgfsetdash{}{0pt}%
\pgfpathmoveto{\pgfqpoint{4.180862in}{1.378420in}}%
\pgfpathcurveto{\pgfqpoint{4.187995in}{1.378420in}}{\pgfqpoint{4.194837in}{1.381254in}}{\pgfqpoint{4.199881in}{1.386298in}}%
\pgfpathcurveto{\pgfqpoint{4.204924in}{1.391341in}}{\pgfqpoint{4.207758in}{1.398183in}}{\pgfqpoint{4.207758in}{1.405316in}}%
\pgfpathcurveto{\pgfqpoint{4.207758in}{1.412449in}}{\pgfqpoint{4.204924in}{1.419290in}}{\pgfqpoint{4.199881in}{1.424334in}}%
\pgfpathcurveto{\pgfqpoint{4.194837in}{1.429378in}}{\pgfqpoint{4.187995in}{1.432211in}}{\pgfqpoint{4.180862in}{1.432211in}}%
\pgfpathcurveto{\pgfqpoint{4.173730in}{1.432211in}}{\pgfqpoint{4.166888in}{1.429378in}}{\pgfqpoint{4.161844in}{1.424334in}}%
\pgfpathcurveto{\pgfqpoint{4.156801in}{1.419290in}}{\pgfqpoint{4.153967in}{1.412449in}}{\pgfqpoint{4.153967in}{1.405316in}}%
\pgfpathcurveto{\pgfqpoint{4.153967in}{1.398183in}}{\pgfqpoint{4.156801in}{1.391341in}}{\pgfqpoint{4.161844in}{1.386298in}}%
\pgfpathcurveto{\pgfqpoint{4.166888in}{1.381254in}}{\pgfqpoint{4.173730in}{1.378420in}}{\pgfqpoint{4.180862in}{1.378420in}}%
\pgfpathclose%
\pgfusepath{stroke,fill}%
\end{pgfscope}%
\begin{pgfscope}%
\pgfpathrectangle{\pgfqpoint{2.867647in}{0.500000in}}{\pgfqpoint{1.764706in}{1.700000in}}%
\pgfusepath{clip}%
\pgfsetbuttcap%
\pgfsetroundjoin%
\definecolor{currentfill}{rgb}{0.973271,0.850724,0.762998}%
\pgfsetfillcolor{currentfill}%
\pgfsetlinewidth{0.311001pt}%
\definecolor{currentstroke}{rgb}{1.000000,1.000000,1.000000}%
\pgfsetstrokecolor{currentstroke}%
\pgfsetdash{}{0pt}%
\pgfpathmoveto{\pgfqpoint{4.159080in}{1.013145in}}%
\pgfpathcurveto{\pgfqpoint{4.166212in}{1.013145in}}{\pgfqpoint{4.173054in}{1.015979in}}{\pgfqpoint{4.178098in}{1.021022in}}%
\pgfpathcurveto{\pgfqpoint{4.183141in}{1.026066in}}{\pgfqpoint{4.185975in}{1.032908in}}{\pgfqpoint{4.185975in}{1.040041in}}%
\pgfpathcurveto{\pgfqpoint{4.185975in}{1.047173in}}{\pgfqpoint{4.183141in}{1.054015in}}{\pgfqpoint{4.178098in}{1.059059in}}%
\pgfpathcurveto{\pgfqpoint{4.173054in}{1.064102in}}{\pgfqpoint{4.166212in}{1.066936in}}{\pgfqpoint{4.159080in}{1.066936in}}%
\pgfpathcurveto{\pgfqpoint{4.151947in}{1.066936in}}{\pgfqpoint{4.145105in}{1.064102in}}{\pgfqpoint{4.140061in}{1.059059in}}%
\pgfpathcurveto{\pgfqpoint{4.135018in}{1.054015in}}{\pgfqpoint{4.132184in}{1.047173in}}{\pgfqpoint{4.132184in}{1.040041in}}%
\pgfpathcurveto{\pgfqpoint{4.132184in}{1.032908in}}{\pgfqpoint{4.135018in}{1.026066in}}{\pgfqpoint{4.140061in}{1.021022in}}%
\pgfpathcurveto{\pgfqpoint{4.145105in}{1.015979in}}{\pgfqpoint{4.151947in}{1.013145in}}{\pgfqpoint{4.159080in}{1.013145in}}%
\pgfpathclose%
\pgfusepath{stroke,fill}%
\end{pgfscope}%
\begin{pgfscope}%
\pgfpathrectangle{\pgfqpoint{2.867647in}{0.500000in}}{\pgfqpoint{1.764706in}{1.700000in}}%
\pgfusepath{clip}%
\pgfsetbuttcap%
\pgfsetroundjoin%
\definecolor{currentfill}{rgb}{0.883342,0.198306,0.260142}%
\pgfsetfillcolor{currentfill}%
\pgfsetlinewidth{0.311001pt}%
\definecolor{currentstroke}{rgb}{1.000000,1.000000,1.000000}%
\pgfsetstrokecolor{currentstroke}%
\pgfsetdash{}{0pt}%
\pgfpathmoveto{\pgfqpoint{3.878473in}{0.799224in}}%
\pgfpathcurveto{\pgfqpoint{3.885606in}{0.799224in}}{\pgfqpoint{3.892448in}{0.802058in}}{\pgfqpoint{3.897491in}{0.807101in}}%
\pgfpathcurveto{\pgfqpoint{3.902535in}{0.812145in}}{\pgfqpoint{3.905369in}{0.818987in}}{\pgfqpoint{3.905369in}{0.826119in}}%
\pgfpathcurveto{\pgfqpoint{3.905369in}{0.833252in}}{\pgfqpoint{3.902535in}{0.840094in}}{\pgfqpoint{3.897491in}{0.845138in}}%
\pgfpathcurveto{\pgfqpoint{3.892448in}{0.850181in}}{\pgfqpoint{3.885606in}{0.853015in}}{\pgfqpoint{3.878473in}{0.853015in}}%
\pgfpathcurveto{\pgfqpoint{3.871340in}{0.853015in}}{\pgfqpoint{3.864499in}{0.850181in}}{\pgfqpoint{3.859455in}{0.845138in}}%
\pgfpathcurveto{\pgfqpoint{3.854411in}{0.840094in}}{\pgfqpoint{3.851577in}{0.833252in}}{\pgfqpoint{3.851577in}{0.826119in}}%
\pgfpathcurveto{\pgfqpoint{3.851577in}{0.818987in}}{\pgfqpoint{3.854411in}{0.812145in}}{\pgfqpoint{3.859455in}{0.807101in}}%
\pgfpathcurveto{\pgfqpoint{3.864499in}{0.802058in}}{\pgfqpoint{3.871340in}{0.799224in}}{\pgfqpoint{3.878473in}{0.799224in}}%
\pgfpathclose%
\pgfusepath{stroke,fill}%
\end{pgfscope}%
\begin{pgfscope}%
\pgfpathrectangle{\pgfqpoint{2.867647in}{0.500000in}}{\pgfqpoint{1.764706in}{1.700000in}}%
\pgfusepath{clip}%
\pgfsetbuttcap%
\pgfsetroundjoin%
\definecolor{currentfill}{rgb}{0.969359,0.803954,0.693832}%
\pgfsetfillcolor{currentfill}%
\pgfsetlinewidth{0.311001pt}%
\definecolor{currentstroke}{rgb}{1.000000,1.000000,1.000000}%
\pgfsetstrokecolor{currentstroke}%
\pgfsetdash{}{0pt}%
\pgfpathmoveto{\pgfqpoint{4.119399in}{0.951664in}}%
\pgfpathcurveto{\pgfqpoint{4.126532in}{0.951664in}}{\pgfqpoint{4.133374in}{0.954498in}}{\pgfqpoint{4.138418in}{0.959541in}}%
\pgfpathcurveto{\pgfqpoint{4.143461in}{0.964585in}}{\pgfqpoint{4.146295in}{0.971427in}}{\pgfqpoint{4.146295in}{0.978559in}}%
\pgfpathcurveto{\pgfqpoint{4.146295in}{0.985692in}}{\pgfqpoint{4.143461in}{0.992534in}}{\pgfqpoint{4.138418in}{0.997577in}}%
\pgfpathcurveto{\pgfqpoint{4.133374in}{1.002621in}}{\pgfqpoint{4.126532in}{1.005455in}}{\pgfqpoint{4.119399in}{1.005455in}}%
\pgfpathcurveto{\pgfqpoint{4.112267in}{1.005455in}}{\pgfqpoint{4.105425in}{1.002621in}}{\pgfqpoint{4.100381in}{0.997577in}}%
\pgfpathcurveto{\pgfqpoint{4.095338in}{0.992534in}}{\pgfqpoint{4.092504in}{0.985692in}}{\pgfqpoint{4.092504in}{0.978559in}}%
\pgfpathcurveto{\pgfqpoint{4.092504in}{0.971427in}}{\pgfqpoint{4.095338in}{0.964585in}}{\pgfqpoint{4.100381in}{0.959541in}}%
\pgfpathcurveto{\pgfqpoint{4.105425in}{0.954498in}}{\pgfqpoint{4.112267in}{0.951664in}}{\pgfqpoint{4.119399in}{0.951664in}}%
\pgfpathclose%
\pgfusepath{stroke,fill}%
\end{pgfscope}%
\begin{pgfscope}%
\pgfpathrectangle{\pgfqpoint{2.867647in}{0.500000in}}{\pgfqpoint{1.764706in}{1.700000in}}%
\pgfusepath{clip}%
\pgfsetbuttcap%
\pgfsetroundjoin%
\definecolor{currentfill}{rgb}{0.980678,0.914765,0.856766}%
\pgfsetfillcolor{currentfill}%
\pgfsetlinewidth{0.311001pt}%
\definecolor{currentstroke}{rgb}{1.000000,1.000000,1.000000}%
\pgfsetstrokecolor{currentstroke}%
\pgfsetdash{}{0pt}%
\pgfpathmoveto{\pgfqpoint{4.176311in}{1.326808in}}%
\pgfpathcurveto{\pgfqpoint{4.183444in}{1.326808in}}{\pgfqpoint{4.190286in}{1.329642in}}{\pgfqpoint{4.195329in}{1.334685in}}%
\pgfpathcurveto{\pgfqpoint{4.200373in}{1.339729in}}{\pgfqpoint{4.203207in}{1.346571in}}{\pgfqpoint{4.203207in}{1.353704in}}%
\pgfpathcurveto{\pgfqpoint{4.203207in}{1.360836in}}{\pgfqpoint{4.200373in}{1.367678in}}{\pgfqpoint{4.195329in}{1.372722in}}%
\pgfpathcurveto{\pgfqpoint{4.190286in}{1.377765in}}{\pgfqpoint{4.183444in}{1.380599in}}{\pgfqpoint{4.176311in}{1.380599in}}%
\pgfpathcurveto{\pgfqpoint{4.169178in}{1.380599in}}{\pgfqpoint{4.162337in}{1.377765in}}{\pgfqpoint{4.157293in}{1.372722in}}%
\pgfpathcurveto{\pgfqpoint{4.152249in}{1.367678in}}{\pgfqpoint{4.149415in}{1.360836in}}{\pgfqpoint{4.149415in}{1.353704in}}%
\pgfpathcurveto{\pgfqpoint{4.149415in}{1.346571in}}{\pgfqpoint{4.152249in}{1.339729in}}{\pgfqpoint{4.157293in}{1.334685in}}%
\pgfpathcurveto{\pgfqpoint{4.162337in}{1.329642in}}{\pgfqpoint{4.169178in}{1.326808in}}{\pgfqpoint{4.176311in}{1.326808in}}%
\pgfpathclose%
\pgfusepath{stroke,fill}%
\end{pgfscope}%
\begin{pgfscope}%
\pgfpathrectangle{\pgfqpoint{2.867647in}{0.500000in}}{\pgfqpoint{1.764706in}{1.700000in}}%
\pgfusepath{clip}%
\pgfsetbuttcap%
\pgfsetroundjoin%
\definecolor{currentfill}{rgb}{0.976961,0.885681,0.814303}%
\pgfsetfillcolor{currentfill}%
\pgfsetlinewidth{0.311001pt}%
\definecolor{currentstroke}{rgb}{1.000000,1.000000,1.000000}%
\pgfsetstrokecolor{currentstroke}%
\pgfsetdash{}{0pt}%
\pgfpathmoveto{\pgfqpoint{4.240186in}{1.335114in}}%
\pgfpathcurveto{\pgfqpoint{4.247319in}{1.335114in}}{\pgfqpoint{4.254160in}{1.337947in}}{\pgfqpoint{4.259204in}{1.342991in}}%
\pgfpathcurveto{\pgfqpoint{4.264248in}{1.348035in}}{\pgfqpoint{4.267082in}{1.354876in}}{\pgfqpoint{4.267082in}{1.362009in}}%
\pgfpathcurveto{\pgfqpoint{4.267082in}{1.369142in}}{\pgfqpoint{4.264248in}{1.375984in}}{\pgfqpoint{4.259204in}{1.381027in}}%
\pgfpathcurveto{\pgfqpoint{4.254160in}{1.386071in}}{\pgfqpoint{4.247319in}{1.388905in}}{\pgfqpoint{4.240186in}{1.388905in}}%
\pgfpathcurveto{\pgfqpoint{4.233053in}{1.388905in}}{\pgfqpoint{4.226211in}{1.386071in}}{\pgfqpoint{4.221168in}{1.381027in}}%
\pgfpathcurveto{\pgfqpoint{4.216124in}{1.375984in}}{\pgfqpoint{4.213290in}{1.369142in}}{\pgfqpoint{4.213290in}{1.362009in}}%
\pgfpathcurveto{\pgfqpoint{4.213290in}{1.354876in}}{\pgfqpoint{4.216124in}{1.348035in}}{\pgfqpoint{4.221168in}{1.342991in}}%
\pgfpathcurveto{\pgfqpoint{4.226211in}{1.337947in}}{\pgfqpoint{4.233053in}{1.335114in}}{\pgfqpoint{4.240186in}{1.335114in}}%
\pgfpathclose%
\pgfusepath{stroke,fill}%
\end{pgfscope}%
\begin{pgfscope}%
\pgfpathrectangle{\pgfqpoint{2.867647in}{0.500000in}}{\pgfqpoint{1.764706in}{1.700000in}}%
\pgfusepath{clip}%
\pgfsetbuttcap%
\pgfsetroundjoin%
\definecolor{currentfill}{rgb}{0.978376,0.897317,0.831308}%
\pgfsetfillcolor{currentfill}%
\pgfsetlinewidth{0.311001pt}%
\definecolor{currentstroke}{rgb}{1.000000,1.000000,1.000000}%
\pgfsetstrokecolor{currentstroke}%
\pgfsetdash{}{0pt}%
\pgfpathmoveto{\pgfqpoint{4.111183in}{1.557541in}}%
\pgfpathcurveto{\pgfqpoint{4.118315in}{1.557541in}}{\pgfqpoint{4.125157in}{1.560375in}}{\pgfqpoint{4.130201in}{1.565418in}}%
\pgfpathcurveto{\pgfqpoint{4.135244in}{1.570462in}}{\pgfqpoint{4.138078in}{1.577304in}}{\pgfqpoint{4.138078in}{1.584436in}}%
\pgfpathcurveto{\pgfqpoint{4.138078in}{1.591569in}}{\pgfqpoint{4.135244in}{1.598411in}}{\pgfqpoint{4.130201in}{1.603455in}}%
\pgfpathcurveto{\pgfqpoint{4.125157in}{1.608498in}}{\pgfqpoint{4.118315in}{1.611332in}}{\pgfqpoint{4.111183in}{1.611332in}}%
\pgfpathcurveto{\pgfqpoint{4.104050in}{1.611332in}}{\pgfqpoint{4.097208in}{1.608498in}}{\pgfqpoint{4.092165in}{1.603455in}}%
\pgfpathcurveto{\pgfqpoint{4.087121in}{1.598411in}}{\pgfqpoint{4.084287in}{1.591569in}}{\pgfqpoint{4.084287in}{1.584436in}}%
\pgfpathcurveto{\pgfqpoint{4.084287in}{1.577304in}}{\pgfqpoint{4.087121in}{1.570462in}}{\pgfqpoint{4.092165in}{1.565418in}}%
\pgfpathcurveto{\pgfqpoint{4.097208in}{1.560375in}}{\pgfqpoint{4.104050in}{1.557541in}}{\pgfqpoint{4.111183in}{1.557541in}}%
\pgfpathclose%
\pgfusepath{stroke,fill}%
\end{pgfscope}%
\begin{pgfscope}%
\pgfpathrectangle{\pgfqpoint{2.867647in}{0.500000in}}{\pgfqpoint{1.764706in}{1.700000in}}%
\pgfusepath{clip}%
\pgfsetbuttcap%
\pgfsetroundjoin%
\definecolor{currentfill}{rgb}{0.965592,0.726236,0.584384}%
\pgfsetfillcolor{currentfill}%
\pgfsetlinewidth{0.311001pt}%
\definecolor{currentstroke}{rgb}{1.000000,1.000000,1.000000}%
\pgfsetstrokecolor{currentstroke}%
\pgfsetdash{}{0pt}%
\pgfpathmoveto{\pgfqpoint{4.027307in}{1.544067in}}%
\pgfpathcurveto{\pgfqpoint{4.034440in}{1.544067in}}{\pgfqpoint{4.041282in}{1.546901in}}{\pgfqpoint{4.046325in}{1.551945in}}%
\pgfpathcurveto{\pgfqpoint{4.051369in}{1.556988in}}{\pgfqpoint{4.054203in}{1.563830in}}{\pgfqpoint{4.054203in}{1.570963in}}%
\pgfpathcurveto{\pgfqpoint{4.054203in}{1.578095in}}{\pgfqpoint{4.051369in}{1.584937in}}{\pgfqpoint{4.046325in}{1.589981in}}%
\pgfpathcurveto{\pgfqpoint{4.041282in}{1.595024in}}{\pgfqpoint{4.034440in}{1.597858in}}{\pgfqpoint{4.027307in}{1.597858in}}%
\pgfpathcurveto{\pgfqpoint{4.020174in}{1.597858in}}{\pgfqpoint{4.013333in}{1.595024in}}{\pgfqpoint{4.008289in}{1.589981in}}%
\pgfpathcurveto{\pgfqpoint{4.003245in}{1.584937in}}{\pgfqpoint{4.000412in}{1.578095in}}{\pgfqpoint{4.000412in}{1.570963in}}%
\pgfpathcurveto{\pgfqpoint{4.000412in}{1.563830in}}{\pgfqpoint{4.003245in}{1.556988in}}{\pgfqpoint{4.008289in}{1.551945in}}%
\pgfpathcurveto{\pgfqpoint{4.013333in}{1.546901in}}{\pgfqpoint{4.020174in}{1.544067in}}{\pgfqpoint{4.027307in}{1.544067in}}%
\pgfpathclose%
\pgfusepath{stroke,fill}%
\end{pgfscope}%
\begin{pgfscope}%
\pgfpathrectangle{\pgfqpoint{2.867647in}{0.500000in}}{\pgfqpoint{1.764706in}{1.700000in}}%
\pgfusepath{clip}%
\pgfsetbuttcap%
\pgfsetroundjoin%
\definecolor{currentfill}{rgb}{0.976961,0.885681,0.814303}%
\pgfsetfillcolor{currentfill}%
\pgfsetlinewidth{0.311001pt}%
\definecolor{currentstroke}{rgb}{1.000000,1.000000,1.000000}%
\pgfsetstrokecolor{currentstroke}%
\pgfsetdash{}{0pt}%
\pgfpathmoveto{\pgfqpoint{4.130504in}{1.451262in}}%
\pgfpathcurveto{\pgfqpoint{4.137637in}{1.451262in}}{\pgfqpoint{4.144478in}{1.454096in}}{\pgfqpoint{4.149522in}{1.459139in}}%
\pgfpathcurveto{\pgfqpoint{4.154566in}{1.464183in}}{\pgfqpoint{4.157399in}{1.471025in}}{\pgfqpoint{4.157399in}{1.478158in}}%
\pgfpathcurveto{\pgfqpoint{4.157399in}{1.485290in}}{\pgfqpoint{4.154566in}{1.492132in}}{\pgfqpoint{4.149522in}{1.497176in}}%
\pgfpathcurveto{\pgfqpoint{4.144478in}{1.502219in}}{\pgfqpoint{4.137637in}{1.505053in}}{\pgfqpoint{4.130504in}{1.505053in}}%
\pgfpathcurveto{\pgfqpoint{4.123371in}{1.505053in}}{\pgfqpoint{4.116529in}{1.502219in}}{\pgfqpoint{4.111486in}{1.497176in}}%
\pgfpathcurveto{\pgfqpoint{4.106442in}{1.492132in}}{\pgfqpoint{4.103608in}{1.485290in}}{\pgfqpoint{4.103608in}{1.478158in}}%
\pgfpathcurveto{\pgfqpoint{4.103608in}{1.471025in}}{\pgfqpoint{4.106442in}{1.464183in}}{\pgfqpoint{4.111486in}{1.459139in}}%
\pgfpathcurveto{\pgfqpoint{4.116529in}{1.454096in}}{\pgfqpoint{4.123371in}{1.451262in}}{\pgfqpoint{4.130504in}{1.451262in}}%
\pgfpathclose%
\pgfusepath{stroke,fill}%
\end{pgfscope}%
\begin{pgfscope}%
\pgfpathrectangle{\pgfqpoint{2.867647in}{0.500000in}}{\pgfqpoint{1.764706in}{1.700000in}}%
\pgfusepath{clip}%
\pgfsetbuttcap%
\pgfsetroundjoin%
\definecolor{currentfill}{rgb}{0.964433,0.670254,0.515093}%
\pgfsetfillcolor{currentfill}%
\pgfsetlinewidth{0.311001pt}%
\definecolor{currentstroke}{rgb}{1.000000,1.000000,1.000000}%
\pgfsetstrokecolor{currentstroke}%
\pgfsetdash{}{0pt}%
\pgfpathmoveto{\pgfqpoint{4.255954in}{1.040576in}}%
\pgfpathcurveto{\pgfqpoint{4.263086in}{1.040576in}}{\pgfqpoint{4.269928in}{1.043410in}}{\pgfqpoint{4.274972in}{1.048453in}}%
\pgfpathcurveto{\pgfqpoint{4.280015in}{1.053497in}}{\pgfqpoint{4.282849in}{1.060339in}}{\pgfqpoint{4.282849in}{1.067471in}}%
\pgfpathcurveto{\pgfqpoint{4.282849in}{1.074604in}}{\pgfqpoint{4.280015in}{1.081446in}}{\pgfqpoint{4.274972in}{1.086490in}}%
\pgfpathcurveto{\pgfqpoint{4.269928in}{1.091533in}}{\pgfqpoint{4.263086in}{1.094367in}}{\pgfqpoint{4.255954in}{1.094367in}}%
\pgfpathcurveto{\pgfqpoint{4.248821in}{1.094367in}}{\pgfqpoint{4.241979in}{1.091533in}}{\pgfqpoint{4.236935in}{1.086490in}}%
\pgfpathcurveto{\pgfqpoint{4.231892in}{1.081446in}}{\pgfqpoint{4.229058in}{1.074604in}}{\pgfqpoint{4.229058in}{1.067471in}}%
\pgfpathcurveto{\pgfqpoint{4.229058in}{1.060339in}}{\pgfqpoint{4.231892in}{1.053497in}}{\pgfqpoint{4.236935in}{1.048453in}}%
\pgfpathcurveto{\pgfqpoint{4.241979in}{1.043410in}}{\pgfqpoint{4.248821in}{1.040576in}}{\pgfqpoint{4.255954in}{1.040576in}}%
\pgfpathclose%
\pgfusepath{stroke,fill}%
\end{pgfscope}%
\begin{pgfscope}%
\pgfpathrectangle{\pgfqpoint{2.867647in}{0.500000in}}{\pgfqpoint{1.764706in}{1.700000in}}%
\pgfusepath{clip}%
\pgfsetbuttcap%
\pgfsetroundjoin%
\definecolor{currentfill}{rgb}{0.953816,0.463738,0.317699}%
\pgfsetfillcolor{currentfill}%
\pgfsetlinewidth{0.311001pt}%
\definecolor{currentstroke}{rgb}{1.000000,1.000000,1.000000}%
\pgfsetstrokecolor{currentstroke}%
\pgfsetdash{}{0pt}%
\pgfpathmoveto{\pgfqpoint{4.034638in}{1.846216in}}%
\pgfpathcurveto{\pgfqpoint{4.041771in}{1.846216in}}{\pgfqpoint{4.048613in}{1.849050in}}{\pgfqpoint{4.053657in}{1.854094in}}%
\pgfpathcurveto{\pgfqpoint{4.058700in}{1.859138in}}{\pgfqpoint{4.061534in}{1.865979in}}{\pgfqpoint{4.061534in}{1.873112in}}%
\pgfpathcurveto{\pgfqpoint{4.061534in}{1.880245in}}{\pgfqpoint{4.058700in}{1.887087in}}{\pgfqpoint{4.053657in}{1.892130in}}%
\pgfpathcurveto{\pgfqpoint{4.048613in}{1.897174in}}{\pgfqpoint{4.041771in}{1.900008in}}{\pgfqpoint{4.034638in}{1.900008in}}%
\pgfpathcurveto{\pgfqpoint{4.027506in}{1.900008in}}{\pgfqpoint{4.020664in}{1.897174in}}{\pgfqpoint{4.015620in}{1.892130in}}%
\pgfpathcurveto{\pgfqpoint{4.010577in}{1.887087in}}{\pgfqpoint{4.007743in}{1.880245in}}{\pgfqpoint{4.007743in}{1.873112in}}%
\pgfpathcurveto{\pgfqpoint{4.007743in}{1.865979in}}{\pgfqpoint{4.010577in}{1.859138in}}{\pgfqpoint{4.015620in}{1.854094in}}%
\pgfpathcurveto{\pgfqpoint{4.020664in}{1.849050in}}{\pgfqpoint{4.027506in}{1.846216in}}{\pgfqpoint{4.034638in}{1.846216in}}%
\pgfpathclose%
\pgfusepath{stroke,fill}%
\end{pgfscope}%
\begin{pgfscope}%
\pgfpathrectangle{\pgfqpoint{2.867647in}{0.500000in}}{\pgfqpoint{1.764706in}{1.700000in}}%
\pgfusepath{clip}%
\pgfsetbuttcap%
\pgfsetroundjoin%
\definecolor{currentfill}{rgb}{0.976287,0.879862,0.805788}%
\pgfsetfillcolor{currentfill}%
\pgfsetlinewidth{0.311001pt}%
\definecolor{currentstroke}{rgb}{1.000000,1.000000,1.000000}%
\pgfsetstrokecolor{currentstroke}%
\pgfsetdash{}{0pt}%
\pgfpathmoveto{\pgfqpoint{4.220765in}{1.463308in}}%
\pgfpathcurveto{\pgfqpoint{4.227898in}{1.463308in}}{\pgfqpoint{4.234740in}{1.466142in}}{\pgfqpoint{4.239784in}{1.471186in}}%
\pgfpathcurveto{\pgfqpoint{4.244827in}{1.476229in}}{\pgfqpoint{4.247661in}{1.483071in}}{\pgfqpoint{4.247661in}{1.490204in}}%
\pgfpathcurveto{\pgfqpoint{4.247661in}{1.497337in}}{\pgfqpoint{4.244827in}{1.504178in}}{\pgfqpoint{4.239784in}{1.509222in}}%
\pgfpathcurveto{\pgfqpoint{4.234740in}{1.514266in}}{\pgfqpoint{4.227898in}{1.517099in}}{\pgfqpoint{4.220765in}{1.517099in}}%
\pgfpathcurveto{\pgfqpoint{4.213633in}{1.517099in}}{\pgfqpoint{4.206791in}{1.514266in}}{\pgfqpoint{4.201747in}{1.509222in}}%
\pgfpathcurveto{\pgfqpoint{4.196704in}{1.504178in}}{\pgfqpoint{4.193870in}{1.497337in}}{\pgfqpoint{4.193870in}{1.490204in}}%
\pgfpathcurveto{\pgfqpoint{4.193870in}{1.483071in}}{\pgfqpoint{4.196704in}{1.476229in}}{\pgfqpoint{4.201747in}{1.471186in}}%
\pgfpathcurveto{\pgfqpoint{4.206791in}{1.466142in}}{\pgfqpoint{4.213633in}{1.463308in}}{\pgfqpoint{4.220765in}{1.463308in}}%
\pgfpathclose%
\pgfusepath{stroke,fill}%
\end{pgfscope}%
\begin{pgfscope}%
\pgfpathrectangle{\pgfqpoint{2.867647in}{0.500000in}}{\pgfqpoint{1.764706in}{1.700000in}}%
\pgfusepath{clip}%
\pgfsetbuttcap%
\pgfsetroundjoin%
\definecolor{currentfill}{rgb}{0.970718,0.821518,0.719872}%
\pgfsetfillcolor{currentfill}%
\pgfsetlinewidth{0.311001pt}%
\definecolor{currentstroke}{rgb}{1.000000,1.000000,1.000000}%
\pgfsetstrokecolor{currentstroke}%
\pgfsetdash{}{0pt}%
\pgfpathmoveto{\pgfqpoint{4.095294in}{1.199921in}}%
\pgfpathcurveto{\pgfqpoint{4.102427in}{1.199921in}}{\pgfqpoint{4.109268in}{1.202755in}}{\pgfqpoint{4.114312in}{1.207799in}}%
\pgfpathcurveto{\pgfqpoint{4.119356in}{1.212843in}}{\pgfqpoint{4.122190in}{1.219684in}}{\pgfqpoint{4.122190in}{1.226817in}}%
\pgfpathcurveto{\pgfqpoint{4.122190in}{1.233950in}}{\pgfqpoint{4.119356in}{1.240792in}}{\pgfqpoint{4.114312in}{1.245835in}}%
\pgfpathcurveto{\pgfqpoint{4.109268in}{1.250879in}}{\pgfqpoint{4.102427in}{1.253713in}}{\pgfqpoint{4.095294in}{1.253713in}}%
\pgfpathcurveto{\pgfqpoint{4.088161in}{1.253713in}}{\pgfqpoint{4.081320in}{1.250879in}}{\pgfqpoint{4.076276in}{1.245835in}}%
\pgfpathcurveto{\pgfqpoint{4.071232in}{1.240792in}}{\pgfqpoint{4.068398in}{1.233950in}}{\pgfqpoint{4.068398in}{1.226817in}}%
\pgfpathcurveto{\pgfqpoint{4.068398in}{1.219684in}}{\pgfqpoint{4.071232in}{1.212843in}}{\pgfqpoint{4.076276in}{1.207799in}}%
\pgfpathcurveto{\pgfqpoint{4.081320in}{1.202755in}}{\pgfqpoint{4.088161in}{1.199921in}}{\pgfqpoint{4.095294in}{1.199921in}}%
\pgfpathclose%
\pgfusepath{stroke,fill}%
\end{pgfscope}%
\begin{pgfscope}%
\pgfpathrectangle{\pgfqpoint{2.867647in}{0.500000in}}{\pgfqpoint{1.764706in}{1.700000in}}%
\pgfusepath{clip}%
\pgfsetbuttcap%
\pgfsetroundjoin%
\definecolor{currentfill}{rgb}{0.965302,0.713942,0.568499}%
\pgfsetfillcolor{currentfill}%
\pgfsetlinewidth{0.311001pt}%
\definecolor{currentstroke}{rgb}{1.000000,1.000000,1.000000}%
\pgfsetstrokecolor{currentstroke}%
\pgfsetdash{}{0pt}%
\pgfpathmoveto{\pgfqpoint{4.141263in}{0.921521in}}%
\pgfpathcurveto{\pgfqpoint{4.148396in}{0.921521in}}{\pgfqpoint{4.155238in}{0.924355in}}{\pgfqpoint{4.160281in}{0.929398in}}%
\pgfpathcurveto{\pgfqpoint{4.165325in}{0.934442in}}{\pgfqpoint{4.168159in}{0.941284in}}{\pgfqpoint{4.168159in}{0.948416in}}%
\pgfpathcurveto{\pgfqpoint{4.168159in}{0.955549in}}{\pgfqpoint{4.165325in}{0.962391in}}{\pgfqpoint{4.160281in}{0.967435in}}%
\pgfpathcurveto{\pgfqpoint{4.155238in}{0.972478in}}{\pgfqpoint{4.148396in}{0.975312in}}{\pgfqpoint{4.141263in}{0.975312in}}%
\pgfpathcurveto{\pgfqpoint{4.134130in}{0.975312in}}{\pgfqpoint{4.127289in}{0.972478in}}{\pgfqpoint{4.122245in}{0.967435in}}%
\pgfpathcurveto{\pgfqpoint{4.117201in}{0.962391in}}{\pgfqpoint{4.114368in}{0.955549in}}{\pgfqpoint{4.114368in}{0.948416in}}%
\pgfpathcurveto{\pgfqpoint{4.114368in}{0.941284in}}{\pgfqpoint{4.117201in}{0.934442in}}{\pgfqpoint{4.122245in}{0.929398in}}%
\pgfpathcurveto{\pgfqpoint{4.127289in}{0.924355in}}{\pgfqpoint{4.134130in}{0.921521in}}{\pgfqpoint{4.141263in}{0.921521in}}%
\pgfpathclose%
\pgfusepath{stroke,fill}%
\end{pgfscope}%
\begin{pgfscope}%
\pgfpathrectangle{\pgfqpoint{2.867647in}{0.500000in}}{\pgfqpoint{1.764706in}{1.700000in}}%
\pgfusepath{clip}%
\pgfsetbuttcap%
\pgfsetroundjoin%
\definecolor{currentfill}{rgb}{0.955103,0.477872,0.328626}%
\pgfsetfillcolor{currentfill}%
\pgfsetlinewidth{0.311001pt}%
\definecolor{currentstroke}{rgb}{1.000000,1.000000,1.000000}%
\pgfsetstrokecolor{currentstroke}%
\pgfsetdash{}{0pt}%
\pgfpathmoveto{\pgfqpoint{4.346382in}{1.206532in}}%
\pgfpathcurveto{\pgfqpoint{4.353515in}{1.206532in}}{\pgfqpoint{4.360357in}{1.209366in}}{\pgfqpoint{4.365400in}{1.214410in}}%
\pgfpathcurveto{\pgfqpoint{4.370444in}{1.219453in}}{\pgfqpoint{4.373278in}{1.226295in}}{\pgfqpoint{4.373278in}{1.233428in}}%
\pgfpathcurveto{\pgfqpoint{4.373278in}{1.240561in}}{\pgfqpoint{4.370444in}{1.247402in}}{\pgfqpoint{4.365400in}{1.252446in}}%
\pgfpathcurveto{\pgfqpoint{4.360357in}{1.257490in}}{\pgfqpoint{4.353515in}{1.260323in}}{\pgfqpoint{4.346382in}{1.260323in}}%
\pgfpathcurveto{\pgfqpoint{4.339250in}{1.260323in}}{\pgfqpoint{4.332408in}{1.257490in}}{\pgfqpoint{4.327364in}{1.252446in}}%
\pgfpathcurveto{\pgfqpoint{4.322321in}{1.247402in}}{\pgfqpoint{4.319487in}{1.240561in}}{\pgfqpoint{4.319487in}{1.233428in}}%
\pgfpathcurveto{\pgfqpoint{4.319487in}{1.226295in}}{\pgfqpoint{4.322321in}{1.219453in}}{\pgfqpoint{4.327364in}{1.214410in}}%
\pgfpathcurveto{\pgfqpoint{4.332408in}{1.209366in}}{\pgfqpoint{4.339250in}{1.206532in}}{\pgfqpoint{4.346382in}{1.206532in}}%
\pgfpathclose%
\pgfusepath{stroke,fill}%
\end{pgfscope}%
\begin{pgfscope}%
\pgfpathrectangle{\pgfqpoint{2.867647in}{0.500000in}}{\pgfqpoint{1.764706in}{1.700000in}}%
\pgfusepath{clip}%
\pgfsetbuttcap%
\pgfsetroundjoin%
\definecolor{currentfill}{rgb}{0.973832,0.856556,0.771584}%
\pgfsetfillcolor{currentfill}%
\pgfsetlinewidth{0.311001pt}%
\definecolor{currentstroke}{rgb}{1.000000,1.000000,1.000000}%
\pgfsetstrokecolor{currentstroke}%
\pgfsetdash{}{0pt}%
\pgfpathmoveto{\pgfqpoint{4.245610in}{1.419599in}}%
\pgfpathcurveto{\pgfqpoint{4.252743in}{1.419599in}}{\pgfqpoint{4.259584in}{1.422433in}}{\pgfqpoint{4.264628in}{1.427477in}}%
\pgfpathcurveto{\pgfqpoint{4.269672in}{1.432520in}}{\pgfqpoint{4.272505in}{1.439362in}}{\pgfqpoint{4.272505in}{1.446495in}}%
\pgfpathcurveto{\pgfqpoint{4.272505in}{1.453628in}}{\pgfqpoint{4.269672in}{1.460469in}}{\pgfqpoint{4.264628in}{1.465513in}}%
\pgfpathcurveto{\pgfqpoint{4.259584in}{1.470557in}}{\pgfqpoint{4.252743in}{1.473391in}}{\pgfqpoint{4.245610in}{1.473391in}}%
\pgfpathcurveto{\pgfqpoint{4.238477in}{1.473391in}}{\pgfqpoint{4.231635in}{1.470557in}}{\pgfqpoint{4.226592in}{1.465513in}}%
\pgfpathcurveto{\pgfqpoint{4.221548in}{1.460469in}}{\pgfqpoint{4.218714in}{1.453628in}}{\pgfqpoint{4.218714in}{1.446495in}}%
\pgfpathcurveto{\pgfqpoint{4.218714in}{1.439362in}}{\pgfqpoint{4.221548in}{1.432520in}}{\pgfqpoint{4.226592in}{1.427477in}}%
\pgfpathcurveto{\pgfqpoint{4.231635in}{1.422433in}}{\pgfqpoint{4.238477in}{1.419599in}}{\pgfqpoint{4.245610in}{1.419599in}}%
\pgfpathclose%
\pgfusepath{stroke,fill}%
\end{pgfscope}%
\begin{pgfscope}%
\pgfpathrectangle{\pgfqpoint{2.867647in}{0.500000in}}{\pgfqpoint{1.764706in}{1.700000in}}%
\pgfusepath{clip}%
\pgfsetbuttcap%
\pgfsetroundjoin%
\definecolor{currentfill}{rgb}{0.976961,0.885681,0.814303}%
\pgfsetfillcolor{currentfill}%
\pgfsetlinewidth{0.311001pt}%
\definecolor{currentstroke}{rgb}{1.000000,1.000000,1.000000}%
\pgfsetstrokecolor{currentstroke}%
\pgfsetdash{}{0pt}%
\pgfpathmoveto{\pgfqpoint{4.236130in}{1.260524in}}%
\pgfpathcurveto{\pgfqpoint{4.243263in}{1.260524in}}{\pgfqpoint{4.250105in}{1.263358in}}{\pgfqpoint{4.255148in}{1.268401in}}%
\pgfpathcurveto{\pgfqpoint{4.260192in}{1.273445in}}{\pgfqpoint{4.263026in}{1.280287in}}{\pgfqpoint{4.263026in}{1.287419in}}%
\pgfpathcurveto{\pgfqpoint{4.263026in}{1.294552in}}{\pgfqpoint{4.260192in}{1.301394in}}{\pgfqpoint{4.255148in}{1.306438in}}%
\pgfpathcurveto{\pgfqpoint{4.250105in}{1.311481in}}{\pgfqpoint{4.243263in}{1.314315in}}{\pgfqpoint{4.236130in}{1.314315in}}%
\pgfpathcurveto{\pgfqpoint{4.228997in}{1.314315in}}{\pgfqpoint{4.222156in}{1.311481in}}{\pgfqpoint{4.217112in}{1.306438in}}%
\pgfpathcurveto{\pgfqpoint{4.212068in}{1.301394in}}{\pgfqpoint{4.209234in}{1.294552in}}{\pgfqpoint{4.209234in}{1.287419in}}%
\pgfpathcurveto{\pgfqpoint{4.209234in}{1.280287in}}{\pgfqpoint{4.212068in}{1.273445in}}{\pgfqpoint{4.217112in}{1.268401in}}%
\pgfpathcurveto{\pgfqpoint{4.222156in}{1.263358in}}{\pgfqpoint{4.228997in}{1.260524in}}{\pgfqpoint{4.236130in}{1.260524in}}%
\pgfpathclose%
\pgfusepath{stroke,fill}%
\end{pgfscope}%
\begin{pgfscope}%
\pgfpathrectangle{\pgfqpoint{2.867647in}{0.500000in}}{\pgfqpoint{1.764706in}{1.700000in}}%
\pgfusepath{clip}%
\pgfsetbuttcap%
\pgfsetroundjoin%
\definecolor{currentfill}{rgb}{0.651546,0.096802,0.354541}%
\pgfsetfillcolor{currentfill}%
\pgfsetlinewidth{0.311001pt}%
\definecolor{currentstroke}{rgb}{1.000000,1.000000,1.000000}%
\pgfsetstrokecolor{currentstroke}%
\pgfsetdash{}{0pt}%
\pgfpathmoveto{\pgfqpoint{4.248843in}{0.800867in}}%
\pgfpathcurveto{\pgfqpoint{4.255976in}{0.800867in}}{\pgfqpoint{4.262818in}{0.803701in}}{\pgfqpoint{4.267861in}{0.808745in}}%
\pgfpathcurveto{\pgfqpoint{4.272905in}{0.813789in}}{\pgfqpoint{4.275739in}{0.820630in}}{\pgfqpoint{4.275739in}{0.827763in}}%
\pgfpathcurveto{\pgfqpoint{4.275739in}{0.834896in}}{\pgfqpoint{4.272905in}{0.841737in}}{\pgfqpoint{4.267861in}{0.846781in}}%
\pgfpathcurveto{\pgfqpoint{4.262818in}{0.851825in}}{\pgfqpoint{4.255976in}{0.854659in}}{\pgfqpoint{4.248843in}{0.854659in}}%
\pgfpathcurveto{\pgfqpoint{4.241710in}{0.854659in}}{\pgfqpoint{4.234869in}{0.851825in}}{\pgfqpoint{4.229825in}{0.846781in}}%
\pgfpathcurveto{\pgfqpoint{4.224781in}{0.841737in}}{\pgfqpoint{4.221948in}{0.834896in}}{\pgfqpoint{4.221948in}{0.827763in}}%
\pgfpathcurveto{\pgfqpoint{4.221948in}{0.820630in}}{\pgfqpoint{4.224781in}{0.813789in}}{\pgfqpoint{4.229825in}{0.808745in}}%
\pgfpathcurveto{\pgfqpoint{4.234869in}{0.803701in}}{\pgfqpoint{4.241710in}{0.800867in}}{\pgfqpoint{4.248843in}{0.800867in}}%
\pgfpathclose%
\pgfusepath{stroke,fill}%
\end{pgfscope}%
\begin{pgfscope}%
\pgfpathrectangle{\pgfqpoint{2.867647in}{0.500000in}}{\pgfqpoint{1.764706in}{1.700000in}}%
\pgfusepath{clip}%
\pgfsetbuttcap%
\pgfsetroundjoin%
\definecolor{currentfill}{rgb}{0.971202,0.827364,0.728520}%
\pgfsetfillcolor{currentfill}%
\pgfsetlinewidth{0.311001pt}%
\definecolor{currentstroke}{rgb}{1.000000,1.000000,1.000000}%
\pgfsetstrokecolor{currentstroke}%
\pgfsetdash{}{0pt}%
\pgfpathmoveto{\pgfqpoint{4.081087in}{1.687593in}}%
\pgfpathcurveto{\pgfqpoint{4.088220in}{1.687593in}}{\pgfqpoint{4.095061in}{1.690427in}}{\pgfqpoint{4.100105in}{1.695470in}}%
\pgfpathcurveto{\pgfqpoint{4.105149in}{1.700514in}}{\pgfqpoint{4.107983in}{1.707356in}}{\pgfqpoint{4.107983in}{1.714488in}}%
\pgfpathcurveto{\pgfqpoint{4.107983in}{1.721621in}}{\pgfqpoint{4.105149in}{1.728463in}}{\pgfqpoint{4.100105in}{1.733507in}}%
\pgfpathcurveto{\pgfqpoint{4.095061in}{1.738550in}}{\pgfqpoint{4.088220in}{1.741384in}}{\pgfqpoint{4.081087in}{1.741384in}}%
\pgfpathcurveto{\pgfqpoint{4.073954in}{1.741384in}}{\pgfqpoint{4.067112in}{1.738550in}}{\pgfqpoint{4.062069in}{1.733507in}}%
\pgfpathcurveto{\pgfqpoint{4.057025in}{1.728463in}}{\pgfqpoint{4.054191in}{1.721621in}}{\pgfqpoint{4.054191in}{1.714488in}}%
\pgfpathcurveto{\pgfqpoint{4.054191in}{1.707356in}}{\pgfqpoint{4.057025in}{1.700514in}}{\pgfqpoint{4.062069in}{1.695470in}}%
\pgfpathcurveto{\pgfqpoint{4.067112in}{1.690427in}}{\pgfqpoint{4.073954in}{1.687593in}}{\pgfqpoint{4.081087in}{1.687593in}}%
\pgfpathclose%
\pgfusepath{stroke,fill}%
\end{pgfscope}%
\begin{pgfscope}%
\pgfpathrectangle{\pgfqpoint{2.867647in}{0.500000in}}{\pgfqpoint{1.764706in}{1.700000in}}%
\pgfusepath{clip}%
\pgfsetbuttcap%
\pgfsetroundjoin%
\definecolor{currentfill}{rgb}{0.964433,0.670254,0.515093}%
\pgfsetfillcolor{currentfill}%
\pgfsetlinewidth{0.311001pt}%
\definecolor{currentstroke}{rgb}{1.000000,1.000000,1.000000}%
\pgfsetstrokecolor{currentstroke}%
\pgfsetdash{}{0pt}%
\pgfpathmoveto{\pgfqpoint{3.966591in}{1.689523in}}%
\pgfpathcurveto{\pgfqpoint{3.973724in}{1.689523in}}{\pgfqpoint{3.980566in}{1.692357in}}{\pgfqpoint{3.985609in}{1.697400in}}%
\pgfpathcurveto{\pgfqpoint{3.990653in}{1.702444in}}{\pgfqpoint{3.993487in}{1.709286in}}{\pgfqpoint{3.993487in}{1.716419in}}%
\pgfpathcurveto{\pgfqpoint{3.993487in}{1.723551in}}{\pgfqpoint{3.990653in}{1.730393in}}{\pgfqpoint{3.985609in}{1.735437in}}%
\pgfpathcurveto{\pgfqpoint{3.980566in}{1.740480in}}{\pgfqpoint{3.973724in}{1.743314in}}{\pgfqpoint{3.966591in}{1.743314in}}%
\pgfpathcurveto{\pgfqpoint{3.959458in}{1.743314in}}{\pgfqpoint{3.952617in}{1.740480in}}{\pgfqpoint{3.947573in}{1.735437in}}%
\pgfpathcurveto{\pgfqpoint{3.942529in}{1.730393in}}{\pgfqpoint{3.939695in}{1.723551in}}{\pgfqpoint{3.939695in}{1.716419in}}%
\pgfpathcurveto{\pgfqpoint{3.939695in}{1.709286in}}{\pgfqpoint{3.942529in}{1.702444in}}{\pgfqpoint{3.947573in}{1.697400in}}%
\pgfpathcurveto{\pgfqpoint{3.952617in}{1.692357in}}{\pgfqpoint{3.959458in}{1.689523in}}{\pgfqpoint{3.966591in}{1.689523in}}%
\pgfpathclose%
\pgfusepath{stroke,fill}%
\end{pgfscope}%
\begin{pgfscope}%
\pgfpathrectangle{\pgfqpoint{2.867647in}{0.500000in}}{\pgfqpoint{1.764706in}{1.700000in}}%
\pgfusepath{clip}%
\pgfsetbuttcap%
\pgfsetroundjoin%
\definecolor{currentfill}{rgb}{0.970255,0.815666,0.711203}%
\pgfsetfillcolor{currentfill}%
\pgfsetlinewidth{0.311001pt}%
\definecolor{currentstroke}{rgb}{1.000000,1.000000,1.000000}%
\pgfsetstrokecolor{currentstroke}%
\pgfsetdash{}{0pt}%
\pgfpathmoveto{\pgfqpoint{4.270431in}{1.373709in}}%
\pgfpathcurveto{\pgfqpoint{4.277564in}{1.373709in}}{\pgfqpoint{4.284406in}{1.376543in}}{\pgfqpoint{4.289449in}{1.381586in}}%
\pgfpathcurveto{\pgfqpoint{4.294493in}{1.386630in}}{\pgfqpoint{4.297327in}{1.393472in}}{\pgfqpoint{4.297327in}{1.400605in}}%
\pgfpathcurveto{\pgfqpoint{4.297327in}{1.407737in}}{\pgfqpoint{4.294493in}{1.414579in}}{\pgfqpoint{4.289449in}{1.419623in}}%
\pgfpathcurveto{\pgfqpoint{4.284406in}{1.424666in}}{\pgfqpoint{4.277564in}{1.427500in}}{\pgfqpoint{4.270431in}{1.427500in}}%
\pgfpathcurveto{\pgfqpoint{4.263298in}{1.427500in}}{\pgfqpoint{4.256457in}{1.424666in}}{\pgfqpoint{4.251413in}{1.419623in}}%
\pgfpathcurveto{\pgfqpoint{4.246369in}{1.414579in}}{\pgfqpoint{4.243535in}{1.407737in}}{\pgfqpoint{4.243535in}{1.400605in}}%
\pgfpathcurveto{\pgfqpoint{4.243535in}{1.393472in}}{\pgfqpoint{4.246369in}{1.386630in}}{\pgfqpoint{4.251413in}{1.381586in}}%
\pgfpathcurveto{\pgfqpoint{4.256457in}{1.376543in}}{\pgfqpoint{4.263298in}{1.373709in}}{\pgfqpoint{4.270431in}{1.373709in}}%
\pgfpathclose%
\pgfusepath{stroke,fill}%
\end{pgfscope}%
\begin{pgfscope}%
\pgfpathrectangle{\pgfqpoint{2.867647in}{0.500000in}}{\pgfqpoint{1.764706in}{1.700000in}}%
\pgfusepath{clip}%
\pgfsetbuttcap%
\pgfsetroundjoin%
\definecolor{currentfill}{rgb}{0.965302,0.713942,0.568499}%
\pgfsetfillcolor{currentfill}%
\pgfsetlinewidth{0.311001pt}%
\definecolor{currentstroke}{rgb}{1.000000,1.000000,1.000000}%
\pgfsetstrokecolor{currentstroke}%
\pgfsetdash{}{0pt}%
\pgfpathmoveto{\pgfqpoint{4.002663in}{1.729696in}}%
\pgfpathcurveto{\pgfqpoint{4.009796in}{1.729696in}}{\pgfqpoint{4.016637in}{1.732530in}}{\pgfqpoint{4.021681in}{1.737574in}}%
\pgfpathcurveto{\pgfqpoint{4.026725in}{1.742617in}}{\pgfqpoint{4.029559in}{1.749459in}}{\pgfqpoint{4.029559in}{1.756592in}}%
\pgfpathcurveto{\pgfqpoint{4.029559in}{1.763725in}}{\pgfqpoint{4.026725in}{1.770566in}}{\pgfqpoint{4.021681in}{1.775610in}}%
\pgfpathcurveto{\pgfqpoint{4.016637in}{1.780654in}}{\pgfqpoint{4.009796in}{1.783488in}}{\pgfqpoint{4.002663in}{1.783488in}}%
\pgfpathcurveto{\pgfqpoint{3.995530in}{1.783488in}}{\pgfqpoint{3.988689in}{1.780654in}}{\pgfqpoint{3.983645in}{1.775610in}}%
\pgfpathcurveto{\pgfqpoint{3.978601in}{1.770566in}}{\pgfqpoint{3.975767in}{1.763725in}}{\pgfqpoint{3.975767in}{1.756592in}}%
\pgfpathcurveto{\pgfqpoint{3.975767in}{1.749459in}}{\pgfqpoint{3.978601in}{1.742617in}}{\pgfqpoint{3.983645in}{1.737574in}}%
\pgfpathcurveto{\pgfqpoint{3.988689in}{1.732530in}}{\pgfqpoint{3.995530in}{1.729696in}}{\pgfqpoint{4.002663in}{1.729696in}}%
\pgfpathclose%
\pgfusepath{stroke,fill}%
\end{pgfscope}%
\begin{pgfscope}%
\pgfpathrectangle{\pgfqpoint{2.867647in}{0.500000in}}{\pgfqpoint{1.764706in}{1.700000in}}%
\pgfusepath{clip}%
\pgfsetbuttcap%
\pgfsetroundjoin%
\definecolor{currentfill}{rgb}{0.981377,0.920617,0.865369}%
\pgfsetfillcolor{currentfill}%
\pgfsetlinewidth{0.311001pt}%
\definecolor{currentstroke}{rgb}{1.000000,1.000000,1.000000}%
\pgfsetstrokecolor{currentstroke}%
\pgfsetdash{}{0pt}%
\pgfpathmoveto{\pgfqpoint{4.193320in}{1.285589in}}%
\pgfpathcurveto{\pgfqpoint{4.200453in}{1.285589in}}{\pgfqpoint{4.207295in}{1.288423in}}{\pgfqpoint{4.212338in}{1.293467in}}%
\pgfpathcurveto{\pgfqpoint{4.217382in}{1.298510in}}{\pgfqpoint{4.220216in}{1.305352in}}{\pgfqpoint{4.220216in}{1.312485in}}%
\pgfpathcurveto{\pgfqpoint{4.220216in}{1.319618in}}{\pgfqpoint{4.217382in}{1.326459in}}{\pgfqpoint{4.212338in}{1.331503in}}%
\pgfpathcurveto{\pgfqpoint{4.207295in}{1.336547in}}{\pgfqpoint{4.200453in}{1.339380in}}{\pgfqpoint{4.193320in}{1.339380in}}%
\pgfpathcurveto{\pgfqpoint{4.186187in}{1.339380in}}{\pgfqpoint{4.179346in}{1.336547in}}{\pgfqpoint{4.174302in}{1.331503in}}%
\pgfpathcurveto{\pgfqpoint{4.169258in}{1.326459in}}{\pgfqpoint{4.166425in}{1.319618in}}{\pgfqpoint{4.166425in}{1.312485in}}%
\pgfpathcurveto{\pgfqpoint{4.166425in}{1.305352in}}{\pgfqpoint{4.169258in}{1.298510in}}{\pgfqpoint{4.174302in}{1.293467in}}%
\pgfpathcurveto{\pgfqpoint{4.179346in}{1.288423in}}{\pgfqpoint{4.186187in}{1.285589in}}{\pgfqpoint{4.193320in}{1.285589in}}%
\pgfpathclose%
\pgfusepath{stroke,fill}%
\end{pgfscope}%
\begin{pgfscope}%
\pgfpathrectangle{\pgfqpoint{2.867647in}{0.500000in}}{\pgfqpoint{1.764706in}{1.700000in}}%
\pgfusepath{clip}%
\pgfsetbuttcap%
\pgfsetroundjoin%
\definecolor{currentfill}{rgb}{0.978376,0.897317,0.831308}%
\pgfsetfillcolor{currentfill}%
\pgfsetlinewidth{0.311001pt}%
\definecolor{currentstroke}{rgb}{1.000000,1.000000,1.000000}%
\pgfsetstrokecolor{currentstroke}%
\pgfsetdash{}{0pt}%
\pgfpathmoveto{\pgfqpoint{4.219442in}{1.195936in}}%
\pgfpathcurveto{\pgfqpoint{4.226575in}{1.195936in}}{\pgfqpoint{4.233417in}{1.198770in}}{\pgfqpoint{4.238460in}{1.203813in}}%
\pgfpathcurveto{\pgfqpoint{4.243504in}{1.208857in}}{\pgfqpoint{4.246338in}{1.215699in}}{\pgfqpoint{4.246338in}{1.222832in}}%
\pgfpathcurveto{\pgfqpoint{4.246338in}{1.229964in}}{\pgfqpoint{4.243504in}{1.236806in}}{\pgfqpoint{4.238460in}{1.241850in}}%
\pgfpathcurveto{\pgfqpoint{4.233417in}{1.246893in}}{\pgfqpoint{4.226575in}{1.249727in}}{\pgfqpoint{4.219442in}{1.249727in}}%
\pgfpathcurveto{\pgfqpoint{4.212309in}{1.249727in}}{\pgfqpoint{4.205468in}{1.246893in}}{\pgfqpoint{4.200424in}{1.241850in}}%
\pgfpathcurveto{\pgfqpoint{4.195380in}{1.236806in}}{\pgfqpoint{4.192547in}{1.229964in}}{\pgfqpoint{4.192547in}{1.222832in}}%
\pgfpathcurveto{\pgfqpoint{4.192547in}{1.215699in}}{\pgfqpoint{4.195380in}{1.208857in}}{\pgfqpoint{4.200424in}{1.203813in}}%
\pgfpathcurveto{\pgfqpoint{4.205468in}{1.198770in}}{\pgfqpoint{4.212309in}{1.195936in}}{\pgfqpoint{4.219442in}{1.195936in}}%
\pgfpathclose%
\pgfusepath{stroke,fill}%
\end{pgfscope}%
\begin{pgfscope}%
\pgfpathrectangle{\pgfqpoint{2.867647in}{0.500000in}}{\pgfqpoint{1.764706in}{1.700000in}}%
\pgfusepath{clip}%
\pgfsetbuttcap%
\pgfsetroundjoin%
\definecolor{currentfill}{rgb}{0.975018,0.868213,0.788710}%
\pgfsetfillcolor{currentfill}%
\pgfsetlinewidth{0.311001pt}%
\definecolor{currentstroke}{rgb}{1.000000,1.000000,1.000000}%
\pgfsetstrokecolor{currentstroke}%
\pgfsetdash{}{0pt}%
\pgfpathmoveto{\pgfqpoint{4.122787in}{1.222289in}}%
\pgfpathcurveto{\pgfqpoint{4.129920in}{1.222289in}}{\pgfqpoint{4.136761in}{1.225123in}}{\pgfqpoint{4.141805in}{1.230167in}}%
\pgfpathcurveto{\pgfqpoint{4.146849in}{1.235210in}}{\pgfqpoint{4.149683in}{1.242052in}}{\pgfqpoint{4.149683in}{1.249185in}}%
\pgfpathcurveto{\pgfqpoint{4.149683in}{1.256318in}}{\pgfqpoint{4.146849in}{1.263159in}}{\pgfqpoint{4.141805in}{1.268203in}}%
\pgfpathcurveto{\pgfqpoint{4.136761in}{1.273247in}}{\pgfqpoint{4.129920in}{1.276081in}}{\pgfqpoint{4.122787in}{1.276081in}}%
\pgfpathcurveto{\pgfqpoint{4.115654in}{1.276081in}}{\pgfqpoint{4.108812in}{1.273247in}}{\pgfqpoint{4.103769in}{1.268203in}}%
\pgfpathcurveto{\pgfqpoint{4.098725in}{1.263159in}}{\pgfqpoint{4.095891in}{1.256318in}}{\pgfqpoint{4.095891in}{1.249185in}}%
\pgfpathcurveto{\pgfqpoint{4.095891in}{1.242052in}}{\pgfqpoint{4.098725in}{1.235210in}}{\pgfqpoint{4.103769in}{1.230167in}}%
\pgfpathcurveto{\pgfqpoint{4.108812in}{1.225123in}}{\pgfqpoint{4.115654in}{1.222289in}}{\pgfqpoint{4.122787in}{1.222289in}}%
\pgfpathclose%
\pgfusepath{stroke,fill}%
\end{pgfscope}%
\begin{pgfscope}%
\pgfpathrectangle{\pgfqpoint{2.867647in}{0.500000in}}{\pgfqpoint{1.764706in}{1.700000in}}%
\pgfusepath{clip}%
\pgfsetbuttcap%
\pgfsetroundjoin%
\definecolor{currentfill}{rgb}{0.980678,0.914765,0.856766}%
\pgfsetfillcolor{currentfill}%
\pgfsetlinewidth{0.311001pt}%
\definecolor{currentstroke}{rgb}{1.000000,1.000000,1.000000}%
\pgfsetstrokecolor{currentstroke}%
\pgfsetdash{}{0pt}%
\pgfpathmoveto{\pgfqpoint{4.158854in}{1.484211in}}%
\pgfpathcurveto{\pgfqpoint{4.165987in}{1.484211in}}{\pgfqpoint{4.172829in}{1.487045in}}{\pgfqpoint{4.177872in}{1.492089in}}%
\pgfpathcurveto{\pgfqpoint{4.182916in}{1.497132in}}{\pgfqpoint{4.185750in}{1.503974in}}{\pgfqpoint{4.185750in}{1.511107in}}%
\pgfpathcurveto{\pgfqpoint{4.185750in}{1.518240in}}{\pgfqpoint{4.182916in}{1.525081in}}{\pgfqpoint{4.177872in}{1.530125in}}%
\pgfpathcurveto{\pgfqpoint{4.172829in}{1.535169in}}{\pgfqpoint{4.165987in}{1.538003in}}{\pgfqpoint{4.158854in}{1.538003in}}%
\pgfpathcurveto{\pgfqpoint{4.151721in}{1.538003in}}{\pgfqpoint{4.144880in}{1.535169in}}{\pgfqpoint{4.139836in}{1.530125in}}%
\pgfpathcurveto{\pgfqpoint{4.134792in}{1.525081in}}{\pgfqpoint{4.131958in}{1.518240in}}{\pgfqpoint{4.131958in}{1.511107in}}%
\pgfpathcurveto{\pgfqpoint{4.131958in}{1.503974in}}{\pgfqpoint{4.134792in}{1.497132in}}{\pgfqpoint{4.139836in}{1.492089in}}%
\pgfpathcurveto{\pgfqpoint{4.144880in}{1.487045in}}{\pgfqpoint{4.151721in}{1.484211in}}{\pgfqpoint{4.158854in}{1.484211in}}%
\pgfpathclose%
\pgfusepath{stroke,fill}%
\end{pgfscope}%
\begin{pgfscope}%
\pgfpathrectangle{\pgfqpoint{2.867647in}{0.500000in}}{\pgfqpoint{1.764706in}{1.700000in}}%
\pgfusepath{clip}%
\pgfsetbuttcap%
\pgfsetroundjoin%
\definecolor{currentfill}{rgb}{0.981377,0.920617,0.865369}%
\pgfsetfillcolor{currentfill}%
\pgfsetlinewidth{0.311001pt}%
\definecolor{currentstroke}{rgb}{1.000000,1.000000,1.000000}%
\pgfsetstrokecolor{currentstroke}%
\pgfsetdash{}{0pt}%
\pgfpathmoveto{\pgfqpoint{4.194106in}{1.207905in}}%
\pgfpathcurveto{\pgfqpoint{4.201239in}{1.207905in}}{\pgfqpoint{4.208080in}{1.210739in}}{\pgfqpoint{4.213124in}{1.215782in}}%
\pgfpathcurveto{\pgfqpoint{4.218168in}{1.220826in}}{\pgfqpoint{4.221001in}{1.227668in}}{\pgfqpoint{4.221001in}{1.234801in}}%
\pgfpathcurveto{\pgfqpoint{4.221001in}{1.241933in}}{\pgfqpoint{4.218168in}{1.248775in}}{\pgfqpoint{4.213124in}{1.253819in}}%
\pgfpathcurveto{\pgfqpoint{4.208080in}{1.258862in}}{\pgfqpoint{4.201239in}{1.261696in}}{\pgfqpoint{4.194106in}{1.261696in}}%
\pgfpathcurveto{\pgfqpoint{4.186973in}{1.261696in}}{\pgfqpoint{4.180131in}{1.258862in}}{\pgfqpoint{4.175088in}{1.253819in}}%
\pgfpathcurveto{\pgfqpoint{4.170044in}{1.248775in}}{\pgfqpoint{4.167210in}{1.241933in}}{\pgfqpoint{4.167210in}{1.234801in}}%
\pgfpathcurveto{\pgfqpoint{4.167210in}{1.227668in}}{\pgfqpoint{4.170044in}{1.220826in}}{\pgfqpoint{4.175088in}{1.215782in}}%
\pgfpathcurveto{\pgfqpoint{4.180131in}{1.210739in}}{\pgfqpoint{4.186973in}{1.207905in}}{\pgfqpoint{4.194106in}{1.207905in}}%
\pgfpathclose%
\pgfusepath{stroke,fill}%
\end{pgfscope}%
\begin{pgfscope}%
\pgfpathrectangle{\pgfqpoint{2.867647in}{0.500000in}}{\pgfqpoint{1.764706in}{1.700000in}}%
\pgfusepath{clip}%
\pgfsetbuttcap%
\pgfsetroundjoin%
\definecolor{currentfill}{rgb}{0.969359,0.803954,0.693832}%
\pgfsetfillcolor{currentfill}%
\pgfsetlinewidth{0.311001pt}%
\definecolor{currentstroke}{rgb}{1.000000,1.000000,1.000000}%
\pgfsetstrokecolor{currentstroke}%
\pgfsetdash{}{0pt}%
\pgfpathmoveto{\pgfqpoint{4.073034in}{1.480478in}}%
\pgfpathcurveto{\pgfqpoint{4.080167in}{1.480478in}}{\pgfqpoint{4.087009in}{1.483312in}}{\pgfqpoint{4.092052in}{1.488356in}}%
\pgfpathcurveto{\pgfqpoint{4.097096in}{1.493400in}}{\pgfqpoint{4.099930in}{1.500241in}}{\pgfqpoint{4.099930in}{1.507374in}}%
\pgfpathcurveto{\pgfqpoint{4.099930in}{1.514507in}}{\pgfqpoint{4.097096in}{1.521348in}}{\pgfqpoint{4.092052in}{1.526392in}}%
\pgfpathcurveto{\pgfqpoint{4.087009in}{1.531436in}}{\pgfqpoint{4.080167in}{1.534270in}}{\pgfqpoint{4.073034in}{1.534270in}}%
\pgfpathcurveto{\pgfqpoint{4.065901in}{1.534270in}}{\pgfqpoint{4.059060in}{1.531436in}}{\pgfqpoint{4.054016in}{1.526392in}}%
\pgfpathcurveto{\pgfqpoint{4.048972in}{1.521348in}}{\pgfqpoint{4.046138in}{1.514507in}}{\pgfqpoint{4.046138in}{1.507374in}}%
\pgfpathcurveto{\pgfqpoint{4.046138in}{1.500241in}}{\pgfqpoint{4.048972in}{1.493400in}}{\pgfqpoint{4.054016in}{1.488356in}}%
\pgfpathcurveto{\pgfqpoint{4.059060in}{1.483312in}}{\pgfqpoint{4.065901in}{1.480478in}}{\pgfqpoint{4.073034in}{1.480478in}}%
\pgfpathclose%
\pgfusepath{stroke,fill}%
\end{pgfscope}%
\begin{pgfscope}%
\pgfpathrectangle{\pgfqpoint{2.867647in}{0.500000in}}{\pgfqpoint{1.764706in}{1.700000in}}%
\pgfusepath{clip}%
\pgfsetbuttcap%
\pgfsetroundjoin%
\definecolor{currentfill}{rgb}{0.975644,0.874038,0.797253}%
\pgfsetfillcolor{currentfill}%
\pgfsetlinewidth{0.311001pt}%
\definecolor{currentstroke}{rgb}{1.000000,1.000000,1.000000}%
\pgfsetstrokecolor{currentstroke}%
\pgfsetdash{}{0pt}%
\pgfpathmoveto{\pgfqpoint{4.235152in}{1.414171in}}%
\pgfpathcurveto{\pgfqpoint{4.242285in}{1.414171in}}{\pgfqpoint{4.249126in}{1.417005in}}{\pgfqpoint{4.254170in}{1.422048in}}%
\pgfpathcurveto{\pgfqpoint{4.259214in}{1.427092in}}{\pgfqpoint{4.262048in}{1.433934in}}{\pgfqpoint{4.262048in}{1.441066in}}%
\pgfpathcurveto{\pgfqpoint{4.262048in}{1.448199in}}{\pgfqpoint{4.259214in}{1.455041in}}{\pgfqpoint{4.254170in}{1.460085in}}%
\pgfpathcurveto{\pgfqpoint{4.249126in}{1.465128in}}{\pgfqpoint{4.242285in}{1.467962in}}{\pgfqpoint{4.235152in}{1.467962in}}%
\pgfpathcurveto{\pgfqpoint{4.228019in}{1.467962in}}{\pgfqpoint{4.221177in}{1.465128in}}{\pgfqpoint{4.216134in}{1.460085in}}%
\pgfpathcurveto{\pgfqpoint{4.211090in}{1.455041in}}{\pgfqpoint{4.208256in}{1.448199in}}{\pgfqpoint{4.208256in}{1.441066in}}%
\pgfpathcurveto{\pgfqpoint{4.208256in}{1.433934in}}{\pgfqpoint{4.211090in}{1.427092in}}{\pgfqpoint{4.216134in}{1.422048in}}%
\pgfpathcurveto{\pgfqpoint{4.221177in}{1.417005in}}{\pgfqpoint{4.228019in}{1.414171in}}{\pgfqpoint{4.235152in}{1.414171in}}%
\pgfpathclose%
\pgfusepath{stroke,fill}%
\end{pgfscope}%
\begin{pgfscope}%
\pgfpathrectangle{\pgfqpoint{2.867647in}{0.500000in}}{\pgfqpoint{1.764706in}{1.700000in}}%
\pgfusepath{clip}%
\pgfsetbuttcap%
\pgfsetroundjoin%
\definecolor{currentfill}{rgb}{0.972726,0.844889,0.754401}%
\pgfsetfillcolor{currentfill}%
\pgfsetlinewidth{0.311001pt}%
\definecolor{currentstroke}{rgb}{1.000000,1.000000,1.000000}%
\pgfsetstrokecolor{currentstroke}%
\pgfsetdash{}{0pt}%
\pgfpathmoveto{\pgfqpoint{4.170747in}{1.019465in}}%
\pgfpathcurveto{\pgfqpoint{4.177880in}{1.019465in}}{\pgfqpoint{4.184721in}{1.022299in}}{\pgfqpoint{4.189765in}{1.027343in}}%
\pgfpathcurveto{\pgfqpoint{4.194809in}{1.032386in}}{\pgfqpoint{4.197643in}{1.039228in}}{\pgfqpoint{4.197643in}{1.046361in}}%
\pgfpathcurveto{\pgfqpoint{4.197643in}{1.053494in}}{\pgfqpoint{4.194809in}{1.060335in}}{\pgfqpoint{4.189765in}{1.065379in}}%
\pgfpathcurveto{\pgfqpoint{4.184721in}{1.070423in}}{\pgfqpoint{4.177880in}{1.073257in}}{\pgfqpoint{4.170747in}{1.073257in}}%
\pgfpathcurveto{\pgfqpoint{4.163614in}{1.073257in}}{\pgfqpoint{4.156772in}{1.070423in}}{\pgfqpoint{4.151729in}{1.065379in}}%
\pgfpathcurveto{\pgfqpoint{4.146685in}{1.060335in}}{\pgfqpoint{4.143851in}{1.053494in}}{\pgfqpoint{4.143851in}{1.046361in}}%
\pgfpathcurveto{\pgfqpoint{4.143851in}{1.039228in}}{\pgfqpoint{4.146685in}{1.032386in}}{\pgfqpoint{4.151729in}{1.027343in}}%
\pgfpathcurveto{\pgfqpoint{4.156772in}{1.022299in}}{\pgfqpoint{4.163614in}{1.019465in}}{\pgfqpoint{4.170747in}{1.019465in}}%
\pgfpathclose%
\pgfusepath{stroke,fill}%
\end{pgfscope}%
\begin{pgfscope}%
\pgfpathrectangle{\pgfqpoint{2.867647in}{0.500000in}}{\pgfqpoint{1.764706in}{1.700000in}}%
\pgfusepath{clip}%
\pgfsetbuttcap%
\pgfsetroundjoin%
\definecolor{currentfill}{rgb}{0.966120,0.744512,0.608720}%
\pgfsetfillcolor{currentfill}%
\pgfsetlinewidth{0.311001pt}%
\definecolor{currentstroke}{rgb}{1.000000,1.000000,1.000000}%
\pgfsetstrokecolor{currentstroke}%
\pgfsetdash{}{0pt}%
\pgfpathmoveto{\pgfqpoint{4.263488in}{1.103176in}}%
\pgfpathcurveto{\pgfqpoint{4.270621in}{1.103176in}}{\pgfqpoint{4.277463in}{1.106010in}}{\pgfqpoint{4.282507in}{1.111053in}}%
\pgfpathcurveto{\pgfqpoint{4.287550in}{1.116097in}}{\pgfqpoint{4.290384in}{1.122939in}}{\pgfqpoint{4.290384in}{1.130071in}}%
\pgfpathcurveto{\pgfqpoint{4.290384in}{1.137204in}}{\pgfqpoint{4.287550in}{1.144046in}}{\pgfqpoint{4.282507in}{1.149090in}}%
\pgfpathcurveto{\pgfqpoint{4.277463in}{1.154133in}}{\pgfqpoint{4.270621in}{1.156967in}}{\pgfqpoint{4.263488in}{1.156967in}}%
\pgfpathcurveto{\pgfqpoint{4.256356in}{1.156967in}}{\pgfqpoint{4.249514in}{1.154133in}}{\pgfqpoint{4.244470in}{1.149090in}}%
\pgfpathcurveto{\pgfqpoint{4.239427in}{1.144046in}}{\pgfqpoint{4.236593in}{1.137204in}}{\pgfqpoint{4.236593in}{1.130071in}}%
\pgfpathcurveto{\pgfqpoint{4.236593in}{1.122939in}}{\pgfqpoint{4.239427in}{1.116097in}}{\pgfqpoint{4.244470in}{1.111053in}}%
\pgfpathcurveto{\pgfqpoint{4.249514in}{1.106010in}}{\pgfqpoint{4.256356in}{1.103176in}}{\pgfqpoint{4.263488in}{1.103176in}}%
\pgfpathclose%
\pgfusepath{stroke,fill}%
\end{pgfscope}%
\begin{pgfscope}%
\pgfpathrectangle{\pgfqpoint{2.867647in}{0.500000in}}{\pgfqpoint{1.764706in}{1.700000in}}%
\pgfusepath{clip}%
\pgfsetbuttcap%
\pgfsetroundjoin%
\definecolor{currentfill}{rgb}{0.967735,0.780441,0.659127}%
\pgfsetfillcolor{currentfill}%
\pgfsetlinewidth{0.311001pt}%
\definecolor{currentstroke}{rgb}{1.000000,1.000000,1.000000}%
\pgfsetstrokecolor{currentstroke}%
\pgfsetdash{}{0pt}%
\pgfpathmoveto{\pgfqpoint{4.031140in}{1.597240in}}%
\pgfpathcurveto{\pgfqpoint{4.038273in}{1.597240in}}{\pgfqpoint{4.045114in}{1.600074in}}{\pgfqpoint{4.050158in}{1.605117in}}%
\pgfpathcurveto{\pgfqpoint{4.055201in}{1.610161in}}{\pgfqpoint{4.058035in}{1.617003in}}{\pgfqpoint{4.058035in}{1.624136in}}%
\pgfpathcurveto{\pgfqpoint{4.058035in}{1.631268in}}{\pgfqpoint{4.055201in}{1.638110in}}{\pgfqpoint{4.050158in}{1.643154in}}%
\pgfpathcurveto{\pgfqpoint{4.045114in}{1.648197in}}{\pgfqpoint{4.038273in}{1.651031in}}{\pgfqpoint{4.031140in}{1.651031in}}%
\pgfpathcurveto{\pgfqpoint{4.024007in}{1.651031in}}{\pgfqpoint{4.017165in}{1.648197in}}{\pgfqpoint{4.012122in}{1.643154in}}%
\pgfpathcurveto{\pgfqpoint{4.007078in}{1.638110in}}{\pgfqpoint{4.004244in}{1.631268in}}{\pgfqpoint{4.004244in}{1.624136in}}%
\pgfpathcurveto{\pgfqpoint{4.004244in}{1.617003in}}{\pgfqpoint{4.007078in}{1.610161in}}{\pgfqpoint{4.012122in}{1.605117in}}%
\pgfpathcurveto{\pgfqpoint{4.017165in}{1.600074in}}{\pgfqpoint{4.024007in}{1.597240in}}{\pgfqpoint{4.031140in}{1.597240in}}%
\pgfpathclose%
\pgfusepath{stroke,fill}%
\end{pgfscope}%
\begin{pgfscope}%
\pgfpathrectangle{\pgfqpoint{2.867647in}{0.500000in}}{\pgfqpoint{1.764706in}{1.700000in}}%
\pgfusepath{clip}%
\pgfsetbuttcap%
\pgfsetroundjoin%
\definecolor{currentfill}{rgb}{0.979891,0.908948,0.848279}%
\pgfsetfillcolor{currentfill}%
\pgfsetlinewidth{0.311001pt}%
\definecolor{currentstroke}{rgb}{1.000000,1.000000,1.000000}%
\pgfsetstrokecolor{currentstroke}%
\pgfsetdash{}{0pt}%
\pgfpathmoveto{\pgfqpoint{4.139478in}{1.565280in}}%
\pgfpathcurveto{\pgfqpoint{4.146611in}{1.565280in}}{\pgfqpoint{4.153453in}{1.568114in}}{\pgfqpoint{4.158497in}{1.573158in}}%
\pgfpathcurveto{\pgfqpoint{4.163540in}{1.578201in}}{\pgfqpoint{4.166374in}{1.585043in}}{\pgfqpoint{4.166374in}{1.592176in}}%
\pgfpathcurveto{\pgfqpoint{4.166374in}{1.599309in}}{\pgfqpoint{4.163540in}{1.606150in}}{\pgfqpoint{4.158497in}{1.611194in}}%
\pgfpathcurveto{\pgfqpoint{4.153453in}{1.616238in}}{\pgfqpoint{4.146611in}{1.619072in}}{\pgfqpoint{4.139478in}{1.619072in}}%
\pgfpathcurveto{\pgfqpoint{4.132346in}{1.619072in}}{\pgfqpoint{4.125504in}{1.616238in}}{\pgfqpoint{4.120460in}{1.611194in}}%
\pgfpathcurveto{\pgfqpoint{4.115417in}{1.606150in}}{\pgfqpoint{4.112583in}{1.599309in}}{\pgfqpoint{4.112583in}{1.592176in}}%
\pgfpathcurveto{\pgfqpoint{4.112583in}{1.585043in}}{\pgfqpoint{4.115417in}{1.578201in}}{\pgfqpoint{4.120460in}{1.573158in}}%
\pgfpathcurveto{\pgfqpoint{4.125504in}{1.568114in}}{\pgfqpoint{4.132346in}{1.565280in}}{\pgfqpoint{4.139478in}{1.565280in}}%
\pgfpathclose%
\pgfusepath{stroke,fill}%
\end{pgfscope}%
\begin{pgfscope}%
\pgfpathrectangle{\pgfqpoint{2.867647in}{0.500000in}}{\pgfqpoint{1.764706in}{1.700000in}}%
\pgfusepath{clip}%
\pgfsetbuttcap%
\pgfsetroundjoin%
\definecolor{currentfill}{rgb}{0.980678,0.914765,0.856766}%
\pgfsetfillcolor{currentfill}%
\pgfsetlinewidth{0.311001pt}%
\definecolor{currentstroke}{rgb}{1.000000,1.000000,1.000000}%
\pgfsetstrokecolor{currentstroke}%
\pgfsetdash{}{0pt}%
\pgfpathmoveto{\pgfqpoint{4.162452in}{1.254082in}}%
\pgfpathcurveto{\pgfqpoint{4.169585in}{1.254082in}}{\pgfqpoint{4.176427in}{1.256916in}}{\pgfqpoint{4.181470in}{1.261959in}}%
\pgfpathcurveto{\pgfqpoint{4.186514in}{1.267003in}}{\pgfqpoint{4.189348in}{1.273845in}}{\pgfqpoint{4.189348in}{1.280977in}}%
\pgfpathcurveto{\pgfqpoint{4.189348in}{1.288110in}}{\pgfqpoint{4.186514in}{1.294952in}}{\pgfqpoint{4.181470in}{1.299996in}}%
\pgfpathcurveto{\pgfqpoint{4.176427in}{1.305039in}}{\pgfqpoint{4.169585in}{1.307873in}}{\pgfqpoint{4.162452in}{1.307873in}}%
\pgfpathcurveto{\pgfqpoint{4.155319in}{1.307873in}}{\pgfqpoint{4.148478in}{1.305039in}}{\pgfqpoint{4.143434in}{1.299996in}}%
\pgfpathcurveto{\pgfqpoint{4.138391in}{1.294952in}}{\pgfqpoint{4.135557in}{1.288110in}}{\pgfqpoint{4.135557in}{1.280977in}}%
\pgfpathcurveto{\pgfqpoint{4.135557in}{1.273845in}}{\pgfqpoint{4.138391in}{1.267003in}}{\pgfqpoint{4.143434in}{1.261959in}}%
\pgfpathcurveto{\pgfqpoint{4.148478in}{1.256916in}}{\pgfqpoint{4.155319in}{1.254082in}}{\pgfqpoint{4.162452in}{1.254082in}}%
\pgfpathclose%
\pgfusepath{stroke,fill}%
\end{pgfscope}%
\begin{pgfscope}%
\pgfpathrectangle{\pgfqpoint{2.867647in}{0.500000in}}{\pgfqpoint{1.764706in}{1.700000in}}%
\pgfusepath{clip}%
\pgfsetbuttcap%
\pgfsetroundjoin%
\definecolor{currentfill}{rgb}{0.980678,0.914765,0.856766}%
\pgfsetfillcolor{currentfill}%
\pgfsetlinewidth{0.311001pt}%
\definecolor{currentstroke}{rgb}{1.000000,1.000000,1.000000}%
\pgfsetstrokecolor{currentstroke}%
\pgfsetdash{}{0pt}%
\pgfpathmoveto{\pgfqpoint{4.200794in}{1.347812in}}%
\pgfpathcurveto{\pgfqpoint{4.207927in}{1.347812in}}{\pgfqpoint{4.214769in}{1.350646in}}{\pgfqpoint{4.219812in}{1.355689in}}%
\pgfpathcurveto{\pgfqpoint{4.224856in}{1.360733in}}{\pgfqpoint{4.227690in}{1.367575in}}{\pgfqpoint{4.227690in}{1.374707in}}%
\pgfpathcurveto{\pgfqpoint{4.227690in}{1.381840in}}{\pgfqpoint{4.224856in}{1.388682in}}{\pgfqpoint{4.219812in}{1.393726in}}%
\pgfpathcurveto{\pgfqpoint{4.214769in}{1.398769in}}{\pgfqpoint{4.207927in}{1.401603in}}{\pgfqpoint{4.200794in}{1.401603in}}%
\pgfpathcurveto{\pgfqpoint{4.193661in}{1.401603in}}{\pgfqpoint{4.186820in}{1.398769in}}{\pgfqpoint{4.181776in}{1.393726in}}%
\pgfpathcurveto{\pgfqpoint{4.176732in}{1.388682in}}{\pgfqpoint{4.173898in}{1.381840in}}{\pgfqpoint{4.173898in}{1.374707in}}%
\pgfpathcurveto{\pgfqpoint{4.173898in}{1.367575in}}{\pgfqpoint{4.176732in}{1.360733in}}{\pgfqpoint{4.181776in}{1.355689in}}%
\pgfpathcurveto{\pgfqpoint{4.186820in}{1.350646in}}{\pgfqpoint{4.193661in}{1.347812in}}{\pgfqpoint{4.200794in}{1.347812in}}%
\pgfpathclose%
\pgfusepath{stroke,fill}%
\end{pgfscope}%
\begin{pgfscope}%
\pgfpathrectangle{\pgfqpoint{2.867647in}{0.500000in}}{\pgfqpoint{1.764706in}{1.700000in}}%
\pgfusepath{clip}%
\pgfsetbuttcap%
\pgfsetroundjoin%
\definecolor{currentfill}{rgb}{0.978376,0.897317,0.831308}%
\pgfsetfillcolor{currentfill}%
\pgfsetlinewidth{0.311001pt}%
\definecolor{currentstroke}{rgb}{1.000000,1.000000,1.000000}%
\pgfsetstrokecolor{currentstroke}%
\pgfsetdash{}{0pt}%
\pgfpathmoveto{\pgfqpoint{4.133620in}{1.068295in}}%
\pgfpathcurveto{\pgfqpoint{4.140753in}{1.068295in}}{\pgfqpoint{4.147594in}{1.071129in}}{\pgfqpoint{4.152638in}{1.076172in}}%
\pgfpathcurveto{\pgfqpoint{4.157682in}{1.081216in}}{\pgfqpoint{4.160515in}{1.088058in}}{\pgfqpoint{4.160515in}{1.095190in}}%
\pgfpathcurveto{\pgfqpoint{4.160515in}{1.102323in}}{\pgfqpoint{4.157682in}{1.109165in}}{\pgfqpoint{4.152638in}{1.114208in}}%
\pgfpathcurveto{\pgfqpoint{4.147594in}{1.119252in}}{\pgfqpoint{4.140753in}{1.122086in}}{\pgfqpoint{4.133620in}{1.122086in}}%
\pgfpathcurveto{\pgfqpoint{4.126487in}{1.122086in}}{\pgfqpoint{4.119645in}{1.119252in}}{\pgfqpoint{4.114602in}{1.114208in}}%
\pgfpathcurveto{\pgfqpoint{4.109558in}{1.109165in}}{\pgfqpoint{4.106724in}{1.102323in}}{\pgfqpoint{4.106724in}{1.095190in}}%
\pgfpathcurveto{\pgfqpoint{4.106724in}{1.088058in}}{\pgfqpoint{4.109558in}{1.081216in}}{\pgfqpoint{4.114602in}{1.076172in}}%
\pgfpathcurveto{\pgfqpoint{4.119645in}{1.071129in}}{\pgfqpoint{4.126487in}{1.068295in}}{\pgfqpoint{4.133620in}{1.068295in}}%
\pgfpathclose%
\pgfusepath{stroke,fill}%
\end{pgfscope}%
\begin{pgfscope}%
\pgfpathrectangle{\pgfqpoint{2.867647in}{0.500000in}}{\pgfqpoint{1.764706in}{1.700000in}}%
\pgfusepath{clip}%
\pgfsetbuttcap%
\pgfsetroundjoin%
\definecolor{currentfill}{rgb}{0.979124,0.903132,0.839793}%
\pgfsetfillcolor{currentfill}%
\pgfsetlinewidth{0.311001pt}%
\definecolor{currentstroke}{rgb}{1.000000,1.000000,1.000000}%
\pgfsetstrokecolor{currentstroke}%
\pgfsetdash{}{0pt}%
\pgfpathmoveto{\pgfqpoint{4.160281in}{1.348185in}}%
\pgfpathcurveto{\pgfqpoint{4.167413in}{1.348185in}}{\pgfqpoint{4.174255in}{1.351019in}}{\pgfqpoint{4.179299in}{1.356063in}}%
\pgfpathcurveto{\pgfqpoint{4.184342in}{1.361106in}}{\pgfqpoint{4.187176in}{1.367948in}}{\pgfqpoint{4.187176in}{1.375081in}}%
\pgfpathcurveto{\pgfqpoint{4.187176in}{1.382214in}}{\pgfqpoint{4.184342in}{1.389055in}}{\pgfqpoint{4.179299in}{1.394099in}}%
\pgfpathcurveto{\pgfqpoint{4.174255in}{1.399143in}}{\pgfqpoint{4.167413in}{1.401976in}}{\pgfqpoint{4.160281in}{1.401976in}}%
\pgfpathcurveto{\pgfqpoint{4.153148in}{1.401976in}}{\pgfqpoint{4.146306in}{1.399143in}}{\pgfqpoint{4.141262in}{1.394099in}}%
\pgfpathcurveto{\pgfqpoint{4.136219in}{1.389055in}}{\pgfqpoint{4.133385in}{1.382214in}}{\pgfqpoint{4.133385in}{1.375081in}}%
\pgfpathcurveto{\pgfqpoint{4.133385in}{1.367948in}}{\pgfqpoint{4.136219in}{1.361106in}}{\pgfqpoint{4.141262in}{1.356063in}}%
\pgfpathcurveto{\pgfqpoint{4.146306in}{1.351019in}}{\pgfqpoint{4.153148in}{1.348185in}}{\pgfqpoint{4.160281in}{1.348185in}}%
\pgfpathclose%
\pgfusepath{stroke,fill}%
\end{pgfscope}%
\begin{pgfscope}%
\pgfpathrectangle{\pgfqpoint{2.867647in}{0.500000in}}{\pgfqpoint{1.764706in}{1.700000in}}%
\pgfusepath{clip}%
\pgfsetbuttcap%
\pgfsetroundjoin%
\definecolor{currentfill}{rgb}{0.974412,0.862387,0.780156}%
\pgfsetfillcolor{currentfill}%
\pgfsetlinewidth{0.311001pt}%
\definecolor{currentstroke}{rgb}{1.000000,1.000000,1.000000}%
\pgfsetstrokecolor{currentstroke}%
\pgfsetdash{}{0pt}%
\pgfpathmoveto{\pgfqpoint{4.226959in}{1.496034in}}%
\pgfpathcurveto{\pgfqpoint{4.234092in}{1.496034in}}{\pgfqpoint{4.240933in}{1.498868in}}{\pgfqpoint{4.245977in}{1.503912in}}%
\pgfpathcurveto{\pgfqpoint{4.251021in}{1.508956in}}{\pgfqpoint{4.253854in}{1.515797in}}{\pgfqpoint{4.253854in}{1.522930in}}%
\pgfpathcurveto{\pgfqpoint{4.253854in}{1.530063in}}{\pgfqpoint{4.251021in}{1.536904in}}{\pgfqpoint{4.245977in}{1.541948in}}%
\pgfpathcurveto{\pgfqpoint{4.240933in}{1.546992in}}{\pgfqpoint{4.234092in}{1.549826in}}{\pgfqpoint{4.226959in}{1.549826in}}%
\pgfpathcurveto{\pgfqpoint{4.219826in}{1.549826in}}{\pgfqpoint{4.212984in}{1.546992in}}{\pgfqpoint{4.207941in}{1.541948in}}%
\pgfpathcurveto{\pgfqpoint{4.202897in}{1.536904in}}{\pgfqpoint{4.200063in}{1.530063in}}{\pgfqpoint{4.200063in}{1.522930in}}%
\pgfpathcurveto{\pgfqpoint{4.200063in}{1.515797in}}{\pgfqpoint{4.202897in}{1.508956in}}{\pgfqpoint{4.207941in}{1.503912in}}%
\pgfpathcurveto{\pgfqpoint{4.212984in}{1.498868in}}{\pgfqpoint{4.219826in}{1.496034in}}{\pgfqpoint{4.226959in}{1.496034in}}%
\pgfpathclose%
\pgfusepath{stroke,fill}%
\end{pgfscope}%
\begin{pgfscope}%
\pgfpathrectangle{\pgfqpoint{2.867647in}{0.500000in}}{\pgfqpoint{1.764706in}{1.700000in}}%
\pgfusepath{clip}%
\pgfsetbuttcap%
\pgfsetroundjoin%
\definecolor{currentfill}{rgb}{0.972201,0.839051,0.745789}%
\pgfsetfillcolor{currentfill}%
\pgfsetlinewidth{0.311001pt}%
\definecolor{currentstroke}{rgb}{1.000000,1.000000,1.000000}%
\pgfsetstrokecolor{currentstroke}%
\pgfsetdash{}{0pt}%
\pgfpathmoveto{\pgfqpoint{4.152680in}{1.647580in}}%
\pgfpathcurveto{\pgfqpoint{4.159813in}{1.647580in}}{\pgfqpoint{4.166655in}{1.650414in}}{\pgfqpoint{4.171698in}{1.655457in}}%
\pgfpathcurveto{\pgfqpoint{4.176742in}{1.660501in}}{\pgfqpoint{4.179576in}{1.667343in}}{\pgfqpoint{4.179576in}{1.674475in}}%
\pgfpathcurveto{\pgfqpoint{4.179576in}{1.681608in}}{\pgfqpoint{4.176742in}{1.688450in}}{\pgfqpoint{4.171698in}{1.693493in}}%
\pgfpathcurveto{\pgfqpoint{4.166655in}{1.698537in}}{\pgfqpoint{4.159813in}{1.701371in}}{\pgfqpoint{4.152680in}{1.701371in}}%
\pgfpathcurveto{\pgfqpoint{4.145547in}{1.701371in}}{\pgfqpoint{4.138706in}{1.698537in}}{\pgfqpoint{4.133662in}{1.693493in}}%
\pgfpathcurveto{\pgfqpoint{4.128618in}{1.688450in}}{\pgfqpoint{4.125784in}{1.681608in}}{\pgfqpoint{4.125784in}{1.674475in}}%
\pgfpathcurveto{\pgfqpoint{4.125784in}{1.667343in}}{\pgfqpoint{4.128618in}{1.660501in}}{\pgfqpoint{4.133662in}{1.655457in}}%
\pgfpathcurveto{\pgfqpoint{4.138706in}{1.650414in}}{\pgfqpoint{4.145547in}{1.647580in}}{\pgfqpoint{4.152680in}{1.647580in}}%
\pgfpathclose%
\pgfusepath{stroke,fill}%
\end{pgfscope}%
\begin{pgfscope}%
\pgfpathrectangle{\pgfqpoint{2.867647in}{0.500000in}}{\pgfqpoint{1.764706in}{1.700000in}}%
\pgfusepath{clip}%
\pgfsetbuttcap%
\pgfsetroundjoin%
\definecolor{currentfill}{rgb}{0.979124,0.903132,0.839793}%
\pgfsetfillcolor{currentfill}%
\pgfsetlinewidth{0.311001pt}%
\definecolor{currentstroke}{rgb}{1.000000,1.000000,1.000000}%
\pgfsetstrokecolor{currentstroke}%
\pgfsetdash{}{0pt}%
\pgfpathmoveto{\pgfqpoint{4.150679in}{1.268130in}}%
\pgfpathcurveto{\pgfqpoint{4.157812in}{1.268130in}}{\pgfqpoint{4.164654in}{1.270964in}}{\pgfqpoint{4.169697in}{1.276008in}}%
\pgfpathcurveto{\pgfqpoint{4.174741in}{1.281051in}}{\pgfqpoint{4.177575in}{1.287893in}}{\pgfqpoint{4.177575in}{1.295026in}}%
\pgfpathcurveto{\pgfqpoint{4.177575in}{1.302159in}}{\pgfqpoint{4.174741in}{1.309000in}}{\pgfqpoint{4.169697in}{1.314044in}}%
\pgfpathcurveto{\pgfqpoint{4.164654in}{1.319088in}}{\pgfqpoint{4.157812in}{1.321921in}}{\pgfqpoint{4.150679in}{1.321921in}}%
\pgfpathcurveto{\pgfqpoint{4.143546in}{1.321921in}}{\pgfqpoint{4.136705in}{1.319088in}}{\pgfqpoint{4.131661in}{1.314044in}}%
\pgfpathcurveto{\pgfqpoint{4.126617in}{1.309000in}}{\pgfqpoint{4.123784in}{1.302159in}}{\pgfqpoint{4.123784in}{1.295026in}}%
\pgfpathcurveto{\pgfqpoint{4.123784in}{1.287893in}}{\pgfqpoint{4.126617in}{1.281051in}}{\pgfqpoint{4.131661in}{1.276008in}}%
\pgfpathcurveto{\pgfqpoint{4.136705in}{1.270964in}}{\pgfqpoint{4.143546in}{1.268130in}}{\pgfqpoint{4.150679in}{1.268130in}}%
\pgfpathclose%
\pgfusepath{stroke,fill}%
\end{pgfscope}%
\begin{pgfscope}%
\pgfpathrectangle{\pgfqpoint{2.867647in}{0.500000in}}{\pgfqpoint{1.764706in}{1.700000in}}%
\pgfusepath{clip}%
\pgfsetbuttcap%
\pgfsetroundjoin%
\definecolor{currentfill}{rgb}{0.981377,0.920617,0.865369}%
\pgfsetfillcolor{currentfill}%
\pgfsetlinewidth{0.311001pt}%
\definecolor{currentstroke}{rgb}{1.000000,1.000000,1.000000}%
\pgfsetstrokecolor{currentstroke}%
\pgfsetdash{}{0pt}%
\pgfpathmoveto{\pgfqpoint{4.191142in}{1.203787in}}%
\pgfpathcurveto{\pgfqpoint{4.198275in}{1.203787in}}{\pgfqpoint{4.205116in}{1.206620in}}{\pgfqpoint{4.210160in}{1.211664in}}%
\pgfpathcurveto{\pgfqpoint{4.215204in}{1.216708in}}{\pgfqpoint{4.218038in}{1.223549in}}{\pgfqpoint{4.218038in}{1.230682in}}%
\pgfpathcurveto{\pgfqpoint{4.218038in}{1.237815in}}{\pgfqpoint{4.215204in}{1.244657in}}{\pgfqpoint{4.210160in}{1.249700in}}%
\pgfpathcurveto{\pgfqpoint{4.205116in}{1.254744in}}{\pgfqpoint{4.198275in}{1.257578in}}{\pgfqpoint{4.191142in}{1.257578in}}%
\pgfpathcurveto{\pgfqpoint{4.184009in}{1.257578in}}{\pgfqpoint{4.177167in}{1.254744in}}{\pgfqpoint{4.172124in}{1.249700in}}%
\pgfpathcurveto{\pgfqpoint{4.167080in}{1.244657in}}{\pgfqpoint{4.164246in}{1.237815in}}{\pgfqpoint{4.164246in}{1.230682in}}%
\pgfpathcurveto{\pgfqpoint{4.164246in}{1.223549in}}{\pgfqpoint{4.167080in}{1.216708in}}{\pgfqpoint{4.172124in}{1.211664in}}%
\pgfpathcurveto{\pgfqpoint{4.177167in}{1.206620in}}{\pgfqpoint{4.184009in}{1.203787in}}{\pgfqpoint{4.191142in}{1.203787in}}%
\pgfpathclose%
\pgfusepath{stroke,fill}%
\end{pgfscope}%
\begin{pgfscope}%
\pgfpathrectangle{\pgfqpoint{2.867647in}{0.500000in}}{\pgfqpoint{1.764706in}{1.700000in}}%
\pgfusepath{clip}%
\pgfsetbuttcap%
\pgfsetroundjoin%
\definecolor{currentfill}{rgb}{0.972201,0.839051,0.745789}%
\pgfsetfillcolor{currentfill}%
\pgfsetlinewidth{0.311001pt}%
\definecolor{currentstroke}{rgb}{1.000000,1.000000,1.000000}%
\pgfsetstrokecolor{currentstroke}%
\pgfsetdash{}{0pt}%
\pgfpathmoveto{\pgfqpoint{4.243684in}{1.152588in}}%
\pgfpathcurveto{\pgfqpoint{4.250817in}{1.152588in}}{\pgfqpoint{4.257659in}{1.155422in}}{\pgfqpoint{4.262702in}{1.160466in}}%
\pgfpathcurveto{\pgfqpoint{4.267746in}{1.165509in}}{\pgfqpoint{4.270580in}{1.172351in}}{\pgfqpoint{4.270580in}{1.179484in}}%
\pgfpathcurveto{\pgfqpoint{4.270580in}{1.186617in}}{\pgfqpoint{4.267746in}{1.193458in}}{\pgfqpoint{4.262702in}{1.198502in}}%
\pgfpathcurveto{\pgfqpoint{4.257659in}{1.203546in}}{\pgfqpoint{4.250817in}{1.206380in}}{\pgfqpoint{4.243684in}{1.206380in}}%
\pgfpathcurveto{\pgfqpoint{4.236551in}{1.206380in}}{\pgfqpoint{4.229710in}{1.203546in}}{\pgfqpoint{4.224666in}{1.198502in}}%
\pgfpathcurveto{\pgfqpoint{4.219622in}{1.193458in}}{\pgfqpoint{4.216789in}{1.186617in}}{\pgfqpoint{4.216789in}{1.179484in}}%
\pgfpathcurveto{\pgfqpoint{4.216789in}{1.172351in}}{\pgfqpoint{4.219622in}{1.165509in}}{\pgfqpoint{4.224666in}{1.160466in}}%
\pgfpathcurveto{\pgfqpoint{4.229710in}{1.155422in}}{\pgfqpoint{4.236551in}{1.152588in}}{\pgfqpoint{4.243684in}{1.152588in}}%
\pgfpathclose%
\pgfusepath{stroke,fill}%
\end{pgfscope}%
\begin{pgfscope}%
\pgfpathrectangle{\pgfqpoint{2.867647in}{0.500000in}}{\pgfqpoint{1.764706in}{1.700000in}}%
\pgfusepath{clip}%
\pgfsetbuttcap%
\pgfsetroundjoin%
\definecolor{currentfill}{rgb}{0.972201,0.839051,0.745789}%
\pgfsetfillcolor{currentfill}%
\pgfsetlinewidth{0.311001pt}%
\definecolor{currentstroke}{rgb}{1.000000,1.000000,1.000000}%
\pgfsetstrokecolor{currentstroke}%
\pgfsetdash{}{0pt}%
\pgfpathmoveto{\pgfqpoint{4.065748in}{0.994879in}}%
\pgfpathcurveto{\pgfqpoint{4.072881in}{0.994879in}}{\pgfqpoint{4.079723in}{0.997713in}}{\pgfqpoint{4.084766in}{1.002757in}}%
\pgfpathcurveto{\pgfqpoint{4.089810in}{1.007800in}}{\pgfqpoint{4.092644in}{1.014642in}}{\pgfqpoint{4.092644in}{1.021775in}}%
\pgfpathcurveto{\pgfqpoint{4.092644in}{1.028908in}}{\pgfqpoint{4.089810in}{1.035749in}}{\pgfqpoint{4.084766in}{1.040793in}}%
\pgfpathcurveto{\pgfqpoint{4.079723in}{1.045837in}}{\pgfqpoint{4.072881in}{1.048671in}}{\pgfqpoint{4.065748in}{1.048671in}}%
\pgfpathcurveto{\pgfqpoint{4.058615in}{1.048671in}}{\pgfqpoint{4.051774in}{1.045837in}}{\pgfqpoint{4.046730in}{1.040793in}}%
\pgfpathcurveto{\pgfqpoint{4.041686in}{1.035749in}}{\pgfqpoint{4.038852in}{1.028908in}}{\pgfqpoint{4.038852in}{1.021775in}}%
\pgfpathcurveto{\pgfqpoint{4.038852in}{1.014642in}}{\pgfqpoint{4.041686in}{1.007800in}}{\pgfqpoint{4.046730in}{1.002757in}}%
\pgfpathcurveto{\pgfqpoint{4.051774in}{0.997713in}}{\pgfqpoint{4.058615in}{0.994879in}}{\pgfqpoint{4.065748in}{0.994879in}}%
\pgfpathclose%
\pgfusepath{stroke,fill}%
\end{pgfscope}%
\begin{pgfscope}%
\pgfpathrectangle{\pgfqpoint{2.867647in}{0.500000in}}{\pgfqpoint{1.764706in}{1.700000in}}%
\pgfusepath{clip}%
\pgfsetbuttcap%
\pgfsetroundjoin%
\definecolor{currentfill}{rgb}{0.934351,0.329284,0.247753}%
\pgfsetfillcolor{currentfill}%
\pgfsetlinewidth{0.311001pt}%
\definecolor{currentstroke}{rgb}{1.000000,1.000000,1.000000}%
\pgfsetstrokecolor{currentstroke}%
\pgfsetdash{}{0pt}%
\pgfpathmoveto{\pgfqpoint{4.361951in}{1.173055in}}%
\pgfpathcurveto{\pgfqpoint{4.369084in}{1.173055in}}{\pgfqpoint{4.375926in}{1.175889in}}{\pgfqpoint{4.380970in}{1.180932in}}%
\pgfpathcurveto{\pgfqpoint{4.386013in}{1.185976in}}{\pgfqpoint{4.388847in}{1.192818in}}{\pgfqpoint{4.388847in}{1.199951in}}%
\pgfpathcurveto{\pgfqpoint{4.388847in}{1.207083in}}{\pgfqpoint{4.386013in}{1.213925in}}{\pgfqpoint{4.380970in}{1.218969in}}%
\pgfpathcurveto{\pgfqpoint{4.375926in}{1.224012in}}{\pgfqpoint{4.369084in}{1.226846in}}{\pgfqpoint{4.361951in}{1.226846in}}%
\pgfpathcurveto{\pgfqpoint{4.354819in}{1.226846in}}{\pgfqpoint{4.347977in}{1.224012in}}{\pgfqpoint{4.342933in}{1.218969in}}%
\pgfpathcurveto{\pgfqpoint{4.337890in}{1.213925in}}{\pgfqpoint{4.335056in}{1.207083in}}{\pgfqpoint{4.335056in}{1.199951in}}%
\pgfpathcurveto{\pgfqpoint{4.335056in}{1.192818in}}{\pgfqpoint{4.337890in}{1.185976in}}{\pgfqpoint{4.342933in}{1.180932in}}%
\pgfpathcurveto{\pgfqpoint{4.347977in}{1.175889in}}{\pgfqpoint{4.354819in}{1.173055in}}{\pgfqpoint{4.361951in}{1.173055in}}%
\pgfpathclose%
\pgfusepath{stroke,fill}%
\end{pgfscope}%
\begin{pgfscope}%
\pgfpathrectangle{\pgfqpoint{2.867647in}{0.500000in}}{\pgfqpoint{1.764706in}{1.700000in}}%
\pgfusepath{clip}%
\pgfsetbuttcap%
\pgfsetroundjoin%
\definecolor{currentfill}{rgb}{0.978376,0.897317,0.831308}%
\pgfsetfillcolor{currentfill}%
\pgfsetlinewidth{0.311001pt}%
\definecolor{currentstroke}{rgb}{1.000000,1.000000,1.000000}%
\pgfsetstrokecolor{currentstroke}%
\pgfsetdash{}{0pt}%
\pgfpathmoveto{\pgfqpoint{4.148557in}{1.388635in}}%
\pgfpathcurveto{\pgfqpoint{4.155690in}{1.388635in}}{\pgfqpoint{4.162532in}{1.391468in}}{\pgfqpoint{4.167575in}{1.396512in}}%
\pgfpathcurveto{\pgfqpoint{4.172619in}{1.401556in}}{\pgfqpoint{4.175453in}{1.408397in}}{\pgfqpoint{4.175453in}{1.415530in}}%
\pgfpathcurveto{\pgfqpoint{4.175453in}{1.422663in}}{\pgfqpoint{4.172619in}{1.429505in}}{\pgfqpoint{4.167575in}{1.434548in}}%
\pgfpathcurveto{\pgfqpoint{4.162532in}{1.439592in}}{\pgfqpoint{4.155690in}{1.442426in}}{\pgfqpoint{4.148557in}{1.442426in}}%
\pgfpathcurveto{\pgfqpoint{4.141424in}{1.442426in}}{\pgfqpoint{4.134583in}{1.439592in}}{\pgfqpoint{4.129539in}{1.434548in}}%
\pgfpathcurveto{\pgfqpoint{4.124495in}{1.429505in}}{\pgfqpoint{4.121662in}{1.422663in}}{\pgfqpoint{4.121662in}{1.415530in}}%
\pgfpathcurveto{\pgfqpoint{4.121662in}{1.408397in}}{\pgfqpoint{4.124495in}{1.401556in}}{\pgfqpoint{4.129539in}{1.396512in}}%
\pgfpathcurveto{\pgfqpoint{4.134583in}{1.391468in}}{\pgfqpoint{4.141424in}{1.388635in}}{\pgfqpoint{4.148557in}{1.388635in}}%
\pgfpathclose%
\pgfusepath{stroke,fill}%
\end{pgfscope}%
\begin{pgfscope}%
\pgfpathrectangle{\pgfqpoint{2.867647in}{0.500000in}}{\pgfqpoint{1.764706in}{1.700000in}}%
\pgfusepath{clip}%
\pgfsetbuttcap%
\pgfsetroundjoin%
\definecolor{currentfill}{rgb}{0.979891,0.908948,0.848279}%
\pgfsetfillcolor{currentfill}%
\pgfsetlinewidth{0.311001pt}%
\definecolor{currentstroke}{rgb}{1.000000,1.000000,1.000000}%
\pgfsetstrokecolor{currentstroke}%
\pgfsetdash{}{0pt}%
\pgfpathmoveto{\pgfqpoint{4.196995in}{1.440356in}}%
\pgfpathcurveto{\pgfqpoint{4.204128in}{1.440356in}}{\pgfqpoint{4.210969in}{1.443190in}}{\pgfqpoint{4.216013in}{1.448234in}}%
\pgfpathcurveto{\pgfqpoint{4.221057in}{1.453278in}}{\pgfqpoint{4.223890in}{1.460119in}}{\pgfqpoint{4.223890in}{1.467252in}}%
\pgfpathcurveto{\pgfqpoint{4.223890in}{1.474385in}}{\pgfqpoint{4.221057in}{1.481227in}}{\pgfqpoint{4.216013in}{1.486270in}}%
\pgfpathcurveto{\pgfqpoint{4.210969in}{1.491314in}}{\pgfqpoint{4.204128in}{1.494148in}}{\pgfqpoint{4.196995in}{1.494148in}}%
\pgfpathcurveto{\pgfqpoint{4.189862in}{1.494148in}}{\pgfqpoint{4.183020in}{1.491314in}}{\pgfqpoint{4.177977in}{1.486270in}}%
\pgfpathcurveto{\pgfqpoint{4.172933in}{1.481227in}}{\pgfqpoint{4.170099in}{1.474385in}}{\pgfqpoint{4.170099in}{1.467252in}}%
\pgfpathcurveto{\pgfqpoint{4.170099in}{1.460119in}}{\pgfqpoint{4.172933in}{1.453278in}}{\pgfqpoint{4.177977in}{1.448234in}}%
\pgfpathcurveto{\pgfqpoint{4.183020in}{1.443190in}}{\pgfqpoint{4.189862in}{1.440356in}}{\pgfqpoint{4.196995in}{1.440356in}}%
\pgfpathclose%
\pgfusepath{stroke,fill}%
\end{pgfscope}%
\begin{pgfscope}%
\pgfpathrectangle{\pgfqpoint{2.867647in}{0.500000in}}{\pgfqpoint{1.764706in}{1.700000in}}%
\pgfusepath{clip}%
\pgfsetbuttcap%
\pgfsetroundjoin%
\definecolor{currentfill}{rgb}{0.977657,0.891500,0.822809}%
\pgfsetfillcolor{currentfill}%
\pgfsetlinewidth{0.311001pt}%
\definecolor{currentstroke}{rgb}{1.000000,1.000000,1.000000}%
\pgfsetstrokecolor{currentstroke}%
\pgfsetdash{}{0pt}%
\pgfpathmoveto{\pgfqpoint{4.114962in}{1.504170in}}%
\pgfpathcurveto{\pgfqpoint{4.122095in}{1.504170in}}{\pgfqpoint{4.128936in}{1.507004in}}{\pgfqpoint{4.133980in}{1.512047in}}%
\pgfpathcurveto{\pgfqpoint{4.139024in}{1.517091in}}{\pgfqpoint{4.141857in}{1.523933in}}{\pgfqpoint{4.141857in}{1.531066in}}%
\pgfpathcurveto{\pgfqpoint{4.141857in}{1.538198in}}{\pgfqpoint{4.139024in}{1.545040in}}{\pgfqpoint{4.133980in}{1.550084in}}%
\pgfpathcurveto{\pgfqpoint{4.128936in}{1.555127in}}{\pgfqpoint{4.122095in}{1.557961in}}{\pgfqpoint{4.114962in}{1.557961in}}%
\pgfpathcurveto{\pgfqpoint{4.107829in}{1.557961in}}{\pgfqpoint{4.100987in}{1.555127in}}{\pgfqpoint{4.095944in}{1.550084in}}%
\pgfpathcurveto{\pgfqpoint{4.090900in}{1.545040in}}{\pgfqpoint{4.088066in}{1.538198in}}{\pgfqpoint{4.088066in}{1.531066in}}%
\pgfpathcurveto{\pgfqpoint{4.088066in}{1.523933in}}{\pgfqpoint{4.090900in}{1.517091in}}{\pgfqpoint{4.095944in}{1.512047in}}%
\pgfpathcurveto{\pgfqpoint{4.100987in}{1.507004in}}{\pgfqpoint{4.107829in}{1.504170in}}{\pgfqpoint{4.114962in}{1.504170in}}%
\pgfpathclose%
\pgfusepath{stroke,fill}%
\end{pgfscope}%
\begin{pgfscope}%
\pgfpathrectangle{\pgfqpoint{2.867647in}{0.500000in}}{\pgfqpoint{1.764706in}{1.700000in}}%
\pgfusepath{clip}%
\pgfsetbuttcap%
\pgfsetroundjoin%
\definecolor{currentfill}{rgb}{0.962532,0.599594,0.438051}%
\pgfsetfillcolor{currentfill}%
\pgfsetlinewidth{0.311001pt}%
\definecolor{currentstroke}{rgb}{1.000000,1.000000,1.000000}%
\pgfsetstrokecolor{currentstroke}%
\pgfsetdash{}{0pt}%
\pgfpathmoveto{\pgfqpoint{4.318573in}{1.453392in}}%
\pgfpathcurveto{\pgfqpoint{4.325706in}{1.453392in}}{\pgfqpoint{4.332547in}{1.456226in}}{\pgfqpoint{4.337591in}{1.461269in}}%
\pgfpathcurveto{\pgfqpoint{4.342635in}{1.466313in}}{\pgfqpoint{4.345468in}{1.473155in}}{\pgfqpoint{4.345468in}{1.480287in}}%
\pgfpathcurveto{\pgfqpoint{4.345468in}{1.487420in}}{\pgfqpoint{4.342635in}{1.494262in}}{\pgfqpoint{4.337591in}{1.499306in}}%
\pgfpathcurveto{\pgfqpoint{4.332547in}{1.504349in}}{\pgfqpoint{4.325706in}{1.507183in}}{\pgfqpoint{4.318573in}{1.507183in}}%
\pgfpathcurveto{\pgfqpoint{4.311440in}{1.507183in}}{\pgfqpoint{4.304598in}{1.504349in}}{\pgfqpoint{4.299555in}{1.499306in}}%
\pgfpathcurveto{\pgfqpoint{4.294511in}{1.494262in}}{\pgfqpoint{4.291677in}{1.487420in}}{\pgfqpoint{4.291677in}{1.480287in}}%
\pgfpathcurveto{\pgfqpoint{4.291677in}{1.473155in}}{\pgfqpoint{4.294511in}{1.466313in}}{\pgfqpoint{4.299555in}{1.461269in}}%
\pgfpathcurveto{\pgfqpoint{4.304598in}{1.456226in}}{\pgfqpoint{4.311440in}{1.453392in}}{\pgfqpoint{4.318573in}{1.453392in}}%
\pgfpathclose%
\pgfusepath{stroke,fill}%
\end{pgfscope}%
\begin{pgfscope}%
\pgfpathrectangle{\pgfqpoint{2.867647in}{0.500000in}}{\pgfqpoint{1.764706in}{1.700000in}}%
\pgfusepath{clip}%
\pgfsetbuttcap%
\pgfsetroundjoin%
\definecolor{currentfill}{rgb}{0.973832,0.856556,0.771584}%
\pgfsetfillcolor{currentfill}%
\pgfsetlinewidth{0.311001pt}%
\definecolor{currentstroke}{rgb}{1.000000,1.000000,1.000000}%
\pgfsetstrokecolor{currentstroke}%
\pgfsetdash{}{0pt}%
\pgfpathmoveto{\pgfqpoint{4.111516in}{1.206918in}}%
\pgfpathcurveto{\pgfqpoint{4.118649in}{1.206918in}}{\pgfqpoint{4.125491in}{1.209752in}}{\pgfqpoint{4.130535in}{1.214796in}}%
\pgfpathcurveto{\pgfqpoint{4.135578in}{1.219839in}}{\pgfqpoint{4.138412in}{1.226681in}}{\pgfqpoint{4.138412in}{1.233814in}}%
\pgfpathcurveto{\pgfqpoint{4.138412in}{1.240947in}}{\pgfqpoint{4.135578in}{1.247788in}}{\pgfqpoint{4.130535in}{1.252832in}}%
\pgfpathcurveto{\pgfqpoint{4.125491in}{1.257876in}}{\pgfqpoint{4.118649in}{1.260709in}}{\pgfqpoint{4.111516in}{1.260709in}}%
\pgfpathcurveto{\pgfqpoint{4.104384in}{1.260709in}}{\pgfqpoint{4.097542in}{1.257876in}}{\pgfqpoint{4.092498in}{1.252832in}}%
\pgfpathcurveto{\pgfqpoint{4.087455in}{1.247788in}}{\pgfqpoint{4.084621in}{1.240947in}}{\pgfqpoint{4.084621in}{1.233814in}}%
\pgfpathcurveto{\pgfqpoint{4.084621in}{1.226681in}}{\pgfqpoint{4.087455in}{1.219839in}}{\pgfqpoint{4.092498in}{1.214796in}}%
\pgfpathcurveto{\pgfqpoint{4.097542in}{1.209752in}}{\pgfqpoint{4.104384in}{1.206918in}}{\pgfqpoint{4.111516in}{1.206918in}}%
\pgfpathclose%
\pgfusepath{stroke,fill}%
\end{pgfscope}%
\begin{pgfscope}%
\pgfpathrectangle{\pgfqpoint{2.867647in}{0.500000in}}{\pgfqpoint{1.764706in}{1.700000in}}%
\pgfusepath{clip}%
\pgfsetbuttcap%
\pgfsetroundjoin%
\definecolor{currentfill}{rgb}{0.966328,0.750560,0.616961}%
\pgfsetfillcolor{currentfill}%
\pgfsetlinewidth{0.311001pt}%
\definecolor{currentstroke}{rgb}{1.000000,1.000000,1.000000}%
\pgfsetstrokecolor{currentstroke}%
\pgfsetdash{}{0pt}%
\pgfpathmoveto{\pgfqpoint{4.069878in}{1.436804in}}%
\pgfpathcurveto{\pgfqpoint{4.077011in}{1.436804in}}{\pgfqpoint{4.083853in}{1.439638in}}{\pgfqpoint{4.088896in}{1.444682in}}%
\pgfpathcurveto{\pgfqpoint{4.093940in}{1.449725in}}{\pgfqpoint{4.096774in}{1.456567in}}{\pgfqpoint{4.096774in}{1.463700in}}%
\pgfpathcurveto{\pgfqpoint{4.096774in}{1.470833in}}{\pgfqpoint{4.093940in}{1.477674in}}{\pgfqpoint{4.088896in}{1.482718in}}%
\pgfpathcurveto{\pgfqpoint{4.083853in}{1.487762in}}{\pgfqpoint{4.077011in}{1.490595in}}{\pgfqpoint{4.069878in}{1.490595in}}%
\pgfpathcurveto{\pgfqpoint{4.062745in}{1.490595in}}{\pgfqpoint{4.055904in}{1.487762in}}{\pgfqpoint{4.050860in}{1.482718in}}%
\pgfpathcurveto{\pgfqpoint{4.045816in}{1.477674in}}{\pgfqpoint{4.042983in}{1.470833in}}{\pgfqpoint{4.042983in}{1.463700in}}%
\pgfpathcurveto{\pgfqpoint{4.042983in}{1.456567in}}{\pgfqpoint{4.045816in}{1.449725in}}{\pgfqpoint{4.050860in}{1.444682in}}%
\pgfpathcurveto{\pgfqpoint{4.055904in}{1.439638in}}{\pgfqpoint{4.062745in}{1.436804in}}{\pgfqpoint{4.069878in}{1.436804in}}%
\pgfpathclose%
\pgfusepath{stroke,fill}%
\end{pgfscope}%
\begin{pgfscope}%
\pgfpathrectangle{\pgfqpoint{2.867647in}{0.500000in}}{\pgfqpoint{1.764706in}{1.700000in}}%
\pgfusepath{clip}%
\pgfsetbuttcap%
\pgfsetroundjoin%
\definecolor{currentfill}{rgb}{0.973832,0.856556,0.771584}%
\pgfsetfillcolor{currentfill}%
\pgfsetlinewidth{0.311001pt}%
\definecolor{currentstroke}{rgb}{1.000000,1.000000,1.000000}%
\pgfsetstrokecolor{currentstroke}%
\pgfsetdash{}{0pt}%
\pgfpathmoveto{\pgfqpoint{4.081728in}{1.531038in}}%
\pgfpathcurveto{\pgfqpoint{4.088861in}{1.531038in}}{\pgfqpoint{4.095702in}{1.533872in}}{\pgfqpoint{4.100746in}{1.538915in}}%
\pgfpathcurveto{\pgfqpoint{4.105790in}{1.543959in}}{\pgfqpoint{4.108624in}{1.550801in}}{\pgfqpoint{4.108624in}{1.557934in}}%
\pgfpathcurveto{\pgfqpoint{4.108624in}{1.565066in}}{\pgfqpoint{4.105790in}{1.571908in}}{\pgfqpoint{4.100746in}{1.576952in}}%
\pgfpathcurveto{\pgfqpoint{4.095702in}{1.581995in}}{\pgfqpoint{4.088861in}{1.584829in}}{\pgfqpoint{4.081728in}{1.584829in}}%
\pgfpathcurveto{\pgfqpoint{4.074595in}{1.584829in}}{\pgfqpoint{4.067753in}{1.581995in}}{\pgfqpoint{4.062710in}{1.576952in}}%
\pgfpathcurveto{\pgfqpoint{4.057666in}{1.571908in}}{\pgfqpoint{4.054832in}{1.565066in}}{\pgfqpoint{4.054832in}{1.557934in}}%
\pgfpathcurveto{\pgfqpoint{4.054832in}{1.550801in}}{\pgfqpoint{4.057666in}{1.543959in}}{\pgfqpoint{4.062710in}{1.538915in}}%
\pgfpathcurveto{\pgfqpoint{4.067753in}{1.533872in}}{\pgfqpoint{4.074595in}{1.531038in}}{\pgfqpoint{4.081728in}{1.531038in}}%
\pgfpathclose%
\pgfusepath{stroke,fill}%
\end{pgfscope}%
\begin{pgfscope}%
\pgfpathrectangle{\pgfqpoint{2.867647in}{0.500000in}}{\pgfqpoint{1.764706in}{1.700000in}}%
\pgfusepath{clip}%
\pgfsetbuttcap%
\pgfsetroundjoin%
\definecolor{currentfill}{rgb}{0.979891,0.908948,0.848279}%
\pgfsetfillcolor{currentfill}%
\pgfsetlinewidth{0.311001pt}%
\definecolor{currentstroke}{rgb}{1.000000,1.000000,1.000000}%
\pgfsetstrokecolor{currentstroke}%
\pgfsetdash{}{0pt}%
\pgfpathmoveto{\pgfqpoint{4.144555in}{1.108754in}}%
\pgfpathcurveto{\pgfqpoint{4.151688in}{1.108754in}}{\pgfqpoint{4.158530in}{1.111588in}}{\pgfqpoint{4.163573in}{1.116631in}}%
\pgfpathcurveto{\pgfqpoint{4.168617in}{1.121675in}}{\pgfqpoint{4.171451in}{1.128517in}}{\pgfqpoint{4.171451in}{1.135650in}}%
\pgfpathcurveto{\pgfqpoint{4.171451in}{1.142782in}}{\pgfqpoint{4.168617in}{1.149624in}}{\pgfqpoint{4.163573in}{1.154668in}}%
\pgfpathcurveto{\pgfqpoint{4.158530in}{1.159711in}}{\pgfqpoint{4.151688in}{1.162545in}}{\pgfqpoint{4.144555in}{1.162545in}}%
\pgfpathcurveto{\pgfqpoint{4.137422in}{1.162545in}}{\pgfqpoint{4.130581in}{1.159711in}}{\pgfqpoint{4.125537in}{1.154668in}}%
\pgfpathcurveto{\pgfqpoint{4.120493in}{1.149624in}}{\pgfqpoint{4.117659in}{1.142782in}}{\pgfqpoint{4.117659in}{1.135650in}}%
\pgfpathcurveto{\pgfqpoint{4.117659in}{1.128517in}}{\pgfqpoint{4.120493in}{1.121675in}}{\pgfqpoint{4.125537in}{1.116631in}}%
\pgfpathcurveto{\pgfqpoint{4.130581in}{1.111588in}}{\pgfqpoint{4.137422in}{1.108754in}}{\pgfqpoint{4.144555in}{1.108754in}}%
\pgfpathclose%
\pgfusepath{stroke,fill}%
\end{pgfscope}%
\begin{pgfscope}%
\pgfpathrectangle{\pgfqpoint{2.867647in}{0.500000in}}{\pgfqpoint{1.764706in}{1.700000in}}%
\pgfusepath{clip}%
\pgfsetbuttcap%
\pgfsetroundjoin%
\definecolor{currentfill}{rgb}{0.980678,0.914765,0.856766}%
\pgfsetfillcolor{currentfill}%
\pgfsetlinewidth{0.311001pt}%
\definecolor{currentstroke}{rgb}{1.000000,1.000000,1.000000}%
\pgfsetstrokecolor{currentstroke}%
\pgfsetdash{}{0pt}%
\pgfpathmoveto{\pgfqpoint{4.165375in}{1.413545in}}%
\pgfpathcurveto{\pgfqpoint{4.172507in}{1.413545in}}{\pgfqpoint{4.179349in}{1.416379in}}{\pgfqpoint{4.184393in}{1.421423in}}%
\pgfpathcurveto{\pgfqpoint{4.189436in}{1.426467in}}{\pgfqpoint{4.192270in}{1.433308in}}{\pgfqpoint{4.192270in}{1.440441in}}%
\pgfpathcurveto{\pgfqpoint{4.192270in}{1.447574in}}{\pgfqpoint{4.189436in}{1.454415in}}{\pgfqpoint{4.184393in}{1.459459in}}%
\pgfpathcurveto{\pgfqpoint{4.179349in}{1.464503in}}{\pgfqpoint{4.172507in}{1.467337in}}{\pgfqpoint{4.165375in}{1.467337in}}%
\pgfpathcurveto{\pgfqpoint{4.158242in}{1.467337in}}{\pgfqpoint{4.151400in}{1.464503in}}{\pgfqpoint{4.146356in}{1.459459in}}%
\pgfpathcurveto{\pgfqpoint{4.141313in}{1.454415in}}{\pgfqpoint{4.138479in}{1.447574in}}{\pgfqpoint{4.138479in}{1.440441in}}%
\pgfpathcurveto{\pgfqpoint{4.138479in}{1.433308in}}{\pgfqpoint{4.141313in}{1.426467in}}{\pgfqpoint{4.146356in}{1.421423in}}%
\pgfpathcurveto{\pgfqpoint{4.151400in}{1.416379in}}{\pgfqpoint{4.158242in}{1.413545in}}{\pgfqpoint{4.165375in}{1.413545in}}%
\pgfpathclose%
\pgfusepath{stroke,fill}%
\end{pgfscope}%
\begin{pgfscope}%
\pgfpathrectangle{\pgfqpoint{2.867647in}{0.500000in}}{\pgfqpoint{1.764706in}{1.700000in}}%
\pgfusepath{clip}%
\pgfsetbuttcap%
\pgfsetroundjoin%
\definecolor{currentfill}{rgb}{0.978376,0.897317,0.831308}%
\pgfsetfillcolor{currentfill}%
\pgfsetlinewidth{0.311001pt}%
\definecolor{currentstroke}{rgb}{1.000000,1.000000,1.000000}%
\pgfsetstrokecolor{currentstroke}%
\pgfsetdash{}{0pt}%
\pgfpathmoveto{\pgfqpoint{4.151383in}{1.400825in}}%
\pgfpathcurveto{\pgfqpoint{4.158516in}{1.400825in}}{\pgfqpoint{4.165357in}{1.403659in}}{\pgfqpoint{4.170401in}{1.408703in}}%
\pgfpathcurveto{\pgfqpoint{4.175445in}{1.413747in}}{\pgfqpoint{4.178279in}{1.420588in}}{\pgfqpoint{4.178279in}{1.427721in}}%
\pgfpathcurveto{\pgfqpoint{4.178279in}{1.434854in}}{\pgfqpoint{4.175445in}{1.441696in}}{\pgfqpoint{4.170401in}{1.446739in}}%
\pgfpathcurveto{\pgfqpoint{4.165357in}{1.451783in}}{\pgfqpoint{4.158516in}{1.454617in}}{\pgfqpoint{4.151383in}{1.454617in}}%
\pgfpathcurveto{\pgfqpoint{4.144250in}{1.454617in}}{\pgfqpoint{4.137408in}{1.451783in}}{\pgfqpoint{4.132365in}{1.446739in}}%
\pgfpathcurveto{\pgfqpoint{4.127321in}{1.441696in}}{\pgfqpoint{4.124487in}{1.434854in}}{\pgfqpoint{4.124487in}{1.427721in}}%
\pgfpathcurveto{\pgfqpoint{4.124487in}{1.420588in}}{\pgfqpoint{4.127321in}{1.413747in}}{\pgfqpoint{4.132365in}{1.408703in}}%
\pgfpathcurveto{\pgfqpoint{4.137408in}{1.403659in}}{\pgfqpoint{4.144250in}{1.400825in}}{\pgfqpoint{4.151383in}{1.400825in}}%
\pgfpathclose%
\pgfusepath{stroke,fill}%
\end{pgfscope}%
\begin{pgfscope}%
\pgfpathrectangle{\pgfqpoint{2.867647in}{0.500000in}}{\pgfqpoint{1.764706in}{1.700000in}}%
\pgfusepath{clip}%
\pgfsetbuttcap%
\pgfsetroundjoin%
\definecolor{currentfill}{rgb}{0.959229,0.533075,0.374889}%
\pgfsetfillcolor{currentfill}%
\pgfsetlinewidth{0.311001pt}%
\definecolor{currentstroke}{rgb}{1.000000,1.000000,1.000000}%
\pgfsetstrokecolor{currentstroke}%
\pgfsetdash{}{0pt}%
\pgfpathmoveto{\pgfqpoint{4.115367in}{1.804998in}}%
\pgfpathcurveto{\pgfqpoint{4.122500in}{1.804998in}}{\pgfqpoint{4.129341in}{1.807832in}}{\pgfqpoint{4.134385in}{1.812875in}}%
\pgfpathcurveto{\pgfqpoint{4.139429in}{1.817919in}}{\pgfqpoint{4.142262in}{1.824761in}}{\pgfqpoint{4.142262in}{1.831893in}}%
\pgfpathcurveto{\pgfqpoint{4.142262in}{1.839026in}}{\pgfqpoint{4.139429in}{1.845868in}}{\pgfqpoint{4.134385in}{1.850911in}}%
\pgfpathcurveto{\pgfqpoint{4.129341in}{1.855955in}}{\pgfqpoint{4.122500in}{1.858789in}}{\pgfqpoint{4.115367in}{1.858789in}}%
\pgfpathcurveto{\pgfqpoint{4.108234in}{1.858789in}}{\pgfqpoint{4.101392in}{1.855955in}}{\pgfqpoint{4.096349in}{1.850911in}}%
\pgfpathcurveto{\pgfqpoint{4.091305in}{1.845868in}}{\pgfqpoint{4.088471in}{1.839026in}}{\pgfqpoint{4.088471in}{1.831893in}}%
\pgfpathcurveto{\pgfqpoint{4.088471in}{1.824761in}}{\pgfqpoint{4.091305in}{1.817919in}}{\pgfqpoint{4.096349in}{1.812875in}}%
\pgfpathcurveto{\pgfqpoint{4.101392in}{1.807832in}}{\pgfqpoint{4.108234in}{1.804998in}}{\pgfqpoint{4.115367in}{1.804998in}}%
\pgfpathclose%
\pgfusepath{stroke,fill}%
\end{pgfscope}%
\begin{pgfscope}%
\pgfpathrectangle{\pgfqpoint{2.867647in}{0.500000in}}{\pgfqpoint{1.764706in}{1.700000in}}%
\pgfusepath{clip}%
\pgfsetbuttcap%
\pgfsetroundjoin%
\definecolor{currentfill}{rgb}{0.969359,0.803954,0.693832}%
\pgfsetfillcolor{currentfill}%
\pgfsetlinewidth{0.311001pt}%
\definecolor{currentstroke}{rgb}{1.000000,1.000000,1.000000}%
\pgfsetstrokecolor{currentstroke}%
\pgfsetdash{}{0pt}%
\pgfpathmoveto{\pgfqpoint{4.216524in}{1.590563in}}%
\pgfpathcurveto{\pgfqpoint{4.223657in}{1.590563in}}{\pgfqpoint{4.230499in}{1.593397in}}{\pgfqpoint{4.235542in}{1.598440in}}%
\pgfpathcurveto{\pgfqpoint{4.240586in}{1.603484in}}{\pgfqpoint{4.243420in}{1.610326in}}{\pgfqpoint{4.243420in}{1.617458in}}%
\pgfpathcurveto{\pgfqpoint{4.243420in}{1.624591in}}{\pgfqpoint{4.240586in}{1.631433in}}{\pgfqpoint{4.235542in}{1.636477in}}%
\pgfpathcurveto{\pgfqpoint{4.230499in}{1.641520in}}{\pgfqpoint{4.223657in}{1.644354in}}{\pgfqpoint{4.216524in}{1.644354in}}%
\pgfpathcurveto{\pgfqpoint{4.209391in}{1.644354in}}{\pgfqpoint{4.202550in}{1.641520in}}{\pgfqpoint{4.197506in}{1.636477in}}%
\pgfpathcurveto{\pgfqpoint{4.192462in}{1.631433in}}{\pgfqpoint{4.189628in}{1.624591in}}{\pgfqpoint{4.189628in}{1.617458in}}%
\pgfpathcurveto{\pgfqpoint{4.189628in}{1.610326in}}{\pgfqpoint{4.192462in}{1.603484in}}{\pgfqpoint{4.197506in}{1.598440in}}%
\pgfpathcurveto{\pgfqpoint{4.202550in}{1.593397in}}{\pgfqpoint{4.209391in}{1.590563in}}{\pgfqpoint{4.216524in}{1.590563in}}%
\pgfpathclose%
\pgfusepath{stroke,fill}%
\end{pgfscope}%
\begin{pgfscope}%
\pgfpathrectangle{\pgfqpoint{2.867647in}{0.500000in}}{\pgfqpoint{1.764706in}{1.700000in}}%
\pgfusepath{clip}%
\pgfsetbuttcap%
\pgfsetroundjoin%
\definecolor{currentfill}{rgb}{0.953816,0.463738,0.317699}%
\pgfsetfillcolor{currentfill}%
\pgfsetlinewidth{0.311001pt}%
\definecolor{currentstroke}{rgb}{1.000000,1.000000,1.000000}%
\pgfsetstrokecolor{currentstroke}%
\pgfsetdash{}{0pt}%
\pgfpathmoveto{\pgfqpoint{3.977513in}{1.837716in}}%
\pgfpathcurveto{\pgfqpoint{3.984645in}{1.837716in}}{\pgfqpoint{3.991487in}{1.840550in}}{\pgfqpoint{3.996531in}{1.845593in}}%
\pgfpathcurveto{\pgfqpoint{4.001574in}{1.850637in}}{\pgfqpoint{4.004408in}{1.857479in}}{\pgfqpoint{4.004408in}{1.864612in}}%
\pgfpathcurveto{\pgfqpoint{4.004408in}{1.871744in}}{\pgfqpoint{4.001574in}{1.878586in}}{\pgfqpoint{3.996531in}{1.883630in}}%
\pgfpathcurveto{\pgfqpoint{3.991487in}{1.888673in}}{\pgfqpoint{3.984645in}{1.891507in}}{\pgfqpoint{3.977513in}{1.891507in}}%
\pgfpathcurveto{\pgfqpoint{3.970380in}{1.891507in}}{\pgfqpoint{3.963538in}{1.888673in}}{\pgfqpoint{3.958494in}{1.883630in}}%
\pgfpathcurveto{\pgfqpoint{3.953451in}{1.878586in}}{\pgfqpoint{3.950617in}{1.871744in}}{\pgfqpoint{3.950617in}{1.864612in}}%
\pgfpathcurveto{\pgfqpoint{3.950617in}{1.857479in}}{\pgfqpoint{3.953451in}{1.850637in}}{\pgfqpoint{3.958494in}{1.845593in}}%
\pgfpathcurveto{\pgfqpoint{3.963538in}{1.840550in}}{\pgfqpoint{3.970380in}{1.837716in}}{\pgfqpoint{3.977513in}{1.837716in}}%
\pgfpathclose%
\pgfusepath{stroke,fill}%
\end{pgfscope}%
\begin{pgfscope}%
\pgfpathrectangle{\pgfqpoint{2.867647in}{0.500000in}}{\pgfqpoint{1.764706in}{1.700000in}}%
\pgfusepath{clip}%
\pgfsetbuttcap%
\pgfsetroundjoin%
\definecolor{currentfill}{rgb}{0.966328,0.750560,0.616961}%
\pgfsetfillcolor{currentfill}%
\pgfsetlinewidth{0.311001pt}%
\definecolor{currentstroke}{rgb}{1.000000,1.000000,1.000000}%
\pgfsetstrokecolor{currentstroke}%
\pgfsetdash{}{0pt}%
\pgfpathmoveto{\pgfqpoint{4.098853in}{1.736533in}}%
\pgfpathcurveto{\pgfqpoint{4.105986in}{1.736533in}}{\pgfqpoint{4.112828in}{1.739367in}}{\pgfqpoint{4.117871in}{1.744410in}}%
\pgfpathcurveto{\pgfqpoint{4.122915in}{1.749454in}}{\pgfqpoint{4.125749in}{1.756296in}}{\pgfqpoint{4.125749in}{1.763429in}}%
\pgfpathcurveto{\pgfqpoint{4.125749in}{1.770561in}}{\pgfqpoint{4.122915in}{1.777403in}}{\pgfqpoint{4.117871in}{1.782447in}}%
\pgfpathcurveto{\pgfqpoint{4.112828in}{1.787490in}}{\pgfqpoint{4.105986in}{1.790324in}}{\pgfqpoint{4.098853in}{1.790324in}}%
\pgfpathcurveto{\pgfqpoint{4.091720in}{1.790324in}}{\pgfqpoint{4.084879in}{1.787490in}}{\pgfqpoint{4.079835in}{1.782447in}}%
\pgfpathcurveto{\pgfqpoint{4.074791in}{1.777403in}}{\pgfqpoint{4.071957in}{1.770561in}}{\pgfqpoint{4.071957in}{1.763429in}}%
\pgfpathcurveto{\pgfqpoint{4.071957in}{1.756296in}}{\pgfqpoint{4.074791in}{1.749454in}}{\pgfqpoint{4.079835in}{1.744410in}}%
\pgfpathcurveto{\pgfqpoint{4.084879in}{1.739367in}}{\pgfqpoint{4.091720in}{1.736533in}}{\pgfqpoint{4.098853in}{1.736533in}}%
\pgfpathclose%
\pgfusepath{stroke,fill}%
\end{pgfscope}%
\begin{pgfscope}%
\pgfpathrectangle{\pgfqpoint{2.867647in}{0.500000in}}{\pgfqpoint{1.764706in}{1.700000in}}%
\pgfusepath{clip}%
\pgfsetbuttcap%
\pgfsetroundjoin%
\definecolor{currentfill}{rgb}{0.970718,0.821518,0.719872}%
\pgfsetfillcolor{currentfill}%
\pgfsetlinewidth{0.311001pt}%
\definecolor{currentstroke}{rgb}{1.000000,1.000000,1.000000}%
\pgfsetstrokecolor{currentstroke}%
\pgfsetdash{}{0pt}%
\pgfpathmoveto{\pgfqpoint{4.259708in}{1.188110in}}%
\pgfpathcurveto{\pgfqpoint{4.266840in}{1.188110in}}{\pgfqpoint{4.273682in}{1.190944in}}{\pgfqpoint{4.278726in}{1.195987in}}%
\pgfpathcurveto{\pgfqpoint{4.283769in}{1.201031in}}{\pgfqpoint{4.286603in}{1.207873in}}{\pgfqpoint{4.286603in}{1.215005in}}%
\pgfpathcurveto{\pgfqpoint{4.286603in}{1.222138in}}{\pgfqpoint{4.283769in}{1.228980in}}{\pgfqpoint{4.278726in}{1.234024in}}%
\pgfpathcurveto{\pgfqpoint{4.273682in}{1.239067in}}{\pgfqpoint{4.266840in}{1.241901in}}{\pgfqpoint{4.259708in}{1.241901in}}%
\pgfpathcurveto{\pgfqpoint{4.252575in}{1.241901in}}{\pgfqpoint{4.245733in}{1.239067in}}{\pgfqpoint{4.240689in}{1.234024in}}%
\pgfpathcurveto{\pgfqpoint{4.235646in}{1.228980in}}{\pgfqpoint{4.232812in}{1.222138in}}{\pgfqpoint{4.232812in}{1.215005in}}%
\pgfpathcurveto{\pgfqpoint{4.232812in}{1.207873in}}{\pgfqpoint{4.235646in}{1.201031in}}{\pgfqpoint{4.240689in}{1.195987in}}%
\pgfpathcurveto{\pgfqpoint{4.245733in}{1.190944in}}{\pgfqpoint{4.252575in}{1.188110in}}{\pgfqpoint{4.259708in}{1.188110in}}%
\pgfpathclose%
\pgfusepath{stroke,fill}%
\end{pgfscope}%
\begin{pgfscope}%
\pgfpathrectangle{\pgfqpoint{2.867647in}{0.500000in}}{\pgfqpoint{1.764706in}{1.700000in}}%
\pgfusepath{clip}%
\pgfsetbuttcap%
\pgfsetroundjoin%
\definecolor{currentfill}{rgb}{0.979891,0.908948,0.848279}%
\pgfsetfillcolor{currentfill}%
\pgfsetlinewidth{0.311001pt}%
\definecolor{currentstroke}{rgb}{1.000000,1.000000,1.000000}%
\pgfsetstrokecolor{currentstroke}%
\pgfsetdash{}{0pt}%
\pgfpathmoveto{\pgfqpoint{4.220370in}{1.288935in}}%
\pgfpathcurveto{\pgfqpoint{4.227503in}{1.288935in}}{\pgfqpoint{4.234344in}{1.291769in}}{\pgfqpoint{4.239388in}{1.296812in}}%
\pgfpathcurveto{\pgfqpoint{4.244432in}{1.301856in}}{\pgfqpoint{4.247266in}{1.308698in}}{\pgfqpoint{4.247266in}{1.315831in}}%
\pgfpathcurveto{\pgfqpoint{4.247266in}{1.322963in}}{\pgfqpoint{4.244432in}{1.329805in}}{\pgfqpoint{4.239388in}{1.334849in}}%
\pgfpathcurveto{\pgfqpoint{4.234344in}{1.339892in}}{\pgfqpoint{4.227503in}{1.342726in}}{\pgfqpoint{4.220370in}{1.342726in}}%
\pgfpathcurveto{\pgfqpoint{4.213237in}{1.342726in}}{\pgfqpoint{4.206396in}{1.339892in}}{\pgfqpoint{4.201352in}{1.334849in}}%
\pgfpathcurveto{\pgfqpoint{4.196308in}{1.329805in}}{\pgfqpoint{4.193474in}{1.322963in}}{\pgfqpoint{4.193474in}{1.315831in}}%
\pgfpathcurveto{\pgfqpoint{4.193474in}{1.308698in}}{\pgfqpoint{4.196308in}{1.301856in}}{\pgfqpoint{4.201352in}{1.296812in}}%
\pgfpathcurveto{\pgfqpoint{4.206396in}{1.291769in}}{\pgfqpoint{4.213237in}{1.288935in}}{\pgfqpoint{4.220370in}{1.288935in}}%
\pgfpathclose%
\pgfusepath{stroke,fill}%
\end{pgfscope}%
\begin{pgfscope}%
\pgfpathrectangle{\pgfqpoint{2.867647in}{0.500000in}}{\pgfqpoint{1.764706in}{1.700000in}}%
\pgfusepath{clip}%
\pgfsetbuttcap%
\pgfsetroundjoin%
\definecolor{currentfill}{rgb}{0.972201,0.839051,0.745789}%
\pgfsetfillcolor{currentfill}%
\pgfsetlinewidth{0.311001pt}%
\definecolor{currentstroke}{rgb}{1.000000,1.000000,1.000000}%
\pgfsetstrokecolor{currentstroke}%
\pgfsetdash{}{0pt}%
\pgfpathmoveto{\pgfqpoint{4.263434in}{1.337658in}}%
\pgfpathcurveto{\pgfqpoint{4.270566in}{1.337658in}}{\pgfqpoint{4.277408in}{1.340492in}}{\pgfqpoint{4.282452in}{1.345536in}}%
\pgfpathcurveto{\pgfqpoint{4.287495in}{1.350580in}}{\pgfqpoint{4.290329in}{1.357421in}}{\pgfqpoint{4.290329in}{1.364554in}}%
\pgfpathcurveto{\pgfqpoint{4.290329in}{1.371687in}}{\pgfqpoint{4.287495in}{1.378529in}}{\pgfqpoint{4.282452in}{1.383572in}}%
\pgfpathcurveto{\pgfqpoint{4.277408in}{1.388616in}}{\pgfqpoint{4.270566in}{1.391450in}}{\pgfqpoint{4.263434in}{1.391450in}}%
\pgfpathcurveto{\pgfqpoint{4.256301in}{1.391450in}}{\pgfqpoint{4.249459in}{1.388616in}}{\pgfqpoint{4.244415in}{1.383572in}}%
\pgfpathcurveto{\pgfqpoint{4.239372in}{1.378529in}}{\pgfqpoint{4.236538in}{1.371687in}}{\pgfqpoint{4.236538in}{1.364554in}}%
\pgfpathcurveto{\pgfqpoint{4.236538in}{1.357421in}}{\pgfqpoint{4.239372in}{1.350580in}}{\pgfqpoint{4.244415in}{1.345536in}}%
\pgfpathcurveto{\pgfqpoint{4.249459in}{1.340492in}}{\pgfqpoint{4.256301in}{1.337658in}}{\pgfqpoint{4.263434in}{1.337658in}}%
\pgfpathclose%
\pgfusepath{stroke,fill}%
\end{pgfscope}%
\begin{pgfscope}%
\pgfpathrectangle{\pgfqpoint{2.867647in}{0.500000in}}{\pgfqpoint{1.764706in}{1.700000in}}%
\pgfusepath{clip}%
\pgfsetbuttcap%
\pgfsetroundjoin%
\definecolor{currentfill}{rgb}{0.975644,0.874038,0.797253}%
\pgfsetfillcolor{currentfill}%
\pgfsetlinewidth{0.311001pt}%
\definecolor{currentstroke}{rgb}{1.000000,1.000000,1.000000}%
\pgfsetstrokecolor{currentstroke}%
\pgfsetdash{}{0pt}%
\pgfpathmoveto{\pgfqpoint{4.093608in}{1.540190in}}%
\pgfpathcurveto{\pgfqpoint{4.100741in}{1.540190in}}{\pgfqpoint{4.107583in}{1.543024in}}{\pgfqpoint{4.112626in}{1.548068in}}%
\pgfpathcurveto{\pgfqpoint{4.117670in}{1.553112in}}{\pgfqpoint{4.120504in}{1.559953in}}{\pgfqpoint{4.120504in}{1.567086in}}%
\pgfpathcurveto{\pgfqpoint{4.120504in}{1.574219in}}{\pgfqpoint{4.117670in}{1.581060in}}{\pgfqpoint{4.112626in}{1.586104in}}%
\pgfpathcurveto{\pgfqpoint{4.107583in}{1.591148in}}{\pgfqpoint{4.100741in}{1.593982in}}{\pgfqpoint{4.093608in}{1.593982in}}%
\pgfpathcurveto{\pgfqpoint{4.086475in}{1.593982in}}{\pgfqpoint{4.079634in}{1.591148in}}{\pgfqpoint{4.074590in}{1.586104in}}%
\pgfpathcurveto{\pgfqpoint{4.069546in}{1.581060in}}{\pgfqpoint{4.066713in}{1.574219in}}{\pgfqpoint{4.066713in}{1.567086in}}%
\pgfpathcurveto{\pgfqpoint{4.066713in}{1.559953in}}{\pgfqpoint{4.069546in}{1.553112in}}{\pgfqpoint{4.074590in}{1.548068in}}%
\pgfpathcurveto{\pgfqpoint{4.079634in}{1.543024in}}{\pgfqpoint{4.086475in}{1.540190in}}{\pgfqpoint{4.093608in}{1.540190in}}%
\pgfpathclose%
\pgfusepath{stroke,fill}%
\end{pgfscope}%
\begin{pgfscope}%
\pgfpathrectangle{\pgfqpoint{2.867647in}{0.500000in}}{\pgfqpoint{1.764706in}{1.700000in}}%
\pgfusepath{clip}%
\pgfsetbuttcap%
\pgfsetroundjoin%
\definecolor{currentfill}{rgb}{0.976287,0.879862,0.805788}%
\pgfsetfillcolor{currentfill}%
\pgfsetlinewidth{0.311001pt}%
\definecolor{currentstroke}{rgb}{1.000000,1.000000,1.000000}%
\pgfsetstrokecolor{currentstroke}%
\pgfsetdash{}{0pt}%
\pgfpathmoveto{\pgfqpoint{4.242778in}{1.319982in}}%
\pgfpathcurveto{\pgfqpoint{4.249911in}{1.319982in}}{\pgfqpoint{4.256752in}{1.322815in}}{\pgfqpoint{4.261796in}{1.327859in}}%
\pgfpathcurveto{\pgfqpoint{4.266840in}{1.332903in}}{\pgfqpoint{4.269673in}{1.339744in}}{\pgfqpoint{4.269673in}{1.346877in}}%
\pgfpathcurveto{\pgfqpoint{4.269673in}{1.354010in}}{\pgfqpoint{4.266840in}{1.360852in}}{\pgfqpoint{4.261796in}{1.365895in}}%
\pgfpathcurveto{\pgfqpoint{4.256752in}{1.370939in}}{\pgfqpoint{4.249911in}{1.373773in}}{\pgfqpoint{4.242778in}{1.373773in}}%
\pgfpathcurveto{\pgfqpoint{4.235645in}{1.373773in}}{\pgfqpoint{4.228803in}{1.370939in}}{\pgfqpoint{4.223760in}{1.365895in}}%
\pgfpathcurveto{\pgfqpoint{4.218716in}{1.360852in}}{\pgfqpoint{4.215882in}{1.354010in}}{\pgfqpoint{4.215882in}{1.346877in}}%
\pgfpathcurveto{\pgfqpoint{4.215882in}{1.339744in}}{\pgfqpoint{4.218716in}{1.332903in}}{\pgfqpoint{4.223760in}{1.327859in}}%
\pgfpathcurveto{\pgfqpoint{4.228803in}{1.322815in}}{\pgfqpoint{4.235645in}{1.319982in}}{\pgfqpoint{4.242778in}{1.319982in}}%
\pgfpathclose%
\pgfusepath{stroke,fill}%
\end{pgfscope}%
\begin{pgfscope}%
\pgfpathrectangle{\pgfqpoint{2.867647in}{0.500000in}}{\pgfqpoint{1.764706in}{1.700000in}}%
\pgfusepath{clip}%
\pgfsetbuttcap%
\pgfsetroundjoin%
\definecolor{currentfill}{rgb}{0.980678,0.914765,0.856766}%
\pgfsetfillcolor{currentfill}%
\pgfsetlinewidth{0.311001pt}%
\definecolor{currentstroke}{rgb}{1.000000,1.000000,1.000000}%
\pgfsetstrokecolor{currentstroke}%
\pgfsetdash{}{0pt}%
\pgfpathmoveto{\pgfqpoint{4.168401in}{1.455724in}}%
\pgfpathcurveto{\pgfqpoint{4.175534in}{1.455724in}}{\pgfqpoint{4.182376in}{1.458557in}}{\pgfqpoint{4.187419in}{1.463601in}}%
\pgfpathcurveto{\pgfqpoint{4.192463in}{1.468645in}}{\pgfqpoint{4.195297in}{1.475486in}}{\pgfqpoint{4.195297in}{1.482619in}}%
\pgfpathcurveto{\pgfqpoint{4.195297in}{1.489752in}}{\pgfqpoint{4.192463in}{1.496594in}}{\pgfqpoint{4.187419in}{1.501637in}}%
\pgfpathcurveto{\pgfqpoint{4.182376in}{1.506681in}}{\pgfqpoint{4.175534in}{1.509515in}}{\pgfqpoint{4.168401in}{1.509515in}}%
\pgfpathcurveto{\pgfqpoint{4.161268in}{1.509515in}}{\pgfqpoint{4.154427in}{1.506681in}}{\pgfqpoint{4.149383in}{1.501637in}}%
\pgfpathcurveto{\pgfqpoint{4.144339in}{1.496594in}}{\pgfqpoint{4.141506in}{1.489752in}}{\pgfqpoint{4.141506in}{1.482619in}}%
\pgfpathcurveto{\pgfqpoint{4.141506in}{1.475486in}}{\pgfqpoint{4.144339in}{1.468645in}}{\pgfqpoint{4.149383in}{1.463601in}}%
\pgfpathcurveto{\pgfqpoint{4.154427in}{1.458557in}}{\pgfqpoint{4.161268in}{1.455724in}}{\pgfqpoint{4.168401in}{1.455724in}}%
\pgfpathclose%
\pgfusepath{stroke,fill}%
\end{pgfscope}%
\begin{pgfscope}%
\pgfpathrectangle{\pgfqpoint{2.867647in}{0.500000in}}{\pgfqpoint{1.764706in}{1.700000in}}%
\pgfusepath{clip}%
\pgfsetbuttcap%
\pgfsetroundjoin%
\definecolor{currentfill}{rgb}{0.976287,0.879862,0.805788}%
\pgfsetfillcolor{currentfill}%
\pgfsetlinewidth{0.311001pt}%
\definecolor{currentstroke}{rgb}{1.000000,1.000000,1.000000}%
\pgfsetstrokecolor{currentstroke}%
\pgfsetdash{}{0pt}%
\pgfpathmoveto{\pgfqpoint{4.203535in}{1.113350in}}%
\pgfpathcurveto{\pgfqpoint{4.210667in}{1.113350in}}{\pgfqpoint{4.217509in}{1.116184in}}{\pgfqpoint{4.222553in}{1.121227in}}%
\pgfpathcurveto{\pgfqpoint{4.227596in}{1.126271in}}{\pgfqpoint{4.230430in}{1.133113in}}{\pgfqpoint{4.230430in}{1.140245in}}%
\pgfpathcurveto{\pgfqpoint{4.230430in}{1.147378in}}{\pgfqpoint{4.227596in}{1.154220in}}{\pgfqpoint{4.222553in}{1.159264in}}%
\pgfpathcurveto{\pgfqpoint{4.217509in}{1.164307in}}{\pgfqpoint{4.210667in}{1.167141in}}{\pgfqpoint{4.203535in}{1.167141in}}%
\pgfpathcurveto{\pgfqpoint{4.196402in}{1.167141in}}{\pgfqpoint{4.189560in}{1.164307in}}{\pgfqpoint{4.184516in}{1.159264in}}%
\pgfpathcurveto{\pgfqpoint{4.179473in}{1.154220in}}{\pgfqpoint{4.176639in}{1.147378in}}{\pgfqpoint{4.176639in}{1.140245in}}%
\pgfpathcurveto{\pgfqpoint{4.176639in}{1.133113in}}{\pgfqpoint{4.179473in}{1.126271in}}{\pgfqpoint{4.184516in}{1.121227in}}%
\pgfpathcurveto{\pgfqpoint{4.189560in}{1.116184in}}{\pgfqpoint{4.196402in}{1.113350in}}{\pgfqpoint{4.203535in}{1.113350in}}%
\pgfpathclose%
\pgfusepath{stroke,fill}%
\end{pgfscope}%
\begin{pgfscope}%
\pgfpathrectangle{\pgfqpoint{2.867647in}{0.500000in}}{\pgfqpoint{1.764706in}{1.700000in}}%
\pgfusepath{clip}%
\pgfsetbuttcap%
\pgfsetroundjoin%
\definecolor{currentfill}{rgb}{0.979891,0.908948,0.848279}%
\pgfsetfillcolor{currentfill}%
\pgfsetlinewidth{0.311001pt}%
\definecolor{currentstroke}{rgb}{1.000000,1.000000,1.000000}%
\pgfsetstrokecolor{currentstroke}%
\pgfsetdash{}{0pt}%
\pgfpathmoveto{\pgfqpoint{4.209469in}{1.211824in}}%
\pgfpathcurveto{\pgfqpoint{4.216602in}{1.211824in}}{\pgfqpoint{4.223443in}{1.214658in}}{\pgfqpoint{4.228487in}{1.219702in}}%
\pgfpathcurveto{\pgfqpoint{4.233531in}{1.224745in}}{\pgfqpoint{4.236365in}{1.231587in}}{\pgfqpoint{4.236365in}{1.238720in}}%
\pgfpathcurveto{\pgfqpoint{4.236365in}{1.245853in}}{\pgfqpoint{4.233531in}{1.252694in}}{\pgfqpoint{4.228487in}{1.257738in}}%
\pgfpathcurveto{\pgfqpoint{4.223443in}{1.262782in}}{\pgfqpoint{4.216602in}{1.265616in}}{\pgfqpoint{4.209469in}{1.265616in}}%
\pgfpathcurveto{\pgfqpoint{4.202336in}{1.265616in}}{\pgfqpoint{4.195494in}{1.262782in}}{\pgfqpoint{4.190451in}{1.257738in}}%
\pgfpathcurveto{\pgfqpoint{4.185407in}{1.252694in}}{\pgfqpoint{4.182573in}{1.245853in}}{\pgfqpoint{4.182573in}{1.238720in}}%
\pgfpathcurveto{\pgfqpoint{4.182573in}{1.231587in}}{\pgfqpoint{4.185407in}{1.224745in}}{\pgfqpoint{4.190451in}{1.219702in}}%
\pgfpathcurveto{\pgfqpoint{4.195494in}{1.214658in}}{\pgfqpoint{4.202336in}{1.211824in}}{\pgfqpoint{4.209469in}{1.211824in}}%
\pgfpathclose%
\pgfusepath{stroke,fill}%
\end{pgfscope}%
\begin{pgfscope}%
\pgfpathrectangle{\pgfqpoint{2.867647in}{0.500000in}}{\pgfqpoint{1.764706in}{1.700000in}}%
\pgfusepath{clip}%
\pgfsetbuttcap%
\pgfsetroundjoin%
\definecolor{currentfill}{rgb}{0.965440,0.720101,0.576404}%
\pgfsetfillcolor{currentfill}%
\pgfsetlinewidth{0.311001pt}%
\definecolor{currentstroke}{rgb}{1.000000,1.000000,1.000000}%
\pgfsetstrokecolor{currentstroke}%
\pgfsetdash{}{0pt}%
\pgfpathmoveto{\pgfqpoint{4.008795in}{1.038588in}}%
\pgfpathcurveto{\pgfqpoint{4.015928in}{1.038588in}}{\pgfqpoint{4.022769in}{1.041422in}}{\pgfqpoint{4.027813in}{1.046465in}}%
\pgfpathcurveto{\pgfqpoint{4.032857in}{1.051509in}}{\pgfqpoint{4.035690in}{1.058351in}}{\pgfqpoint{4.035690in}{1.065484in}}%
\pgfpathcurveto{\pgfqpoint{4.035690in}{1.072616in}}{\pgfqpoint{4.032857in}{1.079458in}}{\pgfqpoint{4.027813in}{1.084502in}}%
\pgfpathcurveto{\pgfqpoint{4.022769in}{1.089545in}}{\pgfqpoint{4.015928in}{1.092379in}}{\pgfqpoint{4.008795in}{1.092379in}}%
\pgfpathcurveto{\pgfqpoint{4.001662in}{1.092379in}}{\pgfqpoint{3.994820in}{1.089545in}}{\pgfqpoint{3.989777in}{1.084502in}}%
\pgfpathcurveto{\pgfqpoint{3.984733in}{1.079458in}}{\pgfqpoint{3.981899in}{1.072616in}}{\pgfqpoint{3.981899in}{1.065484in}}%
\pgfpathcurveto{\pgfqpoint{3.981899in}{1.058351in}}{\pgfqpoint{3.984733in}{1.051509in}}{\pgfqpoint{3.989777in}{1.046465in}}%
\pgfpathcurveto{\pgfqpoint{3.994820in}{1.041422in}}{\pgfqpoint{4.001662in}{1.038588in}}{\pgfqpoint{4.008795in}{1.038588in}}%
\pgfpathclose%
\pgfusepath{stroke,fill}%
\end{pgfscope}%
\begin{pgfscope}%
\pgfpathrectangle{\pgfqpoint{2.867647in}{0.500000in}}{\pgfqpoint{1.764706in}{1.700000in}}%
\pgfusepath{clip}%
\pgfsetbuttcap%
\pgfsetroundjoin%
\definecolor{currentfill}{rgb}{0.979891,0.908948,0.848279}%
\pgfsetfillcolor{currentfill}%
\pgfsetlinewidth{0.311001pt}%
\definecolor{currentstroke}{rgb}{1.000000,1.000000,1.000000}%
\pgfsetstrokecolor{currentstroke}%
\pgfsetdash{}{0pt}%
\pgfpathmoveto{\pgfqpoint{4.212497in}{1.384527in}}%
\pgfpathcurveto{\pgfqpoint{4.219629in}{1.384527in}}{\pgfqpoint{4.226471in}{1.387361in}}{\pgfqpoint{4.231515in}{1.392405in}}%
\pgfpathcurveto{\pgfqpoint{4.236558in}{1.397448in}}{\pgfqpoint{4.239392in}{1.404290in}}{\pgfqpoint{4.239392in}{1.411423in}}%
\pgfpathcurveto{\pgfqpoint{4.239392in}{1.418556in}}{\pgfqpoint{4.236558in}{1.425397in}}{\pgfqpoint{4.231515in}{1.430441in}}%
\pgfpathcurveto{\pgfqpoint{4.226471in}{1.435485in}}{\pgfqpoint{4.219629in}{1.438319in}}{\pgfqpoint{4.212497in}{1.438319in}}%
\pgfpathcurveto{\pgfqpoint{4.205364in}{1.438319in}}{\pgfqpoint{4.198522in}{1.435485in}}{\pgfqpoint{4.193479in}{1.430441in}}%
\pgfpathcurveto{\pgfqpoint{4.188435in}{1.425397in}}{\pgfqpoint{4.185601in}{1.418556in}}{\pgfqpoint{4.185601in}{1.411423in}}%
\pgfpathcurveto{\pgfqpoint{4.185601in}{1.404290in}}{\pgfqpoint{4.188435in}{1.397448in}}{\pgfqpoint{4.193479in}{1.392405in}}%
\pgfpathcurveto{\pgfqpoint{4.198522in}{1.387361in}}{\pgfqpoint{4.205364in}{1.384527in}}{\pgfqpoint{4.212497in}{1.384527in}}%
\pgfpathclose%
\pgfusepath{stroke,fill}%
\end{pgfscope}%
\begin{pgfscope}%
\pgfpathrectangle{\pgfqpoint{2.867647in}{0.500000in}}{\pgfqpoint{1.764706in}{1.700000in}}%
\pgfusepath{clip}%
\pgfsetbuttcap%
\pgfsetroundjoin%
\definecolor{currentfill}{rgb}{0.979891,0.908948,0.848279}%
\pgfsetfillcolor{currentfill}%
\pgfsetlinewidth{0.311001pt}%
\definecolor{currentstroke}{rgb}{1.000000,1.000000,1.000000}%
\pgfsetstrokecolor{currentstroke}%
\pgfsetdash{}{0pt}%
\pgfpathmoveto{\pgfqpoint{4.151370in}{1.558441in}}%
\pgfpathcurveto{\pgfqpoint{4.158503in}{1.558441in}}{\pgfqpoint{4.165345in}{1.561275in}}{\pgfqpoint{4.170389in}{1.566319in}}%
\pgfpathcurveto{\pgfqpoint{4.175432in}{1.571362in}}{\pgfqpoint{4.178266in}{1.578204in}}{\pgfqpoint{4.178266in}{1.585337in}}%
\pgfpathcurveto{\pgfqpoint{4.178266in}{1.592470in}}{\pgfqpoint{4.175432in}{1.599311in}}{\pgfqpoint{4.170389in}{1.604355in}}%
\pgfpathcurveto{\pgfqpoint{4.165345in}{1.609399in}}{\pgfqpoint{4.158503in}{1.612233in}}{\pgfqpoint{4.151370in}{1.612233in}}%
\pgfpathcurveto{\pgfqpoint{4.144238in}{1.612233in}}{\pgfqpoint{4.137396in}{1.609399in}}{\pgfqpoint{4.132352in}{1.604355in}}%
\pgfpathcurveto{\pgfqpoint{4.127309in}{1.599311in}}{\pgfqpoint{4.124475in}{1.592470in}}{\pgfqpoint{4.124475in}{1.585337in}}%
\pgfpathcurveto{\pgfqpoint{4.124475in}{1.578204in}}{\pgfqpoint{4.127309in}{1.571362in}}{\pgfqpoint{4.132352in}{1.566319in}}%
\pgfpathcurveto{\pgfqpoint{4.137396in}{1.561275in}}{\pgfqpoint{4.144238in}{1.558441in}}{\pgfqpoint{4.151370in}{1.558441in}}%
\pgfpathclose%
\pgfusepath{stroke,fill}%
\end{pgfscope}%
\begin{pgfscope}%
\pgfpathrectangle{\pgfqpoint{2.867647in}{0.500000in}}{\pgfqpoint{1.764706in}{1.700000in}}%
\pgfusepath{clip}%
\pgfsetbuttcap%
\pgfsetroundjoin%
\definecolor{currentfill}{rgb}{0.879259,0.192033,0.262681}%
\pgfsetfillcolor{currentfill}%
\pgfsetlinewidth{0.311001pt}%
\definecolor{currentstroke}{rgb}{1.000000,1.000000,1.000000}%
\pgfsetstrokecolor{currentstroke}%
\pgfsetdash{}{0pt}%
\pgfpathmoveto{\pgfqpoint{4.193924in}{1.818305in}}%
\pgfpathcurveto{\pgfqpoint{4.201057in}{1.818305in}}{\pgfqpoint{4.207899in}{1.821139in}}{\pgfqpoint{4.212942in}{1.826183in}}%
\pgfpathcurveto{\pgfqpoint{4.217986in}{1.831226in}}{\pgfqpoint{4.220820in}{1.838068in}}{\pgfqpoint{4.220820in}{1.845201in}}%
\pgfpathcurveto{\pgfqpoint{4.220820in}{1.852334in}}{\pgfqpoint{4.217986in}{1.859175in}}{\pgfqpoint{4.212942in}{1.864219in}}%
\pgfpathcurveto{\pgfqpoint{4.207899in}{1.869263in}}{\pgfqpoint{4.201057in}{1.872097in}}{\pgfqpoint{4.193924in}{1.872097in}}%
\pgfpathcurveto{\pgfqpoint{4.186791in}{1.872097in}}{\pgfqpoint{4.179950in}{1.869263in}}{\pgfqpoint{4.174906in}{1.864219in}}%
\pgfpathcurveto{\pgfqpoint{4.169862in}{1.859175in}}{\pgfqpoint{4.167028in}{1.852334in}}{\pgfqpoint{4.167028in}{1.845201in}}%
\pgfpathcurveto{\pgfqpoint{4.167028in}{1.838068in}}{\pgfqpoint{4.169862in}{1.831226in}}{\pgfqpoint{4.174906in}{1.826183in}}%
\pgfpathcurveto{\pgfqpoint{4.179950in}{1.821139in}}{\pgfqpoint{4.186791in}{1.818305in}}{\pgfqpoint{4.193924in}{1.818305in}}%
\pgfpathclose%
\pgfusepath{stroke,fill}%
\end{pgfscope}%
\begin{pgfscope}%
\pgfpathrectangle{\pgfqpoint{2.867647in}{0.500000in}}{\pgfqpoint{1.764706in}{1.700000in}}%
\pgfusepath{clip}%
\pgfsetbuttcap%
\pgfsetroundjoin%
\definecolor{currentfill}{rgb}{0.973271,0.850724,0.762998}%
\pgfsetfillcolor{currentfill}%
\pgfsetlinewidth{0.311001pt}%
\definecolor{currentstroke}{rgb}{1.000000,1.000000,1.000000}%
\pgfsetstrokecolor{currentstroke}%
\pgfsetdash{}{0pt}%
\pgfpathmoveto{\pgfqpoint{4.103943in}{0.985043in}}%
\pgfpathcurveto{\pgfqpoint{4.111076in}{0.985043in}}{\pgfqpoint{4.117918in}{0.987877in}}{\pgfqpoint{4.122961in}{0.992920in}}%
\pgfpathcurveto{\pgfqpoint{4.128005in}{0.997964in}}{\pgfqpoint{4.130839in}{1.004806in}}{\pgfqpoint{4.130839in}{1.011938in}}%
\pgfpathcurveto{\pgfqpoint{4.130839in}{1.019071in}}{\pgfqpoint{4.128005in}{1.025913in}}{\pgfqpoint{4.122961in}{1.030957in}}%
\pgfpathcurveto{\pgfqpoint{4.117918in}{1.036000in}}{\pgfqpoint{4.111076in}{1.038834in}}{\pgfqpoint{4.103943in}{1.038834in}}%
\pgfpathcurveto{\pgfqpoint{4.096810in}{1.038834in}}{\pgfqpoint{4.089969in}{1.036000in}}{\pgfqpoint{4.084925in}{1.030957in}}%
\pgfpathcurveto{\pgfqpoint{4.079881in}{1.025913in}}{\pgfqpoint{4.077048in}{1.019071in}}{\pgfqpoint{4.077048in}{1.011938in}}%
\pgfpathcurveto{\pgfqpoint{4.077048in}{1.004806in}}{\pgfqpoint{4.079881in}{0.997964in}}{\pgfqpoint{4.084925in}{0.992920in}}%
\pgfpathcurveto{\pgfqpoint{4.089969in}{0.987877in}}{\pgfqpoint{4.096810in}{0.985043in}}{\pgfqpoint{4.103943in}{0.985043in}}%
\pgfpathclose%
\pgfusepath{stroke,fill}%
\end{pgfscope}%
\begin{pgfscope}%
\pgfpathrectangle{\pgfqpoint{2.867647in}{0.500000in}}{\pgfqpoint{1.764706in}{1.700000in}}%
\pgfusepath{clip}%
\pgfsetbuttcap%
\pgfsetroundjoin%
\definecolor{currentfill}{rgb}{0.981377,0.920617,0.865369}%
\pgfsetfillcolor{currentfill}%
\pgfsetlinewidth{0.311001pt}%
\definecolor{currentstroke}{rgb}{1.000000,1.000000,1.000000}%
\pgfsetstrokecolor{currentstroke}%
\pgfsetdash{}{0pt}%
\pgfpathmoveto{\pgfqpoint{4.200255in}{1.323715in}}%
\pgfpathcurveto{\pgfqpoint{4.207388in}{1.323715in}}{\pgfqpoint{4.214230in}{1.326549in}}{\pgfqpoint{4.219273in}{1.331593in}}%
\pgfpathcurveto{\pgfqpoint{4.224317in}{1.336637in}}{\pgfqpoint{4.227151in}{1.343478in}}{\pgfqpoint{4.227151in}{1.350611in}}%
\pgfpathcurveto{\pgfqpoint{4.227151in}{1.357744in}}{\pgfqpoint{4.224317in}{1.364586in}}{\pgfqpoint{4.219273in}{1.369629in}}%
\pgfpathcurveto{\pgfqpoint{4.214230in}{1.374673in}}{\pgfqpoint{4.207388in}{1.377507in}}{\pgfqpoint{4.200255in}{1.377507in}}%
\pgfpathcurveto{\pgfqpoint{4.193122in}{1.377507in}}{\pgfqpoint{4.186281in}{1.374673in}}{\pgfqpoint{4.181237in}{1.369629in}}%
\pgfpathcurveto{\pgfqpoint{4.176193in}{1.364586in}}{\pgfqpoint{4.173360in}{1.357744in}}{\pgfqpoint{4.173360in}{1.350611in}}%
\pgfpathcurveto{\pgfqpoint{4.173360in}{1.343478in}}{\pgfqpoint{4.176193in}{1.336637in}}{\pgfqpoint{4.181237in}{1.331593in}}%
\pgfpathcurveto{\pgfqpoint{4.186281in}{1.326549in}}{\pgfqpoint{4.193122in}{1.323715in}}{\pgfqpoint{4.200255in}{1.323715in}}%
\pgfpathclose%
\pgfusepath{stroke,fill}%
\end{pgfscope}%
\begin{pgfscope}%
\pgfpathrectangle{\pgfqpoint{2.867647in}{0.500000in}}{\pgfqpoint{1.764706in}{1.700000in}}%
\pgfusepath{clip}%
\pgfsetbuttcap%
\pgfsetroundjoin%
\definecolor{currentfill}{rgb}{0.979124,0.903132,0.839793}%
\pgfsetfillcolor{currentfill}%
\pgfsetlinewidth{0.311001pt}%
\definecolor{currentstroke}{rgb}{1.000000,1.000000,1.000000}%
\pgfsetstrokecolor{currentstroke}%
\pgfsetdash{}{0pt}%
\pgfpathmoveto{\pgfqpoint{4.133204in}{1.110023in}}%
\pgfpathcurveto{\pgfqpoint{4.140336in}{1.110023in}}{\pgfqpoint{4.147178in}{1.112857in}}{\pgfqpoint{4.152222in}{1.117900in}}%
\pgfpathcurveto{\pgfqpoint{4.157265in}{1.122944in}}{\pgfqpoint{4.160099in}{1.129786in}}{\pgfqpoint{4.160099in}{1.136919in}}%
\pgfpathcurveto{\pgfqpoint{4.160099in}{1.144051in}}{\pgfqpoint{4.157265in}{1.150893in}}{\pgfqpoint{4.152222in}{1.155937in}}%
\pgfpathcurveto{\pgfqpoint{4.147178in}{1.160980in}}{\pgfqpoint{4.140336in}{1.163814in}}{\pgfqpoint{4.133204in}{1.163814in}}%
\pgfpathcurveto{\pgfqpoint{4.126071in}{1.163814in}}{\pgfqpoint{4.119229in}{1.160980in}}{\pgfqpoint{4.114185in}{1.155937in}}%
\pgfpathcurveto{\pgfqpoint{4.109142in}{1.150893in}}{\pgfqpoint{4.106308in}{1.144051in}}{\pgfqpoint{4.106308in}{1.136919in}}%
\pgfpathcurveto{\pgfqpoint{4.106308in}{1.129786in}}{\pgfqpoint{4.109142in}{1.122944in}}{\pgfqpoint{4.114185in}{1.117900in}}%
\pgfpathcurveto{\pgfqpoint{4.119229in}{1.112857in}}{\pgfqpoint{4.126071in}{1.110023in}}{\pgfqpoint{4.133204in}{1.110023in}}%
\pgfpathclose%
\pgfusepath{stroke,fill}%
\end{pgfscope}%
\begin{pgfscope}%
\pgfpathrectangle{\pgfqpoint{2.867647in}{0.500000in}}{\pgfqpoint{1.764706in}{1.700000in}}%
\pgfusepath{clip}%
\pgfsetbuttcap%
\pgfsetroundjoin%
\definecolor{currentfill}{rgb}{0.975644,0.874038,0.797253}%
\pgfsetfillcolor{currentfill}%
\pgfsetlinewidth{0.311001pt}%
\definecolor{currentstroke}{rgb}{1.000000,1.000000,1.000000}%
\pgfsetstrokecolor{currentstroke}%
\pgfsetdash{}{0pt}%
\pgfpathmoveto{\pgfqpoint{4.093638in}{1.541129in}}%
\pgfpathcurveto{\pgfqpoint{4.100771in}{1.541129in}}{\pgfqpoint{4.107612in}{1.543962in}}{\pgfqpoint{4.112656in}{1.549006in}}%
\pgfpathcurveto{\pgfqpoint{4.117699in}{1.554050in}}{\pgfqpoint{4.120533in}{1.560891in}}{\pgfqpoint{4.120533in}{1.568024in}}%
\pgfpathcurveto{\pgfqpoint{4.120533in}{1.575157in}}{\pgfqpoint{4.117699in}{1.581999in}}{\pgfqpoint{4.112656in}{1.587042in}}%
\pgfpathcurveto{\pgfqpoint{4.107612in}{1.592086in}}{\pgfqpoint{4.100771in}{1.594920in}}{\pgfqpoint{4.093638in}{1.594920in}}%
\pgfpathcurveto{\pgfqpoint{4.086505in}{1.594920in}}{\pgfqpoint{4.079663in}{1.592086in}}{\pgfqpoint{4.074620in}{1.587042in}}%
\pgfpathcurveto{\pgfqpoint{4.069576in}{1.581999in}}{\pgfqpoint{4.066742in}{1.575157in}}{\pgfqpoint{4.066742in}{1.568024in}}%
\pgfpathcurveto{\pgfqpoint{4.066742in}{1.560891in}}{\pgfqpoint{4.069576in}{1.554050in}}{\pgfqpoint{4.074620in}{1.549006in}}%
\pgfpathcurveto{\pgfqpoint{4.079663in}{1.543962in}}{\pgfqpoint{4.086505in}{1.541129in}}{\pgfqpoint{4.093638in}{1.541129in}}%
\pgfpathclose%
\pgfusepath{stroke,fill}%
\end{pgfscope}%
\begin{pgfscope}%
\pgfpathrectangle{\pgfqpoint{2.867647in}{0.500000in}}{\pgfqpoint{1.764706in}{1.700000in}}%
\pgfusepath{clip}%
\pgfsetbuttcap%
\pgfsetroundjoin%
\definecolor{currentfill}{rgb}{0.979891,0.908948,0.848279}%
\pgfsetfillcolor{currentfill}%
\pgfsetlinewidth{0.311001pt}%
\definecolor{currentstroke}{rgb}{1.000000,1.000000,1.000000}%
\pgfsetstrokecolor{currentstroke}%
\pgfsetdash{}{0pt}%
\pgfpathmoveto{\pgfqpoint{4.150994in}{1.552612in}}%
\pgfpathcurveto{\pgfqpoint{4.158126in}{1.552612in}}{\pgfqpoint{4.164968in}{1.555446in}}{\pgfqpoint{4.170012in}{1.560490in}}%
\pgfpathcurveto{\pgfqpoint{4.175055in}{1.565533in}}{\pgfqpoint{4.177889in}{1.572375in}}{\pgfqpoint{4.177889in}{1.579508in}}%
\pgfpathcurveto{\pgfqpoint{4.177889in}{1.586641in}}{\pgfqpoint{4.175055in}{1.593482in}}{\pgfqpoint{4.170012in}{1.598526in}}%
\pgfpathcurveto{\pgfqpoint{4.164968in}{1.603570in}}{\pgfqpoint{4.158126in}{1.606404in}}{\pgfqpoint{4.150994in}{1.606404in}}%
\pgfpathcurveto{\pgfqpoint{4.143861in}{1.606404in}}{\pgfqpoint{4.137019in}{1.603570in}}{\pgfqpoint{4.131975in}{1.598526in}}%
\pgfpathcurveto{\pgfqpoint{4.126932in}{1.593482in}}{\pgfqpoint{4.124098in}{1.586641in}}{\pgfqpoint{4.124098in}{1.579508in}}%
\pgfpathcurveto{\pgfqpoint{4.124098in}{1.572375in}}{\pgfqpoint{4.126932in}{1.565533in}}{\pgfqpoint{4.131975in}{1.560490in}}%
\pgfpathcurveto{\pgfqpoint{4.137019in}{1.555446in}}{\pgfqpoint{4.143861in}{1.552612in}}{\pgfqpoint{4.150994in}{1.552612in}}%
\pgfpathclose%
\pgfusepath{stroke,fill}%
\end{pgfscope}%
\begin{pgfscope}%
\pgfpathrectangle{\pgfqpoint{2.867647in}{0.500000in}}{\pgfqpoint{1.764706in}{1.700000in}}%
\pgfusepath{clip}%
\pgfsetbuttcap%
\pgfsetroundjoin%
\definecolor{currentfill}{rgb}{0.967398,0.774513,0.650573}%
\pgfsetfillcolor{currentfill}%
\pgfsetlinewidth{0.311001pt}%
\definecolor{currentstroke}{rgb}{1.000000,1.000000,1.000000}%
\pgfsetstrokecolor{currentstroke}%
\pgfsetdash{}{0pt}%
\pgfpathmoveto{\pgfqpoint{4.246358in}{1.554495in}}%
\pgfpathcurveto{\pgfqpoint{4.253491in}{1.554495in}}{\pgfqpoint{4.260333in}{1.557329in}}{\pgfqpoint{4.265376in}{1.562372in}}%
\pgfpathcurveto{\pgfqpoint{4.270420in}{1.567416in}}{\pgfqpoint{4.273254in}{1.574258in}}{\pgfqpoint{4.273254in}{1.581391in}}%
\pgfpathcurveto{\pgfqpoint{4.273254in}{1.588523in}}{\pgfqpoint{4.270420in}{1.595365in}}{\pgfqpoint{4.265376in}{1.600409in}}%
\pgfpathcurveto{\pgfqpoint{4.260333in}{1.605452in}}{\pgfqpoint{4.253491in}{1.608286in}}{\pgfqpoint{4.246358in}{1.608286in}}%
\pgfpathcurveto{\pgfqpoint{4.239226in}{1.608286in}}{\pgfqpoint{4.232384in}{1.605452in}}{\pgfqpoint{4.227340in}{1.600409in}}%
\pgfpathcurveto{\pgfqpoint{4.222297in}{1.595365in}}{\pgfqpoint{4.219463in}{1.588523in}}{\pgfqpoint{4.219463in}{1.581391in}}%
\pgfpathcurveto{\pgfqpoint{4.219463in}{1.574258in}}{\pgfqpoint{4.222297in}{1.567416in}}{\pgfqpoint{4.227340in}{1.562372in}}%
\pgfpathcurveto{\pgfqpoint{4.232384in}{1.557329in}}{\pgfqpoint{4.239226in}{1.554495in}}{\pgfqpoint{4.246358in}{1.554495in}}%
\pgfpathclose%
\pgfusepath{stroke,fill}%
\end{pgfscope}%
\begin{pgfscope}%
\pgfpathrectangle{\pgfqpoint{2.867647in}{0.500000in}}{\pgfqpoint{1.764706in}{1.700000in}}%
\pgfusepath{clip}%
\pgfsetbuttcap%
\pgfsetroundjoin%
\definecolor{currentfill}{rgb}{0.975644,0.874038,0.797253}%
\pgfsetfillcolor{currentfill}%
\pgfsetlinewidth{0.311001pt}%
\definecolor{currentstroke}{rgb}{1.000000,1.000000,1.000000}%
\pgfsetstrokecolor{currentstroke}%
\pgfsetdash{}{0pt}%
\pgfpathmoveto{\pgfqpoint{4.095952in}{1.036228in}}%
\pgfpathcurveto{\pgfqpoint{4.103085in}{1.036228in}}{\pgfqpoint{4.109927in}{1.039062in}}{\pgfqpoint{4.114971in}{1.044106in}}%
\pgfpathcurveto{\pgfqpoint{4.120014in}{1.049149in}}{\pgfqpoint{4.122848in}{1.055991in}}{\pgfqpoint{4.122848in}{1.063124in}}%
\pgfpathcurveto{\pgfqpoint{4.122848in}{1.070257in}}{\pgfqpoint{4.120014in}{1.077098in}}{\pgfqpoint{4.114971in}{1.082142in}}%
\pgfpathcurveto{\pgfqpoint{4.109927in}{1.087186in}}{\pgfqpoint{4.103085in}{1.090020in}}{\pgfqpoint{4.095952in}{1.090020in}}%
\pgfpathcurveto{\pgfqpoint{4.088820in}{1.090020in}}{\pgfqpoint{4.081978in}{1.087186in}}{\pgfqpoint{4.076934in}{1.082142in}}%
\pgfpathcurveto{\pgfqpoint{4.071891in}{1.077098in}}{\pgfqpoint{4.069057in}{1.070257in}}{\pgfqpoint{4.069057in}{1.063124in}}%
\pgfpathcurveto{\pgfqpoint{4.069057in}{1.055991in}}{\pgfqpoint{4.071891in}{1.049149in}}{\pgfqpoint{4.076934in}{1.044106in}}%
\pgfpathcurveto{\pgfqpoint{4.081978in}{1.039062in}}{\pgfqpoint{4.088820in}{1.036228in}}{\pgfqpoint{4.095952in}{1.036228in}}%
\pgfpathclose%
\pgfusepath{stroke,fill}%
\end{pgfscope}%
\begin{pgfscope}%
\pgfpathrectangle{\pgfqpoint{2.867647in}{0.500000in}}{\pgfqpoint{1.764706in}{1.700000in}}%
\pgfusepath{clip}%
\pgfsetbuttcap%
\pgfsetroundjoin%
\definecolor{currentfill}{rgb}{0.980678,0.914765,0.856766}%
\pgfsetfillcolor{currentfill}%
\pgfsetlinewidth{0.311001pt}%
\definecolor{currentstroke}{rgb}{1.000000,1.000000,1.000000}%
\pgfsetstrokecolor{currentstroke}%
\pgfsetdash{}{0pt}%
\pgfpathmoveto{\pgfqpoint{4.184954in}{1.466175in}}%
\pgfpathcurveto{\pgfqpoint{4.192087in}{1.466175in}}{\pgfqpoint{4.198929in}{1.469009in}}{\pgfqpoint{4.203973in}{1.474052in}}%
\pgfpathcurveto{\pgfqpoint{4.209016in}{1.479096in}}{\pgfqpoint{4.211850in}{1.485938in}}{\pgfqpoint{4.211850in}{1.493071in}}%
\pgfpathcurveto{\pgfqpoint{4.211850in}{1.500203in}}{\pgfqpoint{4.209016in}{1.507045in}}{\pgfqpoint{4.203973in}{1.512089in}}%
\pgfpathcurveto{\pgfqpoint{4.198929in}{1.517132in}}{\pgfqpoint{4.192087in}{1.519966in}}{\pgfqpoint{4.184954in}{1.519966in}}%
\pgfpathcurveto{\pgfqpoint{4.177822in}{1.519966in}}{\pgfqpoint{4.170980in}{1.517132in}}{\pgfqpoint{4.165936in}{1.512089in}}%
\pgfpathcurveto{\pgfqpoint{4.160893in}{1.507045in}}{\pgfqpoint{4.158059in}{1.500203in}}{\pgfqpoint{4.158059in}{1.493071in}}%
\pgfpathcurveto{\pgfqpoint{4.158059in}{1.485938in}}{\pgfqpoint{4.160893in}{1.479096in}}{\pgfqpoint{4.165936in}{1.474052in}}%
\pgfpathcurveto{\pgfqpoint{4.170980in}{1.469009in}}{\pgfqpoint{4.177822in}{1.466175in}}{\pgfqpoint{4.184954in}{1.466175in}}%
\pgfpathclose%
\pgfusepath{stroke,fill}%
\end{pgfscope}%
\begin{pgfscope}%
\pgfpathrectangle{\pgfqpoint{2.867647in}{0.500000in}}{\pgfqpoint{1.764706in}{1.700000in}}%
\pgfusepath{clip}%
\pgfsetbuttcap%
\pgfsetroundjoin%
\definecolor{currentfill}{rgb}{0.971202,0.827364,0.728520}%
\pgfsetfillcolor{currentfill}%
\pgfsetlinewidth{0.311001pt}%
\definecolor{currentstroke}{rgb}{1.000000,1.000000,1.000000}%
\pgfsetstrokecolor{currentstroke}%
\pgfsetdash{}{0pt}%
\pgfpathmoveto{\pgfqpoint{4.113585in}{1.327941in}}%
\pgfpathcurveto{\pgfqpoint{4.120717in}{1.327941in}}{\pgfqpoint{4.127559in}{1.330775in}}{\pgfqpoint{4.132603in}{1.335819in}}%
\pgfpathcurveto{\pgfqpoint{4.137646in}{1.340862in}}{\pgfqpoint{4.140480in}{1.347704in}}{\pgfqpoint{4.140480in}{1.354837in}}%
\pgfpathcurveto{\pgfqpoint{4.140480in}{1.361970in}}{\pgfqpoint{4.137646in}{1.368811in}}{\pgfqpoint{4.132603in}{1.373855in}}%
\pgfpathcurveto{\pgfqpoint{4.127559in}{1.378899in}}{\pgfqpoint{4.120717in}{1.381733in}}{\pgfqpoint{4.113585in}{1.381733in}}%
\pgfpathcurveto{\pgfqpoint{4.106452in}{1.381733in}}{\pgfqpoint{4.099610in}{1.378899in}}{\pgfqpoint{4.094566in}{1.373855in}}%
\pgfpathcurveto{\pgfqpoint{4.089523in}{1.368811in}}{\pgfqpoint{4.086689in}{1.361970in}}{\pgfqpoint{4.086689in}{1.354837in}}%
\pgfpathcurveto{\pgfqpoint{4.086689in}{1.347704in}}{\pgfqpoint{4.089523in}{1.340862in}}{\pgfqpoint{4.094566in}{1.335819in}}%
\pgfpathcurveto{\pgfqpoint{4.099610in}{1.330775in}}{\pgfqpoint{4.106452in}{1.327941in}}{\pgfqpoint{4.113585in}{1.327941in}}%
\pgfpathclose%
\pgfusepath{stroke,fill}%
\end{pgfscope}%
\begin{pgfscope}%
\pgfpathrectangle{\pgfqpoint{2.867647in}{0.500000in}}{\pgfqpoint{1.764706in}{1.700000in}}%
\pgfusepath{clip}%
\pgfsetbuttcap%
\pgfsetroundjoin%
\definecolor{currentfill}{rgb}{0.967398,0.774513,0.650573}%
\pgfsetfillcolor{currentfill}%
\pgfsetlinewidth{0.311001pt}%
\definecolor{currentstroke}{rgb}{1.000000,1.000000,1.000000}%
\pgfsetstrokecolor{currentstroke}%
\pgfsetdash{}{0pt}%
\pgfpathmoveto{\pgfqpoint{4.276314in}{1.181712in}}%
\pgfpathcurveto{\pgfqpoint{4.283446in}{1.181712in}}{\pgfqpoint{4.290288in}{1.184546in}}{\pgfqpoint{4.295332in}{1.189590in}}%
\pgfpathcurveto{\pgfqpoint{4.300375in}{1.194634in}}{\pgfqpoint{4.303209in}{1.201475in}}{\pgfqpoint{4.303209in}{1.208608in}}%
\pgfpathcurveto{\pgfqpoint{4.303209in}{1.215741in}}{\pgfqpoint{4.300375in}{1.222583in}}{\pgfqpoint{4.295332in}{1.227626in}}%
\pgfpathcurveto{\pgfqpoint{4.290288in}{1.232670in}}{\pgfqpoint{4.283446in}{1.235504in}}{\pgfqpoint{4.276314in}{1.235504in}}%
\pgfpathcurveto{\pgfqpoint{4.269181in}{1.235504in}}{\pgfqpoint{4.262339in}{1.232670in}}{\pgfqpoint{4.257295in}{1.227626in}}%
\pgfpathcurveto{\pgfqpoint{4.252252in}{1.222583in}}{\pgfqpoint{4.249418in}{1.215741in}}{\pgfqpoint{4.249418in}{1.208608in}}%
\pgfpathcurveto{\pgfqpoint{4.249418in}{1.201475in}}{\pgfqpoint{4.252252in}{1.194634in}}{\pgfqpoint{4.257295in}{1.189590in}}%
\pgfpathcurveto{\pgfqpoint{4.262339in}{1.184546in}}{\pgfqpoint{4.269181in}{1.181712in}}{\pgfqpoint{4.276314in}{1.181712in}}%
\pgfpathclose%
\pgfusepath{stroke,fill}%
\end{pgfscope}%
\begin{pgfscope}%
\pgfpathrectangle{\pgfqpoint{2.867647in}{0.500000in}}{\pgfqpoint{1.764706in}{1.700000in}}%
\pgfusepath{clip}%
\pgfsetbuttcap%
\pgfsetroundjoin%
\definecolor{currentfill}{rgb}{0.980678,0.914765,0.856766}%
\pgfsetfillcolor{currentfill}%
\pgfsetlinewidth{0.311001pt}%
\definecolor{currentstroke}{rgb}{1.000000,1.000000,1.000000}%
\pgfsetstrokecolor{currentstroke}%
\pgfsetdash{}{0pt}%
\pgfpathmoveto{\pgfqpoint{4.212677in}{1.246193in}}%
\pgfpathcurveto{\pgfqpoint{4.219810in}{1.246193in}}{\pgfqpoint{4.226651in}{1.249027in}}{\pgfqpoint{4.231695in}{1.254071in}}%
\pgfpathcurveto{\pgfqpoint{4.236739in}{1.259115in}}{\pgfqpoint{4.239573in}{1.265956in}}{\pgfqpoint{4.239573in}{1.273089in}}%
\pgfpathcurveto{\pgfqpoint{4.239573in}{1.280222in}}{\pgfqpoint{4.236739in}{1.287064in}}{\pgfqpoint{4.231695in}{1.292107in}}%
\pgfpathcurveto{\pgfqpoint{4.226651in}{1.297151in}}{\pgfqpoint{4.219810in}{1.299985in}}{\pgfqpoint{4.212677in}{1.299985in}}%
\pgfpathcurveto{\pgfqpoint{4.205544in}{1.299985in}}{\pgfqpoint{4.198703in}{1.297151in}}{\pgfqpoint{4.193659in}{1.292107in}}%
\pgfpathcurveto{\pgfqpoint{4.188615in}{1.287064in}}{\pgfqpoint{4.185781in}{1.280222in}}{\pgfqpoint{4.185781in}{1.273089in}}%
\pgfpathcurveto{\pgfqpoint{4.185781in}{1.265956in}}{\pgfqpoint{4.188615in}{1.259115in}}{\pgfqpoint{4.193659in}{1.254071in}}%
\pgfpathcurveto{\pgfqpoint{4.198703in}{1.249027in}}{\pgfqpoint{4.205544in}{1.246193in}}{\pgfqpoint{4.212677in}{1.246193in}}%
\pgfpathclose%
\pgfusepath{stroke,fill}%
\end{pgfscope}%
\begin{pgfscope}%
\pgfpathrectangle{\pgfqpoint{2.867647in}{0.500000in}}{\pgfqpoint{1.764706in}{1.700000in}}%
\pgfusepath{clip}%
\pgfsetbuttcap%
\pgfsetroundjoin%
\definecolor{currentfill}{rgb}{0.970255,0.815666,0.711203}%
\pgfsetfillcolor{currentfill}%
\pgfsetlinewidth{0.311001pt}%
\definecolor{currentstroke}{rgb}{1.000000,1.000000,1.000000}%
\pgfsetstrokecolor{currentstroke}%
\pgfsetdash{}{0pt}%
\pgfpathmoveto{\pgfqpoint{4.269994in}{1.391124in}}%
\pgfpathcurveto{\pgfqpoint{4.277127in}{1.391124in}}{\pgfqpoint{4.283968in}{1.393958in}}{\pgfqpoint{4.289012in}{1.399002in}}%
\pgfpathcurveto{\pgfqpoint{4.294056in}{1.404045in}}{\pgfqpoint{4.296890in}{1.410887in}}{\pgfqpoint{4.296890in}{1.418020in}}%
\pgfpathcurveto{\pgfqpoint{4.296890in}{1.425153in}}{\pgfqpoint{4.294056in}{1.431994in}}{\pgfqpoint{4.289012in}{1.437038in}}%
\pgfpathcurveto{\pgfqpoint{4.283968in}{1.442082in}}{\pgfqpoint{4.277127in}{1.444916in}}{\pgfqpoint{4.269994in}{1.444916in}}%
\pgfpathcurveto{\pgfqpoint{4.262861in}{1.444916in}}{\pgfqpoint{4.256019in}{1.442082in}}{\pgfqpoint{4.250976in}{1.437038in}}%
\pgfpathcurveto{\pgfqpoint{4.245932in}{1.431994in}}{\pgfqpoint{4.243098in}{1.425153in}}{\pgfqpoint{4.243098in}{1.418020in}}%
\pgfpathcurveto{\pgfqpoint{4.243098in}{1.410887in}}{\pgfqpoint{4.245932in}{1.404045in}}{\pgfqpoint{4.250976in}{1.399002in}}%
\pgfpathcurveto{\pgfqpoint{4.256019in}{1.393958in}}{\pgfqpoint{4.262861in}{1.391124in}}{\pgfqpoint{4.269994in}{1.391124in}}%
\pgfpathclose%
\pgfusepath{stroke,fill}%
\end{pgfscope}%
\begin{pgfscope}%
\pgfpathrectangle{\pgfqpoint{2.867647in}{0.500000in}}{\pgfqpoint{1.764706in}{1.700000in}}%
\pgfusepath{clip}%
\pgfsetbuttcap%
\pgfsetroundjoin%
\definecolor{currentfill}{rgb}{0.965302,0.713942,0.568499}%
\pgfsetfillcolor{currentfill}%
\pgfsetlinewidth{0.311001pt}%
\definecolor{currentstroke}{rgb}{1.000000,1.000000,1.000000}%
\pgfsetstrokecolor{currentstroke}%
\pgfsetdash{}{0pt}%
\pgfpathmoveto{\pgfqpoint{4.064166in}{1.415700in}}%
\pgfpathcurveto{\pgfqpoint{4.071299in}{1.415700in}}{\pgfqpoint{4.078141in}{1.418534in}}{\pgfqpoint{4.083184in}{1.423577in}}%
\pgfpathcurveto{\pgfqpoint{4.088228in}{1.428621in}}{\pgfqpoint{4.091062in}{1.435463in}}{\pgfqpoint{4.091062in}{1.442596in}}%
\pgfpathcurveto{\pgfqpoint{4.091062in}{1.449728in}}{\pgfqpoint{4.088228in}{1.456570in}}{\pgfqpoint{4.083184in}{1.461614in}}%
\pgfpathcurveto{\pgfqpoint{4.078141in}{1.466657in}}{\pgfqpoint{4.071299in}{1.469491in}}{\pgfqpoint{4.064166in}{1.469491in}}%
\pgfpathcurveto{\pgfqpoint{4.057033in}{1.469491in}}{\pgfqpoint{4.050192in}{1.466657in}}{\pgfqpoint{4.045148in}{1.461614in}}%
\pgfpathcurveto{\pgfqpoint{4.040104in}{1.456570in}}{\pgfqpoint{4.037271in}{1.449728in}}{\pgfqpoint{4.037271in}{1.442596in}}%
\pgfpathcurveto{\pgfqpoint{4.037271in}{1.435463in}}{\pgfqpoint{4.040104in}{1.428621in}}{\pgfqpoint{4.045148in}{1.423577in}}%
\pgfpathcurveto{\pgfqpoint{4.050192in}{1.418534in}}{\pgfqpoint{4.057033in}{1.415700in}}{\pgfqpoint{4.064166in}{1.415700in}}%
\pgfpathclose%
\pgfusepath{stroke,fill}%
\end{pgfscope}%
\begin{pgfscope}%
\pgfpathrectangle{\pgfqpoint{2.867647in}{0.500000in}}{\pgfqpoint{1.764706in}{1.700000in}}%
\pgfusepath{clip}%
\pgfsetbuttcap%
\pgfsetroundjoin%
\definecolor{currentfill}{rgb}{0.980678,0.914765,0.856766}%
\pgfsetfillcolor{currentfill}%
\pgfsetlinewidth{0.311001pt}%
\definecolor{currentstroke}{rgb}{1.000000,1.000000,1.000000}%
\pgfsetstrokecolor{currentstroke}%
\pgfsetdash{}{0pt}%
\pgfpathmoveto{\pgfqpoint{4.180872in}{1.401568in}}%
\pgfpathcurveto{\pgfqpoint{4.188005in}{1.401568in}}{\pgfqpoint{4.194847in}{1.404402in}}{\pgfqpoint{4.199891in}{1.409446in}}%
\pgfpathcurveto{\pgfqpoint{4.204934in}{1.414490in}}{\pgfqpoint{4.207768in}{1.421331in}}{\pgfqpoint{4.207768in}{1.428464in}}%
\pgfpathcurveto{\pgfqpoint{4.207768in}{1.435597in}}{\pgfqpoint{4.204934in}{1.442439in}}{\pgfqpoint{4.199891in}{1.447482in}}%
\pgfpathcurveto{\pgfqpoint{4.194847in}{1.452526in}}{\pgfqpoint{4.188005in}{1.455360in}}{\pgfqpoint{4.180872in}{1.455360in}}%
\pgfpathcurveto{\pgfqpoint{4.173740in}{1.455360in}}{\pgfqpoint{4.166898in}{1.452526in}}{\pgfqpoint{4.161854in}{1.447482in}}%
\pgfpathcurveto{\pgfqpoint{4.156811in}{1.442439in}}{\pgfqpoint{4.153977in}{1.435597in}}{\pgfqpoint{4.153977in}{1.428464in}}%
\pgfpathcurveto{\pgfqpoint{4.153977in}{1.421331in}}{\pgfqpoint{4.156811in}{1.414490in}}{\pgfqpoint{4.161854in}{1.409446in}}%
\pgfpathcurveto{\pgfqpoint{4.166898in}{1.404402in}}{\pgfqpoint{4.173740in}{1.401568in}}{\pgfqpoint{4.180872in}{1.401568in}}%
\pgfpathclose%
\pgfusepath{stroke,fill}%
\end{pgfscope}%
\begin{pgfscope}%
\pgfpathrectangle{\pgfqpoint{2.867647in}{0.500000in}}{\pgfqpoint{1.764706in}{1.700000in}}%
\pgfusepath{clip}%
\pgfsetbuttcap%
\pgfsetroundjoin%
\definecolor{currentfill}{rgb}{0.960421,0.553286,0.393191}%
\pgfsetfillcolor{currentfill}%
\pgfsetlinewidth{0.311001pt}%
\definecolor{currentstroke}{rgb}{1.000000,1.000000,1.000000}%
\pgfsetstrokecolor{currentstroke}%
\pgfsetdash{}{0pt}%
\pgfpathmoveto{\pgfqpoint{4.012238in}{0.835984in}}%
\pgfpathcurveto{\pgfqpoint{4.019371in}{0.835984in}}{\pgfqpoint{4.026212in}{0.838818in}}{\pgfqpoint{4.031256in}{0.843862in}}%
\pgfpathcurveto{\pgfqpoint{4.036300in}{0.848905in}}{\pgfqpoint{4.039134in}{0.855747in}}{\pgfqpoint{4.039134in}{0.862880in}}%
\pgfpathcurveto{\pgfqpoint{4.039134in}{0.870013in}}{\pgfqpoint{4.036300in}{0.876854in}}{\pgfqpoint{4.031256in}{0.881898in}}%
\pgfpathcurveto{\pgfqpoint{4.026212in}{0.886942in}}{\pgfqpoint{4.019371in}{0.889776in}}{\pgfqpoint{4.012238in}{0.889776in}}%
\pgfpathcurveto{\pgfqpoint{4.005105in}{0.889776in}}{\pgfqpoint{3.998264in}{0.886942in}}{\pgfqpoint{3.993220in}{0.881898in}}%
\pgfpathcurveto{\pgfqpoint{3.988176in}{0.876854in}}{\pgfqpoint{3.985342in}{0.870013in}}{\pgfqpoint{3.985342in}{0.862880in}}%
\pgfpathcurveto{\pgfqpoint{3.985342in}{0.855747in}}{\pgfqpoint{3.988176in}{0.848905in}}{\pgfqpoint{3.993220in}{0.843862in}}%
\pgfpathcurveto{\pgfqpoint{3.998264in}{0.838818in}}{\pgfqpoint{4.005105in}{0.835984in}}{\pgfqpoint{4.012238in}{0.835984in}}%
\pgfpathclose%
\pgfusepath{stroke,fill}%
\end{pgfscope}%
\begin{pgfscope}%
\pgfpathrectangle{\pgfqpoint{2.867647in}{0.500000in}}{\pgfqpoint{1.764706in}{1.700000in}}%
\pgfusepath{clip}%
\pgfsetbuttcap%
\pgfsetroundjoin%
\definecolor{currentfill}{rgb}{0.978376,0.897317,0.831308}%
\pgfsetfillcolor{currentfill}%
\pgfsetlinewidth{0.311001pt}%
\definecolor{currentstroke}{rgb}{1.000000,1.000000,1.000000}%
\pgfsetstrokecolor{currentstroke}%
\pgfsetdash{}{0pt}%
\pgfpathmoveto{\pgfqpoint{4.111476in}{1.537800in}}%
\pgfpathcurveto{\pgfqpoint{4.118608in}{1.537800in}}{\pgfqpoint{4.125450in}{1.540634in}}{\pgfqpoint{4.130494in}{1.545677in}}%
\pgfpathcurveto{\pgfqpoint{4.135537in}{1.550721in}}{\pgfqpoint{4.138371in}{1.557563in}}{\pgfqpoint{4.138371in}{1.564695in}}%
\pgfpathcurveto{\pgfqpoint{4.138371in}{1.571828in}}{\pgfqpoint{4.135537in}{1.578670in}}{\pgfqpoint{4.130494in}{1.583714in}}%
\pgfpathcurveto{\pgfqpoint{4.125450in}{1.588757in}}{\pgfqpoint{4.118608in}{1.591591in}}{\pgfqpoint{4.111476in}{1.591591in}}%
\pgfpathcurveto{\pgfqpoint{4.104343in}{1.591591in}}{\pgfqpoint{4.097501in}{1.588757in}}{\pgfqpoint{4.092457in}{1.583714in}}%
\pgfpathcurveto{\pgfqpoint{4.087414in}{1.578670in}}{\pgfqpoint{4.084580in}{1.571828in}}{\pgfqpoint{4.084580in}{1.564695in}}%
\pgfpathcurveto{\pgfqpoint{4.084580in}{1.557563in}}{\pgfqpoint{4.087414in}{1.550721in}}{\pgfqpoint{4.092457in}{1.545677in}}%
\pgfpathcurveto{\pgfqpoint{4.097501in}{1.540634in}}{\pgfqpoint{4.104343in}{1.537800in}}{\pgfqpoint{4.111476in}{1.537800in}}%
\pgfpathclose%
\pgfusepath{stroke,fill}%
\end{pgfscope}%
\begin{pgfscope}%
\pgfpathrectangle{\pgfqpoint{2.867647in}{0.500000in}}{\pgfqpoint{1.764706in}{1.700000in}}%
\pgfusepath{clip}%
\pgfsetbuttcap%
\pgfsetroundjoin%
\definecolor{currentfill}{rgb}{0.969803,0.809811,0.702523}%
\pgfsetfillcolor{currentfill}%
\pgfsetlinewidth{0.311001pt}%
\definecolor{currentstroke}{rgb}{1.000000,1.000000,1.000000}%
\pgfsetstrokecolor{currentstroke}%
\pgfsetdash{}{0pt}%
\pgfpathmoveto{\pgfqpoint{4.264296in}{1.189747in}}%
\pgfpathcurveto{\pgfqpoint{4.271429in}{1.189747in}}{\pgfqpoint{4.278271in}{1.192581in}}{\pgfqpoint{4.283314in}{1.197624in}}%
\pgfpathcurveto{\pgfqpoint{4.288358in}{1.202668in}}{\pgfqpoint{4.291192in}{1.209510in}}{\pgfqpoint{4.291192in}{1.216642in}}%
\pgfpathcurveto{\pgfqpoint{4.291192in}{1.223775in}}{\pgfqpoint{4.288358in}{1.230617in}}{\pgfqpoint{4.283314in}{1.235661in}}%
\pgfpathcurveto{\pgfqpoint{4.278271in}{1.240704in}}{\pgfqpoint{4.271429in}{1.243538in}}{\pgfqpoint{4.264296in}{1.243538in}}%
\pgfpathcurveto{\pgfqpoint{4.257163in}{1.243538in}}{\pgfqpoint{4.250322in}{1.240704in}}{\pgfqpoint{4.245278in}{1.235661in}}%
\pgfpathcurveto{\pgfqpoint{4.240234in}{1.230617in}}{\pgfqpoint{4.237400in}{1.223775in}}{\pgfqpoint{4.237400in}{1.216642in}}%
\pgfpathcurveto{\pgfqpoint{4.237400in}{1.209510in}}{\pgfqpoint{4.240234in}{1.202668in}}{\pgfqpoint{4.245278in}{1.197624in}}%
\pgfpathcurveto{\pgfqpoint{4.250322in}{1.192581in}}{\pgfqpoint{4.257163in}{1.189747in}}{\pgfqpoint{4.264296in}{1.189747in}}%
\pgfpathclose%
\pgfusepath{stroke,fill}%
\end{pgfscope}%
\begin{pgfscope}%
\pgfpathrectangle{\pgfqpoint{2.867647in}{0.500000in}}{\pgfqpoint{1.764706in}{1.700000in}}%
\pgfusepath{clip}%
\pgfsetbuttcap%
\pgfsetroundjoin%
\definecolor{currentfill}{rgb}{0.977657,0.891500,0.822809}%
\pgfsetfillcolor{currentfill}%
\pgfsetlinewidth{0.311001pt}%
\definecolor{currentstroke}{rgb}{1.000000,1.000000,1.000000}%
\pgfsetstrokecolor{currentstroke}%
\pgfsetdash{}{0pt}%
\pgfpathmoveto{\pgfqpoint{4.130499in}{1.463710in}}%
\pgfpathcurveto{\pgfqpoint{4.137632in}{1.463710in}}{\pgfqpoint{4.144474in}{1.466544in}}{\pgfqpoint{4.149517in}{1.471588in}}%
\pgfpathcurveto{\pgfqpoint{4.154561in}{1.476632in}}{\pgfqpoint{4.157395in}{1.483473in}}{\pgfqpoint{4.157395in}{1.490606in}}%
\pgfpathcurveto{\pgfqpoint{4.157395in}{1.497739in}}{\pgfqpoint{4.154561in}{1.504581in}}{\pgfqpoint{4.149517in}{1.509624in}}%
\pgfpathcurveto{\pgfqpoint{4.144474in}{1.514668in}}{\pgfqpoint{4.137632in}{1.517502in}}{\pgfqpoint{4.130499in}{1.517502in}}%
\pgfpathcurveto{\pgfqpoint{4.123366in}{1.517502in}}{\pgfqpoint{4.116525in}{1.514668in}}{\pgfqpoint{4.111481in}{1.509624in}}%
\pgfpathcurveto{\pgfqpoint{4.106437in}{1.504581in}}{\pgfqpoint{4.103603in}{1.497739in}}{\pgfqpoint{4.103603in}{1.490606in}}%
\pgfpathcurveto{\pgfqpoint{4.103603in}{1.483473in}}{\pgfqpoint{4.106437in}{1.476632in}}{\pgfqpoint{4.111481in}{1.471588in}}%
\pgfpathcurveto{\pgfqpoint{4.116525in}{1.466544in}}{\pgfqpoint{4.123366in}{1.463710in}}{\pgfqpoint{4.130499in}{1.463710in}}%
\pgfpathclose%
\pgfusepath{stroke,fill}%
\end{pgfscope}%
\begin{pgfscope}%
\pgfpathrectangle{\pgfqpoint{2.867647in}{0.500000in}}{\pgfqpoint{1.764706in}{1.700000in}}%
\pgfusepath{clip}%
\pgfsetbuttcap%
\pgfsetroundjoin%
\definecolor{currentfill}{rgb}{0.970255,0.815666,0.711203}%
\pgfsetfillcolor{currentfill}%
\pgfsetlinewidth{0.311001pt}%
\definecolor{currentstroke}{rgb}{1.000000,1.000000,1.000000}%
\pgfsetstrokecolor{currentstroke}%
\pgfsetdash{}{0pt}%
\pgfpathmoveto{\pgfqpoint{4.268697in}{1.406747in}}%
\pgfpathcurveto{\pgfqpoint{4.275830in}{1.406747in}}{\pgfqpoint{4.282672in}{1.409581in}}{\pgfqpoint{4.287715in}{1.414624in}}%
\pgfpathcurveto{\pgfqpoint{4.292759in}{1.419668in}}{\pgfqpoint{4.295593in}{1.426510in}}{\pgfqpoint{4.295593in}{1.433643in}}%
\pgfpathcurveto{\pgfqpoint{4.295593in}{1.440775in}}{\pgfqpoint{4.292759in}{1.447617in}}{\pgfqpoint{4.287715in}{1.452661in}}%
\pgfpathcurveto{\pgfqpoint{4.282672in}{1.457704in}}{\pgfqpoint{4.275830in}{1.460538in}}{\pgfqpoint{4.268697in}{1.460538in}}%
\pgfpathcurveto{\pgfqpoint{4.261564in}{1.460538in}}{\pgfqpoint{4.254723in}{1.457704in}}{\pgfqpoint{4.249679in}{1.452661in}}%
\pgfpathcurveto{\pgfqpoint{4.244635in}{1.447617in}}{\pgfqpoint{4.241802in}{1.440775in}}{\pgfqpoint{4.241802in}{1.433643in}}%
\pgfpathcurveto{\pgfqpoint{4.241802in}{1.426510in}}{\pgfqpoint{4.244635in}{1.419668in}}{\pgfqpoint{4.249679in}{1.414624in}}%
\pgfpathcurveto{\pgfqpoint{4.254723in}{1.409581in}}{\pgfqpoint{4.261564in}{1.406747in}}{\pgfqpoint{4.268697in}{1.406747in}}%
\pgfpathclose%
\pgfusepath{stroke,fill}%
\end{pgfscope}%
\begin{pgfscope}%
\pgfpathrectangle{\pgfqpoint{2.867647in}{0.500000in}}{\pgfqpoint{1.764706in}{1.700000in}}%
\pgfusepath{clip}%
\pgfsetbuttcap%
\pgfsetroundjoin%
\definecolor{currentfill}{rgb}{0.971202,0.827364,0.728520}%
\pgfsetfillcolor{currentfill}%
\pgfsetlinewidth{0.311001pt}%
\definecolor{currentstroke}{rgb}{1.000000,1.000000,1.000000}%
\pgfsetstrokecolor{currentstroke}%
\pgfsetdash{}{0pt}%
\pgfpathmoveto{\pgfqpoint{4.260257in}{1.422128in}}%
\pgfpathcurveto{\pgfqpoint{4.267390in}{1.422128in}}{\pgfqpoint{4.274231in}{1.424962in}}{\pgfqpoint{4.279275in}{1.430006in}}%
\pgfpathcurveto{\pgfqpoint{4.284319in}{1.435050in}}{\pgfqpoint{4.287153in}{1.441891in}}{\pgfqpoint{4.287153in}{1.449024in}}%
\pgfpathcurveto{\pgfqpoint{4.287153in}{1.456157in}}{\pgfqpoint{4.284319in}{1.462998in}}{\pgfqpoint{4.279275in}{1.468042in}}%
\pgfpathcurveto{\pgfqpoint{4.274231in}{1.473086in}}{\pgfqpoint{4.267390in}{1.475920in}}{\pgfqpoint{4.260257in}{1.475920in}}%
\pgfpathcurveto{\pgfqpoint{4.253124in}{1.475920in}}{\pgfqpoint{4.246282in}{1.473086in}}{\pgfqpoint{4.241239in}{1.468042in}}%
\pgfpathcurveto{\pgfqpoint{4.236195in}{1.462998in}}{\pgfqpoint{4.233361in}{1.456157in}}{\pgfqpoint{4.233361in}{1.449024in}}%
\pgfpathcurveto{\pgfqpoint{4.233361in}{1.441891in}}{\pgfqpoint{4.236195in}{1.435050in}}{\pgfqpoint{4.241239in}{1.430006in}}%
\pgfpathcurveto{\pgfqpoint{4.246282in}{1.424962in}}{\pgfqpoint{4.253124in}{1.422128in}}{\pgfqpoint{4.260257in}{1.422128in}}%
\pgfpathclose%
\pgfusepath{stroke,fill}%
\end{pgfscope}%
\begin{pgfscope}%
\pgfpathrectangle{\pgfqpoint{2.867647in}{0.500000in}}{\pgfqpoint{1.764706in}{1.700000in}}%
\pgfusepath{clip}%
\pgfsetbuttcap%
\pgfsetroundjoin%
\definecolor{currentfill}{rgb}{0.965592,0.726236,0.584384}%
\pgfsetfillcolor{currentfill}%
\pgfsetlinewidth{0.311001pt}%
\definecolor{currentstroke}{rgb}{1.000000,1.000000,1.000000}%
\pgfsetstrokecolor{currentstroke}%
\pgfsetdash{}{0pt}%
\pgfpathmoveto{\pgfqpoint{4.079543in}{1.287999in}}%
\pgfpathcurveto{\pgfqpoint{4.086675in}{1.287999in}}{\pgfqpoint{4.093517in}{1.290833in}}{\pgfqpoint{4.098561in}{1.295877in}}%
\pgfpathcurveto{\pgfqpoint{4.103604in}{1.300921in}}{\pgfqpoint{4.106438in}{1.307762in}}{\pgfqpoint{4.106438in}{1.314895in}}%
\pgfpathcurveto{\pgfqpoint{4.106438in}{1.322028in}}{\pgfqpoint{4.103604in}{1.328870in}}{\pgfqpoint{4.098561in}{1.333913in}}%
\pgfpathcurveto{\pgfqpoint{4.093517in}{1.338957in}}{\pgfqpoint{4.086675in}{1.341791in}}{\pgfqpoint{4.079543in}{1.341791in}}%
\pgfpathcurveto{\pgfqpoint{4.072410in}{1.341791in}}{\pgfqpoint{4.065568in}{1.338957in}}{\pgfqpoint{4.060524in}{1.333913in}}%
\pgfpathcurveto{\pgfqpoint{4.055481in}{1.328870in}}{\pgfqpoint{4.052647in}{1.322028in}}{\pgfqpoint{4.052647in}{1.314895in}}%
\pgfpathcurveto{\pgfqpoint{4.052647in}{1.307762in}}{\pgfqpoint{4.055481in}{1.300921in}}{\pgfqpoint{4.060524in}{1.295877in}}%
\pgfpathcurveto{\pgfqpoint{4.065568in}{1.290833in}}{\pgfqpoint{4.072410in}{1.287999in}}{\pgfqpoint{4.079543in}{1.287999in}}%
\pgfpathclose%
\pgfusepath{stroke,fill}%
\end{pgfscope}%
\begin{pgfscope}%
\pgfpathrectangle{\pgfqpoint{2.867647in}{0.500000in}}{\pgfqpoint{1.764706in}{1.700000in}}%
\pgfusepath{clip}%
\pgfsetbuttcap%
\pgfsetroundjoin%
\definecolor{currentfill}{rgb}{0.976287,0.879862,0.805788}%
\pgfsetfillcolor{currentfill}%
\pgfsetlinewidth{0.311001pt}%
\definecolor{currentstroke}{rgb}{1.000000,1.000000,1.000000}%
\pgfsetstrokecolor{currentstroke}%
\pgfsetdash{}{0pt}%
\pgfpathmoveto{\pgfqpoint{4.112560in}{1.486369in}}%
\pgfpathcurveto{\pgfqpoint{4.119692in}{1.486369in}}{\pgfqpoint{4.126534in}{1.489203in}}{\pgfqpoint{4.131578in}{1.494247in}}%
\pgfpathcurveto{\pgfqpoint{4.136621in}{1.499290in}}{\pgfqpoint{4.139455in}{1.506132in}}{\pgfqpoint{4.139455in}{1.513265in}}%
\pgfpathcurveto{\pgfqpoint{4.139455in}{1.520398in}}{\pgfqpoint{4.136621in}{1.527239in}}{\pgfqpoint{4.131578in}{1.532283in}}%
\pgfpathcurveto{\pgfqpoint{4.126534in}{1.537327in}}{\pgfqpoint{4.119692in}{1.540161in}}{\pgfqpoint{4.112560in}{1.540161in}}%
\pgfpathcurveto{\pgfqpoint{4.105427in}{1.540161in}}{\pgfqpoint{4.098585in}{1.537327in}}{\pgfqpoint{4.093541in}{1.532283in}}%
\pgfpathcurveto{\pgfqpoint{4.088498in}{1.527239in}}{\pgfqpoint{4.085664in}{1.520398in}}{\pgfqpoint{4.085664in}{1.513265in}}%
\pgfpathcurveto{\pgfqpoint{4.085664in}{1.506132in}}{\pgfqpoint{4.088498in}{1.499290in}}{\pgfqpoint{4.093541in}{1.494247in}}%
\pgfpathcurveto{\pgfqpoint{4.098585in}{1.489203in}}{\pgfqpoint{4.105427in}{1.486369in}}{\pgfqpoint{4.112560in}{1.486369in}}%
\pgfpathclose%
\pgfusepath{stroke,fill}%
\end{pgfscope}%
\begin{pgfscope}%
\pgfpathrectangle{\pgfqpoint{2.867647in}{0.500000in}}{\pgfqpoint{1.764706in}{1.700000in}}%
\pgfusepath{clip}%
\pgfsetbuttcap%
\pgfsetroundjoin%
\definecolor{currentfill}{rgb}{0.976287,0.879862,0.805788}%
\pgfsetfillcolor{currentfill}%
\pgfsetlinewidth{0.311001pt}%
\definecolor{currentstroke}{rgb}{1.000000,1.000000,1.000000}%
\pgfsetstrokecolor{currentstroke}%
\pgfsetdash{}{0pt}%
\pgfpathmoveto{\pgfqpoint{4.138274in}{1.275374in}}%
\pgfpathcurveto{\pgfqpoint{4.145407in}{1.275374in}}{\pgfqpoint{4.152249in}{1.278208in}}{\pgfqpoint{4.157293in}{1.283252in}}%
\pgfpathcurveto{\pgfqpoint{4.162336in}{1.288296in}}{\pgfqpoint{4.165170in}{1.295137in}}{\pgfqpoint{4.165170in}{1.302270in}}%
\pgfpathcurveto{\pgfqpoint{4.165170in}{1.309403in}}{\pgfqpoint{4.162336in}{1.316245in}}{\pgfqpoint{4.157293in}{1.321288in}}%
\pgfpathcurveto{\pgfqpoint{4.152249in}{1.326332in}}{\pgfqpoint{4.145407in}{1.329166in}}{\pgfqpoint{4.138274in}{1.329166in}}%
\pgfpathcurveto{\pgfqpoint{4.131142in}{1.329166in}}{\pgfqpoint{4.124300in}{1.326332in}}{\pgfqpoint{4.119256in}{1.321288in}}%
\pgfpathcurveto{\pgfqpoint{4.114213in}{1.316245in}}{\pgfqpoint{4.111379in}{1.309403in}}{\pgfqpoint{4.111379in}{1.302270in}}%
\pgfpathcurveto{\pgfqpoint{4.111379in}{1.295137in}}{\pgfqpoint{4.114213in}{1.288296in}}{\pgfqpoint{4.119256in}{1.283252in}}%
\pgfpathcurveto{\pgfqpoint{4.124300in}{1.278208in}}{\pgfqpoint{4.131142in}{1.275374in}}{\pgfqpoint{4.138274in}{1.275374in}}%
\pgfpathclose%
\pgfusepath{stroke,fill}%
\end{pgfscope}%
\begin{pgfscope}%
\pgfpathrectangle{\pgfqpoint{2.867647in}{0.500000in}}{\pgfqpoint{1.764706in}{1.700000in}}%
\pgfusepath{clip}%
\pgfsetbuttcap%
\pgfsetroundjoin%
\definecolor{currentfill}{rgb}{0.974412,0.862387,0.780156}%
\pgfsetfillcolor{currentfill}%
\pgfsetlinewidth{0.311001pt}%
\definecolor{currentstroke}{rgb}{1.000000,1.000000,1.000000}%
\pgfsetstrokecolor{currentstroke}%
\pgfsetdash{}{0pt}%
\pgfpathmoveto{\pgfqpoint{4.127956in}{1.363023in}}%
\pgfpathcurveto{\pgfqpoint{4.135088in}{1.363023in}}{\pgfqpoint{4.141930in}{1.365857in}}{\pgfqpoint{4.146974in}{1.370900in}}%
\pgfpathcurveto{\pgfqpoint{4.152017in}{1.375944in}}{\pgfqpoint{4.154851in}{1.382785in}}{\pgfqpoint{4.154851in}{1.389918in}}%
\pgfpathcurveto{\pgfqpoint{4.154851in}{1.397051in}}{\pgfqpoint{4.152017in}{1.403893in}}{\pgfqpoint{4.146974in}{1.408936in}}%
\pgfpathcurveto{\pgfqpoint{4.141930in}{1.413980in}}{\pgfqpoint{4.135088in}{1.416814in}}{\pgfqpoint{4.127956in}{1.416814in}}%
\pgfpathcurveto{\pgfqpoint{4.120823in}{1.416814in}}{\pgfqpoint{4.113981in}{1.413980in}}{\pgfqpoint{4.108937in}{1.408936in}}%
\pgfpathcurveto{\pgfqpoint{4.103894in}{1.403893in}}{\pgfqpoint{4.101060in}{1.397051in}}{\pgfqpoint{4.101060in}{1.389918in}}%
\pgfpathcurveto{\pgfqpoint{4.101060in}{1.382785in}}{\pgfqpoint{4.103894in}{1.375944in}}{\pgfqpoint{4.108937in}{1.370900in}}%
\pgfpathcurveto{\pgfqpoint{4.113981in}{1.365857in}}{\pgfqpoint{4.120823in}{1.363023in}}{\pgfqpoint{4.127956in}{1.363023in}}%
\pgfpathclose%
\pgfusepath{stroke,fill}%
\end{pgfscope}%
\begin{pgfscope}%
\pgfpathrectangle{\pgfqpoint{2.867647in}{0.500000in}}{\pgfqpoint{1.764706in}{1.700000in}}%
\pgfusepath{clip}%
\pgfsetbuttcap%
\pgfsetroundjoin%
\definecolor{currentfill}{rgb}{0.965169,0.707764,0.560659}%
\pgfsetfillcolor{currentfill}%
\pgfsetlinewidth{0.311001pt}%
\definecolor{currentstroke}{rgb}{1.000000,1.000000,1.000000}%
\pgfsetstrokecolor{currentstroke}%
\pgfsetdash{}{0pt}%
\pgfpathmoveto{\pgfqpoint{3.989662in}{1.711756in}}%
\pgfpathcurveto{\pgfqpoint{3.996794in}{1.711756in}}{\pgfqpoint{4.003636in}{1.714590in}}{\pgfqpoint{4.008680in}{1.719633in}}%
\pgfpathcurveto{\pgfqpoint{4.013723in}{1.724677in}}{\pgfqpoint{4.016557in}{1.731519in}}{\pgfqpoint{4.016557in}{1.738652in}}%
\pgfpathcurveto{\pgfqpoint{4.016557in}{1.745784in}}{\pgfqpoint{4.013723in}{1.752626in}}{\pgfqpoint{4.008680in}{1.757670in}}%
\pgfpathcurveto{\pgfqpoint{4.003636in}{1.762713in}}{\pgfqpoint{3.996794in}{1.765547in}}{\pgfqpoint{3.989662in}{1.765547in}}%
\pgfpathcurveto{\pgfqpoint{3.982529in}{1.765547in}}{\pgfqpoint{3.975687in}{1.762713in}}{\pgfqpoint{3.970643in}{1.757670in}}%
\pgfpathcurveto{\pgfqpoint{3.965600in}{1.752626in}}{\pgfqpoint{3.962766in}{1.745784in}}{\pgfqpoint{3.962766in}{1.738652in}}%
\pgfpathcurveto{\pgfqpoint{3.962766in}{1.731519in}}{\pgfqpoint{3.965600in}{1.724677in}}{\pgfqpoint{3.970643in}{1.719633in}}%
\pgfpathcurveto{\pgfqpoint{3.975687in}{1.714590in}}{\pgfqpoint{3.982529in}{1.711756in}}{\pgfqpoint{3.989662in}{1.711756in}}%
\pgfpathclose%
\pgfusepath{stroke,fill}%
\end{pgfscope}%
\begin{pgfscope}%
\pgfpathrectangle{\pgfqpoint{2.867647in}{0.500000in}}{\pgfqpoint{1.764706in}{1.700000in}}%
\pgfusepath{clip}%
\pgfsetbuttcap%
\pgfsetroundjoin%
\definecolor{currentfill}{rgb}{0.961115,0.566634,0.405693}%
\pgfsetfillcolor{currentfill}%
\pgfsetlinewidth{0.311001pt}%
\definecolor{currentstroke}{rgb}{1.000000,1.000000,1.000000}%
\pgfsetstrokecolor{currentstroke}%
\pgfsetdash{}{0pt}%
\pgfpathmoveto{\pgfqpoint{4.311570in}{1.509396in}}%
\pgfpathcurveto{\pgfqpoint{4.318703in}{1.509396in}}{\pgfqpoint{4.325544in}{1.512230in}}{\pgfqpoint{4.330588in}{1.517273in}}%
\pgfpathcurveto{\pgfqpoint{4.335632in}{1.522317in}}{\pgfqpoint{4.338466in}{1.529159in}}{\pgfqpoint{4.338466in}{1.536291in}}%
\pgfpathcurveto{\pgfqpoint{4.338466in}{1.543424in}}{\pgfqpoint{4.335632in}{1.550266in}}{\pgfqpoint{4.330588in}{1.555310in}}%
\pgfpathcurveto{\pgfqpoint{4.325544in}{1.560353in}}{\pgfqpoint{4.318703in}{1.563187in}}{\pgfqpoint{4.311570in}{1.563187in}}%
\pgfpathcurveto{\pgfqpoint{4.304437in}{1.563187in}}{\pgfqpoint{4.297596in}{1.560353in}}{\pgfqpoint{4.292552in}{1.555310in}}%
\pgfpathcurveto{\pgfqpoint{4.287508in}{1.550266in}}{\pgfqpoint{4.284674in}{1.543424in}}{\pgfqpoint{4.284674in}{1.536291in}}%
\pgfpathcurveto{\pgfqpoint{4.284674in}{1.529159in}}{\pgfqpoint{4.287508in}{1.522317in}}{\pgfqpoint{4.292552in}{1.517273in}}%
\pgfpathcurveto{\pgfqpoint{4.297596in}{1.512230in}}{\pgfqpoint{4.304437in}{1.509396in}}{\pgfqpoint{4.311570in}{1.509396in}}%
\pgfpathclose%
\pgfusepath{stroke,fill}%
\end{pgfscope}%
\begin{pgfscope}%
\pgfpathrectangle{\pgfqpoint{2.867647in}{0.500000in}}{\pgfqpoint{1.764706in}{1.700000in}}%
\pgfusepath{clip}%
\pgfsetbuttcap%
\pgfsetroundjoin%
\definecolor{currentfill}{rgb}{0.963559,0.632016,0.472047}%
\pgfsetfillcolor{currentfill}%
\pgfsetlinewidth{0.311001pt}%
\definecolor{currentstroke}{rgb}{1.000000,1.000000,1.000000}%
\pgfsetstrokecolor{currentstroke}%
\pgfsetdash{}{0pt}%
\pgfpathmoveto{\pgfqpoint{3.981074in}{0.885577in}}%
\pgfpathcurveto{\pgfqpoint{3.988207in}{0.885577in}}{\pgfqpoint{3.995049in}{0.888411in}}{\pgfqpoint{4.000092in}{0.893455in}}%
\pgfpathcurveto{\pgfqpoint{4.005136in}{0.898498in}}{\pgfqpoint{4.007970in}{0.905340in}}{\pgfqpoint{4.007970in}{0.912473in}}%
\pgfpathcurveto{\pgfqpoint{4.007970in}{0.919606in}}{\pgfqpoint{4.005136in}{0.926447in}}{\pgfqpoint{4.000092in}{0.931491in}}%
\pgfpathcurveto{\pgfqpoint{3.995049in}{0.936535in}}{\pgfqpoint{3.988207in}{0.939369in}}{\pgfqpoint{3.981074in}{0.939369in}}%
\pgfpathcurveto{\pgfqpoint{3.973941in}{0.939369in}}{\pgfqpoint{3.967100in}{0.936535in}}{\pgfqpoint{3.962056in}{0.931491in}}%
\pgfpathcurveto{\pgfqpoint{3.957012in}{0.926447in}}{\pgfqpoint{3.954178in}{0.919606in}}{\pgfqpoint{3.954178in}{0.912473in}}%
\pgfpathcurveto{\pgfqpoint{3.954178in}{0.905340in}}{\pgfqpoint{3.957012in}{0.898498in}}{\pgfqpoint{3.962056in}{0.893455in}}%
\pgfpathcurveto{\pgfqpoint{3.967100in}{0.888411in}}{\pgfqpoint{3.973941in}{0.885577in}}{\pgfqpoint{3.981074in}{0.885577in}}%
\pgfpathclose%
\pgfusepath{stroke,fill}%
\end{pgfscope}%
\begin{pgfscope}%
\pgfpathrectangle{\pgfqpoint{2.867647in}{0.500000in}}{\pgfqpoint{1.764706in}{1.700000in}}%
\pgfusepath{clip}%
\pgfsetbuttcap%
\pgfsetroundjoin%
\definecolor{currentfill}{rgb}{0.964173,0.657587,0.500469}%
\pgfsetfillcolor{currentfill}%
\pgfsetlinewidth{0.311001pt}%
\definecolor{currentstroke}{rgb}{1.000000,1.000000,1.000000}%
\pgfsetstrokecolor{currentstroke}%
\pgfsetdash{}{0pt}%
\pgfpathmoveto{\pgfqpoint{4.212099in}{1.678625in}}%
\pgfpathcurveto{\pgfqpoint{4.219232in}{1.678625in}}{\pgfqpoint{4.226073in}{1.681458in}}{\pgfqpoint{4.231117in}{1.686502in}}%
\pgfpathcurveto{\pgfqpoint{4.236161in}{1.691546in}}{\pgfqpoint{4.238994in}{1.698387in}}{\pgfqpoint{4.238994in}{1.705520in}}%
\pgfpathcurveto{\pgfqpoint{4.238994in}{1.712653in}}{\pgfqpoint{4.236161in}{1.719495in}}{\pgfqpoint{4.231117in}{1.724538in}}%
\pgfpathcurveto{\pgfqpoint{4.226073in}{1.729582in}}{\pgfqpoint{4.219232in}{1.732416in}}{\pgfqpoint{4.212099in}{1.732416in}}%
\pgfpathcurveto{\pgfqpoint{4.204966in}{1.732416in}}{\pgfqpoint{4.198124in}{1.729582in}}{\pgfqpoint{4.193081in}{1.724538in}}%
\pgfpathcurveto{\pgfqpoint{4.188037in}{1.719495in}}{\pgfqpoint{4.185203in}{1.712653in}}{\pgfqpoint{4.185203in}{1.705520in}}%
\pgfpathcurveto{\pgfqpoint{4.185203in}{1.698387in}}{\pgfqpoint{4.188037in}{1.691546in}}{\pgfqpoint{4.193081in}{1.686502in}}%
\pgfpathcurveto{\pgfqpoint{4.198124in}{1.681458in}}{\pgfqpoint{4.204966in}{1.678625in}}{\pgfqpoint{4.212099in}{1.678625in}}%
\pgfpathclose%
\pgfusepath{stroke,fill}%
\end{pgfscope}%
\begin{pgfscope}%
\pgfpathrectangle{\pgfqpoint{2.867647in}{0.500000in}}{\pgfqpoint{1.764706in}{1.700000in}}%
\pgfusepath{clip}%
\pgfsetbuttcap%
\pgfsetroundjoin%
\definecolor{currentfill}{rgb}{0.965928,0.738443,0.600540}%
\pgfsetfillcolor{currentfill}%
\pgfsetlinewidth{0.311001pt}%
\definecolor{currentstroke}{rgb}{1.000000,1.000000,1.000000}%
\pgfsetstrokecolor{currentstroke}%
\pgfsetdash{}{0pt}%
\pgfpathmoveto{\pgfqpoint{4.045322in}{1.745042in}}%
\pgfpathcurveto{\pgfqpoint{4.052455in}{1.745042in}}{\pgfqpoint{4.059297in}{1.747876in}}{\pgfqpoint{4.064340in}{1.752919in}}%
\pgfpathcurveto{\pgfqpoint{4.069384in}{1.757963in}}{\pgfqpoint{4.072218in}{1.764805in}}{\pgfqpoint{4.072218in}{1.771938in}}%
\pgfpathcurveto{\pgfqpoint{4.072218in}{1.779070in}}{\pgfqpoint{4.069384in}{1.785912in}}{\pgfqpoint{4.064340in}{1.790956in}}%
\pgfpathcurveto{\pgfqpoint{4.059297in}{1.795999in}}{\pgfqpoint{4.052455in}{1.798833in}}{\pgfqpoint{4.045322in}{1.798833in}}%
\pgfpathcurveto{\pgfqpoint{4.038189in}{1.798833in}}{\pgfqpoint{4.031348in}{1.795999in}}{\pgfqpoint{4.026304in}{1.790956in}}%
\pgfpathcurveto{\pgfqpoint{4.021260in}{1.785912in}}{\pgfqpoint{4.018426in}{1.779070in}}{\pgfqpoint{4.018426in}{1.771938in}}%
\pgfpathcurveto{\pgfqpoint{4.018426in}{1.764805in}}{\pgfqpoint{4.021260in}{1.757963in}}{\pgfqpoint{4.026304in}{1.752919in}}%
\pgfpathcurveto{\pgfqpoint{4.031348in}{1.747876in}}{\pgfqpoint{4.038189in}{1.745042in}}{\pgfqpoint{4.045322in}{1.745042in}}%
\pgfpathclose%
\pgfusepath{stroke,fill}%
\end{pgfscope}%
\begin{pgfscope}%
\pgfpathrectangle{\pgfqpoint{2.867647in}{0.500000in}}{\pgfqpoint{1.764706in}{1.700000in}}%
\pgfusepath{clip}%
\pgfsetbuttcap%
\pgfsetroundjoin%
\definecolor{currentfill}{rgb}{0.978376,0.897317,0.831308}%
\pgfsetfillcolor{currentfill}%
\pgfsetlinewidth{0.311001pt}%
\definecolor{currentstroke}{rgb}{1.000000,1.000000,1.000000}%
\pgfsetstrokecolor{currentstroke}%
\pgfsetdash{}{0pt}%
\pgfpathmoveto{\pgfqpoint{4.150439in}{1.290332in}}%
\pgfpathcurveto{\pgfqpoint{4.157571in}{1.290332in}}{\pgfqpoint{4.164413in}{1.293166in}}{\pgfqpoint{4.169457in}{1.298209in}}%
\pgfpathcurveto{\pgfqpoint{4.174500in}{1.303253in}}{\pgfqpoint{4.177334in}{1.310095in}}{\pgfqpoint{4.177334in}{1.317228in}}%
\pgfpathcurveto{\pgfqpoint{4.177334in}{1.324360in}}{\pgfqpoint{4.174500in}{1.331202in}}{\pgfqpoint{4.169457in}{1.336246in}}%
\pgfpathcurveto{\pgfqpoint{4.164413in}{1.341289in}}{\pgfqpoint{4.157571in}{1.344123in}}{\pgfqpoint{4.150439in}{1.344123in}}%
\pgfpathcurveto{\pgfqpoint{4.143306in}{1.344123in}}{\pgfqpoint{4.136464in}{1.341289in}}{\pgfqpoint{4.131420in}{1.336246in}}%
\pgfpathcurveto{\pgfqpoint{4.126377in}{1.331202in}}{\pgfqpoint{4.123543in}{1.324360in}}{\pgfqpoint{4.123543in}{1.317228in}}%
\pgfpathcurveto{\pgfqpoint{4.123543in}{1.310095in}}{\pgfqpoint{4.126377in}{1.303253in}}{\pgfqpoint{4.131420in}{1.298209in}}%
\pgfpathcurveto{\pgfqpoint{4.136464in}{1.293166in}}{\pgfqpoint{4.143306in}{1.290332in}}{\pgfqpoint{4.150439in}{1.290332in}}%
\pgfpathclose%
\pgfusepath{stroke,fill}%
\end{pgfscope}%
\begin{pgfscope}%
\pgfpathrectangle{\pgfqpoint{2.867647in}{0.500000in}}{\pgfqpoint{1.764706in}{1.700000in}}%
\pgfusepath{clip}%
\pgfsetbuttcap%
\pgfsetroundjoin%
\definecolor{currentfill}{rgb}{0.975018,0.868213,0.788710}%
\pgfsetfillcolor{currentfill}%
\pgfsetlinewidth{0.311001pt}%
\definecolor{currentstroke}{rgb}{1.000000,1.000000,1.000000}%
\pgfsetstrokecolor{currentstroke}%
\pgfsetdash{}{0pt}%
\pgfpathmoveto{\pgfqpoint{4.201834in}{1.550881in}}%
\pgfpathcurveto{\pgfqpoint{4.208967in}{1.550881in}}{\pgfqpoint{4.215809in}{1.553714in}}{\pgfqpoint{4.220853in}{1.558758in}}%
\pgfpathcurveto{\pgfqpoint{4.225896in}{1.563802in}}{\pgfqpoint{4.228730in}{1.570643in}}{\pgfqpoint{4.228730in}{1.577776in}}%
\pgfpathcurveto{\pgfqpoint{4.228730in}{1.584909in}}{\pgfqpoint{4.225896in}{1.591751in}}{\pgfqpoint{4.220853in}{1.596794in}}%
\pgfpathcurveto{\pgfqpoint{4.215809in}{1.601838in}}{\pgfqpoint{4.208967in}{1.604672in}}{\pgfqpoint{4.201834in}{1.604672in}}%
\pgfpathcurveto{\pgfqpoint{4.194702in}{1.604672in}}{\pgfqpoint{4.187860in}{1.601838in}}{\pgfqpoint{4.182816in}{1.596794in}}%
\pgfpathcurveto{\pgfqpoint{4.177773in}{1.591751in}}{\pgfqpoint{4.174939in}{1.584909in}}{\pgfqpoint{4.174939in}{1.577776in}}%
\pgfpathcurveto{\pgfqpoint{4.174939in}{1.570643in}}{\pgfqpoint{4.177773in}{1.563802in}}{\pgfqpoint{4.182816in}{1.558758in}}%
\pgfpathcurveto{\pgfqpoint{4.187860in}{1.553714in}}{\pgfqpoint{4.194702in}{1.550881in}}{\pgfqpoint{4.201834in}{1.550881in}}%
\pgfpathclose%
\pgfusepath{stroke,fill}%
\end{pgfscope}%
\begin{pgfscope}%
\pgfpathrectangle{\pgfqpoint{2.867647in}{0.500000in}}{\pgfqpoint{1.764706in}{1.700000in}}%
\pgfusepath{clip}%
\pgfsetbuttcap%
\pgfsetroundjoin%
\definecolor{currentfill}{rgb}{0.973832,0.856556,0.771584}%
\pgfsetfillcolor{currentfill}%
\pgfsetlinewidth{0.311001pt}%
\definecolor{currentstroke}{rgb}{1.000000,1.000000,1.000000}%
\pgfsetstrokecolor{currentstroke}%
\pgfsetdash{}{0pt}%
\pgfpathmoveto{\pgfqpoint{4.101732in}{1.663324in}}%
\pgfpathcurveto{\pgfqpoint{4.108865in}{1.663324in}}{\pgfqpoint{4.115706in}{1.666158in}}{\pgfqpoint{4.120750in}{1.671202in}}%
\pgfpathcurveto{\pgfqpoint{4.125794in}{1.676245in}}{\pgfqpoint{4.128628in}{1.683087in}}{\pgfqpoint{4.128628in}{1.690220in}}%
\pgfpathcurveto{\pgfqpoint{4.128628in}{1.697353in}}{\pgfqpoint{4.125794in}{1.704194in}}{\pgfqpoint{4.120750in}{1.709238in}}%
\pgfpathcurveto{\pgfqpoint{4.115706in}{1.714282in}}{\pgfqpoint{4.108865in}{1.717116in}}{\pgfqpoint{4.101732in}{1.717116in}}%
\pgfpathcurveto{\pgfqpoint{4.094599in}{1.717116in}}{\pgfqpoint{4.087757in}{1.714282in}}{\pgfqpoint{4.082714in}{1.709238in}}%
\pgfpathcurveto{\pgfqpoint{4.077670in}{1.704194in}}{\pgfqpoint{4.074836in}{1.697353in}}{\pgfqpoint{4.074836in}{1.690220in}}%
\pgfpathcurveto{\pgfqpoint{4.074836in}{1.683087in}}{\pgfqpoint{4.077670in}{1.676245in}}{\pgfqpoint{4.082714in}{1.671202in}}%
\pgfpathcurveto{\pgfqpoint{4.087757in}{1.666158in}}{\pgfqpoint{4.094599in}{1.663324in}}{\pgfqpoint{4.101732in}{1.663324in}}%
\pgfpathclose%
\pgfusepath{stroke,fill}%
\end{pgfscope}%
\begin{pgfscope}%
\pgfpathrectangle{\pgfqpoint{2.867647in}{0.500000in}}{\pgfqpoint{1.764706in}{1.700000in}}%
\pgfusepath{clip}%
\pgfsetbuttcap%
\pgfsetroundjoin%
\definecolor{currentfill}{rgb}{0.979891,0.908948,0.848279}%
\pgfsetfillcolor{currentfill}%
\pgfsetlinewidth{0.311001pt}%
\definecolor{currentstroke}{rgb}{1.000000,1.000000,1.000000}%
\pgfsetstrokecolor{currentstroke}%
\pgfsetdash{}{0pt}%
\pgfpathmoveto{\pgfqpoint{4.156744in}{1.544129in}}%
\pgfpathcurveto{\pgfqpoint{4.163877in}{1.544129in}}{\pgfqpoint{4.170719in}{1.546962in}}{\pgfqpoint{4.175762in}{1.552006in}}%
\pgfpathcurveto{\pgfqpoint{4.180806in}{1.557050in}}{\pgfqpoint{4.183640in}{1.563891in}}{\pgfqpoint{4.183640in}{1.571024in}}%
\pgfpathcurveto{\pgfqpoint{4.183640in}{1.578157in}}{\pgfqpoint{4.180806in}{1.584999in}}{\pgfqpoint{4.175762in}{1.590042in}}%
\pgfpathcurveto{\pgfqpoint{4.170719in}{1.595086in}}{\pgfqpoint{4.163877in}{1.597920in}}{\pgfqpoint{4.156744in}{1.597920in}}%
\pgfpathcurveto{\pgfqpoint{4.149611in}{1.597920in}}{\pgfqpoint{4.142770in}{1.595086in}}{\pgfqpoint{4.137726in}{1.590042in}}%
\pgfpathcurveto{\pgfqpoint{4.132682in}{1.584999in}}{\pgfqpoint{4.129848in}{1.578157in}}{\pgfqpoint{4.129848in}{1.571024in}}%
\pgfpathcurveto{\pgfqpoint{4.129848in}{1.563891in}}{\pgfqpoint{4.132682in}{1.557050in}}{\pgfqpoint{4.137726in}{1.552006in}}%
\pgfpathcurveto{\pgfqpoint{4.142770in}{1.546962in}}{\pgfqpoint{4.149611in}{1.544129in}}{\pgfqpoint{4.156744in}{1.544129in}}%
\pgfpathclose%
\pgfusepath{stroke,fill}%
\end{pgfscope}%
\begin{pgfscope}%
\pgfpathrectangle{\pgfqpoint{2.867647in}{0.500000in}}{\pgfqpoint{1.764706in}{1.700000in}}%
\pgfusepath{clip}%
\pgfsetbuttcap%
\pgfsetroundjoin%
\definecolor{currentfill}{rgb}{0.971694,0.833208,0.737161}%
\pgfsetfillcolor{currentfill}%
\pgfsetlinewidth{0.311001pt}%
\definecolor{currentstroke}{rgb}{1.000000,1.000000,1.000000}%
\pgfsetstrokecolor{currentstroke}%
\pgfsetdash{}{0pt}%
\pgfpathmoveto{\pgfqpoint{4.115729in}{0.969793in}}%
\pgfpathcurveto{\pgfqpoint{4.122861in}{0.969793in}}{\pgfqpoint{4.129703in}{0.972627in}}{\pgfqpoint{4.134747in}{0.977670in}}%
\pgfpathcurveto{\pgfqpoint{4.139790in}{0.982714in}}{\pgfqpoint{4.142624in}{0.989556in}}{\pgfqpoint{4.142624in}{0.996688in}}%
\pgfpathcurveto{\pgfqpoint{4.142624in}{1.003821in}}{\pgfqpoint{4.139790in}{1.010663in}}{\pgfqpoint{4.134747in}{1.015707in}}%
\pgfpathcurveto{\pgfqpoint{4.129703in}{1.020750in}}{\pgfqpoint{4.122861in}{1.023584in}}{\pgfqpoint{4.115729in}{1.023584in}}%
\pgfpathcurveto{\pgfqpoint{4.108596in}{1.023584in}}{\pgfqpoint{4.101754in}{1.020750in}}{\pgfqpoint{4.096711in}{1.015707in}}%
\pgfpathcurveto{\pgfqpoint{4.091667in}{1.010663in}}{\pgfqpoint{4.088833in}{1.003821in}}{\pgfqpoint{4.088833in}{0.996688in}}%
\pgfpathcurveto{\pgfqpoint{4.088833in}{0.989556in}}{\pgfqpoint{4.091667in}{0.982714in}}{\pgfqpoint{4.096711in}{0.977670in}}%
\pgfpathcurveto{\pgfqpoint{4.101754in}{0.972627in}}{\pgfqpoint{4.108596in}{0.969793in}}{\pgfqpoint{4.115729in}{0.969793in}}%
\pgfpathclose%
\pgfusepath{stroke,fill}%
\end{pgfscope}%
\begin{pgfscope}%
\pgfpathrectangle{\pgfqpoint{2.867647in}{0.500000in}}{\pgfqpoint{1.764706in}{1.700000in}}%
\pgfusepath{clip}%
\pgfsetbuttcap%
\pgfsetroundjoin%
\definecolor{currentfill}{rgb}{0.981377,0.920617,0.865369}%
\pgfsetfillcolor{currentfill}%
\pgfsetlinewidth{0.311001pt}%
\definecolor{currentstroke}{rgb}{1.000000,1.000000,1.000000}%
\pgfsetstrokecolor{currentstroke}%
\pgfsetdash{}{0pt}%
\pgfpathmoveto{\pgfqpoint{4.173588in}{1.247979in}}%
\pgfpathcurveto{\pgfqpoint{4.180721in}{1.247979in}}{\pgfqpoint{4.187563in}{1.250813in}}{\pgfqpoint{4.192606in}{1.255856in}}%
\pgfpathcurveto{\pgfqpoint{4.197650in}{1.260900in}}{\pgfqpoint{4.200484in}{1.267742in}}{\pgfqpoint{4.200484in}{1.274875in}}%
\pgfpathcurveto{\pgfqpoint{4.200484in}{1.282007in}}{\pgfqpoint{4.197650in}{1.288849in}}{\pgfqpoint{4.192606in}{1.293893in}}%
\pgfpathcurveto{\pgfqpoint{4.187563in}{1.298936in}}{\pgfqpoint{4.180721in}{1.301770in}}{\pgfqpoint{4.173588in}{1.301770in}}%
\pgfpathcurveto{\pgfqpoint{4.166455in}{1.301770in}}{\pgfqpoint{4.159614in}{1.298936in}}{\pgfqpoint{4.154570in}{1.293893in}}%
\pgfpathcurveto{\pgfqpoint{4.149526in}{1.288849in}}{\pgfqpoint{4.146693in}{1.282007in}}{\pgfqpoint{4.146693in}{1.274875in}}%
\pgfpathcurveto{\pgfqpoint{4.146693in}{1.267742in}}{\pgfqpoint{4.149526in}{1.260900in}}{\pgfqpoint{4.154570in}{1.255856in}}%
\pgfpathcurveto{\pgfqpoint{4.159614in}{1.250813in}}{\pgfqpoint{4.166455in}{1.247979in}}{\pgfqpoint{4.173588in}{1.247979in}}%
\pgfpathclose%
\pgfusepath{stroke,fill}%
\end{pgfscope}%
\begin{pgfscope}%
\pgfpathrectangle{\pgfqpoint{2.867647in}{0.500000in}}{\pgfqpoint{1.764706in}{1.700000in}}%
\pgfusepath{clip}%
\pgfsetbuttcap%
\pgfsetroundjoin%
\definecolor{currentfill}{rgb}{0.979124,0.903132,0.839793}%
\pgfsetfillcolor{currentfill}%
\pgfsetlinewidth{0.311001pt}%
\definecolor{currentstroke}{rgb}{1.000000,1.000000,1.000000}%
\pgfsetstrokecolor{currentstroke}%
\pgfsetdash{}{0pt}%
\pgfpathmoveto{\pgfqpoint{4.216075in}{1.402236in}}%
\pgfpathcurveto{\pgfqpoint{4.223207in}{1.402236in}}{\pgfqpoint{4.230049in}{1.405069in}}{\pgfqpoint{4.235093in}{1.410113in}}%
\pgfpathcurveto{\pgfqpoint{4.240136in}{1.415157in}}{\pgfqpoint{4.242970in}{1.421998in}}{\pgfqpoint{4.242970in}{1.429131in}}%
\pgfpathcurveto{\pgfqpoint{4.242970in}{1.436264in}}{\pgfqpoint{4.240136in}{1.443106in}}{\pgfqpoint{4.235093in}{1.448149in}}%
\pgfpathcurveto{\pgfqpoint{4.230049in}{1.453193in}}{\pgfqpoint{4.223207in}{1.456027in}}{\pgfqpoint{4.216075in}{1.456027in}}%
\pgfpathcurveto{\pgfqpoint{4.208942in}{1.456027in}}{\pgfqpoint{4.202100in}{1.453193in}}{\pgfqpoint{4.197056in}{1.448149in}}%
\pgfpathcurveto{\pgfqpoint{4.192013in}{1.443106in}}{\pgfqpoint{4.189179in}{1.436264in}}{\pgfqpoint{4.189179in}{1.429131in}}%
\pgfpathcurveto{\pgfqpoint{4.189179in}{1.421998in}}{\pgfqpoint{4.192013in}{1.415157in}}{\pgfqpoint{4.197056in}{1.410113in}}%
\pgfpathcurveto{\pgfqpoint{4.202100in}{1.405069in}}{\pgfqpoint{4.208942in}{1.402236in}}{\pgfqpoint{4.216075in}{1.402236in}}%
\pgfpathclose%
\pgfusepath{stroke,fill}%
\end{pgfscope}%
\begin{pgfscope}%
\pgfpathrectangle{\pgfqpoint{2.867647in}{0.500000in}}{\pgfqpoint{1.764706in}{1.700000in}}%
\pgfusepath{clip}%
\pgfsetbuttcap%
\pgfsetroundjoin%
\definecolor{currentfill}{rgb}{0.980678,0.914765,0.856766}%
\pgfsetfillcolor{currentfill}%
\pgfsetlinewidth{0.311001pt}%
\definecolor{currentstroke}{rgb}{1.000000,1.000000,1.000000}%
\pgfsetstrokecolor{currentstroke}%
\pgfsetdash{}{0pt}%
\pgfpathmoveto{\pgfqpoint{4.183712in}{1.164848in}}%
\pgfpathcurveto{\pgfqpoint{4.190845in}{1.164848in}}{\pgfqpoint{4.197686in}{1.167682in}}{\pgfqpoint{4.202730in}{1.172726in}}%
\pgfpathcurveto{\pgfqpoint{4.207774in}{1.177769in}}{\pgfqpoint{4.210608in}{1.184611in}}{\pgfqpoint{4.210608in}{1.191744in}}%
\pgfpathcurveto{\pgfqpoint{4.210608in}{1.198877in}}{\pgfqpoint{4.207774in}{1.205718in}}{\pgfqpoint{4.202730in}{1.210762in}}%
\pgfpathcurveto{\pgfqpoint{4.197686in}{1.215806in}}{\pgfqpoint{4.190845in}{1.218640in}}{\pgfqpoint{4.183712in}{1.218640in}}%
\pgfpathcurveto{\pgfqpoint{4.176579in}{1.218640in}}{\pgfqpoint{4.169737in}{1.215806in}}{\pgfqpoint{4.164694in}{1.210762in}}%
\pgfpathcurveto{\pgfqpoint{4.159650in}{1.205718in}}{\pgfqpoint{4.156816in}{1.198877in}}{\pgfqpoint{4.156816in}{1.191744in}}%
\pgfpathcurveto{\pgfqpoint{4.156816in}{1.184611in}}{\pgfqpoint{4.159650in}{1.177769in}}{\pgfqpoint{4.164694in}{1.172726in}}%
\pgfpathcurveto{\pgfqpoint{4.169737in}{1.167682in}}{\pgfqpoint{4.176579in}{1.164848in}}{\pgfqpoint{4.183712in}{1.164848in}}%
\pgfpathclose%
\pgfusepath{stroke,fill}%
\end{pgfscope}%
\begin{pgfscope}%
\pgfpathrectangle{\pgfqpoint{2.867647in}{0.500000in}}{\pgfqpoint{1.764706in}{1.700000in}}%
\pgfusepath{clip}%
\pgfsetbuttcap%
\pgfsetroundjoin%
\definecolor{currentfill}{rgb}{0.966328,0.750560,0.616961}%
\pgfsetfillcolor{currentfill}%
\pgfsetlinewidth{0.311001pt}%
\definecolor{currentstroke}{rgb}{1.000000,1.000000,1.000000}%
\pgfsetstrokecolor{currentstroke}%
\pgfsetdash{}{0pt}%
\pgfpathmoveto{\pgfqpoint{4.272971in}{1.495869in}}%
\pgfpathcurveto{\pgfqpoint{4.280104in}{1.495869in}}{\pgfqpoint{4.286945in}{1.498703in}}{\pgfqpoint{4.291989in}{1.503747in}}%
\pgfpathcurveto{\pgfqpoint{4.297033in}{1.508791in}}{\pgfqpoint{4.299867in}{1.515632in}}{\pgfqpoint{4.299867in}{1.522765in}}%
\pgfpathcurveto{\pgfqpoint{4.299867in}{1.529898in}}{\pgfqpoint{4.297033in}{1.536740in}}{\pgfqpoint{4.291989in}{1.541783in}}%
\pgfpathcurveto{\pgfqpoint{4.286945in}{1.546827in}}{\pgfqpoint{4.280104in}{1.549661in}}{\pgfqpoint{4.272971in}{1.549661in}}%
\pgfpathcurveto{\pgfqpoint{4.265838in}{1.549661in}}{\pgfqpoint{4.258996in}{1.546827in}}{\pgfqpoint{4.253953in}{1.541783in}}%
\pgfpathcurveto{\pgfqpoint{4.248909in}{1.536740in}}{\pgfqpoint{4.246075in}{1.529898in}}{\pgfqpoint{4.246075in}{1.522765in}}%
\pgfpathcurveto{\pgfqpoint{4.246075in}{1.515632in}}{\pgfqpoint{4.248909in}{1.508791in}}{\pgfqpoint{4.253953in}{1.503747in}}%
\pgfpathcurveto{\pgfqpoint{4.258996in}{1.498703in}}{\pgfqpoint{4.265838in}{1.495869in}}{\pgfqpoint{4.272971in}{1.495869in}}%
\pgfpathclose%
\pgfusepath{stroke,fill}%
\end{pgfscope}%
\begin{pgfscope}%
\pgfpathrectangle{\pgfqpoint{2.867647in}{0.500000in}}{\pgfqpoint{1.764706in}{1.700000in}}%
\pgfusepath{clip}%
\pgfsetbuttcap%
\pgfsetroundjoin%
\definecolor{currentfill}{rgb}{0.976961,0.885681,0.814303}%
\pgfsetfillcolor{currentfill}%
\pgfsetlinewidth{0.311001pt}%
\definecolor{currentstroke}{rgb}{1.000000,1.000000,1.000000}%
\pgfsetstrokecolor{currentstroke}%
\pgfsetdash{}{0pt}%
\pgfpathmoveto{\pgfqpoint{4.108124in}{1.076315in}}%
\pgfpathcurveto{\pgfqpoint{4.115257in}{1.076315in}}{\pgfqpoint{4.122099in}{1.079149in}}{\pgfqpoint{4.127142in}{1.084193in}}%
\pgfpathcurveto{\pgfqpoint{4.132186in}{1.089236in}}{\pgfqpoint{4.135020in}{1.096078in}}{\pgfqpoint{4.135020in}{1.103211in}}%
\pgfpathcurveto{\pgfqpoint{4.135020in}{1.110344in}}{\pgfqpoint{4.132186in}{1.117185in}}{\pgfqpoint{4.127142in}{1.122229in}}%
\pgfpathcurveto{\pgfqpoint{4.122099in}{1.127273in}}{\pgfqpoint{4.115257in}{1.130107in}}{\pgfqpoint{4.108124in}{1.130107in}}%
\pgfpathcurveto{\pgfqpoint{4.100991in}{1.130107in}}{\pgfqpoint{4.094150in}{1.127273in}}{\pgfqpoint{4.089106in}{1.122229in}}%
\pgfpathcurveto{\pgfqpoint{4.084062in}{1.117185in}}{\pgfqpoint{4.081229in}{1.110344in}}{\pgfqpoint{4.081229in}{1.103211in}}%
\pgfpathcurveto{\pgfqpoint{4.081229in}{1.096078in}}{\pgfqpoint{4.084062in}{1.089236in}}{\pgfqpoint{4.089106in}{1.084193in}}%
\pgfpathcurveto{\pgfqpoint{4.094150in}{1.079149in}}{\pgfqpoint{4.100991in}{1.076315in}}{\pgfqpoint{4.108124in}{1.076315in}}%
\pgfpathclose%
\pgfusepath{stroke,fill}%
\end{pgfscope}%
\begin{pgfscope}%
\pgfpathrectangle{\pgfqpoint{2.867647in}{0.500000in}}{\pgfqpoint{1.764706in}{1.700000in}}%
\pgfusepath{clip}%
\pgfsetbuttcap%
\pgfsetroundjoin%
\definecolor{currentfill}{rgb}{0.971694,0.833208,0.737161}%
\pgfsetfillcolor{currentfill}%
\pgfsetlinewidth{0.311001pt}%
\definecolor{currentstroke}{rgb}{1.000000,1.000000,1.000000}%
\pgfsetstrokecolor{currentstroke}%
\pgfsetdash{}{0pt}%
\pgfpathmoveto{\pgfqpoint{4.109296in}{1.243539in}}%
\pgfpathcurveto{\pgfqpoint{4.116429in}{1.243539in}}{\pgfqpoint{4.123270in}{1.246373in}}{\pgfqpoint{4.128314in}{1.251416in}}%
\pgfpathcurveto{\pgfqpoint{4.133358in}{1.256460in}}{\pgfqpoint{4.136192in}{1.263302in}}{\pgfqpoint{4.136192in}{1.270434in}}%
\pgfpathcurveto{\pgfqpoint{4.136192in}{1.277567in}}{\pgfqpoint{4.133358in}{1.284409in}}{\pgfqpoint{4.128314in}{1.289453in}}%
\pgfpathcurveto{\pgfqpoint{4.123270in}{1.294496in}}{\pgfqpoint{4.116429in}{1.297330in}}{\pgfqpoint{4.109296in}{1.297330in}}%
\pgfpathcurveto{\pgfqpoint{4.102163in}{1.297330in}}{\pgfqpoint{4.095321in}{1.294496in}}{\pgfqpoint{4.090278in}{1.289453in}}%
\pgfpathcurveto{\pgfqpoint{4.085234in}{1.284409in}}{\pgfqpoint{4.082400in}{1.277567in}}{\pgfqpoint{4.082400in}{1.270434in}}%
\pgfpathcurveto{\pgfqpoint{4.082400in}{1.263302in}}{\pgfqpoint{4.085234in}{1.256460in}}{\pgfqpoint{4.090278in}{1.251416in}}%
\pgfpathcurveto{\pgfqpoint{4.095321in}{1.246373in}}{\pgfqpoint{4.102163in}{1.243539in}}{\pgfqpoint{4.109296in}{1.243539in}}%
\pgfpathclose%
\pgfusepath{stroke,fill}%
\end{pgfscope}%
\begin{pgfscope}%
\pgfpathrectangle{\pgfqpoint{2.867647in}{0.500000in}}{\pgfqpoint{1.764706in}{1.700000in}}%
\pgfusepath{clip}%
\pgfsetbuttcap%
\pgfsetroundjoin%
\definecolor{currentfill}{rgb}{0.979124,0.903132,0.839793}%
\pgfsetfillcolor{currentfill}%
\pgfsetlinewidth{0.311001pt}%
\definecolor{currentstroke}{rgb}{1.000000,1.000000,1.000000}%
\pgfsetstrokecolor{currentstroke}%
\pgfsetdash{}{0pt}%
\pgfpathmoveto{\pgfqpoint{4.139353in}{1.469877in}}%
\pgfpathcurveto{\pgfqpoint{4.146485in}{1.469877in}}{\pgfqpoint{4.153327in}{1.472711in}}{\pgfqpoint{4.158371in}{1.477754in}}%
\pgfpathcurveto{\pgfqpoint{4.163414in}{1.482798in}}{\pgfqpoint{4.166248in}{1.489640in}}{\pgfqpoint{4.166248in}{1.496772in}}%
\pgfpathcurveto{\pgfqpoint{4.166248in}{1.503905in}}{\pgfqpoint{4.163414in}{1.510747in}}{\pgfqpoint{4.158371in}{1.515791in}}%
\pgfpathcurveto{\pgfqpoint{4.153327in}{1.520834in}}{\pgfqpoint{4.146485in}{1.523668in}}{\pgfqpoint{4.139353in}{1.523668in}}%
\pgfpathcurveto{\pgfqpoint{4.132220in}{1.523668in}}{\pgfqpoint{4.125378in}{1.520834in}}{\pgfqpoint{4.120334in}{1.515791in}}%
\pgfpathcurveto{\pgfqpoint{4.115291in}{1.510747in}}{\pgfqpoint{4.112457in}{1.503905in}}{\pgfqpoint{4.112457in}{1.496772in}}%
\pgfpathcurveto{\pgfqpoint{4.112457in}{1.489640in}}{\pgfqpoint{4.115291in}{1.482798in}}{\pgfqpoint{4.120334in}{1.477754in}}%
\pgfpathcurveto{\pgfqpoint{4.125378in}{1.472711in}}{\pgfqpoint{4.132220in}{1.469877in}}{\pgfqpoint{4.139353in}{1.469877in}}%
\pgfpathclose%
\pgfusepath{stroke,fill}%
\end{pgfscope}%
\begin{pgfscope}%
\pgfpathrectangle{\pgfqpoint{2.867647in}{0.500000in}}{\pgfqpoint{1.764706in}{1.700000in}}%
\pgfusepath{clip}%
\pgfsetbuttcap%
\pgfsetroundjoin%
\definecolor{currentfill}{rgb}{0.963379,0.625574,0.465113}%
\pgfsetfillcolor{currentfill}%
\pgfsetlinewidth{0.311001pt}%
\definecolor{currentstroke}{rgb}{1.000000,1.000000,1.000000}%
\pgfsetstrokecolor{currentstroke}%
\pgfsetdash{}{0pt}%
\pgfpathmoveto{\pgfqpoint{4.119670in}{0.876453in}}%
\pgfpathcurveto{\pgfqpoint{4.126803in}{0.876453in}}{\pgfqpoint{4.133645in}{0.879287in}}{\pgfqpoint{4.138688in}{0.884331in}}%
\pgfpathcurveto{\pgfqpoint{4.143732in}{0.889375in}}{\pgfqpoint{4.146566in}{0.896216in}}{\pgfqpoint{4.146566in}{0.903349in}}%
\pgfpathcurveto{\pgfqpoint{4.146566in}{0.910482in}}{\pgfqpoint{4.143732in}{0.917324in}}{\pgfqpoint{4.138688in}{0.922367in}}%
\pgfpathcurveto{\pgfqpoint{4.133645in}{0.927411in}}{\pgfqpoint{4.126803in}{0.930245in}}{\pgfqpoint{4.119670in}{0.930245in}}%
\pgfpathcurveto{\pgfqpoint{4.112537in}{0.930245in}}{\pgfqpoint{4.105696in}{0.927411in}}{\pgfqpoint{4.100652in}{0.922367in}}%
\pgfpathcurveto{\pgfqpoint{4.095608in}{0.917324in}}{\pgfqpoint{4.092775in}{0.910482in}}{\pgfqpoint{4.092775in}{0.903349in}}%
\pgfpathcurveto{\pgfqpoint{4.092775in}{0.896216in}}{\pgfqpoint{4.095608in}{0.889375in}}{\pgfqpoint{4.100652in}{0.884331in}}%
\pgfpathcurveto{\pgfqpoint{4.105696in}{0.879287in}}{\pgfqpoint{4.112537in}{0.876453in}}{\pgfqpoint{4.119670in}{0.876453in}}%
\pgfpathclose%
\pgfusepath{stroke,fill}%
\end{pgfscope}%
\begin{pgfscope}%
\pgfpathrectangle{\pgfqpoint{2.867647in}{0.500000in}}{\pgfqpoint{1.764706in}{1.700000in}}%
\pgfusepath{clip}%
\pgfsetbuttcap%
\pgfsetroundjoin%
\definecolor{currentfill}{rgb}{0.962283,0.593046,0.431453}%
\pgfsetfillcolor{currentfill}%
\pgfsetlinewidth{0.311001pt}%
\definecolor{currentstroke}{rgb}{1.000000,1.000000,1.000000}%
\pgfsetstrokecolor{currentstroke}%
\pgfsetdash{}{0pt}%
\pgfpathmoveto{\pgfqpoint{3.926500in}{1.712782in}}%
\pgfpathcurveto{\pgfqpoint{3.933633in}{1.712782in}}{\pgfqpoint{3.940475in}{1.715616in}}{\pgfqpoint{3.945518in}{1.720659in}}%
\pgfpathcurveto{\pgfqpoint{3.950562in}{1.725703in}}{\pgfqpoint{3.953396in}{1.732545in}}{\pgfqpoint{3.953396in}{1.739678in}}%
\pgfpathcurveto{\pgfqpoint{3.953396in}{1.746810in}}{\pgfqpoint{3.950562in}{1.753652in}}{\pgfqpoint{3.945518in}{1.758696in}}%
\pgfpathcurveto{\pgfqpoint{3.940475in}{1.763739in}}{\pgfqpoint{3.933633in}{1.766573in}}{\pgfqpoint{3.926500in}{1.766573in}}%
\pgfpathcurveto{\pgfqpoint{3.919367in}{1.766573in}}{\pgfqpoint{3.912526in}{1.763739in}}{\pgfqpoint{3.907482in}{1.758696in}}%
\pgfpathcurveto{\pgfqpoint{3.902438in}{1.753652in}}{\pgfqpoint{3.899604in}{1.746810in}}{\pgfqpoint{3.899604in}{1.739678in}}%
\pgfpathcurveto{\pgfqpoint{3.899604in}{1.732545in}}{\pgfqpoint{3.902438in}{1.725703in}}{\pgfqpoint{3.907482in}{1.720659in}}%
\pgfpathcurveto{\pgfqpoint{3.912526in}{1.715616in}}{\pgfqpoint{3.919367in}{1.712782in}}{\pgfqpoint{3.926500in}{1.712782in}}%
\pgfpathclose%
\pgfusepath{stroke,fill}%
\end{pgfscope}%
\begin{pgfscope}%
\pgfpathrectangle{\pgfqpoint{2.867647in}{0.500000in}}{\pgfqpoint{1.764706in}{1.700000in}}%
\pgfusepath{clip}%
\pgfsetbuttcap%
\pgfsetroundjoin%
\definecolor{currentfill}{rgb}{0.964920,0.695342,0.545192}%
\pgfsetfillcolor{currentfill}%
\pgfsetlinewidth{0.311001pt}%
\definecolor{currentstroke}{rgb}{1.000000,1.000000,1.000000}%
\pgfsetstrokecolor{currentstroke}%
\pgfsetdash{}{0pt}%
\pgfpathmoveto{\pgfqpoint{4.011709in}{1.562684in}}%
\pgfpathcurveto{\pgfqpoint{4.018842in}{1.562684in}}{\pgfqpoint{4.025683in}{1.565518in}}{\pgfqpoint{4.030727in}{1.570562in}}%
\pgfpathcurveto{\pgfqpoint{4.035771in}{1.575605in}}{\pgfqpoint{4.038605in}{1.582447in}}{\pgfqpoint{4.038605in}{1.589580in}}%
\pgfpathcurveto{\pgfqpoint{4.038605in}{1.596713in}}{\pgfqpoint{4.035771in}{1.603554in}}{\pgfqpoint{4.030727in}{1.608598in}}%
\pgfpathcurveto{\pgfqpoint{4.025683in}{1.613641in}}{\pgfqpoint{4.018842in}{1.616475in}}{\pgfqpoint{4.011709in}{1.616475in}}%
\pgfpathcurveto{\pgfqpoint{4.004576in}{1.616475in}}{\pgfqpoint{3.997735in}{1.613641in}}{\pgfqpoint{3.992691in}{1.608598in}}%
\pgfpathcurveto{\pgfqpoint{3.987647in}{1.603554in}}{\pgfqpoint{3.984813in}{1.596713in}}{\pgfqpoint{3.984813in}{1.589580in}}%
\pgfpathcurveto{\pgfqpoint{3.984813in}{1.582447in}}{\pgfqpoint{3.987647in}{1.575605in}}{\pgfqpoint{3.992691in}{1.570562in}}%
\pgfpathcurveto{\pgfqpoint{3.997735in}{1.565518in}}{\pgfqpoint{4.004576in}{1.562684in}}{\pgfqpoint{4.011709in}{1.562684in}}%
\pgfpathclose%
\pgfusepath{stroke,fill}%
\end{pgfscope}%
\begin{pgfscope}%
\pgfpathrectangle{\pgfqpoint{2.867647in}{0.500000in}}{\pgfqpoint{1.764706in}{1.700000in}}%
\pgfusepath{clip}%
\pgfsetbuttcap%
\pgfsetroundjoin%
\definecolor{currentfill}{rgb}{0.962283,0.593046,0.431453}%
\pgfsetfillcolor{currentfill}%
\pgfsetlinewidth{0.311001pt}%
\definecolor{currentstroke}{rgb}{1.000000,1.000000,1.000000}%
\pgfsetstrokecolor{currentstroke}%
\pgfsetdash{}{0pt}%
\pgfpathmoveto{\pgfqpoint{3.950130in}{0.980244in}}%
\pgfpathcurveto{\pgfqpoint{3.957263in}{0.980244in}}{\pgfqpoint{3.964105in}{0.983078in}}{\pgfqpoint{3.969149in}{0.988121in}}%
\pgfpathcurveto{\pgfqpoint{3.974192in}{0.993165in}}{\pgfqpoint{3.977026in}{1.000007in}}{\pgfqpoint{3.977026in}{1.007140in}}%
\pgfpathcurveto{\pgfqpoint{3.977026in}{1.014272in}}{\pgfqpoint{3.974192in}{1.021114in}}{\pgfqpoint{3.969149in}{1.026158in}}%
\pgfpathcurveto{\pgfqpoint{3.964105in}{1.031201in}}{\pgfqpoint{3.957263in}{1.034035in}}{\pgfqpoint{3.950130in}{1.034035in}}%
\pgfpathcurveto{\pgfqpoint{3.942998in}{1.034035in}}{\pgfqpoint{3.936156in}{1.031201in}}{\pgfqpoint{3.931112in}{1.026158in}}%
\pgfpathcurveto{\pgfqpoint{3.926069in}{1.021114in}}{\pgfqpoint{3.923235in}{1.014272in}}{\pgfqpoint{3.923235in}{1.007140in}}%
\pgfpathcurveto{\pgfqpoint{3.923235in}{1.000007in}}{\pgfqpoint{3.926069in}{0.993165in}}{\pgfqpoint{3.931112in}{0.988121in}}%
\pgfpathcurveto{\pgfqpoint{3.936156in}{0.983078in}}{\pgfqpoint{3.942998in}{0.980244in}}{\pgfqpoint{3.950130in}{0.980244in}}%
\pgfpathclose%
\pgfusepath{stroke,fill}%
\end{pgfscope}%
\begin{pgfscope}%
\pgfpathrectangle{\pgfqpoint{2.867647in}{0.500000in}}{\pgfqpoint{1.764706in}{1.700000in}}%
\pgfusepath{clip}%
\pgfsetbuttcap%
\pgfsetroundjoin%
\definecolor{currentfill}{rgb}{0.979124,0.903132,0.839793}%
\pgfsetfillcolor{currentfill}%
\pgfsetlinewidth{0.311001pt}%
\definecolor{currentstroke}{rgb}{1.000000,1.000000,1.000000}%
\pgfsetstrokecolor{currentstroke}%
\pgfsetdash{}{0pt}%
\pgfpathmoveto{\pgfqpoint{4.149116in}{1.237437in}}%
\pgfpathcurveto{\pgfqpoint{4.156249in}{1.237437in}}{\pgfqpoint{4.163090in}{1.240271in}}{\pgfqpoint{4.168134in}{1.245315in}}%
\pgfpathcurveto{\pgfqpoint{4.173178in}{1.250358in}}{\pgfqpoint{4.176012in}{1.257200in}}{\pgfqpoint{4.176012in}{1.264333in}}%
\pgfpathcurveto{\pgfqpoint{4.176012in}{1.271466in}}{\pgfqpoint{4.173178in}{1.278307in}}{\pgfqpoint{4.168134in}{1.283351in}}%
\pgfpathcurveto{\pgfqpoint{4.163090in}{1.288395in}}{\pgfqpoint{4.156249in}{1.291229in}}{\pgfqpoint{4.149116in}{1.291229in}}%
\pgfpathcurveto{\pgfqpoint{4.141983in}{1.291229in}}{\pgfqpoint{4.135141in}{1.288395in}}{\pgfqpoint{4.130098in}{1.283351in}}%
\pgfpathcurveto{\pgfqpoint{4.125054in}{1.278307in}}{\pgfqpoint{4.122220in}{1.271466in}}{\pgfqpoint{4.122220in}{1.264333in}}%
\pgfpathcurveto{\pgfqpoint{4.122220in}{1.257200in}}{\pgfqpoint{4.125054in}{1.250358in}}{\pgfqpoint{4.130098in}{1.245315in}}%
\pgfpathcurveto{\pgfqpoint{4.135141in}{1.240271in}}{\pgfqpoint{4.141983in}{1.237437in}}{\pgfqpoint{4.149116in}{1.237437in}}%
\pgfpathclose%
\pgfusepath{stroke,fill}%
\end{pgfscope}%
\begin{pgfscope}%
\pgfpathrectangle{\pgfqpoint{2.867647in}{0.500000in}}{\pgfqpoint{1.764706in}{1.700000in}}%
\pgfusepath{clip}%
\pgfsetbuttcap%
\pgfsetroundjoin%
\definecolor{currentfill}{rgb}{0.981377,0.920617,0.865369}%
\pgfsetfillcolor{currentfill}%
\pgfsetlinewidth{0.311001pt}%
\definecolor{currentstroke}{rgb}{1.000000,1.000000,1.000000}%
\pgfsetstrokecolor{currentstroke}%
\pgfsetdash{}{0pt}%
\pgfpathmoveto{\pgfqpoint{4.175290in}{1.179239in}}%
\pgfpathcurveto{\pgfqpoint{4.182423in}{1.179239in}}{\pgfqpoint{4.189264in}{1.182073in}}{\pgfqpoint{4.194308in}{1.187116in}}%
\pgfpathcurveto{\pgfqpoint{4.199352in}{1.192160in}}{\pgfqpoint{4.202186in}{1.199002in}}{\pgfqpoint{4.202186in}{1.206135in}}%
\pgfpathcurveto{\pgfqpoint{4.202186in}{1.213267in}}{\pgfqpoint{4.199352in}{1.220109in}}{\pgfqpoint{4.194308in}{1.225153in}}%
\pgfpathcurveto{\pgfqpoint{4.189264in}{1.230196in}}{\pgfqpoint{4.182423in}{1.233030in}}{\pgfqpoint{4.175290in}{1.233030in}}%
\pgfpathcurveto{\pgfqpoint{4.168157in}{1.233030in}}{\pgfqpoint{4.161315in}{1.230196in}}{\pgfqpoint{4.156272in}{1.225153in}}%
\pgfpathcurveto{\pgfqpoint{4.151228in}{1.220109in}}{\pgfqpoint{4.148394in}{1.213267in}}{\pgfqpoint{4.148394in}{1.206135in}}%
\pgfpathcurveto{\pgfqpoint{4.148394in}{1.199002in}}{\pgfqpoint{4.151228in}{1.192160in}}{\pgfqpoint{4.156272in}{1.187116in}}%
\pgfpathcurveto{\pgfqpoint{4.161315in}{1.182073in}}{\pgfqpoint{4.168157in}{1.179239in}}{\pgfqpoint{4.175290in}{1.179239in}}%
\pgfpathclose%
\pgfusepath{stroke,fill}%
\end{pgfscope}%
\begin{pgfscope}%
\pgfpathrectangle{\pgfqpoint{2.867647in}{0.500000in}}{\pgfqpoint{1.764706in}{1.700000in}}%
\pgfusepath{clip}%
\pgfsetbuttcap%
\pgfsetroundjoin%
\definecolor{currentfill}{rgb}{0.964920,0.695342,0.545192}%
\pgfsetfillcolor{currentfill}%
\pgfsetlinewidth{0.311001pt}%
\definecolor{currentstroke}{rgb}{1.000000,1.000000,1.000000}%
\pgfsetstrokecolor{currentstroke}%
\pgfsetdash{}{0pt}%
\pgfpathmoveto{\pgfqpoint{4.303158in}{1.416448in}}%
\pgfpathcurveto{\pgfqpoint{4.310291in}{1.416448in}}{\pgfqpoint{4.317132in}{1.419282in}}{\pgfqpoint{4.322176in}{1.424325in}}%
\pgfpathcurveto{\pgfqpoint{4.327220in}{1.429369in}}{\pgfqpoint{4.330053in}{1.436211in}}{\pgfqpoint{4.330053in}{1.443344in}}%
\pgfpathcurveto{\pgfqpoint{4.330053in}{1.450476in}}{\pgfqpoint{4.327220in}{1.457318in}}{\pgfqpoint{4.322176in}{1.462362in}}%
\pgfpathcurveto{\pgfqpoint{4.317132in}{1.467405in}}{\pgfqpoint{4.310291in}{1.470239in}}{\pgfqpoint{4.303158in}{1.470239in}}%
\pgfpathcurveto{\pgfqpoint{4.296025in}{1.470239in}}{\pgfqpoint{4.289183in}{1.467405in}}{\pgfqpoint{4.284140in}{1.462362in}}%
\pgfpathcurveto{\pgfqpoint{4.279096in}{1.457318in}}{\pgfqpoint{4.276262in}{1.450476in}}{\pgfqpoint{4.276262in}{1.443344in}}%
\pgfpathcurveto{\pgfqpoint{4.276262in}{1.436211in}}{\pgfqpoint{4.279096in}{1.429369in}}{\pgfqpoint{4.284140in}{1.424325in}}%
\pgfpathcurveto{\pgfqpoint{4.289183in}{1.419282in}}{\pgfqpoint{4.296025in}{1.416448in}}{\pgfqpoint{4.303158in}{1.416448in}}%
\pgfpathclose%
\pgfusepath{stroke,fill}%
\end{pgfscope}%
\begin{pgfscope}%
\pgfpathrectangle{\pgfqpoint{2.867647in}{0.500000in}}{\pgfqpoint{1.764706in}{1.700000in}}%
\pgfusepath{clip}%
\pgfsetbuttcap%
\pgfsetroundjoin%
\definecolor{currentfill}{rgb}{0.963379,0.625574,0.465113}%
\pgfsetfillcolor{currentfill}%
\pgfsetlinewidth{0.311001pt}%
\definecolor{currentstroke}{rgb}{1.000000,1.000000,1.000000}%
\pgfsetstrokecolor{currentstroke}%
\pgfsetdash{}{0pt}%
\pgfpathmoveto{\pgfqpoint{4.238299in}{0.987868in}}%
\pgfpathcurveto{\pgfqpoint{4.245432in}{0.987868in}}{\pgfqpoint{4.252274in}{0.990701in}}{\pgfqpoint{4.257318in}{0.995745in}}%
\pgfpathcurveto{\pgfqpoint{4.262361in}{1.000789in}}{\pgfqpoint{4.265195in}{1.007630in}}{\pgfqpoint{4.265195in}{1.014763in}}%
\pgfpathcurveto{\pgfqpoint{4.265195in}{1.021896in}}{\pgfqpoint{4.262361in}{1.028738in}}{\pgfqpoint{4.257318in}{1.033781in}}%
\pgfpathcurveto{\pgfqpoint{4.252274in}{1.038825in}}{\pgfqpoint{4.245432in}{1.041659in}}{\pgfqpoint{4.238299in}{1.041659in}}%
\pgfpathcurveto{\pgfqpoint{4.231167in}{1.041659in}}{\pgfqpoint{4.224325in}{1.038825in}}{\pgfqpoint{4.219281in}{1.033781in}}%
\pgfpathcurveto{\pgfqpoint{4.214238in}{1.028738in}}{\pgfqpoint{4.211404in}{1.021896in}}{\pgfqpoint{4.211404in}{1.014763in}}%
\pgfpathcurveto{\pgfqpoint{4.211404in}{1.007630in}}{\pgfqpoint{4.214238in}{1.000789in}}{\pgfqpoint{4.219281in}{0.995745in}}%
\pgfpathcurveto{\pgfqpoint{4.224325in}{0.990701in}}{\pgfqpoint{4.231167in}{0.987868in}}{\pgfqpoint{4.238299in}{0.987868in}}%
\pgfpathclose%
\pgfusepath{stroke,fill}%
\end{pgfscope}%
\begin{pgfscope}%
\pgfpathrectangle{\pgfqpoint{2.867647in}{0.500000in}}{\pgfqpoint{1.764706in}{1.700000in}}%
\pgfusepath{clip}%
\pgfsetbuttcap%
\pgfsetroundjoin%
\definecolor{currentfill}{rgb}{0.875073,0.185874,0.265297}%
\pgfsetfillcolor{currentfill}%
\pgfsetlinewidth{0.311001pt}%
\definecolor{currentstroke}{rgb}{1.000000,1.000000,1.000000}%
\pgfsetstrokecolor{currentstroke}%
\pgfsetdash{}{0pt}%
\pgfpathmoveto{\pgfqpoint{4.393489in}{1.382523in}}%
\pgfpathcurveto{\pgfqpoint{4.400622in}{1.382523in}}{\pgfqpoint{4.407464in}{1.385357in}}{\pgfqpoint{4.412507in}{1.390401in}}%
\pgfpathcurveto{\pgfqpoint{4.417551in}{1.395444in}}{\pgfqpoint{4.420385in}{1.402286in}}{\pgfqpoint{4.420385in}{1.409419in}}%
\pgfpathcurveto{\pgfqpoint{4.420385in}{1.416552in}}{\pgfqpoint{4.417551in}{1.423393in}}{\pgfqpoint{4.412507in}{1.428437in}}%
\pgfpathcurveto{\pgfqpoint{4.407464in}{1.433481in}}{\pgfqpoint{4.400622in}{1.436314in}}{\pgfqpoint{4.393489in}{1.436314in}}%
\pgfpathcurveto{\pgfqpoint{4.386356in}{1.436314in}}{\pgfqpoint{4.379515in}{1.433481in}}{\pgfqpoint{4.374471in}{1.428437in}}%
\pgfpathcurveto{\pgfqpoint{4.369427in}{1.423393in}}{\pgfqpoint{4.366593in}{1.416552in}}{\pgfqpoint{4.366593in}{1.409419in}}%
\pgfpathcurveto{\pgfqpoint{4.366593in}{1.402286in}}{\pgfqpoint{4.369427in}{1.395444in}}{\pgfqpoint{4.374471in}{1.390401in}}%
\pgfpathcurveto{\pgfqpoint{4.379515in}{1.385357in}}{\pgfqpoint{4.386356in}{1.382523in}}{\pgfqpoint{4.393489in}{1.382523in}}%
\pgfpathclose%
\pgfusepath{stroke,fill}%
\end{pgfscope}%
\begin{pgfscope}%
\pgfpathrectangle{\pgfqpoint{2.867647in}{0.500000in}}{\pgfqpoint{1.764706in}{1.700000in}}%
\pgfusepath{clip}%
\pgfsetbuttcap%
\pgfsetroundjoin%
\definecolor{currentfill}{rgb}{0.972201,0.839051,0.745789}%
\pgfsetfillcolor{currentfill}%
\pgfsetlinewidth{0.311001pt}%
\definecolor{currentstroke}{rgb}{1.000000,1.000000,1.000000}%
\pgfsetstrokecolor{currentstroke}%
\pgfsetdash{}{0pt}%
\pgfpathmoveto{\pgfqpoint{4.083507in}{1.673962in}}%
\pgfpathcurveto{\pgfqpoint{4.090640in}{1.673962in}}{\pgfqpoint{4.097482in}{1.676796in}}{\pgfqpoint{4.102525in}{1.681840in}}%
\pgfpathcurveto{\pgfqpoint{4.107569in}{1.686883in}}{\pgfqpoint{4.110403in}{1.693725in}}{\pgfqpoint{4.110403in}{1.700858in}}%
\pgfpathcurveto{\pgfqpoint{4.110403in}{1.707991in}}{\pgfqpoint{4.107569in}{1.714832in}}{\pgfqpoint{4.102525in}{1.719876in}}%
\pgfpathcurveto{\pgfqpoint{4.097482in}{1.724920in}}{\pgfqpoint{4.090640in}{1.727753in}}{\pgfqpoint{4.083507in}{1.727753in}}%
\pgfpathcurveto{\pgfqpoint{4.076374in}{1.727753in}}{\pgfqpoint{4.069533in}{1.724920in}}{\pgfqpoint{4.064489in}{1.719876in}}%
\pgfpathcurveto{\pgfqpoint{4.059445in}{1.714832in}}{\pgfqpoint{4.056611in}{1.707991in}}{\pgfqpoint{4.056611in}{1.700858in}}%
\pgfpathcurveto{\pgfqpoint{4.056611in}{1.693725in}}{\pgfqpoint{4.059445in}{1.686883in}}{\pgfqpoint{4.064489in}{1.681840in}}%
\pgfpathcurveto{\pgfqpoint{4.069533in}{1.676796in}}{\pgfqpoint{4.076374in}{1.673962in}}{\pgfqpoint{4.083507in}{1.673962in}}%
\pgfpathclose%
\pgfusepath{stroke,fill}%
\end{pgfscope}%
\begin{pgfscope}%
\pgfpathrectangle{\pgfqpoint{2.867647in}{0.500000in}}{\pgfqpoint{1.764706in}{1.700000in}}%
\pgfusepath{clip}%
\pgfsetbuttcap%
\pgfsetroundjoin%
\definecolor{currentfill}{rgb}{0.963884,0.644842,0.486120}%
\pgfsetfillcolor{currentfill}%
\pgfsetlinewidth{0.311001pt}%
\definecolor{currentstroke}{rgb}{1.000000,1.000000,1.000000}%
\pgfsetstrokecolor{currentstroke}%
\pgfsetdash{}{0pt}%
\pgfpathmoveto{\pgfqpoint{4.229158in}{1.660890in}}%
\pgfpathcurveto{\pgfqpoint{4.236291in}{1.660890in}}{\pgfqpoint{4.243133in}{1.663724in}}{\pgfqpoint{4.248177in}{1.668768in}}%
\pgfpathcurveto{\pgfqpoint{4.253220in}{1.673812in}}{\pgfqpoint{4.256054in}{1.680653in}}{\pgfqpoint{4.256054in}{1.687786in}}%
\pgfpathcurveto{\pgfqpoint{4.256054in}{1.694919in}}{\pgfqpoint{4.253220in}{1.701761in}}{\pgfqpoint{4.248177in}{1.706804in}}%
\pgfpathcurveto{\pgfqpoint{4.243133in}{1.711848in}}{\pgfqpoint{4.236291in}{1.714682in}}{\pgfqpoint{4.229158in}{1.714682in}}%
\pgfpathcurveto{\pgfqpoint{4.222026in}{1.714682in}}{\pgfqpoint{4.215184in}{1.711848in}}{\pgfqpoint{4.210140in}{1.706804in}}%
\pgfpathcurveto{\pgfqpoint{4.205097in}{1.701761in}}{\pgfqpoint{4.202263in}{1.694919in}}{\pgfqpoint{4.202263in}{1.687786in}}%
\pgfpathcurveto{\pgfqpoint{4.202263in}{1.680653in}}{\pgfqpoint{4.205097in}{1.673812in}}{\pgfqpoint{4.210140in}{1.668768in}}%
\pgfpathcurveto{\pgfqpoint{4.215184in}{1.663724in}}{\pgfqpoint{4.222026in}{1.660890in}}{\pgfqpoint{4.229158in}{1.660890in}}%
\pgfpathclose%
\pgfusepath{stroke,fill}%
\end{pgfscope}%
\begin{pgfscope}%
\pgfpathrectangle{\pgfqpoint{2.867647in}{0.500000in}}{\pgfqpoint{1.764706in}{1.700000in}}%
\pgfusepath{clip}%
\pgfsetbuttcap%
\pgfsetroundjoin%
\definecolor{currentfill}{rgb}{0.973271,0.850724,0.762998}%
\pgfsetfillcolor{currentfill}%
\pgfsetlinewidth{0.311001pt}%
\definecolor{currentstroke}{rgb}{1.000000,1.000000,1.000000}%
\pgfsetstrokecolor{currentstroke}%
\pgfsetdash{}{0pt}%
\pgfpathmoveto{\pgfqpoint{4.194551in}{1.585645in}}%
\pgfpathcurveto{\pgfqpoint{4.201684in}{1.585645in}}{\pgfqpoint{4.208526in}{1.588479in}}{\pgfqpoint{4.213569in}{1.593523in}}%
\pgfpathcurveto{\pgfqpoint{4.218613in}{1.598566in}}{\pgfqpoint{4.221447in}{1.605408in}}{\pgfqpoint{4.221447in}{1.612541in}}%
\pgfpathcurveto{\pgfqpoint{4.221447in}{1.619674in}}{\pgfqpoint{4.218613in}{1.626515in}}{\pgfqpoint{4.213569in}{1.631559in}}%
\pgfpathcurveto{\pgfqpoint{4.208526in}{1.636603in}}{\pgfqpoint{4.201684in}{1.639436in}}{\pgfqpoint{4.194551in}{1.639436in}}%
\pgfpathcurveto{\pgfqpoint{4.187418in}{1.639436in}}{\pgfqpoint{4.180577in}{1.636603in}}{\pgfqpoint{4.175533in}{1.631559in}}%
\pgfpathcurveto{\pgfqpoint{4.170489in}{1.626515in}}{\pgfqpoint{4.167655in}{1.619674in}}{\pgfqpoint{4.167655in}{1.612541in}}%
\pgfpathcurveto{\pgfqpoint{4.167655in}{1.605408in}}{\pgfqpoint{4.170489in}{1.598566in}}{\pgfqpoint{4.175533in}{1.593523in}}%
\pgfpathcurveto{\pgfqpoint{4.180577in}{1.588479in}}{\pgfqpoint{4.187418in}{1.585645in}}{\pgfqpoint{4.194551in}{1.585645in}}%
\pgfpathclose%
\pgfusepath{stroke,fill}%
\end{pgfscope}%
\begin{pgfscope}%
\pgfpathrectangle{\pgfqpoint{2.867647in}{0.500000in}}{\pgfqpoint{1.764706in}{1.700000in}}%
\pgfusepath{clip}%
\pgfsetbuttcap%
\pgfsetroundjoin%
\definecolor{currentfill}{rgb}{0.965753,0.732351,0.592427}%
\pgfsetfillcolor{currentfill}%
\pgfsetlinewidth{0.311001pt}%
\definecolor{currentstroke}{rgb}{1.000000,1.000000,1.000000}%
\pgfsetstrokecolor{currentstroke}%
\pgfsetdash{}{0pt}%
\pgfpathmoveto{\pgfqpoint{4.151809in}{0.935297in}}%
\pgfpathcurveto{\pgfqpoint{4.158942in}{0.935297in}}{\pgfqpoint{4.165783in}{0.938131in}}{\pgfqpoint{4.170827in}{0.943175in}}%
\pgfpathcurveto{\pgfqpoint{4.175871in}{0.948219in}}{\pgfqpoint{4.178704in}{0.955060in}}{\pgfqpoint{4.178704in}{0.962193in}}%
\pgfpathcurveto{\pgfqpoint{4.178704in}{0.969326in}}{\pgfqpoint{4.175871in}{0.976168in}}{\pgfqpoint{4.170827in}{0.981211in}}%
\pgfpathcurveto{\pgfqpoint{4.165783in}{0.986255in}}{\pgfqpoint{4.158942in}{0.989089in}}{\pgfqpoint{4.151809in}{0.989089in}}%
\pgfpathcurveto{\pgfqpoint{4.144676in}{0.989089in}}{\pgfqpoint{4.137834in}{0.986255in}}{\pgfqpoint{4.132791in}{0.981211in}}%
\pgfpathcurveto{\pgfqpoint{4.127747in}{0.976168in}}{\pgfqpoint{4.124913in}{0.969326in}}{\pgfqpoint{4.124913in}{0.962193in}}%
\pgfpathcurveto{\pgfqpoint{4.124913in}{0.955060in}}{\pgfqpoint{4.127747in}{0.948219in}}{\pgfqpoint{4.132791in}{0.943175in}}%
\pgfpathcurveto{\pgfqpoint{4.137834in}{0.938131in}}{\pgfqpoint{4.144676in}{0.935297in}}{\pgfqpoint{4.151809in}{0.935297in}}%
\pgfpathclose%
\pgfusepath{stroke,fill}%
\end{pgfscope}%
\begin{pgfscope}%
\pgfpathrectangle{\pgfqpoint{2.867647in}{0.500000in}}{\pgfqpoint{1.764706in}{1.700000in}}%
\pgfusepath{clip}%
\pgfsetbuttcap%
\pgfsetroundjoin%
\definecolor{currentfill}{rgb}{0.971202,0.827364,0.728520}%
\pgfsetfillcolor{currentfill}%
\pgfsetlinewidth{0.311001pt}%
\definecolor{currentstroke}{rgb}{1.000000,1.000000,1.000000}%
\pgfsetstrokecolor{currentstroke}%
\pgfsetdash{}{0pt}%
\pgfpathmoveto{\pgfqpoint{4.112955in}{1.359712in}}%
\pgfpathcurveto{\pgfqpoint{4.120088in}{1.359712in}}{\pgfqpoint{4.126929in}{1.362546in}}{\pgfqpoint{4.131973in}{1.367589in}}%
\pgfpathcurveto{\pgfqpoint{4.137017in}{1.372633in}}{\pgfqpoint{4.139851in}{1.379475in}}{\pgfqpoint{4.139851in}{1.386608in}}%
\pgfpathcurveto{\pgfqpoint{4.139851in}{1.393740in}}{\pgfqpoint{4.137017in}{1.400582in}}{\pgfqpoint{4.131973in}{1.405626in}}%
\pgfpathcurveto{\pgfqpoint{4.126929in}{1.410669in}}{\pgfqpoint{4.120088in}{1.413503in}}{\pgfqpoint{4.112955in}{1.413503in}}%
\pgfpathcurveto{\pgfqpoint{4.105822in}{1.413503in}}{\pgfqpoint{4.098980in}{1.410669in}}{\pgfqpoint{4.093937in}{1.405626in}}%
\pgfpathcurveto{\pgfqpoint{4.088893in}{1.400582in}}{\pgfqpoint{4.086059in}{1.393740in}}{\pgfqpoint{4.086059in}{1.386608in}}%
\pgfpathcurveto{\pgfqpoint{4.086059in}{1.379475in}}{\pgfqpoint{4.088893in}{1.372633in}}{\pgfqpoint{4.093937in}{1.367589in}}%
\pgfpathcurveto{\pgfqpoint{4.098980in}{1.362546in}}{\pgfqpoint{4.105822in}{1.359712in}}{\pgfqpoint{4.112955in}{1.359712in}}%
\pgfpathclose%
\pgfusepath{stroke,fill}%
\end{pgfscope}%
\begin{pgfscope}%
\pgfpathrectangle{\pgfqpoint{2.867647in}{0.500000in}}{\pgfqpoint{1.764706in}{1.700000in}}%
\pgfusepath{clip}%
\pgfsetbuttcap%
\pgfsetroundjoin%
\definecolor{currentfill}{rgb}{0.964173,0.657587,0.500469}%
\pgfsetfillcolor{currentfill}%
\pgfsetlinewidth{0.311001pt}%
\definecolor{currentstroke}{rgb}{1.000000,1.000000,1.000000}%
\pgfsetstrokecolor{currentstroke}%
\pgfsetdash{}{0pt}%
\pgfpathmoveto{\pgfqpoint{4.064851in}{0.870674in}}%
\pgfpathcurveto{\pgfqpoint{4.071984in}{0.870674in}}{\pgfqpoint{4.078826in}{0.873508in}}{\pgfqpoint{4.083869in}{0.878552in}}%
\pgfpathcurveto{\pgfqpoint{4.088913in}{0.883595in}}{\pgfqpoint{4.091747in}{0.890437in}}{\pgfqpoint{4.091747in}{0.897570in}}%
\pgfpathcurveto{\pgfqpoint{4.091747in}{0.904703in}}{\pgfqpoint{4.088913in}{0.911544in}}{\pgfqpoint{4.083869in}{0.916588in}}%
\pgfpathcurveto{\pgfqpoint{4.078826in}{0.921632in}}{\pgfqpoint{4.071984in}{0.924466in}}{\pgfqpoint{4.064851in}{0.924466in}}%
\pgfpathcurveto{\pgfqpoint{4.057718in}{0.924466in}}{\pgfqpoint{4.050877in}{0.921632in}}{\pgfqpoint{4.045833in}{0.916588in}}%
\pgfpathcurveto{\pgfqpoint{4.040789in}{0.911544in}}{\pgfqpoint{4.037955in}{0.904703in}}{\pgfqpoint{4.037955in}{0.897570in}}%
\pgfpathcurveto{\pgfqpoint{4.037955in}{0.890437in}}{\pgfqpoint{4.040789in}{0.883595in}}{\pgfqpoint{4.045833in}{0.878552in}}%
\pgfpathcurveto{\pgfqpoint{4.050877in}{0.873508in}}{\pgfqpoint{4.057718in}{0.870674in}}{\pgfqpoint{4.064851in}{0.870674in}}%
\pgfpathclose%
\pgfusepath{stroke,fill}%
\end{pgfscope}%
\begin{pgfscope}%
\pgfpathrectangle{\pgfqpoint{2.867647in}{0.500000in}}{\pgfqpoint{1.764706in}{1.700000in}}%
\pgfusepath{clip}%
\pgfsetbuttcap%
\pgfsetroundjoin%
\definecolor{currentfill}{rgb}{0.966812,0.762584,0.633643}%
\pgfsetfillcolor{currentfill}%
\pgfsetlinewidth{0.311001pt}%
\definecolor{currentstroke}{rgb}{1.000000,1.000000,1.000000}%
\pgfsetstrokecolor{currentstroke}%
\pgfsetdash{}{0pt}%
\pgfpathmoveto{\pgfqpoint{4.084903in}{0.917930in}}%
\pgfpathcurveto{\pgfqpoint{4.092035in}{0.917930in}}{\pgfqpoint{4.098877in}{0.920764in}}{\pgfqpoint{4.103921in}{0.925808in}}%
\pgfpathcurveto{\pgfqpoint{4.108964in}{0.930852in}}{\pgfqpoint{4.111798in}{0.937693in}}{\pgfqpoint{4.111798in}{0.944826in}}%
\pgfpathcurveto{\pgfqpoint{4.111798in}{0.951959in}}{\pgfqpoint{4.108964in}{0.958801in}}{\pgfqpoint{4.103921in}{0.963844in}}%
\pgfpathcurveto{\pgfqpoint{4.098877in}{0.968888in}}{\pgfqpoint{4.092035in}{0.971722in}}{\pgfqpoint{4.084903in}{0.971722in}}%
\pgfpathcurveto{\pgfqpoint{4.077770in}{0.971722in}}{\pgfqpoint{4.070928in}{0.968888in}}{\pgfqpoint{4.065885in}{0.963844in}}%
\pgfpathcurveto{\pgfqpoint{4.060841in}{0.958801in}}{\pgfqpoint{4.058007in}{0.951959in}}{\pgfqpoint{4.058007in}{0.944826in}}%
\pgfpathcurveto{\pgfqpoint{4.058007in}{0.937693in}}{\pgfqpoint{4.060841in}{0.930852in}}{\pgfqpoint{4.065885in}{0.925808in}}%
\pgfpathcurveto{\pgfqpoint{4.070928in}{0.920764in}}{\pgfqpoint{4.077770in}{0.917930in}}{\pgfqpoint{4.084903in}{0.917930in}}%
\pgfpathclose%
\pgfusepath{stroke,fill}%
\end{pgfscope}%
\begin{pgfscope}%
\pgfpathrectangle{\pgfqpoint{2.867647in}{0.500000in}}{\pgfqpoint{1.764706in}{1.700000in}}%
\pgfusepath{clip}%
\pgfsetbuttcap%
\pgfsetroundjoin%
\definecolor{currentfill}{rgb}{0.976287,0.879862,0.805788}%
\pgfsetfillcolor{currentfill}%
\pgfsetlinewidth{0.311001pt}%
\definecolor{currentstroke}{rgb}{1.000000,1.000000,1.000000}%
\pgfsetstrokecolor{currentstroke}%
\pgfsetdash{}{0pt}%
\pgfpathmoveto{\pgfqpoint{4.241317in}{1.278534in}}%
\pgfpathcurveto{\pgfqpoint{4.248450in}{1.278534in}}{\pgfqpoint{4.255291in}{1.281368in}}{\pgfqpoint{4.260335in}{1.286411in}}%
\pgfpathcurveto{\pgfqpoint{4.265379in}{1.291455in}}{\pgfqpoint{4.268213in}{1.298297in}}{\pgfqpoint{4.268213in}{1.305430in}}%
\pgfpathcurveto{\pgfqpoint{4.268213in}{1.312562in}}{\pgfqpoint{4.265379in}{1.319404in}}{\pgfqpoint{4.260335in}{1.324448in}}%
\pgfpathcurveto{\pgfqpoint{4.255291in}{1.329491in}}{\pgfqpoint{4.248450in}{1.332325in}}{\pgfqpoint{4.241317in}{1.332325in}}%
\pgfpathcurveto{\pgfqpoint{4.234184in}{1.332325in}}{\pgfqpoint{4.227342in}{1.329491in}}{\pgfqpoint{4.222299in}{1.324448in}}%
\pgfpathcurveto{\pgfqpoint{4.217255in}{1.319404in}}{\pgfqpoint{4.214421in}{1.312562in}}{\pgfqpoint{4.214421in}{1.305430in}}%
\pgfpathcurveto{\pgfqpoint{4.214421in}{1.298297in}}{\pgfqpoint{4.217255in}{1.291455in}}{\pgfqpoint{4.222299in}{1.286411in}}%
\pgfpathcurveto{\pgfqpoint{4.227342in}{1.281368in}}{\pgfqpoint{4.234184in}{1.278534in}}{\pgfqpoint{4.241317in}{1.278534in}}%
\pgfpathclose%
\pgfusepath{stroke,fill}%
\end{pgfscope}%
\begin{pgfscope}%
\pgfpathrectangle{\pgfqpoint{2.867647in}{0.500000in}}{\pgfqpoint{1.764706in}{1.700000in}}%
\pgfusepath{clip}%
\pgfsetbuttcap%
\pgfsetroundjoin%
\definecolor{currentfill}{rgb}{0.961734,0.579886,0.418445}%
\pgfsetfillcolor{currentfill}%
\pgfsetlinewidth{0.311001pt}%
\definecolor{currentstroke}{rgb}{1.000000,1.000000,1.000000}%
\pgfsetstrokecolor{currentstroke}%
\pgfsetdash{}{0pt}%
\pgfpathmoveto{\pgfqpoint{3.943479in}{0.973786in}}%
\pgfpathcurveto{\pgfqpoint{3.950612in}{0.973786in}}{\pgfqpoint{3.957454in}{0.976620in}}{\pgfqpoint{3.962497in}{0.981664in}}%
\pgfpathcurveto{\pgfqpoint{3.967541in}{0.986707in}}{\pgfqpoint{3.970375in}{0.993549in}}{\pgfqpoint{3.970375in}{1.000682in}}%
\pgfpathcurveto{\pgfqpoint{3.970375in}{1.007815in}}{\pgfqpoint{3.967541in}{1.014656in}}{\pgfqpoint{3.962497in}{1.019700in}}%
\pgfpathcurveto{\pgfqpoint{3.957454in}{1.024744in}}{\pgfqpoint{3.950612in}{1.027577in}}{\pgfqpoint{3.943479in}{1.027577in}}%
\pgfpathcurveto{\pgfqpoint{3.936346in}{1.027577in}}{\pgfqpoint{3.929505in}{1.024744in}}{\pgfqpoint{3.924461in}{1.019700in}}%
\pgfpathcurveto{\pgfqpoint{3.919417in}{1.014656in}}{\pgfqpoint{3.916583in}{1.007815in}}{\pgfqpoint{3.916583in}{1.000682in}}%
\pgfpathcurveto{\pgfqpoint{3.916583in}{0.993549in}}{\pgfqpoint{3.919417in}{0.986707in}}{\pgfqpoint{3.924461in}{0.981664in}}%
\pgfpathcurveto{\pgfqpoint{3.929505in}{0.976620in}}{\pgfqpoint{3.936346in}{0.973786in}}{\pgfqpoint{3.943479in}{0.973786in}}%
\pgfpathclose%
\pgfusepath{stroke,fill}%
\end{pgfscope}%
\begin{pgfscope}%
\pgfpathrectangle{\pgfqpoint{2.867647in}{0.500000in}}{\pgfqpoint{1.764706in}{1.700000in}}%
\pgfusepath{clip}%
\pgfsetbuttcap%
\pgfsetroundjoin%
\definecolor{currentfill}{rgb}{0.977657,0.891500,0.822809}%
\pgfsetfillcolor{currentfill}%
\pgfsetlinewidth{0.311001pt}%
\definecolor{currentstroke}{rgb}{1.000000,1.000000,1.000000}%
\pgfsetstrokecolor{currentstroke}%
\pgfsetdash{}{0pt}%
\pgfpathmoveto{\pgfqpoint{4.132576in}{1.613720in}}%
\pgfpathcurveto{\pgfqpoint{4.139709in}{1.613720in}}{\pgfqpoint{4.146551in}{1.616554in}}{\pgfqpoint{4.151595in}{1.621597in}}%
\pgfpathcurveto{\pgfqpoint{4.156638in}{1.626641in}}{\pgfqpoint{4.159472in}{1.633483in}}{\pgfqpoint{4.159472in}{1.640615in}}%
\pgfpathcurveto{\pgfqpoint{4.159472in}{1.647748in}}{\pgfqpoint{4.156638in}{1.654590in}}{\pgfqpoint{4.151595in}{1.659634in}}%
\pgfpathcurveto{\pgfqpoint{4.146551in}{1.664677in}}{\pgfqpoint{4.139709in}{1.667511in}}{\pgfqpoint{4.132576in}{1.667511in}}%
\pgfpathcurveto{\pgfqpoint{4.125444in}{1.667511in}}{\pgfqpoint{4.118602in}{1.664677in}}{\pgfqpoint{4.113558in}{1.659634in}}%
\pgfpathcurveto{\pgfqpoint{4.108515in}{1.654590in}}{\pgfqpoint{4.105681in}{1.647748in}}{\pgfqpoint{4.105681in}{1.640615in}}%
\pgfpathcurveto{\pgfqpoint{4.105681in}{1.633483in}}{\pgfqpoint{4.108515in}{1.626641in}}{\pgfqpoint{4.113558in}{1.621597in}}%
\pgfpathcurveto{\pgfqpoint{4.118602in}{1.616554in}}{\pgfqpoint{4.125444in}{1.613720in}}{\pgfqpoint{4.132576in}{1.613720in}}%
\pgfpathclose%
\pgfusepath{stroke,fill}%
\end{pgfscope}%
\begin{pgfscope}%
\pgfpathrectangle{\pgfqpoint{2.867647in}{0.500000in}}{\pgfqpoint{1.764706in}{1.700000in}}%
\pgfusepath{clip}%
\pgfsetbuttcap%
\pgfsetroundjoin%
\definecolor{currentfill}{rgb}{0.967735,0.780441,0.659127}%
\pgfsetfillcolor{currentfill}%
\pgfsetlinewidth{0.311001pt}%
\definecolor{currentstroke}{rgb}{1.000000,1.000000,1.000000}%
\pgfsetstrokecolor{currentstroke}%
\pgfsetdash{}{0pt}%
\pgfpathmoveto{\pgfqpoint{4.036337in}{0.943475in}}%
\pgfpathcurveto{\pgfqpoint{4.043470in}{0.943475in}}{\pgfqpoint{4.050312in}{0.946309in}}{\pgfqpoint{4.055355in}{0.951353in}}%
\pgfpathcurveto{\pgfqpoint{4.060399in}{0.956396in}}{\pgfqpoint{4.063233in}{0.963238in}}{\pgfqpoint{4.063233in}{0.970371in}}%
\pgfpathcurveto{\pgfqpoint{4.063233in}{0.977504in}}{\pgfqpoint{4.060399in}{0.984345in}}{\pgfqpoint{4.055355in}{0.989389in}}%
\pgfpathcurveto{\pgfqpoint{4.050312in}{0.994432in}}{\pgfqpoint{4.043470in}{0.997266in}}{\pgfqpoint{4.036337in}{0.997266in}}%
\pgfpathcurveto{\pgfqpoint{4.029204in}{0.997266in}}{\pgfqpoint{4.022363in}{0.994432in}}{\pgfqpoint{4.017319in}{0.989389in}}%
\pgfpathcurveto{\pgfqpoint{4.012275in}{0.984345in}}{\pgfqpoint{4.009441in}{0.977504in}}{\pgfqpoint{4.009441in}{0.970371in}}%
\pgfpathcurveto{\pgfqpoint{4.009441in}{0.963238in}}{\pgfqpoint{4.012275in}{0.956396in}}{\pgfqpoint{4.017319in}{0.951353in}}%
\pgfpathcurveto{\pgfqpoint{4.022363in}{0.946309in}}{\pgfqpoint{4.029204in}{0.943475in}}{\pgfqpoint{4.036337in}{0.943475in}}%
\pgfpathclose%
\pgfusepath{stroke,fill}%
\end{pgfscope}%
\begin{pgfscope}%
\pgfpathrectangle{\pgfqpoint{2.867647in}{0.500000in}}{\pgfqpoint{1.764706in}{1.700000in}}%
\pgfusepath{clip}%
\pgfsetbuttcap%
\pgfsetroundjoin%
\definecolor{currentfill}{rgb}{0.767484,0.092136,0.323905}%
\pgfsetfillcolor{currentfill}%
\pgfsetlinewidth{0.311001pt}%
\definecolor{currentstroke}{rgb}{1.000000,1.000000,1.000000}%
\pgfsetstrokecolor{currentstroke}%
\pgfsetdash{}{0pt}%
\pgfpathmoveto{\pgfqpoint{3.827418in}{1.886444in}}%
\pgfpathcurveto{\pgfqpoint{3.834551in}{1.886444in}}{\pgfqpoint{3.841393in}{1.889277in}}{\pgfqpoint{3.846436in}{1.894321in}}%
\pgfpathcurveto{\pgfqpoint{3.851480in}{1.899365in}}{\pgfqpoint{3.854314in}{1.906206in}}{\pgfqpoint{3.854314in}{1.913339in}}%
\pgfpathcurveto{\pgfqpoint{3.854314in}{1.920472in}}{\pgfqpoint{3.851480in}{1.927314in}}{\pgfqpoint{3.846436in}{1.932357in}}%
\pgfpathcurveto{\pgfqpoint{3.841393in}{1.937401in}}{\pgfqpoint{3.834551in}{1.940235in}}{\pgfqpoint{3.827418in}{1.940235in}}%
\pgfpathcurveto{\pgfqpoint{3.820285in}{1.940235in}}{\pgfqpoint{3.813444in}{1.937401in}}{\pgfqpoint{3.808400in}{1.932357in}}%
\pgfpathcurveto{\pgfqpoint{3.803356in}{1.927314in}}{\pgfqpoint{3.800523in}{1.920472in}}{\pgfqpoint{3.800523in}{1.913339in}}%
\pgfpathcurveto{\pgfqpoint{3.800523in}{1.906206in}}{\pgfqpoint{3.803356in}{1.899365in}}{\pgfqpoint{3.808400in}{1.894321in}}%
\pgfpathcurveto{\pgfqpoint{3.813444in}{1.889277in}}{\pgfqpoint{3.820285in}{1.886444in}}{\pgfqpoint{3.827418in}{1.886444in}}%
\pgfpathclose%
\pgfusepath{stroke,fill}%
\end{pgfscope}%
\begin{pgfscope}%
\pgfpathrectangle{\pgfqpoint{2.867647in}{0.500000in}}{\pgfqpoint{1.764706in}{1.700000in}}%
\pgfusepath{clip}%
\pgfsetbuttcap%
\pgfsetroundjoin%
\definecolor{currentfill}{rgb}{0.977657,0.891500,0.822809}%
\pgfsetfillcolor{currentfill}%
\pgfsetlinewidth{0.311001pt}%
\definecolor{currentstroke}{rgb}{1.000000,1.000000,1.000000}%
\pgfsetstrokecolor{currentstroke}%
\pgfsetdash{}{0pt}%
\pgfpathmoveto{\pgfqpoint{4.160943in}{1.581375in}}%
\pgfpathcurveto{\pgfqpoint{4.168076in}{1.581375in}}{\pgfqpoint{4.174918in}{1.584209in}}{\pgfqpoint{4.179961in}{1.589252in}}%
\pgfpathcurveto{\pgfqpoint{4.185005in}{1.594296in}}{\pgfqpoint{4.187839in}{1.601138in}}{\pgfqpoint{4.187839in}{1.608270in}}%
\pgfpathcurveto{\pgfqpoint{4.187839in}{1.615403in}}{\pgfqpoint{4.185005in}{1.622245in}}{\pgfqpoint{4.179961in}{1.627288in}}%
\pgfpathcurveto{\pgfqpoint{4.174918in}{1.632332in}}{\pgfqpoint{4.168076in}{1.635166in}}{\pgfqpoint{4.160943in}{1.635166in}}%
\pgfpathcurveto{\pgfqpoint{4.153810in}{1.635166in}}{\pgfqpoint{4.146969in}{1.632332in}}{\pgfqpoint{4.141925in}{1.627288in}}%
\pgfpathcurveto{\pgfqpoint{4.136881in}{1.622245in}}{\pgfqpoint{4.134047in}{1.615403in}}{\pgfqpoint{4.134047in}{1.608270in}}%
\pgfpathcurveto{\pgfqpoint{4.134047in}{1.601138in}}{\pgfqpoint{4.136881in}{1.594296in}}{\pgfqpoint{4.141925in}{1.589252in}}%
\pgfpathcurveto{\pgfqpoint{4.146969in}{1.584209in}}{\pgfqpoint{4.153810in}{1.581375in}}{\pgfqpoint{4.160943in}{1.581375in}}%
\pgfpathclose%
\pgfusepath{stroke,fill}%
\end{pgfscope}%
\begin{pgfscope}%
\pgfpathrectangle{\pgfqpoint{2.867647in}{0.500000in}}{\pgfqpoint{1.764706in}{1.700000in}}%
\pgfusepath{clip}%
\pgfsetbuttcap%
\pgfsetroundjoin%
\definecolor{currentfill}{rgb}{0.963559,0.632016,0.472047}%
\pgfsetfillcolor{currentfill}%
\pgfsetlinewidth{0.311001pt}%
\definecolor{currentstroke}{rgb}{1.000000,1.000000,1.000000}%
\pgfsetstrokecolor{currentstroke}%
\pgfsetdash{}{0pt}%
\pgfpathmoveto{\pgfqpoint{4.280470in}{1.070509in}}%
\pgfpathcurveto{\pgfqpoint{4.287603in}{1.070509in}}{\pgfqpoint{4.294444in}{1.073342in}}{\pgfqpoint{4.299488in}{1.078386in}}%
\pgfpathcurveto{\pgfqpoint{4.304532in}{1.083430in}}{\pgfqpoint{4.307366in}{1.090271in}}{\pgfqpoint{4.307366in}{1.097404in}}%
\pgfpathcurveto{\pgfqpoint{4.307366in}{1.104537in}}{\pgfqpoint{4.304532in}{1.111379in}}{\pgfqpoint{4.299488in}{1.116422in}}%
\pgfpathcurveto{\pgfqpoint{4.294444in}{1.121466in}}{\pgfqpoint{4.287603in}{1.124300in}}{\pgfqpoint{4.280470in}{1.124300in}}%
\pgfpathcurveto{\pgfqpoint{4.273337in}{1.124300in}}{\pgfqpoint{4.266495in}{1.121466in}}{\pgfqpoint{4.261452in}{1.116422in}}%
\pgfpathcurveto{\pgfqpoint{4.256408in}{1.111379in}}{\pgfqpoint{4.253574in}{1.104537in}}{\pgfqpoint{4.253574in}{1.097404in}}%
\pgfpathcurveto{\pgfqpoint{4.253574in}{1.090271in}}{\pgfqpoint{4.256408in}{1.083430in}}{\pgfqpoint{4.261452in}{1.078386in}}%
\pgfpathcurveto{\pgfqpoint{4.266495in}{1.073342in}}{\pgfqpoint{4.273337in}{1.070509in}}{\pgfqpoint{4.280470in}{1.070509in}}%
\pgfpathclose%
\pgfusepath{stroke,fill}%
\end{pgfscope}%
\begin{pgfscope}%
\pgfpathrectangle{\pgfqpoint{2.867647in}{0.500000in}}{\pgfqpoint{1.764706in}{1.700000in}}%
\pgfusepath{clip}%
\pgfsetbuttcap%
\pgfsetroundjoin%
\definecolor{currentfill}{rgb}{0.962985,0.612625,0.451451}%
\pgfsetfillcolor{currentfill}%
\pgfsetlinewidth{0.311001pt}%
\definecolor{currentstroke}{rgb}{1.000000,1.000000,1.000000}%
\pgfsetstrokecolor{currentstroke}%
\pgfsetdash{}{0pt}%
\pgfpathmoveto{\pgfqpoint{4.032734in}{1.801290in}}%
\pgfpathcurveto{\pgfqpoint{4.039867in}{1.801290in}}{\pgfqpoint{4.046709in}{1.804124in}}{\pgfqpoint{4.051752in}{1.809168in}}%
\pgfpathcurveto{\pgfqpoint{4.056796in}{1.814212in}}{\pgfqpoint{4.059630in}{1.821053in}}{\pgfqpoint{4.059630in}{1.828186in}}%
\pgfpathcurveto{\pgfqpoint{4.059630in}{1.835319in}}{\pgfqpoint{4.056796in}{1.842161in}}{\pgfqpoint{4.051752in}{1.847204in}}%
\pgfpathcurveto{\pgfqpoint{4.046709in}{1.852248in}}{\pgfqpoint{4.039867in}{1.855082in}}{\pgfqpoint{4.032734in}{1.855082in}}%
\pgfpathcurveto{\pgfqpoint{4.025601in}{1.855082in}}{\pgfqpoint{4.018760in}{1.852248in}}{\pgfqpoint{4.013716in}{1.847204in}}%
\pgfpathcurveto{\pgfqpoint{4.008672in}{1.842161in}}{\pgfqpoint{4.005838in}{1.835319in}}{\pgfqpoint{4.005838in}{1.828186in}}%
\pgfpathcurveto{\pgfqpoint{4.005838in}{1.821053in}}{\pgfqpoint{4.008672in}{1.814212in}}{\pgfqpoint{4.013716in}{1.809168in}}%
\pgfpathcurveto{\pgfqpoint{4.018760in}{1.804124in}}{\pgfqpoint{4.025601in}{1.801290in}}{\pgfqpoint{4.032734in}{1.801290in}}%
\pgfpathclose%
\pgfusepath{stroke,fill}%
\end{pgfscope}%
\begin{pgfscope}%
\pgfpathrectangle{\pgfqpoint{2.867647in}{0.500000in}}{\pgfqpoint{1.764706in}{1.700000in}}%
\pgfusepath{clip}%
\pgfsetbuttcap%
\pgfsetroundjoin%
\definecolor{currentfill}{rgb}{0.972201,0.839051,0.745789}%
\pgfsetfillcolor{currentfill}%
\pgfsetlinewidth{0.311001pt}%
\definecolor{currentstroke}{rgb}{1.000000,1.000000,1.000000}%
\pgfsetstrokecolor{currentstroke}%
\pgfsetdash{}{0pt}%
\pgfpathmoveto{\pgfqpoint{4.088530in}{1.149067in}}%
\pgfpathcurveto{\pgfqpoint{4.095663in}{1.149067in}}{\pgfqpoint{4.102504in}{1.151901in}}{\pgfqpoint{4.107548in}{1.156945in}}%
\pgfpathcurveto{\pgfqpoint{4.112592in}{1.161989in}}{\pgfqpoint{4.115426in}{1.168830in}}{\pgfqpoint{4.115426in}{1.175963in}}%
\pgfpathcurveto{\pgfqpoint{4.115426in}{1.183096in}}{\pgfqpoint{4.112592in}{1.189938in}}{\pgfqpoint{4.107548in}{1.194981in}}%
\pgfpathcurveto{\pgfqpoint{4.102504in}{1.200025in}}{\pgfqpoint{4.095663in}{1.202859in}}{\pgfqpoint{4.088530in}{1.202859in}}%
\pgfpathcurveto{\pgfqpoint{4.081397in}{1.202859in}}{\pgfqpoint{4.074555in}{1.200025in}}{\pgfqpoint{4.069512in}{1.194981in}}%
\pgfpathcurveto{\pgfqpoint{4.064468in}{1.189938in}}{\pgfqpoint{4.061634in}{1.183096in}}{\pgfqpoint{4.061634in}{1.175963in}}%
\pgfpathcurveto{\pgfqpoint{4.061634in}{1.168830in}}{\pgfqpoint{4.064468in}{1.161989in}}{\pgfqpoint{4.069512in}{1.156945in}}%
\pgfpathcurveto{\pgfqpoint{4.074555in}{1.151901in}}{\pgfqpoint{4.081397in}{1.149067in}}{\pgfqpoint{4.088530in}{1.149067in}}%
\pgfpathclose%
\pgfusepath{stroke,fill}%
\end{pgfscope}%
\begin{pgfscope}%
\pgfpathrectangle{\pgfqpoint{2.867647in}{0.500000in}}{\pgfqpoint{1.764706in}{1.700000in}}%
\pgfusepath{clip}%
\pgfsetbuttcap%
\pgfsetroundjoin%
\definecolor{currentfill}{rgb}{0.970255,0.815666,0.711203}%
\pgfsetfillcolor{currentfill}%
\pgfsetlinewidth{0.311001pt}%
\definecolor{currentstroke}{rgb}{1.000000,1.000000,1.000000}%
\pgfsetstrokecolor{currentstroke}%
\pgfsetdash{}{0pt}%
\pgfpathmoveto{\pgfqpoint{4.116621in}{1.692101in}}%
\pgfpathcurveto{\pgfqpoint{4.123753in}{1.692101in}}{\pgfqpoint{4.130595in}{1.694935in}}{\pgfqpoint{4.135639in}{1.699979in}}%
\pgfpathcurveto{\pgfqpoint{4.140682in}{1.705022in}}{\pgfqpoint{4.143516in}{1.711864in}}{\pgfqpoint{4.143516in}{1.718997in}}%
\pgfpathcurveto{\pgfqpoint{4.143516in}{1.726130in}}{\pgfqpoint{4.140682in}{1.732971in}}{\pgfqpoint{4.135639in}{1.738015in}}%
\pgfpathcurveto{\pgfqpoint{4.130595in}{1.743059in}}{\pgfqpoint{4.123753in}{1.745893in}}{\pgfqpoint{4.116621in}{1.745893in}}%
\pgfpathcurveto{\pgfqpoint{4.109488in}{1.745893in}}{\pgfqpoint{4.102646in}{1.743059in}}{\pgfqpoint{4.097602in}{1.738015in}}%
\pgfpathcurveto{\pgfqpoint{4.092559in}{1.732971in}}{\pgfqpoint{4.089725in}{1.726130in}}{\pgfqpoint{4.089725in}{1.718997in}}%
\pgfpathcurveto{\pgfqpoint{4.089725in}{1.711864in}}{\pgfqpoint{4.092559in}{1.705022in}}{\pgfqpoint{4.097602in}{1.699979in}}%
\pgfpathcurveto{\pgfqpoint{4.102646in}{1.694935in}}{\pgfqpoint{4.109488in}{1.692101in}}{\pgfqpoint{4.116621in}{1.692101in}}%
\pgfpathclose%
\pgfusepath{stroke,fill}%
\end{pgfscope}%
\begin{pgfscope}%
\pgfpathrectangle{\pgfqpoint{2.867647in}{0.500000in}}{\pgfqpoint{1.764706in}{1.700000in}}%
\pgfusepath{clip}%
\pgfsetbuttcap%
\pgfsetroundjoin%
\definecolor{currentfill}{rgb}{0.971202,0.827364,0.728520}%
\pgfsetfillcolor{currentfill}%
\pgfsetlinewidth{0.311001pt}%
\definecolor{currentstroke}{rgb}{1.000000,1.000000,1.000000}%
\pgfsetstrokecolor{currentstroke}%
\pgfsetdash{}{0pt}%
\pgfpathmoveto{\pgfqpoint{4.191973in}{1.031662in}}%
\pgfpathcurveto{\pgfqpoint{4.199106in}{1.031662in}}{\pgfqpoint{4.205948in}{1.034496in}}{\pgfqpoint{4.210991in}{1.039540in}}%
\pgfpathcurveto{\pgfqpoint{4.216035in}{1.044583in}}{\pgfqpoint{4.218869in}{1.051425in}}{\pgfqpoint{4.218869in}{1.058558in}}%
\pgfpathcurveto{\pgfqpoint{4.218869in}{1.065691in}}{\pgfqpoint{4.216035in}{1.072532in}}{\pgfqpoint{4.210991in}{1.077576in}}%
\pgfpathcurveto{\pgfqpoint{4.205948in}{1.082620in}}{\pgfqpoint{4.199106in}{1.085454in}}{\pgfqpoint{4.191973in}{1.085454in}}%
\pgfpathcurveto{\pgfqpoint{4.184841in}{1.085454in}}{\pgfqpoint{4.177999in}{1.082620in}}{\pgfqpoint{4.172955in}{1.077576in}}%
\pgfpathcurveto{\pgfqpoint{4.167912in}{1.072532in}}{\pgfqpoint{4.165078in}{1.065691in}}{\pgfqpoint{4.165078in}{1.058558in}}%
\pgfpathcurveto{\pgfqpoint{4.165078in}{1.051425in}}{\pgfqpoint{4.167912in}{1.044583in}}{\pgfqpoint{4.172955in}{1.039540in}}%
\pgfpathcurveto{\pgfqpoint{4.177999in}{1.034496in}}{\pgfqpoint{4.184841in}{1.031662in}}{\pgfqpoint{4.191973in}{1.031662in}}%
\pgfpathclose%
\pgfusepath{stroke,fill}%
\end{pgfscope}%
\begin{pgfscope}%
\pgfpathrectangle{\pgfqpoint{2.867647in}{0.500000in}}{\pgfqpoint{1.764706in}{1.700000in}}%
\pgfusepath{clip}%
\pgfsetbuttcap%
\pgfsetroundjoin%
\definecolor{currentfill}{rgb}{0.978376,0.897317,0.831308}%
\pgfsetfillcolor{currentfill}%
\pgfsetlinewidth{0.311001pt}%
\definecolor{currentstroke}{rgb}{1.000000,1.000000,1.000000}%
\pgfsetstrokecolor{currentstroke}%
\pgfsetdash{}{0pt}%
\pgfpathmoveto{\pgfqpoint{4.120749in}{1.112735in}}%
\pgfpathcurveto{\pgfqpoint{4.127882in}{1.112735in}}{\pgfqpoint{4.134724in}{1.115569in}}{\pgfqpoint{4.139768in}{1.120613in}}%
\pgfpathcurveto{\pgfqpoint{4.144811in}{1.125657in}}{\pgfqpoint{4.147645in}{1.132498in}}{\pgfqpoint{4.147645in}{1.139631in}}%
\pgfpathcurveto{\pgfqpoint{4.147645in}{1.146764in}}{\pgfqpoint{4.144811in}{1.153606in}}{\pgfqpoint{4.139768in}{1.158649in}}%
\pgfpathcurveto{\pgfqpoint{4.134724in}{1.163693in}}{\pgfqpoint{4.127882in}{1.166527in}}{\pgfqpoint{4.120749in}{1.166527in}}%
\pgfpathcurveto{\pgfqpoint{4.113617in}{1.166527in}}{\pgfqpoint{4.106775in}{1.163693in}}{\pgfqpoint{4.101731in}{1.158649in}}%
\pgfpathcurveto{\pgfqpoint{4.096688in}{1.153606in}}{\pgfqpoint{4.093854in}{1.146764in}}{\pgfqpoint{4.093854in}{1.139631in}}%
\pgfpathcurveto{\pgfqpoint{4.093854in}{1.132498in}}{\pgfqpoint{4.096688in}{1.125657in}}{\pgfqpoint{4.101731in}{1.120613in}}%
\pgfpathcurveto{\pgfqpoint{4.106775in}{1.115569in}}{\pgfqpoint{4.113617in}{1.112735in}}{\pgfqpoint{4.120749in}{1.112735in}}%
\pgfpathclose%
\pgfusepath{stroke,fill}%
\end{pgfscope}%
\begin{pgfscope}%
\pgfpathrectangle{\pgfqpoint{2.867647in}{0.500000in}}{\pgfqpoint{1.764706in}{1.700000in}}%
\pgfusepath{clip}%
\pgfsetbuttcap%
\pgfsetroundjoin%
\definecolor{currentfill}{rgb}{0.976287,0.879862,0.805788}%
\pgfsetfillcolor{currentfill}%
\pgfsetlinewidth{0.311001pt}%
\definecolor{currentstroke}{rgb}{1.000000,1.000000,1.000000}%
\pgfsetstrokecolor{currentstroke}%
\pgfsetdash{}{0pt}%
\pgfpathmoveto{\pgfqpoint{4.212854in}{1.129480in}}%
\pgfpathcurveto{\pgfqpoint{4.219987in}{1.129480in}}{\pgfqpoint{4.226829in}{1.132314in}}{\pgfqpoint{4.231873in}{1.137358in}}%
\pgfpathcurveto{\pgfqpoint{4.236916in}{1.142401in}}{\pgfqpoint{4.239750in}{1.149243in}}{\pgfqpoint{4.239750in}{1.156376in}}%
\pgfpathcurveto{\pgfqpoint{4.239750in}{1.163509in}}{\pgfqpoint{4.236916in}{1.170350in}}{\pgfqpoint{4.231873in}{1.175394in}}%
\pgfpathcurveto{\pgfqpoint{4.226829in}{1.180438in}}{\pgfqpoint{4.219987in}{1.183272in}}{\pgfqpoint{4.212854in}{1.183272in}}%
\pgfpathcurveto{\pgfqpoint{4.205722in}{1.183272in}}{\pgfqpoint{4.198880in}{1.180438in}}{\pgfqpoint{4.193836in}{1.175394in}}%
\pgfpathcurveto{\pgfqpoint{4.188793in}{1.170350in}}{\pgfqpoint{4.185959in}{1.163509in}}{\pgfqpoint{4.185959in}{1.156376in}}%
\pgfpathcurveto{\pgfqpoint{4.185959in}{1.149243in}}{\pgfqpoint{4.188793in}{1.142401in}}{\pgfqpoint{4.193836in}{1.137358in}}%
\pgfpathcurveto{\pgfqpoint{4.198880in}{1.132314in}}{\pgfqpoint{4.205722in}{1.129480in}}{\pgfqpoint{4.212854in}{1.129480in}}%
\pgfpathclose%
\pgfusepath{stroke,fill}%
\end{pgfscope}%
\begin{pgfscope}%
\pgfpathrectangle{\pgfqpoint{2.867647in}{0.500000in}}{\pgfqpoint{1.764706in}{1.700000in}}%
\pgfusepath{clip}%
\pgfsetbuttcap%
\pgfsetroundjoin%
\definecolor{currentfill}{rgb}{0.965302,0.713942,0.568499}%
\pgfsetfillcolor{currentfill}%
\pgfsetlinewidth{0.311001pt}%
\definecolor{currentstroke}{rgb}{1.000000,1.000000,1.000000}%
\pgfsetstrokecolor{currentstroke}%
\pgfsetdash{}{0pt}%
\pgfpathmoveto{\pgfqpoint{3.994124in}{0.954508in}}%
\pgfpathcurveto{\pgfqpoint{4.001257in}{0.954508in}}{\pgfqpoint{4.008099in}{0.957342in}}{\pgfqpoint{4.013142in}{0.962385in}}%
\pgfpathcurveto{\pgfqpoint{4.018186in}{0.967429in}}{\pgfqpoint{4.021020in}{0.974271in}}{\pgfqpoint{4.021020in}{0.981403in}}%
\pgfpathcurveto{\pgfqpoint{4.021020in}{0.988536in}}{\pgfqpoint{4.018186in}{0.995378in}}{\pgfqpoint{4.013142in}{1.000422in}}%
\pgfpathcurveto{\pgfqpoint{4.008099in}{1.005465in}}{\pgfqpoint{4.001257in}{1.008299in}}{\pgfqpoint{3.994124in}{1.008299in}}%
\pgfpathcurveto{\pgfqpoint{3.986991in}{1.008299in}}{\pgfqpoint{3.980150in}{1.005465in}}{\pgfqpoint{3.975106in}{1.000422in}}%
\pgfpathcurveto{\pgfqpoint{3.970062in}{0.995378in}}{\pgfqpoint{3.967228in}{0.988536in}}{\pgfqpoint{3.967228in}{0.981403in}}%
\pgfpathcurveto{\pgfqpoint{3.967228in}{0.974271in}}{\pgfqpoint{3.970062in}{0.967429in}}{\pgfqpoint{3.975106in}{0.962385in}}%
\pgfpathcurveto{\pgfqpoint{3.980150in}{0.957342in}}{\pgfqpoint{3.986991in}{0.954508in}}{\pgfqpoint{3.994124in}{0.954508in}}%
\pgfpathclose%
\pgfusepath{stroke,fill}%
\end{pgfscope}%
\begin{pgfscope}%
\pgfpathrectangle{\pgfqpoint{2.867647in}{0.500000in}}{\pgfqpoint{1.764706in}{1.700000in}}%
\pgfusepath{clip}%
\pgfsetbuttcap%
\pgfsetroundjoin%
\definecolor{currentfill}{rgb}{0.981377,0.920617,0.865369}%
\pgfsetfillcolor{currentfill}%
\pgfsetlinewidth{0.311001pt}%
\definecolor{currentstroke}{rgb}{1.000000,1.000000,1.000000}%
\pgfsetstrokecolor{currentstroke}%
\pgfsetdash{}{0pt}%
\pgfpathmoveto{\pgfqpoint{4.172326in}{1.285774in}}%
\pgfpathcurveto{\pgfqpoint{4.179458in}{1.285774in}}{\pgfqpoint{4.186300in}{1.288608in}}{\pgfqpoint{4.191344in}{1.293652in}}%
\pgfpathcurveto{\pgfqpoint{4.196387in}{1.298696in}}{\pgfqpoint{4.199221in}{1.305537in}}{\pgfqpoint{4.199221in}{1.312670in}}%
\pgfpathcurveto{\pgfqpoint{4.199221in}{1.319803in}}{\pgfqpoint{4.196387in}{1.326645in}}{\pgfqpoint{4.191344in}{1.331688in}}%
\pgfpathcurveto{\pgfqpoint{4.186300in}{1.336732in}}{\pgfqpoint{4.179458in}{1.339566in}}{\pgfqpoint{4.172326in}{1.339566in}}%
\pgfpathcurveto{\pgfqpoint{4.165193in}{1.339566in}}{\pgfqpoint{4.158351in}{1.336732in}}{\pgfqpoint{4.153307in}{1.331688in}}%
\pgfpathcurveto{\pgfqpoint{4.148264in}{1.326645in}}{\pgfqpoint{4.145430in}{1.319803in}}{\pgfqpoint{4.145430in}{1.312670in}}%
\pgfpathcurveto{\pgfqpoint{4.145430in}{1.305537in}}{\pgfqpoint{4.148264in}{1.298696in}}{\pgfqpoint{4.153307in}{1.293652in}}%
\pgfpathcurveto{\pgfqpoint{4.158351in}{1.288608in}}{\pgfqpoint{4.165193in}{1.285774in}}{\pgfqpoint{4.172326in}{1.285774in}}%
\pgfpathclose%
\pgfusepath{stroke,fill}%
\end{pgfscope}%
\begin{pgfscope}%
\pgfpathrectangle{\pgfqpoint{2.867647in}{0.500000in}}{\pgfqpoint{1.764706in}{1.700000in}}%
\pgfusepath{clip}%
\pgfsetbuttcap%
\pgfsetroundjoin%
\definecolor{currentfill}{rgb}{0.980678,0.914765,0.856766}%
\pgfsetfillcolor{currentfill}%
\pgfsetlinewidth{0.311001pt}%
\definecolor{currentstroke}{rgb}{1.000000,1.000000,1.000000}%
\pgfsetstrokecolor{currentstroke}%
\pgfsetdash{}{0pt}%
\pgfpathmoveto{\pgfqpoint{4.208166in}{1.350920in}}%
\pgfpathcurveto{\pgfqpoint{4.215299in}{1.350920in}}{\pgfqpoint{4.222140in}{1.353754in}}{\pgfqpoint{4.227184in}{1.358798in}}%
\pgfpathcurveto{\pgfqpoint{4.232228in}{1.363841in}}{\pgfqpoint{4.235062in}{1.370683in}}{\pgfqpoint{4.235062in}{1.377816in}}%
\pgfpathcurveto{\pgfqpoint{4.235062in}{1.384949in}}{\pgfqpoint{4.232228in}{1.391790in}}{\pgfqpoint{4.227184in}{1.396834in}}%
\pgfpathcurveto{\pgfqpoint{4.222140in}{1.401878in}}{\pgfqpoint{4.215299in}{1.404711in}}{\pgfqpoint{4.208166in}{1.404711in}}%
\pgfpathcurveto{\pgfqpoint{4.201033in}{1.404711in}}{\pgfqpoint{4.194192in}{1.401878in}}{\pgfqpoint{4.189148in}{1.396834in}}%
\pgfpathcurveto{\pgfqpoint{4.184104in}{1.391790in}}{\pgfqpoint{4.181270in}{1.384949in}}{\pgfqpoint{4.181270in}{1.377816in}}%
\pgfpathcurveto{\pgfqpoint{4.181270in}{1.370683in}}{\pgfqpoint{4.184104in}{1.363841in}}{\pgfqpoint{4.189148in}{1.358798in}}%
\pgfpathcurveto{\pgfqpoint{4.194192in}{1.353754in}}{\pgfqpoint{4.201033in}{1.350920in}}{\pgfqpoint{4.208166in}{1.350920in}}%
\pgfpathclose%
\pgfusepath{stroke,fill}%
\end{pgfscope}%
\begin{pgfscope}%
\pgfpathrectangle{\pgfqpoint{2.867647in}{0.500000in}}{\pgfqpoint{1.764706in}{1.700000in}}%
\pgfusepath{clip}%
\pgfsetbuttcap%
\pgfsetroundjoin%
\definecolor{currentfill}{rgb}{0.957848,0.512613,0.357119}%
\pgfsetfillcolor{currentfill}%
\pgfsetlinewidth{0.311001pt}%
\definecolor{currentstroke}{rgb}{1.000000,1.000000,1.000000}%
\pgfsetstrokecolor{currentstroke}%
\pgfsetdash{}{0pt}%
\pgfpathmoveto{\pgfqpoint{3.890088in}{1.705433in}}%
\pgfpathcurveto{\pgfqpoint{3.897221in}{1.705433in}}{\pgfqpoint{3.904063in}{1.708267in}}{\pgfqpoint{3.909107in}{1.713310in}}%
\pgfpathcurveto{\pgfqpoint{3.914150in}{1.718354in}}{\pgfqpoint{3.916984in}{1.725196in}}{\pgfqpoint{3.916984in}{1.732328in}}%
\pgfpathcurveto{\pgfqpoint{3.916984in}{1.739461in}}{\pgfqpoint{3.914150in}{1.746303in}}{\pgfqpoint{3.909107in}{1.751347in}}%
\pgfpathcurveto{\pgfqpoint{3.904063in}{1.756390in}}{\pgfqpoint{3.897221in}{1.759224in}}{\pgfqpoint{3.890088in}{1.759224in}}%
\pgfpathcurveto{\pgfqpoint{3.882956in}{1.759224in}}{\pgfqpoint{3.876114in}{1.756390in}}{\pgfqpoint{3.871070in}{1.751347in}}%
\pgfpathcurveto{\pgfqpoint{3.866027in}{1.746303in}}{\pgfqpoint{3.863193in}{1.739461in}}{\pgfqpoint{3.863193in}{1.732328in}}%
\pgfpathcurveto{\pgfqpoint{3.863193in}{1.725196in}}{\pgfqpoint{3.866027in}{1.718354in}}{\pgfqpoint{3.871070in}{1.713310in}}%
\pgfpathcurveto{\pgfqpoint{3.876114in}{1.708267in}}{\pgfqpoint{3.882956in}{1.705433in}}{\pgfqpoint{3.890088in}{1.705433in}}%
\pgfpathclose%
\pgfusepath{stroke,fill}%
\end{pgfscope}%
\begin{pgfscope}%
\pgfpathrectangle{\pgfqpoint{2.867647in}{0.500000in}}{\pgfqpoint{1.764706in}{1.700000in}}%
\pgfusepath{clip}%
\pgfsetbuttcap%
\pgfsetroundjoin%
\definecolor{currentfill}{rgb}{0.981377,0.920617,0.865369}%
\pgfsetfillcolor{currentfill}%
\pgfsetlinewidth{0.311001pt}%
\definecolor{currentstroke}{rgb}{1.000000,1.000000,1.000000}%
\pgfsetstrokecolor{currentstroke}%
\pgfsetdash{}{0pt}%
\pgfpathmoveto{\pgfqpoint{4.177991in}{1.173814in}}%
\pgfpathcurveto{\pgfqpoint{4.185124in}{1.173814in}}{\pgfqpoint{4.191965in}{1.176647in}}{\pgfqpoint{4.197009in}{1.181691in}}%
\pgfpathcurveto{\pgfqpoint{4.202053in}{1.186735in}}{\pgfqpoint{4.204887in}{1.193576in}}{\pgfqpoint{4.204887in}{1.200709in}}%
\pgfpathcurveto{\pgfqpoint{4.204887in}{1.207842in}}{\pgfqpoint{4.202053in}{1.214684in}}{\pgfqpoint{4.197009in}{1.219727in}}%
\pgfpathcurveto{\pgfqpoint{4.191965in}{1.224771in}}{\pgfqpoint{4.185124in}{1.227605in}}{\pgfqpoint{4.177991in}{1.227605in}}%
\pgfpathcurveto{\pgfqpoint{4.170858in}{1.227605in}}{\pgfqpoint{4.164016in}{1.224771in}}{\pgfqpoint{4.158973in}{1.219727in}}%
\pgfpathcurveto{\pgfqpoint{4.153929in}{1.214684in}}{\pgfqpoint{4.151095in}{1.207842in}}{\pgfqpoint{4.151095in}{1.200709in}}%
\pgfpathcurveto{\pgfqpoint{4.151095in}{1.193576in}}{\pgfqpoint{4.153929in}{1.186735in}}{\pgfqpoint{4.158973in}{1.181691in}}%
\pgfpathcurveto{\pgfqpoint{4.164016in}{1.176647in}}{\pgfqpoint{4.170858in}{1.173814in}}{\pgfqpoint{4.177991in}{1.173814in}}%
\pgfpathclose%
\pgfusepath{stroke,fill}%
\end{pgfscope}%
\begin{pgfscope}%
\pgfpathrectangle{\pgfqpoint{2.867647in}{0.500000in}}{\pgfqpoint{1.764706in}{1.700000in}}%
\pgfusepath{clip}%
\pgfsetbuttcap%
\pgfsetroundjoin%
\definecolor{currentfill}{rgb}{0.974412,0.862387,0.780156}%
\pgfsetfillcolor{currentfill}%
\pgfsetlinewidth{0.311001pt}%
\definecolor{currentstroke}{rgb}{1.000000,1.000000,1.000000}%
\pgfsetstrokecolor{currentstroke}%
\pgfsetdash{}{0pt}%
\pgfpathmoveto{\pgfqpoint{4.209417in}{1.543876in}}%
\pgfpathcurveto{\pgfqpoint{4.216549in}{1.543876in}}{\pgfqpoint{4.223391in}{1.546710in}}{\pgfqpoint{4.228435in}{1.551754in}}%
\pgfpathcurveto{\pgfqpoint{4.233478in}{1.556797in}}{\pgfqpoint{4.236312in}{1.563639in}}{\pgfqpoint{4.236312in}{1.570772in}}%
\pgfpathcurveto{\pgfqpoint{4.236312in}{1.577905in}}{\pgfqpoint{4.233478in}{1.584746in}}{\pgfqpoint{4.228435in}{1.589790in}}%
\pgfpathcurveto{\pgfqpoint{4.223391in}{1.594834in}}{\pgfqpoint{4.216549in}{1.597668in}}{\pgfqpoint{4.209417in}{1.597668in}}%
\pgfpathcurveto{\pgfqpoint{4.202284in}{1.597668in}}{\pgfqpoint{4.195442in}{1.594834in}}{\pgfqpoint{4.190398in}{1.589790in}}%
\pgfpathcurveto{\pgfqpoint{4.185355in}{1.584746in}}{\pgfqpoint{4.182521in}{1.577905in}}{\pgfqpoint{4.182521in}{1.570772in}}%
\pgfpathcurveto{\pgfqpoint{4.182521in}{1.563639in}}{\pgfqpoint{4.185355in}{1.556797in}}{\pgfqpoint{4.190398in}{1.551754in}}%
\pgfpathcurveto{\pgfqpoint{4.195442in}{1.546710in}}{\pgfqpoint{4.202284in}{1.543876in}}{\pgfqpoint{4.209417in}{1.543876in}}%
\pgfpathclose%
\pgfusepath{stroke,fill}%
\end{pgfscope}%
\begin{pgfscope}%
\pgfpathrectangle{\pgfqpoint{2.867647in}{0.500000in}}{\pgfqpoint{1.764706in}{1.700000in}}%
\pgfusepath{clip}%
\pgfsetbuttcap%
\pgfsetroundjoin%
\definecolor{currentfill}{rgb}{0.974412,0.862387,0.780156}%
\pgfsetfillcolor{currentfill}%
\pgfsetlinewidth{0.311001pt}%
\definecolor{currentstroke}{rgb}{1.000000,1.000000,1.000000}%
\pgfsetstrokecolor{currentstroke}%
\pgfsetdash{}{0pt}%
\pgfpathmoveto{\pgfqpoint{4.081393in}{1.559638in}}%
\pgfpathcurveto{\pgfqpoint{4.088526in}{1.559638in}}{\pgfqpoint{4.095368in}{1.562472in}}{\pgfqpoint{4.100411in}{1.567515in}}%
\pgfpathcurveto{\pgfqpoint{4.105455in}{1.572559in}}{\pgfqpoint{4.108289in}{1.579401in}}{\pgfqpoint{4.108289in}{1.586534in}}%
\pgfpathcurveto{\pgfqpoint{4.108289in}{1.593666in}}{\pgfqpoint{4.105455in}{1.600508in}}{\pgfqpoint{4.100411in}{1.605552in}}%
\pgfpathcurveto{\pgfqpoint{4.095368in}{1.610595in}}{\pgfqpoint{4.088526in}{1.613429in}}{\pgfqpoint{4.081393in}{1.613429in}}%
\pgfpathcurveto{\pgfqpoint{4.074260in}{1.613429in}}{\pgfqpoint{4.067419in}{1.610595in}}{\pgfqpoint{4.062375in}{1.605552in}}%
\pgfpathcurveto{\pgfqpoint{4.057331in}{1.600508in}}{\pgfqpoint{4.054497in}{1.593666in}}{\pgfqpoint{4.054497in}{1.586534in}}%
\pgfpathcurveto{\pgfqpoint{4.054497in}{1.579401in}}{\pgfqpoint{4.057331in}{1.572559in}}{\pgfqpoint{4.062375in}{1.567515in}}%
\pgfpathcurveto{\pgfqpoint{4.067419in}{1.562472in}}{\pgfqpoint{4.074260in}{1.559638in}}{\pgfqpoint{4.081393in}{1.559638in}}%
\pgfpathclose%
\pgfusepath{stroke,fill}%
\end{pgfscope}%
\begin{pgfscope}%
\pgfpathrectangle{\pgfqpoint{2.867647in}{0.500000in}}{\pgfqpoint{1.764706in}{1.700000in}}%
\pgfusepath{clip}%
\pgfsetbuttcap%
\pgfsetroundjoin%
\definecolor{currentfill}{rgb}{0.971202,0.827364,0.728520}%
\pgfsetfillcolor{currentfill}%
\pgfsetlinewidth{0.311001pt}%
\definecolor{currentstroke}{rgb}{1.000000,1.000000,1.000000}%
\pgfsetstrokecolor{currentstroke}%
\pgfsetdash{}{0pt}%
\pgfpathmoveto{\pgfqpoint{4.270549in}{1.323648in}}%
\pgfpathcurveto{\pgfqpoint{4.277682in}{1.323648in}}{\pgfqpoint{4.284524in}{1.326481in}}{\pgfqpoint{4.289567in}{1.331525in}}%
\pgfpathcurveto{\pgfqpoint{4.294611in}{1.336569in}}{\pgfqpoint{4.297445in}{1.343410in}}{\pgfqpoint{4.297445in}{1.350543in}}%
\pgfpathcurveto{\pgfqpoint{4.297445in}{1.357676in}}{\pgfqpoint{4.294611in}{1.364518in}}{\pgfqpoint{4.289567in}{1.369561in}}%
\pgfpathcurveto{\pgfqpoint{4.284524in}{1.374605in}}{\pgfqpoint{4.277682in}{1.377439in}}{\pgfqpoint{4.270549in}{1.377439in}}%
\pgfpathcurveto{\pgfqpoint{4.263417in}{1.377439in}}{\pgfqpoint{4.256575in}{1.374605in}}{\pgfqpoint{4.251531in}{1.369561in}}%
\pgfpathcurveto{\pgfqpoint{4.246488in}{1.364518in}}{\pgfqpoint{4.243654in}{1.357676in}}{\pgfqpoint{4.243654in}{1.350543in}}%
\pgfpathcurveto{\pgfqpoint{4.243654in}{1.343410in}}{\pgfqpoint{4.246488in}{1.336569in}}{\pgfqpoint{4.251531in}{1.331525in}}%
\pgfpathcurveto{\pgfqpoint{4.256575in}{1.326481in}}{\pgfqpoint{4.263417in}{1.323648in}}{\pgfqpoint{4.270549in}{1.323648in}}%
\pgfpathclose%
\pgfusepath{stroke,fill}%
\end{pgfscope}%
\begin{pgfscope}%
\pgfpathrectangle{\pgfqpoint{2.867647in}{0.500000in}}{\pgfqpoint{1.764706in}{1.700000in}}%
\pgfusepath{clip}%
\pgfsetbuttcap%
\pgfsetroundjoin%
\definecolor{currentfill}{rgb}{0.978376,0.897317,0.831308}%
\pgfsetfillcolor{currentfill}%
\pgfsetlinewidth{0.311001pt}%
\definecolor{currentstroke}{rgb}{1.000000,1.000000,1.000000}%
\pgfsetstrokecolor{currentstroke}%
\pgfsetdash{}{0pt}%
\pgfpathmoveto{\pgfqpoint{4.127947in}{1.169239in}}%
\pgfpathcurveto{\pgfqpoint{4.135080in}{1.169239in}}{\pgfqpoint{4.141922in}{1.172072in}}{\pgfqpoint{4.146965in}{1.177116in}}%
\pgfpathcurveto{\pgfqpoint{4.152009in}{1.182160in}}{\pgfqpoint{4.154843in}{1.189001in}}{\pgfqpoint{4.154843in}{1.196134in}}%
\pgfpathcurveto{\pgfqpoint{4.154843in}{1.203267in}}{\pgfqpoint{4.152009in}{1.210109in}}{\pgfqpoint{4.146965in}{1.215152in}}%
\pgfpathcurveto{\pgfqpoint{4.141922in}{1.220196in}}{\pgfqpoint{4.135080in}{1.223030in}}{\pgfqpoint{4.127947in}{1.223030in}}%
\pgfpathcurveto{\pgfqpoint{4.120814in}{1.223030in}}{\pgfqpoint{4.113973in}{1.220196in}}{\pgfqpoint{4.108929in}{1.215152in}}%
\pgfpathcurveto{\pgfqpoint{4.103885in}{1.210109in}}{\pgfqpoint{4.101052in}{1.203267in}}{\pgfqpoint{4.101052in}{1.196134in}}%
\pgfpathcurveto{\pgfqpoint{4.101052in}{1.189001in}}{\pgfqpoint{4.103885in}{1.182160in}}{\pgfqpoint{4.108929in}{1.177116in}}%
\pgfpathcurveto{\pgfqpoint{4.113973in}{1.172072in}}{\pgfqpoint{4.120814in}{1.169239in}}{\pgfqpoint{4.127947in}{1.169239in}}%
\pgfpathclose%
\pgfusepath{stroke,fill}%
\end{pgfscope}%
\begin{pgfscope}%
\pgfpathrectangle{\pgfqpoint{2.867647in}{0.500000in}}{\pgfqpoint{1.764706in}{1.700000in}}%
\pgfusepath{clip}%
\pgfsetbuttcap%
\pgfsetroundjoin%
\definecolor{currentfill}{rgb}{0.975644,0.874038,0.797253}%
\pgfsetfillcolor{currentfill}%
\pgfsetlinewidth{0.311001pt}%
\definecolor{currentstroke}{rgb}{1.000000,1.000000,1.000000}%
\pgfsetstrokecolor{currentstroke}%
\pgfsetdash{}{0pt}%
\pgfpathmoveto{\pgfqpoint{4.206919in}{1.530798in}}%
\pgfpathcurveto{\pgfqpoint{4.214052in}{1.530798in}}{\pgfqpoint{4.220894in}{1.533631in}}{\pgfqpoint{4.225937in}{1.538675in}}%
\pgfpathcurveto{\pgfqpoint{4.230981in}{1.543719in}}{\pgfqpoint{4.233815in}{1.550560in}}{\pgfqpoint{4.233815in}{1.557693in}}%
\pgfpathcurveto{\pgfqpoint{4.233815in}{1.564826in}}{\pgfqpoint{4.230981in}{1.571668in}}{\pgfqpoint{4.225937in}{1.576711in}}%
\pgfpathcurveto{\pgfqpoint{4.220894in}{1.581755in}}{\pgfqpoint{4.214052in}{1.584589in}}{\pgfqpoint{4.206919in}{1.584589in}}%
\pgfpathcurveto{\pgfqpoint{4.199786in}{1.584589in}}{\pgfqpoint{4.192945in}{1.581755in}}{\pgfqpoint{4.187901in}{1.576711in}}%
\pgfpathcurveto{\pgfqpoint{4.182857in}{1.571668in}}{\pgfqpoint{4.180023in}{1.564826in}}{\pgfqpoint{4.180023in}{1.557693in}}%
\pgfpathcurveto{\pgfqpoint{4.180023in}{1.550560in}}{\pgfqpoint{4.182857in}{1.543719in}}{\pgfqpoint{4.187901in}{1.538675in}}%
\pgfpathcurveto{\pgfqpoint{4.192945in}{1.533631in}}{\pgfqpoint{4.199786in}{1.530798in}}{\pgfqpoint{4.206919in}{1.530798in}}%
\pgfpathclose%
\pgfusepath{stroke,fill}%
\end{pgfscope}%
\begin{pgfscope}%
\pgfpathrectangle{\pgfqpoint{2.867647in}{0.500000in}}{\pgfqpoint{1.764706in}{1.700000in}}%
\pgfusepath{clip}%
\pgfsetbuttcap%
\pgfsetroundjoin%
\definecolor{currentfill}{rgb}{0.964799,0.689101,0.537560}%
\pgfsetfillcolor{currentfill}%
\pgfsetlinewidth{0.311001pt}%
\definecolor{currentstroke}{rgb}{1.000000,1.000000,1.000000}%
\pgfsetstrokecolor{currentstroke}%
\pgfsetdash{}{0pt}%
\pgfpathmoveto{\pgfqpoint{3.983067in}{0.950421in}}%
\pgfpathcurveto{\pgfqpoint{3.990200in}{0.950421in}}{\pgfqpoint{3.997042in}{0.953255in}}{\pgfqpoint{4.002085in}{0.958299in}}%
\pgfpathcurveto{\pgfqpoint{4.007129in}{0.963343in}}{\pgfqpoint{4.009963in}{0.970184in}}{\pgfqpoint{4.009963in}{0.977317in}}%
\pgfpathcurveto{\pgfqpoint{4.009963in}{0.984450in}}{\pgfqpoint{4.007129in}{0.991292in}}{\pgfqpoint{4.002085in}{0.996335in}}%
\pgfpathcurveto{\pgfqpoint{3.997042in}{1.001379in}}{\pgfqpoint{3.990200in}{1.004213in}}{\pgfqpoint{3.983067in}{1.004213in}}%
\pgfpathcurveto{\pgfqpoint{3.975934in}{1.004213in}}{\pgfqpoint{3.969093in}{1.001379in}}{\pgfqpoint{3.964049in}{0.996335in}}%
\pgfpathcurveto{\pgfqpoint{3.959005in}{0.991292in}}{\pgfqpoint{3.956172in}{0.984450in}}{\pgfqpoint{3.956172in}{0.977317in}}%
\pgfpathcurveto{\pgfqpoint{3.956172in}{0.970184in}}{\pgfqpoint{3.959005in}{0.963343in}}{\pgfqpoint{3.964049in}{0.958299in}}%
\pgfpathcurveto{\pgfqpoint{3.969093in}{0.953255in}}{\pgfqpoint{3.975934in}{0.950421in}}{\pgfqpoint{3.983067in}{0.950421in}}%
\pgfpathclose%
\pgfusepath{stroke,fill}%
\end{pgfscope}%
\begin{pgfscope}%
\pgfpathrectangle{\pgfqpoint{2.867647in}{0.500000in}}{\pgfqpoint{1.764706in}{1.700000in}}%
\pgfusepath{clip}%
\pgfsetbuttcap%
\pgfsetroundjoin%
\definecolor{currentfill}{rgb}{0.977657,0.891500,0.822809}%
\pgfsetfillcolor{currentfill}%
\pgfsetlinewidth{0.311001pt}%
\definecolor{currentstroke}{rgb}{1.000000,1.000000,1.000000}%
\pgfsetstrokecolor{currentstroke}%
\pgfsetdash{}{0pt}%
\pgfpathmoveto{\pgfqpoint{4.207408in}{1.491689in}}%
\pgfpathcurveto{\pgfqpoint{4.214540in}{1.491689in}}{\pgfqpoint{4.221382in}{1.494523in}}{\pgfqpoint{4.226426in}{1.499567in}}%
\pgfpathcurveto{\pgfqpoint{4.231469in}{1.504610in}}{\pgfqpoint{4.234303in}{1.511452in}}{\pgfqpoint{4.234303in}{1.518585in}}%
\pgfpathcurveto{\pgfqpoint{4.234303in}{1.525718in}}{\pgfqpoint{4.231469in}{1.532559in}}{\pgfqpoint{4.226426in}{1.537603in}}%
\pgfpathcurveto{\pgfqpoint{4.221382in}{1.542647in}}{\pgfqpoint{4.214540in}{1.545480in}}{\pgfqpoint{4.207408in}{1.545480in}}%
\pgfpathcurveto{\pgfqpoint{4.200275in}{1.545480in}}{\pgfqpoint{4.193433in}{1.542647in}}{\pgfqpoint{4.188389in}{1.537603in}}%
\pgfpathcurveto{\pgfqpoint{4.183346in}{1.532559in}}{\pgfqpoint{4.180512in}{1.525718in}}{\pgfqpoint{4.180512in}{1.518585in}}%
\pgfpathcurveto{\pgfqpoint{4.180512in}{1.511452in}}{\pgfqpoint{4.183346in}{1.504610in}}{\pgfqpoint{4.188389in}{1.499567in}}%
\pgfpathcurveto{\pgfqpoint{4.193433in}{1.494523in}}{\pgfqpoint{4.200275in}{1.491689in}}{\pgfqpoint{4.207408in}{1.491689in}}%
\pgfpathclose%
\pgfusepath{stroke,fill}%
\end{pgfscope}%
\begin{pgfscope}%
\pgfpathrectangle{\pgfqpoint{2.867647in}{0.500000in}}{\pgfqpoint{1.764706in}{1.700000in}}%
\pgfusepath{clip}%
\pgfsetbuttcap%
\pgfsetroundjoin%
\definecolor{currentfill}{rgb}{0.968105,0.786346,0.667739}%
\pgfsetfillcolor{currentfill}%
\pgfsetlinewidth{0.311001pt}%
\definecolor{currentstroke}{rgb}{1.000000,1.000000,1.000000}%
\pgfsetstrokecolor{currentstroke}%
\pgfsetdash{}{0pt}%
\pgfpathmoveto{\pgfqpoint{4.057569in}{1.714892in}}%
\pgfpathcurveto{\pgfqpoint{4.064702in}{1.714892in}}{\pgfqpoint{4.071544in}{1.717726in}}{\pgfqpoint{4.076587in}{1.722769in}}%
\pgfpathcurveto{\pgfqpoint{4.081631in}{1.727813in}}{\pgfqpoint{4.084465in}{1.734654in}}{\pgfqpoint{4.084465in}{1.741787in}}%
\pgfpathcurveto{\pgfqpoint{4.084465in}{1.748920in}}{\pgfqpoint{4.081631in}{1.755762in}}{\pgfqpoint{4.076587in}{1.760805in}}%
\pgfpathcurveto{\pgfqpoint{4.071544in}{1.765849in}}{\pgfqpoint{4.064702in}{1.768683in}}{\pgfqpoint{4.057569in}{1.768683in}}%
\pgfpathcurveto{\pgfqpoint{4.050436in}{1.768683in}}{\pgfqpoint{4.043595in}{1.765849in}}{\pgfqpoint{4.038551in}{1.760805in}}%
\pgfpathcurveto{\pgfqpoint{4.033507in}{1.755762in}}{\pgfqpoint{4.030673in}{1.748920in}}{\pgfqpoint{4.030673in}{1.741787in}}%
\pgfpathcurveto{\pgfqpoint{4.030673in}{1.734654in}}{\pgfqpoint{4.033507in}{1.727813in}}{\pgfqpoint{4.038551in}{1.722769in}}%
\pgfpathcurveto{\pgfqpoint{4.043595in}{1.717726in}}{\pgfqpoint{4.050436in}{1.714892in}}{\pgfqpoint{4.057569in}{1.714892in}}%
\pgfpathclose%
\pgfusepath{stroke,fill}%
\end{pgfscope}%
\begin{pgfscope}%
\pgfpathrectangle{\pgfqpoint{2.867647in}{0.500000in}}{\pgfqpoint{1.764706in}{1.700000in}}%
\pgfusepath{clip}%
\pgfsetbuttcap%
\pgfsetroundjoin%
\definecolor{currentfill}{rgb}{0.976961,0.885681,0.814303}%
\pgfsetfillcolor{currentfill}%
\pgfsetlinewidth{0.311001pt}%
\definecolor{currentstroke}{rgb}{1.000000,1.000000,1.000000}%
\pgfsetstrokecolor{currentstroke}%
\pgfsetdash{}{0pt}%
\pgfpathmoveto{\pgfqpoint{4.202729in}{1.522208in}}%
\pgfpathcurveto{\pgfqpoint{4.209862in}{1.522208in}}{\pgfqpoint{4.216703in}{1.525042in}}{\pgfqpoint{4.221747in}{1.530086in}}%
\pgfpathcurveto{\pgfqpoint{4.226791in}{1.535130in}}{\pgfqpoint{4.229625in}{1.541971in}}{\pgfqpoint{4.229625in}{1.549104in}}%
\pgfpathcurveto{\pgfqpoint{4.229625in}{1.556237in}}{\pgfqpoint{4.226791in}{1.563079in}}{\pgfqpoint{4.221747in}{1.568122in}}%
\pgfpathcurveto{\pgfqpoint{4.216703in}{1.573166in}}{\pgfqpoint{4.209862in}{1.576000in}}{\pgfqpoint{4.202729in}{1.576000in}}%
\pgfpathcurveto{\pgfqpoint{4.195596in}{1.576000in}}{\pgfqpoint{4.188754in}{1.573166in}}{\pgfqpoint{4.183711in}{1.568122in}}%
\pgfpathcurveto{\pgfqpoint{4.178667in}{1.563079in}}{\pgfqpoint{4.175833in}{1.556237in}}{\pgfqpoint{4.175833in}{1.549104in}}%
\pgfpathcurveto{\pgfqpoint{4.175833in}{1.541971in}}{\pgfqpoint{4.178667in}{1.535130in}}{\pgfqpoint{4.183711in}{1.530086in}}%
\pgfpathcurveto{\pgfqpoint{4.188754in}{1.525042in}}{\pgfqpoint{4.195596in}{1.522208in}}{\pgfqpoint{4.202729in}{1.522208in}}%
\pgfpathclose%
\pgfusepath{stroke,fill}%
\end{pgfscope}%
\begin{pgfscope}%
\pgfpathrectangle{\pgfqpoint{2.867647in}{0.500000in}}{\pgfqpoint{1.764706in}{1.700000in}}%
\pgfusepath{clip}%
\pgfsetbuttcap%
\pgfsetroundjoin%
\definecolor{currentfill}{rgb}{0.965753,0.732351,0.592427}%
\pgfsetfillcolor{currentfill}%
\pgfsetlinewidth{0.311001pt}%
\definecolor{currentstroke}{rgb}{1.000000,1.000000,1.000000}%
\pgfsetstrokecolor{currentstroke}%
\pgfsetdash{}{0pt}%
\pgfpathmoveto{\pgfqpoint{4.044223in}{1.496411in}}%
\pgfpathcurveto{\pgfqpoint{4.051356in}{1.496411in}}{\pgfqpoint{4.058197in}{1.499245in}}{\pgfqpoint{4.063241in}{1.504288in}}%
\pgfpathcurveto{\pgfqpoint{4.068285in}{1.509332in}}{\pgfqpoint{4.071119in}{1.516174in}}{\pgfqpoint{4.071119in}{1.523306in}}%
\pgfpathcurveto{\pgfqpoint{4.071119in}{1.530439in}}{\pgfqpoint{4.068285in}{1.537281in}}{\pgfqpoint{4.063241in}{1.542325in}}%
\pgfpathcurveto{\pgfqpoint{4.058197in}{1.547368in}}{\pgfqpoint{4.051356in}{1.550202in}}{\pgfqpoint{4.044223in}{1.550202in}}%
\pgfpathcurveto{\pgfqpoint{4.037090in}{1.550202in}}{\pgfqpoint{4.030248in}{1.547368in}}{\pgfqpoint{4.025205in}{1.542325in}}%
\pgfpathcurveto{\pgfqpoint{4.020161in}{1.537281in}}{\pgfqpoint{4.017327in}{1.530439in}}{\pgfqpoint{4.017327in}{1.523306in}}%
\pgfpathcurveto{\pgfqpoint{4.017327in}{1.516174in}}{\pgfqpoint{4.020161in}{1.509332in}}{\pgfqpoint{4.025205in}{1.504288in}}%
\pgfpathcurveto{\pgfqpoint{4.030248in}{1.499245in}}{\pgfqpoint{4.037090in}{1.496411in}}{\pgfqpoint{4.044223in}{1.496411in}}%
\pgfpathclose%
\pgfusepath{stroke,fill}%
\end{pgfscope}%
\begin{pgfscope}%
\pgfpathrectangle{\pgfqpoint{2.867647in}{0.500000in}}{\pgfqpoint{1.764706in}{1.700000in}}%
\pgfusepath{clip}%
\pgfsetbuttcap%
\pgfsetroundjoin%
\definecolor{currentfill}{rgb}{0.975644,0.874038,0.797253}%
\pgfsetfillcolor{currentfill}%
\pgfsetlinewidth{0.311001pt}%
\definecolor{currentstroke}{rgb}{1.000000,1.000000,1.000000}%
\pgfsetstrokecolor{currentstroke}%
\pgfsetdash{}{0pt}%
\pgfpathmoveto{\pgfqpoint{4.232413in}{1.440542in}}%
\pgfpathcurveto{\pgfqpoint{4.239546in}{1.440542in}}{\pgfqpoint{4.246388in}{1.443376in}}{\pgfqpoint{4.251432in}{1.448420in}}%
\pgfpathcurveto{\pgfqpoint{4.256475in}{1.453463in}}{\pgfqpoint{4.259309in}{1.460305in}}{\pgfqpoint{4.259309in}{1.467438in}}%
\pgfpathcurveto{\pgfqpoint{4.259309in}{1.474571in}}{\pgfqpoint{4.256475in}{1.481412in}}{\pgfqpoint{4.251432in}{1.486456in}}%
\pgfpathcurveto{\pgfqpoint{4.246388in}{1.491500in}}{\pgfqpoint{4.239546in}{1.494334in}}{\pgfqpoint{4.232413in}{1.494334in}}%
\pgfpathcurveto{\pgfqpoint{4.225281in}{1.494334in}}{\pgfqpoint{4.218439in}{1.491500in}}{\pgfqpoint{4.213395in}{1.486456in}}%
\pgfpathcurveto{\pgfqpoint{4.208352in}{1.481412in}}{\pgfqpoint{4.205518in}{1.474571in}}{\pgfqpoint{4.205518in}{1.467438in}}%
\pgfpathcurveto{\pgfqpoint{4.205518in}{1.460305in}}{\pgfqpoint{4.208352in}{1.453463in}}{\pgfqpoint{4.213395in}{1.448420in}}%
\pgfpathcurveto{\pgfqpoint{4.218439in}{1.443376in}}{\pgfqpoint{4.225281in}{1.440542in}}{\pgfqpoint{4.232413in}{1.440542in}}%
\pgfpathclose%
\pgfusepath{stroke,fill}%
\end{pgfscope}%
\begin{pgfscope}%
\pgfpathrectangle{\pgfqpoint{2.867647in}{0.500000in}}{\pgfqpoint{1.764706in}{1.700000in}}%
\pgfusepath{clip}%
\pgfsetbuttcap%
\pgfsetroundjoin%
\definecolor{currentfill}{rgb}{0.973832,0.856556,0.771584}%
\pgfsetfillcolor{currentfill}%
\pgfsetlinewidth{0.311001pt}%
\definecolor{currentstroke}{rgb}{1.000000,1.000000,1.000000}%
\pgfsetstrokecolor{currentstroke}%
\pgfsetdash{}{0pt}%
\pgfpathmoveto{\pgfqpoint{4.124397in}{1.361892in}}%
\pgfpathcurveto{\pgfqpoint{4.131530in}{1.361892in}}{\pgfqpoint{4.138372in}{1.364726in}}{\pgfqpoint{4.143415in}{1.369770in}}%
\pgfpathcurveto{\pgfqpoint{4.148459in}{1.374813in}}{\pgfqpoint{4.151293in}{1.381655in}}{\pgfqpoint{4.151293in}{1.388788in}}%
\pgfpathcurveto{\pgfqpoint{4.151293in}{1.395921in}}{\pgfqpoint{4.148459in}{1.402762in}}{\pgfqpoint{4.143415in}{1.407806in}}%
\pgfpathcurveto{\pgfqpoint{4.138372in}{1.412850in}}{\pgfqpoint{4.131530in}{1.415684in}}{\pgfqpoint{4.124397in}{1.415684in}}%
\pgfpathcurveto{\pgfqpoint{4.117264in}{1.415684in}}{\pgfqpoint{4.110423in}{1.412850in}}{\pgfqpoint{4.105379in}{1.407806in}}%
\pgfpathcurveto{\pgfqpoint{4.100335in}{1.402762in}}{\pgfqpoint{4.097501in}{1.395921in}}{\pgfqpoint{4.097501in}{1.388788in}}%
\pgfpathcurveto{\pgfqpoint{4.097501in}{1.381655in}}{\pgfqpoint{4.100335in}{1.374813in}}{\pgfqpoint{4.105379in}{1.369770in}}%
\pgfpathcurveto{\pgfqpoint{4.110423in}{1.364726in}}{\pgfqpoint{4.117264in}{1.361892in}}{\pgfqpoint{4.124397in}{1.361892in}}%
\pgfpathclose%
\pgfusepath{stroke,fill}%
\end{pgfscope}%
\begin{pgfscope}%
\pgfpathrectangle{\pgfqpoint{2.867647in}{0.500000in}}{\pgfqpoint{1.764706in}{1.700000in}}%
\pgfusepath{clip}%
\pgfsetbuttcap%
\pgfsetroundjoin%
\definecolor{currentfill}{rgb}{0.978376,0.897317,0.831308}%
\pgfsetfillcolor{currentfill}%
\pgfsetlinewidth{0.311001pt}%
\definecolor{currentstroke}{rgb}{1.000000,1.000000,1.000000}%
\pgfsetstrokecolor{currentstroke}%
\pgfsetdash{}{0pt}%
\pgfpathmoveto{\pgfqpoint{4.229109in}{1.320081in}}%
\pgfpathcurveto{\pgfqpoint{4.236242in}{1.320081in}}{\pgfqpoint{4.243084in}{1.322915in}}{\pgfqpoint{4.248128in}{1.327959in}}%
\pgfpathcurveto{\pgfqpoint{4.253171in}{1.333002in}}{\pgfqpoint{4.256005in}{1.339844in}}{\pgfqpoint{4.256005in}{1.346977in}}%
\pgfpathcurveto{\pgfqpoint{4.256005in}{1.354110in}}{\pgfqpoint{4.253171in}{1.360951in}}{\pgfqpoint{4.248128in}{1.365995in}}%
\pgfpathcurveto{\pgfqpoint{4.243084in}{1.371039in}}{\pgfqpoint{4.236242in}{1.373872in}}{\pgfqpoint{4.229109in}{1.373872in}}%
\pgfpathcurveto{\pgfqpoint{4.221977in}{1.373872in}}{\pgfqpoint{4.215135in}{1.371039in}}{\pgfqpoint{4.210091in}{1.365995in}}%
\pgfpathcurveto{\pgfqpoint{4.205048in}{1.360951in}}{\pgfqpoint{4.202214in}{1.354110in}}{\pgfqpoint{4.202214in}{1.346977in}}%
\pgfpathcurveto{\pgfqpoint{4.202214in}{1.339844in}}{\pgfqpoint{4.205048in}{1.333002in}}{\pgfqpoint{4.210091in}{1.327959in}}%
\pgfpathcurveto{\pgfqpoint{4.215135in}{1.322915in}}{\pgfqpoint{4.221977in}{1.320081in}}{\pgfqpoint{4.229109in}{1.320081in}}%
\pgfpathclose%
\pgfusepath{stroke,fill}%
\end{pgfscope}%
\begin{pgfscope}%
\pgfpathrectangle{\pgfqpoint{2.867647in}{0.500000in}}{\pgfqpoint{1.764706in}{1.700000in}}%
\pgfusepath{clip}%
\pgfsetbuttcap%
\pgfsetroundjoin%
\definecolor{currentfill}{rgb}{0.946260,0.398132,0.274897}%
\pgfsetfillcolor{currentfill}%
\pgfsetlinewidth{0.311001pt}%
\definecolor{currentstroke}{rgb}{1.000000,1.000000,1.000000}%
\pgfsetstrokecolor{currentstroke}%
\pgfsetdash{}{0pt}%
\pgfpathmoveto{\pgfqpoint{3.854414in}{1.675737in}}%
\pgfpathcurveto{\pgfqpoint{3.861547in}{1.675737in}}{\pgfqpoint{3.868389in}{1.678570in}}{\pgfqpoint{3.873432in}{1.683614in}}%
\pgfpathcurveto{\pgfqpoint{3.878476in}{1.688658in}}{\pgfqpoint{3.881310in}{1.695499in}}{\pgfqpoint{3.881310in}{1.702632in}}%
\pgfpathcurveto{\pgfqpoint{3.881310in}{1.709765in}}{\pgfqpoint{3.878476in}{1.716607in}}{\pgfqpoint{3.873432in}{1.721650in}}%
\pgfpathcurveto{\pgfqpoint{3.868389in}{1.726694in}}{\pgfqpoint{3.861547in}{1.729528in}}{\pgfqpoint{3.854414in}{1.729528in}}%
\pgfpathcurveto{\pgfqpoint{3.847282in}{1.729528in}}{\pgfqpoint{3.840440in}{1.726694in}}{\pgfqpoint{3.835396in}{1.721650in}}%
\pgfpathcurveto{\pgfqpoint{3.830353in}{1.716607in}}{\pgfqpoint{3.827519in}{1.709765in}}{\pgfqpoint{3.827519in}{1.702632in}}%
\pgfpathcurveto{\pgfqpoint{3.827519in}{1.695499in}}{\pgfqpoint{3.830353in}{1.688658in}}{\pgfqpoint{3.835396in}{1.683614in}}%
\pgfpathcurveto{\pgfqpoint{3.840440in}{1.678570in}}{\pgfqpoint{3.847282in}{1.675737in}}{\pgfqpoint{3.854414in}{1.675737in}}%
\pgfpathclose%
\pgfusepath{stroke,fill}%
\end{pgfscope}%
\begin{pgfscope}%
\pgfpathrectangle{\pgfqpoint{2.867647in}{0.500000in}}{\pgfqpoint{1.764706in}{1.700000in}}%
\pgfusepath{clip}%
\pgfsetbuttcap%
\pgfsetroundjoin%
\definecolor{currentfill}{rgb}{0.980678,0.914765,0.856766}%
\pgfsetfillcolor{currentfill}%
\pgfsetlinewidth{0.311001pt}%
\definecolor{currentstroke}{rgb}{1.000000,1.000000,1.000000}%
\pgfsetstrokecolor{currentstroke}%
\pgfsetdash{}{0pt}%
\pgfpathmoveto{\pgfqpoint{4.156428in}{1.150584in}}%
\pgfpathcurveto{\pgfqpoint{4.163561in}{1.150584in}}{\pgfqpoint{4.170402in}{1.153418in}}{\pgfqpoint{4.175446in}{1.158462in}}%
\pgfpathcurveto{\pgfqpoint{4.180490in}{1.163506in}}{\pgfqpoint{4.183324in}{1.170347in}}{\pgfqpoint{4.183324in}{1.177480in}}%
\pgfpathcurveto{\pgfqpoint{4.183324in}{1.184613in}}{\pgfqpoint{4.180490in}{1.191455in}}{\pgfqpoint{4.175446in}{1.196498in}}%
\pgfpathcurveto{\pgfqpoint{4.170402in}{1.201542in}}{\pgfqpoint{4.163561in}{1.204376in}}{\pgfqpoint{4.156428in}{1.204376in}}%
\pgfpathcurveto{\pgfqpoint{4.149295in}{1.204376in}}{\pgfqpoint{4.142453in}{1.201542in}}{\pgfqpoint{4.137410in}{1.196498in}}%
\pgfpathcurveto{\pgfqpoint{4.132366in}{1.191455in}}{\pgfqpoint{4.129532in}{1.184613in}}{\pgfqpoint{4.129532in}{1.177480in}}%
\pgfpathcurveto{\pgfqpoint{4.129532in}{1.170347in}}{\pgfqpoint{4.132366in}{1.163506in}}{\pgfqpoint{4.137410in}{1.158462in}}%
\pgfpathcurveto{\pgfqpoint{4.142453in}{1.153418in}}{\pgfqpoint{4.149295in}{1.150584in}}{\pgfqpoint{4.156428in}{1.150584in}}%
\pgfpathclose%
\pgfusepath{stroke,fill}%
\end{pgfscope}%
\begin{pgfscope}%
\pgfpathrectangle{\pgfqpoint{2.867647in}{0.500000in}}{\pgfqpoint{1.764706in}{1.700000in}}%
\pgfusepath{clip}%
\pgfsetbuttcap%
\pgfsetroundjoin%
\definecolor{currentfill}{rgb}{0.981377,0.920617,0.865369}%
\pgfsetfillcolor{currentfill}%
\pgfsetlinewidth{0.311001pt}%
\definecolor{currentstroke}{rgb}{1.000000,1.000000,1.000000}%
\pgfsetstrokecolor{currentstroke}%
\pgfsetdash{}{0pt}%
\pgfpathmoveto{\pgfqpoint{4.183765in}{1.299145in}}%
\pgfpathcurveto{\pgfqpoint{4.190898in}{1.299145in}}{\pgfqpoint{4.197740in}{1.301979in}}{\pgfqpoint{4.202783in}{1.307022in}}%
\pgfpathcurveto{\pgfqpoint{4.207827in}{1.312066in}}{\pgfqpoint{4.210661in}{1.318908in}}{\pgfqpoint{4.210661in}{1.326040in}}%
\pgfpathcurveto{\pgfqpoint{4.210661in}{1.333173in}}{\pgfqpoint{4.207827in}{1.340015in}}{\pgfqpoint{4.202783in}{1.345058in}}%
\pgfpathcurveto{\pgfqpoint{4.197740in}{1.350102in}}{\pgfqpoint{4.190898in}{1.352936in}}{\pgfqpoint{4.183765in}{1.352936in}}%
\pgfpathcurveto{\pgfqpoint{4.176632in}{1.352936in}}{\pgfqpoint{4.169791in}{1.350102in}}{\pgfqpoint{4.164747in}{1.345058in}}%
\pgfpathcurveto{\pgfqpoint{4.159703in}{1.340015in}}{\pgfqpoint{4.156870in}{1.333173in}}{\pgfqpoint{4.156870in}{1.326040in}}%
\pgfpathcurveto{\pgfqpoint{4.156870in}{1.318908in}}{\pgfqpoint{4.159703in}{1.312066in}}{\pgfqpoint{4.164747in}{1.307022in}}%
\pgfpathcurveto{\pgfqpoint{4.169791in}{1.301979in}}{\pgfqpoint{4.176632in}{1.299145in}}{\pgfqpoint{4.183765in}{1.299145in}}%
\pgfpathclose%
\pgfusepath{stroke,fill}%
\end{pgfscope}%
\begin{pgfscope}%
\pgfpathrectangle{\pgfqpoint{2.867647in}{0.500000in}}{\pgfqpoint{1.764706in}{1.700000in}}%
\pgfusepath{clip}%
\pgfsetbuttcap%
\pgfsetroundjoin%
\definecolor{currentfill}{rgb}{0.973832,0.856556,0.771584}%
\pgfsetfillcolor{currentfill}%
\pgfsetlinewidth{0.311001pt}%
\definecolor{currentstroke}{rgb}{1.000000,1.000000,1.000000}%
\pgfsetstrokecolor{currentstroke}%
\pgfsetdash{}{0pt}%
\pgfpathmoveto{\pgfqpoint{4.160816in}{1.017460in}}%
\pgfpathcurveto{\pgfqpoint{4.167949in}{1.017460in}}{\pgfqpoint{4.174791in}{1.020294in}}{\pgfqpoint{4.179835in}{1.025337in}}%
\pgfpathcurveto{\pgfqpoint{4.184878in}{1.030381in}}{\pgfqpoint{4.187712in}{1.037223in}}{\pgfqpoint{4.187712in}{1.044356in}}%
\pgfpathcurveto{\pgfqpoint{4.187712in}{1.051488in}}{\pgfqpoint{4.184878in}{1.058330in}}{\pgfqpoint{4.179835in}{1.063374in}}%
\pgfpathcurveto{\pgfqpoint{4.174791in}{1.068417in}}{\pgfqpoint{4.167949in}{1.071251in}}{\pgfqpoint{4.160816in}{1.071251in}}%
\pgfpathcurveto{\pgfqpoint{4.153684in}{1.071251in}}{\pgfqpoint{4.146842in}{1.068417in}}{\pgfqpoint{4.141798in}{1.063374in}}%
\pgfpathcurveto{\pgfqpoint{4.136755in}{1.058330in}}{\pgfqpoint{4.133921in}{1.051488in}}{\pgfqpoint{4.133921in}{1.044356in}}%
\pgfpathcurveto{\pgfqpoint{4.133921in}{1.037223in}}{\pgfqpoint{4.136755in}{1.030381in}}{\pgfqpoint{4.141798in}{1.025337in}}%
\pgfpathcurveto{\pgfqpoint{4.146842in}{1.020294in}}{\pgfqpoint{4.153684in}{1.017460in}}{\pgfqpoint{4.160816in}{1.017460in}}%
\pgfpathclose%
\pgfusepath{stroke,fill}%
\end{pgfscope}%
\begin{pgfscope}%
\pgfpathrectangle{\pgfqpoint{2.867647in}{0.500000in}}{\pgfqpoint{1.764706in}{1.700000in}}%
\pgfusepath{clip}%
\pgfsetbuttcap%
\pgfsetroundjoin%
\definecolor{currentfill}{rgb}{0.981377,0.920617,0.865369}%
\pgfsetfillcolor{currentfill}%
\pgfsetlinewidth{0.311001pt}%
\definecolor{currentstroke}{rgb}{1.000000,1.000000,1.000000}%
\pgfsetstrokecolor{currentstroke}%
\pgfsetdash{}{0pt}%
\pgfpathmoveto{\pgfqpoint{4.184765in}{1.220926in}}%
\pgfpathcurveto{\pgfqpoint{4.191897in}{1.220926in}}{\pgfqpoint{4.198739in}{1.223760in}}{\pgfqpoint{4.203783in}{1.228804in}}%
\pgfpathcurveto{\pgfqpoint{4.208826in}{1.233848in}}{\pgfqpoint{4.211660in}{1.240689in}}{\pgfqpoint{4.211660in}{1.247822in}}%
\pgfpathcurveto{\pgfqpoint{4.211660in}{1.254955in}}{\pgfqpoint{4.208826in}{1.261797in}}{\pgfqpoint{4.203783in}{1.266840in}}%
\pgfpathcurveto{\pgfqpoint{4.198739in}{1.271884in}}{\pgfqpoint{4.191897in}{1.274718in}}{\pgfqpoint{4.184765in}{1.274718in}}%
\pgfpathcurveto{\pgfqpoint{4.177632in}{1.274718in}}{\pgfqpoint{4.170790in}{1.271884in}}{\pgfqpoint{4.165746in}{1.266840in}}%
\pgfpathcurveto{\pgfqpoint{4.160703in}{1.261797in}}{\pgfqpoint{4.157869in}{1.254955in}}{\pgfqpoint{4.157869in}{1.247822in}}%
\pgfpathcurveto{\pgfqpoint{4.157869in}{1.240689in}}{\pgfqpoint{4.160703in}{1.233848in}}{\pgfqpoint{4.165746in}{1.228804in}}%
\pgfpathcurveto{\pgfqpoint{4.170790in}{1.223760in}}{\pgfqpoint{4.177632in}{1.220926in}}{\pgfqpoint{4.184765in}{1.220926in}}%
\pgfpathclose%
\pgfusepath{stroke,fill}%
\end{pgfscope}%
\begin{pgfscope}%
\pgfpathrectangle{\pgfqpoint{2.867647in}{0.500000in}}{\pgfqpoint{1.764706in}{1.700000in}}%
\pgfusepath{clip}%
\pgfsetbuttcap%
\pgfsetroundjoin%
\definecolor{currentfill}{rgb}{0.966560,0.756582,0.625273}%
\pgfsetfillcolor{currentfill}%
\pgfsetlinewidth{0.311001pt}%
\definecolor{currentstroke}{rgb}{1.000000,1.000000,1.000000}%
\pgfsetstrokecolor{currentstroke}%
\pgfsetdash{}{0pt}%
\pgfpathmoveto{\pgfqpoint{4.013001in}{0.997864in}}%
\pgfpathcurveto{\pgfqpoint{4.020134in}{0.997864in}}{\pgfqpoint{4.026975in}{1.000698in}}{\pgfqpoint{4.032019in}{1.005741in}}%
\pgfpathcurveto{\pgfqpoint{4.037063in}{1.010785in}}{\pgfqpoint{4.039897in}{1.017627in}}{\pgfqpoint{4.039897in}{1.024760in}}%
\pgfpathcurveto{\pgfqpoint{4.039897in}{1.031892in}}{\pgfqpoint{4.037063in}{1.038734in}}{\pgfqpoint{4.032019in}{1.043778in}}%
\pgfpathcurveto{\pgfqpoint{4.026975in}{1.048821in}}{\pgfqpoint{4.020134in}{1.051655in}}{\pgfqpoint{4.013001in}{1.051655in}}%
\pgfpathcurveto{\pgfqpoint{4.005868in}{1.051655in}}{\pgfqpoint{3.999026in}{1.048821in}}{\pgfqpoint{3.993983in}{1.043778in}}%
\pgfpathcurveto{\pgfqpoint{3.988939in}{1.038734in}}{\pgfqpoint{3.986105in}{1.031892in}}{\pgfqpoint{3.986105in}{1.024760in}}%
\pgfpathcurveto{\pgfqpoint{3.986105in}{1.017627in}}{\pgfqpoint{3.988939in}{1.010785in}}{\pgfqpoint{3.993983in}{1.005741in}}%
\pgfpathcurveto{\pgfqpoint{3.999026in}{1.000698in}}{\pgfqpoint{4.005868in}{0.997864in}}{\pgfqpoint{4.013001in}{0.997864in}}%
\pgfpathclose%
\pgfusepath{stroke,fill}%
\end{pgfscope}%
\begin{pgfscope}%
\pgfpathrectangle{\pgfqpoint{2.867647in}{0.500000in}}{\pgfqpoint{1.764706in}{1.700000in}}%
\pgfusepath{clip}%
\pgfsetbuttcap%
\pgfsetroundjoin%
\definecolor{currentfill}{rgb}{0.964679,0.682838,0.530002}%
\pgfsetfillcolor{currentfill}%
\pgfsetlinewidth{0.311001pt}%
\definecolor{currentstroke}{rgb}{1.000000,1.000000,1.000000}%
\pgfsetstrokecolor{currentstroke}%
\pgfsetdash{}{0pt}%
\pgfpathmoveto{\pgfqpoint{4.212181in}{1.666826in}}%
\pgfpathcurveto{\pgfqpoint{4.219314in}{1.666826in}}{\pgfqpoint{4.226155in}{1.669660in}}{\pgfqpoint{4.231199in}{1.674703in}}%
\pgfpathcurveto{\pgfqpoint{4.236243in}{1.679747in}}{\pgfqpoint{4.239076in}{1.686589in}}{\pgfqpoint{4.239076in}{1.693721in}}%
\pgfpathcurveto{\pgfqpoint{4.239076in}{1.700854in}}{\pgfqpoint{4.236243in}{1.707696in}}{\pgfqpoint{4.231199in}{1.712740in}}%
\pgfpathcurveto{\pgfqpoint{4.226155in}{1.717783in}}{\pgfqpoint{4.219314in}{1.720617in}}{\pgfqpoint{4.212181in}{1.720617in}}%
\pgfpathcurveto{\pgfqpoint{4.205048in}{1.720617in}}{\pgfqpoint{4.198206in}{1.717783in}}{\pgfqpoint{4.193163in}{1.712740in}}%
\pgfpathcurveto{\pgfqpoint{4.188119in}{1.707696in}}{\pgfqpoint{4.185285in}{1.700854in}}{\pgfqpoint{4.185285in}{1.693721in}}%
\pgfpathcurveto{\pgfqpoint{4.185285in}{1.686589in}}{\pgfqpoint{4.188119in}{1.679747in}}{\pgfqpoint{4.193163in}{1.674703in}}%
\pgfpathcurveto{\pgfqpoint{4.198206in}{1.669660in}}{\pgfqpoint{4.205048in}{1.666826in}}{\pgfqpoint{4.212181in}{1.666826in}}%
\pgfpathclose%
\pgfusepath{stroke,fill}%
\end{pgfscope}%
\begin{pgfscope}%
\pgfpathrectangle{\pgfqpoint{2.867647in}{0.500000in}}{\pgfqpoint{1.764706in}{1.700000in}}%
\pgfusepath{clip}%
\pgfsetbuttcap%
\pgfsetroundjoin%
\definecolor{currentfill}{rgb}{0.981377,0.920617,0.865369}%
\pgfsetfillcolor{currentfill}%
\pgfsetlinewidth{0.311001pt}%
\definecolor{currentstroke}{rgb}{1.000000,1.000000,1.000000}%
\pgfsetstrokecolor{currentstroke}%
\pgfsetdash{}{0pt}%
\pgfpathmoveto{\pgfqpoint{4.195064in}{1.359640in}}%
\pgfpathcurveto{\pgfqpoint{4.202197in}{1.359640in}}{\pgfqpoint{4.209038in}{1.362474in}}{\pgfqpoint{4.214082in}{1.367518in}}%
\pgfpathcurveto{\pgfqpoint{4.219125in}{1.372561in}}{\pgfqpoint{4.221959in}{1.379403in}}{\pgfqpoint{4.221959in}{1.386536in}}%
\pgfpathcurveto{\pgfqpoint{4.221959in}{1.393669in}}{\pgfqpoint{4.219125in}{1.400510in}}{\pgfqpoint{4.214082in}{1.405554in}}%
\pgfpathcurveto{\pgfqpoint{4.209038in}{1.410598in}}{\pgfqpoint{4.202197in}{1.413432in}}{\pgfqpoint{4.195064in}{1.413432in}}%
\pgfpathcurveto{\pgfqpoint{4.187931in}{1.413432in}}{\pgfqpoint{4.181089in}{1.410598in}}{\pgfqpoint{4.176046in}{1.405554in}}%
\pgfpathcurveto{\pgfqpoint{4.171002in}{1.400510in}}{\pgfqpoint{4.168168in}{1.393669in}}{\pgfqpoint{4.168168in}{1.386536in}}%
\pgfpathcurveto{\pgfqpoint{4.168168in}{1.379403in}}{\pgfqpoint{4.171002in}{1.372561in}}{\pgfqpoint{4.176046in}{1.367518in}}%
\pgfpathcurveto{\pgfqpoint{4.181089in}{1.362474in}}{\pgfqpoint{4.187931in}{1.359640in}}{\pgfqpoint{4.195064in}{1.359640in}}%
\pgfpathclose%
\pgfusepath{stroke,fill}%
\end{pgfscope}%
\begin{pgfscope}%
\pgfpathrectangle{\pgfqpoint{2.867647in}{0.500000in}}{\pgfqpoint{1.764706in}{1.700000in}}%
\pgfusepath{clip}%
\pgfsetbuttcap%
\pgfsetroundjoin%
\definecolor{currentfill}{rgb}{0.964799,0.689101,0.537560}%
\pgfsetfillcolor{currentfill}%
\pgfsetlinewidth{0.311001pt}%
\definecolor{currentstroke}{rgb}{1.000000,1.000000,1.000000}%
\pgfsetstrokecolor{currentstroke}%
\pgfsetdash{}{0pt}%
\pgfpathmoveto{\pgfqpoint{4.311419in}{1.263683in}}%
\pgfpathcurveto{\pgfqpoint{4.318552in}{1.263683in}}{\pgfqpoint{4.325394in}{1.266517in}}{\pgfqpoint{4.330437in}{1.271561in}}%
\pgfpathcurveto{\pgfqpoint{4.335481in}{1.276605in}}{\pgfqpoint{4.338315in}{1.283446in}}{\pgfqpoint{4.338315in}{1.290579in}}%
\pgfpathcurveto{\pgfqpoint{4.338315in}{1.297712in}}{\pgfqpoint{4.335481in}{1.304554in}}{\pgfqpoint{4.330437in}{1.309597in}}%
\pgfpathcurveto{\pgfqpoint{4.325394in}{1.314641in}}{\pgfqpoint{4.318552in}{1.317475in}}{\pgfqpoint{4.311419in}{1.317475in}}%
\pgfpathcurveto{\pgfqpoint{4.304286in}{1.317475in}}{\pgfqpoint{4.297445in}{1.314641in}}{\pgfqpoint{4.292401in}{1.309597in}}%
\pgfpathcurveto{\pgfqpoint{4.287358in}{1.304554in}}{\pgfqpoint{4.284524in}{1.297712in}}{\pgfqpoint{4.284524in}{1.290579in}}%
\pgfpathcurveto{\pgfqpoint{4.284524in}{1.283446in}}{\pgfqpoint{4.287358in}{1.276605in}}{\pgfqpoint{4.292401in}{1.271561in}}%
\pgfpathcurveto{\pgfqpoint{4.297445in}{1.266517in}}{\pgfqpoint{4.304286in}{1.263683in}}{\pgfqpoint{4.311419in}{1.263683in}}%
\pgfpathclose%
\pgfusepath{stroke,fill}%
\end{pgfscope}%
\begin{pgfscope}%
\pgfpathrectangle{\pgfqpoint{2.867647in}{0.500000in}}{\pgfqpoint{1.764706in}{1.700000in}}%
\pgfusepath{clip}%
\pgfsetbuttcap%
\pgfsetroundjoin%
\definecolor{currentfill}{rgb}{0.966560,0.756582,0.625273}%
\pgfsetfillcolor{currentfill}%
\pgfsetlinewidth{0.311001pt}%
\definecolor{currentstroke}{rgb}{1.000000,1.000000,1.000000}%
\pgfsetstrokecolor{currentstroke}%
\pgfsetdash{}{0pt}%
\pgfpathmoveto{\pgfqpoint{4.032361in}{0.924208in}}%
\pgfpathcurveto{\pgfqpoint{4.039494in}{0.924208in}}{\pgfqpoint{4.046335in}{0.927042in}}{\pgfqpoint{4.051379in}{0.932085in}}%
\pgfpathcurveto{\pgfqpoint{4.056423in}{0.937129in}}{\pgfqpoint{4.059257in}{0.943971in}}{\pgfqpoint{4.059257in}{0.951104in}}%
\pgfpathcurveto{\pgfqpoint{4.059257in}{0.958236in}}{\pgfqpoint{4.056423in}{0.965078in}}{\pgfqpoint{4.051379in}{0.970122in}}%
\pgfpathcurveto{\pgfqpoint{4.046335in}{0.975165in}}{\pgfqpoint{4.039494in}{0.977999in}}{\pgfqpoint{4.032361in}{0.977999in}}%
\pgfpathcurveto{\pgfqpoint{4.025228in}{0.977999in}}{\pgfqpoint{4.018387in}{0.975165in}}{\pgfqpoint{4.013343in}{0.970122in}}%
\pgfpathcurveto{\pgfqpoint{4.008299in}{0.965078in}}{\pgfqpoint{4.005465in}{0.958236in}}{\pgfqpoint{4.005465in}{0.951104in}}%
\pgfpathcurveto{\pgfqpoint{4.005465in}{0.943971in}}{\pgfqpoint{4.008299in}{0.937129in}}{\pgfqpoint{4.013343in}{0.932085in}}%
\pgfpathcurveto{\pgfqpoint{4.018387in}{0.927042in}}{\pgfqpoint{4.025228in}{0.924208in}}{\pgfqpoint{4.032361in}{0.924208in}}%
\pgfpathclose%
\pgfusepath{stroke,fill}%
\end{pgfscope}%
\begin{pgfscope}%
\pgfpathrectangle{\pgfqpoint{2.867647in}{0.500000in}}{\pgfqpoint{1.764706in}{1.700000in}}%
\pgfusepath{clip}%
\pgfsetbuttcap%
\pgfsetroundjoin%
\definecolor{currentfill}{rgb}{0.977657,0.891500,0.822809}%
\pgfsetfillcolor{currentfill}%
\pgfsetlinewidth{0.311001pt}%
\definecolor{currentstroke}{rgb}{1.000000,1.000000,1.000000}%
\pgfsetstrokecolor{currentstroke}%
\pgfsetdash{}{0pt}%
\pgfpathmoveto{\pgfqpoint{4.146924in}{1.365491in}}%
\pgfpathcurveto{\pgfqpoint{4.154057in}{1.365491in}}{\pgfqpoint{4.160899in}{1.368325in}}{\pgfqpoint{4.165942in}{1.373368in}}%
\pgfpathcurveto{\pgfqpoint{4.170986in}{1.378412in}}{\pgfqpoint{4.173820in}{1.385254in}}{\pgfqpoint{4.173820in}{1.392387in}}%
\pgfpathcurveto{\pgfqpoint{4.173820in}{1.399519in}}{\pgfqpoint{4.170986in}{1.406361in}}{\pgfqpoint{4.165942in}{1.411405in}}%
\pgfpathcurveto{\pgfqpoint{4.160899in}{1.416448in}}{\pgfqpoint{4.154057in}{1.419282in}}{\pgfqpoint{4.146924in}{1.419282in}}%
\pgfpathcurveto{\pgfqpoint{4.139791in}{1.419282in}}{\pgfqpoint{4.132950in}{1.416448in}}{\pgfqpoint{4.127906in}{1.411405in}}%
\pgfpathcurveto{\pgfqpoint{4.122862in}{1.406361in}}{\pgfqpoint{4.120028in}{1.399519in}}{\pgfqpoint{4.120028in}{1.392387in}}%
\pgfpathcurveto{\pgfqpoint{4.120028in}{1.385254in}}{\pgfqpoint{4.122862in}{1.378412in}}{\pgfqpoint{4.127906in}{1.373368in}}%
\pgfpathcurveto{\pgfqpoint{4.132950in}{1.368325in}}{\pgfqpoint{4.139791in}{1.365491in}}{\pgfqpoint{4.146924in}{1.365491in}}%
\pgfpathclose%
\pgfusepath{stroke,fill}%
\end{pgfscope}%
\begin{pgfscope}%
\pgfpathrectangle{\pgfqpoint{2.867647in}{0.500000in}}{\pgfqpoint{1.764706in}{1.700000in}}%
\pgfusepath{clip}%
\pgfsetbuttcap%
\pgfsetroundjoin%
\definecolor{currentfill}{rgb}{0.981377,0.920617,0.865369}%
\pgfsetfillcolor{currentfill}%
\pgfsetlinewidth{0.311001pt}%
\definecolor{currentstroke}{rgb}{1.000000,1.000000,1.000000}%
\pgfsetstrokecolor{currentstroke}%
\pgfsetdash{}{0pt}%
\pgfpathmoveto{\pgfqpoint{4.204115in}{1.314433in}}%
\pgfpathcurveto{\pgfqpoint{4.211248in}{1.314433in}}{\pgfqpoint{4.218090in}{1.317266in}}{\pgfqpoint{4.223133in}{1.322310in}}%
\pgfpathcurveto{\pgfqpoint{4.228177in}{1.327354in}}{\pgfqpoint{4.231011in}{1.334195in}}{\pgfqpoint{4.231011in}{1.341328in}}%
\pgfpathcurveto{\pgfqpoint{4.231011in}{1.348461in}}{\pgfqpoint{4.228177in}{1.355303in}}{\pgfqpoint{4.223133in}{1.360346in}}%
\pgfpathcurveto{\pgfqpoint{4.218090in}{1.365390in}}{\pgfqpoint{4.211248in}{1.368224in}}{\pgfqpoint{4.204115in}{1.368224in}}%
\pgfpathcurveto{\pgfqpoint{4.196982in}{1.368224in}}{\pgfqpoint{4.190141in}{1.365390in}}{\pgfqpoint{4.185097in}{1.360346in}}%
\pgfpathcurveto{\pgfqpoint{4.180053in}{1.355303in}}{\pgfqpoint{4.177220in}{1.348461in}}{\pgfqpoint{4.177220in}{1.341328in}}%
\pgfpathcurveto{\pgfqpoint{4.177220in}{1.334195in}}{\pgfqpoint{4.180053in}{1.327354in}}{\pgfqpoint{4.185097in}{1.322310in}}%
\pgfpathcurveto{\pgfqpoint{4.190141in}{1.317266in}}{\pgfqpoint{4.196982in}{1.314433in}}{\pgfqpoint{4.204115in}{1.314433in}}%
\pgfpathclose%
\pgfusepath{stroke,fill}%
\end{pgfscope}%
\begin{pgfscope}%
\pgfpathrectangle{\pgfqpoint{2.867647in}{0.500000in}}{\pgfqpoint{1.764706in}{1.700000in}}%
\pgfusepath{clip}%
\pgfsetbuttcap%
\pgfsetroundjoin%
\definecolor{currentfill}{rgb}{0.962765,0.606121,0.444717}%
\pgfsetfillcolor{currentfill}%
\pgfsetlinewidth{0.311001pt}%
\definecolor{currentstroke}{rgb}{1.000000,1.000000,1.000000}%
\pgfsetstrokecolor{currentstroke}%
\pgfsetdash{}{0pt}%
\pgfpathmoveto{\pgfqpoint{4.321468in}{1.429374in}}%
\pgfpathcurveto{\pgfqpoint{4.328600in}{1.429374in}}{\pgfqpoint{4.335442in}{1.432208in}}{\pgfqpoint{4.340486in}{1.437252in}}%
\pgfpathcurveto{\pgfqpoint{4.345529in}{1.442295in}}{\pgfqpoint{4.348363in}{1.449137in}}{\pgfqpoint{4.348363in}{1.456270in}}%
\pgfpathcurveto{\pgfqpoint{4.348363in}{1.463403in}}{\pgfqpoint{4.345529in}{1.470244in}}{\pgfqpoint{4.340486in}{1.475288in}}%
\pgfpathcurveto{\pgfqpoint{4.335442in}{1.480332in}}{\pgfqpoint{4.328600in}{1.483166in}}{\pgfqpoint{4.321468in}{1.483166in}}%
\pgfpathcurveto{\pgfqpoint{4.314335in}{1.483166in}}{\pgfqpoint{4.307493in}{1.480332in}}{\pgfqpoint{4.302449in}{1.475288in}}%
\pgfpathcurveto{\pgfqpoint{4.297406in}{1.470244in}}{\pgfqpoint{4.294572in}{1.463403in}}{\pgfqpoint{4.294572in}{1.456270in}}%
\pgfpathcurveto{\pgfqpoint{4.294572in}{1.449137in}}{\pgfqpoint{4.297406in}{1.442295in}}{\pgfqpoint{4.302449in}{1.437252in}}%
\pgfpathcurveto{\pgfqpoint{4.307493in}{1.432208in}}{\pgfqpoint{4.314335in}{1.429374in}}{\pgfqpoint{4.321468in}{1.429374in}}%
\pgfpathclose%
\pgfusepath{stroke,fill}%
\end{pgfscope}%
\begin{pgfscope}%
\pgfpathrectangle{\pgfqpoint{2.867647in}{0.500000in}}{\pgfqpoint{1.764706in}{1.700000in}}%
\pgfusepath{clip}%
\pgfsetbuttcap%
\pgfsetroundjoin%
\definecolor{currentfill}{rgb}{0.974412,0.862387,0.780156}%
\pgfsetfillcolor{currentfill}%
\pgfsetlinewidth{0.311001pt}%
\definecolor{currentstroke}{rgb}{1.000000,1.000000,1.000000}%
\pgfsetstrokecolor{currentstroke}%
\pgfsetdash{}{0pt}%
\pgfpathmoveto{\pgfqpoint{4.131930in}{1.003482in}}%
\pgfpathcurveto{\pgfqpoint{4.139063in}{1.003482in}}{\pgfqpoint{4.145904in}{1.006316in}}{\pgfqpoint{4.150948in}{1.011359in}}%
\pgfpathcurveto{\pgfqpoint{4.155992in}{1.016403in}}{\pgfqpoint{4.158826in}{1.023245in}}{\pgfqpoint{4.158826in}{1.030378in}}%
\pgfpathcurveto{\pgfqpoint{4.158826in}{1.037510in}}{\pgfqpoint{4.155992in}{1.044352in}}{\pgfqpoint{4.150948in}{1.049396in}}%
\pgfpathcurveto{\pgfqpoint{4.145904in}{1.054439in}}{\pgfqpoint{4.139063in}{1.057273in}}{\pgfqpoint{4.131930in}{1.057273in}}%
\pgfpathcurveto{\pgfqpoint{4.124797in}{1.057273in}}{\pgfqpoint{4.117955in}{1.054439in}}{\pgfqpoint{4.112912in}{1.049396in}}%
\pgfpathcurveto{\pgfqpoint{4.107868in}{1.044352in}}{\pgfqpoint{4.105034in}{1.037510in}}{\pgfqpoint{4.105034in}{1.030378in}}%
\pgfpathcurveto{\pgfqpoint{4.105034in}{1.023245in}}{\pgfqpoint{4.107868in}{1.016403in}}{\pgfqpoint{4.112912in}{1.011359in}}%
\pgfpathcurveto{\pgfqpoint{4.117955in}{1.006316in}}{\pgfqpoint{4.124797in}{1.003482in}}{\pgfqpoint{4.131930in}{1.003482in}}%
\pgfpathclose%
\pgfusepath{stroke,fill}%
\end{pgfscope}%
\begin{pgfscope}%
\pgfpathrectangle{\pgfqpoint{2.867647in}{0.500000in}}{\pgfqpoint{1.764706in}{1.700000in}}%
\pgfusepath{clip}%
\pgfsetbuttcap%
\pgfsetroundjoin%
\definecolor{currentfill}{rgb}{0.979124,0.903132,0.839793}%
\pgfsetfillcolor{currentfill}%
\pgfsetlinewidth{0.311001pt}%
\definecolor{currentstroke}{rgb}{1.000000,1.000000,1.000000}%
\pgfsetstrokecolor{currentstroke}%
\pgfsetdash{}{0pt}%
\pgfpathmoveto{\pgfqpoint{4.138557in}{1.077730in}}%
\pgfpathcurveto{\pgfqpoint{4.145690in}{1.077730in}}{\pgfqpoint{4.152532in}{1.080564in}}{\pgfqpoint{4.157575in}{1.085608in}}%
\pgfpathcurveto{\pgfqpoint{4.162619in}{1.090651in}}{\pgfqpoint{4.165453in}{1.097493in}}{\pgfqpoint{4.165453in}{1.104626in}}%
\pgfpathcurveto{\pgfqpoint{4.165453in}{1.111759in}}{\pgfqpoint{4.162619in}{1.118600in}}{\pgfqpoint{4.157575in}{1.123644in}}%
\pgfpathcurveto{\pgfqpoint{4.152532in}{1.128688in}}{\pgfqpoint{4.145690in}{1.131522in}}{\pgfqpoint{4.138557in}{1.131522in}}%
\pgfpathcurveto{\pgfqpoint{4.131424in}{1.131522in}}{\pgfqpoint{4.124583in}{1.128688in}}{\pgfqpoint{4.119539in}{1.123644in}}%
\pgfpathcurveto{\pgfqpoint{4.114495in}{1.118600in}}{\pgfqpoint{4.111662in}{1.111759in}}{\pgfqpoint{4.111662in}{1.104626in}}%
\pgfpathcurveto{\pgfqpoint{4.111662in}{1.097493in}}{\pgfqpoint{4.114495in}{1.090651in}}{\pgfqpoint{4.119539in}{1.085608in}}%
\pgfpathcurveto{\pgfqpoint{4.124583in}{1.080564in}}{\pgfqpoint{4.131424in}{1.077730in}}{\pgfqpoint{4.138557in}{1.077730in}}%
\pgfpathclose%
\pgfusepath{stroke,fill}%
\end{pgfscope}%
\begin{pgfscope}%
\pgfpathrectangle{\pgfqpoint{2.867647in}{0.500000in}}{\pgfqpoint{1.764706in}{1.700000in}}%
\pgfusepath{clip}%
\pgfsetbuttcap%
\pgfsetroundjoin%
\definecolor{currentfill}{rgb}{0.967398,0.774513,0.650573}%
\pgfsetfillcolor{currentfill}%
\pgfsetlinewidth{0.311001pt}%
\definecolor{currentstroke}{rgb}{1.000000,1.000000,1.000000}%
\pgfsetstrokecolor{currentstroke}%
\pgfsetdash{}{0pt}%
\pgfpathmoveto{\pgfqpoint{4.023596in}{1.009233in}}%
\pgfpathcurveto{\pgfqpoint{4.030729in}{1.009233in}}{\pgfqpoint{4.037571in}{1.012067in}}{\pgfqpoint{4.042615in}{1.017111in}}%
\pgfpathcurveto{\pgfqpoint{4.047658in}{1.022154in}}{\pgfqpoint{4.050492in}{1.028996in}}{\pgfqpoint{4.050492in}{1.036129in}}%
\pgfpathcurveto{\pgfqpoint{4.050492in}{1.043262in}}{\pgfqpoint{4.047658in}{1.050103in}}{\pgfqpoint{4.042615in}{1.055147in}}%
\pgfpathcurveto{\pgfqpoint{4.037571in}{1.060191in}}{\pgfqpoint{4.030729in}{1.063024in}}{\pgfqpoint{4.023596in}{1.063024in}}%
\pgfpathcurveto{\pgfqpoint{4.016464in}{1.063024in}}{\pgfqpoint{4.009622in}{1.060191in}}{\pgfqpoint{4.004578in}{1.055147in}}%
\pgfpathcurveto{\pgfqpoint{3.999535in}{1.050103in}}{\pgfqpoint{3.996701in}{1.043262in}}{\pgfqpoint{3.996701in}{1.036129in}}%
\pgfpathcurveto{\pgfqpoint{3.996701in}{1.028996in}}{\pgfqpoint{3.999535in}{1.022154in}}{\pgfqpoint{4.004578in}{1.017111in}}%
\pgfpathcurveto{\pgfqpoint{4.009622in}{1.012067in}}{\pgfqpoint{4.016464in}{1.009233in}}{\pgfqpoint{4.023596in}{1.009233in}}%
\pgfpathclose%
\pgfusepath{stroke,fill}%
\end{pgfscope}%
\begin{pgfscope}%
\pgfpathrectangle{\pgfqpoint{2.867647in}{0.500000in}}{\pgfqpoint{1.764706in}{1.700000in}}%
\pgfusepath{clip}%
\pgfsetbuttcap%
\pgfsetroundjoin%
\definecolor{currentfill}{rgb}{0.965169,0.707764,0.560659}%
\pgfsetfillcolor{currentfill}%
\pgfsetlinewidth{0.311001pt}%
\definecolor{currentstroke}{rgb}{1.000000,1.000000,1.000000}%
\pgfsetstrokecolor{currentstroke}%
\pgfsetdash{}{0pt}%
\pgfpathmoveto{\pgfqpoint{4.273591in}{1.101094in}}%
\pgfpathcurveto{\pgfqpoint{4.280724in}{1.101094in}}{\pgfqpoint{4.287565in}{1.103928in}}{\pgfqpoint{4.292609in}{1.108972in}}%
\pgfpathcurveto{\pgfqpoint{4.297653in}{1.114015in}}{\pgfqpoint{4.300487in}{1.120857in}}{\pgfqpoint{4.300487in}{1.127990in}}%
\pgfpathcurveto{\pgfqpoint{4.300487in}{1.135123in}}{\pgfqpoint{4.297653in}{1.141964in}}{\pgfqpoint{4.292609in}{1.147008in}}%
\pgfpathcurveto{\pgfqpoint{4.287565in}{1.152052in}}{\pgfqpoint{4.280724in}{1.154886in}}{\pgfqpoint{4.273591in}{1.154886in}}%
\pgfpathcurveto{\pgfqpoint{4.266458in}{1.154886in}}{\pgfqpoint{4.259617in}{1.152052in}}{\pgfqpoint{4.254573in}{1.147008in}}%
\pgfpathcurveto{\pgfqpoint{4.249529in}{1.141964in}}{\pgfqpoint{4.246695in}{1.135123in}}{\pgfqpoint{4.246695in}{1.127990in}}%
\pgfpathcurveto{\pgfqpoint{4.246695in}{1.120857in}}{\pgfqpoint{4.249529in}{1.114015in}}{\pgfqpoint{4.254573in}{1.108972in}}%
\pgfpathcurveto{\pgfqpoint{4.259617in}{1.103928in}}{\pgfqpoint{4.266458in}{1.101094in}}{\pgfqpoint{4.273591in}{1.101094in}}%
\pgfpathclose%
\pgfusepath{stroke,fill}%
\end{pgfscope}%
\begin{pgfscope}%
\pgfpathrectangle{\pgfqpoint{2.867647in}{0.500000in}}{\pgfqpoint{1.764706in}{1.700000in}}%
\pgfusepath{clip}%
\pgfsetbuttcap%
\pgfsetroundjoin%
\definecolor{currentfill}{rgb}{0.975018,0.868213,0.788710}%
\pgfsetfillcolor{currentfill}%
\pgfsetlinewidth{0.311001pt}%
\definecolor{currentstroke}{rgb}{1.000000,1.000000,1.000000}%
\pgfsetstrokecolor{currentstroke}%
\pgfsetdash{}{0pt}%
\pgfpathmoveto{\pgfqpoint{4.202639in}{1.548906in}}%
\pgfpathcurveto{\pgfqpoint{4.209772in}{1.548906in}}{\pgfqpoint{4.216614in}{1.551740in}}{\pgfqpoint{4.221657in}{1.556783in}}%
\pgfpathcurveto{\pgfqpoint{4.226701in}{1.561827in}}{\pgfqpoint{4.229535in}{1.568669in}}{\pgfqpoint{4.229535in}{1.575801in}}%
\pgfpathcurveto{\pgfqpoint{4.229535in}{1.582934in}}{\pgfqpoint{4.226701in}{1.589776in}}{\pgfqpoint{4.221657in}{1.594820in}}%
\pgfpathcurveto{\pgfqpoint{4.216614in}{1.599863in}}{\pgfqpoint{4.209772in}{1.602697in}}{\pgfqpoint{4.202639in}{1.602697in}}%
\pgfpathcurveto{\pgfqpoint{4.195506in}{1.602697in}}{\pgfqpoint{4.188665in}{1.599863in}}{\pgfqpoint{4.183621in}{1.594820in}}%
\pgfpathcurveto{\pgfqpoint{4.178577in}{1.589776in}}{\pgfqpoint{4.175743in}{1.582934in}}{\pgfqpoint{4.175743in}{1.575801in}}%
\pgfpathcurveto{\pgfqpoint{4.175743in}{1.568669in}}{\pgfqpoint{4.178577in}{1.561827in}}{\pgfqpoint{4.183621in}{1.556783in}}%
\pgfpathcurveto{\pgfqpoint{4.188665in}{1.551740in}}{\pgfqpoint{4.195506in}{1.548906in}}{\pgfqpoint{4.202639in}{1.548906in}}%
\pgfpathclose%
\pgfusepath{stroke,fill}%
\end{pgfscope}%
\begin{pgfscope}%
\pgfpathrectangle{\pgfqpoint{2.867647in}{0.500000in}}{\pgfqpoint{1.764706in}{1.700000in}}%
\pgfusepath{clip}%
\pgfsetbuttcap%
\pgfsetroundjoin%
\definecolor{currentfill}{rgb}{0.911533,0.252926,0.244703}%
\pgfsetfillcolor{currentfill}%
\pgfsetlinewidth{0.311001pt}%
\definecolor{currentstroke}{rgb}{1.000000,1.000000,1.000000}%
\pgfsetstrokecolor{currentstroke}%
\pgfsetdash{}{0pt}%
\pgfpathmoveto{\pgfqpoint{4.213998in}{0.851459in}}%
\pgfpathcurveto{\pgfqpoint{4.221131in}{0.851459in}}{\pgfqpoint{4.227972in}{0.854293in}}{\pgfqpoint{4.233016in}{0.859336in}}%
\pgfpathcurveto{\pgfqpoint{4.238060in}{0.864380in}}{\pgfqpoint{4.240893in}{0.871222in}}{\pgfqpoint{4.240893in}{0.878354in}}%
\pgfpathcurveto{\pgfqpoint{4.240893in}{0.885487in}}{\pgfqpoint{4.238060in}{0.892329in}}{\pgfqpoint{4.233016in}{0.897373in}}%
\pgfpathcurveto{\pgfqpoint{4.227972in}{0.902416in}}{\pgfqpoint{4.221131in}{0.905250in}}{\pgfqpoint{4.213998in}{0.905250in}}%
\pgfpathcurveto{\pgfqpoint{4.206865in}{0.905250in}}{\pgfqpoint{4.200023in}{0.902416in}}{\pgfqpoint{4.194980in}{0.897373in}}%
\pgfpathcurveto{\pgfqpoint{4.189936in}{0.892329in}}{\pgfqpoint{4.187102in}{0.885487in}}{\pgfqpoint{4.187102in}{0.878354in}}%
\pgfpathcurveto{\pgfqpoint{4.187102in}{0.871222in}}{\pgfqpoint{4.189936in}{0.864380in}}{\pgfqpoint{4.194980in}{0.859336in}}%
\pgfpathcurveto{\pgfqpoint{4.200023in}{0.854293in}}{\pgfqpoint{4.206865in}{0.851459in}}{\pgfqpoint{4.213998in}{0.851459in}}%
\pgfpathclose%
\pgfusepath{stroke,fill}%
\end{pgfscope}%
\begin{pgfscope}%
\pgfpathrectangle{\pgfqpoint{2.867647in}{0.500000in}}{\pgfqpoint{1.764706in}{1.700000in}}%
\pgfusepath{clip}%
\pgfsetbuttcap%
\pgfsetroundjoin%
\definecolor{currentfill}{rgb}{0.968931,0.798091,0.685123}%
\pgfsetfillcolor{currentfill}%
\pgfsetlinewidth{0.311001pt}%
\definecolor{currentstroke}{rgb}{1.000000,1.000000,1.000000}%
\pgfsetstrokecolor{currentstroke}%
\pgfsetdash{}{0pt}%
\pgfpathmoveto{\pgfqpoint{4.072678in}{0.940749in}}%
\pgfpathcurveto{\pgfqpoint{4.079811in}{0.940749in}}{\pgfqpoint{4.086653in}{0.943583in}}{\pgfqpoint{4.091696in}{0.948626in}}%
\pgfpathcurveto{\pgfqpoint{4.096740in}{0.953670in}}{\pgfqpoint{4.099574in}{0.960512in}}{\pgfqpoint{4.099574in}{0.967645in}}%
\pgfpathcurveto{\pgfqpoint{4.099574in}{0.974777in}}{\pgfqpoint{4.096740in}{0.981619in}}{\pgfqpoint{4.091696in}{0.986663in}}%
\pgfpathcurveto{\pgfqpoint{4.086653in}{0.991706in}}{\pgfqpoint{4.079811in}{0.994540in}}{\pgfqpoint{4.072678in}{0.994540in}}%
\pgfpathcurveto{\pgfqpoint{4.065545in}{0.994540in}}{\pgfqpoint{4.058704in}{0.991706in}}{\pgfqpoint{4.053660in}{0.986663in}}%
\pgfpathcurveto{\pgfqpoint{4.048616in}{0.981619in}}{\pgfqpoint{4.045782in}{0.974777in}}{\pgfqpoint{4.045782in}{0.967645in}}%
\pgfpathcurveto{\pgfqpoint{4.045782in}{0.960512in}}{\pgfqpoint{4.048616in}{0.953670in}}{\pgfqpoint{4.053660in}{0.948626in}}%
\pgfpathcurveto{\pgfqpoint{4.058704in}{0.943583in}}{\pgfqpoint{4.065545in}{0.940749in}}{\pgfqpoint{4.072678in}{0.940749in}}%
\pgfpathclose%
\pgfusepath{stroke,fill}%
\end{pgfscope}%
\begin{pgfscope}%
\pgfpathrectangle{\pgfqpoint{2.867647in}{0.500000in}}{\pgfqpoint{1.764706in}{1.700000in}}%
\pgfusepath{clip}%
\pgfsetbuttcap%
\pgfsetroundjoin%
\definecolor{currentfill}{rgb}{0.965592,0.726236,0.584384}%
\pgfsetfillcolor{currentfill}%
\pgfsetlinewidth{0.311001pt}%
\definecolor{currentstroke}{rgb}{1.000000,1.000000,1.000000}%
\pgfsetstrokecolor{currentstroke}%
\pgfsetdash{}{0pt}%
\pgfpathmoveto{\pgfqpoint{3.993773in}{1.679416in}}%
\pgfpathcurveto{\pgfqpoint{4.000906in}{1.679416in}}{\pgfqpoint{4.007748in}{1.682250in}}{\pgfqpoint{4.012792in}{1.687293in}}%
\pgfpathcurveto{\pgfqpoint{4.017835in}{1.692337in}}{\pgfqpoint{4.020669in}{1.699179in}}{\pgfqpoint{4.020669in}{1.706312in}}%
\pgfpathcurveto{\pgfqpoint{4.020669in}{1.713444in}}{\pgfqpoint{4.017835in}{1.720286in}}{\pgfqpoint{4.012792in}{1.725330in}}%
\pgfpathcurveto{\pgfqpoint{4.007748in}{1.730373in}}{\pgfqpoint{4.000906in}{1.733207in}}{\pgfqpoint{3.993773in}{1.733207in}}%
\pgfpathcurveto{\pgfqpoint{3.986641in}{1.733207in}}{\pgfqpoint{3.979799in}{1.730373in}}{\pgfqpoint{3.974755in}{1.725330in}}%
\pgfpathcurveto{\pgfqpoint{3.969712in}{1.720286in}}{\pgfqpoint{3.966878in}{1.713444in}}{\pgfqpoint{3.966878in}{1.706312in}}%
\pgfpathcurveto{\pgfqpoint{3.966878in}{1.699179in}}{\pgfqpoint{3.969712in}{1.692337in}}{\pgfqpoint{3.974755in}{1.687293in}}%
\pgfpathcurveto{\pgfqpoint{3.979799in}{1.682250in}}{\pgfqpoint{3.986641in}{1.679416in}}{\pgfqpoint{3.993773in}{1.679416in}}%
\pgfpathclose%
\pgfusepath{stroke,fill}%
\end{pgfscope}%
\begin{pgfscope}%
\pgfpathrectangle{\pgfqpoint{2.867647in}{0.500000in}}{\pgfqpoint{1.764706in}{1.700000in}}%
\pgfusepath{clip}%
\pgfsetbuttcap%
\pgfsetroundjoin%
\definecolor{currentfill}{rgb}{0.955103,0.477872,0.328626}%
\pgfsetfillcolor{currentfill}%
\pgfsetlinewidth{0.311001pt}%
\definecolor{currentstroke}{rgb}{1.000000,1.000000,1.000000}%
\pgfsetstrokecolor{currentstroke}%
\pgfsetdash{}{0pt}%
\pgfpathmoveto{\pgfqpoint{4.325058in}{1.112073in}}%
\pgfpathcurveto{\pgfqpoint{4.332191in}{1.112073in}}{\pgfqpoint{4.339033in}{1.114907in}}{\pgfqpoint{4.344076in}{1.119950in}}%
\pgfpathcurveto{\pgfqpoint{4.349120in}{1.124994in}}{\pgfqpoint{4.351954in}{1.131836in}}{\pgfqpoint{4.351954in}{1.138969in}}%
\pgfpathcurveto{\pgfqpoint{4.351954in}{1.146101in}}{\pgfqpoint{4.349120in}{1.152943in}}{\pgfqpoint{4.344076in}{1.157987in}}%
\pgfpathcurveto{\pgfqpoint{4.339033in}{1.163030in}}{\pgfqpoint{4.332191in}{1.165864in}}{\pgfqpoint{4.325058in}{1.165864in}}%
\pgfpathcurveto{\pgfqpoint{4.317925in}{1.165864in}}{\pgfqpoint{4.311084in}{1.163030in}}{\pgfqpoint{4.306040in}{1.157987in}}%
\pgfpathcurveto{\pgfqpoint{4.300996in}{1.152943in}}{\pgfqpoint{4.298162in}{1.146101in}}{\pgfqpoint{4.298162in}{1.138969in}}%
\pgfpathcurveto{\pgfqpoint{4.298162in}{1.131836in}}{\pgfqpoint{4.300996in}{1.124994in}}{\pgfqpoint{4.306040in}{1.119950in}}%
\pgfpathcurveto{\pgfqpoint{4.311084in}{1.114907in}}{\pgfqpoint{4.317925in}{1.112073in}}{\pgfqpoint{4.325058in}{1.112073in}}%
\pgfpathclose%
\pgfusepath{stroke,fill}%
\end{pgfscope}%
\begin{pgfscope}%
\pgfpathrectangle{\pgfqpoint{2.867647in}{0.500000in}}{\pgfqpoint{1.764706in}{1.700000in}}%
\pgfusepath{clip}%
\pgfsetbuttcap%
\pgfsetroundjoin%
\definecolor{currentfill}{rgb}{0.919781,0.275262,0.242460}%
\pgfsetfillcolor{currentfill}%
\pgfsetlinewidth{0.311001pt}%
\definecolor{currentstroke}{rgb}{1.000000,1.000000,1.000000}%
\pgfsetstrokecolor{currentstroke}%
\pgfsetdash{}{0pt}%
\pgfpathmoveto{\pgfqpoint{4.203433in}{0.846512in}}%
\pgfpathcurveto{\pgfqpoint{4.210566in}{0.846512in}}{\pgfqpoint{4.217408in}{0.849346in}}{\pgfqpoint{4.222452in}{0.854389in}}%
\pgfpathcurveto{\pgfqpoint{4.227495in}{0.859433in}}{\pgfqpoint{4.230329in}{0.866275in}}{\pgfqpoint{4.230329in}{0.873408in}}%
\pgfpathcurveto{\pgfqpoint{4.230329in}{0.880540in}}{\pgfqpoint{4.227495in}{0.887382in}}{\pgfqpoint{4.222452in}{0.892426in}}%
\pgfpathcurveto{\pgfqpoint{4.217408in}{0.897469in}}{\pgfqpoint{4.210566in}{0.900303in}}{\pgfqpoint{4.203433in}{0.900303in}}%
\pgfpathcurveto{\pgfqpoint{4.196301in}{0.900303in}}{\pgfqpoint{4.189459in}{0.897469in}}{\pgfqpoint{4.184415in}{0.892426in}}%
\pgfpathcurveto{\pgfqpoint{4.179372in}{0.887382in}}{\pgfqpoint{4.176538in}{0.880540in}}{\pgfqpoint{4.176538in}{0.873408in}}%
\pgfpathcurveto{\pgfqpoint{4.176538in}{0.866275in}}{\pgfqpoint{4.179372in}{0.859433in}}{\pgfqpoint{4.184415in}{0.854389in}}%
\pgfpathcurveto{\pgfqpoint{4.189459in}{0.849346in}}{\pgfqpoint{4.196301in}{0.846512in}}{\pgfqpoint{4.203433in}{0.846512in}}%
\pgfpathclose%
\pgfusepath{stroke,fill}%
\end{pgfscope}%
\begin{pgfscope}%
\pgfpathrectangle{\pgfqpoint{2.867647in}{0.500000in}}{\pgfqpoint{1.764706in}{1.700000in}}%
\pgfusepath{clip}%
\pgfsetbuttcap%
\pgfsetroundjoin%
\definecolor{currentfill}{rgb}{0.975018,0.868213,0.788710}%
\pgfsetfillcolor{currentfill}%
\pgfsetlinewidth{0.311001pt}%
\definecolor{currentstroke}{rgb}{1.000000,1.000000,1.000000}%
\pgfsetstrokecolor{currentstroke}%
\pgfsetdash{}{0pt}%
\pgfpathmoveto{\pgfqpoint{4.105218in}{1.147664in}}%
\pgfpathcurveto{\pgfqpoint{4.112350in}{1.147664in}}{\pgfqpoint{4.119192in}{1.150497in}}{\pgfqpoint{4.124236in}{1.155541in}}%
\pgfpathcurveto{\pgfqpoint{4.129279in}{1.160585in}}{\pgfqpoint{4.132113in}{1.167426in}}{\pgfqpoint{4.132113in}{1.174559in}}%
\pgfpathcurveto{\pgfqpoint{4.132113in}{1.181692in}}{\pgfqpoint{4.129279in}{1.188534in}}{\pgfqpoint{4.124236in}{1.193577in}}%
\pgfpathcurveto{\pgfqpoint{4.119192in}{1.198621in}}{\pgfqpoint{4.112350in}{1.201455in}}{\pgfqpoint{4.105218in}{1.201455in}}%
\pgfpathcurveto{\pgfqpoint{4.098085in}{1.201455in}}{\pgfqpoint{4.091243in}{1.198621in}}{\pgfqpoint{4.086199in}{1.193577in}}%
\pgfpathcurveto{\pgfqpoint{4.081156in}{1.188534in}}{\pgfqpoint{4.078322in}{1.181692in}}{\pgfqpoint{4.078322in}{1.174559in}}%
\pgfpathcurveto{\pgfqpoint{4.078322in}{1.167426in}}{\pgfqpoint{4.081156in}{1.160585in}}{\pgfqpoint{4.086199in}{1.155541in}}%
\pgfpathcurveto{\pgfqpoint{4.091243in}{1.150497in}}{\pgfqpoint{4.098085in}{1.147664in}}{\pgfqpoint{4.105218in}{1.147664in}}%
\pgfpathclose%
\pgfusepath{stroke,fill}%
\end{pgfscope}%
\begin{pgfscope}%
\pgfpathrectangle{\pgfqpoint{2.867647in}{0.500000in}}{\pgfqpoint{1.764706in}{1.700000in}}%
\pgfusepath{clip}%
\pgfsetbuttcap%
\pgfsetroundjoin%
\definecolor{currentfill}{rgb}{0.975018,0.868213,0.788710}%
\pgfsetfillcolor{currentfill}%
\pgfsetlinewidth{0.311001pt}%
\definecolor{currentstroke}{rgb}{1.000000,1.000000,1.000000}%
\pgfsetstrokecolor{currentstroke}%
\pgfsetdash{}{0pt}%
\pgfpathmoveto{\pgfqpoint{4.250987in}{1.296803in}}%
\pgfpathcurveto{\pgfqpoint{4.258120in}{1.296803in}}{\pgfqpoint{4.264962in}{1.299637in}}{\pgfqpoint{4.270005in}{1.304680in}}%
\pgfpathcurveto{\pgfqpoint{4.275049in}{1.309724in}}{\pgfqpoint{4.277883in}{1.316566in}}{\pgfqpoint{4.277883in}{1.323699in}}%
\pgfpathcurveto{\pgfqpoint{4.277883in}{1.330831in}}{\pgfqpoint{4.275049in}{1.337673in}}{\pgfqpoint{4.270005in}{1.342717in}}%
\pgfpathcurveto{\pgfqpoint{4.264962in}{1.347760in}}{\pgfqpoint{4.258120in}{1.350594in}}{\pgfqpoint{4.250987in}{1.350594in}}%
\pgfpathcurveto{\pgfqpoint{4.243854in}{1.350594in}}{\pgfqpoint{4.237013in}{1.347760in}}{\pgfqpoint{4.231969in}{1.342717in}}%
\pgfpathcurveto{\pgfqpoint{4.226925in}{1.337673in}}{\pgfqpoint{4.224091in}{1.330831in}}{\pgfqpoint{4.224091in}{1.323699in}}%
\pgfpathcurveto{\pgfqpoint{4.224091in}{1.316566in}}{\pgfqpoint{4.226925in}{1.309724in}}{\pgfqpoint{4.231969in}{1.304680in}}%
\pgfpathcurveto{\pgfqpoint{4.237013in}{1.299637in}}{\pgfqpoint{4.243854in}{1.296803in}}{\pgfqpoint{4.250987in}{1.296803in}}%
\pgfpathclose%
\pgfusepath{stroke,fill}%
\end{pgfscope}%
\begin{pgfscope}%
\pgfpathrectangle{\pgfqpoint{2.867647in}{0.500000in}}{\pgfqpoint{1.764706in}{1.700000in}}%
\pgfusepath{clip}%
\pgfsetbuttcap%
\pgfsetroundjoin%
\definecolor{currentfill}{rgb}{0.972201,0.839051,0.745789}%
\pgfsetfillcolor{currentfill}%
\pgfsetlinewidth{0.311001pt}%
\definecolor{currentstroke}{rgb}{1.000000,1.000000,1.000000}%
\pgfsetstrokecolor{currentstroke}%
\pgfsetdash{}{0pt}%
\pgfpathmoveto{\pgfqpoint{4.113258in}{1.390884in}}%
\pgfpathcurveto{\pgfqpoint{4.120391in}{1.390884in}}{\pgfqpoint{4.127232in}{1.393718in}}{\pgfqpoint{4.132276in}{1.398762in}}%
\pgfpathcurveto{\pgfqpoint{4.137320in}{1.403805in}}{\pgfqpoint{4.140153in}{1.410647in}}{\pgfqpoint{4.140153in}{1.417780in}}%
\pgfpathcurveto{\pgfqpoint{4.140153in}{1.424913in}}{\pgfqpoint{4.137320in}{1.431754in}}{\pgfqpoint{4.132276in}{1.436798in}}%
\pgfpathcurveto{\pgfqpoint{4.127232in}{1.441842in}}{\pgfqpoint{4.120391in}{1.444676in}}{\pgfqpoint{4.113258in}{1.444676in}}%
\pgfpathcurveto{\pgfqpoint{4.106125in}{1.444676in}}{\pgfqpoint{4.099283in}{1.441842in}}{\pgfqpoint{4.094240in}{1.436798in}}%
\pgfpathcurveto{\pgfqpoint{4.089196in}{1.431754in}}{\pgfqpoint{4.086362in}{1.424913in}}{\pgfqpoint{4.086362in}{1.417780in}}%
\pgfpathcurveto{\pgfqpoint{4.086362in}{1.410647in}}{\pgfqpoint{4.089196in}{1.403805in}}{\pgfqpoint{4.094240in}{1.398762in}}%
\pgfpathcurveto{\pgfqpoint{4.099283in}{1.393718in}}{\pgfqpoint{4.106125in}{1.390884in}}{\pgfqpoint{4.113258in}{1.390884in}}%
\pgfpathclose%
\pgfusepath{stroke,fill}%
\end{pgfscope}%
\begin{pgfscope}%
\pgfpathrectangle{\pgfqpoint{2.867647in}{0.500000in}}{\pgfqpoint{1.764706in}{1.700000in}}%
\pgfusepath{clip}%
\pgfsetbuttcap%
\pgfsetroundjoin%
\definecolor{currentfill}{rgb}{0.979891,0.908948,0.848279}%
\pgfsetfillcolor{currentfill}%
\pgfsetlinewidth{0.311001pt}%
\definecolor{currentstroke}{rgb}{1.000000,1.000000,1.000000}%
\pgfsetstrokecolor{currentstroke}%
\pgfsetdash{}{0pt}%
\pgfpathmoveto{\pgfqpoint{4.204016in}{1.184738in}}%
\pgfpathcurveto{\pgfqpoint{4.211149in}{1.184738in}}{\pgfqpoint{4.217990in}{1.187572in}}{\pgfqpoint{4.223034in}{1.192616in}}%
\pgfpathcurveto{\pgfqpoint{4.228078in}{1.197659in}}{\pgfqpoint{4.230911in}{1.204501in}}{\pgfqpoint{4.230911in}{1.211634in}}%
\pgfpathcurveto{\pgfqpoint{4.230911in}{1.218767in}}{\pgfqpoint{4.228078in}{1.225608in}}{\pgfqpoint{4.223034in}{1.230652in}}%
\pgfpathcurveto{\pgfqpoint{4.217990in}{1.235696in}}{\pgfqpoint{4.211149in}{1.238530in}}{\pgfqpoint{4.204016in}{1.238530in}}%
\pgfpathcurveto{\pgfqpoint{4.196883in}{1.238530in}}{\pgfqpoint{4.190041in}{1.235696in}}{\pgfqpoint{4.184998in}{1.230652in}}%
\pgfpathcurveto{\pgfqpoint{4.179954in}{1.225608in}}{\pgfqpoint{4.177120in}{1.218767in}}{\pgfqpoint{4.177120in}{1.211634in}}%
\pgfpathcurveto{\pgfqpoint{4.177120in}{1.204501in}}{\pgfqpoint{4.179954in}{1.197659in}}{\pgfqpoint{4.184998in}{1.192616in}}%
\pgfpathcurveto{\pgfqpoint{4.190041in}{1.187572in}}{\pgfqpoint{4.196883in}{1.184738in}}{\pgfqpoint{4.204016in}{1.184738in}}%
\pgfpathclose%
\pgfusepath{stroke,fill}%
\end{pgfscope}%
\begin{pgfscope}%
\pgfpathrectangle{\pgfqpoint{2.867647in}{0.500000in}}{\pgfqpoint{1.764706in}{1.700000in}}%
\pgfusepath{clip}%
\pgfsetbuttcap%
\pgfsetroundjoin%
\definecolor{currentfill}{rgb}{0.981377,0.920617,0.865369}%
\pgfsetfillcolor{currentfill}%
\pgfsetlinewidth{0.311001pt}%
\definecolor{currentstroke}{rgb}{1.000000,1.000000,1.000000}%
\pgfsetstrokecolor{currentstroke}%
\pgfsetdash{}{0pt}%
\pgfpathmoveto{\pgfqpoint{4.182328in}{1.246946in}}%
\pgfpathcurveto{\pgfqpoint{4.189461in}{1.246946in}}{\pgfqpoint{4.196303in}{1.249780in}}{\pgfqpoint{4.201346in}{1.254824in}}%
\pgfpathcurveto{\pgfqpoint{4.206390in}{1.259867in}}{\pgfqpoint{4.209224in}{1.266709in}}{\pgfqpoint{4.209224in}{1.273842in}}%
\pgfpathcurveto{\pgfqpoint{4.209224in}{1.280975in}}{\pgfqpoint{4.206390in}{1.287816in}}{\pgfqpoint{4.201346in}{1.292860in}}%
\pgfpathcurveto{\pgfqpoint{4.196303in}{1.297904in}}{\pgfqpoint{4.189461in}{1.300737in}}{\pgfqpoint{4.182328in}{1.300737in}}%
\pgfpathcurveto{\pgfqpoint{4.175195in}{1.300737in}}{\pgfqpoint{4.168354in}{1.297904in}}{\pgfqpoint{4.163310in}{1.292860in}}%
\pgfpathcurveto{\pgfqpoint{4.158266in}{1.287816in}}{\pgfqpoint{4.155432in}{1.280975in}}{\pgfqpoint{4.155432in}{1.273842in}}%
\pgfpathcurveto{\pgfqpoint{4.155432in}{1.266709in}}{\pgfqpoint{4.158266in}{1.259867in}}{\pgfqpoint{4.163310in}{1.254824in}}%
\pgfpathcurveto{\pgfqpoint{4.168354in}{1.249780in}}{\pgfqpoint{4.175195in}{1.246946in}}{\pgfqpoint{4.182328in}{1.246946in}}%
\pgfpathclose%
\pgfusepath{stroke,fill}%
\end{pgfscope}%
\begin{pgfscope}%
\pgfpathrectangle{\pgfqpoint{2.867647in}{0.500000in}}{\pgfqpoint{1.764706in}{1.700000in}}%
\pgfusepath{clip}%
\pgfsetbuttcap%
\pgfsetroundjoin%
\definecolor{currentfill}{rgb}{0.965592,0.726236,0.584384}%
\pgfsetfillcolor{currentfill}%
\pgfsetlinewidth{0.311001pt}%
\definecolor{currentstroke}{rgb}{1.000000,1.000000,1.000000}%
\pgfsetstrokecolor{currentstroke}%
\pgfsetdash{}{0pt}%
\pgfpathmoveto{\pgfqpoint{4.283421in}{1.477842in}}%
\pgfpathcurveto{\pgfqpoint{4.290554in}{1.477842in}}{\pgfqpoint{4.297395in}{1.480676in}}{\pgfqpoint{4.302439in}{1.485719in}}%
\pgfpathcurveto{\pgfqpoint{4.307483in}{1.490763in}}{\pgfqpoint{4.310317in}{1.497605in}}{\pgfqpoint{4.310317in}{1.504737in}}%
\pgfpathcurveto{\pgfqpoint{4.310317in}{1.511870in}}{\pgfqpoint{4.307483in}{1.518712in}}{\pgfqpoint{4.302439in}{1.523756in}}%
\pgfpathcurveto{\pgfqpoint{4.297395in}{1.528799in}}{\pgfqpoint{4.290554in}{1.531633in}}{\pgfqpoint{4.283421in}{1.531633in}}%
\pgfpathcurveto{\pgfqpoint{4.276288in}{1.531633in}}{\pgfqpoint{4.269446in}{1.528799in}}{\pgfqpoint{4.264403in}{1.523756in}}%
\pgfpathcurveto{\pgfqpoint{4.259359in}{1.518712in}}{\pgfqpoint{4.256525in}{1.511870in}}{\pgfqpoint{4.256525in}{1.504737in}}%
\pgfpathcurveto{\pgfqpoint{4.256525in}{1.497605in}}{\pgfqpoint{4.259359in}{1.490763in}}{\pgfqpoint{4.264403in}{1.485719in}}%
\pgfpathcurveto{\pgfqpoint{4.269446in}{1.480676in}}{\pgfqpoint{4.276288in}{1.477842in}}{\pgfqpoint{4.283421in}{1.477842in}}%
\pgfpathclose%
\pgfusepath{stroke,fill}%
\end{pgfscope}%
\begin{pgfscope}%
\pgfpathrectangle{\pgfqpoint{2.867647in}{0.500000in}}{\pgfqpoint{1.764706in}{1.700000in}}%
\pgfusepath{clip}%
\pgfsetbuttcap%
\pgfsetroundjoin%
\definecolor{currentfill}{rgb}{0.967092,0.768560,0.642079}%
\pgfsetfillcolor{currentfill}%
\pgfsetlinewidth{0.311001pt}%
\definecolor{currentstroke}{rgb}{1.000000,1.000000,1.000000}%
\pgfsetstrokecolor{currentstroke}%
\pgfsetdash{}{0pt}%
\pgfpathmoveto{\pgfqpoint{4.118584in}{1.717421in}}%
\pgfpathcurveto{\pgfqpoint{4.125717in}{1.717421in}}{\pgfqpoint{4.132558in}{1.720254in}}{\pgfqpoint{4.137602in}{1.725298in}}%
\pgfpathcurveto{\pgfqpoint{4.142646in}{1.730342in}}{\pgfqpoint{4.145479in}{1.737183in}}{\pgfqpoint{4.145479in}{1.744316in}}%
\pgfpathcurveto{\pgfqpoint{4.145479in}{1.751449in}}{\pgfqpoint{4.142646in}{1.758291in}}{\pgfqpoint{4.137602in}{1.763334in}}%
\pgfpathcurveto{\pgfqpoint{4.132558in}{1.768378in}}{\pgfqpoint{4.125717in}{1.771212in}}{\pgfqpoint{4.118584in}{1.771212in}}%
\pgfpathcurveto{\pgfqpoint{4.111451in}{1.771212in}}{\pgfqpoint{4.104609in}{1.768378in}}{\pgfqpoint{4.099566in}{1.763334in}}%
\pgfpathcurveto{\pgfqpoint{4.094522in}{1.758291in}}{\pgfqpoint{4.091688in}{1.751449in}}{\pgfqpoint{4.091688in}{1.744316in}}%
\pgfpathcurveto{\pgfqpoint{4.091688in}{1.737183in}}{\pgfqpoint{4.094522in}{1.730342in}}{\pgfqpoint{4.099566in}{1.725298in}}%
\pgfpathcurveto{\pgfqpoint{4.104609in}{1.720254in}}{\pgfqpoint{4.111451in}{1.717421in}}{\pgfqpoint{4.118584in}{1.717421in}}%
\pgfpathclose%
\pgfusepath{stroke,fill}%
\end{pgfscope}%
\begin{pgfscope}%
\pgfpathrectangle{\pgfqpoint{2.867647in}{0.500000in}}{\pgfqpoint{1.764706in}{1.700000in}}%
\pgfusepath{clip}%
\pgfsetbuttcap%
\pgfsetroundjoin%
\definecolor{currentfill}{rgb}{0.979891,0.908948,0.848279}%
\pgfsetfillcolor{currentfill}%
\pgfsetlinewidth{0.311001pt}%
\definecolor{currentstroke}{rgb}{1.000000,1.000000,1.000000}%
\pgfsetstrokecolor{currentstroke}%
\pgfsetdash{}{0pt}%
\pgfpathmoveto{\pgfqpoint{4.207115in}{1.397165in}}%
\pgfpathcurveto{\pgfqpoint{4.214247in}{1.397165in}}{\pgfqpoint{4.221089in}{1.399999in}}{\pgfqpoint{4.226133in}{1.405042in}}%
\pgfpathcurveto{\pgfqpoint{4.231176in}{1.410086in}}{\pgfqpoint{4.234010in}{1.416928in}}{\pgfqpoint{4.234010in}{1.424060in}}%
\pgfpathcurveto{\pgfqpoint{4.234010in}{1.431193in}}{\pgfqpoint{4.231176in}{1.438035in}}{\pgfqpoint{4.226133in}{1.443079in}}%
\pgfpathcurveto{\pgfqpoint{4.221089in}{1.448122in}}{\pgfqpoint{4.214247in}{1.450956in}}{\pgfqpoint{4.207115in}{1.450956in}}%
\pgfpathcurveto{\pgfqpoint{4.199982in}{1.450956in}}{\pgfqpoint{4.193140in}{1.448122in}}{\pgfqpoint{4.188096in}{1.443079in}}%
\pgfpathcurveto{\pgfqpoint{4.183053in}{1.438035in}}{\pgfqpoint{4.180219in}{1.431193in}}{\pgfqpoint{4.180219in}{1.424060in}}%
\pgfpathcurveto{\pgfqpoint{4.180219in}{1.416928in}}{\pgfqpoint{4.183053in}{1.410086in}}{\pgfqpoint{4.188096in}{1.405042in}}%
\pgfpathcurveto{\pgfqpoint{4.193140in}{1.399999in}}{\pgfqpoint{4.199982in}{1.397165in}}{\pgfqpoint{4.207115in}{1.397165in}}%
\pgfpathclose%
\pgfusepath{stroke,fill}%
\end{pgfscope}%
\begin{pgfscope}%
\pgfpathrectangle{\pgfqpoint{2.867647in}{0.500000in}}{\pgfqpoint{1.764706in}{1.700000in}}%
\pgfusepath{clip}%
\pgfsetbuttcap%
\pgfsetroundjoin%
\definecolor{currentfill}{rgb}{0.976287,0.879862,0.805788}%
\pgfsetfillcolor{currentfill}%
\pgfsetlinewidth{0.311001pt}%
\definecolor{currentstroke}{rgb}{1.000000,1.000000,1.000000}%
\pgfsetstrokecolor{currentstroke}%
\pgfsetdash{}{0pt}%
\pgfpathmoveto{\pgfqpoint{4.133385in}{1.621427in}}%
\pgfpathcurveto{\pgfqpoint{4.140518in}{1.621427in}}{\pgfqpoint{4.147360in}{1.624261in}}{\pgfqpoint{4.152403in}{1.629305in}}%
\pgfpathcurveto{\pgfqpoint{4.157447in}{1.634348in}}{\pgfqpoint{4.160281in}{1.641190in}}{\pgfqpoint{4.160281in}{1.648323in}}%
\pgfpathcurveto{\pgfqpoint{4.160281in}{1.655456in}}{\pgfqpoint{4.157447in}{1.662297in}}{\pgfqpoint{4.152403in}{1.667341in}}%
\pgfpathcurveto{\pgfqpoint{4.147360in}{1.672385in}}{\pgfqpoint{4.140518in}{1.675219in}}{\pgfqpoint{4.133385in}{1.675219in}}%
\pgfpathcurveto{\pgfqpoint{4.126252in}{1.675219in}}{\pgfqpoint{4.119411in}{1.672385in}}{\pgfqpoint{4.114367in}{1.667341in}}%
\pgfpathcurveto{\pgfqpoint{4.109323in}{1.662297in}}{\pgfqpoint{4.106489in}{1.655456in}}{\pgfqpoint{4.106489in}{1.648323in}}%
\pgfpathcurveto{\pgfqpoint{4.106489in}{1.641190in}}{\pgfqpoint{4.109323in}{1.634348in}}{\pgfqpoint{4.114367in}{1.629305in}}%
\pgfpathcurveto{\pgfqpoint{4.119411in}{1.624261in}}{\pgfqpoint{4.126252in}{1.621427in}}{\pgfqpoint{4.133385in}{1.621427in}}%
\pgfpathclose%
\pgfusepath{stroke,fill}%
\end{pgfscope}%
\begin{pgfscope}%
\pgfpathrectangle{\pgfqpoint{2.867647in}{0.500000in}}{\pgfqpoint{1.764706in}{1.700000in}}%
\pgfusepath{clip}%
\pgfsetbuttcap%
\pgfsetroundjoin%
\definecolor{currentfill}{rgb}{0.963728,0.638439,0.479050}%
\pgfsetfillcolor{currentfill}%
\pgfsetlinewidth{0.311001pt}%
\definecolor{currentstroke}{rgb}{1.000000,1.000000,1.000000}%
\pgfsetstrokecolor{currentstroke}%
\pgfsetdash{}{0pt}%
\pgfpathmoveto{\pgfqpoint{4.318834in}{1.405127in}}%
\pgfpathcurveto{\pgfqpoint{4.325966in}{1.405127in}}{\pgfqpoint{4.332808in}{1.407961in}}{\pgfqpoint{4.337852in}{1.413004in}}%
\pgfpathcurveto{\pgfqpoint{4.342895in}{1.418048in}}{\pgfqpoint{4.345729in}{1.424890in}}{\pgfqpoint{4.345729in}{1.432022in}}%
\pgfpathcurveto{\pgfqpoint{4.345729in}{1.439155in}}{\pgfqpoint{4.342895in}{1.445997in}}{\pgfqpoint{4.337852in}{1.451041in}}%
\pgfpathcurveto{\pgfqpoint{4.332808in}{1.456084in}}{\pgfqpoint{4.325966in}{1.458918in}}{\pgfqpoint{4.318834in}{1.458918in}}%
\pgfpathcurveto{\pgfqpoint{4.311701in}{1.458918in}}{\pgfqpoint{4.304859in}{1.456084in}}{\pgfqpoint{4.299815in}{1.451041in}}%
\pgfpathcurveto{\pgfqpoint{4.294772in}{1.445997in}}{\pgfqpoint{4.291938in}{1.439155in}}{\pgfqpoint{4.291938in}{1.432022in}}%
\pgfpathcurveto{\pgfqpoint{4.291938in}{1.424890in}}{\pgfqpoint{4.294772in}{1.418048in}}{\pgfqpoint{4.299815in}{1.413004in}}%
\pgfpathcurveto{\pgfqpoint{4.304859in}{1.407961in}}{\pgfqpoint{4.311701in}{1.405127in}}{\pgfqpoint{4.318834in}{1.405127in}}%
\pgfpathclose%
\pgfusepath{stroke,fill}%
\end{pgfscope}%
\begin{pgfscope}%
\pgfpathrectangle{\pgfqpoint{2.867647in}{0.500000in}}{\pgfqpoint{1.764706in}{1.700000in}}%
\pgfusepath{clip}%
\pgfsetbuttcap%
\pgfsetroundjoin%
\definecolor{currentfill}{rgb}{0.966120,0.744512,0.608720}%
\pgfsetfillcolor{currentfill}%
\pgfsetlinewidth{0.311001pt}%
\definecolor{currentstroke}{rgb}{1.000000,1.000000,1.000000}%
\pgfsetstrokecolor{currentstroke}%
\pgfsetdash{}{0pt}%
\pgfpathmoveto{\pgfqpoint{4.078773in}{1.249671in}}%
\pgfpathcurveto{\pgfqpoint{4.085906in}{1.249671in}}{\pgfqpoint{4.092748in}{1.252505in}}{\pgfqpoint{4.097791in}{1.257548in}}%
\pgfpathcurveto{\pgfqpoint{4.102835in}{1.262592in}}{\pgfqpoint{4.105669in}{1.269434in}}{\pgfqpoint{4.105669in}{1.276566in}}%
\pgfpathcurveto{\pgfqpoint{4.105669in}{1.283699in}}{\pgfqpoint{4.102835in}{1.290541in}}{\pgfqpoint{4.097791in}{1.295585in}}%
\pgfpathcurveto{\pgfqpoint{4.092748in}{1.300628in}}{\pgfqpoint{4.085906in}{1.303462in}}{\pgfqpoint{4.078773in}{1.303462in}}%
\pgfpathcurveto{\pgfqpoint{4.071640in}{1.303462in}}{\pgfqpoint{4.064799in}{1.300628in}}{\pgfqpoint{4.059755in}{1.295585in}}%
\pgfpathcurveto{\pgfqpoint{4.054711in}{1.290541in}}{\pgfqpoint{4.051877in}{1.283699in}}{\pgfqpoint{4.051877in}{1.276566in}}%
\pgfpathcurveto{\pgfqpoint{4.051877in}{1.269434in}}{\pgfqpoint{4.054711in}{1.262592in}}{\pgfqpoint{4.059755in}{1.257548in}}%
\pgfpathcurveto{\pgfqpoint{4.064799in}{1.252505in}}{\pgfqpoint{4.071640in}{1.249671in}}{\pgfqpoint{4.078773in}{1.249671in}}%
\pgfpathclose%
\pgfusepath{stroke,fill}%
\end{pgfscope}%
\begin{pgfscope}%
\pgfpathrectangle{\pgfqpoint{2.867647in}{0.500000in}}{\pgfqpoint{1.764706in}{1.700000in}}%
\pgfusepath{clip}%
\pgfsetbuttcap%
\pgfsetroundjoin%
\definecolor{currentfill}{rgb}{0.973832,0.856556,0.771584}%
\pgfsetfillcolor{currentfill}%
\pgfsetlinewidth{0.311001pt}%
\definecolor{currentstroke}{rgb}{1.000000,1.000000,1.000000}%
\pgfsetstrokecolor{currentstroke}%
\pgfsetdash{}{0pt}%
\pgfpathmoveto{\pgfqpoint{4.117227in}{1.655301in}}%
\pgfpathcurveto{\pgfqpoint{4.124360in}{1.655301in}}{\pgfqpoint{4.131202in}{1.658135in}}{\pgfqpoint{4.136245in}{1.663179in}}%
\pgfpathcurveto{\pgfqpoint{4.141289in}{1.668223in}}{\pgfqpoint{4.144123in}{1.675064in}}{\pgfqpoint{4.144123in}{1.682197in}}%
\pgfpathcurveto{\pgfqpoint{4.144123in}{1.689330in}}{\pgfqpoint{4.141289in}{1.696172in}}{\pgfqpoint{4.136245in}{1.701215in}}%
\pgfpathcurveto{\pgfqpoint{4.131202in}{1.706259in}}{\pgfqpoint{4.124360in}{1.709093in}}{\pgfqpoint{4.117227in}{1.709093in}}%
\pgfpathcurveto{\pgfqpoint{4.110094in}{1.709093in}}{\pgfqpoint{4.103253in}{1.706259in}}{\pgfqpoint{4.098209in}{1.701215in}}%
\pgfpathcurveto{\pgfqpoint{4.093165in}{1.696172in}}{\pgfqpoint{4.090331in}{1.689330in}}{\pgfqpoint{4.090331in}{1.682197in}}%
\pgfpathcurveto{\pgfqpoint{4.090331in}{1.675064in}}{\pgfqpoint{4.093165in}{1.668223in}}{\pgfqpoint{4.098209in}{1.663179in}}%
\pgfpathcurveto{\pgfqpoint{4.103253in}{1.658135in}}{\pgfqpoint{4.110094in}{1.655301in}}{\pgfqpoint{4.117227in}{1.655301in}}%
\pgfpathclose%
\pgfusepath{stroke,fill}%
\end{pgfscope}%
\begin{pgfscope}%
\pgfpathrectangle{\pgfqpoint{2.867647in}{0.500000in}}{\pgfqpoint{1.764706in}{1.700000in}}%
\pgfusepath{clip}%
\pgfsetbuttcap%
\pgfsetroundjoin%
\definecolor{currentfill}{rgb}{0.981377,0.920617,0.865369}%
\pgfsetfillcolor{currentfill}%
\pgfsetlinewidth{0.311001pt}%
\definecolor{currentstroke}{rgb}{1.000000,1.000000,1.000000}%
\pgfsetstrokecolor{currentstroke}%
\pgfsetdash{}{0pt}%
\pgfpathmoveto{\pgfqpoint{4.181411in}{1.288115in}}%
\pgfpathcurveto{\pgfqpoint{4.188544in}{1.288115in}}{\pgfqpoint{4.195385in}{1.290949in}}{\pgfqpoint{4.200429in}{1.295993in}}%
\pgfpathcurveto{\pgfqpoint{4.205473in}{1.301037in}}{\pgfqpoint{4.208307in}{1.307878in}}{\pgfqpoint{4.208307in}{1.315011in}}%
\pgfpathcurveto{\pgfqpoint{4.208307in}{1.322144in}}{\pgfqpoint{4.205473in}{1.328985in}}{\pgfqpoint{4.200429in}{1.334029in}}%
\pgfpathcurveto{\pgfqpoint{4.195385in}{1.339073in}}{\pgfqpoint{4.188544in}{1.341907in}}{\pgfqpoint{4.181411in}{1.341907in}}%
\pgfpathcurveto{\pgfqpoint{4.174278in}{1.341907in}}{\pgfqpoint{4.167436in}{1.339073in}}{\pgfqpoint{4.162393in}{1.334029in}}%
\pgfpathcurveto{\pgfqpoint{4.157349in}{1.328985in}}{\pgfqpoint{4.154515in}{1.322144in}}{\pgfqpoint{4.154515in}{1.315011in}}%
\pgfpathcurveto{\pgfqpoint{4.154515in}{1.307878in}}{\pgfqpoint{4.157349in}{1.301037in}}{\pgfqpoint{4.162393in}{1.295993in}}%
\pgfpathcurveto{\pgfqpoint{4.167436in}{1.290949in}}{\pgfqpoint{4.174278in}{1.288115in}}{\pgfqpoint{4.181411in}{1.288115in}}%
\pgfpathclose%
\pgfusepath{stroke,fill}%
\end{pgfscope}%
\begin{pgfscope}%
\pgfpathrectangle{\pgfqpoint{2.867647in}{0.500000in}}{\pgfqpoint{1.764706in}{1.700000in}}%
\pgfusepath{clip}%
\pgfsetbuttcap%
\pgfsetroundjoin%
\definecolor{currentfill}{rgb}{0.966120,0.744512,0.608720}%
\pgfsetfillcolor{currentfill}%
\pgfsetlinewidth{0.311001pt}%
\definecolor{currentstroke}{rgb}{1.000000,1.000000,1.000000}%
\pgfsetstrokecolor{currentstroke}%
\pgfsetdash{}{0pt}%
\pgfpathmoveto{\pgfqpoint{4.008277in}{1.629419in}}%
\pgfpathcurveto{\pgfqpoint{4.015409in}{1.629419in}}{\pgfqpoint{4.022251in}{1.632253in}}{\pgfqpoint{4.027295in}{1.637297in}}%
\pgfpathcurveto{\pgfqpoint{4.032338in}{1.642340in}}{\pgfqpoint{4.035172in}{1.649182in}}{\pgfqpoint{4.035172in}{1.656315in}}%
\pgfpathcurveto{\pgfqpoint{4.035172in}{1.663448in}}{\pgfqpoint{4.032338in}{1.670289in}}{\pgfqpoint{4.027295in}{1.675333in}}%
\pgfpathcurveto{\pgfqpoint{4.022251in}{1.680377in}}{\pgfqpoint{4.015409in}{1.683211in}}{\pgfqpoint{4.008277in}{1.683211in}}%
\pgfpathcurveto{\pgfqpoint{4.001144in}{1.683211in}}{\pgfqpoint{3.994302in}{1.680377in}}{\pgfqpoint{3.989258in}{1.675333in}}%
\pgfpathcurveto{\pgfqpoint{3.984215in}{1.670289in}}{\pgfqpoint{3.981381in}{1.663448in}}{\pgfqpoint{3.981381in}{1.656315in}}%
\pgfpathcurveto{\pgfqpoint{3.981381in}{1.649182in}}{\pgfqpoint{3.984215in}{1.642340in}}{\pgfqpoint{3.989258in}{1.637297in}}%
\pgfpathcurveto{\pgfqpoint{3.994302in}{1.632253in}}{\pgfqpoint{4.001144in}{1.629419in}}{\pgfqpoint{4.008277in}{1.629419in}}%
\pgfpathclose%
\pgfusepath{stroke,fill}%
\end{pgfscope}%
\begin{pgfscope}%
\pgfpathrectangle{\pgfqpoint{2.867647in}{0.500000in}}{\pgfqpoint{1.764706in}{1.700000in}}%
\pgfusepath{clip}%
\pgfsetbuttcap%
\pgfsetroundjoin%
\definecolor{currentfill}{rgb}{0.981377,0.920617,0.865369}%
\pgfsetfillcolor{currentfill}%
\pgfsetlinewidth{0.311001pt}%
\definecolor{currentstroke}{rgb}{1.000000,1.000000,1.000000}%
\pgfsetstrokecolor{currentstroke}%
\pgfsetdash{}{0pt}%
\pgfpathmoveto{\pgfqpoint{4.189176in}{1.202434in}}%
\pgfpathcurveto{\pgfqpoint{4.196309in}{1.202434in}}{\pgfqpoint{4.203151in}{1.205268in}}{\pgfqpoint{4.208194in}{1.210312in}}%
\pgfpathcurveto{\pgfqpoint{4.213238in}{1.215355in}}{\pgfqpoint{4.216072in}{1.222197in}}{\pgfqpoint{4.216072in}{1.229330in}}%
\pgfpathcurveto{\pgfqpoint{4.216072in}{1.236463in}}{\pgfqpoint{4.213238in}{1.243304in}}{\pgfqpoint{4.208194in}{1.248348in}}%
\pgfpathcurveto{\pgfqpoint{4.203151in}{1.253392in}}{\pgfqpoint{4.196309in}{1.256225in}}{\pgfqpoint{4.189176in}{1.256225in}}%
\pgfpathcurveto{\pgfqpoint{4.182043in}{1.256225in}}{\pgfqpoint{4.175202in}{1.253392in}}{\pgfqpoint{4.170158in}{1.248348in}}%
\pgfpathcurveto{\pgfqpoint{4.165114in}{1.243304in}}{\pgfqpoint{4.162280in}{1.236463in}}{\pgfqpoint{4.162280in}{1.229330in}}%
\pgfpathcurveto{\pgfqpoint{4.162280in}{1.222197in}}{\pgfqpoint{4.165114in}{1.215355in}}{\pgfqpoint{4.170158in}{1.210312in}}%
\pgfpathcurveto{\pgfqpoint{4.175202in}{1.205268in}}{\pgfqpoint{4.182043in}{1.202434in}}{\pgfqpoint{4.189176in}{1.202434in}}%
\pgfpathclose%
\pgfusepath{stroke,fill}%
\end{pgfscope}%
\begin{pgfscope}%
\pgfpathrectangle{\pgfqpoint{2.867647in}{0.500000in}}{\pgfqpoint{1.764706in}{1.700000in}}%
\pgfusepath{clip}%
\pgfsetbuttcap%
\pgfsetroundjoin%
\definecolor{currentfill}{rgb}{0.976287,0.879862,0.805788}%
\pgfsetfillcolor{currentfill}%
\pgfsetlinewidth{0.311001pt}%
\definecolor{currentstroke}{rgb}{1.000000,1.000000,1.000000}%
\pgfsetstrokecolor{currentstroke}%
\pgfsetdash{}{0pt}%
\pgfpathmoveto{\pgfqpoint{4.197659in}{1.537350in}}%
\pgfpathcurveto{\pgfqpoint{4.204792in}{1.537350in}}{\pgfqpoint{4.211634in}{1.540184in}}{\pgfqpoint{4.216677in}{1.545227in}}%
\pgfpathcurveto{\pgfqpoint{4.221721in}{1.550271in}}{\pgfqpoint{4.224555in}{1.557113in}}{\pgfqpoint{4.224555in}{1.564245in}}%
\pgfpathcurveto{\pgfqpoint{4.224555in}{1.571378in}}{\pgfqpoint{4.221721in}{1.578220in}}{\pgfqpoint{4.216677in}{1.583264in}}%
\pgfpathcurveto{\pgfqpoint{4.211634in}{1.588307in}}{\pgfqpoint{4.204792in}{1.591141in}}{\pgfqpoint{4.197659in}{1.591141in}}%
\pgfpathcurveto{\pgfqpoint{4.190526in}{1.591141in}}{\pgfqpoint{4.183685in}{1.588307in}}{\pgfqpoint{4.178641in}{1.583264in}}%
\pgfpathcurveto{\pgfqpoint{4.173597in}{1.578220in}}{\pgfqpoint{4.170763in}{1.571378in}}{\pgfqpoint{4.170763in}{1.564245in}}%
\pgfpathcurveto{\pgfqpoint{4.170763in}{1.557113in}}{\pgfqpoint{4.173597in}{1.550271in}}{\pgfqpoint{4.178641in}{1.545227in}}%
\pgfpathcurveto{\pgfqpoint{4.183685in}{1.540184in}}{\pgfqpoint{4.190526in}{1.537350in}}{\pgfqpoint{4.197659in}{1.537350in}}%
\pgfpathclose%
\pgfusepath{stroke,fill}%
\end{pgfscope}%
\begin{pgfscope}%
\pgfpathrectangle{\pgfqpoint{2.867647in}{0.500000in}}{\pgfqpoint{1.764706in}{1.700000in}}%
\pgfusepath{clip}%
\pgfsetbuttcap%
\pgfsetroundjoin%
\definecolor{currentfill}{rgb}{0.981377,0.920617,0.865369}%
\pgfsetfillcolor{currentfill}%
\pgfsetlinewidth{0.311001pt}%
\definecolor{currentstroke}{rgb}{1.000000,1.000000,1.000000}%
\pgfsetstrokecolor{currentstroke}%
\pgfsetdash{}{0pt}%
\pgfpathmoveto{\pgfqpoint{4.170091in}{1.239240in}}%
\pgfpathcurveto{\pgfqpoint{4.177224in}{1.239240in}}{\pgfqpoint{4.184066in}{1.242074in}}{\pgfqpoint{4.189110in}{1.247117in}}%
\pgfpathcurveto{\pgfqpoint{4.194153in}{1.252161in}}{\pgfqpoint{4.196987in}{1.259003in}}{\pgfqpoint{4.196987in}{1.266136in}}%
\pgfpathcurveto{\pgfqpoint{4.196987in}{1.273268in}}{\pgfqpoint{4.194153in}{1.280110in}}{\pgfqpoint{4.189110in}{1.285154in}}%
\pgfpathcurveto{\pgfqpoint{4.184066in}{1.290197in}}{\pgfqpoint{4.177224in}{1.293031in}}{\pgfqpoint{4.170091in}{1.293031in}}%
\pgfpathcurveto{\pgfqpoint{4.162959in}{1.293031in}}{\pgfqpoint{4.156117in}{1.290197in}}{\pgfqpoint{4.151073in}{1.285154in}}%
\pgfpathcurveto{\pgfqpoint{4.146030in}{1.280110in}}{\pgfqpoint{4.143196in}{1.273268in}}{\pgfqpoint{4.143196in}{1.266136in}}%
\pgfpathcurveto{\pgfqpoint{4.143196in}{1.259003in}}{\pgfqpoint{4.146030in}{1.252161in}}{\pgfqpoint{4.151073in}{1.247117in}}%
\pgfpathcurveto{\pgfqpoint{4.156117in}{1.242074in}}{\pgfqpoint{4.162959in}{1.239240in}}{\pgfqpoint{4.170091in}{1.239240in}}%
\pgfpathclose%
\pgfusepath{stroke,fill}%
\end{pgfscope}%
\begin{pgfscope}%
\pgfpathrectangle{\pgfqpoint{2.867647in}{0.500000in}}{\pgfqpoint{1.764706in}{1.700000in}}%
\pgfusepath{clip}%
\pgfsetbuttcap%
\pgfsetroundjoin%
\definecolor{currentfill}{rgb}{0.980678,0.914765,0.856766}%
\pgfsetfillcolor{currentfill}%
\pgfsetlinewidth{0.311001pt}%
\definecolor{currentstroke}{rgb}{1.000000,1.000000,1.000000}%
\pgfsetstrokecolor{currentstroke}%
\pgfsetdash{}{0pt}%
\pgfpathmoveto{\pgfqpoint{4.204486in}{1.213346in}}%
\pgfpathcurveto{\pgfqpoint{4.211619in}{1.213346in}}{\pgfqpoint{4.218460in}{1.216180in}}{\pgfqpoint{4.223504in}{1.221224in}}%
\pgfpathcurveto{\pgfqpoint{4.228548in}{1.226267in}}{\pgfqpoint{4.231382in}{1.233109in}}{\pgfqpoint{4.231382in}{1.240242in}}%
\pgfpathcurveto{\pgfqpoint{4.231382in}{1.247375in}}{\pgfqpoint{4.228548in}{1.254216in}}{\pgfqpoint{4.223504in}{1.259260in}}%
\pgfpathcurveto{\pgfqpoint{4.218460in}{1.264303in}}{\pgfqpoint{4.211619in}{1.267137in}}{\pgfqpoint{4.204486in}{1.267137in}}%
\pgfpathcurveto{\pgfqpoint{4.197353in}{1.267137in}}{\pgfqpoint{4.190511in}{1.264303in}}{\pgfqpoint{4.185468in}{1.259260in}}%
\pgfpathcurveto{\pgfqpoint{4.180424in}{1.254216in}}{\pgfqpoint{4.177590in}{1.247375in}}{\pgfqpoint{4.177590in}{1.240242in}}%
\pgfpathcurveto{\pgfqpoint{4.177590in}{1.233109in}}{\pgfqpoint{4.180424in}{1.226267in}}{\pgfqpoint{4.185468in}{1.221224in}}%
\pgfpathcurveto{\pgfqpoint{4.190511in}{1.216180in}}{\pgfqpoint{4.197353in}{1.213346in}}{\pgfqpoint{4.204486in}{1.213346in}}%
\pgfpathclose%
\pgfusepath{stroke,fill}%
\end{pgfscope}%
\begin{pgfscope}%
\pgfpathrectangle{\pgfqpoint{2.867647in}{0.500000in}}{\pgfqpoint{1.764706in}{1.700000in}}%
\pgfusepath{clip}%
\pgfsetbuttcap%
\pgfsetroundjoin%
\definecolor{currentfill}{rgb}{0.979891,0.908948,0.848279}%
\pgfsetfillcolor{currentfill}%
\pgfsetlinewidth{0.311001pt}%
\definecolor{currentstroke}{rgb}{1.000000,1.000000,1.000000}%
\pgfsetstrokecolor{currentstroke}%
\pgfsetdash{}{0pt}%
\pgfpathmoveto{\pgfqpoint{4.207031in}{1.210703in}}%
\pgfpathcurveto{\pgfqpoint{4.214164in}{1.210703in}}{\pgfqpoint{4.221006in}{1.213536in}}{\pgfqpoint{4.226049in}{1.218580in}}%
\pgfpathcurveto{\pgfqpoint{4.231093in}{1.223624in}}{\pgfqpoint{4.233927in}{1.230465in}}{\pgfqpoint{4.233927in}{1.237598in}}%
\pgfpathcurveto{\pgfqpoint{4.233927in}{1.244731in}}{\pgfqpoint{4.231093in}{1.251573in}}{\pgfqpoint{4.226049in}{1.256616in}}%
\pgfpathcurveto{\pgfqpoint{4.221006in}{1.261660in}}{\pgfqpoint{4.214164in}{1.264494in}}{\pgfqpoint{4.207031in}{1.264494in}}%
\pgfpathcurveto{\pgfqpoint{4.199898in}{1.264494in}}{\pgfqpoint{4.193057in}{1.261660in}}{\pgfqpoint{4.188013in}{1.256616in}}%
\pgfpathcurveto{\pgfqpoint{4.182969in}{1.251573in}}{\pgfqpoint{4.180136in}{1.244731in}}{\pgfqpoint{4.180136in}{1.237598in}}%
\pgfpathcurveto{\pgfqpoint{4.180136in}{1.230465in}}{\pgfqpoint{4.182969in}{1.223624in}}{\pgfqpoint{4.188013in}{1.218580in}}%
\pgfpathcurveto{\pgfqpoint{4.193057in}{1.213536in}}{\pgfqpoint{4.199898in}{1.210703in}}{\pgfqpoint{4.207031in}{1.210703in}}%
\pgfpathclose%
\pgfusepath{stroke,fill}%
\end{pgfscope}%
\begin{pgfscope}%
\pgfpathrectangle{\pgfqpoint{2.867647in}{0.500000in}}{\pgfqpoint{1.764706in}{1.700000in}}%
\pgfusepath{clip}%
\pgfsetbuttcap%
\pgfsetroundjoin%
\definecolor{currentfill}{rgb}{0.964306,0.663930,0.507747}%
\pgfsetfillcolor{currentfill}%
\pgfsetlinewidth{0.311001pt}%
\definecolor{currentstroke}{rgb}{1.000000,1.000000,1.000000}%
\pgfsetstrokecolor{currentstroke}%
\pgfsetdash{}{0pt}%
\pgfpathmoveto{\pgfqpoint{3.985039in}{1.025818in}}%
\pgfpathcurveto{\pgfqpoint{3.992171in}{1.025818in}}{\pgfqpoint{3.999013in}{1.028652in}}{\pgfqpoint{4.004057in}{1.033695in}}%
\pgfpathcurveto{\pgfqpoint{4.009100in}{1.038739in}}{\pgfqpoint{4.011934in}{1.045581in}}{\pgfqpoint{4.011934in}{1.052713in}}%
\pgfpathcurveto{\pgfqpoint{4.011934in}{1.059846in}}{\pgfqpoint{4.009100in}{1.066688in}}{\pgfqpoint{4.004057in}{1.071732in}}%
\pgfpathcurveto{\pgfqpoint{3.999013in}{1.076775in}}{\pgfqpoint{3.992171in}{1.079609in}}{\pgfqpoint{3.985039in}{1.079609in}}%
\pgfpathcurveto{\pgfqpoint{3.977906in}{1.079609in}}{\pgfqpoint{3.971064in}{1.076775in}}{\pgfqpoint{3.966020in}{1.071732in}}%
\pgfpathcurveto{\pgfqpoint{3.960977in}{1.066688in}}{\pgfqpoint{3.958143in}{1.059846in}}{\pgfqpoint{3.958143in}{1.052713in}}%
\pgfpathcurveto{\pgfqpoint{3.958143in}{1.045581in}}{\pgfqpoint{3.960977in}{1.038739in}}{\pgfqpoint{3.966020in}{1.033695in}}%
\pgfpathcurveto{\pgfqpoint{3.971064in}{1.028652in}}{\pgfqpoint{3.977906in}{1.025818in}}{\pgfqpoint{3.985039in}{1.025818in}}%
\pgfpathclose%
\pgfusepath{stroke,fill}%
\end{pgfscope}%
\begin{pgfscope}%
\pgfpathrectangle{\pgfqpoint{2.867647in}{0.500000in}}{\pgfqpoint{1.764706in}{1.700000in}}%
\pgfusepath{clip}%
\pgfsetbuttcap%
\pgfsetroundjoin%
\definecolor{currentfill}{rgb}{0.966328,0.750560,0.616961}%
\pgfsetfillcolor{currentfill}%
\pgfsetlinewidth{0.311001pt}%
\definecolor{currentstroke}{rgb}{1.000000,1.000000,1.000000}%
\pgfsetstrokecolor{currentstroke}%
\pgfsetdash{}{0pt}%
\pgfpathmoveto{\pgfqpoint{4.296190in}{1.333079in}}%
\pgfpathcurveto{\pgfqpoint{4.303323in}{1.333079in}}{\pgfqpoint{4.310165in}{1.335913in}}{\pgfqpoint{4.315208in}{1.340956in}}%
\pgfpathcurveto{\pgfqpoint{4.320252in}{1.346000in}}{\pgfqpoint{4.323086in}{1.352842in}}{\pgfqpoint{4.323086in}{1.359975in}}%
\pgfpathcurveto{\pgfqpoint{4.323086in}{1.367107in}}{\pgfqpoint{4.320252in}{1.373949in}}{\pgfqpoint{4.315208in}{1.378993in}}%
\pgfpathcurveto{\pgfqpoint{4.310165in}{1.384036in}}{\pgfqpoint{4.303323in}{1.386870in}}{\pgfqpoint{4.296190in}{1.386870in}}%
\pgfpathcurveto{\pgfqpoint{4.289057in}{1.386870in}}{\pgfqpoint{4.282216in}{1.384036in}}{\pgfqpoint{4.277172in}{1.378993in}}%
\pgfpathcurveto{\pgfqpoint{4.272128in}{1.373949in}}{\pgfqpoint{4.269294in}{1.367107in}}{\pgfqpoint{4.269294in}{1.359975in}}%
\pgfpathcurveto{\pgfqpoint{4.269294in}{1.352842in}}{\pgfqpoint{4.272128in}{1.346000in}}{\pgfqpoint{4.277172in}{1.340956in}}%
\pgfpathcurveto{\pgfqpoint{4.282216in}{1.335913in}}{\pgfqpoint{4.289057in}{1.333079in}}{\pgfqpoint{4.296190in}{1.333079in}}%
\pgfpathclose%
\pgfusepath{stroke,fill}%
\end{pgfscope}%
\begin{pgfscope}%
\pgfpathrectangle{\pgfqpoint{2.867647in}{0.500000in}}{\pgfqpoint{1.764706in}{1.700000in}}%
\pgfusepath{clip}%
\pgfsetbuttcap%
\pgfsetroundjoin%
\definecolor{currentfill}{rgb}{0.981377,0.920617,0.865369}%
\pgfsetfillcolor{currentfill}%
\pgfsetlinewidth{0.311001pt}%
\definecolor{currentstroke}{rgb}{1.000000,1.000000,1.000000}%
\pgfsetstrokecolor{currentstroke}%
\pgfsetdash{}{0pt}%
\pgfpathmoveto{\pgfqpoint{4.194522in}{1.205091in}}%
\pgfpathcurveto{\pgfqpoint{4.201654in}{1.205091in}}{\pgfqpoint{4.208496in}{1.207925in}}{\pgfqpoint{4.213540in}{1.212968in}}%
\pgfpathcurveto{\pgfqpoint{4.218583in}{1.218012in}}{\pgfqpoint{4.221417in}{1.224854in}}{\pgfqpoint{4.221417in}{1.231987in}}%
\pgfpathcurveto{\pgfqpoint{4.221417in}{1.239119in}}{\pgfqpoint{4.218583in}{1.245961in}}{\pgfqpoint{4.213540in}{1.251005in}}%
\pgfpathcurveto{\pgfqpoint{4.208496in}{1.256048in}}{\pgfqpoint{4.201654in}{1.258882in}}{\pgfqpoint{4.194522in}{1.258882in}}%
\pgfpathcurveto{\pgfqpoint{4.187389in}{1.258882in}}{\pgfqpoint{4.180547in}{1.256048in}}{\pgfqpoint{4.175503in}{1.251005in}}%
\pgfpathcurveto{\pgfqpoint{4.170460in}{1.245961in}}{\pgfqpoint{4.167626in}{1.239119in}}{\pgfqpoint{4.167626in}{1.231987in}}%
\pgfpathcurveto{\pgfqpoint{4.167626in}{1.224854in}}{\pgfqpoint{4.170460in}{1.218012in}}{\pgfqpoint{4.175503in}{1.212968in}}%
\pgfpathcurveto{\pgfqpoint{4.180547in}{1.207925in}}{\pgfqpoint{4.187389in}{1.205091in}}{\pgfqpoint{4.194522in}{1.205091in}}%
\pgfpathclose%
\pgfusepath{stroke,fill}%
\end{pgfscope}%
\begin{pgfscope}%
\pgfpathrectangle{\pgfqpoint{2.867647in}{0.500000in}}{\pgfqpoint{1.764706in}{1.700000in}}%
\pgfusepath{clip}%
\pgfsetbuttcap%
\pgfsetroundjoin%
\definecolor{currentfill}{rgb}{0.965928,0.738443,0.600540}%
\pgfsetfillcolor{currentfill}%
\pgfsetlinewidth{0.311001pt}%
\definecolor{currentstroke}{rgb}{1.000000,1.000000,1.000000}%
\pgfsetstrokecolor{currentstroke}%
\pgfsetdash{}{0pt}%
\pgfpathmoveto{\pgfqpoint{4.035205in}{1.529507in}}%
\pgfpathcurveto{\pgfqpoint{4.042338in}{1.529507in}}{\pgfqpoint{4.049180in}{1.532341in}}{\pgfqpoint{4.054224in}{1.537385in}}%
\pgfpathcurveto{\pgfqpoint{4.059267in}{1.542428in}}{\pgfqpoint{4.062101in}{1.549270in}}{\pgfqpoint{4.062101in}{1.556403in}}%
\pgfpathcurveto{\pgfqpoint{4.062101in}{1.563536in}}{\pgfqpoint{4.059267in}{1.570377in}}{\pgfqpoint{4.054224in}{1.575421in}}%
\pgfpathcurveto{\pgfqpoint{4.049180in}{1.580465in}}{\pgfqpoint{4.042338in}{1.583299in}}{\pgfqpoint{4.035205in}{1.583299in}}%
\pgfpathcurveto{\pgfqpoint{4.028073in}{1.583299in}}{\pgfqpoint{4.021231in}{1.580465in}}{\pgfqpoint{4.016187in}{1.575421in}}%
\pgfpathcurveto{\pgfqpoint{4.011144in}{1.570377in}}{\pgfqpoint{4.008310in}{1.563536in}}{\pgfqpoint{4.008310in}{1.556403in}}%
\pgfpathcurveto{\pgfqpoint{4.008310in}{1.549270in}}{\pgfqpoint{4.011144in}{1.542428in}}{\pgfqpoint{4.016187in}{1.537385in}}%
\pgfpathcurveto{\pgfqpoint{4.021231in}{1.532341in}}{\pgfqpoint{4.028073in}{1.529507in}}{\pgfqpoint{4.035205in}{1.529507in}}%
\pgfpathclose%
\pgfusepath{stroke,fill}%
\end{pgfscope}%
\begin{pgfscope}%
\pgfpathrectangle{\pgfqpoint{2.867647in}{0.500000in}}{\pgfqpoint{1.764706in}{1.700000in}}%
\pgfusepath{clip}%
\pgfsetbuttcap%
\pgfsetroundjoin%
\definecolor{currentfill}{rgb}{0.979124,0.903132,0.839793}%
\pgfsetfillcolor{currentfill}%
\pgfsetlinewidth{0.311001pt}%
\definecolor{currentstroke}{rgb}{1.000000,1.000000,1.000000}%
\pgfsetstrokecolor{currentstroke}%
\pgfsetdash{}{0pt}%
\pgfpathmoveto{\pgfqpoint{4.223745in}{1.343170in}}%
\pgfpathcurveto{\pgfqpoint{4.230878in}{1.343170in}}{\pgfqpoint{4.237720in}{1.346004in}}{\pgfqpoint{4.242763in}{1.351047in}}%
\pgfpathcurveto{\pgfqpoint{4.247807in}{1.356091in}}{\pgfqpoint{4.250641in}{1.362932in}}{\pgfqpoint{4.250641in}{1.370065in}}%
\pgfpathcurveto{\pgfqpoint{4.250641in}{1.377198in}}{\pgfqpoint{4.247807in}{1.384040in}}{\pgfqpoint{4.242763in}{1.389083in}}%
\pgfpathcurveto{\pgfqpoint{4.237720in}{1.394127in}}{\pgfqpoint{4.230878in}{1.396961in}}{\pgfqpoint{4.223745in}{1.396961in}}%
\pgfpathcurveto{\pgfqpoint{4.216612in}{1.396961in}}{\pgfqpoint{4.209771in}{1.394127in}}{\pgfqpoint{4.204727in}{1.389083in}}%
\pgfpathcurveto{\pgfqpoint{4.199684in}{1.384040in}}{\pgfqpoint{4.196850in}{1.377198in}}{\pgfqpoint{4.196850in}{1.370065in}}%
\pgfpathcurveto{\pgfqpoint{4.196850in}{1.362932in}}{\pgfqpoint{4.199684in}{1.356091in}}{\pgfqpoint{4.204727in}{1.351047in}}%
\pgfpathcurveto{\pgfqpoint{4.209771in}{1.346004in}}{\pgfqpoint{4.216612in}{1.343170in}}{\pgfqpoint{4.223745in}{1.343170in}}%
\pgfpathclose%
\pgfusepath{stroke,fill}%
\end{pgfscope}%
\begin{pgfscope}%
\pgfpathrectangle{\pgfqpoint{2.867647in}{0.500000in}}{\pgfqpoint{1.764706in}{1.700000in}}%
\pgfusepath{clip}%
\pgfsetbuttcap%
\pgfsetroundjoin%
\definecolor{currentfill}{rgb}{0.976961,0.885681,0.814303}%
\pgfsetfillcolor{currentfill}%
\pgfsetlinewidth{0.311001pt}%
\definecolor{currentstroke}{rgb}{1.000000,1.000000,1.000000}%
\pgfsetstrokecolor{currentstroke}%
\pgfsetdash{}{0pt}%
\pgfpathmoveto{\pgfqpoint{4.214626in}{1.148178in}}%
\pgfpathcurveto{\pgfqpoint{4.221759in}{1.148178in}}{\pgfqpoint{4.228600in}{1.151011in}}{\pgfqpoint{4.233644in}{1.156055in}}%
\pgfpathcurveto{\pgfqpoint{4.238688in}{1.161099in}}{\pgfqpoint{4.241522in}{1.167940in}}{\pgfqpoint{4.241522in}{1.175073in}}%
\pgfpathcurveto{\pgfqpoint{4.241522in}{1.182206in}}{\pgfqpoint{4.238688in}{1.189048in}}{\pgfqpoint{4.233644in}{1.194091in}}%
\pgfpathcurveto{\pgfqpoint{4.228600in}{1.199135in}}{\pgfqpoint{4.221759in}{1.201969in}}{\pgfqpoint{4.214626in}{1.201969in}}%
\pgfpathcurveto{\pgfqpoint{4.207493in}{1.201969in}}{\pgfqpoint{4.200651in}{1.199135in}}{\pgfqpoint{4.195608in}{1.194091in}}%
\pgfpathcurveto{\pgfqpoint{4.190564in}{1.189048in}}{\pgfqpoint{4.187730in}{1.182206in}}{\pgfqpoint{4.187730in}{1.175073in}}%
\pgfpathcurveto{\pgfqpoint{4.187730in}{1.167940in}}{\pgfqpoint{4.190564in}{1.161099in}}{\pgfqpoint{4.195608in}{1.156055in}}%
\pgfpathcurveto{\pgfqpoint{4.200651in}{1.151011in}}{\pgfqpoint{4.207493in}{1.148178in}}{\pgfqpoint{4.214626in}{1.148178in}}%
\pgfpathclose%
\pgfusepath{stroke,fill}%
\end{pgfscope}%
\begin{pgfscope}%
\pgfpathrectangle{\pgfqpoint{2.867647in}{0.500000in}}{\pgfqpoint{1.764706in}{1.700000in}}%
\pgfusepath{clip}%
\pgfsetbuttcap%
\pgfsetroundjoin%
\definecolor{currentfill}{rgb}{0.969803,0.809811,0.702523}%
\pgfsetfillcolor{currentfill}%
\pgfsetlinewidth{0.311001pt}%
\definecolor{currentstroke}{rgb}{1.000000,1.000000,1.000000}%
\pgfsetstrokecolor{currentstroke}%
\pgfsetdash{}{0pt}%
\pgfpathmoveto{\pgfqpoint{4.270116in}{1.216605in}}%
\pgfpathcurveto{\pgfqpoint{4.277249in}{1.216605in}}{\pgfqpoint{4.284091in}{1.219439in}}{\pgfqpoint{4.289134in}{1.224483in}}%
\pgfpathcurveto{\pgfqpoint{4.294178in}{1.229526in}}{\pgfqpoint{4.297012in}{1.236368in}}{\pgfqpoint{4.297012in}{1.243501in}}%
\pgfpathcurveto{\pgfqpoint{4.297012in}{1.250633in}}{\pgfqpoint{4.294178in}{1.257475in}}{\pgfqpoint{4.289134in}{1.262519in}}%
\pgfpathcurveto{\pgfqpoint{4.284091in}{1.267562in}}{\pgfqpoint{4.277249in}{1.270396in}}{\pgfqpoint{4.270116in}{1.270396in}}%
\pgfpathcurveto{\pgfqpoint{4.262983in}{1.270396in}}{\pgfqpoint{4.256142in}{1.267562in}}{\pgfqpoint{4.251098in}{1.262519in}}%
\pgfpathcurveto{\pgfqpoint{4.246054in}{1.257475in}}{\pgfqpoint{4.243220in}{1.250633in}}{\pgfqpoint{4.243220in}{1.243501in}}%
\pgfpathcurveto{\pgfqpoint{4.243220in}{1.236368in}}{\pgfqpoint{4.246054in}{1.229526in}}{\pgfqpoint{4.251098in}{1.224483in}}%
\pgfpathcurveto{\pgfqpoint{4.256142in}{1.219439in}}{\pgfqpoint{4.262983in}{1.216605in}}{\pgfqpoint{4.270116in}{1.216605in}}%
\pgfpathclose%
\pgfusepath{stroke,fill}%
\end{pgfscope}%
\begin{pgfscope}%
\pgfpathrectangle{\pgfqpoint{2.867647in}{0.500000in}}{\pgfqpoint{1.764706in}{1.700000in}}%
\pgfusepath{clip}%
\pgfsetbuttcap%
\pgfsetroundjoin%
\definecolor{currentfill}{rgb}{0.972726,0.844889,0.754401}%
\pgfsetfillcolor{currentfill}%
\pgfsetlinewidth{0.311001pt}%
\definecolor{currentstroke}{rgb}{1.000000,1.000000,1.000000}%
\pgfsetstrokecolor{currentstroke}%
\pgfsetdash{}{0pt}%
\pgfpathmoveto{\pgfqpoint{4.118914in}{1.288620in}}%
\pgfpathcurveto{\pgfqpoint{4.126047in}{1.288620in}}{\pgfqpoint{4.132889in}{1.291454in}}{\pgfqpoint{4.137932in}{1.296497in}}%
\pgfpathcurveto{\pgfqpoint{4.142976in}{1.301541in}}{\pgfqpoint{4.145810in}{1.308383in}}{\pgfqpoint{4.145810in}{1.315515in}}%
\pgfpathcurveto{\pgfqpoint{4.145810in}{1.322648in}}{\pgfqpoint{4.142976in}{1.329490in}}{\pgfqpoint{4.137932in}{1.334534in}}%
\pgfpathcurveto{\pgfqpoint{4.132889in}{1.339577in}}{\pgfqpoint{4.126047in}{1.342411in}}{\pgfqpoint{4.118914in}{1.342411in}}%
\pgfpathcurveto{\pgfqpoint{4.111781in}{1.342411in}}{\pgfqpoint{4.104940in}{1.339577in}}{\pgfqpoint{4.099896in}{1.334534in}}%
\pgfpathcurveto{\pgfqpoint{4.094852in}{1.329490in}}{\pgfqpoint{4.092018in}{1.322648in}}{\pgfqpoint{4.092018in}{1.315515in}}%
\pgfpathcurveto{\pgfqpoint{4.092018in}{1.308383in}}{\pgfqpoint{4.094852in}{1.301541in}}{\pgfqpoint{4.099896in}{1.296497in}}%
\pgfpathcurveto{\pgfqpoint{4.104940in}{1.291454in}}{\pgfqpoint{4.111781in}{1.288620in}}{\pgfqpoint{4.118914in}{1.288620in}}%
\pgfpathclose%
\pgfusepath{stroke,fill}%
\end{pgfscope}%
\begin{pgfscope}%
\pgfpathrectangle{\pgfqpoint{2.867647in}{0.500000in}}{\pgfqpoint{1.764706in}{1.700000in}}%
\pgfusepath{clip}%
\pgfsetbuttcap%
\pgfsetroundjoin%
\definecolor{currentfill}{rgb}{0.980678,0.914765,0.856766}%
\pgfsetfillcolor{currentfill}%
\pgfsetlinewidth{0.311001pt}%
\definecolor{currentstroke}{rgb}{1.000000,1.000000,1.000000}%
\pgfsetstrokecolor{currentstroke}%
\pgfsetdash{}{0pt}%
\pgfpathmoveto{\pgfqpoint{4.204843in}{1.368130in}}%
\pgfpathcurveto{\pgfqpoint{4.211976in}{1.368130in}}{\pgfqpoint{4.218817in}{1.370964in}}{\pgfqpoint{4.223861in}{1.376007in}}%
\pgfpathcurveto{\pgfqpoint{4.228905in}{1.381051in}}{\pgfqpoint{4.231738in}{1.387893in}}{\pgfqpoint{4.231738in}{1.395025in}}%
\pgfpathcurveto{\pgfqpoint{4.231738in}{1.402158in}}{\pgfqpoint{4.228905in}{1.409000in}}{\pgfqpoint{4.223861in}{1.414044in}}%
\pgfpathcurveto{\pgfqpoint{4.218817in}{1.419087in}}{\pgfqpoint{4.211976in}{1.421921in}}{\pgfqpoint{4.204843in}{1.421921in}}%
\pgfpathcurveto{\pgfqpoint{4.197710in}{1.421921in}}{\pgfqpoint{4.190868in}{1.419087in}}{\pgfqpoint{4.185825in}{1.414044in}}%
\pgfpathcurveto{\pgfqpoint{4.180781in}{1.409000in}}{\pgfqpoint{4.177947in}{1.402158in}}{\pgfqpoint{4.177947in}{1.395025in}}%
\pgfpathcurveto{\pgfqpoint{4.177947in}{1.387893in}}{\pgfqpoint{4.180781in}{1.381051in}}{\pgfqpoint{4.185825in}{1.376007in}}%
\pgfpathcurveto{\pgfqpoint{4.190868in}{1.370964in}}{\pgfqpoint{4.197710in}{1.368130in}}{\pgfqpoint{4.204843in}{1.368130in}}%
\pgfpathclose%
\pgfusepath{stroke,fill}%
\end{pgfscope}%
\begin{pgfscope}%
\pgfpathrectangle{\pgfqpoint{2.867647in}{0.500000in}}{\pgfqpoint{1.764706in}{1.700000in}}%
\pgfusepath{clip}%
\pgfsetbuttcap%
\pgfsetroundjoin%
\definecolor{currentfill}{rgb}{0.967735,0.780441,0.659127}%
\pgfsetfillcolor{currentfill}%
\pgfsetlinewidth{0.311001pt}%
\definecolor{currentstroke}{rgb}{1.000000,1.000000,1.000000}%
\pgfsetstrokecolor{currentstroke}%
\pgfsetdash{}{0pt}%
\pgfpathmoveto{\pgfqpoint{4.051243in}{1.714619in}}%
\pgfpathcurveto{\pgfqpoint{4.058376in}{1.714619in}}{\pgfqpoint{4.065218in}{1.717453in}}{\pgfqpoint{4.070261in}{1.722497in}}%
\pgfpathcurveto{\pgfqpoint{4.075305in}{1.727541in}}{\pgfqpoint{4.078139in}{1.734382in}}{\pgfqpoint{4.078139in}{1.741515in}}%
\pgfpathcurveto{\pgfqpoint{4.078139in}{1.748648in}}{\pgfqpoint{4.075305in}{1.755489in}}{\pgfqpoint{4.070261in}{1.760533in}}%
\pgfpathcurveto{\pgfqpoint{4.065218in}{1.765577in}}{\pgfqpoint{4.058376in}{1.768411in}}{\pgfqpoint{4.051243in}{1.768411in}}%
\pgfpathcurveto{\pgfqpoint{4.044110in}{1.768411in}}{\pgfqpoint{4.037269in}{1.765577in}}{\pgfqpoint{4.032225in}{1.760533in}}%
\pgfpathcurveto{\pgfqpoint{4.027181in}{1.755489in}}{\pgfqpoint{4.024347in}{1.748648in}}{\pgfqpoint{4.024347in}{1.741515in}}%
\pgfpathcurveto{\pgfqpoint{4.024347in}{1.734382in}}{\pgfqpoint{4.027181in}{1.727541in}}{\pgfqpoint{4.032225in}{1.722497in}}%
\pgfpathcurveto{\pgfqpoint{4.037269in}{1.717453in}}{\pgfqpoint{4.044110in}{1.714619in}}{\pgfqpoint{4.051243in}{1.714619in}}%
\pgfpathclose%
\pgfusepath{stroke,fill}%
\end{pgfscope}%
\begin{pgfscope}%
\pgfpathrectangle{\pgfqpoint{2.867647in}{0.500000in}}{\pgfqpoint{1.764706in}{1.700000in}}%
\pgfusepath{clip}%
\pgfsetbuttcap%
\pgfsetroundjoin%
\definecolor{currentfill}{rgb}{0.966328,0.750560,0.616961}%
\pgfsetfillcolor{currentfill}%
\pgfsetlinewidth{0.311001pt}%
\definecolor{currentstroke}{rgb}{1.000000,1.000000,1.000000}%
\pgfsetstrokecolor{currentstroke}%
\pgfsetdash{}{0pt}%
\pgfpathmoveto{\pgfqpoint{4.249366in}{1.074060in}}%
\pgfpathcurveto{\pgfqpoint{4.256499in}{1.074060in}}{\pgfqpoint{4.263341in}{1.076894in}}{\pgfqpoint{4.268385in}{1.081938in}}%
\pgfpathcurveto{\pgfqpoint{4.273428in}{1.086982in}}{\pgfqpoint{4.276262in}{1.093823in}}{\pgfqpoint{4.276262in}{1.100956in}}%
\pgfpathcurveto{\pgfqpoint{4.276262in}{1.108089in}}{\pgfqpoint{4.273428in}{1.114931in}}{\pgfqpoint{4.268385in}{1.119974in}}%
\pgfpathcurveto{\pgfqpoint{4.263341in}{1.125018in}}{\pgfqpoint{4.256499in}{1.127852in}}{\pgfqpoint{4.249366in}{1.127852in}}%
\pgfpathcurveto{\pgfqpoint{4.242234in}{1.127852in}}{\pgfqpoint{4.235392in}{1.125018in}}{\pgfqpoint{4.230348in}{1.119974in}}%
\pgfpathcurveto{\pgfqpoint{4.225305in}{1.114931in}}{\pgfqpoint{4.222471in}{1.108089in}}{\pgfqpoint{4.222471in}{1.100956in}}%
\pgfpathcurveto{\pgfqpoint{4.222471in}{1.093823in}}{\pgfqpoint{4.225305in}{1.086982in}}{\pgfqpoint{4.230348in}{1.081938in}}%
\pgfpathcurveto{\pgfqpoint{4.235392in}{1.076894in}}{\pgfqpoint{4.242234in}{1.074060in}}{\pgfqpoint{4.249366in}{1.074060in}}%
\pgfpathclose%
\pgfusepath{stroke,fill}%
\end{pgfscope}%
\begin{pgfscope}%
\pgfpathrectangle{\pgfqpoint{2.867647in}{0.500000in}}{\pgfqpoint{1.764706in}{1.700000in}}%
\pgfusepath{clip}%
\pgfsetbuttcap%
\pgfsetroundjoin%
\definecolor{currentfill}{rgb}{0.961433,0.573272,0.412036}%
\pgfsetfillcolor{currentfill}%
\pgfsetlinewidth{0.311001pt}%
\definecolor{currentstroke}{rgb}{1.000000,1.000000,1.000000}%
\pgfsetstrokecolor{currentstroke}%
\pgfsetdash{}{0pt}%
\pgfpathmoveto{\pgfqpoint{3.994146in}{1.533217in}}%
\pgfpathcurveto{\pgfqpoint{4.001279in}{1.533217in}}{\pgfqpoint{4.008121in}{1.536051in}}{\pgfqpoint{4.013164in}{1.541095in}}%
\pgfpathcurveto{\pgfqpoint{4.018208in}{1.546138in}}{\pgfqpoint{4.021042in}{1.552980in}}{\pgfqpoint{4.021042in}{1.560113in}}%
\pgfpathcurveto{\pgfqpoint{4.021042in}{1.567246in}}{\pgfqpoint{4.018208in}{1.574087in}}{\pgfqpoint{4.013164in}{1.579131in}}%
\pgfpathcurveto{\pgfqpoint{4.008121in}{1.584175in}}{\pgfqpoint{4.001279in}{1.587009in}}{\pgfqpoint{3.994146in}{1.587009in}}%
\pgfpathcurveto{\pgfqpoint{3.987013in}{1.587009in}}{\pgfqpoint{3.980172in}{1.584175in}}{\pgfqpoint{3.975128in}{1.579131in}}%
\pgfpathcurveto{\pgfqpoint{3.970084in}{1.574087in}}{\pgfqpoint{3.967250in}{1.567246in}}{\pgfqpoint{3.967250in}{1.560113in}}%
\pgfpathcurveto{\pgfqpoint{3.967250in}{1.552980in}}{\pgfqpoint{3.970084in}{1.546138in}}{\pgfqpoint{3.975128in}{1.541095in}}%
\pgfpathcurveto{\pgfqpoint{3.980172in}{1.536051in}}{\pgfqpoint{3.987013in}{1.533217in}}{\pgfqpoint{3.994146in}{1.533217in}}%
\pgfpathclose%
\pgfusepath{stroke,fill}%
\end{pgfscope}%
\begin{pgfscope}%
\pgfpathrectangle{\pgfqpoint{2.867647in}{0.500000in}}{\pgfqpoint{1.764706in}{1.700000in}}%
\pgfusepath{clip}%
\pgfsetbuttcap%
\pgfsetroundjoin%
\definecolor{currentfill}{rgb}{0.975644,0.874038,0.797253}%
\pgfsetfillcolor{currentfill}%
\pgfsetlinewidth{0.311001pt}%
\definecolor{currentstroke}{rgb}{1.000000,1.000000,1.000000}%
\pgfsetstrokecolor{currentstroke}%
\pgfsetdash{}{0pt}%
\pgfpathmoveto{\pgfqpoint{4.231865in}{1.183252in}}%
\pgfpathcurveto{\pgfqpoint{4.238998in}{1.183252in}}{\pgfqpoint{4.245840in}{1.186086in}}{\pgfqpoint{4.250883in}{1.191130in}}%
\pgfpathcurveto{\pgfqpoint{4.255927in}{1.196173in}}{\pgfqpoint{4.258761in}{1.203015in}}{\pgfqpoint{4.258761in}{1.210148in}}%
\pgfpathcurveto{\pgfqpoint{4.258761in}{1.217280in}}{\pgfqpoint{4.255927in}{1.224122in}}{\pgfqpoint{4.250883in}{1.229166in}}%
\pgfpathcurveto{\pgfqpoint{4.245840in}{1.234209in}}{\pgfqpoint{4.238998in}{1.237043in}}{\pgfqpoint{4.231865in}{1.237043in}}%
\pgfpathcurveto{\pgfqpoint{4.224732in}{1.237043in}}{\pgfqpoint{4.217891in}{1.234209in}}{\pgfqpoint{4.212847in}{1.229166in}}%
\pgfpathcurveto{\pgfqpoint{4.207803in}{1.224122in}}{\pgfqpoint{4.204969in}{1.217280in}}{\pgfqpoint{4.204969in}{1.210148in}}%
\pgfpathcurveto{\pgfqpoint{4.204969in}{1.203015in}}{\pgfqpoint{4.207803in}{1.196173in}}{\pgfqpoint{4.212847in}{1.191130in}}%
\pgfpathcurveto{\pgfqpoint{4.217891in}{1.186086in}}{\pgfqpoint{4.224732in}{1.183252in}}{\pgfqpoint{4.231865in}{1.183252in}}%
\pgfpathclose%
\pgfusepath{stroke,fill}%
\end{pgfscope}%
\begin{pgfscope}%
\pgfpathrectangle{\pgfqpoint{2.867647in}{0.500000in}}{\pgfqpoint{1.764706in}{1.700000in}}%
\pgfusepath{clip}%
\pgfsetbuttcap%
\pgfsetroundjoin%
\definecolor{currentfill}{rgb}{0.977657,0.891500,0.822809}%
\pgfsetfillcolor{currentfill}%
\pgfsetlinewidth{0.311001pt}%
\definecolor{currentstroke}{rgb}{1.000000,1.000000,1.000000}%
\pgfsetstrokecolor{currentstroke}%
\pgfsetdash{}{0pt}%
\pgfpathmoveto{\pgfqpoint{4.198328in}{1.118544in}}%
\pgfpathcurveto{\pgfqpoint{4.205461in}{1.118544in}}{\pgfqpoint{4.212303in}{1.121378in}}{\pgfqpoint{4.217346in}{1.126422in}}%
\pgfpathcurveto{\pgfqpoint{4.222390in}{1.131466in}}{\pgfqpoint{4.225224in}{1.138307in}}{\pgfqpoint{4.225224in}{1.145440in}}%
\pgfpathcurveto{\pgfqpoint{4.225224in}{1.152573in}}{\pgfqpoint{4.222390in}{1.159414in}}{\pgfqpoint{4.217346in}{1.164458in}}%
\pgfpathcurveto{\pgfqpoint{4.212303in}{1.169502in}}{\pgfqpoint{4.205461in}{1.172336in}}{\pgfqpoint{4.198328in}{1.172336in}}%
\pgfpathcurveto{\pgfqpoint{4.191195in}{1.172336in}}{\pgfqpoint{4.184354in}{1.169502in}}{\pgfqpoint{4.179310in}{1.164458in}}%
\pgfpathcurveto{\pgfqpoint{4.174266in}{1.159414in}}{\pgfqpoint{4.171432in}{1.152573in}}{\pgfqpoint{4.171432in}{1.145440in}}%
\pgfpathcurveto{\pgfqpoint{4.171432in}{1.138307in}}{\pgfqpoint{4.174266in}{1.131466in}}{\pgfqpoint{4.179310in}{1.126422in}}%
\pgfpathcurveto{\pgfqpoint{4.184354in}{1.121378in}}{\pgfqpoint{4.191195in}{1.118544in}}{\pgfqpoint{4.198328in}{1.118544in}}%
\pgfpathclose%
\pgfusepath{stroke,fill}%
\end{pgfscope}%
\begin{pgfscope}%
\pgfpathrectangle{\pgfqpoint{2.867647in}{0.500000in}}{\pgfqpoint{1.764706in}{1.700000in}}%
\pgfusepath{clip}%
\pgfsetbuttcap%
\pgfsetroundjoin%
\definecolor{currentfill}{rgb}{0.980678,0.914765,0.856766}%
\pgfsetfillcolor{currentfill}%
\pgfsetlinewidth{0.311001pt}%
\definecolor{currentstroke}{rgb}{1.000000,1.000000,1.000000}%
\pgfsetstrokecolor{currentstroke}%
\pgfsetdash{}{0pt}%
\pgfpathmoveto{\pgfqpoint{4.168816in}{1.144054in}}%
\pgfpathcurveto{\pgfqpoint{4.175949in}{1.144054in}}{\pgfqpoint{4.182790in}{1.146888in}}{\pgfqpoint{4.187834in}{1.151931in}}%
\pgfpathcurveto{\pgfqpoint{4.192878in}{1.156975in}}{\pgfqpoint{4.195712in}{1.163817in}}{\pgfqpoint{4.195712in}{1.170950in}}%
\pgfpathcurveto{\pgfqpoint{4.195712in}{1.178082in}}{\pgfqpoint{4.192878in}{1.184924in}}{\pgfqpoint{4.187834in}{1.189968in}}%
\pgfpathcurveto{\pgfqpoint{4.182790in}{1.195011in}}{\pgfqpoint{4.175949in}{1.197845in}}{\pgfqpoint{4.168816in}{1.197845in}}%
\pgfpathcurveto{\pgfqpoint{4.161683in}{1.197845in}}{\pgfqpoint{4.154842in}{1.195011in}}{\pgfqpoint{4.149798in}{1.189968in}}%
\pgfpathcurveto{\pgfqpoint{4.144754in}{1.184924in}}{\pgfqpoint{4.141920in}{1.178082in}}{\pgfqpoint{4.141920in}{1.170950in}}%
\pgfpathcurveto{\pgfqpoint{4.141920in}{1.163817in}}{\pgfqpoint{4.144754in}{1.156975in}}{\pgfqpoint{4.149798in}{1.151931in}}%
\pgfpathcurveto{\pgfqpoint{4.154842in}{1.146888in}}{\pgfqpoint{4.161683in}{1.144054in}}{\pgfqpoint{4.168816in}{1.144054in}}%
\pgfpathclose%
\pgfusepath{stroke,fill}%
\end{pgfscope}%
\begin{pgfscope}%
\pgfpathrectangle{\pgfqpoint{2.867647in}{0.500000in}}{\pgfqpoint{1.764706in}{1.700000in}}%
\pgfusepath{clip}%
\pgfsetbuttcap%
\pgfsetroundjoin%
\definecolor{currentfill}{rgb}{0.969803,0.809811,0.702523}%
\pgfsetfillcolor{currentfill}%
\pgfsetlinewidth{0.311001pt}%
\definecolor{currentstroke}{rgb}{1.000000,1.000000,1.000000}%
\pgfsetstrokecolor{currentstroke}%
\pgfsetdash{}{0pt}%
\pgfpathmoveto{\pgfqpoint{4.100613in}{1.384475in}}%
\pgfpathcurveto{\pgfqpoint{4.107746in}{1.384475in}}{\pgfqpoint{4.114588in}{1.387309in}}{\pgfqpoint{4.119631in}{1.392353in}}%
\pgfpathcurveto{\pgfqpoint{4.124675in}{1.397397in}}{\pgfqpoint{4.127509in}{1.404238in}}{\pgfqpoint{4.127509in}{1.411371in}}%
\pgfpathcurveto{\pgfqpoint{4.127509in}{1.418504in}}{\pgfqpoint{4.124675in}{1.425346in}}{\pgfqpoint{4.119631in}{1.430389in}}%
\pgfpathcurveto{\pgfqpoint{4.114588in}{1.435433in}}{\pgfqpoint{4.107746in}{1.438267in}}{\pgfqpoint{4.100613in}{1.438267in}}%
\pgfpathcurveto{\pgfqpoint{4.093480in}{1.438267in}}{\pgfqpoint{4.086639in}{1.435433in}}{\pgfqpoint{4.081595in}{1.430389in}}%
\pgfpathcurveto{\pgfqpoint{4.076551in}{1.425346in}}{\pgfqpoint{4.073717in}{1.418504in}}{\pgfqpoint{4.073717in}{1.411371in}}%
\pgfpathcurveto{\pgfqpoint{4.073717in}{1.404238in}}{\pgfqpoint{4.076551in}{1.397397in}}{\pgfqpoint{4.081595in}{1.392353in}}%
\pgfpathcurveto{\pgfqpoint{4.086639in}{1.387309in}}{\pgfqpoint{4.093480in}{1.384475in}}{\pgfqpoint{4.100613in}{1.384475in}}%
\pgfpathclose%
\pgfusepath{stroke,fill}%
\end{pgfscope}%
\begin{pgfscope}%
\pgfpathrectangle{\pgfqpoint{2.867647in}{0.500000in}}{\pgfqpoint{1.764706in}{1.700000in}}%
\pgfusepath{clip}%
\pgfsetbuttcap%
\pgfsetroundjoin%
\definecolor{currentfill}{rgb}{0.965440,0.720101,0.576404}%
\pgfsetfillcolor{currentfill}%
\pgfsetlinewidth{0.311001pt}%
\definecolor{currentstroke}{rgb}{1.000000,1.000000,1.000000}%
\pgfsetstrokecolor{currentstroke}%
\pgfsetdash{}{0pt}%
\pgfpathmoveto{\pgfqpoint{4.292910in}{1.437717in}}%
\pgfpathcurveto{\pgfqpoint{4.300043in}{1.437717in}}{\pgfqpoint{4.306885in}{1.440551in}}{\pgfqpoint{4.311928in}{1.445595in}}%
\pgfpathcurveto{\pgfqpoint{4.316972in}{1.450638in}}{\pgfqpoint{4.319806in}{1.457480in}}{\pgfqpoint{4.319806in}{1.464613in}}%
\pgfpathcurveto{\pgfqpoint{4.319806in}{1.471746in}}{\pgfqpoint{4.316972in}{1.478587in}}{\pgfqpoint{4.311928in}{1.483631in}}%
\pgfpathcurveto{\pgfqpoint{4.306885in}{1.488675in}}{\pgfqpoint{4.300043in}{1.491509in}}{\pgfqpoint{4.292910in}{1.491509in}}%
\pgfpathcurveto{\pgfqpoint{4.285777in}{1.491509in}}{\pgfqpoint{4.278936in}{1.488675in}}{\pgfqpoint{4.273892in}{1.483631in}}%
\pgfpathcurveto{\pgfqpoint{4.268848in}{1.478587in}}{\pgfqpoint{4.266015in}{1.471746in}}{\pgfqpoint{4.266015in}{1.464613in}}%
\pgfpathcurveto{\pgfqpoint{4.266015in}{1.457480in}}{\pgfqpoint{4.268848in}{1.450638in}}{\pgfqpoint{4.273892in}{1.445595in}}%
\pgfpathcurveto{\pgfqpoint{4.278936in}{1.440551in}}{\pgfqpoint{4.285777in}{1.437717in}}{\pgfqpoint{4.292910in}{1.437717in}}%
\pgfpathclose%
\pgfusepath{stroke,fill}%
\end{pgfscope}%
\begin{pgfscope}%
\pgfpathrectangle{\pgfqpoint{2.867647in}{0.500000in}}{\pgfqpoint{1.764706in}{1.700000in}}%
\pgfusepath{clip}%
\pgfsetbuttcap%
\pgfsetroundjoin%
\definecolor{currentfill}{rgb}{0.964799,0.689101,0.537560}%
\pgfsetfillcolor{currentfill}%
\pgfsetlinewidth{0.311001pt}%
\definecolor{currentstroke}{rgb}{1.000000,1.000000,1.000000}%
\pgfsetstrokecolor{currentstroke}%
\pgfsetdash{}{0pt}%
\pgfpathmoveto{\pgfqpoint{4.280729in}{1.109963in}}%
\pgfpathcurveto{\pgfqpoint{4.287861in}{1.109963in}}{\pgfqpoint{4.294703in}{1.112797in}}{\pgfqpoint{4.299747in}{1.117840in}}%
\pgfpathcurveto{\pgfqpoint{4.304790in}{1.122884in}}{\pgfqpoint{4.307624in}{1.129726in}}{\pgfqpoint{4.307624in}{1.136858in}}%
\pgfpathcurveto{\pgfqpoint{4.307624in}{1.143991in}}{\pgfqpoint{4.304790in}{1.150833in}}{\pgfqpoint{4.299747in}{1.155877in}}%
\pgfpathcurveto{\pgfqpoint{4.294703in}{1.160920in}}{\pgfqpoint{4.287861in}{1.163754in}}{\pgfqpoint{4.280729in}{1.163754in}}%
\pgfpathcurveto{\pgfqpoint{4.273596in}{1.163754in}}{\pgfqpoint{4.266754in}{1.160920in}}{\pgfqpoint{4.261710in}{1.155877in}}%
\pgfpathcurveto{\pgfqpoint{4.256667in}{1.150833in}}{\pgfqpoint{4.253833in}{1.143991in}}{\pgfqpoint{4.253833in}{1.136858in}}%
\pgfpathcurveto{\pgfqpoint{4.253833in}{1.129726in}}{\pgfqpoint{4.256667in}{1.122884in}}{\pgfqpoint{4.261710in}{1.117840in}}%
\pgfpathcurveto{\pgfqpoint{4.266754in}{1.112797in}}{\pgfqpoint{4.273596in}{1.109963in}}{\pgfqpoint{4.280729in}{1.109963in}}%
\pgfpathclose%
\pgfusepath{stroke,fill}%
\end{pgfscope}%
\begin{pgfscope}%
\pgfpathrectangle{\pgfqpoint{2.867647in}{0.500000in}}{\pgfqpoint{1.764706in}{1.700000in}}%
\pgfusepath{clip}%
\pgfsetbuttcap%
\pgfsetroundjoin%
\definecolor{currentfill}{rgb}{0.968931,0.798091,0.685123}%
\pgfsetfillcolor{currentfill}%
\pgfsetlinewidth{0.311001pt}%
\definecolor{currentstroke}{rgb}{1.000000,1.000000,1.000000}%
\pgfsetstrokecolor{currentstroke}%
\pgfsetdash{}{0pt}%
\pgfpathmoveto{\pgfqpoint{4.189974in}{1.006762in}}%
\pgfpathcurveto{\pgfqpoint{4.197107in}{1.006762in}}{\pgfqpoint{4.203949in}{1.009596in}}{\pgfqpoint{4.208992in}{1.014640in}}%
\pgfpathcurveto{\pgfqpoint{4.214036in}{1.019684in}}{\pgfqpoint{4.216870in}{1.026525in}}{\pgfqpoint{4.216870in}{1.033658in}}%
\pgfpathcurveto{\pgfqpoint{4.216870in}{1.040791in}}{\pgfqpoint{4.214036in}{1.047633in}}{\pgfqpoint{4.208992in}{1.052676in}}%
\pgfpathcurveto{\pgfqpoint{4.203949in}{1.057720in}}{\pgfqpoint{4.197107in}{1.060554in}}{\pgfqpoint{4.189974in}{1.060554in}}%
\pgfpathcurveto{\pgfqpoint{4.182841in}{1.060554in}}{\pgfqpoint{4.176000in}{1.057720in}}{\pgfqpoint{4.170956in}{1.052676in}}%
\pgfpathcurveto{\pgfqpoint{4.165912in}{1.047633in}}{\pgfqpoint{4.163078in}{1.040791in}}{\pgfqpoint{4.163078in}{1.033658in}}%
\pgfpathcurveto{\pgfqpoint{4.163078in}{1.026525in}}{\pgfqpoint{4.165912in}{1.019684in}}{\pgfqpoint{4.170956in}{1.014640in}}%
\pgfpathcurveto{\pgfqpoint{4.176000in}{1.009596in}}{\pgfqpoint{4.182841in}{1.006762in}}{\pgfqpoint{4.189974in}{1.006762in}}%
\pgfpathclose%
\pgfusepath{stroke,fill}%
\end{pgfscope}%
\begin{pgfscope}%
\pgfpathrectangle{\pgfqpoint{2.867647in}{0.500000in}}{\pgfqpoint{1.764706in}{1.700000in}}%
\pgfusepath{clip}%
\pgfsetbuttcap%
\pgfsetroundjoin%
\definecolor{currentfill}{rgb}{0.979124,0.903132,0.839793}%
\pgfsetfillcolor{currentfill}%
\pgfsetlinewidth{0.311001pt}%
\definecolor{currentstroke}{rgb}{1.000000,1.000000,1.000000}%
\pgfsetstrokecolor{currentstroke}%
\pgfsetdash{}{0pt}%
\pgfpathmoveto{\pgfqpoint{4.128843in}{1.108284in}}%
\pgfpathcurveto{\pgfqpoint{4.135976in}{1.108284in}}{\pgfqpoint{4.142817in}{1.111118in}}{\pgfqpoint{4.147861in}{1.116162in}}%
\pgfpathcurveto{\pgfqpoint{4.152905in}{1.121205in}}{\pgfqpoint{4.155739in}{1.128047in}}{\pgfqpoint{4.155739in}{1.135180in}}%
\pgfpathcurveto{\pgfqpoint{4.155739in}{1.142313in}}{\pgfqpoint{4.152905in}{1.149154in}}{\pgfqpoint{4.147861in}{1.154198in}}%
\pgfpathcurveto{\pgfqpoint{4.142817in}{1.159242in}}{\pgfqpoint{4.135976in}{1.162076in}}{\pgfqpoint{4.128843in}{1.162076in}}%
\pgfpathcurveto{\pgfqpoint{4.121710in}{1.162076in}}{\pgfqpoint{4.114868in}{1.159242in}}{\pgfqpoint{4.109825in}{1.154198in}}%
\pgfpathcurveto{\pgfqpoint{4.104781in}{1.149154in}}{\pgfqpoint{4.101947in}{1.142313in}}{\pgfqpoint{4.101947in}{1.135180in}}%
\pgfpathcurveto{\pgfqpoint{4.101947in}{1.128047in}}{\pgfqpoint{4.104781in}{1.121205in}}{\pgfqpoint{4.109825in}{1.116162in}}%
\pgfpathcurveto{\pgfqpoint{4.114868in}{1.111118in}}{\pgfqpoint{4.121710in}{1.108284in}}{\pgfqpoint{4.128843in}{1.108284in}}%
\pgfpathclose%
\pgfusepath{stroke,fill}%
\end{pgfscope}%
\begin{pgfscope}%
\pgfpathrectangle{\pgfqpoint{2.867647in}{0.500000in}}{\pgfqpoint{1.764706in}{1.700000in}}%
\pgfusepath{clip}%
\pgfsetbuttcap%
\pgfsetroundjoin%
\definecolor{currentfill}{rgb}{0.945204,0.390623,0.270949}%
\pgfsetfillcolor{currentfill}%
\pgfsetlinewidth{0.311001pt}%
\definecolor{currentstroke}{rgb}{1.000000,1.000000,1.000000}%
\pgfsetstrokecolor{currentstroke}%
\pgfsetdash{}{0pt}%
\pgfpathmoveto{\pgfqpoint{3.864782in}{1.663727in}}%
\pgfpathcurveto{\pgfqpoint{3.871915in}{1.663727in}}{\pgfqpoint{3.878756in}{1.666560in}}{\pgfqpoint{3.883800in}{1.671604in}}%
\pgfpathcurveto{\pgfqpoint{3.888844in}{1.676648in}}{\pgfqpoint{3.891677in}{1.683489in}}{\pgfqpoint{3.891677in}{1.690622in}}%
\pgfpathcurveto{\pgfqpoint{3.891677in}{1.697755in}}{\pgfqpoint{3.888844in}{1.704597in}}{\pgfqpoint{3.883800in}{1.709640in}}%
\pgfpathcurveto{\pgfqpoint{3.878756in}{1.714684in}}{\pgfqpoint{3.871915in}{1.717518in}}{\pgfqpoint{3.864782in}{1.717518in}}%
\pgfpathcurveto{\pgfqpoint{3.857649in}{1.717518in}}{\pgfqpoint{3.850807in}{1.714684in}}{\pgfqpoint{3.845764in}{1.709640in}}%
\pgfpathcurveto{\pgfqpoint{3.840720in}{1.704597in}}{\pgfqpoint{3.837886in}{1.697755in}}{\pgfqpoint{3.837886in}{1.690622in}}%
\pgfpathcurveto{\pgfqpoint{3.837886in}{1.683489in}}{\pgfqpoint{3.840720in}{1.676648in}}{\pgfqpoint{3.845764in}{1.671604in}}%
\pgfpathcurveto{\pgfqpoint{3.850807in}{1.666560in}}{\pgfqpoint{3.857649in}{1.663727in}}{\pgfqpoint{3.864782in}{1.663727in}}%
\pgfpathclose%
\pgfusepath{stroke,fill}%
\end{pgfscope}%
\begin{pgfscope}%
\pgfpathrectangle{\pgfqpoint{2.867647in}{0.500000in}}{\pgfqpoint{1.764706in}{1.700000in}}%
\pgfusepath{clip}%
\pgfsetbuttcap%
\pgfsetroundjoin%
\definecolor{currentfill}{rgb}{0.965753,0.732351,0.592427}%
\pgfsetfillcolor{currentfill}%
\pgfsetlinewidth{0.311001pt}%
\definecolor{currentstroke}{rgb}{1.000000,1.000000,1.000000}%
\pgfsetstrokecolor{currentstroke}%
\pgfsetdash{}{0pt}%
\pgfpathmoveto{\pgfqpoint{4.051010in}{0.901712in}}%
\pgfpathcurveto{\pgfqpoint{4.058143in}{0.901712in}}{\pgfqpoint{4.064984in}{0.904545in}}{\pgfqpoint{4.070028in}{0.909589in}}%
\pgfpathcurveto{\pgfqpoint{4.075072in}{0.914633in}}{\pgfqpoint{4.077906in}{0.921474in}}{\pgfqpoint{4.077906in}{0.928607in}}%
\pgfpathcurveto{\pgfqpoint{4.077906in}{0.935740in}}{\pgfqpoint{4.075072in}{0.942582in}}{\pgfqpoint{4.070028in}{0.947625in}}%
\pgfpathcurveto{\pgfqpoint{4.064984in}{0.952669in}}{\pgfqpoint{4.058143in}{0.955503in}}{\pgfqpoint{4.051010in}{0.955503in}}%
\pgfpathcurveto{\pgfqpoint{4.043877in}{0.955503in}}{\pgfqpoint{4.037035in}{0.952669in}}{\pgfqpoint{4.031992in}{0.947625in}}%
\pgfpathcurveto{\pgfqpoint{4.026948in}{0.942582in}}{\pgfqpoint{4.024114in}{0.935740in}}{\pgfqpoint{4.024114in}{0.928607in}}%
\pgfpathcurveto{\pgfqpoint{4.024114in}{0.921474in}}{\pgfqpoint{4.026948in}{0.914633in}}{\pgfqpoint{4.031992in}{0.909589in}}%
\pgfpathcurveto{\pgfqpoint{4.037035in}{0.904545in}}{\pgfqpoint{4.043877in}{0.901712in}}{\pgfqpoint{4.051010in}{0.901712in}}%
\pgfpathclose%
\pgfusepath{stroke,fill}%
\end{pgfscope}%
\begin{pgfscope}%
\pgfpathrectangle{\pgfqpoint{2.867647in}{0.500000in}}{\pgfqpoint{1.764706in}{1.700000in}}%
\pgfusepath{clip}%
\pgfsetbuttcap%
\pgfsetroundjoin%
\definecolor{currentfill}{rgb}{0.977657,0.891500,0.822809}%
\pgfsetfillcolor{currentfill}%
\pgfsetlinewidth{0.311001pt}%
\definecolor{currentstroke}{rgb}{1.000000,1.000000,1.000000}%
\pgfsetstrokecolor{currentstroke}%
\pgfsetdash{}{0pt}%
\pgfpathmoveto{\pgfqpoint{4.109643in}{1.599028in}}%
\pgfpathcurveto{\pgfqpoint{4.116776in}{1.599028in}}{\pgfqpoint{4.123617in}{1.601862in}}{\pgfqpoint{4.128661in}{1.606906in}}%
\pgfpathcurveto{\pgfqpoint{4.133705in}{1.611949in}}{\pgfqpoint{4.136539in}{1.618791in}}{\pgfqpoint{4.136539in}{1.625924in}}%
\pgfpathcurveto{\pgfqpoint{4.136539in}{1.633056in}}{\pgfqpoint{4.133705in}{1.639898in}}{\pgfqpoint{4.128661in}{1.644942in}}%
\pgfpathcurveto{\pgfqpoint{4.123617in}{1.649985in}}{\pgfqpoint{4.116776in}{1.652819in}}{\pgfqpoint{4.109643in}{1.652819in}}%
\pgfpathcurveto{\pgfqpoint{4.102510in}{1.652819in}}{\pgfqpoint{4.095668in}{1.649985in}}{\pgfqpoint{4.090625in}{1.644942in}}%
\pgfpathcurveto{\pgfqpoint{4.085581in}{1.639898in}}{\pgfqpoint{4.082747in}{1.633056in}}{\pgfqpoint{4.082747in}{1.625924in}}%
\pgfpathcurveto{\pgfqpoint{4.082747in}{1.618791in}}{\pgfqpoint{4.085581in}{1.611949in}}{\pgfqpoint{4.090625in}{1.606906in}}%
\pgfpathcurveto{\pgfqpoint{4.095668in}{1.601862in}}{\pgfqpoint{4.102510in}{1.599028in}}{\pgfqpoint{4.109643in}{1.599028in}}%
\pgfpathclose%
\pgfusepath{stroke,fill}%
\end{pgfscope}%
\begin{pgfscope}%
\pgfpathrectangle{\pgfqpoint{2.867647in}{0.500000in}}{\pgfqpoint{1.764706in}{1.700000in}}%
\pgfusepath{clip}%
\pgfsetbuttcap%
\pgfsetroundjoin%
\definecolor{currentfill}{rgb}{0.970255,0.815666,0.711203}%
\pgfsetfillcolor{currentfill}%
\pgfsetlinewidth{0.311001pt}%
\definecolor{currentstroke}{rgb}{1.000000,1.000000,1.000000}%
\pgfsetstrokecolor{currentstroke}%
\pgfsetdash{}{0pt}%
\pgfpathmoveto{\pgfqpoint{4.125261in}{0.961333in}}%
\pgfpathcurveto{\pgfqpoint{4.132394in}{0.961333in}}{\pgfqpoint{4.139235in}{0.964167in}}{\pgfqpoint{4.144279in}{0.969211in}}%
\pgfpathcurveto{\pgfqpoint{4.149323in}{0.974254in}}{\pgfqpoint{4.152157in}{0.981096in}}{\pgfqpoint{4.152157in}{0.988229in}}%
\pgfpathcurveto{\pgfqpoint{4.152157in}{0.995362in}}{\pgfqpoint{4.149323in}{1.002203in}}{\pgfqpoint{4.144279in}{1.007247in}}%
\pgfpathcurveto{\pgfqpoint{4.139235in}{1.012291in}}{\pgfqpoint{4.132394in}{1.015125in}}{\pgfqpoint{4.125261in}{1.015125in}}%
\pgfpathcurveto{\pgfqpoint{4.118128in}{1.015125in}}{\pgfqpoint{4.111286in}{1.012291in}}{\pgfqpoint{4.106243in}{1.007247in}}%
\pgfpathcurveto{\pgfqpoint{4.101199in}{1.002203in}}{\pgfqpoint{4.098365in}{0.995362in}}{\pgfqpoint{4.098365in}{0.988229in}}%
\pgfpathcurveto{\pgfqpoint{4.098365in}{0.981096in}}{\pgfqpoint{4.101199in}{0.974254in}}{\pgfqpoint{4.106243in}{0.969211in}}%
\pgfpathcurveto{\pgfqpoint{4.111286in}{0.964167in}}{\pgfqpoint{4.118128in}{0.961333in}}{\pgfqpoint{4.125261in}{0.961333in}}%
\pgfpathclose%
\pgfusepath{stroke,fill}%
\end{pgfscope}%
\begin{pgfscope}%
\pgfpathrectangle{\pgfqpoint{2.867647in}{0.500000in}}{\pgfqpoint{1.764706in}{1.700000in}}%
\pgfusepath{clip}%
\pgfsetbuttcap%
\pgfsetroundjoin%
\definecolor{currentfill}{rgb}{0.971202,0.827364,0.728520}%
\pgfsetfillcolor{currentfill}%
\pgfsetlinewidth{0.311001pt}%
\definecolor{currentstroke}{rgb}{1.000000,1.000000,1.000000}%
\pgfsetstrokecolor{currentstroke}%
\pgfsetdash{}{0pt}%
\pgfpathmoveto{\pgfqpoint{4.093205in}{1.182710in}}%
\pgfpathcurveto{\pgfqpoint{4.100338in}{1.182710in}}{\pgfqpoint{4.107179in}{1.185544in}}{\pgfqpoint{4.112223in}{1.190588in}}%
\pgfpathcurveto{\pgfqpoint{4.117267in}{1.195631in}}{\pgfqpoint{4.120101in}{1.202473in}}{\pgfqpoint{4.120101in}{1.209606in}}%
\pgfpathcurveto{\pgfqpoint{4.120101in}{1.216739in}}{\pgfqpoint{4.117267in}{1.223580in}}{\pgfqpoint{4.112223in}{1.228624in}}%
\pgfpathcurveto{\pgfqpoint{4.107179in}{1.233668in}}{\pgfqpoint{4.100338in}{1.236501in}}{\pgfqpoint{4.093205in}{1.236501in}}%
\pgfpathcurveto{\pgfqpoint{4.086072in}{1.236501in}}{\pgfqpoint{4.079230in}{1.233668in}}{\pgfqpoint{4.074187in}{1.228624in}}%
\pgfpathcurveto{\pgfqpoint{4.069143in}{1.223580in}}{\pgfqpoint{4.066309in}{1.216739in}}{\pgfqpoint{4.066309in}{1.209606in}}%
\pgfpathcurveto{\pgfqpoint{4.066309in}{1.202473in}}{\pgfqpoint{4.069143in}{1.195631in}}{\pgfqpoint{4.074187in}{1.190588in}}%
\pgfpathcurveto{\pgfqpoint{4.079230in}{1.185544in}}{\pgfqpoint{4.086072in}{1.182710in}}{\pgfqpoint{4.093205in}{1.182710in}}%
\pgfpathclose%
\pgfusepath{stroke,fill}%
\end{pgfscope}%
\begin{pgfscope}%
\pgfpathrectangle{\pgfqpoint{2.867647in}{0.500000in}}{\pgfqpoint{1.764706in}{1.700000in}}%
\pgfusepath{clip}%
\pgfsetbuttcap%
\pgfsetroundjoin%
\definecolor{currentfill}{rgb}{0.976287,0.879862,0.805788}%
\pgfsetfillcolor{currentfill}%
\pgfsetlinewidth{0.311001pt}%
\definecolor{currentstroke}{rgb}{1.000000,1.000000,1.000000}%
\pgfsetstrokecolor{currentstroke}%
\pgfsetdash{}{0pt}%
\pgfpathmoveto{\pgfqpoint{4.102472in}{1.516476in}}%
\pgfpathcurveto{\pgfqpoint{4.109605in}{1.516476in}}{\pgfqpoint{4.116446in}{1.519310in}}{\pgfqpoint{4.121490in}{1.524354in}}%
\pgfpathcurveto{\pgfqpoint{4.126534in}{1.529398in}}{\pgfqpoint{4.129368in}{1.536239in}}{\pgfqpoint{4.129368in}{1.543372in}}%
\pgfpathcurveto{\pgfqpoint{4.129368in}{1.550505in}}{\pgfqpoint{4.126534in}{1.557347in}}{\pgfqpoint{4.121490in}{1.562390in}}%
\pgfpathcurveto{\pgfqpoint{4.116446in}{1.567434in}}{\pgfqpoint{4.109605in}{1.570268in}}{\pgfqpoint{4.102472in}{1.570268in}}%
\pgfpathcurveto{\pgfqpoint{4.095339in}{1.570268in}}{\pgfqpoint{4.088498in}{1.567434in}}{\pgfqpoint{4.083454in}{1.562390in}}%
\pgfpathcurveto{\pgfqpoint{4.078410in}{1.557347in}}{\pgfqpoint{4.075576in}{1.550505in}}{\pgfqpoint{4.075576in}{1.543372in}}%
\pgfpathcurveto{\pgfqpoint{4.075576in}{1.536239in}}{\pgfqpoint{4.078410in}{1.529398in}}{\pgfqpoint{4.083454in}{1.524354in}}%
\pgfpathcurveto{\pgfqpoint{4.088498in}{1.519310in}}{\pgfqpoint{4.095339in}{1.516476in}}{\pgfqpoint{4.102472in}{1.516476in}}%
\pgfpathclose%
\pgfusepath{stroke,fill}%
\end{pgfscope}%
\begin{pgfscope}%
\pgfpathrectangle{\pgfqpoint{2.867647in}{0.500000in}}{\pgfqpoint{1.764706in}{1.700000in}}%
\pgfusepath{clip}%
\pgfsetbuttcap%
\pgfsetroundjoin%
\definecolor{currentfill}{rgb}{0.971694,0.833208,0.737161}%
\pgfsetfillcolor{currentfill}%
\pgfsetlinewidth{0.311001pt}%
\definecolor{currentstroke}{rgb}{1.000000,1.000000,1.000000}%
\pgfsetstrokecolor{currentstroke}%
\pgfsetdash{}{0pt}%
\pgfpathmoveto{\pgfqpoint{4.081353in}{1.679036in}}%
\pgfpathcurveto{\pgfqpoint{4.088486in}{1.679036in}}{\pgfqpoint{4.095328in}{1.681870in}}{\pgfqpoint{4.100371in}{1.686914in}}%
\pgfpathcurveto{\pgfqpoint{4.105415in}{1.691958in}}{\pgfqpoint{4.108249in}{1.698799in}}{\pgfqpoint{4.108249in}{1.705932in}}%
\pgfpathcurveto{\pgfqpoint{4.108249in}{1.713065in}}{\pgfqpoint{4.105415in}{1.719907in}}{\pgfqpoint{4.100371in}{1.724950in}}%
\pgfpathcurveto{\pgfqpoint{4.095328in}{1.729994in}}{\pgfqpoint{4.088486in}{1.732828in}}{\pgfqpoint{4.081353in}{1.732828in}}%
\pgfpathcurveto{\pgfqpoint{4.074220in}{1.732828in}}{\pgfqpoint{4.067379in}{1.729994in}}{\pgfqpoint{4.062335in}{1.724950in}}%
\pgfpathcurveto{\pgfqpoint{4.057291in}{1.719907in}}{\pgfqpoint{4.054458in}{1.713065in}}{\pgfqpoint{4.054458in}{1.705932in}}%
\pgfpathcurveto{\pgfqpoint{4.054458in}{1.698799in}}{\pgfqpoint{4.057291in}{1.691958in}}{\pgfqpoint{4.062335in}{1.686914in}}%
\pgfpathcurveto{\pgfqpoint{4.067379in}{1.681870in}}{\pgfqpoint{4.074220in}{1.679036in}}{\pgfqpoint{4.081353in}{1.679036in}}%
\pgfpathclose%
\pgfusepath{stroke,fill}%
\end{pgfscope}%
\begin{pgfscope}%
\pgfpathrectangle{\pgfqpoint{2.867647in}{0.500000in}}{\pgfqpoint{1.764706in}{1.700000in}}%
\pgfusepath{clip}%
\pgfsetbuttcap%
\pgfsetroundjoin%
\definecolor{currentfill}{rgb}{0.977657,0.891500,0.822809}%
\pgfsetfillcolor{currentfill}%
\pgfsetlinewidth{0.311001pt}%
\definecolor{currentstroke}{rgb}{1.000000,1.000000,1.000000}%
\pgfsetstrokecolor{currentstroke}%
\pgfsetdash{}{0pt}%
\pgfpathmoveto{\pgfqpoint{4.149015in}{1.355692in}}%
\pgfpathcurveto{\pgfqpoint{4.156148in}{1.355692in}}{\pgfqpoint{4.162990in}{1.358526in}}{\pgfqpoint{4.168033in}{1.363570in}}%
\pgfpathcurveto{\pgfqpoint{4.173077in}{1.368613in}}{\pgfqpoint{4.175911in}{1.375455in}}{\pgfqpoint{4.175911in}{1.382588in}}%
\pgfpathcurveto{\pgfqpoint{4.175911in}{1.389721in}}{\pgfqpoint{4.173077in}{1.396562in}}{\pgfqpoint{4.168033in}{1.401606in}}%
\pgfpathcurveto{\pgfqpoint{4.162990in}{1.406649in}}{\pgfqpoint{4.156148in}{1.409483in}}{\pgfqpoint{4.149015in}{1.409483in}}%
\pgfpathcurveto{\pgfqpoint{4.141882in}{1.409483in}}{\pgfqpoint{4.135041in}{1.406649in}}{\pgfqpoint{4.129997in}{1.401606in}}%
\pgfpathcurveto{\pgfqpoint{4.124953in}{1.396562in}}{\pgfqpoint{4.122120in}{1.389721in}}{\pgfqpoint{4.122120in}{1.382588in}}%
\pgfpathcurveto{\pgfqpoint{4.122120in}{1.375455in}}{\pgfqpoint{4.124953in}{1.368613in}}{\pgfqpoint{4.129997in}{1.363570in}}%
\pgfpathcurveto{\pgfqpoint{4.135041in}{1.358526in}}{\pgfqpoint{4.141882in}{1.355692in}}{\pgfqpoint{4.149015in}{1.355692in}}%
\pgfpathclose%
\pgfusepath{stroke,fill}%
\end{pgfscope}%
\begin{pgfscope}%
\pgfpathrectangle{\pgfqpoint{2.867647in}{0.500000in}}{\pgfqpoint{1.764706in}{1.700000in}}%
\pgfusepath{clip}%
\pgfsetbuttcap%
\pgfsetroundjoin%
\definecolor{currentfill}{rgb}{0.963884,0.644842,0.486120}%
\pgfsetfillcolor{currentfill}%
\pgfsetlinewidth{0.311001pt}%
\definecolor{currentstroke}{rgb}{1.000000,1.000000,1.000000}%
\pgfsetstrokecolor{currentstroke}%
\pgfsetdash{}{0pt}%
\pgfpathmoveto{\pgfqpoint{4.066867in}{1.789128in}}%
\pgfpathcurveto{\pgfqpoint{4.074000in}{1.789128in}}{\pgfqpoint{4.080841in}{1.791962in}}{\pgfqpoint{4.085885in}{1.797005in}}%
\pgfpathcurveto{\pgfqpoint{4.090929in}{1.802049in}}{\pgfqpoint{4.093763in}{1.808891in}}{\pgfqpoint{4.093763in}{1.816023in}}%
\pgfpathcurveto{\pgfqpoint{4.093763in}{1.823156in}}{\pgfqpoint{4.090929in}{1.829998in}}{\pgfqpoint{4.085885in}{1.835042in}}%
\pgfpathcurveto{\pgfqpoint{4.080841in}{1.840085in}}{\pgfqpoint{4.074000in}{1.842919in}}{\pgfqpoint{4.066867in}{1.842919in}}%
\pgfpathcurveto{\pgfqpoint{4.059734in}{1.842919in}}{\pgfqpoint{4.052893in}{1.840085in}}{\pgfqpoint{4.047849in}{1.835042in}}%
\pgfpathcurveto{\pgfqpoint{4.042805in}{1.829998in}}{\pgfqpoint{4.039971in}{1.823156in}}{\pgfqpoint{4.039971in}{1.816023in}}%
\pgfpathcurveto{\pgfqpoint{4.039971in}{1.808891in}}{\pgfqpoint{4.042805in}{1.802049in}}{\pgfqpoint{4.047849in}{1.797005in}}%
\pgfpathcurveto{\pgfqpoint{4.052893in}{1.791962in}}{\pgfqpoint{4.059734in}{1.789128in}}{\pgfqpoint{4.066867in}{1.789128in}}%
\pgfpathclose%
\pgfusepath{stroke,fill}%
\end{pgfscope}%
\begin{pgfscope}%
\pgfpathrectangle{\pgfqpoint{2.867647in}{0.500000in}}{\pgfqpoint{1.764706in}{1.700000in}}%
\pgfusepath{clip}%
\pgfsetbuttcap%
\pgfsetroundjoin%
\definecolor{currentfill}{rgb}{0.977657,0.891500,0.822809}%
\pgfsetfillcolor{currentfill}%
\pgfsetlinewidth{0.311001pt}%
\definecolor{currentstroke}{rgb}{1.000000,1.000000,1.000000}%
\pgfsetstrokecolor{currentstroke}%
\pgfsetdash{}{0pt}%
\pgfpathmoveto{\pgfqpoint{4.227534in}{1.396939in}}%
\pgfpathcurveto{\pgfqpoint{4.234667in}{1.396939in}}{\pgfqpoint{4.241509in}{1.399773in}}{\pgfqpoint{4.246552in}{1.404816in}}%
\pgfpathcurveto{\pgfqpoint{4.251596in}{1.409860in}}{\pgfqpoint{4.254430in}{1.416702in}}{\pgfqpoint{4.254430in}{1.423834in}}%
\pgfpathcurveto{\pgfqpoint{4.254430in}{1.430967in}}{\pgfqpoint{4.251596in}{1.437809in}}{\pgfqpoint{4.246552in}{1.442853in}}%
\pgfpathcurveto{\pgfqpoint{4.241509in}{1.447896in}}{\pgfqpoint{4.234667in}{1.450730in}}{\pgfqpoint{4.227534in}{1.450730in}}%
\pgfpathcurveto{\pgfqpoint{4.220401in}{1.450730in}}{\pgfqpoint{4.213560in}{1.447896in}}{\pgfqpoint{4.208516in}{1.442853in}}%
\pgfpathcurveto{\pgfqpoint{4.203472in}{1.437809in}}{\pgfqpoint{4.200638in}{1.430967in}}{\pgfqpoint{4.200638in}{1.423834in}}%
\pgfpathcurveto{\pgfqpoint{4.200638in}{1.416702in}}{\pgfqpoint{4.203472in}{1.409860in}}{\pgfqpoint{4.208516in}{1.404816in}}%
\pgfpathcurveto{\pgfqpoint{4.213560in}{1.399773in}}{\pgfqpoint{4.220401in}{1.396939in}}{\pgfqpoint{4.227534in}{1.396939in}}%
\pgfpathclose%
\pgfusepath{stroke,fill}%
\end{pgfscope}%
\begin{pgfscope}%
\pgfpathrectangle{\pgfqpoint{2.867647in}{0.500000in}}{\pgfqpoint{1.764706in}{1.700000in}}%
\pgfusepath{clip}%
\pgfsetbuttcap%
\pgfsetroundjoin%
\definecolor{currentfill}{rgb}{0.965440,0.720101,0.576404}%
\pgfsetfillcolor{currentfill}%
\pgfsetlinewidth{0.311001pt}%
\definecolor{currentstroke}{rgb}{1.000000,1.000000,1.000000}%
\pgfsetstrokecolor{currentstroke}%
\pgfsetdash{}{0pt}%
\pgfpathmoveto{\pgfqpoint{4.054686in}{1.454945in}}%
\pgfpathcurveto{\pgfqpoint{4.061818in}{1.454945in}}{\pgfqpoint{4.068660in}{1.457779in}}{\pgfqpoint{4.073704in}{1.462823in}}%
\pgfpathcurveto{\pgfqpoint{4.078747in}{1.467866in}}{\pgfqpoint{4.081581in}{1.474708in}}{\pgfqpoint{4.081581in}{1.481841in}}%
\pgfpathcurveto{\pgfqpoint{4.081581in}{1.488974in}}{\pgfqpoint{4.078747in}{1.495815in}}{\pgfqpoint{4.073704in}{1.500859in}}%
\pgfpathcurveto{\pgfqpoint{4.068660in}{1.505903in}}{\pgfqpoint{4.061818in}{1.508736in}}{\pgfqpoint{4.054686in}{1.508736in}}%
\pgfpathcurveto{\pgfqpoint{4.047553in}{1.508736in}}{\pgfqpoint{4.040711in}{1.505903in}}{\pgfqpoint{4.035667in}{1.500859in}}%
\pgfpathcurveto{\pgfqpoint{4.030624in}{1.495815in}}{\pgfqpoint{4.027790in}{1.488974in}}{\pgfqpoint{4.027790in}{1.481841in}}%
\pgfpathcurveto{\pgfqpoint{4.027790in}{1.474708in}}{\pgfqpoint{4.030624in}{1.467866in}}{\pgfqpoint{4.035667in}{1.462823in}}%
\pgfpathcurveto{\pgfqpoint{4.040711in}{1.457779in}}{\pgfqpoint{4.047553in}{1.454945in}}{\pgfqpoint{4.054686in}{1.454945in}}%
\pgfpathclose%
\pgfusepath{stroke,fill}%
\end{pgfscope}%
\begin{pgfscope}%
\pgfpathrectangle{\pgfqpoint{2.867647in}{0.500000in}}{\pgfqpoint{1.764706in}{1.700000in}}%
\pgfusepath{clip}%
\pgfsetbuttcap%
\pgfsetroundjoin%
\definecolor{currentfill}{rgb}{0.979124,0.903132,0.839793}%
\pgfsetfillcolor{currentfill}%
\pgfsetlinewidth{0.311001pt}%
\definecolor{currentstroke}{rgb}{1.000000,1.000000,1.000000}%
\pgfsetstrokecolor{currentstroke}%
\pgfsetdash{}{0pt}%
\pgfpathmoveto{\pgfqpoint{4.141196in}{1.212924in}}%
\pgfpathcurveto{\pgfqpoint{4.148328in}{1.212924in}}{\pgfqpoint{4.155170in}{1.215758in}}{\pgfqpoint{4.160214in}{1.220802in}}%
\pgfpathcurveto{\pgfqpoint{4.165257in}{1.225845in}}{\pgfqpoint{4.168091in}{1.232687in}}{\pgfqpoint{4.168091in}{1.239820in}}%
\pgfpathcurveto{\pgfqpoint{4.168091in}{1.246953in}}{\pgfqpoint{4.165257in}{1.253794in}}{\pgfqpoint{4.160214in}{1.258838in}}%
\pgfpathcurveto{\pgfqpoint{4.155170in}{1.263882in}}{\pgfqpoint{4.148328in}{1.266716in}}{\pgfqpoint{4.141196in}{1.266716in}}%
\pgfpathcurveto{\pgfqpoint{4.134063in}{1.266716in}}{\pgfqpoint{4.127221in}{1.263882in}}{\pgfqpoint{4.122177in}{1.258838in}}%
\pgfpathcurveto{\pgfqpoint{4.117134in}{1.253794in}}{\pgfqpoint{4.114300in}{1.246953in}}{\pgfqpoint{4.114300in}{1.239820in}}%
\pgfpathcurveto{\pgfqpoint{4.114300in}{1.232687in}}{\pgfqpoint{4.117134in}{1.225845in}}{\pgfqpoint{4.122177in}{1.220802in}}%
\pgfpathcurveto{\pgfqpoint{4.127221in}{1.215758in}}{\pgfqpoint{4.134063in}{1.212924in}}{\pgfqpoint{4.141196in}{1.212924in}}%
\pgfpathclose%
\pgfusepath{stroke,fill}%
\end{pgfscope}%
\begin{pgfscope}%
\pgfpathrectangle{\pgfqpoint{2.867647in}{0.500000in}}{\pgfqpoint{1.764706in}{1.700000in}}%
\pgfusepath{clip}%
\pgfsetbuttcap%
\pgfsetroundjoin%
\definecolor{currentfill}{rgb}{0.972201,0.839051,0.745789}%
\pgfsetfillcolor{currentfill}%
\pgfsetlinewidth{0.311001pt}%
\definecolor{currentstroke}{rgb}{1.000000,1.000000,1.000000}%
\pgfsetstrokecolor{currentstroke}%
\pgfsetdash{}{0pt}%
\pgfpathmoveto{\pgfqpoint{4.254273in}{1.199614in}}%
\pgfpathcurveto{\pgfqpoint{4.261406in}{1.199614in}}{\pgfqpoint{4.268247in}{1.202448in}}{\pgfqpoint{4.273291in}{1.207492in}}%
\pgfpathcurveto{\pgfqpoint{4.278335in}{1.212535in}}{\pgfqpoint{4.281169in}{1.219377in}}{\pgfqpoint{4.281169in}{1.226510in}}%
\pgfpathcurveto{\pgfqpoint{4.281169in}{1.233643in}}{\pgfqpoint{4.278335in}{1.240484in}}{\pgfqpoint{4.273291in}{1.245528in}}%
\pgfpathcurveto{\pgfqpoint{4.268247in}{1.250572in}}{\pgfqpoint{4.261406in}{1.253406in}}{\pgfqpoint{4.254273in}{1.253406in}}%
\pgfpathcurveto{\pgfqpoint{4.247140in}{1.253406in}}{\pgfqpoint{4.240299in}{1.250572in}}{\pgfqpoint{4.235255in}{1.245528in}}%
\pgfpathcurveto{\pgfqpoint{4.230211in}{1.240484in}}{\pgfqpoint{4.227377in}{1.233643in}}{\pgfqpoint{4.227377in}{1.226510in}}%
\pgfpathcurveto{\pgfqpoint{4.227377in}{1.219377in}}{\pgfqpoint{4.230211in}{1.212535in}}{\pgfqpoint{4.235255in}{1.207492in}}%
\pgfpathcurveto{\pgfqpoint{4.240299in}{1.202448in}}{\pgfqpoint{4.247140in}{1.199614in}}{\pgfqpoint{4.254273in}{1.199614in}}%
\pgfpathclose%
\pgfusepath{stroke,fill}%
\end{pgfscope}%
\begin{pgfscope}%
\pgfpathrectangle{\pgfqpoint{2.867647in}{0.500000in}}{\pgfqpoint{1.764706in}{1.700000in}}%
\pgfusepath{clip}%
\pgfsetbuttcap%
\pgfsetroundjoin%
\definecolor{currentfill}{rgb}{0.964799,0.689101,0.537560}%
\pgfsetfillcolor{currentfill}%
\pgfsetlinewidth{0.311001pt}%
\definecolor{currentstroke}{rgb}{1.000000,1.000000,1.000000}%
\pgfsetstrokecolor{currentstroke}%
\pgfsetdash{}{0pt}%
\pgfpathmoveto{\pgfqpoint{4.007771in}{1.571396in}}%
\pgfpathcurveto{\pgfqpoint{4.014904in}{1.571396in}}{\pgfqpoint{4.021745in}{1.574230in}}{\pgfqpoint{4.026789in}{1.579273in}}%
\pgfpathcurveto{\pgfqpoint{4.031833in}{1.584317in}}{\pgfqpoint{4.034666in}{1.591159in}}{\pgfqpoint{4.034666in}{1.598291in}}%
\pgfpathcurveto{\pgfqpoint{4.034666in}{1.605424in}}{\pgfqpoint{4.031833in}{1.612266in}}{\pgfqpoint{4.026789in}{1.617310in}}%
\pgfpathcurveto{\pgfqpoint{4.021745in}{1.622353in}}{\pgfqpoint{4.014904in}{1.625187in}}{\pgfqpoint{4.007771in}{1.625187in}}%
\pgfpathcurveto{\pgfqpoint{4.000638in}{1.625187in}}{\pgfqpoint{3.993796in}{1.622353in}}{\pgfqpoint{3.988753in}{1.617310in}}%
\pgfpathcurveto{\pgfqpoint{3.983709in}{1.612266in}}{\pgfqpoint{3.980875in}{1.605424in}}{\pgfqpoint{3.980875in}{1.598291in}}%
\pgfpathcurveto{\pgfqpoint{3.980875in}{1.591159in}}{\pgfqpoint{3.983709in}{1.584317in}}{\pgfqpoint{3.988753in}{1.579273in}}%
\pgfpathcurveto{\pgfqpoint{3.993796in}{1.574230in}}{\pgfqpoint{4.000638in}{1.571396in}}{\pgfqpoint{4.007771in}{1.571396in}}%
\pgfpathclose%
\pgfusepath{stroke,fill}%
\end{pgfscope}%
\begin{pgfscope}%
\pgfpathrectangle{\pgfqpoint{2.867647in}{0.500000in}}{\pgfqpoint{1.764706in}{1.700000in}}%
\pgfusepath{clip}%
\pgfsetbuttcap%
\pgfsetroundjoin%
\definecolor{currentfill}{rgb}{0.973271,0.850724,0.762998}%
\pgfsetfillcolor{currentfill}%
\pgfsetlinewidth{0.311001pt}%
\definecolor{currentstroke}{rgb}{1.000000,1.000000,1.000000}%
\pgfsetstrokecolor{currentstroke}%
\pgfsetdash{}{0pt}%
\pgfpathmoveto{\pgfqpoint{4.123161in}{1.282893in}}%
\pgfpathcurveto{\pgfqpoint{4.130294in}{1.282893in}}{\pgfqpoint{4.137136in}{1.285727in}}{\pgfqpoint{4.142179in}{1.290770in}}%
\pgfpathcurveto{\pgfqpoint{4.147223in}{1.295814in}}{\pgfqpoint{4.150057in}{1.302655in}}{\pgfqpoint{4.150057in}{1.309788in}}%
\pgfpathcurveto{\pgfqpoint{4.150057in}{1.316921in}}{\pgfqpoint{4.147223in}{1.323763in}}{\pgfqpoint{4.142179in}{1.328806in}}%
\pgfpathcurveto{\pgfqpoint{4.137136in}{1.333850in}}{\pgfqpoint{4.130294in}{1.336684in}}{\pgfqpoint{4.123161in}{1.336684in}}%
\pgfpathcurveto{\pgfqpoint{4.116028in}{1.336684in}}{\pgfqpoint{4.109187in}{1.333850in}}{\pgfqpoint{4.104143in}{1.328806in}}%
\pgfpathcurveto{\pgfqpoint{4.099099in}{1.323763in}}{\pgfqpoint{4.096266in}{1.316921in}}{\pgfqpoint{4.096266in}{1.309788in}}%
\pgfpathcurveto{\pgfqpoint{4.096266in}{1.302655in}}{\pgfqpoint{4.099099in}{1.295814in}}{\pgfqpoint{4.104143in}{1.290770in}}%
\pgfpathcurveto{\pgfqpoint{4.109187in}{1.285727in}}{\pgfqpoint{4.116028in}{1.282893in}}{\pgfqpoint{4.123161in}{1.282893in}}%
\pgfpathclose%
\pgfusepath{stroke,fill}%
\end{pgfscope}%
\begin{pgfscope}%
\pgfpathrectangle{\pgfqpoint{2.867647in}{0.500000in}}{\pgfqpoint{1.764706in}{1.700000in}}%
\pgfusepath{clip}%
\pgfsetbuttcap%
\pgfsetroundjoin%
\definecolor{currentfill}{rgb}{0.979124,0.903132,0.839793}%
\pgfsetfillcolor{currentfill}%
\pgfsetlinewidth{0.311001pt}%
\definecolor{currentstroke}{rgb}{1.000000,1.000000,1.000000}%
\pgfsetstrokecolor{currentstroke}%
\pgfsetdash{}{0pt}%
\pgfpathmoveto{\pgfqpoint{4.224448in}{1.291320in}}%
\pgfpathcurveto{\pgfqpoint{4.231581in}{1.291320in}}{\pgfqpoint{4.238422in}{1.294154in}}{\pgfqpoint{4.243466in}{1.299198in}}%
\pgfpathcurveto{\pgfqpoint{4.248510in}{1.304241in}}{\pgfqpoint{4.251343in}{1.311083in}}{\pgfqpoint{4.251343in}{1.318216in}}%
\pgfpathcurveto{\pgfqpoint{4.251343in}{1.325349in}}{\pgfqpoint{4.248510in}{1.332190in}}{\pgfqpoint{4.243466in}{1.337234in}}%
\pgfpathcurveto{\pgfqpoint{4.238422in}{1.342278in}}{\pgfqpoint{4.231581in}{1.345112in}}{\pgfqpoint{4.224448in}{1.345112in}}%
\pgfpathcurveto{\pgfqpoint{4.217315in}{1.345112in}}{\pgfqpoint{4.210473in}{1.342278in}}{\pgfqpoint{4.205430in}{1.337234in}}%
\pgfpathcurveto{\pgfqpoint{4.200386in}{1.332190in}}{\pgfqpoint{4.197552in}{1.325349in}}{\pgfqpoint{4.197552in}{1.318216in}}%
\pgfpathcurveto{\pgfqpoint{4.197552in}{1.311083in}}{\pgfqpoint{4.200386in}{1.304241in}}{\pgfqpoint{4.205430in}{1.299198in}}%
\pgfpathcurveto{\pgfqpoint{4.210473in}{1.294154in}}{\pgfqpoint{4.217315in}{1.291320in}}{\pgfqpoint{4.224448in}{1.291320in}}%
\pgfpathclose%
\pgfusepath{stroke,fill}%
\end{pgfscope}%
\begin{pgfscope}%
\pgfpathrectangle{\pgfqpoint{2.867647in}{0.500000in}}{\pgfqpoint{1.764706in}{1.700000in}}%
\pgfusepath{clip}%
\pgfsetbuttcap%
\pgfsetroundjoin%
\definecolor{currentfill}{rgb}{0.973271,0.850724,0.762998}%
\pgfsetfillcolor{currentfill}%
\pgfsetlinewidth{0.311001pt}%
\definecolor{currentstroke}{rgb}{1.000000,1.000000,1.000000}%
\pgfsetstrokecolor{currentstroke}%
\pgfsetdash{}{0pt}%
\pgfpathmoveto{\pgfqpoint{4.086158in}{1.118711in}}%
\pgfpathcurveto{\pgfqpoint{4.093291in}{1.118711in}}{\pgfqpoint{4.100132in}{1.121545in}}{\pgfqpoint{4.105176in}{1.126589in}}%
\pgfpathcurveto{\pgfqpoint{4.110220in}{1.131633in}}{\pgfqpoint{4.113054in}{1.138474in}}{\pgfqpoint{4.113054in}{1.145607in}}%
\pgfpathcurveto{\pgfqpoint{4.113054in}{1.152740in}}{\pgfqpoint{4.110220in}{1.159582in}}{\pgfqpoint{4.105176in}{1.164625in}}%
\pgfpathcurveto{\pgfqpoint{4.100132in}{1.169669in}}{\pgfqpoint{4.093291in}{1.172503in}}{\pgfqpoint{4.086158in}{1.172503in}}%
\pgfpathcurveto{\pgfqpoint{4.079025in}{1.172503in}}{\pgfqpoint{4.072183in}{1.169669in}}{\pgfqpoint{4.067140in}{1.164625in}}%
\pgfpathcurveto{\pgfqpoint{4.062096in}{1.159582in}}{\pgfqpoint{4.059262in}{1.152740in}}{\pgfqpoint{4.059262in}{1.145607in}}%
\pgfpathcurveto{\pgfqpoint{4.059262in}{1.138474in}}{\pgfqpoint{4.062096in}{1.131633in}}{\pgfqpoint{4.067140in}{1.126589in}}%
\pgfpathcurveto{\pgfqpoint{4.072183in}{1.121545in}}{\pgfqpoint{4.079025in}{1.118711in}}{\pgfqpoint{4.086158in}{1.118711in}}%
\pgfpathclose%
\pgfusepath{stroke,fill}%
\end{pgfscope}%
\begin{pgfscope}%
\pgfpathrectangle{\pgfqpoint{2.867647in}{0.500000in}}{\pgfqpoint{1.764706in}{1.700000in}}%
\pgfusepath{clip}%
\pgfsetbuttcap%
\pgfsetroundjoin%
\definecolor{currentfill}{rgb}{0.964032,0.651225,0.493258}%
\pgfsetfillcolor{currentfill}%
\pgfsetlinewidth{0.311001pt}%
\definecolor{currentstroke}{rgb}{1.000000,1.000000,1.000000}%
\pgfsetstrokecolor{currentstroke}%
\pgfsetdash{}{0pt}%
\pgfpathmoveto{\pgfqpoint{4.043746in}{1.788851in}}%
\pgfpathcurveto{\pgfqpoint{4.050879in}{1.788851in}}{\pgfqpoint{4.057721in}{1.791685in}}{\pgfqpoint{4.062764in}{1.796729in}}%
\pgfpathcurveto{\pgfqpoint{4.067808in}{1.801772in}}{\pgfqpoint{4.070642in}{1.808614in}}{\pgfqpoint{4.070642in}{1.815747in}}%
\pgfpathcurveto{\pgfqpoint{4.070642in}{1.822880in}}{\pgfqpoint{4.067808in}{1.829721in}}{\pgfqpoint{4.062764in}{1.834765in}}%
\pgfpathcurveto{\pgfqpoint{4.057721in}{1.839809in}}{\pgfqpoint{4.050879in}{1.842643in}}{\pgfqpoint{4.043746in}{1.842643in}}%
\pgfpathcurveto{\pgfqpoint{4.036613in}{1.842643in}}{\pgfqpoint{4.029772in}{1.839809in}}{\pgfqpoint{4.024728in}{1.834765in}}%
\pgfpathcurveto{\pgfqpoint{4.019684in}{1.829721in}}{\pgfqpoint{4.016851in}{1.822880in}}{\pgfqpoint{4.016851in}{1.815747in}}%
\pgfpathcurveto{\pgfqpoint{4.016851in}{1.808614in}}{\pgfqpoint{4.019684in}{1.801772in}}{\pgfqpoint{4.024728in}{1.796729in}}%
\pgfpathcurveto{\pgfqpoint{4.029772in}{1.791685in}}{\pgfqpoint{4.036613in}{1.788851in}}{\pgfqpoint{4.043746in}{1.788851in}}%
\pgfpathclose%
\pgfusepath{stroke,fill}%
\end{pgfscope}%
\begin{pgfscope}%
\pgfpathrectangle{\pgfqpoint{2.867647in}{0.500000in}}{\pgfqpoint{1.764706in}{1.700000in}}%
\pgfusepath{clip}%
\pgfsetbuttcap%
\pgfsetroundjoin%
\definecolor{currentfill}{rgb}{0.978376,0.897317,0.831308}%
\pgfsetfillcolor{currentfill}%
\pgfsetlinewidth{0.311001pt}%
\definecolor{currentstroke}{rgb}{1.000000,1.000000,1.000000}%
\pgfsetstrokecolor{currentstroke}%
\pgfsetdash{}{0pt}%
\pgfpathmoveto{\pgfqpoint{4.180737in}{1.546126in}}%
\pgfpathcurveto{\pgfqpoint{4.187869in}{1.546126in}}{\pgfqpoint{4.194711in}{1.548959in}}{\pgfqpoint{4.199755in}{1.554003in}}%
\pgfpathcurveto{\pgfqpoint{4.204798in}{1.559047in}}{\pgfqpoint{4.207632in}{1.565888in}}{\pgfqpoint{4.207632in}{1.573021in}}%
\pgfpathcurveto{\pgfqpoint{4.207632in}{1.580154in}}{\pgfqpoint{4.204798in}{1.586996in}}{\pgfqpoint{4.199755in}{1.592039in}}%
\pgfpathcurveto{\pgfqpoint{4.194711in}{1.597083in}}{\pgfqpoint{4.187869in}{1.599917in}}{\pgfqpoint{4.180737in}{1.599917in}}%
\pgfpathcurveto{\pgfqpoint{4.173604in}{1.599917in}}{\pgfqpoint{4.166762in}{1.597083in}}{\pgfqpoint{4.161718in}{1.592039in}}%
\pgfpathcurveto{\pgfqpoint{4.156675in}{1.586996in}}{\pgfqpoint{4.153841in}{1.580154in}}{\pgfqpoint{4.153841in}{1.573021in}}%
\pgfpathcurveto{\pgfqpoint{4.153841in}{1.565888in}}{\pgfqpoint{4.156675in}{1.559047in}}{\pgfqpoint{4.161718in}{1.554003in}}%
\pgfpathcurveto{\pgfqpoint{4.166762in}{1.548959in}}{\pgfqpoint{4.173604in}{1.546126in}}{\pgfqpoint{4.180737in}{1.546126in}}%
\pgfpathclose%
\pgfusepath{stroke,fill}%
\end{pgfscope}%
\begin{pgfscope}%
\pgfpathrectangle{\pgfqpoint{2.867647in}{0.500000in}}{\pgfqpoint{1.764706in}{1.700000in}}%
\pgfusepath{clip}%
\pgfsetbuttcap%
\pgfsetroundjoin%
\definecolor{currentfill}{rgb}{0.960778,0.559972,0.399412}%
\pgfsetfillcolor{currentfill}%
\pgfsetlinewidth{0.311001pt}%
\definecolor{currentstroke}{rgb}{1.000000,1.000000,1.000000}%
\pgfsetstrokecolor{currentstroke}%
\pgfsetdash{}{0pt}%
\pgfpathmoveto{\pgfqpoint{3.973358in}{1.577723in}}%
\pgfpathcurveto{\pgfqpoint{3.980491in}{1.577723in}}{\pgfqpoint{3.987332in}{1.580557in}}{\pgfqpoint{3.992376in}{1.585601in}}%
\pgfpathcurveto{\pgfqpoint{3.997420in}{1.590644in}}{\pgfqpoint{4.000254in}{1.597486in}}{\pgfqpoint{4.000254in}{1.604619in}}%
\pgfpathcurveto{\pgfqpoint{4.000254in}{1.611752in}}{\pgfqpoint{3.997420in}{1.618593in}}{\pgfqpoint{3.992376in}{1.623637in}}%
\pgfpathcurveto{\pgfqpoint{3.987332in}{1.628681in}}{\pgfqpoint{3.980491in}{1.631515in}}{\pgfqpoint{3.973358in}{1.631515in}}%
\pgfpathcurveto{\pgfqpoint{3.966225in}{1.631515in}}{\pgfqpoint{3.959384in}{1.628681in}}{\pgfqpoint{3.954340in}{1.623637in}}%
\pgfpathcurveto{\pgfqpoint{3.949296in}{1.618593in}}{\pgfqpoint{3.946462in}{1.611752in}}{\pgfqpoint{3.946462in}{1.604619in}}%
\pgfpathcurveto{\pgfqpoint{3.946462in}{1.597486in}}{\pgfqpoint{3.949296in}{1.590644in}}{\pgfqpoint{3.954340in}{1.585601in}}%
\pgfpathcurveto{\pgfqpoint{3.959384in}{1.580557in}}{\pgfqpoint{3.966225in}{1.577723in}}{\pgfqpoint{3.973358in}{1.577723in}}%
\pgfpathclose%
\pgfusepath{stroke,fill}%
\end{pgfscope}%
\begin{pgfscope}%
\pgfpathrectangle{\pgfqpoint{2.867647in}{0.500000in}}{\pgfqpoint{1.764706in}{1.700000in}}%
\pgfusepath{clip}%
\pgfsetbuttcap%
\pgfsetroundjoin%
\definecolor{currentfill}{rgb}{0.975644,0.874038,0.797253}%
\pgfsetfillcolor{currentfill}%
\pgfsetlinewidth{0.311001pt}%
\definecolor{currentstroke}{rgb}{1.000000,1.000000,1.000000}%
\pgfsetstrokecolor{currentstroke}%
\pgfsetdash{}{0pt}%
\pgfpathmoveto{\pgfqpoint{4.085648in}{1.566119in}}%
\pgfpathcurveto{\pgfqpoint{4.092780in}{1.566119in}}{\pgfqpoint{4.099622in}{1.568953in}}{\pgfqpoint{4.104666in}{1.573997in}}%
\pgfpathcurveto{\pgfqpoint{4.109709in}{1.579041in}}{\pgfqpoint{4.112543in}{1.585882in}}{\pgfqpoint{4.112543in}{1.593015in}}%
\pgfpathcurveto{\pgfqpoint{4.112543in}{1.600148in}}{\pgfqpoint{4.109709in}{1.606990in}}{\pgfqpoint{4.104666in}{1.612033in}}%
\pgfpathcurveto{\pgfqpoint{4.099622in}{1.617077in}}{\pgfqpoint{4.092780in}{1.619911in}}{\pgfqpoint{4.085648in}{1.619911in}}%
\pgfpathcurveto{\pgfqpoint{4.078515in}{1.619911in}}{\pgfqpoint{4.071673in}{1.617077in}}{\pgfqpoint{4.066630in}{1.612033in}}%
\pgfpathcurveto{\pgfqpoint{4.061586in}{1.606990in}}{\pgfqpoint{4.058752in}{1.600148in}}{\pgfqpoint{4.058752in}{1.593015in}}%
\pgfpathcurveto{\pgfqpoint{4.058752in}{1.585882in}}{\pgfqpoint{4.061586in}{1.579041in}}{\pgfqpoint{4.066630in}{1.573997in}}%
\pgfpathcurveto{\pgfqpoint{4.071673in}{1.568953in}}{\pgfqpoint{4.078515in}{1.566119in}}{\pgfqpoint{4.085648in}{1.566119in}}%
\pgfpathclose%
\pgfusepath{stroke,fill}%
\end{pgfscope}%
\begin{pgfscope}%
\pgfpathrectangle{\pgfqpoint{2.867647in}{0.500000in}}{\pgfqpoint{1.764706in}{1.700000in}}%
\pgfusepath{clip}%
\pgfsetbuttcap%
\pgfsetroundjoin%
\definecolor{currentfill}{rgb}{0.963728,0.638439,0.479050}%
\pgfsetfillcolor{currentfill}%
\pgfsetlinewidth{0.311001pt}%
\definecolor{currentstroke}{rgb}{1.000000,1.000000,1.000000}%
\pgfsetstrokecolor{currentstroke}%
\pgfsetdash{}{0pt}%
\pgfpathmoveto{\pgfqpoint{3.962808in}{0.929403in}}%
\pgfpathcurveto{\pgfqpoint{3.969941in}{0.929403in}}{\pgfqpoint{3.976782in}{0.932237in}}{\pgfqpoint{3.981826in}{0.937280in}}%
\pgfpathcurveto{\pgfqpoint{3.986870in}{0.942324in}}{\pgfqpoint{3.989703in}{0.949166in}}{\pgfqpoint{3.989703in}{0.956299in}}%
\pgfpathcurveto{\pgfqpoint{3.989703in}{0.963431in}}{\pgfqpoint{3.986870in}{0.970273in}}{\pgfqpoint{3.981826in}{0.975317in}}%
\pgfpathcurveto{\pgfqpoint{3.976782in}{0.980360in}}{\pgfqpoint{3.969941in}{0.983194in}}{\pgfqpoint{3.962808in}{0.983194in}}%
\pgfpathcurveto{\pgfqpoint{3.955675in}{0.983194in}}{\pgfqpoint{3.948833in}{0.980360in}}{\pgfqpoint{3.943790in}{0.975317in}}%
\pgfpathcurveto{\pgfqpoint{3.938746in}{0.970273in}}{\pgfqpoint{3.935912in}{0.963431in}}{\pgfqpoint{3.935912in}{0.956299in}}%
\pgfpathcurveto{\pgfqpoint{3.935912in}{0.949166in}}{\pgfqpoint{3.938746in}{0.942324in}}{\pgfqpoint{3.943790in}{0.937280in}}%
\pgfpathcurveto{\pgfqpoint{3.948833in}{0.932237in}}{\pgfqpoint{3.955675in}{0.929403in}}{\pgfqpoint{3.962808in}{0.929403in}}%
\pgfpathclose%
\pgfusepath{stroke,fill}%
\end{pgfscope}%
\begin{pgfscope}%
\pgfpathrectangle{\pgfqpoint{2.867647in}{0.500000in}}{\pgfqpoint{1.764706in}{1.700000in}}%
\pgfusepath{clip}%
\pgfsetbuttcap%
\pgfsetroundjoin%
\definecolor{currentfill}{rgb}{0.978376,0.897317,0.831308}%
\pgfsetfillcolor{currentfill}%
\pgfsetlinewidth{0.311001pt}%
\definecolor{currentstroke}{rgb}{1.000000,1.000000,1.000000}%
\pgfsetstrokecolor{currentstroke}%
\pgfsetdash{}{0pt}%
\pgfpathmoveto{\pgfqpoint{4.213788in}{1.446696in}}%
\pgfpathcurveto{\pgfqpoint{4.220920in}{1.446696in}}{\pgfqpoint{4.227762in}{1.449530in}}{\pgfqpoint{4.232806in}{1.454574in}}%
\pgfpathcurveto{\pgfqpoint{4.237849in}{1.459617in}}{\pgfqpoint{4.240683in}{1.466459in}}{\pgfqpoint{4.240683in}{1.473592in}}%
\pgfpathcurveto{\pgfqpoint{4.240683in}{1.480725in}}{\pgfqpoint{4.237849in}{1.487566in}}{\pgfqpoint{4.232806in}{1.492610in}}%
\pgfpathcurveto{\pgfqpoint{4.227762in}{1.497654in}}{\pgfqpoint{4.220920in}{1.500487in}}{\pgfqpoint{4.213788in}{1.500487in}}%
\pgfpathcurveto{\pgfqpoint{4.206655in}{1.500487in}}{\pgfqpoint{4.199813in}{1.497654in}}{\pgfqpoint{4.194769in}{1.492610in}}%
\pgfpathcurveto{\pgfqpoint{4.189726in}{1.487566in}}{\pgfqpoint{4.186892in}{1.480725in}}{\pgfqpoint{4.186892in}{1.473592in}}%
\pgfpathcurveto{\pgfqpoint{4.186892in}{1.466459in}}{\pgfqpoint{4.189726in}{1.459617in}}{\pgfqpoint{4.194769in}{1.454574in}}%
\pgfpathcurveto{\pgfqpoint{4.199813in}{1.449530in}}{\pgfqpoint{4.206655in}{1.446696in}}{\pgfqpoint{4.213788in}{1.446696in}}%
\pgfpathclose%
\pgfusepath{stroke,fill}%
\end{pgfscope}%
\begin{pgfscope}%
\pgfpathrectangle{\pgfqpoint{2.867647in}{0.500000in}}{\pgfqpoint{1.764706in}{1.700000in}}%
\pgfusepath{clip}%
\pgfsetbuttcap%
\pgfsetroundjoin%
\definecolor{currentfill}{rgb}{0.960421,0.553286,0.393191}%
\pgfsetfillcolor{currentfill}%
\pgfsetlinewidth{0.311001pt}%
\definecolor{currentstroke}{rgb}{1.000000,1.000000,1.000000}%
\pgfsetstrokecolor{currentstroke}%
\pgfsetdash{}{0pt}%
\pgfpathmoveto{\pgfqpoint{4.055650in}{1.819714in}}%
\pgfpathcurveto{\pgfqpoint{4.062782in}{1.819714in}}{\pgfqpoint{4.069624in}{1.822548in}}{\pgfqpoint{4.074668in}{1.827592in}}%
\pgfpathcurveto{\pgfqpoint{4.079711in}{1.832635in}}{\pgfqpoint{4.082545in}{1.839477in}}{\pgfqpoint{4.082545in}{1.846610in}}%
\pgfpathcurveto{\pgfqpoint{4.082545in}{1.853743in}}{\pgfqpoint{4.079711in}{1.860584in}}{\pgfqpoint{4.074668in}{1.865628in}}%
\pgfpathcurveto{\pgfqpoint{4.069624in}{1.870672in}}{\pgfqpoint{4.062782in}{1.873505in}}{\pgfqpoint{4.055650in}{1.873505in}}%
\pgfpathcurveto{\pgfqpoint{4.048517in}{1.873505in}}{\pgfqpoint{4.041675in}{1.870672in}}{\pgfqpoint{4.036631in}{1.865628in}}%
\pgfpathcurveto{\pgfqpoint{4.031588in}{1.860584in}}{\pgfqpoint{4.028754in}{1.853743in}}{\pgfqpoint{4.028754in}{1.846610in}}%
\pgfpathcurveto{\pgfqpoint{4.028754in}{1.839477in}}{\pgfqpoint{4.031588in}{1.832635in}}{\pgfqpoint{4.036631in}{1.827592in}}%
\pgfpathcurveto{\pgfqpoint{4.041675in}{1.822548in}}{\pgfqpoint{4.048517in}{1.819714in}}{\pgfqpoint{4.055650in}{1.819714in}}%
\pgfpathclose%
\pgfusepath{stroke,fill}%
\end{pgfscope}%
\begin{pgfscope}%
\pgfpathrectangle{\pgfqpoint{2.867647in}{0.500000in}}{\pgfqpoint{1.764706in}{1.700000in}}%
\pgfusepath{clip}%
\pgfsetbuttcap%
\pgfsetroundjoin%
\definecolor{currentfill}{rgb}{0.960043,0.546576,0.387029}%
\pgfsetfillcolor{currentfill}%
\pgfsetlinewidth{0.311001pt}%
\definecolor{currentstroke}{rgb}{1.000000,1.000000,1.000000}%
\pgfsetstrokecolor{currentstroke}%
\pgfsetdash{}{0pt}%
\pgfpathmoveto{\pgfqpoint{4.217202in}{0.932862in}}%
\pgfpathcurveto{\pgfqpoint{4.224335in}{0.932862in}}{\pgfqpoint{4.231176in}{0.935696in}}{\pgfqpoint{4.236220in}{0.940740in}}%
\pgfpathcurveto{\pgfqpoint{4.241263in}{0.945783in}}{\pgfqpoint{4.244097in}{0.952625in}}{\pgfqpoint{4.244097in}{0.959758in}}%
\pgfpathcurveto{\pgfqpoint{4.244097in}{0.966890in}}{\pgfqpoint{4.241263in}{0.973732in}}{\pgfqpoint{4.236220in}{0.978776in}}%
\pgfpathcurveto{\pgfqpoint{4.231176in}{0.983819in}}{\pgfqpoint{4.224335in}{0.986653in}}{\pgfqpoint{4.217202in}{0.986653in}}%
\pgfpathcurveto{\pgfqpoint{4.210069in}{0.986653in}}{\pgfqpoint{4.203227in}{0.983819in}}{\pgfqpoint{4.198184in}{0.978776in}}%
\pgfpathcurveto{\pgfqpoint{4.193140in}{0.973732in}}{\pgfqpoint{4.190306in}{0.966890in}}{\pgfqpoint{4.190306in}{0.959758in}}%
\pgfpathcurveto{\pgfqpoint{4.190306in}{0.952625in}}{\pgfqpoint{4.193140in}{0.945783in}}{\pgfqpoint{4.198184in}{0.940740in}}%
\pgfpathcurveto{\pgfqpoint{4.203227in}{0.935696in}}{\pgfqpoint{4.210069in}{0.932862in}}{\pgfqpoint{4.217202in}{0.932862in}}%
\pgfpathclose%
\pgfusepath{stroke,fill}%
\end{pgfscope}%
\begin{pgfscope}%
\pgfpathrectangle{\pgfqpoint{2.867647in}{0.500000in}}{\pgfqpoint{1.764706in}{1.700000in}}%
\pgfusepath{clip}%
\pgfsetbuttcap%
\pgfsetroundjoin%
\definecolor{currentfill}{rgb}{0.980678,0.914765,0.856766}%
\pgfsetfillcolor{currentfill}%
\pgfsetlinewidth{0.311001pt}%
\definecolor{currentstroke}{rgb}{1.000000,1.000000,1.000000}%
\pgfsetstrokecolor{currentstroke}%
\pgfsetdash{}{0pt}%
\pgfpathmoveto{\pgfqpoint{4.155285in}{1.197697in}}%
\pgfpathcurveto{\pgfqpoint{4.162418in}{1.197697in}}{\pgfqpoint{4.169260in}{1.200531in}}{\pgfqpoint{4.174303in}{1.205575in}}%
\pgfpathcurveto{\pgfqpoint{4.179347in}{1.210619in}}{\pgfqpoint{4.182181in}{1.217460in}}{\pgfqpoint{4.182181in}{1.224593in}}%
\pgfpathcurveto{\pgfqpoint{4.182181in}{1.231726in}}{\pgfqpoint{4.179347in}{1.238568in}}{\pgfqpoint{4.174303in}{1.243611in}}%
\pgfpathcurveto{\pgfqpoint{4.169260in}{1.248655in}}{\pgfqpoint{4.162418in}{1.251489in}}{\pgfqpoint{4.155285in}{1.251489in}}%
\pgfpathcurveto{\pgfqpoint{4.148152in}{1.251489in}}{\pgfqpoint{4.141311in}{1.248655in}}{\pgfqpoint{4.136267in}{1.243611in}}%
\pgfpathcurveto{\pgfqpoint{4.131223in}{1.238568in}}{\pgfqpoint{4.128389in}{1.231726in}}{\pgfqpoint{4.128389in}{1.224593in}}%
\pgfpathcurveto{\pgfqpoint{4.128389in}{1.217460in}}{\pgfqpoint{4.131223in}{1.210619in}}{\pgfqpoint{4.136267in}{1.205575in}}%
\pgfpathcurveto{\pgfqpoint{4.141311in}{1.200531in}}{\pgfqpoint{4.148152in}{1.197697in}}{\pgfqpoint{4.155285in}{1.197697in}}%
\pgfpathclose%
\pgfusepath{stroke,fill}%
\end{pgfscope}%
\begin{pgfscope}%
\pgfpathrectangle{\pgfqpoint{2.867647in}{0.500000in}}{\pgfqpoint{1.764706in}{1.700000in}}%
\pgfusepath{clip}%
\pgfsetbuttcap%
\pgfsetroundjoin%
\definecolor{currentfill}{rgb}{0.976287,0.879862,0.805788}%
\pgfsetfillcolor{currentfill}%
\pgfsetlinewidth{0.311001pt}%
\definecolor{currentstroke}{rgb}{1.000000,1.000000,1.000000}%
\pgfsetstrokecolor{currentstroke}%
\pgfsetdash{}{0pt}%
\pgfpathmoveto{\pgfqpoint{4.157356in}{1.601164in}}%
\pgfpathcurveto{\pgfqpoint{4.164488in}{1.601164in}}{\pgfqpoint{4.171330in}{1.603998in}}{\pgfqpoint{4.176374in}{1.609042in}}%
\pgfpathcurveto{\pgfqpoint{4.181417in}{1.614086in}}{\pgfqpoint{4.184251in}{1.620927in}}{\pgfqpoint{4.184251in}{1.628060in}}%
\pgfpathcurveto{\pgfqpoint{4.184251in}{1.635193in}}{\pgfqpoint{4.181417in}{1.642034in}}{\pgfqpoint{4.176374in}{1.647078in}}%
\pgfpathcurveto{\pgfqpoint{4.171330in}{1.652122in}}{\pgfqpoint{4.164488in}{1.654956in}}{\pgfqpoint{4.157356in}{1.654956in}}%
\pgfpathcurveto{\pgfqpoint{4.150223in}{1.654956in}}{\pgfqpoint{4.143381in}{1.652122in}}{\pgfqpoint{4.138337in}{1.647078in}}%
\pgfpathcurveto{\pgfqpoint{4.133294in}{1.642034in}}{\pgfqpoint{4.130460in}{1.635193in}}{\pgfqpoint{4.130460in}{1.628060in}}%
\pgfpathcurveto{\pgfqpoint{4.130460in}{1.620927in}}{\pgfqpoint{4.133294in}{1.614086in}}{\pgfqpoint{4.138337in}{1.609042in}}%
\pgfpathcurveto{\pgfqpoint{4.143381in}{1.603998in}}{\pgfqpoint{4.150223in}{1.601164in}}{\pgfqpoint{4.157356in}{1.601164in}}%
\pgfpathclose%
\pgfusepath{stroke,fill}%
\end{pgfscope}%
\begin{pgfscope}%
\pgfpathrectangle{\pgfqpoint{2.867647in}{0.500000in}}{\pgfqpoint{1.764706in}{1.700000in}}%
\pgfusepath{clip}%
\pgfsetbuttcap%
\pgfsetroundjoin%
\definecolor{currentfill}{rgb}{0.979124,0.903132,0.839793}%
\pgfsetfillcolor{currentfill}%
\pgfsetlinewidth{0.311001pt}%
\definecolor{currentstroke}{rgb}{1.000000,1.000000,1.000000}%
\pgfsetstrokecolor{currentstroke}%
\pgfsetdash{}{0pt}%
\pgfpathmoveto{\pgfqpoint{4.136040in}{1.182728in}}%
\pgfpathcurveto{\pgfqpoint{4.143173in}{1.182728in}}{\pgfqpoint{4.150014in}{1.185562in}}{\pgfqpoint{4.155058in}{1.190606in}}%
\pgfpathcurveto{\pgfqpoint{4.160102in}{1.195649in}}{\pgfqpoint{4.162936in}{1.202491in}}{\pgfqpoint{4.162936in}{1.209624in}}%
\pgfpathcurveto{\pgfqpoint{4.162936in}{1.216757in}}{\pgfqpoint{4.160102in}{1.223598in}}{\pgfqpoint{4.155058in}{1.228642in}}%
\pgfpathcurveto{\pgfqpoint{4.150014in}{1.233686in}}{\pgfqpoint{4.143173in}{1.236519in}}{\pgfqpoint{4.136040in}{1.236519in}}%
\pgfpathcurveto{\pgfqpoint{4.128907in}{1.236519in}}{\pgfqpoint{4.122066in}{1.233686in}}{\pgfqpoint{4.117022in}{1.228642in}}%
\pgfpathcurveto{\pgfqpoint{4.111978in}{1.223598in}}{\pgfqpoint{4.109144in}{1.216757in}}{\pgfqpoint{4.109144in}{1.209624in}}%
\pgfpathcurveto{\pgfqpoint{4.109144in}{1.202491in}}{\pgfqpoint{4.111978in}{1.195649in}}{\pgfqpoint{4.117022in}{1.190606in}}%
\pgfpathcurveto{\pgfqpoint{4.122066in}{1.185562in}}{\pgfqpoint{4.128907in}{1.182728in}}{\pgfqpoint{4.136040in}{1.182728in}}%
\pgfpathclose%
\pgfusepath{stroke,fill}%
\end{pgfscope}%
\begin{pgfscope}%
\pgfpathrectangle{\pgfqpoint{2.867647in}{0.500000in}}{\pgfqpoint{1.764706in}{1.700000in}}%
\pgfusepath{clip}%
\pgfsetbuttcap%
\pgfsetroundjoin%
\definecolor{currentfill}{rgb}{0.980678,0.914765,0.856766}%
\pgfsetfillcolor{currentfill}%
\pgfsetlinewidth{0.311001pt}%
\definecolor{currentstroke}{rgb}{1.000000,1.000000,1.000000}%
\pgfsetstrokecolor{currentstroke}%
\pgfsetdash{}{0pt}%
\pgfpathmoveto{\pgfqpoint{4.171343in}{1.410195in}}%
\pgfpathcurveto{\pgfqpoint{4.178476in}{1.410195in}}{\pgfqpoint{4.185318in}{1.413028in}}{\pgfqpoint{4.190361in}{1.418072in}}%
\pgfpathcurveto{\pgfqpoint{4.195405in}{1.423116in}}{\pgfqpoint{4.198239in}{1.429957in}}{\pgfqpoint{4.198239in}{1.437090in}}%
\pgfpathcurveto{\pgfqpoint{4.198239in}{1.444223in}}{\pgfqpoint{4.195405in}{1.451065in}}{\pgfqpoint{4.190361in}{1.456108in}}%
\pgfpathcurveto{\pgfqpoint{4.185318in}{1.461152in}}{\pgfqpoint{4.178476in}{1.463986in}}{\pgfqpoint{4.171343in}{1.463986in}}%
\pgfpathcurveto{\pgfqpoint{4.164210in}{1.463986in}}{\pgfqpoint{4.157369in}{1.461152in}}{\pgfqpoint{4.152325in}{1.456108in}}%
\pgfpathcurveto{\pgfqpoint{4.147281in}{1.451065in}}{\pgfqpoint{4.144448in}{1.444223in}}{\pgfqpoint{4.144448in}{1.437090in}}%
\pgfpathcurveto{\pgfqpoint{4.144448in}{1.429957in}}{\pgfqpoint{4.147281in}{1.423116in}}{\pgfqpoint{4.152325in}{1.418072in}}%
\pgfpathcurveto{\pgfqpoint{4.157369in}{1.413028in}}{\pgfqpoint{4.164210in}{1.410195in}}{\pgfqpoint{4.171343in}{1.410195in}}%
\pgfpathclose%
\pgfusepath{stroke,fill}%
\end{pgfscope}%
\begin{pgfscope}%
\pgfpathrectangle{\pgfqpoint{2.867647in}{0.500000in}}{\pgfqpoint{1.764706in}{1.700000in}}%
\pgfusepath{clip}%
\pgfsetbuttcap%
\pgfsetroundjoin%
\definecolor{currentfill}{rgb}{0.950017,0.427714,0.292447}%
\pgfsetfillcolor{currentfill}%
\pgfsetlinewidth{0.311001pt}%
\definecolor{currentstroke}{rgb}{1.000000,1.000000,1.000000}%
\pgfsetstrokecolor{currentstroke}%
\pgfsetdash{}{0pt}%
\pgfpathmoveto{\pgfqpoint{3.856480in}{1.693386in}}%
\pgfpathcurveto{\pgfqpoint{3.863613in}{1.693386in}}{\pgfqpoint{3.870455in}{1.696220in}}{\pgfqpoint{3.875499in}{1.701264in}}%
\pgfpathcurveto{\pgfqpoint{3.880542in}{1.706307in}}{\pgfqpoint{3.883376in}{1.713149in}}{\pgfqpoint{3.883376in}{1.720282in}}%
\pgfpathcurveto{\pgfqpoint{3.883376in}{1.727415in}}{\pgfqpoint{3.880542in}{1.734256in}}{\pgfqpoint{3.875499in}{1.739300in}}%
\pgfpathcurveto{\pgfqpoint{3.870455in}{1.744344in}}{\pgfqpoint{3.863613in}{1.747178in}}{\pgfqpoint{3.856480in}{1.747178in}}%
\pgfpathcurveto{\pgfqpoint{3.849348in}{1.747178in}}{\pgfqpoint{3.842506in}{1.744344in}}{\pgfqpoint{3.837462in}{1.739300in}}%
\pgfpathcurveto{\pgfqpoint{3.832419in}{1.734256in}}{\pgfqpoint{3.829585in}{1.727415in}}{\pgfqpoint{3.829585in}{1.720282in}}%
\pgfpathcurveto{\pgfqpoint{3.829585in}{1.713149in}}{\pgfqpoint{3.832419in}{1.706307in}}{\pgfqpoint{3.837462in}{1.701264in}}%
\pgfpathcurveto{\pgfqpoint{3.842506in}{1.696220in}}{\pgfqpoint{3.849348in}{1.693386in}}{\pgfqpoint{3.856480in}{1.693386in}}%
\pgfpathclose%
\pgfusepath{stroke,fill}%
\end{pgfscope}%
\begin{pgfscope}%
\pgfpathrectangle{\pgfqpoint{2.867647in}{0.500000in}}{\pgfqpoint{1.764706in}{1.700000in}}%
\pgfusepath{clip}%
\pgfsetbuttcap%
\pgfsetroundjoin%
\definecolor{currentfill}{rgb}{0.965928,0.738443,0.600540}%
\pgfsetfillcolor{currentfill}%
\pgfsetlinewidth{0.311001pt}%
\definecolor{currentstroke}{rgb}{1.000000,1.000000,1.000000}%
\pgfsetstrokecolor{currentstroke}%
\pgfsetdash{}{0pt}%
\pgfpathmoveto{\pgfqpoint{4.050945in}{1.481072in}}%
\pgfpathcurveto{\pgfqpoint{4.058078in}{1.481072in}}{\pgfqpoint{4.064919in}{1.483906in}}{\pgfqpoint{4.069963in}{1.488950in}}%
\pgfpathcurveto{\pgfqpoint{4.075007in}{1.493994in}}{\pgfqpoint{4.077841in}{1.500835in}}{\pgfqpoint{4.077841in}{1.507968in}}%
\pgfpathcurveto{\pgfqpoint{4.077841in}{1.515101in}}{\pgfqpoint{4.075007in}{1.521943in}}{\pgfqpoint{4.069963in}{1.526986in}}%
\pgfpathcurveto{\pgfqpoint{4.064919in}{1.532030in}}{\pgfqpoint{4.058078in}{1.534864in}}{\pgfqpoint{4.050945in}{1.534864in}}%
\pgfpathcurveto{\pgfqpoint{4.043812in}{1.534864in}}{\pgfqpoint{4.036970in}{1.532030in}}{\pgfqpoint{4.031927in}{1.526986in}}%
\pgfpathcurveto{\pgfqpoint{4.026883in}{1.521943in}}{\pgfqpoint{4.024049in}{1.515101in}}{\pgfqpoint{4.024049in}{1.507968in}}%
\pgfpathcurveto{\pgfqpoint{4.024049in}{1.500835in}}{\pgfqpoint{4.026883in}{1.493994in}}{\pgfqpoint{4.031927in}{1.488950in}}%
\pgfpathcurveto{\pgfqpoint{4.036970in}{1.483906in}}{\pgfqpoint{4.043812in}{1.481072in}}{\pgfqpoint{4.050945in}{1.481072in}}%
\pgfpathclose%
\pgfusepath{stroke,fill}%
\end{pgfscope}%
\begin{pgfscope}%
\pgfpathrectangle{\pgfqpoint{2.867647in}{0.500000in}}{\pgfqpoint{1.764706in}{1.700000in}}%
\pgfusepath{clip}%
\pgfsetbuttcap%
\pgfsetroundjoin%
\definecolor{currentfill}{rgb}{0.955697,0.484891,0.334214}%
\pgfsetfillcolor{currentfill}%
\pgfsetlinewidth{0.311001pt}%
\definecolor{currentstroke}{rgb}{1.000000,1.000000,1.000000}%
\pgfsetstrokecolor{currentstroke}%
\pgfsetdash{}{0pt}%
\pgfpathmoveto{\pgfqpoint{4.334617in}{1.473032in}}%
\pgfpathcurveto{\pgfqpoint{4.341750in}{1.473032in}}{\pgfqpoint{4.348592in}{1.475866in}}{\pgfqpoint{4.353635in}{1.480910in}}%
\pgfpathcurveto{\pgfqpoint{4.358679in}{1.485954in}}{\pgfqpoint{4.361513in}{1.492795in}}{\pgfqpoint{4.361513in}{1.499928in}}%
\pgfpathcurveto{\pgfqpoint{4.361513in}{1.507061in}}{\pgfqpoint{4.358679in}{1.513903in}}{\pgfqpoint{4.353635in}{1.518946in}}%
\pgfpathcurveto{\pgfqpoint{4.348592in}{1.523990in}}{\pgfqpoint{4.341750in}{1.526824in}}{\pgfqpoint{4.334617in}{1.526824in}}%
\pgfpathcurveto{\pgfqpoint{4.327484in}{1.526824in}}{\pgfqpoint{4.320643in}{1.523990in}}{\pgfqpoint{4.315599in}{1.518946in}}%
\pgfpathcurveto{\pgfqpoint{4.310555in}{1.513903in}}{\pgfqpoint{4.307722in}{1.507061in}}{\pgfqpoint{4.307722in}{1.499928in}}%
\pgfpathcurveto{\pgfqpoint{4.307722in}{1.492795in}}{\pgfqpoint{4.310555in}{1.485954in}}{\pgfqpoint{4.315599in}{1.480910in}}%
\pgfpathcurveto{\pgfqpoint{4.320643in}{1.475866in}}{\pgfqpoint{4.327484in}{1.473032in}}{\pgfqpoint{4.334617in}{1.473032in}}%
\pgfpathclose%
\pgfusepath{stroke,fill}%
\end{pgfscope}%
\begin{pgfscope}%
\pgfpathrectangle{\pgfqpoint{2.867647in}{0.500000in}}{\pgfqpoint{1.764706in}{1.700000in}}%
\pgfusepath{clip}%
\pgfsetbuttcap%
\pgfsetroundjoin%
\definecolor{currentfill}{rgb}{0.976961,0.885681,0.814303}%
\pgfsetfillcolor{currentfill}%
\pgfsetlinewidth{0.311001pt}%
\definecolor{currentstroke}{rgb}{1.000000,1.000000,1.000000}%
\pgfsetstrokecolor{currentstroke}%
\pgfsetdash{}{0pt}%
\pgfpathmoveto{\pgfqpoint{4.192906in}{1.541476in}}%
\pgfpathcurveto{\pgfqpoint{4.200039in}{1.541476in}}{\pgfqpoint{4.206881in}{1.544310in}}{\pgfqpoint{4.211925in}{1.549354in}}%
\pgfpathcurveto{\pgfqpoint{4.216968in}{1.554398in}}{\pgfqpoint{4.219802in}{1.561239in}}{\pgfqpoint{4.219802in}{1.568372in}}%
\pgfpathcurveto{\pgfqpoint{4.219802in}{1.575505in}}{\pgfqpoint{4.216968in}{1.582347in}}{\pgfqpoint{4.211925in}{1.587390in}}%
\pgfpathcurveto{\pgfqpoint{4.206881in}{1.592434in}}{\pgfqpoint{4.200039in}{1.595268in}}{\pgfqpoint{4.192906in}{1.595268in}}%
\pgfpathcurveto{\pgfqpoint{4.185774in}{1.595268in}}{\pgfqpoint{4.178932in}{1.592434in}}{\pgfqpoint{4.173888in}{1.587390in}}%
\pgfpathcurveto{\pgfqpoint{4.168845in}{1.582347in}}{\pgfqpoint{4.166011in}{1.575505in}}{\pgfqpoint{4.166011in}{1.568372in}}%
\pgfpathcurveto{\pgfqpoint{4.166011in}{1.561239in}}{\pgfqpoint{4.168845in}{1.554398in}}{\pgfqpoint{4.173888in}{1.549354in}}%
\pgfpathcurveto{\pgfqpoint{4.178932in}{1.544310in}}{\pgfqpoint{4.185774in}{1.541476in}}{\pgfqpoint{4.192906in}{1.541476in}}%
\pgfpathclose%
\pgfusepath{stroke,fill}%
\end{pgfscope}%
\begin{pgfscope}%
\pgfpathrectangle{\pgfqpoint{2.867647in}{0.500000in}}{\pgfqpoint{1.764706in}{1.700000in}}%
\pgfusepath{clip}%
\pgfsetbuttcap%
\pgfsetroundjoin%
\definecolor{currentfill}{rgb}{0.969359,0.803954,0.693832}%
\pgfsetfillcolor{currentfill}%
\pgfsetlinewidth{0.311001pt}%
\definecolor{currentstroke}{rgb}{1.000000,1.000000,1.000000}%
\pgfsetstrokecolor{currentstroke}%
\pgfsetdash{}{0pt}%
\pgfpathmoveto{\pgfqpoint{4.049833in}{0.962051in}}%
\pgfpathcurveto{\pgfqpoint{4.056966in}{0.962051in}}{\pgfqpoint{4.063808in}{0.964885in}}{\pgfqpoint{4.068851in}{0.969928in}}%
\pgfpathcurveto{\pgfqpoint{4.073895in}{0.974972in}}{\pgfqpoint{4.076729in}{0.981814in}}{\pgfqpoint{4.076729in}{0.988946in}}%
\pgfpathcurveto{\pgfqpoint{4.076729in}{0.996079in}}{\pgfqpoint{4.073895in}{1.002921in}}{\pgfqpoint{4.068851in}{1.007965in}}%
\pgfpathcurveto{\pgfqpoint{4.063808in}{1.013008in}}{\pgfqpoint{4.056966in}{1.015842in}}{\pgfqpoint{4.049833in}{1.015842in}}%
\pgfpathcurveto{\pgfqpoint{4.042700in}{1.015842in}}{\pgfqpoint{4.035859in}{1.013008in}}{\pgfqpoint{4.030815in}{1.007965in}}%
\pgfpathcurveto{\pgfqpoint{4.025771in}{1.002921in}}{\pgfqpoint{4.022937in}{0.996079in}}{\pgfqpoint{4.022937in}{0.988946in}}%
\pgfpathcurveto{\pgfqpoint{4.022937in}{0.981814in}}{\pgfqpoint{4.025771in}{0.974972in}}{\pgfqpoint{4.030815in}{0.969928in}}%
\pgfpathcurveto{\pgfqpoint{4.035859in}{0.964885in}}{\pgfqpoint{4.042700in}{0.962051in}}{\pgfqpoint{4.049833in}{0.962051in}}%
\pgfpathclose%
\pgfusepath{stroke,fill}%
\end{pgfscope}%
\begin{pgfscope}%
\pgfpathrectangle{\pgfqpoint{2.867647in}{0.500000in}}{\pgfqpoint{1.764706in}{1.700000in}}%
\pgfusepath{clip}%
\pgfsetbuttcap%
\pgfsetroundjoin%
\definecolor{currentfill}{rgb}{0.965440,0.720101,0.576404}%
\pgfsetfillcolor{currentfill}%
\pgfsetlinewidth{0.311001pt}%
\definecolor{currentstroke}{rgb}{1.000000,1.000000,1.000000}%
\pgfsetstrokecolor{currentstroke}%
\pgfsetdash{}{0pt}%
\pgfpathmoveto{\pgfqpoint{3.993522in}{1.687889in}}%
\pgfpathcurveto{\pgfqpoint{4.000655in}{1.687889in}}{\pgfqpoint{4.007497in}{1.690723in}}{\pgfqpoint{4.012540in}{1.695767in}}%
\pgfpathcurveto{\pgfqpoint{4.017584in}{1.700810in}}{\pgfqpoint{4.020418in}{1.707652in}}{\pgfqpoint{4.020418in}{1.714785in}}%
\pgfpathcurveto{\pgfqpoint{4.020418in}{1.721918in}}{\pgfqpoint{4.017584in}{1.728759in}}{\pgfqpoint{4.012540in}{1.733803in}}%
\pgfpathcurveto{\pgfqpoint{4.007497in}{1.738847in}}{\pgfqpoint{4.000655in}{1.741681in}}{\pgfqpoint{3.993522in}{1.741681in}}%
\pgfpathcurveto{\pgfqpoint{3.986389in}{1.741681in}}{\pgfqpoint{3.979548in}{1.738847in}}{\pgfqpoint{3.974504in}{1.733803in}}%
\pgfpathcurveto{\pgfqpoint{3.969460in}{1.728759in}}{\pgfqpoint{3.966626in}{1.721918in}}{\pgfqpoint{3.966626in}{1.714785in}}%
\pgfpathcurveto{\pgfqpoint{3.966626in}{1.707652in}}{\pgfqpoint{3.969460in}{1.700810in}}{\pgfqpoint{3.974504in}{1.695767in}}%
\pgfpathcurveto{\pgfqpoint{3.979548in}{1.690723in}}{\pgfqpoint{3.986389in}{1.687889in}}{\pgfqpoint{3.993522in}{1.687889in}}%
\pgfpathclose%
\pgfusepath{stroke,fill}%
\end{pgfscope}%
\begin{pgfscope}%
\pgfpathrectangle{\pgfqpoint{2.867647in}{0.500000in}}{\pgfqpoint{1.764706in}{1.700000in}}%
\pgfusepath{clip}%
\pgfsetbuttcap%
\pgfsetroundjoin%
\definecolor{currentfill}{rgb}{0.980678,0.914765,0.856766}%
\pgfsetfillcolor{currentfill}%
\pgfsetlinewidth{0.311001pt}%
\definecolor{currentstroke}{rgb}{1.000000,1.000000,1.000000}%
\pgfsetstrokecolor{currentstroke}%
\pgfsetdash{}{0pt}%
\pgfpathmoveto{\pgfqpoint{4.201481in}{1.399062in}}%
\pgfpathcurveto{\pgfqpoint{4.208614in}{1.399062in}}{\pgfqpoint{4.215456in}{1.401896in}}{\pgfqpoint{4.220499in}{1.406940in}}%
\pgfpathcurveto{\pgfqpoint{4.225543in}{1.411983in}}{\pgfqpoint{4.228377in}{1.418825in}}{\pgfqpoint{4.228377in}{1.425958in}}%
\pgfpathcurveto{\pgfqpoint{4.228377in}{1.433091in}}{\pgfqpoint{4.225543in}{1.439932in}}{\pgfqpoint{4.220499in}{1.444976in}}%
\pgfpathcurveto{\pgfqpoint{4.215456in}{1.450020in}}{\pgfqpoint{4.208614in}{1.452853in}}{\pgfqpoint{4.201481in}{1.452853in}}%
\pgfpathcurveto{\pgfqpoint{4.194348in}{1.452853in}}{\pgfqpoint{4.187507in}{1.450020in}}{\pgfqpoint{4.182463in}{1.444976in}}%
\pgfpathcurveto{\pgfqpoint{4.177419in}{1.439932in}}{\pgfqpoint{4.174586in}{1.433091in}}{\pgfqpoint{4.174586in}{1.425958in}}%
\pgfpathcurveto{\pgfqpoint{4.174586in}{1.418825in}}{\pgfqpoint{4.177419in}{1.411983in}}{\pgfqpoint{4.182463in}{1.406940in}}%
\pgfpathcurveto{\pgfqpoint{4.187507in}{1.401896in}}{\pgfqpoint{4.194348in}{1.399062in}}{\pgfqpoint{4.201481in}{1.399062in}}%
\pgfpathclose%
\pgfusepath{stroke,fill}%
\end{pgfscope}%
\begin{pgfscope}%
\pgfpathrectangle{\pgfqpoint{2.867647in}{0.500000in}}{\pgfqpoint{1.764706in}{1.700000in}}%
\pgfusepath{clip}%
\pgfsetbuttcap%
\pgfsetroundjoin%
\definecolor{currentfill}{rgb}{0.917171,0.267738,0.242941}%
\pgfsetfillcolor{currentfill}%
\pgfsetlinewidth{0.311001pt}%
\definecolor{currentstroke}{rgb}{1.000000,1.000000,1.000000}%
\pgfsetstrokecolor{currentstroke}%
\pgfsetdash{}{0pt}%
\pgfpathmoveto{\pgfqpoint{3.851070in}{1.636093in}}%
\pgfpathcurveto{\pgfqpoint{3.858202in}{1.636093in}}{\pgfqpoint{3.865044in}{1.638927in}}{\pgfqpoint{3.870088in}{1.643971in}}%
\pgfpathcurveto{\pgfqpoint{3.875131in}{1.649014in}}{\pgfqpoint{3.877965in}{1.655856in}}{\pgfqpoint{3.877965in}{1.662989in}}%
\pgfpathcurveto{\pgfqpoint{3.877965in}{1.670122in}}{\pgfqpoint{3.875131in}{1.676963in}}{\pgfqpoint{3.870088in}{1.682007in}}%
\pgfpathcurveto{\pgfqpoint{3.865044in}{1.687051in}}{\pgfqpoint{3.858202in}{1.689885in}}{\pgfqpoint{3.851070in}{1.689885in}}%
\pgfpathcurveto{\pgfqpoint{3.843937in}{1.689885in}}{\pgfqpoint{3.837095in}{1.687051in}}{\pgfqpoint{3.832051in}{1.682007in}}%
\pgfpathcurveto{\pgfqpoint{3.827008in}{1.676963in}}{\pgfqpoint{3.824174in}{1.670122in}}{\pgfqpoint{3.824174in}{1.662989in}}%
\pgfpathcurveto{\pgfqpoint{3.824174in}{1.655856in}}{\pgfqpoint{3.827008in}{1.649014in}}{\pgfqpoint{3.832051in}{1.643971in}}%
\pgfpathcurveto{\pgfqpoint{3.837095in}{1.638927in}}{\pgfqpoint{3.843937in}{1.636093in}}{\pgfqpoint{3.851070in}{1.636093in}}%
\pgfpathclose%
\pgfusepath{stroke,fill}%
\end{pgfscope}%
\begin{pgfscope}%
\pgfpathrectangle{\pgfqpoint{2.867647in}{0.500000in}}{\pgfqpoint{1.764706in}{1.700000in}}%
\pgfusepath{clip}%
\pgfsetbuttcap%
\pgfsetroundjoin%
\definecolor{currentfill}{rgb}{0.974412,0.862387,0.780156}%
\pgfsetfillcolor{currentfill}%
\pgfsetlinewidth{0.311001pt}%
\definecolor{currentstroke}{rgb}{1.000000,1.000000,1.000000}%
\pgfsetstrokecolor{currentstroke}%
\pgfsetdash{}{0pt}%
\pgfpathmoveto{\pgfqpoint{4.089389in}{1.520311in}}%
\pgfpathcurveto{\pgfqpoint{4.096522in}{1.520311in}}{\pgfqpoint{4.103364in}{1.523145in}}{\pgfqpoint{4.108407in}{1.528188in}}%
\pgfpathcurveto{\pgfqpoint{4.113451in}{1.533232in}}{\pgfqpoint{4.116285in}{1.540074in}}{\pgfqpoint{4.116285in}{1.547206in}}%
\pgfpathcurveto{\pgfqpoint{4.116285in}{1.554339in}}{\pgfqpoint{4.113451in}{1.561181in}}{\pgfqpoint{4.108407in}{1.566225in}}%
\pgfpathcurveto{\pgfqpoint{4.103364in}{1.571268in}}{\pgfqpoint{4.096522in}{1.574102in}}{\pgfqpoint{4.089389in}{1.574102in}}%
\pgfpathcurveto{\pgfqpoint{4.082256in}{1.574102in}}{\pgfqpoint{4.075415in}{1.571268in}}{\pgfqpoint{4.070371in}{1.566225in}}%
\pgfpathcurveto{\pgfqpoint{4.065327in}{1.561181in}}{\pgfqpoint{4.062493in}{1.554339in}}{\pgfqpoint{4.062493in}{1.547206in}}%
\pgfpathcurveto{\pgfqpoint{4.062493in}{1.540074in}}{\pgfqpoint{4.065327in}{1.533232in}}{\pgfqpoint{4.070371in}{1.528188in}}%
\pgfpathcurveto{\pgfqpoint{4.075415in}{1.523145in}}{\pgfqpoint{4.082256in}{1.520311in}}{\pgfqpoint{4.089389in}{1.520311in}}%
\pgfpathclose%
\pgfusepath{stroke,fill}%
\end{pgfscope}%
\begin{pgfscope}%
\pgfpathrectangle{\pgfqpoint{2.867647in}{0.500000in}}{\pgfqpoint{1.764706in}{1.700000in}}%
\pgfusepath{clip}%
\pgfsetbuttcap%
\pgfsetroundjoin%
\definecolor{currentfill}{rgb}{0.975018,0.868213,0.788710}%
\pgfsetfillcolor{currentfill}%
\pgfsetlinewidth{0.311001pt}%
\definecolor{currentstroke}{rgb}{1.000000,1.000000,1.000000}%
\pgfsetstrokecolor{currentstroke}%
\pgfsetdash{}{0pt}%
\pgfpathmoveto{\pgfqpoint{4.088023in}{1.064462in}}%
\pgfpathcurveto{\pgfqpoint{4.095156in}{1.064462in}}{\pgfqpoint{4.101998in}{1.067296in}}{\pgfqpoint{4.107042in}{1.072340in}}%
\pgfpathcurveto{\pgfqpoint{4.112085in}{1.077384in}}{\pgfqpoint{4.114919in}{1.084225in}}{\pgfqpoint{4.114919in}{1.091358in}}%
\pgfpathcurveto{\pgfqpoint{4.114919in}{1.098491in}}{\pgfqpoint{4.112085in}{1.105333in}}{\pgfqpoint{4.107042in}{1.110376in}}%
\pgfpathcurveto{\pgfqpoint{4.101998in}{1.115420in}}{\pgfqpoint{4.095156in}{1.118254in}}{\pgfqpoint{4.088023in}{1.118254in}}%
\pgfpathcurveto{\pgfqpoint{4.080891in}{1.118254in}}{\pgfqpoint{4.074049in}{1.115420in}}{\pgfqpoint{4.069005in}{1.110376in}}%
\pgfpathcurveto{\pgfqpoint{4.063962in}{1.105333in}}{\pgfqpoint{4.061128in}{1.098491in}}{\pgfqpoint{4.061128in}{1.091358in}}%
\pgfpathcurveto{\pgfqpoint{4.061128in}{1.084225in}}{\pgfqpoint{4.063962in}{1.077384in}}{\pgfqpoint{4.069005in}{1.072340in}}%
\pgfpathcurveto{\pgfqpoint{4.074049in}{1.067296in}}{\pgfqpoint{4.080891in}{1.064462in}}{\pgfqpoint{4.088023in}{1.064462in}}%
\pgfpathclose%
\pgfusepath{stroke,fill}%
\end{pgfscope}%
\begin{pgfscope}%
\pgfpathrectangle{\pgfqpoint{2.867647in}{0.500000in}}{\pgfqpoint{1.764706in}{1.700000in}}%
\pgfusepath{clip}%
\pgfsetbuttcap%
\pgfsetroundjoin%
\definecolor{currentfill}{rgb}{0.973271,0.850724,0.762998}%
\pgfsetfillcolor{currentfill}%
\pgfsetlinewidth{0.311001pt}%
\definecolor{currentstroke}{rgb}{1.000000,1.000000,1.000000}%
\pgfsetstrokecolor{currentstroke}%
\pgfsetdash{}{0pt}%
\pgfpathmoveto{\pgfqpoint{4.099654in}{0.989224in}}%
\pgfpathcurveto{\pgfqpoint{4.106787in}{0.989224in}}{\pgfqpoint{4.113629in}{0.992058in}}{\pgfqpoint{4.118672in}{0.997101in}}%
\pgfpathcurveto{\pgfqpoint{4.123716in}{1.002145in}}{\pgfqpoint{4.126550in}{1.008987in}}{\pgfqpoint{4.126550in}{1.016119in}}%
\pgfpathcurveto{\pgfqpoint{4.126550in}{1.023252in}}{\pgfqpoint{4.123716in}{1.030094in}}{\pgfqpoint{4.118672in}{1.035138in}}%
\pgfpathcurveto{\pgfqpoint{4.113629in}{1.040181in}}{\pgfqpoint{4.106787in}{1.043015in}}{\pgfqpoint{4.099654in}{1.043015in}}%
\pgfpathcurveto{\pgfqpoint{4.092521in}{1.043015in}}{\pgfqpoint{4.085680in}{1.040181in}}{\pgfqpoint{4.080636in}{1.035138in}}%
\pgfpathcurveto{\pgfqpoint{4.075592in}{1.030094in}}{\pgfqpoint{4.072759in}{1.023252in}}{\pgfqpoint{4.072759in}{1.016119in}}%
\pgfpathcurveto{\pgfqpoint{4.072759in}{1.008987in}}{\pgfqpoint{4.075592in}{1.002145in}}{\pgfqpoint{4.080636in}{0.997101in}}%
\pgfpathcurveto{\pgfqpoint{4.085680in}{0.992058in}}{\pgfqpoint{4.092521in}{0.989224in}}{\pgfqpoint{4.099654in}{0.989224in}}%
\pgfpathclose%
\pgfusepath{stroke,fill}%
\end{pgfscope}%
\begin{pgfscope}%
\pgfpathrectangle{\pgfqpoint{2.867647in}{0.500000in}}{\pgfqpoint{1.764706in}{1.700000in}}%
\pgfusepath{clip}%
\pgfsetbuttcap%
\pgfsetroundjoin%
\definecolor{currentfill}{rgb}{0.965753,0.732351,0.592427}%
\pgfsetfillcolor{currentfill}%
\pgfsetlinewidth{0.311001pt}%
\definecolor{currentstroke}{rgb}{1.000000,1.000000,1.000000}%
\pgfsetstrokecolor{currentstroke}%
\pgfsetdash{}{0pt}%
\pgfpathmoveto{\pgfqpoint{4.021943in}{1.568642in}}%
\pgfpathcurveto{\pgfqpoint{4.029076in}{1.568642in}}{\pgfqpoint{4.035918in}{1.571476in}}{\pgfqpoint{4.040962in}{1.576520in}}%
\pgfpathcurveto{\pgfqpoint{4.046005in}{1.581563in}}{\pgfqpoint{4.048839in}{1.588405in}}{\pgfqpoint{4.048839in}{1.595538in}}%
\pgfpathcurveto{\pgfqpoint{4.048839in}{1.602671in}}{\pgfqpoint{4.046005in}{1.609512in}}{\pgfqpoint{4.040962in}{1.614556in}}%
\pgfpathcurveto{\pgfqpoint{4.035918in}{1.619600in}}{\pgfqpoint{4.029076in}{1.622434in}}{\pgfqpoint{4.021943in}{1.622434in}}%
\pgfpathcurveto{\pgfqpoint{4.014811in}{1.622434in}}{\pgfqpoint{4.007969in}{1.619600in}}{\pgfqpoint{4.002925in}{1.614556in}}%
\pgfpathcurveto{\pgfqpoint{3.997882in}{1.609512in}}{\pgfqpoint{3.995048in}{1.602671in}}{\pgfqpoint{3.995048in}{1.595538in}}%
\pgfpathcurveto{\pgfqpoint{3.995048in}{1.588405in}}{\pgfqpoint{3.997882in}{1.581563in}}{\pgfqpoint{4.002925in}{1.576520in}}%
\pgfpathcurveto{\pgfqpoint{4.007969in}{1.571476in}}{\pgfqpoint{4.014811in}{1.568642in}}{\pgfqpoint{4.021943in}{1.568642in}}%
\pgfpathclose%
\pgfusepath{stroke,fill}%
\end{pgfscope}%
\begin{pgfscope}%
\pgfpathrectangle{\pgfqpoint{2.867647in}{0.500000in}}{\pgfqpoint{1.764706in}{1.700000in}}%
\pgfusepath{clip}%
\pgfsetbuttcap%
\pgfsetroundjoin%
\definecolor{currentfill}{rgb}{0.965042,0.701564,0.552889}%
\pgfsetfillcolor{currentfill}%
\pgfsetlinewidth{0.311001pt}%
\definecolor{currentstroke}{rgb}{1.000000,1.000000,1.000000}%
\pgfsetstrokecolor{currentstroke}%
\pgfsetdash{}{0pt}%
\pgfpathmoveto{\pgfqpoint{4.041387in}{0.888710in}}%
\pgfpathcurveto{\pgfqpoint{4.048520in}{0.888710in}}{\pgfqpoint{4.055362in}{0.891544in}}{\pgfqpoint{4.060405in}{0.896588in}}%
\pgfpathcurveto{\pgfqpoint{4.065449in}{0.901632in}}{\pgfqpoint{4.068283in}{0.908473in}}{\pgfqpoint{4.068283in}{0.915606in}}%
\pgfpathcurveto{\pgfqpoint{4.068283in}{0.922739in}}{\pgfqpoint{4.065449in}{0.929581in}}{\pgfqpoint{4.060405in}{0.934624in}}%
\pgfpathcurveto{\pgfqpoint{4.055362in}{0.939668in}}{\pgfqpoint{4.048520in}{0.942502in}}{\pgfqpoint{4.041387in}{0.942502in}}%
\pgfpathcurveto{\pgfqpoint{4.034254in}{0.942502in}}{\pgfqpoint{4.027413in}{0.939668in}}{\pgfqpoint{4.022369in}{0.934624in}}%
\pgfpathcurveto{\pgfqpoint{4.017325in}{0.929581in}}{\pgfqpoint{4.014492in}{0.922739in}}{\pgfqpoint{4.014492in}{0.915606in}}%
\pgfpathcurveto{\pgfqpoint{4.014492in}{0.908473in}}{\pgfqpoint{4.017325in}{0.901632in}}{\pgfqpoint{4.022369in}{0.896588in}}%
\pgfpathcurveto{\pgfqpoint{4.027413in}{0.891544in}}{\pgfqpoint{4.034254in}{0.888710in}}{\pgfqpoint{4.041387in}{0.888710in}}%
\pgfpathclose%
\pgfusepath{stroke,fill}%
\end{pgfscope}%
\begin{pgfscope}%
\pgfpathrectangle{\pgfqpoint{2.867647in}{0.500000in}}{\pgfqpoint{1.764706in}{1.700000in}}%
\pgfusepath{clip}%
\pgfsetbuttcap%
\pgfsetroundjoin%
\definecolor{currentfill}{rgb}{0.974412,0.862387,0.780156}%
\pgfsetfillcolor{currentfill}%
\pgfsetlinewidth{0.311001pt}%
\definecolor{currentstroke}{rgb}{1.000000,1.000000,1.000000}%
\pgfsetstrokecolor{currentstroke}%
\pgfsetdash{}{0pt}%
\pgfpathmoveto{\pgfqpoint{4.252741in}{1.342064in}}%
\pgfpathcurveto{\pgfqpoint{4.259874in}{1.342064in}}{\pgfqpoint{4.266716in}{1.344898in}}{\pgfqpoint{4.271759in}{1.349941in}}%
\pgfpathcurveto{\pgfqpoint{4.276803in}{1.354985in}}{\pgfqpoint{4.279637in}{1.361827in}}{\pgfqpoint{4.279637in}{1.368960in}}%
\pgfpathcurveto{\pgfqpoint{4.279637in}{1.376092in}}{\pgfqpoint{4.276803in}{1.382934in}}{\pgfqpoint{4.271759in}{1.387978in}}%
\pgfpathcurveto{\pgfqpoint{4.266716in}{1.393021in}}{\pgfqpoint{4.259874in}{1.395855in}}{\pgfqpoint{4.252741in}{1.395855in}}%
\pgfpathcurveto{\pgfqpoint{4.245608in}{1.395855in}}{\pgfqpoint{4.238767in}{1.393021in}}{\pgfqpoint{4.233723in}{1.387978in}}%
\pgfpathcurveto{\pgfqpoint{4.228679in}{1.382934in}}{\pgfqpoint{4.225845in}{1.376092in}}{\pgfqpoint{4.225845in}{1.368960in}}%
\pgfpathcurveto{\pgfqpoint{4.225845in}{1.361827in}}{\pgfqpoint{4.228679in}{1.354985in}}{\pgfqpoint{4.233723in}{1.349941in}}%
\pgfpathcurveto{\pgfqpoint{4.238767in}{1.344898in}}{\pgfqpoint{4.245608in}{1.342064in}}{\pgfqpoint{4.252741in}{1.342064in}}%
\pgfpathclose%
\pgfusepath{stroke,fill}%
\end{pgfscope}%
\begin{pgfscope}%
\pgfpathrectangle{\pgfqpoint{2.867647in}{0.500000in}}{\pgfqpoint{1.764706in}{1.700000in}}%
\pgfusepath{clip}%
\pgfsetbuttcap%
\pgfsetroundjoin%
\definecolor{currentfill}{rgb}{0.964799,0.689101,0.537560}%
\pgfsetfillcolor{currentfill}%
\pgfsetlinewidth{0.311001pt}%
\definecolor{currentstroke}{rgb}{1.000000,1.000000,1.000000}%
\pgfsetstrokecolor{currentstroke}%
\pgfsetdash{}{0pt}%
\pgfpathmoveto{\pgfqpoint{3.982955in}{1.650498in}}%
\pgfpathcurveto{\pgfqpoint{3.990088in}{1.650498in}}{\pgfqpoint{3.996930in}{1.653332in}}{\pgfqpoint{4.001973in}{1.658376in}}%
\pgfpathcurveto{\pgfqpoint{4.007017in}{1.663419in}}{\pgfqpoint{4.009851in}{1.670261in}}{\pgfqpoint{4.009851in}{1.677394in}}%
\pgfpathcurveto{\pgfqpoint{4.009851in}{1.684527in}}{\pgfqpoint{4.007017in}{1.691368in}}{\pgfqpoint{4.001973in}{1.696412in}}%
\pgfpathcurveto{\pgfqpoint{3.996930in}{1.701456in}}{\pgfqpoint{3.990088in}{1.704290in}}{\pgfqpoint{3.982955in}{1.704290in}}%
\pgfpathcurveto{\pgfqpoint{3.975822in}{1.704290in}}{\pgfqpoint{3.968981in}{1.701456in}}{\pgfqpoint{3.963937in}{1.696412in}}%
\pgfpathcurveto{\pgfqpoint{3.958893in}{1.691368in}}{\pgfqpoint{3.956059in}{1.684527in}}{\pgfqpoint{3.956059in}{1.677394in}}%
\pgfpathcurveto{\pgfqpoint{3.956059in}{1.670261in}}{\pgfqpoint{3.958893in}{1.663419in}}{\pgfqpoint{3.963937in}{1.658376in}}%
\pgfpathcurveto{\pgfqpoint{3.968981in}{1.653332in}}{\pgfqpoint{3.975822in}{1.650498in}}{\pgfqpoint{3.982955in}{1.650498in}}%
\pgfpathclose%
\pgfusepath{stroke,fill}%
\end{pgfscope}%
\begin{pgfscope}%
\pgfpathrectangle{\pgfqpoint{2.867647in}{0.500000in}}{\pgfqpoint{1.764706in}{1.700000in}}%
\pgfusepath{clip}%
\pgfsetbuttcap%
\pgfsetroundjoin%
\definecolor{currentfill}{rgb}{0.975644,0.874038,0.797253}%
\pgfsetfillcolor{currentfill}%
\pgfsetlinewidth{0.311001pt}%
\definecolor{currentstroke}{rgb}{1.000000,1.000000,1.000000}%
\pgfsetstrokecolor{currentstroke}%
\pgfsetdash{}{0pt}%
\pgfpathmoveto{\pgfqpoint{4.102372in}{1.027514in}}%
\pgfpathcurveto{\pgfqpoint{4.109505in}{1.027514in}}{\pgfqpoint{4.116347in}{1.030348in}}{\pgfqpoint{4.121390in}{1.035391in}}%
\pgfpathcurveto{\pgfqpoint{4.126434in}{1.040435in}}{\pgfqpoint{4.129268in}{1.047277in}}{\pgfqpoint{4.129268in}{1.054409in}}%
\pgfpathcurveto{\pgfqpoint{4.129268in}{1.061542in}}{\pgfqpoint{4.126434in}{1.068384in}}{\pgfqpoint{4.121390in}{1.073428in}}%
\pgfpathcurveto{\pgfqpoint{4.116347in}{1.078471in}}{\pgfqpoint{4.109505in}{1.081305in}}{\pgfqpoint{4.102372in}{1.081305in}}%
\pgfpathcurveto{\pgfqpoint{4.095239in}{1.081305in}}{\pgfqpoint{4.088398in}{1.078471in}}{\pgfqpoint{4.083354in}{1.073428in}}%
\pgfpathcurveto{\pgfqpoint{4.078310in}{1.068384in}}{\pgfqpoint{4.075476in}{1.061542in}}{\pgfqpoint{4.075476in}{1.054409in}}%
\pgfpathcurveto{\pgfqpoint{4.075476in}{1.047277in}}{\pgfqpoint{4.078310in}{1.040435in}}{\pgfqpoint{4.083354in}{1.035391in}}%
\pgfpathcurveto{\pgfqpoint{4.088398in}{1.030348in}}{\pgfqpoint{4.095239in}{1.027514in}}{\pgfqpoint{4.102372in}{1.027514in}}%
\pgfpathclose%
\pgfusepath{stroke,fill}%
\end{pgfscope}%
\begin{pgfscope}%
\pgfpathrectangle{\pgfqpoint{2.867647in}{0.500000in}}{\pgfqpoint{1.764706in}{1.700000in}}%
\pgfusepath{clip}%
\pgfsetbuttcap%
\pgfsetroundjoin%
\definecolor{currentfill}{rgb}{0.970255,0.815666,0.711203}%
\pgfsetfillcolor{currentfill}%
\pgfsetlinewidth{0.311001pt}%
\definecolor{currentstroke}{rgb}{1.000000,1.000000,1.000000}%
\pgfsetstrokecolor{currentstroke}%
\pgfsetdash{}{0pt}%
\pgfpathmoveto{\pgfqpoint{4.098661in}{1.224088in}}%
\pgfpathcurveto{\pgfqpoint{4.105794in}{1.224088in}}{\pgfqpoint{4.112636in}{1.226922in}}{\pgfqpoint{4.117679in}{1.231965in}}%
\pgfpathcurveto{\pgfqpoint{4.122723in}{1.237009in}}{\pgfqpoint{4.125557in}{1.243851in}}{\pgfqpoint{4.125557in}{1.250983in}}%
\pgfpathcurveto{\pgfqpoint{4.125557in}{1.258116in}}{\pgfqpoint{4.122723in}{1.264958in}}{\pgfqpoint{4.117679in}{1.270002in}}%
\pgfpathcurveto{\pgfqpoint{4.112636in}{1.275045in}}{\pgfqpoint{4.105794in}{1.277879in}}{\pgfqpoint{4.098661in}{1.277879in}}%
\pgfpathcurveto{\pgfqpoint{4.091528in}{1.277879in}}{\pgfqpoint{4.084687in}{1.275045in}}{\pgfqpoint{4.079643in}{1.270002in}}%
\pgfpathcurveto{\pgfqpoint{4.074599in}{1.264958in}}{\pgfqpoint{4.071766in}{1.258116in}}{\pgfqpoint{4.071766in}{1.250983in}}%
\pgfpathcurveto{\pgfqpoint{4.071766in}{1.243851in}}{\pgfqpoint{4.074599in}{1.237009in}}{\pgfqpoint{4.079643in}{1.231965in}}%
\pgfpathcurveto{\pgfqpoint{4.084687in}{1.226922in}}{\pgfqpoint{4.091528in}{1.224088in}}{\pgfqpoint{4.098661in}{1.224088in}}%
\pgfpathclose%
\pgfusepath{stroke,fill}%
\end{pgfscope}%
\begin{pgfscope}%
\pgfpathrectangle{\pgfqpoint{2.867647in}{0.500000in}}{\pgfqpoint{1.764706in}{1.700000in}}%
\pgfusepath{clip}%
\pgfsetbuttcap%
\pgfsetroundjoin%
\definecolor{currentfill}{rgb}{0.971202,0.827364,0.728520}%
\pgfsetfillcolor{currentfill}%
\pgfsetlinewidth{0.311001pt}%
\definecolor{currentstroke}{rgb}{1.000000,1.000000,1.000000}%
\pgfsetstrokecolor{currentstroke}%
\pgfsetdash{}{0pt}%
\pgfpathmoveto{\pgfqpoint{4.052408in}{1.586127in}}%
\pgfpathcurveto{\pgfqpoint{4.059541in}{1.586127in}}{\pgfqpoint{4.066382in}{1.588960in}}{\pgfqpoint{4.071426in}{1.594004in}}%
\pgfpathcurveto{\pgfqpoint{4.076470in}{1.599048in}}{\pgfqpoint{4.079304in}{1.605889in}}{\pgfqpoint{4.079304in}{1.613022in}}%
\pgfpathcurveto{\pgfqpoint{4.079304in}{1.620155in}}{\pgfqpoint{4.076470in}{1.626997in}}{\pgfqpoint{4.071426in}{1.632040in}}%
\pgfpathcurveto{\pgfqpoint{4.066382in}{1.637084in}}{\pgfqpoint{4.059541in}{1.639918in}}{\pgfqpoint{4.052408in}{1.639918in}}%
\pgfpathcurveto{\pgfqpoint{4.045275in}{1.639918in}}{\pgfqpoint{4.038433in}{1.637084in}}{\pgfqpoint{4.033390in}{1.632040in}}%
\pgfpathcurveto{\pgfqpoint{4.028346in}{1.626997in}}{\pgfqpoint{4.025512in}{1.620155in}}{\pgfqpoint{4.025512in}{1.613022in}}%
\pgfpathcurveto{\pgfqpoint{4.025512in}{1.605889in}}{\pgfqpoint{4.028346in}{1.599048in}}{\pgfqpoint{4.033390in}{1.594004in}}%
\pgfpathcurveto{\pgfqpoint{4.038433in}{1.588960in}}{\pgfqpoint{4.045275in}{1.586127in}}{\pgfqpoint{4.052408in}{1.586127in}}%
\pgfpathclose%
\pgfusepath{stroke,fill}%
\end{pgfscope}%
\begin{pgfscope}%
\pgfpathrectangle{\pgfqpoint{2.867647in}{0.500000in}}{\pgfqpoint{1.764706in}{1.700000in}}%
\pgfusepath{clip}%
\pgfsetbuttcap%
\pgfsetroundjoin%
\definecolor{currentfill}{rgb}{0.965042,0.701564,0.552889}%
\pgfsetfillcolor{currentfill}%
\pgfsetlinewidth{0.311001pt}%
\definecolor{currentstroke}{rgb}{1.000000,1.000000,1.000000}%
\pgfsetstrokecolor{currentstroke}%
\pgfsetdash{}{0pt}%
\pgfpathmoveto{\pgfqpoint{3.987823in}{0.957319in}}%
\pgfpathcurveto{\pgfqpoint{3.994955in}{0.957319in}}{\pgfqpoint{4.001797in}{0.960153in}}{\pgfqpoint{4.006841in}{0.965197in}}%
\pgfpathcurveto{\pgfqpoint{4.011884in}{0.970241in}}{\pgfqpoint{4.014718in}{0.977082in}}{\pgfqpoint{4.014718in}{0.984215in}}%
\pgfpathcurveto{\pgfqpoint{4.014718in}{0.991348in}}{\pgfqpoint{4.011884in}{0.998190in}}{\pgfqpoint{4.006841in}{1.003233in}}%
\pgfpathcurveto{\pgfqpoint{4.001797in}{1.008277in}}{\pgfqpoint{3.994955in}{1.011111in}}{\pgfqpoint{3.987823in}{1.011111in}}%
\pgfpathcurveto{\pgfqpoint{3.980690in}{1.011111in}}{\pgfqpoint{3.973848in}{1.008277in}}{\pgfqpoint{3.968804in}{1.003233in}}%
\pgfpathcurveto{\pgfqpoint{3.963761in}{0.998190in}}{\pgfqpoint{3.960927in}{0.991348in}}{\pgfqpoint{3.960927in}{0.984215in}}%
\pgfpathcurveto{\pgfqpoint{3.960927in}{0.977082in}}{\pgfqpoint{3.963761in}{0.970241in}}{\pgfqpoint{3.968804in}{0.965197in}}%
\pgfpathcurveto{\pgfqpoint{3.973848in}{0.960153in}}{\pgfqpoint{3.980690in}{0.957319in}}{\pgfqpoint{3.987823in}{0.957319in}}%
\pgfpathclose%
\pgfusepath{stroke,fill}%
\end{pgfscope}%
\begin{pgfscope}%
\pgfpathrectangle{\pgfqpoint{2.867647in}{0.500000in}}{\pgfqpoint{1.764706in}{1.700000in}}%
\pgfusepath{clip}%
\pgfsetbuttcap%
\pgfsetroundjoin%
\definecolor{currentfill}{rgb}{0.964433,0.670254,0.515093}%
\pgfsetfillcolor{currentfill}%
\pgfsetlinewidth{0.311001pt}%
\definecolor{currentstroke}{rgb}{1.000000,1.000000,1.000000}%
\pgfsetstrokecolor{currentstroke}%
\pgfsetdash{}{0pt}%
\pgfpathmoveto{\pgfqpoint{4.014435in}{0.882637in}}%
\pgfpathcurveto{\pgfqpoint{4.021567in}{0.882637in}}{\pgfqpoint{4.028409in}{0.885471in}}{\pgfqpoint{4.033453in}{0.890515in}}%
\pgfpathcurveto{\pgfqpoint{4.038496in}{0.895559in}}{\pgfqpoint{4.041330in}{0.902400in}}{\pgfqpoint{4.041330in}{0.909533in}}%
\pgfpathcurveto{\pgfqpoint{4.041330in}{0.916666in}}{\pgfqpoint{4.038496in}{0.923508in}}{\pgfqpoint{4.033453in}{0.928551in}}%
\pgfpathcurveto{\pgfqpoint{4.028409in}{0.933595in}}{\pgfqpoint{4.021567in}{0.936429in}}{\pgfqpoint{4.014435in}{0.936429in}}%
\pgfpathcurveto{\pgfqpoint{4.007302in}{0.936429in}}{\pgfqpoint{4.000460in}{0.933595in}}{\pgfqpoint{3.995416in}{0.928551in}}%
\pgfpathcurveto{\pgfqpoint{3.990373in}{0.923508in}}{\pgfqpoint{3.987539in}{0.916666in}}{\pgfqpoint{3.987539in}{0.909533in}}%
\pgfpathcurveto{\pgfqpoint{3.987539in}{0.902400in}}{\pgfqpoint{3.990373in}{0.895559in}}{\pgfqpoint{3.995416in}{0.890515in}}%
\pgfpathcurveto{\pgfqpoint{4.000460in}{0.885471in}}{\pgfqpoint{4.007302in}{0.882637in}}{\pgfqpoint{4.014435in}{0.882637in}}%
\pgfpathclose%
\pgfusepath{stroke,fill}%
\end{pgfscope}%
\begin{pgfscope}%
\pgfpathrectangle{\pgfqpoint{2.867647in}{0.500000in}}{\pgfqpoint{1.764706in}{1.700000in}}%
\pgfusepath{clip}%
\pgfsetbuttcap%
\pgfsetroundjoin%
\definecolor{currentfill}{rgb}{0.980678,0.914765,0.856766}%
\pgfsetfillcolor{currentfill}%
\pgfsetlinewidth{0.311001pt}%
\definecolor{currentstroke}{rgb}{1.000000,1.000000,1.000000}%
\pgfsetstrokecolor{currentstroke}%
\pgfsetdash{}{0pt}%
\pgfpathmoveto{\pgfqpoint{4.196539in}{1.200270in}}%
\pgfpathcurveto{\pgfqpoint{4.203672in}{1.200270in}}{\pgfqpoint{4.210513in}{1.203104in}}{\pgfqpoint{4.215557in}{1.208148in}}%
\pgfpathcurveto{\pgfqpoint{4.220601in}{1.213191in}}{\pgfqpoint{4.223434in}{1.220033in}}{\pgfqpoint{4.223434in}{1.227166in}}%
\pgfpathcurveto{\pgfqpoint{4.223434in}{1.234299in}}{\pgfqpoint{4.220601in}{1.241140in}}{\pgfqpoint{4.215557in}{1.246184in}}%
\pgfpathcurveto{\pgfqpoint{4.210513in}{1.251228in}}{\pgfqpoint{4.203672in}{1.254062in}}{\pgfqpoint{4.196539in}{1.254062in}}%
\pgfpathcurveto{\pgfqpoint{4.189406in}{1.254062in}}{\pgfqpoint{4.182564in}{1.251228in}}{\pgfqpoint{4.177521in}{1.246184in}}%
\pgfpathcurveto{\pgfqpoint{4.172477in}{1.241140in}}{\pgfqpoint{4.169643in}{1.234299in}}{\pgfqpoint{4.169643in}{1.227166in}}%
\pgfpathcurveto{\pgfqpoint{4.169643in}{1.220033in}}{\pgfqpoint{4.172477in}{1.213191in}}{\pgfqpoint{4.177521in}{1.208148in}}%
\pgfpathcurveto{\pgfqpoint{4.182564in}{1.203104in}}{\pgfqpoint{4.189406in}{1.200270in}}{\pgfqpoint{4.196539in}{1.200270in}}%
\pgfpathclose%
\pgfusepath{stroke,fill}%
\end{pgfscope}%
\begin{pgfscope}%
\pgfpathrectangle{\pgfqpoint{2.867647in}{0.500000in}}{\pgfqpoint{1.764706in}{1.700000in}}%
\pgfusepath{clip}%
\pgfsetbuttcap%
\pgfsetroundjoin%
\definecolor{currentfill}{rgb}{0.968931,0.798091,0.685123}%
\pgfsetfillcolor{currentfill}%
\pgfsetlinewidth{0.311001pt}%
\definecolor{currentstroke}{rgb}{1.000000,1.000000,1.000000}%
\pgfsetstrokecolor{currentstroke}%
\pgfsetdash{}{0pt}%
\pgfpathmoveto{\pgfqpoint{4.225187in}{1.059273in}}%
\pgfpathcurveto{\pgfqpoint{4.232320in}{1.059273in}}{\pgfqpoint{4.239162in}{1.062107in}}{\pgfqpoint{4.244206in}{1.067150in}}%
\pgfpathcurveto{\pgfqpoint{4.249249in}{1.072194in}}{\pgfqpoint{4.252083in}{1.079036in}}{\pgfqpoint{4.252083in}{1.086168in}}%
\pgfpathcurveto{\pgfqpoint{4.252083in}{1.093301in}}{\pgfqpoint{4.249249in}{1.100143in}}{\pgfqpoint{4.244206in}{1.105187in}}%
\pgfpathcurveto{\pgfqpoint{4.239162in}{1.110230in}}{\pgfqpoint{4.232320in}{1.113064in}}{\pgfqpoint{4.225187in}{1.113064in}}%
\pgfpathcurveto{\pgfqpoint{4.218055in}{1.113064in}}{\pgfqpoint{4.211213in}{1.110230in}}{\pgfqpoint{4.206169in}{1.105187in}}%
\pgfpathcurveto{\pgfqpoint{4.201126in}{1.100143in}}{\pgfqpoint{4.198292in}{1.093301in}}{\pgfqpoint{4.198292in}{1.086168in}}%
\pgfpathcurveto{\pgfqpoint{4.198292in}{1.079036in}}{\pgfqpoint{4.201126in}{1.072194in}}{\pgfqpoint{4.206169in}{1.067150in}}%
\pgfpathcurveto{\pgfqpoint{4.211213in}{1.062107in}}{\pgfqpoint{4.218055in}{1.059273in}}{\pgfqpoint{4.225187in}{1.059273in}}%
\pgfpathclose%
\pgfusepath{stroke,fill}%
\end{pgfscope}%
\begin{pgfscope}%
\pgfpathrectangle{\pgfqpoint{2.867647in}{0.500000in}}{\pgfqpoint{1.764706in}{1.700000in}}%
\pgfusepath{clip}%
\pgfsetbuttcap%
\pgfsetroundjoin%
\definecolor{currentfill}{rgb}{0.963559,0.632016,0.472047}%
\pgfsetfillcolor{currentfill}%
\pgfsetlinewidth{0.311001pt}%
\definecolor{currentstroke}{rgb}{1.000000,1.000000,1.000000}%
\pgfsetstrokecolor{currentstroke}%
\pgfsetdash{}{0pt}%
\pgfpathmoveto{\pgfqpoint{4.044432in}{1.418235in}}%
\pgfpathcurveto{\pgfqpoint{4.051565in}{1.418235in}}{\pgfqpoint{4.058407in}{1.421069in}}{\pgfqpoint{4.063450in}{1.426113in}}%
\pgfpathcurveto{\pgfqpoint{4.068494in}{1.431156in}}{\pgfqpoint{4.071328in}{1.437998in}}{\pgfqpoint{4.071328in}{1.445131in}}%
\pgfpathcurveto{\pgfqpoint{4.071328in}{1.452263in}}{\pgfqpoint{4.068494in}{1.459105in}}{\pgfqpoint{4.063450in}{1.464149in}}%
\pgfpathcurveto{\pgfqpoint{4.058407in}{1.469192in}}{\pgfqpoint{4.051565in}{1.472026in}}{\pgfqpoint{4.044432in}{1.472026in}}%
\pgfpathcurveto{\pgfqpoint{4.037299in}{1.472026in}}{\pgfqpoint{4.030458in}{1.469192in}}{\pgfqpoint{4.025414in}{1.464149in}}%
\pgfpathcurveto{\pgfqpoint{4.020370in}{1.459105in}}{\pgfqpoint{4.017536in}{1.452263in}}{\pgfqpoint{4.017536in}{1.445131in}}%
\pgfpathcurveto{\pgfqpoint{4.017536in}{1.437998in}}{\pgfqpoint{4.020370in}{1.431156in}}{\pgfqpoint{4.025414in}{1.426113in}}%
\pgfpathcurveto{\pgfqpoint{4.030458in}{1.421069in}}{\pgfqpoint{4.037299in}{1.418235in}}{\pgfqpoint{4.044432in}{1.418235in}}%
\pgfpathclose%
\pgfusepath{stroke,fill}%
\end{pgfscope}%
\begin{pgfscope}%
\pgfpathrectangle{\pgfqpoint{2.867647in}{0.500000in}}{\pgfqpoint{1.764706in}{1.700000in}}%
\pgfusepath{clip}%
\pgfsetbuttcap%
\pgfsetroundjoin%
\definecolor{currentfill}{rgb}{0.965592,0.726236,0.584384}%
\pgfsetfillcolor{currentfill}%
\pgfsetlinewidth{0.311001pt}%
\definecolor{currentstroke}{rgb}{1.000000,1.000000,1.000000}%
\pgfsetstrokecolor{currentstroke}%
\pgfsetdash{}{0pt}%
\pgfpathmoveto{\pgfqpoint{4.042310in}{1.497998in}}%
\pgfpathcurveto{\pgfqpoint{4.049443in}{1.497998in}}{\pgfqpoint{4.056285in}{1.500832in}}{\pgfqpoint{4.061328in}{1.505875in}}%
\pgfpathcurveto{\pgfqpoint{4.066372in}{1.510919in}}{\pgfqpoint{4.069206in}{1.517761in}}{\pgfqpoint{4.069206in}{1.524894in}}%
\pgfpathcurveto{\pgfqpoint{4.069206in}{1.532026in}}{\pgfqpoint{4.066372in}{1.538868in}}{\pgfqpoint{4.061328in}{1.543912in}}%
\pgfpathcurveto{\pgfqpoint{4.056285in}{1.548955in}}{\pgfqpoint{4.049443in}{1.551789in}}{\pgfqpoint{4.042310in}{1.551789in}}%
\pgfpathcurveto{\pgfqpoint{4.035177in}{1.551789in}}{\pgfqpoint{4.028336in}{1.548955in}}{\pgfqpoint{4.023292in}{1.543912in}}%
\pgfpathcurveto{\pgfqpoint{4.018248in}{1.538868in}}{\pgfqpoint{4.015415in}{1.532026in}}{\pgfqpoint{4.015415in}{1.524894in}}%
\pgfpathcurveto{\pgfqpoint{4.015415in}{1.517761in}}{\pgfqpoint{4.018248in}{1.510919in}}{\pgfqpoint{4.023292in}{1.505875in}}%
\pgfpathcurveto{\pgfqpoint{4.028336in}{1.500832in}}{\pgfqpoint{4.035177in}{1.497998in}}{\pgfqpoint{4.042310in}{1.497998in}}%
\pgfpathclose%
\pgfusepath{stroke,fill}%
\end{pgfscope}%
\begin{pgfscope}%
\pgfpathrectangle{\pgfqpoint{2.867647in}{0.500000in}}{\pgfqpoint{1.764706in}{1.700000in}}%
\pgfusepath{clip}%
\pgfsetbuttcap%
\pgfsetroundjoin%
\definecolor{currentfill}{rgb}{0.975644,0.874038,0.797253}%
\pgfsetfillcolor{currentfill}%
\pgfsetlinewidth{0.311001pt}%
\definecolor{currentstroke}{rgb}{1.000000,1.000000,1.000000}%
\pgfsetstrokecolor{currentstroke}%
\pgfsetdash{}{0pt}%
\pgfpathmoveto{\pgfqpoint{4.241464in}{1.371223in}}%
\pgfpathcurveto{\pgfqpoint{4.248597in}{1.371223in}}{\pgfqpoint{4.255438in}{1.374057in}}{\pgfqpoint{4.260482in}{1.379100in}}%
\pgfpathcurveto{\pgfqpoint{4.265526in}{1.384144in}}{\pgfqpoint{4.268360in}{1.390986in}}{\pgfqpoint{4.268360in}{1.398118in}}%
\pgfpathcurveto{\pgfqpoint{4.268360in}{1.405251in}}{\pgfqpoint{4.265526in}{1.412093in}}{\pgfqpoint{4.260482in}{1.417137in}}%
\pgfpathcurveto{\pgfqpoint{4.255438in}{1.422180in}}{\pgfqpoint{4.248597in}{1.425014in}}{\pgfqpoint{4.241464in}{1.425014in}}%
\pgfpathcurveto{\pgfqpoint{4.234331in}{1.425014in}}{\pgfqpoint{4.227489in}{1.422180in}}{\pgfqpoint{4.222446in}{1.417137in}}%
\pgfpathcurveto{\pgfqpoint{4.217402in}{1.412093in}}{\pgfqpoint{4.214568in}{1.405251in}}{\pgfqpoint{4.214568in}{1.398118in}}%
\pgfpathcurveto{\pgfqpoint{4.214568in}{1.390986in}}{\pgfqpoint{4.217402in}{1.384144in}}{\pgfqpoint{4.222446in}{1.379100in}}%
\pgfpathcurveto{\pgfqpoint{4.227489in}{1.374057in}}{\pgfqpoint{4.234331in}{1.371223in}}{\pgfqpoint{4.241464in}{1.371223in}}%
\pgfpathclose%
\pgfusepath{stroke,fill}%
\end{pgfscope}%
\begin{pgfscope}%
\pgfpathrectangle{\pgfqpoint{2.867647in}{0.500000in}}{\pgfqpoint{1.764706in}{1.700000in}}%
\pgfusepath{clip}%
\pgfsetbuttcap%
\pgfsetroundjoin%
\definecolor{currentfill}{rgb}{0.964032,0.651225,0.493258}%
\pgfsetfillcolor{currentfill}%
\pgfsetlinewidth{0.311001pt}%
\definecolor{currentstroke}{rgb}{1.000000,1.000000,1.000000}%
\pgfsetstrokecolor{currentstroke}%
\pgfsetdash{}{0pt}%
\pgfpathmoveto{\pgfqpoint{4.064212in}{0.868232in}}%
\pgfpathcurveto{\pgfqpoint{4.071345in}{0.868232in}}{\pgfqpoint{4.078186in}{0.871066in}}{\pgfqpoint{4.083230in}{0.876110in}}%
\pgfpathcurveto{\pgfqpoint{4.088274in}{0.881154in}}{\pgfqpoint{4.091107in}{0.887995in}}{\pgfqpoint{4.091107in}{0.895128in}}%
\pgfpathcurveto{\pgfqpoint{4.091107in}{0.902261in}}{\pgfqpoint{4.088274in}{0.909103in}}{\pgfqpoint{4.083230in}{0.914146in}}%
\pgfpathcurveto{\pgfqpoint{4.078186in}{0.919190in}}{\pgfqpoint{4.071345in}{0.922024in}}{\pgfqpoint{4.064212in}{0.922024in}}%
\pgfpathcurveto{\pgfqpoint{4.057079in}{0.922024in}}{\pgfqpoint{4.050237in}{0.919190in}}{\pgfqpoint{4.045194in}{0.914146in}}%
\pgfpathcurveto{\pgfqpoint{4.040150in}{0.909103in}}{\pgfqpoint{4.037316in}{0.902261in}}{\pgfqpoint{4.037316in}{0.895128in}}%
\pgfpathcurveto{\pgfqpoint{4.037316in}{0.887995in}}{\pgfqpoint{4.040150in}{0.881154in}}{\pgfqpoint{4.045194in}{0.876110in}}%
\pgfpathcurveto{\pgfqpoint{4.050237in}{0.871066in}}{\pgfqpoint{4.057079in}{0.868232in}}{\pgfqpoint{4.064212in}{0.868232in}}%
\pgfpathclose%
\pgfusepath{stroke,fill}%
\end{pgfscope}%
\begin{pgfscope}%
\pgfpathrectangle{\pgfqpoint{2.867647in}{0.500000in}}{\pgfqpoint{1.764706in}{1.700000in}}%
\pgfusepath{clip}%
\pgfsetbuttcap%
\pgfsetroundjoin%
\definecolor{currentfill}{rgb}{0.978376,0.897317,0.831308}%
\pgfsetfillcolor{currentfill}%
\pgfsetlinewidth{0.311001pt}%
\definecolor{currentstroke}{rgb}{1.000000,1.000000,1.000000}%
\pgfsetstrokecolor{currentstroke}%
\pgfsetdash{}{0pt}%
\pgfpathmoveto{\pgfqpoint{4.185434in}{1.526793in}}%
\pgfpathcurveto{\pgfqpoint{4.192567in}{1.526793in}}{\pgfqpoint{4.199409in}{1.529627in}}{\pgfqpoint{4.204452in}{1.534671in}}%
\pgfpathcurveto{\pgfqpoint{4.209496in}{1.539714in}}{\pgfqpoint{4.212330in}{1.546556in}}{\pgfqpoint{4.212330in}{1.553689in}}%
\pgfpathcurveto{\pgfqpoint{4.212330in}{1.560822in}}{\pgfqpoint{4.209496in}{1.567663in}}{\pgfqpoint{4.204452in}{1.572707in}}%
\pgfpathcurveto{\pgfqpoint{4.199409in}{1.577751in}}{\pgfqpoint{4.192567in}{1.580585in}}{\pgfqpoint{4.185434in}{1.580585in}}%
\pgfpathcurveto{\pgfqpoint{4.178301in}{1.580585in}}{\pgfqpoint{4.171460in}{1.577751in}}{\pgfqpoint{4.166416in}{1.572707in}}%
\pgfpathcurveto{\pgfqpoint{4.161372in}{1.567663in}}{\pgfqpoint{4.158538in}{1.560822in}}{\pgfqpoint{4.158538in}{1.553689in}}%
\pgfpathcurveto{\pgfqpoint{4.158538in}{1.546556in}}{\pgfqpoint{4.161372in}{1.539714in}}{\pgfqpoint{4.166416in}{1.534671in}}%
\pgfpathcurveto{\pgfqpoint{4.171460in}{1.529627in}}{\pgfqpoint{4.178301in}{1.526793in}}{\pgfqpoint{4.185434in}{1.526793in}}%
\pgfpathclose%
\pgfusepath{stroke,fill}%
\end{pgfscope}%
\begin{pgfscope}%
\pgfpathrectangle{\pgfqpoint{2.867647in}{0.500000in}}{\pgfqpoint{1.764706in}{1.700000in}}%
\pgfusepath{clip}%
\pgfsetbuttcap%
\pgfsetroundjoin%
\definecolor{currentfill}{rgb}{0.972726,0.844889,0.754401}%
\pgfsetfillcolor{currentfill}%
\pgfsetlinewidth{0.311001pt}%
\definecolor{currentstroke}{rgb}{1.000000,1.000000,1.000000}%
\pgfsetstrokecolor{currentstroke}%
\pgfsetdash{}{0pt}%
\pgfpathmoveto{\pgfqpoint{4.116695in}{1.270399in}}%
\pgfpathcurveto{\pgfqpoint{4.123828in}{1.270399in}}{\pgfqpoint{4.130669in}{1.273233in}}{\pgfqpoint{4.135713in}{1.278276in}}%
\pgfpathcurveto{\pgfqpoint{4.140757in}{1.283320in}}{\pgfqpoint{4.143591in}{1.290162in}}{\pgfqpoint{4.143591in}{1.297295in}}%
\pgfpathcurveto{\pgfqpoint{4.143591in}{1.304427in}}{\pgfqpoint{4.140757in}{1.311269in}}{\pgfqpoint{4.135713in}{1.316313in}}%
\pgfpathcurveto{\pgfqpoint{4.130669in}{1.321356in}}{\pgfqpoint{4.123828in}{1.324190in}}{\pgfqpoint{4.116695in}{1.324190in}}%
\pgfpathcurveto{\pgfqpoint{4.109562in}{1.324190in}}{\pgfqpoint{4.102720in}{1.321356in}}{\pgfqpoint{4.097677in}{1.316313in}}%
\pgfpathcurveto{\pgfqpoint{4.092633in}{1.311269in}}{\pgfqpoint{4.089799in}{1.304427in}}{\pgfqpoint{4.089799in}{1.297295in}}%
\pgfpathcurveto{\pgfqpoint{4.089799in}{1.290162in}}{\pgfqpoint{4.092633in}{1.283320in}}{\pgfqpoint{4.097677in}{1.278276in}}%
\pgfpathcurveto{\pgfqpoint{4.102720in}{1.273233in}}{\pgfqpoint{4.109562in}{1.270399in}}{\pgfqpoint{4.116695in}{1.270399in}}%
\pgfpathclose%
\pgfusepath{stroke,fill}%
\end{pgfscope}%
\begin{pgfscope}%
\pgfpathrectangle{\pgfqpoint{2.867647in}{0.500000in}}{\pgfqpoint{1.764706in}{1.700000in}}%
\pgfusepath{clip}%
\pgfsetbuttcap%
\pgfsetroundjoin%
\definecolor{currentfill}{rgb}{0.953126,0.456614,0.312398}%
\pgfsetfillcolor{currentfill}%
\pgfsetlinewidth{0.311001pt}%
\definecolor{currentstroke}{rgb}{1.000000,1.000000,1.000000}%
\pgfsetstrokecolor{currentstroke}%
\pgfsetdash{}{0pt}%
\pgfpathmoveto{\pgfqpoint{4.318277in}{1.555247in}}%
\pgfpathcurveto{\pgfqpoint{4.325410in}{1.555247in}}{\pgfqpoint{4.332251in}{1.558081in}}{\pgfqpoint{4.337295in}{1.563125in}}%
\pgfpathcurveto{\pgfqpoint{4.342339in}{1.568168in}}{\pgfqpoint{4.345173in}{1.575010in}}{\pgfqpoint{4.345173in}{1.582143in}}%
\pgfpathcurveto{\pgfqpoint{4.345173in}{1.589276in}}{\pgfqpoint{4.342339in}{1.596117in}}{\pgfqpoint{4.337295in}{1.601161in}}%
\pgfpathcurveto{\pgfqpoint{4.332251in}{1.606205in}}{\pgfqpoint{4.325410in}{1.609038in}}{\pgfqpoint{4.318277in}{1.609038in}}%
\pgfpathcurveto{\pgfqpoint{4.311144in}{1.609038in}}{\pgfqpoint{4.304302in}{1.606205in}}{\pgfqpoint{4.299259in}{1.601161in}}%
\pgfpathcurveto{\pgfqpoint{4.294215in}{1.596117in}}{\pgfqpoint{4.291381in}{1.589276in}}{\pgfqpoint{4.291381in}{1.582143in}}%
\pgfpathcurveto{\pgfqpoint{4.291381in}{1.575010in}}{\pgfqpoint{4.294215in}{1.568168in}}{\pgfqpoint{4.299259in}{1.563125in}}%
\pgfpathcurveto{\pgfqpoint{4.304302in}{1.558081in}}{\pgfqpoint{4.311144in}{1.555247in}}{\pgfqpoint{4.318277in}{1.555247in}}%
\pgfpathclose%
\pgfusepath{stroke,fill}%
\end{pgfscope}%
\begin{pgfscope}%
\pgfpathrectangle{\pgfqpoint{2.867647in}{0.500000in}}{\pgfqpoint{1.764706in}{1.700000in}}%
\pgfusepath{clip}%
\pgfsetbuttcap%
\pgfsetroundjoin%
\definecolor{currentfill}{rgb}{0.976287,0.879862,0.805788}%
\pgfsetfillcolor{currentfill}%
\pgfsetlinewidth{0.311001pt}%
\definecolor{currentstroke}{rgb}{1.000000,1.000000,1.000000}%
\pgfsetstrokecolor{currentstroke}%
\pgfsetdash{}{0pt}%
\pgfpathmoveto{\pgfqpoint{4.115904in}{1.632864in}}%
\pgfpathcurveto{\pgfqpoint{4.123037in}{1.632864in}}{\pgfqpoint{4.129878in}{1.635698in}}{\pgfqpoint{4.134922in}{1.640742in}}%
\pgfpathcurveto{\pgfqpoint{4.139966in}{1.645785in}}{\pgfqpoint{4.142800in}{1.652627in}}{\pgfqpoint{4.142800in}{1.659760in}}%
\pgfpathcurveto{\pgfqpoint{4.142800in}{1.666893in}}{\pgfqpoint{4.139966in}{1.673734in}}{\pgfqpoint{4.134922in}{1.678778in}}%
\pgfpathcurveto{\pgfqpoint{4.129878in}{1.683822in}}{\pgfqpoint{4.123037in}{1.686656in}}{\pgfqpoint{4.115904in}{1.686656in}}%
\pgfpathcurveto{\pgfqpoint{4.108771in}{1.686656in}}{\pgfqpoint{4.101930in}{1.683822in}}{\pgfqpoint{4.096886in}{1.678778in}}%
\pgfpathcurveto{\pgfqpoint{4.091842in}{1.673734in}}{\pgfqpoint{4.089008in}{1.666893in}}{\pgfqpoint{4.089008in}{1.659760in}}%
\pgfpathcurveto{\pgfqpoint{4.089008in}{1.652627in}}{\pgfqpoint{4.091842in}{1.645785in}}{\pgfqpoint{4.096886in}{1.640742in}}%
\pgfpathcurveto{\pgfqpoint{4.101930in}{1.635698in}}{\pgfqpoint{4.108771in}{1.632864in}}{\pgfqpoint{4.115904in}{1.632864in}}%
\pgfpathclose%
\pgfusepath{stroke,fill}%
\end{pgfscope}%
\begin{pgfscope}%
\pgfpathrectangle{\pgfqpoint{2.867647in}{0.500000in}}{\pgfqpoint{1.764706in}{1.700000in}}%
\pgfusepath{clip}%
\pgfsetbuttcap%
\pgfsetroundjoin%
\definecolor{currentfill}{rgb}{0.977657,0.891500,0.822809}%
\pgfsetfillcolor{currentfill}%
\pgfsetlinewidth{0.311001pt}%
\definecolor{currentstroke}{rgb}{1.000000,1.000000,1.000000}%
\pgfsetstrokecolor{currentstroke}%
\pgfsetdash{}{0pt}%
\pgfpathmoveto{\pgfqpoint{4.146966in}{1.295065in}}%
\pgfpathcurveto{\pgfqpoint{4.154099in}{1.295065in}}{\pgfqpoint{4.160941in}{1.297899in}}{\pgfqpoint{4.165985in}{1.302942in}}%
\pgfpathcurveto{\pgfqpoint{4.171028in}{1.307986in}}{\pgfqpoint{4.173862in}{1.314828in}}{\pgfqpoint{4.173862in}{1.321960in}}%
\pgfpathcurveto{\pgfqpoint{4.173862in}{1.329093in}}{\pgfqpoint{4.171028in}{1.335935in}}{\pgfqpoint{4.165985in}{1.340978in}}%
\pgfpathcurveto{\pgfqpoint{4.160941in}{1.346022in}}{\pgfqpoint{4.154099in}{1.348856in}}{\pgfqpoint{4.146966in}{1.348856in}}%
\pgfpathcurveto{\pgfqpoint{4.139834in}{1.348856in}}{\pgfqpoint{4.132992in}{1.346022in}}{\pgfqpoint{4.127948in}{1.340978in}}%
\pgfpathcurveto{\pgfqpoint{4.122905in}{1.335935in}}{\pgfqpoint{4.120071in}{1.329093in}}{\pgfqpoint{4.120071in}{1.321960in}}%
\pgfpathcurveto{\pgfqpoint{4.120071in}{1.314828in}}{\pgfqpoint{4.122905in}{1.307986in}}{\pgfqpoint{4.127948in}{1.302942in}}%
\pgfpathcurveto{\pgfqpoint{4.132992in}{1.297899in}}{\pgfqpoint{4.139834in}{1.295065in}}{\pgfqpoint{4.146966in}{1.295065in}}%
\pgfpathclose%
\pgfusepath{stroke,fill}%
\end{pgfscope}%
\begin{pgfscope}%
\pgfpathrectangle{\pgfqpoint{2.867647in}{0.500000in}}{\pgfqpoint{1.764706in}{1.700000in}}%
\pgfusepath{clip}%
\pgfsetbuttcap%
\pgfsetroundjoin%
\definecolor{currentfill}{rgb}{0.970718,0.821518,0.719872}%
\pgfsetfillcolor{currentfill}%
\pgfsetlinewidth{0.311001pt}%
\definecolor{currentstroke}{rgb}{1.000000,1.000000,1.000000}%
\pgfsetstrokecolor{currentstroke}%
\pgfsetdash{}{0pt}%
\pgfpathmoveto{\pgfqpoint{4.176249in}{1.006661in}}%
\pgfpathcurveto{\pgfqpoint{4.183382in}{1.006661in}}{\pgfqpoint{4.190224in}{1.009495in}}{\pgfqpoint{4.195267in}{1.014538in}}%
\pgfpathcurveto{\pgfqpoint{4.200311in}{1.019582in}}{\pgfqpoint{4.203145in}{1.026424in}}{\pgfqpoint{4.203145in}{1.033557in}}%
\pgfpathcurveto{\pgfqpoint{4.203145in}{1.040689in}}{\pgfqpoint{4.200311in}{1.047531in}}{\pgfqpoint{4.195267in}{1.052575in}}%
\pgfpathcurveto{\pgfqpoint{4.190224in}{1.057618in}}{\pgfqpoint{4.183382in}{1.060452in}}{\pgfqpoint{4.176249in}{1.060452in}}%
\pgfpathcurveto{\pgfqpoint{4.169116in}{1.060452in}}{\pgfqpoint{4.162275in}{1.057618in}}{\pgfqpoint{4.157231in}{1.052575in}}%
\pgfpathcurveto{\pgfqpoint{4.152187in}{1.047531in}}{\pgfqpoint{4.149353in}{1.040689in}}{\pgfqpoint{4.149353in}{1.033557in}}%
\pgfpathcurveto{\pgfqpoint{4.149353in}{1.026424in}}{\pgfqpoint{4.152187in}{1.019582in}}{\pgfqpoint{4.157231in}{1.014538in}}%
\pgfpathcurveto{\pgfqpoint{4.162275in}{1.009495in}}{\pgfqpoint{4.169116in}{1.006661in}}{\pgfqpoint{4.176249in}{1.006661in}}%
\pgfpathclose%
\pgfusepath{stroke,fill}%
\end{pgfscope}%
\begin{pgfscope}%
\pgfpathrectangle{\pgfqpoint{2.867647in}{0.500000in}}{\pgfqpoint{1.764706in}{1.700000in}}%
\pgfusepath{clip}%
\pgfsetbuttcap%
\pgfsetroundjoin%
\definecolor{currentfill}{rgb}{0.979891,0.908948,0.848279}%
\pgfsetfillcolor{currentfill}%
\pgfsetlinewidth{0.311001pt}%
\definecolor{currentstroke}{rgb}{1.000000,1.000000,1.000000}%
\pgfsetstrokecolor{currentstroke}%
\pgfsetdash{}{0pt}%
\pgfpathmoveto{\pgfqpoint{4.216997in}{1.332568in}}%
\pgfpathcurveto{\pgfqpoint{4.224129in}{1.332568in}}{\pgfqpoint{4.230971in}{1.335402in}}{\pgfqpoint{4.236015in}{1.340446in}}%
\pgfpathcurveto{\pgfqpoint{4.241058in}{1.345489in}}{\pgfqpoint{4.243892in}{1.352331in}}{\pgfqpoint{4.243892in}{1.359464in}}%
\pgfpathcurveto{\pgfqpoint{4.243892in}{1.366597in}}{\pgfqpoint{4.241058in}{1.373438in}}{\pgfqpoint{4.236015in}{1.378482in}}%
\pgfpathcurveto{\pgfqpoint{4.230971in}{1.383526in}}{\pgfqpoint{4.224129in}{1.386360in}}{\pgfqpoint{4.216997in}{1.386360in}}%
\pgfpathcurveto{\pgfqpoint{4.209864in}{1.386360in}}{\pgfqpoint{4.203022in}{1.383526in}}{\pgfqpoint{4.197978in}{1.378482in}}%
\pgfpathcurveto{\pgfqpoint{4.192935in}{1.373438in}}{\pgfqpoint{4.190101in}{1.366597in}}{\pgfqpoint{4.190101in}{1.359464in}}%
\pgfpathcurveto{\pgfqpoint{4.190101in}{1.352331in}}{\pgfqpoint{4.192935in}{1.345489in}}{\pgfqpoint{4.197978in}{1.340446in}}%
\pgfpathcurveto{\pgfqpoint{4.203022in}{1.335402in}}{\pgfqpoint{4.209864in}{1.332568in}}{\pgfqpoint{4.216997in}{1.332568in}}%
\pgfpathclose%
\pgfusepath{stroke,fill}%
\end{pgfscope}%
\begin{pgfscope}%
\pgfpathrectangle{\pgfqpoint{2.867647in}{0.500000in}}{\pgfqpoint{1.764706in}{1.700000in}}%
\pgfusepath{clip}%
\pgfsetbuttcap%
\pgfsetroundjoin%
\definecolor{currentfill}{rgb}{0.981377,0.920617,0.865369}%
\pgfsetfillcolor{currentfill}%
\pgfsetlinewidth{0.311001pt}%
\definecolor{currentstroke}{rgb}{1.000000,1.000000,1.000000}%
\pgfsetstrokecolor{currentstroke}%
\pgfsetdash{}{0pt}%
\pgfpathmoveto{\pgfqpoint{4.183228in}{1.360904in}}%
\pgfpathcurveto{\pgfqpoint{4.190361in}{1.360904in}}{\pgfqpoint{4.197202in}{1.363737in}}{\pgfqpoint{4.202246in}{1.368781in}}%
\pgfpathcurveto{\pgfqpoint{4.207290in}{1.373825in}}{\pgfqpoint{4.210124in}{1.380666in}}{\pgfqpoint{4.210124in}{1.387799in}}%
\pgfpathcurveto{\pgfqpoint{4.210124in}{1.394932in}}{\pgfqpoint{4.207290in}{1.401774in}}{\pgfqpoint{4.202246in}{1.406817in}}%
\pgfpathcurveto{\pgfqpoint{4.197202in}{1.411861in}}{\pgfqpoint{4.190361in}{1.414695in}}{\pgfqpoint{4.183228in}{1.414695in}}%
\pgfpathcurveto{\pgfqpoint{4.176095in}{1.414695in}}{\pgfqpoint{4.169254in}{1.411861in}}{\pgfqpoint{4.164210in}{1.406817in}}%
\pgfpathcurveto{\pgfqpoint{4.159166in}{1.401774in}}{\pgfqpoint{4.156332in}{1.394932in}}{\pgfqpoint{4.156332in}{1.387799in}}%
\pgfpathcurveto{\pgfqpoint{4.156332in}{1.380666in}}{\pgfqpoint{4.159166in}{1.373825in}}{\pgfqpoint{4.164210in}{1.368781in}}%
\pgfpathcurveto{\pgfqpoint{4.169254in}{1.363737in}}{\pgfqpoint{4.176095in}{1.360904in}}{\pgfqpoint{4.183228in}{1.360904in}}%
\pgfpathclose%
\pgfusepath{stroke,fill}%
\end{pgfscope}%
\begin{pgfscope}%
\pgfpathrectangle{\pgfqpoint{2.867647in}{0.500000in}}{\pgfqpoint{1.764706in}{1.700000in}}%
\pgfusepath{clip}%
\pgfsetbuttcap%
\pgfsetroundjoin%
\definecolor{currentfill}{rgb}{0.979891,0.908948,0.848279}%
\pgfsetfillcolor{currentfill}%
\pgfsetlinewidth{0.311001pt}%
\definecolor{currentstroke}{rgb}{1.000000,1.000000,1.000000}%
\pgfsetstrokecolor{currentstroke}%
\pgfsetdash{}{0pt}%
\pgfpathmoveto{\pgfqpoint{4.183990in}{1.138415in}}%
\pgfpathcurveto{\pgfqpoint{4.191123in}{1.138415in}}{\pgfqpoint{4.197964in}{1.141249in}}{\pgfqpoint{4.203008in}{1.146292in}}%
\pgfpathcurveto{\pgfqpoint{4.208052in}{1.151336in}}{\pgfqpoint{4.210886in}{1.158178in}}{\pgfqpoint{4.210886in}{1.165311in}}%
\pgfpathcurveto{\pgfqpoint{4.210886in}{1.172443in}}{\pgfqpoint{4.208052in}{1.179285in}}{\pgfqpoint{4.203008in}{1.184329in}}%
\pgfpathcurveto{\pgfqpoint{4.197964in}{1.189372in}}{\pgfqpoint{4.191123in}{1.192206in}}{\pgfqpoint{4.183990in}{1.192206in}}%
\pgfpathcurveto{\pgfqpoint{4.176857in}{1.192206in}}{\pgfqpoint{4.170016in}{1.189372in}}{\pgfqpoint{4.164972in}{1.184329in}}%
\pgfpathcurveto{\pgfqpoint{4.159928in}{1.179285in}}{\pgfqpoint{4.157094in}{1.172443in}}{\pgfqpoint{4.157094in}{1.165311in}}%
\pgfpathcurveto{\pgfqpoint{4.157094in}{1.158178in}}{\pgfqpoint{4.159928in}{1.151336in}}{\pgfqpoint{4.164972in}{1.146292in}}%
\pgfpathcurveto{\pgfqpoint{4.170016in}{1.141249in}}{\pgfqpoint{4.176857in}{1.138415in}}{\pgfqpoint{4.183990in}{1.138415in}}%
\pgfpathclose%
\pgfusepath{stroke,fill}%
\end{pgfscope}%
\begin{pgfscope}%
\pgfpathrectangle{\pgfqpoint{2.867647in}{0.500000in}}{\pgfqpoint{1.764706in}{1.700000in}}%
\pgfusepath{clip}%
\pgfsetbuttcap%
\pgfsetroundjoin%
\definecolor{currentfill}{rgb}{0.962018,0.586477,0.424918}%
\pgfsetfillcolor{currentfill}%
\pgfsetlinewidth{0.311001pt}%
\definecolor{currentstroke}{rgb}{1.000000,1.000000,1.000000}%
\pgfsetstrokecolor{currentstroke}%
\pgfsetdash{}{0pt}%
\pgfpathmoveto{\pgfqpoint{4.038013in}{0.844310in}}%
\pgfpathcurveto{\pgfqpoint{4.045146in}{0.844310in}}{\pgfqpoint{4.051988in}{0.847144in}}{\pgfqpoint{4.057031in}{0.852188in}}%
\pgfpathcurveto{\pgfqpoint{4.062075in}{0.857231in}}{\pgfqpoint{4.064909in}{0.864073in}}{\pgfqpoint{4.064909in}{0.871206in}}%
\pgfpathcurveto{\pgfqpoint{4.064909in}{0.878339in}}{\pgfqpoint{4.062075in}{0.885180in}}{\pgfqpoint{4.057031in}{0.890224in}}%
\pgfpathcurveto{\pgfqpoint{4.051988in}{0.895268in}}{\pgfqpoint{4.045146in}{0.898101in}}{\pgfqpoint{4.038013in}{0.898101in}}%
\pgfpathcurveto{\pgfqpoint{4.030881in}{0.898101in}}{\pgfqpoint{4.024039in}{0.895268in}}{\pgfqpoint{4.018995in}{0.890224in}}%
\pgfpathcurveto{\pgfqpoint{4.013952in}{0.885180in}}{\pgfqpoint{4.011118in}{0.878339in}}{\pgfqpoint{4.011118in}{0.871206in}}%
\pgfpathcurveto{\pgfqpoint{4.011118in}{0.864073in}}{\pgfqpoint{4.013952in}{0.857231in}}{\pgfqpoint{4.018995in}{0.852188in}}%
\pgfpathcurveto{\pgfqpoint{4.024039in}{0.847144in}}{\pgfqpoint{4.030881in}{0.844310in}}{\pgfqpoint{4.038013in}{0.844310in}}%
\pgfpathclose%
\pgfusepath{stroke,fill}%
\end{pgfscope}%
\begin{pgfscope}%
\pgfpathrectangle{\pgfqpoint{2.867647in}{0.500000in}}{\pgfqpoint{1.764706in}{1.700000in}}%
\pgfusepath{clip}%
\pgfsetbuttcap%
\pgfsetroundjoin%
\definecolor{currentfill}{rgb}{0.961433,0.573272,0.412036}%
\pgfsetfillcolor{currentfill}%
\pgfsetlinewidth{0.311001pt}%
\definecolor{currentstroke}{rgb}{1.000000,1.000000,1.000000}%
\pgfsetstrokecolor{currentstroke}%
\pgfsetdash{}{0pt}%
\pgfpathmoveto{\pgfqpoint{4.013218in}{1.160515in}}%
\pgfpathcurveto{\pgfqpoint{4.020350in}{1.160515in}}{\pgfqpoint{4.027192in}{1.163349in}}{\pgfqpoint{4.032236in}{1.168393in}}%
\pgfpathcurveto{\pgfqpoint{4.037279in}{1.173436in}}{\pgfqpoint{4.040113in}{1.180278in}}{\pgfqpoint{4.040113in}{1.187411in}}%
\pgfpathcurveto{\pgfqpoint{4.040113in}{1.194544in}}{\pgfqpoint{4.037279in}{1.201385in}}{\pgfqpoint{4.032236in}{1.206429in}}%
\pgfpathcurveto{\pgfqpoint{4.027192in}{1.211473in}}{\pgfqpoint{4.020350in}{1.214307in}}{\pgfqpoint{4.013218in}{1.214307in}}%
\pgfpathcurveto{\pgfqpoint{4.006085in}{1.214307in}}{\pgfqpoint{3.999243in}{1.211473in}}{\pgfqpoint{3.994199in}{1.206429in}}%
\pgfpathcurveto{\pgfqpoint{3.989156in}{1.201385in}}{\pgfqpoint{3.986322in}{1.194544in}}{\pgfqpoint{3.986322in}{1.187411in}}%
\pgfpathcurveto{\pgfqpoint{3.986322in}{1.180278in}}{\pgfqpoint{3.989156in}{1.173436in}}{\pgfqpoint{3.994199in}{1.168393in}}%
\pgfpathcurveto{\pgfqpoint{3.999243in}{1.163349in}}{\pgfqpoint{4.006085in}{1.160515in}}{\pgfqpoint{4.013218in}{1.160515in}}%
\pgfpathclose%
\pgfusepath{stroke,fill}%
\end{pgfscope}%
\begin{pgfscope}%
\pgfpathrectangle{\pgfqpoint{2.867647in}{0.500000in}}{\pgfqpoint{1.764706in}{1.700000in}}%
\pgfusepath{clip}%
\pgfsetbuttcap%
\pgfsetroundjoin%
\definecolor{currentfill}{rgb}{0.976961,0.885681,0.814303}%
\pgfsetfillcolor{currentfill}%
\pgfsetlinewidth{0.311001pt}%
\definecolor{currentstroke}{rgb}{1.000000,1.000000,1.000000}%
\pgfsetstrokecolor{currentstroke}%
\pgfsetdash{}{0pt}%
\pgfpathmoveto{\pgfqpoint{4.109427in}{1.114615in}}%
\pgfpathcurveto{\pgfqpoint{4.116560in}{1.114615in}}{\pgfqpoint{4.123401in}{1.117449in}}{\pgfqpoint{4.128445in}{1.122493in}}%
\pgfpathcurveto{\pgfqpoint{4.133489in}{1.127536in}}{\pgfqpoint{4.136322in}{1.134378in}}{\pgfqpoint{4.136322in}{1.141511in}}%
\pgfpathcurveto{\pgfqpoint{4.136322in}{1.148644in}}{\pgfqpoint{4.133489in}{1.155485in}}{\pgfqpoint{4.128445in}{1.160529in}}%
\pgfpathcurveto{\pgfqpoint{4.123401in}{1.165572in}}{\pgfqpoint{4.116560in}{1.168406in}}{\pgfqpoint{4.109427in}{1.168406in}}%
\pgfpathcurveto{\pgfqpoint{4.102294in}{1.168406in}}{\pgfqpoint{4.095452in}{1.165572in}}{\pgfqpoint{4.090409in}{1.160529in}}%
\pgfpathcurveto{\pgfqpoint{4.085365in}{1.155485in}}{\pgfqpoint{4.082531in}{1.148644in}}{\pgfqpoint{4.082531in}{1.141511in}}%
\pgfpathcurveto{\pgfqpoint{4.082531in}{1.134378in}}{\pgfqpoint{4.085365in}{1.127536in}}{\pgfqpoint{4.090409in}{1.122493in}}%
\pgfpathcurveto{\pgfqpoint{4.095452in}{1.117449in}}{\pgfqpoint{4.102294in}{1.114615in}}{\pgfqpoint{4.109427in}{1.114615in}}%
\pgfpathclose%
\pgfusepath{stroke,fill}%
\end{pgfscope}%
\begin{pgfscope}%
\pgfpathrectangle{\pgfqpoint{2.867647in}{0.500000in}}{\pgfqpoint{1.764706in}{1.700000in}}%
\pgfusepath{clip}%
\pgfsetbuttcap%
\pgfsetroundjoin%
\definecolor{currentfill}{rgb}{0.966812,0.762584,0.633643}%
\pgfsetfillcolor{currentfill}%
\pgfsetlinewidth{0.311001pt}%
\definecolor{currentstroke}{rgb}{1.000000,1.000000,1.000000}%
\pgfsetstrokecolor{currentstroke}%
\pgfsetdash{}{0pt}%
\pgfpathmoveto{\pgfqpoint{4.034183in}{1.076044in}}%
\pgfpathcurveto{\pgfqpoint{4.041316in}{1.076044in}}{\pgfqpoint{4.048157in}{1.078878in}}{\pgfqpoint{4.053201in}{1.083922in}}%
\pgfpathcurveto{\pgfqpoint{4.058245in}{1.088966in}}{\pgfqpoint{4.061078in}{1.095807in}}{\pgfqpoint{4.061078in}{1.102940in}}%
\pgfpathcurveto{\pgfqpoint{4.061078in}{1.110073in}}{\pgfqpoint{4.058245in}{1.116915in}}{\pgfqpoint{4.053201in}{1.121958in}}%
\pgfpathcurveto{\pgfqpoint{4.048157in}{1.127002in}}{\pgfqpoint{4.041316in}{1.129836in}}{\pgfqpoint{4.034183in}{1.129836in}}%
\pgfpathcurveto{\pgfqpoint{4.027050in}{1.129836in}}{\pgfqpoint{4.020208in}{1.127002in}}{\pgfqpoint{4.015165in}{1.121958in}}%
\pgfpathcurveto{\pgfqpoint{4.010121in}{1.116915in}}{\pgfqpoint{4.007287in}{1.110073in}}{\pgfqpoint{4.007287in}{1.102940in}}%
\pgfpathcurveto{\pgfqpoint{4.007287in}{1.095807in}}{\pgfqpoint{4.010121in}{1.088966in}}{\pgfqpoint{4.015165in}{1.083922in}}%
\pgfpathcurveto{\pgfqpoint{4.020208in}{1.078878in}}{\pgfqpoint{4.027050in}{1.076044in}}{\pgfqpoint{4.034183in}{1.076044in}}%
\pgfpathclose%
\pgfusepath{stroke,fill}%
\end{pgfscope}%
\begin{pgfscope}%
\pgfpathrectangle{\pgfqpoint{2.867647in}{0.500000in}}{\pgfqpoint{1.764706in}{1.700000in}}%
\pgfusepath{clip}%
\pgfsetbuttcap%
\pgfsetroundjoin%
\definecolor{currentfill}{rgb}{0.967735,0.780441,0.659127}%
\pgfsetfillcolor{currentfill}%
\pgfsetlinewidth{0.311001pt}%
\definecolor{currentstroke}{rgb}{1.000000,1.000000,1.000000}%
\pgfsetstrokecolor{currentstroke}%
\pgfsetdash{}{0pt}%
\pgfpathmoveto{\pgfqpoint{4.126847in}{1.705365in}}%
\pgfpathcurveto{\pgfqpoint{4.133980in}{1.705365in}}{\pgfqpoint{4.140822in}{1.708199in}}{\pgfqpoint{4.145865in}{1.713243in}}%
\pgfpathcurveto{\pgfqpoint{4.150909in}{1.718287in}}{\pgfqpoint{4.153743in}{1.725128in}}{\pgfqpoint{4.153743in}{1.732261in}}%
\pgfpathcurveto{\pgfqpoint{4.153743in}{1.739394in}}{\pgfqpoint{4.150909in}{1.746236in}}{\pgfqpoint{4.145865in}{1.751279in}}%
\pgfpathcurveto{\pgfqpoint{4.140822in}{1.756323in}}{\pgfqpoint{4.133980in}{1.759157in}}{\pgfqpoint{4.126847in}{1.759157in}}%
\pgfpathcurveto{\pgfqpoint{4.119714in}{1.759157in}}{\pgfqpoint{4.112873in}{1.756323in}}{\pgfqpoint{4.107829in}{1.751279in}}%
\pgfpathcurveto{\pgfqpoint{4.102785in}{1.746236in}}{\pgfqpoint{4.099951in}{1.739394in}}{\pgfqpoint{4.099951in}{1.732261in}}%
\pgfpathcurveto{\pgfqpoint{4.099951in}{1.725128in}}{\pgfqpoint{4.102785in}{1.718287in}}{\pgfqpoint{4.107829in}{1.713243in}}%
\pgfpathcurveto{\pgfqpoint{4.112873in}{1.708199in}}{\pgfqpoint{4.119714in}{1.705365in}}{\pgfqpoint{4.126847in}{1.705365in}}%
\pgfpathclose%
\pgfusepath{stroke,fill}%
\end{pgfscope}%
\begin{pgfscope}%
\pgfpathrectangle{\pgfqpoint{2.867647in}{0.500000in}}{\pgfqpoint{1.764706in}{1.700000in}}%
\pgfusepath{clip}%
\pgfsetbuttcap%
\pgfsetroundjoin%
\definecolor{currentfill}{rgb}{0.891169,0.211218,0.255359}%
\pgfsetfillcolor{currentfill}%
\pgfsetlinewidth{0.311001pt}%
\definecolor{currentstroke}{rgb}{1.000000,1.000000,1.000000}%
\pgfsetstrokecolor{currentstroke}%
\pgfsetdash{}{0pt}%
\pgfpathmoveto{\pgfqpoint{3.931396in}{1.886020in}}%
\pgfpathcurveto{\pgfqpoint{3.938529in}{1.886020in}}{\pgfqpoint{3.945371in}{1.888854in}}{\pgfqpoint{3.950415in}{1.893898in}}%
\pgfpathcurveto{\pgfqpoint{3.955458in}{1.898942in}}{\pgfqpoint{3.958292in}{1.905783in}}{\pgfqpoint{3.958292in}{1.912916in}}%
\pgfpathcurveto{\pgfqpoint{3.958292in}{1.920049in}}{\pgfqpoint{3.955458in}{1.926891in}}{\pgfqpoint{3.950415in}{1.931934in}}%
\pgfpathcurveto{\pgfqpoint{3.945371in}{1.936978in}}{\pgfqpoint{3.938529in}{1.939812in}}{\pgfqpoint{3.931396in}{1.939812in}}%
\pgfpathcurveto{\pgfqpoint{3.924264in}{1.939812in}}{\pgfqpoint{3.917422in}{1.936978in}}{\pgfqpoint{3.912378in}{1.931934in}}%
\pgfpathcurveto{\pgfqpoint{3.907335in}{1.926891in}}{\pgfqpoint{3.904501in}{1.920049in}}{\pgfqpoint{3.904501in}{1.912916in}}%
\pgfpathcurveto{\pgfqpoint{3.904501in}{1.905783in}}{\pgfqpoint{3.907335in}{1.898942in}}{\pgfqpoint{3.912378in}{1.893898in}}%
\pgfpathcurveto{\pgfqpoint{3.917422in}{1.888854in}}{\pgfqpoint{3.924264in}{1.886020in}}{\pgfqpoint{3.931396in}{1.886020in}}%
\pgfpathclose%
\pgfusepath{stroke,fill}%
\end{pgfscope}%
\begin{pgfscope}%
\pgfpathrectangle{\pgfqpoint{2.867647in}{0.500000in}}{\pgfqpoint{1.764706in}{1.700000in}}%
\pgfusepath{clip}%
\pgfsetbuttcap%
\pgfsetroundjoin%
\definecolor{currentfill}{rgb}{0.965928,0.738443,0.600540}%
\pgfsetfillcolor{currentfill}%
\pgfsetlinewidth{0.311001pt}%
\definecolor{currentstroke}{rgb}{1.000000,1.000000,1.000000}%
\pgfsetstrokecolor{currentstroke}%
\pgfsetdash{}{0pt}%
\pgfpathmoveto{\pgfqpoint{4.061141in}{1.177790in}}%
\pgfpathcurveto{\pgfqpoint{4.068274in}{1.177790in}}{\pgfqpoint{4.075115in}{1.180624in}}{\pgfqpoint{4.080159in}{1.185668in}}%
\pgfpathcurveto{\pgfqpoint{4.085203in}{1.190711in}}{\pgfqpoint{4.088036in}{1.197553in}}{\pgfqpoint{4.088036in}{1.204686in}}%
\pgfpathcurveto{\pgfqpoint{4.088036in}{1.211819in}}{\pgfqpoint{4.085203in}{1.218660in}}{\pgfqpoint{4.080159in}{1.223704in}}%
\pgfpathcurveto{\pgfqpoint{4.075115in}{1.228748in}}{\pgfqpoint{4.068274in}{1.231581in}}{\pgfqpoint{4.061141in}{1.231581in}}%
\pgfpathcurveto{\pgfqpoint{4.054008in}{1.231581in}}{\pgfqpoint{4.047166in}{1.228748in}}{\pgfqpoint{4.042123in}{1.223704in}}%
\pgfpathcurveto{\pgfqpoint{4.037079in}{1.218660in}}{\pgfqpoint{4.034245in}{1.211819in}}{\pgfqpoint{4.034245in}{1.204686in}}%
\pgfpathcurveto{\pgfqpoint{4.034245in}{1.197553in}}{\pgfqpoint{4.037079in}{1.190711in}}{\pgfqpoint{4.042123in}{1.185668in}}%
\pgfpathcurveto{\pgfqpoint{4.047166in}{1.180624in}}{\pgfqpoint{4.054008in}{1.177790in}}{\pgfqpoint{4.061141in}{1.177790in}}%
\pgfpathclose%
\pgfusepath{stroke,fill}%
\end{pgfscope}%
\begin{pgfscope}%
\pgfpathrectangle{\pgfqpoint{2.867647in}{0.500000in}}{\pgfqpoint{1.764706in}{1.700000in}}%
\pgfusepath{clip}%
\pgfsetbuttcap%
\pgfsetroundjoin%
\definecolor{currentfill}{rgb}{0.979891,0.908948,0.848279}%
\pgfsetfillcolor{currentfill}%
\pgfsetlinewidth{0.311001pt}%
\definecolor{currentstroke}{rgb}{1.000000,1.000000,1.000000}%
\pgfsetstrokecolor{currentstroke}%
\pgfsetdash{}{0pt}%
\pgfpathmoveto{\pgfqpoint{4.162982in}{1.400761in}}%
\pgfpathcurveto{\pgfqpoint{4.170115in}{1.400761in}}{\pgfqpoint{4.176957in}{1.403595in}}{\pgfqpoint{4.182001in}{1.408639in}}%
\pgfpathcurveto{\pgfqpoint{4.187044in}{1.413683in}}{\pgfqpoint{4.189878in}{1.420524in}}{\pgfqpoint{4.189878in}{1.427657in}}%
\pgfpathcurveto{\pgfqpoint{4.189878in}{1.434790in}}{\pgfqpoint{4.187044in}{1.441631in}}{\pgfqpoint{4.182001in}{1.446675in}}%
\pgfpathcurveto{\pgfqpoint{4.176957in}{1.451719in}}{\pgfqpoint{4.170115in}{1.454553in}}{\pgfqpoint{4.162982in}{1.454553in}}%
\pgfpathcurveto{\pgfqpoint{4.155850in}{1.454553in}}{\pgfqpoint{4.149008in}{1.451719in}}{\pgfqpoint{4.143964in}{1.446675in}}%
\pgfpathcurveto{\pgfqpoint{4.138921in}{1.441631in}}{\pgfqpoint{4.136087in}{1.434790in}}{\pgfqpoint{4.136087in}{1.427657in}}%
\pgfpathcurveto{\pgfqpoint{4.136087in}{1.420524in}}{\pgfqpoint{4.138921in}{1.413683in}}{\pgfqpoint{4.143964in}{1.408639in}}%
\pgfpathcurveto{\pgfqpoint{4.149008in}{1.403595in}}{\pgfqpoint{4.155850in}{1.400761in}}{\pgfqpoint{4.162982in}{1.400761in}}%
\pgfpathclose%
\pgfusepath{stroke,fill}%
\end{pgfscope}%
\begin{pgfscope}%
\pgfpathrectangle{\pgfqpoint{2.867647in}{0.500000in}}{\pgfqpoint{1.764706in}{1.700000in}}%
\pgfusepath{clip}%
\pgfsetbuttcap%
\pgfsetroundjoin%
\definecolor{currentfill}{rgb}{0.980678,0.914765,0.856766}%
\pgfsetfillcolor{currentfill}%
\pgfsetlinewidth{0.311001pt}%
\definecolor{currentstroke}{rgb}{1.000000,1.000000,1.000000}%
\pgfsetstrokecolor{currentstroke}%
\pgfsetdash{}{0pt}%
\pgfpathmoveto{\pgfqpoint{4.210492in}{1.325467in}}%
\pgfpathcurveto{\pgfqpoint{4.217624in}{1.325467in}}{\pgfqpoint{4.224466in}{1.328301in}}{\pgfqpoint{4.229510in}{1.333345in}}%
\pgfpathcurveto{\pgfqpoint{4.234553in}{1.338388in}}{\pgfqpoint{4.237387in}{1.345230in}}{\pgfqpoint{4.237387in}{1.352363in}}%
\pgfpathcurveto{\pgfqpoint{4.237387in}{1.359496in}}{\pgfqpoint{4.234553in}{1.366337in}}{\pgfqpoint{4.229510in}{1.371381in}}%
\pgfpathcurveto{\pgfqpoint{4.224466in}{1.376425in}}{\pgfqpoint{4.217624in}{1.379259in}}{\pgfqpoint{4.210492in}{1.379259in}}%
\pgfpathcurveto{\pgfqpoint{4.203359in}{1.379259in}}{\pgfqpoint{4.196517in}{1.376425in}}{\pgfqpoint{4.191473in}{1.371381in}}%
\pgfpathcurveto{\pgfqpoint{4.186430in}{1.366337in}}{\pgfqpoint{4.183596in}{1.359496in}}{\pgfqpoint{4.183596in}{1.352363in}}%
\pgfpathcurveto{\pgfqpoint{4.183596in}{1.345230in}}{\pgfqpoint{4.186430in}{1.338388in}}{\pgfqpoint{4.191473in}{1.333345in}}%
\pgfpathcurveto{\pgfqpoint{4.196517in}{1.328301in}}{\pgfqpoint{4.203359in}{1.325467in}}{\pgfqpoint{4.210492in}{1.325467in}}%
\pgfpathclose%
\pgfusepath{stroke,fill}%
\end{pgfscope}%
\begin{pgfscope}%
\pgfpathrectangle{\pgfqpoint{2.867647in}{0.500000in}}{\pgfqpoint{1.764706in}{1.700000in}}%
\pgfusepath{clip}%
\pgfsetbuttcap%
\pgfsetroundjoin%
\definecolor{currentfill}{rgb}{0.973271,0.850724,0.762998}%
\pgfsetfillcolor{currentfill}%
\pgfsetlinewidth{0.311001pt}%
\definecolor{currentstroke}{rgb}{1.000000,1.000000,1.000000}%
\pgfsetstrokecolor{currentstroke}%
\pgfsetdash{}{0pt}%
\pgfpathmoveto{\pgfqpoint{4.238880in}{1.473040in}}%
\pgfpathcurveto{\pgfqpoint{4.246012in}{1.473040in}}{\pgfqpoint{4.252854in}{1.475874in}}{\pgfqpoint{4.257898in}{1.480918in}}%
\pgfpathcurveto{\pgfqpoint{4.262941in}{1.485961in}}{\pgfqpoint{4.265775in}{1.492803in}}{\pgfqpoint{4.265775in}{1.499936in}}%
\pgfpathcurveto{\pgfqpoint{4.265775in}{1.507069in}}{\pgfqpoint{4.262941in}{1.513910in}}{\pgfqpoint{4.257898in}{1.518954in}}%
\pgfpathcurveto{\pgfqpoint{4.252854in}{1.523998in}}{\pgfqpoint{4.246012in}{1.526832in}}{\pgfqpoint{4.238880in}{1.526832in}}%
\pgfpathcurveto{\pgfqpoint{4.231747in}{1.526832in}}{\pgfqpoint{4.224905in}{1.523998in}}{\pgfqpoint{4.219862in}{1.518954in}}%
\pgfpathcurveto{\pgfqpoint{4.214818in}{1.513910in}}{\pgfqpoint{4.211984in}{1.507069in}}{\pgfqpoint{4.211984in}{1.499936in}}%
\pgfpathcurveto{\pgfqpoint{4.211984in}{1.492803in}}{\pgfqpoint{4.214818in}{1.485961in}}{\pgfqpoint{4.219862in}{1.480918in}}%
\pgfpathcurveto{\pgfqpoint{4.224905in}{1.475874in}}{\pgfqpoint{4.231747in}{1.473040in}}{\pgfqpoint{4.238880in}{1.473040in}}%
\pgfpathclose%
\pgfusepath{stroke,fill}%
\end{pgfscope}%
\begin{pgfscope}%
\pgfpathrectangle{\pgfqpoint{2.867647in}{0.500000in}}{\pgfqpoint{1.764706in}{1.700000in}}%
\pgfusepath{clip}%
\pgfsetbuttcap%
\pgfsetroundjoin%
\definecolor{currentfill}{rgb}{0.980678,0.914765,0.856766}%
\pgfsetfillcolor{currentfill}%
\pgfsetlinewidth{0.311001pt}%
\definecolor{currentstroke}{rgb}{1.000000,1.000000,1.000000}%
\pgfsetstrokecolor{currentstroke}%
\pgfsetdash{}{0pt}%
\pgfpathmoveto{\pgfqpoint{4.158038in}{1.478605in}}%
\pgfpathcurveto{\pgfqpoint{4.165171in}{1.478605in}}{\pgfqpoint{4.172013in}{1.481439in}}{\pgfqpoint{4.177056in}{1.486482in}}%
\pgfpathcurveto{\pgfqpoint{4.182100in}{1.491526in}}{\pgfqpoint{4.184934in}{1.498368in}}{\pgfqpoint{4.184934in}{1.505500in}}%
\pgfpathcurveto{\pgfqpoint{4.184934in}{1.512633in}}{\pgfqpoint{4.182100in}{1.519475in}}{\pgfqpoint{4.177056in}{1.524519in}}%
\pgfpathcurveto{\pgfqpoint{4.172013in}{1.529562in}}{\pgfqpoint{4.165171in}{1.532396in}}{\pgfqpoint{4.158038in}{1.532396in}}%
\pgfpathcurveto{\pgfqpoint{4.150905in}{1.532396in}}{\pgfqpoint{4.144064in}{1.529562in}}{\pgfqpoint{4.139020in}{1.524519in}}%
\pgfpathcurveto{\pgfqpoint{4.133976in}{1.519475in}}{\pgfqpoint{4.131142in}{1.512633in}}{\pgfqpoint{4.131142in}{1.505500in}}%
\pgfpathcurveto{\pgfqpoint{4.131142in}{1.498368in}}{\pgfqpoint{4.133976in}{1.491526in}}{\pgfqpoint{4.139020in}{1.486482in}}%
\pgfpathcurveto{\pgfqpoint{4.144064in}{1.481439in}}{\pgfqpoint{4.150905in}{1.478605in}}{\pgfqpoint{4.158038in}{1.478605in}}%
\pgfpathclose%
\pgfusepath{stroke,fill}%
\end{pgfscope}%
\begin{pgfscope}%
\pgfpathrectangle{\pgfqpoint{2.867647in}{0.500000in}}{\pgfqpoint{1.764706in}{1.700000in}}%
\pgfusepath{clip}%
\pgfsetbuttcap%
\pgfsetroundjoin%
\definecolor{currentfill}{rgb}{0.980678,0.914765,0.856766}%
\pgfsetfillcolor{currentfill}%
\pgfsetlinewidth{0.311001pt}%
\definecolor{currentstroke}{rgb}{1.000000,1.000000,1.000000}%
\pgfsetstrokecolor{currentstroke}%
\pgfsetdash{}{0pt}%
\pgfpathmoveto{\pgfqpoint{4.166376in}{1.507833in}}%
\pgfpathcurveto{\pgfqpoint{4.173509in}{1.507833in}}{\pgfqpoint{4.180351in}{1.510667in}}{\pgfqpoint{4.185394in}{1.515710in}}%
\pgfpathcurveto{\pgfqpoint{4.190438in}{1.520754in}}{\pgfqpoint{4.193272in}{1.527596in}}{\pgfqpoint{4.193272in}{1.534728in}}%
\pgfpathcurveto{\pgfqpoint{4.193272in}{1.541861in}}{\pgfqpoint{4.190438in}{1.548703in}}{\pgfqpoint{4.185394in}{1.553747in}}%
\pgfpathcurveto{\pgfqpoint{4.180351in}{1.558790in}}{\pgfqpoint{4.173509in}{1.561624in}}{\pgfqpoint{4.166376in}{1.561624in}}%
\pgfpathcurveto{\pgfqpoint{4.159243in}{1.561624in}}{\pgfqpoint{4.152402in}{1.558790in}}{\pgfqpoint{4.147358in}{1.553747in}}%
\pgfpathcurveto{\pgfqpoint{4.142314in}{1.548703in}}{\pgfqpoint{4.139480in}{1.541861in}}{\pgfqpoint{4.139480in}{1.534728in}}%
\pgfpathcurveto{\pgfqpoint{4.139480in}{1.527596in}}{\pgfqpoint{4.142314in}{1.520754in}}{\pgfqpoint{4.147358in}{1.515710in}}%
\pgfpathcurveto{\pgfqpoint{4.152402in}{1.510667in}}{\pgfqpoint{4.159243in}{1.507833in}}{\pgfqpoint{4.166376in}{1.507833in}}%
\pgfpathclose%
\pgfusepath{stroke,fill}%
\end{pgfscope}%
\begin{pgfscope}%
\pgfpathrectangle{\pgfqpoint{2.867647in}{0.500000in}}{\pgfqpoint{1.764706in}{1.700000in}}%
\pgfusepath{clip}%
\pgfsetbuttcap%
\pgfsetroundjoin%
\definecolor{currentfill}{rgb}{0.976961,0.885681,0.814303}%
\pgfsetfillcolor{currentfill}%
\pgfsetlinewidth{0.311001pt}%
\definecolor{currentstroke}{rgb}{1.000000,1.000000,1.000000}%
\pgfsetstrokecolor{currentstroke}%
\pgfsetdash{}{0pt}%
\pgfpathmoveto{\pgfqpoint{4.220036in}{1.163196in}}%
\pgfpathcurveto{\pgfqpoint{4.227169in}{1.163196in}}{\pgfqpoint{4.234010in}{1.166029in}}{\pgfqpoint{4.239054in}{1.171073in}}%
\pgfpathcurveto{\pgfqpoint{4.244098in}{1.176117in}}{\pgfqpoint{4.246932in}{1.182958in}}{\pgfqpoint{4.246932in}{1.190091in}}%
\pgfpathcurveto{\pgfqpoint{4.246932in}{1.197224in}}{\pgfqpoint{4.244098in}{1.204066in}}{\pgfqpoint{4.239054in}{1.209109in}}%
\pgfpathcurveto{\pgfqpoint{4.234010in}{1.214153in}}{\pgfqpoint{4.227169in}{1.216987in}}{\pgfqpoint{4.220036in}{1.216987in}}%
\pgfpathcurveto{\pgfqpoint{4.212903in}{1.216987in}}{\pgfqpoint{4.206061in}{1.214153in}}{\pgfqpoint{4.201018in}{1.209109in}}%
\pgfpathcurveto{\pgfqpoint{4.195974in}{1.204066in}}{\pgfqpoint{4.193140in}{1.197224in}}{\pgfqpoint{4.193140in}{1.190091in}}%
\pgfpathcurveto{\pgfqpoint{4.193140in}{1.182958in}}{\pgfqpoint{4.195974in}{1.176117in}}{\pgfqpoint{4.201018in}{1.171073in}}%
\pgfpathcurveto{\pgfqpoint{4.206061in}{1.166029in}}{\pgfqpoint{4.212903in}{1.163196in}}{\pgfqpoint{4.220036in}{1.163196in}}%
\pgfpathclose%
\pgfusepath{stroke,fill}%
\end{pgfscope}%
\begin{pgfscope}%
\pgfpathrectangle{\pgfqpoint{2.867647in}{0.500000in}}{\pgfqpoint{1.764706in}{1.700000in}}%
\pgfusepath{clip}%
\pgfsetbuttcap%
\pgfsetroundjoin%
\definecolor{currentfill}{rgb}{0.968509,0.792226,0.676405}%
\pgfsetfillcolor{currentfill}%
\pgfsetlinewidth{0.311001pt}%
\definecolor{currentstroke}{rgb}{1.000000,1.000000,1.000000}%
\pgfsetstrokecolor{currentstroke}%
\pgfsetdash{}{0pt}%
\pgfpathmoveto{\pgfqpoint{4.280825in}{1.374241in}}%
\pgfpathcurveto{\pgfqpoint{4.287958in}{1.374241in}}{\pgfqpoint{4.294800in}{1.377075in}}{\pgfqpoint{4.299843in}{1.382118in}}%
\pgfpathcurveto{\pgfqpoint{4.304887in}{1.387162in}}{\pgfqpoint{4.307721in}{1.394004in}}{\pgfqpoint{4.307721in}{1.401136in}}%
\pgfpathcurveto{\pgfqpoint{4.307721in}{1.408269in}}{\pgfqpoint{4.304887in}{1.415111in}}{\pgfqpoint{4.299843in}{1.420155in}}%
\pgfpathcurveto{\pgfqpoint{4.294800in}{1.425198in}}{\pgfqpoint{4.287958in}{1.428032in}}{\pgfqpoint{4.280825in}{1.428032in}}%
\pgfpathcurveto{\pgfqpoint{4.273692in}{1.428032in}}{\pgfqpoint{4.266851in}{1.425198in}}{\pgfqpoint{4.261807in}{1.420155in}}%
\pgfpathcurveto{\pgfqpoint{4.256763in}{1.415111in}}{\pgfqpoint{4.253929in}{1.408269in}}{\pgfqpoint{4.253929in}{1.401136in}}%
\pgfpathcurveto{\pgfqpoint{4.253929in}{1.394004in}}{\pgfqpoint{4.256763in}{1.387162in}}{\pgfqpoint{4.261807in}{1.382118in}}%
\pgfpathcurveto{\pgfqpoint{4.266851in}{1.377075in}}{\pgfqpoint{4.273692in}{1.374241in}}{\pgfqpoint{4.280825in}{1.374241in}}%
\pgfpathclose%
\pgfusepath{stroke,fill}%
\end{pgfscope}%
\begin{pgfscope}%
\pgfpathrectangle{\pgfqpoint{2.867647in}{0.500000in}}{\pgfqpoint{1.764706in}{1.700000in}}%
\pgfusepath{clip}%
\pgfsetbuttcap%
\pgfsetroundjoin%
\definecolor{currentfill}{rgb}{0.980678,0.914765,0.856766}%
\pgfsetfillcolor{currentfill}%
\pgfsetlinewidth{0.311001pt}%
\definecolor{currentstroke}{rgb}{1.000000,1.000000,1.000000}%
\pgfsetstrokecolor{currentstroke}%
\pgfsetdash{}{0pt}%
\pgfpathmoveto{\pgfqpoint{4.208057in}{1.229158in}}%
\pgfpathcurveto{\pgfqpoint{4.215190in}{1.229158in}}{\pgfqpoint{4.222032in}{1.231992in}}{\pgfqpoint{4.227075in}{1.237036in}}%
\pgfpathcurveto{\pgfqpoint{4.232119in}{1.242080in}}{\pgfqpoint{4.234953in}{1.248921in}}{\pgfqpoint{4.234953in}{1.256054in}}%
\pgfpathcurveto{\pgfqpoint{4.234953in}{1.263187in}}{\pgfqpoint{4.232119in}{1.270028in}}{\pgfqpoint{4.227075in}{1.275072in}}%
\pgfpathcurveto{\pgfqpoint{4.222032in}{1.280116in}}{\pgfqpoint{4.215190in}{1.282950in}}{\pgfqpoint{4.208057in}{1.282950in}}%
\pgfpathcurveto{\pgfqpoint{4.200924in}{1.282950in}}{\pgfqpoint{4.194083in}{1.280116in}}{\pgfqpoint{4.189039in}{1.275072in}}%
\pgfpathcurveto{\pgfqpoint{4.183995in}{1.270028in}}{\pgfqpoint{4.181162in}{1.263187in}}{\pgfqpoint{4.181162in}{1.256054in}}%
\pgfpathcurveto{\pgfqpoint{4.181162in}{1.248921in}}{\pgfqpoint{4.183995in}{1.242080in}}{\pgfqpoint{4.189039in}{1.237036in}}%
\pgfpathcurveto{\pgfqpoint{4.194083in}{1.231992in}}{\pgfqpoint{4.200924in}{1.229158in}}{\pgfqpoint{4.208057in}{1.229158in}}%
\pgfpathclose%
\pgfusepath{stroke,fill}%
\end{pgfscope}%
\begin{pgfscope}%
\pgfpathrectangle{\pgfqpoint{2.867647in}{0.500000in}}{\pgfqpoint{1.764706in}{1.700000in}}%
\pgfusepath{clip}%
\pgfsetbuttcap%
\pgfsetroundjoin%
\definecolor{currentfill}{rgb}{0.964306,0.663930,0.507747}%
\pgfsetfillcolor{currentfill}%
\pgfsetlinewidth{0.311001pt}%
\definecolor{currentstroke}{rgb}{1.000000,1.000000,1.000000}%
\pgfsetstrokecolor{currentstroke}%
\pgfsetdash{}{0pt}%
\pgfpathmoveto{\pgfqpoint{4.085481in}{0.876249in}}%
\pgfpathcurveto{\pgfqpoint{4.092614in}{0.876249in}}{\pgfqpoint{4.099455in}{0.879083in}}{\pgfqpoint{4.104499in}{0.884127in}}%
\pgfpathcurveto{\pgfqpoint{4.109543in}{0.889171in}}{\pgfqpoint{4.112377in}{0.896012in}}{\pgfqpoint{4.112377in}{0.903145in}}%
\pgfpathcurveto{\pgfqpoint{4.112377in}{0.910278in}}{\pgfqpoint{4.109543in}{0.917120in}}{\pgfqpoint{4.104499in}{0.922163in}}%
\pgfpathcurveto{\pgfqpoint{4.099455in}{0.927207in}}{\pgfqpoint{4.092614in}{0.930041in}}{\pgfqpoint{4.085481in}{0.930041in}}%
\pgfpathcurveto{\pgfqpoint{4.078348in}{0.930041in}}{\pgfqpoint{4.071506in}{0.927207in}}{\pgfqpoint{4.066463in}{0.922163in}}%
\pgfpathcurveto{\pgfqpoint{4.061419in}{0.917120in}}{\pgfqpoint{4.058585in}{0.910278in}}{\pgfqpoint{4.058585in}{0.903145in}}%
\pgfpathcurveto{\pgfqpoint{4.058585in}{0.896012in}}{\pgfqpoint{4.061419in}{0.889171in}}{\pgfqpoint{4.066463in}{0.884127in}}%
\pgfpathcurveto{\pgfqpoint{4.071506in}{0.879083in}}{\pgfqpoint{4.078348in}{0.876249in}}{\pgfqpoint{4.085481in}{0.876249in}}%
\pgfpathclose%
\pgfusepath{stroke,fill}%
\end{pgfscope}%
\begin{pgfscope}%
\pgfpathrectangle{\pgfqpoint{2.867647in}{0.500000in}}{\pgfqpoint{1.764706in}{1.700000in}}%
\pgfusepath{clip}%
\pgfsetbuttcap%
\pgfsetroundjoin%
\definecolor{currentfill}{rgb}{0.953816,0.463738,0.317699}%
\pgfsetfillcolor{currentfill}%
\pgfsetlinewidth{0.311001pt}%
\definecolor{currentstroke}{rgb}{1.000000,1.000000,1.000000}%
\pgfsetstrokecolor{currentstroke}%
\pgfsetdash{}{0pt}%
\pgfpathmoveto{\pgfqpoint{3.906117in}{0.966014in}}%
\pgfpathcurveto{\pgfqpoint{3.913249in}{0.966014in}}{\pgfqpoint{3.920091in}{0.968848in}}{\pgfqpoint{3.925135in}{0.973891in}}%
\pgfpathcurveto{\pgfqpoint{3.930178in}{0.978935in}}{\pgfqpoint{3.933012in}{0.985777in}}{\pgfqpoint{3.933012in}{0.992909in}}%
\pgfpathcurveto{\pgfqpoint{3.933012in}{1.000042in}}{\pgfqpoint{3.930178in}{1.006884in}}{\pgfqpoint{3.925135in}{1.011928in}}%
\pgfpathcurveto{\pgfqpoint{3.920091in}{1.016971in}}{\pgfqpoint{3.913249in}{1.019805in}}{\pgfqpoint{3.906117in}{1.019805in}}%
\pgfpathcurveto{\pgfqpoint{3.898984in}{1.019805in}}{\pgfqpoint{3.892142in}{1.016971in}}{\pgfqpoint{3.887098in}{1.011928in}}%
\pgfpathcurveto{\pgfqpoint{3.882055in}{1.006884in}}{\pgfqpoint{3.879221in}{1.000042in}}{\pgfqpoint{3.879221in}{0.992909in}}%
\pgfpathcurveto{\pgfqpoint{3.879221in}{0.985777in}}{\pgfqpoint{3.882055in}{0.978935in}}{\pgfqpoint{3.887098in}{0.973891in}}%
\pgfpathcurveto{\pgfqpoint{3.892142in}{0.968848in}}{\pgfqpoint{3.898984in}{0.966014in}}{\pgfqpoint{3.906117in}{0.966014in}}%
\pgfpathclose%
\pgfusepath{stroke,fill}%
\end{pgfscope}%
\begin{pgfscope}%
\pgfpathrectangle{\pgfqpoint{2.867647in}{0.500000in}}{\pgfqpoint{1.764706in}{1.700000in}}%
\pgfusepath{clip}%
\pgfsetbuttcap%
\pgfsetroundjoin%
\definecolor{currentfill}{rgb}{0.981377,0.920617,0.865369}%
\pgfsetfillcolor{currentfill}%
\pgfsetlinewidth{0.311001pt}%
\definecolor{currentstroke}{rgb}{1.000000,1.000000,1.000000}%
\pgfsetstrokecolor{currentstroke}%
\pgfsetdash{}{0pt}%
\pgfpathmoveto{\pgfqpoint{4.189865in}{1.221631in}}%
\pgfpathcurveto{\pgfqpoint{4.196997in}{1.221631in}}{\pgfqpoint{4.203839in}{1.224465in}}{\pgfqpoint{4.208883in}{1.229508in}}%
\pgfpathcurveto{\pgfqpoint{4.213926in}{1.234552in}}{\pgfqpoint{4.216760in}{1.241394in}}{\pgfqpoint{4.216760in}{1.248526in}}%
\pgfpathcurveto{\pgfqpoint{4.216760in}{1.255659in}}{\pgfqpoint{4.213926in}{1.262501in}}{\pgfqpoint{4.208883in}{1.267545in}}%
\pgfpathcurveto{\pgfqpoint{4.203839in}{1.272588in}}{\pgfqpoint{4.196997in}{1.275422in}}{\pgfqpoint{4.189865in}{1.275422in}}%
\pgfpathcurveto{\pgfqpoint{4.182732in}{1.275422in}}{\pgfqpoint{4.175890in}{1.272588in}}{\pgfqpoint{4.170846in}{1.267545in}}%
\pgfpathcurveto{\pgfqpoint{4.165803in}{1.262501in}}{\pgfqpoint{4.162969in}{1.255659in}}{\pgfqpoint{4.162969in}{1.248526in}}%
\pgfpathcurveto{\pgfqpoint{4.162969in}{1.241394in}}{\pgfqpoint{4.165803in}{1.234552in}}{\pgfqpoint{4.170846in}{1.229508in}}%
\pgfpathcurveto{\pgfqpoint{4.175890in}{1.224465in}}{\pgfqpoint{4.182732in}{1.221631in}}{\pgfqpoint{4.189865in}{1.221631in}}%
\pgfpathclose%
\pgfusepath{stroke,fill}%
\end{pgfscope}%
\begin{pgfscope}%
\pgfpathrectangle{\pgfqpoint{2.867647in}{0.500000in}}{\pgfqpoint{1.764706in}{1.700000in}}%
\pgfusepath{clip}%
\pgfsetbuttcap%
\pgfsetroundjoin%
\definecolor{currentfill}{rgb}{0.972726,0.844889,0.754401}%
\pgfsetfillcolor{currentfill}%
\pgfsetlinewidth{0.311001pt}%
\definecolor{currentstroke}{rgb}{1.000000,1.000000,1.000000}%
\pgfsetstrokecolor{currentstroke}%
\pgfsetdash{}{0pt}%
\pgfpathmoveto{\pgfqpoint{4.233043in}{1.499555in}}%
\pgfpathcurveto{\pgfqpoint{4.240176in}{1.499555in}}{\pgfqpoint{4.247018in}{1.502389in}}{\pgfqpoint{4.252062in}{1.507432in}}%
\pgfpathcurveto{\pgfqpoint{4.257105in}{1.512476in}}{\pgfqpoint{4.259939in}{1.519318in}}{\pgfqpoint{4.259939in}{1.526450in}}%
\pgfpathcurveto{\pgfqpoint{4.259939in}{1.533583in}}{\pgfqpoint{4.257105in}{1.540425in}}{\pgfqpoint{4.252062in}{1.545469in}}%
\pgfpathcurveto{\pgfqpoint{4.247018in}{1.550512in}}{\pgfqpoint{4.240176in}{1.553346in}}{\pgfqpoint{4.233043in}{1.553346in}}%
\pgfpathcurveto{\pgfqpoint{4.225911in}{1.553346in}}{\pgfqpoint{4.219069in}{1.550512in}}{\pgfqpoint{4.214025in}{1.545469in}}%
\pgfpathcurveto{\pgfqpoint{4.208982in}{1.540425in}}{\pgfqpoint{4.206148in}{1.533583in}}{\pgfqpoint{4.206148in}{1.526450in}}%
\pgfpathcurveto{\pgfqpoint{4.206148in}{1.519318in}}{\pgfqpoint{4.208982in}{1.512476in}}{\pgfqpoint{4.214025in}{1.507432in}}%
\pgfpathcurveto{\pgfqpoint{4.219069in}{1.502389in}}{\pgfqpoint{4.225911in}{1.499555in}}{\pgfqpoint{4.233043in}{1.499555in}}%
\pgfpathclose%
\pgfusepath{stroke,fill}%
\end{pgfscope}%
\begin{pgfscope}%
\pgfpathrectangle{\pgfqpoint{2.867647in}{0.500000in}}{\pgfqpoint{1.764706in}{1.700000in}}%
\pgfusepath{clip}%
\pgfsetbuttcap%
\pgfsetroundjoin%
\definecolor{currentfill}{rgb}{0.980678,0.914765,0.856766}%
\pgfsetfillcolor{currentfill}%
\pgfsetlinewidth{0.311001pt}%
\definecolor{currentstroke}{rgb}{1.000000,1.000000,1.000000}%
\pgfsetstrokecolor{currentstroke}%
\pgfsetdash{}{0pt}%
\pgfpathmoveto{\pgfqpoint{4.192190in}{1.173849in}}%
\pgfpathcurveto{\pgfqpoint{4.199323in}{1.173849in}}{\pgfqpoint{4.206165in}{1.176683in}}{\pgfqpoint{4.211208in}{1.181726in}}%
\pgfpathcurveto{\pgfqpoint{4.216252in}{1.186770in}}{\pgfqpoint{4.219086in}{1.193612in}}{\pgfqpoint{4.219086in}{1.200745in}}%
\pgfpathcurveto{\pgfqpoint{4.219086in}{1.207877in}}{\pgfqpoint{4.216252in}{1.214719in}}{\pgfqpoint{4.211208in}{1.219763in}}%
\pgfpathcurveto{\pgfqpoint{4.206165in}{1.224806in}}{\pgfqpoint{4.199323in}{1.227640in}}{\pgfqpoint{4.192190in}{1.227640in}}%
\pgfpathcurveto{\pgfqpoint{4.185057in}{1.227640in}}{\pgfqpoint{4.178216in}{1.224806in}}{\pgfqpoint{4.173172in}{1.219763in}}%
\pgfpathcurveto{\pgfqpoint{4.168128in}{1.214719in}}{\pgfqpoint{4.165295in}{1.207877in}}{\pgfqpoint{4.165295in}{1.200745in}}%
\pgfpathcurveto{\pgfqpoint{4.165295in}{1.193612in}}{\pgfqpoint{4.168128in}{1.186770in}}{\pgfqpoint{4.173172in}{1.181726in}}%
\pgfpathcurveto{\pgfqpoint{4.178216in}{1.176683in}}{\pgfqpoint{4.185057in}{1.173849in}}{\pgfqpoint{4.192190in}{1.173849in}}%
\pgfpathclose%
\pgfusepath{stroke,fill}%
\end{pgfscope}%
\begin{pgfscope}%
\pgfpathrectangle{\pgfqpoint{2.867647in}{0.500000in}}{\pgfqpoint{1.764706in}{1.700000in}}%
\pgfusepath{clip}%
\pgfsetbuttcap%
\pgfsetroundjoin%
\definecolor{currentfill}{rgb}{0.979124,0.903132,0.839793}%
\pgfsetfillcolor{currentfill}%
\pgfsetlinewidth{0.311001pt}%
\definecolor{currentstroke}{rgb}{1.000000,1.000000,1.000000}%
\pgfsetstrokecolor{currentstroke}%
\pgfsetdash{}{0pt}%
\pgfpathmoveto{\pgfqpoint{4.148957in}{1.239018in}}%
\pgfpathcurveto{\pgfqpoint{4.156090in}{1.239018in}}{\pgfqpoint{4.162932in}{1.241852in}}{\pgfqpoint{4.167975in}{1.246895in}}%
\pgfpathcurveto{\pgfqpoint{4.173019in}{1.251939in}}{\pgfqpoint{4.175853in}{1.258781in}}{\pgfqpoint{4.175853in}{1.265913in}}%
\pgfpathcurveto{\pgfqpoint{4.175853in}{1.273046in}}{\pgfqpoint{4.173019in}{1.279888in}}{\pgfqpoint{4.167975in}{1.284932in}}%
\pgfpathcurveto{\pgfqpoint{4.162932in}{1.289975in}}{\pgfqpoint{4.156090in}{1.292809in}}{\pgfqpoint{4.148957in}{1.292809in}}%
\pgfpathcurveto{\pgfqpoint{4.141824in}{1.292809in}}{\pgfqpoint{4.134983in}{1.289975in}}{\pgfqpoint{4.129939in}{1.284932in}}%
\pgfpathcurveto{\pgfqpoint{4.124896in}{1.279888in}}{\pgfqpoint{4.122062in}{1.273046in}}{\pgfqpoint{4.122062in}{1.265913in}}%
\pgfpathcurveto{\pgfqpoint{4.122062in}{1.258781in}}{\pgfqpoint{4.124896in}{1.251939in}}{\pgfqpoint{4.129939in}{1.246895in}}%
\pgfpathcurveto{\pgfqpoint{4.134983in}{1.241852in}}{\pgfqpoint{4.141824in}{1.239018in}}{\pgfqpoint{4.148957in}{1.239018in}}%
\pgfpathclose%
\pgfusepath{stroke,fill}%
\end{pgfscope}%
\begin{pgfscope}%
\pgfpathrectangle{\pgfqpoint{2.867647in}{0.500000in}}{\pgfqpoint{1.764706in}{1.700000in}}%
\pgfusepath{clip}%
\pgfsetbuttcap%
\pgfsetroundjoin%
\definecolor{currentfill}{rgb}{0.968931,0.798091,0.685123}%
\pgfsetfillcolor{currentfill}%
\pgfsetlinewidth{0.311001pt}%
\definecolor{currentstroke}{rgb}{1.000000,1.000000,1.000000}%
\pgfsetstrokecolor{currentstroke}%
\pgfsetdash{}{0pt}%
\pgfpathmoveto{\pgfqpoint{4.156458in}{1.672903in}}%
\pgfpathcurveto{\pgfqpoint{4.163591in}{1.672903in}}{\pgfqpoint{4.170433in}{1.675737in}}{\pgfqpoint{4.175476in}{1.680781in}}%
\pgfpathcurveto{\pgfqpoint{4.180520in}{1.685825in}}{\pgfqpoint{4.183354in}{1.692666in}}{\pgfqpoint{4.183354in}{1.699799in}}%
\pgfpathcurveto{\pgfqpoint{4.183354in}{1.706932in}}{\pgfqpoint{4.180520in}{1.713774in}}{\pgfqpoint{4.175476in}{1.718817in}}%
\pgfpathcurveto{\pgfqpoint{4.170433in}{1.723861in}}{\pgfqpoint{4.163591in}{1.726695in}}{\pgfqpoint{4.156458in}{1.726695in}}%
\pgfpathcurveto{\pgfqpoint{4.149325in}{1.726695in}}{\pgfqpoint{4.142484in}{1.723861in}}{\pgfqpoint{4.137440in}{1.718817in}}%
\pgfpathcurveto{\pgfqpoint{4.132396in}{1.713774in}}{\pgfqpoint{4.129562in}{1.706932in}}{\pgfqpoint{4.129562in}{1.699799in}}%
\pgfpathcurveto{\pgfqpoint{4.129562in}{1.692666in}}{\pgfqpoint{4.132396in}{1.685825in}}{\pgfqpoint{4.137440in}{1.680781in}}%
\pgfpathcurveto{\pgfqpoint{4.142484in}{1.675737in}}{\pgfqpoint{4.149325in}{1.672903in}}{\pgfqpoint{4.156458in}{1.672903in}}%
\pgfpathclose%
\pgfusepath{stroke,fill}%
\end{pgfscope}%
\begin{pgfscope}%
\pgfpathrectangle{\pgfqpoint{2.867647in}{0.500000in}}{\pgfqpoint{1.764706in}{1.700000in}}%
\pgfusepath{clip}%
\pgfsetbuttcap%
\pgfsetroundjoin%
\definecolor{currentfill}{rgb}{0.973271,0.850724,0.762998}%
\pgfsetfillcolor{currentfill}%
\pgfsetlinewidth{0.311001pt}%
\definecolor{currentstroke}{rgb}{1.000000,1.000000,1.000000}%
\pgfsetstrokecolor{currentstroke}%
\pgfsetdash{}{0pt}%
\pgfpathmoveto{\pgfqpoint{4.068148in}{1.613457in}}%
\pgfpathcurveto{\pgfqpoint{4.075281in}{1.613457in}}{\pgfqpoint{4.082123in}{1.616290in}}{\pgfqpoint{4.087166in}{1.621334in}}%
\pgfpathcurveto{\pgfqpoint{4.092210in}{1.626378in}}{\pgfqpoint{4.095044in}{1.633219in}}{\pgfqpoint{4.095044in}{1.640352in}}%
\pgfpathcurveto{\pgfqpoint{4.095044in}{1.647485in}}{\pgfqpoint{4.092210in}{1.654327in}}{\pgfqpoint{4.087166in}{1.659370in}}%
\pgfpathcurveto{\pgfqpoint{4.082123in}{1.664414in}}{\pgfqpoint{4.075281in}{1.667248in}}{\pgfqpoint{4.068148in}{1.667248in}}%
\pgfpathcurveto{\pgfqpoint{4.061015in}{1.667248in}}{\pgfqpoint{4.054174in}{1.664414in}}{\pgfqpoint{4.049130in}{1.659370in}}%
\pgfpathcurveto{\pgfqpoint{4.044086in}{1.654327in}}{\pgfqpoint{4.041253in}{1.647485in}}{\pgfqpoint{4.041253in}{1.640352in}}%
\pgfpathcurveto{\pgfqpoint{4.041253in}{1.633219in}}{\pgfqpoint{4.044086in}{1.626378in}}{\pgfqpoint{4.049130in}{1.621334in}}%
\pgfpathcurveto{\pgfqpoint{4.054174in}{1.616290in}}{\pgfqpoint{4.061015in}{1.613457in}}{\pgfqpoint{4.068148in}{1.613457in}}%
\pgfpathclose%
\pgfusepath{stroke,fill}%
\end{pgfscope}%
\begin{pgfscope}%
\pgfpathrectangle{\pgfqpoint{2.867647in}{0.500000in}}{\pgfqpoint{1.764706in}{1.700000in}}%
\pgfusepath{clip}%
\pgfsetbuttcap%
\pgfsetroundjoin%
\definecolor{currentfill}{rgb}{0.966560,0.756582,0.625273}%
\pgfsetfillcolor{currentfill}%
\pgfsetlinewidth{0.311001pt}%
\definecolor{currentstroke}{rgb}{1.000000,1.000000,1.000000}%
\pgfsetstrokecolor{currentstroke}%
\pgfsetdash{}{0pt}%
\pgfpathmoveto{\pgfqpoint{4.177318in}{0.969063in}}%
\pgfpathcurveto{\pgfqpoint{4.184450in}{0.969063in}}{\pgfqpoint{4.191292in}{0.971897in}}{\pgfqpoint{4.196336in}{0.976941in}}%
\pgfpathcurveto{\pgfqpoint{4.201379in}{0.981984in}}{\pgfqpoint{4.204213in}{0.988826in}}{\pgfqpoint{4.204213in}{0.995959in}}%
\pgfpathcurveto{\pgfqpoint{4.204213in}{1.003092in}}{\pgfqpoint{4.201379in}{1.009933in}}{\pgfqpoint{4.196336in}{1.014977in}}%
\pgfpathcurveto{\pgfqpoint{4.191292in}{1.020021in}}{\pgfqpoint{4.184450in}{1.022855in}}{\pgfqpoint{4.177318in}{1.022855in}}%
\pgfpathcurveto{\pgfqpoint{4.170185in}{1.022855in}}{\pgfqpoint{4.163343in}{1.020021in}}{\pgfqpoint{4.158299in}{1.014977in}}%
\pgfpathcurveto{\pgfqpoint{4.153256in}{1.009933in}}{\pgfqpoint{4.150422in}{1.003092in}}{\pgfqpoint{4.150422in}{0.995959in}}%
\pgfpathcurveto{\pgfqpoint{4.150422in}{0.988826in}}{\pgfqpoint{4.153256in}{0.981984in}}{\pgfqpoint{4.158299in}{0.976941in}}%
\pgfpathcurveto{\pgfqpoint{4.163343in}{0.971897in}}{\pgfqpoint{4.170185in}{0.969063in}}{\pgfqpoint{4.177318in}{0.969063in}}%
\pgfpathclose%
\pgfusepath{stroke,fill}%
\end{pgfscope}%
\begin{pgfscope}%
\pgfpathrectangle{\pgfqpoint{2.867647in}{0.500000in}}{\pgfqpoint{1.764706in}{1.700000in}}%
\pgfusepath{clip}%
\pgfsetbuttcap%
\pgfsetroundjoin%
\definecolor{currentfill}{rgb}{0.960778,0.559972,0.399412}%
\pgfsetfillcolor{currentfill}%
\pgfsetlinewidth{0.311001pt}%
\definecolor{currentstroke}{rgb}{1.000000,1.000000,1.000000}%
\pgfsetstrokecolor{currentstroke}%
\pgfsetdash{}{0pt}%
\pgfpathmoveto{\pgfqpoint{3.943008in}{0.890650in}}%
\pgfpathcurveto{\pgfqpoint{3.950141in}{0.890650in}}{\pgfqpoint{3.956983in}{0.893483in}}{\pgfqpoint{3.962027in}{0.898527in}}%
\pgfpathcurveto{\pgfqpoint{3.967070in}{0.903571in}}{\pgfqpoint{3.969904in}{0.910412in}}{\pgfqpoint{3.969904in}{0.917545in}}%
\pgfpathcurveto{\pgfqpoint{3.969904in}{0.924678in}}{\pgfqpoint{3.967070in}{0.931520in}}{\pgfqpoint{3.962027in}{0.936563in}}%
\pgfpathcurveto{\pgfqpoint{3.956983in}{0.941607in}}{\pgfqpoint{3.950141in}{0.944441in}}{\pgfqpoint{3.943008in}{0.944441in}}%
\pgfpathcurveto{\pgfqpoint{3.935876in}{0.944441in}}{\pgfqpoint{3.929034in}{0.941607in}}{\pgfqpoint{3.923990in}{0.936563in}}%
\pgfpathcurveto{\pgfqpoint{3.918947in}{0.931520in}}{\pgfqpoint{3.916113in}{0.924678in}}{\pgfqpoint{3.916113in}{0.917545in}}%
\pgfpathcurveto{\pgfqpoint{3.916113in}{0.910412in}}{\pgfqpoint{3.918947in}{0.903571in}}{\pgfqpoint{3.923990in}{0.898527in}}%
\pgfpathcurveto{\pgfqpoint{3.929034in}{0.893483in}}{\pgfqpoint{3.935876in}{0.890650in}}{\pgfqpoint{3.943008in}{0.890650in}}%
\pgfpathclose%
\pgfusepath{stroke,fill}%
\end{pgfscope}%
\begin{pgfscope}%
\pgfpathrectangle{\pgfqpoint{2.867647in}{0.500000in}}{\pgfqpoint{1.764706in}{1.700000in}}%
\pgfusepath{clip}%
\pgfsetbuttcap%
\pgfsetroundjoin%
\definecolor{currentfill}{rgb}{0.969359,0.803954,0.693832}%
\pgfsetfillcolor{currentfill}%
\pgfsetlinewidth{0.311001pt}%
\definecolor{currentstroke}{rgb}{1.000000,1.000000,1.000000}%
\pgfsetstrokecolor{currentstroke}%
\pgfsetdash{}{0pt}%
\pgfpathmoveto{\pgfqpoint{4.271323in}{1.212387in}}%
\pgfpathcurveto{\pgfqpoint{4.278456in}{1.212387in}}{\pgfqpoint{4.285297in}{1.215221in}}{\pgfqpoint{4.290341in}{1.220265in}}%
\pgfpathcurveto{\pgfqpoint{4.295385in}{1.225309in}}{\pgfqpoint{4.298219in}{1.232150in}}{\pgfqpoint{4.298219in}{1.239283in}}%
\pgfpathcurveto{\pgfqpoint{4.298219in}{1.246416in}}{\pgfqpoint{4.295385in}{1.253258in}}{\pgfqpoint{4.290341in}{1.258301in}}%
\pgfpathcurveto{\pgfqpoint{4.285297in}{1.263345in}}{\pgfqpoint{4.278456in}{1.266179in}}{\pgfqpoint{4.271323in}{1.266179in}}%
\pgfpathcurveto{\pgfqpoint{4.264190in}{1.266179in}}{\pgfqpoint{4.257348in}{1.263345in}}{\pgfqpoint{4.252305in}{1.258301in}}%
\pgfpathcurveto{\pgfqpoint{4.247261in}{1.253258in}}{\pgfqpoint{4.244427in}{1.246416in}}{\pgfqpoint{4.244427in}{1.239283in}}%
\pgfpathcurveto{\pgfqpoint{4.244427in}{1.232150in}}{\pgfqpoint{4.247261in}{1.225309in}}{\pgfqpoint{4.252305in}{1.220265in}}%
\pgfpathcurveto{\pgfqpoint{4.257348in}{1.215221in}}{\pgfqpoint{4.264190in}{1.212387in}}{\pgfqpoint{4.271323in}{1.212387in}}%
\pgfpathclose%
\pgfusepath{stroke,fill}%
\end{pgfscope}%
\begin{pgfscope}%
\pgfpathrectangle{\pgfqpoint{2.867647in}{0.500000in}}{\pgfqpoint{1.764706in}{1.700000in}}%
\pgfusepath{clip}%
\pgfsetbuttcap%
\pgfsetroundjoin%
\definecolor{currentfill}{rgb}{0.979124,0.903132,0.839793}%
\pgfsetfillcolor{currentfill}%
\pgfsetlinewidth{0.311001pt}%
\definecolor{currentstroke}{rgb}{1.000000,1.000000,1.000000}%
\pgfsetstrokecolor{currentstroke}%
\pgfsetdash{}{0pt}%
\pgfpathmoveto{\pgfqpoint{4.148366in}{1.448843in}}%
\pgfpathcurveto{\pgfqpoint{4.155499in}{1.448843in}}{\pgfqpoint{4.162340in}{1.451677in}}{\pgfqpoint{4.167384in}{1.456721in}}%
\pgfpathcurveto{\pgfqpoint{4.172428in}{1.461765in}}{\pgfqpoint{4.175262in}{1.468606in}}{\pgfqpoint{4.175262in}{1.475739in}}%
\pgfpathcurveto{\pgfqpoint{4.175262in}{1.482872in}}{\pgfqpoint{4.172428in}{1.489713in}}{\pgfqpoint{4.167384in}{1.494757in}}%
\pgfpathcurveto{\pgfqpoint{4.162340in}{1.499801in}}{\pgfqpoint{4.155499in}{1.502635in}}{\pgfqpoint{4.148366in}{1.502635in}}%
\pgfpathcurveto{\pgfqpoint{4.141233in}{1.502635in}}{\pgfqpoint{4.134391in}{1.499801in}}{\pgfqpoint{4.129348in}{1.494757in}}%
\pgfpathcurveto{\pgfqpoint{4.124304in}{1.489713in}}{\pgfqpoint{4.121470in}{1.482872in}}{\pgfqpoint{4.121470in}{1.475739in}}%
\pgfpathcurveto{\pgfqpoint{4.121470in}{1.468606in}}{\pgfqpoint{4.124304in}{1.461765in}}{\pgfqpoint{4.129348in}{1.456721in}}%
\pgfpathcurveto{\pgfqpoint{4.134391in}{1.451677in}}{\pgfqpoint{4.141233in}{1.448843in}}{\pgfqpoint{4.148366in}{1.448843in}}%
\pgfpathclose%
\pgfusepath{stroke,fill}%
\end{pgfscope}%
\begin{pgfscope}%
\pgfpathrectangle{\pgfqpoint{2.867647in}{0.500000in}}{\pgfqpoint{1.764706in}{1.700000in}}%
\pgfusepath{clip}%
\pgfsetbuttcap%
\pgfsetroundjoin%
\definecolor{currentfill}{rgb}{0.965928,0.738443,0.600540}%
\pgfsetfillcolor{currentfill}%
\pgfsetlinewidth{0.311001pt}%
\definecolor{currentstroke}{rgb}{1.000000,1.000000,1.000000}%
\pgfsetstrokecolor{currentstroke}%
\pgfsetdash{}{0pt}%
\pgfpathmoveto{\pgfqpoint{4.043162in}{1.507874in}}%
\pgfpathcurveto{\pgfqpoint{4.050295in}{1.507874in}}{\pgfqpoint{4.057137in}{1.510708in}}{\pgfqpoint{4.062180in}{1.515751in}}%
\pgfpathcurveto{\pgfqpoint{4.067224in}{1.520795in}}{\pgfqpoint{4.070058in}{1.527637in}}{\pgfqpoint{4.070058in}{1.534770in}}%
\pgfpathcurveto{\pgfqpoint{4.070058in}{1.541902in}}{\pgfqpoint{4.067224in}{1.548744in}}{\pgfqpoint{4.062180in}{1.553788in}}%
\pgfpathcurveto{\pgfqpoint{4.057137in}{1.558831in}}{\pgfqpoint{4.050295in}{1.561665in}}{\pgfqpoint{4.043162in}{1.561665in}}%
\pgfpathcurveto{\pgfqpoint{4.036029in}{1.561665in}}{\pgfqpoint{4.029188in}{1.558831in}}{\pgfqpoint{4.024144in}{1.553788in}}%
\pgfpathcurveto{\pgfqpoint{4.019100in}{1.548744in}}{\pgfqpoint{4.016266in}{1.541902in}}{\pgfqpoint{4.016266in}{1.534770in}}%
\pgfpathcurveto{\pgfqpoint{4.016266in}{1.527637in}}{\pgfqpoint{4.019100in}{1.520795in}}{\pgfqpoint{4.024144in}{1.515751in}}%
\pgfpathcurveto{\pgfqpoint{4.029188in}{1.510708in}}{\pgfqpoint{4.036029in}{1.507874in}}{\pgfqpoint{4.043162in}{1.507874in}}%
\pgfpathclose%
\pgfusepath{stroke,fill}%
\end{pgfscope}%
\begin{pgfscope}%
\pgfpathrectangle{\pgfqpoint{2.867647in}{0.500000in}}{\pgfqpoint{1.764706in}{1.700000in}}%
\pgfusepath{clip}%
\pgfsetbuttcap%
\pgfsetroundjoin%
\definecolor{currentfill}{rgb}{0.958791,0.526283,0.368909}%
\pgfsetfillcolor{currentfill}%
\pgfsetlinewidth{0.311001pt}%
\definecolor{currentstroke}{rgb}{1.000000,1.000000,1.000000}%
\pgfsetstrokecolor{currentstroke}%
\pgfsetdash{}{0pt}%
\pgfpathmoveto{\pgfqpoint{3.961109in}{0.848518in}}%
\pgfpathcurveto{\pgfqpoint{3.968242in}{0.848518in}}{\pgfqpoint{3.975084in}{0.851352in}}{\pgfqpoint{3.980127in}{0.856395in}}%
\pgfpathcurveto{\pgfqpoint{3.985171in}{0.861439in}}{\pgfqpoint{3.988005in}{0.868281in}}{\pgfqpoint{3.988005in}{0.875413in}}%
\pgfpathcurveto{\pgfqpoint{3.988005in}{0.882546in}}{\pgfqpoint{3.985171in}{0.889388in}}{\pgfqpoint{3.980127in}{0.894432in}}%
\pgfpathcurveto{\pgfqpoint{3.975084in}{0.899475in}}{\pgfqpoint{3.968242in}{0.902309in}}{\pgfqpoint{3.961109in}{0.902309in}}%
\pgfpathcurveto{\pgfqpoint{3.953976in}{0.902309in}}{\pgfqpoint{3.947135in}{0.899475in}}{\pgfqpoint{3.942091in}{0.894432in}}%
\pgfpathcurveto{\pgfqpoint{3.937047in}{0.889388in}}{\pgfqpoint{3.934214in}{0.882546in}}{\pgfqpoint{3.934214in}{0.875413in}}%
\pgfpathcurveto{\pgfqpoint{3.934214in}{0.868281in}}{\pgfqpoint{3.937047in}{0.861439in}}{\pgfqpoint{3.942091in}{0.856395in}}%
\pgfpathcurveto{\pgfqpoint{3.947135in}{0.851352in}}{\pgfqpoint{3.953976in}{0.848518in}}{\pgfqpoint{3.961109in}{0.848518in}}%
\pgfpathclose%
\pgfusepath{stroke,fill}%
\end{pgfscope}%
\begin{pgfscope}%
\pgfpathrectangle{\pgfqpoint{2.867647in}{0.500000in}}{\pgfqpoint{1.764706in}{1.700000in}}%
\pgfusepath{clip}%
\pgfsetbuttcap%
\pgfsetroundjoin%
\definecolor{currentfill}{rgb}{0.958331,0.519463,0.362986}%
\pgfsetfillcolor{currentfill}%
\pgfsetlinewidth{0.311001pt}%
\definecolor{currentstroke}{rgb}{1.000000,1.000000,1.000000}%
\pgfsetstrokecolor{currentstroke}%
\pgfsetdash{}{0pt}%
\pgfpathmoveto{\pgfqpoint{4.347422in}{1.284743in}}%
\pgfpathcurveto{\pgfqpoint{4.354555in}{1.284743in}}{\pgfqpoint{4.361396in}{1.287577in}}{\pgfqpoint{4.366440in}{1.292620in}}%
\pgfpathcurveto{\pgfqpoint{4.371484in}{1.297664in}}{\pgfqpoint{4.374318in}{1.304506in}}{\pgfqpoint{4.374318in}{1.311638in}}%
\pgfpathcurveto{\pgfqpoint{4.374318in}{1.318771in}}{\pgfqpoint{4.371484in}{1.325613in}}{\pgfqpoint{4.366440in}{1.330657in}}%
\pgfpathcurveto{\pgfqpoint{4.361396in}{1.335700in}}{\pgfqpoint{4.354555in}{1.338534in}}{\pgfqpoint{4.347422in}{1.338534in}}%
\pgfpathcurveto{\pgfqpoint{4.340289in}{1.338534in}}{\pgfqpoint{4.333447in}{1.335700in}}{\pgfqpoint{4.328404in}{1.330657in}}%
\pgfpathcurveto{\pgfqpoint{4.323360in}{1.325613in}}{\pgfqpoint{4.320526in}{1.318771in}}{\pgfqpoint{4.320526in}{1.311638in}}%
\pgfpathcurveto{\pgfqpoint{4.320526in}{1.304506in}}{\pgfqpoint{4.323360in}{1.297664in}}{\pgfqpoint{4.328404in}{1.292620in}}%
\pgfpathcurveto{\pgfqpoint{4.333447in}{1.287577in}}{\pgfqpoint{4.340289in}{1.284743in}}{\pgfqpoint{4.347422in}{1.284743in}}%
\pgfpathclose%
\pgfusepath{stroke,fill}%
\end{pgfscope}%
\begin{pgfscope}%
\pgfpathrectangle{\pgfqpoint{2.867647in}{0.500000in}}{\pgfqpoint{1.764706in}{1.700000in}}%
\pgfusepath{clip}%
\pgfsetbuttcap%
\pgfsetroundjoin%
\definecolor{currentfill}{rgb}{0.971694,0.833208,0.737161}%
\pgfsetfillcolor{currentfill}%
\pgfsetlinewidth{0.311001pt}%
\definecolor{currentstroke}{rgb}{1.000000,1.000000,1.000000}%
\pgfsetstrokecolor{currentstroke}%
\pgfsetdash{}{0pt}%
\pgfpathmoveto{\pgfqpoint{4.115313in}{1.310493in}}%
\pgfpathcurveto{\pgfqpoint{4.122446in}{1.310493in}}{\pgfqpoint{4.129288in}{1.313327in}}{\pgfqpoint{4.134332in}{1.318371in}}%
\pgfpathcurveto{\pgfqpoint{4.139375in}{1.323415in}}{\pgfqpoint{4.142209in}{1.330256in}}{\pgfqpoint{4.142209in}{1.337389in}}%
\pgfpathcurveto{\pgfqpoint{4.142209in}{1.344522in}}{\pgfqpoint{4.139375in}{1.351364in}}{\pgfqpoint{4.134332in}{1.356407in}}%
\pgfpathcurveto{\pgfqpoint{4.129288in}{1.361451in}}{\pgfqpoint{4.122446in}{1.364285in}}{\pgfqpoint{4.115313in}{1.364285in}}%
\pgfpathcurveto{\pgfqpoint{4.108181in}{1.364285in}}{\pgfqpoint{4.101339in}{1.361451in}}{\pgfqpoint{4.096295in}{1.356407in}}%
\pgfpathcurveto{\pgfqpoint{4.091252in}{1.351364in}}{\pgfqpoint{4.088418in}{1.344522in}}{\pgfqpoint{4.088418in}{1.337389in}}%
\pgfpathcurveto{\pgfqpoint{4.088418in}{1.330256in}}{\pgfqpoint{4.091252in}{1.323415in}}{\pgfqpoint{4.096295in}{1.318371in}}%
\pgfpathcurveto{\pgfqpoint{4.101339in}{1.313327in}}{\pgfqpoint{4.108181in}{1.310493in}}{\pgfqpoint{4.115313in}{1.310493in}}%
\pgfpathclose%
\pgfusepath{stroke,fill}%
\end{pgfscope}%
\begin{pgfscope}%
\pgfpathrectangle{\pgfqpoint{2.867647in}{0.500000in}}{\pgfqpoint{1.764706in}{1.700000in}}%
\pgfusepath{clip}%
\pgfsetbuttcap%
\pgfsetroundjoin%
\definecolor{currentfill}{rgb}{0.969803,0.809811,0.702523}%
\pgfsetfillcolor{currentfill}%
\pgfsetlinewidth{0.311001pt}%
\definecolor{currentstroke}{rgb}{1.000000,1.000000,1.000000}%
\pgfsetstrokecolor{currentstroke}%
\pgfsetdash{}{0pt}%
\pgfpathmoveto{\pgfqpoint{4.084108in}{1.174295in}}%
\pgfpathcurveto{\pgfqpoint{4.091241in}{1.174295in}}{\pgfqpoint{4.098082in}{1.177129in}}{\pgfqpoint{4.103126in}{1.182173in}}%
\pgfpathcurveto{\pgfqpoint{4.108170in}{1.187216in}}{\pgfqpoint{4.111004in}{1.194058in}}{\pgfqpoint{4.111004in}{1.201191in}}%
\pgfpathcurveto{\pgfqpoint{4.111004in}{1.208324in}}{\pgfqpoint{4.108170in}{1.215165in}}{\pgfqpoint{4.103126in}{1.220209in}}%
\pgfpathcurveto{\pgfqpoint{4.098082in}{1.225253in}}{\pgfqpoint{4.091241in}{1.228087in}}{\pgfqpoint{4.084108in}{1.228087in}}%
\pgfpathcurveto{\pgfqpoint{4.076975in}{1.228087in}}{\pgfqpoint{4.070133in}{1.225253in}}{\pgfqpoint{4.065090in}{1.220209in}}%
\pgfpathcurveto{\pgfqpoint{4.060046in}{1.215165in}}{\pgfqpoint{4.057212in}{1.208324in}}{\pgfqpoint{4.057212in}{1.201191in}}%
\pgfpathcurveto{\pgfqpoint{4.057212in}{1.194058in}}{\pgfqpoint{4.060046in}{1.187216in}}{\pgfqpoint{4.065090in}{1.182173in}}%
\pgfpathcurveto{\pgfqpoint{4.070133in}{1.177129in}}{\pgfqpoint{4.076975in}{1.174295in}}{\pgfqpoint{4.084108in}{1.174295in}}%
\pgfpathclose%
\pgfusepath{stroke,fill}%
\end{pgfscope}%
\begin{pgfscope}%
\pgfpathrectangle{\pgfqpoint{2.867647in}{0.500000in}}{\pgfqpoint{1.764706in}{1.700000in}}%
\pgfusepath{clip}%
\pgfsetbuttcap%
\pgfsetroundjoin%
\definecolor{currentfill}{rgb}{0.971694,0.833208,0.737161}%
\pgfsetfillcolor{currentfill}%
\pgfsetlinewidth{0.311001pt}%
\definecolor{currentstroke}{rgb}{1.000000,1.000000,1.000000}%
\pgfsetstrokecolor{currentstroke}%
\pgfsetdash{}{0pt}%
\pgfpathmoveto{\pgfqpoint{4.105147in}{1.226792in}}%
\pgfpathcurveto{\pgfqpoint{4.112279in}{1.226792in}}{\pgfqpoint{4.119121in}{1.229626in}}{\pgfqpoint{4.124165in}{1.234670in}}%
\pgfpathcurveto{\pgfqpoint{4.129208in}{1.239714in}}{\pgfqpoint{4.132042in}{1.246555in}}{\pgfqpoint{4.132042in}{1.253688in}}%
\pgfpathcurveto{\pgfqpoint{4.132042in}{1.260821in}}{\pgfqpoint{4.129208in}{1.267663in}}{\pgfqpoint{4.124165in}{1.272706in}}%
\pgfpathcurveto{\pgfqpoint{4.119121in}{1.277750in}}{\pgfqpoint{4.112279in}{1.280584in}}{\pgfqpoint{4.105147in}{1.280584in}}%
\pgfpathcurveto{\pgfqpoint{4.098014in}{1.280584in}}{\pgfqpoint{4.091172in}{1.277750in}}{\pgfqpoint{4.086128in}{1.272706in}}%
\pgfpathcurveto{\pgfqpoint{4.081085in}{1.267663in}}{\pgfqpoint{4.078251in}{1.260821in}}{\pgfqpoint{4.078251in}{1.253688in}}%
\pgfpathcurveto{\pgfqpoint{4.078251in}{1.246555in}}{\pgfqpoint{4.081085in}{1.239714in}}{\pgfqpoint{4.086128in}{1.234670in}}%
\pgfpathcurveto{\pgfqpoint{4.091172in}{1.229626in}}{\pgfqpoint{4.098014in}{1.226792in}}{\pgfqpoint{4.105147in}{1.226792in}}%
\pgfpathclose%
\pgfusepath{stroke,fill}%
\end{pgfscope}%
\begin{pgfscope}%
\pgfpathrectangle{\pgfqpoint{2.867647in}{0.500000in}}{\pgfqpoint{1.764706in}{1.700000in}}%
\pgfusepath{clip}%
\pgfsetbuttcap%
\pgfsetroundjoin%
\definecolor{currentfill}{rgb}{0.978376,0.897317,0.831308}%
\pgfsetfillcolor{currentfill}%
\pgfsetlinewidth{0.311001pt}%
\definecolor{currentstroke}{rgb}{1.000000,1.000000,1.000000}%
\pgfsetstrokecolor{currentstroke}%
\pgfsetdash{}{0pt}%
\pgfpathmoveto{\pgfqpoint{4.171037in}{1.082016in}}%
\pgfpathcurveto{\pgfqpoint{4.178170in}{1.082016in}}{\pgfqpoint{4.185012in}{1.084850in}}{\pgfqpoint{4.190056in}{1.089894in}}%
\pgfpathcurveto{\pgfqpoint{4.195099in}{1.094937in}}{\pgfqpoint{4.197933in}{1.101779in}}{\pgfqpoint{4.197933in}{1.108912in}}%
\pgfpathcurveto{\pgfqpoint{4.197933in}{1.116045in}}{\pgfqpoint{4.195099in}{1.122886in}}{\pgfqpoint{4.190056in}{1.127930in}}%
\pgfpathcurveto{\pgfqpoint{4.185012in}{1.132974in}}{\pgfqpoint{4.178170in}{1.135807in}}{\pgfqpoint{4.171037in}{1.135807in}}%
\pgfpathcurveto{\pgfqpoint{4.163905in}{1.135807in}}{\pgfqpoint{4.157063in}{1.132974in}}{\pgfqpoint{4.152019in}{1.127930in}}%
\pgfpathcurveto{\pgfqpoint{4.146976in}{1.122886in}}{\pgfqpoint{4.144142in}{1.116045in}}{\pgfqpoint{4.144142in}{1.108912in}}%
\pgfpathcurveto{\pgfqpoint{4.144142in}{1.101779in}}{\pgfqpoint{4.146976in}{1.094937in}}{\pgfqpoint{4.152019in}{1.089894in}}%
\pgfpathcurveto{\pgfqpoint{4.157063in}{1.084850in}}{\pgfqpoint{4.163905in}{1.082016in}}{\pgfqpoint{4.171037in}{1.082016in}}%
\pgfpathclose%
\pgfusepath{stroke,fill}%
\end{pgfscope}%
\begin{pgfscope}%
\pgfpathrectangle{\pgfqpoint{2.867647in}{0.500000in}}{\pgfqpoint{1.764706in}{1.700000in}}%
\pgfusepath{clip}%
\pgfsetbuttcap%
\pgfsetroundjoin%
\definecolor{currentfill}{rgb}{0.978376,0.897317,0.831308}%
\pgfsetfillcolor{currentfill}%
\pgfsetlinewidth{0.311001pt}%
\definecolor{currentstroke}{rgb}{1.000000,1.000000,1.000000}%
\pgfsetstrokecolor{currentstroke}%
\pgfsetdash{}{0pt}%
\pgfpathmoveto{\pgfqpoint{4.145121in}{1.241408in}}%
\pgfpathcurveto{\pgfqpoint{4.152253in}{1.241408in}}{\pgfqpoint{4.159095in}{1.244242in}}{\pgfqpoint{4.164139in}{1.249286in}}%
\pgfpathcurveto{\pgfqpoint{4.169182in}{1.254330in}}{\pgfqpoint{4.172016in}{1.261171in}}{\pgfqpoint{4.172016in}{1.268304in}}%
\pgfpathcurveto{\pgfqpoint{4.172016in}{1.275437in}}{\pgfqpoint{4.169182in}{1.282278in}}{\pgfqpoint{4.164139in}{1.287322in}}%
\pgfpathcurveto{\pgfqpoint{4.159095in}{1.292366in}}{\pgfqpoint{4.152253in}{1.295200in}}{\pgfqpoint{4.145121in}{1.295200in}}%
\pgfpathcurveto{\pgfqpoint{4.137988in}{1.295200in}}{\pgfqpoint{4.131146in}{1.292366in}}{\pgfqpoint{4.126102in}{1.287322in}}%
\pgfpathcurveto{\pgfqpoint{4.121059in}{1.282278in}}{\pgfqpoint{4.118225in}{1.275437in}}{\pgfqpoint{4.118225in}{1.268304in}}%
\pgfpathcurveto{\pgfqpoint{4.118225in}{1.261171in}}{\pgfqpoint{4.121059in}{1.254330in}}{\pgfqpoint{4.126102in}{1.249286in}}%
\pgfpathcurveto{\pgfqpoint{4.131146in}{1.244242in}}{\pgfqpoint{4.137988in}{1.241408in}}{\pgfqpoint{4.145121in}{1.241408in}}%
\pgfpathclose%
\pgfusepath{stroke,fill}%
\end{pgfscope}%
\begin{pgfscope}%
\pgfpathrectangle{\pgfqpoint{2.867647in}{0.500000in}}{\pgfqpoint{1.764706in}{1.700000in}}%
\pgfusepath{clip}%
\pgfsetbuttcap%
\pgfsetroundjoin%
\definecolor{currentfill}{rgb}{0.966120,0.744512,0.608720}%
\pgfsetfillcolor{currentfill}%
\pgfsetlinewidth{0.311001pt}%
\definecolor{currentstroke}{rgb}{1.000000,1.000000,1.000000}%
\pgfsetstrokecolor{currentstroke}%
\pgfsetdash{}{0pt}%
\pgfpathmoveto{\pgfqpoint{4.073468in}{1.414479in}}%
\pgfpathcurveto{\pgfqpoint{4.080601in}{1.414479in}}{\pgfqpoint{4.087442in}{1.417313in}}{\pgfqpoint{4.092486in}{1.422357in}}%
\pgfpathcurveto{\pgfqpoint{4.097530in}{1.427400in}}{\pgfqpoint{4.100364in}{1.434242in}}{\pgfqpoint{4.100364in}{1.441375in}}%
\pgfpathcurveto{\pgfqpoint{4.100364in}{1.448508in}}{\pgfqpoint{4.097530in}{1.455349in}}{\pgfqpoint{4.092486in}{1.460393in}}%
\pgfpathcurveto{\pgfqpoint{4.087442in}{1.465436in}}{\pgfqpoint{4.080601in}{1.468270in}}{\pgfqpoint{4.073468in}{1.468270in}}%
\pgfpathcurveto{\pgfqpoint{4.066335in}{1.468270in}}{\pgfqpoint{4.059493in}{1.465436in}}{\pgfqpoint{4.054450in}{1.460393in}}%
\pgfpathcurveto{\pgfqpoint{4.049406in}{1.455349in}}{\pgfqpoint{4.046572in}{1.448508in}}{\pgfqpoint{4.046572in}{1.441375in}}%
\pgfpathcurveto{\pgfqpoint{4.046572in}{1.434242in}}{\pgfqpoint{4.049406in}{1.427400in}}{\pgfqpoint{4.054450in}{1.422357in}}%
\pgfpathcurveto{\pgfqpoint{4.059493in}{1.417313in}}{\pgfqpoint{4.066335in}{1.414479in}}{\pgfqpoint{4.073468in}{1.414479in}}%
\pgfpathclose%
\pgfusepath{stroke,fill}%
\end{pgfscope}%
\begin{pgfscope}%
\pgfpathrectangle{\pgfqpoint{2.867647in}{0.500000in}}{\pgfqpoint{1.764706in}{1.700000in}}%
\pgfusepath{clip}%
\pgfsetbuttcap%
\pgfsetroundjoin%
\definecolor{currentfill}{rgb}{0.979891,0.908948,0.848279}%
\pgfsetfillcolor{currentfill}%
\pgfsetlinewidth{0.311001pt}%
\definecolor{currentstroke}{rgb}{1.000000,1.000000,1.000000}%
\pgfsetstrokecolor{currentstroke}%
\pgfsetdash{}{0pt}%
\pgfpathmoveto{\pgfqpoint{4.203053in}{1.402705in}}%
\pgfpathcurveto{\pgfqpoint{4.210186in}{1.402705in}}{\pgfqpoint{4.217028in}{1.405539in}}{\pgfqpoint{4.222071in}{1.410583in}}%
\pgfpathcurveto{\pgfqpoint{4.227115in}{1.415626in}}{\pgfqpoint{4.229949in}{1.422468in}}{\pgfqpoint{4.229949in}{1.429601in}}%
\pgfpathcurveto{\pgfqpoint{4.229949in}{1.436734in}}{\pgfqpoint{4.227115in}{1.443575in}}{\pgfqpoint{4.222071in}{1.448619in}}%
\pgfpathcurveto{\pgfqpoint{4.217028in}{1.453663in}}{\pgfqpoint{4.210186in}{1.456497in}}{\pgfqpoint{4.203053in}{1.456497in}}%
\pgfpathcurveto{\pgfqpoint{4.195920in}{1.456497in}}{\pgfqpoint{4.189079in}{1.453663in}}{\pgfqpoint{4.184035in}{1.448619in}}%
\pgfpathcurveto{\pgfqpoint{4.178991in}{1.443575in}}{\pgfqpoint{4.176158in}{1.436734in}}{\pgfqpoint{4.176158in}{1.429601in}}%
\pgfpathcurveto{\pgfqpoint{4.176158in}{1.422468in}}{\pgfqpoint{4.178991in}{1.415626in}}{\pgfqpoint{4.184035in}{1.410583in}}%
\pgfpathcurveto{\pgfqpoint{4.189079in}{1.405539in}}{\pgfqpoint{4.195920in}{1.402705in}}{\pgfqpoint{4.203053in}{1.402705in}}%
\pgfpathclose%
\pgfusepath{stroke,fill}%
\end{pgfscope}%
\begin{pgfscope}%
\pgfpathrectangle{\pgfqpoint{2.867647in}{0.500000in}}{\pgfqpoint{1.764706in}{1.700000in}}%
\pgfusepath{clip}%
\pgfsetbuttcap%
\pgfsetroundjoin%
\definecolor{currentfill}{rgb}{0.972201,0.839051,0.745789}%
\pgfsetfillcolor{currentfill}%
\pgfsetlinewidth{0.311001pt}%
\definecolor{currentstroke}{rgb}{1.000000,1.000000,1.000000}%
\pgfsetstrokecolor{currentstroke}%
\pgfsetdash{}{0pt}%
\pgfpathmoveto{\pgfqpoint{4.060575in}{1.638107in}}%
\pgfpathcurveto{\pgfqpoint{4.067708in}{1.638107in}}{\pgfqpoint{4.074549in}{1.640940in}}{\pgfqpoint{4.079593in}{1.645984in}}%
\pgfpathcurveto{\pgfqpoint{4.084637in}{1.651028in}}{\pgfqpoint{4.087470in}{1.657869in}}{\pgfqpoint{4.087470in}{1.665002in}}%
\pgfpathcurveto{\pgfqpoint{4.087470in}{1.672135in}}{\pgfqpoint{4.084637in}{1.678977in}}{\pgfqpoint{4.079593in}{1.684020in}}%
\pgfpathcurveto{\pgfqpoint{4.074549in}{1.689064in}}{\pgfqpoint{4.067708in}{1.691898in}}{\pgfqpoint{4.060575in}{1.691898in}}%
\pgfpathcurveto{\pgfqpoint{4.053442in}{1.691898in}}{\pgfqpoint{4.046600in}{1.689064in}}{\pgfqpoint{4.041557in}{1.684020in}}%
\pgfpathcurveto{\pgfqpoint{4.036513in}{1.678977in}}{\pgfqpoint{4.033679in}{1.672135in}}{\pgfqpoint{4.033679in}{1.665002in}}%
\pgfpathcurveto{\pgfqpoint{4.033679in}{1.657869in}}{\pgfqpoint{4.036513in}{1.651028in}}{\pgfqpoint{4.041557in}{1.645984in}}%
\pgfpathcurveto{\pgfqpoint{4.046600in}{1.640940in}}{\pgfqpoint{4.053442in}{1.638107in}}{\pgfqpoint{4.060575in}{1.638107in}}%
\pgfpathclose%
\pgfusepath{stroke,fill}%
\end{pgfscope}%
\begin{pgfscope}%
\pgfpathrectangle{\pgfqpoint{2.867647in}{0.500000in}}{\pgfqpoint{1.764706in}{1.700000in}}%
\pgfusepath{clip}%
\pgfsetbuttcap%
\pgfsetroundjoin%
\definecolor{currentfill}{rgb}{0.979891,0.908948,0.848279}%
\pgfsetfillcolor{currentfill}%
\pgfsetlinewidth{0.311001pt}%
\definecolor{currentstroke}{rgb}{1.000000,1.000000,1.000000}%
\pgfsetstrokecolor{currentstroke}%
\pgfsetdash{}{0pt}%
\pgfpathmoveto{\pgfqpoint{4.183585in}{1.498836in}}%
\pgfpathcurveto{\pgfqpoint{4.190718in}{1.498836in}}{\pgfqpoint{4.197559in}{1.501670in}}{\pgfqpoint{4.202603in}{1.506713in}}%
\pgfpathcurveto{\pgfqpoint{4.207647in}{1.511757in}}{\pgfqpoint{4.210481in}{1.518599in}}{\pgfqpoint{4.210481in}{1.525731in}}%
\pgfpathcurveto{\pgfqpoint{4.210481in}{1.532864in}}{\pgfqpoint{4.207647in}{1.539706in}}{\pgfqpoint{4.202603in}{1.544749in}}%
\pgfpathcurveto{\pgfqpoint{4.197559in}{1.549793in}}{\pgfqpoint{4.190718in}{1.552627in}}{\pgfqpoint{4.183585in}{1.552627in}}%
\pgfpathcurveto{\pgfqpoint{4.176452in}{1.552627in}}{\pgfqpoint{4.169610in}{1.549793in}}{\pgfqpoint{4.164567in}{1.544749in}}%
\pgfpathcurveto{\pgfqpoint{4.159523in}{1.539706in}}{\pgfqpoint{4.156689in}{1.532864in}}{\pgfqpoint{4.156689in}{1.525731in}}%
\pgfpathcurveto{\pgfqpoint{4.156689in}{1.518599in}}{\pgfqpoint{4.159523in}{1.511757in}}{\pgfqpoint{4.164567in}{1.506713in}}%
\pgfpathcurveto{\pgfqpoint{4.169610in}{1.501670in}}{\pgfqpoint{4.176452in}{1.498836in}}{\pgfqpoint{4.183585in}{1.498836in}}%
\pgfpathclose%
\pgfusepath{stroke,fill}%
\end{pgfscope}%
\begin{pgfscope}%
\pgfpathrectangle{\pgfqpoint{2.867647in}{0.500000in}}{\pgfqpoint{1.764706in}{1.700000in}}%
\pgfusepath{clip}%
\pgfsetbuttcap%
\pgfsetroundjoin%
\definecolor{currentfill}{rgb}{0.967092,0.768560,0.642079}%
\pgfsetfillcolor{currentfill}%
\pgfsetlinewidth{0.311001pt}%
\definecolor{currentstroke}{rgb}{1.000000,1.000000,1.000000}%
\pgfsetstrokecolor{currentstroke}%
\pgfsetdash{}{0pt}%
\pgfpathmoveto{\pgfqpoint{4.170472in}{0.968255in}}%
\pgfpathcurveto{\pgfqpoint{4.177605in}{0.968255in}}{\pgfqpoint{4.184446in}{0.971089in}}{\pgfqpoint{4.189490in}{0.976132in}}%
\pgfpathcurveto{\pgfqpoint{4.194534in}{0.981176in}}{\pgfqpoint{4.197368in}{0.988018in}}{\pgfqpoint{4.197368in}{0.995151in}}%
\pgfpathcurveto{\pgfqpoint{4.197368in}{1.002283in}}{\pgfqpoint{4.194534in}{1.009125in}}{\pgfqpoint{4.189490in}{1.014169in}}%
\pgfpathcurveto{\pgfqpoint{4.184446in}{1.019212in}}{\pgfqpoint{4.177605in}{1.022046in}}{\pgfqpoint{4.170472in}{1.022046in}}%
\pgfpathcurveto{\pgfqpoint{4.163339in}{1.022046in}}{\pgfqpoint{4.156497in}{1.019212in}}{\pgfqpoint{4.151454in}{1.014169in}}%
\pgfpathcurveto{\pgfqpoint{4.146410in}{1.009125in}}{\pgfqpoint{4.143576in}{1.002283in}}{\pgfqpoint{4.143576in}{0.995151in}}%
\pgfpathcurveto{\pgfqpoint{4.143576in}{0.988018in}}{\pgfqpoint{4.146410in}{0.981176in}}{\pgfqpoint{4.151454in}{0.976132in}}%
\pgfpathcurveto{\pgfqpoint{4.156497in}{0.971089in}}{\pgfqpoint{4.163339in}{0.968255in}}{\pgfqpoint{4.170472in}{0.968255in}}%
\pgfpathclose%
\pgfusepath{stroke,fill}%
\end{pgfscope}%
\begin{pgfscope}%
\pgfpathrectangle{\pgfqpoint{2.867647in}{0.500000in}}{\pgfqpoint{1.764706in}{1.700000in}}%
\pgfusepath{clip}%
\pgfsetbuttcap%
\pgfsetroundjoin%
\definecolor{currentfill}{rgb}{0.961734,0.579886,0.418445}%
\pgfsetfillcolor{currentfill}%
\pgfsetlinewidth{0.311001pt}%
\definecolor{currentstroke}{rgb}{1.000000,1.000000,1.000000}%
\pgfsetstrokecolor{currentstroke}%
\pgfsetdash{}{0pt}%
\pgfpathmoveto{\pgfqpoint{4.323529in}{1.180306in}}%
\pgfpathcurveto{\pgfqpoint{4.330662in}{1.180306in}}{\pgfqpoint{4.337504in}{1.183140in}}{\pgfqpoint{4.342548in}{1.188183in}}%
\pgfpathcurveto{\pgfqpoint{4.347591in}{1.193227in}}{\pgfqpoint{4.350425in}{1.200069in}}{\pgfqpoint{4.350425in}{1.207201in}}%
\pgfpathcurveto{\pgfqpoint{4.350425in}{1.214334in}}{\pgfqpoint{4.347591in}{1.221176in}}{\pgfqpoint{4.342548in}{1.226220in}}%
\pgfpathcurveto{\pgfqpoint{4.337504in}{1.231263in}}{\pgfqpoint{4.330662in}{1.234097in}}{\pgfqpoint{4.323529in}{1.234097in}}%
\pgfpathcurveto{\pgfqpoint{4.316397in}{1.234097in}}{\pgfqpoint{4.309555in}{1.231263in}}{\pgfqpoint{4.304511in}{1.226220in}}%
\pgfpathcurveto{\pgfqpoint{4.299468in}{1.221176in}}{\pgfqpoint{4.296634in}{1.214334in}}{\pgfqpoint{4.296634in}{1.207201in}}%
\pgfpathcurveto{\pgfqpoint{4.296634in}{1.200069in}}{\pgfqpoint{4.299468in}{1.193227in}}{\pgfqpoint{4.304511in}{1.188183in}}%
\pgfpathcurveto{\pgfqpoint{4.309555in}{1.183140in}}{\pgfqpoint{4.316397in}{1.180306in}}{\pgfqpoint{4.323529in}{1.180306in}}%
\pgfpathclose%
\pgfusepath{stroke,fill}%
\end{pgfscope}%
\begin{pgfscope}%
\pgfpathrectangle{\pgfqpoint{2.867647in}{0.500000in}}{\pgfqpoint{1.764706in}{1.700000in}}%
\pgfusepath{clip}%
\pgfsetbuttcap%
\pgfsetroundjoin%
\definecolor{currentfill}{rgb}{0.969803,0.809811,0.702523}%
\pgfsetfillcolor{currentfill}%
\pgfsetlinewidth{0.311001pt}%
\definecolor{currentstroke}{rgb}{1.000000,1.000000,1.000000}%
\pgfsetstrokecolor{currentstroke}%
\pgfsetdash{}{0pt}%
\pgfpathmoveto{\pgfqpoint{4.204146in}{1.608033in}}%
\pgfpathcurveto{\pgfqpoint{4.211279in}{1.608033in}}{\pgfqpoint{4.218121in}{1.610867in}}{\pgfqpoint{4.223164in}{1.615910in}}%
\pgfpathcurveto{\pgfqpoint{4.228208in}{1.620954in}}{\pgfqpoint{4.231042in}{1.627796in}}{\pgfqpoint{4.231042in}{1.634928in}}%
\pgfpathcurveto{\pgfqpoint{4.231042in}{1.642061in}}{\pgfqpoint{4.228208in}{1.648903in}}{\pgfqpoint{4.223164in}{1.653946in}}%
\pgfpathcurveto{\pgfqpoint{4.218121in}{1.658990in}}{\pgfqpoint{4.211279in}{1.661824in}}{\pgfqpoint{4.204146in}{1.661824in}}%
\pgfpathcurveto{\pgfqpoint{4.197013in}{1.661824in}}{\pgfqpoint{4.190172in}{1.658990in}}{\pgfqpoint{4.185128in}{1.653946in}}%
\pgfpathcurveto{\pgfqpoint{4.180084in}{1.648903in}}{\pgfqpoint{4.177250in}{1.642061in}}{\pgfqpoint{4.177250in}{1.634928in}}%
\pgfpathcurveto{\pgfqpoint{4.177250in}{1.627796in}}{\pgfqpoint{4.180084in}{1.620954in}}{\pgfqpoint{4.185128in}{1.615910in}}%
\pgfpathcurveto{\pgfqpoint{4.190172in}{1.610867in}}{\pgfqpoint{4.197013in}{1.608033in}}{\pgfqpoint{4.204146in}{1.608033in}}%
\pgfpathclose%
\pgfusepath{stroke,fill}%
\end{pgfscope}%
\begin{pgfscope}%
\pgfpathrectangle{\pgfqpoint{2.867647in}{0.500000in}}{\pgfqpoint{1.764706in}{1.700000in}}%
\pgfusepath{clip}%
\pgfsetbuttcap%
\pgfsetroundjoin%
\definecolor{currentfill}{rgb}{0.978376,0.897317,0.831308}%
\pgfsetfillcolor{currentfill}%
\pgfsetlinewidth{0.311001pt}%
\definecolor{currentstroke}{rgb}{1.000000,1.000000,1.000000}%
\pgfsetstrokecolor{currentstroke}%
\pgfsetdash{}{0pt}%
\pgfpathmoveto{\pgfqpoint{4.150554in}{1.073952in}}%
\pgfpathcurveto{\pgfqpoint{4.157687in}{1.073952in}}{\pgfqpoint{4.164529in}{1.076786in}}{\pgfqpoint{4.169572in}{1.081829in}}%
\pgfpathcurveto{\pgfqpoint{4.174616in}{1.086873in}}{\pgfqpoint{4.177450in}{1.093715in}}{\pgfqpoint{4.177450in}{1.100847in}}%
\pgfpathcurveto{\pgfqpoint{4.177450in}{1.107980in}}{\pgfqpoint{4.174616in}{1.114822in}}{\pgfqpoint{4.169572in}{1.119866in}}%
\pgfpathcurveto{\pgfqpoint{4.164529in}{1.124909in}}{\pgfqpoint{4.157687in}{1.127743in}}{\pgfqpoint{4.150554in}{1.127743in}}%
\pgfpathcurveto{\pgfqpoint{4.143422in}{1.127743in}}{\pgfqpoint{4.136580in}{1.124909in}}{\pgfqpoint{4.131536in}{1.119866in}}%
\pgfpathcurveto{\pgfqpoint{4.126493in}{1.114822in}}{\pgfqpoint{4.123659in}{1.107980in}}{\pgfqpoint{4.123659in}{1.100847in}}%
\pgfpathcurveto{\pgfqpoint{4.123659in}{1.093715in}}{\pgfqpoint{4.126493in}{1.086873in}}{\pgfqpoint{4.131536in}{1.081829in}}%
\pgfpathcurveto{\pgfqpoint{4.136580in}{1.076786in}}{\pgfqpoint{4.143422in}{1.073952in}}{\pgfqpoint{4.150554in}{1.073952in}}%
\pgfpathclose%
\pgfusepath{stroke,fill}%
\end{pgfscope}%
\begin{pgfscope}%
\pgfpathrectangle{\pgfqpoint{2.867647in}{0.500000in}}{\pgfqpoint{1.764706in}{1.700000in}}%
\pgfusepath{clip}%
\pgfsetbuttcap%
\pgfsetroundjoin%
\definecolor{currentfill}{rgb}{0.979124,0.903132,0.839793}%
\pgfsetfillcolor{currentfill}%
\pgfsetlinewidth{0.311001pt}%
\definecolor{currentstroke}{rgb}{1.000000,1.000000,1.000000}%
\pgfsetstrokecolor{currentstroke}%
\pgfsetdash{}{0pt}%
\pgfpathmoveto{\pgfqpoint{4.139860in}{1.091397in}}%
\pgfpathcurveto{\pgfqpoint{4.146992in}{1.091397in}}{\pgfqpoint{4.153834in}{1.094231in}}{\pgfqpoint{4.158878in}{1.099274in}}%
\pgfpathcurveto{\pgfqpoint{4.163921in}{1.104318in}}{\pgfqpoint{4.166755in}{1.111160in}}{\pgfqpoint{4.166755in}{1.118293in}}%
\pgfpathcurveto{\pgfqpoint{4.166755in}{1.125425in}}{\pgfqpoint{4.163921in}{1.132267in}}{\pgfqpoint{4.158878in}{1.137311in}}%
\pgfpathcurveto{\pgfqpoint{4.153834in}{1.142354in}}{\pgfqpoint{4.146992in}{1.145188in}}{\pgfqpoint{4.139860in}{1.145188in}}%
\pgfpathcurveto{\pgfqpoint{4.132727in}{1.145188in}}{\pgfqpoint{4.125885in}{1.142354in}}{\pgfqpoint{4.120842in}{1.137311in}}%
\pgfpathcurveto{\pgfqpoint{4.115798in}{1.132267in}}{\pgfqpoint{4.112964in}{1.125425in}}{\pgfqpoint{4.112964in}{1.118293in}}%
\pgfpathcurveto{\pgfqpoint{4.112964in}{1.111160in}}{\pgfqpoint{4.115798in}{1.104318in}}{\pgfqpoint{4.120842in}{1.099274in}}%
\pgfpathcurveto{\pgfqpoint{4.125885in}{1.094231in}}{\pgfqpoint{4.132727in}{1.091397in}}{\pgfqpoint{4.139860in}{1.091397in}}%
\pgfpathclose%
\pgfusepath{stroke,fill}%
\end{pgfscope}%
\begin{pgfscope}%
\pgfpathrectangle{\pgfqpoint{2.867647in}{0.500000in}}{\pgfqpoint{1.764706in}{1.700000in}}%
\pgfusepath{clip}%
\pgfsetbuttcap%
\pgfsetroundjoin%
\definecolor{currentfill}{rgb}{0.971202,0.827364,0.728520}%
\pgfsetfillcolor{currentfill}%
\pgfsetlinewidth{0.311001pt}%
\definecolor{currentstroke}{rgb}{1.000000,1.000000,1.000000}%
\pgfsetstrokecolor{currentstroke}%
\pgfsetdash{}{0pt}%
\pgfpathmoveto{\pgfqpoint{4.099804in}{1.214432in}}%
\pgfpathcurveto{\pgfqpoint{4.106937in}{1.214432in}}{\pgfqpoint{4.113779in}{1.217266in}}{\pgfqpoint{4.118822in}{1.222310in}}%
\pgfpathcurveto{\pgfqpoint{4.123866in}{1.227353in}}{\pgfqpoint{4.126700in}{1.234195in}}{\pgfqpoint{4.126700in}{1.241328in}}%
\pgfpathcurveto{\pgfqpoint{4.126700in}{1.248461in}}{\pgfqpoint{4.123866in}{1.255302in}}{\pgfqpoint{4.118822in}{1.260346in}}%
\pgfpathcurveto{\pgfqpoint{4.113779in}{1.265390in}}{\pgfqpoint{4.106937in}{1.268224in}}{\pgfqpoint{4.099804in}{1.268224in}}%
\pgfpathcurveto{\pgfqpoint{4.092671in}{1.268224in}}{\pgfqpoint{4.085830in}{1.265390in}}{\pgfqpoint{4.080786in}{1.260346in}}%
\pgfpathcurveto{\pgfqpoint{4.075742in}{1.255302in}}{\pgfqpoint{4.072908in}{1.248461in}}{\pgfqpoint{4.072908in}{1.241328in}}%
\pgfpathcurveto{\pgfqpoint{4.072908in}{1.234195in}}{\pgfqpoint{4.075742in}{1.227353in}}{\pgfqpoint{4.080786in}{1.222310in}}%
\pgfpathcurveto{\pgfqpoint{4.085830in}{1.217266in}}{\pgfqpoint{4.092671in}{1.214432in}}{\pgfqpoint{4.099804in}{1.214432in}}%
\pgfpathclose%
\pgfusepath{stroke,fill}%
\end{pgfscope}%
\begin{pgfscope}%
\pgfpathrectangle{\pgfqpoint{2.867647in}{0.500000in}}{\pgfqpoint{1.764706in}{1.700000in}}%
\pgfusepath{clip}%
\pgfsetbuttcap%
\pgfsetroundjoin%
\definecolor{currentfill}{rgb}{0.972726,0.844889,0.754401}%
\pgfsetfillcolor{currentfill}%
\pgfsetlinewidth{0.311001pt}%
\definecolor{currentstroke}{rgb}{1.000000,1.000000,1.000000}%
\pgfsetstrokecolor{currentstroke}%
\pgfsetdash{}{0pt}%
\pgfpathmoveto{\pgfqpoint{4.253967in}{1.405524in}}%
\pgfpathcurveto{\pgfqpoint{4.261100in}{1.405524in}}{\pgfqpoint{4.267942in}{1.408358in}}{\pgfqpoint{4.272986in}{1.413402in}}%
\pgfpathcurveto{\pgfqpoint{4.278029in}{1.418446in}}{\pgfqpoint{4.280863in}{1.425287in}}{\pgfqpoint{4.280863in}{1.432420in}}%
\pgfpathcurveto{\pgfqpoint{4.280863in}{1.439553in}}{\pgfqpoint{4.278029in}{1.446395in}}{\pgfqpoint{4.272986in}{1.451438in}}%
\pgfpathcurveto{\pgfqpoint{4.267942in}{1.456482in}}{\pgfqpoint{4.261100in}{1.459316in}}{\pgfqpoint{4.253967in}{1.459316in}}%
\pgfpathcurveto{\pgfqpoint{4.246835in}{1.459316in}}{\pgfqpoint{4.239993in}{1.456482in}}{\pgfqpoint{4.234949in}{1.451438in}}%
\pgfpathcurveto{\pgfqpoint{4.229906in}{1.446395in}}{\pgfqpoint{4.227072in}{1.439553in}}{\pgfqpoint{4.227072in}{1.432420in}}%
\pgfpathcurveto{\pgfqpoint{4.227072in}{1.425287in}}{\pgfqpoint{4.229906in}{1.418446in}}{\pgfqpoint{4.234949in}{1.413402in}}%
\pgfpathcurveto{\pgfqpoint{4.239993in}{1.408358in}}{\pgfqpoint{4.246835in}{1.405524in}}{\pgfqpoint{4.253967in}{1.405524in}}%
\pgfpathclose%
\pgfusepath{stroke,fill}%
\end{pgfscope}%
\begin{pgfscope}%
\pgfpathrectangle{\pgfqpoint{2.867647in}{0.500000in}}{\pgfqpoint{1.764706in}{1.700000in}}%
\pgfusepath{clip}%
\pgfsetbuttcap%
\pgfsetroundjoin%
\definecolor{currentfill}{rgb}{0.965169,0.707764,0.560659}%
\pgfsetfillcolor{currentfill}%
\pgfsetlinewidth{0.311001pt}%
\definecolor{currentstroke}{rgb}{1.000000,1.000000,1.000000}%
\pgfsetstrokecolor{currentstroke}%
\pgfsetdash{}{0pt}%
\pgfpathmoveto{\pgfqpoint{4.297220in}{1.437131in}}%
\pgfpathcurveto{\pgfqpoint{4.304353in}{1.437131in}}{\pgfqpoint{4.311194in}{1.439965in}}{\pgfqpoint{4.316238in}{1.445009in}}%
\pgfpathcurveto{\pgfqpoint{4.321282in}{1.450052in}}{\pgfqpoint{4.324116in}{1.456894in}}{\pgfqpoint{4.324116in}{1.464027in}}%
\pgfpathcurveto{\pgfqpoint{4.324116in}{1.471160in}}{\pgfqpoint{4.321282in}{1.478001in}}{\pgfqpoint{4.316238in}{1.483045in}}%
\pgfpathcurveto{\pgfqpoint{4.311194in}{1.488089in}}{\pgfqpoint{4.304353in}{1.490923in}}{\pgfqpoint{4.297220in}{1.490923in}}%
\pgfpathcurveto{\pgfqpoint{4.290087in}{1.490923in}}{\pgfqpoint{4.283245in}{1.488089in}}{\pgfqpoint{4.278202in}{1.483045in}}%
\pgfpathcurveto{\pgfqpoint{4.273158in}{1.478001in}}{\pgfqpoint{4.270324in}{1.471160in}}{\pgfqpoint{4.270324in}{1.464027in}}%
\pgfpathcurveto{\pgfqpoint{4.270324in}{1.456894in}}{\pgfqpoint{4.273158in}{1.450052in}}{\pgfqpoint{4.278202in}{1.445009in}}%
\pgfpathcurveto{\pgfqpoint{4.283245in}{1.439965in}}{\pgfqpoint{4.290087in}{1.437131in}}{\pgfqpoint{4.297220in}{1.437131in}}%
\pgfpathclose%
\pgfusepath{stroke,fill}%
\end{pgfscope}%
\begin{pgfscope}%
\pgfpathrectangle{\pgfqpoint{2.867647in}{0.500000in}}{\pgfqpoint{1.764706in}{1.700000in}}%
\pgfusepath{clip}%
\pgfsetbuttcap%
\pgfsetroundjoin%
\definecolor{currentfill}{rgb}{0.979124,0.903132,0.839793}%
\pgfsetfillcolor{currentfill}%
\pgfsetlinewidth{0.311001pt}%
\definecolor{currentstroke}{rgb}{1.000000,1.000000,1.000000}%
\pgfsetstrokecolor{currentstroke}%
\pgfsetdash{}{0pt}%
\pgfpathmoveto{\pgfqpoint{4.114523in}{1.570367in}}%
\pgfpathcurveto{\pgfqpoint{4.121656in}{1.570367in}}{\pgfqpoint{4.128498in}{1.573200in}}{\pgfqpoint{4.133541in}{1.578244in}}%
\pgfpathcurveto{\pgfqpoint{4.138585in}{1.583288in}}{\pgfqpoint{4.141419in}{1.590129in}}{\pgfqpoint{4.141419in}{1.597262in}}%
\pgfpathcurveto{\pgfqpoint{4.141419in}{1.604395in}}{\pgfqpoint{4.138585in}{1.611237in}}{\pgfqpoint{4.133541in}{1.616280in}}%
\pgfpathcurveto{\pgfqpoint{4.128498in}{1.621324in}}{\pgfqpoint{4.121656in}{1.624158in}}{\pgfqpoint{4.114523in}{1.624158in}}%
\pgfpathcurveto{\pgfqpoint{4.107390in}{1.624158in}}{\pgfqpoint{4.100549in}{1.621324in}}{\pgfqpoint{4.095505in}{1.616280in}}%
\pgfpathcurveto{\pgfqpoint{4.090461in}{1.611237in}}{\pgfqpoint{4.087628in}{1.604395in}}{\pgfqpoint{4.087628in}{1.597262in}}%
\pgfpathcurveto{\pgfqpoint{4.087628in}{1.590129in}}{\pgfqpoint{4.090461in}{1.583288in}}{\pgfqpoint{4.095505in}{1.578244in}}%
\pgfpathcurveto{\pgfqpoint{4.100549in}{1.573200in}}{\pgfqpoint{4.107390in}{1.570367in}}{\pgfqpoint{4.114523in}{1.570367in}}%
\pgfpathclose%
\pgfusepath{stroke,fill}%
\end{pgfscope}%
\begin{pgfscope}%
\pgfpathrectangle{\pgfqpoint{2.867647in}{0.500000in}}{\pgfqpoint{1.764706in}{1.700000in}}%
\pgfusepath{clip}%
\pgfsetbuttcap%
\pgfsetroundjoin%
\definecolor{currentfill}{rgb}{0.975018,0.868213,0.788710}%
\pgfsetfillcolor{currentfill}%
\pgfsetlinewidth{0.311001pt}%
\definecolor{currentstroke}{rgb}{1.000000,1.000000,1.000000}%
\pgfsetstrokecolor{currentstroke}%
\pgfsetdash{}{0pt}%
\pgfpathmoveto{\pgfqpoint{4.118894in}{1.216171in}}%
\pgfpathcurveto{\pgfqpoint{4.126026in}{1.216171in}}{\pgfqpoint{4.132868in}{1.219005in}}{\pgfqpoint{4.137912in}{1.224048in}}%
\pgfpathcurveto{\pgfqpoint{4.142955in}{1.229092in}}{\pgfqpoint{4.145789in}{1.235934in}}{\pgfqpoint{4.145789in}{1.243067in}}%
\pgfpathcurveto{\pgfqpoint{4.145789in}{1.250199in}}{\pgfqpoint{4.142955in}{1.257041in}}{\pgfqpoint{4.137912in}{1.262085in}}%
\pgfpathcurveto{\pgfqpoint{4.132868in}{1.267128in}}{\pgfqpoint{4.126026in}{1.269962in}}{\pgfqpoint{4.118894in}{1.269962in}}%
\pgfpathcurveto{\pgfqpoint{4.111761in}{1.269962in}}{\pgfqpoint{4.104919in}{1.267128in}}{\pgfqpoint{4.099875in}{1.262085in}}%
\pgfpathcurveto{\pgfqpoint{4.094832in}{1.257041in}}{\pgfqpoint{4.091998in}{1.250199in}}{\pgfqpoint{4.091998in}{1.243067in}}%
\pgfpathcurveto{\pgfqpoint{4.091998in}{1.235934in}}{\pgfqpoint{4.094832in}{1.229092in}}{\pgfqpoint{4.099875in}{1.224048in}}%
\pgfpathcurveto{\pgfqpoint{4.104919in}{1.219005in}}{\pgfqpoint{4.111761in}{1.216171in}}{\pgfqpoint{4.118894in}{1.216171in}}%
\pgfpathclose%
\pgfusepath{stroke,fill}%
\end{pgfscope}%
\begin{pgfscope}%
\pgfpathrectangle{\pgfqpoint{2.867647in}{0.500000in}}{\pgfqpoint{1.764706in}{1.700000in}}%
\pgfusepath{clip}%
\pgfsetbuttcap%
\pgfsetroundjoin%
\definecolor{currentfill}{rgb}{0.977657,0.891500,0.822809}%
\pgfsetfillcolor{currentfill}%
\pgfsetlinewidth{0.311001pt}%
\definecolor{currentstroke}{rgb}{1.000000,1.000000,1.000000}%
\pgfsetstrokecolor{currentstroke}%
\pgfsetdash{}{0pt}%
\pgfpathmoveto{\pgfqpoint{4.155120in}{1.593610in}}%
\pgfpathcurveto{\pgfqpoint{4.162253in}{1.593610in}}{\pgfqpoint{4.169094in}{1.596444in}}{\pgfqpoint{4.174138in}{1.601487in}}%
\pgfpathcurveto{\pgfqpoint{4.179182in}{1.606531in}}{\pgfqpoint{4.182016in}{1.613373in}}{\pgfqpoint{4.182016in}{1.620506in}}%
\pgfpathcurveto{\pgfqpoint{4.182016in}{1.627638in}}{\pgfqpoint{4.179182in}{1.634480in}}{\pgfqpoint{4.174138in}{1.639524in}}%
\pgfpathcurveto{\pgfqpoint{4.169094in}{1.644567in}}{\pgfqpoint{4.162253in}{1.647401in}}{\pgfqpoint{4.155120in}{1.647401in}}%
\pgfpathcurveto{\pgfqpoint{4.147987in}{1.647401in}}{\pgfqpoint{4.141145in}{1.644567in}}{\pgfqpoint{4.136102in}{1.639524in}}%
\pgfpathcurveto{\pgfqpoint{4.131058in}{1.634480in}}{\pgfqpoint{4.128224in}{1.627638in}}{\pgfqpoint{4.128224in}{1.620506in}}%
\pgfpathcurveto{\pgfqpoint{4.128224in}{1.613373in}}{\pgfqpoint{4.131058in}{1.606531in}}{\pgfqpoint{4.136102in}{1.601487in}}%
\pgfpathcurveto{\pgfqpoint{4.141145in}{1.596444in}}{\pgfqpoint{4.147987in}{1.593610in}}{\pgfqpoint{4.155120in}{1.593610in}}%
\pgfpathclose%
\pgfusepath{stroke,fill}%
\end{pgfscope}%
\begin{pgfscope}%
\pgfpathrectangle{\pgfqpoint{2.867647in}{0.500000in}}{\pgfqpoint{1.764706in}{1.700000in}}%
\pgfusepath{clip}%
\pgfsetbuttcap%
\pgfsetroundjoin%
\definecolor{currentfill}{rgb}{0.976287,0.879862,0.805788}%
\pgfsetfillcolor{currentfill}%
\pgfsetlinewidth{0.311001pt}%
\definecolor{currentstroke}{rgb}{1.000000,1.000000,1.000000}%
\pgfsetstrokecolor{currentstroke}%
\pgfsetdash{}{0pt}%
\pgfpathmoveto{\pgfqpoint{4.200056in}{1.102382in}}%
\pgfpathcurveto{\pgfqpoint{4.207189in}{1.102382in}}{\pgfqpoint{4.214030in}{1.105216in}}{\pgfqpoint{4.219074in}{1.110259in}}%
\pgfpathcurveto{\pgfqpoint{4.224118in}{1.115303in}}{\pgfqpoint{4.226952in}{1.122145in}}{\pgfqpoint{4.226952in}{1.129278in}}%
\pgfpathcurveto{\pgfqpoint{4.226952in}{1.136410in}}{\pgfqpoint{4.224118in}{1.143252in}}{\pgfqpoint{4.219074in}{1.148296in}}%
\pgfpathcurveto{\pgfqpoint{4.214030in}{1.153339in}}{\pgfqpoint{4.207189in}{1.156173in}}{\pgfqpoint{4.200056in}{1.156173in}}%
\pgfpathcurveto{\pgfqpoint{4.192923in}{1.156173in}}{\pgfqpoint{4.186081in}{1.153339in}}{\pgfqpoint{4.181038in}{1.148296in}}%
\pgfpathcurveto{\pgfqpoint{4.175994in}{1.143252in}}{\pgfqpoint{4.173160in}{1.136410in}}{\pgfqpoint{4.173160in}{1.129278in}}%
\pgfpathcurveto{\pgfqpoint{4.173160in}{1.122145in}}{\pgfqpoint{4.175994in}{1.115303in}}{\pgfqpoint{4.181038in}{1.110259in}}%
\pgfpathcurveto{\pgfqpoint{4.186081in}{1.105216in}}{\pgfqpoint{4.192923in}{1.102382in}}{\pgfqpoint{4.200056in}{1.102382in}}%
\pgfpathclose%
\pgfusepath{stroke,fill}%
\end{pgfscope}%
\begin{pgfscope}%
\pgfpathrectangle{\pgfqpoint{2.867647in}{0.500000in}}{\pgfqpoint{1.764706in}{1.700000in}}%
\pgfusepath{clip}%
\pgfsetbuttcap%
\pgfsetroundjoin%
\definecolor{currentfill}{rgb}{0.977657,0.891500,0.822809}%
\pgfsetfillcolor{currentfill}%
\pgfsetlinewidth{0.311001pt}%
\definecolor{currentstroke}{rgb}{1.000000,1.000000,1.000000}%
\pgfsetstrokecolor{currentstroke}%
\pgfsetdash{}{0pt}%
\pgfpathmoveto{\pgfqpoint{4.204207in}{1.130346in}}%
\pgfpathcurveto{\pgfqpoint{4.211340in}{1.130346in}}{\pgfqpoint{4.218182in}{1.133180in}}{\pgfqpoint{4.223225in}{1.138224in}}%
\pgfpathcurveto{\pgfqpoint{4.228269in}{1.143268in}}{\pgfqpoint{4.231103in}{1.150109in}}{\pgfqpoint{4.231103in}{1.157242in}}%
\pgfpathcurveto{\pgfqpoint{4.231103in}{1.164375in}}{\pgfqpoint{4.228269in}{1.171217in}}{\pgfqpoint{4.223225in}{1.176260in}}%
\pgfpathcurveto{\pgfqpoint{4.218182in}{1.181304in}}{\pgfqpoint{4.211340in}{1.184138in}}{\pgfqpoint{4.204207in}{1.184138in}}%
\pgfpathcurveto{\pgfqpoint{4.197074in}{1.184138in}}{\pgfqpoint{4.190233in}{1.181304in}}{\pgfqpoint{4.185189in}{1.176260in}}%
\pgfpathcurveto{\pgfqpoint{4.180145in}{1.171217in}}{\pgfqpoint{4.177311in}{1.164375in}}{\pgfqpoint{4.177311in}{1.157242in}}%
\pgfpathcurveto{\pgfqpoint{4.177311in}{1.150109in}}{\pgfqpoint{4.180145in}{1.143268in}}{\pgfqpoint{4.185189in}{1.138224in}}%
\pgfpathcurveto{\pgfqpoint{4.190233in}{1.133180in}}{\pgfqpoint{4.197074in}{1.130346in}}{\pgfqpoint{4.204207in}{1.130346in}}%
\pgfpathclose%
\pgfusepath{stroke,fill}%
\end{pgfscope}%
\begin{pgfscope}%
\pgfpathrectangle{\pgfqpoint{2.867647in}{0.500000in}}{\pgfqpoint{1.764706in}{1.700000in}}%
\pgfusepath{clip}%
\pgfsetbuttcap%
\pgfsetroundjoin%
\definecolor{currentfill}{rgb}{0.981377,0.920617,0.865369}%
\pgfsetfillcolor{currentfill}%
\pgfsetlinewidth{0.311001pt}%
\definecolor{currentstroke}{rgb}{1.000000,1.000000,1.000000}%
\pgfsetstrokecolor{currentstroke}%
\pgfsetdash{}{0pt}%
\pgfpathmoveto{\pgfqpoint{4.174626in}{1.207723in}}%
\pgfpathcurveto{\pgfqpoint{4.181759in}{1.207723in}}{\pgfqpoint{4.188600in}{1.210557in}}{\pgfqpoint{4.193644in}{1.215600in}}%
\pgfpathcurveto{\pgfqpoint{4.198688in}{1.220644in}}{\pgfqpoint{4.201522in}{1.227486in}}{\pgfqpoint{4.201522in}{1.234618in}}%
\pgfpathcurveto{\pgfqpoint{4.201522in}{1.241751in}}{\pgfqpoint{4.198688in}{1.248593in}}{\pgfqpoint{4.193644in}{1.253637in}}%
\pgfpathcurveto{\pgfqpoint{4.188600in}{1.258680in}}{\pgfqpoint{4.181759in}{1.261514in}}{\pgfqpoint{4.174626in}{1.261514in}}%
\pgfpathcurveto{\pgfqpoint{4.167493in}{1.261514in}}{\pgfqpoint{4.160651in}{1.258680in}}{\pgfqpoint{4.155608in}{1.253637in}}%
\pgfpathcurveto{\pgfqpoint{4.150564in}{1.248593in}}{\pgfqpoint{4.147730in}{1.241751in}}{\pgfqpoint{4.147730in}{1.234618in}}%
\pgfpathcurveto{\pgfqpoint{4.147730in}{1.227486in}}{\pgfqpoint{4.150564in}{1.220644in}}{\pgfqpoint{4.155608in}{1.215600in}}%
\pgfpathcurveto{\pgfqpoint{4.160651in}{1.210557in}}{\pgfqpoint{4.167493in}{1.207723in}}{\pgfqpoint{4.174626in}{1.207723in}}%
\pgfpathclose%
\pgfusepath{stroke,fill}%
\end{pgfscope}%
\begin{pgfscope}%
\pgfpathrectangle{\pgfqpoint{2.867647in}{0.500000in}}{\pgfqpoint{1.764706in}{1.700000in}}%
\pgfusepath{clip}%
\pgfsetbuttcap%
\pgfsetroundjoin%
\definecolor{currentfill}{rgb}{0.964306,0.663930,0.507747}%
\pgfsetfillcolor{currentfill}%
\pgfsetlinewidth{0.311001pt}%
\definecolor{currentstroke}{rgb}{1.000000,1.000000,1.000000}%
\pgfsetstrokecolor{currentstroke}%
\pgfsetdash{}{0pt}%
\pgfpathmoveto{\pgfqpoint{4.010374in}{1.096874in}}%
\pgfpathcurveto{\pgfqpoint{4.017507in}{1.096874in}}{\pgfqpoint{4.024349in}{1.099708in}}{\pgfqpoint{4.029392in}{1.104752in}}%
\pgfpathcurveto{\pgfqpoint{4.034436in}{1.109796in}}{\pgfqpoint{4.037270in}{1.116637in}}{\pgfqpoint{4.037270in}{1.123770in}}%
\pgfpathcurveto{\pgfqpoint{4.037270in}{1.130903in}}{\pgfqpoint{4.034436in}{1.137745in}}{\pgfqpoint{4.029392in}{1.142788in}}%
\pgfpathcurveto{\pgfqpoint{4.024349in}{1.147832in}}{\pgfqpoint{4.017507in}{1.150666in}}{\pgfqpoint{4.010374in}{1.150666in}}%
\pgfpathcurveto{\pgfqpoint{4.003241in}{1.150666in}}{\pgfqpoint{3.996400in}{1.147832in}}{\pgfqpoint{3.991356in}{1.142788in}}%
\pgfpathcurveto{\pgfqpoint{3.986312in}{1.137745in}}{\pgfqpoint{3.983478in}{1.130903in}}{\pgfqpoint{3.983478in}{1.123770in}}%
\pgfpathcurveto{\pgfqpoint{3.983478in}{1.116637in}}{\pgfqpoint{3.986312in}{1.109796in}}{\pgfqpoint{3.991356in}{1.104752in}}%
\pgfpathcurveto{\pgfqpoint{3.996400in}{1.099708in}}{\pgfqpoint{4.003241in}{1.096874in}}{\pgfqpoint{4.010374in}{1.096874in}}%
\pgfpathclose%
\pgfusepath{stroke,fill}%
\end{pgfscope}%
\begin{pgfscope}%
\pgfpathrectangle{\pgfqpoint{2.867647in}{0.500000in}}{\pgfqpoint{1.764706in}{1.700000in}}%
\pgfusepath{clip}%
\pgfsetbuttcap%
\pgfsetroundjoin%
\definecolor{currentfill}{rgb}{0.971694,0.833208,0.737161}%
\pgfsetfillcolor{currentfill}%
\pgfsetlinewidth{0.311001pt}%
\definecolor{currentstroke}{rgb}{1.000000,1.000000,1.000000}%
\pgfsetstrokecolor{currentstroke}%
\pgfsetdash{}{0pt}%
\pgfpathmoveto{\pgfqpoint{4.092266in}{1.465770in}}%
\pgfpathcurveto{\pgfqpoint{4.099399in}{1.465770in}}{\pgfqpoint{4.106240in}{1.468604in}}{\pgfqpoint{4.111284in}{1.473648in}}%
\pgfpathcurveto{\pgfqpoint{4.116328in}{1.478691in}}{\pgfqpoint{4.119161in}{1.485533in}}{\pgfqpoint{4.119161in}{1.492666in}}%
\pgfpathcurveto{\pgfqpoint{4.119161in}{1.499799in}}{\pgfqpoint{4.116328in}{1.506640in}}{\pgfqpoint{4.111284in}{1.511684in}}%
\pgfpathcurveto{\pgfqpoint{4.106240in}{1.516728in}}{\pgfqpoint{4.099399in}{1.519561in}}{\pgfqpoint{4.092266in}{1.519561in}}%
\pgfpathcurveto{\pgfqpoint{4.085133in}{1.519561in}}{\pgfqpoint{4.078291in}{1.516728in}}{\pgfqpoint{4.073248in}{1.511684in}}%
\pgfpathcurveto{\pgfqpoint{4.068204in}{1.506640in}}{\pgfqpoint{4.065370in}{1.499799in}}{\pgfqpoint{4.065370in}{1.492666in}}%
\pgfpathcurveto{\pgfqpoint{4.065370in}{1.485533in}}{\pgfqpoint{4.068204in}{1.478691in}}{\pgfqpoint{4.073248in}{1.473648in}}%
\pgfpathcurveto{\pgfqpoint{4.078291in}{1.468604in}}{\pgfqpoint{4.085133in}{1.465770in}}{\pgfqpoint{4.092266in}{1.465770in}}%
\pgfpathclose%
\pgfusepath{stroke,fill}%
\end{pgfscope}%
\begin{pgfscope}%
\pgfpathrectangle{\pgfqpoint{2.867647in}{0.500000in}}{\pgfqpoint{1.764706in}{1.700000in}}%
\pgfusepath{clip}%
\pgfsetbuttcap%
\pgfsetroundjoin%
\definecolor{currentfill}{rgb}{0.963884,0.644842,0.486120}%
\pgfsetfillcolor{currentfill}%
\pgfsetlinewidth{0.311001pt}%
\definecolor{currentstroke}{rgb}{1.000000,1.000000,1.000000}%
\pgfsetstrokecolor{currentstroke}%
\pgfsetdash{}{0pt}%
\pgfpathmoveto{\pgfqpoint{4.014342in}{1.119976in}}%
\pgfpathcurveto{\pgfqpoint{4.021474in}{1.119976in}}{\pgfqpoint{4.028316in}{1.122810in}}{\pgfqpoint{4.033360in}{1.127854in}}%
\pgfpathcurveto{\pgfqpoint{4.038403in}{1.132898in}}{\pgfqpoint{4.041237in}{1.139739in}}{\pgfqpoint{4.041237in}{1.146872in}}%
\pgfpathcurveto{\pgfqpoint{4.041237in}{1.154005in}}{\pgfqpoint{4.038403in}{1.160847in}}{\pgfqpoint{4.033360in}{1.165890in}}%
\pgfpathcurveto{\pgfqpoint{4.028316in}{1.170934in}}{\pgfqpoint{4.021474in}{1.173768in}}{\pgfqpoint{4.014342in}{1.173768in}}%
\pgfpathcurveto{\pgfqpoint{4.007209in}{1.173768in}}{\pgfqpoint{4.000367in}{1.170934in}}{\pgfqpoint{3.995324in}{1.165890in}}%
\pgfpathcurveto{\pgfqpoint{3.990280in}{1.160847in}}{\pgfqpoint{3.987446in}{1.154005in}}{\pgfqpoint{3.987446in}{1.146872in}}%
\pgfpathcurveto{\pgfqpoint{3.987446in}{1.139739in}}{\pgfqpoint{3.990280in}{1.132898in}}{\pgfqpoint{3.995324in}{1.127854in}}%
\pgfpathcurveto{\pgfqpoint{4.000367in}{1.122810in}}{\pgfqpoint{4.007209in}{1.119976in}}{\pgfqpoint{4.014342in}{1.119976in}}%
\pgfpathclose%
\pgfusepath{stroke,fill}%
\end{pgfscope}%
\begin{pgfscope}%
\pgfpathrectangle{\pgfqpoint{2.867647in}{0.500000in}}{\pgfqpoint{1.764706in}{1.700000in}}%
\pgfusepath{clip}%
\pgfsetbuttcap%
\pgfsetroundjoin%
\definecolor{currentfill}{rgb}{0.963884,0.644842,0.486120}%
\pgfsetfillcolor{currentfill}%
\pgfsetlinewidth{0.311001pt}%
\definecolor{currentstroke}{rgb}{1.000000,1.000000,1.000000}%
\pgfsetstrokecolor{currentstroke}%
\pgfsetdash{}{0pt}%
\pgfpathmoveto{\pgfqpoint{4.061411in}{1.324269in}}%
\pgfpathcurveto{\pgfqpoint{4.068544in}{1.324269in}}{\pgfqpoint{4.075386in}{1.327102in}}{\pgfqpoint{4.080430in}{1.332146in}}%
\pgfpathcurveto{\pgfqpoint{4.085473in}{1.337190in}}{\pgfqpoint{4.088307in}{1.344031in}}{\pgfqpoint{4.088307in}{1.351164in}}%
\pgfpathcurveto{\pgfqpoint{4.088307in}{1.358297in}}{\pgfqpoint{4.085473in}{1.365139in}}{\pgfqpoint{4.080430in}{1.370182in}}%
\pgfpathcurveto{\pgfqpoint{4.075386in}{1.375226in}}{\pgfqpoint{4.068544in}{1.378060in}}{\pgfqpoint{4.061411in}{1.378060in}}%
\pgfpathcurveto{\pgfqpoint{4.054279in}{1.378060in}}{\pgfqpoint{4.047437in}{1.375226in}}{\pgfqpoint{4.042393in}{1.370182in}}%
\pgfpathcurveto{\pgfqpoint{4.037350in}{1.365139in}}{\pgfqpoint{4.034516in}{1.358297in}}{\pgfqpoint{4.034516in}{1.351164in}}%
\pgfpathcurveto{\pgfqpoint{4.034516in}{1.344031in}}{\pgfqpoint{4.037350in}{1.337190in}}{\pgfqpoint{4.042393in}{1.332146in}}%
\pgfpathcurveto{\pgfqpoint{4.047437in}{1.327102in}}{\pgfqpoint{4.054279in}{1.324269in}}{\pgfqpoint{4.061411in}{1.324269in}}%
\pgfpathclose%
\pgfusepath{stroke,fill}%
\end{pgfscope}%
\begin{pgfscope}%
\pgfpathrectangle{\pgfqpoint{2.867647in}{0.500000in}}{\pgfqpoint{1.764706in}{1.700000in}}%
\pgfusepath{clip}%
\pgfsetbuttcap%
\pgfsetroundjoin%
\definecolor{currentfill}{rgb}{0.944085,0.383081,0.267220}%
\pgfsetfillcolor{currentfill}%
\pgfsetlinewidth{0.311001pt}%
\definecolor{currentstroke}{rgb}{1.000000,1.000000,1.000000}%
\pgfsetstrokecolor{currentstroke}%
\pgfsetdash{}{0pt}%
\pgfpathmoveto{\pgfqpoint{4.305990in}{1.019293in}}%
\pgfpathcurveto{\pgfqpoint{4.313123in}{1.019293in}}{\pgfqpoint{4.319964in}{1.022127in}}{\pgfqpoint{4.325008in}{1.027171in}}%
\pgfpathcurveto{\pgfqpoint{4.330052in}{1.032214in}}{\pgfqpoint{4.332885in}{1.039056in}}{\pgfqpoint{4.332885in}{1.046189in}}%
\pgfpathcurveto{\pgfqpoint{4.332885in}{1.053322in}}{\pgfqpoint{4.330052in}{1.060163in}}{\pgfqpoint{4.325008in}{1.065207in}}%
\pgfpathcurveto{\pgfqpoint{4.319964in}{1.070251in}}{\pgfqpoint{4.313123in}{1.073085in}}{\pgfqpoint{4.305990in}{1.073085in}}%
\pgfpathcurveto{\pgfqpoint{4.298857in}{1.073085in}}{\pgfqpoint{4.292015in}{1.070251in}}{\pgfqpoint{4.286972in}{1.065207in}}%
\pgfpathcurveto{\pgfqpoint{4.281928in}{1.060163in}}{\pgfqpoint{4.279094in}{1.053322in}}{\pgfqpoint{4.279094in}{1.046189in}}%
\pgfpathcurveto{\pgfqpoint{4.279094in}{1.039056in}}{\pgfqpoint{4.281928in}{1.032214in}}{\pgfqpoint{4.286972in}{1.027171in}}%
\pgfpathcurveto{\pgfqpoint{4.292015in}{1.022127in}}{\pgfqpoint{4.298857in}{1.019293in}}{\pgfqpoint{4.305990in}{1.019293in}}%
\pgfpathclose%
\pgfusepath{stroke,fill}%
\end{pgfscope}%
\begin{pgfscope}%
\pgfpathrectangle{\pgfqpoint{2.867647in}{0.500000in}}{\pgfqpoint{1.764706in}{1.700000in}}%
\pgfusepath{clip}%
\pgfsetbuttcap%
\pgfsetroundjoin%
\definecolor{currentfill}{rgb}{0.977657,0.891500,0.822809}%
\pgfsetfillcolor{currentfill}%
\pgfsetlinewidth{0.311001pt}%
\definecolor{currentstroke}{rgb}{1.000000,1.000000,1.000000}%
\pgfsetstrokecolor{currentstroke}%
\pgfsetdash{}{0pt}%
\pgfpathmoveto{\pgfqpoint{4.134945in}{1.221755in}}%
\pgfpathcurveto{\pgfqpoint{4.142078in}{1.221755in}}{\pgfqpoint{4.148919in}{1.224589in}}{\pgfqpoint{4.153963in}{1.229633in}}%
\pgfpathcurveto{\pgfqpoint{4.159007in}{1.234676in}}{\pgfqpoint{4.161841in}{1.241518in}}{\pgfqpoint{4.161841in}{1.248651in}}%
\pgfpathcurveto{\pgfqpoint{4.161841in}{1.255784in}}{\pgfqpoint{4.159007in}{1.262625in}}{\pgfqpoint{4.153963in}{1.267669in}}%
\pgfpathcurveto{\pgfqpoint{4.148919in}{1.272712in}}{\pgfqpoint{4.142078in}{1.275546in}}{\pgfqpoint{4.134945in}{1.275546in}}%
\pgfpathcurveto{\pgfqpoint{4.127812in}{1.275546in}}{\pgfqpoint{4.120970in}{1.272712in}}{\pgfqpoint{4.115927in}{1.267669in}}%
\pgfpathcurveto{\pgfqpoint{4.110883in}{1.262625in}}{\pgfqpoint{4.108049in}{1.255784in}}{\pgfqpoint{4.108049in}{1.248651in}}%
\pgfpathcurveto{\pgfqpoint{4.108049in}{1.241518in}}{\pgfqpoint{4.110883in}{1.234676in}}{\pgfqpoint{4.115927in}{1.229633in}}%
\pgfpathcurveto{\pgfqpoint{4.120970in}{1.224589in}}{\pgfqpoint{4.127812in}{1.221755in}}{\pgfqpoint{4.134945in}{1.221755in}}%
\pgfpathclose%
\pgfusepath{stroke,fill}%
\end{pgfscope}%
\begin{pgfscope}%
\pgfpathrectangle{\pgfqpoint{2.867647in}{0.500000in}}{\pgfqpoint{1.764706in}{1.700000in}}%
\pgfusepath{clip}%
\pgfsetbuttcap%
\pgfsetroundjoin%
\definecolor{currentfill}{rgb}{0.978376,0.897317,0.831308}%
\pgfsetfillcolor{currentfill}%
\pgfsetlinewidth{0.311001pt}%
\definecolor{currentstroke}{rgb}{1.000000,1.000000,1.000000}%
\pgfsetstrokecolor{currentstroke}%
\pgfsetdash{}{0pt}%
\pgfpathmoveto{\pgfqpoint{4.138166in}{1.069198in}}%
\pgfpathcurveto{\pgfqpoint{4.145299in}{1.069198in}}{\pgfqpoint{4.152140in}{1.072032in}}{\pgfqpoint{4.157184in}{1.077075in}}%
\pgfpathcurveto{\pgfqpoint{4.162228in}{1.082119in}}{\pgfqpoint{4.165062in}{1.088961in}}{\pgfqpoint{4.165062in}{1.096093in}}%
\pgfpathcurveto{\pgfqpoint{4.165062in}{1.103226in}}{\pgfqpoint{4.162228in}{1.110068in}}{\pgfqpoint{4.157184in}{1.115112in}}%
\pgfpathcurveto{\pgfqpoint{4.152140in}{1.120155in}}{\pgfqpoint{4.145299in}{1.122989in}}{\pgfqpoint{4.138166in}{1.122989in}}%
\pgfpathcurveto{\pgfqpoint{4.131033in}{1.122989in}}{\pgfqpoint{4.124191in}{1.120155in}}{\pgfqpoint{4.119148in}{1.115112in}}%
\pgfpathcurveto{\pgfqpoint{4.114104in}{1.110068in}}{\pgfqpoint{4.111270in}{1.103226in}}{\pgfqpoint{4.111270in}{1.096093in}}%
\pgfpathcurveto{\pgfqpoint{4.111270in}{1.088961in}}{\pgfqpoint{4.114104in}{1.082119in}}{\pgfqpoint{4.119148in}{1.077075in}}%
\pgfpathcurveto{\pgfqpoint{4.124191in}{1.072032in}}{\pgfqpoint{4.131033in}{1.069198in}}{\pgfqpoint{4.138166in}{1.069198in}}%
\pgfpathclose%
\pgfusepath{stroke,fill}%
\end{pgfscope}%
\begin{pgfscope}%
\pgfpathrectangle{\pgfqpoint{2.867647in}{0.500000in}}{\pgfqpoint{1.764706in}{1.700000in}}%
\pgfusepath{clip}%
\pgfsetbuttcap%
\pgfsetroundjoin%
\definecolor{currentfill}{rgb}{0.930781,0.313740,0.244688}%
\pgfsetfillcolor{currentfill}%
\pgfsetlinewidth{0.311001pt}%
\definecolor{currentstroke}{rgb}{1.000000,1.000000,1.000000}%
\pgfsetstrokecolor{currentstroke}%
\pgfsetdash{}{0pt}%
\pgfpathmoveto{\pgfqpoint{4.182439in}{1.802346in}}%
\pgfpathcurveto{\pgfqpoint{4.189572in}{1.802346in}}{\pgfqpoint{4.196413in}{1.805180in}}{\pgfqpoint{4.201457in}{1.810224in}}%
\pgfpathcurveto{\pgfqpoint{4.206501in}{1.815267in}}{\pgfqpoint{4.209335in}{1.822109in}}{\pgfqpoint{4.209335in}{1.829242in}}%
\pgfpathcurveto{\pgfqpoint{4.209335in}{1.836375in}}{\pgfqpoint{4.206501in}{1.843216in}}{\pgfqpoint{4.201457in}{1.848260in}}%
\pgfpathcurveto{\pgfqpoint{4.196413in}{1.853303in}}{\pgfqpoint{4.189572in}{1.856137in}}{\pgfqpoint{4.182439in}{1.856137in}}%
\pgfpathcurveto{\pgfqpoint{4.175306in}{1.856137in}}{\pgfqpoint{4.168464in}{1.853303in}}{\pgfqpoint{4.163421in}{1.848260in}}%
\pgfpathcurveto{\pgfqpoint{4.158377in}{1.843216in}}{\pgfqpoint{4.155543in}{1.836375in}}{\pgfqpoint{4.155543in}{1.829242in}}%
\pgfpathcurveto{\pgfqpoint{4.155543in}{1.822109in}}{\pgfqpoint{4.158377in}{1.815267in}}{\pgfqpoint{4.163421in}{1.810224in}}%
\pgfpathcurveto{\pgfqpoint{4.168464in}{1.805180in}}{\pgfqpoint{4.175306in}{1.802346in}}{\pgfqpoint{4.182439in}{1.802346in}}%
\pgfpathclose%
\pgfusepath{stroke,fill}%
\end{pgfscope}%
\begin{pgfscope}%
\pgfpathrectangle{\pgfqpoint{2.867647in}{0.500000in}}{\pgfqpoint{1.764706in}{1.700000in}}%
\pgfusepath{clip}%
\pgfsetbuttcap%
\pgfsetroundjoin%
\definecolor{currentfill}{rgb}{0.977657,0.891500,0.822809}%
\pgfsetfillcolor{currentfill}%
\pgfsetlinewidth{0.311001pt}%
\definecolor{currentstroke}{rgb}{1.000000,1.000000,1.000000}%
\pgfsetstrokecolor{currentstroke}%
\pgfsetdash{}{0pt}%
\pgfpathmoveto{\pgfqpoint{4.118283in}{1.125998in}}%
\pgfpathcurveto{\pgfqpoint{4.125416in}{1.125998in}}{\pgfqpoint{4.132257in}{1.128832in}}{\pgfqpoint{4.137301in}{1.133876in}}%
\pgfpathcurveto{\pgfqpoint{4.142345in}{1.138920in}}{\pgfqpoint{4.145179in}{1.145761in}}{\pgfqpoint{4.145179in}{1.152894in}}%
\pgfpathcurveto{\pgfqpoint{4.145179in}{1.160027in}}{\pgfqpoint{4.142345in}{1.166869in}}{\pgfqpoint{4.137301in}{1.171912in}}%
\pgfpathcurveto{\pgfqpoint{4.132257in}{1.176956in}}{\pgfqpoint{4.125416in}{1.179790in}}{\pgfqpoint{4.118283in}{1.179790in}}%
\pgfpathcurveto{\pgfqpoint{4.111150in}{1.179790in}}{\pgfqpoint{4.104308in}{1.176956in}}{\pgfqpoint{4.099265in}{1.171912in}}%
\pgfpathcurveto{\pgfqpoint{4.094221in}{1.166869in}}{\pgfqpoint{4.091387in}{1.160027in}}{\pgfqpoint{4.091387in}{1.152894in}}%
\pgfpathcurveto{\pgfqpoint{4.091387in}{1.145761in}}{\pgfqpoint{4.094221in}{1.138920in}}{\pgfqpoint{4.099265in}{1.133876in}}%
\pgfpathcurveto{\pgfqpoint{4.104308in}{1.128832in}}{\pgfqpoint{4.111150in}{1.125998in}}{\pgfqpoint{4.118283in}{1.125998in}}%
\pgfpathclose%
\pgfusepath{stroke,fill}%
\end{pgfscope}%
\begin{pgfscope}%
\pgfpathrectangle{\pgfqpoint{2.867647in}{0.500000in}}{\pgfqpoint{1.764706in}{1.700000in}}%
\pgfusepath{clip}%
\pgfsetbuttcap%
\pgfsetroundjoin%
\definecolor{currentfill}{rgb}{0.976287,0.879862,0.805788}%
\pgfsetfillcolor{currentfill}%
\pgfsetlinewidth{0.311001pt}%
\definecolor{currentstroke}{rgb}{1.000000,1.000000,1.000000}%
\pgfsetstrokecolor{currentstroke}%
\pgfsetdash{}{0pt}%
\pgfpathmoveto{\pgfqpoint{4.223277in}{1.155645in}}%
\pgfpathcurveto{\pgfqpoint{4.230410in}{1.155645in}}{\pgfqpoint{4.237251in}{1.158478in}}{\pgfqpoint{4.242295in}{1.163522in}}%
\pgfpathcurveto{\pgfqpoint{4.247339in}{1.168566in}}{\pgfqpoint{4.250173in}{1.175407in}}{\pgfqpoint{4.250173in}{1.182540in}}%
\pgfpathcurveto{\pgfqpoint{4.250173in}{1.189673in}}{\pgfqpoint{4.247339in}{1.196515in}}{\pgfqpoint{4.242295in}{1.201558in}}%
\pgfpathcurveto{\pgfqpoint{4.237251in}{1.206602in}}{\pgfqpoint{4.230410in}{1.209436in}}{\pgfqpoint{4.223277in}{1.209436in}}%
\pgfpathcurveto{\pgfqpoint{4.216144in}{1.209436in}}{\pgfqpoint{4.209302in}{1.206602in}}{\pgfqpoint{4.204259in}{1.201558in}}%
\pgfpathcurveto{\pgfqpoint{4.199215in}{1.196515in}}{\pgfqpoint{4.196381in}{1.189673in}}{\pgfqpoint{4.196381in}{1.182540in}}%
\pgfpathcurveto{\pgfqpoint{4.196381in}{1.175407in}}{\pgfqpoint{4.199215in}{1.168566in}}{\pgfqpoint{4.204259in}{1.163522in}}%
\pgfpathcurveto{\pgfqpoint{4.209302in}{1.158478in}}{\pgfqpoint{4.216144in}{1.155645in}}{\pgfqpoint{4.223277in}{1.155645in}}%
\pgfpathclose%
\pgfusepath{stroke,fill}%
\end{pgfscope}%
\begin{pgfscope}%
\pgfpathrectangle{\pgfqpoint{2.867647in}{0.500000in}}{\pgfqpoint{1.764706in}{1.700000in}}%
\pgfusepath{clip}%
\pgfsetbuttcap%
\pgfsetroundjoin%
\definecolor{currentfill}{rgb}{0.962532,0.599594,0.438051}%
\pgfsetfillcolor{currentfill}%
\pgfsetlinewidth{0.311001pt}%
\definecolor{currentstroke}{rgb}{1.000000,1.000000,1.000000}%
\pgfsetstrokecolor{currentstroke}%
\pgfsetdash{}{0pt}%
\pgfpathmoveto{\pgfqpoint{4.049582in}{0.848812in}}%
\pgfpathcurveto{\pgfqpoint{4.056715in}{0.848812in}}{\pgfqpoint{4.063557in}{0.851646in}}{\pgfqpoint{4.068601in}{0.856690in}}%
\pgfpathcurveto{\pgfqpoint{4.073644in}{0.861734in}}{\pgfqpoint{4.076478in}{0.868575in}}{\pgfqpoint{4.076478in}{0.875708in}}%
\pgfpathcurveto{\pgfqpoint{4.076478in}{0.882841in}}{\pgfqpoint{4.073644in}{0.889683in}}{\pgfqpoint{4.068601in}{0.894726in}}%
\pgfpathcurveto{\pgfqpoint{4.063557in}{0.899770in}}{\pgfqpoint{4.056715in}{0.902604in}}{\pgfqpoint{4.049582in}{0.902604in}}%
\pgfpathcurveto{\pgfqpoint{4.042450in}{0.902604in}}{\pgfqpoint{4.035608in}{0.899770in}}{\pgfqpoint{4.030564in}{0.894726in}}%
\pgfpathcurveto{\pgfqpoint{4.025521in}{0.889683in}}{\pgfqpoint{4.022687in}{0.882841in}}{\pgfqpoint{4.022687in}{0.875708in}}%
\pgfpathcurveto{\pgfqpoint{4.022687in}{0.868575in}}{\pgfqpoint{4.025521in}{0.861734in}}{\pgfqpoint{4.030564in}{0.856690in}}%
\pgfpathcurveto{\pgfqpoint{4.035608in}{0.851646in}}{\pgfqpoint{4.042450in}{0.848812in}}{\pgfqpoint{4.049582in}{0.848812in}}%
\pgfpathclose%
\pgfusepath{stroke,fill}%
\end{pgfscope}%
\begin{pgfscope}%
\pgfpathrectangle{\pgfqpoint{2.867647in}{0.500000in}}{\pgfqpoint{1.764706in}{1.700000in}}%
\pgfusepath{clip}%
\pgfsetbuttcap%
\pgfsetroundjoin%
\definecolor{currentfill}{rgb}{0.977657,0.891500,0.822809}%
\pgfsetfillcolor{currentfill}%
\pgfsetlinewidth{0.311001pt}%
\definecolor{currentstroke}{rgb}{1.000000,1.000000,1.000000}%
\pgfsetstrokecolor{currentstroke}%
\pgfsetdash{}{0pt}%
\pgfpathmoveto{\pgfqpoint{4.110726in}{1.518433in}}%
\pgfpathcurveto{\pgfqpoint{4.117859in}{1.518433in}}{\pgfqpoint{4.124701in}{1.521267in}}{\pgfqpoint{4.129744in}{1.526311in}}%
\pgfpathcurveto{\pgfqpoint{4.134788in}{1.531354in}}{\pgfqpoint{4.137622in}{1.538196in}}{\pgfqpoint{4.137622in}{1.545329in}}%
\pgfpathcurveto{\pgfqpoint{4.137622in}{1.552462in}}{\pgfqpoint{4.134788in}{1.559303in}}{\pgfqpoint{4.129744in}{1.564347in}}%
\pgfpathcurveto{\pgfqpoint{4.124701in}{1.569391in}}{\pgfqpoint{4.117859in}{1.572225in}}{\pgfqpoint{4.110726in}{1.572225in}}%
\pgfpathcurveto{\pgfqpoint{4.103593in}{1.572225in}}{\pgfqpoint{4.096752in}{1.569391in}}{\pgfqpoint{4.091708in}{1.564347in}}%
\pgfpathcurveto{\pgfqpoint{4.086665in}{1.559303in}}{\pgfqpoint{4.083831in}{1.552462in}}{\pgfqpoint{4.083831in}{1.545329in}}%
\pgfpathcurveto{\pgfqpoint{4.083831in}{1.538196in}}{\pgfqpoint{4.086665in}{1.531354in}}{\pgfqpoint{4.091708in}{1.526311in}}%
\pgfpathcurveto{\pgfqpoint{4.096752in}{1.521267in}}{\pgfqpoint{4.103593in}{1.518433in}}{\pgfqpoint{4.110726in}{1.518433in}}%
\pgfpathclose%
\pgfusepath{stroke,fill}%
\end{pgfscope}%
\begin{pgfscope}%
\pgfpathrectangle{\pgfqpoint{2.867647in}{0.500000in}}{\pgfqpoint{1.764706in}{1.700000in}}%
\pgfusepath{clip}%
\pgfsetbuttcap%
\pgfsetroundjoin%
\definecolor{currentfill}{rgb}{0.969803,0.809811,0.702523}%
\pgfsetfillcolor{currentfill}%
\pgfsetlinewidth{0.311001pt}%
\definecolor{currentstroke}{rgb}{1.000000,1.000000,1.000000}%
\pgfsetstrokecolor{currentstroke}%
\pgfsetdash{}{0pt}%
\pgfpathmoveto{\pgfqpoint{4.083869in}{1.459075in}}%
\pgfpathcurveto{\pgfqpoint{4.091002in}{1.459075in}}{\pgfqpoint{4.097844in}{1.461909in}}{\pgfqpoint{4.102888in}{1.466953in}}%
\pgfpathcurveto{\pgfqpoint{4.107931in}{1.471996in}}{\pgfqpoint{4.110765in}{1.478838in}}{\pgfqpoint{4.110765in}{1.485971in}}%
\pgfpathcurveto{\pgfqpoint{4.110765in}{1.493104in}}{\pgfqpoint{4.107931in}{1.499945in}}{\pgfqpoint{4.102888in}{1.504989in}}%
\pgfpathcurveto{\pgfqpoint{4.097844in}{1.510033in}}{\pgfqpoint{4.091002in}{1.512867in}}{\pgfqpoint{4.083869in}{1.512867in}}%
\pgfpathcurveto{\pgfqpoint{4.076737in}{1.512867in}}{\pgfqpoint{4.069895in}{1.510033in}}{\pgfqpoint{4.064851in}{1.504989in}}%
\pgfpathcurveto{\pgfqpoint{4.059808in}{1.499945in}}{\pgfqpoint{4.056974in}{1.493104in}}{\pgfqpoint{4.056974in}{1.485971in}}%
\pgfpathcurveto{\pgfqpoint{4.056974in}{1.478838in}}{\pgfqpoint{4.059808in}{1.471996in}}{\pgfqpoint{4.064851in}{1.466953in}}%
\pgfpathcurveto{\pgfqpoint{4.069895in}{1.461909in}}{\pgfqpoint{4.076737in}{1.459075in}}{\pgfqpoint{4.083869in}{1.459075in}}%
\pgfpathclose%
\pgfusepath{stroke,fill}%
\end{pgfscope}%
\begin{pgfscope}%
\pgfpathrectangle{\pgfqpoint{2.867647in}{0.500000in}}{\pgfqpoint{1.764706in}{1.700000in}}%
\pgfusepath{clip}%
\pgfsetbuttcap%
\pgfsetroundjoin%
\definecolor{currentfill}{rgb}{0.926767,0.298233,0.242855}%
\pgfsetfillcolor{currentfill}%
\pgfsetlinewidth{0.311001pt}%
\definecolor{currentstroke}{rgb}{1.000000,1.000000,1.000000}%
\pgfsetstrokecolor{currentstroke}%
\pgfsetdash{}{0pt}%
\pgfpathmoveto{\pgfqpoint{3.782115in}{1.731313in}}%
\pgfpathcurveto{\pgfqpoint{3.789248in}{1.731313in}}{\pgfqpoint{3.796090in}{1.734147in}}{\pgfqpoint{3.801134in}{1.739191in}}%
\pgfpathcurveto{\pgfqpoint{3.806177in}{1.744234in}}{\pgfqpoint{3.809011in}{1.751076in}}{\pgfqpoint{3.809011in}{1.758209in}}%
\pgfpathcurveto{\pgfqpoint{3.809011in}{1.765342in}}{\pgfqpoint{3.806177in}{1.772183in}}{\pgfqpoint{3.801134in}{1.777227in}}%
\pgfpathcurveto{\pgfqpoint{3.796090in}{1.782271in}}{\pgfqpoint{3.789248in}{1.785104in}}{\pgfqpoint{3.782115in}{1.785104in}}%
\pgfpathcurveto{\pgfqpoint{3.774983in}{1.785104in}}{\pgfqpoint{3.768141in}{1.782271in}}{\pgfqpoint{3.763097in}{1.777227in}}%
\pgfpathcurveto{\pgfqpoint{3.758054in}{1.772183in}}{\pgfqpoint{3.755220in}{1.765342in}}{\pgfqpoint{3.755220in}{1.758209in}}%
\pgfpathcurveto{\pgfqpoint{3.755220in}{1.751076in}}{\pgfqpoint{3.758054in}{1.744234in}}{\pgfqpoint{3.763097in}{1.739191in}}%
\pgfpathcurveto{\pgfqpoint{3.768141in}{1.734147in}}{\pgfqpoint{3.774983in}{1.731313in}}{\pgfqpoint{3.782115in}{1.731313in}}%
\pgfpathclose%
\pgfusepath{stroke,fill}%
\end{pgfscope}%
\begin{pgfscope}%
\pgfpathrectangle{\pgfqpoint{2.867647in}{0.500000in}}{\pgfqpoint{1.764706in}{1.700000in}}%
\pgfusepath{clip}%
\pgfsetbuttcap%
\pgfsetroundjoin%
\definecolor{currentfill}{rgb}{0.964032,0.651225,0.493258}%
\pgfsetfillcolor{currentfill}%
\pgfsetlinewidth{0.311001pt}%
\definecolor{currentstroke}{rgb}{1.000000,1.000000,1.000000}%
\pgfsetstrokecolor{currentstroke}%
\pgfsetdash{}{0pt}%
\pgfpathmoveto{\pgfqpoint{3.958753in}{1.718965in}}%
\pgfpathcurveto{\pgfqpoint{3.965886in}{1.718965in}}{\pgfqpoint{3.972728in}{1.721798in}}{\pgfqpoint{3.977771in}{1.726842in}}%
\pgfpathcurveto{\pgfqpoint{3.982815in}{1.731886in}}{\pgfqpoint{3.985649in}{1.738727in}}{\pgfqpoint{3.985649in}{1.745860in}}%
\pgfpathcurveto{\pgfqpoint{3.985649in}{1.752993in}}{\pgfqpoint{3.982815in}{1.759835in}}{\pgfqpoint{3.977771in}{1.764878in}}%
\pgfpathcurveto{\pgfqpoint{3.972728in}{1.769922in}}{\pgfqpoint{3.965886in}{1.772756in}}{\pgfqpoint{3.958753in}{1.772756in}}%
\pgfpathcurveto{\pgfqpoint{3.951620in}{1.772756in}}{\pgfqpoint{3.944779in}{1.769922in}}{\pgfqpoint{3.939735in}{1.764878in}}%
\pgfpathcurveto{\pgfqpoint{3.934691in}{1.759835in}}{\pgfqpoint{3.931858in}{1.752993in}}{\pgfqpoint{3.931858in}{1.745860in}}%
\pgfpathcurveto{\pgfqpoint{3.931858in}{1.738727in}}{\pgfqpoint{3.934691in}{1.731886in}}{\pgfqpoint{3.939735in}{1.726842in}}%
\pgfpathcurveto{\pgfqpoint{3.944779in}{1.721798in}}{\pgfqpoint{3.951620in}{1.718965in}}{\pgfqpoint{3.958753in}{1.718965in}}%
\pgfpathclose%
\pgfusepath{stroke,fill}%
\end{pgfscope}%
\begin{pgfscope}%
\pgfpathrectangle{\pgfqpoint{2.867647in}{0.500000in}}{\pgfqpoint{1.764706in}{1.700000in}}%
\pgfusepath{clip}%
\pgfsetbuttcap%
\pgfsetroundjoin%
\definecolor{currentfill}{rgb}{0.972201,0.839051,0.745789}%
\pgfsetfillcolor{currentfill}%
\pgfsetlinewidth{0.311001pt}%
\definecolor{currentstroke}{rgb}{1.000000,1.000000,1.000000}%
\pgfsetstrokecolor{currentstroke}%
\pgfsetdash{}{0pt}%
\pgfpathmoveto{\pgfqpoint{4.248443in}{1.451463in}}%
\pgfpathcurveto{\pgfqpoint{4.255576in}{1.451463in}}{\pgfqpoint{4.262418in}{1.454297in}}{\pgfqpoint{4.267462in}{1.459341in}}%
\pgfpathcurveto{\pgfqpoint{4.272505in}{1.464384in}}{\pgfqpoint{4.275339in}{1.471226in}}{\pgfqpoint{4.275339in}{1.478359in}}%
\pgfpathcurveto{\pgfqpoint{4.275339in}{1.485492in}}{\pgfqpoint{4.272505in}{1.492333in}}{\pgfqpoint{4.267462in}{1.497377in}}%
\pgfpathcurveto{\pgfqpoint{4.262418in}{1.502421in}}{\pgfqpoint{4.255576in}{1.505255in}}{\pgfqpoint{4.248443in}{1.505255in}}%
\pgfpathcurveto{\pgfqpoint{4.241311in}{1.505255in}}{\pgfqpoint{4.234469in}{1.502421in}}{\pgfqpoint{4.229425in}{1.497377in}}%
\pgfpathcurveto{\pgfqpoint{4.224382in}{1.492333in}}{\pgfqpoint{4.221548in}{1.485492in}}{\pgfqpoint{4.221548in}{1.478359in}}%
\pgfpathcurveto{\pgfqpoint{4.221548in}{1.471226in}}{\pgfqpoint{4.224382in}{1.464384in}}{\pgfqpoint{4.229425in}{1.459341in}}%
\pgfpathcurveto{\pgfqpoint{4.234469in}{1.454297in}}{\pgfqpoint{4.241311in}{1.451463in}}{\pgfqpoint{4.248443in}{1.451463in}}%
\pgfpathclose%
\pgfusepath{stroke,fill}%
\end{pgfscope}%
\begin{pgfscope}%
\pgfpathrectangle{\pgfqpoint{2.867647in}{0.500000in}}{\pgfqpoint{1.764706in}{1.700000in}}%
\pgfusepath{clip}%
\pgfsetbuttcap%
\pgfsetroundjoin%
\definecolor{currentfill}{rgb}{0.981377,0.920617,0.865369}%
\pgfsetfillcolor{currentfill}%
\pgfsetlinewidth{0.311001pt}%
\definecolor{currentstroke}{rgb}{1.000000,1.000000,1.000000}%
\pgfsetstrokecolor{currentstroke}%
\pgfsetdash{}{0pt}%
\pgfpathmoveto{\pgfqpoint{4.180472in}{1.300705in}}%
\pgfpathcurveto{\pgfqpoint{4.187605in}{1.300705in}}{\pgfqpoint{4.194447in}{1.303539in}}{\pgfqpoint{4.199490in}{1.308583in}}%
\pgfpathcurveto{\pgfqpoint{4.204534in}{1.313626in}}{\pgfqpoint{4.207368in}{1.320468in}}{\pgfqpoint{4.207368in}{1.327601in}}%
\pgfpathcurveto{\pgfqpoint{4.207368in}{1.334734in}}{\pgfqpoint{4.204534in}{1.341575in}}{\pgfqpoint{4.199490in}{1.346619in}}%
\pgfpathcurveto{\pgfqpoint{4.194447in}{1.351663in}}{\pgfqpoint{4.187605in}{1.354497in}}{\pgfqpoint{4.180472in}{1.354497in}}%
\pgfpathcurveto{\pgfqpoint{4.173339in}{1.354497in}}{\pgfqpoint{4.166498in}{1.351663in}}{\pgfqpoint{4.161454in}{1.346619in}}%
\pgfpathcurveto{\pgfqpoint{4.156410in}{1.341575in}}{\pgfqpoint{4.153576in}{1.334734in}}{\pgfqpoint{4.153576in}{1.327601in}}%
\pgfpathcurveto{\pgfqpoint{4.153576in}{1.320468in}}{\pgfqpoint{4.156410in}{1.313626in}}{\pgfqpoint{4.161454in}{1.308583in}}%
\pgfpathcurveto{\pgfqpoint{4.166498in}{1.303539in}}{\pgfqpoint{4.173339in}{1.300705in}}{\pgfqpoint{4.180472in}{1.300705in}}%
\pgfpathclose%
\pgfusepath{stroke,fill}%
\end{pgfscope}%
\begin{pgfscope}%
\pgfpathrectangle{\pgfqpoint{2.867647in}{0.500000in}}{\pgfqpoint{1.764706in}{1.700000in}}%
\pgfusepath{clip}%
\pgfsetbuttcap%
\pgfsetroundjoin%
\definecolor{currentfill}{rgb}{0.980678,0.914765,0.856766}%
\pgfsetfillcolor{currentfill}%
\pgfsetlinewidth{0.311001pt}%
\definecolor{currentstroke}{rgb}{1.000000,1.000000,1.000000}%
\pgfsetstrokecolor{currentstroke}%
\pgfsetdash{}{0pt}%
\pgfpathmoveto{\pgfqpoint{4.172419in}{1.311550in}}%
\pgfpathcurveto{\pgfqpoint{4.179552in}{1.311550in}}{\pgfqpoint{4.186393in}{1.314384in}}{\pgfqpoint{4.191437in}{1.319428in}}%
\pgfpathcurveto{\pgfqpoint{4.196481in}{1.324471in}}{\pgfqpoint{4.199315in}{1.331313in}}{\pgfqpoint{4.199315in}{1.338446in}}%
\pgfpathcurveto{\pgfqpoint{4.199315in}{1.345579in}}{\pgfqpoint{4.196481in}{1.352420in}}{\pgfqpoint{4.191437in}{1.357464in}}%
\pgfpathcurveto{\pgfqpoint{4.186393in}{1.362508in}}{\pgfqpoint{4.179552in}{1.365342in}}{\pgfqpoint{4.172419in}{1.365342in}}%
\pgfpathcurveto{\pgfqpoint{4.165286in}{1.365342in}}{\pgfqpoint{4.158444in}{1.362508in}}{\pgfqpoint{4.153401in}{1.357464in}}%
\pgfpathcurveto{\pgfqpoint{4.148357in}{1.352420in}}{\pgfqpoint{4.145523in}{1.345579in}}{\pgfqpoint{4.145523in}{1.338446in}}%
\pgfpathcurveto{\pgfqpoint{4.145523in}{1.331313in}}{\pgfqpoint{4.148357in}{1.324471in}}{\pgfqpoint{4.153401in}{1.319428in}}%
\pgfpathcurveto{\pgfqpoint{4.158444in}{1.314384in}}{\pgfqpoint{4.165286in}{1.311550in}}{\pgfqpoint{4.172419in}{1.311550in}}%
\pgfpathclose%
\pgfusepath{stroke,fill}%
\end{pgfscope}%
\begin{pgfscope}%
\pgfpathrectangle{\pgfqpoint{2.867647in}{0.500000in}}{\pgfqpoint{1.764706in}{1.700000in}}%
\pgfusepath{clip}%
\pgfsetbuttcap%
\pgfsetroundjoin%
\definecolor{currentfill}{rgb}{0.965592,0.726236,0.584384}%
\pgfsetfillcolor{currentfill}%
\pgfsetlinewidth{0.311001pt}%
\definecolor{currentstroke}{rgb}{1.000000,1.000000,1.000000}%
\pgfsetstrokecolor{currentstroke}%
\pgfsetdash{}{0pt}%
\pgfpathmoveto{\pgfqpoint{4.303957in}{1.297392in}}%
\pgfpathcurveto{\pgfqpoint{4.311090in}{1.297392in}}{\pgfqpoint{4.317932in}{1.300226in}}{\pgfqpoint{4.322976in}{1.305269in}}%
\pgfpathcurveto{\pgfqpoint{4.328019in}{1.310313in}}{\pgfqpoint{4.330853in}{1.317155in}}{\pgfqpoint{4.330853in}{1.324288in}}%
\pgfpathcurveto{\pgfqpoint{4.330853in}{1.331420in}}{\pgfqpoint{4.328019in}{1.338262in}}{\pgfqpoint{4.322976in}{1.343306in}}%
\pgfpathcurveto{\pgfqpoint{4.317932in}{1.348349in}}{\pgfqpoint{4.311090in}{1.351183in}}{\pgfqpoint{4.303957in}{1.351183in}}%
\pgfpathcurveto{\pgfqpoint{4.296825in}{1.351183in}}{\pgfqpoint{4.289983in}{1.348349in}}{\pgfqpoint{4.284939in}{1.343306in}}%
\pgfpathcurveto{\pgfqpoint{4.279896in}{1.338262in}}{\pgfqpoint{4.277062in}{1.331420in}}{\pgfqpoint{4.277062in}{1.324288in}}%
\pgfpathcurveto{\pgfqpoint{4.277062in}{1.317155in}}{\pgfqpoint{4.279896in}{1.310313in}}{\pgfqpoint{4.284939in}{1.305269in}}%
\pgfpathcurveto{\pgfqpoint{4.289983in}{1.300226in}}{\pgfqpoint{4.296825in}{1.297392in}}{\pgfqpoint{4.303957in}{1.297392in}}%
\pgfpathclose%
\pgfusepath{stroke,fill}%
\end{pgfscope}%
\begin{pgfscope}%
\pgfpathrectangle{\pgfqpoint{2.867647in}{0.500000in}}{\pgfqpoint{1.764706in}{1.700000in}}%
\pgfusepath{clip}%
\pgfsetbuttcap%
\pgfsetroundjoin%
\definecolor{currentfill}{rgb}{0.976287,0.879862,0.805788}%
\pgfsetfillcolor{currentfill}%
\pgfsetlinewidth{0.311001pt}%
\definecolor{currentstroke}{rgb}{1.000000,1.000000,1.000000}%
\pgfsetstrokecolor{currentstroke}%
\pgfsetdash{}{0pt}%
\pgfpathmoveto{\pgfqpoint{4.121139in}{1.466035in}}%
\pgfpathcurveto{\pgfqpoint{4.128272in}{1.466035in}}{\pgfqpoint{4.135114in}{1.468869in}}{\pgfqpoint{4.140157in}{1.473913in}}%
\pgfpathcurveto{\pgfqpoint{4.145201in}{1.478957in}}{\pgfqpoint{4.148035in}{1.485798in}}{\pgfqpoint{4.148035in}{1.492931in}}%
\pgfpathcurveto{\pgfqpoint{4.148035in}{1.500064in}}{\pgfqpoint{4.145201in}{1.506906in}}{\pgfqpoint{4.140157in}{1.511949in}}%
\pgfpathcurveto{\pgfqpoint{4.135114in}{1.516993in}}{\pgfqpoint{4.128272in}{1.519827in}}{\pgfqpoint{4.121139in}{1.519827in}}%
\pgfpathcurveto{\pgfqpoint{4.114006in}{1.519827in}}{\pgfqpoint{4.107165in}{1.516993in}}{\pgfqpoint{4.102121in}{1.511949in}}%
\pgfpathcurveto{\pgfqpoint{4.097077in}{1.506906in}}{\pgfqpoint{4.094244in}{1.500064in}}{\pgfqpoint{4.094244in}{1.492931in}}%
\pgfpathcurveto{\pgfqpoint{4.094244in}{1.485798in}}{\pgfqpoint{4.097077in}{1.478957in}}{\pgfqpoint{4.102121in}{1.473913in}}%
\pgfpathcurveto{\pgfqpoint{4.107165in}{1.468869in}}{\pgfqpoint{4.114006in}{1.466035in}}{\pgfqpoint{4.121139in}{1.466035in}}%
\pgfpathclose%
\pgfusepath{stroke,fill}%
\end{pgfscope}%
\begin{pgfscope}%
\pgfpathrectangle{\pgfqpoint{2.867647in}{0.500000in}}{\pgfqpoint{1.764706in}{1.700000in}}%
\pgfusepath{clip}%
\pgfsetbuttcap%
\pgfsetroundjoin%
\definecolor{currentfill}{rgb}{0.970255,0.815666,0.711203}%
\pgfsetfillcolor{currentfill}%
\pgfsetlinewidth{0.311001pt}%
\definecolor{currentstroke}{rgb}{1.000000,1.000000,1.000000}%
\pgfsetstrokecolor{currentstroke}%
\pgfsetdash{}{0pt}%
\pgfpathmoveto{\pgfqpoint{4.046326in}{1.020909in}}%
\pgfpathcurveto{\pgfqpoint{4.053459in}{1.020909in}}{\pgfqpoint{4.060300in}{1.023743in}}{\pgfqpoint{4.065344in}{1.028786in}}%
\pgfpathcurveto{\pgfqpoint{4.070388in}{1.033830in}}{\pgfqpoint{4.073222in}{1.040672in}}{\pgfqpoint{4.073222in}{1.047804in}}%
\pgfpathcurveto{\pgfqpoint{4.073222in}{1.054937in}}{\pgfqpoint{4.070388in}{1.061779in}}{\pgfqpoint{4.065344in}{1.066822in}}%
\pgfpathcurveto{\pgfqpoint{4.060300in}{1.071866in}}{\pgfqpoint{4.053459in}{1.074700in}}{\pgfqpoint{4.046326in}{1.074700in}}%
\pgfpathcurveto{\pgfqpoint{4.039193in}{1.074700in}}{\pgfqpoint{4.032351in}{1.071866in}}{\pgfqpoint{4.027308in}{1.066822in}}%
\pgfpathcurveto{\pgfqpoint{4.022264in}{1.061779in}}{\pgfqpoint{4.019430in}{1.054937in}}{\pgfqpoint{4.019430in}{1.047804in}}%
\pgfpathcurveto{\pgfqpoint{4.019430in}{1.040672in}}{\pgfqpoint{4.022264in}{1.033830in}}{\pgfqpoint{4.027308in}{1.028786in}}%
\pgfpathcurveto{\pgfqpoint{4.032351in}{1.023743in}}{\pgfqpoint{4.039193in}{1.020909in}}{\pgfqpoint{4.046326in}{1.020909in}}%
\pgfpathclose%
\pgfusepath{stroke,fill}%
\end{pgfscope}%
\begin{pgfscope}%
\pgfpathrectangle{\pgfqpoint{2.867647in}{0.500000in}}{\pgfqpoint{1.764706in}{1.700000in}}%
\pgfusepath{clip}%
\pgfsetbuttcap%
\pgfsetroundjoin%
\definecolor{currentfill}{rgb}{0.965169,0.707764,0.560659}%
\pgfsetfillcolor{currentfill}%
\pgfsetlinewidth{0.311001pt}%
\definecolor{currentstroke}{rgb}{1.000000,1.000000,1.000000}%
\pgfsetstrokecolor{currentstroke}%
\pgfsetdash{}{0pt}%
\pgfpathmoveto{\pgfqpoint{4.006875in}{1.738030in}}%
\pgfpathcurveto{\pgfqpoint{4.014008in}{1.738030in}}{\pgfqpoint{4.020849in}{1.740864in}}{\pgfqpoint{4.025893in}{1.745908in}}%
\pgfpathcurveto{\pgfqpoint{4.030937in}{1.750951in}}{\pgfqpoint{4.033771in}{1.757793in}}{\pgfqpoint{4.033771in}{1.764926in}}%
\pgfpathcurveto{\pgfqpoint{4.033771in}{1.772059in}}{\pgfqpoint{4.030937in}{1.778900in}}{\pgfqpoint{4.025893in}{1.783944in}}%
\pgfpathcurveto{\pgfqpoint{4.020849in}{1.788988in}}{\pgfqpoint{4.014008in}{1.791821in}}{\pgfqpoint{4.006875in}{1.791821in}}%
\pgfpathcurveto{\pgfqpoint{3.999742in}{1.791821in}}{\pgfqpoint{3.992900in}{1.788988in}}{\pgfqpoint{3.987857in}{1.783944in}}%
\pgfpathcurveto{\pgfqpoint{3.982813in}{1.778900in}}{\pgfqpoint{3.979979in}{1.772059in}}{\pgfqpoint{3.979979in}{1.764926in}}%
\pgfpathcurveto{\pgfqpoint{3.979979in}{1.757793in}}{\pgfqpoint{3.982813in}{1.750951in}}{\pgfqpoint{3.987857in}{1.745908in}}%
\pgfpathcurveto{\pgfqpoint{3.992900in}{1.740864in}}{\pgfqpoint{3.999742in}{1.738030in}}{\pgfqpoint{4.006875in}{1.738030in}}%
\pgfpathclose%
\pgfusepath{stroke,fill}%
\end{pgfscope}%
\begin{pgfscope}%
\pgfpathrectangle{\pgfqpoint{2.867647in}{0.500000in}}{\pgfqpoint{1.764706in}{1.700000in}}%
\pgfusepath{clip}%
\pgfsetbuttcap%
\pgfsetroundjoin%
\definecolor{currentfill}{rgb}{0.980678,0.914765,0.856766}%
\pgfsetfillcolor{currentfill}%
\pgfsetlinewidth{0.311001pt}%
\definecolor{currentstroke}{rgb}{1.000000,1.000000,1.000000}%
\pgfsetstrokecolor{currentstroke}%
\pgfsetdash{}{0pt}%
\pgfpathmoveto{\pgfqpoint{4.176328in}{1.384041in}}%
\pgfpathcurveto{\pgfqpoint{4.183461in}{1.384041in}}{\pgfqpoint{4.190303in}{1.386875in}}{\pgfqpoint{4.195346in}{1.391919in}}%
\pgfpathcurveto{\pgfqpoint{4.200390in}{1.396962in}}{\pgfqpoint{4.203224in}{1.403804in}}{\pgfqpoint{4.203224in}{1.410937in}}%
\pgfpathcurveto{\pgfqpoint{4.203224in}{1.418070in}}{\pgfqpoint{4.200390in}{1.424911in}}{\pgfqpoint{4.195346in}{1.429955in}}%
\pgfpathcurveto{\pgfqpoint{4.190303in}{1.434999in}}{\pgfqpoint{4.183461in}{1.437833in}}{\pgfqpoint{4.176328in}{1.437833in}}%
\pgfpathcurveto{\pgfqpoint{4.169195in}{1.437833in}}{\pgfqpoint{4.162354in}{1.434999in}}{\pgfqpoint{4.157310in}{1.429955in}}%
\pgfpathcurveto{\pgfqpoint{4.152266in}{1.424911in}}{\pgfqpoint{4.149433in}{1.418070in}}{\pgfqpoint{4.149433in}{1.410937in}}%
\pgfpathcurveto{\pgfqpoint{4.149433in}{1.403804in}}{\pgfqpoint{4.152266in}{1.396962in}}{\pgfqpoint{4.157310in}{1.391919in}}%
\pgfpathcurveto{\pgfqpoint{4.162354in}{1.386875in}}{\pgfqpoint{4.169195in}{1.384041in}}{\pgfqpoint{4.176328in}{1.384041in}}%
\pgfpathclose%
\pgfusepath{stroke,fill}%
\end{pgfscope}%
\begin{pgfscope}%
\pgfpathrectangle{\pgfqpoint{2.867647in}{0.500000in}}{\pgfqpoint{1.764706in}{1.700000in}}%
\pgfusepath{clip}%
\pgfsetbuttcap%
\pgfsetroundjoin%
\definecolor{currentfill}{rgb}{0.980678,0.914765,0.856766}%
\pgfsetfillcolor{currentfill}%
\pgfsetlinewidth{0.311001pt}%
\definecolor{currentstroke}{rgb}{1.000000,1.000000,1.000000}%
\pgfsetstrokecolor{currentstroke}%
\pgfsetdash{}{0pt}%
\pgfpathmoveto{\pgfqpoint{4.163953in}{1.430443in}}%
\pgfpathcurveto{\pgfqpoint{4.171086in}{1.430443in}}{\pgfqpoint{4.177927in}{1.433277in}}{\pgfqpoint{4.182971in}{1.438321in}}%
\pgfpathcurveto{\pgfqpoint{4.188015in}{1.443365in}}{\pgfqpoint{4.190849in}{1.450206in}}{\pgfqpoint{4.190849in}{1.457339in}}%
\pgfpathcurveto{\pgfqpoint{4.190849in}{1.464472in}}{\pgfqpoint{4.188015in}{1.471314in}}{\pgfqpoint{4.182971in}{1.476357in}}%
\pgfpathcurveto{\pgfqpoint{4.177927in}{1.481401in}}{\pgfqpoint{4.171086in}{1.484235in}}{\pgfqpoint{4.163953in}{1.484235in}}%
\pgfpathcurveto{\pgfqpoint{4.156820in}{1.484235in}}{\pgfqpoint{4.149978in}{1.481401in}}{\pgfqpoint{4.144935in}{1.476357in}}%
\pgfpathcurveto{\pgfqpoint{4.139891in}{1.471314in}}{\pgfqpoint{4.137057in}{1.464472in}}{\pgfqpoint{4.137057in}{1.457339in}}%
\pgfpathcurveto{\pgfqpoint{4.137057in}{1.450206in}}{\pgfqpoint{4.139891in}{1.443365in}}{\pgfqpoint{4.144935in}{1.438321in}}%
\pgfpathcurveto{\pgfqpoint{4.149978in}{1.433277in}}{\pgfqpoint{4.156820in}{1.430443in}}{\pgfqpoint{4.163953in}{1.430443in}}%
\pgfpathclose%
\pgfusepath{stroke,fill}%
\end{pgfscope}%
\begin{pgfscope}%
\pgfpathrectangle{\pgfqpoint{2.867647in}{0.500000in}}{\pgfqpoint{1.764706in}{1.700000in}}%
\pgfusepath{clip}%
\pgfsetbuttcap%
\pgfsetroundjoin%
\definecolor{currentfill}{rgb}{0.964173,0.657587,0.500469}%
\pgfsetfillcolor{currentfill}%
\pgfsetlinewidth{0.311001pt}%
\definecolor{currentstroke}{rgb}{1.000000,1.000000,1.000000}%
\pgfsetstrokecolor{currentstroke}%
\pgfsetdash{}{0pt}%
\pgfpathmoveto{\pgfqpoint{4.265703in}{1.051964in}}%
\pgfpathcurveto{\pgfqpoint{4.272836in}{1.051964in}}{\pgfqpoint{4.279678in}{1.054798in}}{\pgfqpoint{4.284721in}{1.059842in}}%
\pgfpathcurveto{\pgfqpoint{4.289765in}{1.064886in}}{\pgfqpoint{4.292599in}{1.071727in}}{\pgfqpoint{4.292599in}{1.078860in}}%
\pgfpathcurveto{\pgfqpoint{4.292599in}{1.085993in}}{\pgfqpoint{4.289765in}{1.092835in}}{\pgfqpoint{4.284721in}{1.097878in}}%
\pgfpathcurveto{\pgfqpoint{4.279678in}{1.102922in}}{\pgfqpoint{4.272836in}{1.105756in}}{\pgfqpoint{4.265703in}{1.105756in}}%
\pgfpathcurveto{\pgfqpoint{4.258570in}{1.105756in}}{\pgfqpoint{4.251729in}{1.102922in}}{\pgfqpoint{4.246685in}{1.097878in}}%
\pgfpathcurveto{\pgfqpoint{4.241641in}{1.092835in}}{\pgfqpoint{4.238807in}{1.085993in}}{\pgfqpoint{4.238807in}{1.078860in}}%
\pgfpathcurveto{\pgfqpoint{4.238807in}{1.071727in}}{\pgfqpoint{4.241641in}{1.064886in}}{\pgfqpoint{4.246685in}{1.059842in}}%
\pgfpathcurveto{\pgfqpoint{4.251729in}{1.054798in}}{\pgfqpoint{4.258570in}{1.051964in}}{\pgfqpoint{4.265703in}{1.051964in}}%
\pgfpathclose%
\pgfusepath{stroke,fill}%
\end{pgfscope}%
\begin{pgfscope}%
\pgfpathrectangle{\pgfqpoint{2.867647in}{0.500000in}}{\pgfqpoint{1.764706in}{1.700000in}}%
\pgfusepath{clip}%
\pgfsetbuttcap%
\pgfsetroundjoin%
\definecolor{currentfill}{rgb}{0.970255,0.815666,0.711203}%
\pgfsetfillcolor{currentfill}%
\pgfsetlinewidth{0.311001pt}%
\definecolor{currentstroke}{rgb}{1.000000,1.000000,1.000000}%
\pgfsetstrokecolor{currentstroke}%
\pgfsetdash{}{0pt}%
\pgfpathmoveto{\pgfqpoint{4.271967in}{1.252049in}}%
\pgfpathcurveto{\pgfqpoint{4.279100in}{1.252049in}}{\pgfqpoint{4.285941in}{1.254883in}}{\pgfqpoint{4.290985in}{1.259927in}}%
\pgfpathcurveto{\pgfqpoint{4.296029in}{1.264970in}}{\pgfqpoint{4.298862in}{1.271812in}}{\pgfqpoint{4.298862in}{1.278945in}}%
\pgfpathcurveto{\pgfqpoint{4.298862in}{1.286078in}}{\pgfqpoint{4.296029in}{1.292919in}}{\pgfqpoint{4.290985in}{1.297963in}}%
\pgfpathcurveto{\pgfqpoint{4.285941in}{1.303007in}}{\pgfqpoint{4.279100in}{1.305841in}}{\pgfqpoint{4.271967in}{1.305841in}}%
\pgfpathcurveto{\pgfqpoint{4.264834in}{1.305841in}}{\pgfqpoint{4.257992in}{1.303007in}}{\pgfqpoint{4.252949in}{1.297963in}}%
\pgfpathcurveto{\pgfqpoint{4.247905in}{1.292919in}}{\pgfqpoint{4.245071in}{1.286078in}}{\pgfqpoint{4.245071in}{1.278945in}}%
\pgfpathcurveto{\pgfqpoint{4.245071in}{1.271812in}}{\pgfqpoint{4.247905in}{1.264970in}}{\pgfqpoint{4.252949in}{1.259927in}}%
\pgfpathcurveto{\pgfqpoint{4.257992in}{1.254883in}}{\pgfqpoint{4.264834in}{1.252049in}}{\pgfqpoint{4.271967in}{1.252049in}}%
\pgfpathclose%
\pgfusepath{stroke,fill}%
\end{pgfscope}%
\begin{pgfscope}%
\pgfpathrectangle{\pgfqpoint{2.867647in}{0.500000in}}{\pgfqpoint{1.764706in}{1.700000in}}%
\pgfusepath{clip}%
\pgfsetbuttcap%
\pgfsetroundjoin%
\definecolor{currentfill}{rgb}{0.965440,0.720101,0.576404}%
\pgfsetfillcolor{currentfill}%
\pgfsetlinewidth{0.311001pt}%
\definecolor{currentstroke}{rgb}{1.000000,1.000000,1.000000}%
\pgfsetstrokecolor{currentstroke}%
\pgfsetdash{}{0pt}%
\pgfpathmoveto{\pgfqpoint{4.131413in}{0.915730in}}%
\pgfpathcurveto{\pgfqpoint{4.138546in}{0.915730in}}{\pgfqpoint{4.145388in}{0.918564in}}{\pgfqpoint{4.150431in}{0.923608in}}%
\pgfpathcurveto{\pgfqpoint{4.155475in}{0.928651in}}{\pgfqpoint{4.158309in}{0.935493in}}{\pgfqpoint{4.158309in}{0.942626in}}%
\pgfpathcurveto{\pgfqpoint{4.158309in}{0.949758in}}{\pgfqpoint{4.155475in}{0.956600in}}{\pgfqpoint{4.150431in}{0.961644in}}%
\pgfpathcurveto{\pgfqpoint{4.145388in}{0.966687in}}{\pgfqpoint{4.138546in}{0.969521in}}{\pgfqpoint{4.131413in}{0.969521in}}%
\pgfpathcurveto{\pgfqpoint{4.124280in}{0.969521in}}{\pgfqpoint{4.117439in}{0.966687in}}{\pgfqpoint{4.112395in}{0.961644in}}%
\pgfpathcurveto{\pgfqpoint{4.107351in}{0.956600in}}{\pgfqpoint{4.104517in}{0.949758in}}{\pgfqpoint{4.104517in}{0.942626in}}%
\pgfpathcurveto{\pgfqpoint{4.104517in}{0.935493in}}{\pgfqpoint{4.107351in}{0.928651in}}{\pgfqpoint{4.112395in}{0.923608in}}%
\pgfpathcurveto{\pgfqpoint{4.117439in}{0.918564in}}{\pgfqpoint{4.124280in}{0.915730in}}{\pgfqpoint{4.131413in}{0.915730in}}%
\pgfpathclose%
\pgfusepath{stroke,fill}%
\end{pgfscope}%
\begin{pgfscope}%
\pgfpathrectangle{\pgfqpoint{2.867647in}{0.500000in}}{\pgfqpoint{1.764706in}{1.700000in}}%
\pgfusepath{clip}%
\pgfsetbuttcap%
\pgfsetroundjoin%
\definecolor{currentfill}{rgb}{0.976961,0.885681,0.814303}%
\pgfsetfillcolor{currentfill}%
\pgfsetlinewidth{0.311001pt}%
\definecolor{currentstroke}{rgb}{1.000000,1.000000,1.000000}%
\pgfsetstrokecolor{currentstroke}%
\pgfsetdash{}{0pt}%
\pgfpathmoveto{\pgfqpoint{4.109295in}{1.507871in}}%
\pgfpathcurveto{\pgfqpoint{4.116428in}{1.507871in}}{\pgfqpoint{4.123270in}{1.510705in}}{\pgfqpoint{4.128314in}{1.515749in}}%
\pgfpathcurveto{\pgfqpoint{4.133357in}{1.520793in}}{\pgfqpoint{4.136191in}{1.527634in}}{\pgfqpoint{4.136191in}{1.534767in}}%
\pgfpathcurveto{\pgfqpoint{4.136191in}{1.541900in}}{\pgfqpoint{4.133357in}{1.548741in}}{\pgfqpoint{4.128314in}{1.553785in}}%
\pgfpathcurveto{\pgfqpoint{4.123270in}{1.558829in}}{\pgfqpoint{4.116428in}{1.561663in}}{\pgfqpoint{4.109295in}{1.561663in}}%
\pgfpathcurveto{\pgfqpoint{4.102163in}{1.561663in}}{\pgfqpoint{4.095321in}{1.558829in}}{\pgfqpoint{4.090277in}{1.553785in}}%
\pgfpathcurveto{\pgfqpoint{4.085234in}{1.548741in}}{\pgfqpoint{4.082400in}{1.541900in}}{\pgfqpoint{4.082400in}{1.534767in}}%
\pgfpathcurveto{\pgfqpoint{4.082400in}{1.527634in}}{\pgfqpoint{4.085234in}{1.520793in}}{\pgfqpoint{4.090277in}{1.515749in}}%
\pgfpathcurveto{\pgfqpoint{4.095321in}{1.510705in}}{\pgfqpoint{4.102163in}{1.507871in}}{\pgfqpoint{4.109295in}{1.507871in}}%
\pgfpathclose%
\pgfusepath{stroke,fill}%
\end{pgfscope}%
\begin{pgfscope}%
\pgfpathrectangle{\pgfqpoint{2.867647in}{0.500000in}}{\pgfqpoint{1.764706in}{1.700000in}}%
\pgfusepath{clip}%
\pgfsetbuttcap%
\pgfsetroundjoin%
\definecolor{currentfill}{rgb}{0.980678,0.914765,0.856766}%
\pgfsetfillcolor{currentfill}%
\pgfsetlinewidth{0.311001pt}%
\definecolor{currentstroke}{rgb}{1.000000,1.000000,1.000000}%
\pgfsetstrokecolor{currentstroke}%
\pgfsetdash{}{0pt}%
\pgfpathmoveto{\pgfqpoint{4.171184in}{1.484786in}}%
\pgfpathcurveto{\pgfqpoint{4.178317in}{1.484786in}}{\pgfqpoint{4.185158in}{1.487620in}}{\pgfqpoint{4.190202in}{1.492663in}}%
\pgfpathcurveto{\pgfqpoint{4.195246in}{1.497707in}}{\pgfqpoint{4.198080in}{1.504549in}}{\pgfqpoint{4.198080in}{1.511681in}}%
\pgfpathcurveto{\pgfqpoint{4.198080in}{1.518814in}}{\pgfqpoint{4.195246in}{1.525656in}}{\pgfqpoint{4.190202in}{1.530700in}}%
\pgfpathcurveto{\pgfqpoint{4.185158in}{1.535743in}}{\pgfqpoint{4.178317in}{1.538577in}}{\pgfqpoint{4.171184in}{1.538577in}}%
\pgfpathcurveto{\pgfqpoint{4.164051in}{1.538577in}}{\pgfqpoint{4.157210in}{1.535743in}}{\pgfqpoint{4.152166in}{1.530700in}}%
\pgfpathcurveto{\pgfqpoint{4.147122in}{1.525656in}}{\pgfqpoint{4.144288in}{1.518814in}}{\pgfqpoint{4.144288in}{1.511681in}}%
\pgfpathcurveto{\pgfqpoint{4.144288in}{1.504549in}}{\pgfqpoint{4.147122in}{1.497707in}}{\pgfqpoint{4.152166in}{1.492663in}}%
\pgfpathcurveto{\pgfqpoint{4.157210in}{1.487620in}}{\pgfqpoint{4.164051in}{1.484786in}}{\pgfqpoint{4.171184in}{1.484786in}}%
\pgfpathclose%
\pgfusepath{stroke,fill}%
\end{pgfscope}%
\begin{pgfscope}%
\pgfpathrectangle{\pgfqpoint{2.867647in}{0.500000in}}{\pgfqpoint{1.764706in}{1.700000in}}%
\pgfusepath{clip}%
\pgfsetbuttcap%
\pgfsetroundjoin%
\definecolor{currentfill}{rgb}{0.977657,0.891500,0.822809}%
\pgfsetfillcolor{currentfill}%
\pgfsetlinewidth{0.311001pt}%
\definecolor{currentstroke}{rgb}{1.000000,1.000000,1.000000}%
\pgfsetstrokecolor{currentstroke}%
\pgfsetdash{}{0pt}%
\pgfpathmoveto{\pgfqpoint{4.147219in}{1.328715in}}%
\pgfpathcurveto{\pgfqpoint{4.154352in}{1.328715in}}{\pgfqpoint{4.161193in}{1.331549in}}{\pgfqpoint{4.166237in}{1.336593in}}%
\pgfpathcurveto{\pgfqpoint{4.171281in}{1.341636in}}{\pgfqpoint{4.174114in}{1.348478in}}{\pgfqpoint{4.174114in}{1.355611in}}%
\pgfpathcurveto{\pgfqpoint{4.174114in}{1.362743in}}{\pgfqpoint{4.171281in}{1.369585in}}{\pgfqpoint{4.166237in}{1.374629in}}%
\pgfpathcurveto{\pgfqpoint{4.161193in}{1.379672in}}{\pgfqpoint{4.154352in}{1.382506in}}{\pgfqpoint{4.147219in}{1.382506in}}%
\pgfpathcurveto{\pgfqpoint{4.140086in}{1.382506in}}{\pgfqpoint{4.133244in}{1.379672in}}{\pgfqpoint{4.128201in}{1.374629in}}%
\pgfpathcurveto{\pgfqpoint{4.123157in}{1.369585in}}{\pgfqpoint{4.120323in}{1.362743in}}{\pgfqpoint{4.120323in}{1.355611in}}%
\pgfpathcurveto{\pgfqpoint{4.120323in}{1.348478in}}{\pgfqpoint{4.123157in}{1.341636in}}{\pgfqpoint{4.128201in}{1.336593in}}%
\pgfpathcurveto{\pgfqpoint{4.133244in}{1.331549in}}{\pgfqpoint{4.140086in}{1.328715in}}{\pgfqpoint{4.147219in}{1.328715in}}%
\pgfpathclose%
\pgfusepath{stroke,fill}%
\end{pgfscope}%
\begin{pgfscope}%
\pgfpathrectangle{\pgfqpoint{2.867647in}{0.500000in}}{\pgfqpoint{1.764706in}{1.700000in}}%
\pgfusepath{clip}%
\pgfsetbuttcap%
\pgfsetroundjoin%
\definecolor{currentfill}{rgb}{0.958791,0.526283,0.368909}%
\pgfsetfillcolor{currentfill}%
\pgfsetlinewidth{0.311001pt}%
\definecolor{currentstroke}{rgb}{1.000000,1.000000,1.000000}%
\pgfsetstrokecolor{currentstroke}%
\pgfsetdash{}{0pt}%
\pgfpathmoveto{\pgfqpoint{3.981475in}{1.110834in}}%
\pgfpathcurveto{\pgfqpoint{3.988608in}{1.110834in}}{\pgfqpoint{3.995450in}{1.113668in}}{\pgfqpoint{4.000493in}{1.118711in}}%
\pgfpathcurveto{\pgfqpoint{4.005537in}{1.123755in}}{\pgfqpoint{4.008371in}{1.130597in}}{\pgfqpoint{4.008371in}{1.137729in}}%
\pgfpathcurveto{\pgfqpoint{4.008371in}{1.144862in}}{\pgfqpoint{4.005537in}{1.151704in}}{\pgfqpoint{4.000493in}{1.156748in}}%
\pgfpathcurveto{\pgfqpoint{3.995450in}{1.161791in}}{\pgfqpoint{3.988608in}{1.164625in}}{\pgfqpoint{3.981475in}{1.164625in}}%
\pgfpathcurveto{\pgfqpoint{3.974342in}{1.164625in}}{\pgfqpoint{3.967501in}{1.161791in}}{\pgfqpoint{3.962457in}{1.156748in}}%
\pgfpathcurveto{\pgfqpoint{3.957413in}{1.151704in}}{\pgfqpoint{3.954579in}{1.144862in}}{\pgfqpoint{3.954579in}{1.137729in}}%
\pgfpathcurveto{\pgfqpoint{3.954579in}{1.130597in}}{\pgfqpoint{3.957413in}{1.123755in}}{\pgfqpoint{3.962457in}{1.118711in}}%
\pgfpathcurveto{\pgfqpoint{3.967501in}{1.113668in}}{\pgfqpoint{3.974342in}{1.110834in}}{\pgfqpoint{3.981475in}{1.110834in}}%
\pgfpathclose%
\pgfusepath{stroke,fill}%
\end{pgfscope}%
\begin{pgfscope}%
\pgfpathrectangle{\pgfqpoint{2.867647in}{0.500000in}}{\pgfqpoint{1.764706in}{1.700000in}}%
\pgfusepath{clip}%
\pgfsetbuttcap%
\pgfsetroundjoin%
\definecolor{currentfill}{rgb}{0.964173,0.657587,0.500469}%
\pgfsetfillcolor{currentfill}%
\pgfsetlinewidth{0.311001pt}%
\definecolor{currentstroke}{rgb}{1.000000,1.000000,1.000000}%
\pgfsetstrokecolor{currentstroke}%
\pgfsetdash{}{0pt}%
\pgfpathmoveto{\pgfqpoint{3.977289in}{1.002513in}}%
\pgfpathcurveto{\pgfqpoint{3.984422in}{1.002513in}}{\pgfqpoint{3.991264in}{1.005347in}}{\pgfqpoint{3.996307in}{1.010391in}}%
\pgfpathcurveto{\pgfqpoint{4.001351in}{1.015435in}}{\pgfqpoint{4.004185in}{1.022276in}}{\pgfqpoint{4.004185in}{1.029409in}}%
\pgfpathcurveto{\pgfqpoint{4.004185in}{1.036542in}}{\pgfqpoint{4.001351in}{1.043384in}}{\pgfqpoint{3.996307in}{1.048427in}}%
\pgfpathcurveto{\pgfqpoint{3.991264in}{1.053471in}}{\pgfqpoint{3.984422in}{1.056305in}}{\pgfqpoint{3.977289in}{1.056305in}}%
\pgfpathcurveto{\pgfqpoint{3.970156in}{1.056305in}}{\pgfqpoint{3.963315in}{1.053471in}}{\pgfqpoint{3.958271in}{1.048427in}}%
\pgfpathcurveto{\pgfqpoint{3.953227in}{1.043384in}}{\pgfqpoint{3.950393in}{1.036542in}}{\pgfqpoint{3.950393in}{1.029409in}}%
\pgfpathcurveto{\pgfqpoint{3.950393in}{1.022276in}}{\pgfqpoint{3.953227in}{1.015435in}}{\pgfqpoint{3.958271in}{1.010391in}}%
\pgfpathcurveto{\pgfqpoint{3.963315in}{1.005347in}}{\pgfqpoint{3.970156in}{1.002513in}}{\pgfqpoint{3.977289in}{1.002513in}}%
\pgfpathclose%
\pgfusepath{stroke,fill}%
\end{pgfscope}%
\begin{pgfscope}%
\pgfpathrectangle{\pgfqpoint{2.867647in}{0.500000in}}{\pgfqpoint{1.764706in}{1.700000in}}%
\pgfusepath{clip}%
\pgfsetbuttcap%
\pgfsetroundjoin%
\definecolor{currentfill}{rgb}{0.973832,0.856556,0.771584}%
\pgfsetfillcolor{currentfill}%
\pgfsetlinewidth{0.311001pt}%
\definecolor{currentstroke}{rgb}{1.000000,1.000000,1.000000}%
\pgfsetstrokecolor{currentstroke}%
\pgfsetdash{}{0pt}%
\pgfpathmoveto{\pgfqpoint{4.078968in}{1.037327in}}%
\pgfpathcurveto{\pgfqpoint{4.086101in}{1.037327in}}{\pgfqpoint{4.092942in}{1.040161in}}{\pgfqpoint{4.097986in}{1.045205in}}%
\pgfpathcurveto{\pgfqpoint{4.103030in}{1.050248in}}{\pgfqpoint{4.105864in}{1.057090in}}{\pgfqpoint{4.105864in}{1.064223in}}%
\pgfpathcurveto{\pgfqpoint{4.105864in}{1.071356in}}{\pgfqpoint{4.103030in}{1.078197in}}{\pgfqpoint{4.097986in}{1.083241in}}%
\pgfpathcurveto{\pgfqpoint{4.092942in}{1.088285in}}{\pgfqpoint{4.086101in}{1.091118in}}{\pgfqpoint{4.078968in}{1.091118in}}%
\pgfpathcurveto{\pgfqpoint{4.071835in}{1.091118in}}{\pgfqpoint{4.064994in}{1.088285in}}{\pgfqpoint{4.059950in}{1.083241in}}%
\pgfpathcurveto{\pgfqpoint{4.054906in}{1.078197in}}{\pgfqpoint{4.052072in}{1.071356in}}{\pgfqpoint{4.052072in}{1.064223in}}%
\pgfpathcurveto{\pgfqpoint{4.052072in}{1.057090in}}{\pgfqpoint{4.054906in}{1.050248in}}{\pgfqpoint{4.059950in}{1.045205in}}%
\pgfpathcurveto{\pgfqpoint{4.064994in}{1.040161in}}{\pgfqpoint{4.071835in}{1.037327in}}{\pgfqpoint{4.078968in}{1.037327in}}%
\pgfpathclose%
\pgfusepath{stroke,fill}%
\end{pgfscope}%
\begin{pgfscope}%
\pgfpathrectangle{\pgfqpoint{2.867647in}{0.500000in}}{\pgfqpoint{1.764706in}{1.700000in}}%
\pgfusepath{clip}%
\pgfsetbuttcap%
\pgfsetroundjoin%
\definecolor{currentfill}{rgb}{0.980678,0.914765,0.856766}%
\pgfsetfillcolor{currentfill}%
\pgfsetlinewidth{0.311001pt}%
\definecolor{currentstroke}{rgb}{1.000000,1.000000,1.000000}%
\pgfsetstrokecolor{currentstroke}%
\pgfsetdash{}{0pt}%
\pgfpathmoveto{\pgfqpoint{4.153678in}{1.497133in}}%
\pgfpathcurveto{\pgfqpoint{4.160811in}{1.497133in}}{\pgfqpoint{4.167653in}{1.499967in}}{\pgfqpoint{4.172696in}{1.505011in}}%
\pgfpathcurveto{\pgfqpoint{4.177740in}{1.510055in}}{\pgfqpoint{4.180574in}{1.516896in}}{\pgfqpoint{4.180574in}{1.524029in}}%
\pgfpathcurveto{\pgfqpoint{4.180574in}{1.531162in}}{\pgfqpoint{4.177740in}{1.538004in}}{\pgfqpoint{4.172696in}{1.543047in}}%
\pgfpathcurveto{\pgfqpoint{4.167653in}{1.548091in}}{\pgfqpoint{4.160811in}{1.550925in}}{\pgfqpoint{4.153678in}{1.550925in}}%
\pgfpathcurveto{\pgfqpoint{4.146545in}{1.550925in}}{\pgfqpoint{4.139704in}{1.548091in}}{\pgfqpoint{4.134660in}{1.543047in}}%
\pgfpathcurveto{\pgfqpoint{4.129616in}{1.538004in}}{\pgfqpoint{4.126782in}{1.531162in}}{\pgfqpoint{4.126782in}{1.524029in}}%
\pgfpathcurveto{\pgfqpoint{4.126782in}{1.516896in}}{\pgfqpoint{4.129616in}{1.510055in}}{\pgfqpoint{4.134660in}{1.505011in}}%
\pgfpathcurveto{\pgfqpoint{4.139704in}{1.499967in}}{\pgfqpoint{4.146545in}{1.497133in}}{\pgfqpoint{4.153678in}{1.497133in}}%
\pgfpathclose%
\pgfusepath{stroke,fill}%
\end{pgfscope}%
\begin{pgfscope}%
\pgfpathrectangle{\pgfqpoint{2.867647in}{0.500000in}}{\pgfqpoint{1.764706in}{1.700000in}}%
\pgfusepath{clip}%
\pgfsetbuttcap%
\pgfsetroundjoin%
\definecolor{currentfill}{rgb}{0.976287,0.879862,0.805788}%
\pgfsetfillcolor{currentfill}%
\pgfsetlinewidth{0.311001pt}%
\definecolor{currentstroke}{rgb}{1.000000,1.000000,1.000000}%
\pgfsetstrokecolor{currentstroke}%
\pgfsetdash{}{0pt}%
\pgfpathmoveto{\pgfqpoint{4.219867in}{1.473654in}}%
\pgfpathcurveto{\pgfqpoint{4.227000in}{1.473654in}}{\pgfqpoint{4.233842in}{1.476488in}}{\pgfqpoint{4.238885in}{1.481532in}}%
\pgfpathcurveto{\pgfqpoint{4.243929in}{1.486575in}}{\pgfqpoint{4.246763in}{1.493417in}}{\pgfqpoint{4.246763in}{1.500550in}}%
\pgfpathcurveto{\pgfqpoint{4.246763in}{1.507683in}}{\pgfqpoint{4.243929in}{1.514524in}}{\pgfqpoint{4.238885in}{1.519568in}}%
\pgfpathcurveto{\pgfqpoint{4.233842in}{1.524612in}}{\pgfqpoint{4.227000in}{1.527446in}}{\pgfqpoint{4.219867in}{1.527446in}}%
\pgfpathcurveto{\pgfqpoint{4.212734in}{1.527446in}}{\pgfqpoint{4.205893in}{1.524612in}}{\pgfqpoint{4.200849in}{1.519568in}}%
\pgfpathcurveto{\pgfqpoint{4.195806in}{1.514524in}}{\pgfqpoint{4.192972in}{1.507683in}}{\pgfqpoint{4.192972in}{1.500550in}}%
\pgfpathcurveto{\pgfqpoint{4.192972in}{1.493417in}}{\pgfqpoint{4.195806in}{1.486575in}}{\pgfqpoint{4.200849in}{1.481532in}}%
\pgfpathcurveto{\pgfqpoint{4.205893in}{1.476488in}}{\pgfqpoint{4.212734in}{1.473654in}}{\pgfqpoint{4.219867in}{1.473654in}}%
\pgfpathclose%
\pgfusepath{stroke,fill}%
\end{pgfscope}%
\begin{pgfscope}%
\pgfpathrectangle{\pgfqpoint{2.867647in}{0.500000in}}{\pgfqpoint{1.764706in}{1.700000in}}%
\pgfusepath{clip}%
\pgfsetbuttcap%
\pgfsetroundjoin%
\definecolor{currentfill}{rgb}{0.970718,0.821518,0.719872}%
\pgfsetfillcolor{currentfill}%
\pgfsetlinewidth{0.311001pt}%
\definecolor{currentstroke}{rgb}{1.000000,1.000000,1.000000}%
\pgfsetstrokecolor{currentstroke}%
\pgfsetdash{}{0pt}%
\pgfpathmoveto{\pgfqpoint{4.212543in}{1.584986in}}%
\pgfpathcurveto{\pgfqpoint{4.219676in}{1.584986in}}{\pgfqpoint{4.226518in}{1.587820in}}{\pgfqpoint{4.231562in}{1.592864in}}%
\pgfpathcurveto{\pgfqpoint{4.236605in}{1.597907in}}{\pgfqpoint{4.239439in}{1.604749in}}{\pgfqpoint{4.239439in}{1.611882in}}%
\pgfpathcurveto{\pgfqpoint{4.239439in}{1.619015in}}{\pgfqpoint{4.236605in}{1.625856in}}{\pgfqpoint{4.231562in}{1.630900in}}%
\pgfpathcurveto{\pgfqpoint{4.226518in}{1.635944in}}{\pgfqpoint{4.219676in}{1.638777in}}{\pgfqpoint{4.212543in}{1.638777in}}%
\pgfpathcurveto{\pgfqpoint{4.205411in}{1.638777in}}{\pgfqpoint{4.198569in}{1.635944in}}{\pgfqpoint{4.193525in}{1.630900in}}%
\pgfpathcurveto{\pgfqpoint{4.188482in}{1.625856in}}{\pgfqpoint{4.185648in}{1.619015in}}{\pgfqpoint{4.185648in}{1.611882in}}%
\pgfpathcurveto{\pgfqpoint{4.185648in}{1.604749in}}{\pgfqpoint{4.188482in}{1.597907in}}{\pgfqpoint{4.193525in}{1.592864in}}%
\pgfpathcurveto{\pgfqpoint{4.198569in}{1.587820in}}{\pgfqpoint{4.205411in}{1.584986in}}{\pgfqpoint{4.212543in}{1.584986in}}%
\pgfpathclose%
\pgfusepath{stroke,fill}%
\end{pgfscope}%
\begin{pgfscope}%
\pgfpathrectangle{\pgfqpoint{2.867647in}{0.500000in}}{\pgfqpoint{1.764706in}{1.700000in}}%
\pgfusepath{clip}%
\pgfsetbuttcap%
\pgfsetroundjoin%
\definecolor{currentfill}{rgb}{0.962985,0.612625,0.451451}%
\pgfsetfillcolor{currentfill}%
\pgfsetlinewidth{0.311001pt}%
\definecolor{currentstroke}{rgb}{1.000000,1.000000,1.000000}%
\pgfsetstrokecolor{currentstroke}%
\pgfsetdash{}{0pt}%
\pgfpathmoveto{\pgfqpoint{3.997391in}{1.792805in}}%
\pgfpathcurveto{\pgfqpoint{4.004524in}{1.792805in}}{\pgfqpoint{4.011365in}{1.795639in}}{\pgfqpoint{4.016409in}{1.800682in}}%
\pgfpathcurveto{\pgfqpoint{4.021453in}{1.805726in}}{\pgfqpoint{4.024287in}{1.812568in}}{\pgfqpoint{4.024287in}{1.819700in}}%
\pgfpathcurveto{\pgfqpoint{4.024287in}{1.826833in}}{\pgfqpoint{4.021453in}{1.833675in}}{\pgfqpoint{4.016409in}{1.838718in}}%
\pgfpathcurveto{\pgfqpoint{4.011365in}{1.843762in}}{\pgfqpoint{4.004524in}{1.846596in}}{\pgfqpoint{3.997391in}{1.846596in}}%
\pgfpathcurveto{\pgfqpoint{3.990258in}{1.846596in}}{\pgfqpoint{3.983416in}{1.843762in}}{\pgfqpoint{3.978373in}{1.838718in}}%
\pgfpathcurveto{\pgfqpoint{3.973329in}{1.833675in}}{\pgfqpoint{3.970495in}{1.826833in}}{\pgfqpoint{3.970495in}{1.819700in}}%
\pgfpathcurveto{\pgfqpoint{3.970495in}{1.812568in}}{\pgfqpoint{3.973329in}{1.805726in}}{\pgfqpoint{3.978373in}{1.800682in}}%
\pgfpathcurveto{\pgfqpoint{3.983416in}{1.795639in}}{\pgfqpoint{3.990258in}{1.792805in}}{\pgfqpoint{3.997391in}{1.792805in}}%
\pgfpathclose%
\pgfusepath{stroke,fill}%
\end{pgfscope}%
\begin{pgfscope}%
\pgfpathrectangle{\pgfqpoint{2.867647in}{0.500000in}}{\pgfqpoint{1.764706in}{1.700000in}}%
\pgfusepath{clip}%
\pgfsetbuttcap%
\pgfsetroundjoin%
\definecolor{currentfill}{rgb}{0.968931,0.798091,0.685123}%
\pgfsetfillcolor{currentfill}%
\pgfsetlinewidth{0.311001pt}%
\definecolor{currentstroke}{rgb}{1.000000,1.000000,1.000000}%
\pgfsetstrokecolor{currentstroke}%
\pgfsetdash{}{0pt}%
\pgfpathmoveto{\pgfqpoint{4.237678in}{1.550358in}}%
\pgfpathcurveto{\pgfqpoint{4.244811in}{1.550358in}}{\pgfqpoint{4.251653in}{1.553192in}}{\pgfqpoint{4.256696in}{1.558236in}}%
\pgfpathcurveto{\pgfqpoint{4.261740in}{1.563279in}}{\pgfqpoint{4.264574in}{1.570121in}}{\pgfqpoint{4.264574in}{1.577254in}}%
\pgfpathcurveto{\pgfqpoint{4.264574in}{1.584387in}}{\pgfqpoint{4.261740in}{1.591228in}}{\pgfqpoint{4.256696in}{1.596272in}}%
\pgfpathcurveto{\pgfqpoint{4.251653in}{1.601316in}}{\pgfqpoint{4.244811in}{1.604150in}}{\pgfqpoint{4.237678in}{1.604150in}}%
\pgfpathcurveto{\pgfqpoint{4.230545in}{1.604150in}}{\pgfqpoint{4.223704in}{1.601316in}}{\pgfqpoint{4.218660in}{1.596272in}}%
\pgfpathcurveto{\pgfqpoint{4.213616in}{1.591228in}}{\pgfqpoint{4.210783in}{1.584387in}}{\pgfqpoint{4.210783in}{1.577254in}}%
\pgfpathcurveto{\pgfqpoint{4.210783in}{1.570121in}}{\pgfqpoint{4.213616in}{1.563279in}}{\pgfqpoint{4.218660in}{1.558236in}}%
\pgfpathcurveto{\pgfqpoint{4.223704in}{1.553192in}}{\pgfqpoint{4.230545in}{1.550358in}}{\pgfqpoint{4.237678in}{1.550358in}}%
\pgfpathclose%
\pgfusepath{stroke,fill}%
\end{pgfscope}%
\begin{pgfscope}%
\pgfpathrectangle{\pgfqpoint{2.867647in}{0.500000in}}{\pgfqpoint{1.764706in}{1.700000in}}%
\pgfusepath{clip}%
\pgfsetbuttcap%
\pgfsetroundjoin%
\definecolor{currentfill}{rgb}{0.970718,0.821518,0.719872}%
\pgfsetfillcolor{currentfill}%
\pgfsetlinewidth{0.311001pt}%
\definecolor{currentstroke}{rgb}{1.000000,1.000000,1.000000}%
\pgfsetstrokecolor{currentstroke}%
\pgfsetdash{}{0pt}%
\pgfpathmoveto{\pgfqpoint{4.060939in}{1.075960in}}%
\pgfpathcurveto{\pgfqpoint{4.068072in}{1.075960in}}{\pgfqpoint{4.074914in}{1.078794in}}{\pgfqpoint{4.079958in}{1.083838in}}%
\pgfpathcurveto{\pgfqpoint{4.085001in}{1.088881in}}{\pgfqpoint{4.087835in}{1.095723in}}{\pgfqpoint{4.087835in}{1.102856in}}%
\pgfpathcurveto{\pgfqpoint{4.087835in}{1.109989in}}{\pgfqpoint{4.085001in}{1.116830in}}{\pgfqpoint{4.079958in}{1.121874in}}%
\pgfpathcurveto{\pgfqpoint{4.074914in}{1.126918in}}{\pgfqpoint{4.068072in}{1.129752in}}{\pgfqpoint{4.060939in}{1.129752in}}%
\pgfpathcurveto{\pgfqpoint{4.053807in}{1.129752in}}{\pgfqpoint{4.046965in}{1.126918in}}{\pgfqpoint{4.041921in}{1.121874in}}%
\pgfpathcurveto{\pgfqpoint{4.036878in}{1.116830in}}{\pgfqpoint{4.034044in}{1.109989in}}{\pgfqpoint{4.034044in}{1.102856in}}%
\pgfpathcurveto{\pgfqpoint{4.034044in}{1.095723in}}{\pgfqpoint{4.036878in}{1.088881in}}{\pgfqpoint{4.041921in}{1.083838in}}%
\pgfpathcurveto{\pgfqpoint{4.046965in}{1.078794in}}{\pgfqpoint{4.053807in}{1.075960in}}{\pgfqpoint{4.060939in}{1.075960in}}%
\pgfpathclose%
\pgfusepath{stroke,fill}%
\end{pgfscope}%
\begin{pgfscope}%
\pgfpathrectangle{\pgfqpoint{2.867647in}{0.500000in}}{\pgfqpoint{1.764706in}{1.700000in}}%
\pgfusepath{clip}%
\pgfsetbuttcap%
\pgfsetroundjoin%
\definecolor{currentfill}{rgb}{0.960778,0.559972,0.399412}%
\pgfsetfillcolor{currentfill}%
\pgfsetlinewidth{0.311001pt}%
\definecolor{currentstroke}{rgb}{1.000000,1.000000,1.000000}%
\pgfsetstrokecolor{currentstroke}%
\pgfsetdash{}{0pt}%
\pgfpathmoveto{\pgfqpoint{3.965556in}{1.055945in}}%
\pgfpathcurveto{\pgfqpoint{3.972689in}{1.055945in}}{\pgfqpoint{3.979531in}{1.058779in}}{\pgfqpoint{3.984574in}{1.063822in}}%
\pgfpathcurveto{\pgfqpoint{3.989618in}{1.068866in}}{\pgfqpoint{3.992452in}{1.075708in}}{\pgfqpoint{3.992452in}{1.082840in}}%
\pgfpathcurveto{\pgfqpoint{3.992452in}{1.089973in}}{\pgfqpoint{3.989618in}{1.096815in}}{\pgfqpoint{3.984574in}{1.101859in}}%
\pgfpathcurveto{\pgfqpoint{3.979531in}{1.106902in}}{\pgfqpoint{3.972689in}{1.109736in}}{\pgfqpoint{3.965556in}{1.109736in}}%
\pgfpathcurveto{\pgfqpoint{3.958423in}{1.109736in}}{\pgfqpoint{3.951582in}{1.106902in}}{\pgfqpoint{3.946538in}{1.101859in}}%
\pgfpathcurveto{\pgfqpoint{3.941494in}{1.096815in}}{\pgfqpoint{3.938661in}{1.089973in}}{\pgfqpoint{3.938661in}{1.082840in}}%
\pgfpathcurveto{\pgfqpoint{3.938661in}{1.075708in}}{\pgfqpoint{3.941494in}{1.068866in}}{\pgfqpoint{3.946538in}{1.063822in}}%
\pgfpathcurveto{\pgfqpoint{3.951582in}{1.058779in}}{\pgfqpoint{3.958423in}{1.055945in}}{\pgfqpoint{3.965556in}{1.055945in}}%
\pgfpathclose%
\pgfusepath{stroke,fill}%
\end{pgfscope}%
\begin{pgfscope}%
\pgfpathrectangle{\pgfqpoint{2.867647in}{0.500000in}}{\pgfqpoint{1.764706in}{1.700000in}}%
\pgfusepath{clip}%
\pgfsetbuttcap%
\pgfsetroundjoin%
\definecolor{currentfill}{rgb}{0.979124,0.903132,0.839793}%
\pgfsetfillcolor{currentfill}%
\pgfsetlinewidth{0.311001pt}%
\definecolor{currentstroke}{rgb}{1.000000,1.000000,1.000000}%
\pgfsetstrokecolor{currentstroke}%
\pgfsetdash{}{0pt}%
\pgfpathmoveto{\pgfqpoint{4.214507in}{1.203695in}}%
\pgfpathcurveto{\pgfqpoint{4.221640in}{1.203695in}}{\pgfqpoint{4.228481in}{1.206528in}}{\pgfqpoint{4.233525in}{1.211572in}}%
\pgfpathcurveto{\pgfqpoint{4.238569in}{1.216616in}}{\pgfqpoint{4.241402in}{1.223457in}}{\pgfqpoint{4.241402in}{1.230590in}}%
\pgfpathcurveto{\pgfqpoint{4.241402in}{1.237723in}}{\pgfqpoint{4.238569in}{1.244565in}}{\pgfqpoint{4.233525in}{1.249608in}}%
\pgfpathcurveto{\pgfqpoint{4.228481in}{1.254652in}}{\pgfqpoint{4.221640in}{1.257486in}}{\pgfqpoint{4.214507in}{1.257486in}}%
\pgfpathcurveto{\pgfqpoint{4.207374in}{1.257486in}}{\pgfqpoint{4.200532in}{1.254652in}}{\pgfqpoint{4.195489in}{1.249608in}}%
\pgfpathcurveto{\pgfqpoint{4.190445in}{1.244565in}}{\pgfqpoint{4.187611in}{1.237723in}}{\pgfqpoint{4.187611in}{1.230590in}}%
\pgfpathcurveto{\pgfqpoint{4.187611in}{1.223457in}}{\pgfqpoint{4.190445in}{1.216616in}}{\pgfqpoint{4.195489in}{1.211572in}}%
\pgfpathcurveto{\pgfqpoint{4.200532in}{1.206528in}}{\pgfqpoint{4.207374in}{1.203695in}}{\pgfqpoint{4.214507in}{1.203695in}}%
\pgfpathclose%
\pgfusepath{stroke,fill}%
\end{pgfscope}%
\begin{pgfscope}%
\pgfpathrectangle{\pgfqpoint{2.867647in}{0.500000in}}{\pgfqpoint{1.764706in}{1.700000in}}%
\pgfusepath{clip}%
\pgfsetbuttcap%
\pgfsetroundjoin%
\definecolor{currentfill}{rgb}{0.966812,0.762584,0.633643}%
\pgfsetfillcolor{currentfill}%
\pgfsetlinewidth{0.311001pt}%
\definecolor{currentstroke}{rgb}{1.000000,1.000000,1.000000}%
\pgfsetstrokecolor{currentstroke}%
\pgfsetdash{}{0pt}%
\pgfpathmoveto{\pgfqpoint{4.252301in}{1.549505in}}%
\pgfpathcurveto{\pgfqpoint{4.259434in}{1.549505in}}{\pgfqpoint{4.266275in}{1.552339in}}{\pgfqpoint{4.271319in}{1.557383in}}%
\pgfpathcurveto{\pgfqpoint{4.276363in}{1.562426in}}{\pgfqpoint{4.279197in}{1.569268in}}{\pgfqpoint{4.279197in}{1.576401in}}%
\pgfpathcurveto{\pgfqpoint{4.279197in}{1.583534in}}{\pgfqpoint{4.276363in}{1.590375in}}{\pgfqpoint{4.271319in}{1.595419in}}%
\pgfpathcurveto{\pgfqpoint{4.266275in}{1.600463in}}{\pgfqpoint{4.259434in}{1.603297in}}{\pgfqpoint{4.252301in}{1.603297in}}%
\pgfpathcurveto{\pgfqpoint{4.245168in}{1.603297in}}{\pgfqpoint{4.238326in}{1.600463in}}{\pgfqpoint{4.233283in}{1.595419in}}%
\pgfpathcurveto{\pgfqpoint{4.228239in}{1.590375in}}{\pgfqpoint{4.225405in}{1.583534in}}{\pgfqpoint{4.225405in}{1.576401in}}%
\pgfpathcurveto{\pgfqpoint{4.225405in}{1.569268in}}{\pgfqpoint{4.228239in}{1.562426in}}{\pgfqpoint{4.233283in}{1.557383in}}%
\pgfpathcurveto{\pgfqpoint{4.238326in}{1.552339in}}{\pgfqpoint{4.245168in}{1.549505in}}{\pgfqpoint{4.252301in}{1.549505in}}%
\pgfpathclose%
\pgfusepath{stroke,fill}%
\end{pgfscope}%
\begin{pgfscope}%
\pgfpathrectangle{\pgfqpoint{2.867647in}{0.500000in}}{\pgfqpoint{1.764706in}{1.700000in}}%
\pgfusepath{clip}%
\pgfsetbuttcap%
\pgfsetroundjoin%
\definecolor{currentfill}{rgb}{0.965302,0.713942,0.568499}%
\pgfsetfillcolor{currentfill}%
\pgfsetlinewidth{0.311001pt}%
\definecolor{currentstroke}{rgb}{1.000000,1.000000,1.000000}%
\pgfsetstrokecolor{currentstroke}%
\pgfsetdash{}{0pt}%
\pgfpathmoveto{\pgfqpoint{4.001873in}{0.926317in}}%
\pgfpathcurveto{\pgfqpoint{4.009006in}{0.926317in}}{\pgfqpoint{4.015847in}{0.929151in}}{\pgfqpoint{4.020891in}{0.934195in}}%
\pgfpathcurveto{\pgfqpoint{4.025935in}{0.939239in}}{\pgfqpoint{4.028768in}{0.946080in}}{\pgfqpoint{4.028768in}{0.953213in}}%
\pgfpathcurveto{\pgfqpoint{4.028768in}{0.960346in}}{\pgfqpoint{4.025935in}{0.967188in}}{\pgfqpoint{4.020891in}{0.972231in}}%
\pgfpathcurveto{\pgfqpoint{4.015847in}{0.977275in}}{\pgfqpoint{4.009006in}{0.980109in}}{\pgfqpoint{4.001873in}{0.980109in}}%
\pgfpathcurveto{\pgfqpoint{3.994740in}{0.980109in}}{\pgfqpoint{3.987898in}{0.977275in}}{\pgfqpoint{3.982855in}{0.972231in}}%
\pgfpathcurveto{\pgfqpoint{3.977811in}{0.967188in}}{\pgfqpoint{3.974977in}{0.960346in}}{\pgfqpoint{3.974977in}{0.953213in}}%
\pgfpathcurveto{\pgfqpoint{3.974977in}{0.946080in}}{\pgfqpoint{3.977811in}{0.939239in}}{\pgfqpoint{3.982855in}{0.934195in}}%
\pgfpathcurveto{\pgfqpoint{3.987898in}{0.929151in}}{\pgfqpoint{3.994740in}{0.926317in}}{\pgfqpoint{4.001873in}{0.926317in}}%
\pgfpathclose%
\pgfusepath{stroke,fill}%
\end{pgfscope}%
\begin{pgfscope}%
\pgfpathrectangle{\pgfqpoint{2.867647in}{0.500000in}}{\pgfqpoint{1.764706in}{1.700000in}}%
\pgfusepath{clip}%
\pgfsetbuttcap%
\pgfsetroundjoin%
\definecolor{currentfill}{rgb}{0.971694,0.833208,0.737161}%
\pgfsetfillcolor{currentfill}%
\pgfsetlinewidth{0.311001pt}%
\definecolor{currentstroke}{rgb}{1.000000,1.000000,1.000000}%
\pgfsetstrokecolor{currentstroke}%
\pgfsetdash{}{0pt}%
\pgfpathmoveto{\pgfqpoint{4.064808in}{0.987098in}}%
\pgfpathcurveto{\pgfqpoint{4.071941in}{0.987098in}}{\pgfqpoint{4.078782in}{0.989932in}}{\pgfqpoint{4.083826in}{0.994976in}}%
\pgfpathcurveto{\pgfqpoint{4.088870in}{1.000019in}}{\pgfqpoint{4.091704in}{1.006861in}}{\pgfqpoint{4.091704in}{1.013994in}}%
\pgfpathcurveto{\pgfqpoint{4.091704in}{1.021127in}}{\pgfqpoint{4.088870in}{1.027968in}}{\pgfqpoint{4.083826in}{1.033012in}}%
\pgfpathcurveto{\pgfqpoint{4.078782in}{1.038056in}}{\pgfqpoint{4.071941in}{1.040890in}}{\pgfqpoint{4.064808in}{1.040890in}}%
\pgfpathcurveto{\pgfqpoint{4.057675in}{1.040890in}}{\pgfqpoint{4.050833in}{1.038056in}}{\pgfqpoint{4.045790in}{1.033012in}}%
\pgfpathcurveto{\pgfqpoint{4.040746in}{1.027968in}}{\pgfqpoint{4.037912in}{1.021127in}}{\pgfqpoint{4.037912in}{1.013994in}}%
\pgfpathcurveto{\pgfqpoint{4.037912in}{1.006861in}}{\pgfqpoint{4.040746in}{1.000019in}}{\pgfqpoint{4.045790in}{0.994976in}}%
\pgfpathcurveto{\pgfqpoint{4.050833in}{0.989932in}}{\pgfqpoint{4.057675in}{0.987098in}}{\pgfqpoint{4.064808in}{0.987098in}}%
\pgfpathclose%
\pgfusepath{stroke,fill}%
\end{pgfscope}%
\begin{pgfscope}%
\pgfpathrectangle{\pgfqpoint{2.867647in}{0.500000in}}{\pgfqpoint{1.764706in}{1.700000in}}%
\pgfusepath{clip}%
\pgfsetbuttcap%
\pgfsetroundjoin%
\definecolor{currentfill}{rgb}{0.974412,0.862387,0.780156}%
\pgfsetfillcolor{currentfill}%
\pgfsetlinewidth{0.311001pt}%
\definecolor{currentstroke}{rgb}{1.000000,1.000000,1.000000}%
\pgfsetstrokecolor{currentstroke}%
\pgfsetdash{}{0pt}%
\pgfpathmoveto{\pgfqpoint{4.117685in}{1.430597in}}%
\pgfpathcurveto{\pgfqpoint{4.124818in}{1.430597in}}{\pgfqpoint{4.131660in}{1.433431in}}{\pgfqpoint{4.136703in}{1.438475in}}%
\pgfpathcurveto{\pgfqpoint{4.141747in}{1.443518in}}{\pgfqpoint{4.144581in}{1.450360in}}{\pgfqpoint{4.144581in}{1.457493in}}%
\pgfpathcurveto{\pgfqpoint{4.144581in}{1.464626in}}{\pgfqpoint{4.141747in}{1.471467in}}{\pgfqpoint{4.136703in}{1.476511in}}%
\pgfpathcurveto{\pgfqpoint{4.131660in}{1.481555in}}{\pgfqpoint{4.124818in}{1.484389in}}{\pgfqpoint{4.117685in}{1.484389in}}%
\pgfpathcurveto{\pgfqpoint{4.110552in}{1.484389in}}{\pgfqpoint{4.103711in}{1.481555in}}{\pgfqpoint{4.098667in}{1.476511in}}%
\pgfpathcurveto{\pgfqpoint{4.093623in}{1.471467in}}{\pgfqpoint{4.090789in}{1.464626in}}{\pgfqpoint{4.090789in}{1.457493in}}%
\pgfpathcurveto{\pgfqpoint{4.090789in}{1.450360in}}{\pgfqpoint{4.093623in}{1.443518in}}{\pgfqpoint{4.098667in}{1.438475in}}%
\pgfpathcurveto{\pgfqpoint{4.103711in}{1.433431in}}{\pgfqpoint{4.110552in}{1.430597in}}{\pgfqpoint{4.117685in}{1.430597in}}%
\pgfpathclose%
\pgfusepath{stroke,fill}%
\end{pgfscope}%
\begin{pgfscope}%
\pgfpathrectangle{\pgfqpoint{2.867647in}{0.500000in}}{\pgfqpoint{1.764706in}{1.700000in}}%
\pgfusepath{clip}%
\pgfsetbuttcap%
\pgfsetroundjoin%
\definecolor{currentfill}{rgb}{0.962532,0.599594,0.438051}%
\pgfsetfillcolor{currentfill}%
\pgfsetlinewidth{0.311001pt}%
\definecolor{currentstroke}{rgb}{1.000000,1.000000,1.000000}%
\pgfsetstrokecolor{currentstroke}%
\pgfsetdash{}{0pt}%
\pgfpathmoveto{\pgfqpoint{3.948791in}{0.919858in}}%
\pgfpathcurveto{\pgfqpoint{3.955924in}{0.919858in}}{\pgfqpoint{3.962766in}{0.922692in}}{\pgfqpoint{3.967809in}{0.927735in}}%
\pgfpathcurveto{\pgfqpoint{3.972853in}{0.932779in}}{\pgfqpoint{3.975687in}{0.939620in}}{\pgfqpoint{3.975687in}{0.946753in}}%
\pgfpathcurveto{\pgfqpoint{3.975687in}{0.953886in}}{\pgfqpoint{3.972853in}{0.960728in}}{\pgfqpoint{3.967809in}{0.965771in}}%
\pgfpathcurveto{\pgfqpoint{3.962766in}{0.970815in}}{\pgfqpoint{3.955924in}{0.973649in}}{\pgfqpoint{3.948791in}{0.973649in}}%
\pgfpathcurveto{\pgfqpoint{3.941658in}{0.973649in}}{\pgfqpoint{3.934817in}{0.970815in}}{\pgfqpoint{3.929773in}{0.965771in}}%
\pgfpathcurveto{\pgfqpoint{3.924729in}{0.960728in}}{\pgfqpoint{3.921895in}{0.953886in}}{\pgfqpoint{3.921895in}{0.946753in}}%
\pgfpathcurveto{\pgfqpoint{3.921895in}{0.939620in}}{\pgfqpoint{3.924729in}{0.932779in}}{\pgfqpoint{3.929773in}{0.927735in}}%
\pgfpathcurveto{\pgfqpoint{3.934817in}{0.922692in}}{\pgfqpoint{3.941658in}{0.919858in}}{\pgfqpoint{3.948791in}{0.919858in}}%
\pgfpathclose%
\pgfusepath{stroke,fill}%
\end{pgfscope}%
\begin{pgfscope}%
\pgfpathrectangle{\pgfqpoint{2.867647in}{0.500000in}}{\pgfqpoint{1.764706in}{1.700000in}}%
\pgfusepath{clip}%
\pgfsetbuttcap%
\pgfsetroundjoin%
\definecolor{currentfill}{rgb}{0.970255,0.815666,0.711203}%
\pgfsetfillcolor{currentfill}%
\pgfsetlinewidth{0.311001pt}%
\definecolor{currentstroke}{rgb}{1.000000,1.000000,1.000000}%
\pgfsetstrokecolor{currentstroke}%
\pgfsetdash{}{0pt}%
\pgfpathmoveto{\pgfqpoint{4.163659in}{1.654988in}}%
\pgfpathcurveto{\pgfqpoint{4.170792in}{1.654988in}}{\pgfqpoint{4.177634in}{1.657822in}}{\pgfqpoint{4.182677in}{1.662866in}}%
\pgfpathcurveto{\pgfqpoint{4.187721in}{1.667909in}}{\pgfqpoint{4.190555in}{1.674751in}}{\pgfqpoint{4.190555in}{1.681884in}}%
\pgfpathcurveto{\pgfqpoint{4.190555in}{1.689017in}}{\pgfqpoint{4.187721in}{1.695858in}}{\pgfqpoint{4.182677in}{1.700902in}}%
\pgfpathcurveto{\pgfqpoint{4.177634in}{1.705946in}}{\pgfqpoint{4.170792in}{1.708780in}}{\pgfqpoint{4.163659in}{1.708780in}}%
\pgfpathcurveto{\pgfqpoint{4.156526in}{1.708780in}}{\pgfqpoint{4.149685in}{1.705946in}}{\pgfqpoint{4.144641in}{1.700902in}}%
\pgfpathcurveto{\pgfqpoint{4.139597in}{1.695858in}}{\pgfqpoint{4.136763in}{1.689017in}}{\pgfqpoint{4.136763in}{1.681884in}}%
\pgfpathcurveto{\pgfqpoint{4.136763in}{1.674751in}}{\pgfqpoint{4.139597in}{1.667909in}}{\pgfqpoint{4.144641in}{1.662866in}}%
\pgfpathcurveto{\pgfqpoint{4.149685in}{1.657822in}}{\pgfqpoint{4.156526in}{1.654988in}}{\pgfqpoint{4.163659in}{1.654988in}}%
\pgfpathclose%
\pgfusepath{stroke,fill}%
\end{pgfscope}%
\begin{pgfscope}%
\pgfpathrectangle{\pgfqpoint{2.867647in}{0.500000in}}{\pgfqpoint{1.764706in}{1.700000in}}%
\pgfusepath{clip}%
\pgfsetbuttcap%
\pgfsetroundjoin%
\definecolor{currentfill}{rgb}{0.980678,0.914765,0.856766}%
\pgfsetfillcolor{currentfill}%
\pgfsetlinewidth{0.311001pt}%
\definecolor{currentstroke}{rgb}{1.000000,1.000000,1.000000}%
\pgfsetstrokecolor{currentstroke}%
\pgfsetdash{}{0pt}%
\pgfpathmoveto{\pgfqpoint{4.177829in}{1.161914in}}%
\pgfpathcurveto{\pgfqpoint{4.184962in}{1.161914in}}{\pgfqpoint{4.191803in}{1.164748in}}{\pgfqpoint{4.196847in}{1.169792in}}%
\pgfpathcurveto{\pgfqpoint{4.201891in}{1.174835in}}{\pgfqpoint{4.204725in}{1.181677in}}{\pgfqpoint{4.204725in}{1.188810in}}%
\pgfpathcurveto{\pgfqpoint{4.204725in}{1.195943in}}{\pgfqpoint{4.201891in}{1.202784in}}{\pgfqpoint{4.196847in}{1.207828in}}%
\pgfpathcurveto{\pgfqpoint{4.191803in}{1.212872in}}{\pgfqpoint{4.184962in}{1.215706in}}{\pgfqpoint{4.177829in}{1.215706in}}%
\pgfpathcurveto{\pgfqpoint{4.170696in}{1.215706in}}{\pgfqpoint{4.163855in}{1.212872in}}{\pgfqpoint{4.158811in}{1.207828in}}%
\pgfpathcurveto{\pgfqpoint{4.153767in}{1.202784in}}{\pgfqpoint{4.150933in}{1.195943in}}{\pgfqpoint{4.150933in}{1.188810in}}%
\pgfpathcurveto{\pgfqpoint{4.150933in}{1.181677in}}{\pgfqpoint{4.153767in}{1.174835in}}{\pgfqpoint{4.158811in}{1.169792in}}%
\pgfpathcurveto{\pgfqpoint{4.163855in}{1.164748in}}{\pgfqpoint{4.170696in}{1.161914in}}{\pgfqpoint{4.177829in}{1.161914in}}%
\pgfpathclose%
\pgfusepath{stroke,fill}%
\end{pgfscope}%
\begin{pgfscope}%
\pgfpathrectangle{\pgfqpoint{2.867647in}{0.500000in}}{\pgfqpoint{1.764706in}{1.700000in}}%
\pgfusepath{clip}%
\pgfsetbuttcap%
\pgfsetroundjoin%
\definecolor{currentfill}{rgb}{0.977657,0.891500,0.822809}%
\pgfsetfillcolor{currentfill}%
\pgfsetlinewidth{0.311001pt}%
\definecolor{currentstroke}{rgb}{1.000000,1.000000,1.000000}%
\pgfsetstrokecolor{currentstroke}%
\pgfsetdash{}{0pt}%
\pgfpathmoveto{\pgfqpoint{4.146872in}{1.055321in}}%
\pgfpathcurveto{\pgfqpoint{4.154005in}{1.055321in}}{\pgfqpoint{4.160846in}{1.058155in}}{\pgfqpoint{4.165890in}{1.063198in}}%
\pgfpathcurveto{\pgfqpoint{4.170934in}{1.068242in}}{\pgfqpoint{4.173768in}{1.075084in}}{\pgfqpoint{4.173768in}{1.082216in}}%
\pgfpathcurveto{\pgfqpoint{4.173768in}{1.089349in}}{\pgfqpoint{4.170934in}{1.096191in}}{\pgfqpoint{4.165890in}{1.101235in}}%
\pgfpathcurveto{\pgfqpoint{4.160846in}{1.106278in}}{\pgfqpoint{4.154005in}{1.109112in}}{\pgfqpoint{4.146872in}{1.109112in}}%
\pgfpathcurveto{\pgfqpoint{4.139739in}{1.109112in}}{\pgfqpoint{4.132897in}{1.106278in}}{\pgfqpoint{4.127854in}{1.101235in}}%
\pgfpathcurveto{\pgfqpoint{4.122810in}{1.096191in}}{\pgfqpoint{4.119976in}{1.089349in}}{\pgfqpoint{4.119976in}{1.082216in}}%
\pgfpathcurveto{\pgfqpoint{4.119976in}{1.075084in}}{\pgfqpoint{4.122810in}{1.068242in}}{\pgfqpoint{4.127854in}{1.063198in}}%
\pgfpathcurveto{\pgfqpoint{4.132897in}{1.058155in}}{\pgfqpoint{4.139739in}{1.055321in}}{\pgfqpoint{4.146872in}{1.055321in}}%
\pgfpathclose%
\pgfusepath{stroke,fill}%
\end{pgfscope}%
\begin{pgfscope}%
\pgfpathrectangle{\pgfqpoint{2.867647in}{0.500000in}}{\pgfqpoint{1.764706in}{1.700000in}}%
\pgfusepath{clip}%
\pgfsetbuttcap%
\pgfsetroundjoin%
\definecolor{currentfill}{rgb}{0.953816,0.463738,0.317699}%
\pgfsetfillcolor{currentfill}%
\pgfsetlinewidth{0.311001pt}%
\definecolor{currentstroke}{rgb}{1.000000,1.000000,1.000000}%
\pgfsetstrokecolor{currentstroke}%
\pgfsetdash{}{0pt}%
\pgfpathmoveto{\pgfqpoint{3.991373in}{1.164643in}}%
\pgfpathcurveto{\pgfqpoint{3.998506in}{1.164643in}}{\pgfqpoint{4.005347in}{1.167476in}}{\pgfqpoint{4.010391in}{1.172520in}}%
\pgfpathcurveto{\pgfqpoint{4.015435in}{1.177564in}}{\pgfqpoint{4.018269in}{1.184405in}}{\pgfqpoint{4.018269in}{1.191538in}}%
\pgfpathcurveto{\pgfqpoint{4.018269in}{1.198671in}}{\pgfqpoint{4.015435in}{1.205513in}}{\pgfqpoint{4.010391in}{1.210556in}}%
\pgfpathcurveto{\pgfqpoint{4.005347in}{1.215600in}}{\pgfqpoint{3.998506in}{1.218434in}}{\pgfqpoint{3.991373in}{1.218434in}}%
\pgfpathcurveto{\pgfqpoint{3.984240in}{1.218434in}}{\pgfqpoint{3.977398in}{1.215600in}}{\pgfqpoint{3.972355in}{1.210556in}}%
\pgfpathcurveto{\pgfqpoint{3.967311in}{1.205513in}}{\pgfqpoint{3.964477in}{1.198671in}}{\pgfqpoint{3.964477in}{1.191538in}}%
\pgfpathcurveto{\pgfqpoint{3.964477in}{1.184405in}}{\pgfqpoint{3.967311in}{1.177564in}}{\pgfqpoint{3.972355in}{1.172520in}}%
\pgfpathcurveto{\pgfqpoint{3.977398in}{1.167476in}}{\pgfqpoint{3.984240in}{1.164643in}}{\pgfqpoint{3.991373in}{1.164643in}}%
\pgfpathclose%
\pgfusepath{stroke,fill}%
\end{pgfscope}%
\begin{pgfscope}%
\pgfpathrectangle{\pgfqpoint{2.867647in}{0.500000in}}{\pgfqpoint{1.764706in}{1.700000in}}%
\pgfusepath{clip}%
\pgfsetbuttcap%
\pgfsetroundjoin%
\definecolor{currentfill}{rgb}{0.953816,0.463738,0.317699}%
\pgfsetfillcolor{currentfill}%
\pgfsetlinewidth{0.311001pt}%
\definecolor{currentstroke}{rgb}{1.000000,1.000000,1.000000}%
\pgfsetstrokecolor{currentstroke}%
\pgfsetdash{}{0pt}%
\pgfpathmoveto{\pgfqpoint{4.227271in}{0.918567in}}%
\pgfpathcurveto{\pgfqpoint{4.234404in}{0.918567in}}{\pgfqpoint{4.241246in}{0.921401in}}{\pgfqpoint{4.246290in}{0.926445in}}%
\pgfpathcurveto{\pgfqpoint{4.251333in}{0.931488in}}{\pgfqpoint{4.254167in}{0.938330in}}{\pgfqpoint{4.254167in}{0.945463in}}%
\pgfpathcurveto{\pgfqpoint{4.254167in}{0.952596in}}{\pgfqpoint{4.251333in}{0.959437in}}{\pgfqpoint{4.246290in}{0.964481in}}%
\pgfpathcurveto{\pgfqpoint{4.241246in}{0.969525in}}{\pgfqpoint{4.234404in}{0.972359in}}{\pgfqpoint{4.227271in}{0.972359in}}%
\pgfpathcurveto{\pgfqpoint{4.220139in}{0.972359in}}{\pgfqpoint{4.213297in}{0.969525in}}{\pgfqpoint{4.208253in}{0.964481in}}%
\pgfpathcurveto{\pgfqpoint{4.203210in}{0.959437in}}{\pgfqpoint{4.200376in}{0.952596in}}{\pgfqpoint{4.200376in}{0.945463in}}%
\pgfpathcurveto{\pgfqpoint{4.200376in}{0.938330in}}{\pgfqpoint{4.203210in}{0.931488in}}{\pgfqpoint{4.208253in}{0.926445in}}%
\pgfpathcurveto{\pgfqpoint{4.213297in}{0.921401in}}{\pgfqpoint{4.220139in}{0.918567in}}{\pgfqpoint{4.227271in}{0.918567in}}%
\pgfpathclose%
\pgfusepath{stroke,fill}%
\end{pgfscope}%
\begin{pgfscope}%
\pgfpathrectangle{\pgfqpoint{2.867647in}{0.500000in}}{\pgfqpoint{1.764706in}{1.700000in}}%
\pgfusepath{clip}%
\pgfsetbuttcap%
\pgfsetroundjoin%
\definecolor{currentfill}{rgb}{0.967735,0.780441,0.659127}%
\pgfsetfillcolor{currentfill}%
\pgfsetlinewidth{0.311001pt}%
\definecolor{currentstroke}{rgb}{1.000000,1.000000,1.000000}%
\pgfsetstrokecolor{currentstroke}%
\pgfsetdash{}{0pt}%
\pgfpathmoveto{\pgfqpoint{4.236986in}{1.569901in}}%
\pgfpathcurveto{\pgfqpoint{4.244119in}{1.569901in}}{\pgfqpoint{4.250960in}{1.572735in}}{\pgfqpoint{4.256004in}{1.577778in}}%
\pgfpathcurveto{\pgfqpoint{4.261048in}{1.582822in}}{\pgfqpoint{4.263882in}{1.589664in}}{\pgfqpoint{4.263882in}{1.596797in}}%
\pgfpathcurveto{\pgfqpoint{4.263882in}{1.603929in}}{\pgfqpoint{4.261048in}{1.610771in}}{\pgfqpoint{4.256004in}{1.615815in}}%
\pgfpathcurveto{\pgfqpoint{4.250960in}{1.620858in}}{\pgfqpoint{4.244119in}{1.623692in}}{\pgfqpoint{4.236986in}{1.623692in}}%
\pgfpathcurveto{\pgfqpoint{4.229853in}{1.623692in}}{\pgfqpoint{4.223011in}{1.620858in}}{\pgfqpoint{4.217968in}{1.615815in}}%
\pgfpathcurveto{\pgfqpoint{4.212924in}{1.610771in}}{\pgfqpoint{4.210090in}{1.603929in}}{\pgfqpoint{4.210090in}{1.596797in}}%
\pgfpathcurveto{\pgfqpoint{4.210090in}{1.589664in}}{\pgfqpoint{4.212924in}{1.582822in}}{\pgfqpoint{4.217968in}{1.577778in}}%
\pgfpathcurveto{\pgfqpoint{4.223011in}{1.572735in}}{\pgfqpoint{4.229853in}{1.569901in}}{\pgfqpoint{4.236986in}{1.569901in}}%
\pgfpathclose%
\pgfusepath{stroke,fill}%
\end{pgfscope}%
\begin{pgfscope}%
\pgfpathrectangle{\pgfqpoint{2.867647in}{0.500000in}}{\pgfqpoint{1.764706in}{1.700000in}}%
\pgfusepath{clip}%
\pgfsetbuttcap%
\pgfsetroundjoin%
\definecolor{currentfill}{rgb}{0.979891,0.908948,0.848279}%
\pgfsetfillcolor{currentfill}%
\pgfsetlinewidth{0.311001pt}%
\definecolor{currentstroke}{rgb}{1.000000,1.000000,1.000000}%
\pgfsetstrokecolor{currentstroke}%
\pgfsetdash{}{0pt}%
\pgfpathmoveto{\pgfqpoint{4.138895in}{1.139224in}}%
\pgfpathcurveto{\pgfqpoint{4.146028in}{1.139224in}}{\pgfqpoint{4.152869in}{1.142057in}}{\pgfqpoint{4.157913in}{1.147101in}}%
\pgfpathcurveto{\pgfqpoint{4.162957in}{1.152145in}}{\pgfqpoint{4.165791in}{1.158986in}}{\pgfqpoint{4.165791in}{1.166119in}}%
\pgfpathcurveto{\pgfqpoint{4.165791in}{1.173252in}}{\pgfqpoint{4.162957in}{1.180094in}}{\pgfqpoint{4.157913in}{1.185137in}}%
\pgfpathcurveto{\pgfqpoint{4.152869in}{1.190181in}}{\pgfqpoint{4.146028in}{1.193015in}}{\pgfqpoint{4.138895in}{1.193015in}}%
\pgfpathcurveto{\pgfqpoint{4.131762in}{1.193015in}}{\pgfqpoint{4.124920in}{1.190181in}}{\pgfqpoint{4.119877in}{1.185137in}}%
\pgfpathcurveto{\pgfqpoint{4.114833in}{1.180094in}}{\pgfqpoint{4.111999in}{1.173252in}}{\pgfqpoint{4.111999in}{1.166119in}}%
\pgfpathcurveto{\pgfqpoint{4.111999in}{1.158986in}}{\pgfqpoint{4.114833in}{1.152145in}}{\pgfqpoint{4.119877in}{1.147101in}}%
\pgfpathcurveto{\pgfqpoint{4.124920in}{1.142057in}}{\pgfqpoint{4.131762in}{1.139224in}}{\pgfqpoint{4.138895in}{1.139224in}}%
\pgfpathclose%
\pgfusepath{stroke,fill}%
\end{pgfscope}%
\begin{pgfscope}%
\pgfpathrectangle{\pgfqpoint{2.867647in}{0.500000in}}{\pgfqpoint{1.764706in}{1.700000in}}%
\pgfusepath{clip}%
\pgfsetbuttcap%
\pgfsetroundjoin%
\definecolor{currentfill}{rgb}{0.973271,0.850724,0.762998}%
\pgfsetfillcolor{currentfill}%
\pgfsetlinewidth{0.311001pt}%
\definecolor{currentstroke}{rgb}{1.000000,1.000000,1.000000}%
\pgfsetstrokecolor{currentstroke}%
\pgfsetdash{}{0pt}%
\pgfpathmoveto{\pgfqpoint{4.107146in}{1.193531in}}%
\pgfpathcurveto{\pgfqpoint{4.114279in}{1.193531in}}{\pgfqpoint{4.121121in}{1.196365in}}{\pgfqpoint{4.126164in}{1.201409in}}%
\pgfpathcurveto{\pgfqpoint{4.131208in}{1.206452in}}{\pgfqpoint{4.134042in}{1.213294in}}{\pgfqpoint{4.134042in}{1.220427in}}%
\pgfpathcurveto{\pgfqpoint{4.134042in}{1.227560in}}{\pgfqpoint{4.131208in}{1.234401in}}{\pgfqpoint{4.126164in}{1.239445in}}%
\pgfpathcurveto{\pgfqpoint{4.121121in}{1.244489in}}{\pgfqpoint{4.114279in}{1.247322in}}{\pgfqpoint{4.107146in}{1.247322in}}%
\pgfpathcurveto{\pgfqpoint{4.100013in}{1.247322in}}{\pgfqpoint{4.093172in}{1.244489in}}{\pgfqpoint{4.088128in}{1.239445in}}%
\pgfpathcurveto{\pgfqpoint{4.083084in}{1.234401in}}{\pgfqpoint{4.080251in}{1.227560in}}{\pgfqpoint{4.080251in}{1.220427in}}%
\pgfpathcurveto{\pgfqpoint{4.080251in}{1.213294in}}{\pgfqpoint{4.083084in}{1.206452in}}{\pgfqpoint{4.088128in}{1.201409in}}%
\pgfpathcurveto{\pgfqpoint{4.093172in}{1.196365in}}{\pgfqpoint{4.100013in}{1.193531in}}{\pgfqpoint{4.107146in}{1.193531in}}%
\pgfpathclose%
\pgfusepath{stroke,fill}%
\end{pgfscope}%
\begin{pgfscope}%
\pgfpathrectangle{\pgfqpoint{2.867647in}{0.500000in}}{\pgfqpoint{1.764706in}{1.700000in}}%
\pgfusepath{clip}%
\pgfsetbuttcap%
\pgfsetroundjoin%
\definecolor{currentfill}{rgb}{0.979891,0.908948,0.848279}%
\pgfsetfillcolor{currentfill}%
\pgfsetlinewidth{0.311001pt}%
\definecolor{currentstroke}{rgb}{1.000000,1.000000,1.000000}%
\pgfsetstrokecolor{currentstroke}%
\pgfsetdash{}{0pt}%
\pgfpathmoveto{\pgfqpoint{4.220020in}{1.279385in}}%
\pgfpathcurveto{\pgfqpoint{4.227153in}{1.279385in}}{\pgfqpoint{4.233995in}{1.282218in}}{\pgfqpoint{4.239038in}{1.287262in}}%
\pgfpathcurveto{\pgfqpoint{4.244082in}{1.292306in}}{\pgfqpoint{4.246916in}{1.299147in}}{\pgfqpoint{4.246916in}{1.306280in}}%
\pgfpathcurveto{\pgfqpoint{4.246916in}{1.313413in}}{\pgfqpoint{4.244082in}{1.320255in}}{\pgfqpoint{4.239038in}{1.325298in}}%
\pgfpathcurveto{\pgfqpoint{4.233995in}{1.330342in}}{\pgfqpoint{4.227153in}{1.333176in}}{\pgfqpoint{4.220020in}{1.333176in}}%
\pgfpathcurveto{\pgfqpoint{4.212887in}{1.333176in}}{\pgfqpoint{4.206046in}{1.330342in}}{\pgfqpoint{4.201002in}{1.325298in}}%
\pgfpathcurveto{\pgfqpoint{4.195959in}{1.320255in}}{\pgfqpoint{4.193125in}{1.313413in}}{\pgfqpoint{4.193125in}{1.306280in}}%
\pgfpathcurveto{\pgfqpoint{4.193125in}{1.299147in}}{\pgfqpoint{4.195959in}{1.292306in}}{\pgfqpoint{4.201002in}{1.287262in}}%
\pgfpathcurveto{\pgfqpoint{4.206046in}{1.282218in}}{\pgfqpoint{4.212887in}{1.279385in}}{\pgfqpoint{4.220020in}{1.279385in}}%
\pgfpathclose%
\pgfusepath{stroke,fill}%
\end{pgfscope}%
\begin{pgfscope}%
\pgfpathrectangle{\pgfqpoint{2.867647in}{0.500000in}}{\pgfqpoint{1.764706in}{1.700000in}}%
\pgfusepath{clip}%
\pgfsetbuttcap%
\pgfsetroundjoin%
\definecolor{currentfill}{rgb}{0.965592,0.726236,0.584384}%
\pgfsetfillcolor{currentfill}%
\pgfsetlinewidth{0.311001pt}%
\definecolor{currentstroke}{rgb}{1.000000,1.000000,1.000000}%
\pgfsetstrokecolor{currentstroke}%
\pgfsetdash{}{0pt}%
\pgfpathmoveto{\pgfqpoint{4.028622in}{1.742040in}}%
\pgfpathcurveto{\pgfqpoint{4.035755in}{1.742040in}}{\pgfqpoint{4.042597in}{1.744874in}}{\pgfqpoint{4.047641in}{1.749917in}}%
\pgfpathcurveto{\pgfqpoint{4.052684in}{1.754961in}}{\pgfqpoint{4.055518in}{1.761803in}}{\pgfqpoint{4.055518in}{1.768935in}}%
\pgfpathcurveto{\pgfqpoint{4.055518in}{1.776068in}}{\pgfqpoint{4.052684in}{1.782910in}}{\pgfqpoint{4.047641in}{1.787954in}}%
\pgfpathcurveto{\pgfqpoint{4.042597in}{1.792997in}}{\pgfqpoint{4.035755in}{1.795831in}}{\pgfqpoint{4.028622in}{1.795831in}}%
\pgfpathcurveto{\pgfqpoint{4.021490in}{1.795831in}}{\pgfqpoint{4.014648in}{1.792997in}}{\pgfqpoint{4.009604in}{1.787954in}}%
\pgfpathcurveto{\pgfqpoint{4.004561in}{1.782910in}}{\pgfqpoint{4.001727in}{1.776068in}}{\pgfqpoint{4.001727in}{1.768935in}}%
\pgfpathcurveto{\pgfqpoint{4.001727in}{1.761803in}}{\pgfqpoint{4.004561in}{1.754961in}}{\pgfqpoint{4.009604in}{1.749917in}}%
\pgfpathcurveto{\pgfqpoint{4.014648in}{1.744874in}}{\pgfqpoint{4.021490in}{1.742040in}}{\pgfqpoint{4.028622in}{1.742040in}}%
\pgfpathclose%
\pgfusepath{stroke,fill}%
\end{pgfscope}%
\begin{pgfscope}%
\pgfpathrectangle{\pgfqpoint{2.867647in}{0.500000in}}{\pgfqpoint{1.764706in}{1.700000in}}%
\pgfusepath{clip}%
\pgfsetbuttcap%
\pgfsetroundjoin%
\definecolor{currentfill}{rgb}{0.964558,0.676556,0.522514}%
\pgfsetfillcolor{currentfill}%
\pgfsetlinewidth{0.311001pt}%
\definecolor{currentstroke}{rgb}{1.000000,1.000000,1.000000}%
\pgfsetstrokecolor{currentstroke}%
\pgfsetdash{}{0pt}%
\pgfpathmoveto{\pgfqpoint{4.128845in}{1.751645in}}%
\pgfpathcurveto{\pgfqpoint{4.135978in}{1.751645in}}{\pgfqpoint{4.142820in}{1.754479in}}{\pgfqpoint{4.147863in}{1.759523in}}%
\pgfpathcurveto{\pgfqpoint{4.152907in}{1.764567in}}{\pgfqpoint{4.155741in}{1.771408in}}{\pgfqpoint{4.155741in}{1.778541in}}%
\pgfpathcurveto{\pgfqpoint{4.155741in}{1.785674in}}{\pgfqpoint{4.152907in}{1.792516in}}{\pgfqpoint{4.147863in}{1.797559in}}%
\pgfpathcurveto{\pgfqpoint{4.142820in}{1.802603in}}{\pgfqpoint{4.135978in}{1.805437in}}{\pgfqpoint{4.128845in}{1.805437in}}%
\pgfpathcurveto{\pgfqpoint{4.121712in}{1.805437in}}{\pgfqpoint{4.114871in}{1.802603in}}{\pgfqpoint{4.109827in}{1.797559in}}%
\pgfpathcurveto{\pgfqpoint{4.104784in}{1.792516in}}{\pgfqpoint{4.101950in}{1.785674in}}{\pgfqpoint{4.101950in}{1.778541in}}%
\pgfpathcurveto{\pgfqpoint{4.101950in}{1.771408in}}{\pgfqpoint{4.104784in}{1.764567in}}{\pgfqpoint{4.109827in}{1.759523in}}%
\pgfpathcurveto{\pgfqpoint{4.114871in}{1.754479in}}{\pgfqpoint{4.121712in}{1.751645in}}{\pgfqpoint{4.128845in}{1.751645in}}%
\pgfpathclose%
\pgfusepath{stroke,fill}%
\end{pgfscope}%
\begin{pgfscope}%
\pgfpathrectangle{\pgfqpoint{2.867647in}{0.500000in}}{\pgfqpoint{1.764706in}{1.700000in}}%
\pgfusepath{clip}%
\pgfsetbuttcap%
\pgfsetroundjoin%
\definecolor{currentfill}{rgb}{0.152013,0.081591,0.202705}%
\pgfsetfillcolor{currentfill}%
\pgfsetlinewidth{0.803000pt}%
\definecolor{currentstroke}{rgb}{0.152013,0.081591,0.202705}%
\pgfsetstrokecolor{currentstroke}%
\pgfsetdash{}{0pt}%
\pgfsys@defobject{currentmarker}{\pgfqpoint{-0.033333in}{-0.033333in}}{\pgfqpoint{0.033333in}{0.033333in}}{%
\pgfpathmoveto{\pgfqpoint{0.000000in}{-0.033333in}}%
\pgfpathcurveto{\pgfqpoint{0.008840in}{-0.033333in}}{\pgfqpoint{0.017319in}{-0.029821in}}{\pgfqpoint{0.023570in}{-0.023570in}}%
\pgfpathcurveto{\pgfqpoint{0.029821in}{-0.017319in}}{\pgfqpoint{0.033333in}{-0.008840in}}{\pgfqpoint{0.033333in}{0.000000in}}%
\pgfpathcurveto{\pgfqpoint{0.033333in}{0.008840in}}{\pgfqpoint{0.029821in}{0.017319in}}{\pgfqpoint{0.023570in}{0.023570in}}%
\pgfpathcurveto{\pgfqpoint{0.017319in}{0.029821in}}{\pgfqpoint{0.008840in}{0.033333in}}{\pgfqpoint{0.000000in}{0.033333in}}%
\pgfpathcurveto{\pgfqpoint{-0.008840in}{0.033333in}}{\pgfqpoint{-0.017319in}{0.029821in}}{\pgfqpoint{-0.023570in}{0.023570in}}%
\pgfpathcurveto{\pgfqpoint{-0.029821in}{0.017319in}}{\pgfqpoint{-0.033333in}{0.008840in}}{\pgfqpoint{-0.033333in}{0.000000in}}%
\pgfpathcurveto{\pgfqpoint{-0.033333in}{-0.008840in}}{\pgfqpoint{-0.029821in}{-0.017319in}}{\pgfqpoint{-0.023570in}{-0.023570in}}%
\pgfpathcurveto{\pgfqpoint{-0.017319in}{-0.029821in}}{\pgfqpoint{-0.008840in}{-0.033333in}}{\pgfqpoint{0.000000in}{-0.033333in}}%
\pgfpathclose%
\pgfusepath{stroke,fill}%
}%
\end{pgfscope}%
\begin{pgfscope}%
\pgfpathrectangle{\pgfqpoint{2.867647in}{0.500000in}}{\pgfqpoint{1.764706in}{1.700000in}}%
\pgfusepath{clip}%
\pgfsetbuttcap%
\pgfsetroundjoin%
\definecolor{currentfill}{rgb}{0.419253,0.121738,0.336404}%
\pgfsetfillcolor{currentfill}%
\pgfsetlinewidth{0.803000pt}%
\definecolor{currentstroke}{rgb}{0.419253,0.121738,0.336404}%
\pgfsetstrokecolor{currentstroke}%
\pgfsetdash{}{0pt}%
\pgfsys@defobject{currentmarker}{\pgfqpoint{-0.033333in}{-0.033333in}}{\pgfqpoint{0.033333in}{0.033333in}}{%
\pgfpathmoveto{\pgfqpoint{0.000000in}{-0.033333in}}%
\pgfpathcurveto{\pgfqpoint{0.008840in}{-0.033333in}}{\pgfqpoint{0.017319in}{-0.029821in}}{\pgfqpoint{0.023570in}{-0.023570in}}%
\pgfpathcurveto{\pgfqpoint{0.029821in}{-0.017319in}}{\pgfqpoint{0.033333in}{-0.008840in}}{\pgfqpoint{0.033333in}{0.000000in}}%
\pgfpathcurveto{\pgfqpoint{0.033333in}{0.008840in}}{\pgfqpoint{0.029821in}{0.017319in}}{\pgfqpoint{0.023570in}{0.023570in}}%
\pgfpathcurveto{\pgfqpoint{0.017319in}{0.029821in}}{\pgfqpoint{0.008840in}{0.033333in}}{\pgfqpoint{0.000000in}{0.033333in}}%
\pgfpathcurveto{\pgfqpoint{-0.008840in}{0.033333in}}{\pgfqpoint{-0.017319in}{0.029821in}}{\pgfqpoint{-0.023570in}{0.023570in}}%
\pgfpathcurveto{\pgfqpoint{-0.029821in}{0.017319in}}{\pgfqpoint{-0.033333in}{0.008840in}}{\pgfqpoint{-0.033333in}{0.000000in}}%
\pgfpathcurveto{\pgfqpoint{-0.033333in}{-0.008840in}}{\pgfqpoint{-0.029821in}{-0.017319in}}{\pgfqpoint{-0.023570in}{-0.023570in}}%
\pgfpathcurveto{\pgfqpoint{-0.017319in}{-0.029821in}}{\pgfqpoint{-0.008840in}{-0.033333in}}{\pgfqpoint{0.000000in}{-0.033333in}}%
\pgfpathclose%
\pgfusepath{stroke,fill}%
}%
\end{pgfscope}%
\begin{pgfscope}%
\pgfpathrectangle{\pgfqpoint{2.867647in}{0.500000in}}{\pgfqpoint{1.764706in}{1.700000in}}%
\pgfusepath{clip}%
\pgfsetbuttcap%
\pgfsetroundjoin%
\definecolor{currentfill}{rgb}{0.711081,0.087612,0.343064}%
\pgfsetfillcolor{currentfill}%
\pgfsetlinewidth{0.803000pt}%
\definecolor{currentstroke}{rgb}{0.711081,0.087612,0.343064}%
\pgfsetstrokecolor{currentstroke}%
\pgfsetdash{}{0pt}%
\pgfsys@defobject{currentmarker}{\pgfqpoint{-0.033333in}{-0.033333in}}{\pgfqpoint{0.033333in}{0.033333in}}{%
\pgfpathmoveto{\pgfqpoint{0.000000in}{-0.033333in}}%
\pgfpathcurveto{\pgfqpoint{0.008840in}{-0.033333in}}{\pgfqpoint{0.017319in}{-0.029821in}}{\pgfqpoint{0.023570in}{-0.023570in}}%
\pgfpathcurveto{\pgfqpoint{0.029821in}{-0.017319in}}{\pgfqpoint{0.033333in}{-0.008840in}}{\pgfqpoint{0.033333in}{0.000000in}}%
\pgfpathcurveto{\pgfqpoint{0.033333in}{0.008840in}}{\pgfqpoint{0.029821in}{0.017319in}}{\pgfqpoint{0.023570in}{0.023570in}}%
\pgfpathcurveto{\pgfqpoint{0.017319in}{0.029821in}}{\pgfqpoint{0.008840in}{0.033333in}}{\pgfqpoint{0.000000in}{0.033333in}}%
\pgfpathcurveto{\pgfqpoint{-0.008840in}{0.033333in}}{\pgfqpoint{-0.017319in}{0.029821in}}{\pgfqpoint{-0.023570in}{0.023570in}}%
\pgfpathcurveto{\pgfqpoint{-0.029821in}{0.017319in}}{\pgfqpoint{-0.033333in}{0.008840in}}{\pgfqpoint{-0.033333in}{0.000000in}}%
\pgfpathcurveto{\pgfqpoint{-0.033333in}{-0.008840in}}{\pgfqpoint{-0.029821in}{-0.017319in}}{\pgfqpoint{-0.023570in}{-0.023570in}}%
\pgfpathcurveto{\pgfqpoint{-0.017319in}{-0.029821in}}{\pgfqpoint{-0.008840in}{-0.033333in}}{\pgfqpoint{0.000000in}{-0.033333in}}%
\pgfpathclose%
\pgfusepath{stroke,fill}%
}%
\end{pgfscope}%
\begin{pgfscope}%
\pgfpathrectangle{\pgfqpoint{2.867647in}{0.500000in}}{\pgfqpoint{1.764706in}{1.700000in}}%
\pgfusepath{clip}%
\pgfsetbuttcap%
\pgfsetroundjoin%
\definecolor{currentfill}{rgb}{0.922239,0.282873,0.242296}%
\pgfsetfillcolor{currentfill}%
\pgfsetlinewidth{0.803000pt}%
\definecolor{currentstroke}{rgb}{0.922239,0.282873,0.242296}%
\pgfsetstrokecolor{currentstroke}%
\pgfsetdash{}{0pt}%
\pgfsys@defobject{currentmarker}{\pgfqpoint{-0.033333in}{-0.033333in}}{\pgfqpoint{0.033333in}{0.033333in}}{%
\pgfpathmoveto{\pgfqpoint{0.000000in}{-0.033333in}}%
\pgfpathcurveto{\pgfqpoint{0.008840in}{-0.033333in}}{\pgfqpoint{0.017319in}{-0.029821in}}{\pgfqpoint{0.023570in}{-0.023570in}}%
\pgfpathcurveto{\pgfqpoint{0.029821in}{-0.017319in}}{\pgfqpoint{0.033333in}{-0.008840in}}{\pgfqpoint{0.033333in}{0.000000in}}%
\pgfpathcurveto{\pgfqpoint{0.033333in}{0.008840in}}{\pgfqpoint{0.029821in}{0.017319in}}{\pgfqpoint{0.023570in}{0.023570in}}%
\pgfpathcurveto{\pgfqpoint{0.017319in}{0.029821in}}{\pgfqpoint{0.008840in}{0.033333in}}{\pgfqpoint{0.000000in}{0.033333in}}%
\pgfpathcurveto{\pgfqpoint{-0.008840in}{0.033333in}}{\pgfqpoint{-0.017319in}{0.029821in}}{\pgfqpoint{-0.023570in}{0.023570in}}%
\pgfpathcurveto{\pgfqpoint{-0.029821in}{0.017319in}}{\pgfqpoint{-0.033333in}{0.008840in}}{\pgfqpoint{-0.033333in}{0.000000in}}%
\pgfpathcurveto{\pgfqpoint{-0.033333in}{-0.008840in}}{\pgfqpoint{-0.029821in}{-0.017319in}}{\pgfqpoint{-0.023570in}{-0.023570in}}%
\pgfpathcurveto{\pgfqpoint{-0.017319in}{-0.029821in}}{\pgfqpoint{-0.008840in}{-0.033333in}}{\pgfqpoint{0.000000in}{-0.033333in}}%
\pgfpathclose%
\pgfusepath{stroke,fill}%
}%
\end{pgfscope}%
\begin{pgfscope}%
\pgfpathrectangle{\pgfqpoint{2.867647in}{0.500000in}}{\pgfqpoint{1.764706in}{1.700000in}}%
\pgfusepath{clip}%
\pgfsetbuttcap%
\pgfsetroundjoin%
\definecolor{currentfill}{rgb}{0.962532,0.599594,0.438051}%
\pgfsetfillcolor{currentfill}%
\pgfsetlinewidth{0.803000pt}%
\definecolor{currentstroke}{rgb}{0.962532,0.599594,0.438051}%
\pgfsetstrokecolor{currentstroke}%
\pgfsetdash{}{0pt}%
\pgfsys@defobject{currentmarker}{\pgfqpoint{-0.033333in}{-0.033333in}}{\pgfqpoint{0.033333in}{0.033333in}}{%
\pgfpathmoveto{\pgfqpoint{0.000000in}{-0.033333in}}%
\pgfpathcurveto{\pgfqpoint{0.008840in}{-0.033333in}}{\pgfqpoint{0.017319in}{-0.029821in}}{\pgfqpoint{0.023570in}{-0.023570in}}%
\pgfpathcurveto{\pgfqpoint{0.029821in}{-0.017319in}}{\pgfqpoint{0.033333in}{-0.008840in}}{\pgfqpoint{0.033333in}{0.000000in}}%
\pgfpathcurveto{\pgfqpoint{0.033333in}{0.008840in}}{\pgfqpoint{0.029821in}{0.017319in}}{\pgfqpoint{0.023570in}{0.023570in}}%
\pgfpathcurveto{\pgfqpoint{0.017319in}{0.029821in}}{\pgfqpoint{0.008840in}{0.033333in}}{\pgfqpoint{0.000000in}{0.033333in}}%
\pgfpathcurveto{\pgfqpoint{-0.008840in}{0.033333in}}{\pgfqpoint{-0.017319in}{0.029821in}}{\pgfqpoint{-0.023570in}{0.023570in}}%
\pgfpathcurveto{\pgfqpoint{-0.029821in}{0.017319in}}{\pgfqpoint{-0.033333in}{0.008840in}}{\pgfqpoint{-0.033333in}{0.000000in}}%
\pgfpathcurveto{\pgfqpoint{-0.033333in}{-0.008840in}}{\pgfqpoint{-0.029821in}{-0.017319in}}{\pgfqpoint{-0.023570in}{-0.023570in}}%
\pgfpathcurveto{\pgfqpoint{-0.017319in}{-0.029821in}}{\pgfqpoint{-0.008840in}{-0.033333in}}{\pgfqpoint{0.000000in}{-0.033333in}}%
\pgfpathclose%
\pgfusepath{stroke,fill}%
}%
\end{pgfscope}%
\begin{pgfscope}%
\pgfpathrectangle{\pgfqpoint{2.867647in}{0.500000in}}{\pgfqpoint{1.764706in}{1.700000in}}%
\pgfusepath{clip}%
\pgfsetbuttcap%
\pgfsetroundjoin%
\definecolor{currentfill}{rgb}{0.975644,0.874038,0.797253}%
\pgfsetfillcolor{currentfill}%
\pgfsetlinewidth{0.803000pt}%
\definecolor{currentstroke}{rgb}{0.975644,0.874038,0.797253}%
\pgfsetstrokecolor{currentstroke}%
\pgfsetdash{}{0pt}%
\pgfsys@defobject{currentmarker}{\pgfqpoint{-0.033333in}{-0.033333in}}{\pgfqpoint{0.033333in}{0.033333in}}{%
\pgfpathmoveto{\pgfqpoint{0.000000in}{-0.033333in}}%
\pgfpathcurveto{\pgfqpoint{0.008840in}{-0.033333in}}{\pgfqpoint{0.017319in}{-0.029821in}}{\pgfqpoint{0.023570in}{-0.023570in}}%
\pgfpathcurveto{\pgfqpoint{0.029821in}{-0.017319in}}{\pgfqpoint{0.033333in}{-0.008840in}}{\pgfqpoint{0.033333in}{0.000000in}}%
\pgfpathcurveto{\pgfqpoint{0.033333in}{0.008840in}}{\pgfqpoint{0.029821in}{0.017319in}}{\pgfqpoint{0.023570in}{0.023570in}}%
\pgfpathcurveto{\pgfqpoint{0.017319in}{0.029821in}}{\pgfqpoint{0.008840in}{0.033333in}}{\pgfqpoint{0.000000in}{0.033333in}}%
\pgfpathcurveto{\pgfqpoint{-0.008840in}{0.033333in}}{\pgfqpoint{-0.017319in}{0.029821in}}{\pgfqpoint{-0.023570in}{0.023570in}}%
\pgfpathcurveto{\pgfqpoint{-0.029821in}{0.017319in}}{\pgfqpoint{-0.033333in}{0.008840in}}{\pgfqpoint{-0.033333in}{0.000000in}}%
\pgfpathcurveto{\pgfqpoint{-0.033333in}{-0.008840in}}{\pgfqpoint{-0.029821in}{-0.017319in}}{\pgfqpoint{-0.023570in}{-0.023570in}}%
\pgfpathcurveto{\pgfqpoint{-0.017319in}{-0.029821in}}{\pgfqpoint{-0.008840in}{-0.033333in}}{\pgfqpoint{0.000000in}{-0.033333in}}%
\pgfpathclose%
\pgfusepath{stroke,fill}%
}%
\end{pgfscope}%
\begin{pgfscope}%
\pgfsetrectcap%
\pgfsetmiterjoin%
\pgfsetlinewidth{1.003750pt}%
\definecolor{currentstroke}{rgb}{0.150000,0.150000,0.150000}%
\pgfsetstrokecolor{currentstroke}%
\pgfsetdash{}{0pt}%
\pgfpathmoveto{\pgfqpoint{2.867647in}{0.500000in}}%
\pgfpathlineto{\pgfqpoint{2.867647in}{2.200000in}}%
\pgfusepath{stroke}%
\end{pgfscope}%
\begin{pgfscope}%
\pgfsetrectcap%
\pgfsetmiterjoin%
\pgfsetlinewidth{1.003750pt}%
\definecolor{currentstroke}{rgb}{0.150000,0.150000,0.150000}%
\pgfsetstrokecolor{currentstroke}%
\pgfsetdash{}{0pt}%
\pgfpathmoveto{\pgfqpoint{4.632353in}{0.500000in}}%
\pgfpathlineto{\pgfqpoint{4.632353in}{2.200000in}}%
\pgfusepath{stroke}%
\end{pgfscope}%
\begin{pgfscope}%
\pgfsetrectcap%
\pgfsetmiterjoin%
\pgfsetlinewidth{1.003750pt}%
\definecolor{currentstroke}{rgb}{0.150000,0.150000,0.150000}%
\pgfsetstrokecolor{currentstroke}%
\pgfsetdash{}{0pt}%
\pgfpathmoveto{\pgfqpoint{2.867647in}{0.500000in}}%
\pgfpathlineto{\pgfqpoint{4.632353in}{0.500000in}}%
\pgfusepath{stroke}%
\end{pgfscope}%
\begin{pgfscope}%
\pgfsetrectcap%
\pgfsetmiterjoin%
\pgfsetlinewidth{1.003750pt}%
\definecolor{currentstroke}{rgb}{0.150000,0.150000,0.150000}%
\pgfsetstrokecolor{currentstroke}%
\pgfsetdash{}{0pt}%
\pgfpathmoveto{\pgfqpoint{2.867647in}{2.200000in}}%
\pgfpathlineto{\pgfqpoint{4.632353in}{2.200000in}}%
\pgfusepath{stroke}%
\end{pgfscope}%
\begin{pgfscope}%
\definecolor{textcolor}{rgb}{0.150000,0.150000,0.150000}%
\pgfsetstrokecolor{textcolor}%
\pgfsetfillcolor{textcolor}%
\pgftext[x=3.750000in,y=2.283333in,,base]{\color{textcolor}\rmfamily\fontsize{9.600000}{11.520000}\selectfont Iteration 5000}%
\end{pgfscope}%
\begin{pgfscope}%
\pgfsetbuttcap%
\pgfsetmiterjoin%
\definecolor{currentfill}{rgb}{1.000000,1.000000,1.000000}%
\pgfsetfillcolor{currentfill}%
\pgfsetlinewidth{0.000000pt}%
\definecolor{currentstroke}{rgb}{0.000000,0.000000,0.000000}%
\pgfsetstrokecolor{currentstroke}%
\pgfsetstrokeopacity{0.000000}%
\pgfsetdash{}{0pt}%
\pgfpathmoveto{\pgfqpoint{4.985294in}{0.500000in}}%
\pgfpathlineto{\pgfqpoint{6.750000in}{0.500000in}}%
\pgfpathlineto{\pgfqpoint{6.750000in}{2.200000in}}%
\pgfpathlineto{\pgfqpoint{4.985294in}{2.200000in}}%
\pgfpathclose%
\pgfusepath{fill}%
\end{pgfscope}%
\begin{pgfscope}%
\pgfpathrectangle{\pgfqpoint{4.985294in}{0.500000in}}{\pgfqpoint{1.764706in}{1.700000in}}%
\pgfusepath{clip}%
\pgfsetbuttcap%
\pgfsetroundjoin%
\definecolor{currentfill}{rgb}{0.975018,0.868213,0.788710}%
\pgfsetfillcolor{currentfill}%
\pgfsetlinewidth{0.311001pt}%
\definecolor{currentstroke}{rgb}{1.000000,1.000000,1.000000}%
\pgfsetstrokecolor{currentstroke}%
\pgfsetdash{}{0pt}%
\pgfpathmoveto{\pgfqpoint{5.403671in}{1.150775in}}%
\pgfpathcurveto{\pgfqpoint{5.410804in}{1.150775in}}{\pgfqpoint{5.417645in}{1.153609in}}{\pgfqpoint{5.422689in}{1.158653in}}%
\pgfpathcurveto{\pgfqpoint{5.427733in}{1.163696in}}{\pgfqpoint{5.430567in}{1.170538in}}{\pgfqpoint{5.430567in}{1.177671in}}%
\pgfpathcurveto{\pgfqpoint{5.430567in}{1.184804in}}{\pgfqpoint{5.427733in}{1.191645in}}{\pgfqpoint{5.422689in}{1.196689in}}%
\pgfpathcurveto{\pgfqpoint{5.417645in}{1.201733in}}{\pgfqpoint{5.410804in}{1.204567in}}{\pgfqpoint{5.403671in}{1.204567in}}%
\pgfpathcurveto{\pgfqpoint{5.396538in}{1.204567in}}{\pgfqpoint{5.389696in}{1.201733in}}{\pgfqpoint{5.384653in}{1.196689in}}%
\pgfpathcurveto{\pgfqpoint{5.379609in}{1.191645in}}{\pgfqpoint{5.376775in}{1.184804in}}{\pgfqpoint{5.376775in}{1.177671in}}%
\pgfpathcurveto{\pgfqpoint{5.376775in}{1.170538in}}{\pgfqpoint{5.379609in}{1.163696in}}{\pgfqpoint{5.384653in}{1.158653in}}%
\pgfpathcurveto{\pgfqpoint{5.389696in}{1.153609in}}{\pgfqpoint{5.396538in}{1.150775in}}{\pgfqpoint{5.403671in}{1.150775in}}%
\pgfpathclose%
\pgfusepath{stroke,fill}%
\end{pgfscope}%
\begin{pgfscope}%
\pgfpathrectangle{\pgfqpoint{4.985294in}{0.500000in}}{\pgfqpoint{1.764706in}{1.700000in}}%
\pgfusepath{clip}%
\pgfsetbuttcap%
\pgfsetroundjoin%
\definecolor{currentfill}{rgb}{0.970255,0.815666,0.711203}%
\pgfsetfillcolor{currentfill}%
\pgfsetlinewidth{0.311001pt}%
\definecolor{currentstroke}{rgb}{1.000000,1.000000,1.000000}%
\pgfsetstrokecolor{currentstroke}%
\pgfsetdash{}{0pt}%
\pgfpathmoveto{\pgfqpoint{5.359327in}{1.404620in}}%
\pgfpathcurveto{\pgfqpoint{5.366460in}{1.404620in}}{\pgfqpoint{5.373301in}{1.407454in}}{\pgfqpoint{5.378345in}{1.412497in}}%
\pgfpathcurveto{\pgfqpoint{5.383389in}{1.417541in}}{\pgfqpoint{5.386222in}{1.424383in}}{\pgfqpoint{5.386222in}{1.431516in}}%
\pgfpathcurveto{\pgfqpoint{5.386222in}{1.438648in}}{\pgfqpoint{5.383389in}{1.445490in}}{\pgfqpoint{5.378345in}{1.450534in}}%
\pgfpathcurveto{\pgfqpoint{5.373301in}{1.455577in}}{\pgfqpoint{5.366460in}{1.458411in}}{\pgfqpoint{5.359327in}{1.458411in}}%
\pgfpathcurveto{\pgfqpoint{5.352194in}{1.458411in}}{\pgfqpoint{5.345352in}{1.455577in}}{\pgfqpoint{5.340309in}{1.450534in}}%
\pgfpathcurveto{\pgfqpoint{5.335265in}{1.445490in}}{\pgfqpoint{5.332431in}{1.438648in}}{\pgfqpoint{5.332431in}{1.431516in}}%
\pgfpathcurveto{\pgfqpoint{5.332431in}{1.424383in}}{\pgfqpoint{5.335265in}{1.417541in}}{\pgfqpoint{5.340309in}{1.412497in}}%
\pgfpathcurveto{\pgfqpoint{5.345352in}{1.407454in}}{\pgfqpoint{5.352194in}{1.404620in}}{\pgfqpoint{5.359327in}{1.404620in}}%
\pgfpathclose%
\pgfusepath{stroke,fill}%
\end{pgfscope}%
\begin{pgfscope}%
\pgfpathrectangle{\pgfqpoint{4.985294in}{0.500000in}}{\pgfqpoint{1.764706in}{1.700000in}}%
\pgfusepath{clip}%
\pgfsetbuttcap%
\pgfsetroundjoin%
\definecolor{currentfill}{rgb}{0.981377,0.920617,0.865369}%
\pgfsetfillcolor{currentfill}%
\pgfsetlinewidth{0.311001pt}%
\definecolor{currentstroke}{rgb}{1.000000,1.000000,1.000000}%
\pgfsetstrokecolor{currentstroke}%
\pgfsetdash{}{0pt}%
\pgfpathmoveto{\pgfqpoint{6.305872in}{1.282297in}}%
\pgfpathcurveto{\pgfqpoint{6.313005in}{1.282297in}}{\pgfqpoint{6.319847in}{1.285131in}}{\pgfqpoint{6.324891in}{1.290175in}}%
\pgfpathcurveto{\pgfqpoint{6.329934in}{1.295218in}}{\pgfqpoint{6.332768in}{1.302060in}}{\pgfqpoint{6.332768in}{1.309193in}}%
\pgfpathcurveto{\pgfqpoint{6.332768in}{1.316326in}}{\pgfqpoint{6.329934in}{1.323167in}}{\pgfqpoint{6.324891in}{1.328211in}}%
\pgfpathcurveto{\pgfqpoint{6.319847in}{1.333255in}}{\pgfqpoint{6.313005in}{1.336088in}}{\pgfqpoint{6.305872in}{1.336088in}}%
\pgfpathcurveto{\pgfqpoint{6.298740in}{1.336088in}}{\pgfqpoint{6.291898in}{1.333255in}}{\pgfqpoint{6.286854in}{1.328211in}}%
\pgfpathcurveto{\pgfqpoint{6.281811in}{1.323167in}}{\pgfqpoint{6.278977in}{1.316326in}}{\pgfqpoint{6.278977in}{1.309193in}}%
\pgfpathcurveto{\pgfqpoint{6.278977in}{1.302060in}}{\pgfqpoint{6.281811in}{1.295218in}}{\pgfqpoint{6.286854in}{1.290175in}}%
\pgfpathcurveto{\pgfqpoint{6.291898in}{1.285131in}}{\pgfqpoint{6.298740in}{1.282297in}}{\pgfqpoint{6.305872in}{1.282297in}}%
\pgfpathclose%
\pgfusepath{stroke,fill}%
\end{pgfscope}%
\begin{pgfscope}%
\pgfpathrectangle{\pgfqpoint{4.985294in}{0.500000in}}{\pgfqpoint{1.764706in}{1.700000in}}%
\pgfusepath{clip}%
\pgfsetbuttcap%
\pgfsetroundjoin%
\definecolor{currentfill}{rgb}{0.981377,0.920617,0.865369}%
\pgfsetfillcolor{currentfill}%
\pgfsetlinewidth{0.311001pt}%
\definecolor{currentstroke}{rgb}{1.000000,1.000000,1.000000}%
\pgfsetstrokecolor{currentstroke}%
\pgfsetdash{}{0pt}%
\pgfpathmoveto{\pgfqpoint{6.318422in}{1.324510in}}%
\pgfpathcurveto{\pgfqpoint{6.325555in}{1.324510in}}{\pgfqpoint{6.332397in}{1.327344in}}{\pgfqpoint{6.337440in}{1.332387in}}%
\pgfpathcurveto{\pgfqpoint{6.342484in}{1.337431in}}{\pgfqpoint{6.345318in}{1.344273in}}{\pgfqpoint{6.345318in}{1.351405in}}%
\pgfpathcurveto{\pgfqpoint{6.345318in}{1.358538in}}{\pgfqpoint{6.342484in}{1.365380in}}{\pgfqpoint{6.337440in}{1.370424in}}%
\pgfpathcurveto{\pgfqpoint{6.332397in}{1.375467in}}{\pgfqpoint{6.325555in}{1.378301in}}{\pgfqpoint{6.318422in}{1.378301in}}%
\pgfpathcurveto{\pgfqpoint{6.311289in}{1.378301in}}{\pgfqpoint{6.304448in}{1.375467in}}{\pgfqpoint{6.299404in}{1.370424in}}%
\pgfpathcurveto{\pgfqpoint{6.294360in}{1.365380in}}{\pgfqpoint{6.291527in}{1.358538in}}{\pgfqpoint{6.291527in}{1.351405in}}%
\pgfpathcurveto{\pgfqpoint{6.291527in}{1.344273in}}{\pgfqpoint{6.294360in}{1.337431in}}{\pgfqpoint{6.299404in}{1.332387in}}%
\pgfpathcurveto{\pgfqpoint{6.304448in}{1.327344in}}{\pgfqpoint{6.311289in}{1.324510in}}{\pgfqpoint{6.318422in}{1.324510in}}%
\pgfpathclose%
\pgfusepath{stroke,fill}%
\end{pgfscope}%
\begin{pgfscope}%
\pgfpathrectangle{\pgfqpoint{4.985294in}{0.500000in}}{\pgfqpoint{1.764706in}{1.700000in}}%
\pgfusepath{clip}%
\pgfsetbuttcap%
\pgfsetroundjoin%
\definecolor{currentfill}{rgb}{0.974412,0.862387,0.780156}%
\pgfsetfillcolor{currentfill}%
\pgfsetlinewidth{0.311001pt}%
\definecolor{currentstroke}{rgb}{1.000000,1.000000,1.000000}%
\pgfsetstrokecolor{currentstroke}%
\pgfsetdash{}{0pt}%
\pgfpathmoveto{\pgfqpoint{5.472169in}{1.378189in}}%
\pgfpathcurveto{\pgfqpoint{5.479301in}{1.378189in}}{\pgfqpoint{5.486143in}{1.381023in}}{\pgfqpoint{5.491187in}{1.386066in}}%
\pgfpathcurveto{\pgfqpoint{5.496230in}{1.391110in}}{\pgfqpoint{5.499064in}{1.397952in}}{\pgfqpoint{5.499064in}{1.405084in}}%
\pgfpathcurveto{\pgfqpoint{5.499064in}{1.412217in}}{\pgfqpoint{5.496230in}{1.419059in}}{\pgfqpoint{5.491187in}{1.424103in}}%
\pgfpathcurveto{\pgfqpoint{5.486143in}{1.429146in}}{\pgfqpoint{5.479301in}{1.431980in}}{\pgfqpoint{5.472169in}{1.431980in}}%
\pgfpathcurveto{\pgfqpoint{5.465036in}{1.431980in}}{\pgfqpoint{5.458194in}{1.429146in}}{\pgfqpoint{5.453150in}{1.424103in}}%
\pgfpathcurveto{\pgfqpoint{5.448107in}{1.419059in}}{\pgfqpoint{5.445273in}{1.412217in}}{\pgfqpoint{5.445273in}{1.405084in}}%
\pgfpathcurveto{\pgfqpoint{5.445273in}{1.397952in}}{\pgfqpoint{5.448107in}{1.391110in}}{\pgfqpoint{5.453150in}{1.386066in}}%
\pgfpathcurveto{\pgfqpoint{5.458194in}{1.381023in}}{\pgfqpoint{5.465036in}{1.378189in}}{\pgfqpoint{5.472169in}{1.378189in}}%
\pgfpathclose%
\pgfusepath{stroke,fill}%
\end{pgfscope}%
\begin{pgfscope}%
\pgfpathrectangle{\pgfqpoint{4.985294in}{0.500000in}}{\pgfqpoint{1.764706in}{1.700000in}}%
\pgfusepath{clip}%
\pgfsetbuttcap%
\pgfsetroundjoin%
\definecolor{currentfill}{rgb}{0.969359,0.803954,0.693832}%
\pgfsetfillcolor{currentfill}%
\pgfsetlinewidth{0.311001pt}%
\definecolor{currentstroke}{rgb}{1.000000,1.000000,1.000000}%
\pgfsetstrokecolor{currentstroke}%
\pgfsetdash{}{0pt}%
\pgfpathmoveto{\pgfqpoint{5.545355in}{1.048068in}}%
\pgfpathcurveto{\pgfqpoint{5.552487in}{1.048068in}}{\pgfqpoint{5.559329in}{1.050902in}}{\pgfqpoint{5.564373in}{1.055946in}}%
\pgfpathcurveto{\pgfqpoint{5.569416in}{1.060990in}}{\pgfqpoint{5.572250in}{1.067831in}}{\pgfqpoint{5.572250in}{1.074964in}}%
\pgfpathcurveto{\pgfqpoint{5.572250in}{1.082097in}}{\pgfqpoint{5.569416in}{1.088939in}}{\pgfqpoint{5.564373in}{1.093982in}}%
\pgfpathcurveto{\pgfqpoint{5.559329in}{1.099026in}}{\pgfqpoint{5.552487in}{1.101860in}}{\pgfqpoint{5.545355in}{1.101860in}}%
\pgfpathcurveto{\pgfqpoint{5.538222in}{1.101860in}}{\pgfqpoint{5.531380in}{1.099026in}}{\pgfqpoint{5.526336in}{1.093982in}}%
\pgfpathcurveto{\pgfqpoint{5.521293in}{1.088939in}}{\pgfqpoint{5.518459in}{1.082097in}}{\pgfqpoint{5.518459in}{1.074964in}}%
\pgfpathcurveto{\pgfqpoint{5.518459in}{1.067831in}}{\pgfqpoint{5.521293in}{1.060990in}}{\pgfqpoint{5.526336in}{1.055946in}}%
\pgfpathcurveto{\pgfqpoint{5.531380in}{1.050902in}}{\pgfqpoint{5.538222in}{1.048068in}}{\pgfqpoint{5.545355in}{1.048068in}}%
\pgfpathclose%
\pgfusepath{stroke,fill}%
\end{pgfscope}%
\begin{pgfscope}%
\pgfpathrectangle{\pgfqpoint{4.985294in}{0.500000in}}{\pgfqpoint{1.764706in}{1.700000in}}%
\pgfusepath{clip}%
\pgfsetbuttcap%
\pgfsetroundjoin%
\definecolor{currentfill}{rgb}{0.973832,0.856556,0.771584}%
\pgfsetfillcolor{currentfill}%
\pgfsetlinewidth{0.311001pt}%
\definecolor{currentstroke}{rgb}{1.000000,1.000000,1.000000}%
\pgfsetstrokecolor{currentstroke}%
\pgfsetdash{}{0pt}%
\pgfpathmoveto{\pgfqpoint{5.429727in}{1.080916in}}%
\pgfpathcurveto{\pgfqpoint{5.436860in}{1.080916in}}{\pgfqpoint{5.443702in}{1.083750in}}{\pgfqpoint{5.448745in}{1.088794in}}%
\pgfpathcurveto{\pgfqpoint{5.453789in}{1.093837in}}{\pgfqpoint{5.456623in}{1.100679in}}{\pgfqpoint{5.456623in}{1.107812in}}%
\pgfpathcurveto{\pgfqpoint{5.456623in}{1.114945in}}{\pgfqpoint{5.453789in}{1.121786in}}{\pgfqpoint{5.448745in}{1.126830in}}%
\pgfpathcurveto{\pgfqpoint{5.443702in}{1.131874in}}{\pgfqpoint{5.436860in}{1.134708in}}{\pgfqpoint{5.429727in}{1.134708in}}%
\pgfpathcurveto{\pgfqpoint{5.422594in}{1.134708in}}{\pgfqpoint{5.415753in}{1.131874in}}{\pgfqpoint{5.410709in}{1.126830in}}%
\pgfpathcurveto{\pgfqpoint{5.405666in}{1.121786in}}{\pgfqpoint{5.402832in}{1.114945in}}{\pgfqpoint{5.402832in}{1.107812in}}%
\pgfpathcurveto{\pgfqpoint{5.402832in}{1.100679in}}{\pgfqpoint{5.405666in}{1.093837in}}{\pgfqpoint{5.410709in}{1.088794in}}%
\pgfpathcurveto{\pgfqpoint{5.415753in}{1.083750in}}{\pgfqpoint{5.422594in}{1.080916in}}{\pgfqpoint{5.429727in}{1.080916in}}%
\pgfpathclose%
\pgfusepath{stroke,fill}%
\end{pgfscope}%
\begin{pgfscope}%
\pgfpathrectangle{\pgfqpoint{4.985294in}{0.500000in}}{\pgfqpoint{1.764706in}{1.700000in}}%
\pgfusepath{clip}%
\pgfsetbuttcap%
\pgfsetroundjoin%
\definecolor{currentfill}{rgb}{0.964799,0.689101,0.537560}%
\pgfsetfillcolor{currentfill}%
\pgfsetlinewidth{0.311001pt}%
\definecolor{currentstroke}{rgb}{1.000000,1.000000,1.000000}%
\pgfsetstrokecolor{currentstroke}%
\pgfsetdash{}{0pt}%
\pgfpathmoveto{\pgfqpoint{5.590912in}{1.052564in}}%
\pgfpathcurveto{\pgfqpoint{5.598045in}{1.052564in}}{\pgfqpoint{5.604886in}{1.055398in}}{\pgfqpoint{5.609930in}{1.060441in}}%
\pgfpathcurveto{\pgfqpoint{5.614974in}{1.065485in}}{\pgfqpoint{5.617807in}{1.072327in}}{\pgfqpoint{5.617807in}{1.079459in}}%
\pgfpathcurveto{\pgfqpoint{5.617807in}{1.086592in}}{\pgfqpoint{5.614974in}{1.093434in}}{\pgfqpoint{5.609930in}{1.098478in}}%
\pgfpathcurveto{\pgfqpoint{5.604886in}{1.103521in}}{\pgfqpoint{5.598045in}{1.106355in}}{\pgfqpoint{5.590912in}{1.106355in}}%
\pgfpathcurveto{\pgfqpoint{5.583779in}{1.106355in}}{\pgfqpoint{5.576937in}{1.103521in}}{\pgfqpoint{5.571894in}{1.098478in}}%
\pgfpathcurveto{\pgfqpoint{5.566850in}{1.093434in}}{\pgfqpoint{5.564016in}{1.086592in}}{\pgfqpoint{5.564016in}{1.079459in}}%
\pgfpathcurveto{\pgfqpoint{5.564016in}{1.072327in}}{\pgfqpoint{5.566850in}{1.065485in}}{\pgfqpoint{5.571894in}{1.060441in}}%
\pgfpathcurveto{\pgfqpoint{5.576937in}{1.055398in}}{\pgfqpoint{5.583779in}{1.052564in}}{\pgfqpoint{5.590912in}{1.052564in}}%
\pgfpathclose%
\pgfusepath{stroke,fill}%
\end{pgfscope}%
\begin{pgfscope}%
\pgfpathrectangle{\pgfqpoint{4.985294in}{0.500000in}}{\pgfqpoint{1.764706in}{1.700000in}}%
\pgfusepath{clip}%
\pgfsetbuttcap%
\pgfsetroundjoin%
\definecolor{currentfill}{rgb}{0.979124,0.903132,0.839793}%
\pgfsetfillcolor{currentfill}%
\pgfsetlinewidth{0.311001pt}%
\definecolor{currentstroke}{rgb}{1.000000,1.000000,1.000000}%
\pgfsetstrokecolor{currentstroke}%
\pgfsetdash{}{0pt}%
\pgfpathmoveto{\pgfqpoint{5.399878in}{1.388926in}}%
\pgfpathcurveto{\pgfqpoint{5.407011in}{1.388926in}}{\pgfqpoint{5.413853in}{1.391760in}}{\pgfqpoint{5.418896in}{1.396804in}}%
\pgfpathcurveto{\pgfqpoint{5.423940in}{1.401848in}}{\pgfqpoint{5.426774in}{1.408689in}}{\pgfqpoint{5.426774in}{1.415822in}}%
\pgfpathcurveto{\pgfqpoint{5.426774in}{1.422955in}}{\pgfqpoint{5.423940in}{1.429796in}}{\pgfqpoint{5.418896in}{1.434840in}}%
\pgfpathcurveto{\pgfqpoint{5.413853in}{1.439884in}}{\pgfqpoint{5.407011in}{1.442718in}}{\pgfqpoint{5.399878in}{1.442718in}}%
\pgfpathcurveto{\pgfqpoint{5.392745in}{1.442718in}}{\pgfqpoint{5.385904in}{1.439884in}}{\pgfqpoint{5.380860in}{1.434840in}}%
\pgfpathcurveto{\pgfqpoint{5.375816in}{1.429796in}}{\pgfqpoint{5.372982in}{1.422955in}}{\pgfqpoint{5.372982in}{1.415822in}}%
\pgfpathcurveto{\pgfqpoint{5.372982in}{1.408689in}}{\pgfqpoint{5.375816in}{1.401848in}}{\pgfqpoint{5.380860in}{1.396804in}}%
\pgfpathcurveto{\pgfqpoint{5.385904in}{1.391760in}}{\pgfqpoint{5.392745in}{1.388926in}}{\pgfqpoint{5.399878in}{1.388926in}}%
\pgfpathclose%
\pgfusepath{stroke,fill}%
\end{pgfscope}%
\begin{pgfscope}%
\pgfpathrectangle{\pgfqpoint{4.985294in}{0.500000in}}{\pgfqpoint{1.764706in}{1.700000in}}%
\pgfusepath{clip}%
\pgfsetbuttcap%
\pgfsetroundjoin%
\definecolor{currentfill}{rgb}{0.964679,0.682838,0.530002}%
\pgfsetfillcolor{currentfill}%
\pgfsetlinewidth{0.311001pt}%
\definecolor{currentstroke}{rgb}{1.000000,1.000000,1.000000}%
\pgfsetstrokecolor{currentstroke}%
\pgfsetdash{}{0pt}%
\pgfpathmoveto{\pgfqpoint{6.261419in}{1.705667in}}%
\pgfpathcurveto{\pgfqpoint{6.268552in}{1.705667in}}{\pgfqpoint{6.275394in}{1.708501in}}{\pgfqpoint{6.280437in}{1.713545in}}%
\pgfpathcurveto{\pgfqpoint{6.285481in}{1.718588in}}{\pgfqpoint{6.288315in}{1.725430in}}{\pgfqpoint{6.288315in}{1.732563in}}%
\pgfpathcurveto{\pgfqpoint{6.288315in}{1.739696in}}{\pgfqpoint{6.285481in}{1.746537in}}{\pgfqpoint{6.280437in}{1.751581in}}%
\pgfpathcurveto{\pgfqpoint{6.275394in}{1.756625in}}{\pgfqpoint{6.268552in}{1.759459in}}{\pgfqpoint{6.261419in}{1.759459in}}%
\pgfpathcurveto{\pgfqpoint{6.254286in}{1.759459in}}{\pgfqpoint{6.247445in}{1.756625in}}{\pgfqpoint{6.242401in}{1.751581in}}%
\pgfpathcurveto{\pgfqpoint{6.237357in}{1.746537in}}{\pgfqpoint{6.234523in}{1.739696in}}{\pgfqpoint{6.234523in}{1.732563in}}%
\pgfpathcurveto{\pgfqpoint{6.234523in}{1.725430in}}{\pgfqpoint{6.237357in}{1.718588in}}{\pgfqpoint{6.242401in}{1.713545in}}%
\pgfpathcurveto{\pgfqpoint{6.247445in}{1.708501in}}{\pgfqpoint{6.254286in}{1.705667in}}{\pgfqpoint{6.261419in}{1.705667in}}%
\pgfpathclose%
\pgfusepath{stroke,fill}%
\end{pgfscope}%
\begin{pgfscope}%
\pgfpathrectangle{\pgfqpoint{4.985294in}{0.500000in}}{\pgfqpoint{1.764706in}{1.700000in}}%
\pgfusepath{clip}%
\pgfsetbuttcap%
\pgfsetroundjoin%
\definecolor{currentfill}{rgb}{0.974412,0.862387,0.780156}%
\pgfsetfillcolor{currentfill}%
\pgfsetlinewidth{0.311001pt}%
\definecolor{currentstroke}{rgb}{1.000000,1.000000,1.000000}%
\pgfsetstrokecolor{currentstroke}%
\pgfsetdash{}{0pt}%
\pgfpathmoveto{\pgfqpoint{6.257146in}{1.262994in}}%
\pgfpathcurveto{\pgfqpoint{6.264279in}{1.262994in}}{\pgfqpoint{6.271120in}{1.265828in}}{\pgfqpoint{6.276164in}{1.270872in}}%
\pgfpathcurveto{\pgfqpoint{6.281208in}{1.275916in}}{\pgfqpoint{6.284042in}{1.282757in}}{\pgfqpoint{6.284042in}{1.289890in}}%
\pgfpathcurveto{\pgfqpoint{6.284042in}{1.297023in}}{\pgfqpoint{6.281208in}{1.303865in}}{\pgfqpoint{6.276164in}{1.308908in}}%
\pgfpathcurveto{\pgfqpoint{6.271120in}{1.313952in}}{\pgfqpoint{6.264279in}{1.316786in}}{\pgfqpoint{6.257146in}{1.316786in}}%
\pgfpathcurveto{\pgfqpoint{6.250013in}{1.316786in}}{\pgfqpoint{6.243171in}{1.313952in}}{\pgfqpoint{6.238128in}{1.308908in}}%
\pgfpathcurveto{\pgfqpoint{6.233084in}{1.303865in}}{\pgfqpoint{6.230250in}{1.297023in}}{\pgfqpoint{6.230250in}{1.289890in}}%
\pgfpathcurveto{\pgfqpoint{6.230250in}{1.282757in}}{\pgfqpoint{6.233084in}{1.275916in}}{\pgfqpoint{6.238128in}{1.270872in}}%
\pgfpathcurveto{\pgfqpoint{6.243171in}{1.265828in}}{\pgfqpoint{6.250013in}{1.262994in}}{\pgfqpoint{6.257146in}{1.262994in}}%
\pgfpathclose%
\pgfusepath{stroke,fill}%
\end{pgfscope}%
\begin{pgfscope}%
\pgfpathrectangle{\pgfqpoint{4.985294in}{0.500000in}}{\pgfqpoint{1.764706in}{1.700000in}}%
\pgfusepath{clip}%
\pgfsetbuttcap%
\pgfsetroundjoin%
\definecolor{currentfill}{rgb}{0.966120,0.744512,0.608720}%
\pgfsetfillcolor{currentfill}%
\pgfsetlinewidth{0.311001pt}%
\definecolor{currentstroke}{rgb}{1.000000,1.000000,1.000000}%
\pgfsetstrokecolor{currentstroke}%
\pgfsetdash{}{0pt}%
\pgfpathmoveto{\pgfqpoint{6.151101in}{1.675981in}}%
\pgfpathcurveto{\pgfqpoint{6.158234in}{1.675981in}}{\pgfqpoint{6.165075in}{1.678815in}}{\pgfqpoint{6.170119in}{1.683859in}}%
\pgfpathcurveto{\pgfqpoint{6.175163in}{1.688902in}}{\pgfqpoint{6.177997in}{1.695744in}}{\pgfqpoint{6.177997in}{1.702877in}}%
\pgfpathcurveto{\pgfqpoint{6.177997in}{1.710010in}}{\pgfqpoint{6.175163in}{1.716851in}}{\pgfqpoint{6.170119in}{1.721895in}}%
\pgfpathcurveto{\pgfqpoint{6.165075in}{1.726939in}}{\pgfqpoint{6.158234in}{1.729772in}}{\pgfqpoint{6.151101in}{1.729772in}}%
\pgfpathcurveto{\pgfqpoint{6.143968in}{1.729772in}}{\pgfqpoint{6.137127in}{1.726939in}}{\pgfqpoint{6.132083in}{1.721895in}}%
\pgfpathcurveto{\pgfqpoint{6.127039in}{1.716851in}}{\pgfqpoint{6.124205in}{1.710010in}}{\pgfqpoint{6.124205in}{1.702877in}}%
\pgfpathcurveto{\pgfqpoint{6.124205in}{1.695744in}}{\pgfqpoint{6.127039in}{1.688902in}}{\pgfqpoint{6.132083in}{1.683859in}}%
\pgfpathcurveto{\pgfqpoint{6.137127in}{1.678815in}}{\pgfqpoint{6.143968in}{1.675981in}}{\pgfqpoint{6.151101in}{1.675981in}}%
\pgfpathclose%
\pgfusepath{stroke,fill}%
\end{pgfscope}%
\begin{pgfscope}%
\pgfpathrectangle{\pgfqpoint{4.985294in}{0.500000in}}{\pgfqpoint{1.764706in}{1.700000in}}%
\pgfusepath{clip}%
\pgfsetbuttcap%
\pgfsetroundjoin%
\definecolor{currentfill}{rgb}{0.976287,0.879862,0.805788}%
\pgfsetfillcolor{currentfill}%
\pgfsetlinewidth{0.311001pt}%
\definecolor{currentstroke}{rgb}{1.000000,1.000000,1.000000}%
\pgfsetstrokecolor{currentstroke}%
\pgfsetdash{}{0pt}%
\pgfpathmoveto{\pgfqpoint{6.319739in}{1.512849in}}%
\pgfpathcurveto{\pgfqpoint{6.326872in}{1.512849in}}{\pgfqpoint{6.333714in}{1.515683in}}{\pgfqpoint{6.338758in}{1.520727in}}%
\pgfpathcurveto{\pgfqpoint{6.343801in}{1.525771in}}{\pgfqpoint{6.346635in}{1.532612in}}{\pgfqpoint{6.346635in}{1.539745in}}%
\pgfpathcurveto{\pgfqpoint{6.346635in}{1.546878in}}{\pgfqpoint{6.343801in}{1.553720in}}{\pgfqpoint{6.338758in}{1.558763in}}%
\pgfpathcurveto{\pgfqpoint{6.333714in}{1.563807in}}{\pgfqpoint{6.326872in}{1.566641in}}{\pgfqpoint{6.319739in}{1.566641in}}%
\pgfpathcurveto{\pgfqpoint{6.312607in}{1.566641in}}{\pgfqpoint{6.305765in}{1.563807in}}{\pgfqpoint{6.300721in}{1.558763in}}%
\pgfpathcurveto{\pgfqpoint{6.295678in}{1.553720in}}{\pgfqpoint{6.292844in}{1.546878in}}{\pgfqpoint{6.292844in}{1.539745in}}%
\pgfpathcurveto{\pgfqpoint{6.292844in}{1.532612in}}{\pgfqpoint{6.295678in}{1.525771in}}{\pgfqpoint{6.300721in}{1.520727in}}%
\pgfpathcurveto{\pgfqpoint{6.305765in}{1.515683in}}{\pgfqpoint{6.312607in}{1.512849in}}{\pgfqpoint{6.319739in}{1.512849in}}%
\pgfpathclose%
\pgfusepath{stroke,fill}%
\end{pgfscope}%
\begin{pgfscope}%
\pgfpathrectangle{\pgfqpoint{4.985294in}{0.500000in}}{\pgfqpoint{1.764706in}{1.700000in}}%
\pgfusepath{clip}%
\pgfsetbuttcap%
\pgfsetroundjoin%
\definecolor{currentfill}{rgb}{0.972726,0.844889,0.754401}%
\pgfsetfillcolor{currentfill}%
\pgfsetlinewidth{0.311001pt}%
\definecolor{currentstroke}{rgb}{1.000000,1.000000,1.000000}%
\pgfsetstrokecolor{currentstroke}%
\pgfsetdash{}{0pt}%
\pgfpathmoveto{\pgfqpoint{6.321735in}{1.548804in}}%
\pgfpathcurveto{\pgfqpoint{6.328868in}{1.548804in}}{\pgfqpoint{6.335709in}{1.551638in}}{\pgfqpoint{6.340753in}{1.556682in}}%
\pgfpathcurveto{\pgfqpoint{6.345797in}{1.561726in}}{\pgfqpoint{6.348631in}{1.568567in}}{\pgfqpoint{6.348631in}{1.575700in}}%
\pgfpathcurveto{\pgfqpoint{6.348631in}{1.582833in}}{\pgfqpoint{6.345797in}{1.589675in}}{\pgfqpoint{6.340753in}{1.594718in}}%
\pgfpathcurveto{\pgfqpoint{6.335709in}{1.599762in}}{\pgfqpoint{6.328868in}{1.602596in}}{\pgfqpoint{6.321735in}{1.602596in}}%
\pgfpathcurveto{\pgfqpoint{6.314602in}{1.602596in}}{\pgfqpoint{6.307761in}{1.599762in}}{\pgfqpoint{6.302717in}{1.594718in}}%
\pgfpathcurveto{\pgfqpoint{6.297673in}{1.589675in}}{\pgfqpoint{6.294839in}{1.582833in}}{\pgfqpoint{6.294839in}{1.575700in}}%
\pgfpathcurveto{\pgfqpoint{6.294839in}{1.568567in}}{\pgfqpoint{6.297673in}{1.561726in}}{\pgfqpoint{6.302717in}{1.556682in}}%
\pgfpathcurveto{\pgfqpoint{6.307761in}{1.551638in}}{\pgfqpoint{6.314602in}{1.548804in}}{\pgfqpoint{6.321735in}{1.548804in}}%
\pgfpathclose%
\pgfusepath{stroke,fill}%
\end{pgfscope}%
\begin{pgfscope}%
\pgfpathrectangle{\pgfqpoint{4.985294in}{0.500000in}}{\pgfqpoint{1.764706in}{1.700000in}}%
\pgfusepath{clip}%
\pgfsetbuttcap%
\pgfsetroundjoin%
\definecolor{currentfill}{rgb}{0.972201,0.839051,0.745789}%
\pgfsetfillcolor{currentfill}%
\pgfsetlinewidth{0.311001pt}%
\definecolor{currentstroke}{rgb}{1.000000,1.000000,1.000000}%
\pgfsetstrokecolor{currentstroke}%
\pgfsetdash{}{0pt}%
\pgfpathmoveto{\pgfqpoint{5.407152in}{1.527428in}}%
\pgfpathcurveto{\pgfqpoint{5.414284in}{1.527428in}}{\pgfqpoint{5.421126in}{1.530262in}}{\pgfqpoint{5.426170in}{1.535306in}}%
\pgfpathcurveto{\pgfqpoint{5.431213in}{1.540349in}}{\pgfqpoint{5.434047in}{1.547191in}}{\pgfqpoint{5.434047in}{1.554324in}}%
\pgfpathcurveto{\pgfqpoint{5.434047in}{1.561457in}}{\pgfqpoint{5.431213in}{1.568298in}}{\pgfqpoint{5.426170in}{1.573342in}}%
\pgfpathcurveto{\pgfqpoint{5.421126in}{1.578386in}}{\pgfqpoint{5.414284in}{1.581220in}}{\pgfqpoint{5.407152in}{1.581220in}}%
\pgfpathcurveto{\pgfqpoint{5.400019in}{1.581220in}}{\pgfqpoint{5.393177in}{1.578386in}}{\pgfqpoint{5.388133in}{1.573342in}}%
\pgfpathcurveto{\pgfqpoint{5.383090in}{1.568298in}}{\pgfqpoint{5.380256in}{1.561457in}}{\pgfqpoint{5.380256in}{1.554324in}}%
\pgfpathcurveto{\pgfqpoint{5.380256in}{1.547191in}}{\pgfqpoint{5.383090in}{1.540349in}}{\pgfqpoint{5.388133in}{1.535306in}}%
\pgfpathcurveto{\pgfqpoint{5.393177in}{1.530262in}}{\pgfqpoint{5.400019in}{1.527428in}}{\pgfqpoint{5.407152in}{1.527428in}}%
\pgfpathclose%
\pgfusepath{stroke,fill}%
\end{pgfscope}%
\begin{pgfscope}%
\pgfpathrectangle{\pgfqpoint{4.985294in}{0.500000in}}{\pgfqpoint{1.764706in}{1.700000in}}%
\pgfusepath{clip}%
\pgfsetbuttcap%
\pgfsetroundjoin%
\definecolor{currentfill}{rgb}{0.974412,0.862387,0.780156}%
\pgfsetfillcolor{currentfill}%
\pgfsetlinewidth{0.311001pt}%
\definecolor{currentstroke}{rgb}{1.000000,1.000000,1.000000}%
\pgfsetstrokecolor{currentstroke}%
\pgfsetdash{}{0pt}%
\pgfpathmoveto{\pgfqpoint{6.267352in}{1.356329in}}%
\pgfpathcurveto{\pgfqpoint{6.274484in}{1.356329in}}{\pgfqpoint{6.281326in}{1.359163in}}{\pgfqpoint{6.286370in}{1.364207in}}%
\pgfpathcurveto{\pgfqpoint{6.291413in}{1.369250in}}{\pgfqpoint{6.294247in}{1.376092in}}{\pgfqpoint{6.294247in}{1.383225in}}%
\pgfpathcurveto{\pgfqpoint{6.294247in}{1.390358in}}{\pgfqpoint{6.291413in}{1.397199in}}{\pgfqpoint{6.286370in}{1.402243in}}%
\pgfpathcurveto{\pgfqpoint{6.281326in}{1.407287in}}{\pgfqpoint{6.274484in}{1.410121in}}{\pgfqpoint{6.267352in}{1.410121in}}%
\pgfpathcurveto{\pgfqpoint{6.260219in}{1.410121in}}{\pgfqpoint{6.253377in}{1.407287in}}{\pgfqpoint{6.248333in}{1.402243in}}%
\pgfpathcurveto{\pgfqpoint{6.243290in}{1.397199in}}{\pgfqpoint{6.240456in}{1.390358in}}{\pgfqpoint{6.240456in}{1.383225in}}%
\pgfpathcurveto{\pgfqpoint{6.240456in}{1.376092in}}{\pgfqpoint{6.243290in}{1.369250in}}{\pgfqpoint{6.248333in}{1.364207in}}%
\pgfpathcurveto{\pgfqpoint{6.253377in}{1.359163in}}{\pgfqpoint{6.260219in}{1.356329in}}{\pgfqpoint{6.267352in}{1.356329in}}%
\pgfpathclose%
\pgfusepath{stroke,fill}%
\end{pgfscope}%
\begin{pgfscope}%
\pgfpathrectangle{\pgfqpoint{4.985294in}{0.500000in}}{\pgfqpoint{1.764706in}{1.700000in}}%
\pgfusepath{clip}%
\pgfsetbuttcap%
\pgfsetroundjoin%
\definecolor{currentfill}{rgb}{0.975018,0.868213,0.788710}%
\pgfsetfillcolor{currentfill}%
\pgfsetlinewidth{0.311001pt}%
\definecolor{currentstroke}{rgb}{1.000000,1.000000,1.000000}%
\pgfsetstrokecolor{currentstroke}%
\pgfsetdash{}{0pt}%
\pgfpathmoveto{\pgfqpoint{5.481175in}{1.069169in}}%
\pgfpathcurveto{\pgfqpoint{5.488307in}{1.069169in}}{\pgfqpoint{5.495149in}{1.072002in}}{\pgfqpoint{5.500193in}{1.077046in}}%
\pgfpathcurveto{\pgfqpoint{5.505236in}{1.082090in}}{\pgfqpoint{5.508070in}{1.088931in}}{\pgfqpoint{5.508070in}{1.096064in}}%
\pgfpathcurveto{\pgfqpoint{5.508070in}{1.103197in}}{\pgfqpoint{5.505236in}{1.110039in}}{\pgfqpoint{5.500193in}{1.115082in}}%
\pgfpathcurveto{\pgfqpoint{5.495149in}{1.120126in}}{\pgfqpoint{5.488307in}{1.122960in}}{\pgfqpoint{5.481175in}{1.122960in}}%
\pgfpathcurveto{\pgfqpoint{5.474042in}{1.122960in}}{\pgfqpoint{5.467200in}{1.120126in}}{\pgfqpoint{5.462157in}{1.115082in}}%
\pgfpathcurveto{\pgfqpoint{5.457113in}{1.110039in}}{\pgfqpoint{5.454279in}{1.103197in}}{\pgfqpoint{5.454279in}{1.096064in}}%
\pgfpathcurveto{\pgfqpoint{5.454279in}{1.088931in}}{\pgfqpoint{5.457113in}{1.082090in}}{\pgfqpoint{5.462157in}{1.077046in}}%
\pgfpathcurveto{\pgfqpoint{5.467200in}{1.072002in}}{\pgfqpoint{5.474042in}{1.069169in}}{\pgfqpoint{5.481175in}{1.069169in}}%
\pgfpathclose%
\pgfusepath{stroke,fill}%
\end{pgfscope}%
\begin{pgfscope}%
\pgfpathrectangle{\pgfqpoint{4.985294in}{0.500000in}}{\pgfqpoint{1.764706in}{1.700000in}}%
\pgfusepath{clip}%
\pgfsetbuttcap%
\pgfsetroundjoin%
\definecolor{currentfill}{rgb}{0.966812,0.762584,0.633643}%
\pgfsetfillcolor{currentfill}%
\pgfsetlinewidth{0.311001pt}%
\definecolor{currentstroke}{rgb}{1.000000,1.000000,1.000000}%
\pgfsetstrokecolor{currentstroke}%
\pgfsetdash{}{0pt}%
\pgfpathmoveto{\pgfqpoint{6.213787in}{1.341245in}}%
\pgfpathcurveto{\pgfqpoint{6.220920in}{1.341245in}}{\pgfqpoint{6.227762in}{1.344079in}}{\pgfqpoint{6.232805in}{1.349122in}}%
\pgfpathcurveto{\pgfqpoint{6.237849in}{1.354166in}}{\pgfqpoint{6.240683in}{1.361008in}}{\pgfqpoint{6.240683in}{1.368141in}}%
\pgfpathcurveto{\pgfqpoint{6.240683in}{1.375273in}}{\pgfqpoint{6.237849in}{1.382115in}}{\pgfqpoint{6.232805in}{1.387159in}}%
\pgfpathcurveto{\pgfqpoint{6.227762in}{1.392202in}}{\pgfqpoint{6.220920in}{1.395036in}}{\pgfqpoint{6.213787in}{1.395036in}}%
\pgfpathcurveto{\pgfqpoint{6.206655in}{1.395036in}}{\pgfqpoint{6.199813in}{1.392202in}}{\pgfqpoint{6.194769in}{1.387159in}}%
\pgfpathcurveto{\pgfqpoint{6.189726in}{1.382115in}}{\pgfqpoint{6.186892in}{1.375273in}}{\pgfqpoint{6.186892in}{1.368141in}}%
\pgfpathcurveto{\pgfqpoint{6.186892in}{1.361008in}}{\pgfqpoint{6.189726in}{1.354166in}}{\pgfqpoint{6.194769in}{1.349122in}}%
\pgfpathcurveto{\pgfqpoint{6.199813in}{1.344079in}}{\pgfqpoint{6.206655in}{1.341245in}}{\pgfqpoint{6.213787in}{1.341245in}}%
\pgfpathclose%
\pgfusepath{stroke,fill}%
\end{pgfscope}%
\begin{pgfscope}%
\pgfpathrectangle{\pgfqpoint{4.985294in}{0.500000in}}{\pgfqpoint{1.764706in}{1.700000in}}%
\pgfusepath{clip}%
\pgfsetbuttcap%
\pgfsetroundjoin%
\definecolor{currentfill}{rgb}{0.980678,0.914765,0.856766}%
\pgfsetfillcolor{currentfill}%
\pgfsetlinewidth{0.311001pt}%
\definecolor{currentstroke}{rgb}{1.000000,1.000000,1.000000}%
\pgfsetstrokecolor{currentstroke}%
\pgfsetdash{}{0pt}%
\pgfpathmoveto{\pgfqpoint{5.417167in}{1.260622in}}%
\pgfpathcurveto{\pgfqpoint{5.424299in}{1.260622in}}{\pgfqpoint{5.431141in}{1.263456in}}{\pgfqpoint{5.436185in}{1.268500in}}%
\pgfpathcurveto{\pgfqpoint{5.441228in}{1.273544in}}{\pgfqpoint{5.444062in}{1.280385in}}{\pgfqpoint{5.444062in}{1.287518in}}%
\pgfpathcurveto{\pgfqpoint{5.444062in}{1.294651in}}{\pgfqpoint{5.441228in}{1.301492in}}{\pgfqpoint{5.436185in}{1.306536in}}%
\pgfpathcurveto{\pgfqpoint{5.431141in}{1.311580in}}{\pgfqpoint{5.424299in}{1.314414in}}{\pgfqpoint{5.417167in}{1.314414in}}%
\pgfpathcurveto{\pgfqpoint{5.410034in}{1.314414in}}{\pgfqpoint{5.403192in}{1.311580in}}{\pgfqpoint{5.398148in}{1.306536in}}%
\pgfpathcurveto{\pgfqpoint{5.393105in}{1.301492in}}{\pgfqpoint{5.390271in}{1.294651in}}{\pgfqpoint{5.390271in}{1.287518in}}%
\pgfpathcurveto{\pgfqpoint{5.390271in}{1.280385in}}{\pgfqpoint{5.393105in}{1.273544in}}{\pgfqpoint{5.398148in}{1.268500in}}%
\pgfpathcurveto{\pgfqpoint{5.403192in}{1.263456in}}{\pgfqpoint{5.410034in}{1.260622in}}{\pgfqpoint{5.417167in}{1.260622in}}%
\pgfpathclose%
\pgfusepath{stroke,fill}%
\end{pgfscope}%
\begin{pgfscope}%
\pgfpathrectangle{\pgfqpoint{4.985294in}{0.500000in}}{\pgfqpoint{1.764706in}{1.700000in}}%
\pgfusepath{clip}%
\pgfsetbuttcap%
\pgfsetroundjoin%
\definecolor{currentfill}{rgb}{0.966328,0.750560,0.616961}%
\pgfsetfillcolor{currentfill}%
\pgfsetlinewidth{0.311001pt}%
\definecolor{currentstroke}{rgb}{1.000000,1.000000,1.000000}%
\pgfsetstrokecolor{currentstroke}%
\pgfsetdash{}{0pt}%
\pgfpathmoveto{\pgfqpoint{6.396110in}{1.222688in}}%
\pgfpathcurveto{\pgfqpoint{6.403243in}{1.222688in}}{\pgfqpoint{6.410085in}{1.225522in}}{\pgfqpoint{6.415128in}{1.230565in}}%
\pgfpathcurveto{\pgfqpoint{6.420172in}{1.235609in}}{\pgfqpoint{6.423006in}{1.242451in}}{\pgfqpoint{6.423006in}{1.249583in}}%
\pgfpathcurveto{\pgfqpoint{6.423006in}{1.256716in}}{\pgfqpoint{6.420172in}{1.263558in}}{\pgfqpoint{6.415128in}{1.268602in}}%
\pgfpathcurveto{\pgfqpoint{6.410085in}{1.273645in}}{\pgfqpoint{6.403243in}{1.276479in}}{\pgfqpoint{6.396110in}{1.276479in}}%
\pgfpathcurveto{\pgfqpoint{6.388977in}{1.276479in}}{\pgfqpoint{6.382136in}{1.273645in}}{\pgfqpoint{6.377092in}{1.268602in}}%
\pgfpathcurveto{\pgfqpoint{6.372048in}{1.263558in}}{\pgfqpoint{6.369214in}{1.256716in}}{\pgfqpoint{6.369214in}{1.249583in}}%
\pgfpathcurveto{\pgfqpoint{6.369214in}{1.242451in}}{\pgfqpoint{6.372048in}{1.235609in}}{\pgfqpoint{6.377092in}{1.230565in}}%
\pgfpathcurveto{\pgfqpoint{6.382136in}{1.225522in}}{\pgfqpoint{6.388977in}{1.222688in}}{\pgfqpoint{6.396110in}{1.222688in}}%
\pgfpathclose%
\pgfusepath{stroke,fill}%
\end{pgfscope}%
\begin{pgfscope}%
\pgfpathrectangle{\pgfqpoint{4.985294in}{0.500000in}}{\pgfqpoint{1.764706in}{1.700000in}}%
\pgfusepath{clip}%
\pgfsetbuttcap%
\pgfsetroundjoin%
\definecolor{currentfill}{rgb}{0.964306,0.663930,0.507747}%
\pgfsetfillcolor{currentfill}%
\pgfsetlinewidth{0.311001pt}%
\definecolor{currentstroke}{rgb}{1.000000,1.000000,1.000000}%
\pgfsetstrokecolor{currentstroke}%
\pgfsetdash{}{0pt}%
\pgfpathmoveto{\pgfqpoint{5.316745in}{1.361010in}}%
\pgfpathcurveto{\pgfqpoint{5.323878in}{1.361010in}}{\pgfqpoint{5.330719in}{1.363844in}}{\pgfqpoint{5.335763in}{1.368887in}}%
\pgfpathcurveto{\pgfqpoint{5.340807in}{1.373931in}}{\pgfqpoint{5.343641in}{1.380773in}}{\pgfqpoint{5.343641in}{1.387905in}}%
\pgfpathcurveto{\pgfqpoint{5.343641in}{1.395038in}}{\pgfqpoint{5.340807in}{1.401880in}}{\pgfqpoint{5.335763in}{1.406923in}}%
\pgfpathcurveto{\pgfqpoint{5.330719in}{1.411967in}}{\pgfqpoint{5.323878in}{1.414801in}}{\pgfqpoint{5.316745in}{1.414801in}}%
\pgfpathcurveto{\pgfqpoint{5.309612in}{1.414801in}}{\pgfqpoint{5.302770in}{1.411967in}}{\pgfqpoint{5.297727in}{1.406923in}}%
\pgfpathcurveto{\pgfqpoint{5.292683in}{1.401880in}}{\pgfqpoint{5.289849in}{1.395038in}}{\pgfqpoint{5.289849in}{1.387905in}}%
\pgfpathcurveto{\pgfqpoint{5.289849in}{1.380773in}}{\pgfqpoint{5.292683in}{1.373931in}}{\pgfqpoint{5.297727in}{1.368887in}}%
\pgfpathcurveto{\pgfqpoint{5.302770in}{1.363844in}}{\pgfqpoint{5.309612in}{1.361010in}}{\pgfqpoint{5.316745in}{1.361010in}}%
\pgfpathclose%
\pgfusepath{stroke,fill}%
\end{pgfscope}%
\begin{pgfscope}%
\pgfpathrectangle{\pgfqpoint{4.985294in}{0.500000in}}{\pgfqpoint{1.764706in}{1.700000in}}%
\pgfusepath{clip}%
\pgfsetbuttcap%
\pgfsetroundjoin%
\definecolor{currentfill}{rgb}{0.974412,0.862387,0.780156}%
\pgfsetfillcolor{currentfill}%
\pgfsetlinewidth{0.311001pt}%
\definecolor{currentstroke}{rgb}{1.000000,1.000000,1.000000}%
\pgfsetstrokecolor{currentstroke}%
\pgfsetdash{}{0pt}%
\pgfpathmoveto{\pgfqpoint{6.254064in}{1.238895in}}%
\pgfpathcurveto{\pgfqpoint{6.261197in}{1.238895in}}{\pgfqpoint{6.268038in}{1.241729in}}{\pgfqpoint{6.273082in}{1.246773in}}%
\pgfpathcurveto{\pgfqpoint{6.278126in}{1.251816in}}{\pgfqpoint{6.280960in}{1.258658in}}{\pgfqpoint{6.280960in}{1.265791in}}%
\pgfpathcurveto{\pgfqpoint{6.280960in}{1.272924in}}{\pgfqpoint{6.278126in}{1.279765in}}{\pgfqpoint{6.273082in}{1.284809in}}%
\pgfpathcurveto{\pgfqpoint{6.268038in}{1.289853in}}{\pgfqpoint{6.261197in}{1.292686in}}{\pgfqpoint{6.254064in}{1.292686in}}%
\pgfpathcurveto{\pgfqpoint{6.246931in}{1.292686in}}{\pgfqpoint{6.240089in}{1.289853in}}{\pgfqpoint{6.235046in}{1.284809in}}%
\pgfpathcurveto{\pgfqpoint{6.230002in}{1.279765in}}{\pgfqpoint{6.227168in}{1.272924in}}{\pgfqpoint{6.227168in}{1.265791in}}%
\pgfpathcurveto{\pgfqpoint{6.227168in}{1.258658in}}{\pgfqpoint{6.230002in}{1.251816in}}{\pgfqpoint{6.235046in}{1.246773in}}%
\pgfpathcurveto{\pgfqpoint{6.240089in}{1.241729in}}{\pgfqpoint{6.246931in}{1.238895in}}{\pgfqpoint{6.254064in}{1.238895in}}%
\pgfpathclose%
\pgfusepath{stroke,fill}%
\end{pgfscope}%
\begin{pgfscope}%
\pgfpathrectangle{\pgfqpoint{4.985294in}{0.500000in}}{\pgfqpoint{1.764706in}{1.700000in}}%
\pgfusepath{clip}%
\pgfsetbuttcap%
\pgfsetroundjoin%
\definecolor{currentfill}{rgb}{0.965042,0.701564,0.552889}%
\pgfsetfillcolor{currentfill}%
\pgfsetlinewidth{0.311001pt}%
\definecolor{currentstroke}{rgb}{1.000000,1.000000,1.000000}%
\pgfsetstrokecolor{currentstroke}%
\pgfsetdash{}{0pt}%
\pgfpathmoveto{\pgfqpoint{5.342834in}{1.464395in}}%
\pgfpathcurveto{\pgfqpoint{5.349967in}{1.464395in}}{\pgfqpoint{5.356809in}{1.467229in}}{\pgfqpoint{5.361852in}{1.472273in}}%
\pgfpathcurveto{\pgfqpoint{5.366896in}{1.477316in}}{\pgfqpoint{5.369730in}{1.484158in}}{\pgfqpoint{5.369730in}{1.491291in}}%
\pgfpathcurveto{\pgfqpoint{5.369730in}{1.498424in}}{\pgfqpoint{5.366896in}{1.505265in}}{\pgfqpoint{5.361852in}{1.510309in}}%
\pgfpathcurveto{\pgfqpoint{5.356809in}{1.515353in}}{\pgfqpoint{5.349967in}{1.518186in}}{\pgfqpoint{5.342834in}{1.518186in}}%
\pgfpathcurveto{\pgfqpoint{5.335701in}{1.518186in}}{\pgfqpoint{5.328860in}{1.515353in}}{\pgfqpoint{5.323816in}{1.510309in}}%
\pgfpathcurveto{\pgfqpoint{5.318772in}{1.505265in}}{\pgfqpoint{5.315938in}{1.498424in}}{\pgfqpoint{5.315938in}{1.491291in}}%
\pgfpathcurveto{\pgfqpoint{5.315938in}{1.484158in}}{\pgfqpoint{5.318772in}{1.477316in}}{\pgfqpoint{5.323816in}{1.472273in}}%
\pgfpathcurveto{\pgfqpoint{5.328860in}{1.467229in}}{\pgfqpoint{5.335701in}{1.464395in}}{\pgfqpoint{5.342834in}{1.464395in}}%
\pgfpathclose%
\pgfusepath{stroke,fill}%
\end{pgfscope}%
\begin{pgfscope}%
\pgfpathrectangle{\pgfqpoint{4.985294in}{0.500000in}}{\pgfqpoint{1.764706in}{1.700000in}}%
\pgfusepath{clip}%
\pgfsetbuttcap%
\pgfsetroundjoin%
\definecolor{currentfill}{rgb}{0.966812,0.762584,0.633643}%
\pgfsetfillcolor{currentfill}%
\pgfsetlinewidth{0.311001pt}%
\definecolor{currentstroke}{rgb}{1.000000,1.000000,1.000000}%
\pgfsetstrokecolor{currentstroke}%
\pgfsetdash{}{0pt}%
\pgfpathmoveto{\pgfqpoint{5.379329in}{1.092762in}}%
\pgfpathcurveto{\pgfqpoint{5.386462in}{1.092762in}}{\pgfqpoint{5.393304in}{1.095596in}}{\pgfqpoint{5.398347in}{1.100639in}}%
\pgfpathcurveto{\pgfqpoint{5.403391in}{1.105683in}}{\pgfqpoint{5.406225in}{1.112525in}}{\pgfqpoint{5.406225in}{1.119658in}}%
\pgfpathcurveto{\pgfqpoint{5.406225in}{1.126790in}}{\pgfqpoint{5.403391in}{1.133632in}}{\pgfqpoint{5.398347in}{1.138676in}}%
\pgfpathcurveto{\pgfqpoint{5.393304in}{1.143719in}}{\pgfqpoint{5.386462in}{1.146553in}}{\pgfqpoint{5.379329in}{1.146553in}}%
\pgfpathcurveto{\pgfqpoint{5.372196in}{1.146553in}}{\pgfqpoint{5.365355in}{1.143719in}}{\pgfqpoint{5.360311in}{1.138676in}}%
\pgfpathcurveto{\pgfqpoint{5.355267in}{1.133632in}}{\pgfqpoint{5.352433in}{1.126790in}}{\pgfqpoint{5.352433in}{1.119658in}}%
\pgfpathcurveto{\pgfqpoint{5.352433in}{1.112525in}}{\pgfqpoint{5.355267in}{1.105683in}}{\pgfqpoint{5.360311in}{1.100639in}}%
\pgfpathcurveto{\pgfqpoint{5.365355in}{1.095596in}}{\pgfqpoint{5.372196in}{1.092762in}}{\pgfqpoint{5.379329in}{1.092762in}}%
\pgfpathclose%
\pgfusepath{stroke,fill}%
\end{pgfscope}%
\begin{pgfscope}%
\pgfpathrectangle{\pgfqpoint{4.985294in}{0.500000in}}{\pgfqpoint{1.764706in}{1.700000in}}%
\pgfusepath{clip}%
\pgfsetbuttcap%
\pgfsetroundjoin%
\definecolor{currentfill}{rgb}{0.966328,0.750560,0.616961}%
\pgfsetfillcolor{currentfill}%
\pgfsetlinewidth{0.311001pt}%
\definecolor{currentstroke}{rgb}{1.000000,1.000000,1.000000}%
\pgfsetstrokecolor{currentstroke}%
\pgfsetdash{}{0pt}%
\pgfpathmoveto{\pgfqpoint{5.481322in}{1.690445in}}%
\pgfpathcurveto{\pgfqpoint{5.488455in}{1.690445in}}{\pgfqpoint{5.495296in}{1.693279in}}{\pgfqpoint{5.500340in}{1.698323in}}%
\pgfpathcurveto{\pgfqpoint{5.505384in}{1.703367in}}{\pgfqpoint{5.508218in}{1.710208in}}{\pgfqpoint{5.508218in}{1.717341in}}%
\pgfpathcurveto{\pgfqpoint{5.508218in}{1.724474in}}{\pgfqpoint{5.505384in}{1.731316in}}{\pgfqpoint{5.500340in}{1.736359in}}%
\pgfpathcurveto{\pgfqpoint{5.495296in}{1.741403in}}{\pgfqpoint{5.488455in}{1.744237in}}{\pgfqpoint{5.481322in}{1.744237in}}%
\pgfpathcurveto{\pgfqpoint{5.474189in}{1.744237in}}{\pgfqpoint{5.467347in}{1.741403in}}{\pgfqpoint{5.462304in}{1.736359in}}%
\pgfpathcurveto{\pgfqpoint{5.457260in}{1.731316in}}{\pgfqpoint{5.454426in}{1.724474in}}{\pgfqpoint{5.454426in}{1.717341in}}%
\pgfpathcurveto{\pgfqpoint{5.454426in}{1.710208in}}{\pgfqpoint{5.457260in}{1.703367in}}{\pgfqpoint{5.462304in}{1.698323in}}%
\pgfpathcurveto{\pgfqpoint{5.467347in}{1.693279in}}{\pgfqpoint{5.474189in}{1.690445in}}{\pgfqpoint{5.481322in}{1.690445in}}%
\pgfpathclose%
\pgfusepath{stroke,fill}%
\end{pgfscope}%
\begin{pgfscope}%
\pgfpathrectangle{\pgfqpoint{4.985294in}{0.500000in}}{\pgfqpoint{1.764706in}{1.700000in}}%
\pgfusepath{clip}%
\pgfsetbuttcap%
\pgfsetroundjoin%
\definecolor{currentfill}{rgb}{0.977657,0.891500,0.822809}%
\pgfsetfillcolor{currentfill}%
\pgfsetlinewidth{0.311001pt}%
\definecolor{currentstroke}{rgb}{1.000000,1.000000,1.000000}%
\pgfsetstrokecolor{currentstroke}%
\pgfsetdash{}{0pt}%
\pgfpathmoveto{\pgfqpoint{6.302109in}{1.547110in}}%
\pgfpathcurveto{\pgfqpoint{6.309242in}{1.547110in}}{\pgfqpoint{6.316084in}{1.549944in}}{\pgfqpoint{6.321128in}{1.554988in}}%
\pgfpathcurveto{\pgfqpoint{6.326171in}{1.560032in}}{\pgfqpoint{6.329005in}{1.566873in}}{\pgfqpoint{6.329005in}{1.574006in}}%
\pgfpathcurveto{\pgfqpoint{6.329005in}{1.581139in}}{\pgfqpoint{6.326171in}{1.587981in}}{\pgfqpoint{6.321128in}{1.593024in}}%
\pgfpathcurveto{\pgfqpoint{6.316084in}{1.598068in}}{\pgfqpoint{6.309242in}{1.600902in}}{\pgfqpoint{6.302109in}{1.600902in}}%
\pgfpathcurveto{\pgfqpoint{6.294977in}{1.600902in}}{\pgfqpoint{6.288135in}{1.598068in}}{\pgfqpoint{6.283091in}{1.593024in}}%
\pgfpathcurveto{\pgfqpoint{6.278048in}{1.587981in}}{\pgfqpoint{6.275214in}{1.581139in}}{\pgfqpoint{6.275214in}{1.574006in}}%
\pgfpathcurveto{\pgfqpoint{6.275214in}{1.566873in}}{\pgfqpoint{6.278048in}{1.560032in}}{\pgfqpoint{6.283091in}{1.554988in}}%
\pgfpathcurveto{\pgfqpoint{6.288135in}{1.549944in}}{\pgfqpoint{6.294977in}{1.547110in}}{\pgfqpoint{6.302109in}{1.547110in}}%
\pgfpathclose%
\pgfusepath{stroke,fill}%
\end{pgfscope}%
\begin{pgfscope}%
\pgfpathrectangle{\pgfqpoint{4.985294in}{0.500000in}}{\pgfqpoint{1.764706in}{1.700000in}}%
\pgfusepath{clip}%
\pgfsetbuttcap%
\pgfsetroundjoin%
\definecolor{currentfill}{rgb}{0.978376,0.897317,0.831308}%
\pgfsetfillcolor{currentfill}%
\pgfsetlinewidth{0.311001pt}%
\definecolor{currentstroke}{rgb}{1.000000,1.000000,1.000000}%
\pgfsetstrokecolor{currentstroke}%
\pgfsetdash{}{0pt}%
\pgfpathmoveto{\pgfqpoint{6.302270in}{1.157587in}}%
\pgfpathcurveto{\pgfqpoint{6.309402in}{1.157587in}}{\pgfqpoint{6.316244in}{1.160421in}}{\pgfqpoint{6.321288in}{1.165464in}}%
\pgfpathcurveto{\pgfqpoint{6.326331in}{1.170508in}}{\pgfqpoint{6.329165in}{1.177350in}}{\pgfqpoint{6.329165in}{1.184483in}}%
\pgfpathcurveto{\pgfqpoint{6.329165in}{1.191615in}}{\pgfqpoint{6.326331in}{1.198457in}}{\pgfqpoint{6.321288in}{1.203501in}}%
\pgfpathcurveto{\pgfqpoint{6.316244in}{1.208544in}}{\pgfqpoint{6.309402in}{1.211378in}}{\pgfqpoint{6.302270in}{1.211378in}}%
\pgfpathcurveto{\pgfqpoint{6.295137in}{1.211378in}}{\pgfqpoint{6.288295in}{1.208544in}}{\pgfqpoint{6.283251in}{1.203501in}}%
\pgfpathcurveto{\pgfqpoint{6.278208in}{1.198457in}}{\pgfqpoint{6.275374in}{1.191615in}}{\pgfqpoint{6.275374in}{1.184483in}}%
\pgfpathcurveto{\pgfqpoint{6.275374in}{1.177350in}}{\pgfqpoint{6.278208in}{1.170508in}}{\pgfqpoint{6.283251in}{1.165464in}}%
\pgfpathcurveto{\pgfqpoint{6.288295in}{1.160421in}}{\pgfqpoint{6.295137in}{1.157587in}}{\pgfqpoint{6.302270in}{1.157587in}}%
\pgfpathclose%
\pgfusepath{stroke,fill}%
\end{pgfscope}%
\begin{pgfscope}%
\pgfpathrectangle{\pgfqpoint{4.985294in}{0.500000in}}{\pgfqpoint{1.764706in}{1.700000in}}%
\pgfusepath{clip}%
\pgfsetbuttcap%
\pgfsetroundjoin%
\definecolor{currentfill}{rgb}{0.967092,0.768560,0.642079}%
\pgfsetfillcolor{currentfill}%
\pgfsetlinewidth{0.311001pt}%
\definecolor{currentstroke}{rgb}{1.000000,1.000000,1.000000}%
\pgfsetstrokecolor{currentstroke}%
\pgfsetdash{}{0pt}%
\pgfpathmoveto{\pgfqpoint{6.193024in}{1.701674in}}%
\pgfpathcurveto{\pgfqpoint{6.200157in}{1.701674in}}{\pgfqpoint{6.206999in}{1.704508in}}{\pgfqpoint{6.212043in}{1.709551in}}%
\pgfpathcurveto{\pgfqpoint{6.217086in}{1.714595in}}{\pgfqpoint{6.219920in}{1.721437in}}{\pgfqpoint{6.219920in}{1.728570in}}%
\pgfpathcurveto{\pgfqpoint{6.219920in}{1.735702in}}{\pgfqpoint{6.217086in}{1.742544in}}{\pgfqpoint{6.212043in}{1.747588in}}%
\pgfpathcurveto{\pgfqpoint{6.206999in}{1.752631in}}{\pgfqpoint{6.200157in}{1.755465in}}{\pgfqpoint{6.193024in}{1.755465in}}%
\pgfpathcurveto{\pgfqpoint{6.185892in}{1.755465in}}{\pgfqpoint{6.179050in}{1.752631in}}{\pgfqpoint{6.174006in}{1.747588in}}%
\pgfpathcurveto{\pgfqpoint{6.168963in}{1.742544in}}{\pgfqpoint{6.166129in}{1.735702in}}{\pgfqpoint{6.166129in}{1.728570in}}%
\pgfpathcurveto{\pgfqpoint{6.166129in}{1.721437in}}{\pgfqpoint{6.168963in}{1.714595in}}{\pgfqpoint{6.174006in}{1.709551in}}%
\pgfpathcurveto{\pgfqpoint{6.179050in}{1.704508in}}{\pgfqpoint{6.185892in}{1.701674in}}{\pgfqpoint{6.193024in}{1.701674in}}%
\pgfpathclose%
\pgfusepath{stroke,fill}%
\end{pgfscope}%
\begin{pgfscope}%
\pgfpathrectangle{\pgfqpoint{4.985294in}{0.500000in}}{\pgfqpoint{1.764706in}{1.700000in}}%
\pgfusepath{clip}%
\pgfsetbuttcap%
\pgfsetroundjoin%
\definecolor{currentfill}{rgb}{0.974412,0.862387,0.780156}%
\pgfsetfillcolor{currentfill}%
\pgfsetlinewidth{0.311001pt}%
\definecolor{currentstroke}{rgb}{1.000000,1.000000,1.000000}%
\pgfsetstrokecolor{currentstroke}%
\pgfsetdash{}{0pt}%
\pgfpathmoveto{\pgfqpoint{5.368989in}{1.300279in}}%
\pgfpathcurveto{\pgfqpoint{5.376122in}{1.300279in}}{\pgfqpoint{5.382964in}{1.303113in}}{\pgfqpoint{5.388007in}{1.308156in}}%
\pgfpathcurveto{\pgfqpoint{5.393051in}{1.313200in}}{\pgfqpoint{5.395885in}{1.320042in}}{\pgfqpoint{5.395885in}{1.327175in}}%
\pgfpathcurveto{\pgfqpoint{5.395885in}{1.334307in}}{\pgfqpoint{5.393051in}{1.341149in}}{\pgfqpoint{5.388007in}{1.346193in}}%
\pgfpathcurveto{\pgfqpoint{5.382964in}{1.351236in}}{\pgfqpoint{5.376122in}{1.354070in}}{\pgfqpoint{5.368989in}{1.354070in}}%
\pgfpathcurveto{\pgfqpoint{5.361856in}{1.354070in}}{\pgfqpoint{5.355015in}{1.351236in}}{\pgfqpoint{5.349971in}{1.346193in}}%
\pgfpathcurveto{\pgfqpoint{5.344927in}{1.341149in}}{\pgfqpoint{5.342093in}{1.334307in}}{\pgfqpoint{5.342093in}{1.327175in}}%
\pgfpathcurveto{\pgfqpoint{5.342093in}{1.320042in}}{\pgfqpoint{5.344927in}{1.313200in}}{\pgfqpoint{5.349971in}{1.308156in}}%
\pgfpathcurveto{\pgfqpoint{5.355015in}{1.303113in}}{\pgfqpoint{5.361856in}{1.300279in}}{\pgfqpoint{5.368989in}{1.300279in}}%
\pgfpathclose%
\pgfusepath{stroke,fill}%
\end{pgfscope}%
\begin{pgfscope}%
\pgfpathrectangle{\pgfqpoint{4.985294in}{0.500000in}}{\pgfqpoint{1.764706in}{1.700000in}}%
\pgfusepath{clip}%
\pgfsetbuttcap%
\pgfsetroundjoin%
\definecolor{currentfill}{rgb}{0.966560,0.756582,0.625273}%
\pgfsetfillcolor{currentfill}%
\pgfsetlinewidth{0.311001pt}%
\definecolor{currentstroke}{rgb}{1.000000,1.000000,1.000000}%
\pgfsetstrokecolor{currentstroke}%
\pgfsetdash{}{0pt}%
\pgfpathmoveto{\pgfqpoint{6.396464in}{1.364303in}}%
\pgfpathcurveto{\pgfqpoint{6.403597in}{1.364303in}}{\pgfqpoint{6.410439in}{1.367137in}}{\pgfqpoint{6.415483in}{1.372181in}}%
\pgfpathcurveto{\pgfqpoint{6.420526in}{1.377224in}}{\pgfqpoint{6.423360in}{1.384066in}}{\pgfqpoint{6.423360in}{1.391199in}}%
\pgfpathcurveto{\pgfqpoint{6.423360in}{1.398332in}}{\pgfqpoint{6.420526in}{1.405173in}}{\pgfqpoint{6.415483in}{1.410217in}}%
\pgfpathcurveto{\pgfqpoint{6.410439in}{1.415261in}}{\pgfqpoint{6.403597in}{1.418095in}}{\pgfqpoint{6.396464in}{1.418095in}}%
\pgfpathcurveto{\pgfqpoint{6.389332in}{1.418095in}}{\pgfqpoint{6.382490in}{1.415261in}}{\pgfqpoint{6.377446in}{1.410217in}}%
\pgfpathcurveto{\pgfqpoint{6.372403in}{1.405173in}}{\pgfqpoint{6.369569in}{1.398332in}}{\pgfqpoint{6.369569in}{1.391199in}}%
\pgfpathcurveto{\pgfqpoint{6.369569in}{1.384066in}}{\pgfqpoint{6.372403in}{1.377224in}}{\pgfqpoint{6.377446in}{1.372181in}}%
\pgfpathcurveto{\pgfqpoint{6.382490in}{1.367137in}}{\pgfqpoint{6.389332in}{1.364303in}}{\pgfqpoint{6.396464in}{1.364303in}}%
\pgfpathclose%
\pgfusepath{stroke,fill}%
\end{pgfscope}%
\begin{pgfscope}%
\pgfpathrectangle{\pgfqpoint{4.985294in}{0.500000in}}{\pgfqpoint{1.764706in}{1.700000in}}%
\pgfusepath{clip}%
\pgfsetbuttcap%
\pgfsetroundjoin%
\definecolor{currentfill}{rgb}{0.968105,0.786346,0.667739}%
\pgfsetfillcolor{currentfill}%
\pgfsetlinewidth{0.311001pt}%
\definecolor{currentstroke}{rgb}{1.000000,1.000000,1.000000}%
\pgfsetstrokecolor{currentstroke}%
\pgfsetdash{}{0pt}%
\pgfpathmoveto{\pgfqpoint{5.572375in}{0.976922in}}%
\pgfpathcurveto{\pgfqpoint{5.579508in}{0.976922in}}{\pgfqpoint{5.586349in}{0.979756in}}{\pgfqpoint{5.591393in}{0.984800in}}%
\pgfpathcurveto{\pgfqpoint{5.596437in}{0.989844in}}{\pgfqpoint{5.599271in}{0.996685in}}{\pgfqpoint{5.599271in}{1.003818in}}%
\pgfpathcurveto{\pgfqpoint{5.599271in}{1.010951in}}{\pgfqpoint{5.596437in}{1.017793in}}{\pgfqpoint{5.591393in}{1.022836in}}%
\pgfpathcurveto{\pgfqpoint{5.586349in}{1.027880in}}{\pgfqpoint{5.579508in}{1.030714in}}{\pgfqpoint{5.572375in}{1.030714in}}%
\pgfpathcurveto{\pgfqpoint{5.565242in}{1.030714in}}{\pgfqpoint{5.558400in}{1.027880in}}{\pgfqpoint{5.553357in}{1.022836in}}%
\pgfpathcurveto{\pgfqpoint{5.548313in}{1.017793in}}{\pgfqpoint{5.545479in}{1.010951in}}{\pgfqpoint{5.545479in}{1.003818in}}%
\pgfpathcurveto{\pgfqpoint{5.545479in}{0.996685in}}{\pgfqpoint{5.548313in}{0.989844in}}{\pgfqpoint{5.553357in}{0.984800in}}%
\pgfpathcurveto{\pgfqpoint{5.558400in}{0.979756in}}{\pgfqpoint{5.565242in}{0.976922in}}{\pgfqpoint{5.572375in}{0.976922in}}%
\pgfpathclose%
\pgfusepath{stroke,fill}%
\end{pgfscope}%
\begin{pgfscope}%
\pgfpathrectangle{\pgfqpoint{4.985294in}{0.500000in}}{\pgfqpoint{1.764706in}{1.700000in}}%
\pgfusepath{clip}%
\pgfsetbuttcap%
\pgfsetroundjoin%
\definecolor{currentfill}{rgb}{0.966560,0.756582,0.625273}%
\pgfsetfillcolor{currentfill}%
\pgfsetlinewidth{0.311001pt}%
\definecolor{currentstroke}{rgb}{1.000000,1.000000,1.000000}%
\pgfsetstrokecolor{currentstroke}%
\pgfsetdash{}{0pt}%
\pgfpathmoveto{\pgfqpoint{6.197005in}{1.484604in}}%
\pgfpathcurveto{\pgfqpoint{6.204138in}{1.484604in}}{\pgfqpoint{6.210980in}{1.487438in}}{\pgfqpoint{6.216023in}{1.492481in}}%
\pgfpathcurveto{\pgfqpoint{6.221067in}{1.497525in}}{\pgfqpoint{6.223901in}{1.504367in}}{\pgfqpoint{6.223901in}{1.511500in}}%
\pgfpathcurveto{\pgfqpoint{6.223901in}{1.518632in}}{\pgfqpoint{6.221067in}{1.525474in}}{\pgfqpoint{6.216023in}{1.530518in}}%
\pgfpathcurveto{\pgfqpoint{6.210980in}{1.535561in}}{\pgfqpoint{6.204138in}{1.538395in}}{\pgfqpoint{6.197005in}{1.538395in}}%
\pgfpathcurveto{\pgfqpoint{6.189872in}{1.538395in}}{\pgfqpoint{6.183031in}{1.535561in}}{\pgfqpoint{6.177987in}{1.530518in}}%
\pgfpathcurveto{\pgfqpoint{6.172943in}{1.525474in}}{\pgfqpoint{6.170109in}{1.518632in}}{\pgfqpoint{6.170109in}{1.511500in}}%
\pgfpathcurveto{\pgfqpoint{6.170109in}{1.504367in}}{\pgfqpoint{6.172943in}{1.497525in}}{\pgfqpoint{6.177987in}{1.492481in}}%
\pgfpathcurveto{\pgfqpoint{6.183031in}{1.487438in}}{\pgfqpoint{6.189872in}{1.484604in}}{\pgfqpoint{6.197005in}{1.484604in}}%
\pgfpathclose%
\pgfusepath{stroke,fill}%
\end{pgfscope}%
\begin{pgfscope}%
\pgfpathrectangle{\pgfqpoint{4.985294in}{0.500000in}}{\pgfqpoint{1.764706in}{1.700000in}}%
\pgfusepath{clip}%
\pgfsetbuttcap%
\pgfsetroundjoin%
\definecolor{currentfill}{rgb}{0.980678,0.914765,0.856766}%
\pgfsetfillcolor{currentfill}%
\pgfsetlinewidth{0.311001pt}%
\definecolor{currentstroke}{rgb}{1.000000,1.000000,1.000000}%
\pgfsetstrokecolor{currentstroke}%
\pgfsetdash{}{0pt}%
\pgfpathmoveto{\pgfqpoint{5.416981in}{1.335389in}}%
\pgfpathcurveto{\pgfqpoint{5.424114in}{1.335389in}}{\pgfqpoint{5.430956in}{1.338223in}}{\pgfqpoint{5.435999in}{1.343267in}}%
\pgfpathcurveto{\pgfqpoint{5.441043in}{1.348311in}}{\pgfqpoint{5.443877in}{1.355152in}}{\pgfqpoint{5.443877in}{1.362285in}}%
\pgfpathcurveto{\pgfqpoint{5.443877in}{1.369418in}}{\pgfqpoint{5.441043in}{1.376260in}}{\pgfqpoint{5.435999in}{1.381303in}}%
\pgfpathcurveto{\pgfqpoint{5.430956in}{1.386347in}}{\pgfqpoint{5.424114in}{1.389181in}}{\pgfqpoint{5.416981in}{1.389181in}}%
\pgfpathcurveto{\pgfqpoint{5.409848in}{1.389181in}}{\pgfqpoint{5.403007in}{1.386347in}}{\pgfqpoint{5.397963in}{1.381303in}}%
\pgfpathcurveto{\pgfqpoint{5.392919in}{1.376260in}}{\pgfqpoint{5.390085in}{1.369418in}}{\pgfqpoint{5.390085in}{1.362285in}}%
\pgfpathcurveto{\pgfqpoint{5.390085in}{1.355152in}}{\pgfqpoint{5.392919in}{1.348311in}}{\pgfqpoint{5.397963in}{1.343267in}}%
\pgfpathcurveto{\pgfqpoint{5.403007in}{1.338223in}}{\pgfqpoint{5.409848in}{1.335389in}}{\pgfqpoint{5.416981in}{1.335389in}}%
\pgfpathclose%
\pgfusepath{stroke,fill}%
\end{pgfscope}%
\begin{pgfscope}%
\pgfpathrectangle{\pgfqpoint{4.985294in}{0.500000in}}{\pgfqpoint{1.764706in}{1.700000in}}%
\pgfusepath{clip}%
\pgfsetbuttcap%
\pgfsetroundjoin%
\definecolor{currentfill}{rgb}{0.965169,0.707764,0.560659}%
\pgfsetfillcolor{currentfill}%
\pgfsetlinewidth{0.311001pt}%
\definecolor{currentstroke}{rgb}{1.000000,1.000000,1.000000}%
\pgfsetstrokecolor{currentstroke}%
\pgfsetdash{}{0pt}%
\pgfpathmoveto{\pgfqpoint{5.353733in}{1.500389in}}%
\pgfpathcurveto{\pgfqpoint{5.360865in}{1.500389in}}{\pgfqpoint{5.367707in}{1.503223in}}{\pgfqpoint{5.372751in}{1.508267in}}%
\pgfpathcurveto{\pgfqpoint{5.377794in}{1.513310in}}{\pgfqpoint{5.380628in}{1.520152in}}{\pgfqpoint{5.380628in}{1.527285in}}%
\pgfpathcurveto{\pgfqpoint{5.380628in}{1.534418in}}{\pgfqpoint{5.377794in}{1.541259in}}{\pgfqpoint{5.372751in}{1.546303in}}%
\pgfpathcurveto{\pgfqpoint{5.367707in}{1.551347in}}{\pgfqpoint{5.360865in}{1.554181in}}{\pgfqpoint{5.353733in}{1.554181in}}%
\pgfpathcurveto{\pgfqpoint{5.346600in}{1.554181in}}{\pgfqpoint{5.339758in}{1.551347in}}{\pgfqpoint{5.334714in}{1.546303in}}%
\pgfpathcurveto{\pgfqpoint{5.329671in}{1.541259in}}{\pgfqpoint{5.326837in}{1.534418in}}{\pgfqpoint{5.326837in}{1.527285in}}%
\pgfpathcurveto{\pgfqpoint{5.326837in}{1.520152in}}{\pgfqpoint{5.329671in}{1.513310in}}{\pgfqpoint{5.334714in}{1.508267in}}%
\pgfpathcurveto{\pgfqpoint{5.339758in}{1.503223in}}{\pgfqpoint{5.346600in}{1.500389in}}{\pgfqpoint{5.353733in}{1.500389in}}%
\pgfpathclose%
\pgfusepath{stroke,fill}%
\end{pgfscope}%
\begin{pgfscope}%
\pgfpathrectangle{\pgfqpoint{4.985294in}{0.500000in}}{\pgfqpoint{1.764706in}{1.700000in}}%
\pgfusepath{clip}%
\pgfsetbuttcap%
\pgfsetroundjoin%
\definecolor{currentfill}{rgb}{0.972726,0.844889,0.754401}%
\pgfsetfillcolor{currentfill}%
\pgfsetlinewidth{0.311001pt}%
\definecolor{currentstroke}{rgb}{1.000000,1.000000,1.000000}%
\pgfsetstrokecolor{currentstroke}%
\pgfsetdash{}{0pt}%
\pgfpathmoveto{\pgfqpoint{6.243358in}{1.234142in}}%
\pgfpathcurveto{\pgfqpoint{6.250491in}{1.234142in}}{\pgfqpoint{6.257333in}{1.236976in}}{\pgfqpoint{6.262377in}{1.242020in}}%
\pgfpathcurveto{\pgfqpoint{6.267420in}{1.247064in}}{\pgfqpoint{6.270254in}{1.253905in}}{\pgfqpoint{6.270254in}{1.261038in}}%
\pgfpathcurveto{\pgfqpoint{6.270254in}{1.268171in}}{\pgfqpoint{6.267420in}{1.275013in}}{\pgfqpoint{6.262377in}{1.280056in}}%
\pgfpathcurveto{\pgfqpoint{6.257333in}{1.285100in}}{\pgfqpoint{6.250491in}{1.287934in}}{\pgfqpoint{6.243358in}{1.287934in}}%
\pgfpathcurveto{\pgfqpoint{6.236226in}{1.287934in}}{\pgfqpoint{6.229384in}{1.285100in}}{\pgfqpoint{6.224340in}{1.280056in}}%
\pgfpathcurveto{\pgfqpoint{6.219297in}{1.275013in}}{\pgfqpoint{6.216463in}{1.268171in}}{\pgfqpoint{6.216463in}{1.261038in}}%
\pgfpathcurveto{\pgfqpoint{6.216463in}{1.253905in}}{\pgfqpoint{6.219297in}{1.247064in}}{\pgfqpoint{6.224340in}{1.242020in}}%
\pgfpathcurveto{\pgfqpoint{6.229384in}{1.236976in}}{\pgfqpoint{6.236226in}{1.234142in}}{\pgfqpoint{6.243358in}{1.234142in}}%
\pgfpathclose%
\pgfusepath{stroke,fill}%
\end{pgfscope}%
\begin{pgfscope}%
\pgfpathrectangle{\pgfqpoint{4.985294in}{0.500000in}}{\pgfqpoint{1.764706in}{1.700000in}}%
\pgfusepath{clip}%
\pgfsetbuttcap%
\pgfsetroundjoin%
\definecolor{currentfill}{rgb}{0.978376,0.897317,0.831308}%
\pgfsetfillcolor{currentfill}%
\pgfsetlinewidth{0.311001pt}%
\definecolor{currentstroke}{rgb}{1.000000,1.000000,1.000000}%
\pgfsetstrokecolor{currentstroke}%
\pgfsetdash{}{0pt}%
\pgfpathmoveto{\pgfqpoint{6.283662in}{1.394943in}}%
\pgfpathcurveto{\pgfqpoint{6.290795in}{1.394943in}}{\pgfqpoint{6.297637in}{1.397777in}}{\pgfqpoint{6.302681in}{1.402820in}}%
\pgfpathcurveto{\pgfqpoint{6.307724in}{1.407864in}}{\pgfqpoint{6.310558in}{1.414706in}}{\pgfqpoint{6.310558in}{1.421839in}}%
\pgfpathcurveto{\pgfqpoint{6.310558in}{1.428971in}}{\pgfqpoint{6.307724in}{1.435813in}}{\pgfqpoint{6.302681in}{1.440857in}}%
\pgfpathcurveto{\pgfqpoint{6.297637in}{1.445900in}}{\pgfqpoint{6.290795in}{1.448734in}}{\pgfqpoint{6.283662in}{1.448734in}}%
\pgfpathcurveto{\pgfqpoint{6.276530in}{1.448734in}}{\pgfqpoint{6.269688in}{1.445900in}}{\pgfqpoint{6.264644in}{1.440857in}}%
\pgfpathcurveto{\pgfqpoint{6.259601in}{1.435813in}}{\pgfqpoint{6.256767in}{1.428971in}}{\pgfqpoint{6.256767in}{1.421839in}}%
\pgfpathcurveto{\pgfqpoint{6.256767in}{1.414706in}}{\pgfqpoint{6.259601in}{1.407864in}}{\pgfqpoint{6.264644in}{1.402820in}}%
\pgfpathcurveto{\pgfqpoint{6.269688in}{1.397777in}}{\pgfqpoint{6.276530in}{1.394943in}}{\pgfqpoint{6.283662in}{1.394943in}}%
\pgfpathclose%
\pgfusepath{stroke,fill}%
\end{pgfscope}%
\begin{pgfscope}%
\pgfpathrectangle{\pgfqpoint{4.985294in}{0.500000in}}{\pgfqpoint{1.764706in}{1.700000in}}%
\pgfusepath{clip}%
\pgfsetbuttcap%
\pgfsetroundjoin%
\definecolor{currentfill}{rgb}{0.964433,0.670254,0.515093}%
\pgfsetfillcolor{currentfill}%
\pgfsetlinewidth{0.311001pt}%
\definecolor{currentstroke}{rgb}{1.000000,1.000000,1.000000}%
\pgfsetstrokecolor{currentstroke}%
\pgfsetdash{}{0pt}%
\pgfpathmoveto{\pgfqpoint{6.172999in}{1.190455in}}%
\pgfpathcurveto{\pgfqpoint{6.180132in}{1.190455in}}{\pgfqpoint{6.186974in}{1.193289in}}{\pgfqpoint{6.192018in}{1.198333in}}%
\pgfpathcurveto{\pgfqpoint{6.197061in}{1.203376in}}{\pgfqpoint{6.199895in}{1.210218in}}{\pgfqpoint{6.199895in}{1.217351in}}%
\pgfpathcurveto{\pgfqpoint{6.199895in}{1.224484in}}{\pgfqpoint{6.197061in}{1.231325in}}{\pgfqpoint{6.192018in}{1.236369in}}%
\pgfpathcurveto{\pgfqpoint{6.186974in}{1.241413in}}{\pgfqpoint{6.180132in}{1.244246in}}{\pgfqpoint{6.172999in}{1.244246in}}%
\pgfpathcurveto{\pgfqpoint{6.165867in}{1.244246in}}{\pgfqpoint{6.159025in}{1.241413in}}{\pgfqpoint{6.153981in}{1.236369in}}%
\pgfpathcurveto{\pgfqpoint{6.148938in}{1.231325in}}{\pgfqpoint{6.146104in}{1.224484in}}{\pgfqpoint{6.146104in}{1.217351in}}%
\pgfpathcurveto{\pgfqpoint{6.146104in}{1.210218in}}{\pgfqpoint{6.148938in}{1.203376in}}{\pgfqpoint{6.153981in}{1.198333in}}%
\pgfpathcurveto{\pgfqpoint{6.159025in}{1.193289in}}{\pgfqpoint{6.165867in}{1.190455in}}{\pgfqpoint{6.172999in}{1.190455in}}%
\pgfpathclose%
\pgfusepath{stroke,fill}%
\end{pgfscope}%
\begin{pgfscope}%
\pgfpathrectangle{\pgfqpoint{4.985294in}{0.500000in}}{\pgfqpoint{1.764706in}{1.700000in}}%
\pgfusepath{clip}%
\pgfsetbuttcap%
\pgfsetroundjoin%
\definecolor{currentfill}{rgb}{0.967735,0.780441,0.659127}%
\pgfsetfillcolor{currentfill}%
\pgfsetlinewidth{0.311001pt}%
\definecolor{currentstroke}{rgb}{1.000000,1.000000,1.000000}%
\pgfsetstrokecolor{currentstroke}%
\pgfsetdash{}{0pt}%
\pgfpathmoveto{\pgfqpoint{6.279418in}{1.659419in}}%
\pgfpathcurveto{\pgfqpoint{6.286551in}{1.659419in}}{\pgfqpoint{6.293392in}{1.662253in}}{\pgfqpoint{6.298436in}{1.667297in}}%
\pgfpathcurveto{\pgfqpoint{6.303480in}{1.672341in}}{\pgfqpoint{6.306314in}{1.679182in}}{\pgfqpoint{6.306314in}{1.686315in}}%
\pgfpathcurveto{\pgfqpoint{6.306314in}{1.693448in}}{\pgfqpoint{6.303480in}{1.700290in}}{\pgfqpoint{6.298436in}{1.705333in}}%
\pgfpathcurveto{\pgfqpoint{6.293392in}{1.710377in}}{\pgfqpoint{6.286551in}{1.713211in}}{\pgfqpoint{6.279418in}{1.713211in}}%
\pgfpathcurveto{\pgfqpoint{6.272285in}{1.713211in}}{\pgfqpoint{6.265443in}{1.710377in}}{\pgfqpoint{6.260400in}{1.705333in}}%
\pgfpathcurveto{\pgfqpoint{6.255356in}{1.700290in}}{\pgfqpoint{6.252522in}{1.693448in}}{\pgfqpoint{6.252522in}{1.686315in}}%
\pgfpathcurveto{\pgfqpoint{6.252522in}{1.679182in}}{\pgfqpoint{6.255356in}{1.672341in}}{\pgfqpoint{6.260400in}{1.667297in}}%
\pgfpathcurveto{\pgfqpoint{6.265443in}{1.662253in}}{\pgfqpoint{6.272285in}{1.659419in}}{\pgfqpoint{6.279418in}{1.659419in}}%
\pgfpathclose%
\pgfusepath{stroke,fill}%
\end{pgfscope}%
\begin{pgfscope}%
\pgfpathrectangle{\pgfqpoint{4.985294in}{0.500000in}}{\pgfqpoint{1.764706in}{1.700000in}}%
\pgfusepath{clip}%
\pgfsetbuttcap%
\pgfsetroundjoin%
\definecolor{currentfill}{rgb}{0.954476,0.470822,0.323110}%
\pgfsetfillcolor{currentfill}%
\pgfsetlinewidth{0.311001pt}%
\definecolor{currentstroke}{rgb}{1.000000,1.000000,1.000000}%
\pgfsetstrokecolor{currentstroke}%
\pgfsetdash{}{0pt}%
\pgfpathmoveto{\pgfqpoint{6.215020in}{1.784042in}}%
\pgfpathcurveto{\pgfqpoint{6.222152in}{1.784042in}}{\pgfqpoint{6.228994in}{1.786875in}}{\pgfqpoint{6.234038in}{1.791919in}}%
\pgfpathcurveto{\pgfqpoint{6.239081in}{1.796963in}}{\pgfqpoint{6.241915in}{1.803804in}}{\pgfqpoint{6.241915in}{1.810937in}}%
\pgfpathcurveto{\pgfqpoint{6.241915in}{1.818070in}}{\pgfqpoint{6.239081in}{1.824912in}}{\pgfqpoint{6.234038in}{1.829955in}}%
\pgfpathcurveto{\pgfqpoint{6.228994in}{1.834999in}}{\pgfqpoint{6.222152in}{1.837833in}}{\pgfqpoint{6.215020in}{1.837833in}}%
\pgfpathcurveto{\pgfqpoint{6.207887in}{1.837833in}}{\pgfqpoint{6.201045in}{1.834999in}}{\pgfqpoint{6.196001in}{1.829955in}}%
\pgfpathcurveto{\pgfqpoint{6.190958in}{1.824912in}}{\pgfqpoint{6.188124in}{1.818070in}}{\pgfqpoint{6.188124in}{1.810937in}}%
\pgfpathcurveto{\pgfqpoint{6.188124in}{1.803804in}}{\pgfqpoint{6.190958in}{1.796963in}}{\pgfqpoint{6.196001in}{1.791919in}}%
\pgfpathcurveto{\pgfqpoint{6.201045in}{1.786875in}}{\pgfqpoint{6.207887in}{1.784042in}}{\pgfqpoint{6.215020in}{1.784042in}}%
\pgfpathclose%
\pgfusepath{stroke,fill}%
\end{pgfscope}%
\begin{pgfscope}%
\pgfpathrectangle{\pgfqpoint{4.985294in}{0.500000in}}{\pgfqpoint{1.764706in}{1.700000in}}%
\pgfusepath{clip}%
\pgfsetbuttcap%
\pgfsetroundjoin%
\definecolor{currentfill}{rgb}{0.973832,0.856556,0.771584}%
\pgfsetfillcolor{currentfill}%
\pgfsetlinewidth{0.311001pt}%
\definecolor{currentstroke}{rgb}{1.000000,1.000000,1.000000}%
\pgfsetstrokecolor{currentstroke}%
\pgfsetdash{}{0pt}%
\pgfpathmoveto{\pgfqpoint{6.199109in}{1.608898in}}%
\pgfpathcurveto{\pgfqpoint{6.206242in}{1.608898in}}{\pgfqpoint{6.213084in}{1.611732in}}{\pgfqpoint{6.218128in}{1.616776in}}%
\pgfpathcurveto{\pgfqpoint{6.223171in}{1.621819in}}{\pgfqpoint{6.226005in}{1.628661in}}{\pgfqpoint{6.226005in}{1.635794in}}%
\pgfpathcurveto{\pgfqpoint{6.226005in}{1.642927in}}{\pgfqpoint{6.223171in}{1.649768in}}{\pgfqpoint{6.218128in}{1.654812in}}%
\pgfpathcurveto{\pgfqpoint{6.213084in}{1.659856in}}{\pgfqpoint{6.206242in}{1.662689in}}{\pgfqpoint{6.199109in}{1.662689in}}%
\pgfpathcurveto{\pgfqpoint{6.191977in}{1.662689in}}{\pgfqpoint{6.185135in}{1.659856in}}{\pgfqpoint{6.180091in}{1.654812in}}%
\pgfpathcurveto{\pgfqpoint{6.175048in}{1.649768in}}{\pgfqpoint{6.172214in}{1.642927in}}{\pgfqpoint{6.172214in}{1.635794in}}%
\pgfpathcurveto{\pgfqpoint{6.172214in}{1.628661in}}{\pgfqpoint{6.175048in}{1.621819in}}{\pgfqpoint{6.180091in}{1.616776in}}%
\pgfpathcurveto{\pgfqpoint{6.185135in}{1.611732in}}{\pgfqpoint{6.191977in}{1.608898in}}{\pgfqpoint{6.199109in}{1.608898in}}%
\pgfpathclose%
\pgfusepath{stroke,fill}%
\end{pgfscope}%
\begin{pgfscope}%
\pgfpathrectangle{\pgfqpoint{4.985294in}{0.500000in}}{\pgfqpoint{1.764706in}{1.700000in}}%
\pgfusepath{clip}%
\pgfsetbuttcap%
\pgfsetroundjoin%
\definecolor{currentfill}{rgb}{0.979891,0.908948,0.848279}%
\pgfsetfillcolor{currentfill}%
\pgfsetlinewidth{0.311001pt}%
\definecolor{currentstroke}{rgb}{1.000000,1.000000,1.000000}%
\pgfsetstrokecolor{currentstroke}%
\pgfsetdash{}{0pt}%
\pgfpathmoveto{\pgfqpoint{5.398956in}{1.322001in}}%
\pgfpathcurveto{\pgfqpoint{5.406089in}{1.322001in}}{\pgfqpoint{5.412931in}{1.324835in}}{\pgfqpoint{5.417974in}{1.329878in}}%
\pgfpathcurveto{\pgfqpoint{5.423018in}{1.334922in}}{\pgfqpoint{5.425852in}{1.341764in}}{\pgfqpoint{5.425852in}{1.348896in}}%
\pgfpathcurveto{\pgfqpoint{5.425852in}{1.356029in}}{\pgfqpoint{5.423018in}{1.362871in}}{\pgfqpoint{5.417974in}{1.367915in}}%
\pgfpathcurveto{\pgfqpoint{5.412931in}{1.372958in}}{\pgfqpoint{5.406089in}{1.375792in}}{\pgfqpoint{5.398956in}{1.375792in}}%
\pgfpathcurveto{\pgfqpoint{5.391823in}{1.375792in}}{\pgfqpoint{5.384982in}{1.372958in}}{\pgfqpoint{5.379938in}{1.367915in}}%
\pgfpathcurveto{\pgfqpoint{5.374894in}{1.362871in}}{\pgfqpoint{5.372061in}{1.356029in}}{\pgfqpoint{5.372061in}{1.348896in}}%
\pgfpathcurveto{\pgfqpoint{5.372061in}{1.341764in}}{\pgfqpoint{5.374894in}{1.334922in}}{\pgfqpoint{5.379938in}{1.329878in}}%
\pgfpathcurveto{\pgfqpoint{5.384982in}{1.324835in}}{\pgfqpoint{5.391823in}{1.322001in}}{\pgfqpoint{5.398956in}{1.322001in}}%
\pgfpathclose%
\pgfusepath{stroke,fill}%
\end{pgfscope}%
\begin{pgfscope}%
\pgfpathrectangle{\pgfqpoint{4.985294in}{0.500000in}}{\pgfqpoint{1.764706in}{1.700000in}}%
\pgfusepath{clip}%
\pgfsetbuttcap%
\pgfsetroundjoin%
\definecolor{currentfill}{rgb}{0.976961,0.885681,0.814303}%
\pgfsetfillcolor{currentfill}%
\pgfsetlinewidth{0.311001pt}%
\definecolor{currentstroke}{rgb}{1.000000,1.000000,1.000000}%
\pgfsetstrokecolor{currentstroke}%
\pgfsetdash{}{0pt}%
\pgfpathmoveto{\pgfqpoint{5.464696in}{1.168226in}}%
\pgfpathcurveto{\pgfqpoint{5.471829in}{1.168226in}}{\pgfqpoint{5.478670in}{1.171060in}}{\pgfqpoint{5.483714in}{1.176103in}}%
\pgfpathcurveto{\pgfqpoint{5.488758in}{1.181147in}}{\pgfqpoint{5.491592in}{1.187989in}}{\pgfqpoint{5.491592in}{1.195121in}}%
\pgfpathcurveto{\pgfqpoint{5.491592in}{1.202254in}}{\pgfqpoint{5.488758in}{1.209096in}}{\pgfqpoint{5.483714in}{1.214140in}}%
\pgfpathcurveto{\pgfqpoint{5.478670in}{1.219183in}}{\pgfqpoint{5.471829in}{1.222017in}}{\pgfqpoint{5.464696in}{1.222017in}}%
\pgfpathcurveto{\pgfqpoint{5.457563in}{1.222017in}}{\pgfqpoint{5.450721in}{1.219183in}}{\pgfqpoint{5.445678in}{1.214140in}}%
\pgfpathcurveto{\pgfqpoint{5.440634in}{1.209096in}}{\pgfqpoint{5.437800in}{1.202254in}}{\pgfqpoint{5.437800in}{1.195121in}}%
\pgfpathcurveto{\pgfqpoint{5.437800in}{1.187989in}}{\pgfqpoint{5.440634in}{1.181147in}}{\pgfqpoint{5.445678in}{1.176103in}}%
\pgfpathcurveto{\pgfqpoint{5.450721in}{1.171060in}}{\pgfqpoint{5.457563in}{1.168226in}}{\pgfqpoint{5.464696in}{1.168226in}}%
\pgfpathclose%
\pgfusepath{stroke,fill}%
\end{pgfscope}%
\begin{pgfscope}%
\pgfpathrectangle{\pgfqpoint{4.985294in}{0.500000in}}{\pgfqpoint{1.764706in}{1.700000in}}%
\pgfusepath{clip}%
\pgfsetbuttcap%
\pgfsetroundjoin%
\definecolor{currentfill}{rgb}{0.969359,0.803954,0.693832}%
\pgfsetfillcolor{currentfill}%
\pgfsetlinewidth{0.311001pt}%
\definecolor{currentstroke}{rgb}{1.000000,1.000000,1.000000}%
\pgfsetstrokecolor{currentstroke}%
\pgfsetdash{}{0pt}%
\pgfpathmoveto{\pgfqpoint{6.227027in}{1.279937in}}%
\pgfpathcurveto{\pgfqpoint{6.234160in}{1.279937in}}{\pgfqpoint{6.241002in}{1.282771in}}{\pgfqpoint{6.246045in}{1.287815in}}%
\pgfpathcurveto{\pgfqpoint{6.251089in}{1.292858in}}{\pgfqpoint{6.253923in}{1.299700in}}{\pgfqpoint{6.253923in}{1.306833in}}%
\pgfpathcurveto{\pgfqpoint{6.253923in}{1.313966in}}{\pgfqpoint{6.251089in}{1.320807in}}{\pgfqpoint{6.246045in}{1.325851in}}%
\pgfpathcurveto{\pgfqpoint{6.241002in}{1.330895in}}{\pgfqpoint{6.234160in}{1.333728in}}{\pgfqpoint{6.227027in}{1.333728in}}%
\pgfpathcurveto{\pgfqpoint{6.219894in}{1.333728in}}{\pgfqpoint{6.213053in}{1.330895in}}{\pgfqpoint{6.208009in}{1.325851in}}%
\pgfpathcurveto{\pgfqpoint{6.202965in}{1.320807in}}{\pgfqpoint{6.200132in}{1.313966in}}{\pgfqpoint{6.200132in}{1.306833in}}%
\pgfpathcurveto{\pgfqpoint{6.200132in}{1.299700in}}{\pgfqpoint{6.202965in}{1.292858in}}{\pgfqpoint{6.208009in}{1.287815in}}%
\pgfpathcurveto{\pgfqpoint{6.213053in}{1.282771in}}{\pgfqpoint{6.219894in}{1.279937in}}{\pgfqpoint{6.227027in}{1.279937in}}%
\pgfpathclose%
\pgfusepath{stroke,fill}%
\end{pgfscope}%
\begin{pgfscope}%
\pgfpathrectangle{\pgfqpoint{4.985294in}{0.500000in}}{\pgfqpoint{1.764706in}{1.700000in}}%
\pgfusepath{clip}%
\pgfsetbuttcap%
\pgfsetroundjoin%
\definecolor{currentfill}{rgb}{0.965928,0.738443,0.600540}%
\pgfsetfillcolor{currentfill}%
\pgfsetlinewidth{0.311001pt}%
\definecolor{currentstroke}{rgb}{1.000000,1.000000,1.000000}%
\pgfsetstrokecolor{currentstroke}%
\pgfsetdash{}{0pt}%
\pgfpathmoveto{\pgfqpoint{5.516626in}{1.238231in}}%
\pgfpathcurveto{\pgfqpoint{5.523759in}{1.238231in}}{\pgfqpoint{5.530601in}{1.241065in}}{\pgfqpoint{5.535645in}{1.246109in}}%
\pgfpathcurveto{\pgfqpoint{5.540688in}{1.251152in}}{\pgfqpoint{5.543522in}{1.257994in}}{\pgfqpoint{5.543522in}{1.265127in}}%
\pgfpathcurveto{\pgfqpoint{5.543522in}{1.272260in}}{\pgfqpoint{5.540688in}{1.279101in}}{\pgfqpoint{5.535645in}{1.284145in}}%
\pgfpathcurveto{\pgfqpoint{5.530601in}{1.289189in}}{\pgfqpoint{5.523759in}{1.292023in}}{\pgfqpoint{5.516626in}{1.292023in}}%
\pgfpathcurveto{\pgfqpoint{5.509494in}{1.292023in}}{\pgfqpoint{5.502652in}{1.289189in}}{\pgfqpoint{5.497608in}{1.284145in}}%
\pgfpathcurveto{\pgfqpoint{5.492565in}{1.279101in}}{\pgfqpoint{5.489731in}{1.272260in}}{\pgfqpoint{5.489731in}{1.265127in}}%
\pgfpathcurveto{\pgfqpoint{5.489731in}{1.257994in}}{\pgfqpoint{5.492565in}{1.251152in}}{\pgfqpoint{5.497608in}{1.246109in}}%
\pgfpathcurveto{\pgfqpoint{5.502652in}{1.241065in}}{\pgfqpoint{5.509494in}{1.238231in}}{\pgfqpoint{5.516626in}{1.238231in}}%
\pgfpathclose%
\pgfusepath{stroke,fill}%
\end{pgfscope}%
\begin{pgfscope}%
\pgfpathrectangle{\pgfqpoint{4.985294in}{0.500000in}}{\pgfqpoint{1.764706in}{1.700000in}}%
\pgfusepath{clip}%
\pgfsetbuttcap%
\pgfsetroundjoin%
\definecolor{currentfill}{rgb}{0.970718,0.821518,0.719872}%
\pgfsetfillcolor{currentfill}%
\pgfsetlinewidth{0.311001pt}%
\definecolor{currentstroke}{rgb}{1.000000,1.000000,1.000000}%
\pgfsetstrokecolor{currentstroke}%
\pgfsetdash{}{0pt}%
\pgfpathmoveto{\pgfqpoint{5.468991in}{1.631141in}}%
\pgfpathcurveto{\pgfqpoint{5.476123in}{1.631141in}}{\pgfqpoint{5.482965in}{1.633975in}}{\pgfqpoint{5.488009in}{1.639018in}}%
\pgfpathcurveto{\pgfqpoint{5.493052in}{1.644062in}}{\pgfqpoint{5.495886in}{1.650904in}}{\pgfqpoint{5.495886in}{1.658036in}}%
\pgfpathcurveto{\pgfqpoint{5.495886in}{1.665169in}}{\pgfqpoint{5.493052in}{1.672011in}}{\pgfqpoint{5.488009in}{1.677055in}}%
\pgfpathcurveto{\pgfqpoint{5.482965in}{1.682098in}}{\pgfqpoint{5.476123in}{1.684932in}}{\pgfqpoint{5.468991in}{1.684932in}}%
\pgfpathcurveto{\pgfqpoint{5.461858in}{1.684932in}}{\pgfqpoint{5.455016in}{1.682098in}}{\pgfqpoint{5.449972in}{1.677055in}}%
\pgfpathcurveto{\pgfqpoint{5.444929in}{1.672011in}}{\pgfqpoint{5.442095in}{1.665169in}}{\pgfqpoint{5.442095in}{1.658036in}}%
\pgfpathcurveto{\pgfqpoint{5.442095in}{1.650904in}}{\pgfqpoint{5.444929in}{1.644062in}}{\pgfqpoint{5.449972in}{1.639018in}}%
\pgfpathcurveto{\pgfqpoint{5.455016in}{1.633975in}}{\pgfqpoint{5.461858in}{1.631141in}}{\pgfqpoint{5.468991in}{1.631141in}}%
\pgfpathclose%
\pgfusepath{stroke,fill}%
\end{pgfscope}%
\begin{pgfscope}%
\pgfpathrectangle{\pgfqpoint{4.985294in}{0.500000in}}{\pgfqpoint{1.764706in}{1.700000in}}%
\pgfusepath{clip}%
\pgfsetbuttcap%
\pgfsetroundjoin%
\definecolor{currentfill}{rgb}{0.974412,0.862387,0.780156}%
\pgfsetfillcolor{currentfill}%
\pgfsetlinewidth{0.311001pt}%
\definecolor{currentstroke}{rgb}{1.000000,1.000000,1.000000}%
\pgfsetstrokecolor{currentstroke}%
\pgfsetdash{}{0pt}%
\pgfpathmoveto{\pgfqpoint{5.498481in}{1.048977in}}%
\pgfpathcurveto{\pgfqpoint{5.505614in}{1.048977in}}{\pgfqpoint{5.512456in}{1.051811in}}{\pgfqpoint{5.517499in}{1.056854in}}%
\pgfpathcurveto{\pgfqpoint{5.522543in}{1.061898in}}{\pgfqpoint{5.525377in}{1.068740in}}{\pgfqpoint{5.525377in}{1.075872in}}%
\pgfpathcurveto{\pgfqpoint{5.525377in}{1.083005in}}{\pgfqpoint{5.522543in}{1.089847in}}{\pgfqpoint{5.517499in}{1.094891in}}%
\pgfpathcurveto{\pgfqpoint{5.512456in}{1.099934in}}{\pgfqpoint{5.505614in}{1.102768in}}{\pgfqpoint{5.498481in}{1.102768in}}%
\pgfpathcurveto{\pgfqpoint{5.491348in}{1.102768in}}{\pgfqpoint{5.484507in}{1.099934in}}{\pgfqpoint{5.479463in}{1.094891in}}%
\pgfpathcurveto{\pgfqpoint{5.474419in}{1.089847in}}{\pgfqpoint{5.471585in}{1.083005in}}{\pgfqpoint{5.471585in}{1.075872in}}%
\pgfpathcurveto{\pgfqpoint{5.471585in}{1.068740in}}{\pgfqpoint{5.474419in}{1.061898in}}{\pgfqpoint{5.479463in}{1.056854in}}%
\pgfpathcurveto{\pgfqpoint{5.484507in}{1.051811in}}{\pgfqpoint{5.491348in}{1.048977in}}{\pgfqpoint{5.498481in}{1.048977in}}%
\pgfpathclose%
\pgfusepath{stroke,fill}%
\end{pgfscope}%
\begin{pgfscope}%
\pgfpathrectangle{\pgfqpoint{4.985294in}{0.500000in}}{\pgfqpoint{1.764706in}{1.700000in}}%
\pgfusepath{clip}%
\pgfsetbuttcap%
\pgfsetroundjoin%
\definecolor{currentfill}{rgb}{0.980678,0.914765,0.856766}%
\pgfsetfillcolor{currentfill}%
\pgfsetlinewidth{0.311001pt}%
\definecolor{currentstroke}{rgb}{1.000000,1.000000,1.000000}%
\pgfsetstrokecolor{currentstroke}%
\pgfsetdash{}{0pt}%
\pgfpathmoveto{\pgfqpoint{6.327667in}{1.280532in}}%
\pgfpathcurveto{\pgfqpoint{6.334800in}{1.280532in}}{\pgfqpoint{6.341642in}{1.283366in}}{\pgfqpoint{6.346685in}{1.288410in}}%
\pgfpathcurveto{\pgfqpoint{6.351729in}{1.293454in}}{\pgfqpoint{6.354563in}{1.300295in}}{\pgfqpoint{6.354563in}{1.307428in}}%
\pgfpathcurveto{\pgfqpoint{6.354563in}{1.314561in}}{\pgfqpoint{6.351729in}{1.321403in}}{\pgfqpoint{6.346685in}{1.326446in}}%
\pgfpathcurveto{\pgfqpoint{6.341642in}{1.331490in}}{\pgfqpoint{6.334800in}{1.334324in}}{\pgfqpoint{6.327667in}{1.334324in}}%
\pgfpathcurveto{\pgfqpoint{6.320534in}{1.334324in}}{\pgfqpoint{6.313693in}{1.331490in}}{\pgfqpoint{6.308649in}{1.326446in}}%
\pgfpathcurveto{\pgfqpoint{6.303605in}{1.321403in}}{\pgfqpoint{6.300771in}{1.314561in}}{\pgfqpoint{6.300771in}{1.307428in}}%
\pgfpathcurveto{\pgfqpoint{6.300771in}{1.300295in}}{\pgfqpoint{6.303605in}{1.293454in}}{\pgfqpoint{6.308649in}{1.288410in}}%
\pgfpathcurveto{\pgfqpoint{6.313693in}{1.283366in}}{\pgfqpoint{6.320534in}{1.280532in}}{\pgfqpoint{6.327667in}{1.280532in}}%
\pgfpathclose%
\pgfusepath{stroke,fill}%
\end{pgfscope}%
\begin{pgfscope}%
\pgfpathrectangle{\pgfqpoint{4.985294in}{0.500000in}}{\pgfqpoint{1.764706in}{1.700000in}}%
\pgfusepath{clip}%
\pgfsetbuttcap%
\pgfsetroundjoin%
\definecolor{currentfill}{rgb}{0.951650,0.442241,0.302145}%
\pgfsetfillcolor{currentfill}%
\pgfsetlinewidth{0.311001pt}%
\definecolor{currentstroke}{rgb}{1.000000,1.000000,1.000000}%
\pgfsetstrokecolor{currentstroke}%
\pgfsetdash{}{0pt}%
\pgfpathmoveto{\pgfqpoint{5.644689in}{1.776829in}}%
\pgfpathcurveto{\pgfqpoint{5.651822in}{1.776829in}}{\pgfqpoint{5.658664in}{1.779663in}}{\pgfqpoint{5.663707in}{1.784706in}}%
\pgfpathcurveto{\pgfqpoint{5.668751in}{1.789750in}}{\pgfqpoint{5.671585in}{1.796592in}}{\pgfqpoint{5.671585in}{1.803724in}}%
\pgfpathcurveto{\pgfqpoint{5.671585in}{1.810857in}}{\pgfqpoint{5.668751in}{1.817699in}}{\pgfqpoint{5.663707in}{1.822742in}}%
\pgfpathcurveto{\pgfqpoint{5.658664in}{1.827786in}}{\pgfqpoint{5.651822in}{1.830620in}}{\pgfqpoint{5.644689in}{1.830620in}}%
\pgfpathcurveto{\pgfqpoint{5.637557in}{1.830620in}}{\pgfqpoint{5.630715in}{1.827786in}}{\pgfqpoint{5.625671in}{1.822742in}}%
\pgfpathcurveto{\pgfqpoint{5.620628in}{1.817699in}}{\pgfqpoint{5.617794in}{1.810857in}}{\pgfqpoint{5.617794in}{1.803724in}}%
\pgfpathcurveto{\pgfqpoint{5.617794in}{1.796592in}}{\pgfqpoint{5.620628in}{1.789750in}}{\pgfqpoint{5.625671in}{1.784706in}}%
\pgfpathcurveto{\pgfqpoint{5.630715in}{1.779663in}}{\pgfqpoint{5.637557in}{1.776829in}}{\pgfqpoint{5.644689in}{1.776829in}}%
\pgfpathclose%
\pgfusepath{stroke,fill}%
\end{pgfscope}%
\begin{pgfscope}%
\pgfpathrectangle{\pgfqpoint{4.985294in}{0.500000in}}{\pgfqpoint{1.764706in}{1.700000in}}%
\pgfusepath{clip}%
\pgfsetbuttcap%
\pgfsetroundjoin%
\definecolor{currentfill}{rgb}{0.975018,0.868213,0.788710}%
\pgfsetfillcolor{currentfill}%
\pgfsetlinewidth{0.311001pt}%
\definecolor{currentstroke}{rgb}{1.000000,1.000000,1.000000}%
\pgfsetstrokecolor{currentstroke}%
\pgfsetdash{}{0pt}%
\pgfpathmoveto{\pgfqpoint{5.488415in}{1.074860in}}%
\pgfpathcurveto{\pgfqpoint{5.495548in}{1.074860in}}{\pgfqpoint{5.502390in}{1.077694in}}{\pgfqpoint{5.507434in}{1.082738in}}%
\pgfpathcurveto{\pgfqpoint{5.512477in}{1.087782in}}{\pgfqpoint{5.515311in}{1.094623in}}{\pgfqpoint{5.515311in}{1.101756in}}%
\pgfpathcurveto{\pgfqpoint{5.515311in}{1.108889in}}{\pgfqpoint{5.512477in}{1.115731in}}{\pgfqpoint{5.507434in}{1.120774in}}%
\pgfpathcurveto{\pgfqpoint{5.502390in}{1.125818in}}{\pgfqpoint{5.495548in}{1.128652in}}{\pgfqpoint{5.488415in}{1.128652in}}%
\pgfpathcurveto{\pgfqpoint{5.481283in}{1.128652in}}{\pgfqpoint{5.474441in}{1.125818in}}{\pgfqpoint{5.469397in}{1.120774in}}%
\pgfpathcurveto{\pgfqpoint{5.464354in}{1.115731in}}{\pgfqpoint{5.461520in}{1.108889in}}{\pgfqpoint{5.461520in}{1.101756in}}%
\pgfpathcurveto{\pgfqpoint{5.461520in}{1.094623in}}{\pgfqpoint{5.464354in}{1.087782in}}{\pgfqpoint{5.469397in}{1.082738in}}%
\pgfpathcurveto{\pgfqpoint{5.474441in}{1.077694in}}{\pgfqpoint{5.481283in}{1.074860in}}{\pgfqpoint{5.488415in}{1.074860in}}%
\pgfpathclose%
\pgfusepath{stroke,fill}%
\end{pgfscope}%
\begin{pgfscope}%
\pgfpathrectangle{\pgfqpoint{4.985294in}{0.500000in}}{\pgfqpoint{1.764706in}{1.700000in}}%
\pgfusepath{clip}%
\pgfsetbuttcap%
\pgfsetroundjoin%
\definecolor{currentfill}{rgb}{0.979891,0.908948,0.848279}%
\pgfsetfillcolor{currentfill}%
\pgfsetlinewidth{0.311001pt}%
\definecolor{currentstroke}{rgb}{1.000000,1.000000,1.000000}%
\pgfsetstrokecolor{currentstroke}%
\pgfsetdash{}{0pt}%
\pgfpathmoveto{\pgfqpoint{6.321714in}{1.445073in}}%
\pgfpathcurveto{\pgfqpoint{6.328847in}{1.445073in}}{\pgfqpoint{6.335688in}{1.447906in}}{\pgfqpoint{6.340732in}{1.452950in}}%
\pgfpathcurveto{\pgfqpoint{6.345776in}{1.457994in}}{\pgfqpoint{6.348609in}{1.464835in}}{\pgfqpoint{6.348609in}{1.471968in}}%
\pgfpathcurveto{\pgfqpoint{6.348609in}{1.479101in}}{\pgfqpoint{6.345776in}{1.485943in}}{\pgfqpoint{6.340732in}{1.490986in}}%
\pgfpathcurveto{\pgfqpoint{6.335688in}{1.496030in}}{\pgfqpoint{6.328847in}{1.498864in}}{\pgfqpoint{6.321714in}{1.498864in}}%
\pgfpathcurveto{\pgfqpoint{6.314581in}{1.498864in}}{\pgfqpoint{6.307739in}{1.496030in}}{\pgfqpoint{6.302696in}{1.490986in}}%
\pgfpathcurveto{\pgfqpoint{6.297652in}{1.485943in}}{\pgfqpoint{6.294818in}{1.479101in}}{\pgfqpoint{6.294818in}{1.471968in}}%
\pgfpathcurveto{\pgfqpoint{6.294818in}{1.464835in}}{\pgfqpoint{6.297652in}{1.457994in}}{\pgfqpoint{6.302696in}{1.452950in}}%
\pgfpathcurveto{\pgfqpoint{6.307739in}{1.447906in}}{\pgfqpoint{6.314581in}{1.445073in}}{\pgfqpoint{6.321714in}{1.445073in}}%
\pgfpathclose%
\pgfusepath{stroke,fill}%
\end{pgfscope}%
\begin{pgfscope}%
\pgfpathrectangle{\pgfqpoint{4.985294in}{0.500000in}}{\pgfqpoint{1.764706in}{1.700000in}}%
\pgfusepath{clip}%
\pgfsetbuttcap%
\pgfsetroundjoin%
\definecolor{currentfill}{rgb}{0.960421,0.553286,0.393191}%
\pgfsetfillcolor{currentfill}%
\pgfsetlinewidth{0.311001pt}%
\definecolor{currentstroke}{rgb}{1.000000,1.000000,1.000000}%
\pgfsetstrokecolor{currentstroke}%
\pgfsetdash{}{0pt}%
\pgfpathmoveto{\pgfqpoint{6.129092in}{1.114444in}}%
\pgfpathcurveto{\pgfqpoint{6.136225in}{1.114444in}}{\pgfqpoint{6.143067in}{1.117278in}}{\pgfqpoint{6.148111in}{1.122322in}}%
\pgfpathcurveto{\pgfqpoint{6.153154in}{1.127366in}}{\pgfqpoint{6.155988in}{1.134207in}}{\pgfqpoint{6.155988in}{1.141340in}}%
\pgfpathcurveto{\pgfqpoint{6.155988in}{1.148473in}}{\pgfqpoint{6.153154in}{1.155315in}}{\pgfqpoint{6.148111in}{1.160358in}}%
\pgfpathcurveto{\pgfqpoint{6.143067in}{1.165402in}}{\pgfqpoint{6.136225in}{1.168236in}}{\pgfqpoint{6.129092in}{1.168236in}}%
\pgfpathcurveto{\pgfqpoint{6.121960in}{1.168236in}}{\pgfqpoint{6.115118in}{1.165402in}}{\pgfqpoint{6.110074in}{1.160358in}}%
\pgfpathcurveto{\pgfqpoint{6.105031in}{1.155315in}}{\pgfqpoint{6.102197in}{1.148473in}}{\pgfqpoint{6.102197in}{1.141340in}}%
\pgfpathcurveto{\pgfqpoint{6.102197in}{1.134207in}}{\pgfqpoint{6.105031in}{1.127366in}}{\pgfqpoint{6.110074in}{1.122322in}}%
\pgfpathcurveto{\pgfqpoint{6.115118in}{1.117278in}}{\pgfqpoint{6.121960in}{1.114444in}}{\pgfqpoint{6.129092in}{1.114444in}}%
\pgfpathclose%
\pgfusepath{stroke,fill}%
\end{pgfscope}%
\begin{pgfscope}%
\pgfpathrectangle{\pgfqpoint{4.985294in}{0.500000in}}{\pgfqpoint{1.764706in}{1.700000in}}%
\pgfusepath{clip}%
\pgfsetbuttcap%
\pgfsetroundjoin%
\definecolor{currentfill}{rgb}{0.966328,0.750560,0.616961}%
\pgfsetfillcolor{currentfill}%
\pgfsetlinewidth{0.311001pt}%
\definecolor{currentstroke}{rgb}{1.000000,1.000000,1.000000}%
\pgfsetstrokecolor{currentstroke}%
\pgfsetdash{}{0pt}%
\pgfpathmoveto{\pgfqpoint{5.568716in}{1.627130in}}%
\pgfpathcurveto{\pgfqpoint{5.575848in}{1.627130in}}{\pgfqpoint{5.582690in}{1.629963in}}{\pgfqpoint{5.587734in}{1.635007in}}%
\pgfpathcurveto{\pgfqpoint{5.592777in}{1.640051in}}{\pgfqpoint{5.595611in}{1.646892in}}{\pgfqpoint{5.595611in}{1.654025in}}%
\pgfpathcurveto{\pgfqpoint{5.595611in}{1.661158in}}{\pgfqpoint{5.592777in}{1.668000in}}{\pgfqpoint{5.587734in}{1.673043in}}%
\pgfpathcurveto{\pgfqpoint{5.582690in}{1.678087in}}{\pgfqpoint{5.575848in}{1.680921in}}{\pgfqpoint{5.568716in}{1.680921in}}%
\pgfpathcurveto{\pgfqpoint{5.561583in}{1.680921in}}{\pgfqpoint{5.554741in}{1.678087in}}{\pgfqpoint{5.549697in}{1.673043in}}%
\pgfpathcurveto{\pgfqpoint{5.544654in}{1.668000in}}{\pgfqpoint{5.541820in}{1.661158in}}{\pgfqpoint{5.541820in}{1.654025in}}%
\pgfpathcurveto{\pgfqpoint{5.541820in}{1.646892in}}{\pgfqpoint{5.544654in}{1.640051in}}{\pgfqpoint{5.549697in}{1.635007in}}%
\pgfpathcurveto{\pgfqpoint{5.554741in}{1.629963in}}{\pgfqpoint{5.561583in}{1.627130in}}{\pgfqpoint{5.568716in}{1.627130in}}%
\pgfpathclose%
\pgfusepath{stroke,fill}%
\end{pgfscope}%
\begin{pgfscope}%
\pgfpathrectangle{\pgfqpoint{4.985294in}{0.500000in}}{\pgfqpoint{1.764706in}{1.700000in}}%
\pgfusepath{clip}%
\pgfsetbuttcap%
\pgfsetroundjoin%
\definecolor{currentfill}{rgb}{0.967735,0.780441,0.659127}%
\pgfsetfillcolor{currentfill}%
\pgfsetlinewidth{0.311001pt}%
\definecolor{currentstroke}{rgb}{1.000000,1.000000,1.000000}%
\pgfsetstrokecolor{currentstroke}%
\pgfsetdash{}{0pt}%
\pgfpathmoveto{\pgfqpoint{6.353087in}{1.521984in}}%
\pgfpathcurveto{\pgfqpoint{6.360220in}{1.521984in}}{\pgfqpoint{6.367062in}{1.524818in}}{\pgfqpoint{6.372105in}{1.529862in}}%
\pgfpathcurveto{\pgfqpoint{6.377149in}{1.534906in}}{\pgfqpoint{6.379983in}{1.541747in}}{\pgfqpoint{6.379983in}{1.548880in}}%
\pgfpathcurveto{\pgfqpoint{6.379983in}{1.556013in}}{\pgfqpoint{6.377149in}{1.562854in}}{\pgfqpoint{6.372105in}{1.567898in}}%
\pgfpathcurveto{\pgfqpoint{6.367062in}{1.572942in}}{\pgfqpoint{6.360220in}{1.575776in}}{\pgfqpoint{6.353087in}{1.575776in}}%
\pgfpathcurveto{\pgfqpoint{6.345954in}{1.575776in}}{\pgfqpoint{6.339113in}{1.572942in}}{\pgfqpoint{6.334069in}{1.567898in}}%
\pgfpathcurveto{\pgfqpoint{6.329025in}{1.562854in}}{\pgfqpoint{6.326192in}{1.556013in}}{\pgfqpoint{6.326192in}{1.548880in}}%
\pgfpathcurveto{\pgfqpoint{6.326192in}{1.541747in}}{\pgfqpoint{6.329025in}{1.534906in}}{\pgfqpoint{6.334069in}{1.529862in}}%
\pgfpathcurveto{\pgfqpoint{6.339113in}{1.524818in}}{\pgfqpoint{6.345954in}{1.521984in}}{\pgfqpoint{6.353087in}{1.521984in}}%
\pgfpathclose%
\pgfusepath{stroke,fill}%
\end{pgfscope}%
\begin{pgfscope}%
\pgfpathrectangle{\pgfqpoint{4.985294in}{0.500000in}}{\pgfqpoint{1.764706in}{1.700000in}}%
\pgfusepath{clip}%
\pgfsetbuttcap%
\pgfsetroundjoin%
\definecolor{currentfill}{rgb}{0.965042,0.701564,0.552889}%
\pgfsetfillcolor{currentfill}%
\pgfsetlinewidth{0.311001pt}%
\definecolor{currentstroke}{rgb}{1.000000,1.000000,1.000000}%
\pgfsetstrokecolor{currentstroke}%
\pgfsetdash{}{0pt}%
\pgfpathmoveto{\pgfqpoint{5.520802in}{0.902437in}}%
\pgfpathcurveto{\pgfqpoint{5.527934in}{0.902437in}}{\pgfqpoint{5.534776in}{0.905271in}}{\pgfqpoint{5.539820in}{0.910314in}}%
\pgfpathcurveto{\pgfqpoint{5.544863in}{0.915358in}}{\pgfqpoint{5.547697in}{0.922200in}}{\pgfqpoint{5.547697in}{0.929332in}}%
\pgfpathcurveto{\pgfqpoint{5.547697in}{0.936465in}}{\pgfqpoint{5.544863in}{0.943307in}}{\pgfqpoint{5.539820in}{0.948350in}}%
\pgfpathcurveto{\pgfqpoint{5.534776in}{0.953394in}}{\pgfqpoint{5.527934in}{0.956228in}}{\pgfqpoint{5.520802in}{0.956228in}}%
\pgfpathcurveto{\pgfqpoint{5.513669in}{0.956228in}}{\pgfqpoint{5.506827in}{0.953394in}}{\pgfqpoint{5.501784in}{0.948350in}}%
\pgfpathcurveto{\pgfqpoint{5.496740in}{0.943307in}}{\pgfqpoint{5.493906in}{0.936465in}}{\pgfqpoint{5.493906in}{0.929332in}}%
\pgfpathcurveto{\pgfqpoint{5.493906in}{0.922200in}}{\pgfqpoint{5.496740in}{0.915358in}}{\pgfqpoint{5.501784in}{0.910314in}}%
\pgfpathcurveto{\pgfqpoint{5.506827in}{0.905271in}}{\pgfqpoint{5.513669in}{0.902437in}}{\pgfqpoint{5.520802in}{0.902437in}}%
\pgfpathclose%
\pgfusepath{stroke,fill}%
\end{pgfscope}%
\begin{pgfscope}%
\pgfpathrectangle{\pgfqpoint{4.985294in}{0.500000in}}{\pgfqpoint{1.764706in}{1.700000in}}%
\pgfusepath{clip}%
\pgfsetbuttcap%
\pgfsetroundjoin%
\definecolor{currentfill}{rgb}{0.979891,0.908948,0.848279}%
\pgfsetfillcolor{currentfill}%
\pgfsetlinewidth{0.311001pt}%
\definecolor{currentstroke}{rgb}{1.000000,1.000000,1.000000}%
\pgfsetstrokecolor{currentstroke}%
\pgfsetdash{}{0pt}%
\pgfpathmoveto{\pgfqpoint{6.284954in}{1.504717in}}%
\pgfpathcurveto{\pgfqpoint{6.292087in}{1.504717in}}{\pgfqpoint{6.298929in}{1.507551in}}{\pgfqpoint{6.303972in}{1.512595in}}%
\pgfpathcurveto{\pgfqpoint{6.309016in}{1.517639in}}{\pgfqpoint{6.311850in}{1.524480in}}{\pgfqpoint{6.311850in}{1.531613in}}%
\pgfpathcurveto{\pgfqpoint{6.311850in}{1.538746in}}{\pgfqpoint{6.309016in}{1.545588in}}{\pgfqpoint{6.303972in}{1.550631in}}%
\pgfpathcurveto{\pgfqpoint{6.298929in}{1.555675in}}{\pgfqpoint{6.292087in}{1.558509in}}{\pgfqpoint{6.284954in}{1.558509in}}%
\pgfpathcurveto{\pgfqpoint{6.277821in}{1.558509in}}{\pgfqpoint{6.270980in}{1.555675in}}{\pgfqpoint{6.265936in}{1.550631in}}%
\pgfpathcurveto{\pgfqpoint{6.260892in}{1.545588in}}{\pgfqpoint{6.258058in}{1.538746in}}{\pgfqpoint{6.258058in}{1.531613in}}%
\pgfpathcurveto{\pgfqpoint{6.258058in}{1.524480in}}{\pgfqpoint{6.260892in}{1.517639in}}{\pgfqpoint{6.265936in}{1.512595in}}%
\pgfpathcurveto{\pgfqpoint{6.270980in}{1.507551in}}{\pgfqpoint{6.277821in}{1.504717in}}{\pgfqpoint{6.284954in}{1.504717in}}%
\pgfpathclose%
\pgfusepath{stroke,fill}%
\end{pgfscope}%
\begin{pgfscope}%
\pgfpathrectangle{\pgfqpoint{4.985294in}{0.500000in}}{\pgfqpoint{1.764706in}{1.700000in}}%
\pgfusepath{clip}%
\pgfsetbuttcap%
\pgfsetroundjoin%
\definecolor{currentfill}{rgb}{0.972201,0.839051,0.745789}%
\pgfsetfillcolor{currentfill}%
\pgfsetlinewidth{0.311001pt}%
\definecolor{currentstroke}{rgb}{1.000000,1.000000,1.000000}%
\pgfsetstrokecolor{currentstroke}%
\pgfsetdash{}{0pt}%
\pgfpathmoveto{\pgfqpoint{5.413315in}{1.088541in}}%
\pgfpathcurveto{\pgfqpoint{5.420447in}{1.088541in}}{\pgfqpoint{5.427289in}{1.091375in}}{\pgfqpoint{5.432333in}{1.096418in}}%
\pgfpathcurveto{\pgfqpoint{5.437376in}{1.101462in}}{\pgfqpoint{5.440210in}{1.108304in}}{\pgfqpoint{5.440210in}{1.115436in}}%
\pgfpathcurveto{\pgfqpoint{5.440210in}{1.122569in}}{\pgfqpoint{5.437376in}{1.129411in}}{\pgfqpoint{5.432333in}{1.134455in}}%
\pgfpathcurveto{\pgfqpoint{5.427289in}{1.139498in}}{\pgfqpoint{5.420447in}{1.142332in}}{\pgfqpoint{5.413315in}{1.142332in}}%
\pgfpathcurveto{\pgfqpoint{5.406182in}{1.142332in}}{\pgfqpoint{5.399340in}{1.139498in}}{\pgfqpoint{5.394297in}{1.134455in}}%
\pgfpathcurveto{\pgfqpoint{5.389253in}{1.129411in}}{\pgfqpoint{5.386419in}{1.122569in}}{\pgfqpoint{5.386419in}{1.115436in}}%
\pgfpathcurveto{\pgfqpoint{5.386419in}{1.108304in}}{\pgfqpoint{5.389253in}{1.101462in}}{\pgfqpoint{5.394297in}{1.096418in}}%
\pgfpathcurveto{\pgfqpoint{5.399340in}{1.091375in}}{\pgfqpoint{5.406182in}{1.088541in}}{\pgfqpoint{5.413315in}{1.088541in}}%
\pgfpathclose%
\pgfusepath{stroke,fill}%
\end{pgfscope}%
\begin{pgfscope}%
\pgfpathrectangle{\pgfqpoint{4.985294in}{0.500000in}}{\pgfqpoint{1.764706in}{1.700000in}}%
\pgfusepath{clip}%
\pgfsetbuttcap%
\pgfsetroundjoin%
\definecolor{currentfill}{rgb}{0.962018,0.586477,0.424918}%
\pgfsetfillcolor{currentfill}%
\pgfsetlinewidth{0.311001pt}%
\definecolor{currentstroke}{rgb}{1.000000,1.000000,1.000000}%
\pgfsetstrokecolor{currentstroke}%
\pgfsetdash{}{0pt}%
\pgfpathmoveto{\pgfqpoint{6.159784in}{1.451380in}}%
\pgfpathcurveto{\pgfqpoint{6.166917in}{1.451380in}}{\pgfqpoint{6.173759in}{1.454214in}}{\pgfqpoint{6.178803in}{1.459257in}}%
\pgfpathcurveto{\pgfqpoint{6.183846in}{1.464301in}}{\pgfqpoint{6.186680in}{1.471143in}}{\pgfqpoint{6.186680in}{1.478275in}}%
\pgfpathcurveto{\pgfqpoint{6.186680in}{1.485408in}}{\pgfqpoint{6.183846in}{1.492250in}}{\pgfqpoint{6.178803in}{1.497294in}}%
\pgfpathcurveto{\pgfqpoint{6.173759in}{1.502337in}}{\pgfqpoint{6.166917in}{1.505171in}}{\pgfqpoint{6.159784in}{1.505171in}}%
\pgfpathcurveto{\pgfqpoint{6.152652in}{1.505171in}}{\pgfqpoint{6.145810in}{1.502337in}}{\pgfqpoint{6.140766in}{1.497294in}}%
\pgfpathcurveto{\pgfqpoint{6.135723in}{1.492250in}}{\pgfqpoint{6.132889in}{1.485408in}}{\pgfqpoint{6.132889in}{1.478275in}}%
\pgfpathcurveto{\pgfqpoint{6.132889in}{1.471143in}}{\pgfqpoint{6.135723in}{1.464301in}}{\pgfqpoint{6.140766in}{1.459257in}}%
\pgfpathcurveto{\pgfqpoint{6.145810in}{1.454214in}}{\pgfqpoint{6.152652in}{1.451380in}}{\pgfqpoint{6.159784in}{1.451380in}}%
\pgfpathclose%
\pgfusepath{stroke,fill}%
\end{pgfscope}%
\begin{pgfscope}%
\pgfpathrectangle{\pgfqpoint{4.985294in}{0.500000in}}{\pgfqpoint{1.764706in}{1.700000in}}%
\pgfusepath{clip}%
\pgfsetbuttcap%
\pgfsetroundjoin%
\definecolor{currentfill}{rgb}{0.968105,0.786346,0.667739}%
\pgfsetfillcolor{currentfill}%
\pgfsetlinewidth{0.311001pt}%
\definecolor{currentstroke}{rgb}{1.000000,1.000000,1.000000}%
\pgfsetstrokecolor{currentstroke}%
\pgfsetdash{}{0pt}%
\pgfpathmoveto{\pgfqpoint{6.391320in}{1.330049in}}%
\pgfpathcurveto{\pgfqpoint{6.398453in}{1.330049in}}{\pgfqpoint{6.405295in}{1.332883in}}{\pgfqpoint{6.410338in}{1.337927in}}%
\pgfpathcurveto{\pgfqpoint{6.415382in}{1.342971in}}{\pgfqpoint{6.418216in}{1.349812in}}{\pgfqpoint{6.418216in}{1.356945in}}%
\pgfpathcurveto{\pgfqpoint{6.418216in}{1.364078in}}{\pgfqpoint{6.415382in}{1.370920in}}{\pgfqpoint{6.410338in}{1.375963in}}%
\pgfpathcurveto{\pgfqpoint{6.405295in}{1.381007in}}{\pgfqpoint{6.398453in}{1.383841in}}{\pgfqpoint{6.391320in}{1.383841in}}%
\pgfpathcurveto{\pgfqpoint{6.384187in}{1.383841in}}{\pgfqpoint{6.377346in}{1.381007in}}{\pgfqpoint{6.372302in}{1.375963in}}%
\pgfpathcurveto{\pgfqpoint{6.367258in}{1.370920in}}{\pgfqpoint{6.364424in}{1.364078in}}{\pgfqpoint{6.364424in}{1.356945in}}%
\pgfpathcurveto{\pgfqpoint{6.364424in}{1.349812in}}{\pgfqpoint{6.367258in}{1.342971in}}{\pgfqpoint{6.372302in}{1.337927in}}%
\pgfpathcurveto{\pgfqpoint{6.377346in}{1.332883in}}{\pgfqpoint{6.384187in}{1.330049in}}{\pgfqpoint{6.391320in}{1.330049in}}%
\pgfpathclose%
\pgfusepath{stroke,fill}%
\end{pgfscope}%
\begin{pgfscope}%
\pgfpathrectangle{\pgfqpoint{4.985294in}{0.500000in}}{\pgfqpoint{1.764706in}{1.700000in}}%
\pgfusepath{clip}%
\pgfsetbuttcap%
\pgfsetroundjoin%
\definecolor{currentfill}{rgb}{0.908486,0.245685,0.245983}%
\pgfsetfillcolor{currentfill}%
\pgfsetlinewidth{0.311001pt}%
\definecolor{currentstroke}{rgb}{1.000000,1.000000,1.000000}%
\pgfsetstrokecolor{currentstroke}%
\pgfsetdash{}{0pt}%
\pgfpathmoveto{\pgfqpoint{5.261407in}{1.416670in}}%
\pgfpathcurveto{\pgfqpoint{5.268540in}{1.416670in}}{\pgfqpoint{5.275382in}{1.419504in}}{\pgfqpoint{5.280425in}{1.424548in}}%
\pgfpathcurveto{\pgfqpoint{5.285469in}{1.429591in}}{\pgfqpoint{5.288303in}{1.436433in}}{\pgfqpoint{5.288303in}{1.443566in}}%
\pgfpathcurveto{\pgfqpoint{5.288303in}{1.450698in}}{\pgfqpoint{5.285469in}{1.457540in}}{\pgfqpoint{5.280425in}{1.462584in}}%
\pgfpathcurveto{\pgfqpoint{5.275382in}{1.467627in}}{\pgfqpoint{5.268540in}{1.470461in}}{\pgfqpoint{5.261407in}{1.470461in}}%
\pgfpathcurveto{\pgfqpoint{5.254274in}{1.470461in}}{\pgfqpoint{5.247433in}{1.467627in}}{\pgfqpoint{5.242389in}{1.462584in}}%
\pgfpathcurveto{\pgfqpoint{5.237345in}{1.457540in}}{\pgfqpoint{5.234511in}{1.450698in}}{\pgfqpoint{5.234511in}{1.443566in}}%
\pgfpathcurveto{\pgfqpoint{5.234511in}{1.436433in}}{\pgfqpoint{5.237345in}{1.429591in}}{\pgfqpoint{5.242389in}{1.424548in}}%
\pgfpathcurveto{\pgfqpoint{5.247433in}{1.419504in}}{\pgfqpoint{5.254274in}{1.416670in}}{\pgfqpoint{5.261407in}{1.416670in}}%
\pgfpathclose%
\pgfusepath{stroke,fill}%
\end{pgfscope}%
\begin{pgfscope}%
\pgfpathrectangle{\pgfqpoint{4.985294in}{0.500000in}}{\pgfqpoint{1.764706in}{1.700000in}}%
\pgfusepath{clip}%
\pgfsetbuttcap%
\pgfsetroundjoin%
\definecolor{currentfill}{rgb}{0.976287,0.879862,0.805788}%
\pgfsetfillcolor{currentfill}%
\pgfsetlinewidth{0.311001pt}%
\definecolor{currentstroke}{rgb}{1.000000,1.000000,1.000000}%
\pgfsetstrokecolor{currentstroke}%
\pgfsetdash{}{0pt}%
\pgfpathmoveto{\pgfqpoint{6.350490in}{1.390077in}}%
\pgfpathcurveto{\pgfqpoint{6.357623in}{1.390077in}}{\pgfqpoint{6.364465in}{1.392911in}}{\pgfqpoint{6.369509in}{1.397955in}}%
\pgfpathcurveto{\pgfqpoint{6.374552in}{1.402998in}}{\pgfqpoint{6.377386in}{1.409840in}}{\pgfqpoint{6.377386in}{1.416973in}}%
\pgfpathcurveto{\pgfqpoint{6.377386in}{1.424106in}}{\pgfqpoint{6.374552in}{1.430947in}}{\pgfqpoint{6.369509in}{1.435991in}}%
\pgfpathcurveto{\pgfqpoint{6.364465in}{1.441035in}}{\pgfqpoint{6.357623in}{1.443869in}}{\pgfqpoint{6.350490in}{1.443869in}}%
\pgfpathcurveto{\pgfqpoint{6.343358in}{1.443869in}}{\pgfqpoint{6.336516in}{1.441035in}}{\pgfqpoint{6.331472in}{1.435991in}}%
\pgfpathcurveto{\pgfqpoint{6.326429in}{1.430947in}}{\pgfqpoint{6.323595in}{1.424106in}}{\pgfqpoint{6.323595in}{1.416973in}}%
\pgfpathcurveto{\pgfqpoint{6.323595in}{1.409840in}}{\pgfqpoint{6.326429in}{1.402998in}}{\pgfqpoint{6.331472in}{1.397955in}}%
\pgfpathcurveto{\pgfqpoint{6.336516in}{1.392911in}}{\pgfqpoint{6.343358in}{1.390077in}}{\pgfqpoint{6.350490in}{1.390077in}}%
\pgfpathclose%
\pgfusepath{stroke,fill}%
\end{pgfscope}%
\begin{pgfscope}%
\pgfpathrectangle{\pgfqpoint{4.985294in}{0.500000in}}{\pgfqpoint{1.764706in}{1.700000in}}%
\pgfusepath{clip}%
\pgfsetbuttcap%
\pgfsetroundjoin%
\definecolor{currentfill}{rgb}{0.975018,0.868213,0.788710}%
\pgfsetfillcolor{currentfill}%
\pgfsetlinewidth{0.311001pt}%
\definecolor{currentstroke}{rgb}{1.000000,1.000000,1.000000}%
\pgfsetstrokecolor{currentstroke}%
\pgfsetdash{}{0pt}%
\pgfpathmoveto{\pgfqpoint{5.400853in}{1.159472in}}%
\pgfpathcurveto{\pgfqpoint{5.407986in}{1.159472in}}{\pgfqpoint{5.414827in}{1.162306in}}{\pgfqpoint{5.419871in}{1.167350in}}%
\pgfpathcurveto{\pgfqpoint{5.424915in}{1.172393in}}{\pgfqpoint{5.427749in}{1.179235in}}{\pgfqpoint{5.427749in}{1.186368in}}%
\pgfpathcurveto{\pgfqpoint{5.427749in}{1.193501in}}{\pgfqpoint{5.424915in}{1.200342in}}{\pgfqpoint{5.419871in}{1.205386in}}%
\pgfpathcurveto{\pgfqpoint{5.414827in}{1.210430in}}{\pgfqpoint{5.407986in}{1.213264in}}{\pgfqpoint{5.400853in}{1.213264in}}%
\pgfpathcurveto{\pgfqpoint{5.393720in}{1.213264in}}{\pgfqpoint{5.386878in}{1.210430in}}{\pgfqpoint{5.381835in}{1.205386in}}%
\pgfpathcurveto{\pgfqpoint{5.376791in}{1.200342in}}{\pgfqpoint{5.373957in}{1.193501in}}{\pgfqpoint{5.373957in}{1.186368in}}%
\pgfpathcurveto{\pgfqpoint{5.373957in}{1.179235in}}{\pgfqpoint{5.376791in}{1.172393in}}{\pgfqpoint{5.381835in}{1.167350in}}%
\pgfpathcurveto{\pgfqpoint{5.386878in}{1.162306in}}{\pgfqpoint{5.393720in}{1.159472in}}{\pgfqpoint{5.400853in}{1.159472in}}%
\pgfpathclose%
\pgfusepath{stroke,fill}%
\end{pgfscope}%
\begin{pgfscope}%
\pgfpathrectangle{\pgfqpoint{4.985294in}{0.500000in}}{\pgfqpoint{1.764706in}{1.700000in}}%
\pgfusepath{clip}%
\pgfsetbuttcap%
\pgfsetroundjoin%
\definecolor{currentfill}{rgb}{0.967092,0.768560,0.642079}%
\pgfsetfillcolor{currentfill}%
\pgfsetlinewidth{0.311001pt}%
\definecolor{currentstroke}{rgb}{1.000000,1.000000,1.000000}%
\pgfsetstrokecolor{currentstroke}%
\pgfsetdash{}{0pt}%
\pgfpathmoveto{\pgfqpoint{5.573990in}{0.949655in}}%
\pgfpathcurveto{\pgfqpoint{5.581122in}{0.949655in}}{\pgfqpoint{5.587964in}{0.952489in}}{\pgfqpoint{5.593008in}{0.957533in}}%
\pgfpathcurveto{\pgfqpoint{5.598051in}{0.962577in}}{\pgfqpoint{5.600885in}{0.969418in}}{\pgfqpoint{5.600885in}{0.976551in}}%
\pgfpathcurveto{\pgfqpoint{5.600885in}{0.983684in}}{\pgfqpoint{5.598051in}{0.990526in}}{\pgfqpoint{5.593008in}{0.995569in}}%
\pgfpathcurveto{\pgfqpoint{5.587964in}{1.000613in}}{\pgfqpoint{5.581122in}{1.003447in}}{\pgfqpoint{5.573990in}{1.003447in}}%
\pgfpathcurveto{\pgfqpoint{5.566857in}{1.003447in}}{\pgfqpoint{5.560015in}{1.000613in}}{\pgfqpoint{5.554972in}{0.995569in}}%
\pgfpathcurveto{\pgfqpoint{5.549928in}{0.990526in}}{\pgfqpoint{5.547094in}{0.983684in}}{\pgfqpoint{5.547094in}{0.976551in}}%
\pgfpathcurveto{\pgfqpoint{5.547094in}{0.969418in}}{\pgfqpoint{5.549928in}{0.962577in}}{\pgfqpoint{5.554972in}{0.957533in}}%
\pgfpathcurveto{\pgfqpoint{5.560015in}{0.952489in}}{\pgfqpoint{5.566857in}{0.949655in}}{\pgfqpoint{5.573990in}{0.949655in}}%
\pgfpathclose%
\pgfusepath{stroke,fill}%
\end{pgfscope}%
\begin{pgfscope}%
\pgfpathrectangle{\pgfqpoint{4.985294in}{0.500000in}}{\pgfqpoint{1.764706in}{1.700000in}}%
\pgfusepath{clip}%
\pgfsetbuttcap%
\pgfsetroundjoin%
\definecolor{currentfill}{rgb}{0.958331,0.519463,0.362986}%
\pgfsetfillcolor{currentfill}%
\pgfsetlinewidth{0.311001pt}%
\definecolor{currentstroke}{rgb}{1.000000,1.000000,1.000000}%
\pgfsetstrokecolor{currentstroke}%
\pgfsetdash{}{0pt}%
\pgfpathmoveto{\pgfqpoint{5.294095in}{1.224311in}}%
\pgfpathcurveto{\pgfqpoint{5.301228in}{1.224311in}}{\pgfqpoint{5.308070in}{1.227145in}}{\pgfqpoint{5.313113in}{1.232189in}}%
\pgfpathcurveto{\pgfqpoint{5.318157in}{1.237233in}}{\pgfqpoint{5.320991in}{1.244074in}}{\pgfqpoint{5.320991in}{1.251207in}}%
\pgfpathcurveto{\pgfqpoint{5.320991in}{1.258340in}}{\pgfqpoint{5.318157in}{1.265182in}}{\pgfqpoint{5.313113in}{1.270225in}}%
\pgfpathcurveto{\pgfqpoint{5.308070in}{1.275269in}}{\pgfqpoint{5.301228in}{1.278103in}}{\pgfqpoint{5.294095in}{1.278103in}}%
\pgfpathcurveto{\pgfqpoint{5.286962in}{1.278103in}}{\pgfqpoint{5.280121in}{1.275269in}}{\pgfqpoint{5.275077in}{1.270225in}}%
\pgfpathcurveto{\pgfqpoint{5.270033in}{1.265182in}}{\pgfqpoint{5.267200in}{1.258340in}}{\pgfqpoint{5.267200in}{1.251207in}}%
\pgfpathcurveto{\pgfqpoint{5.267200in}{1.244074in}}{\pgfqpoint{5.270033in}{1.237233in}}{\pgfqpoint{5.275077in}{1.232189in}}%
\pgfpathcurveto{\pgfqpoint{5.280121in}{1.227145in}}{\pgfqpoint{5.286962in}{1.224311in}}{\pgfqpoint{5.294095in}{1.224311in}}%
\pgfpathclose%
\pgfusepath{stroke,fill}%
\end{pgfscope}%
\begin{pgfscope}%
\pgfpathrectangle{\pgfqpoint{4.985294in}{0.500000in}}{\pgfqpoint{1.764706in}{1.700000in}}%
\pgfusepath{clip}%
\pgfsetbuttcap%
\pgfsetroundjoin%
\definecolor{currentfill}{rgb}{0.974412,0.862387,0.780156}%
\pgfsetfillcolor{currentfill}%
\pgfsetlinewidth{0.311001pt}%
\definecolor{currentstroke}{rgb}{1.000000,1.000000,1.000000}%
\pgfsetstrokecolor{currentstroke}%
\pgfsetdash{}{0pt}%
\pgfpathmoveto{\pgfqpoint{5.470977in}{1.569105in}}%
\pgfpathcurveto{\pgfqpoint{5.478110in}{1.569105in}}{\pgfqpoint{5.484952in}{1.571939in}}{\pgfqpoint{5.489996in}{1.576983in}}%
\pgfpathcurveto{\pgfqpoint{5.495039in}{1.582026in}}{\pgfqpoint{5.497873in}{1.588868in}}{\pgfqpoint{5.497873in}{1.596001in}}%
\pgfpathcurveto{\pgfqpoint{5.497873in}{1.603134in}}{\pgfqpoint{5.495039in}{1.609975in}}{\pgfqpoint{5.489996in}{1.615019in}}%
\pgfpathcurveto{\pgfqpoint{5.484952in}{1.620063in}}{\pgfqpoint{5.478110in}{1.622896in}}{\pgfqpoint{5.470977in}{1.622896in}}%
\pgfpathcurveto{\pgfqpoint{5.463845in}{1.622896in}}{\pgfqpoint{5.457003in}{1.620063in}}{\pgfqpoint{5.451959in}{1.615019in}}%
\pgfpathcurveto{\pgfqpoint{5.446916in}{1.609975in}}{\pgfqpoint{5.444082in}{1.603134in}}{\pgfqpoint{5.444082in}{1.596001in}}%
\pgfpathcurveto{\pgfqpoint{5.444082in}{1.588868in}}{\pgfqpoint{5.446916in}{1.582026in}}{\pgfqpoint{5.451959in}{1.576983in}}%
\pgfpathcurveto{\pgfqpoint{5.457003in}{1.571939in}}{\pgfqpoint{5.463845in}{1.569105in}}{\pgfqpoint{5.470977in}{1.569105in}}%
\pgfpathclose%
\pgfusepath{stroke,fill}%
\end{pgfscope}%
\begin{pgfscope}%
\pgfpathrectangle{\pgfqpoint{4.985294in}{0.500000in}}{\pgfqpoint{1.764706in}{1.700000in}}%
\pgfusepath{clip}%
\pgfsetbuttcap%
\pgfsetroundjoin%
\definecolor{currentfill}{rgb}{0.980678,0.914765,0.856766}%
\pgfsetfillcolor{currentfill}%
\pgfsetlinewidth{0.311001pt}%
\definecolor{currentstroke}{rgb}{1.000000,1.000000,1.000000}%
\pgfsetstrokecolor{currentstroke}%
\pgfsetdash{}{0pt}%
\pgfpathmoveto{\pgfqpoint{6.285763in}{1.503303in}}%
\pgfpathcurveto{\pgfqpoint{6.292896in}{1.503303in}}{\pgfqpoint{6.299737in}{1.506136in}}{\pgfqpoint{6.304781in}{1.511180in}}%
\pgfpathcurveto{\pgfqpoint{6.309825in}{1.516224in}}{\pgfqpoint{6.312658in}{1.523065in}}{\pgfqpoint{6.312658in}{1.530198in}}%
\pgfpathcurveto{\pgfqpoint{6.312658in}{1.537331in}}{\pgfqpoint{6.309825in}{1.544173in}}{\pgfqpoint{6.304781in}{1.549216in}}%
\pgfpathcurveto{\pgfqpoint{6.299737in}{1.554260in}}{\pgfqpoint{6.292896in}{1.557094in}}{\pgfqpoint{6.285763in}{1.557094in}}%
\pgfpathcurveto{\pgfqpoint{6.278630in}{1.557094in}}{\pgfqpoint{6.271788in}{1.554260in}}{\pgfqpoint{6.266745in}{1.549216in}}%
\pgfpathcurveto{\pgfqpoint{6.261701in}{1.544173in}}{\pgfqpoint{6.258867in}{1.537331in}}{\pgfqpoint{6.258867in}{1.530198in}}%
\pgfpathcurveto{\pgfqpoint{6.258867in}{1.523065in}}{\pgfqpoint{6.261701in}{1.516224in}}{\pgfqpoint{6.266745in}{1.511180in}}%
\pgfpathcurveto{\pgfqpoint{6.271788in}{1.506136in}}{\pgfqpoint{6.278630in}{1.503303in}}{\pgfqpoint{6.285763in}{1.503303in}}%
\pgfpathclose%
\pgfusepath{stroke,fill}%
\end{pgfscope}%
\begin{pgfscope}%
\pgfpathrectangle{\pgfqpoint{4.985294in}{0.500000in}}{\pgfqpoint{1.764706in}{1.700000in}}%
\pgfusepath{clip}%
\pgfsetbuttcap%
\pgfsetroundjoin%
\definecolor{currentfill}{rgb}{0.962018,0.586477,0.424918}%
\pgfsetfillcolor{currentfill}%
\pgfsetlinewidth{0.311001pt}%
\definecolor{currentstroke}{rgb}{1.000000,1.000000,1.000000}%
\pgfsetstrokecolor{currentstroke}%
\pgfsetdash{}{0pt}%
\pgfpathmoveto{\pgfqpoint{6.421481in}{1.427596in}}%
\pgfpathcurveto{\pgfqpoint{6.428614in}{1.427596in}}{\pgfqpoint{6.435455in}{1.430430in}}{\pgfqpoint{6.440499in}{1.435473in}}%
\pgfpathcurveto{\pgfqpoint{6.445543in}{1.440517in}}{\pgfqpoint{6.448376in}{1.447359in}}{\pgfqpoint{6.448376in}{1.454491in}}%
\pgfpathcurveto{\pgfqpoint{6.448376in}{1.461624in}}{\pgfqpoint{6.445543in}{1.468466in}}{\pgfqpoint{6.440499in}{1.473509in}}%
\pgfpathcurveto{\pgfqpoint{6.435455in}{1.478553in}}{\pgfqpoint{6.428614in}{1.481387in}}{\pgfqpoint{6.421481in}{1.481387in}}%
\pgfpathcurveto{\pgfqpoint{6.414348in}{1.481387in}}{\pgfqpoint{6.407506in}{1.478553in}}{\pgfqpoint{6.402463in}{1.473509in}}%
\pgfpathcurveto{\pgfqpoint{6.397419in}{1.468466in}}{\pgfqpoint{6.394585in}{1.461624in}}{\pgfqpoint{6.394585in}{1.454491in}}%
\pgfpathcurveto{\pgfqpoint{6.394585in}{1.447359in}}{\pgfqpoint{6.397419in}{1.440517in}}{\pgfqpoint{6.402463in}{1.435473in}}%
\pgfpathcurveto{\pgfqpoint{6.407506in}{1.430430in}}{\pgfqpoint{6.414348in}{1.427596in}}{\pgfqpoint{6.421481in}{1.427596in}}%
\pgfpathclose%
\pgfusepath{stroke,fill}%
\end{pgfscope}%
\begin{pgfscope}%
\pgfpathrectangle{\pgfqpoint{4.985294in}{0.500000in}}{\pgfqpoint{1.764706in}{1.700000in}}%
\pgfusepath{clip}%
\pgfsetbuttcap%
\pgfsetroundjoin%
\definecolor{currentfill}{rgb}{0.978376,0.897317,0.831308}%
\pgfsetfillcolor{currentfill}%
\pgfsetlinewidth{0.311001pt}%
\definecolor{currentstroke}{rgb}{1.000000,1.000000,1.000000}%
\pgfsetstrokecolor{currentstroke}%
\pgfsetdash{}{0pt}%
\pgfpathmoveto{\pgfqpoint{6.285429in}{1.365108in}}%
\pgfpathcurveto{\pgfqpoint{6.292562in}{1.365108in}}{\pgfqpoint{6.299403in}{1.367942in}}{\pgfqpoint{6.304447in}{1.372985in}}%
\pgfpathcurveto{\pgfqpoint{6.309491in}{1.378029in}}{\pgfqpoint{6.312325in}{1.384871in}}{\pgfqpoint{6.312325in}{1.392004in}}%
\pgfpathcurveto{\pgfqpoint{6.312325in}{1.399136in}}{\pgfqpoint{6.309491in}{1.405978in}}{\pgfqpoint{6.304447in}{1.411022in}}%
\pgfpathcurveto{\pgfqpoint{6.299403in}{1.416065in}}{\pgfqpoint{6.292562in}{1.418899in}}{\pgfqpoint{6.285429in}{1.418899in}}%
\pgfpathcurveto{\pgfqpoint{6.278296in}{1.418899in}}{\pgfqpoint{6.271454in}{1.416065in}}{\pgfqpoint{6.266411in}{1.411022in}}%
\pgfpathcurveto{\pgfqpoint{6.261367in}{1.405978in}}{\pgfqpoint{6.258533in}{1.399136in}}{\pgfqpoint{6.258533in}{1.392004in}}%
\pgfpathcurveto{\pgfqpoint{6.258533in}{1.384871in}}{\pgfqpoint{6.261367in}{1.378029in}}{\pgfqpoint{6.266411in}{1.372985in}}%
\pgfpathcurveto{\pgfqpoint{6.271454in}{1.367942in}}{\pgfqpoint{6.278296in}{1.365108in}}{\pgfqpoint{6.285429in}{1.365108in}}%
\pgfpathclose%
\pgfusepath{stroke,fill}%
\end{pgfscope}%
\begin{pgfscope}%
\pgfpathrectangle{\pgfqpoint{4.985294in}{0.500000in}}{\pgfqpoint{1.764706in}{1.700000in}}%
\pgfusepath{clip}%
\pgfsetbuttcap%
\pgfsetroundjoin%
\definecolor{currentfill}{rgb}{0.974412,0.862387,0.780156}%
\pgfsetfillcolor{currentfill}%
\pgfsetlinewidth{0.311001pt}%
\definecolor{currentstroke}{rgb}{1.000000,1.000000,1.000000}%
\pgfsetstrokecolor{currentstroke}%
\pgfsetdash{}{0pt}%
\pgfpathmoveto{\pgfqpoint{6.365643in}{1.354245in}}%
\pgfpathcurveto{\pgfqpoint{6.372775in}{1.354245in}}{\pgfqpoint{6.379617in}{1.357079in}}{\pgfqpoint{6.384661in}{1.362122in}}%
\pgfpathcurveto{\pgfqpoint{6.389704in}{1.367166in}}{\pgfqpoint{6.392538in}{1.374008in}}{\pgfqpoint{6.392538in}{1.381141in}}%
\pgfpathcurveto{\pgfqpoint{6.392538in}{1.388273in}}{\pgfqpoint{6.389704in}{1.395115in}}{\pgfqpoint{6.384661in}{1.400159in}}%
\pgfpathcurveto{\pgfqpoint{6.379617in}{1.405202in}}{\pgfqpoint{6.372775in}{1.408036in}}{\pgfqpoint{6.365643in}{1.408036in}}%
\pgfpathcurveto{\pgfqpoint{6.358510in}{1.408036in}}{\pgfqpoint{6.351668in}{1.405202in}}{\pgfqpoint{6.346625in}{1.400159in}}%
\pgfpathcurveto{\pgfqpoint{6.341581in}{1.395115in}}{\pgfqpoint{6.338747in}{1.388273in}}{\pgfqpoint{6.338747in}{1.381141in}}%
\pgfpathcurveto{\pgfqpoint{6.338747in}{1.374008in}}{\pgfqpoint{6.341581in}{1.367166in}}{\pgfqpoint{6.346625in}{1.362122in}}%
\pgfpathcurveto{\pgfqpoint{6.351668in}{1.357079in}}{\pgfqpoint{6.358510in}{1.354245in}}{\pgfqpoint{6.365643in}{1.354245in}}%
\pgfpathclose%
\pgfusepath{stroke,fill}%
\end{pgfscope}%
\begin{pgfscope}%
\pgfpathrectangle{\pgfqpoint{4.985294in}{0.500000in}}{\pgfqpoint{1.764706in}{1.700000in}}%
\pgfusepath{clip}%
\pgfsetbuttcap%
\pgfsetroundjoin%
\definecolor{currentfill}{rgb}{0.973271,0.850724,0.762998}%
\pgfsetfillcolor{currentfill}%
\pgfsetlinewidth{0.311001pt}%
\definecolor{currentstroke}{rgb}{1.000000,1.000000,1.000000}%
\pgfsetstrokecolor{currentstroke}%
\pgfsetdash{}{0pt}%
\pgfpathmoveto{\pgfqpoint{6.216672in}{1.111229in}}%
\pgfpathcurveto{\pgfqpoint{6.223804in}{1.111229in}}{\pgfqpoint{6.230646in}{1.114063in}}{\pgfqpoint{6.235690in}{1.119107in}}%
\pgfpathcurveto{\pgfqpoint{6.240733in}{1.124150in}}{\pgfqpoint{6.243567in}{1.130992in}}{\pgfqpoint{6.243567in}{1.138125in}}%
\pgfpathcurveto{\pgfqpoint{6.243567in}{1.145258in}}{\pgfqpoint{6.240733in}{1.152099in}}{\pgfqpoint{6.235690in}{1.157143in}}%
\pgfpathcurveto{\pgfqpoint{6.230646in}{1.162187in}}{\pgfqpoint{6.223804in}{1.165021in}}{\pgfqpoint{6.216672in}{1.165021in}}%
\pgfpathcurveto{\pgfqpoint{6.209539in}{1.165021in}}{\pgfqpoint{6.202697in}{1.162187in}}{\pgfqpoint{6.197653in}{1.157143in}}%
\pgfpathcurveto{\pgfqpoint{6.192610in}{1.152099in}}{\pgfqpoint{6.189776in}{1.145258in}}{\pgfqpoint{6.189776in}{1.138125in}}%
\pgfpathcurveto{\pgfqpoint{6.189776in}{1.130992in}}{\pgfqpoint{6.192610in}{1.124150in}}{\pgfqpoint{6.197653in}{1.119107in}}%
\pgfpathcurveto{\pgfqpoint{6.202697in}{1.114063in}}{\pgfqpoint{6.209539in}{1.111229in}}{\pgfqpoint{6.216672in}{1.111229in}}%
\pgfpathclose%
\pgfusepath{stroke,fill}%
\end{pgfscope}%
\begin{pgfscope}%
\pgfpathrectangle{\pgfqpoint{4.985294in}{0.500000in}}{\pgfqpoint{1.764706in}{1.700000in}}%
\pgfusepath{clip}%
\pgfsetbuttcap%
\pgfsetroundjoin%
\definecolor{currentfill}{rgb}{0.980678,0.914765,0.856766}%
\pgfsetfillcolor{currentfill}%
\pgfsetlinewidth{0.311001pt}%
\definecolor{currentstroke}{rgb}{1.000000,1.000000,1.000000}%
\pgfsetstrokecolor{currentstroke}%
\pgfsetdash{}{0pt}%
\pgfpathmoveto{\pgfqpoint{6.296772in}{1.219692in}}%
\pgfpathcurveto{\pgfqpoint{6.303904in}{1.219692in}}{\pgfqpoint{6.310746in}{1.222526in}}{\pgfqpoint{6.315790in}{1.227569in}}%
\pgfpathcurveto{\pgfqpoint{6.320833in}{1.232613in}}{\pgfqpoint{6.323667in}{1.239454in}}{\pgfqpoint{6.323667in}{1.246587in}}%
\pgfpathcurveto{\pgfqpoint{6.323667in}{1.253720in}}{\pgfqpoint{6.320833in}{1.260562in}}{\pgfqpoint{6.315790in}{1.265605in}}%
\pgfpathcurveto{\pgfqpoint{6.310746in}{1.270649in}}{\pgfqpoint{6.303904in}{1.273483in}}{\pgfqpoint{6.296772in}{1.273483in}}%
\pgfpathcurveto{\pgfqpoint{6.289639in}{1.273483in}}{\pgfqpoint{6.282797in}{1.270649in}}{\pgfqpoint{6.277753in}{1.265605in}}%
\pgfpathcurveto{\pgfqpoint{6.272710in}{1.260562in}}{\pgfqpoint{6.269876in}{1.253720in}}{\pgfqpoint{6.269876in}{1.246587in}}%
\pgfpathcurveto{\pgfqpoint{6.269876in}{1.239454in}}{\pgfqpoint{6.272710in}{1.232613in}}{\pgfqpoint{6.277753in}{1.227569in}}%
\pgfpathcurveto{\pgfqpoint{6.282797in}{1.222526in}}{\pgfqpoint{6.289639in}{1.219692in}}{\pgfqpoint{6.296772in}{1.219692in}}%
\pgfpathclose%
\pgfusepath{stroke,fill}%
\end{pgfscope}%
\begin{pgfscope}%
\pgfpathrectangle{\pgfqpoint{4.985294in}{0.500000in}}{\pgfqpoint{1.764706in}{1.700000in}}%
\pgfusepath{clip}%
\pgfsetbuttcap%
\pgfsetroundjoin%
\definecolor{currentfill}{rgb}{0.969803,0.809811,0.702523}%
\pgfsetfillcolor{currentfill}%
\pgfsetlinewidth{0.311001pt}%
\definecolor{currentstroke}{rgb}{1.000000,1.000000,1.000000}%
\pgfsetstrokecolor{currentstroke}%
\pgfsetdash{}{0pt}%
\pgfpathmoveto{\pgfqpoint{6.355116in}{1.140893in}}%
\pgfpathcurveto{\pgfqpoint{6.362248in}{1.140893in}}{\pgfqpoint{6.369090in}{1.143727in}}{\pgfqpoint{6.374134in}{1.148770in}}%
\pgfpathcurveto{\pgfqpoint{6.379177in}{1.153814in}}{\pgfqpoint{6.382011in}{1.160655in}}{\pgfqpoint{6.382011in}{1.167788in}}%
\pgfpathcurveto{\pgfqpoint{6.382011in}{1.174921in}}{\pgfqpoint{6.379177in}{1.181763in}}{\pgfqpoint{6.374134in}{1.186806in}}%
\pgfpathcurveto{\pgfqpoint{6.369090in}{1.191850in}}{\pgfqpoint{6.362248in}{1.194684in}}{\pgfqpoint{6.355116in}{1.194684in}}%
\pgfpathcurveto{\pgfqpoint{6.347983in}{1.194684in}}{\pgfqpoint{6.341141in}{1.191850in}}{\pgfqpoint{6.336097in}{1.186806in}}%
\pgfpathcurveto{\pgfqpoint{6.331054in}{1.181763in}}{\pgfqpoint{6.328220in}{1.174921in}}{\pgfqpoint{6.328220in}{1.167788in}}%
\pgfpathcurveto{\pgfqpoint{6.328220in}{1.160655in}}{\pgfqpoint{6.331054in}{1.153814in}}{\pgfqpoint{6.336097in}{1.148770in}}%
\pgfpathcurveto{\pgfqpoint{6.341141in}{1.143727in}}{\pgfqpoint{6.347983in}{1.140893in}}{\pgfqpoint{6.355116in}{1.140893in}}%
\pgfpathclose%
\pgfusepath{stroke,fill}%
\end{pgfscope}%
\begin{pgfscope}%
\pgfpathrectangle{\pgfqpoint{4.985294in}{0.500000in}}{\pgfqpoint{1.764706in}{1.700000in}}%
\pgfusepath{clip}%
\pgfsetbuttcap%
\pgfsetroundjoin%
\definecolor{currentfill}{rgb}{0.969359,0.803954,0.693832}%
\pgfsetfillcolor{currentfill}%
\pgfsetlinewidth{0.311001pt}%
\definecolor{currentstroke}{rgb}{1.000000,1.000000,1.000000}%
\pgfsetstrokecolor{currentstroke}%
\pgfsetdash{}{0pt}%
\pgfpathmoveto{\pgfqpoint{6.347428in}{1.115877in}}%
\pgfpathcurveto{\pgfqpoint{6.354561in}{1.115877in}}{\pgfqpoint{6.361403in}{1.118711in}}{\pgfqpoint{6.366446in}{1.123754in}}%
\pgfpathcurveto{\pgfqpoint{6.371490in}{1.128798in}}{\pgfqpoint{6.374324in}{1.135640in}}{\pgfqpoint{6.374324in}{1.142772in}}%
\pgfpathcurveto{\pgfqpoint{6.374324in}{1.149905in}}{\pgfqpoint{6.371490in}{1.156747in}}{\pgfqpoint{6.366446in}{1.161791in}}%
\pgfpathcurveto{\pgfqpoint{6.361403in}{1.166834in}}{\pgfqpoint{6.354561in}{1.169668in}}{\pgfqpoint{6.347428in}{1.169668in}}%
\pgfpathcurveto{\pgfqpoint{6.340295in}{1.169668in}}{\pgfqpoint{6.333454in}{1.166834in}}{\pgfqpoint{6.328410in}{1.161791in}}%
\pgfpathcurveto{\pgfqpoint{6.323366in}{1.156747in}}{\pgfqpoint{6.320532in}{1.149905in}}{\pgfqpoint{6.320532in}{1.142772in}}%
\pgfpathcurveto{\pgfqpoint{6.320532in}{1.135640in}}{\pgfqpoint{6.323366in}{1.128798in}}{\pgfqpoint{6.328410in}{1.123754in}}%
\pgfpathcurveto{\pgfqpoint{6.333454in}{1.118711in}}{\pgfqpoint{6.340295in}{1.115877in}}{\pgfqpoint{6.347428in}{1.115877in}}%
\pgfpathclose%
\pgfusepath{stroke,fill}%
\end{pgfscope}%
\begin{pgfscope}%
\pgfpathrectangle{\pgfqpoint{4.985294in}{0.500000in}}{\pgfqpoint{1.764706in}{1.700000in}}%
\pgfusepath{clip}%
\pgfsetbuttcap%
\pgfsetroundjoin%
\definecolor{currentfill}{rgb}{0.964032,0.651225,0.493258}%
\pgfsetfillcolor{currentfill}%
\pgfsetlinewidth{0.311001pt}%
\definecolor{currentstroke}{rgb}{1.000000,1.000000,1.000000}%
\pgfsetstrokecolor{currentstroke}%
\pgfsetdash{}{0pt}%
\pgfpathmoveto{\pgfqpoint{6.296815in}{0.964171in}}%
\pgfpathcurveto{\pgfqpoint{6.303948in}{0.964171in}}{\pgfqpoint{6.310790in}{0.967004in}}{\pgfqpoint{6.315834in}{0.972048in}}%
\pgfpathcurveto{\pgfqpoint{6.320877in}{0.977092in}}{\pgfqpoint{6.323711in}{0.983933in}}{\pgfqpoint{6.323711in}{0.991066in}}%
\pgfpathcurveto{\pgfqpoint{6.323711in}{0.998199in}}{\pgfqpoint{6.320877in}{1.005041in}}{\pgfqpoint{6.315834in}{1.010084in}}%
\pgfpathcurveto{\pgfqpoint{6.310790in}{1.015128in}}{\pgfqpoint{6.303948in}{1.017962in}}{\pgfqpoint{6.296815in}{1.017962in}}%
\pgfpathcurveto{\pgfqpoint{6.289683in}{1.017962in}}{\pgfqpoint{6.282841in}{1.015128in}}{\pgfqpoint{6.277797in}{1.010084in}}%
\pgfpathcurveto{\pgfqpoint{6.272754in}{1.005041in}}{\pgfqpoint{6.269920in}{0.998199in}}{\pgfqpoint{6.269920in}{0.991066in}}%
\pgfpathcurveto{\pgfqpoint{6.269920in}{0.983933in}}{\pgfqpoint{6.272754in}{0.977092in}}{\pgfqpoint{6.277797in}{0.972048in}}%
\pgfpathcurveto{\pgfqpoint{6.282841in}{0.967004in}}{\pgfqpoint{6.289683in}{0.964171in}}{\pgfqpoint{6.296815in}{0.964171in}}%
\pgfpathclose%
\pgfusepath{stroke,fill}%
\end{pgfscope}%
\begin{pgfscope}%
\pgfpathrectangle{\pgfqpoint{4.985294in}{0.500000in}}{\pgfqpoint{1.764706in}{1.700000in}}%
\pgfusepath{clip}%
\pgfsetbuttcap%
\pgfsetroundjoin%
\definecolor{currentfill}{rgb}{0.964173,0.657587,0.500469}%
\pgfsetfillcolor{currentfill}%
\pgfsetlinewidth{0.311001pt}%
\definecolor{currentstroke}{rgb}{1.000000,1.000000,1.000000}%
\pgfsetstrokecolor{currentstroke}%
\pgfsetdash{}{0pt}%
\pgfpathmoveto{\pgfqpoint{6.161176in}{1.491277in}}%
\pgfpathcurveto{\pgfqpoint{6.168309in}{1.491277in}}{\pgfqpoint{6.175151in}{1.494111in}}{\pgfqpoint{6.180194in}{1.499155in}}%
\pgfpathcurveto{\pgfqpoint{6.185238in}{1.504198in}}{\pgfqpoint{6.188072in}{1.511040in}}{\pgfqpoint{6.188072in}{1.518173in}}%
\pgfpathcurveto{\pgfqpoint{6.188072in}{1.525306in}}{\pgfqpoint{6.185238in}{1.532147in}}{\pgfqpoint{6.180194in}{1.537191in}}%
\pgfpathcurveto{\pgfqpoint{6.175151in}{1.542235in}}{\pgfqpoint{6.168309in}{1.545069in}}{\pgfqpoint{6.161176in}{1.545069in}}%
\pgfpathcurveto{\pgfqpoint{6.154043in}{1.545069in}}{\pgfqpoint{6.147202in}{1.542235in}}{\pgfqpoint{6.142158in}{1.537191in}}%
\pgfpathcurveto{\pgfqpoint{6.137114in}{1.532147in}}{\pgfqpoint{6.134280in}{1.525306in}}{\pgfqpoint{6.134280in}{1.518173in}}%
\pgfpathcurveto{\pgfqpoint{6.134280in}{1.511040in}}{\pgfqpoint{6.137114in}{1.504198in}}{\pgfqpoint{6.142158in}{1.499155in}}%
\pgfpathcurveto{\pgfqpoint{6.147202in}{1.494111in}}{\pgfqpoint{6.154043in}{1.491277in}}{\pgfqpoint{6.161176in}{1.491277in}}%
\pgfpathclose%
\pgfusepath{stroke,fill}%
\end{pgfscope}%
\begin{pgfscope}%
\pgfpathrectangle{\pgfqpoint{4.985294in}{0.500000in}}{\pgfqpoint{1.764706in}{1.700000in}}%
\pgfusepath{clip}%
\pgfsetbuttcap%
\pgfsetroundjoin%
\definecolor{currentfill}{rgb}{0.704578,0.088213,0.344730}%
\pgfsetfillcolor{currentfill}%
\pgfsetlinewidth{0.311001pt}%
\definecolor{currentstroke}{rgb}{1.000000,1.000000,1.000000}%
\pgfsetstrokecolor{currentstroke}%
\pgfsetdash{}{0pt}%
\pgfpathmoveto{\pgfqpoint{5.218520in}{1.247678in}}%
\pgfpathcurveto{\pgfqpoint{5.225653in}{1.247678in}}{\pgfqpoint{5.232494in}{1.250512in}}{\pgfqpoint{5.237538in}{1.255555in}}%
\pgfpathcurveto{\pgfqpoint{5.242582in}{1.260599in}}{\pgfqpoint{5.245415in}{1.267441in}}{\pgfqpoint{5.245415in}{1.274574in}}%
\pgfpathcurveto{\pgfqpoint{5.245415in}{1.281706in}}{\pgfqpoint{5.242582in}{1.288548in}}{\pgfqpoint{5.237538in}{1.293592in}}%
\pgfpathcurveto{\pgfqpoint{5.232494in}{1.298635in}}{\pgfqpoint{5.225653in}{1.301469in}}{\pgfqpoint{5.218520in}{1.301469in}}%
\pgfpathcurveto{\pgfqpoint{5.211387in}{1.301469in}}{\pgfqpoint{5.204545in}{1.298635in}}{\pgfqpoint{5.199502in}{1.293592in}}%
\pgfpathcurveto{\pgfqpoint{5.194458in}{1.288548in}}{\pgfqpoint{5.191624in}{1.281706in}}{\pgfqpoint{5.191624in}{1.274574in}}%
\pgfpathcurveto{\pgfqpoint{5.191624in}{1.267441in}}{\pgfqpoint{5.194458in}{1.260599in}}{\pgfqpoint{5.199502in}{1.255555in}}%
\pgfpathcurveto{\pgfqpoint{5.204545in}{1.250512in}}{\pgfqpoint{5.211387in}{1.247678in}}{\pgfqpoint{5.218520in}{1.247678in}}%
\pgfpathclose%
\pgfusepath{stroke,fill}%
\end{pgfscope}%
\begin{pgfscope}%
\pgfpathrectangle{\pgfqpoint{4.985294in}{0.500000in}}{\pgfqpoint{1.764706in}{1.700000in}}%
\pgfusepath{clip}%
\pgfsetbuttcap%
\pgfsetroundjoin%
\definecolor{currentfill}{rgb}{0.971202,0.827364,0.728520}%
\pgfsetfillcolor{currentfill}%
\pgfsetlinewidth{0.311001pt}%
\definecolor{currentstroke}{rgb}{1.000000,1.000000,1.000000}%
\pgfsetstrokecolor{currentstroke}%
\pgfsetdash{}{0pt}%
\pgfpathmoveto{\pgfqpoint{6.260155in}{1.652365in}}%
\pgfpathcurveto{\pgfqpoint{6.267288in}{1.652365in}}{\pgfqpoint{6.274129in}{1.655198in}}{\pgfqpoint{6.279173in}{1.660242in}}%
\pgfpathcurveto{\pgfqpoint{6.284217in}{1.665286in}}{\pgfqpoint{6.287051in}{1.672127in}}{\pgfqpoint{6.287051in}{1.679260in}}%
\pgfpathcurveto{\pgfqpoint{6.287051in}{1.686393in}}{\pgfqpoint{6.284217in}{1.693235in}}{\pgfqpoint{6.279173in}{1.698278in}}%
\pgfpathcurveto{\pgfqpoint{6.274129in}{1.703322in}}{\pgfqpoint{6.267288in}{1.706156in}}{\pgfqpoint{6.260155in}{1.706156in}}%
\pgfpathcurveto{\pgfqpoint{6.253022in}{1.706156in}}{\pgfqpoint{6.246180in}{1.703322in}}{\pgfqpoint{6.241137in}{1.698278in}}%
\pgfpathcurveto{\pgfqpoint{6.236093in}{1.693235in}}{\pgfqpoint{6.233259in}{1.686393in}}{\pgfqpoint{6.233259in}{1.679260in}}%
\pgfpathcurveto{\pgfqpoint{6.233259in}{1.672127in}}{\pgfqpoint{6.236093in}{1.665286in}}{\pgfqpoint{6.241137in}{1.660242in}}%
\pgfpathcurveto{\pgfqpoint{6.246180in}{1.655198in}}{\pgfqpoint{6.253022in}{1.652365in}}{\pgfqpoint{6.260155in}{1.652365in}}%
\pgfpathclose%
\pgfusepath{stroke,fill}%
\end{pgfscope}%
\begin{pgfscope}%
\pgfpathrectangle{\pgfqpoint{4.985294in}{0.500000in}}{\pgfqpoint{1.764706in}{1.700000in}}%
\pgfusepath{clip}%
\pgfsetbuttcap%
\pgfsetroundjoin%
\definecolor{currentfill}{rgb}{0.963379,0.625574,0.465113}%
\pgfsetfillcolor{currentfill}%
\pgfsetlinewidth{0.311001pt}%
\definecolor{currentstroke}{rgb}{1.000000,1.000000,1.000000}%
\pgfsetstrokecolor{currentstroke}%
\pgfsetdash{}{0pt}%
\pgfpathmoveto{\pgfqpoint{6.102561in}{0.968675in}}%
\pgfpathcurveto{\pgfqpoint{6.109693in}{0.968675in}}{\pgfqpoint{6.116535in}{0.971509in}}{\pgfqpoint{6.121579in}{0.976552in}}%
\pgfpathcurveto{\pgfqpoint{6.126622in}{0.981596in}}{\pgfqpoint{6.129456in}{0.988438in}}{\pgfqpoint{6.129456in}{0.995570in}}%
\pgfpathcurveto{\pgfqpoint{6.129456in}{1.002703in}}{\pgfqpoint{6.126622in}{1.009545in}}{\pgfqpoint{6.121579in}{1.014589in}}%
\pgfpathcurveto{\pgfqpoint{6.116535in}{1.019632in}}{\pgfqpoint{6.109693in}{1.022466in}}{\pgfqpoint{6.102561in}{1.022466in}}%
\pgfpathcurveto{\pgfqpoint{6.095428in}{1.022466in}}{\pgfqpoint{6.088586in}{1.019632in}}{\pgfqpoint{6.083542in}{1.014589in}}%
\pgfpathcurveto{\pgfqpoint{6.078499in}{1.009545in}}{\pgfqpoint{6.075665in}{1.002703in}}{\pgfqpoint{6.075665in}{0.995570in}}%
\pgfpathcurveto{\pgfqpoint{6.075665in}{0.988438in}}{\pgfqpoint{6.078499in}{0.981596in}}{\pgfqpoint{6.083542in}{0.976552in}}%
\pgfpathcurveto{\pgfqpoint{6.088586in}{0.971509in}}{\pgfqpoint{6.095428in}{0.968675in}}{\pgfqpoint{6.102561in}{0.968675in}}%
\pgfpathclose%
\pgfusepath{stroke,fill}%
\end{pgfscope}%
\begin{pgfscope}%
\pgfpathrectangle{\pgfqpoint{4.985294in}{0.500000in}}{\pgfqpoint{1.764706in}{1.700000in}}%
\pgfusepath{clip}%
\pgfsetbuttcap%
\pgfsetroundjoin%
\definecolor{currentfill}{rgb}{0.966120,0.744512,0.608720}%
\pgfsetfillcolor{currentfill}%
\pgfsetlinewidth{0.311001pt}%
\definecolor{currentstroke}{rgb}{1.000000,1.000000,1.000000}%
\pgfsetstrokecolor{currentstroke}%
\pgfsetdash{}{0pt}%
\pgfpathmoveto{\pgfqpoint{5.597768in}{0.980445in}}%
\pgfpathcurveto{\pgfqpoint{5.604901in}{0.980445in}}{\pgfqpoint{5.611743in}{0.983279in}}{\pgfqpoint{5.616786in}{0.988322in}}%
\pgfpathcurveto{\pgfqpoint{5.621830in}{0.993366in}}{\pgfqpoint{5.624664in}{1.000208in}}{\pgfqpoint{5.624664in}{1.007341in}}%
\pgfpathcurveto{\pgfqpoint{5.624664in}{1.014473in}}{\pgfqpoint{5.621830in}{1.021315in}}{\pgfqpoint{5.616786in}{1.026359in}}%
\pgfpathcurveto{\pgfqpoint{5.611743in}{1.031402in}}{\pgfqpoint{5.604901in}{1.034236in}}{\pgfqpoint{5.597768in}{1.034236in}}%
\pgfpathcurveto{\pgfqpoint{5.590635in}{1.034236in}}{\pgfqpoint{5.583794in}{1.031402in}}{\pgfqpoint{5.578750in}{1.026359in}}%
\pgfpathcurveto{\pgfqpoint{5.573706in}{1.021315in}}{\pgfqpoint{5.570872in}{1.014473in}}{\pgfqpoint{5.570872in}{1.007341in}}%
\pgfpathcurveto{\pgfqpoint{5.570872in}{1.000208in}}{\pgfqpoint{5.573706in}{0.993366in}}{\pgfqpoint{5.578750in}{0.988322in}}%
\pgfpathcurveto{\pgfqpoint{5.583794in}{0.983279in}}{\pgfqpoint{5.590635in}{0.980445in}}{\pgfqpoint{5.597768in}{0.980445in}}%
\pgfpathclose%
\pgfusepath{stroke,fill}%
\end{pgfscope}%
\begin{pgfscope}%
\pgfpathrectangle{\pgfqpoint{4.985294in}{0.500000in}}{\pgfqpoint{1.764706in}{1.700000in}}%
\pgfusepath{clip}%
\pgfsetbuttcap%
\pgfsetroundjoin%
\definecolor{currentfill}{rgb}{0.962018,0.586477,0.424918}%
\pgfsetfillcolor{currentfill}%
\pgfsetlinewidth{0.311001pt}%
\definecolor{currentstroke}{rgb}{1.000000,1.000000,1.000000}%
\pgfsetstrokecolor{currentstroke}%
\pgfsetdash{}{0pt}%
\pgfpathmoveto{\pgfqpoint{6.130562in}{1.098837in}}%
\pgfpathcurveto{\pgfqpoint{6.137695in}{1.098837in}}{\pgfqpoint{6.144536in}{1.101671in}}{\pgfqpoint{6.149580in}{1.106715in}}%
\pgfpathcurveto{\pgfqpoint{6.154624in}{1.111759in}}{\pgfqpoint{6.157458in}{1.118600in}}{\pgfqpoint{6.157458in}{1.125733in}}%
\pgfpathcurveto{\pgfqpoint{6.157458in}{1.132866in}}{\pgfqpoint{6.154624in}{1.139708in}}{\pgfqpoint{6.149580in}{1.144751in}}%
\pgfpathcurveto{\pgfqpoint{6.144536in}{1.149795in}}{\pgfqpoint{6.137695in}{1.152629in}}{\pgfqpoint{6.130562in}{1.152629in}}%
\pgfpathcurveto{\pgfqpoint{6.123429in}{1.152629in}}{\pgfqpoint{6.116587in}{1.149795in}}{\pgfqpoint{6.111544in}{1.144751in}}%
\pgfpathcurveto{\pgfqpoint{6.106500in}{1.139708in}}{\pgfqpoint{6.103666in}{1.132866in}}{\pgfqpoint{6.103666in}{1.125733in}}%
\pgfpathcurveto{\pgfqpoint{6.103666in}{1.118600in}}{\pgfqpoint{6.106500in}{1.111759in}}{\pgfqpoint{6.111544in}{1.106715in}}%
\pgfpathcurveto{\pgfqpoint{6.116587in}{1.101671in}}{\pgfqpoint{6.123429in}{1.098837in}}{\pgfqpoint{6.130562in}{1.098837in}}%
\pgfpathclose%
\pgfusepath{stroke,fill}%
\end{pgfscope}%
\begin{pgfscope}%
\pgfpathrectangle{\pgfqpoint{4.985294in}{0.500000in}}{\pgfqpoint{1.764706in}{1.700000in}}%
\pgfusepath{clip}%
\pgfsetbuttcap%
\pgfsetroundjoin%
\definecolor{currentfill}{rgb}{0.975644,0.874038,0.797253}%
\pgfsetfillcolor{currentfill}%
\pgfsetlinewidth{0.311001pt}%
\definecolor{currentstroke}{rgb}{1.000000,1.000000,1.000000}%
\pgfsetstrokecolor{currentstroke}%
\pgfsetdash{}{0pt}%
\pgfpathmoveto{\pgfqpoint{6.238850in}{1.078545in}}%
\pgfpathcurveto{\pgfqpoint{6.245982in}{1.078545in}}{\pgfqpoint{6.252824in}{1.081379in}}{\pgfqpoint{6.257868in}{1.086422in}}%
\pgfpathcurveto{\pgfqpoint{6.262911in}{1.091466in}}{\pgfqpoint{6.265745in}{1.098308in}}{\pgfqpoint{6.265745in}{1.105441in}}%
\pgfpathcurveto{\pgfqpoint{6.265745in}{1.112573in}}{\pgfqpoint{6.262911in}{1.119415in}}{\pgfqpoint{6.257868in}{1.124459in}}%
\pgfpathcurveto{\pgfqpoint{6.252824in}{1.129502in}}{\pgfqpoint{6.245982in}{1.132336in}}{\pgfqpoint{6.238850in}{1.132336in}}%
\pgfpathcurveto{\pgfqpoint{6.231717in}{1.132336in}}{\pgfqpoint{6.224875in}{1.129502in}}{\pgfqpoint{6.219831in}{1.124459in}}%
\pgfpathcurveto{\pgfqpoint{6.214788in}{1.119415in}}{\pgfqpoint{6.211954in}{1.112573in}}{\pgfqpoint{6.211954in}{1.105441in}}%
\pgfpathcurveto{\pgfqpoint{6.211954in}{1.098308in}}{\pgfqpoint{6.214788in}{1.091466in}}{\pgfqpoint{6.219831in}{1.086422in}}%
\pgfpathcurveto{\pgfqpoint{6.224875in}{1.081379in}}{\pgfqpoint{6.231717in}{1.078545in}}{\pgfqpoint{6.238850in}{1.078545in}}%
\pgfpathclose%
\pgfusepath{stroke,fill}%
\end{pgfscope}%
\begin{pgfscope}%
\pgfpathrectangle{\pgfqpoint{4.985294in}{0.500000in}}{\pgfqpoint{1.764706in}{1.700000in}}%
\pgfusepath{clip}%
\pgfsetbuttcap%
\pgfsetroundjoin%
\definecolor{currentfill}{rgb}{0.977657,0.891500,0.822809}%
\pgfsetfillcolor{currentfill}%
\pgfsetlinewidth{0.311001pt}%
\definecolor{currentstroke}{rgb}{1.000000,1.000000,1.000000}%
\pgfsetstrokecolor{currentstroke}%
\pgfsetdash{}{0pt}%
\pgfpathmoveto{\pgfqpoint{6.264620in}{1.180219in}}%
\pgfpathcurveto{\pgfqpoint{6.271752in}{1.180219in}}{\pgfqpoint{6.278594in}{1.183053in}}{\pgfqpoint{6.283638in}{1.188097in}}%
\pgfpathcurveto{\pgfqpoint{6.288681in}{1.193141in}}{\pgfqpoint{6.291515in}{1.199982in}}{\pgfqpoint{6.291515in}{1.207115in}}%
\pgfpathcurveto{\pgfqpoint{6.291515in}{1.214248in}}{\pgfqpoint{6.288681in}{1.221089in}}{\pgfqpoint{6.283638in}{1.226133in}}%
\pgfpathcurveto{\pgfqpoint{6.278594in}{1.231177in}}{\pgfqpoint{6.271752in}{1.234011in}}{\pgfqpoint{6.264620in}{1.234011in}}%
\pgfpathcurveto{\pgfqpoint{6.257487in}{1.234011in}}{\pgfqpoint{6.250645in}{1.231177in}}{\pgfqpoint{6.245601in}{1.226133in}}%
\pgfpathcurveto{\pgfqpoint{6.240558in}{1.221089in}}{\pgfqpoint{6.237724in}{1.214248in}}{\pgfqpoint{6.237724in}{1.207115in}}%
\pgfpathcurveto{\pgfqpoint{6.237724in}{1.199982in}}{\pgfqpoint{6.240558in}{1.193141in}}{\pgfqpoint{6.245601in}{1.188097in}}%
\pgfpathcurveto{\pgfqpoint{6.250645in}{1.183053in}}{\pgfqpoint{6.257487in}{1.180219in}}{\pgfqpoint{6.264620in}{1.180219in}}%
\pgfpathclose%
\pgfusepath{stroke,fill}%
\end{pgfscope}%
\begin{pgfscope}%
\pgfpathrectangle{\pgfqpoint{4.985294in}{0.500000in}}{\pgfqpoint{1.764706in}{1.700000in}}%
\pgfusepath{clip}%
\pgfsetbuttcap%
\pgfsetroundjoin%
\definecolor{currentfill}{rgb}{0.967735,0.780441,0.659127}%
\pgfsetfillcolor{currentfill}%
\pgfsetlinewidth{0.311001pt}%
\definecolor{currentstroke}{rgb}{1.000000,1.000000,1.000000}%
\pgfsetstrokecolor{currentstroke}%
\pgfsetdash{}{0pt}%
\pgfpathmoveto{\pgfqpoint{5.440124in}{0.994462in}}%
\pgfpathcurveto{\pgfqpoint{5.447257in}{0.994462in}}{\pgfqpoint{5.454098in}{0.997296in}}{\pgfqpoint{5.459142in}{1.002339in}}%
\pgfpathcurveto{\pgfqpoint{5.464186in}{1.007383in}}{\pgfqpoint{5.467020in}{1.014225in}}{\pgfqpoint{5.467020in}{1.021358in}}%
\pgfpathcurveto{\pgfqpoint{5.467020in}{1.028490in}}{\pgfqpoint{5.464186in}{1.035332in}}{\pgfqpoint{5.459142in}{1.040376in}}%
\pgfpathcurveto{\pgfqpoint{5.454098in}{1.045419in}}{\pgfqpoint{5.447257in}{1.048253in}}{\pgfqpoint{5.440124in}{1.048253in}}%
\pgfpathcurveto{\pgfqpoint{5.432991in}{1.048253in}}{\pgfqpoint{5.426150in}{1.045419in}}{\pgfqpoint{5.421106in}{1.040376in}}%
\pgfpathcurveto{\pgfqpoint{5.416062in}{1.035332in}}{\pgfqpoint{5.413228in}{1.028490in}}{\pgfqpoint{5.413228in}{1.021358in}}%
\pgfpathcurveto{\pgfqpoint{5.413228in}{1.014225in}}{\pgfqpoint{5.416062in}{1.007383in}}{\pgfqpoint{5.421106in}{1.002339in}}%
\pgfpathcurveto{\pgfqpoint{5.426150in}{0.997296in}}{\pgfqpoint{5.432991in}{0.994462in}}{\pgfqpoint{5.440124in}{0.994462in}}%
\pgfpathclose%
\pgfusepath{stroke,fill}%
\end{pgfscope}%
\begin{pgfscope}%
\pgfpathrectangle{\pgfqpoint{4.985294in}{0.500000in}}{\pgfqpoint{1.764706in}{1.700000in}}%
\pgfusepath{clip}%
\pgfsetbuttcap%
\pgfsetroundjoin%
\definecolor{currentfill}{rgb}{0.974412,0.862387,0.780156}%
\pgfsetfillcolor{currentfill}%
\pgfsetlinewidth{0.311001pt}%
\definecolor{currentstroke}{rgb}{1.000000,1.000000,1.000000}%
\pgfsetstrokecolor{currentstroke}%
\pgfsetdash{}{0pt}%
\pgfpathmoveto{\pgfqpoint{6.367321in}{1.296454in}}%
\pgfpathcurveto{\pgfqpoint{6.374454in}{1.296454in}}{\pgfqpoint{6.381296in}{1.299287in}}{\pgfqpoint{6.386339in}{1.304331in}}%
\pgfpathcurveto{\pgfqpoint{6.391383in}{1.309375in}}{\pgfqpoint{6.394217in}{1.316216in}}{\pgfqpoint{6.394217in}{1.323349in}}%
\pgfpathcurveto{\pgfqpoint{6.394217in}{1.330482in}}{\pgfqpoint{6.391383in}{1.337324in}}{\pgfqpoint{6.386339in}{1.342367in}}%
\pgfpathcurveto{\pgfqpoint{6.381296in}{1.347411in}}{\pgfqpoint{6.374454in}{1.350245in}}{\pgfqpoint{6.367321in}{1.350245in}}%
\pgfpathcurveto{\pgfqpoint{6.360189in}{1.350245in}}{\pgfqpoint{6.353347in}{1.347411in}}{\pgfqpoint{6.348303in}{1.342367in}}%
\pgfpathcurveto{\pgfqpoint{6.343260in}{1.337324in}}{\pgfqpoint{6.340426in}{1.330482in}}{\pgfqpoint{6.340426in}{1.323349in}}%
\pgfpathcurveto{\pgfqpoint{6.340426in}{1.316216in}}{\pgfqpoint{6.343260in}{1.309375in}}{\pgfqpoint{6.348303in}{1.304331in}}%
\pgfpathcurveto{\pgfqpoint{6.353347in}{1.299287in}}{\pgfqpoint{6.360189in}{1.296454in}}{\pgfqpoint{6.367321in}{1.296454in}}%
\pgfpathclose%
\pgfusepath{stroke,fill}%
\end{pgfscope}%
\begin{pgfscope}%
\pgfpathrectangle{\pgfqpoint{4.985294in}{0.500000in}}{\pgfqpoint{1.764706in}{1.700000in}}%
\pgfusepath{clip}%
\pgfsetbuttcap%
\pgfsetroundjoin%
\definecolor{currentfill}{rgb}{0.976287,0.879862,0.805788}%
\pgfsetfillcolor{currentfill}%
\pgfsetlinewidth{0.311001pt}%
\definecolor{currentstroke}{rgb}{1.000000,1.000000,1.000000}%
\pgfsetstrokecolor{currentstroke}%
\pgfsetdash{}{0pt}%
\pgfpathmoveto{\pgfqpoint{5.471761in}{1.089156in}}%
\pgfpathcurveto{\pgfqpoint{5.478893in}{1.089156in}}{\pgfqpoint{5.485735in}{1.091990in}}{\pgfqpoint{5.490779in}{1.097034in}}%
\pgfpathcurveto{\pgfqpoint{5.495822in}{1.102078in}}{\pgfqpoint{5.498656in}{1.108919in}}{\pgfqpoint{5.498656in}{1.116052in}}%
\pgfpathcurveto{\pgfqpoint{5.498656in}{1.123185in}}{\pgfqpoint{5.495822in}{1.130026in}}{\pgfqpoint{5.490779in}{1.135070in}}%
\pgfpathcurveto{\pgfqpoint{5.485735in}{1.140114in}}{\pgfqpoint{5.478893in}{1.142948in}}{\pgfqpoint{5.471761in}{1.142948in}}%
\pgfpathcurveto{\pgfqpoint{5.464628in}{1.142948in}}{\pgfqpoint{5.457786in}{1.140114in}}{\pgfqpoint{5.452742in}{1.135070in}}%
\pgfpathcurveto{\pgfqpoint{5.447699in}{1.130026in}}{\pgfqpoint{5.444865in}{1.123185in}}{\pgfqpoint{5.444865in}{1.116052in}}%
\pgfpathcurveto{\pgfqpoint{5.444865in}{1.108919in}}{\pgfqpoint{5.447699in}{1.102078in}}{\pgfqpoint{5.452742in}{1.097034in}}%
\pgfpathcurveto{\pgfqpoint{5.457786in}{1.091990in}}{\pgfqpoint{5.464628in}{1.089156in}}{\pgfqpoint{5.471761in}{1.089156in}}%
\pgfpathclose%
\pgfusepath{stroke,fill}%
\end{pgfscope}%
\begin{pgfscope}%
\pgfpathrectangle{\pgfqpoint{4.985294in}{0.500000in}}{\pgfqpoint{1.764706in}{1.700000in}}%
\pgfusepath{clip}%
\pgfsetbuttcap%
\pgfsetroundjoin%
\definecolor{currentfill}{rgb}{0.969359,0.803954,0.693832}%
\pgfsetfillcolor{currentfill}%
\pgfsetlinewidth{0.311001pt}%
\definecolor{currentstroke}{rgb}{1.000000,1.000000,1.000000}%
\pgfsetstrokecolor{currentstroke}%
\pgfsetdash{}{0pt}%
\pgfpathmoveto{\pgfqpoint{6.385349in}{1.342184in}}%
\pgfpathcurveto{\pgfqpoint{6.392482in}{1.342184in}}{\pgfqpoint{6.399324in}{1.345018in}}{\pgfqpoint{6.404368in}{1.350062in}}%
\pgfpathcurveto{\pgfqpoint{6.409411in}{1.355105in}}{\pgfqpoint{6.412245in}{1.361947in}}{\pgfqpoint{6.412245in}{1.369080in}}%
\pgfpathcurveto{\pgfqpoint{6.412245in}{1.376213in}}{\pgfqpoint{6.409411in}{1.383054in}}{\pgfqpoint{6.404368in}{1.388098in}}%
\pgfpathcurveto{\pgfqpoint{6.399324in}{1.393142in}}{\pgfqpoint{6.392482in}{1.395976in}}{\pgfqpoint{6.385349in}{1.395976in}}%
\pgfpathcurveto{\pgfqpoint{6.378217in}{1.395976in}}{\pgfqpoint{6.371375in}{1.393142in}}{\pgfqpoint{6.366331in}{1.388098in}}%
\pgfpathcurveto{\pgfqpoint{6.361288in}{1.383054in}}{\pgfqpoint{6.358454in}{1.376213in}}{\pgfqpoint{6.358454in}{1.369080in}}%
\pgfpathcurveto{\pgfqpoint{6.358454in}{1.361947in}}{\pgfqpoint{6.361288in}{1.355105in}}{\pgfqpoint{6.366331in}{1.350062in}}%
\pgfpathcurveto{\pgfqpoint{6.371375in}{1.345018in}}{\pgfqpoint{6.378217in}{1.342184in}}{\pgfqpoint{6.385349in}{1.342184in}}%
\pgfpathclose%
\pgfusepath{stroke,fill}%
\end{pgfscope}%
\begin{pgfscope}%
\pgfpathrectangle{\pgfqpoint{4.985294in}{0.500000in}}{\pgfqpoint{1.764706in}{1.700000in}}%
\pgfusepath{clip}%
\pgfsetbuttcap%
\pgfsetroundjoin%
\definecolor{currentfill}{rgb}{0.965753,0.732351,0.592427}%
\pgfsetfillcolor{currentfill}%
\pgfsetlinewidth{0.311001pt}%
\definecolor{currentstroke}{rgb}{1.000000,1.000000,1.000000}%
\pgfsetstrokecolor{currentstroke}%
\pgfsetdash{}{0pt}%
\pgfpathmoveto{\pgfqpoint{6.405938in}{1.305676in}}%
\pgfpathcurveto{\pgfqpoint{6.413071in}{1.305676in}}{\pgfqpoint{6.419912in}{1.308509in}}{\pgfqpoint{6.424956in}{1.313553in}}%
\pgfpathcurveto{\pgfqpoint{6.430000in}{1.318597in}}{\pgfqpoint{6.432834in}{1.325438in}}{\pgfqpoint{6.432834in}{1.332571in}}%
\pgfpathcurveto{\pgfqpoint{6.432834in}{1.339704in}}{\pgfqpoint{6.430000in}{1.346546in}}{\pgfqpoint{6.424956in}{1.351589in}}%
\pgfpathcurveto{\pgfqpoint{6.419912in}{1.356633in}}{\pgfqpoint{6.413071in}{1.359467in}}{\pgfqpoint{6.405938in}{1.359467in}}%
\pgfpathcurveto{\pgfqpoint{6.398805in}{1.359467in}}{\pgfqpoint{6.391963in}{1.356633in}}{\pgfqpoint{6.386920in}{1.351589in}}%
\pgfpathcurveto{\pgfqpoint{6.381876in}{1.346546in}}{\pgfqpoint{6.379042in}{1.339704in}}{\pgfqpoint{6.379042in}{1.332571in}}%
\pgfpathcurveto{\pgfqpoint{6.379042in}{1.325438in}}{\pgfqpoint{6.381876in}{1.318597in}}{\pgfqpoint{6.386920in}{1.313553in}}%
\pgfpathcurveto{\pgfqpoint{6.391963in}{1.308509in}}{\pgfqpoint{6.398805in}{1.305676in}}{\pgfqpoint{6.405938in}{1.305676in}}%
\pgfpathclose%
\pgfusepath{stroke,fill}%
\end{pgfscope}%
\begin{pgfscope}%
\pgfpathrectangle{\pgfqpoint{4.985294in}{0.500000in}}{\pgfqpoint{1.764706in}{1.700000in}}%
\pgfusepath{clip}%
\pgfsetbuttcap%
\pgfsetroundjoin%
\definecolor{currentfill}{rgb}{0.970255,0.815666,0.711203}%
\pgfsetfillcolor{currentfill}%
\pgfsetlinewidth{0.311001pt}%
\definecolor{currentstroke}{rgb}{1.000000,1.000000,1.000000}%
\pgfsetstrokecolor{currentstroke}%
\pgfsetdash{}{0pt}%
\pgfpathmoveto{\pgfqpoint{5.523304in}{0.970303in}}%
\pgfpathcurveto{\pgfqpoint{5.530437in}{0.970303in}}{\pgfqpoint{5.537279in}{0.973137in}}{\pgfqpoint{5.542322in}{0.978180in}}%
\pgfpathcurveto{\pgfqpoint{5.547366in}{0.983224in}}{\pgfqpoint{5.550200in}{0.990066in}}{\pgfqpoint{5.550200in}{0.997199in}}%
\pgfpathcurveto{\pgfqpoint{5.550200in}{1.004331in}}{\pgfqpoint{5.547366in}{1.011173in}}{\pgfqpoint{5.542322in}{1.016217in}}%
\pgfpathcurveto{\pgfqpoint{5.537279in}{1.021260in}}{\pgfqpoint{5.530437in}{1.024094in}}{\pgfqpoint{5.523304in}{1.024094in}}%
\pgfpathcurveto{\pgfqpoint{5.516171in}{1.024094in}}{\pgfqpoint{5.509330in}{1.021260in}}{\pgfqpoint{5.504286in}{1.016217in}}%
\pgfpathcurveto{\pgfqpoint{5.499242in}{1.011173in}}{\pgfqpoint{5.496408in}{1.004331in}}{\pgfqpoint{5.496408in}{0.997199in}}%
\pgfpathcurveto{\pgfqpoint{5.496408in}{0.990066in}}{\pgfqpoint{5.499242in}{0.983224in}}{\pgfqpoint{5.504286in}{0.978180in}}%
\pgfpathcurveto{\pgfqpoint{5.509330in}{0.973137in}}{\pgfqpoint{5.516171in}{0.970303in}}{\pgfqpoint{5.523304in}{0.970303in}}%
\pgfpathclose%
\pgfusepath{stroke,fill}%
\end{pgfscope}%
\begin{pgfscope}%
\pgfpathrectangle{\pgfqpoint{4.985294in}{0.500000in}}{\pgfqpoint{1.764706in}{1.700000in}}%
\pgfusepath{clip}%
\pgfsetbuttcap%
\pgfsetroundjoin%
\definecolor{currentfill}{rgb}{0.963190,0.619109,0.458249}%
\pgfsetfillcolor{currentfill}%
\pgfsetlinewidth{0.311001pt}%
\definecolor{currentstroke}{rgb}{1.000000,1.000000,1.000000}%
\pgfsetstrokecolor{currentstroke}%
\pgfsetdash{}{0pt}%
\pgfpathmoveto{\pgfqpoint{6.176599in}{1.262943in}}%
\pgfpathcurveto{\pgfqpoint{6.183732in}{1.262943in}}{\pgfqpoint{6.190574in}{1.265777in}}{\pgfqpoint{6.195617in}{1.270820in}}%
\pgfpathcurveto{\pgfqpoint{6.200661in}{1.275864in}}{\pgfqpoint{6.203495in}{1.282706in}}{\pgfqpoint{6.203495in}{1.289838in}}%
\pgfpathcurveto{\pgfqpoint{6.203495in}{1.296971in}}{\pgfqpoint{6.200661in}{1.303813in}}{\pgfqpoint{6.195617in}{1.308857in}}%
\pgfpathcurveto{\pgfqpoint{6.190574in}{1.313900in}}{\pgfqpoint{6.183732in}{1.316734in}}{\pgfqpoint{6.176599in}{1.316734in}}%
\pgfpathcurveto{\pgfqpoint{6.169466in}{1.316734in}}{\pgfqpoint{6.162625in}{1.313900in}}{\pgfqpoint{6.157581in}{1.308857in}}%
\pgfpathcurveto{\pgfqpoint{6.152537in}{1.303813in}}{\pgfqpoint{6.149704in}{1.296971in}}{\pgfqpoint{6.149704in}{1.289838in}}%
\pgfpathcurveto{\pgfqpoint{6.149704in}{1.282706in}}{\pgfqpoint{6.152537in}{1.275864in}}{\pgfqpoint{6.157581in}{1.270820in}}%
\pgfpathcurveto{\pgfqpoint{6.162625in}{1.265777in}}{\pgfqpoint{6.169466in}{1.262943in}}{\pgfqpoint{6.176599in}{1.262943in}}%
\pgfpathclose%
\pgfusepath{stroke,fill}%
\end{pgfscope}%
\begin{pgfscope}%
\pgfpathrectangle{\pgfqpoint{4.985294in}{0.500000in}}{\pgfqpoint{1.764706in}{1.700000in}}%
\pgfusepath{clip}%
\pgfsetbuttcap%
\pgfsetroundjoin%
\definecolor{currentfill}{rgb}{0.963728,0.638439,0.479050}%
\pgfsetfillcolor{currentfill}%
\pgfsetlinewidth{0.311001pt}%
\definecolor{currentstroke}{rgb}{1.000000,1.000000,1.000000}%
\pgfsetstrokecolor{currentstroke}%
\pgfsetdash{}{0pt}%
\pgfpathmoveto{\pgfqpoint{6.130791in}{1.064658in}}%
\pgfpathcurveto{\pgfqpoint{6.137923in}{1.064658in}}{\pgfqpoint{6.144765in}{1.067492in}}{\pgfqpoint{6.149809in}{1.072535in}}%
\pgfpathcurveto{\pgfqpoint{6.154852in}{1.077579in}}{\pgfqpoint{6.157686in}{1.084421in}}{\pgfqpoint{6.157686in}{1.091554in}}%
\pgfpathcurveto{\pgfqpoint{6.157686in}{1.098686in}}{\pgfqpoint{6.154852in}{1.105528in}}{\pgfqpoint{6.149809in}{1.110572in}}%
\pgfpathcurveto{\pgfqpoint{6.144765in}{1.115615in}}{\pgfqpoint{6.137923in}{1.118449in}}{\pgfqpoint{6.130791in}{1.118449in}}%
\pgfpathcurveto{\pgfqpoint{6.123658in}{1.118449in}}{\pgfqpoint{6.116816in}{1.115615in}}{\pgfqpoint{6.111772in}{1.110572in}}%
\pgfpathcurveto{\pgfqpoint{6.106729in}{1.105528in}}{\pgfqpoint{6.103895in}{1.098686in}}{\pgfqpoint{6.103895in}{1.091554in}}%
\pgfpathcurveto{\pgfqpoint{6.103895in}{1.084421in}}{\pgfqpoint{6.106729in}{1.077579in}}{\pgfqpoint{6.111772in}{1.072535in}}%
\pgfpathcurveto{\pgfqpoint{6.116816in}{1.067492in}}{\pgfqpoint{6.123658in}{1.064658in}}{\pgfqpoint{6.130791in}{1.064658in}}%
\pgfpathclose%
\pgfusepath{stroke,fill}%
\end{pgfscope}%
\begin{pgfscope}%
\pgfpathrectangle{\pgfqpoint{4.985294in}{0.500000in}}{\pgfqpoint{1.764706in}{1.700000in}}%
\pgfusepath{clip}%
\pgfsetbuttcap%
\pgfsetroundjoin%
\definecolor{currentfill}{rgb}{0.979891,0.908948,0.848279}%
\pgfsetfillcolor{currentfill}%
\pgfsetlinewidth{0.311001pt}%
\definecolor{currentstroke}{rgb}{1.000000,1.000000,1.000000}%
\pgfsetstrokecolor{currentstroke}%
\pgfsetdash{}{0pt}%
\pgfpathmoveto{\pgfqpoint{6.326135in}{1.430789in}}%
\pgfpathcurveto{\pgfqpoint{6.333268in}{1.430789in}}{\pgfqpoint{6.340110in}{1.433623in}}{\pgfqpoint{6.345153in}{1.438667in}}%
\pgfpathcurveto{\pgfqpoint{6.350197in}{1.443711in}}{\pgfqpoint{6.353031in}{1.450552in}}{\pgfqpoint{6.353031in}{1.457685in}}%
\pgfpathcurveto{\pgfqpoint{6.353031in}{1.464818in}}{\pgfqpoint{6.350197in}{1.471660in}}{\pgfqpoint{6.345153in}{1.476703in}}%
\pgfpathcurveto{\pgfqpoint{6.340110in}{1.481747in}}{\pgfqpoint{6.333268in}{1.484581in}}{\pgfqpoint{6.326135in}{1.484581in}}%
\pgfpathcurveto{\pgfqpoint{6.319003in}{1.484581in}}{\pgfqpoint{6.312161in}{1.481747in}}{\pgfqpoint{6.307117in}{1.476703in}}%
\pgfpathcurveto{\pgfqpoint{6.302074in}{1.471660in}}{\pgfqpoint{6.299240in}{1.464818in}}{\pgfqpoint{6.299240in}{1.457685in}}%
\pgfpathcurveto{\pgfqpoint{6.299240in}{1.450552in}}{\pgfqpoint{6.302074in}{1.443711in}}{\pgfqpoint{6.307117in}{1.438667in}}%
\pgfpathcurveto{\pgfqpoint{6.312161in}{1.433623in}}{\pgfqpoint{6.319003in}{1.430789in}}{\pgfqpoint{6.326135in}{1.430789in}}%
\pgfpathclose%
\pgfusepath{stroke,fill}%
\end{pgfscope}%
\begin{pgfscope}%
\pgfpathrectangle{\pgfqpoint{4.985294in}{0.500000in}}{\pgfqpoint{1.764706in}{1.700000in}}%
\pgfusepath{clip}%
\pgfsetbuttcap%
\pgfsetroundjoin%
\definecolor{currentfill}{rgb}{0.948235,0.413004,0.283323}%
\pgfsetfillcolor{currentfill}%
\pgfsetlinewidth{0.311001pt}%
\definecolor{currentstroke}{rgb}{1.000000,1.000000,1.000000}%
\pgfsetstrokecolor{currentstroke}%
\pgfsetdash{}{0pt}%
\pgfpathmoveto{\pgfqpoint{6.079658in}{1.067668in}}%
\pgfpathcurveto{\pgfqpoint{6.086791in}{1.067668in}}{\pgfqpoint{6.093632in}{1.070502in}}{\pgfqpoint{6.098676in}{1.075545in}}%
\pgfpathcurveto{\pgfqpoint{6.103720in}{1.080589in}}{\pgfqpoint{6.106554in}{1.087431in}}{\pgfqpoint{6.106554in}{1.094563in}}%
\pgfpathcurveto{\pgfqpoint{6.106554in}{1.101696in}}{\pgfqpoint{6.103720in}{1.108538in}}{\pgfqpoint{6.098676in}{1.113582in}}%
\pgfpathcurveto{\pgfqpoint{6.093632in}{1.118625in}}{\pgfqpoint{6.086791in}{1.121459in}}{\pgfqpoint{6.079658in}{1.121459in}}%
\pgfpathcurveto{\pgfqpoint{6.072525in}{1.121459in}}{\pgfqpoint{6.065683in}{1.118625in}}{\pgfqpoint{6.060640in}{1.113582in}}%
\pgfpathcurveto{\pgfqpoint{6.055596in}{1.108538in}}{\pgfqpoint{6.052762in}{1.101696in}}{\pgfqpoint{6.052762in}{1.094563in}}%
\pgfpathcurveto{\pgfqpoint{6.052762in}{1.087431in}}{\pgfqpoint{6.055596in}{1.080589in}}{\pgfqpoint{6.060640in}{1.075545in}}%
\pgfpathcurveto{\pgfqpoint{6.065683in}{1.070502in}}{\pgfqpoint{6.072525in}{1.067668in}}{\pgfqpoint{6.079658in}{1.067668in}}%
\pgfpathclose%
\pgfusepath{stroke,fill}%
\end{pgfscope}%
\begin{pgfscope}%
\pgfpathrectangle{\pgfqpoint{4.985294in}{0.500000in}}{\pgfqpoint{1.764706in}{1.700000in}}%
\pgfusepath{clip}%
\pgfsetbuttcap%
\pgfsetroundjoin%
\definecolor{currentfill}{rgb}{0.973832,0.856556,0.771584}%
\pgfsetfillcolor{currentfill}%
\pgfsetlinewidth{0.311001pt}%
\definecolor{currentstroke}{rgb}{1.000000,1.000000,1.000000}%
\pgfsetstrokecolor{currentstroke}%
\pgfsetdash{}{0pt}%
\pgfpathmoveto{\pgfqpoint{6.238318in}{1.048033in}}%
\pgfpathcurveto{\pgfqpoint{6.245451in}{1.048033in}}{\pgfqpoint{6.252293in}{1.050867in}}{\pgfqpoint{6.257336in}{1.055911in}}%
\pgfpathcurveto{\pgfqpoint{6.262380in}{1.060954in}}{\pgfqpoint{6.265214in}{1.067796in}}{\pgfqpoint{6.265214in}{1.074929in}}%
\pgfpathcurveto{\pgfqpoint{6.265214in}{1.082062in}}{\pgfqpoint{6.262380in}{1.088903in}}{\pgfqpoint{6.257336in}{1.093947in}}%
\pgfpathcurveto{\pgfqpoint{6.252293in}{1.098991in}}{\pgfqpoint{6.245451in}{1.101825in}}{\pgfqpoint{6.238318in}{1.101825in}}%
\pgfpathcurveto{\pgfqpoint{6.231186in}{1.101825in}}{\pgfqpoint{6.224344in}{1.098991in}}{\pgfqpoint{6.219300in}{1.093947in}}%
\pgfpathcurveto{\pgfqpoint{6.214257in}{1.088903in}}{\pgfqpoint{6.211423in}{1.082062in}}{\pgfqpoint{6.211423in}{1.074929in}}%
\pgfpathcurveto{\pgfqpoint{6.211423in}{1.067796in}}{\pgfqpoint{6.214257in}{1.060954in}}{\pgfqpoint{6.219300in}{1.055911in}}%
\pgfpathcurveto{\pgfqpoint{6.224344in}{1.050867in}}{\pgfqpoint{6.231186in}{1.048033in}}{\pgfqpoint{6.238318in}{1.048033in}}%
\pgfpathclose%
\pgfusepath{stroke,fill}%
\end{pgfscope}%
\begin{pgfscope}%
\pgfpathrectangle{\pgfqpoint{4.985294in}{0.500000in}}{\pgfqpoint{1.764706in}{1.700000in}}%
\pgfusepath{clip}%
\pgfsetbuttcap%
\pgfsetroundjoin%
\definecolor{currentfill}{rgb}{0.975644,0.874038,0.797253}%
\pgfsetfillcolor{currentfill}%
\pgfsetlinewidth{0.311001pt}%
\definecolor{currentstroke}{rgb}{1.000000,1.000000,1.000000}%
\pgfsetstrokecolor{currentstroke}%
\pgfsetdash{}{0pt}%
\pgfpathmoveto{\pgfqpoint{6.289724in}{1.102790in}}%
\pgfpathcurveto{\pgfqpoint{6.296857in}{1.102790in}}{\pgfqpoint{6.303698in}{1.105624in}}{\pgfqpoint{6.308742in}{1.110668in}}%
\pgfpathcurveto{\pgfqpoint{6.313786in}{1.115711in}}{\pgfqpoint{6.316619in}{1.122553in}}{\pgfqpoint{6.316619in}{1.129686in}}%
\pgfpathcurveto{\pgfqpoint{6.316619in}{1.136819in}}{\pgfqpoint{6.313786in}{1.143660in}}{\pgfqpoint{6.308742in}{1.148704in}}%
\pgfpathcurveto{\pgfqpoint{6.303698in}{1.153748in}}{\pgfqpoint{6.296857in}{1.156582in}}{\pgfqpoint{6.289724in}{1.156582in}}%
\pgfpathcurveto{\pgfqpoint{6.282591in}{1.156582in}}{\pgfqpoint{6.275749in}{1.153748in}}{\pgfqpoint{6.270706in}{1.148704in}}%
\pgfpathcurveto{\pgfqpoint{6.265662in}{1.143660in}}{\pgfqpoint{6.262828in}{1.136819in}}{\pgfqpoint{6.262828in}{1.129686in}}%
\pgfpathcurveto{\pgfqpoint{6.262828in}{1.122553in}}{\pgfqpoint{6.265662in}{1.115711in}}{\pgfqpoint{6.270706in}{1.110668in}}%
\pgfpathcurveto{\pgfqpoint{6.275749in}{1.105624in}}{\pgfqpoint{6.282591in}{1.102790in}}{\pgfqpoint{6.289724in}{1.102790in}}%
\pgfpathclose%
\pgfusepath{stroke,fill}%
\end{pgfscope}%
\begin{pgfscope}%
\pgfpathrectangle{\pgfqpoint{4.985294in}{0.500000in}}{\pgfqpoint{1.764706in}{1.700000in}}%
\pgfusepath{clip}%
\pgfsetbuttcap%
\pgfsetroundjoin%
\definecolor{currentfill}{rgb}{0.961734,0.579886,0.418445}%
\pgfsetfillcolor{currentfill}%
\pgfsetlinewidth{0.311001pt}%
\definecolor{currentstroke}{rgb}{1.000000,1.000000,1.000000}%
\pgfsetstrokecolor{currentstroke}%
\pgfsetdash{}{0pt}%
\pgfpathmoveto{\pgfqpoint{6.270988in}{1.725357in}}%
\pgfpathcurveto{\pgfqpoint{6.278121in}{1.725357in}}{\pgfqpoint{6.284962in}{1.728191in}}{\pgfqpoint{6.290006in}{1.733235in}}%
\pgfpathcurveto{\pgfqpoint{6.295050in}{1.738278in}}{\pgfqpoint{6.297884in}{1.745120in}}{\pgfqpoint{6.297884in}{1.752253in}}%
\pgfpathcurveto{\pgfqpoint{6.297884in}{1.759386in}}{\pgfqpoint{6.295050in}{1.766227in}}{\pgfqpoint{6.290006in}{1.771271in}}%
\pgfpathcurveto{\pgfqpoint{6.284962in}{1.776315in}}{\pgfqpoint{6.278121in}{1.779148in}}{\pgfqpoint{6.270988in}{1.779148in}}%
\pgfpathcurveto{\pgfqpoint{6.263855in}{1.779148in}}{\pgfqpoint{6.257013in}{1.776315in}}{\pgfqpoint{6.251970in}{1.771271in}}%
\pgfpathcurveto{\pgfqpoint{6.246926in}{1.766227in}}{\pgfqpoint{6.244092in}{1.759386in}}{\pgfqpoint{6.244092in}{1.752253in}}%
\pgfpathcurveto{\pgfqpoint{6.244092in}{1.745120in}}{\pgfqpoint{6.246926in}{1.738278in}}{\pgfqpoint{6.251970in}{1.733235in}}%
\pgfpathcurveto{\pgfqpoint{6.257013in}{1.728191in}}{\pgfqpoint{6.263855in}{1.725357in}}{\pgfqpoint{6.270988in}{1.725357in}}%
\pgfpathclose%
\pgfusepath{stroke,fill}%
\end{pgfscope}%
\begin{pgfscope}%
\pgfpathrectangle{\pgfqpoint{4.985294in}{0.500000in}}{\pgfqpoint{1.764706in}{1.700000in}}%
\pgfusepath{clip}%
\pgfsetbuttcap%
\pgfsetroundjoin%
\definecolor{currentfill}{rgb}{0.979124,0.903132,0.839793}%
\pgfsetfillcolor{currentfill}%
\pgfsetlinewidth{0.311001pt}%
\definecolor{currentstroke}{rgb}{1.000000,1.000000,1.000000}%
\pgfsetstrokecolor{currentstroke}%
\pgfsetdash{}{0pt}%
\pgfpathmoveto{\pgfqpoint{6.324707in}{1.202565in}}%
\pgfpathcurveto{\pgfqpoint{6.331840in}{1.202565in}}{\pgfqpoint{6.338682in}{1.205399in}}{\pgfqpoint{6.343725in}{1.210442in}}%
\pgfpathcurveto{\pgfqpoint{6.348769in}{1.215486in}}{\pgfqpoint{6.351603in}{1.222328in}}{\pgfqpoint{6.351603in}{1.229460in}}%
\pgfpathcurveto{\pgfqpoint{6.351603in}{1.236593in}}{\pgfqpoint{6.348769in}{1.243435in}}{\pgfqpoint{6.343725in}{1.248479in}}%
\pgfpathcurveto{\pgfqpoint{6.338682in}{1.253522in}}{\pgfqpoint{6.331840in}{1.256356in}}{\pgfqpoint{6.324707in}{1.256356in}}%
\pgfpathcurveto{\pgfqpoint{6.317574in}{1.256356in}}{\pgfqpoint{6.310733in}{1.253522in}}{\pgfqpoint{6.305689in}{1.248479in}}%
\pgfpathcurveto{\pgfqpoint{6.300645in}{1.243435in}}{\pgfqpoint{6.297811in}{1.236593in}}{\pgfqpoint{6.297811in}{1.229460in}}%
\pgfpathcurveto{\pgfqpoint{6.297811in}{1.222328in}}{\pgfqpoint{6.300645in}{1.215486in}}{\pgfqpoint{6.305689in}{1.210442in}}%
\pgfpathcurveto{\pgfqpoint{6.310733in}{1.205399in}}{\pgfqpoint{6.317574in}{1.202565in}}{\pgfqpoint{6.324707in}{1.202565in}}%
\pgfpathclose%
\pgfusepath{stroke,fill}%
\end{pgfscope}%
\begin{pgfscope}%
\pgfpathrectangle{\pgfqpoint{4.985294in}{0.500000in}}{\pgfqpoint{1.764706in}{1.700000in}}%
\pgfusepath{clip}%
\pgfsetbuttcap%
\pgfsetroundjoin%
\definecolor{currentfill}{rgb}{0.968105,0.786346,0.667739}%
\pgfsetfillcolor{currentfill}%
\pgfsetlinewidth{0.311001pt}%
\definecolor{currentstroke}{rgb}{1.000000,1.000000,1.000000}%
\pgfsetstrokecolor{currentstroke}%
\pgfsetdash{}{0pt}%
\pgfpathmoveto{\pgfqpoint{6.314063in}{1.049267in}}%
\pgfpathcurveto{\pgfqpoint{6.321196in}{1.049267in}}{\pgfqpoint{6.328037in}{1.052101in}}{\pgfqpoint{6.333081in}{1.057145in}}%
\pgfpathcurveto{\pgfqpoint{6.338124in}{1.062188in}}{\pgfqpoint{6.340958in}{1.069030in}}{\pgfqpoint{6.340958in}{1.076163in}}%
\pgfpathcurveto{\pgfqpoint{6.340958in}{1.083296in}}{\pgfqpoint{6.338124in}{1.090137in}}{\pgfqpoint{6.333081in}{1.095181in}}%
\pgfpathcurveto{\pgfqpoint{6.328037in}{1.100225in}}{\pgfqpoint{6.321196in}{1.103059in}}{\pgfqpoint{6.314063in}{1.103059in}}%
\pgfpathcurveto{\pgfqpoint{6.306930in}{1.103059in}}{\pgfqpoint{6.300088in}{1.100225in}}{\pgfqpoint{6.295045in}{1.095181in}}%
\pgfpathcurveto{\pgfqpoint{6.290001in}{1.090137in}}{\pgfqpoint{6.287167in}{1.083296in}}{\pgfqpoint{6.287167in}{1.076163in}}%
\pgfpathcurveto{\pgfqpoint{6.287167in}{1.069030in}}{\pgfqpoint{6.290001in}{1.062188in}}{\pgfqpoint{6.295045in}{1.057145in}}%
\pgfpathcurveto{\pgfqpoint{6.300088in}{1.052101in}}{\pgfqpoint{6.306930in}{1.049267in}}{\pgfqpoint{6.314063in}{1.049267in}}%
\pgfpathclose%
\pgfusepath{stroke,fill}%
\end{pgfscope}%
\begin{pgfscope}%
\pgfpathrectangle{\pgfqpoint{4.985294in}{0.500000in}}{\pgfqpoint{1.764706in}{1.700000in}}%
\pgfusepath{clip}%
\pgfsetbuttcap%
\pgfsetroundjoin%
\definecolor{currentfill}{rgb}{0.979891,0.908948,0.848279}%
\pgfsetfillcolor{currentfill}%
\pgfsetlinewidth{0.311001pt}%
\definecolor{currentstroke}{rgb}{1.000000,1.000000,1.000000}%
\pgfsetstrokecolor{currentstroke}%
\pgfsetdash{}{0pt}%
\pgfpathmoveto{\pgfqpoint{6.290790in}{1.401181in}}%
\pgfpathcurveto{\pgfqpoint{6.297923in}{1.401181in}}{\pgfqpoint{6.304765in}{1.404015in}}{\pgfqpoint{6.309808in}{1.409059in}}%
\pgfpathcurveto{\pgfqpoint{6.314852in}{1.414102in}}{\pgfqpoint{6.317686in}{1.420944in}}{\pgfqpoint{6.317686in}{1.428077in}}%
\pgfpathcurveto{\pgfqpoint{6.317686in}{1.435210in}}{\pgfqpoint{6.314852in}{1.442051in}}{\pgfqpoint{6.309808in}{1.447095in}}%
\pgfpathcurveto{\pgfqpoint{6.304765in}{1.452139in}}{\pgfqpoint{6.297923in}{1.454973in}}{\pgfqpoint{6.290790in}{1.454973in}}%
\pgfpathcurveto{\pgfqpoint{6.283657in}{1.454973in}}{\pgfqpoint{6.276816in}{1.452139in}}{\pgfqpoint{6.271772in}{1.447095in}}%
\pgfpathcurveto{\pgfqpoint{6.266728in}{1.442051in}}{\pgfqpoint{6.263894in}{1.435210in}}{\pgfqpoint{6.263894in}{1.428077in}}%
\pgfpathcurveto{\pgfqpoint{6.263894in}{1.420944in}}{\pgfqpoint{6.266728in}{1.414102in}}{\pgfqpoint{6.271772in}{1.409059in}}%
\pgfpathcurveto{\pgfqpoint{6.276816in}{1.404015in}}{\pgfqpoint{6.283657in}{1.401181in}}{\pgfqpoint{6.290790in}{1.401181in}}%
\pgfpathclose%
\pgfusepath{stroke,fill}%
\end{pgfscope}%
\begin{pgfscope}%
\pgfpathrectangle{\pgfqpoint{4.985294in}{0.500000in}}{\pgfqpoint{1.764706in}{1.700000in}}%
\pgfusepath{clip}%
\pgfsetbuttcap%
\pgfsetroundjoin%
\definecolor{currentfill}{rgb}{0.972726,0.844889,0.754401}%
\pgfsetfillcolor{currentfill}%
\pgfsetlinewidth{0.311001pt}%
\definecolor{currentstroke}{rgb}{1.000000,1.000000,1.000000}%
\pgfsetstrokecolor{currentstroke}%
\pgfsetdash{}{0pt}%
\pgfpathmoveto{\pgfqpoint{5.507798in}{1.004848in}}%
\pgfpathcurveto{\pgfqpoint{5.514931in}{1.004848in}}{\pgfqpoint{5.521773in}{1.007682in}}{\pgfqpoint{5.526816in}{1.012726in}}%
\pgfpathcurveto{\pgfqpoint{5.531860in}{1.017770in}}{\pgfqpoint{5.534694in}{1.024611in}}{\pgfqpoint{5.534694in}{1.031744in}}%
\pgfpathcurveto{\pgfqpoint{5.534694in}{1.038877in}}{\pgfqpoint{5.531860in}{1.045718in}}{\pgfqpoint{5.526816in}{1.050762in}}%
\pgfpathcurveto{\pgfqpoint{5.521773in}{1.055806in}}{\pgfqpoint{5.514931in}{1.058640in}}{\pgfqpoint{5.507798in}{1.058640in}}%
\pgfpathcurveto{\pgfqpoint{5.500665in}{1.058640in}}{\pgfqpoint{5.493824in}{1.055806in}}{\pgfqpoint{5.488780in}{1.050762in}}%
\pgfpathcurveto{\pgfqpoint{5.483736in}{1.045718in}}{\pgfqpoint{5.480902in}{1.038877in}}{\pgfqpoint{5.480902in}{1.031744in}}%
\pgfpathcurveto{\pgfqpoint{5.480902in}{1.024611in}}{\pgfqpoint{5.483736in}{1.017770in}}{\pgfqpoint{5.488780in}{1.012726in}}%
\pgfpathcurveto{\pgfqpoint{5.493824in}{1.007682in}}{\pgfqpoint{5.500665in}{1.004848in}}{\pgfqpoint{5.507798in}{1.004848in}}%
\pgfpathclose%
\pgfusepath{stroke,fill}%
\end{pgfscope}%
\begin{pgfscope}%
\pgfpathrectangle{\pgfqpoint{4.985294in}{0.500000in}}{\pgfqpoint{1.764706in}{1.700000in}}%
\pgfusepath{clip}%
\pgfsetbuttcap%
\pgfsetroundjoin%
\definecolor{currentfill}{rgb}{0.958791,0.526283,0.368909}%
\pgfsetfillcolor{currentfill}%
\pgfsetlinewidth{0.311001pt}%
\definecolor{currentstroke}{rgb}{1.000000,1.000000,1.000000}%
\pgfsetstrokecolor{currentstroke}%
\pgfsetdash{}{0pt}%
\pgfpathmoveto{\pgfqpoint{6.118767in}{1.539785in}}%
\pgfpathcurveto{\pgfqpoint{6.125900in}{1.539785in}}{\pgfqpoint{6.132742in}{1.542619in}}{\pgfqpoint{6.137785in}{1.547663in}}%
\pgfpathcurveto{\pgfqpoint{6.142829in}{1.552707in}}{\pgfqpoint{6.145663in}{1.559548in}}{\pgfqpoint{6.145663in}{1.566681in}}%
\pgfpathcurveto{\pgfqpoint{6.145663in}{1.573814in}}{\pgfqpoint{6.142829in}{1.580656in}}{\pgfqpoint{6.137785in}{1.585699in}}%
\pgfpathcurveto{\pgfqpoint{6.132742in}{1.590743in}}{\pgfqpoint{6.125900in}{1.593577in}}{\pgfqpoint{6.118767in}{1.593577in}}%
\pgfpathcurveto{\pgfqpoint{6.111634in}{1.593577in}}{\pgfqpoint{6.104793in}{1.590743in}}{\pgfqpoint{6.099749in}{1.585699in}}%
\pgfpathcurveto{\pgfqpoint{6.094705in}{1.580656in}}{\pgfqpoint{6.091871in}{1.573814in}}{\pgfqpoint{6.091871in}{1.566681in}}%
\pgfpathcurveto{\pgfqpoint{6.091871in}{1.559548in}}{\pgfqpoint{6.094705in}{1.552707in}}{\pgfqpoint{6.099749in}{1.547663in}}%
\pgfpathcurveto{\pgfqpoint{6.104793in}{1.542619in}}{\pgfqpoint{6.111634in}{1.539785in}}{\pgfqpoint{6.118767in}{1.539785in}}%
\pgfpathclose%
\pgfusepath{stroke,fill}%
\end{pgfscope}%
\begin{pgfscope}%
\pgfpathrectangle{\pgfqpoint{4.985294in}{0.500000in}}{\pgfqpoint{1.764706in}{1.700000in}}%
\pgfusepath{clip}%
\pgfsetbuttcap%
\pgfsetroundjoin%
\definecolor{currentfill}{rgb}{0.934351,0.329284,0.247753}%
\pgfsetfillcolor{currentfill}%
\pgfsetlinewidth{0.311001pt}%
\definecolor{currentstroke}{rgb}{1.000000,1.000000,1.000000}%
\pgfsetstrokecolor{currentstroke}%
\pgfsetdash{}{0pt}%
\pgfpathmoveto{\pgfqpoint{6.414902in}{1.050282in}}%
\pgfpathcurveto{\pgfqpoint{6.422035in}{1.050282in}}{\pgfqpoint{6.428876in}{1.053116in}}{\pgfqpoint{6.433920in}{1.058159in}}%
\pgfpathcurveto{\pgfqpoint{6.438964in}{1.063203in}}{\pgfqpoint{6.441798in}{1.070045in}}{\pgfqpoint{6.441798in}{1.077177in}}%
\pgfpathcurveto{\pgfqpoint{6.441798in}{1.084310in}}{\pgfqpoint{6.438964in}{1.091152in}}{\pgfqpoint{6.433920in}{1.096196in}}%
\pgfpathcurveto{\pgfqpoint{6.428876in}{1.101239in}}{\pgfqpoint{6.422035in}{1.104073in}}{\pgfqpoint{6.414902in}{1.104073in}}%
\pgfpathcurveto{\pgfqpoint{6.407769in}{1.104073in}}{\pgfqpoint{6.400928in}{1.101239in}}{\pgfqpoint{6.395884in}{1.096196in}}%
\pgfpathcurveto{\pgfqpoint{6.390840in}{1.091152in}}{\pgfqpoint{6.388006in}{1.084310in}}{\pgfqpoint{6.388006in}{1.077177in}}%
\pgfpathcurveto{\pgfqpoint{6.388006in}{1.070045in}}{\pgfqpoint{6.390840in}{1.063203in}}{\pgfqpoint{6.395884in}{1.058159in}}%
\pgfpathcurveto{\pgfqpoint{6.400928in}{1.053116in}}{\pgfqpoint{6.407769in}{1.050282in}}{\pgfqpoint{6.414902in}{1.050282in}}%
\pgfpathclose%
\pgfusepath{stroke,fill}%
\end{pgfscope}%
\begin{pgfscope}%
\pgfpathrectangle{\pgfqpoint{4.985294in}{0.500000in}}{\pgfqpoint{1.764706in}{1.700000in}}%
\pgfusepath{clip}%
\pgfsetbuttcap%
\pgfsetroundjoin%
\definecolor{currentfill}{rgb}{0.967092,0.768560,0.642079}%
\pgfsetfillcolor{currentfill}%
\pgfsetlinewidth{0.311001pt}%
\definecolor{currentstroke}{rgb}{1.000000,1.000000,1.000000}%
\pgfsetstrokecolor{currentstroke}%
\pgfsetdash{}{0pt}%
\pgfpathmoveto{\pgfqpoint{6.316093in}{1.042702in}}%
\pgfpathcurveto{\pgfqpoint{6.323226in}{1.042702in}}{\pgfqpoint{6.330067in}{1.045536in}}{\pgfqpoint{6.335111in}{1.050580in}}%
\pgfpathcurveto{\pgfqpoint{6.340155in}{1.055623in}}{\pgfqpoint{6.342989in}{1.062465in}}{\pgfqpoint{6.342989in}{1.069598in}}%
\pgfpathcurveto{\pgfqpoint{6.342989in}{1.076731in}}{\pgfqpoint{6.340155in}{1.083572in}}{\pgfqpoint{6.335111in}{1.088616in}}%
\pgfpathcurveto{\pgfqpoint{6.330067in}{1.093660in}}{\pgfqpoint{6.323226in}{1.096494in}}{\pgfqpoint{6.316093in}{1.096494in}}%
\pgfpathcurveto{\pgfqpoint{6.308960in}{1.096494in}}{\pgfqpoint{6.302119in}{1.093660in}}{\pgfqpoint{6.297075in}{1.088616in}}%
\pgfpathcurveto{\pgfqpoint{6.292031in}{1.083572in}}{\pgfqpoint{6.289197in}{1.076731in}}{\pgfqpoint{6.289197in}{1.069598in}}%
\pgfpathcurveto{\pgfqpoint{6.289197in}{1.062465in}}{\pgfqpoint{6.292031in}{1.055623in}}{\pgfqpoint{6.297075in}{1.050580in}}%
\pgfpathcurveto{\pgfqpoint{6.302119in}{1.045536in}}{\pgfqpoint{6.308960in}{1.042702in}}{\pgfqpoint{6.316093in}{1.042702in}}%
\pgfpathclose%
\pgfusepath{stroke,fill}%
\end{pgfscope}%
\begin{pgfscope}%
\pgfpathrectangle{\pgfqpoint{4.985294in}{0.500000in}}{\pgfqpoint{1.764706in}{1.700000in}}%
\pgfusepath{clip}%
\pgfsetbuttcap%
\pgfsetroundjoin%
\definecolor{currentfill}{rgb}{0.969359,0.803954,0.693832}%
\pgfsetfillcolor{currentfill}%
\pgfsetlinewidth{0.311001pt}%
\definecolor{currentstroke}{rgb}{1.000000,1.000000,1.000000}%
\pgfsetstrokecolor{currentstroke}%
\pgfsetdash{}{0pt}%
\pgfpathmoveto{\pgfqpoint{6.226944in}{1.269337in}}%
\pgfpathcurveto{\pgfqpoint{6.234077in}{1.269337in}}{\pgfqpoint{6.240918in}{1.272171in}}{\pgfqpoint{6.245962in}{1.277215in}}%
\pgfpathcurveto{\pgfqpoint{6.251006in}{1.282259in}}{\pgfqpoint{6.253839in}{1.289100in}}{\pgfqpoint{6.253839in}{1.296233in}}%
\pgfpathcurveto{\pgfqpoint{6.253839in}{1.303366in}}{\pgfqpoint{6.251006in}{1.310208in}}{\pgfqpoint{6.245962in}{1.315251in}}%
\pgfpathcurveto{\pgfqpoint{6.240918in}{1.320295in}}{\pgfqpoint{6.234077in}{1.323129in}}{\pgfqpoint{6.226944in}{1.323129in}}%
\pgfpathcurveto{\pgfqpoint{6.219811in}{1.323129in}}{\pgfqpoint{6.212969in}{1.320295in}}{\pgfqpoint{6.207926in}{1.315251in}}%
\pgfpathcurveto{\pgfqpoint{6.202882in}{1.310208in}}{\pgfqpoint{6.200048in}{1.303366in}}{\pgfqpoint{6.200048in}{1.296233in}}%
\pgfpathcurveto{\pgfqpoint{6.200048in}{1.289100in}}{\pgfqpoint{6.202882in}{1.282259in}}{\pgfqpoint{6.207926in}{1.277215in}}%
\pgfpathcurveto{\pgfqpoint{6.212969in}{1.272171in}}{\pgfqpoint{6.219811in}{1.269337in}}{\pgfqpoint{6.226944in}{1.269337in}}%
\pgfpathclose%
\pgfusepath{stroke,fill}%
\end{pgfscope}%
\begin{pgfscope}%
\pgfpathrectangle{\pgfqpoint{4.985294in}{0.500000in}}{\pgfqpoint{1.764706in}{1.700000in}}%
\pgfusepath{clip}%
\pgfsetbuttcap%
\pgfsetroundjoin%
\definecolor{currentfill}{rgb}{0.977657,0.891500,0.822809}%
\pgfsetfillcolor{currentfill}%
\pgfsetlinewidth{0.311001pt}%
\definecolor{currentstroke}{rgb}{1.000000,1.000000,1.000000}%
\pgfsetstrokecolor{currentstroke}%
\pgfsetdash{}{0pt}%
\pgfpathmoveto{\pgfqpoint{5.456479in}{1.466347in}}%
\pgfpathcurveto{\pgfqpoint{5.463612in}{1.466347in}}{\pgfqpoint{5.470454in}{1.469181in}}{\pgfqpoint{5.475498in}{1.474225in}}%
\pgfpathcurveto{\pgfqpoint{5.480541in}{1.479269in}}{\pgfqpoint{5.483375in}{1.486110in}}{\pgfqpoint{5.483375in}{1.493243in}}%
\pgfpathcurveto{\pgfqpoint{5.483375in}{1.500376in}}{\pgfqpoint{5.480541in}{1.507218in}}{\pgfqpoint{5.475498in}{1.512261in}}%
\pgfpathcurveto{\pgfqpoint{5.470454in}{1.517305in}}{\pgfqpoint{5.463612in}{1.520139in}}{\pgfqpoint{5.456479in}{1.520139in}}%
\pgfpathcurveto{\pgfqpoint{5.449347in}{1.520139in}}{\pgfqpoint{5.442505in}{1.517305in}}{\pgfqpoint{5.437461in}{1.512261in}}%
\pgfpathcurveto{\pgfqpoint{5.432418in}{1.507218in}}{\pgfqpoint{5.429584in}{1.500376in}}{\pgfqpoint{5.429584in}{1.493243in}}%
\pgfpathcurveto{\pgfqpoint{5.429584in}{1.486110in}}{\pgfqpoint{5.432418in}{1.479269in}}{\pgfqpoint{5.437461in}{1.474225in}}%
\pgfpathcurveto{\pgfqpoint{5.442505in}{1.469181in}}{\pgfqpoint{5.449347in}{1.466347in}}{\pgfqpoint{5.456479in}{1.466347in}}%
\pgfpathclose%
\pgfusepath{stroke,fill}%
\end{pgfscope}%
\begin{pgfscope}%
\pgfpathrectangle{\pgfqpoint{4.985294in}{0.500000in}}{\pgfqpoint{1.764706in}{1.700000in}}%
\pgfusepath{clip}%
\pgfsetbuttcap%
\pgfsetroundjoin%
\definecolor{currentfill}{rgb}{0.974412,0.862387,0.780156}%
\pgfsetfillcolor{currentfill}%
\pgfsetlinewidth{0.311001pt}%
\definecolor{currentstroke}{rgb}{1.000000,1.000000,1.000000}%
\pgfsetstrokecolor{currentstroke}%
\pgfsetdash{}{0pt}%
\pgfpathmoveto{\pgfqpoint{6.263959in}{1.317809in}}%
\pgfpathcurveto{\pgfqpoint{6.271091in}{1.317809in}}{\pgfqpoint{6.277933in}{1.320643in}}{\pgfqpoint{6.282977in}{1.325686in}}%
\pgfpathcurveto{\pgfqpoint{6.288020in}{1.330730in}}{\pgfqpoint{6.290854in}{1.337572in}}{\pgfqpoint{6.290854in}{1.344704in}}%
\pgfpathcurveto{\pgfqpoint{6.290854in}{1.351837in}}{\pgfqpoint{6.288020in}{1.358679in}}{\pgfqpoint{6.282977in}{1.363723in}}%
\pgfpathcurveto{\pgfqpoint{6.277933in}{1.368766in}}{\pgfqpoint{6.271091in}{1.371600in}}{\pgfqpoint{6.263959in}{1.371600in}}%
\pgfpathcurveto{\pgfqpoint{6.256826in}{1.371600in}}{\pgfqpoint{6.249984in}{1.368766in}}{\pgfqpoint{6.244940in}{1.363723in}}%
\pgfpathcurveto{\pgfqpoint{6.239897in}{1.358679in}}{\pgfqpoint{6.237063in}{1.351837in}}{\pgfqpoint{6.237063in}{1.344704in}}%
\pgfpathcurveto{\pgfqpoint{6.237063in}{1.337572in}}{\pgfqpoint{6.239897in}{1.330730in}}{\pgfqpoint{6.244940in}{1.325686in}}%
\pgfpathcurveto{\pgfqpoint{6.249984in}{1.320643in}}{\pgfqpoint{6.256826in}{1.317809in}}{\pgfqpoint{6.263959in}{1.317809in}}%
\pgfpathclose%
\pgfusepath{stroke,fill}%
\end{pgfscope}%
\begin{pgfscope}%
\pgfpathrectangle{\pgfqpoint{4.985294in}{0.500000in}}{\pgfqpoint{1.764706in}{1.700000in}}%
\pgfusepath{clip}%
\pgfsetbuttcap%
\pgfsetroundjoin%
\definecolor{currentfill}{rgb}{0.964679,0.682838,0.530002}%
\pgfsetfillcolor{currentfill}%
\pgfsetlinewidth{0.311001pt}%
\definecolor{currentstroke}{rgb}{1.000000,1.000000,1.000000}%
\pgfsetstrokecolor{currentstroke}%
\pgfsetdash{}{0pt}%
\pgfpathmoveto{\pgfqpoint{6.116914in}{1.620023in}}%
\pgfpathcurveto{\pgfqpoint{6.124047in}{1.620023in}}{\pgfqpoint{6.130889in}{1.622857in}}{\pgfqpoint{6.135933in}{1.627900in}}%
\pgfpathcurveto{\pgfqpoint{6.140976in}{1.632944in}}{\pgfqpoint{6.143810in}{1.639786in}}{\pgfqpoint{6.143810in}{1.646918in}}%
\pgfpathcurveto{\pgfqpoint{6.143810in}{1.654051in}}{\pgfqpoint{6.140976in}{1.660893in}}{\pgfqpoint{6.135933in}{1.665937in}}%
\pgfpathcurveto{\pgfqpoint{6.130889in}{1.670980in}}{\pgfqpoint{6.124047in}{1.673814in}}{\pgfqpoint{6.116914in}{1.673814in}}%
\pgfpathcurveto{\pgfqpoint{6.109782in}{1.673814in}}{\pgfqpoint{6.102940in}{1.670980in}}{\pgfqpoint{6.097896in}{1.665937in}}%
\pgfpathcurveto{\pgfqpoint{6.092853in}{1.660893in}}{\pgfqpoint{6.090019in}{1.654051in}}{\pgfqpoint{6.090019in}{1.646918in}}%
\pgfpathcurveto{\pgfqpoint{6.090019in}{1.639786in}}{\pgfqpoint{6.092853in}{1.632944in}}{\pgfqpoint{6.097896in}{1.627900in}}%
\pgfpathcurveto{\pgfqpoint{6.102940in}{1.622857in}}{\pgfqpoint{6.109782in}{1.620023in}}{\pgfqpoint{6.116914in}{1.620023in}}%
\pgfpathclose%
\pgfusepath{stroke,fill}%
\end{pgfscope}%
\begin{pgfscope}%
\pgfpathrectangle{\pgfqpoint{4.985294in}{0.500000in}}{\pgfqpoint{1.764706in}{1.700000in}}%
\pgfusepath{clip}%
\pgfsetbuttcap%
\pgfsetroundjoin%
\definecolor{currentfill}{rgb}{0.968105,0.786346,0.667739}%
\pgfsetfillcolor{currentfill}%
\pgfsetlinewidth{0.311001pt}%
\definecolor{currentstroke}{rgb}{1.000000,1.000000,1.000000}%
\pgfsetstrokecolor{currentstroke}%
\pgfsetdash{}{0pt}%
\pgfpathmoveto{\pgfqpoint{5.516325in}{1.431086in}}%
\pgfpathcurveto{\pgfqpoint{5.523458in}{1.431086in}}{\pgfqpoint{5.530300in}{1.433920in}}{\pgfqpoint{5.535343in}{1.438964in}}%
\pgfpathcurveto{\pgfqpoint{5.540387in}{1.444008in}}{\pgfqpoint{5.543221in}{1.450849in}}{\pgfqpoint{5.543221in}{1.457982in}}%
\pgfpathcurveto{\pgfqpoint{5.543221in}{1.465115in}}{\pgfqpoint{5.540387in}{1.471956in}}{\pgfqpoint{5.535343in}{1.477000in}}%
\pgfpathcurveto{\pgfqpoint{5.530300in}{1.482044in}}{\pgfqpoint{5.523458in}{1.484878in}}{\pgfqpoint{5.516325in}{1.484878in}}%
\pgfpathcurveto{\pgfqpoint{5.509192in}{1.484878in}}{\pgfqpoint{5.502351in}{1.482044in}}{\pgfqpoint{5.497307in}{1.477000in}}%
\pgfpathcurveto{\pgfqpoint{5.492263in}{1.471956in}}{\pgfqpoint{5.489430in}{1.465115in}}{\pgfqpoint{5.489430in}{1.457982in}}%
\pgfpathcurveto{\pgfqpoint{5.489430in}{1.450849in}}{\pgfqpoint{5.492263in}{1.444008in}}{\pgfqpoint{5.497307in}{1.438964in}}%
\pgfpathcurveto{\pgfqpoint{5.502351in}{1.433920in}}{\pgfqpoint{5.509192in}{1.431086in}}{\pgfqpoint{5.516325in}{1.431086in}}%
\pgfpathclose%
\pgfusepath{stroke,fill}%
\end{pgfscope}%
\begin{pgfscope}%
\pgfpathrectangle{\pgfqpoint{4.985294in}{0.500000in}}{\pgfqpoint{1.764706in}{1.700000in}}%
\pgfusepath{clip}%
\pgfsetbuttcap%
\pgfsetroundjoin%
\definecolor{currentfill}{rgb}{0.973832,0.856556,0.771584}%
\pgfsetfillcolor{currentfill}%
\pgfsetlinewidth{0.311001pt}%
\definecolor{currentstroke}{rgb}{1.000000,1.000000,1.000000}%
\pgfsetstrokecolor{currentstroke}%
\pgfsetdash{}{0pt}%
\pgfpathmoveto{\pgfqpoint{5.475474in}{1.222961in}}%
\pgfpathcurveto{\pgfqpoint{5.482607in}{1.222961in}}{\pgfqpoint{5.489448in}{1.225795in}}{\pgfqpoint{5.494492in}{1.230839in}}%
\pgfpathcurveto{\pgfqpoint{5.499536in}{1.235883in}}{\pgfqpoint{5.502370in}{1.242724in}}{\pgfqpoint{5.502370in}{1.249857in}}%
\pgfpathcurveto{\pgfqpoint{5.502370in}{1.256990in}}{\pgfqpoint{5.499536in}{1.263832in}}{\pgfqpoint{5.494492in}{1.268875in}}%
\pgfpathcurveto{\pgfqpoint{5.489448in}{1.273919in}}{\pgfqpoint{5.482607in}{1.276753in}}{\pgfqpoint{5.475474in}{1.276753in}}%
\pgfpathcurveto{\pgfqpoint{5.468341in}{1.276753in}}{\pgfqpoint{5.461499in}{1.273919in}}{\pgfqpoint{5.456456in}{1.268875in}}%
\pgfpathcurveto{\pgfqpoint{5.451412in}{1.263832in}}{\pgfqpoint{5.448578in}{1.256990in}}{\pgfqpoint{5.448578in}{1.249857in}}%
\pgfpathcurveto{\pgfqpoint{5.448578in}{1.242724in}}{\pgfqpoint{5.451412in}{1.235883in}}{\pgfqpoint{5.456456in}{1.230839in}}%
\pgfpathcurveto{\pgfqpoint{5.461499in}{1.225795in}}{\pgfqpoint{5.468341in}{1.222961in}}{\pgfqpoint{5.475474in}{1.222961in}}%
\pgfpathclose%
\pgfusepath{stroke,fill}%
\end{pgfscope}%
\begin{pgfscope}%
\pgfpathrectangle{\pgfqpoint{4.985294in}{0.500000in}}{\pgfqpoint{1.764706in}{1.700000in}}%
\pgfusepath{clip}%
\pgfsetbuttcap%
\pgfsetroundjoin%
\definecolor{currentfill}{rgb}{0.818205,0.120806,0.299261}%
\pgfsetfillcolor{currentfill}%
\pgfsetlinewidth{0.311001pt}%
\definecolor{currentstroke}{rgb}{1.000000,1.000000,1.000000}%
\pgfsetstrokecolor{currentstroke}%
\pgfsetdash{}{0pt}%
\pgfpathmoveto{\pgfqpoint{6.471833in}{1.497512in}}%
\pgfpathcurveto{\pgfqpoint{6.478966in}{1.497512in}}{\pgfqpoint{6.485808in}{1.500346in}}{\pgfqpoint{6.490852in}{1.505389in}}%
\pgfpathcurveto{\pgfqpoint{6.495895in}{1.510433in}}{\pgfqpoint{6.498729in}{1.517275in}}{\pgfqpoint{6.498729in}{1.524408in}}%
\pgfpathcurveto{\pgfqpoint{6.498729in}{1.531540in}}{\pgfqpoint{6.495895in}{1.538382in}}{\pgfqpoint{6.490852in}{1.543426in}}%
\pgfpathcurveto{\pgfqpoint{6.485808in}{1.548469in}}{\pgfqpoint{6.478966in}{1.551303in}}{\pgfqpoint{6.471833in}{1.551303in}}%
\pgfpathcurveto{\pgfqpoint{6.464701in}{1.551303in}}{\pgfqpoint{6.457859in}{1.548469in}}{\pgfqpoint{6.452815in}{1.543426in}}%
\pgfpathcurveto{\pgfqpoint{6.447772in}{1.538382in}}{\pgfqpoint{6.444938in}{1.531540in}}{\pgfqpoint{6.444938in}{1.524408in}}%
\pgfpathcurveto{\pgfqpoint{6.444938in}{1.517275in}}{\pgfqpoint{6.447772in}{1.510433in}}{\pgfqpoint{6.452815in}{1.505389in}}%
\pgfpathcurveto{\pgfqpoint{6.457859in}{1.500346in}}{\pgfqpoint{6.464701in}{1.497512in}}{\pgfqpoint{6.471833in}{1.497512in}}%
\pgfpathclose%
\pgfusepath{stroke,fill}%
\end{pgfscope}%
\begin{pgfscope}%
\pgfpathrectangle{\pgfqpoint{4.985294in}{0.500000in}}{\pgfqpoint{1.764706in}{1.700000in}}%
\pgfusepath{clip}%
\pgfsetbuttcap%
\pgfsetroundjoin%
\definecolor{currentfill}{rgb}{0.967092,0.768560,0.642079}%
\pgfsetfillcolor{currentfill}%
\pgfsetlinewidth{0.311001pt}%
\definecolor{currentstroke}{rgb}{1.000000,1.000000,1.000000}%
\pgfsetstrokecolor{currentstroke}%
\pgfsetdash{}{0pt}%
\pgfpathmoveto{\pgfqpoint{6.396387in}{1.340795in}}%
\pgfpathcurveto{\pgfqpoint{6.403520in}{1.340795in}}{\pgfqpoint{6.410362in}{1.343629in}}{\pgfqpoint{6.415405in}{1.348672in}}%
\pgfpathcurveto{\pgfqpoint{6.420449in}{1.353716in}}{\pgfqpoint{6.423283in}{1.360558in}}{\pgfqpoint{6.423283in}{1.367691in}}%
\pgfpathcurveto{\pgfqpoint{6.423283in}{1.374823in}}{\pgfqpoint{6.420449in}{1.381665in}}{\pgfqpoint{6.415405in}{1.386709in}}%
\pgfpathcurveto{\pgfqpoint{6.410362in}{1.391752in}}{\pgfqpoint{6.403520in}{1.394586in}}{\pgfqpoint{6.396387in}{1.394586in}}%
\pgfpathcurveto{\pgfqpoint{6.389254in}{1.394586in}}{\pgfqpoint{6.382413in}{1.391752in}}{\pgfqpoint{6.377369in}{1.386709in}}%
\pgfpathcurveto{\pgfqpoint{6.372325in}{1.381665in}}{\pgfqpoint{6.369492in}{1.374823in}}{\pgfqpoint{6.369492in}{1.367691in}}%
\pgfpathcurveto{\pgfqpoint{6.369492in}{1.360558in}}{\pgfqpoint{6.372325in}{1.353716in}}{\pgfqpoint{6.377369in}{1.348672in}}%
\pgfpathcurveto{\pgfqpoint{6.382413in}{1.343629in}}{\pgfqpoint{6.389254in}{1.340795in}}{\pgfqpoint{6.396387in}{1.340795in}}%
\pgfpathclose%
\pgfusepath{stroke,fill}%
\end{pgfscope}%
\begin{pgfscope}%
\pgfpathrectangle{\pgfqpoint{4.985294in}{0.500000in}}{\pgfqpoint{1.764706in}{1.700000in}}%
\pgfusepath{clip}%
\pgfsetbuttcap%
\pgfsetroundjoin%
\definecolor{currentfill}{rgb}{0.966328,0.750560,0.616961}%
\pgfsetfillcolor{currentfill}%
\pgfsetlinewidth{0.311001pt}%
\definecolor{currentstroke}{rgb}{1.000000,1.000000,1.000000}%
\pgfsetstrokecolor{currentstroke}%
\pgfsetdash{}{0pt}%
\pgfpathmoveto{\pgfqpoint{5.592109in}{0.981275in}}%
\pgfpathcurveto{\pgfqpoint{5.599242in}{0.981275in}}{\pgfqpoint{5.606083in}{0.984109in}}{\pgfqpoint{5.611127in}{0.989153in}}%
\pgfpathcurveto{\pgfqpoint{5.616171in}{0.994196in}}{\pgfqpoint{5.619005in}{1.001038in}}{\pgfqpoint{5.619005in}{1.008171in}}%
\pgfpathcurveto{\pgfqpoint{5.619005in}{1.015304in}}{\pgfqpoint{5.616171in}{1.022145in}}{\pgfqpoint{5.611127in}{1.027189in}}%
\pgfpathcurveto{\pgfqpoint{5.606083in}{1.032233in}}{\pgfqpoint{5.599242in}{1.035066in}}{\pgfqpoint{5.592109in}{1.035066in}}%
\pgfpathcurveto{\pgfqpoint{5.584976in}{1.035066in}}{\pgfqpoint{5.578134in}{1.032233in}}{\pgfqpoint{5.573091in}{1.027189in}}%
\pgfpathcurveto{\pgfqpoint{5.568047in}{1.022145in}}{\pgfqpoint{5.565213in}{1.015304in}}{\pgfqpoint{5.565213in}{1.008171in}}%
\pgfpathcurveto{\pgfqpoint{5.565213in}{1.001038in}}{\pgfqpoint{5.568047in}{0.994196in}}{\pgfqpoint{5.573091in}{0.989153in}}%
\pgfpathcurveto{\pgfqpoint{5.578134in}{0.984109in}}{\pgfqpoint{5.584976in}{0.981275in}}{\pgfqpoint{5.592109in}{0.981275in}}%
\pgfpathclose%
\pgfusepath{stroke,fill}%
\end{pgfscope}%
\begin{pgfscope}%
\pgfpathrectangle{\pgfqpoint{4.985294in}{0.500000in}}{\pgfqpoint{1.764706in}{1.700000in}}%
\pgfusepath{clip}%
\pgfsetbuttcap%
\pgfsetroundjoin%
\definecolor{currentfill}{rgb}{0.968931,0.798091,0.685123}%
\pgfsetfillcolor{currentfill}%
\pgfsetlinewidth{0.311001pt}%
\definecolor{currentstroke}{rgb}{1.000000,1.000000,1.000000}%
\pgfsetstrokecolor{currentstroke}%
\pgfsetdash{}{0pt}%
\pgfpathmoveto{\pgfqpoint{5.498159in}{1.242169in}}%
\pgfpathcurveto{\pgfqpoint{5.505292in}{1.242169in}}{\pgfqpoint{5.512133in}{1.245002in}}{\pgfqpoint{5.517177in}{1.250046in}}%
\pgfpathcurveto{\pgfqpoint{5.522221in}{1.255090in}}{\pgfqpoint{5.525054in}{1.261931in}}{\pgfqpoint{5.525054in}{1.269064in}}%
\pgfpathcurveto{\pgfqpoint{5.525054in}{1.276197in}}{\pgfqpoint{5.522221in}{1.283039in}}{\pgfqpoint{5.517177in}{1.288082in}}%
\pgfpathcurveto{\pgfqpoint{5.512133in}{1.293126in}}{\pgfqpoint{5.505292in}{1.295960in}}{\pgfqpoint{5.498159in}{1.295960in}}%
\pgfpathcurveto{\pgfqpoint{5.491026in}{1.295960in}}{\pgfqpoint{5.484184in}{1.293126in}}{\pgfqpoint{5.479141in}{1.288082in}}%
\pgfpathcurveto{\pgfqpoint{5.474097in}{1.283039in}}{\pgfqpoint{5.471263in}{1.276197in}}{\pgfqpoint{5.471263in}{1.269064in}}%
\pgfpathcurveto{\pgfqpoint{5.471263in}{1.261931in}}{\pgfqpoint{5.474097in}{1.255090in}}{\pgfqpoint{5.479141in}{1.250046in}}%
\pgfpathcurveto{\pgfqpoint{5.484184in}{1.245002in}}{\pgfqpoint{5.491026in}{1.242169in}}{\pgfqpoint{5.498159in}{1.242169in}}%
\pgfpathclose%
\pgfusepath{stroke,fill}%
\end{pgfscope}%
\begin{pgfscope}%
\pgfpathrectangle{\pgfqpoint{4.985294in}{0.500000in}}{\pgfqpoint{1.764706in}{1.700000in}}%
\pgfusepath{clip}%
\pgfsetbuttcap%
\pgfsetroundjoin%
\definecolor{currentfill}{rgb}{0.964173,0.657587,0.500469}%
\pgfsetfillcolor{currentfill}%
\pgfsetlinewidth{0.311001pt}%
\definecolor{currentstroke}{rgb}{1.000000,1.000000,1.000000}%
\pgfsetstrokecolor{currentstroke}%
\pgfsetdash{}{0pt}%
\pgfpathmoveto{\pgfqpoint{6.204292in}{1.737640in}}%
\pgfpathcurveto{\pgfqpoint{6.211425in}{1.737640in}}{\pgfqpoint{6.218266in}{1.740474in}}{\pgfqpoint{6.223310in}{1.745518in}}%
\pgfpathcurveto{\pgfqpoint{6.228354in}{1.750562in}}{\pgfqpoint{6.231188in}{1.757403in}}{\pgfqpoint{6.231188in}{1.764536in}}%
\pgfpathcurveto{\pgfqpoint{6.231188in}{1.771669in}}{\pgfqpoint{6.228354in}{1.778511in}}{\pgfqpoint{6.223310in}{1.783554in}}%
\pgfpathcurveto{\pgfqpoint{6.218266in}{1.788598in}}{\pgfqpoint{6.211425in}{1.791432in}}{\pgfqpoint{6.204292in}{1.791432in}}%
\pgfpathcurveto{\pgfqpoint{6.197159in}{1.791432in}}{\pgfqpoint{6.190317in}{1.788598in}}{\pgfqpoint{6.185274in}{1.783554in}}%
\pgfpathcurveto{\pgfqpoint{6.180230in}{1.778511in}}{\pgfqpoint{6.177396in}{1.771669in}}{\pgfqpoint{6.177396in}{1.764536in}}%
\pgfpathcurveto{\pgfqpoint{6.177396in}{1.757403in}}{\pgfqpoint{6.180230in}{1.750562in}}{\pgfqpoint{6.185274in}{1.745518in}}%
\pgfpathcurveto{\pgfqpoint{6.190317in}{1.740474in}}{\pgfqpoint{6.197159in}{1.737640in}}{\pgfqpoint{6.204292in}{1.737640in}}%
\pgfpathclose%
\pgfusepath{stroke,fill}%
\end{pgfscope}%
\begin{pgfscope}%
\pgfpathrectangle{\pgfqpoint{4.985294in}{0.500000in}}{\pgfqpoint{1.764706in}{1.700000in}}%
\pgfusepath{clip}%
\pgfsetbuttcap%
\pgfsetroundjoin%
\definecolor{currentfill}{rgb}{0.979124,0.903132,0.839793}%
\pgfsetfillcolor{currentfill}%
\pgfsetlinewidth{0.311001pt}%
\definecolor{currentstroke}{rgb}{1.000000,1.000000,1.000000}%
\pgfsetstrokecolor{currentstroke}%
\pgfsetdash{}{0pt}%
\pgfpathmoveto{\pgfqpoint{6.289132in}{1.403852in}}%
\pgfpathcurveto{\pgfqpoint{6.296265in}{1.403852in}}{\pgfqpoint{6.303106in}{1.406686in}}{\pgfqpoint{6.308150in}{1.411730in}}%
\pgfpathcurveto{\pgfqpoint{6.313194in}{1.416773in}}{\pgfqpoint{6.316028in}{1.423615in}}{\pgfqpoint{6.316028in}{1.430748in}}%
\pgfpathcurveto{\pgfqpoint{6.316028in}{1.437881in}}{\pgfqpoint{6.313194in}{1.444722in}}{\pgfqpoint{6.308150in}{1.449766in}}%
\pgfpathcurveto{\pgfqpoint{6.303106in}{1.454810in}}{\pgfqpoint{6.296265in}{1.457644in}}{\pgfqpoint{6.289132in}{1.457644in}}%
\pgfpathcurveto{\pgfqpoint{6.281999in}{1.457644in}}{\pgfqpoint{6.275157in}{1.454810in}}{\pgfqpoint{6.270114in}{1.449766in}}%
\pgfpathcurveto{\pgfqpoint{6.265070in}{1.444722in}}{\pgfqpoint{6.262236in}{1.437881in}}{\pgfqpoint{6.262236in}{1.430748in}}%
\pgfpathcurveto{\pgfqpoint{6.262236in}{1.423615in}}{\pgfqpoint{6.265070in}{1.416773in}}{\pgfqpoint{6.270114in}{1.411730in}}%
\pgfpathcurveto{\pgfqpoint{6.275157in}{1.406686in}}{\pgfqpoint{6.281999in}{1.403852in}}{\pgfqpoint{6.289132in}{1.403852in}}%
\pgfpathclose%
\pgfusepath{stroke,fill}%
\end{pgfscope}%
\begin{pgfscope}%
\pgfpathrectangle{\pgfqpoint{4.985294in}{0.500000in}}{\pgfqpoint{1.764706in}{1.700000in}}%
\pgfusepath{clip}%
\pgfsetbuttcap%
\pgfsetroundjoin%
\definecolor{currentfill}{rgb}{0.976961,0.885681,0.814303}%
\pgfsetfillcolor{currentfill}%
\pgfsetlinewidth{0.311001pt}%
\definecolor{currentstroke}{rgb}{1.000000,1.000000,1.000000}%
\pgfsetstrokecolor{currentstroke}%
\pgfsetdash{}{0pt}%
\pgfpathmoveto{\pgfqpoint{5.400770in}{1.203796in}}%
\pgfpathcurveto{\pgfqpoint{5.407903in}{1.203796in}}{\pgfqpoint{5.414745in}{1.206630in}}{\pgfqpoint{5.419788in}{1.211674in}}%
\pgfpathcurveto{\pgfqpoint{5.424832in}{1.216717in}}{\pgfqpoint{5.427666in}{1.223559in}}{\pgfqpoint{5.427666in}{1.230692in}}%
\pgfpathcurveto{\pgfqpoint{5.427666in}{1.237825in}}{\pgfqpoint{5.424832in}{1.244666in}}{\pgfqpoint{5.419788in}{1.249710in}}%
\pgfpathcurveto{\pgfqpoint{5.414745in}{1.254754in}}{\pgfqpoint{5.407903in}{1.257588in}}{\pgfqpoint{5.400770in}{1.257588in}}%
\pgfpathcurveto{\pgfqpoint{5.393638in}{1.257588in}}{\pgfqpoint{5.386796in}{1.254754in}}{\pgfqpoint{5.381752in}{1.249710in}}%
\pgfpathcurveto{\pgfqpoint{5.376709in}{1.244666in}}{\pgfqpoint{5.373875in}{1.237825in}}{\pgfqpoint{5.373875in}{1.230692in}}%
\pgfpathcurveto{\pgfqpoint{5.373875in}{1.223559in}}{\pgfqpoint{5.376709in}{1.216717in}}{\pgfqpoint{5.381752in}{1.211674in}}%
\pgfpathcurveto{\pgfqpoint{5.386796in}{1.206630in}}{\pgfqpoint{5.393638in}{1.203796in}}{\pgfqpoint{5.400770in}{1.203796in}}%
\pgfpathclose%
\pgfusepath{stroke,fill}%
\end{pgfscope}%
\begin{pgfscope}%
\pgfpathrectangle{\pgfqpoint{4.985294in}{0.500000in}}{\pgfqpoint{1.764706in}{1.700000in}}%
\pgfusepath{clip}%
\pgfsetbuttcap%
\pgfsetroundjoin%
\definecolor{currentfill}{rgb}{0.968931,0.798091,0.685123}%
\pgfsetfillcolor{currentfill}%
\pgfsetlinewidth{0.311001pt}%
\definecolor{currentstroke}{rgb}{1.000000,1.000000,1.000000}%
\pgfsetstrokecolor{currentstroke}%
\pgfsetdash{}{0pt}%
\pgfpathmoveto{\pgfqpoint{5.346703in}{1.324287in}}%
\pgfpathcurveto{\pgfqpoint{5.353836in}{1.324287in}}{\pgfqpoint{5.360677in}{1.327121in}}{\pgfqpoint{5.365721in}{1.332164in}}%
\pgfpathcurveto{\pgfqpoint{5.370765in}{1.337208in}}{\pgfqpoint{5.373599in}{1.344050in}}{\pgfqpoint{5.373599in}{1.351182in}}%
\pgfpathcurveto{\pgfqpoint{5.373599in}{1.358315in}}{\pgfqpoint{5.370765in}{1.365157in}}{\pgfqpoint{5.365721in}{1.370201in}}%
\pgfpathcurveto{\pgfqpoint{5.360677in}{1.375244in}}{\pgfqpoint{5.353836in}{1.378078in}}{\pgfqpoint{5.346703in}{1.378078in}}%
\pgfpathcurveto{\pgfqpoint{5.339570in}{1.378078in}}{\pgfqpoint{5.332728in}{1.375244in}}{\pgfqpoint{5.327685in}{1.370201in}}%
\pgfpathcurveto{\pgfqpoint{5.322641in}{1.365157in}}{\pgfqpoint{5.319807in}{1.358315in}}{\pgfqpoint{5.319807in}{1.351182in}}%
\pgfpathcurveto{\pgfqpoint{5.319807in}{1.344050in}}{\pgfqpoint{5.322641in}{1.337208in}}{\pgfqpoint{5.327685in}{1.332164in}}%
\pgfpathcurveto{\pgfqpoint{5.332728in}{1.327121in}}{\pgfqpoint{5.339570in}{1.324287in}}{\pgfqpoint{5.346703in}{1.324287in}}%
\pgfpathclose%
\pgfusepath{stroke,fill}%
\end{pgfscope}%
\begin{pgfscope}%
\pgfpathrectangle{\pgfqpoint{4.985294in}{0.500000in}}{\pgfqpoint{1.764706in}{1.700000in}}%
\pgfusepath{clip}%
\pgfsetbuttcap%
\pgfsetroundjoin%
\definecolor{currentfill}{rgb}{0.970255,0.815666,0.711203}%
\pgfsetfillcolor{currentfill}%
\pgfsetlinewidth{0.311001pt}%
\definecolor{currentstroke}{rgb}{1.000000,1.000000,1.000000}%
\pgfsetstrokecolor{currentstroke}%
\pgfsetdash{}{0pt}%
\pgfpathmoveto{\pgfqpoint{6.237011in}{0.996396in}}%
\pgfpathcurveto{\pgfqpoint{6.244143in}{0.996396in}}{\pgfqpoint{6.250985in}{0.999230in}}{\pgfqpoint{6.256029in}{1.004274in}}%
\pgfpathcurveto{\pgfqpoint{6.261072in}{1.009318in}}{\pgfqpoint{6.263906in}{1.016159in}}{\pgfqpoint{6.263906in}{1.023292in}}%
\pgfpathcurveto{\pgfqpoint{6.263906in}{1.030425in}}{\pgfqpoint{6.261072in}{1.037267in}}{\pgfqpoint{6.256029in}{1.042310in}}%
\pgfpathcurveto{\pgfqpoint{6.250985in}{1.047354in}}{\pgfqpoint{6.244143in}{1.050188in}}{\pgfqpoint{6.237011in}{1.050188in}}%
\pgfpathcurveto{\pgfqpoint{6.229878in}{1.050188in}}{\pgfqpoint{6.223036in}{1.047354in}}{\pgfqpoint{6.217993in}{1.042310in}}%
\pgfpathcurveto{\pgfqpoint{6.212949in}{1.037267in}}{\pgfqpoint{6.210115in}{1.030425in}}{\pgfqpoint{6.210115in}{1.023292in}}%
\pgfpathcurveto{\pgfqpoint{6.210115in}{1.016159in}}{\pgfqpoint{6.212949in}{1.009318in}}{\pgfqpoint{6.217993in}{1.004274in}}%
\pgfpathcurveto{\pgfqpoint{6.223036in}{0.999230in}}{\pgfqpoint{6.229878in}{0.996396in}}{\pgfqpoint{6.237011in}{0.996396in}}%
\pgfpathclose%
\pgfusepath{stroke,fill}%
\end{pgfscope}%
\begin{pgfscope}%
\pgfpathrectangle{\pgfqpoint{4.985294in}{0.500000in}}{\pgfqpoint{1.764706in}{1.700000in}}%
\pgfusepath{clip}%
\pgfsetbuttcap%
\pgfsetroundjoin%
\definecolor{currentfill}{rgb}{0.965753,0.732351,0.592427}%
\pgfsetfillcolor{currentfill}%
\pgfsetlinewidth{0.311001pt}%
\definecolor{currentstroke}{rgb}{1.000000,1.000000,1.000000}%
\pgfsetstrokecolor{currentstroke}%
\pgfsetdash{}{0pt}%
\pgfpathmoveto{\pgfqpoint{5.541927in}{1.706121in}}%
\pgfpathcurveto{\pgfqpoint{5.549060in}{1.706121in}}{\pgfqpoint{5.555901in}{1.708954in}}{\pgfqpoint{5.560945in}{1.713998in}}%
\pgfpathcurveto{\pgfqpoint{5.565989in}{1.719042in}}{\pgfqpoint{5.568823in}{1.725883in}}{\pgfqpoint{5.568823in}{1.733016in}}%
\pgfpathcurveto{\pgfqpoint{5.568823in}{1.740149in}}{\pgfqpoint{5.565989in}{1.746991in}}{\pgfqpoint{5.560945in}{1.752034in}}%
\pgfpathcurveto{\pgfqpoint{5.555901in}{1.757078in}}{\pgfqpoint{5.549060in}{1.759912in}}{\pgfqpoint{5.541927in}{1.759912in}}%
\pgfpathcurveto{\pgfqpoint{5.534794in}{1.759912in}}{\pgfqpoint{5.527952in}{1.757078in}}{\pgfqpoint{5.522909in}{1.752034in}}%
\pgfpathcurveto{\pgfqpoint{5.517865in}{1.746991in}}{\pgfqpoint{5.515031in}{1.740149in}}{\pgfqpoint{5.515031in}{1.733016in}}%
\pgfpathcurveto{\pgfqpoint{5.515031in}{1.725883in}}{\pgfqpoint{5.517865in}{1.719042in}}{\pgfqpoint{5.522909in}{1.713998in}}%
\pgfpathcurveto{\pgfqpoint{5.527952in}{1.708954in}}{\pgfqpoint{5.534794in}{1.706121in}}{\pgfqpoint{5.541927in}{1.706121in}}%
\pgfpathclose%
\pgfusepath{stroke,fill}%
\end{pgfscope}%
\begin{pgfscope}%
\pgfpathrectangle{\pgfqpoint{4.985294in}{0.500000in}}{\pgfqpoint{1.764706in}{1.700000in}}%
\pgfusepath{clip}%
\pgfsetbuttcap%
\pgfsetroundjoin%
\definecolor{currentfill}{rgb}{0.838502,0.140251,0.287688}%
\pgfsetfillcolor{currentfill}%
\pgfsetlinewidth{0.311001pt}%
\definecolor{currentstroke}{rgb}{1.000000,1.000000,1.000000}%
\pgfsetstrokecolor{currentstroke}%
\pgfsetdash{}{0pt}%
\pgfpathmoveto{\pgfqpoint{6.096183in}{1.429979in}}%
\pgfpathcurveto{\pgfqpoint{6.103315in}{1.429979in}}{\pgfqpoint{6.110157in}{1.432813in}}{\pgfqpoint{6.115201in}{1.437857in}}%
\pgfpathcurveto{\pgfqpoint{6.120244in}{1.442901in}}{\pgfqpoint{6.123078in}{1.449742in}}{\pgfqpoint{6.123078in}{1.456875in}}%
\pgfpathcurveto{\pgfqpoint{6.123078in}{1.464008in}}{\pgfqpoint{6.120244in}{1.470850in}}{\pgfqpoint{6.115201in}{1.475893in}}%
\pgfpathcurveto{\pgfqpoint{6.110157in}{1.480937in}}{\pgfqpoint{6.103315in}{1.483771in}}{\pgfqpoint{6.096183in}{1.483771in}}%
\pgfpathcurveto{\pgfqpoint{6.089050in}{1.483771in}}{\pgfqpoint{6.082208in}{1.480937in}}{\pgfqpoint{6.077164in}{1.475893in}}%
\pgfpathcurveto{\pgfqpoint{6.072121in}{1.470850in}}{\pgfqpoint{6.069287in}{1.464008in}}{\pgfqpoint{6.069287in}{1.456875in}}%
\pgfpathcurveto{\pgfqpoint{6.069287in}{1.449742in}}{\pgfqpoint{6.072121in}{1.442901in}}{\pgfqpoint{6.077164in}{1.437857in}}%
\pgfpathcurveto{\pgfqpoint{6.082208in}{1.432813in}}{\pgfqpoint{6.089050in}{1.429979in}}{\pgfqpoint{6.096183in}{1.429979in}}%
\pgfpathclose%
\pgfusepath{stroke,fill}%
\end{pgfscope}%
\begin{pgfscope}%
\pgfpathrectangle{\pgfqpoint{4.985294in}{0.500000in}}{\pgfqpoint{1.764706in}{1.700000in}}%
\pgfusepath{clip}%
\pgfsetbuttcap%
\pgfsetroundjoin%
\definecolor{currentfill}{rgb}{0.981377,0.920617,0.865369}%
\pgfsetfillcolor{currentfill}%
\pgfsetlinewidth{0.311001pt}%
\definecolor{currentstroke}{rgb}{1.000000,1.000000,1.000000}%
\pgfsetstrokecolor{currentstroke}%
\pgfsetdash{}{0pt}%
\pgfpathmoveto{\pgfqpoint{6.322028in}{1.309261in}}%
\pgfpathcurveto{\pgfqpoint{6.329160in}{1.309261in}}{\pgfqpoint{6.336002in}{1.312095in}}{\pgfqpoint{6.341046in}{1.317138in}}%
\pgfpathcurveto{\pgfqpoint{6.346089in}{1.322182in}}{\pgfqpoint{6.348923in}{1.329024in}}{\pgfqpoint{6.348923in}{1.336157in}}%
\pgfpathcurveto{\pgfqpoint{6.348923in}{1.343289in}}{\pgfqpoint{6.346089in}{1.350131in}}{\pgfqpoint{6.341046in}{1.355175in}}%
\pgfpathcurveto{\pgfqpoint{6.336002in}{1.360218in}}{\pgfqpoint{6.329160in}{1.363052in}}{\pgfqpoint{6.322028in}{1.363052in}}%
\pgfpathcurveto{\pgfqpoint{6.314895in}{1.363052in}}{\pgfqpoint{6.308053in}{1.360218in}}{\pgfqpoint{6.303009in}{1.355175in}}%
\pgfpathcurveto{\pgfqpoint{6.297966in}{1.350131in}}{\pgfqpoint{6.295132in}{1.343289in}}{\pgfqpoint{6.295132in}{1.336157in}}%
\pgfpathcurveto{\pgfqpoint{6.295132in}{1.329024in}}{\pgfqpoint{6.297966in}{1.322182in}}{\pgfqpoint{6.303009in}{1.317138in}}%
\pgfpathcurveto{\pgfqpoint{6.308053in}{1.312095in}}{\pgfqpoint{6.314895in}{1.309261in}}{\pgfqpoint{6.322028in}{1.309261in}}%
\pgfpathclose%
\pgfusepath{stroke,fill}%
\end{pgfscope}%
\begin{pgfscope}%
\pgfpathrectangle{\pgfqpoint{4.985294in}{0.500000in}}{\pgfqpoint{1.764706in}{1.700000in}}%
\pgfusepath{clip}%
\pgfsetbuttcap%
\pgfsetroundjoin%
\definecolor{currentfill}{rgb}{0.843354,0.145567,0.284808}%
\pgfsetfillcolor{currentfill}%
\pgfsetlinewidth{0.311001pt}%
\definecolor{currentstroke}{rgb}{1.000000,1.000000,1.000000}%
\pgfsetstrokecolor{currentstroke}%
\pgfsetdash{}{0pt}%
\pgfpathmoveto{\pgfqpoint{5.762596in}{1.823830in}}%
\pgfpathcurveto{\pgfqpoint{5.769729in}{1.823830in}}{\pgfqpoint{5.776571in}{1.826664in}}{\pgfqpoint{5.781615in}{1.831708in}}%
\pgfpathcurveto{\pgfqpoint{5.786658in}{1.836752in}}{\pgfqpoint{5.789492in}{1.843593in}}{\pgfqpoint{5.789492in}{1.850726in}}%
\pgfpathcurveto{\pgfqpoint{5.789492in}{1.857859in}}{\pgfqpoint{5.786658in}{1.864701in}}{\pgfqpoint{5.781615in}{1.869744in}}%
\pgfpathcurveto{\pgfqpoint{5.776571in}{1.874788in}}{\pgfqpoint{5.769729in}{1.877622in}}{\pgfqpoint{5.762596in}{1.877622in}}%
\pgfpathcurveto{\pgfqpoint{5.755464in}{1.877622in}}{\pgfqpoint{5.748622in}{1.874788in}}{\pgfqpoint{5.743578in}{1.869744in}}%
\pgfpathcurveto{\pgfqpoint{5.738535in}{1.864701in}}{\pgfqpoint{5.735701in}{1.857859in}}{\pgfqpoint{5.735701in}{1.850726in}}%
\pgfpathcurveto{\pgfqpoint{5.735701in}{1.843593in}}{\pgfqpoint{5.738535in}{1.836752in}}{\pgfqpoint{5.743578in}{1.831708in}}%
\pgfpathcurveto{\pgfqpoint{5.748622in}{1.826664in}}{\pgfqpoint{5.755464in}{1.823830in}}{\pgfqpoint{5.762596in}{1.823830in}}%
\pgfpathclose%
\pgfusepath{stroke,fill}%
\end{pgfscope}%
\begin{pgfscope}%
\pgfpathrectangle{\pgfqpoint{4.985294in}{0.500000in}}{\pgfqpoint{1.764706in}{1.700000in}}%
\pgfusepath{clip}%
\pgfsetbuttcap%
\pgfsetroundjoin%
\definecolor{currentfill}{rgb}{0.975018,0.868213,0.788710}%
\pgfsetfillcolor{currentfill}%
\pgfsetlinewidth{0.311001pt}%
\definecolor{currentstroke}{rgb}{1.000000,1.000000,1.000000}%
\pgfsetstrokecolor{currentstroke}%
\pgfsetdash{}{0pt}%
\pgfpathmoveto{\pgfqpoint{5.442481in}{1.084383in}}%
\pgfpathcurveto{\pgfqpoint{5.449613in}{1.084383in}}{\pgfqpoint{5.456455in}{1.087217in}}{\pgfqpoint{5.461499in}{1.092261in}}%
\pgfpathcurveto{\pgfqpoint{5.466542in}{1.097305in}}{\pgfqpoint{5.469376in}{1.104146in}}{\pgfqpoint{5.469376in}{1.111279in}}%
\pgfpathcurveto{\pgfqpoint{5.469376in}{1.118412in}}{\pgfqpoint{5.466542in}{1.125254in}}{\pgfqpoint{5.461499in}{1.130297in}}%
\pgfpathcurveto{\pgfqpoint{5.456455in}{1.135341in}}{\pgfqpoint{5.449613in}{1.138175in}}{\pgfqpoint{5.442481in}{1.138175in}}%
\pgfpathcurveto{\pgfqpoint{5.435348in}{1.138175in}}{\pgfqpoint{5.428506in}{1.135341in}}{\pgfqpoint{5.423462in}{1.130297in}}%
\pgfpathcurveto{\pgfqpoint{5.418419in}{1.125254in}}{\pgfqpoint{5.415585in}{1.118412in}}{\pgfqpoint{5.415585in}{1.111279in}}%
\pgfpathcurveto{\pgfqpoint{5.415585in}{1.104146in}}{\pgfqpoint{5.418419in}{1.097305in}}{\pgfqpoint{5.423462in}{1.092261in}}%
\pgfpathcurveto{\pgfqpoint{5.428506in}{1.087217in}}{\pgfqpoint{5.435348in}{1.084383in}}{\pgfqpoint{5.442481in}{1.084383in}}%
\pgfpathclose%
\pgfusepath{stroke,fill}%
\end{pgfscope}%
\begin{pgfscope}%
\pgfpathrectangle{\pgfqpoint{4.985294in}{0.500000in}}{\pgfqpoint{1.764706in}{1.700000in}}%
\pgfusepath{clip}%
\pgfsetbuttcap%
\pgfsetroundjoin%
\definecolor{currentfill}{rgb}{0.972201,0.839051,0.745789}%
\pgfsetfillcolor{currentfill}%
\pgfsetlinewidth{0.311001pt}%
\definecolor{currentstroke}{rgb}{1.000000,1.000000,1.000000}%
\pgfsetstrokecolor{currentstroke}%
\pgfsetdash{}{0pt}%
\pgfpathmoveto{\pgfqpoint{5.383425in}{1.466178in}}%
\pgfpathcurveto{\pgfqpoint{5.390558in}{1.466178in}}{\pgfqpoint{5.397399in}{1.469012in}}{\pgfqpoint{5.402443in}{1.474055in}}%
\pgfpathcurveto{\pgfqpoint{5.407487in}{1.479099in}}{\pgfqpoint{5.410321in}{1.485941in}}{\pgfqpoint{5.410321in}{1.493073in}}%
\pgfpathcurveto{\pgfqpoint{5.410321in}{1.500206in}}{\pgfqpoint{5.407487in}{1.507048in}}{\pgfqpoint{5.402443in}{1.512092in}}%
\pgfpathcurveto{\pgfqpoint{5.397399in}{1.517135in}}{\pgfqpoint{5.390558in}{1.519969in}}{\pgfqpoint{5.383425in}{1.519969in}}%
\pgfpathcurveto{\pgfqpoint{5.376292in}{1.519969in}}{\pgfqpoint{5.369450in}{1.517135in}}{\pgfqpoint{5.364407in}{1.512092in}}%
\pgfpathcurveto{\pgfqpoint{5.359363in}{1.507048in}}{\pgfqpoint{5.356529in}{1.500206in}}{\pgfqpoint{5.356529in}{1.493073in}}%
\pgfpathcurveto{\pgfqpoint{5.356529in}{1.485941in}}{\pgfqpoint{5.359363in}{1.479099in}}{\pgfqpoint{5.364407in}{1.474055in}}%
\pgfpathcurveto{\pgfqpoint{5.369450in}{1.469012in}}{\pgfqpoint{5.376292in}{1.466178in}}{\pgfqpoint{5.383425in}{1.466178in}}%
\pgfpathclose%
\pgfusepath{stroke,fill}%
\end{pgfscope}%
\begin{pgfscope}%
\pgfpathrectangle{\pgfqpoint{4.985294in}{0.500000in}}{\pgfqpoint{1.764706in}{1.700000in}}%
\pgfusepath{clip}%
\pgfsetbuttcap%
\pgfsetroundjoin%
\definecolor{currentfill}{rgb}{0.965753,0.732351,0.592427}%
\pgfsetfillcolor{currentfill}%
\pgfsetlinewidth{0.311001pt}%
\definecolor{currentstroke}{rgb}{1.000000,1.000000,1.000000}%
\pgfsetstrokecolor{currentstroke}%
\pgfsetdash{}{0pt}%
\pgfpathmoveto{\pgfqpoint{6.254979in}{0.961420in}}%
\pgfpathcurveto{\pgfqpoint{6.262111in}{0.961420in}}{\pgfqpoint{6.268953in}{0.964254in}}{\pgfqpoint{6.273997in}{0.969298in}}%
\pgfpathcurveto{\pgfqpoint{6.279040in}{0.974342in}}{\pgfqpoint{6.281874in}{0.981183in}}{\pgfqpoint{6.281874in}{0.988316in}}%
\pgfpathcurveto{\pgfqpoint{6.281874in}{0.995449in}}{\pgfqpoint{6.279040in}{1.002291in}}{\pgfqpoint{6.273997in}{1.007334in}}%
\pgfpathcurveto{\pgfqpoint{6.268953in}{1.012378in}}{\pgfqpoint{6.262111in}{1.015212in}}{\pgfqpoint{6.254979in}{1.015212in}}%
\pgfpathcurveto{\pgfqpoint{6.247846in}{1.015212in}}{\pgfqpoint{6.241004in}{1.012378in}}{\pgfqpoint{6.235960in}{1.007334in}}%
\pgfpathcurveto{\pgfqpoint{6.230917in}{1.002291in}}{\pgfqpoint{6.228083in}{0.995449in}}{\pgfqpoint{6.228083in}{0.988316in}}%
\pgfpathcurveto{\pgfqpoint{6.228083in}{0.981183in}}{\pgfqpoint{6.230917in}{0.974342in}}{\pgfqpoint{6.235960in}{0.969298in}}%
\pgfpathcurveto{\pgfqpoint{6.241004in}{0.964254in}}{\pgfqpoint{6.247846in}{0.961420in}}{\pgfqpoint{6.254979in}{0.961420in}}%
\pgfpathclose%
\pgfusepath{stroke,fill}%
\end{pgfscope}%
\begin{pgfscope}%
\pgfpathrectangle{\pgfqpoint{4.985294in}{0.500000in}}{\pgfqpoint{1.764706in}{1.700000in}}%
\pgfusepath{clip}%
\pgfsetbuttcap%
\pgfsetroundjoin%
\definecolor{currentfill}{rgb}{0.965592,0.726236,0.584384}%
\pgfsetfillcolor{currentfill}%
\pgfsetlinewidth{0.311001pt}%
\definecolor{currentstroke}{rgb}{1.000000,1.000000,1.000000}%
\pgfsetstrokecolor{currentstroke}%
\pgfsetdash{}{0pt}%
\pgfpathmoveto{\pgfqpoint{5.578308in}{1.644269in}}%
\pgfpathcurveto{\pgfqpoint{5.585441in}{1.644269in}}{\pgfqpoint{5.592282in}{1.647103in}}{\pgfqpoint{5.597326in}{1.652147in}}%
\pgfpathcurveto{\pgfqpoint{5.602370in}{1.657190in}}{\pgfqpoint{5.605204in}{1.664032in}}{\pgfqpoint{5.605204in}{1.671165in}}%
\pgfpathcurveto{\pgfqpoint{5.605204in}{1.678298in}}{\pgfqpoint{5.602370in}{1.685139in}}{\pgfqpoint{5.597326in}{1.690183in}}%
\pgfpathcurveto{\pgfqpoint{5.592282in}{1.695227in}}{\pgfqpoint{5.585441in}{1.698061in}}{\pgfqpoint{5.578308in}{1.698061in}}%
\pgfpathcurveto{\pgfqpoint{5.571175in}{1.698061in}}{\pgfqpoint{5.564333in}{1.695227in}}{\pgfqpoint{5.559290in}{1.690183in}}%
\pgfpathcurveto{\pgfqpoint{5.554246in}{1.685139in}}{\pgfqpoint{5.551412in}{1.678298in}}{\pgfqpoint{5.551412in}{1.671165in}}%
\pgfpathcurveto{\pgfqpoint{5.551412in}{1.664032in}}{\pgfqpoint{5.554246in}{1.657190in}}{\pgfqpoint{5.559290in}{1.652147in}}%
\pgfpathcurveto{\pgfqpoint{5.564333in}{1.647103in}}{\pgfqpoint{5.571175in}{1.644269in}}{\pgfqpoint{5.578308in}{1.644269in}}%
\pgfpathclose%
\pgfusepath{stroke,fill}%
\end{pgfscope}%
\begin{pgfscope}%
\pgfpathrectangle{\pgfqpoint{4.985294in}{0.500000in}}{\pgfqpoint{1.764706in}{1.700000in}}%
\pgfusepath{clip}%
\pgfsetbuttcap%
\pgfsetroundjoin%
\definecolor{currentfill}{rgb}{0.796501,0.105066,0.310630}%
\pgfsetfillcolor{currentfill}%
\pgfsetlinewidth{0.311001pt}%
\definecolor{currentstroke}{rgb}{1.000000,1.000000,1.000000}%
\pgfsetstrokecolor{currentstroke}%
\pgfsetdash{}{0pt}%
\pgfpathmoveto{\pgfqpoint{6.056333in}{1.125609in}}%
\pgfpathcurveto{\pgfqpoint{6.063465in}{1.125609in}}{\pgfqpoint{6.070307in}{1.128443in}}{\pgfqpoint{6.075351in}{1.133487in}}%
\pgfpathcurveto{\pgfqpoint{6.080394in}{1.138531in}}{\pgfqpoint{6.083228in}{1.145372in}}{\pgfqpoint{6.083228in}{1.152505in}}%
\pgfpathcurveto{\pgfqpoint{6.083228in}{1.159638in}}{\pgfqpoint{6.080394in}{1.166480in}}{\pgfqpoint{6.075351in}{1.171523in}}%
\pgfpathcurveto{\pgfqpoint{6.070307in}{1.176567in}}{\pgfqpoint{6.063465in}{1.179401in}}{\pgfqpoint{6.056333in}{1.179401in}}%
\pgfpathcurveto{\pgfqpoint{6.049200in}{1.179401in}}{\pgfqpoint{6.042358in}{1.176567in}}{\pgfqpoint{6.037314in}{1.171523in}}%
\pgfpathcurveto{\pgfqpoint{6.032271in}{1.166480in}}{\pgfqpoint{6.029437in}{1.159638in}}{\pgfqpoint{6.029437in}{1.152505in}}%
\pgfpathcurveto{\pgfqpoint{6.029437in}{1.145372in}}{\pgfqpoint{6.032271in}{1.138531in}}{\pgfqpoint{6.037314in}{1.133487in}}%
\pgfpathcurveto{\pgfqpoint{6.042358in}{1.128443in}}{\pgfqpoint{6.049200in}{1.125609in}}{\pgfqpoint{6.056333in}{1.125609in}}%
\pgfpathclose%
\pgfusepath{stroke,fill}%
\end{pgfscope}%
\begin{pgfscope}%
\pgfpathrectangle{\pgfqpoint{4.985294in}{0.500000in}}{\pgfqpoint{1.764706in}{1.700000in}}%
\pgfusepath{clip}%
\pgfsetbuttcap%
\pgfsetroundjoin%
\definecolor{currentfill}{rgb}{0.942910,0.375495,0.263698}%
\pgfsetfillcolor{currentfill}%
\pgfsetlinewidth{0.311001pt}%
\definecolor{currentstroke}{rgb}{1.000000,1.000000,1.000000}%
\pgfsetstrokecolor{currentstroke}%
\pgfsetdash{}{0pt}%
\pgfpathmoveto{\pgfqpoint{6.445480in}{1.458355in}}%
\pgfpathcurveto{\pgfqpoint{6.452613in}{1.458355in}}{\pgfqpoint{6.459455in}{1.461189in}}{\pgfqpoint{6.464498in}{1.466233in}}%
\pgfpathcurveto{\pgfqpoint{6.469542in}{1.471276in}}{\pgfqpoint{6.472376in}{1.478118in}}{\pgfqpoint{6.472376in}{1.485251in}}%
\pgfpathcurveto{\pgfqpoint{6.472376in}{1.492384in}}{\pgfqpoint{6.469542in}{1.499225in}}{\pgfqpoint{6.464498in}{1.504269in}}%
\pgfpathcurveto{\pgfqpoint{6.459455in}{1.509313in}}{\pgfqpoint{6.452613in}{1.512146in}}{\pgfqpoint{6.445480in}{1.512146in}}%
\pgfpathcurveto{\pgfqpoint{6.438347in}{1.512146in}}{\pgfqpoint{6.431506in}{1.509313in}}{\pgfqpoint{6.426462in}{1.504269in}}%
\pgfpathcurveto{\pgfqpoint{6.421418in}{1.499225in}}{\pgfqpoint{6.418584in}{1.492384in}}{\pgfqpoint{6.418584in}{1.485251in}}%
\pgfpathcurveto{\pgfqpoint{6.418584in}{1.478118in}}{\pgfqpoint{6.421418in}{1.471276in}}{\pgfqpoint{6.426462in}{1.466233in}}%
\pgfpathcurveto{\pgfqpoint{6.431506in}{1.461189in}}{\pgfqpoint{6.438347in}{1.458355in}}{\pgfqpoint{6.445480in}{1.458355in}}%
\pgfpathclose%
\pgfusepath{stroke,fill}%
\end{pgfscope}%
\begin{pgfscope}%
\pgfpathrectangle{\pgfqpoint{4.985294in}{0.500000in}}{\pgfqpoint{1.764706in}{1.700000in}}%
\pgfusepath{clip}%
\pgfsetbuttcap%
\pgfsetroundjoin%
\definecolor{currentfill}{rgb}{0.961115,0.566634,0.405693}%
\pgfsetfillcolor{currentfill}%
\pgfsetlinewidth{0.311001pt}%
\definecolor{currentstroke}{rgb}{1.000000,1.000000,1.000000}%
\pgfsetstrokecolor{currentstroke}%
\pgfsetdash{}{0pt}%
\pgfpathmoveto{\pgfqpoint{5.306249in}{1.196220in}}%
\pgfpathcurveto{\pgfqpoint{5.313382in}{1.196220in}}{\pgfqpoint{5.320224in}{1.199054in}}{\pgfqpoint{5.325267in}{1.204097in}}%
\pgfpathcurveto{\pgfqpoint{5.330311in}{1.209141in}}{\pgfqpoint{5.333145in}{1.215983in}}{\pgfqpoint{5.333145in}{1.223115in}}%
\pgfpathcurveto{\pgfqpoint{5.333145in}{1.230248in}}{\pgfqpoint{5.330311in}{1.237090in}}{\pgfqpoint{5.325267in}{1.242134in}}%
\pgfpathcurveto{\pgfqpoint{5.320224in}{1.247177in}}{\pgfqpoint{5.313382in}{1.250011in}}{\pgfqpoint{5.306249in}{1.250011in}}%
\pgfpathcurveto{\pgfqpoint{5.299116in}{1.250011in}}{\pgfqpoint{5.292275in}{1.247177in}}{\pgfqpoint{5.287231in}{1.242134in}}%
\pgfpathcurveto{\pgfqpoint{5.282187in}{1.237090in}}{\pgfqpoint{5.279353in}{1.230248in}}{\pgfqpoint{5.279353in}{1.223115in}}%
\pgfpathcurveto{\pgfqpoint{5.279353in}{1.215983in}}{\pgfqpoint{5.282187in}{1.209141in}}{\pgfqpoint{5.287231in}{1.204097in}}%
\pgfpathcurveto{\pgfqpoint{5.292275in}{1.199054in}}{\pgfqpoint{5.299116in}{1.196220in}}{\pgfqpoint{5.306249in}{1.196220in}}%
\pgfpathclose%
\pgfusepath{stroke,fill}%
\end{pgfscope}%
\begin{pgfscope}%
\pgfpathrectangle{\pgfqpoint{4.985294in}{0.500000in}}{\pgfqpoint{1.764706in}{1.700000in}}%
\pgfusepath{clip}%
\pgfsetbuttcap%
\pgfsetroundjoin%
\definecolor{currentfill}{rgb}{0.975018,0.868213,0.788710}%
\pgfsetfillcolor{currentfill}%
\pgfsetlinewidth{0.311001pt}%
\definecolor{currentstroke}{rgb}{1.000000,1.000000,1.000000}%
\pgfsetstrokecolor{currentstroke}%
\pgfsetdash{}{0pt}%
\pgfpathmoveto{\pgfqpoint{5.479995in}{1.466976in}}%
\pgfpathcurveto{\pgfqpoint{5.487128in}{1.466976in}}{\pgfqpoint{5.493970in}{1.469810in}}{\pgfqpoint{5.499014in}{1.474853in}}%
\pgfpathcurveto{\pgfqpoint{5.504057in}{1.479897in}}{\pgfqpoint{5.506891in}{1.486739in}}{\pgfqpoint{5.506891in}{1.493871in}}%
\pgfpathcurveto{\pgfqpoint{5.506891in}{1.501004in}}{\pgfqpoint{5.504057in}{1.507846in}}{\pgfqpoint{5.499014in}{1.512890in}}%
\pgfpathcurveto{\pgfqpoint{5.493970in}{1.517933in}}{\pgfqpoint{5.487128in}{1.520767in}}{\pgfqpoint{5.479995in}{1.520767in}}%
\pgfpathcurveto{\pgfqpoint{5.472863in}{1.520767in}}{\pgfqpoint{5.466021in}{1.517933in}}{\pgfqpoint{5.460977in}{1.512890in}}%
\pgfpathcurveto{\pgfqpoint{5.455934in}{1.507846in}}{\pgfqpoint{5.453100in}{1.501004in}}{\pgfqpoint{5.453100in}{1.493871in}}%
\pgfpathcurveto{\pgfqpoint{5.453100in}{1.486739in}}{\pgfqpoint{5.455934in}{1.479897in}}{\pgfqpoint{5.460977in}{1.474853in}}%
\pgfpathcurveto{\pgfqpoint{5.466021in}{1.469810in}}{\pgfqpoint{5.472863in}{1.466976in}}{\pgfqpoint{5.479995in}{1.466976in}}%
\pgfpathclose%
\pgfusepath{stroke,fill}%
\end{pgfscope}%
\begin{pgfscope}%
\pgfpathrectangle{\pgfqpoint{4.985294in}{0.500000in}}{\pgfqpoint{1.764706in}{1.700000in}}%
\pgfusepath{clip}%
\pgfsetbuttcap%
\pgfsetroundjoin%
\definecolor{currentfill}{rgb}{0.838502,0.140251,0.287688}%
\pgfsetfillcolor{currentfill}%
\pgfsetlinewidth{0.311001pt}%
\definecolor{currentstroke}{rgb}{1.000000,1.000000,1.000000}%
\pgfsetstrokecolor{currentstroke}%
\pgfsetdash{}{0pt}%
\pgfpathmoveto{\pgfqpoint{5.912286in}{1.734134in}}%
\pgfpathcurveto{\pgfqpoint{5.919418in}{1.734134in}}{\pgfqpoint{5.926260in}{1.736968in}}{\pgfqpoint{5.931304in}{1.742012in}}%
\pgfpathcurveto{\pgfqpoint{5.936347in}{1.747056in}}{\pgfqpoint{5.939181in}{1.753897in}}{\pgfqpoint{5.939181in}{1.761030in}}%
\pgfpathcurveto{\pgfqpoint{5.939181in}{1.768163in}}{\pgfqpoint{5.936347in}{1.775005in}}{\pgfqpoint{5.931304in}{1.780048in}}%
\pgfpathcurveto{\pgfqpoint{5.926260in}{1.785092in}}{\pgfqpoint{5.919418in}{1.787926in}}{\pgfqpoint{5.912286in}{1.787926in}}%
\pgfpathcurveto{\pgfqpoint{5.905153in}{1.787926in}}{\pgfqpoint{5.898311in}{1.785092in}}{\pgfqpoint{5.893267in}{1.780048in}}%
\pgfpathcurveto{\pgfqpoint{5.888224in}{1.775005in}}{\pgfqpoint{5.885390in}{1.768163in}}{\pgfqpoint{5.885390in}{1.761030in}}%
\pgfpathcurveto{\pgfqpoint{5.885390in}{1.753897in}}{\pgfqpoint{5.888224in}{1.747056in}}{\pgfqpoint{5.893267in}{1.742012in}}%
\pgfpathcurveto{\pgfqpoint{5.898311in}{1.736968in}}{\pgfqpoint{5.905153in}{1.734134in}}{\pgfqpoint{5.912286in}{1.734134in}}%
\pgfpathclose%
\pgfusepath{stroke,fill}%
\end{pgfscope}%
\begin{pgfscope}%
\pgfpathrectangle{\pgfqpoint{4.985294in}{0.500000in}}{\pgfqpoint{1.764706in}{1.700000in}}%
\pgfusepath{clip}%
\pgfsetbuttcap%
\pgfsetroundjoin%
\definecolor{currentfill}{rgb}{0.914423,0.260289,0.243694}%
\pgfsetfillcolor{currentfill}%
\pgfsetlinewidth{0.311001pt}%
\definecolor{currentstroke}{rgb}{1.000000,1.000000,1.000000}%
\pgfsetstrokecolor{currentstroke}%
\pgfsetdash{}{0pt}%
\pgfpathmoveto{\pgfqpoint{6.088838in}{1.121157in}}%
\pgfpathcurveto{\pgfqpoint{6.095971in}{1.121157in}}{\pgfqpoint{6.102813in}{1.123991in}}{\pgfqpoint{6.107857in}{1.129034in}}%
\pgfpathcurveto{\pgfqpoint{6.112900in}{1.134078in}}{\pgfqpoint{6.115734in}{1.140920in}}{\pgfqpoint{6.115734in}{1.148053in}}%
\pgfpathcurveto{\pgfqpoint{6.115734in}{1.155185in}}{\pgfqpoint{6.112900in}{1.162027in}}{\pgfqpoint{6.107857in}{1.167071in}}%
\pgfpathcurveto{\pgfqpoint{6.102813in}{1.172114in}}{\pgfqpoint{6.095971in}{1.174948in}}{\pgfqpoint{6.088838in}{1.174948in}}%
\pgfpathcurveto{\pgfqpoint{6.081706in}{1.174948in}}{\pgfqpoint{6.074864in}{1.172114in}}{\pgfqpoint{6.069820in}{1.167071in}}%
\pgfpathcurveto{\pgfqpoint{6.064777in}{1.162027in}}{\pgfqpoint{6.061943in}{1.155185in}}{\pgfqpoint{6.061943in}{1.148053in}}%
\pgfpathcurveto{\pgfqpoint{6.061943in}{1.140920in}}{\pgfqpoint{6.064777in}{1.134078in}}{\pgfqpoint{6.069820in}{1.129034in}}%
\pgfpathcurveto{\pgfqpoint{6.074864in}{1.123991in}}{\pgfqpoint{6.081706in}{1.121157in}}{\pgfqpoint{6.088838in}{1.121157in}}%
\pgfpathclose%
\pgfusepath{stroke,fill}%
\end{pgfscope}%
\begin{pgfscope}%
\pgfpathrectangle{\pgfqpoint{4.985294in}{0.500000in}}{\pgfqpoint{1.764706in}{1.700000in}}%
\pgfusepath{clip}%
\pgfsetbuttcap%
\pgfsetroundjoin%
\definecolor{currentfill}{rgb}{0.828528,0.130141,0.293475}%
\pgfsetfillcolor{currentfill}%
\pgfsetlinewidth{0.311001pt}%
\definecolor{currentstroke}{rgb}{1.000000,1.000000,1.000000}%
\pgfsetstrokecolor{currentstroke}%
\pgfsetdash{}{0pt}%
\pgfpathmoveto{\pgfqpoint{5.661949in}{1.856063in}}%
\pgfpathcurveto{\pgfqpoint{5.669082in}{1.856063in}}{\pgfqpoint{5.675924in}{1.858897in}}{\pgfqpoint{5.680967in}{1.863940in}}%
\pgfpathcurveto{\pgfqpoint{5.686011in}{1.868984in}}{\pgfqpoint{5.688845in}{1.875826in}}{\pgfqpoint{5.688845in}{1.882958in}}%
\pgfpathcurveto{\pgfqpoint{5.688845in}{1.890091in}}{\pgfqpoint{5.686011in}{1.896933in}}{\pgfqpoint{5.680967in}{1.901977in}}%
\pgfpathcurveto{\pgfqpoint{5.675924in}{1.907020in}}{\pgfqpoint{5.669082in}{1.909854in}}{\pgfqpoint{5.661949in}{1.909854in}}%
\pgfpathcurveto{\pgfqpoint{5.654816in}{1.909854in}}{\pgfqpoint{5.647975in}{1.907020in}}{\pgfqpoint{5.642931in}{1.901977in}}%
\pgfpathcurveto{\pgfqpoint{5.637887in}{1.896933in}}{\pgfqpoint{5.635054in}{1.890091in}}{\pgfqpoint{5.635054in}{1.882958in}}%
\pgfpathcurveto{\pgfqpoint{5.635054in}{1.875826in}}{\pgfqpoint{5.637887in}{1.868984in}}{\pgfqpoint{5.642931in}{1.863940in}}%
\pgfpathcurveto{\pgfqpoint{5.647975in}{1.858897in}}{\pgfqpoint{5.654816in}{1.856063in}}{\pgfqpoint{5.661949in}{1.856063in}}%
\pgfpathclose%
\pgfusepath{stroke,fill}%
\end{pgfscope}%
\begin{pgfscope}%
\pgfpathrectangle{\pgfqpoint{4.985294in}{0.500000in}}{\pgfqpoint{1.764706in}{1.700000in}}%
\pgfusepath{clip}%
\pgfsetbuttcap%
\pgfsetroundjoin%
\definecolor{currentfill}{rgb}{0.976961,0.885681,0.814303}%
\pgfsetfillcolor{currentfill}%
\pgfsetlinewidth{0.311001pt}%
\definecolor{currentstroke}{rgb}{1.000000,1.000000,1.000000}%
\pgfsetstrokecolor{currentstroke}%
\pgfsetdash{}{0pt}%
\pgfpathmoveto{\pgfqpoint{6.253801in}{1.592285in}}%
\pgfpathcurveto{\pgfqpoint{6.260934in}{1.592285in}}{\pgfqpoint{6.267776in}{1.595118in}}{\pgfqpoint{6.272820in}{1.600162in}}%
\pgfpathcurveto{\pgfqpoint{6.277863in}{1.605206in}}{\pgfqpoint{6.280697in}{1.612047in}}{\pgfqpoint{6.280697in}{1.619180in}}%
\pgfpathcurveto{\pgfqpoint{6.280697in}{1.626313in}}{\pgfqpoint{6.277863in}{1.633155in}}{\pgfqpoint{6.272820in}{1.638198in}}%
\pgfpathcurveto{\pgfqpoint{6.267776in}{1.643242in}}{\pgfqpoint{6.260934in}{1.646076in}}{\pgfqpoint{6.253801in}{1.646076in}}%
\pgfpathcurveto{\pgfqpoint{6.246669in}{1.646076in}}{\pgfqpoint{6.239827in}{1.643242in}}{\pgfqpoint{6.234783in}{1.638198in}}%
\pgfpathcurveto{\pgfqpoint{6.229740in}{1.633155in}}{\pgfqpoint{6.226906in}{1.626313in}}{\pgfqpoint{6.226906in}{1.619180in}}%
\pgfpathcurveto{\pgfqpoint{6.226906in}{1.612047in}}{\pgfqpoint{6.229740in}{1.605206in}}{\pgfqpoint{6.234783in}{1.600162in}}%
\pgfpathcurveto{\pgfqpoint{6.239827in}{1.595118in}}{\pgfqpoint{6.246669in}{1.592285in}}{\pgfqpoint{6.253801in}{1.592285in}}%
\pgfpathclose%
\pgfusepath{stroke,fill}%
\end{pgfscope}%
\begin{pgfscope}%
\pgfpathrectangle{\pgfqpoint{4.985294in}{0.500000in}}{\pgfqpoint{1.764706in}{1.700000in}}%
\pgfusepath{clip}%
\pgfsetbuttcap%
\pgfsetroundjoin%
\definecolor{currentfill}{rgb}{0.971694,0.833208,0.737161}%
\pgfsetfillcolor{currentfill}%
\pgfsetlinewidth{0.311001pt}%
\definecolor{currentstroke}{rgb}{1.000000,1.000000,1.000000}%
\pgfsetstrokecolor{currentstroke}%
\pgfsetdash{}{0pt}%
\pgfpathmoveto{\pgfqpoint{5.517505in}{0.994478in}}%
\pgfpathcurveto{\pgfqpoint{5.524638in}{0.994478in}}{\pgfqpoint{5.531480in}{0.997312in}}{\pgfqpoint{5.536523in}{1.002356in}}%
\pgfpathcurveto{\pgfqpoint{5.541567in}{1.007399in}}{\pgfqpoint{5.544401in}{1.014241in}}{\pgfqpoint{5.544401in}{1.021374in}}%
\pgfpathcurveto{\pgfqpoint{5.544401in}{1.028507in}}{\pgfqpoint{5.541567in}{1.035348in}}{\pgfqpoint{5.536523in}{1.040392in}}%
\pgfpathcurveto{\pgfqpoint{5.531480in}{1.045436in}}{\pgfqpoint{5.524638in}{1.048269in}}{\pgfqpoint{5.517505in}{1.048269in}}%
\pgfpathcurveto{\pgfqpoint{5.510372in}{1.048269in}}{\pgfqpoint{5.503531in}{1.045436in}}{\pgfqpoint{5.498487in}{1.040392in}}%
\pgfpathcurveto{\pgfqpoint{5.493443in}{1.035348in}}{\pgfqpoint{5.490609in}{1.028507in}}{\pgfqpoint{5.490609in}{1.021374in}}%
\pgfpathcurveto{\pgfqpoint{5.490609in}{1.014241in}}{\pgfqpoint{5.493443in}{1.007399in}}{\pgfqpoint{5.498487in}{1.002356in}}%
\pgfpathcurveto{\pgfqpoint{5.503531in}{0.997312in}}{\pgfqpoint{5.510372in}{0.994478in}}{\pgfqpoint{5.517505in}{0.994478in}}%
\pgfpathclose%
\pgfusepath{stroke,fill}%
\end{pgfscope}%
\begin{pgfscope}%
\pgfpathrectangle{\pgfqpoint{4.985294in}{0.500000in}}{\pgfqpoint{1.764706in}{1.700000in}}%
\pgfusepath{clip}%
\pgfsetbuttcap%
\pgfsetroundjoin%
\definecolor{currentfill}{rgb}{0.972726,0.844889,0.754401}%
\pgfsetfillcolor{currentfill}%
\pgfsetlinewidth{0.311001pt}%
\definecolor{currentstroke}{rgb}{1.000000,1.000000,1.000000}%
\pgfsetstrokecolor{currentstroke}%
\pgfsetdash{}{0pt}%
\pgfpathmoveto{\pgfqpoint{5.487751in}{1.414897in}}%
\pgfpathcurveto{\pgfqpoint{5.494884in}{1.414897in}}{\pgfqpoint{5.501725in}{1.417730in}}{\pgfqpoint{5.506769in}{1.422774in}}%
\pgfpathcurveto{\pgfqpoint{5.511813in}{1.427818in}}{\pgfqpoint{5.514647in}{1.434659in}}{\pgfqpoint{5.514647in}{1.441792in}}%
\pgfpathcurveto{\pgfqpoint{5.514647in}{1.448925in}}{\pgfqpoint{5.511813in}{1.455767in}}{\pgfqpoint{5.506769in}{1.460810in}}%
\pgfpathcurveto{\pgfqpoint{5.501725in}{1.465854in}}{\pgfqpoint{5.494884in}{1.468688in}}{\pgfqpoint{5.487751in}{1.468688in}}%
\pgfpathcurveto{\pgfqpoint{5.480618in}{1.468688in}}{\pgfqpoint{5.473776in}{1.465854in}}{\pgfqpoint{5.468733in}{1.460810in}}%
\pgfpathcurveto{\pgfqpoint{5.463689in}{1.455767in}}{\pgfqpoint{5.460855in}{1.448925in}}{\pgfqpoint{5.460855in}{1.441792in}}%
\pgfpathcurveto{\pgfqpoint{5.460855in}{1.434659in}}{\pgfqpoint{5.463689in}{1.427818in}}{\pgfqpoint{5.468733in}{1.422774in}}%
\pgfpathcurveto{\pgfqpoint{5.473776in}{1.417730in}}{\pgfqpoint{5.480618in}{1.414897in}}{\pgfqpoint{5.487751in}{1.414897in}}%
\pgfpathclose%
\pgfusepath{stroke,fill}%
\end{pgfscope}%
\begin{pgfscope}%
\pgfpathrectangle{\pgfqpoint{4.985294in}{0.500000in}}{\pgfqpoint{1.764706in}{1.700000in}}%
\pgfusepath{clip}%
\pgfsetbuttcap%
\pgfsetroundjoin%
\definecolor{currentfill}{rgb}{0.838502,0.140251,0.287688}%
\pgfsetfillcolor{currentfill}%
\pgfsetlinewidth{0.311001pt}%
\definecolor{currentstroke}{rgb}{1.000000,1.000000,1.000000}%
\pgfsetstrokecolor{currentstroke}%
\pgfsetdash{}{0pt}%
\pgfpathmoveto{\pgfqpoint{5.973847in}{1.797639in}}%
\pgfpathcurveto{\pgfqpoint{5.980980in}{1.797639in}}{\pgfqpoint{5.987821in}{1.800473in}}{\pgfqpoint{5.992865in}{1.805516in}}%
\pgfpathcurveto{\pgfqpoint{5.997909in}{1.810560in}}{\pgfqpoint{6.000743in}{1.817402in}}{\pgfqpoint{6.000743in}{1.824535in}}%
\pgfpathcurveto{\pgfqpoint{6.000743in}{1.831667in}}{\pgfqpoint{5.997909in}{1.838509in}}{\pgfqpoint{5.992865in}{1.843553in}}%
\pgfpathcurveto{\pgfqpoint{5.987821in}{1.848596in}}{\pgfqpoint{5.980980in}{1.851430in}}{\pgfqpoint{5.973847in}{1.851430in}}%
\pgfpathcurveto{\pgfqpoint{5.966714in}{1.851430in}}{\pgfqpoint{5.959873in}{1.848596in}}{\pgfqpoint{5.954829in}{1.843553in}}%
\pgfpathcurveto{\pgfqpoint{5.949785in}{1.838509in}}{\pgfqpoint{5.946951in}{1.831667in}}{\pgfqpoint{5.946951in}{1.824535in}}%
\pgfpathcurveto{\pgfqpoint{5.946951in}{1.817402in}}{\pgfqpoint{5.949785in}{1.810560in}}{\pgfqpoint{5.954829in}{1.805516in}}%
\pgfpathcurveto{\pgfqpoint{5.959873in}{1.800473in}}{\pgfqpoint{5.966714in}{1.797639in}}{\pgfqpoint{5.973847in}{1.797639in}}%
\pgfpathclose%
\pgfusepath{stroke,fill}%
\end{pgfscope}%
\begin{pgfscope}%
\pgfpathrectangle{\pgfqpoint{4.985294in}{0.500000in}}{\pgfqpoint{1.764706in}{1.700000in}}%
\pgfusepath{clip}%
\pgfsetbuttcap%
\pgfsetroundjoin%
\definecolor{currentfill}{rgb}{0.975018,0.868213,0.788710}%
\pgfsetfillcolor{currentfill}%
\pgfsetlinewidth{0.311001pt}%
\definecolor{currentstroke}{rgb}{1.000000,1.000000,1.000000}%
\pgfsetstrokecolor{currentstroke}%
\pgfsetdash{}{0pt}%
\pgfpathmoveto{\pgfqpoint{5.482502in}{1.542079in}}%
\pgfpathcurveto{\pgfqpoint{5.489635in}{1.542079in}}{\pgfqpoint{5.496476in}{1.544912in}}{\pgfqpoint{5.501520in}{1.549956in}}%
\pgfpathcurveto{\pgfqpoint{5.506564in}{1.555000in}}{\pgfqpoint{5.509398in}{1.561841in}}{\pgfqpoint{5.509398in}{1.568974in}}%
\pgfpathcurveto{\pgfqpoint{5.509398in}{1.576107in}}{\pgfqpoint{5.506564in}{1.582949in}}{\pgfqpoint{5.501520in}{1.587992in}}%
\pgfpathcurveto{\pgfqpoint{5.496476in}{1.593036in}}{\pgfqpoint{5.489635in}{1.595870in}}{\pgfqpoint{5.482502in}{1.595870in}}%
\pgfpathcurveto{\pgfqpoint{5.475369in}{1.595870in}}{\pgfqpoint{5.468527in}{1.593036in}}{\pgfqpoint{5.463484in}{1.587992in}}%
\pgfpathcurveto{\pgfqpoint{5.458440in}{1.582949in}}{\pgfqpoint{5.455606in}{1.576107in}}{\pgfqpoint{5.455606in}{1.568974in}}%
\pgfpathcurveto{\pgfqpoint{5.455606in}{1.561841in}}{\pgfqpoint{5.458440in}{1.555000in}}{\pgfqpoint{5.463484in}{1.549956in}}%
\pgfpathcurveto{\pgfqpoint{5.468527in}{1.544912in}}{\pgfqpoint{5.475369in}{1.542079in}}{\pgfqpoint{5.482502in}{1.542079in}}%
\pgfpathclose%
\pgfusepath{stroke,fill}%
\end{pgfscope}%
\begin{pgfscope}%
\pgfpathrectangle{\pgfqpoint{4.985294in}{0.500000in}}{\pgfqpoint{1.764706in}{1.700000in}}%
\pgfusepath{clip}%
\pgfsetbuttcap%
\pgfsetroundjoin%
\definecolor{currentfill}{rgb}{0.962985,0.612625,0.451451}%
\pgfsetfillcolor{currentfill}%
\pgfsetlinewidth{0.311001pt}%
\definecolor{currentstroke}{rgb}{1.000000,1.000000,1.000000}%
\pgfsetstrokecolor{currentstroke}%
\pgfsetdash{}{0pt}%
\pgfpathmoveto{\pgfqpoint{5.651077in}{0.927820in}}%
\pgfpathcurveto{\pgfqpoint{5.658210in}{0.927820in}}{\pgfqpoint{5.665052in}{0.930653in}}{\pgfqpoint{5.670095in}{0.935697in}}%
\pgfpathcurveto{\pgfqpoint{5.675139in}{0.940741in}}{\pgfqpoint{5.677973in}{0.947582in}}{\pgfqpoint{5.677973in}{0.954715in}}%
\pgfpathcurveto{\pgfqpoint{5.677973in}{0.961848in}}{\pgfqpoint{5.675139in}{0.968690in}}{\pgfqpoint{5.670095in}{0.973733in}}%
\pgfpathcurveto{\pgfqpoint{5.665052in}{0.978777in}}{\pgfqpoint{5.658210in}{0.981611in}}{\pgfqpoint{5.651077in}{0.981611in}}%
\pgfpathcurveto{\pgfqpoint{5.643944in}{0.981611in}}{\pgfqpoint{5.637103in}{0.978777in}}{\pgfqpoint{5.632059in}{0.973733in}}%
\pgfpathcurveto{\pgfqpoint{5.627015in}{0.968690in}}{\pgfqpoint{5.624181in}{0.961848in}}{\pgfqpoint{5.624181in}{0.954715in}}%
\pgfpathcurveto{\pgfqpoint{5.624181in}{0.947582in}}{\pgfqpoint{5.627015in}{0.940741in}}{\pgfqpoint{5.632059in}{0.935697in}}%
\pgfpathcurveto{\pgfqpoint{5.637103in}{0.930653in}}{\pgfqpoint{5.643944in}{0.927820in}}{\pgfqpoint{5.651077in}{0.927820in}}%
\pgfpathclose%
\pgfusepath{stroke,fill}%
\end{pgfscope}%
\begin{pgfscope}%
\pgfpathrectangle{\pgfqpoint{4.985294in}{0.500000in}}{\pgfqpoint{1.764706in}{1.700000in}}%
\pgfusepath{clip}%
\pgfsetbuttcap%
\pgfsetroundjoin%
\definecolor{currentfill}{rgb}{0.980678,0.914765,0.856766}%
\pgfsetfillcolor{currentfill}%
\pgfsetlinewidth{0.311001pt}%
\definecolor{currentstroke}{rgb}{1.000000,1.000000,1.000000}%
\pgfsetstrokecolor{currentstroke}%
\pgfsetdash{}{0pt}%
\pgfpathmoveto{\pgfqpoint{6.299524in}{1.363669in}}%
\pgfpathcurveto{\pgfqpoint{6.306657in}{1.363669in}}{\pgfqpoint{6.313499in}{1.366503in}}{\pgfqpoint{6.318543in}{1.371546in}}%
\pgfpathcurveto{\pgfqpoint{6.323586in}{1.376590in}}{\pgfqpoint{6.326420in}{1.383432in}}{\pgfqpoint{6.326420in}{1.390565in}}%
\pgfpathcurveto{\pgfqpoint{6.326420in}{1.397697in}}{\pgfqpoint{6.323586in}{1.404539in}}{\pgfqpoint{6.318543in}{1.409583in}}%
\pgfpathcurveto{\pgfqpoint{6.313499in}{1.414626in}}{\pgfqpoint{6.306657in}{1.417460in}}{\pgfqpoint{6.299524in}{1.417460in}}%
\pgfpathcurveto{\pgfqpoint{6.292392in}{1.417460in}}{\pgfqpoint{6.285550in}{1.414626in}}{\pgfqpoint{6.280506in}{1.409583in}}%
\pgfpathcurveto{\pgfqpoint{6.275463in}{1.404539in}}{\pgfqpoint{6.272629in}{1.397697in}}{\pgfqpoint{6.272629in}{1.390565in}}%
\pgfpathcurveto{\pgfqpoint{6.272629in}{1.383432in}}{\pgfqpoint{6.275463in}{1.376590in}}{\pgfqpoint{6.280506in}{1.371546in}}%
\pgfpathcurveto{\pgfqpoint{6.285550in}{1.366503in}}{\pgfqpoint{6.292392in}{1.363669in}}{\pgfqpoint{6.299524in}{1.363669in}}%
\pgfpathclose%
\pgfusepath{stroke,fill}%
\end{pgfscope}%
\begin{pgfscope}%
\pgfpathrectangle{\pgfqpoint{4.985294in}{0.500000in}}{\pgfqpoint{1.764706in}{1.700000in}}%
\pgfusepath{clip}%
\pgfsetbuttcap%
\pgfsetroundjoin%
\definecolor{currentfill}{rgb}{0.964173,0.657587,0.500469}%
\pgfsetfillcolor{currentfill}%
\pgfsetlinewidth{0.311001pt}%
\definecolor{currentstroke}{rgb}{1.000000,1.000000,1.000000}%
\pgfsetstrokecolor{currentstroke}%
\pgfsetdash{}{0pt}%
\pgfpathmoveto{\pgfqpoint{6.187546in}{1.332222in}}%
\pgfpathcurveto{\pgfqpoint{6.194678in}{1.332222in}}{\pgfqpoint{6.201520in}{1.335056in}}{\pgfqpoint{6.206564in}{1.340100in}}%
\pgfpathcurveto{\pgfqpoint{6.211607in}{1.345144in}}{\pgfqpoint{6.214441in}{1.351985in}}{\pgfqpoint{6.214441in}{1.359118in}}%
\pgfpathcurveto{\pgfqpoint{6.214441in}{1.366251in}}{\pgfqpoint{6.211607in}{1.373093in}}{\pgfqpoint{6.206564in}{1.378136in}}%
\pgfpathcurveto{\pgfqpoint{6.201520in}{1.383180in}}{\pgfqpoint{6.194678in}{1.386014in}}{\pgfqpoint{6.187546in}{1.386014in}}%
\pgfpathcurveto{\pgfqpoint{6.180413in}{1.386014in}}{\pgfqpoint{6.173571in}{1.383180in}}{\pgfqpoint{6.168527in}{1.378136in}}%
\pgfpathcurveto{\pgfqpoint{6.163484in}{1.373093in}}{\pgfqpoint{6.160650in}{1.366251in}}{\pgfqpoint{6.160650in}{1.359118in}}%
\pgfpathcurveto{\pgfqpoint{6.160650in}{1.351985in}}{\pgfqpoint{6.163484in}{1.345144in}}{\pgfqpoint{6.168527in}{1.340100in}}%
\pgfpathcurveto{\pgfqpoint{6.173571in}{1.335056in}}{\pgfqpoint{6.180413in}{1.332222in}}{\pgfqpoint{6.187546in}{1.332222in}}%
\pgfpathclose%
\pgfusepath{stroke,fill}%
\end{pgfscope}%
\begin{pgfscope}%
\pgfpathrectangle{\pgfqpoint{4.985294in}{0.500000in}}{\pgfqpoint{1.764706in}{1.700000in}}%
\pgfusepath{clip}%
\pgfsetbuttcap%
\pgfsetroundjoin%
\definecolor{currentfill}{rgb}{0.979124,0.903132,0.839793}%
\pgfsetfillcolor{currentfill}%
\pgfsetlinewidth{0.311001pt}%
\definecolor{currentstroke}{rgb}{1.000000,1.000000,1.000000}%
\pgfsetstrokecolor{currentstroke}%
\pgfsetdash{}{0pt}%
\pgfpathmoveto{\pgfqpoint{5.422997in}{1.450219in}}%
\pgfpathcurveto{\pgfqpoint{5.430130in}{1.450219in}}{\pgfqpoint{5.436971in}{1.453052in}}{\pgfqpoint{5.442015in}{1.458096in}}%
\pgfpathcurveto{\pgfqpoint{5.447059in}{1.463140in}}{\pgfqpoint{5.449893in}{1.469981in}}{\pgfqpoint{5.449893in}{1.477114in}}%
\pgfpathcurveto{\pgfqpoint{5.449893in}{1.484247in}}{\pgfqpoint{5.447059in}{1.491089in}}{\pgfqpoint{5.442015in}{1.496132in}}%
\pgfpathcurveto{\pgfqpoint{5.436971in}{1.501176in}}{\pgfqpoint{5.430130in}{1.504010in}}{\pgfqpoint{5.422997in}{1.504010in}}%
\pgfpathcurveto{\pgfqpoint{5.415864in}{1.504010in}}{\pgfqpoint{5.409022in}{1.501176in}}{\pgfqpoint{5.403979in}{1.496132in}}%
\pgfpathcurveto{\pgfqpoint{5.398935in}{1.491089in}}{\pgfqpoint{5.396101in}{1.484247in}}{\pgfqpoint{5.396101in}{1.477114in}}%
\pgfpathcurveto{\pgfqpoint{5.396101in}{1.469981in}}{\pgfqpoint{5.398935in}{1.463140in}}{\pgfqpoint{5.403979in}{1.458096in}}%
\pgfpathcurveto{\pgfqpoint{5.409022in}{1.453052in}}{\pgfqpoint{5.415864in}{1.450219in}}{\pgfqpoint{5.422997in}{1.450219in}}%
\pgfpathclose%
\pgfusepath{stroke,fill}%
\end{pgfscope}%
\begin{pgfscope}%
\pgfpathrectangle{\pgfqpoint{4.985294in}{0.500000in}}{\pgfqpoint{1.764706in}{1.700000in}}%
\pgfusepath{clip}%
\pgfsetbuttcap%
\pgfsetroundjoin%
\definecolor{currentfill}{rgb}{0.966812,0.762584,0.633643}%
\pgfsetfillcolor{currentfill}%
\pgfsetlinewidth{0.311001pt}%
\definecolor{currentstroke}{rgb}{1.000000,1.000000,1.000000}%
\pgfsetstrokecolor{currentstroke}%
\pgfsetdash{}{0pt}%
\pgfpathmoveto{\pgfqpoint{5.359973in}{1.155646in}}%
\pgfpathcurveto{\pgfqpoint{5.367106in}{1.155646in}}{\pgfqpoint{5.373948in}{1.158480in}}{\pgfqpoint{5.378992in}{1.163523in}}%
\pgfpathcurveto{\pgfqpoint{5.384035in}{1.168567in}}{\pgfqpoint{5.386869in}{1.175409in}}{\pgfqpoint{5.386869in}{1.182541in}}%
\pgfpathcurveto{\pgfqpoint{5.386869in}{1.189674in}}{\pgfqpoint{5.384035in}{1.196516in}}{\pgfqpoint{5.378992in}{1.201559in}}%
\pgfpathcurveto{\pgfqpoint{5.373948in}{1.206603in}}{\pgfqpoint{5.367106in}{1.209437in}}{\pgfqpoint{5.359973in}{1.209437in}}%
\pgfpathcurveto{\pgfqpoint{5.352841in}{1.209437in}}{\pgfqpoint{5.345999in}{1.206603in}}{\pgfqpoint{5.340955in}{1.201559in}}%
\pgfpathcurveto{\pgfqpoint{5.335912in}{1.196516in}}{\pgfqpoint{5.333078in}{1.189674in}}{\pgfqpoint{5.333078in}{1.182541in}}%
\pgfpathcurveto{\pgfqpoint{5.333078in}{1.175409in}}{\pgfqpoint{5.335912in}{1.168567in}}{\pgfqpoint{5.340955in}{1.163523in}}%
\pgfpathcurveto{\pgfqpoint{5.345999in}{1.158480in}}{\pgfqpoint{5.352841in}{1.155646in}}{\pgfqpoint{5.359973in}{1.155646in}}%
\pgfpathclose%
\pgfusepath{stroke,fill}%
\end{pgfscope}%
\begin{pgfscope}%
\pgfpathrectangle{\pgfqpoint{4.985294in}{0.500000in}}{\pgfqpoint{1.764706in}{1.700000in}}%
\pgfusepath{clip}%
\pgfsetbuttcap%
\pgfsetroundjoin%
\definecolor{currentfill}{rgb}{0.979124,0.903132,0.839793}%
\pgfsetfillcolor{currentfill}%
\pgfsetlinewidth{0.311001pt}%
\definecolor{currentstroke}{rgb}{1.000000,1.000000,1.000000}%
\pgfsetstrokecolor{currentstroke}%
\pgfsetdash{}{0pt}%
\pgfpathmoveto{\pgfqpoint{6.285865in}{1.293397in}}%
\pgfpathcurveto{\pgfqpoint{6.292997in}{1.293397in}}{\pgfqpoint{6.299839in}{1.296231in}}{\pgfqpoint{6.304883in}{1.301274in}}%
\pgfpathcurveto{\pgfqpoint{6.309926in}{1.306318in}}{\pgfqpoint{6.312760in}{1.313160in}}{\pgfqpoint{6.312760in}{1.320292in}}%
\pgfpathcurveto{\pgfqpoint{6.312760in}{1.327425in}}{\pgfqpoint{6.309926in}{1.334267in}}{\pgfqpoint{6.304883in}{1.339311in}}%
\pgfpathcurveto{\pgfqpoint{6.299839in}{1.344354in}}{\pgfqpoint{6.292997in}{1.347188in}}{\pgfqpoint{6.285865in}{1.347188in}}%
\pgfpathcurveto{\pgfqpoint{6.278732in}{1.347188in}}{\pgfqpoint{6.271890in}{1.344354in}}{\pgfqpoint{6.266846in}{1.339311in}}%
\pgfpathcurveto{\pgfqpoint{6.261803in}{1.334267in}}{\pgfqpoint{6.258969in}{1.327425in}}{\pgfqpoint{6.258969in}{1.320292in}}%
\pgfpathcurveto{\pgfqpoint{6.258969in}{1.313160in}}{\pgfqpoint{6.261803in}{1.306318in}}{\pgfqpoint{6.266846in}{1.301274in}}%
\pgfpathcurveto{\pgfqpoint{6.271890in}{1.296231in}}{\pgfqpoint{6.278732in}{1.293397in}}{\pgfqpoint{6.285865in}{1.293397in}}%
\pgfpathclose%
\pgfusepath{stroke,fill}%
\end{pgfscope}%
\begin{pgfscope}%
\pgfpathrectangle{\pgfqpoint{4.985294in}{0.500000in}}{\pgfqpoint{1.764706in}{1.700000in}}%
\pgfusepath{clip}%
\pgfsetbuttcap%
\pgfsetroundjoin%
\definecolor{currentfill}{rgb}{0.968931,0.798091,0.685123}%
\pgfsetfillcolor{currentfill}%
\pgfsetlinewidth{0.311001pt}%
\definecolor{currentstroke}{rgb}{1.000000,1.000000,1.000000}%
\pgfsetstrokecolor{currentstroke}%
\pgfsetdash{}{0pt}%
\pgfpathmoveto{\pgfqpoint{5.407819in}{1.567592in}}%
\pgfpathcurveto{\pgfqpoint{5.414952in}{1.567592in}}{\pgfqpoint{5.421794in}{1.570426in}}{\pgfqpoint{5.426837in}{1.575469in}}%
\pgfpathcurveto{\pgfqpoint{5.431881in}{1.580513in}}{\pgfqpoint{5.434715in}{1.587355in}}{\pgfqpoint{5.434715in}{1.594488in}}%
\pgfpathcurveto{\pgfqpoint{5.434715in}{1.601620in}}{\pgfqpoint{5.431881in}{1.608462in}}{\pgfqpoint{5.426837in}{1.613506in}}%
\pgfpathcurveto{\pgfqpoint{5.421794in}{1.618549in}}{\pgfqpoint{5.414952in}{1.621383in}}{\pgfqpoint{5.407819in}{1.621383in}}%
\pgfpathcurveto{\pgfqpoint{5.400686in}{1.621383in}}{\pgfqpoint{5.393845in}{1.618549in}}{\pgfqpoint{5.388801in}{1.613506in}}%
\pgfpathcurveto{\pgfqpoint{5.383757in}{1.608462in}}{\pgfqpoint{5.380924in}{1.601620in}}{\pgfqpoint{5.380924in}{1.594488in}}%
\pgfpathcurveto{\pgfqpoint{5.380924in}{1.587355in}}{\pgfqpoint{5.383757in}{1.580513in}}{\pgfqpoint{5.388801in}{1.575469in}}%
\pgfpathcurveto{\pgfqpoint{5.393845in}{1.570426in}}{\pgfqpoint{5.400686in}{1.567592in}}{\pgfqpoint{5.407819in}{1.567592in}}%
\pgfpathclose%
\pgfusepath{stroke,fill}%
\end{pgfscope}%
\begin{pgfscope}%
\pgfpathrectangle{\pgfqpoint{4.985294in}{0.500000in}}{\pgfqpoint{1.764706in}{1.700000in}}%
\pgfusepath{clip}%
\pgfsetbuttcap%
\pgfsetroundjoin%
\definecolor{currentfill}{rgb}{0.961734,0.579886,0.418445}%
\pgfsetfillcolor{currentfill}%
\pgfsetlinewidth{0.311001pt}%
\definecolor{currentstroke}{rgb}{1.000000,1.000000,1.000000}%
\pgfsetstrokecolor{currentstroke}%
\pgfsetdash{}{0pt}%
\pgfpathmoveto{\pgfqpoint{6.407830in}{1.499563in}}%
\pgfpathcurveto{\pgfqpoint{6.414963in}{1.499563in}}{\pgfqpoint{6.421805in}{1.502397in}}{\pgfqpoint{6.426848in}{1.507441in}}%
\pgfpathcurveto{\pgfqpoint{6.431892in}{1.512484in}}{\pgfqpoint{6.434726in}{1.519326in}}{\pgfqpoint{6.434726in}{1.526459in}}%
\pgfpathcurveto{\pgfqpoint{6.434726in}{1.533592in}}{\pgfqpoint{6.431892in}{1.540433in}}{\pgfqpoint{6.426848in}{1.545477in}}%
\pgfpathcurveto{\pgfqpoint{6.421805in}{1.550521in}}{\pgfqpoint{6.414963in}{1.553355in}}{\pgfqpoint{6.407830in}{1.553355in}}%
\pgfpathcurveto{\pgfqpoint{6.400697in}{1.553355in}}{\pgfqpoint{6.393856in}{1.550521in}}{\pgfqpoint{6.388812in}{1.545477in}}%
\pgfpathcurveto{\pgfqpoint{6.383768in}{1.540433in}}{\pgfqpoint{6.380934in}{1.533592in}}{\pgfqpoint{6.380934in}{1.526459in}}%
\pgfpathcurveto{\pgfqpoint{6.380934in}{1.519326in}}{\pgfqpoint{6.383768in}{1.512484in}}{\pgfqpoint{6.388812in}{1.507441in}}%
\pgfpathcurveto{\pgfqpoint{6.393856in}{1.502397in}}{\pgfqpoint{6.400697in}{1.499563in}}{\pgfqpoint{6.407830in}{1.499563in}}%
\pgfpathclose%
\pgfusepath{stroke,fill}%
\end{pgfscope}%
\begin{pgfscope}%
\pgfpathrectangle{\pgfqpoint{4.985294in}{0.500000in}}{\pgfqpoint{1.764706in}{1.700000in}}%
\pgfusepath{clip}%
\pgfsetbuttcap%
\pgfsetroundjoin%
\definecolor{currentfill}{rgb}{0.957344,0.505732,0.351309}%
\pgfsetfillcolor{currentfill}%
\pgfsetlinewidth{0.311001pt}%
\definecolor{currentstroke}{rgb}{1.000000,1.000000,1.000000}%
\pgfsetstrokecolor{currentstroke}%
\pgfsetdash{}{0pt}%
\pgfpathmoveto{\pgfqpoint{6.137926in}{1.478902in}}%
\pgfpathcurveto{\pgfqpoint{6.145059in}{1.478902in}}{\pgfqpoint{6.151901in}{1.481736in}}{\pgfqpoint{6.156945in}{1.486780in}}%
\pgfpathcurveto{\pgfqpoint{6.161988in}{1.491824in}}{\pgfqpoint{6.164822in}{1.498665in}}{\pgfqpoint{6.164822in}{1.505798in}}%
\pgfpathcurveto{\pgfqpoint{6.164822in}{1.512931in}}{\pgfqpoint{6.161988in}{1.519772in}}{\pgfqpoint{6.156945in}{1.524816in}}%
\pgfpathcurveto{\pgfqpoint{6.151901in}{1.529860in}}{\pgfqpoint{6.145059in}{1.532694in}}{\pgfqpoint{6.137926in}{1.532694in}}%
\pgfpathcurveto{\pgfqpoint{6.130794in}{1.532694in}}{\pgfqpoint{6.123952in}{1.529860in}}{\pgfqpoint{6.118908in}{1.524816in}}%
\pgfpathcurveto{\pgfqpoint{6.113865in}{1.519772in}}{\pgfqpoint{6.111031in}{1.512931in}}{\pgfqpoint{6.111031in}{1.505798in}}%
\pgfpathcurveto{\pgfqpoint{6.111031in}{1.498665in}}{\pgfqpoint{6.113865in}{1.491824in}}{\pgfqpoint{6.118908in}{1.486780in}}%
\pgfpathcurveto{\pgfqpoint{6.123952in}{1.481736in}}{\pgfqpoint{6.130794in}{1.478902in}}{\pgfqpoint{6.137926in}{1.478902in}}%
\pgfpathclose%
\pgfusepath{stroke,fill}%
\end{pgfscope}%
\begin{pgfscope}%
\pgfpathrectangle{\pgfqpoint{4.985294in}{0.500000in}}{\pgfqpoint{1.764706in}{1.700000in}}%
\pgfusepath{clip}%
\pgfsetbuttcap%
\pgfsetroundjoin%
\definecolor{currentfill}{rgb}{0.976961,0.885681,0.814303}%
\pgfsetfillcolor{currentfill}%
\pgfsetlinewidth{0.311001pt}%
\definecolor{currentstroke}{rgb}{1.000000,1.000000,1.000000}%
\pgfsetstrokecolor{currentstroke}%
\pgfsetdash{}{0pt}%
\pgfpathmoveto{\pgfqpoint{6.354026in}{1.344642in}}%
\pgfpathcurveto{\pgfqpoint{6.361158in}{1.344642in}}{\pgfqpoint{6.368000in}{1.347476in}}{\pgfqpoint{6.373044in}{1.352520in}}%
\pgfpathcurveto{\pgfqpoint{6.378087in}{1.357563in}}{\pgfqpoint{6.380921in}{1.364405in}}{\pgfqpoint{6.380921in}{1.371538in}}%
\pgfpathcurveto{\pgfqpoint{6.380921in}{1.378670in}}{\pgfqpoint{6.378087in}{1.385512in}}{\pgfqpoint{6.373044in}{1.390556in}}%
\pgfpathcurveto{\pgfqpoint{6.368000in}{1.395599in}}{\pgfqpoint{6.361158in}{1.398433in}}{\pgfqpoint{6.354026in}{1.398433in}}%
\pgfpathcurveto{\pgfqpoint{6.346893in}{1.398433in}}{\pgfqpoint{6.340051in}{1.395599in}}{\pgfqpoint{6.335007in}{1.390556in}}%
\pgfpathcurveto{\pgfqpoint{6.329964in}{1.385512in}}{\pgfqpoint{6.327130in}{1.378670in}}{\pgfqpoint{6.327130in}{1.371538in}}%
\pgfpathcurveto{\pgfqpoint{6.327130in}{1.364405in}}{\pgfqpoint{6.329964in}{1.357563in}}{\pgfqpoint{6.335007in}{1.352520in}}%
\pgfpathcurveto{\pgfqpoint{6.340051in}{1.347476in}}{\pgfqpoint{6.346893in}{1.344642in}}{\pgfqpoint{6.354026in}{1.344642in}}%
\pgfpathclose%
\pgfusepath{stroke,fill}%
\end{pgfscope}%
\begin{pgfscope}%
\pgfpathrectangle{\pgfqpoint{4.985294in}{0.500000in}}{\pgfqpoint{1.764706in}{1.700000in}}%
\pgfusepath{clip}%
\pgfsetbuttcap%
\pgfsetroundjoin%
\definecolor{currentfill}{rgb}{0.964433,0.670254,0.515093}%
\pgfsetfillcolor{currentfill}%
\pgfsetlinewidth{0.311001pt}%
\definecolor{currentstroke}{rgb}{1.000000,1.000000,1.000000}%
\pgfsetstrokecolor{currentstroke}%
\pgfsetdash{}{0pt}%
\pgfpathmoveto{\pgfqpoint{5.550565in}{1.737511in}}%
\pgfpathcurveto{\pgfqpoint{5.557698in}{1.737511in}}{\pgfqpoint{5.564539in}{1.740345in}}{\pgfqpoint{5.569583in}{1.745388in}}%
\pgfpathcurveto{\pgfqpoint{5.574627in}{1.750432in}}{\pgfqpoint{5.577460in}{1.757274in}}{\pgfqpoint{5.577460in}{1.764406in}}%
\pgfpathcurveto{\pgfqpoint{5.577460in}{1.771539in}}{\pgfqpoint{5.574627in}{1.778381in}}{\pgfqpoint{5.569583in}{1.783425in}}%
\pgfpathcurveto{\pgfqpoint{5.564539in}{1.788468in}}{\pgfqpoint{5.557698in}{1.791302in}}{\pgfqpoint{5.550565in}{1.791302in}}%
\pgfpathcurveto{\pgfqpoint{5.543432in}{1.791302in}}{\pgfqpoint{5.536590in}{1.788468in}}{\pgfqpoint{5.531547in}{1.783425in}}%
\pgfpathcurveto{\pgfqpoint{5.526503in}{1.778381in}}{\pgfqpoint{5.523669in}{1.771539in}}{\pgfqpoint{5.523669in}{1.764406in}}%
\pgfpathcurveto{\pgfqpoint{5.523669in}{1.757274in}}{\pgfqpoint{5.526503in}{1.750432in}}{\pgfqpoint{5.531547in}{1.745388in}}%
\pgfpathcurveto{\pgfqpoint{5.536590in}{1.740345in}}{\pgfqpoint{5.543432in}{1.737511in}}{\pgfqpoint{5.550565in}{1.737511in}}%
\pgfpathclose%
\pgfusepath{stroke,fill}%
\end{pgfscope}%
\begin{pgfscope}%
\pgfpathrectangle{\pgfqpoint{4.985294in}{0.500000in}}{\pgfqpoint{1.764706in}{1.700000in}}%
\pgfusepath{clip}%
\pgfsetbuttcap%
\pgfsetroundjoin%
\definecolor{currentfill}{rgb}{0.974412,0.862387,0.780156}%
\pgfsetfillcolor{currentfill}%
\pgfsetlinewidth{0.311001pt}%
\definecolor{currentstroke}{rgb}{1.000000,1.000000,1.000000}%
\pgfsetstrokecolor{currentstroke}%
\pgfsetdash{}{0pt}%
\pgfpathmoveto{\pgfqpoint{6.211640in}{1.623495in}}%
\pgfpathcurveto{\pgfqpoint{6.218773in}{1.623495in}}{\pgfqpoint{6.225614in}{1.626329in}}{\pgfqpoint{6.230658in}{1.631373in}}%
\pgfpathcurveto{\pgfqpoint{6.235701in}{1.636416in}}{\pgfqpoint{6.238535in}{1.643258in}}{\pgfqpoint{6.238535in}{1.650391in}}%
\pgfpathcurveto{\pgfqpoint{6.238535in}{1.657524in}}{\pgfqpoint{6.235701in}{1.664365in}}{\pgfqpoint{6.230658in}{1.669409in}}%
\pgfpathcurveto{\pgfqpoint{6.225614in}{1.674453in}}{\pgfqpoint{6.218773in}{1.677287in}}{\pgfqpoint{6.211640in}{1.677287in}}%
\pgfpathcurveto{\pgfqpoint{6.204507in}{1.677287in}}{\pgfqpoint{6.197665in}{1.674453in}}{\pgfqpoint{6.192622in}{1.669409in}}%
\pgfpathcurveto{\pgfqpoint{6.187578in}{1.664365in}}{\pgfqpoint{6.184744in}{1.657524in}}{\pgfqpoint{6.184744in}{1.650391in}}%
\pgfpathcurveto{\pgfqpoint{6.184744in}{1.643258in}}{\pgfqpoint{6.187578in}{1.636416in}}{\pgfqpoint{6.192622in}{1.631373in}}%
\pgfpathcurveto{\pgfqpoint{6.197665in}{1.626329in}}{\pgfqpoint{6.204507in}{1.623495in}}{\pgfqpoint{6.211640in}{1.623495in}}%
\pgfpathclose%
\pgfusepath{stroke,fill}%
\end{pgfscope}%
\begin{pgfscope}%
\pgfpathrectangle{\pgfqpoint{4.985294in}{0.500000in}}{\pgfqpoint{1.764706in}{1.700000in}}%
\pgfusepath{clip}%
\pgfsetbuttcap%
\pgfsetroundjoin%
\definecolor{currentfill}{rgb}{0.779326,0.096348,0.318766}%
\pgfsetfillcolor{currentfill}%
\pgfsetlinewidth{0.311001pt}%
\definecolor{currentstroke}{rgb}{1.000000,1.000000,1.000000}%
\pgfsetstrokecolor{currentstroke}%
\pgfsetdash{}{0pt}%
\pgfpathmoveto{\pgfqpoint{5.922006in}{1.822965in}}%
\pgfpathcurveto{\pgfqpoint{5.929139in}{1.822965in}}{\pgfqpoint{5.935980in}{1.825799in}}{\pgfqpoint{5.941024in}{1.830843in}}%
\pgfpathcurveto{\pgfqpoint{5.946068in}{1.835886in}}{\pgfqpoint{5.948902in}{1.842728in}}{\pgfqpoint{5.948902in}{1.849861in}}%
\pgfpathcurveto{\pgfqpoint{5.948902in}{1.856994in}}{\pgfqpoint{5.946068in}{1.863835in}}{\pgfqpoint{5.941024in}{1.868879in}}%
\pgfpathcurveto{\pgfqpoint{5.935980in}{1.873923in}}{\pgfqpoint{5.929139in}{1.876756in}}{\pgfqpoint{5.922006in}{1.876756in}}%
\pgfpathcurveto{\pgfqpoint{5.914873in}{1.876756in}}{\pgfqpoint{5.908031in}{1.873923in}}{\pgfqpoint{5.902988in}{1.868879in}}%
\pgfpathcurveto{\pgfqpoint{5.897944in}{1.863835in}}{\pgfqpoint{5.895110in}{1.856994in}}{\pgfqpoint{5.895110in}{1.849861in}}%
\pgfpathcurveto{\pgfqpoint{5.895110in}{1.842728in}}{\pgfqpoint{5.897944in}{1.835886in}}{\pgfqpoint{5.902988in}{1.830843in}}%
\pgfpathcurveto{\pgfqpoint{5.908031in}{1.825799in}}{\pgfqpoint{5.914873in}{1.822965in}}{\pgfqpoint{5.922006in}{1.822965in}}%
\pgfpathclose%
\pgfusepath{stroke,fill}%
\end{pgfscope}%
\begin{pgfscope}%
\pgfpathrectangle{\pgfqpoint{4.985294in}{0.500000in}}{\pgfqpoint{1.764706in}{1.700000in}}%
\pgfusepath{clip}%
\pgfsetbuttcap%
\pgfsetroundjoin%
\definecolor{currentfill}{rgb}{0.970255,0.815666,0.711203}%
\pgfsetfillcolor{currentfill}%
\pgfsetlinewidth{0.311001pt}%
\definecolor{currentstroke}{rgb}{1.000000,1.000000,1.000000}%
\pgfsetstrokecolor{currentstroke}%
\pgfsetdash{}{0pt}%
\pgfpathmoveto{\pgfqpoint{5.502554in}{1.171575in}}%
\pgfpathcurveto{\pgfqpoint{5.509687in}{1.171575in}}{\pgfqpoint{5.516529in}{1.174409in}}{\pgfqpoint{5.521573in}{1.179453in}}%
\pgfpathcurveto{\pgfqpoint{5.526616in}{1.184496in}}{\pgfqpoint{5.529450in}{1.191338in}}{\pgfqpoint{5.529450in}{1.198471in}}%
\pgfpathcurveto{\pgfqpoint{5.529450in}{1.205604in}}{\pgfqpoint{5.526616in}{1.212445in}}{\pgfqpoint{5.521573in}{1.217489in}}%
\pgfpathcurveto{\pgfqpoint{5.516529in}{1.222533in}}{\pgfqpoint{5.509687in}{1.225366in}}{\pgfqpoint{5.502554in}{1.225366in}}%
\pgfpathcurveto{\pgfqpoint{5.495422in}{1.225366in}}{\pgfqpoint{5.488580in}{1.222533in}}{\pgfqpoint{5.483536in}{1.217489in}}%
\pgfpathcurveto{\pgfqpoint{5.478493in}{1.212445in}}{\pgfqpoint{5.475659in}{1.205604in}}{\pgfqpoint{5.475659in}{1.198471in}}%
\pgfpathcurveto{\pgfqpoint{5.475659in}{1.191338in}}{\pgfqpoint{5.478493in}{1.184496in}}{\pgfqpoint{5.483536in}{1.179453in}}%
\pgfpathcurveto{\pgfqpoint{5.488580in}{1.174409in}}{\pgfqpoint{5.495422in}{1.171575in}}{\pgfqpoint{5.502554in}{1.171575in}}%
\pgfpathclose%
\pgfusepath{stroke,fill}%
\end{pgfscope}%
\begin{pgfscope}%
\pgfpathrectangle{\pgfqpoint{4.985294in}{0.500000in}}{\pgfqpoint{1.764706in}{1.700000in}}%
\pgfusepath{clip}%
\pgfsetbuttcap%
\pgfsetroundjoin%
\definecolor{currentfill}{rgb}{0.965753,0.732351,0.592427}%
\pgfsetfillcolor{currentfill}%
\pgfsetlinewidth{0.311001pt}%
\definecolor{currentstroke}{rgb}{1.000000,1.000000,1.000000}%
\pgfsetstrokecolor{currentstroke}%
\pgfsetdash{}{0pt}%
\pgfpathmoveto{\pgfqpoint{6.163132in}{1.551313in}}%
\pgfpathcurveto{\pgfqpoint{6.170264in}{1.551313in}}{\pgfqpoint{6.177106in}{1.554147in}}{\pgfqpoint{6.182150in}{1.559191in}}%
\pgfpathcurveto{\pgfqpoint{6.187193in}{1.564235in}}{\pgfqpoint{6.190027in}{1.571076in}}{\pgfqpoint{6.190027in}{1.578209in}}%
\pgfpathcurveto{\pgfqpoint{6.190027in}{1.585342in}}{\pgfqpoint{6.187193in}{1.592184in}}{\pgfqpoint{6.182150in}{1.597227in}}%
\pgfpathcurveto{\pgfqpoint{6.177106in}{1.602271in}}{\pgfqpoint{6.170264in}{1.605105in}}{\pgfqpoint{6.163132in}{1.605105in}}%
\pgfpathcurveto{\pgfqpoint{6.155999in}{1.605105in}}{\pgfqpoint{6.149157in}{1.602271in}}{\pgfqpoint{6.144114in}{1.597227in}}%
\pgfpathcurveto{\pgfqpoint{6.139070in}{1.592184in}}{\pgfqpoint{6.136236in}{1.585342in}}{\pgfqpoint{6.136236in}{1.578209in}}%
\pgfpathcurveto{\pgfqpoint{6.136236in}{1.571076in}}{\pgfqpoint{6.139070in}{1.564235in}}{\pgfqpoint{6.144114in}{1.559191in}}%
\pgfpathcurveto{\pgfqpoint{6.149157in}{1.554147in}}{\pgfqpoint{6.155999in}{1.551313in}}{\pgfqpoint{6.163132in}{1.551313in}}%
\pgfpathclose%
\pgfusepath{stroke,fill}%
\end{pgfscope}%
\begin{pgfscope}%
\pgfpathrectangle{\pgfqpoint{4.985294in}{0.500000in}}{\pgfqpoint{1.764706in}{1.700000in}}%
\pgfusepath{clip}%
\pgfsetbuttcap%
\pgfsetroundjoin%
\definecolor{currentfill}{rgb}{0.979891,0.908948,0.848279}%
\pgfsetfillcolor{currentfill}%
\pgfsetlinewidth{0.311001pt}%
\definecolor{currentstroke}{rgb}{1.000000,1.000000,1.000000}%
\pgfsetstrokecolor{currentstroke}%
\pgfsetdash{}{0pt}%
\pgfpathmoveto{\pgfqpoint{5.399766in}{1.301237in}}%
\pgfpathcurveto{\pgfqpoint{5.406899in}{1.301237in}}{\pgfqpoint{5.413741in}{1.304071in}}{\pgfqpoint{5.418785in}{1.309114in}}%
\pgfpathcurveto{\pgfqpoint{5.423828in}{1.314158in}}{\pgfqpoint{5.426662in}{1.321000in}}{\pgfqpoint{5.426662in}{1.328132in}}%
\pgfpathcurveto{\pgfqpoint{5.426662in}{1.335265in}}{\pgfqpoint{5.423828in}{1.342107in}}{\pgfqpoint{5.418785in}{1.347151in}}%
\pgfpathcurveto{\pgfqpoint{5.413741in}{1.352194in}}{\pgfqpoint{5.406899in}{1.355028in}}{\pgfqpoint{5.399766in}{1.355028in}}%
\pgfpathcurveto{\pgfqpoint{5.392634in}{1.355028in}}{\pgfqpoint{5.385792in}{1.352194in}}{\pgfqpoint{5.380748in}{1.347151in}}%
\pgfpathcurveto{\pgfqpoint{5.375705in}{1.342107in}}{\pgfqpoint{5.372871in}{1.335265in}}{\pgfqpoint{5.372871in}{1.328132in}}%
\pgfpathcurveto{\pgfqpoint{5.372871in}{1.321000in}}{\pgfqpoint{5.375705in}{1.314158in}}{\pgfqpoint{5.380748in}{1.309114in}}%
\pgfpathcurveto{\pgfqpoint{5.385792in}{1.304071in}}{\pgfqpoint{5.392634in}{1.301237in}}{\pgfqpoint{5.399766in}{1.301237in}}%
\pgfpathclose%
\pgfusepath{stroke,fill}%
\end{pgfscope}%
\begin{pgfscope}%
\pgfpathrectangle{\pgfqpoint{4.985294in}{0.500000in}}{\pgfqpoint{1.764706in}{1.700000in}}%
\pgfusepath{clip}%
\pgfsetbuttcap%
\pgfsetroundjoin%
\definecolor{currentfill}{rgb}{0.963379,0.625574,0.465113}%
\pgfsetfillcolor{currentfill}%
\pgfsetlinewidth{0.311001pt}%
\definecolor{currentstroke}{rgb}{1.000000,1.000000,1.000000}%
\pgfsetstrokecolor{currentstroke}%
\pgfsetdash{}{0pt}%
\pgfpathmoveto{\pgfqpoint{5.645271in}{0.927517in}}%
\pgfpathcurveto{\pgfqpoint{5.652404in}{0.927517in}}{\pgfqpoint{5.659246in}{0.930351in}}{\pgfqpoint{5.664290in}{0.935395in}}%
\pgfpathcurveto{\pgfqpoint{5.669333in}{0.940438in}}{\pgfqpoint{5.672167in}{0.947280in}}{\pgfqpoint{5.672167in}{0.954413in}}%
\pgfpathcurveto{\pgfqpoint{5.672167in}{0.961546in}}{\pgfqpoint{5.669333in}{0.968387in}}{\pgfqpoint{5.664290in}{0.973431in}}%
\pgfpathcurveto{\pgfqpoint{5.659246in}{0.978475in}}{\pgfqpoint{5.652404in}{0.981309in}}{\pgfqpoint{5.645271in}{0.981309in}}%
\pgfpathcurveto{\pgfqpoint{5.638139in}{0.981309in}}{\pgfqpoint{5.631297in}{0.978475in}}{\pgfqpoint{5.626253in}{0.973431in}}%
\pgfpathcurveto{\pgfqpoint{5.621210in}{0.968387in}}{\pgfqpoint{5.618376in}{0.961546in}}{\pgfqpoint{5.618376in}{0.954413in}}%
\pgfpathcurveto{\pgfqpoint{5.618376in}{0.947280in}}{\pgfqpoint{5.621210in}{0.940438in}}{\pgfqpoint{5.626253in}{0.935395in}}%
\pgfpathcurveto{\pgfqpoint{5.631297in}{0.930351in}}{\pgfqpoint{5.638139in}{0.927517in}}{\pgfqpoint{5.645271in}{0.927517in}}%
\pgfpathclose%
\pgfusepath{stroke,fill}%
\end{pgfscope}%
\begin{pgfscope}%
\pgfpathrectangle{\pgfqpoint{4.985294in}{0.500000in}}{\pgfqpoint{1.764706in}{1.700000in}}%
\pgfusepath{clip}%
\pgfsetbuttcap%
\pgfsetroundjoin%
\definecolor{currentfill}{rgb}{0.955697,0.484891,0.334214}%
\pgfsetfillcolor{currentfill}%
\pgfsetlinewidth{0.311001pt}%
\definecolor{currentstroke}{rgb}{1.000000,1.000000,1.000000}%
\pgfsetstrokecolor{currentstroke}%
\pgfsetdash{}{0pt}%
\pgfpathmoveto{\pgfqpoint{6.158006in}{1.346344in}}%
\pgfpathcurveto{\pgfqpoint{6.165139in}{1.346344in}}{\pgfqpoint{6.171980in}{1.349177in}}{\pgfqpoint{6.177024in}{1.354221in}}%
\pgfpathcurveto{\pgfqpoint{6.182067in}{1.359265in}}{\pgfqpoint{6.184901in}{1.366106in}}{\pgfqpoint{6.184901in}{1.373239in}}%
\pgfpathcurveto{\pgfqpoint{6.184901in}{1.380372in}}{\pgfqpoint{6.182067in}{1.387214in}}{\pgfqpoint{6.177024in}{1.392257in}}%
\pgfpathcurveto{\pgfqpoint{6.171980in}{1.397301in}}{\pgfqpoint{6.165139in}{1.400135in}}{\pgfqpoint{6.158006in}{1.400135in}}%
\pgfpathcurveto{\pgfqpoint{6.150873in}{1.400135in}}{\pgfqpoint{6.144031in}{1.397301in}}{\pgfqpoint{6.138988in}{1.392257in}}%
\pgfpathcurveto{\pgfqpoint{6.133944in}{1.387214in}}{\pgfqpoint{6.131110in}{1.380372in}}{\pgfqpoint{6.131110in}{1.373239in}}%
\pgfpathcurveto{\pgfqpoint{6.131110in}{1.366106in}}{\pgfqpoint{6.133944in}{1.359265in}}{\pgfqpoint{6.138988in}{1.354221in}}%
\pgfpathcurveto{\pgfqpoint{6.144031in}{1.349177in}}{\pgfqpoint{6.150873in}{1.346344in}}{\pgfqpoint{6.158006in}{1.346344in}}%
\pgfpathclose%
\pgfusepath{stroke,fill}%
\end{pgfscope}%
\begin{pgfscope}%
\pgfpathrectangle{\pgfqpoint{4.985294in}{0.500000in}}{\pgfqpoint{1.764706in}{1.700000in}}%
\pgfusepath{clip}%
\pgfsetbuttcap%
\pgfsetroundjoin%
\definecolor{currentfill}{rgb}{0.967735,0.780441,0.659127}%
\pgfsetfillcolor{currentfill}%
\pgfsetlinewidth{0.311001pt}%
\definecolor{currentstroke}{rgb}{1.000000,1.000000,1.000000}%
\pgfsetstrokecolor{currentstroke}%
\pgfsetdash{}{0pt}%
\pgfpathmoveto{\pgfqpoint{5.420519in}{1.602200in}}%
\pgfpathcurveto{\pgfqpoint{5.427652in}{1.602200in}}{\pgfqpoint{5.434493in}{1.605034in}}{\pgfqpoint{5.439537in}{1.610077in}}%
\pgfpathcurveto{\pgfqpoint{5.444581in}{1.615121in}}{\pgfqpoint{5.447414in}{1.621963in}}{\pgfqpoint{5.447414in}{1.629095in}}%
\pgfpathcurveto{\pgfqpoint{5.447414in}{1.636228in}}{\pgfqpoint{5.444581in}{1.643070in}}{\pgfqpoint{5.439537in}{1.648114in}}%
\pgfpathcurveto{\pgfqpoint{5.434493in}{1.653157in}}{\pgfqpoint{5.427652in}{1.655991in}}{\pgfqpoint{5.420519in}{1.655991in}}%
\pgfpathcurveto{\pgfqpoint{5.413386in}{1.655991in}}{\pgfqpoint{5.406544in}{1.653157in}}{\pgfqpoint{5.401501in}{1.648114in}}%
\pgfpathcurveto{\pgfqpoint{5.396457in}{1.643070in}}{\pgfqpoint{5.393623in}{1.636228in}}{\pgfqpoint{5.393623in}{1.629095in}}%
\pgfpathcurveto{\pgfqpoint{5.393623in}{1.621963in}}{\pgfqpoint{5.396457in}{1.615121in}}{\pgfqpoint{5.401501in}{1.610077in}}%
\pgfpathcurveto{\pgfqpoint{5.406544in}{1.605034in}}{\pgfqpoint{5.413386in}{1.602200in}}{\pgfqpoint{5.420519in}{1.602200in}}%
\pgfpathclose%
\pgfusepath{stroke,fill}%
\end{pgfscope}%
\begin{pgfscope}%
\pgfpathrectangle{\pgfqpoint{4.985294in}{0.500000in}}{\pgfqpoint{1.764706in}{1.700000in}}%
\pgfusepath{clip}%
\pgfsetbuttcap%
\pgfsetroundjoin%
\definecolor{currentfill}{rgb}{0.969803,0.809811,0.702523}%
\pgfsetfillcolor{currentfill}%
\pgfsetlinewidth{0.311001pt}%
\definecolor{currentstroke}{rgb}{1.000000,1.000000,1.000000}%
\pgfsetstrokecolor{currentstroke}%
\pgfsetdash{}{0pt}%
\pgfpathmoveto{\pgfqpoint{5.479384in}{1.648158in}}%
\pgfpathcurveto{\pgfqpoint{5.486517in}{1.648158in}}{\pgfqpoint{5.493359in}{1.650992in}}{\pgfqpoint{5.498403in}{1.656036in}}%
\pgfpathcurveto{\pgfqpoint{5.503446in}{1.661079in}}{\pgfqpoint{5.506280in}{1.667921in}}{\pgfqpoint{5.506280in}{1.675054in}}%
\pgfpathcurveto{\pgfqpoint{5.506280in}{1.682187in}}{\pgfqpoint{5.503446in}{1.689028in}}{\pgfqpoint{5.498403in}{1.694072in}}%
\pgfpathcurveto{\pgfqpoint{5.493359in}{1.699116in}}{\pgfqpoint{5.486517in}{1.701950in}}{\pgfqpoint{5.479384in}{1.701950in}}%
\pgfpathcurveto{\pgfqpoint{5.472252in}{1.701950in}}{\pgfqpoint{5.465410in}{1.699116in}}{\pgfqpoint{5.460366in}{1.694072in}}%
\pgfpathcurveto{\pgfqpoint{5.455323in}{1.689028in}}{\pgfqpoint{5.452489in}{1.682187in}}{\pgfqpoint{5.452489in}{1.675054in}}%
\pgfpathcurveto{\pgfqpoint{5.452489in}{1.667921in}}{\pgfqpoint{5.455323in}{1.661079in}}{\pgfqpoint{5.460366in}{1.656036in}}%
\pgfpathcurveto{\pgfqpoint{5.465410in}{1.650992in}}{\pgfqpoint{5.472252in}{1.648158in}}{\pgfqpoint{5.479384in}{1.648158in}}%
\pgfpathclose%
\pgfusepath{stroke,fill}%
\end{pgfscope}%
\begin{pgfscope}%
\pgfpathrectangle{\pgfqpoint{4.985294in}{0.500000in}}{\pgfqpoint{1.764706in}{1.700000in}}%
\pgfusepath{clip}%
\pgfsetbuttcap%
\pgfsetroundjoin%
\definecolor{currentfill}{rgb}{0.965042,0.701564,0.552889}%
\pgfsetfillcolor{currentfill}%
\pgfsetlinewidth{0.311001pt}%
\definecolor{currentstroke}{rgb}{1.000000,1.000000,1.000000}%
\pgfsetstrokecolor{currentstroke}%
\pgfsetdash{}{0pt}%
\pgfpathmoveto{\pgfqpoint{6.242801in}{1.710888in}}%
\pgfpathcurveto{\pgfqpoint{6.249934in}{1.710888in}}{\pgfqpoint{6.256775in}{1.713722in}}{\pgfqpoint{6.261819in}{1.718765in}}%
\pgfpathcurveto{\pgfqpoint{6.266863in}{1.723809in}}{\pgfqpoint{6.269696in}{1.730651in}}{\pgfqpoint{6.269696in}{1.737783in}}%
\pgfpathcurveto{\pgfqpoint{6.269696in}{1.744916in}}{\pgfqpoint{6.266863in}{1.751758in}}{\pgfqpoint{6.261819in}{1.756802in}}%
\pgfpathcurveto{\pgfqpoint{6.256775in}{1.761845in}}{\pgfqpoint{6.249934in}{1.764679in}}{\pgfqpoint{6.242801in}{1.764679in}}%
\pgfpathcurveto{\pgfqpoint{6.235668in}{1.764679in}}{\pgfqpoint{6.228826in}{1.761845in}}{\pgfqpoint{6.223783in}{1.756802in}}%
\pgfpathcurveto{\pgfqpoint{6.218739in}{1.751758in}}{\pgfqpoint{6.215905in}{1.744916in}}{\pgfqpoint{6.215905in}{1.737783in}}%
\pgfpathcurveto{\pgfqpoint{6.215905in}{1.730651in}}{\pgfqpoint{6.218739in}{1.723809in}}{\pgfqpoint{6.223783in}{1.718765in}}%
\pgfpathcurveto{\pgfqpoint{6.228826in}{1.713722in}}{\pgfqpoint{6.235668in}{1.710888in}}{\pgfqpoint{6.242801in}{1.710888in}}%
\pgfpathclose%
\pgfusepath{stroke,fill}%
\end{pgfscope}%
\begin{pgfscope}%
\pgfpathrectangle{\pgfqpoint{4.985294in}{0.500000in}}{\pgfqpoint{1.764706in}{1.700000in}}%
\pgfusepath{clip}%
\pgfsetbuttcap%
\pgfsetroundjoin%
\definecolor{currentfill}{rgb}{0.964799,0.689101,0.537560}%
\pgfsetfillcolor{currentfill}%
\pgfsetlinewidth{0.311001pt}%
\definecolor{currentstroke}{rgb}{1.000000,1.000000,1.000000}%
\pgfsetstrokecolor{currentstroke}%
\pgfsetdash{}{0pt}%
\pgfpathmoveto{\pgfqpoint{5.563410in}{1.718748in}}%
\pgfpathcurveto{\pgfqpoint{5.570543in}{1.718748in}}{\pgfqpoint{5.577384in}{1.721582in}}{\pgfqpoint{5.582428in}{1.726626in}}%
\pgfpathcurveto{\pgfqpoint{5.587472in}{1.731669in}}{\pgfqpoint{5.590305in}{1.738511in}}{\pgfqpoint{5.590305in}{1.745644in}}%
\pgfpathcurveto{\pgfqpoint{5.590305in}{1.752777in}}{\pgfqpoint{5.587472in}{1.759618in}}{\pgfqpoint{5.582428in}{1.764662in}}%
\pgfpathcurveto{\pgfqpoint{5.577384in}{1.769706in}}{\pgfqpoint{5.570543in}{1.772540in}}{\pgfqpoint{5.563410in}{1.772540in}}%
\pgfpathcurveto{\pgfqpoint{5.556277in}{1.772540in}}{\pgfqpoint{5.549435in}{1.769706in}}{\pgfqpoint{5.544392in}{1.764662in}}%
\pgfpathcurveto{\pgfqpoint{5.539348in}{1.759618in}}{\pgfqpoint{5.536514in}{1.752777in}}{\pgfqpoint{5.536514in}{1.745644in}}%
\pgfpathcurveto{\pgfqpoint{5.536514in}{1.738511in}}{\pgfqpoint{5.539348in}{1.731669in}}{\pgfqpoint{5.544392in}{1.726626in}}%
\pgfpathcurveto{\pgfqpoint{5.549435in}{1.721582in}}{\pgfqpoint{5.556277in}{1.718748in}}{\pgfqpoint{5.563410in}{1.718748in}}%
\pgfpathclose%
\pgfusepath{stroke,fill}%
\end{pgfscope}%
\begin{pgfscope}%
\pgfpathrectangle{\pgfqpoint{4.985294in}{0.500000in}}{\pgfqpoint{1.764706in}{1.700000in}}%
\pgfusepath{clip}%
\pgfsetbuttcap%
\pgfsetroundjoin%
\definecolor{currentfill}{rgb}{0.979891,0.908948,0.848279}%
\pgfsetfillcolor{currentfill}%
\pgfsetlinewidth{0.311001pt}%
\definecolor{currentstroke}{rgb}{1.000000,1.000000,1.000000}%
\pgfsetstrokecolor{currentstroke}%
\pgfsetdash{}{0pt}%
\pgfpathmoveto{\pgfqpoint{6.314781in}{1.483569in}}%
\pgfpathcurveto{\pgfqpoint{6.321914in}{1.483569in}}{\pgfqpoint{6.328755in}{1.486402in}}{\pgfqpoint{6.333799in}{1.491446in}}%
\pgfpathcurveto{\pgfqpoint{6.338843in}{1.496490in}}{\pgfqpoint{6.341676in}{1.503331in}}{\pgfqpoint{6.341676in}{1.510464in}}%
\pgfpathcurveto{\pgfqpoint{6.341676in}{1.517597in}}{\pgfqpoint{6.338843in}{1.524439in}}{\pgfqpoint{6.333799in}{1.529482in}}%
\pgfpathcurveto{\pgfqpoint{6.328755in}{1.534526in}}{\pgfqpoint{6.321914in}{1.537360in}}{\pgfqpoint{6.314781in}{1.537360in}}%
\pgfpathcurveto{\pgfqpoint{6.307648in}{1.537360in}}{\pgfqpoint{6.300806in}{1.534526in}}{\pgfqpoint{6.295763in}{1.529482in}}%
\pgfpathcurveto{\pgfqpoint{6.290719in}{1.524439in}}{\pgfqpoint{6.287885in}{1.517597in}}{\pgfqpoint{6.287885in}{1.510464in}}%
\pgfpathcurveto{\pgfqpoint{6.287885in}{1.503331in}}{\pgfqpoint{6.290719in}{1.496490in}}{\pgfqpoint{6.295763in}{1.491446in}}%
\pgfpathcurveto{\pgfqpoint{6.300806in}{1.486402in}}{\pgfqpoint{6.307648in}{1.483569in}}{\pgfqpoint{6.314781in}{1.483569in}}%
\pgfpathclose%
\pgfusepath{stroke,fill}%
\end{pgfscope}%
\begin{pgfscope}%
\pgfpathrectangle{\pgfqpoint{4.985294in}{0.500000in}}{\pgfqpoint{1.764706in}{1.700000in}}%
\pgfusepath{clip}%
\pgfsetbuttcap%
\pgfsetroundjoin%
\definecolor{currentfill}{rgb}{0.975644,0.874038,0.797253}%
\pgfsetfillcolor{currentfill}%
\pgfsetlinewidth{0.311001pt}%
\definecolor{currentstroke}{rgb}{1.000000,1.000000,1.000000}%
\pgfsetstrokecolor{currentstroke}%
\pgfsetdash{}{0pt}%
\pgfpathmoveto{\pgfqpoint{5.448496in}{1.539838in}}%
\pgfpathcurveto{\pgfqpoint{5.455629in}{1.539838in}}{\pgfqpoint{5.462471in}{1.542672in}}{\pgfqpoint{5.467515in}{1.547716in}}%
\pgfpathcurveto{\pgfqpoint{5.472558in}{1.552759in}}{\pgfqpoint{5.475392in}{1.559601in}}{\pgfqpoint{5.475392in}{1.566734in}}%
\pgfpathcurveto{\pgfqpoint{5.475392in}{1.573867in}}{\pgfqpoint{5.472558in}{1.580708in}}{\pgfqpoint{5.467515in}{1.585752in}}%
\pgfpathcurveto{\pgfqpoint{5.462471in}{1.590796in}}{\pgfqpoint{5.455629in}{1.593629in}}{\pgfqpoint{5.448496in}{1.593629in}}%
\pgfpathcurveto{\pgfqpoint{5.441364in}{1.593629in}}{\pgfqpoint{5.434522in}{1.590796in}}{\pgfqpoint{5.429478in}{1.585752in}}%
\pgfpathcurveto{\pgfqpoint{5.424435in}{1.580708in}}{\pgfqpoint{5.421601in}{1.573867in}}{\pgfqpoint{5.421601in}{1.566734in}}%
\pgfpathcurveto{\pgfqpoint{5.421601in}{1.559601in}}{\pgfqpoint{5.424435in}{1.552759in}}{\pgfqpoint{5.429478in}{1.547716in}}%
\pgfpathcurveto{\pgfqpoint{5.434522in}{1.542672in}}{\pgfqpoint{5.441364in}{1.539838in}}{\pgfqpoint{5.448496in}{1.539838in}}%
\pgfpathclose%
\pgfusepath{stroke,fill}%
\end{pgfscope}%
\begin{pgfscope}%
\pgfpathrectangle{\pgfqpoint{4.985294in}{0.500000in}}{\pgfqpoint{1.764706in}{1.700000in}}%
\pgfusepath{clip}%
\pgfsetbuttcap%
\pgfsetroundjoin%
\definecolor{currentfill}{rgb}{0.965753,0.732351,0.592427}%
\pgfsetfillcolor{currentfill}%
\pgfsetlinewidth{0.311001pt}%
\definecolor{currentstroke}{rgb}{1.000000,1.000000,1.000000}%
\pgfsetstrokecolor{currentstroke}%
\pgfsetdash{}{0pt}%
\pgfpathmoveto{\pgfqpoint{5.534586in}{1.155681in}}%
\pgfpathcurveto{\pgfqpoint{5.541719in}{1.155681in}}{\pgfqpoint{5.548561in}{1.158515in}}{\pgfqpoint{5.553604in}{1.163558in}}%
\pgfpathcurveto{\pgfqpoint{5.558648in}{1.168602in}}{\pgfqpoint{5.561482in}{1.175444in}}{\pgfqpoint{5.561482in}{1.182577in}}%
\pgfpathcurveto{\pgfqpoint{5.561482in}{1.189709in}}{\pgfqpoint{5.558648in}{1.196551in}}{\pgfqpoint{5.553604in}{1.201595in}}%
\pgfpathcurveto{\pgfqpoint{5.548561in}{1.206638in}}{\pgfqpoint{5.541719in}{1.209472in}}{\pgfqpoint{5.534586in}{1.209472in}}%
\pgfpathcurveto{\pgfqpoint{5.527453in}{1.209472in}}{\pgfqpoint{5.520612in}{1.206638in}}{\pgfqpoint{5.515568in}{1.201595in}}%
\pgfpathcurveto{\pgfqpoint{5.510524in}{1.196551in}}{\pgfqpoint{5.507690in}{1.189709in}}{\pgfqpoint{5.507690in}{1.182577in}}%
\pgfpathcurveto{\pgfqpoint{5.507690in}{1.175444in}}{\pgfqpoint{5.510524in}{1.168602in}}{\pgfqpoint{5.515568in}{1.163558in}}%
\pgfpathcurveto{\pgfqpoint{5.520612in}{1.158515in}}{\pgfqpoint{5.527453in}{1.155681in}}{\pgfqpoint{5.534586in}{1.155681in}}%
\pgfpathclose%
\pgfusepath{stroke,fill}%
\end{pgfscope}%
\begin{pgfscope}%
\pgfpathrectangle{\pgfqpoint{4.985294in}{0.500000in}}{\pgfqpoint{1.764706in}{1.700000in}}%
\pgfusepath{clip}%
\pgfsetbuttcap%
\pgfsetroundjoin%
\definecolor{currentfill}{rgb}{0.964306,0.663930,0.507747}%
\pgfsetfillcolor{currentfill}%
\pgfsetlinewidth{0.311001pt}%
\definecolor{currentstroke}{rgb}{1.000000,1.000000,1.000000}%
\pgfsetstrokecolor{currentstroke}%
\pgfsetdash{}{0pt}%
\pgfpathmoveto{\pgfqpoint{6.107727in}{1.698895in}}%
\pgfpathcurveto{\pgfqpoint{6.114860in}{1.698895in}}{\pgfqpoint{6.121702in}{1.701729in}}{\pgfqpoint{6.126745in}{1.706773in}}%
\pgfpathcurveto{\pgfqpoint{6.131789in}{1.711816in}}{\pgfqpoint{6.134623in}{1.718658in}}{\pgfqpoint{6.134623in}{1.725791in}}%
\pgfpathcurveto{\pgfqpoint{6.134623in}{1.732924in}}{\pgfqpoint{6.131789in}{1.739765in}}{\pgfqpoint{6.126745in}{1.744809in}}%
\pgfpathcurveto{\pgfqpoint{6.121702in}{1.749853in}}{\pgfqpoint{6.114860in}{1.752687in}}{\pgfqpoint{6.107727in}{1.752687in}}%
\pgfpathcurveto{\pgfqpoint{6.100594in}{1.752687in}}{\pgfqpoint{6.093753in}{1.749853in}}{\pgfqpoint{6.088709in}{1.744809in}}%
\pgfpathcurveto{\pgfqpoint{6.083665in}{1.739765in}}{\pgfqpoint{6.080832in}{1.732924in}}{\pgfqpoint{6.080832in}{1.725791in}}%
\pgfpathcurveto{\pgfqpoint{6.080832in}{1.718658in}}{\pgfqpoint{6.083665in}{1.711816in}}{\pgfqpoint{6.088709in}{1.706773in}}%
\pgfpathcurveto{\pgfqpoint{6.093753in}{1.701729in}}{\pgfqpoint{6.100594in}{1.698895in}}{\pgfqpoint{6.107727in}{1.698895in}}%
\pgfpathclose%
\pgfusepath{stroke,fill}%
\end{pgfscope}%
\begin{pgfscope}%
\pgfpathrectangle{\pgfqpoint{4.985294in}{0.500000in}}{\pgfqpoint{1.764706in}{1.700000in}}%
\pgfusepath{clip}%
\pgfsetbuttcap%
\pgfsetroundjoin%
\definecolor{currentfill}{rgb}{0.974412,0.862387,0.780156}%
\pgfsetfillcolor{currentfill}%
\pgfsetlinewidth{0.311001pt}%
\definecolor{currentstroke}{rgb}{1.000000,1.000000,1.000000}%
\pgfsetstrokecolor{currentstroke}%
\pgfsetdash{}{0pt}%
\pgfpathmoveto{\pgfqpoint{6.255292in}{1.249897in}}%
\pgfpathcurveto{\pgfqpoint{6.262424in}{1.249897in}}{\pgfqpoint{6.269266in}{1.252731in}}{\pgfqpoint{6.274310in}{1.257775in}}%
\pgfpathcurveto{\pgfqpoint{6.279353in}{1.262818in}}{\pgfqpoint{6.282187in}{1.269660in}}{\pgfqpoint{6.282187in}{1.276793in}}%
\pgfpathcurveto{\pgfqpoint{6.282187in}{1.283926in}}{\pgfqpoint{6.279353in}{1.290767in}}{\pgfqpoint{6.274310in}{1.295811in}}%
\pgfpathcurveto{\pgfqpoint{6.269266in}{1.300855in}}{\pgfqpoint{6.262424in}{1.303689in}}{\pgfqpoint{6.255292in}{1.303689in}}%
\pgfpathcurveto{\pgfqpoint{6.248159in}{1.303689in}}{\pgfqpoint{6.241317in}{1.300855in}}{\pgfqpoint{6.236273in}{1.295811in}}%
\pgfpathcurveto{\pgfqpoint{6.231230in}{1.290767in}}{\pgfqpoint{6.228396in}{1.283926in}}{\pgfqpoint{6.228396in}{1.276793in}}%
\pgfpathcurveto{\pgfqpoint{6.228396in}{1.269660in}}{\pgfqpoint{6.231230in}{1.262818in}}{\pgfqpoint{6.236273in}{1.257775in}}%
\pgfpathcurveto{\pgfqpoint{6.241317in}{1.252731in}}{\pgfqpoint{6.248159in}{1.249897in}}{\pgfqpoint{6.255292in}{1.249897in}}%
\pgfpathclose%
\pgfusepath{stroke,fill}%
\end{pgfscope}%
\begin{pgfscope}%
\pgfpathrectangle{\pgfqpoint{4.985294in}{0.500000in}}{\pgfqpoint{1.764706in}{1.700000in}}%
\pgfusepath{clip}%
\pgfsetbuttcap%
\pgfsetroundjoin%
\definecolor{currentfill}{rgb}{0.965042,0.701564,0.552889}%
\pgfsetfillcolor{currentfill}%
\pgfsetlinewidth{0.311001pt}%
\definecolor{currentstroke}{rgb}{1.000000,1.000000,1.000000}%
\pgfsetstrokecolor{currentstroke}%
\pgfsetdash{}{0pt}%
\pgfpathmoveto{\pgfqpoint{5.592159in}{0.912136in}}%
\pgfpathcurveto{\pgfqpoint{5.599292in}{0.912136in}}{\pgfqpoint{5.606134in}{0.914970in}}{\pgfqpoint{5.611178in}{0.920014in}}%
\pgfpathcurveto{\pgfqpoint{5.616221in}{0.925058in}}{\pgfqpoint{5.619055in}{0.931899in}}{\pgfqpoint{5.619055in}{0.939032in}}%
\pgfpathcurveto{\pgfqpoint{5.619055in}{0.946165in}}{\pgfqpoint{5.616221in}{0.953007in}}{\pgfqpoint{5.611178in}{0.958050in}}%
\pgfpathcurveto{\pgfqpoint{5.606134in}{0.963094in}}{\pgfqpoint{5.599292in}{0.965928in}}{\pgfqpoint{5.592159in}{0.965928in}}%
\pgfpathcurveto{\pgfqpoint{5.585027in}{0.965928in}}{\pgfqpoint{5.578185in}{0.963094in}}{\pgfqpoint{5.573141in}{0.958050in}}%
\pgfpathcurveto{\pgfqpoint{5.568098in}{0.953007in}}{\pgfqpoint{5.565264in}{0.946165in}}{\pgfqpoint{5.565264in}{0.939032in}}%
\pgfpathcurveto{\pgfqpoint{5.565264in}{0.931899in}}{\pgfqpoint{5.568098in}{0.925058in}}{\pgfqpoint{5.573141in}{0.920014in}}%
\pgfpathcurveto{\pgfqpoint{5.578185in}{0.914970in}}{\pgfqpoint{5.585027in}{0.912136in}}{\pgfqpoint{5.592159in}{0.912136in}}%
\pgfpathclose%
\pgfusepath{stroke,fill}%
\end{pgfscope}%
\begin{pgfscope}%
\pgfpathrectangle{\pgfqpoint{4.985294in}{0.500000in}}{\pgfqpoint{1.764706in}{1.700000in}}%
\pgfusepath{clip}%
\pgfsetbuttcap%
\pgfsetroundjoin%
\definecolor{currentfill}{rgb}{0.970718,0.821518,0.719872}%
\pgfsetfillcolor{currentfill}%
\pgfsetlinewidth{0.311001pt}%
\definecolor{currentstroke}{rgb}{1.000000,1.000000,1.000000}%
\pgfsetstrokecolor{currentstroke}%
\pgfsetdash{}{0pt}%
\pgfpathmoveto{\pgfqpoint{6.297668in}{1.057078in}}%
\pgfpathcurveto{\pgfqpoint{6.304801in}{1.057078in}}{\pgfqpoint{6.311642in}{1.059912in}}{\pgfqpoint{6.316686in}{1.064955in}}%
\pgfpathcurveto{\pgfqpoint{6.321730in}{1.069999in}}{\pgfqpoint{6.324564in}{1.076841in}}{\pgfqpoint{6.324564in}{1.083974in}}%
\pgfpathcurveto{\pgfqpoint{6.324564in}{1.091106in}}{\pgfqpoint{6.321730in}{1.097948in}}{\pgfqpoint{6.316686in}{1.102992in}}%
\pgfpathcurveto{\pgfqpoint{6.311642in}{1.108035in}}{\pgfqpoint{6.304801in}{1.110869in}}{\pgfqpoint{6.297668in}{1.110869in}}%
\pgfpathcurveto{\pgfqpoint{6.290535in}{1.110869in}}{\pgfqpoint{6.283693in}{1.108035in}}{\pgfqpoint{6.278650in}{1.102992in}}%
\pgfpathcurveto{\pgfqpoint{6.273606in}{1.097948in}}{\pgfqpoint{6.270772in}{1.091106in}}{\pgfqpoint{6.270772in}{1.083974in}}%
\pgfpathcurveto{\pgfqpoint{6.270772in}{1.076841in}}{\pgfqpoint{6.273606in}{1.069999in}}{\pgfqpoint{6.278650in}{1.064955in}}%
\pgfpathcurveto{\pgfqpoint{6.283693in}{1.059912in}}{\pgfqpoint{6.290535in}{1.057078in}}{\pgfqpoint{6.297668in}{1.057078in}}%
\pgfpathclose%
\pgfusepath{stroke,fill}%
\end{pgfscope}%
\begin{pgfscope}%
\pgfpathrectangle{\pgfqpoint{4.985294in}{0.500000in}}{\pgfqpoint{1.764706in}{1.700000in}}%
\pgfusepath{clip}%
\pgfsetbuttcap%
\pgfsetroundjoin%
\definecolor{currentfill}{rgb}{0.968105,0.786346,0.667739}%
\pgfsetfillcolor{currentfill}%
\pgfsetlinewidth{0.311001pt}%
\definecolor{currentstroke}{rgb}{1.000000,1.000000,1.000000}%
\pgfsetstrokecolor{currentstroke}%
\pgfsetdash{}{0pt}%
\pgfpathmoveto{\pgfqpoint{5.388823in}{1.091370in}}%
\pgfpathcurveto{\pgfqpoint{5.395956in}{1.091370in}}{\pgfqpoint{5.402797in}{1.094204in}}{\pgfqpoint{5.407841in}{1.099247in}}%
\pgfpathcurveto{\pgfqpoint{5.412884in}{1.104291in}}{\pgfqpoint{5.415718in}{1.111133in}}{\pgfqpoint{5.415718in}{1.118266in}}%
\pgfpathcurveto{\pgfqpoint{5.415718in}{1.125398in}}{\pgfqpoint{5.412884in}{1.132240in}}{\pgfqpoint{5.407841in}{1.137284in}}%
\pgfpathcurveto{\pgfqpoint{5.402797in}{1.142327in}}{\pgfqpoint{5.395956in}{1.145161in}}{\pgfqpoint{5.388823in}{1.145161in}}%
\pgfpathcurveto{\pgfqpoint{5.381690in}{1.145161in}}{\pgfqpoint{5.374848in}{1.142327in}}{\pgfqpoint{5.369805in}{1.137284in}}%
\pgfpathcurveto{\pgfqpoint{5.364761in}{1.132240in}}{\pgfqpoint{5.361927in}{1.125398in}}{\pgfqpoint{5.361927in}{1.118266in}}%
\pgfpathcurveto{\pgfqpoint{5.361927in}{1.111133in}}{\pgfqpoint{5.364761in}{1.104291in}}{\pgfqpoint{5.369805in}{1.099247in}}%
\pgfpathcurveto{\pgfqpoint{5.374848in}{1.094204in}}{\pgfqpoint{5.381690in}{1.091370in}}{\pgfqpoint{5.388823in}{1.091370in}}%
\pgfpathclose%
\pgfusepath{stroke,fill}%
\end{pgfscope}%
\begin{pgfscope}%
\pgfpathrectangle{\pgfqpoint{4.985294in}{0.500000in}}{\pgfqpoint{1.764706in}{1.700000in}}%
\pgfusepath{clip}%
\pgfsetbuttcap%
\pgfsetroundjoin%
\definecolor{currentfill}{rgb}{0.977657,0.891500,0.822809}%
\pgfsetfillcolor{currentfill}%
\pgfsetlinewidth{0.311001pt}%
\definecolor{currentstroke}{rgb}{1.000000,1.000000,1.000000}%
\pgfsetstrokecolor{currentstroke}%
\pgfsetdash{}{0pt}%
\pgfpathmoveto{\pgfqpoint{5.445729in}{1.134566in}}%
\pgfpathcurveto{\pgfqpoint{5.452862in}{1.134566in}}{\pgfqpoint{5.459704in}{1.137399in}}{\pgfqpoint{5.464747in}{1.142443in}}%
\pgfpathcurveto{\pgfqpoint{5.469791in}{1.147487in}}{\pgfqpoint{5.472625in}{1.154328in}}{\pgfqpoint{5.472625in}{1.161461in}}%
\pgfpathcurveto{\pgfqpoint{5.472625in}{1.168594in}}{\pgfqpoint{5.469791in}{1.175436in}}{\pgfqpoint{5.464747in}{1.180479in}}%
\pgfpathcurveto{\pgfqpoint{5.459704in}{1.185523in}}{\pgfqpoint{5.452862in}{1.188357in}}{\pgfqpoint{5.445729in}{1.188357in}}%
\pgfpathcurveto{\pgfqpoint{5.438596in}{1.188357in}}{\pgfqpoint{5.431755in}{1.185523in}}{\pgfqpoint{5.426711in}{1.180479in}}%
\pgfpathcurveto{\pgfqpoint{5.421667in}{1.175436in}}{\pgfqpoint{5.418834in}{1.168594in}}{\pgfqpoint{5.418834in}{1.161461in}}%
\pgfpathcurveto{\pgfqpoint{5.418834in}{1.154328in}}{\pgfqpoint{5.421667in}{1.147487in}}{\pgfqpoint{5.426711in}{1.142443in}}%
\pgfpathcurveto{\pgfqpoint{5.431755in}{1.137399in}}{\pgfqpoint{5.438596in}{1.134566in}}{\pgfqpoint{5.445729in}{1.134566in}}%
\pgfpathclose%
\pgfusepath{stroke,fill}%
\end{pgfscope}%
\begin{pgfscope}%
\pgfpathrectangle{\pgfqpoint{4.985294in}{0.500000in}}{\pgfqpoint{1.764706in}{1.700000in}}%
\pgfusepath{clip}%
\pgfsetbuttcap%
\pgfsetroundjoin%
\definecolor{currentfill}{rgb}{0.973832,0.856556,0.771584}%
\pgfsetfillcolor{currentfill}%
\pgfsetlinewidth{0.311001pt}%
\definecolor{currentstroke}{rgb}{1.000000,1.000000,1.000000}%
\pgfsetstrokecolor{currentstroke}%
\pgfsetdash{}{0pt}%
\pgfpathmoveto{\pgfqpoint{5.473584in}{1.034632in}}%
\pgfpathcurveto{\pgfqpoint{5.480717in}{1.034632in}}{\pgfqpoint{5.487558in}{1.037466in}}{\pgfqpoint{5.492602in}{1.042510in}}%
\pgfpathcurveto{\pgfqpoint{5.497646in}{1.047553in}}{\pgfqpoint{5.500479in}{1.054395in}}{\pgfqpoint{5.500479in}{1.061528in}}%
\pgfpathcurveto{\pgfqpoint{5.500479in}{1.068661in}}{\pgfqpoint{5.497646in}{1.075502in}}{\pgfqpoint{5.492602in}{1.080546in}}%
\pgfpathcurveto{\pgfqpoint{5.487558in}{1.085590in}}{\pgfqpoint{5.480717in}{1.088424in}}{\pgfqpoint{5.473584in}{1.088424in}}%
\pgfpathcurveto{\pgfqpoint{5.466451in}{1.088424in}}{\pgfqpoint{5.459609in}{1.085590in}}{\pgfqpoint{5.454566in}{1.080546in}}%
\pgfpathcurveto{\pgfqpoint{5.449522in}{1.075502in}}{\pgfqpoint{5.446688in}{1.068661in}}{\pgfqpoint{5.446688in}{1.061528in}}%
\pgfpathcurveto{\pgfqpoint{5.446688in}{1.054395in}}{\pgfqpoint{5.449522in}{1.047553in}}{\pgfqpoint{5.454566in}{1.042510in}}%
\pgfpathcurveto{\pgfqpoint{5.459609in}{1.037466in}}{\pgfqpoint{5.466451in}{1.034632in}}{\pgfqpoint{5.473584in}{1.034632in}}%
\pgfpathclose%
\pgfusepath{stroke,fill}%
\end{pgfscope}%
\begin{pgfscope}%
\pgfpathrectangle{\pgfqpoint{4.985294in}{0.500000in}}{\pgfqpoint{1.764706in}{1.700000in}}%
\pgfusepath{clip}%
\pgfsetbuttcap%
\pgfsetroundjoin%
\definecolor{currentfill}{rgb}{0.954476,0.470822,0.323110}%
\pgfsetfillcolor{currentfill}%
\pgfsetlinewidth{0.311001pt}%
\definecolor{currentstroke}{rgb}{1.000000,1.000000,1.000000}%
\pgfsetstrokecolor{currentstroke}%
\pgfsetdash{}{0pt}%
\pgfpathmoveto{\pgfqpoint{5.524412in}{1.794219in}}%
\pgfpathcurveto{\pgfqpoint{5.531545in}{1.794219in}}{\pgfqpoint{5.538387in}{1.797053in}}{\pgfqpoint{5.543430in}{1.802096in}}%
\pgfpathcurveto{\pgfqpoint{5.548474in}{1.807140in}}{\pgfqpoint{5.551308in}{1.813982in}}{\pgfqpoint{5.551308in}{1.821115in}}%
\pgfpathcurveto{\pgfqpoint{5.551308in}{1.828247in}}{\pgfqpoint{5.548474in}{1.835089in}}{\pgfqpoint{5.543430in}{1.840133in}}%
\pgfpathcurveto{\pgfqpoint{5.538387in}{1.845176in}}{\pgfqpoint{5.531545in}{1.848010in}}{\pgfqpoint{5.524412in}{1.848010in}}%
\pgfpathcurveto{\pgfqpoint{5.517279in}{1.848010in}}{\pgfqpoint{5.510438in}{1.845176in}}{\pgfqpoint{5.505394in}{1.840133in}}%
\pgfpathcurveto{\pgfqpoint{5.500350in}{1.835089in}}{\pgfqpoint{5.497516in}{1.828247in}}{\pgfqpoint{5.497516in}{1.821115in}}%
\pgfpathcurveto{\pgfqpoint{5.497516in}{1.813982in}}{\pgfqpoint{5.500350in}{1.807140in}}{\pgfqpoint{5.505394in}{1.802096in}}%
\pgfpathcurveto{\pgfqpoint{5.510438in}{1.797053in}}{\pgfqpoint{5.517279in}{1.794219in}}{\pgfqpoint{5.524412in}{1.794219in}}%
\pgfpathclose%
\pgfusepath{stroke,fill}%
\end{pgfscope}%
\begin{pgfscope}%
\pgfpathrectangle{\pgfqpoint{4.985294in}{0.500000in}}{\pgfqpoint{1.764706in}{1.700000in}}%
\pgfusepath{clip}%
\pgfsetbuttcap%
\pgfsetroundjoin%
\definecolor{currentfill}{rgb}{0.973271,0.850724,0.762998}%
\pgfsetfillcolor{currentfill}%
\pgfsetlinewidth{0.311001pt}%
\definecolor{currentstroke}{rgb}{1.000000,1.000000,1.000000}%
\pgfsetstrokecolor{currentstroke}%
\pgfsetdash{}{0pt}%
\pgfpathmoveto{\pgfqpoint{5.447192in}{1.575947in}}%
\pgfpathcurveto{\pgfqpoint{5.454324in}{1.575947in}}{\pgfqpoint{5.461166in}{1.578781in}}{\pgfqpoint{5.466210in}{1.583825in}}%
\pgfpathcurveto{\pgfqpoint{5.471253in}{1.588868in}}{\pgfqpoint{5.474087in}{1.595710in}}{\pgfqpoint{5.474087in}{1.602843in}}%
\pgfpathcurveto{\pgfqpoint{5.474087in}{1.609976in}}{\pgfqpoint{5.471253in}{1.616817in}}{\pgfqpoint{5.466210in}{1.621861in}}%
\pgfpathcurveto{\pgfqpoint{5.461166in}{1.626905in}}{\pgfqpoint{5.454324in}{1.629739in}}{\pgfqpoint{5.447192in}{1.629739in}}%
\pgfpathcurveto{\pgfqpoint{5.440059in}{1.629739in}}{\pgfqpoint{5.433217in}{1.626905in}}{\pgfqpoint{5.428173in}{1.621861in}}%
\pgfpathcurveto{\pgfqpoint{5.423130in}{1.616817in}}{\pgfqpoint{5.420296in}{1.609976in}}{\pgfqpoint{5.420296in}{1.602843in}}%
\pgfpathcurveto{\pgfqpoint{5.420296in}{1.595710in}}{\pgfqpoint{5.423130in}{1.588868in}}{\pgfqpoint{5.428173in}{1.583825in}}%
\pgfpathcurveto{\pgfqpoint{5.433217in}{1.578781in}}{\pgfqpoint{5.440059in}{1.575947in}}{\pgfqpoint{5.447192in}{1.575947in}}%
\pgfpathclose%
\pgfusepath{stroke,fill}%
\end{pgfscope}%
\begin{pgfscope}%
\pgfpathrectangle{\pgfqpoint{4.985294in}{0.500000in}}{\pgfqpoint{1.764706in}{1.700000in}}%
\pgfusepath{clip}%
\pgfsetbuttcap%
\pgfsetroundjoin%
\definecolor{currentfill}{rgb}{0.965302,0.713942,0.568499}%
\pgfsetfillcolor{currentfill}%
\pgfsetlinewidth{0.311001pt}%
\definecolor{currentstroke}{rgb}{1.000000,1.000000,1.000000}%
\pgfsetstrokecolor{currentstroke}%
\pgfsetdash{}{0pt}%
\pgfpathmoveto{\pgfqpoint{6.356611in}{1.562442in}}%
\pgfpathcurveto{\pgfqpoint{6.363744in}{1.562442in}}{\pgfqpoint{6.370585in}{1.565276in}}{\pgfqpoint{6.375629in}{1.570319in}}%
\pgfpathcurveto{\pgfqpoint{6.380673in}{1.575363in}}{\pgfqpoint{6.383507in}{1.582205in}}{\pgfqpoint{6.383507in}{1.589338in}}%
\pgfpathcurveto{\pgfqpoint{6.383507in}{1.596470in}}{\pgfqpoint{6.380673in}{1.603312in}}{\pgfqpoint{6.375629in}{1.608356in}}%
\pgfpathcurveto{\pgfqpoint{6.370585in}{1.613399in}}{\pgfqpoint{6.363744in}{1.616233in}}{\pgfqpoint{6.356611in}{1.616233in}}%
\pgfpathcurveto{\pgfqpoint{6.349478in}{1.616233in}}{\pgfqpoint{6.342636in}{1.613399in}}{\pgfqpoint{6.337593in}{1.608356in}}%
\pgfpathcurveto{\pgfqpoint{6.332549in}{1.603312in}}{\pgfqpoint{6.329715in}{1.596470in}}{\pgfqpoint{6.329715in}{1.589338in}}%
\pgfpathcurveto{\pgfqpoint{6.329715in}{1.582205in}}{\pgfqpoint{6.332549in}{1.575363in}}{\pgfqpoint{6.337593in}{1.570319in}}%
\pgfpathcurveto{\pgfqpoint{6.342636in}{1.565276in}}{\pgfqpoint{6.349478in}{1.562442in}}{\pgfqpoint{6.356611in}{1.562442in}}%
\pgfpathclose%
\pgfusepath{stroke,fill}%
\end{pgfscope}%
\begin{pgfscope}%
\pgfpathrectangle{\pgfqpoint{4.985294in}{0.500000in}}{\pgfqpoint{1.764706in}{1.700000in}}%
\pgfusepath{clip}%
\pgfsetbuttcap%
\pgfsetroundjoin%
\definecolor{currentfill}{rgb}{0.973832,0.856556,0.771584}%
\pgfsetfillcolor{currentfill}%
\pgfsetlinewidth{0.311001pt}%
\definecolor{currentstroke}{rgb}{1.000000,1.000000,1.000000}%
\pgfsetstrokecolor{currentstroke}%
\pgfsetdash{}{0pt}%
\pgfpathmoveto{\pgfqpoint{5.494434in}{1.029359in}}%
\pgfpathcurveto{\pgfqpoint{5.501567in}{1.029359in}}{\pgfqpoint{5.508408in}{1.032193in}}{\pgfqpoint{5.513452in}{1.037237in}}%
\pgfpathcurveto{\pgfqpoint{5.518496in}{1.042281in}}{\pgfqpoint{5.521329in}{1.049122in}}{\pgfqpoint{5.521329in}{1.056255in}}%
\pgfpathcurveto{\pgfqpoint{5.521329in}{1.063388in}}{\pgfqpoint{5.518496in}{1.070230in}}{\pgfqpoint{5.513452in}{1.075273in}}%
\pgfpathcurveto{\pgfqpoint{5.508408in}{1.080317in}}{\pgfqpoint{5.501567in}{1.083151in}}{\pgfqpoint{5.494434in}{1.083151in}}%
\pgfpathcurveto{\pgfqpoint{5.487301in}{1.083151in}}{\pgfqpoint{5.480459in}{1.080317in}}{\pgfqpoint{5.475416in}{1.075273in}}%
\pgfpathcurveto{\pgfqpoint{5.470372in}{1.070230in}}{\pgfqpoint{5.467538in}{1.063388in}}{\pgfqpoint{5.467538in}{1.056255in}}%
\pgfpathcurveto{\pgfqpoint{5.467538in}{1.049122in}}{\pgfqpoint{5.470372in}{1.042281in}}{\pgfqpoint{5.475416in}{1.037237in}}%
\pgfpathcurveto{\pgfqpoint{5.480459in}{1.032193in}}{\pgfqpoint{5.487301in}{1.029359in}}{\pgfqpoint{5.494434in}{1.029359in}}%
\pgfpathclose%
\pgfusepath{stroke,fill}%
\end{pgfscope}%
\begin{pgfscope}%
\pgfpathrectangle{\pgfqpoint{4.985294in}{0.500000in}}{\pgfqpoint{1.764706in}{1.700000in}}%
\pgfusepath{clip}%
\pgfsetbuttcap%
\pgfsetroundjoin%
\definecolor{currentfill}{rgb}{0.964306,0.663930,0.507747}%
\pgfsetfillcolor{currentfill}%
\pgfsetlinewidth{0.311001pt}%
\definecolor{currentstroke}{rgb}{1.000000,1.000000,1.000000}%
\pgfsetstrokecolor{currentstroke}%
\pgfsetdash{}{0pt}%
\pgfpathmoveto{\pgfqpoint{5.587880in}{1.574071in}}%
\pgfpathcurveto{\pgfqpoint{5.595013in}{1.574071in}}{\pgfqpoint{5.601855in}{1.576905in}}{\pgfqpoint{5.606898in}{1.581948in}}%
\pgfpathcurveto{\pgfqpoint{5.611942in}{1.586992in}}{\pgfqpoint{5.614776in}{1.593834in}}{\pgfqpoint{5.614776in}{1.600966in}}%
\pgfpathcurveto{\pgfqpoint{5.614776in}{1.608099in}}{\pgfqpoint{5.611942in}{1.614941in}}{\pgfqpoint{5.606898in}{1.619985in}}%
\pgfpathcurveto{\pgfqpoint{5.601855in}{1.625028in}}{\pgfqpoint{5.595013in}{1.627862in}}{\pgfqpoint{5.587880in}{1.627862in}}%
\pgfpathcurveto{\pgfqpoint{5.580748in}{1.627862in}}{\pgfqpoint{5.573906in}{1.625028in}}{\pgfqpoint{5.568862in}{1.619985in}}%
\pgfpathcurveto{\pgfqpoint{5.563819in}{1.614941in}}{\pgfqpoint{5.560985in}{1.608099in}}{\pgfqpoint{5.560985in}{1.600966in}}%
\pgfpathcurveto{\pgfqpoint{5.560985in}{1.593834in}}{\pgfqpoint{5.563819in}{1.586992in}}{\pgfqpoint{5.568862in}{1.581948in}}%
\pgfpathcurveto{\pgfqpoint{5.573906in}{1.576905in}}{\pgfqpoint{5.580748in}{1.574071in}}{\pgfqpoint{5.587880in}{1.574071in}}%
\pgfpathclose%
\pgfusepath{stroke,fill}%
\end{pgfscope}%
\begin{pgfscope}%
\pgfpathrectangle{\pgfqpoint{4.985294in}{0.500000in}}{\pgfqpoint{1.764706in}{1.700000in}}%
\pgfusepath{clip}%
\pgfsetbuttcap%
\pgfsetroundjoin%
\definecolor{currentfill}{rgb}{0.972201,0.839051,0.745789}%
\pgfsetfillcolor{currentfill}%
\pgfsetlinewidth{0.311001pt}%
\definecolor{currentstroke}{rgb}{1.000000,1.000000,1.000000}%
\pgfsetstrokecolor{currentstroke}%
\pgfsetdash{}{0pt}%
\pgfpathmoveto{\pgfqpoint{5.524629in}{1.014818in}}%
\pgfpathcurveto{\pgfqpoint{5.531762in}{1.014818in}}{\pgfqpoint{5.538604in}{1.017652in}}{\pgfqpoint{5.543647in}{1.022695in}}%
\pgfpathcurveto{\pgfqpoint{5.548691in}{1.027739in}}{\pgfqpoint{5.551525in}{1.034581in}}{\pgfqpoint{5.551525in}{1.041713in}}%
\pgfpathcurveto{\pgfqpoint{5.551525in}{1.048846in}}{\pgfqpoint{5.548691in}{1.055688in}}{\pgfqpoint{5.543647in}{1.060732in}}%
\pgfpathcurveto{\pgfqpoint{5.538604in}{1.065775in}}{\pgfqpoint{5.531762in}{1.068609in}}{\pgfqpoint{5.524629in}{1.068609in}}%
\pgfpathcurveto{\pgfqpoint{5.517496in}{1.068609in}}{\pgfqpoint{5.510655in}{1.065775in}}{\pgfqpoint{5.505611in}{1.060732in}}%
\pgfpathcurveto{\pgfqpoint{5.500567in}{1.055688in}}{\pgfqpoint{5.497733in}{1.048846in}}{\pgfqpoint{5.497733in}{1.041713in}}%
\pgfpathcurveto{\pgfqpoint{5.497733in}{1.034581in}}{\pgfqpoint{5.500567in}{1.027739in}}{\pgfqpoint{5.505611in}{1.022695in}}%
\pgfpathcurveto{\pgfqpoint{5.510655in}{1.017652in}}{\pgfqpoint{5.517496in}{1.014818in}}{\pgfqpoint{5.524629in}{1.014818in}}%
\pgfpathclose%
\pgfusepath{stroke,fill}%
\end{pgfscope}%
\begin{pgfscope}%
\pgfpathrectangle{\pgfqpoint{4.985294in}{0.500000in}}{\pgfqpoint{1.764706in}{1.700000in}}%
\pgfusepath{clip}%
\pgfsetbuttcap%
\pgfsetroundjoin%
\definecolor{currentfill}{rgb}{0.972726,0.844889,0.754401}%
\pgfsetfillcolor{currentfill}%
\pgfsetlinewidth{0.311001pt}%
\definecolor{currentstroke}{rgb}{1.000000,1.000000,1.000000}%
\pgfsetstrokecolor{currentstroke}%
\pgfsetdash{}{0pt}%
\pgfpathmoveto{\pgfqpoint{5.368286in}{1.400167in}}%
\pgfpathcurveto{\pgfqpoint{5.375419in}{1.400167in}}{\pgfqpoint{5.382261in}{1.403001in}}{\pgfqpoint{5.387304in}{1.408045in}}%
\pgfpathcurveto{\pgfqpoint{5.392348in}{1.413089in}}{\pgfqpoint{5.395182in}{1.419930in}}{\pgfqpoint{5.395182in}{1.427063in}}%
\pgfpathcurveto{\pgfqpoint{5.395182in}{1.434196in}}{\pgfqpoint{5.392348in}{1.441037in}}{\pgfqpoint{5.387304in}{1.446081in}}%
\pgfpathcurveto{\pgfqpoint{5.382261in}{1.451125in}}{\pgfqpoint{5.375419in}{1.453959in}}{\pgfqpoint{5.368286in}{1.453959in}}%
\pgfpathcurveto{\pgfqpoint{5.361153in}{1.453959in}}{\pgfqpoint{5.354312in}{1.451125in}}{\pgfqpoint{5.349268in}{1.446081in}}%
\pgfpathcurveto{\pgfqpoint{5.344224in}{1.441037in}}{\pgfqpoint{5.341390in}{1.434196in}}{\pgfqpoint{5.341390in}{1.427063in}}%
\pgfpathcurveto{\pgfqpoint{5.341390in}{1.419930in}}{\pgfqpoint{5.344224in}{1.413089in}}{\pgfqpoint{5.349268in}{1.408045in}}%
\pgfpathcurveto{\pgfqpoint{5.354312in}{1.403001in}}{\pgfqpoint{5.361153in}{1.400167in}}{\pgfqpoint{5.368286in}{1.400167in}}%
\pgfpathclose%
\pgfusepath{stroke,fill}%
\end{pgfscope}%
\begin{pgfscope}%
\pgfpathrectangle{\pgfqpoint{4.985294in}{0.500000in}}{\pgfqpoint{1.764706in}{1.700000in}}%
\pgfusepath{clip}%
\pgfsetbuttcap%
\pgfsetroundjoin%
\definecolor{currentfill}{rgb}{0.976287,0.879862,0.805788}%
\pgfsetfillcolor{currentfill}%
\pgfsetlinewidth{0.311001pt}%
\definecolor{currentstroke}{rgb}{1.000000,1.000000,1.000000}%
\pgfsetstrokecolor{currentstroke}%
\pgfsetdash{}{0pt}%
\pgfpathmoveto{\pgfqpoint{6.307526in}{1.549235in}}%
\pgfpathcurveto{\pgfqpoint{6.314659in}{1.549235in}}{\pgfqpoint{6.321500in}{1.552069in}}{\pgfqpoint{6.326544in}{1.557112in}}%
\pgfpathcurveto{\pgfqpoint{6.331588in}{1.562156in}}{\pgfqpoint{6.334422in}{1.568998in}}{\pgfqpoint{6.334422in}{1.576130in}}%
\pgfpathcurveto{\pgfqpoint{6.334422in}{1.583263in}}{\pgfqpoint{6.331588in}{1.590105in}}{\pgfqpoint{6.326544in}{1.595149in}}%
\pgfpathcurveto{\pgfqpoint{6.321500in}{1.600192in}}{\pgfqpoint{6.314659in}{1.603026in}}{\pgfqpoint{6.307526in}{1.603026in}}%
\pgfpathcurveto{\pgfqpoint{6.300393in}{1.603026in}}{\pgfqpoint{6.293552in}{1.600192in}}{\pgfqpoint{6.288508in}{1.595149in}}%
\pgfpathcurveto{\pgfqpoint{6.283464in}{1.590105in}}{\pgfqpoint{6.280630in}{1.583263in}}{\pgfqpoint{6.280630in}{1.576130in}}%
\pgfpathcurveto{\pgfqpoint{6.280630in}{1.568998in}}{\pgfqpoint{6.283464in}{1.562156in}}{\pgfqpoint{6.288508in}{1.557112in}}%
\pgfpathcurveto{\pgfqpoint{6.293552in}{1.552069in}}{\pgfqpoint{6.300393in}{1.549235in}}{\pgfqpoint{6.307526in}{1.549235in}}%
\pgfpathclose%
\pgfusepath{stroke,fill}%
\end{pgfscope}%
\begin{pgfscope}%
\pgfpathrectangle{\pgfqpoint{4.985294in}{0.500000in}}{\pgfqpoint{1.764706in}{1.700000in}}%
\pgfusepath{clip}%
\pgfsetbuttcap%
\pgfsetroundjoin%
\definecolor{currentfill}{rgb}{0.974412,0.862387,0.780156}%
\pgfsetfillcolor{currentfill}%
\pgfsetlinewidth{0.311001pt}%
\definecolor{currentstroke}{rgb}{1.000000,1.000000,1.000000}%
\pgfsetstrokecolor{currentstroke}%
\pgfsetdash{}{0pt}%
\pgfpathmoveto{\pgfqpoint{5.478267in}{1.414626in}}%
\pgfpathcurveto{\pgfqpoint{5.485400in}{1.414626in}}{\pgfqpoint{5.492242in}{1.417460in}}{\pgfqpoint{5.497285in}{1.422504in}}%
\pgfpathcurveto{\pgfqpoint{5.502329in}{1.427547in}}{\pgfqpoint{5.505163in}{1.434389in}}{\pgfqpoint{5.505163in}{1.441522in}}%
\pgfpathcurveto{\pgfqpoint{5.505163in}{1.448655in}}{\pgfqpoint{5.502329in}{1.455496in}}{\pgfqpoint{5.497285in}{1.460540in}}%
\pgfpathcurveto{\pgfqpoint{5.492242in}{1.465584in}}{\pgfqpoint{5.485400in}{1.468418in}}{\pgfqpoint{5.478267in}{1.468418in}}%
\pgfpathcurveto{\pgfqpoint{5.471134in}{1.468418in}}{\pgfqpoint{5.464293in}{1.465584in}}{\pgfqpoint{5.459249in}{1.460540in}}%
\pgfpathcurveto{\pgfqpoint{5.454205in}{1.455496in}}{\pgfqpoint{5.451371in}{1.448655in}}{\pgfqpoint{5.451371in}{1.441522in}}%
\pgfpathcurveto{\pgfqpoint{5.451371in}{1.434389in}}{\pgfqpoint{5.454205in}{1.427547in}}{\pgfqpoint{5.459249in}{1.422504in}}%
\pgfpathcurveto{\pgfqpoint{5.464293in}{1.417460in}}{\pgfqpoint{5.471134in}{1.414626in}}{\pgfqpoint{5.478267in}{1.414626in}}%
\pgfpathclose%
\pgfusepath{stroke,fill}%
\end{pgfscope}%
\begin{pgfscope}%
\pgfpathrectangle{\pgfqpoint{4.985294in}{0.500000in}}{\pgfqpoint{1.764706in}{1.700000in}}%
\pgfusepath{clip}%
\pgfsetbuttcap%
\pgfsetroundjoin%
\definecolor{currentfill}{rgb}{0.964679,0.682838,0.530002}%
\pgfsetfillcolor{currentfill}%
\pgfsetlinewidth{0.311001pt}%
\definecolor{currentstroke}{rgb}{1.000000,1.000000,1.000000}%
\pgfsetstrokecolor{currentstroke}%
\pgfsetdash{}{0pt}%
\pgfpathmoveto{\pgfqpoint{5.450702in}{1.692900in}}%
\pgfpathcurveto{\pgfqpoint{5.457835in}{1.692900in}}{\pgfqpoint{5.464677in}{1.695734in}}{\pgfqpoint{5.469721in}{1.700778in}}%
\pgfpathcurveto{\pgfqpoint{5.474764in}{1.705821in}}{\pgfqpoint{5.477598in}{1.712663in}}{\pgfqpoint{5.477598in}{1.719796in}}%
\pgfpathcurveto{\pgfqpoint{5.477598in}{1.726929in}}{\pgfqpoint{5.474764in}{1.733770in}}{\pgfqpoint{5.469721in}{1.738814in}}%
\pgfpathcurveto{\pgfqpoint{5.464677in}{1.743858in}}{\pgfqpoint{5.457835in}{1.746691in}}{\pgfqpoint{5.450702in}{1.746691in}}%
\pgfpathcurveto{\pgfqpoint{5.443570in}{1.746691in}}{\pgfqpoint{5.436728in}{1.743858in}}{\pgfqpoint{5.431684in}{1.738814in}}%
\pgfpathcurveto{\pgfqpoint{5.426641in}{1.733770in}}{\pgfqpoint{5.423807in}{1.726929in}}{\pgfqpoint{5.423807in}{1.719796in}}%
\pgfpathcurveto{\pgfqpoint{5.423807in}{1.712663in}}{\pgfqpoint{5.426641in}{1.705821in}}{\pgfqpoint{5.431684in}{1.700778in}}%
\pgfpathcurveto{\pgfqpoint{5.436728in}{1.695734in}}{\pgfqpoint{5.443570in}{1.692900in}}{\pgfqpoint{5.450702in}{1.692900in}}%
\pgfpathclose%
\pgfusepath{stroke,fill}%
\end{pgfscope}%
\begin{pgfscope}%
\pgfpathrectangle{\pgfqpoint{4.985294in}{0.500000in}}{\pgfqpoint{1.764706in}{1.700000in}}%
\pgfusepath{clip}%
\pgfsetbuttcap%
\pgfsetroundjoin%
\definecolor{currentfill}{rgb}{0.975018,0.868213,0.788710}%
\pgfsetfillcolor{currentfill}%
\pgfsetlinewidth{0.311001pt}%
\definecolor{currentstroke}{rgb}{1.000000,1.000000,1.000000}%
\pgfsetstrokecolor{currentstroke}%
\pgfsetdash{}{0pt}%
\pgfpathmoveto{\pgfqpoint{6.260138in}{1.489271in}}%
\pgfpathcurveto{\pgfqpoint{6.267271in}{1.489271in}}{\pgfqpoint{6.274112in}{1.492105in}}{\pgfqpoint{6.279156in}{1.497148in}}%
\pgfpathcurveto{\pgfqpoint{6.284200in}{1.502192in}}{\pgfqpoint{6.287034in}{1.509034in}}{\pgfqpoint{6.287034in}{1.516166in}}%
\pgfpathcurveto{\pgfqpoint{6.287034in}{1.523299in}}{\pgfqpoint{6.284200in}{1.530141in}}{\pgfqpoint{6.279156in}{1.535185in}}%
\pgfpathcurveto{\pgfqpoint{6.274112in}{1.540228in}}{\pgfqpoint{6.267271in}{1.543062in}}{\pgfqpoint{6.260138in}{1.543062in}}%
\pgfpathcurveto{\pgfqpoint{6.253005in}{1.543062in}}{\pgfqpoint{6.246163in}{1.540228in}}{\pgfqpoint{6.241120in}{1.535185in}}%
\pgfpathcurveto{\pgfqpoint{6.236076in}{1.530141in}}{\pgfqpoint{6.233242in}{1.523299in}}{\pgfqpoint{6.233242in}{1.516166in}}%
\pgfpathcurveto{\pgfqpoint{6.233242in}{1.509034in}}{\pgfqpoint{6.236076in}{1.502192in}}{\pgfqpoint{6.241120in}{1.497148in}}%
\pgfpathcurveto{\pgfqpoint{6.246163in}{1.492105in}}{\pgfqpoint{6.253005in}{1.489271in}}{\pgfqpoint{6.260138in}{1.489271in}}%
\pgfpathclose%
\pgfusepath{stroke,fill}%
\end{pgfscope}%
\begin{pgfscope}%
\pgfpathrectangle{\pgfqpoint{4.985294in}{0.500000in}}{\pgfqpoint{1.764706in}{1.700000in}}%
\pgfusepath{clip}%
\pgfsetbuttcap%
\pgfsetroundjoin%
\definecolor{currentfill}{rgb}{0.969803,0.809811,0.702523}%
\pgfsetfillcolor{currentfill}%
\pgfsetlinewidth{0.311001pt}%
\definecolor{currentstroke}{rgb}{1.000000,1.000000,1.000000}%
\pgfsetstrokecolor{currentstroke}%
\pgfsetdash{}{0pt}%
\pgfpathmoveto{\pgfqpoint{5.431741in}{1.030185in}}%
\pgfpathcurveto{\pgfqpoint{5.438874in}{1.030185in}}{\pgfqpoint{5.445716in}{1.033019in}}{\pgfqpoint{5.450759in}{1.038063in}}%
\pgfpathcurveto{\pgfqpoint{5.455803in}{1.043107in}}{\pgfqpoint{5.458637in}{1.049948in}}{\pgfqpoint{5.458637in}{1.057081in}}%
\pgfpathcurveto{\pgfqpoint{5.458637in}{1.064214in}}{\pgfqpoint{5.455803in}{1.071056in}}{\pgfqpoint{5.450759in}{1.076099in}}%
\pgfpathcurveto{\pgfqpoint{5.445716in}{1.081143in}}{\pgfqpoint{5.438874in}{1.083977in}}{\pgfqpoint{5.431741in}{1.083977in}}%
\pgfpathcurveto{\pgfqpoint{5.424608in}{1.083977in}}{\pgfqpoint{5.417767in}{1.081143in}}{\pgfqpoint{5.412723in}{1.076099in}}%
\pgfpathcurveto{\pgfqpoint{5.407679in}{1.071056in}}{\pgfqpoint{5.404845in}{1.064214in}}{\pgfqpoint{5.404845in}{1.057081in}}%
\pgfpathcurveto{\pgfqpoint{5.404845in}{1.049948in}}{\pgfqpoint{5.407679in}{1.043107in}}{\pgfqpoint{5.412723in}{1.038063in}}%
\pgfpathcurveto{\pgfqpoint{5.417767in}{1.033019in}}{\pgfqpoint{5.424608in}{1.030185in}}{\pgfqpoint{5.431741in}{1.030185in}}%
\pgfpathclose%
\pgfusepath{stroke,fill}%
\end{pgfscope}%
\begin{pgfscope}%
\pgfpathrectangle{\pgfqpoint{4.985294in}{0.500000in}}{\pgfqpoint{1.764706in}{1.700000in}}%
\pgfusepath{clip}%
\pgfsetbuttcap%
\pgfsetroundjoin%
\definecolor{currentfill}{rgb}{0.973271,0.850724,0.762998}%
\pgfsetfillcolor{currentfill}%
\pgfsetlinewidth{0.311001pt}%
\definecolor{currentstroke}{rgb}{1.000000,1.000000,1.000000}%
\pgfsetstrokecolor{currentstroke}%
\pgfsetdash{}{0pt}%
\pgfpathmoveto{\pgfqpoint{6.328549in}{1.126760in}}%
\pgfpathcurveto{\pgfqpoint{6.335682in}{1.126760in}}{\pgfqpoint{6.342524in}{1.129594in}}{\pgfqpoint{6.347567in}{1.134638in}}%
\pgfpathcurveto{\pgfqpoint{6.352611in}{1.139682in}}{\pgfqpoint{6.355445in}{1.146523in}}{\pgfqpoint{6.355445in}{1.153656in}}%
\pgfpathcurveto{\pgfqpoint{6.355445in}{1.160789in}}{\pgfqpoint{6.352611in}{1.167631in}}{\pgfqpoint{6.347567in}{1.172674in}}%
\pgfpathcurveto{\pgfqpoint{6.342524in}{1.177718in}}{\pgfqpoint{6.335682in}{1.180552in}}{\pgfqpoint{6.328549in}{1.180552in}}%
\pgfpathcurveto{\pgfqpoint{6.321416in}{1.180552in}}{\pgfqpoint{6.314575in}{1.177718in}}{\pgfqpoint{6.309531in}{1.172674in}}%
\pgfpathcurveto{\pgfqpoint{6.304487in}{1.167631in}}{\pgfqpoint{6.301653in}{1.160789in}}{\pgfqpoint{6.301653in}{1.153656in}}%
\pgfpathcurveto{\pgfqpoint{6.301653in}{1.146523in}}{\pgfqpoint{6.304487in}{1.139682in}}{\pgfqpoint{6.309531in}{1.134638in}}%
\pgfpathcurveto{\pgfqpoint{6.314575in}{1.129594in}}{\pgfqpoint{6.321416in}{1.126760in}}{\pgfqpoint{6.328549in}{1.126760in}}%
\pgfpathclose%
\pgfusepath{stroke,fill}%
\end{pgfscope}%
\begin{pgfscope}%
\pgfpathrectangle{\pgfqpoint{4.985294in}{0.500000in}}{\pgfqpoint{1.764706in}{1.700000in}}%
\pgfusepath{clip}%
\pgfsetbuttcap%
\pgfsetroundjoin%
\definecolor{currentfill}{rgb}{0.970718,0.821518,0.719872}%
\pgfsetfillcolor{currentfill}%
\pgfsetlinewidth{0.311001pt}%
\definecolor{currentstroke}{rgb}{1.000000,1.000000,1.000000}%
\pgfsetstrokecolor{currentstroke}%
\pgfsetdash{}{0pt}%
\pgfpathmoveto{\pgfqpoint{6.366626in}{1.432373in}}%
\pgfpathcurveto{\pgfqpoint{6.373758in}{1.432373in}}{\pgfqpoint{6.380600in}{1.435207in}}{\pgfqpoint{6.385644in}{1.440251in}}%
\pgfpathcurveto{\pgfqpoint{6.390687in}{1.445294in}}{\pgfqpoint{6.393521in}{1.452136in}}{\pgfqpoint{6.393521in}{1.459269in}}%
\pgfpathcurveto{\pgfqpoint{6.393521in}{1.466402in}}{\pgfqpoint{6.390687in}{1.473243in}}{\pgfqpoint{6.385644in}{1.478287in}}%
\pgfpathcurveto{\pgfqpoint{6.380600in}{1.483331in}}{\pgfqpoint{6.373758in}{1.486165in}}{\pgfqpoint{6.366626in}{1.486165in}}%
\pgfpathcurveto{\pgfqpoint{6.359493in}{1.486165in}}{\pgfqpoint{6.352651in}{1.483331in}}{\pgfqpoint{6.347607in}{1.478287in}}%
\pgfpathcurveto{\pgfqpoint{6.342564in}{1.473243in}}{\pgfqpoint{6.339730in}{1.466402in}}{\pgfqpoint{6.339730in}{1.459269in}}%
\pgfpathcurveto{\pgfqpoint{6.339730in}{1.452136in}}{\pgfqpoint{6.342564in}{1.445294in}}{\pgfqpoint{6.347607in}{1.440251in}}%
\pgfpathcurveto{\pgfqpoint{6.352651in}{1.435207in}}{\pgfqpoint{6.359493in}{1.432373in}}{\pgfqpoint{6.366626in}{1.432373in}}%
\pgfpathclose%
\pgfusepath{stroke,fill}%
\end{pgfscope}%
\begin{pgfscope}%
\pgfpathrectangle{\pgfqpoint{4.985294in}{0.500000in}}{\pgfqpoint{1.764706in}{1.700000in}}%
\pgfusepath{clip}%
\pgfsetbuttcap%
\pgfsetroundjoin%
\definecolor{currentfill}{rgb}{0.965302,0.713942,0.568499}%
\pgfsetfillcolor{currentfill}%
\pgfsetlinewidth{0.311001pt}%
\definecolor{currentstroke}{rgb}{1.000000,1.000000,1.000000}%
\pgfsetstrokecolor{currentstroke}%
\pgfsetdash{}{0pt}%
\pgfpathmoveto{\pgfqpoint{5.441603in}{1.669888in}}%
\pgfpathcurveto{\pgfqpoint{5.448736in}{1.669888in}}{\pgfqpoint{5.455578in}{1.672722in}}{\pgfqpoint{5.460621in}{1.677765in}}%
\pgfpathcurveto{\pgfqpoint{5.465665in}{1.682809in}}{\pgfqpoint{5.468499in}{1.689651in}}{\pgfqpoint{5.468499in}{1.696783in}}%
\pgfpathcurveto{\pgfqpoint{5.468499in}{1.703916in}}{\pgfqpoint{5.465665in}{1.710758in}}{\pgfqpoint{5.460621in}{1.715802in}}%
\pgfpathcurveto{\pgfqpoint{5.455578in}{1.720845in}}{\pgfqpoint{5.448736in}{1.723679in}}{\pgfqpoint{5.441603in}{1.723679in}}%
\pgfpathcurveto{\pgfqpoint{5.434470in}{1.723679in}}{\pgfqpoint{5.427629in}{1.720845in}}{\pgfqpoint{5.422585in}{1.715802in}}%
\pgfpathcurveto{\pgfqpoint{5.417541in}{1.710758in}}{\pgfqpoint{5.414707in}{1.703916in}}{\pgfqpoint{5.414707in}{1.696783in}}%
\pgfpathcurveto{\pgfqpoint{5.414707in}{1.689651in}}{\pgfqpoint{5.417541in}{1.682809in}}{\pgfqpoint{5.422585in}{1.677765in}}%
\pgfpathcurveto{\pgfqpoint{5.427629in}{1.672722in}}{\pgfqpoint{5.434470in}{1.669888in}}{\pgfqpoint{5.441603in}{1.669888in}}%
\pgfpathclose%
\pgfusepath{stroke,fill}%
\end{pgfscope}%
\begin{pgfscope}%
\pgfpathrectangle{\pgfqpoint{4.985294in}{0.500000in}}{\pgfqpoint{1.764706in}{1.700000in}}%
\pgfusepath{clip}%
\pgfsetbuttcap%
\pgfsetroundjoin%
\definecolor{currentfill}{rgb}{0.958331,0.519463,0.362986}%
\pgfsetfillcolor{currentfill}%
\pgfsetlinewidth{0.311001pt}%
\definecolor{currentstroke}{rgb}{1.000000,1.000000,1.000000}%
\pgfsetstrokecolor{currentstroke}%
\pgfsetdash{}{0pt}%
\pgfpathmoveto{\pgfqpoint{5.632592in}{1.603949in}}%
\pgfpathcurveto{\pgfqpoint{5.639725in}{1.603949in}}{\pgfqpoint{5.646566in}{1.606783in}}{\pgfqpoint{5.651610in}{1.611827in}}%
\pgfpathcurveto{\pgfqpoint{5.656654in}{1.616870in}}{\pgfqpoint{5.659488in}{1.623712in}}{\pgfqpoint{5.659488in}{1.630845in}}%
\pgfpathcurveto{\pgfqpoint{5.659488in}{1.637978in}}{\pgfqpoint{5.656654in}{1.644819in}}{\pgfqpoint{5.651610in}{1.649863in}}%
\pgfpathcurveto{\pgfqpoint{5.646566in}{1.654907in}}{\pgfqpoint{5.639725in}{1.657740in}}{\pgfqpoint{5.632592in}{1.657740in}}%
\pgfpathcurveto{\pgfqpoint{5.625459in}{1.657740in}}{\pgfqpoint{5.618617in}{1.654907in}}{\pgfqpoint{5.613574in}{1.649863in}}%
\pgfpathcurveto{\pgfqpoint{5.608530in}{1.644819in}}{\pgfqpoint{5.605696in}{1.637978in}}{\pgfqpoint{5.605696in}{1.630845in}}%
\pgfpathcurveto{\pgfqpoint{5.605696in}{1.623712in}}{\pgfqpoint{5.608530in}{1.616870in}}{\pgfqpoint{5.613574in}{1.611827in}}%
\pgfpathcurveto{\pgfqpoint{5.618617in}{1.606783in}}{\pgfqpoint{5.625459in}{1.603949in}}{\pgfqpoint{5.632592in}{1.603949in}}%
\pgfpathclose%
\pgfusepath{stroke,fill}%
\end{pgfscope}%
\begin{pgfscope}%
\pgfpathrectangle{\pgfqpoint{4.985294in}{0.500000in}}{\pgfqpoint{1.764706in}{1.700000in}}%
\pgfusepath{clip}%
\pgfsetbuttcap%
\pgfsetroundjoin%
\definecolor{currentfill}{rgb}{0.966328,0.750560,0.616961}%
\pgfsetfillcolor{currentfill}%
\pgfsetlinewidth{0.311001pt}%
\definecolor{currentstroke}{rgb}{1.000000,1.000000,1.000000}%
\pgfsetstrokecolor{currentstroke}%
\pgfsetdash{}{0pt}%
\pgfpathmoveto{\pgfqpoint{5.526125in}{1.167857in}}%
\pgfpathcurveto{\pgfqpoint{5.533258in}{1.167857in}}{\pgfqpoint{5.540099in}{1.170690in}}{\pgfqpoint{5.545143in}{1.175734in}}%
\pgfpathcurveto{\pgfqpoint{5.550187in}{1.180778in}}{\pgfqpoint{5.553020in}{1.187619in}}{\pgfqpoint{5.553020in}{1.194752in}}%
\pgfpathcurveto{\pgfqpoint{5.553020in}{1.201885in}}{\pgfqpoint{5.550187in}{1.208727in}}{\pgfqpoint{5.545143in}{1.213770in}}%
\pgfpathcurveto{\pgfqpoint{5.540099in}{1.218814in}}{\pgfqpoint{5.533258in}{1.221648in}}{\pgfqpoint{5.526125in}{1.221648in}}%
\pgfpathcurveto{\pgfqpoint{5.518992in}{1.221648in}}{\pgfqpoint{5.512150in}{1.218814in}}{\pgfqpoint{5.507107in}{1.213770in}}%
\pgfpathcurveto{\pgfqpoint{5.502063in}{1.208727in}}{\pgfqpoint{5.499229in}{1.201885in}}{\pgfqpoint{5.499229in}{1.194752in}}%
\pgfpathcurveto{\pgfqpoint{5.499229in}{1.187619in}}{\pgfqpoint{5.502063in}{1.180778in}}{\pgfqpoint{5.507107in}{1.175734in}}%
\pgfpathcurveto{\pgfqpoint{5.512150in}{1.170690in}}{\pgfqpoint{5.518992in}{1.167857in}}{\pgfqpoint{5.526125in}{1.167857in}}%
\pgfpathclose%
\pgfusepath{stroke,fill}%
\end{pgfscope}%
\begin{pgfscope}%
\pgfpathrectangle{\pgfqpoint{4.985294in}{0.500000in}}{\pgfqpoint{1.764706in}{1.700000in}}%
\pgfusepath{clip}%
\pgfsetbuttcap%
\pgfsetroundjoin%
\definecolor{currentfill}{rgb}{0.972726,0.844889,0.754401}%
\pgfsetfillcolor{currentfill}%
\pgfsetlinewidth{0.311001pt}%
\definecolor{currentstroke}{rgb}{1.000000,1.000000,1.000000}%
\pgfsetstrokecolor{currentstroke}%
\pgfsetdash{}{0pt}%
\pgfpathmoveto{\pgfqpoint{5.494661in}{1.592438in}}%
\pgfpathcurveto{\pgfqpoint{5.501794in}{1.592438in}}{\pgfqpoint{5.508636in}{1.595272in}}{\pgfqpoint{5.513679in}{1.600315in}}%
\pgfpathcurveto{\pgfqpoint{5.518723in}{1.605359in}}{\pgfqpoint{5.521557in}{1.612201in}}{\pgfqpoint{5.521557in}{1.619334in}}%
\pgfpathcurveto{\pgfqpoint{5.521557in}{1.626466in}}{\pgfqpoint{5.518723in}{1.633308in}}{\pgfqpoint{5.513679in}{1.638352in}}%
\pgfpathcurveto{\pgfqpoint{5.508636in}{1.643395in}}{\pgfqpoint{5.501794in}{1.646229in}}{\pgfqpoint{5.494661in}{1.646229in}}%
\pgfpathcurveto{\pgfqpoint{5.487528in}{1.646229in}}{\pgfqpoint{5.480687in}{1.643395in}}{\pgfqpoint{5.475643in}{1.638352in}}%
\pgfpathcurveto{\pgfqpoint{5.470599in}{1.633308in}}{\pgfqpoint{5.467765in}{1.626466in}}{\pgfqpoint{5.467765in}{1.619334in}}%
\pgfpathcurveto{\pgfqpoint{5.467765in}{1.612201in}}{\pgfqpoint{5.470599in}{1.605359in}}{\pgfqpoint{5.475643in}{1.600315in}}%
\pgfpathcurveto{\pgfqpoint{5.480687in}{1.595272in}}{\pgfqpoint{5.487528in}{1.592438in}}{\pgfqpoint{5.494661in}{1.592438in}}%
\pgfpathclose%
\pgfusepath{stroke,fill}%
\end{pgfscope}%
\begin{pgfscope}%
\pgfpathrectangle{\pgfqpoint{4.985294in}{0.500000in}}{\pgfqpoint{1.764706in}{1.700000in}}%
\pgfusepath{clip}%
\pgfsetbuttcap%
\pgfsetroundjoin%
\definecolor{currentfill}{rgb}{0.952404,0.449449,0.307210}%
\pgfsetfillcolor{currentfill}%
\pgfsetlinewidth{0.311001pt}%
\definecolor{currentstroke}{rgb}{1.000000,1.000000,1.000000}%
\pgfsetstrokecolor{currentstroke}%
\pgfsetdash{}{0pt}%
\pgfpathmoveto{\pgfqpoint{6.307063in}{1.719426in}}%
\pgfpathcurveto{\pgfqpoint{6.314196in}{1.719426in}}{\pgfqpoint{6.321037in}{1.722260in}}{\pgfqpoint{6.326081in}{1.727304in}}%
\pgfpathcurveto{\pgfqpoint{6.331125in}{1.732347in}}{\pgfqpoint{6.333959in}{1.739189in}}{\pgfqpoint{6.333959in}{1.746322in}}%
\pgfpathcurveto{\pgfqpoint{6.333959in}{1.753455in}}{\pgfqpoint{6.331125in}{1.760296in}}{\pgfqpoint{6.326081in}{1.765340in}}%
\pgfpathcurveto{\pgfqpoint{6.321037in}{1.770384in}}{\pgfqpoint{6.314196in}{1.773218in}}{\pgfqpoint{6.307063in}{1.773218in}}%
\pgfpathcurveto{\pgfqpoint{6.299930in}{1.773218in}}{\pgfqpoint{6.293088in}{1.770384in}}{\pgfqpoint{6.288045in}{1.765340in}}%
\pgfpathcurveto{\pgfqpoint{6.283001in}{1.760296in}}{\pgfqpoint{6.280167in}{1.753455in}}{\pgfqpoint{6.280167in}{1.746322in}}%
\pgfpathcurveto{\pgfqpoint{6.280167in}{1.739189in}}{\pgfqpoint{6.283001in}{1.732347in}}{\pgfqpoint{6.288045in}{1.727304in}}%
\pgfpathcurveto{\pgfqpoint{6.293088in}{1.722260in}}{\pgfqpoint{6.299930in}{1.719426in}}{\pgfqpoint{6.307063in}{1.719426in}}%
\pgfpathclose%
\pgfusepath{stroke,fill}%
\end{pgfscope}%
\begin{pgfscope}%
\pgfpathrectangle{\pgfqpoint{4.985294in}{0.500000in}}{\pgfqpoint{1.764706in}{1.700000in}}%
\pgfusepath{clip}%
\pgfsetbuttcap%
\pgfsetroundjoin%
\definecolor{currentfill}{rgb}{0.957344,0.505732,0.351309}%
\pgfsetfillcolor{currentfill}%
\pgfsetlinewidth{0.311001pt}%
\definecolor{currentstroke}{rgb}{1.000000,1.000000,1.000000}%
\pgfsetstrokecolor{currentstroke}%
\pgfsetdash{}{0pt}%
\pgfpathmoveto{\pgfqpoint{5.645674in}{1.037148in}}%
\pgfpathcurveto{\pgfqpoint{5.652807in}{1.037148in}}{\pgfqpoint{5.659648in}{1.039982in}}{\pgfqpoint{5.664692in}{1.045026in}}%
\pgfpathcurveto{\pgfqpoint{5.669736in}{1.050070in}}{\pgfqpoint{5.672570in}{1.056911in}}{\pgfqpoint{5.672570in}{1.064044in}}%
\pgfpathcurveto{\pgfqpoint{5.672570in}{1.071177in}}{\pgfqpoint{5.669736in}{1.078019in}}{\pgfqpoint{5.664692in}{1.083062in}}%
\pgfpathcurveto{\pgfqpoint{5.659648in}{1.088106in}}{\pgfqpoint{5.652807in}{1.090940in}}{\pgfqpoint{5.645674in}{1.090940in}}%
\pgfpathcurveto{\pgfqpoint{5.638541in}{1.090940in}}{\pgfqpoint{5.631699in}{1.088106in}}{\pgfqpoint{5.626656in}{1.083062in}}%
\pgfpathcurveto{\pgfqpoint{5.621612in}{1.078019in}}{\pgfqpoint{5.618778in}{1.071177in}}{\pgfqpoint{5.618778in}{1.064044in}}%
\pgfpathcurveto{\pgfqpoint{5.618778in}{1.056911in}}{\pgfqpoint{5.621612in}{1.050070in}}{\pgfqpoint{5.626656in}{1.045026in}}%
\pgfpathcurveto{\pgfqpoint{5.631699in}{1.039982in}}{\pgfqpoint{5.638541in}{1.037148in}}{\pgfqpoint{5.645674in}{1.037148in}}%
\pgfpathclose%
\pgfusepath{stroke,fill}%
\end{pgfscope}%
\begin{pgfscope}%
\pgfpathrectangle{\pgfqpoint{4.985294in}{0.500000in}}{\pgfqpoint{1.764706in}{1.700000in}}%
\pgfusepath{clip}%
\pgfsetbuttcap%
\pgfsetroundjoin%
\definecolor{currentfill}{rgb}{0.979891,0.908948,0.848279}%
\pgfsetfillcolor{currentfill}%
\pgfsetlinewidth{0.311001pt}%
\definecolor{currentstroke}{rgb}{1.000000,1.000000,1.000000}%
\pgfsetstrokecolor{currentstroke}%
\pgfsetdash{}{0pt}%
\pgfpathmoveto{\pgfqpoint{6.325223in}{1.428131in}}%
\pgfpathcurveto{\pgfqpoint{6.332356in}{1.428131in}}{\pgfqpoint{6.339198in}{1.430965in}}{\pgfqpoint{6.344241in}{1.436009in}}%
\pgfpathcurveto{\pgfqpoint{6.349285in}{1.441052in}}{\pgfqpoint{6.352119in}{1.447894in}}{\pgfqpoint{6.352119in}{1.455027in}}%
\pgfpathcurveto{\pgfqpoint{6.352119in}{1.462160in}}{\pgfqpoint{6.349285in}{1.469001in}}{\pgfqpoint{6.344241in}{1.474045in}}%
\pgfpathcurveto{\pgfqpoint{6.339198in}{1.479089in}}{\pgfqpoint{6.332356in}{1.481923in}}{\pgfqpoint{6.325223in}{1.481923in}}%
\pgfpathcurveto{\pgfqpoint{6.318090in}{1.481923in}}{\pgfqpoint{6.311249in}{1.479089in}}{\pgfqpoint{6.306205in}{1.474045in}}%
\pgfpathcurveto{\pgfqpoint{6.301161in}{1.469001in}}{\pgfqpoint{6.298328in}{1.462160in}}{\pgfqpoint{6.298328in}{1.455027in}}%
\pgfpathcurveto{\pgfqpoint{6.298328in}{1.447894in}}{\pgfqpoint{6.301161in}{1.441052in}}{\pgfqpoint{6.306205in}{1.436009in}}%
\pgfpathcurveto{\pgfqpoint{6.311249in}{1.430965in}}{\pgfqpoint{6.318090in}{1.428131in}}{\pgfqpoint{6.325223in}{1.428131in}}%
\pgfpathclose%
\pgfusepath{stroke,fill}%
\end{pgfscope}%
\begin{pgfscope}%
\pgfpathrectangle{\pgfqpoint{4.985294in}{0.500000in}}{\pgfqpoint{1.764706in}{1.700000in}}%
\pgfusepath{clip}%
\pgfsetbuttcap%
\pgfsetroundjoin%
\definecolor{currentfill}{rgb}{0.967398,0.774513,0.650573}%
\pgfsetfillcolor{currentfill}%
\pgfsetlinewidth{0.311001pt}%
\definecolor{currentstroke}{rgb}{1.000000,1.000000,1.000000}%
\pgfsetstrokecolor{currentstroke}%
\pgfsetdash{}{0pt}%
\pgfpathmoveto{\pgfqpoint{5.557550in}{1.576797in}}%
\pgfpathcurveto{\pgfqpoint{5.564683in}{1.576797in}}{\pgfqpoint{5.571524in}{1.579631in}}{\pgfqpoint{5.576568in}{1.584675in}}%
\pgfpathcurveto{\pgfqpoint{5.581612in}{1.589718in}}{\pgfqpoint{5.584445in}{1.596560in}}{\pgfqpoint{5.584445in}{1.603693in}}%
\pgfpathcurveto{\pgfqpoint{5.584445in}{1.610826in}}{\pgfqpoint{5.581612in}{1.617667in}}{\pgfqpoint{5.576568in}{1.622711in}}%
\pgfpathcurveto{\pgfqpoint{5.571524in}{1.627755in}}{\pgfqpoint{5.564683in}{1.630588in}}{\pgfqpoint{5.557550in}{1.630588in}}%
\pgfpathcurveto{\pgfqpoint{5.550417in}{1.630588in}}{\pgfqpoint{5.543575in}{1.627755in}}{\pgfqpoint{5.538532in}{1.622711in}}%
\pgfpathcurveto{\pgfqpoint{5.533488in}{1.617667in}}{\pgfqpoint{5.530654in}{1.610826in}}{\pgfqpoint{5.530654in}{1.603693in}}%
\pgfpathcurveto{\pgfqpoint{5.530654in}{1.596560in}}{\pgfqpoint{5.533488in}{1.589718in}}{\pgfqpoint{5.538532in}{1.584675in}}%
\pgfpathcurveto{\pgfqpoint{5.543575in}{1.579631in}}{\pgfqpoint{5.550417in}{1.576797in}}{\pgfqpoint{5.557550in}{1.576797in}}%
\pgfpathclose%
\pgfusepath{stroke,fill}%
\end{pgfscope}%
\begin{pgfscope}%
\pgfpathrectangle{\pgfqpoint{4.985294in}{0.500000in}}{\pgfqpoint{1.764706in}{1.700000in}}%
\pgfusepath{clip}%
\pgfsetbuttcap%
\pgfsetroundjoin%
\definecolor{currentfill}{rgb}{0.972726,0.844889,0.754401}%
\pgfsetfillcolor{currentfill}%
\pgfsetlinewidth{0.311001pt}%
\definecolor{currentstroke}{rgb}{1.000000,1.000000,1.000000}%
\pgfsetstrokecolor{currentstroke}%
\pgfsetdash{}{0pt}%
\pgfpathmoveto{\pgfqpoint{6.299996in}{1.081322in}}%
\pgfpathcurveto{\pgfqpoint{6.307129in}{1.081322in}}{\pgfqpoint{6.313970in}{1.084156in}}{\pgfqpoint{6.319014in}{1.089200in}}%
\pgfpathcurveto{\pgfqpoint{6.324058in}{1.094244in}}{\pgfqpoint{6.326892in}{1.101085in}}{\pgfqpoint{6.326892in}{1.108218in}}%
\pgfpathcurveto{\pgfqpoint{6.326892in}{1.115351in}}{\pgfqpoint{6.324058in}{1.122193in}}{\pgfqpoint{6.319014in}{1.127236in}}%
\pgfpathcurveto{\pgfqpoint{6.313970in}{1.132280in}}{\pgfqpoint{6.307129in}{1.135114in}}{\pgfqpoint{6.299996in}{1.135114in}}%
\pgfpathcurveto{\pgfqpoint{6.292863in}{1.135114in}}{\pgfqpoint{6.286022in}{1.132280in}}{\pgfqpoint{6.280978in}{1.127236in}}%
\pgfpathcurveto{\pgfqpoint{6.275934in}{1.122193in}}{\pgfqpoint{6.273100in}{1.115351in}}{\pgfqpoint{6.273100in}{1.108218in}}%
\pgfpathcurveto{\pgfqpoint{6.273100in}{1.101085in}}{\pgfqpoint{6.275934in}{1.094244in}}{\pgfqpoint{6.280978in}{1.089200in}}%
\pgfpathcurveto{\pgfqpoint{6.286022in}{1.084156in}}{\pgfqpoint{6.292863in}{1.081322in}}{\pgfqpoint{6.299996in}{1.081322in}}%
\pgfpathclose%
\pgfusepath{stroke,fill}%
\end{pgfscope}%
\begin{pgfscope}%
\pgfpathrectangle{\pgfqpoint{4.985294in}{0.500000in}}{\pgfqpoint{1.764706in}{1.700000in}}%
\pgfusepath{clip}%
\pgfsetbuttcap%
\pgfsetroundjoin%
\definecolor{currentfill}{rgb}{0.977657,0.891500,0.822809}%
\pgfsetfillcolor{currentfill}%
\pgfsetlinewidth{0.311001pt}%
\definecolor{currentstroke}{rgb}{1.000000,1.000000,1.000000}%
\pgfsetstrokecolor{currentstroke}%
\pgfsetdash{}{0pt}%
\pgfpathmoveto{\pgfqpoint{5.453330in}{1.262614in}}%
\pgfpathcurveto{\pgfqpoint{5.460463in}{1.262614in}}{\pgfqpoint{5.467305in}{1.265448in}}{\pgfqpoint{5.472349in}{1.270491in}}%
\pgfpathcurveto{\pgfqpoint{5.477392in}{1.275535in}}{\pgfqpoint{5.480226in}{1.282377in}}{\pgfqpoint{5.480226in}{1.289510in}}%
\pgfpathcurveto{\pgfqpoint{5.480226in}{1.296642in}}{\pgfqpoint{5.477392in}{1.303484in}}{\pgfqpoint{5.472349in}{1.308528in}}%
\pgfpathcurveto{\pgfqpoint{5.467305in}{1.313571in}}{\pgfqpoint{5.460463in}{1.316405in}}{\pgfqpoint{5.453330in}{1.316405in}}%
\pgfpathcurveto{\pgfqpoint{5.446198in}{1.316405in}}{\pgfqpoint{5.439356in}{1.313571in}}{\pgfqpoint{5.434312in}{1.308528in}}%
\pgfpathcurveto{\pgfqpoint{5.429269in}{1.303484in}}{\pgfqpoint{5.426435in}{1.296642in}}{\pgfqpoint{5.426435in}{1.289510in}}%
\pgfpathcurveto{\pgfqpoint{5.426435in}{1.282377in}}{\pgfqpoint{5.429269in}{1.275535in}}{\pgfqpoint{5.434312in}{1.270491in}}%
\pgfpathcurveto{\pgfqpoint{5.439356in}{1.265448in}}{\pgfqpoint{5.446198in}{1.262614in}}{\pgfqpoint{5.453330in}{1.262614in}}%
\pgfpathclose%
\pgfusepath{stroke,fill}%
\end{pgfscope}%
\begin{pgfscope}%
\pgfpathrectangle{\pgfqpoint{4.985294in}{0.500000in}}{\pgfqpoint{1.764706in}{1.700000in}}%
\pgfusepath{clip}%
\pgfsetbuttcap%
\pgfsetroundjoin%
\definecolor{currentfill}{rgb}{0.976287,0.879862,0.805788}%
\pgfsetfillcolor{currentfill}%
\pgfsetlinewidth{0.311001pt}%
\definecolor{currentstroke}{rgb}{1.000000,1.000000,1.000000}%
\pgfsetstrokecolor{currentstroke}%
\pgfsetdash{}{0pt}%
\pgfpathmoveto{\pgfqpoint{6.351939in}{1.391739in}}%
\pgfpathcurveto{\pgfqpoint{6.359072in}{1.391739in}}{\pgfqpoint{6.365914in}{1.394573in}}{\pgfqpoint{6.370957in}{1.399616in}}%
\pgfpathcurveto{\pgfqpoint{6.376001in}{1.404660in}}{\pgfqpoint{6.378835in}{1.411502in}}{\pgfqpoint{6.378835in}{1.418634in}}%
\pgfpathcurveto{\pgfqpoint{6.378835in}{1.425767in}}{\pgfqpoint{6.376001in}{1.432609in}}{\pgfqpoint{6.370957in}{1.437653in}}%
\pgfpathcurveto{\pgfqpoint{6.365914in}{1.442696in}}{\pgfqpoint{6.359072in}{1.445530in}}{\pgfqpoint{6.351939in}{1.445530in}}%
\pgfpathcurveto{\pgfqpoint{6.344806in}{1.445530in}}{\pgfqpoint{6.337965in}{1.442696in}}{\pgfqpoint{6.332921in}{1.437653in}}%
\pgfpathcurveto{\pgfqpoint{6.327877in}{1.432609in}}{\pgfqpoint{6.325044in}{1.425767in}}{\pgfqpoint{6.325044in}{1.418634in}}%
\pgfpathcurveto{\pgfqpoint{6.325044in}{1.411502in}}{\pgfqpoint{6.327877in}{1.404660in}}{\pgfqpoint{6.332921in}{1.399616in}}%
\pgfpathcurveto{\pgfqpoint{6.337965in}{1.394573in}}{\pgfqpoint{6.344806in}{1.391739in}}{\pgfqpoint{6.351939in}{1.391739in}}%
\pgfpathclose%
\pgfusepath{stroke,fill}%
\end{pgfscope}%
\begin{pgfscope}%
\pgfpathrectangle{\pgfqpoint{4.985294in}{0.500000in}}{\pgfqpoint{1.764706in}{1.700000in}}%
\pgfusepath{clip}%
\pgfsetbuttcap%
\pgfsetroundjoin%
\definecolor{currentfill}{rgb}{0.870791,0.179821,0.267974}%
\pgfsetfillcolor{currentfill}%
\pgfsetlinewidth{0.311001pt}%
\definecolor{currentstroke}{rgb}{1.000000,1.000000,1.000000}%
\pgfsetstrokecolor{currentstroke}%
\pgfsetdash{}{0pt}%
\pgfpathmoveto{\pgfqpoint{5.243414in}{1.338186in}}%
\pgfpathcurveto{\pgfqpoint{5.250547in}{1.338186in}}{\pgfqpoint{5.257389in}{1.341020in}}{\pgfqpoint{5.262433in}{1.346063in}}%
\pgfpathcurveto{\pgfqpoint{5.267476in}{1.351107in}}{\pgfqpoint{5.270310in}{1.357949in}}{\pgfqpoint{5.270310in}{1.365081in}}%
\pgfpathcurveto{\pgfqpoint{5.270310in}{1.372214in}}{\pgfqpoint{5.267476in}{1.379056in}}{\pgfqpoint{5.262433in}{1.384100in}}%
\pgfpathcurveto{\pgfqpoint{5.257389in}{1.389143in}}{\pgfqpoint{5.250547in}{1.391977in}}{\pgfqpoint{5.243414in}{1.391977in}}%
\pgfpathcurveto{\pgfqpoint{5.236282in}{1.391977in}}{\pgfqpoint{5.229440in}{1.389143in}}{\pgfqpoint{5.224396in}{1.384100in}}%
\pgfpathcurveto{\pgfqpoint{5.219353in}{1.379056in}}{\pgfqpoint{5.216519in}{1.372214in}}{\pgfqpoint{5.216519in}{1.365081in}}%
\pgfpathcurveto{\pgfqpoint{5.216519in}{1.357949in}}{\pgfqpoint{5.219353in}{1.351107in}}{\pgfqpoint{5.224396in}{1.346063in}}%
\pgfpathcurveto{\pgfqpoint{5.229440in}{1.341020in}}{\pgfqpoint{5.236282in}{1.338186in}}{\pgfqpoint{5.243414in}{1.338186in}}%
\pgfpathclose%
\pgfusepath{stroke,fill}%
\end{pgfscope}%
\begin{pgfscope}%
\pgfpathrectangle{\pgfqpoint{4.985294in}{0.500000in}}{\pgfqpoint{1.764706in}{1.700000in}}%
\pgfusepath{clip}%
\pgfsetbuttcap%
\pgfsetroundjoin%
\definecolor{currentfill}{rgb}{0.981377,0.920617,0.865369}%
\pgfsetfillcolor{currentfill}%
\pgfsetlinewidth{0.311001pt}%
\definecolor{currentstroke}{rgb}{1.000000,1.000000,1.000000}%
\pgfsetstrokecolor{currentstroke}%
\pgfsetdash{}{0pt}%
\pgfpathmoveto{\pgfqpoint{6.303511in}{1.484196in}}%
\pgfpathcurveto{\pgfqpoint{6.310644in}{1.484196in}}{\pgfqpoint{6.317486in}{1.487030in}}{\pgfqpoint{6.322530in}{1.492073in}}%
\pgfpathcurveto{\pgfqpoint{6.327573in}{1.497117in}}{\pgfqpoint{6.330407in}{1.503959in}}{\pgfqpoint{6.330407in}{1.511092in}}%
\pgfpathcurveto{\pgfqpoint{6.330407in}{1.518224in}}{\pgfqpoint{6.327573in}{1.525066in}}{\pgfqpoint{6.322530in}{1.530110in}}%
\pgfpathcurveto{\pgfqpoint{6.317486in}{1.535153in}}{\pgfqpoint{6.310644in}{1.537987in}}{\pgfqpoint{6.303511in}{1.537987in}}%
\pgfpathcurveto{\pgfqpoint{6.296379in}{1.537987in}}{\pgfqpoint{6.289537in}{1.535153in}}{\pgfqpoint{6.284493in}{1.530110in}}%
\pgfpathcurveto{\pgfqpoint{6.279450in}{1.525066in}}{\pgfqpoint{6.276616in}{1.518224in}}{\pgfqpoint{6.276616in}{1.511092in}}%
\pgfpathcurveto{\pgfqpoint{6.276616in}{1.503959in}}{\pgfqpoint{6.279450in}{1.497117in}}{\pgfqpoint{6.284493in}{1.492073in}}%
\pgfpathcurveto{\pgfqpoint{6.289537in}{1.487030in}}{\pgfqpoint{6.296379in}{1.484196in}}{\pgfqpoint{6.303511in}{1.484196in}}%
\pgfpathclose%
\pgfusepath{stroke,fill}%
\end{pgfscope}%
\begin{pgfscope}%
\pgfpathrectangle{\pgfqpoint{4.985294in}{0.500000in}}{\pgfqpoint{1.764706in}{1.700000in}}%
\pgfusepath{clip}%
\pgfsetbuttcap%
\pgfsetroundjoin%
\definecolor{currentfill}{rgb}{0.962765,0.606121,0.444717}%
\pgfsetfillcolor{currentfill}%
\pgfsetlinewidth{0.311001pt}%
\definecolor{currentstroke}{rgb}{1.000000,1.000000,1.000000}%
\pgfsetstrokecolor{currentstroke}%
\pgfsetdash{}{0pt}%
\pgfpathmoveto{\pgfqpoint{5.544450in}{1.269128in}}%
\pgfpathcurveto{\pgfqpoint{5.551583in}{1.269128in}}{\pgfqpoint{5.558425in}{1.271962in}}{\pgfqpoint{5.563468in}{1.277006in}}%
\pgfpathcurveto{\pgfqpoint{5.568512in}{1.282049in}}{\pgfqpoint{5.571346in}{1.288891in}}{\pgfqpoint{5.571346in}{1.296024in}}%
\pgfpathcurveto{\pgfqpoint{5.571346in}{1.303157in}}{\pgfqpoint{5.568512in}{1.309998in}}{\pgfqpoint{5.563468in}{1.315042in}}%
\pgfpathcurveto{\pgfqpoint{5.558425in}{1.320086in}}{\pgfqpoint{5.551583in}{1.322920in}}{\pgfqpoint{5.544450in}{1.322920in}}%
\pgfpathcurveto{\pgfqpoint{5.537317in}{1.322920in}}{\pgfqpoint{5.530476in}{1.320086in}}{\pgfqpoint{5.525432in}{1.315042in}}%
\pgfpathcurveto{\pgfqpoint{5.520388in}{1.309998in}}{\pgfqpoint{5.517554in}{1.303157in}}{\pgfqpoint{5.517554in}{1.296024in}}%
\pgfpathcurveto{\pgfqpoint{5.517554in}{1.288891in}}{\pgfqpoint{5.520388in}{1.282049in}}{\pgfqpoint{5.525432in}{1.277006in}}%
\pgfpathcurveto{\pgfqpoint{5.530476in}{1.271962in}}{\pgfqpoint{5.537317in}{1.269128in}}{\pgfqpoint{5.544450in}{1.269128in}}%
\pgfpathclose%
\pgfusepath{stroke,fill}%
\end{pgfscope}%
\begin{pgfscope}%
\pgfpathrectangle{\pgfqpoint{4.985294in}{0.500000in}}{\pgfqpoint{1.764706in}{1.700000in}}%
\pgfusepath{clip}%
\pgfsetbuttcap%
\pgfsetroundjoin%
\definecolor{currentfill}{rgb}{0.975644,0.874038,0.797253}%
\pgfsetfillcolor{currentfill}%
\pgfsetlinewidth{0.311001pt}%
\definecolor{currentstroke}{rgb}{1.000000,1.000000,1.000000}%
\pgfsetstrokecolor{currentstroke}%
\pgfsetdash{}{0pt}%
\pgfpathmoveto{\pgfqpoint{6.272339in}{1.406277in}}%
\pgfpathcurveto{\pgfqpoint{6.279472in}{1.406277in}}{\pgfqpoint{6.286313in}{1.409111in}}{\pgfqpoint{6.291357in}{1.414154in}}%
\pgfpathcurveto{\pgfqpoint{6.296400in}{1.419198in}}{\pgfqpoint{6.299234in}{1.426040in}}{\pgfqpoint{6.299234in}{1.433173in}}%
\pgfpathcurveto{\pgfqpoint{6.299234in}{1.440305in}}{\pgfqpoint{6.296400in}{1.447147in}}{\pgfqpoint{6.291357in}{1.452191in}}%
\pgfpathcurveto{\pgfqpoint{6.286313in}{1.457234in}}{\pgfqpoint{6.279472in}{1.460068in}}{\pgfqpoint{6.272339in}{1.460068in}}%
\pgfpathcurveto{\pgfqpoint{6.265206in}{1.460068in}}{\pgfqpoint{6.258364in}{1.457234in}}{\pgfqpoint{6.253321in}{1.452191in}}%
\pgfpathcurveto{\pgfqpoint{6.248277in}{1.447147in}}{\pgfqpoint{6.245443in}{1.440305in}}{\pgfqpoint{6.245443in}{1.433173in}}%
\pgfpathcurveto{\pgfqpoint{6.245443in}{1.426040in}}{\pgfqpoint{6.248277in}{1.419198in}}{\pgfqpoint{6.253321in}{1.414154in}}%
\pgfpathcurveto{\pgfqpoint{6.258364in}{1.409111in}}{\pgfqpoint{6.265206in}{1.406277in}}{\pgfqpoint{6.272339in}{1.406277in}}%
\pgfpathclose%
\pgfusepath{stroke,fill}%
\end{pgfscope}%
\begin{pgfscope}%
\pgfpathrectangle{\pgfqpoint{4.985294in}{0.500000in}}{\pgfqpoint{1.764706in}{1.700000in}}%
\pgfusepath{clip}%
\pgfsetbuttcap%
\pgfsetroundjoin%
\definecolor{currentfill}{rgb}{0.963728,0.638439,0.479050}%
\pgfsetfillcolor{currentfill}%
\pgfsetlinewidth{0.311001pt}%
\definecolor{currentstroke}{rgb}{1.000000,1.000000,1.000000}%
\pgfsetstrokecolor{currentstroke}%
\pgfsetdash{}{0pt}%
\pgfpathmoveto{\pgfqpoint{5.604962in}{1.053128in}}%
\pgfpathcurveto{\pgfqpoint{5.612095in}{1.053128in}}{\pgfqpoint{5.618937in}{1.055962in}}{\pgfqpoint{5.623981in}{1.061006in}}%
\pgfpathcurveto{\pgfqpoint{5.629024in}{1.066049in}}{\pgfqpoint{5.631858in}{1.072891in}}{\pgfqpoint{5.631858in}{1.080024in}}%
\pgfpathcurveto{\pgfqpoint{5.631858in}{1.087157in}}{\pgfqpoint{5.629024in}{1.093998in}}{\pgfqpoint{5.623981in}{1.099042in}}%
\pgfpathcurveto{\pgfqpoint{5.618937in}{1.104085in}}{\pgfqpoint{5.612095in}{1.106919in}}{\pgfqpoint{5.604962in}{1.106919in}}%
\pgfpathcurveto{\pgfqpoint{5.597830in}{1.106919in}}{\pgfqpoint{5.590988in}{1.104085in}}{\pgfqpoint{5.585944in}{1.099042in}}%
\pgfpathcurveto{\pgfqpoint{5.580901in}{1.093998in}}{\pgfqpoint{5.578067in}{1.087157in}}{\pgfqpoint{5.578067in}{1.080024in}}%
\pgfpathcurveto{\pgfqpoint{5.578067in}{1.072891in}}{\pgfqpoint{5.580901in}{1.066049in}}{\pgfqpoint{5.585944in}{1.061006in}}%
\pgfpathcurveto{\pgfqpoint{5.590988in}{1.055962in}}{\pgfqpoint{5.597830in}{1.053128in}}{\pgfqpoint{5.604962in}{1.053128in}}%
\pgfpathclose%
\pgfusepath{stroke,fill}%
\end{pgfscope}%
\begin{pgfscope}%
\pgfpathrectangle{\pgfqpoint{4.985294in}{0.500000in}}{\pgfqpoint{1.764706in}{1.700000in}}%
\pgfusepath{clip}%
\pgfsetbuttcap%
\pgfsetroundjoin%
\definecolor{currentfill}{rgb}{0.975644,0.874038,0.797253}%
\pgfsetfillcolor{currentfill}%
\pgfsetlinewidth{0.311001pt}%
\definecolor{currentstroke}{rgb}{1.000000,1.000000,1.000000}%
\pgfsetstrokecolor{currentstroke}%
\pgfsetdash{}{0pt}%
\pgfpathmoveto{\pgfqpoint{5.475909in}{1.520705in}}%
\pgfpathcurveto{\pgfqpoint{5.483042in}{1.520705in}}{\pgfqpoint{5.489883in}{1.523539in}}{\pgfqpoint{5.494927in}{1.528582in}}%
\pgfpathcurveto{\pgfqpoint{5.499971in}{1.533626in}}{\pgfqpoint{5.502804in}{1.540468in}}{\pgfqpoint{5.502804in}{1.547600in}}%
\pgfpathcurveto{\pgfqpoint{5.502804in}{1.554733in}}{\pgfqpoint{5.499971in}{1.561575in}}{\pgfqpoint{5.494927in}{1.566619in}}%
\pgfpathcurveto{\pgfqpoint{5.489883in}{1.571662in}}{\pgfqpoint{5.483042in}{1.574496in}}{\pgfqpoint{5.475909in}{1.574496in}}%
\pgfpathcurveto{\pgfqpoint{5.468776in}{1.574496in}}{\pgfqpoint{5.461934in}{1.571662in}}{\pgfqpoint{5.456891in}{1.566619in}}%
\pgfpathcurveto{\pgfqpoint{5.451847in}{1.561575in}}{\pgfqpoint{5.449013in}{1.554733in}}{\pgfqpoint{5.449013in}{1.547600in}}%
\pgfpathcurveto{\pgfqpoint{5.449013in}{1.540468in}}{\pgfqpoint{5.451847in}{1.533626in}}{\pgfqpoint{5.456891in}{1.528582in}}%
\pgfpathcurveto{\pgfqpoint{5.461934in}{1.523539in}}{\pgfqpoint{5.468776in}{1.520705in}}{\pgfqpoint{5.475909in}{1.520705in}}%
\pgfpathclose%
\pgfusepath{stroke,fill}%
\end{pgfscope}%
\begin{pgfscope}%
\pgfpathrectangle{\pgfqpoint{4.985294in}{0.500000in}}{\pgfqpoint{1.764706in}{1.700000in}}%
\pgfusepath{clip}%
\pgfsetbuttcap%
\pgfsetroundjoin%
\definecolor{currentfill}{rgb}{0.970718,0.821518,0.719872}%
\pgfsetfillcolor{currentfill}%
\pgfsetlinewidth{0.311001pt}%
\definecolor{currentstroke}{rgb}{1.000000,1.000000,1.000000}%
\pgfsetstrokecolor{currentstroke}%
\pgfsetdash{}{0pt}%
\pgfpathmoveto{\pgfqpoint{6.243434in}{1.334327in}}%
\pgfpathcurveto{\pgfqpoint{6.250567in}{1.334327in}}{\pgfqpoint{6.257409in}{1.337161in}}{\pgfqpoint{6.262452in}{1.342205in}}%
\pgfpathcurveto{\pgfqpoint{6.267496in}{1.347249in}}{\pgfqpoint{6.270330in}{1.354090in}}{\pgfqpoint{6.270330in}{1.361223in}}%
\pgfpathcurveto{\pgfqpoint{6.270330in}{1.368356in}}{\pgfqpoint{6.267496in}{1.375197in}}{\pgfqpoint{6.262452in}{1.380241in}}%
\pgfpathcurveto{\pgfqpoint{6.257409in}{1.385285in}}{\pgfqpoint{6.250567in}{1.388119in}}{\pgfqpoint{6.243434in}{1.388119in}}%
\pgfpathcurveto{\pgfqpoint{6.236301in}{1.388119in}}{\pgfqpoint{6.229460in}{1.385285in}}{\pgfqpoint{6.224416in}{1.380241in}}%
\pgfpathcurveto{\pgfqpoint{6.219372in}{1.375197in}}{\pgfqpoint{6.216538in}{1.368356in}}{\pgfqpoint{6.216538in}{1.361223in}}%
\pgfpathcurveto{\pgfqpoint{6.216538in}{1.354090in}}{\pgfqpoint{6.219372in}{1.347249in}}{\pgfqpoint{6.224416in}{1.342205in}}%
\pgfpathcurveto{\pgfqpoint{6.229460in}{1.337161in}}{\pgfqpoint{6.236301in}{1.334327in}}{\pgfqpoint{6.243434in}{1.334327in}}%
\pgfpathclose%
\pgfusepath{stroke,fill}%
\end{pgfscope}%
\begin{pgfscope}%
\pgfpathrectangle{\pgfqpoint{4.985294in}{0.500000in}}{\pgfqpoint{1.764706in}{1.700000in}}%
\pgfusepath{clip}%
\pgfsetbuttcap%
\pgfsetroundjoin%
\definecolor{currentfill}{rgb}{0.972726,0.844889,0.754401}%
\pgfsetfillcolor{currentfill}%
\pgfsetlinewidth{0.311001pt}%
\definecolor{currentstroke}{rgb}{1.000000,1.000000,1.000000}%
\pgfsetstrokecolor{currentstroke}%
\pgfsetdash{}{0pt}%
\pgfpathmoveto{\pgfqpoint{5.454775in}{1.038171in}}%
\pgfpathcurveto{\pgfqpoint{5.461908in}{1.038171in}}{\pgfqpoint{5.468750in}{1.041005in}}{\pgfqpoint{5.473793in}{1.046049in}}%
\pgfpathcurveto{\pgfqpoint{5.478837in}{1.051092in}}{\pgfqpoint{5.481671in}{1.057934in}}{\pgfqpoint{5.481671in}{1.065067in}}%
\pgfpathcurveto{\pgfqpoint{5.481671in}{1.072200in}}{\pgfqpoint{5.478837in}{1.079041in}}{\pgfqpoint{5.473793in}{1.084085in}}%
\pgfpathcurveto{\pgfqpoint{5.468750in}{1.089129in}}{\pgfqpoint{5.461908in}{1.091963in}}{\pgfqpoint{5.454775in}{1.091963in}}%
\pgfpathcurveto{\pgfqpoint{5.447642in}{1.091963in}}{\pgfqpoint{5.440801in}{1.089129in}}{\pgfqpoint{5.435757in}{1.084085in}}%
\pgfpathcurveto{\pgfqpoint{5.430713in}{1.079041in}}{\pgfqpoint{5.427880in}{1.072200in}}{\pgfqpoint{5.427880in}{1.065067in}}%
\pgfpathcurveto{\pgfqpoint{5.427880in}{1.057934in}}{\pgfqpoint{5.430713in}{1.051092in}}{\pgfqpoint{5.435757in}{1.046049in}}%
\pgfpathcurveto{\pgfqpoint{5.440801in}{1.041005in}}{\pgfqpoint{5.447642in}{1.038171in}}{\pgfqpoint{5.454775in}{1.038171in}}%
\pgfpathclose%
\pgfusepath{stroke,fill}%
\end{pgfscope}%
\begin{pgfscope}%
\pgfpathrectangle{\pgfqpoint{4.985294in}{0.500000in}}{\pgfqpoint{1.764706in}{1.700000in}}%
\pgfusepath{clip}%
\pgfsetbuttcap%
\pgfsetroundjoin%
\definecolor{currentfill}{rgb}{0.979124,0.903132,0.839793}%
\pgfsetfillcolor{currentfill}%
\pgfsetlinewidth{0.311001pt}%
\definecolor{currentstroke}{rgb}{1.000000,1.000000,1.000000}%
\pgfsetstrokecolor{currentstroke}%
\pgfsetdash{}{0pt}%
\pgfpathmoveto{\pgfqpoint{6.285087in}{1.264014in}}%
\pgfpathcurveto{\pgfqpoint{6.292220in}{1.264014in}}{\pgfqpoint{6.299061in}{1.266848in}}{\pgfqpoint{6.304105in}{1.271891in}}%
\pgfpathcurveto{\pgfqpoint{6.309149in}{1.276935in}}{\pgfqpoint{6.311983in}{1.283777in}}{\pgfqpoint{6.311983in}{1.290910in}}%
\pgfpathcurveto{\pgfqpoint{6.311983in}{1.298042in}}{\pgfqpoint{6.309149in}{1.304884in}}{\pgfqpoint{6.304105in}{1.309928in}}%
\pgfpathcurveto{\pgfqpoint{6.299061in}{1.314971in}}{\pgfqpoint{6.292220in}{1.317805in}}{\pgfqpoint{6.285087in}{1.317805in}}%
\pgfpathcurveto{\pgfqpoint{6.277954in}{1.317805in}}{\pgfqpoint{6.271112in}{1.314971in}}{\pgfqpoint{6.266069in}{1.309928in}}%
\pgfpathcurveto{\pgfqpoint{6.261025in}{1.304884in}}{\pgfqpoint{6.258191in}{1.298042in}}{\pgfqpoint{6.258191in}{1.290910in}}%
\pgfpathcurveto{\pgfqpoint{6.258191in}{1.283777in}}{\pgfqpoint{6.261025in}{1.276935in}}{\pgfqpoint{6.266069in}{1.271891in}}%
\pgfpathcurveto{\pgfqpoint{6.271112in}{1.266848in}}{\pgfqpoint{6.277954in}{1.264014in}}{\pgfqpoint{6.285087in}{1.264014in}}%
\pgfpathclose%
\pgfusepath{stroke,fill}%
\end{pgfscope}%
\begin{pgfscope}%
\pgfpathrectangle{\pgfqpoint{4.985294in}{0.500000in}}{\pgfqpoint{1.764706in}{1.700000in}}%
\pgfusepath{clip}%
\pgfsetbuttcap%
\pgfsetroundjoin%
\definecolor{currentfill}{rgb}{0.975644,0.874038,0.797253}%
\pgfsetfillcolor{currentfill}%
\pgfsetlinewidth{0.311001pt}%
\definecolor{currentstroke}{rgb}{1.000000,1.000000,1.000000}%
\pgfsetstrokecolor{currentstroke}%
\pgfsetdash{}{0pt}%
\pgfpathmoveto{\pgfqpoint{6.336438in}{1.472023in}}%
\pgfpathcurveto{\pgfqpoint{6.343571in}{1.472023in}}{\pgfqpoint{6.350412in}{1.474857in}}{\pgfqpoint{6.355456in}{1.479901in}}%
\pgfpathcurveto{\pgfqpoint{6.360500in}{1.484944in}}{\pgfqpoint{6.363334in}{1.491786in}}{\pgfqpoint{6.363334in}{1.498919in}}%
\pgfpathcurveto{\pgfqpoint{6.363334in}{1.506051in}}{\pgfqpoint{6.360500in}{1.512893in}}{\pgfqpoint{6.355456in}{1.517937in}}%
\pgfpathcurveto{\pgfqpoint{6.350412in}{1.522980in}}{\pgfqpoint{6.343571in}{1.525814in}}{\pgfqpoint{6.336438in}{1.525814in}}%
\pgfpathcurveto{\pgfqpoint{6.329305in}{1.525814in}}{\pgfqpoint{6.322463in}{1.522980in}}{\pgfqpoint{6.317420in}{1.517937in}}%
\pgfpathcurveto{\pgfqpoint{6.312376in}{1.512893in}}{\pgfqpoint{6.309542in}{1.506051in}}{\pgfqpoint{6.309542in}{1.498919in}}%
\pgfpathcurveto{\pgfqpoint{6.309542in}{1.491786in}}{\pgfqpoint{6.312376in}{1.484944in}}{\pgfqpoint{6.317420in}{1.479901in}}%
\pgfpathcurveto{\pgfqpoint{6.322463in}{1.474857in}}{\pgfqpoint{6.329305in}{1.472023in}}{\pgfqpoint{6.336438in}{1.472023in}}%
\pgfpathclose%
\pgfusepath{stroke,fill}%
\end{pgfscope}%
\begin{pgfscope}%
\pgfpathrectangle{\pgfqpoint{4.985294in}{0.500000in}}{\pgfqpoint{1.764706in}{1.700000in}}%
\pgfusepath{clip}%
\pgfsetbuttcap%
\pgfsetroundjoin%
\definecolor{currentfill}{rgb}{0.979891,0.908948,0.848279}%
\pgfsetfillcolor{currentfill}%
\pgfsetlinewidth{0.311001pt}%
\definecolor{currentstroke}{rgb}{1.000000,1.000000,1.000000}%
\pgfsetstrokecolor{currentstroke}%
\pgfsetdash{}{0pt}%
\pgfpathmoveto{\pgfqpoint{5.433862in}{1.405819in}}%
\pgfpathcurveto{\pgfqpoint{5.440995in}{1.405819in}}{\pgfqpoint{5.447836in}{1.408653in}}{\pgfqpoint{5.452880in}{1.413696in}}%
\pgfpathcurveto{\pgfqpoint{5.457924in}{1.418740in}}{\pgfqpoint{5.460758in}{1.425582in}}{\pgfqpoint{5.460758in}{1.432714in}}%
\pgfpathcurveto{\pgfqpoint{5.460758in}{1.439847in}}{\pgfqpoint{5.457924in}{1.446689in}}{\pgfqpoint{5.452880in}{1.451732in}}%
\pgfpathcurveto{\pgfqpoint{5.447836in}{1.456776in}}{\pgfqpoint{5.440995in}{1.459610in}}{\pgfqpoint{5.433862in}{1.459610in}}%
\pgfpathcurveto{\pgfqpoint{5.426729in}{1.459610in}}{\pgfqpoint{5.419888in}{1.456776in}}{\pgfqpoint{5.414844in}{1.451732in}}%
\pgfpathcurveto{\pgfqpoint{5.409800in}{1.446689in}}{\pgfqpoint{5.406966in}{1.439847in}}{\pgfqpoint{5.406966in}{1.432714in}}%
\pgfpathcurveto{\pgfqpoint{5.406966in}{1.425582in}}{\pgfqpoint{5.409800in}{1.418740in}}{\pgfqpoint{5.414844in}{1.413696in}}%
\pgfpathcurveto{\pgfqpoint{5.419888in}{1.408653in}}{\pgfqpoint{5.426729in}{1.405819in}}{\pgfqpoint{5.433862in}{1.405819in}}%
\pgfpathclose%
\pgfusepath{stroke,fill}%
\end{pgfscope}%
\begin{pgfscope}%
\pgfpathrectangle{\pgfqpoint{4.985294in}{0.500000in}}{\pgfqpoint{1.764706in}{1.700000in}}%
\pgfusepath{clip}%
\pgfsetbuttcap%
\pgfsetroundjoin%
\definecolor{currentfill}{rgb}{0.973271,0.850724,0.762998}%
\pgfsetfillcolor{currentfill}%
\pgfsetlinewidth{0.311001pt}%
\definecolor{currentstroke}{rgb}{1.000000,1.000000,1.000000}%
\pgfsetstrokecolor{currentstroke}%
\pgfsetdash{}{0pt}%
\pgfpathmoveto{\pgfqpoint{5.511632in}{1.034739in}}%
\pgfpathcurveto{\pgfqpoint{5.518765in}{1.034739in}}{\pgfqpoint{5.525606in}{1.037572in}}{\pgfqpoint{5.530650in}{1.042616in}}%
\pgfpathcurveto{\pgfqpoint{5.535694in}{1.047660in}}{\pgfqpoint{5.538528in}{1.054501in}}{\pgfqpoint{5.538528in}{1.061634in}}%
\pgfpathcurveto{\pgfqpoint{5.538528in}{1.068767in}}{\pgfqpoint{5.535694in}{1.075609in}}{\pgfqpoint{5.530650in}{1.080652in}}%
\pgfpathcurveto{\pgfqpoint{5.525606in}{1.085696in}}{\pgfqpoint{5.518765in}{1.088530in}}{\pgfqpoint{5.511632in}{1.088530in}}%
\pgfpathcurveto{\pgfqpoint{5.504499in}{1.088530in}}{\pgfqpoint{5.497657in}{1.085696in}}{\pgfqpoint{5.492614in}{1.080652in}}%
\pgfpathcurveto{\pgfqpoint{5.487570in}{1.075609in}}{\pgfqpoint{5.484736in}{1.068767in}}{\pgfqpoint{5.484736in}{1.061634in}}%
\pgfpathcurveto{\pgfqpoint{5.484736in}{1.054501in}}{\pgfqpoint{5.487570in}{1.047660in}}{\pgfqpoint{5.492614in}{1.042616in}}%
\pgfpathcurveto{\pgfqpoint{5.497657in}{1.037572in}}{\pgfqpoint{5.504499in}{1.034739in}}{\pgfqpoint{5.511632in}{1.034739in}}%
\pgfpathclose%
\pgfusepath{stroke,fill}%
\end{pgfscope}%
\begin{pgfscope}%
\pgfpathrectangle{\pgfqpoint{4.985294in}{0.500000in}}{\pgfqpoint{1.764706in}{1.700000in}}%
\pgfusepath{clip}%
\pgfsetbuttcap%
\pgfsetroundjoin%
\definecolor{currentfill}{rgb}{0.955103,0.477872,0.328626}%
\pgfsetfillcolor{currentfill}%
\pgfsetlinewidth{0.311001pt}%
\definecolor{currentstroke}{rgb}{1.000000,1.000000,1.000000}%
\pgfsetstrokecolor{currentstroke}%
\pgfsetdash{}{0pt}%
\pgfpathmoveto{\pgfqpoint{5.293189in}{1.189829in}}%
\pgfpathcurveto{\pgfqpoint{5.300322in}{1.189829in}}{\pgfqpoint{5.307164in}{1.192663in}}{\pgfqpoint{5.312207in}{1.197707in}}%
\pgfpathcurveto{\pgfqpoint{5.317251in}{1.202751in}}{\pgfqpoint{5.320085in}{1.209592in}}{\pgfqpoint{5.320085in}{1.216725in}}%
\pgfpathcurveto{\pgfqpoint{5.320085in}{1.223858in}}{\pgfqpoint{5.317251in}{1.230700in}}{\pgfqpoint{5.312207in}{1.235743in}}%
\pgfpathcurveto{\pgfqpoint{5.307164in}{1.240787in}}{\pgfqpoint{5.300322in}{1.243621in}}{\pgfqpoint{5.293189in}{1.243621in}}%
\pgfpathcurveto{\pgfqpoint{5.286056in}{1.243621in}}{\pgfqpoint{5.279215in}{1.240787in}}{\pgfqpoint{5.274171in}{1.235743in}}%
\pgfpathcurveto{\pgfqpoint{5.269127in}{1.230700in}}{\pgfqpoint{5.266294in}{1.223858in}}{\pgfqpoint{5.266294in}{1.216725in}}%
\pgfpathcurveto{\pgfqpoint{5.266294in}{1.209592in}}{\pgfqpoint{5.269127in}{1.202751in}}{\pgfqpoint{5.274171in}{1.197707in}}%
\pgfpathcurveto{\pgfqpoint{5.279215in}{1.192663in}}{\pgfqpoint{5.286056in}{1.189829in}}{\pgfqpoint{5.293189in}{1.189829in}}%
\pgfpathclose%
\pgfusepath{stroke,fill}%
\end{pgfscope}%
\begin{pgfscope}%
\pgfpathrectangle{\pgfqpoint{4.985294in}{0.500000in}}{\pgfqpoint{1.764706in}{1.700000in}}%
\pgfusepath{clip}%
\pgfsetbuttcap%
\pgfsetroundjoin%
\definecolor{currentfill}{rgb}{0.964306,0.663930,0.507747}%
\pgfsetfillcolor{currentfill}%
\pgfsetlinewidth{0.311001pt}%
\definecolor{currentstroke}{rgb}{1.000000,1.000000,1.000000}%
\pgfsetstrokecolor{currentstroke}%
\pgfsetdash{}{0pt}%
\pgfpathmoveto{\pgfqpoint{5.347716in}{1.108933in}}%
\pgfpathcurveto{\pgfqpoint{5.354849in}{1.108933in}}{\pgfqpoint{5.361691in}{1.111767in}}{\pgfqpoint{5.366735in}{1.116810in}}%
\pgfpathcurveto{\pgfqpoint{5.371778in}{1.121854in}}{\pgfqpoint{5.374612in}{1.128696in}}{\pgfqpoint{5.374612in}{1.135828in}}%
\pgfpathcurveto{\pgfqpoint{5.374612in}{1.142961in}}{\pgfqpoint{5.371778in}{1.149803in}}{\pgfqpoint{5.366735in}{1.154847in}}%
\pgfpathcurveto{\pgfqpoint{5.361691in}{1.159890in}}{\pgfqpoint{5.354849in}{1.162724in}}{\pgfqpoint{5.347716in}{1.162724in}}%
\pgfpathcurveto{\pgfqpoint{5.340584in}{1.162724in}}{\pgfqpoint{5.333742in}{1.159890in}}{\pgfqpoint{5.328698in}{1.154847in}}%
\pgfpathcurveto{\pgfqpoint{5.323655in}{1.149803in}}{\pgfqpoint{5.320821in}{1.142961in}}{\pgfqpoint{5.320821in}{1.135828in}}%
\pgfpathcurveto{\pgfqpoint{5.320821in}{1.128696in}}{\pgfqpoint{5.323655in}{1.121854in}}{\pgfqpoint{5.328698in}{1.116810in}}%
\pgfpathcurveto{\pgfqpoint{5.333742in}{1.111767in}}{\pgfqpoint{5.340584in}{1.108933in}}{\pgfqpoint{5.347716in}{1.108933in}}%
\pgfpathclose%
\pgfusepath{stroke,fill}%
\end{pgfscope}%
\begin{pgfscope}%
\pgfpathrectangle{\pgfqpoint{4.985294in}{0.500000in}}{\pgfqpoint{1.764706in}{1.700000in}}%
\pgfusepath{clip}%
\pgfsetbuttcap%
\pgfsetroundjoin%
\definecolor{currentfill}{rgb}{0.979891,0.908948,0.848279}%
\pgfsetfillcolor{currentfill}%
\pgfsetlinewidth{0.311001pt}%
\definecolor{currentstroke}{rgb}{1.000000,1.000000,1.000000}%
\pgfsetstrokecolor{currentstroke}%
\pgfsetdash{}{0pt}%
\pgfpathmoveto{\pgfqpoint{6.286101in}{1.460392in}}%
\pgfpathcurveto{\pgfqpoint{6.293234in}{1.460392in}}{\pgfqpoint{6.300075in}{1.463226in}}{\pgfqpoint{6.305119in}{1.468270in}}%
\pgfpathcurveto{\pgfqpoint{6.310163in}{1.473314in}}{\pgfqpoint{6.312997in}{1.480155in}}{\pgfqpoint{6.312997in}{1.487288in}}%
\pgfpathcurveto{\pgfqpoint{6.312997in}{1.494421in}}{\pgfqpoint{6.310163in}{1.501263in}}{\pgfqpoint{6.305119in}{1.506306in}}%
\pgfpathcurveto{\pgfqpoint{6.300075in}{1.511350in}}{\pgfqpoint{6.293234in}{1.514184in}}{\pgfqpoint{6.286101in}{1.514184in}}%
\pgfpathcurveto{\pgfqpoint{6.278968in}{1.514184in}}{\pgfqpoint{6.272126in}{1.511350in}}{\pgfqpoint{6.267083in}{1.506306in}}%
\pgfpathcurveto{\pgfqpoint{6.262039in}{1.501263in}}{\pgfqpoint{6.259205in}{1.494421in}}{\pgfqpoint{6.259205in}{1.487288in}}%
\pgfpathcurveto{\pgfqpoint{6.259205in}{1.480155in}}{\pgfqpoint{6.262039in}{1.473314in}}{\pgfqpoint{6.267083in}{1.468270in}}%
\pgfpathcurveto{\pgfqpoint{6.272126in}{1.463226in}}{\pgfqpoint{6.278968in}{1.460392in}}{\pgfqpoint{6.286101in}{1.460392in}}%
\pgfpathclose%
\pgfusepath{stroke,fill}%
\end{pgfscope}%
\begin{pgfscope}%
\pgfpathrectangle{\pgfqpoint{4.985294in}{0.500000in}}{\pgfqpoint{1.764706in}{1.700000in}}%
\pgfusepath{clip}%
\pgfsetbuttcap%
\pgfsetroundjoin%
\definecolor{currentfill}{rgb}{0.964799,0.689101,0.537560}%
\pgfsetfillcolor{currentfill}%
\pgfsetlinewidth{0.311001pt}%
\definecolor{currentstroke}{rgb}{1.000000,1.000000,1.000000}%
\pgfsetstrokecolor{currentstroke}%
\pgfsetdash{}{0pt}%
\pgfpathmoveto{\pgfqpoint{6.393908in}{1.466156in}}%
\pgfpathcurveto{\pgfqpoint{6.401041in}{1.466156in}}{\pgfqpoint{6.407883in}{1.468990in}}{\pgfqpoint{6.412927in}{1.474034in}}%
\pgfpathcurveto{\pgfqpoint{6.417970in}{1.479077in}}{\pgfqpoint{6.420804in}{1.485919in}}{\pgfqpoint{6.420804in}{1.493052in}}%
\pgfpathcurveto{\pgfqpoint{6.420804in}{1.500185in}}{\pgfqpoint{6.417970in}{1.507026in}}{\pgfqpoint{6.412927in}{1.512070in}}%
\pgfpathcurveto{\pgfqpoint{6.407883in}{1.517113in}}{\pgfqpoint{6.401041in}{1.519947in}}{\pgfqpoint{6.393908in}{1.519947in}}%
\pgfpathcurveto{\pgfqpoint{6.386776in}{1.519947in}}{\pgfqpoint{6.379934in}{1.517113in}}{\pgfqpoint{6.374890in}{1.512070in}}%
\pgfpathcurveto{\pgfqpoint{6.369847in}{1.507026in}}{\pgfqpoint{6.367013in}{1.500185in}}{\pgfqpoint{6.367013in}{1.493052in}}%
\pgfpathcurveto{\pgfqpoint{6.367013in}{1.485919in}}{\pgfqpoint{6.369847in}{1.479077in}}{\pgfqpoint{6.374890in}{1.474034in}}%
\pgfpathcurveto{\pgfqpoint{6.379934in}{1.468990in}}{\pgfqpoint{6.386776in}{1.466156in}}{\pgfqpoint{6.393908in}{1.466156in}}%
\pgfpathclose%
\pgfusepath{stroke,fill}%
\end{pgfscope}%
\begin{pgfscope}%
\pgfpathrectangle{\pgfqpoint{4.985294in}{0.500000in}}{\pgfqpoint{1.764706in}{1.700000in}}%
\pgfusepath{clip}%
\pgfsetbuttcap%
\pgfsetroundjoin%
\definecolor{currentfill}{rgb}{0.963884,0.644842,0.486120}%
\pgfsetfillcolor{currentfill}%
\pgfsetlinewidth{0.311001pt}%
\definecolor{currentstroke}{rgb}{1.000000,1.000000,1.000000}%
\pgfsetstrokecolor{currentstroke}%
\pgfsetdash{}{0pt}%
\pgfpathmoveto{\pgfqpoint{6.183254in}{1.380067in}}%
\pgfpathcurveto{\pgfqpoint{6.190387in}{1.380067in}}{\pgfqpoint{6.197229in}{1.382901in}}{\pgfqpoint{6.202272in}{1.387944in}}%
\pgfpathcurveto{\pgfqpoint{6.207316in}{1.392988in}}{\pgfqpoint{6.210150in}{1.399830in}}{\pgfqpoint{6.210150in}{1.406962in}}%
\pgfpathcurveto{\pgfqpoint{6.210150in}{1.414095in}}{\pgfqpoint{6.207316in}{1.420937in}}{\pgfqpoint{6.202272in}{1.425980in}}%
\pgfpathcurveto{\pgfqpoint{6.197229in}{1.431024in}}{\pgfqpoint{6.190387in}{1.433858in}}{\pgfqpoint{6.183254in}{1.433858in}}%
\pgfpathcurveto{\pgfqpoint{6.176121in}{1.433858in}}{\pgfqpoint{6.169280in}{1.431024in}}{\pgfqpoint{6.164236in}{1.425980in}}%
\pgfpathcurveto{\pgfqpoint{6.159192in}{1.420937in}}{\pgfqpoint{6.156358in}{1.414095in}}{\pgfqpoint{6.156358in}{1.406962in}}%
\pgfpathcurveto{\pgfqpoint{6.156358in}{1.399830in}}{\pgfqpoint{6.159192in}{1.392988in}}{\pgfqpoint{6.164236in}{1.387944in}}%
\pgfpathcurveto{\pgfqpoint{6.169280in}{1.382901in}}{\pgfqpoint{6.176121in}{1.380067in}}{\pgfqpoint{6.183254in}{1.380067in}}%
\pgfpathclose%
\pgfusepath{stroke,fill}%
\end{pgfscope}%
\begin{pgfscope}%
\pgfpathrectangle{\pgfqpoint{4.985294in}{0.500000in}}{\pgfqpoint{1.764706in}{1.700000in}}%
\pgfusepath{clip}%
\pgfsetbuttcap%
\pgfsetroundjoin%
\definecolor{currentfill}{rgb}{0.968931,0.798091,0.685123}%
\pgfsetfillcolor{currentfill}%
\pgfsetlinewidth{0.311001pt}%
\definecolor{currentstroke}{rgb}{1.000000,1.000000,1.000000}%
\pgfsetstrokecolor{currentstroke}%
\pgfsetdash{}{0pt}%
\pgfpathmoveto{\pgfqpoint{5.360199in}{1.199341in}}%
\pgfpathcurveto{\pgfqpoint{5.367332in}{1.199341in}}{\pgfqpoint{5.374173in}{1.202175in}}{\pgfqpoint{5.379217in}{1.207219in}}%
\pgfpathcurveto{\pgfqpoint{5.384261in}{1.212262in}}{\pgfqpoint{5.387094in}{1.219104in}}{\pgfqpoint{5.387094in}{1.226237in}}%
\pgfpathcurveto{\pgfqpoint{5.387094in}{1.233370in}}{\pgfqpoint{5.384261in}{1.240211in}}{\pgfqpoint{5.379217in}{1.245255in}}%
\pgfpathcurveto{\pgfqpoint{5.374173in}{1.250299in}}{\pgfqpoint{5.367332in}{1.253133in}}{\pgfqpoint{5.360199in}{1.253133in}}%
\pgfpathcurveto{\pgfqpoint{5.353066in}{1.253133in}}{\pgfqpoint{5.346224in}{1.250299in}}{\pgfqpoint{5.341181in}{1.245255in}}%
\pgfpathcurveto{\pgfqpoint{5.336137in}{1.240211in}}{\pgfqpoint{5.333303in}{1.233370in}}{\pgfqpoint{5.333303in}{1.226237in}}%
\pgfpathcurveto{\pgfqpoint{5.333303in}{1.219104in}}{\pgfqpoint{5.336137in}{1.212262in}}{\pgfqpoint{5.341181in}{1.207219in}}%
\pgfpathcurveto{\pgfqpoint{5.346224in}{1.202175in}}{\pgfqpoint{5.353066in}{1.199341in}}{\pgfqpoint{5.360199in}{1.199341in}}%
\pgfpathclose%
\pgfusepath{stroke,fill}%
\end{pgfscope}%
\begin{pgfscope}%
\pgfpathrectangle{\pgfqpoint{4.985294in}{0.500000in}}{\pgfqpoint{1.764706in}{1.700000in}}%
\pgfusepath{clip}%
\pgfsetbuttcap%
\pgfsetroundjoin%
\definecolor{currentfill}{rgb}{0.975644,0.874038,0.797253}%
\pgfsetfillcolor{currentfill}%
\pgfsetlinewidth{0.311001pt}%
\definecolor{currentstroke}{rgb}{1.000000,1.000000,1.000000}%
\pgfsetstrokecolor{currentstroke}%
\pgfsetdash{}{0pt}%
\pgfpathmoveto{\pgfqpoint{6.261063in}{1.080610in}}%
\pgfpathcurveto{\pgfqpoint{6.268196in}{1.080610in}}{\pgfqpoint{6.275038in}{1.083444in}}{\pgfqpoint{6.280082in}{1.088488in}}%
\pgfpathcurveto{\pgfqpoint{6.285125in}{1.093531in}}{\pgfqpoint{6.287959in}{1.100373in}}{\pgfqpoint{6.287959in}{1.107506in}}%
\pgfpathcurveto{\pgfqpoint{6.287959in}{1.114638in}}{\pgfqpoint{6.285125in}{1.121480in}}{\pgfqpoint{6.280082in}{1.126524in}}%
\pgfpathcurveto{\pgfqpoint{6.275038in}{1.131567in}}{\pgfqpoint{6.268196in}{1.134401in}}{\pgfqpoint{6.261063in}{1.134401in}}%
\pgfpathcurveto{\pgfqpoint{6.253931in}{1.134401in}}{\pgfqpoint{6.247089in}{1.131567in}}{\pgfqpoint{6.242045in}{1.126524in}}%
\pgfpathcurveto{\pgfqpoint{6.237002in}{1.121480in}}{\pgfqpoint{6.234168in}{1.114638in}}{\pgfqpoint{6.234168in}{1.107506in}}%
\pgfpathcurveto{\pgfqpoint{6.234168in}{1.100373in}}{\pgfqpoint{6.237002in}{1.093531in}}{\pgfqpoint{6.242045in}{1.088488in}}%
\pgfpathcurveto{\pgfqpoint{6.247089in}{1.083444in}}{\pgfqpoint{6.253931in}{1.080610in}}{\pgfqpoint{6.261063in}{1.080610in}}%
\pgfpathclose%
\pgfusepath{stroke,fill}%
\end{pgfscope}%
\begin{pgfscope}%
\pgfpathrectangle{\pgfqpoint{4.985294in}{0.500000in}}{\pgfqpoint{1.764706in}{1.700000in}}%
\pgfusepath{clip}%
\pgfsetbuttcap%
\pgfsetroundjoin%
\definecolor{currentfill}{rgb}{0.969803,0.809811,0.702523}%
\pgfsetfillcolor{currentfill}%
\pgfsetlinewidth{0.311001pt}%
\definecolor{currentstroke}{rgb}{1.000000,1.000000,1.000000}%
\pgfsetstrokecolor{currentstroke}%
\pgfsetdash{}{0pt}%
\pgfpathmoveto{\pgfqpoint{6.380126in}{1.229155in}}%
\pgfpathcurveto{\pgfqpoint{6.387259in}{1.229155in}}{\pgfqpoint{6.394100in}{1.231989in}}{\pgfqpoint{6.399144in}{1.237033in}}%
\pgfpathcurveto{\pgfqpoint{6.404188in}{1.242076in}}{\pgfqpoint{6.407021in}{1.248918in}}{\pgfqpoint{6.407021in}{1.256051in}}%
\pgfpathcurveto{\pgfqpoint{6.407021in}{1.263184in}}{\pgfqpoint{6.404188in}{1.270025in}}{\pgfqpoint{6.399144in}{1.275069in}}%
\pgfpathcurveto{\pgfqpoint{6.394100in}{1.280113in}}{\pgfqpoint{6.387259in}{1.282946in}}{\pgfqpoint{6.380126in}{1.282946in}}%
\pgfpathcurveto{\pgfqpoint{6.372993in}{1.282946in}}{\pgfqpoint{6.366151in}{1.280113in}}{\pgfqpoint{6.361108in}{1.275069in}}%
\pgfpathcurveto{\pgfqpoint{6.356064in}{1.270025in}}{\pgfqpoint{6.353230in}{1.263184in}}{\pgfqpoint{6.353230in}{1.256051in}}%
\pgfpathcurveto{\pgfqpoint{6.353230in}{1.248918in}}{\pgfqpoint{6.356064in}{1.242076in}}{\pgfqpoint{6.361108in}{1.237033in}}%
\pgfpathcurveto{\pgfqpoint{6.366151in}{1.231989in}}{\pgfqpoint{6.372993in}{1.229155in}}{\pgfqpoint{6.380126in}{1.229155in}}%
\pgfpathclose%
\pgfusepath{stroke,fill}%
\end{pgfscope}%
\begin{pgfscope}%
\pgfpathrectangle{\pgfqpoint{4.985294in}{0.500000in}}{\pgfqpoint{1.764706in}{1.700000in}}%
\pgfusepath{clip}%
\pgfsetbuttcap%
\pgfsetroundjoin%
\definecolor{currentfill}{rgb}{0.978376,0.897317,0.831308}%
\pgfsetfillcolor{currentfill}%
\pgfsetlinewidth{0.311001pt}%
\definecolor{currentstroke}{rgb}{1.000000,1.000000,1.000000}%
\pgfsetstrokecolor{currentstroke}%
\pgfsetdash{}{0pt}%
\pgfpathmoveto{\pgfqpoint{5.431700in}{1.158595in}}%
\pgfpathcurveto{\pgfqpoint{5.438833in}{1.158595in}}{\pgfqpoint{5.445674in}{1.161429in}}{\pgfqpoint{5.450718in}{1.166472in}}%
\pgfpathcurveto{\pgfqpoint{5.455762in}{1.171516in}}{\pgfqpoint{5.458595in}{1.178358in}}{\pgfqpoint{5.458595in}{1.185491in}}%
\pgfpathcurveto{\pgfqpoint{5.458595in}{1.192623in}}{\pgfqpoint{5.455762in}{1.199465in}}{\pgfqpoint{5.450718in}{1.204509in}}%
\pgfpathcurveto{\pgfqpoint{5.445674in}{1.209552in}}{\pgfqpoint{5.438833in}{1.212386in}}{\pgfqpoint{5.431700in}{1.212386in}}%
\pgfpathcurveto{\pgfqpoint{5.424567in}{1.212386in}}{\pgfqpoint{5.417725in}{1.209552in}}{\pgfqpoint{5.412682in}{1.204509in}}%
\pgfpathcurveto{\pgfqpoint{5.407638in}{1.199465in}}{\pgfqpoint{5.404804in}{1.192623in}}{\pgfqpoint{5.404804in}{1.185491in}}%
\pgfpathcurveto{\pgfqpoint{5.404804in}{1.178358in}}{\pgfqpoint{5.407638in}{1.171516in}}{\pgfqpoint{5.412682in}{1.166472in}}%
\pgfpathcurveto{\pgfqpoint{5.417725in}{1.161429in}}{\pgfqpoint{5.424567in}{1.158595in}}{\pgfqpoint{5.431700in}{1.158595in}}%
\pgfpathclose%
\pgfusepath{stroke,fill}%
\end{pgfscope}%
\begin{pgfscope}%
\pgfpathrectangle{\pgfqpoint{4.985294in}{0.500000in}}{\pgfqpoint{1.764706in}{1.700000in}}%
\pgfusepath{clip}%
\pgfsetbuttcap%
\pgfsetroundjoin%
\definecolor{currentfill}{rgb}{0.964799,0.689101,0.537560}%
\pgfsetfillcolor{currentfill}%
\pgfsetlinewidth{0.311001pt}%
\definecolor{currentstroke}{rgb}{1.000000,1.000000,1.000000}%
\pgfsetstrokecolor{currentstroke}%
\pgfsetdash{}{0pt}%
\pgfpathmoveto{\pgfqpoint{5.567834in}{0.896816in}}%
\pgfpathcurveto{\pgfqpoint{5.574967in}{0.896816in}}{\pgfqpoint{5.581809in}{0.899650in}}{\pgfqpoint{5.586852in}{0.904694in}}%
\pgfpathcurveto{\pgfqpoint{5.591896in}{0.909737in}}{\pgfqpoint{5.594730in}{0.916579in}}{\pgfqpoint{5.594730in}{0.923712in}}%
\pgfpathcurveto{\pgfqpoint{5.594730in}{0.930845in}}{\pgfqpoint{5.591896in}{0.937686in}}{\pgfqpoint{5.586852in}{0.942730in}}%
\pgfpathcurveto{\pgfqpoint{5.581809in}{0.947774in}}{\pgfqpoint{5.574967in}{0.950608in}}{\pgfqpoint{5.567834in}{0.950608in}}%
\pgfpathcurveto{\pgfqpoint{5.560701in}{0.950608in}}{\pgfqpoint{5.553860in}{0.947774in}}{\pgfqpoint{5.548816in}{0.942730in}}%
\pgfpathcurveto{\pgfqpoint{5.543772in}{0.937686in}}{\pgfqpoint{5.540938in}{0.930845in}}{\pgfqpoint{5.540938in}{0.923712in}}%
\pgfpathcurveto{\pgfqpoint{5.540938in}{0.916579in}}{\pgfqpoint{5.543772in}{0.909737in}}{\pgfqpoint{5.548816in}{0.904694in}}%
\pgfpathcurveto{\pgfqpoint{5.553860in}{0.899650in}}{\pgfqpoint{5.560701in}{0.896816in}}{\pgfqpoint{5.567834in}{0.896816in}}%
\pgfpathclose%
\pgfusepath{stroke,fill}%
\end{pgfscope}%
\begin{pgfscope}%
\pgfpathrectangle{\pgfqpoint{4.985294in}{0.500000in}}{\pgfqpoint{1.764706in}{1.700000in}}%
\pgfusepath{clip}%
\pgfsetbuttcap%
\pgfsetroundjoin%
\definecolor{currentfill}{rgb}{0.970255,0.815666,0.711203}%
\pgfsetfillcolor{currentfill}%
\pgfsetlinewidth{0.311001pt}%
\definecolor{currentstroke}{rgb}{1.000000,1.000000,1.000000}%
\pgfsetstrokecolor{currentstroke}%
\pgfsetdash{}{0pt}%
\pgfpathmoveto{\pgfqpoint{5.468942in}{1.635989in}}%
\pgfpathcurveto{\pgfqpoint{5.476075in}{1.635989in}}{\pgfqpoint{5.482916in}{1.638823in}}{\pgfqpoint{5.487960in}{1.643867in}}%
\pgfpathcurveto{\pgfqpoint{5.493004in}{1.648911in}}{\pgfqpoint{5.495837in}{1.655752in}}{\pgfqpoint{5.495837in}{1.662885in}}%
\pgfpathcurveto{\pgfqpoint{5.495837in}{1.670018in}}{\pgfqpoint{5.493004in}{1.676860in}}{\pgfqpoint{5.487960in}{1.681903in}}%
\pgfpathcurveto{\pgfqpoint{5.482916in}{1.686947in}}{\pgfqpoint{5.476075in}{1.689781in}}{\pgfqpoint{5.468942in}{1.689781in}}%
\pgfpathcurveto{\pgfqpoint{5.461809in}{1.689781in}}{\pgfqpoint{5.454967in}{1.686947in}}{\pgfqpoint{5.449924in}{1.681903in}}%
\pgfpathcurveto{\pgfqpoint{5.444880in}{1.676860in}}{\pgfqpoint{5.442046in}{1.670018in}}{\pgfqpoint{5.442046in}{1.662885in}}%
\pgfpathcurveto{\pgfqpoint{5.442046in}{1.655752in}}{\pgfqpoint{5.444880in}{1.648911in}}{\pgfqpoint{5.449924in}{1.643867in}}%
\pgfpathcurveto{\pgfqpoint{5.454967in}{1.638823in}}{\pgfqpoint{5.461809in}{1.635989in}}{\pgfqpoint{5.468942in}{1.635989in}}%
\pgfpathclose%
\pgfusepath{stroke,fill}%
\end{pgfscope}%
\begin{pgfscope}%
\pgfpathrectangle{\pgfqpoint{4.985294in}{0.500000in}}{\pgfqpoint{1.764706in}{1.700000in}}%
\pgfusepath{clip}%
\pgfsetbuttcap%
\pgfsetroundjoin%
\definecolor{currentfill}{rgb}{0.969803,0.809811,0.702523}%
\pgfsetfillcolor{currentfill}%
\pgfsetlinewidth{0.311001pt}%
\definecolor{currentstroke}{rgb}{1.000000,1.000000,1.000000}%
\pgfsetstrokecolor{currentstroke}%
\pgfsetdash{}{0pt}%
\pgfpathmoveto{\pgfqpoint{6.233674in}{1.670248in}}%
\pgfpathcurveto{\pgfqpoint{6.240807in}{1.670248in}}{\pgfqpoint{6.247648in}{1.673082in}}{\pgfqpoint{6.252692in}{1.678126in}}%
\pgfpathcurveto{\pgfqpoint{6.257736in}{1.683169in}}{\pgfqpoint{6.260570in}{1.690011in}}{\pgfqpoint{6.260570in}{1.697144in}}%
\pgfpathcurveto{\pgfqpoint{6.260570in}{1.704277in}}{\pgfqpoint{6.257736in}{1.711118in}}{\pgfqpoint{6.252692in}{1.716162in}}%
\pgfpathcurveto{\pgfqpoint{6.247648in}{1.721206in}}{\pgfqpoint{6.240807in}{1.724039in}}{\pgfqpoint{6.233674in}{1.724039in}}%
\pgfpathcurveto{\pgfqpoint{6.226541in}{1.724039in}}{\pgfqpoint{6.219699in}{1.721206in}}{\pgfqpoint{6.214656in}{1.716162in}}%
\pgfpathcurveto{\pgfqpoint{6.209612in}{1.711118in}}{\pgfqpoint{6.206778in}{1.704277in}}{\pgfqpoint{6.206778in}{1.697144in}}%
\pgfpathcurveto{\pgfqpoint{6.206778in}{1.690011in}}{\pgfqpoint{6.209612in}{1.683169in}}{\pgfqpoint{6.214656in}{1.678126in}}%
\pgfpathcurveto{\pgfqpoint{6.219699in}{1.673082in}}{\pgfqpoint{6.226541in}{1.670248in}}{\pgfqpoint{6.233674in}{1.670248in}}%
\pgfpathclose%
\pgfusepath{stroke,fill}%
\end{pgfscope}%
\begin{pgfscope}%
\pgfpathrectangle{\pgfqpoint{4.985294in}{0.500000in}}{\pgfqpoint{1.764706in}{1.700000in}}%
\pgfusepath{clip}%
\pgfsetbuttcap%
\pgfsetroundjoin%
\definecolor{currentfill}{rgb}{0.968509,0.792226,0.676405}%
\pgfsetfillcolor{currentfill}%
\pgfsetlinewidth{0.311001pt}%
\definecolor{currentstroke}{rgb}{1.000000,1.000000,1.000000}%
\pgfsetstrokecolor{currentstroke}%
\pgfsetdash{}{0pt}%
\pgfpathmoveto{\pgfqpoint{5.567213in}{0.998658in}}%
\pgfpathcurveto{\pgfqpoint{5.574346in}{0.998658in}}{\pgfqpoint{5.581188in}{1.001492in}}{\pgfqpoint{5.586231in}{1.006536in}}%
\pgfpathcurveto{\pgfqpoint{5.591275in}{1.011579in}}{\pgfqpoint{5.594109in}{1.018421in}}{\pgfqpoint{5.594109in}{1.025554in}}%
\pgfpathcurveto{\pgfqpoint{5.594109in}{1.032686in}}{\pgfqpoint{5.591275in}{1.039528in}}{\pgfqpoint{5.586231in}{1.044572in}}%
\pgfpathcurveto{\pgfqpoint{5.581188in}{1.049615in}}{\pgfqpoint{5.574346in}{1.052449in}}{\pgfqpoint{5.567213in}{1.052449in}}%
\pgfpathcurveto{\pgfqpoint{5.560080in}{1.052449in}}{\pgfqpoint{5.553239in}{1.049615in}}{\pgfqpoint{5.548195in}{1.044572in}}%
\pgfpathcurveto{\pgfqpoint{5.543151in}{1.039528in}}{\pgfqpoint{5.540318in}{1.032686in}}{\pgfqpoint{5.540318in}{1.025554in}}%
\pgfpathcurveto{\pgfqpoint{5.540318in}{1.018421in}}{\pgfqpoint{5.543151in}{1.011579in}}{\pgfqpoint{5.548195in}{1.006536in}}%
\pgfpathcurveto{\pgfqpoint{5.553239in}{1.001492in}}{\pgfqpoint{5.560080in}{0.998658in}}{\pgfqpoint{5.567213in}{0.998658in}}%
\pgfpathclose%
\pgfusepath{stroke,fill}%
\end{pgfscope}%
\begin{pgfscope}%
\pgfpathrectangle{\pgfqpoint{4.985294in}{0.500000in}}{\pgfqpoint{1.764706in}{1.700000in}}%
\pgfusepath{clip}%
\pgfsetbuttcap%
\pgfsetroundjoin%
\definecolor{currentfill}{rgb}{0.972201,0.839051,0.745789}%
\pgfsetfillcolor{currentfill}%
\pgfsetlinewidth{0.311001pt}%
\definecolor{currentstroke}{rgb}{1.000000,1.000000,1.000000}%
\pgfsetstrokecolor{currentstroke}%
\pgfsetdash{}{0pt}%
\pgfpathmoveto{\pgfqpoint{5.501886in}{1.604987in}}%
\pgfpathcurveto{\pgfqpoint{5.509019in}{1.604987in}}{\pgfqpoint{5.515861in}{1.607821in}}{\pgfqpoint{5.520904in}{1.612864in}}%
\pgfpathcurveto{\pgfqpoint{5.525948in}{1.617908in}}{\pgfqpoint{5.528782in}{1.624750in}}{\pgfqpoint{5.528782in}{1.631882in}}%
\pgfpathcurveto{\pgfqpoint{5.528782in}{1.639015in}}{\pgfqpoint{5.525948in}{1.645857in}}{\pgfqpoint{5.520904in}{1.650901in}}%
\pgfpathcurveto{\pgfqpoint{5.515861in}{1.655944in}}{\pgfqpoint{5.509019in}{1.658778in}}{\pgfqpoint{5.501886in}{1.658778in}}%
\pgfpathcurveto{\pgfqpoint{5.494753in}{1.658778in}}{\pgfqpoint{5.487912in}{1.655944in}}{\pgfqpoint{5.482868in}{1.650901in}}%
\pgfpathcurveto{\pgfqpoint{5.477824in}{1.645857in}}{\pgfqpoint{5.474990in}{1.639015in}}{\pgfqpoint{5.474990in}{1.631882in}}%
\pgfpathcurveto{\pgfqpoint{5.474990in}{1.624750in}}{\pgfqpoint{5.477824in}{1.617908in}}{\pgfqpoint{5.482868in}{1.612864in}}%
\pgfpathcurveto{\pgfqpoint{5.487912in}{1.607821in}}{\pgfqpoint{5.494753in}{1.604987in}}{\pgfqpoint{5.501886in}{1.604987in}}%
\pgfpathclose%
\pgfusepath{stroke,fill}%
\end{pgfscope}%
\begin{pgfscope}%
\pgfpathrectangle{\pgfqpoint{4.985294in}{0.500000in}}{\pgfqpoint{1.764706in}{1.700000in}}%
\pgfusepath{clip}%
\pgfsetbuttcap%
\pgfsetroundjoin%
\definecolor{currentfill}{rgb}{0.976287,0.879862,0.805788}%
\pgfsetfillcolor{currentfill}%
\pgfsetlinewidth{0.311001pt}%
\definecolor{currentstroke}{rgb}{1.000000,1.000000,1.000000}%
\pgfsetstrokecolor{currentstroke}%
\pgfsetdash{}{0pt}%
\pgfpathmoveto{\pgfqpoint{6.267226in}{1.247800in}}%
\pgfpathcurveto{\pgfqpoint{6.274359in}{1.247800in}}{\pgfqpoint{6.281201in}{1.250634in}}{\pgfqpoint{6.286245in}{1.255678in}}%
\pgfpathcurveto{\pgfqpoint{6.291288in}{1.260722in}}{\pgfqpoint{6.294122in}{1.267563in}}{\pgfqpoint{6.294122in}{1.274696in}}%
\pgfpathcurveto{\pgfqpoint{6.294122in}{1.281829in}}{\pgfqpoint{6.291288in}{1.288671in}}{\pgfqpoint{6.286245in}{1.293714in}}%
\pgfpathcurveto{\pgfqpoint{6.281201in}{1.298758in}}{\pgfqpoint{6.274359in}{1.301592in}}{\pgfqpoint{6.267226in}{1.301592in}}%
\pgfpathcurveto{\pgfqpoint{6.260094in}{1.301592in}}{\pgfqpoint{6.253252in}{1.298758in}}{\pgfqpoint{6.248208in}{1.293714in}}%
\pgfpathcurveto{\pgfqpoint{6.243165in}{1.288671in}}{\pgfqpoint{6.240331in}{1.281829in}}{\pgfqpoint{6.240331in}{1.274696in}}%
\pgfpathcurveto{\pgfqpoint{6.240331in}{1.267563in}}{\pgfqpoint{6.243165in}{1.260722in}}{\pgfqpoint{6.248208in}{1.255678in}}%
\pgfpathcurveto{\pgfqpoint{6.253252in}{1.250634in}}{\pgfqpoint{6.260094in}{1.247800in}}{\pgfqpoint{6.267226in}{1.247800in}}%
\pgfpathclose%
\pgfusepath{stroke,fill}%
\end{pgfscope}%
\begin{pgfscope}%
\pgfpathrectangle{\pgfqpoint{4.985294in}{0.500000in}}{\pgfqpoint{1.764706in}{1.700000in}}%
\pgfusepath{clip}%
\pgfsetbuttcap%
\pgfsetroundjoin%
\definecolor{currentfill}{rgb}{0.955697,0.484891,0.334214}%
\pgfsetfillcolor{currentfill}%
\pgfsetlinewidth{0.311001pt}%
\definecolor{currentstroke}{rgb}{1.000000,1.000000,1.000000}%
\pgfsetstrokecolor{currentstroke}%
\pgfsetdash{}{0pt}%
\pgfpathmoveto{\pgfqpoint{5.654198in}{1.632510in}}%
\pgfpathcurveto{\pgfqpoint{5.661331in}{1.632510in}}{\pgfqpoint{5.668172in}{1.635344in}}{\pgfqpoint{5.673216in}{1.640388in}}%
\pgfpathcurveto{\pgfqpoint{5.678260in}{1.645431in}}{\pgfqpoint{5.681094in}{1.652273in}}{\pgfqpoint{5.681094in}{1.659406in}}%
\pgfpathcurveto{\pgfqpoint{5.681094in}{1.666539in}}{\pgfqpoint{5.678260in}{1.673380in}}{\pgfqpoint{5.673216in}{1.678424in}}%
\pgfpathcurveto{\pgfqpoint{5.668172in}{1.683468in}}{\pgfqpoint{5.661331in}{1.686302in}}{\pgfqpoint{5.654198in}{1.686302in}}%
\pgfpathcurveto{\pgfqpoint{5.647065in}{1.686302in}}{\pgfqpoint{5.640223in}{1.683468in}}{\pgfqpoint{5.635180in}{1.678424in}}%
\pgfpathcurveto{\pgfqpoint{5.630136in}{1.673380in}}{\pgfqpoint{5.627302in}{1.666539in}}{\pgfqpoint{5.627302in}{1.659406in}}%
\pgfpathcurveto{\pgfqpoint{5.627302in}{1.652273in}}{\pgfqpoint{5.630136in}{1.645431in}}{\pgfqpoint{5.635180in}{1.640388in}}%
\pgfpathcurveto{\pgfqpoint{5.640223in}{1.635344in}}{\pgfqpoint{5.647065in}{1.632510in}}{\pgfqpoint{5.654198in}{1.632510in}}%
\pgfpathclose%
\pgfusepath{stroke,fill}%
\end{pgfscope}%
\begin{pgfscope}%
\pgfpathrectangle{\pgfqpoint{4.985294in}{0.500000in}}{\pgfqpoint{1.764706in}{1.700000in}}%
\pgfusepath{clip}%
\pgfsetbuttcap%
\pgfsetroundjoin%
\definecolor{currentfill}{rgb}{0.962765,0.606121,0.444717}%
\pgfsetfillcolor{currentfill}%
\pgfsetlinewidth{0.311001pt}%
\definecolor{currentstroke}{rgb}{1.000000,1.000000,1.000000}%
\pgfsetstrokecolor{currentstroke}%
\pgfsetdash{}{0pt}%
\pgfpathmoveto{\pgfqpoint{5.441747in}{1.709898in}}%
\pgfpathcurveto{\pgfqpoint{5.448879in}{1.709898in}}{\pgfqpoint{5.455721in}{1.712732in}}{\pgfqpoint{5.460765in}{1.717775in}}%
\pgfpathcurveto{\pgfqpoint{5.465808in}{1.722819in}}{\pgfqpoint{5.468642in}{1.729661in}}{\pgfqpoint{5.468642in}{1.736794in}}%
\pgfpathcurveto{\pgfqpoint{5.468642in}{1.743926in}}{\pgfqpoint{5.465808in}{1.750768in}}{\pgfqpoint{5.460765in}{1.755812in}}%
\pgfpathcurveto{\pgfqpoint{5.455721in}{1.760855in}}{\pgfqpoint{5.448879in}{1.763689in}}{\pgfqpoint{5.441747in}{1.763689in}}%
\pgfpathcurveto{\pgfqpoint{5.434614in}{1.763689in}}{\pgfqpoint{5.427772in}{1.760855in}}{\pgfqpoint{5.422728in}{1.755812in}}%
\pgfpathcurveto{\pgfqpoint{5.417685in}{1.750768in}}{\pgfqpoint{5.414851in}{1.743926in}}{\pgfqpoint{5.414851in}{1.736794in}}%
\pgfpathcurveto{\pgfqpoint{5.414851in}{1.729661in}}{\pgfqpoint{5.417685in}{1.722819in}}{\pgfqpoint{5.422728in}{1.717775in}}%
\pgfpathcurveto{\pgfqpoint{5.427772in}{1.712732in}}{\pgfqpoint{5.434614in}{1.709898in}}{\pgfqpoint{5.441747in}{1.709898in}}%
\pgfpathclose%
\pgfusepath{stroke,fill}%
\end{pgfscope}%
\begin{pgfscope}%
\pgfpathrectangle{\pgfqpoint{4.985294in}{0.500000in}}{\pgfqpoint{1.764706in}{1.700000in}}%
\pgfusepath{clip}%
\pgfsetbuttcap%
\pgfsetroundjoin%
\definecolor{currentfill}{rgb}{0.980678,0.914765,0.856766}%
\pgfsetfillcolor{currentfill}%
\pgfsetlinewidth{0.311001pt}%
\definecolor{currentstroke}{rgb}{1.000000,1.000000,1.000000}%
\pgfsetstrokecolor{currentstroke}%
\pgfsetdash{}{0pt}%
\pgfpathmoveto{\pgfqpoint{6.294618in}{1.468841in}}%
\pgfpathcurveto{\pgfqpoint{6.301751in}{1.468841in}}{\pgfqpoint{6.308592in}{1.471675in}}{\pgfqpoint{6.313636in}{1.476719in}}%
\pgfpathcurveto{\pgfqpoint{6.318680in}{1.481762in}}{\pgfqpoint{6.321514in}{1.488604in}}{\pgfqpoint{6.321514in}{1.495737in}}%
\pgfpathcurveto{\pgfqpoint{6.321514in}{1.502870in}}{\pgfqpoint{6.318680in}{1.509711in}}{\pgfqpoint{6.313636in}{1.514755in}}%
\pgfpathcurveto{\pgfqpoint{6.308592in}{1.519799in}}{\pgfqpoint{6.301751in}{1.522633in}}{\pgfqpoint{6.294618in}{1.522633in}}%
\pgfpathcurveto{\pgfqpoint{6.287485in}{1.522633in}}{\pgfqpoint{6.280643in}{1.519799in}}{\pgfqpoint{6.275600in}{1.514755in}}%
\pgfpathcurveto{\pgfqpoint{6.270556in}{1.509711in}}{\pgfqpoint{6.267722in}{1.502870in}}{\pgfqpoint{6.267722in}{1.495737in}}%
\pgfpathcurveto{\pgfqpoint{6.267722in}{1.488604in}}{\pgfqpoint{6.270556in}{1.481762in}}{\pgfqpoint{6.275600in}{1.476719in}}%
\pgfpathcurveto{\pgfqpoint{6.280643in}{1.471675in}}{\pgfqpoint{6.287485in}{1.468841in}}{\pgfqpoint{6.294618in}{1.468841in}}%
\pgfpathclose%
\pgfusepath{stroke,fill}%
\end{pgfscope}%
\begin{pgfscope}%
\pgfpathrectangle{\pgfqpoint{4.985294in}{0.500000in}}{\pgfqpoint{1.764706in}{1.700000in}}%
\pgfusepath{clip}%
\pgfsetbuttcap%
\pgfsetroundjoin%
\definecolor{currentfill}{rgb}{0.963190,0.619109,0.458249}%
\pgfsetfillcolor{currentfill}%
\pgfsetlinewidth{0.311001pt}%
\definecolor{currentstroke}{rgb}{1.000000,1.000000,1.000000}%
\pgfsetstrokecolor{currentstroke}%
\pgfsetdash{}{0pt}%
\pgfpathmoveto{\pgfqpoint{5.540888in}{1.341132in}}%
\pgfpathcurveto{\pgfqpoint{5.548021in}{1.341132in}}{\pgfqpoint{5.554862in}{1.343966in}}{\pgfqpoint{5.559906in}{1.349010in}}%
\pgfpathcurveto{\pgfqpoint{5.564950in}{1.354053in}}{\pgfqpoint{5.567783in}{1.360895in}}{\pgfqpoint{5.567783in}{1.368028in}}%
\pgfpathcurveto{\pgfqpoint{5.567783in}{1.375161in}}{\pgfqpoint{5.564950in}{1.382002in}}{\pgfqpoint{5.559906in}{1.387046in}}%
\pgfpathcurveto{\pgfqpoint{5.554862in}{1.392090in}}{\pgfqpoint{5.548021in}{1.394923in}}{\pgfqpoint{5.540888in}{1.394923in}}%
\pgfpathcurveto{\pgfqpoint{5.533755in}{1.394923in}}{\pgfqpoint{5.526913in}{1.392090in}}{\pgfqpoint{5.521870in}{1.387046in}}%
\pgfpathcurveto{\pgfqpoint{5.516826in}{1.382002in}}{\pgfqpoint{5.513992in}{1.375161in}}{\pgfqpoint{5.513992in}{1.368028in}}%
\pgfpathcurveto{\pgfqpoint{5.513992in}{1.360895in}}{\pgfqpoint{5.516826in}{1.354053in}}{\pgfqpoint{5.521870in}{1.349010in}}%
\pgfpathcurveto{\pgfqpoint{5.526913in}{1.343966in}}{\pgfqpoint{5.533755in}{1.341132in}}{\pgfqpoint{5.540888in}{1.341132in}}%
\pgfpathclose%
\pgfusepath{stroke,fill}%
\end{pgfscope}%
\begin{pgfscope}%
\pgfpathrectangle{\pgfqpoint{4.985294in}{0.500000in}}{\pgfqpoint{1.764706in}{1.700000in}}%
\pgfusepath{clip}%
\pgfsetbuttcap%
\pgfsetroundjoin%
\definecolor{currentfill}{rgb}{0.973832,0.856556,0.771584}%
\pgfsetfillcolor{currentfill}%
\pgfsetlinewidth{0.311001pt}%
\definecolor{currentstroke}{rgb}{1.000000,1.000000,1.000000}%
\pgfsetstrokecolor{currentstroke}%
\pgfsetdash{}{0pt}%
\pgfpathmoveto{\pgfqpoint{6.242359in}{1.525275in}}%
\pgfpathcurveto{\pgfqpoint{6.249492in}{1.525275in}}{\pgfqpoint{6.256334in}{1.528109in}}{\pgfqpoint{6.261378in}{1.533152in}}%
\pgfpathcurveto{\pgfqpoint{6.266421in}{1.538196in}}{\pgfqpoint{6.269255in}{1.545038in}}{\pgfqpoint{6.269255in}{1.552171in}}%
\pgfpathcurveto{\pgfqpoint{6.269255in}{1.559303in}}{\pgfqpoint{6.266421in}{1.566145in}}{\pgfqpoint{6.261378in}{1.571189in}}%
\pgfpathcurveto{\pgfqpoint{6.256334in}{1.576232in}}{\pgfqpoint{6.249492in}{1.579066in}}{\pgfqpoint{6.242359in}{1.579066in}}%
\pgfpathcurveto{\pgfqpoint{6.235227in}{1.579066in}}{\pgfqpoint{6.228385in}{1.576232in}}{\pgfqpoint{6.223341in}{1.571189in}}%
\pgfpathcurveto{\pgfqpoint{6.218298in}{1.566145in}}{\pgfqpoint{6.215464in}{1.559303in}}{\pgfqpoint{6.215464in}{1.552171in}}%
\pgfpathcurveto{\pgfqpoint{6.215464in}{1.545038in}}{\pgfqpoint{6.218298in}{1.538196in}}{\pgfqpoint{6.223341in}{1.533152in}}%
\pgfpathcurveto{\pgfqpoint{6.228385in}{1.528109in}}{\pgfqpoint{6.235227in}{1.525275in}}{\pgfqpoint{6.242359in}{1.525275in}}%
\pgfpathclose%
\pgfusepath{stroke,fill}%
\end{pgfscope}%
\begin{pgfscope}%
\pgfpathrectangle{\pgfqpoint{4.985294in}{0.500000in}}{\pgfqpoint{1.764706in}{1.700000in}}%
\pgfusepath{clip}%
\pgfsetbuttcap%
\pgfsetroundjoin%
\definecolor{currentfill}{rgb}{0.967735,0.780441,0.659127}%
\pgfsetfillcolor{currentfill}%
\pgfsetlinewidth{0.311001pt}%
\definecolor{currentstroke}{rgb}{1.000000,1.000000,1.000000}%
\pgfsetstrokecolor{currentstroke}%
\pgfsetdash{}{0pt}%
\pgfpathmoveto{\pgfqpoint{6.354672in}{1.518021in}}%
\pgfpathcurveto{\pgfqpoint{6.361805in}{1.518021in}}{\pgfqpoint{6.368647in}{1.520855in}}{\pgfqpoint{6.373690in}{1.525898in}}%
\pgfpathcurveto{\pgfqpoint{6.378734in}{1.530942in}}{\pgfqpoint{6.381568in}{1.537784in}}{\pgfqpoint{6.381568in}{1.544916in}}%
\pgfpathcurveto{\pgfqpoint{6.381568in}{1.552049in}}{\pgfqpoint{6.378734in}{1.558891in}}{\pgfqpoint{6.373690in}{1.563935in}}%
\pgfpathcurveto{\pgfqpoint{6.368647in}{1.568978in}}{\pgfqpoint{6.361805in}{1.571812in}}{\pgfqpoint{6.354672in}{1.571812in}}%
\pgfpathcurveto{\pgfqpoint{6.347539in}{1.571812in}}{\pgfqpoint{6.340698in}{1.568978in}}{\pgfqpoint{6.335654in}{1.563935in}}%
\pgfpathcurveto{\pgfqpoint{6.330610in}{1.558891in}}{\pgfqpoint{6.327777in}{1.552049in}}{\pgfqpoint{6.327777in}{1.544916in}}%
\pgfpathcurveto{\pgfqpoint{6.327777in}{1.537784in}}{\pgfqpoint{6.330610in}{1.530942in}}{\pgfqpoint{6.335654in}{1.525898in}}%
\pgfpathcurveto{\pgfqpoint{6.340698in}{1.520855in}}{\pgfqpoint{6.347539in}{1.518021in}}{\pgfqpoint{6.354672in}{1.518021in}}%
\pgfpathclose%
\pgfusepath{stroke,fill}%
\end{pgfscope}%
\begin{pgfscope}%
\pgfpathrectangle{\pgfqpoint{4.985294in}{0.500000in}}{\pgfqpoint{1.764706in}{1.700000in}}%
\pgfusepath{clip}%
\pgfsetbuttcap%
\pgfsetroundjoin%
\definecolor{currentfill}{rgb}{0.971694,0.833208,0.737161}%
\pgfsetfillcolor{currentfill}%
\pgfsetlinewidth{0.311001pt}%
\definecolor{currentstroke}{rgb}{1.000000,1.000000,1.000000}%
\pgfsetstrokecolor{currentstroke}%
\pgfsetdash{}{0pt}%
\pgfpathmoveto{\pgfqpoint{5.480969in}{1.619997in}}%
\pgfpathcurveto{\pgfqpoint{5.488102in}{1.619997in}}{\pgfqpoint{5.494944in}{1.622830in}}{\pgfqpoint{5.499987in}{1.627874in}}%
\pgfpathcurveto{\pgfqpoint{5.505031in}{1.632918in}}{\pgfqpoint{5.507865in}{1.639759in}}{\pgfqpoint{5.507865in}{1.646892in}}%
\pgfpathcurveto{\pgfqpoint{5.507865in}{1.654025in}}{\pgfqpoint{5.505031in}{1.660867in}}{\pgfqpoint{5.499987in}{1.665910in}}%
\pgfpathcurveto{\pgfqpoint{5.494944in}{1.670954in}}{\pgfqpoint{5.488102in}{1.673788in}}{\pgfqpoint{5.480969in}{1.673788in}}%
\pgfpathcurveto{\pgfqpoint{5.473836in}{1.673788in}}{\pgfqpoint{5.466995in}{1.670954in}}{\pgfqpoint{5.461951in}{1.665910in}}%
\pgfpathcurveto{\pgfqpoint{5.456907in}{1.660867in}}{\pgfqpoint{5.454074in}{1.654025in}}{\pgfqpoint{5.454074in}{1.646892in}}%
\pgfpathcurveto{\pgfqpoint{5.454074in}{1.639759in}}{\pgfqpoint{5.456907in}{1.632918in}}{\pgfqpoint{5.461951in}{1.627874in}}%
\pgfpathcurveto{\pgfqpoint{5.466995in}{1.622830in}}{\pgfqpoint{5.473836in}{1.619997in}}{\pgfqpoint{5.480969in}{1.619997in}}%
\pgfpathclose%
\pgfusepath{stroke,fill}%
\end{pgfscope}%
\begin{pgfscope}%
\pgfpathrectangle{\pgfqpoint{4.985294in}{0.500000in}}{\pgfqpoint{1.764706in}{1.700000in}}%
\pgfusepath{clip}%
\pgfsetbuttcap%
\pgfsetroundjoin%
\definecolor{currentfill}{rgb}{0.974412,0.862387,0.780156}%
\pgfsetfillcolor{currentfill}%
\pgfsetlinewidth{0.311001pt}%
\definecolor{currentstroke}{rgb}{1.000000,1.000000,1.000000}%
\pgfsetstrokecolor{currentstroke}%
\pgfsetdash{}{0pt}%
\pgfpathmoveto{\pgfqpoint{5.469433in}{1.344829in}}%
\pgfpathcurveto{\pgfqpoint{5.476565in}{1.344829in}}{\pgfqpoint{5.483407in}{1.347663in}}{\pgfqpoint{5.488451in}{1.352707in}}%
\pgfpathcurveto{\pgfqpoint{5.493494in}{1.357750in}}{\pgfqpoint{5.496328in}{1.364592in}}{\pgfqpoint{5.496328in}{1.371725in}}%
\pgfpathcurveto{\pgfqpoint{5.496328in}{1.378858in}}{\pgfqpoint{5.493494in}{1.385699in}}{\pgfqpoint{5.488451in}{1.390743in}}%
\pgfpathcurveto{\pgfqpoint{5.483407in}{1.395787in}}{\pgfqpoint{5.476565in}{1.398621in}}{\pgfqpoint{5.469433in}{1.398621in}}%
\pgfpathcurveto{\pgfqpoint{5.462300in}{1.398621in}}{\pgfqpoint{5.455458in}{1.395787in}}{\pgfqpoint{5.450414in}{1.390743in}}%
\pgfpathcurveto{\pgfqpoint{5.445371in}{1.385699in}}{\pgfqpoint{5.442537in}{1.378858in}}{\pgfqpoint{5.442537in}{1.371725in}}%
\pgfpathcurveto{\pgfqpoint{5.442537in}{1.364592in}}{\pgfqpoint{5.445371in}{1.357750in}}{\pgfqpoint{5.450414in}{1.352707in}}%
\pgfpathcurveto{\pgfqpoint{5.455458in}{1.347663in}}{\pgfqpoint{5.462300in}{1.344829in}}{\pgfqpoint{5.469433in}{1.344829in}}%
\pgfpathclose%
\pgfusepath{stroke,fill}%
\end{pgfscope}%
\begin{pgfscope}%
\pgfpathrectangle{\pgfqpoint{4.985294in}{0.500000in}}{\pgfqpoint{1.764706in}{1.700000in}}%
\pgfusepath{clip}%
\pgfsetbuttcap%
\pgfsetroundjoin%
\definecolor{currentfill}{rgb}{0.960421,0.553286,0.393191}%
\pgfsetfillcolor{currentfill}%
\pgfsetlinewidth{0.311001pt}%
\definecolor{currentstroke}{rgb}{1.000000,1.000000,1.000000}%
\pgfsetstrokecolor{currentstroke}%
\pgfsetdash{}{0pt}%
\pgfpathmoveto{\pgfqpoint{5.506870in}{0.862432in}}%
\pgfpathcurveto{\pgfqpoint{5.514003in}{0.862432in}}{\pgfqpoint{5.520845in}{0.865266in}}{\pgfqpoint{5.525888in}{0.870309in}}%
\pgfpathcurveto{\pgfqpoint{5.530932in}{0.875353in}}{\pgfqpoint{5.533766in}{0.882195in}}{\pgfqpoint{5.533766in}{0.889328in}}%
\pgfpathcurveto{\pgfqpoint{5.533766in}{0.896460in}}{\pgfqpoint{5.530932in}{0.903302in}}{\pgfqpoint{5.525888in}{0.908346in}}%
\pgfpathcurveto{\pgfqpoint{5.520845in}{0.913389in}}{\pgfqpoint{5.514003in}{0.916223in}}{\pgfqpoint{5.506870in}{0.916223in}}%
\pgfpathcurveto{\pgfqpoint{5.499737in}{0.916223in}}{\pgfqpoint{5.492896in}{0.913389in}}{\pgfqpoint{5.487852in}{0.908346in}}%
\pgfpathcurveto{\pgfqpoint{5.482808in}{0.903302in}}{\pgfqpoint{5.479975in}{0.896460in}}{\pgfqpoint{5.479975in}{0.889328in}}%
\pgfpathcurveto{\pgfqpoint{5.479975in}{0.882195in}}{\pgfqpoint{5.482808in}{0.875353in}}{\pgfqpoint{5.487852in}{0.870309in}}%
\pgfpathcurveto{\pgfqpoint{5.492896in}{0.865266in}}{\pgfqpoint{5.499737in}{0.862432in}}{\pgfqpoint{5.506870in}{0.862432in}}%
\pgfpathclose%
\pgfusepath{stroke,fill}%
\end{pgfscope}%
\begin{pgfscope}%
\pgfpathrectangle{\pgfqpoint{4.985294in}{0.500000in}}{\pgfqpoint{1.764706in}{1.700000in}}%
\pgfusepath{clip}%
\pgfsetbuttcap%
\pgfsetroundjoin%
\definecolor{currentfill}{rgb}{0.970718,0.821518,0.719872}%
\pgfsetfillcolor{currentfill}%
\pgfsetlinewidth{0.311001pt}%
\definecolor{currentstroke}{rgb}{1.000000,1.000000,1.000000}%
\pgfsetstrokecolor{currentstroke}%
\pgfsetdash{}{0pt}%
\pgfpathmoveto{\pgfqpoint{5.518518in}{1.098499in}}%
\pgfpathcurveto{\pgfqpoint{5.525651in}{1.098499in}}{\pgfqpoint{5.532493in}{1.101333in}}{\pgfqpoint{5.537536in}{1.106377in}}%
\pgfpathcurveto{\pgfqpoint{5.542580in}{1.111420in}}{\pgfqpoint{5.545414in}{1.118262in}}{\pgfqpoint{5.545414in}{1.125395in}}%
\pgfpathcurveto{\pgfqpoint{5.545414in}{1.132528in}}{\pgfqpoint{5.542580in}{1.139369in}}{\pgfqpoint{5.537536in}{1.144413in}}%
\pgfpathcurveto{\pgfqpoint{5.532493in}{1.149457in}}{\pgfqpoint{5.525651in}{1.152290in}}{\pgfqpoint{5.518518in}{1.152290in}}%
\pgfpathcurveto{\pgfqpoint{5.511385in}{1.152290in}}{\pgfqpoint{5.504544in}{1.149457in}}{\pgfqpoint{5.499500in}{1.144413in}}%
\pgfpathcurveto{\pgfqpoint{5.494456in}{1.139369in}}{\pgfqpoint{5.491622in}{1.132528in}}{\pgfqpoint{5.491622in}{1.125395in}}%
\pgfpathcurveto{\pgfqpoint{5.491622in}{1.118262in}}{\pgfqpoint{5.494456in}{1.111420in}}{\pgfqpoint{5.499500in}{1.106377in}}%
\pgfpathcurveto{\pgfqpoint{5.504544in}{1.101333in}}{\pgfqpoint{5.511385in}{1.098499in}}{\pgfqpoint{5.518518in}{1.098499in}}%
\pgfpathclose%
\pgfusepath{stroke,fill}%
\end{pgfscope}%
\begin{pgfscope}%
\pgfpathrectangle{\pgfqpoint{4.985294in}{0.500000in}}{\pgfqpoint{1.764706in}{1.700000in}}%
\pgfusepath{clip}%
\pgfsetbuttcap%
\pgfsetroundjoin%
\definecolor{currentfill}{rgb}{0.965169,0.707764,0.560659}%
\pgfsetfillcolor{currentfill}%
\pgfsetlinewidth{0.311001pt}%
\definecolor{currentstroke}{rgb}{1.000000,1.000000,1.000000}%
\pgfsetstrokecolor{currentstroke}%
\pgfsetdash{}{0pt}%
\pgfpathmoveto{\pgfqpoint{5.348118in}{1.140990in}}%
\pgfpathcurveto{\pgfqpoint{5.355251in}{1.140990in}}{\pgfqpoint{5.362093in}{1.143824in}}{\pgfqpoint{5.367136in}{1.148868in}}%
\pgfpathcurveto{\pgfqpoint{5.372180in}{1.153911in}}{\pgfqpoint{5.375014in}{1.160753in}}{\pgfqpoint{5.375014in}{1.167886in}}%
\pgfpathcurveto{\pgfqpoint{5.375014in}{1.175019in}}{\pgfqpoint{5.372180in}{1.181860in}}{\pgfqpoint{5.367136in}{1.186904in}}%
\pgfpathcurveto{\pgfqpoint{5.362093in}{1.191948in}}{\pgfqpoint{5.355251in}{1.194782in}}{\pgfqpoint{5.348118in}{1.194782in}}%
\pgfpathcurveto{\pgfqpoint{5.340985in}{1.194782in}}{\pgfqpoint{5.334144in}{1.191948in}}{\pgfqpoint{5.329100in}{1.186904in}}%
\pgfpathcurveto{\pgfqpoint{5.324056in}{1.181860in}}{\pgfqpoint{5.321222in}{1.175019in}}{\pgfqpoint{5.321222in}{1.167886in}}%
\pgfpathcurveto{\pgfqpoint{5.321222in}{1.160753in}}{\pgfqpoint{5.324056in}{1.153911in}}{\pgfqpoint{5.329100in}{1.148868in}}%
\pgfpathcurveto{\pgfqpoint{5.334144in}{1.143824in}}{\pgfqpoint{5.340985in}{1.140990in}}{\pgfqpoint{5.348118in}{1.140990in}}%
\pgfpathclose%
\pgfusepath{stroke,fill}%
\end{pgfscope}%
\begin{pgfscope}%
\pgfpathrectangle{\pgfqpoint{4.985294in}{0.500000in}}{\pgfqpoint{1.764706in}{1.700000in}}%
\pgfusepath{clip}%
\pgfsetbuttcap%
\pgfsetroundjoin%
\definecolor{currentfill}{rgb}{0.965042,0.701564,0.552889}%
\pgfsetfillcolor{currentfill}%
\pgfsetlinewidth{0.311001pt}%
\definecolor{currentstroke}{rgb}{1.000000,1.000000,1.000000}%
\pgfsetstrokecolor{currentstroke}%
\pgfsetdash{}{0pt}%
\pgfpathmoveto{\pgfqpoint{5.498645in}{1.721600in}}%
\pgfpathcurveto{\pgfqpoint{5.505777in}{1.721600in}}{\pgfqpoint{5.512619in}{1.724434in}}{\pgfqpoint{5.517663in}{1.729478in}}%
\pgfpathcurveto{\pgfqpoint{5.522706in}{1.734521in}}{\pgfqpoint{5.525540in}{1.741363in}}{\pgfqpoint{5.525540in}{1.748496in}}%
\pgfpathcurveto{\pgfqpoint{5.525540in}{1.755629in}}{\pgfqpoint{5.522706in}{1.762470in}}{\pgfqpoint{5.517663in}{1.767514in}}%
\pgfpathcurveto{\pgfqpoint{5.512619in}{1.772558in}}{\pgfqpoint{5.505777in}{1.775391in}}{\pgfqpoint{5.498645in}{1.775391in}}%
\pgfpathcurveto{\pgfqpoint{5.491512in}{1.775391in}}{\pgfqpoint{5.484670in}{1.772558in}}{\pgfqpoint{5.479626in}{1.767514in}}%
\pgfpathcurveto{\pgfqpoint{5.474583in}{1.762470in}}{\pgfqpoint{5.471749in}{1.755629in}}{\pgfqpoint{5.471749in}{1.748496in}}%
\pgfpathcurveto{\pgfqpoint{5.471749in}{1.741363in}}{\pgfqpoint{5.474583in}{1.734521in}}{\pgfqpoint{5.479626in}{1.729478in}}%
\pgfpathcurveto{\pgfqpoint{5.484670in}{1.724434in}}{\pgfqpoint{5.491512in}{1.721600in}}{\pgfqpoint{5.498645in}{1.721600in}}%
\pgfpathclose%
\pgfusepath{stroke,fill}%
\end{pgfscope}%
\begin{pgfscope}%
\pgfpathrectangle{\pgfqpoint{4.985294in}{0.500000in}}{\pgfqpoint{1.764706in}{1.700000in}}%
\pgfusepath{clip}%
\pgfsetbuttcap%
\pgfsetroundjoin%
\definecolor{currentfill}{rgb}{0.975644,0.874038,0.797253}%
\pgfsetfillcolor{currentfill}%
\pgfsetlinewidth{0.311001pt}%
\definecolor{currentstroke}{rgb}{1.000000,1.000000,1.000000}%
\pgfsetstrokecolor{currentstroke}%
\pgfsetdash{}{0pt}%
\pgfpathmoveto{\pgfqpoint{5.464087in}{1.239987in}}%
\pgfpathcurveto{\pgfqpoint{5.471220in}{1.239987in}}{\pgfqpoint{5.478062in}{1.242821in}}{\pgfqpoint{5.483105in}{1.247865in}}%
\pgfpathcurveto{\pgfqpoint{5.488149in}{1.252909in}}{\pgfqpoint{5.490983in}{1.259750in}}{\pgfqpoint{5.490983in}{1.266883in}}%
\pgfpathcurveto{\pgfqpoint{5.490983in}{1.274016in}}{\pgfqpoint{5.488149in}{1.280857in}}{\pgfqpoint{5.483105in}{1.285901in}}%
\pgfpathcurveto{\pgfqpoint{5.478062in}{1.290945in}}{\pgfqpoint{5.471220in}{1.293779in}}{\pgfqpoint{5.464087in}{1.293779in}}%
\pgfpathcurveto{\pgfqpoint{5.456954in}{1.293779in}}{\pgfqpoint{5.450113in}{1.290945in}}{\pgfqpoint{5.445069in}{1.285901in}}%
\pgfpathcurveto{\pgfqpoint{5.440025in}{1.280857in}}{\pgfqpoint{5.437191in}{1.274016in}}{\pgfqpoint{5.437191in}{1.266883in}}%
\pgfpathcurveto{\pgfqpoint{5.437191in}{1.259750in}}{\pgfqpoint{5.440025in}{1.252909in}}{\pgfqpoint{5.445069in}{1.247865in}}%
\pgfpathcurveto{\pgfqpoint{5.450113in}{1.242821in}}{\pgfqpoint{5.456954in}{1.239987in}}{\pgfqpoint{5.464087in}{1.239987in}}%
\pgfpathclose%
\pgfusepath{stroke,fill}%
\end{pgfscope}%
\begin{pgfscope}%
\pgfpathrectangle{\pgfqpoint{4.985294in}{0.500000in}}{\pgfqpoint{1.764706in}{1.700000in}}%
\pgfusepath{clip}%
\pgfsetbuttcap%
\pgfsetroundjoin%
\definecolor{currentfill}{rgb}{0.973271,0.850724,0.762998}%
\pgfsetfillcolor{currentfill}%
\pgfsetlinewidth{0.311001pt}%
\definecolor{currentstroke}{rgb}{1.000000,1.000000,1.000000}%
\pgfsetstrokecolor{currentstroke}%
\pgfsetdash{}{0pt}%
\pgfpathmoveto{\pgfqpoint{5.444026in}{1.056416in}}%
\pgfpathcurveto{\pgfqpoint{5.451159in}{1.056416in}}{\pgfqpoint{5.458001in}{1.059250in}}{\pgfqpoint{5.463045in}{1.064294in}}%
\pgfpathcurveto{\pgfqpoint{5.468088in}{1.069337in}}{\pgfqpoint{5.470922in}{1.076179in}}{\pgfqpoint{5.470922in}{1.083312in}}%
\pgfpathcurveto{\pgfqpoint{5.470922in}{1.090445in}}{\pgfqpoint{5.468088in}{1.097286in}}{\pgfqpoint{5.463045in}{1.102330in}}%
\pgfpathcurveto{\pgfqpoint{5.458001in}{1.107374in}}{\pgfqpoint{5.451159in}{1.110207in}}{\pgfqpoint{5.444026in}{1.110207in}}%
\pgfpathcurveto{\pgfqpoint{5.436894in}{1.110207in}}{\pgfqpoint{5.430052in}{1.107374in}}{\pgfqpoint{5.425008in}{1.102330in}}%
\pgfpathcurveto{\pgfqpoint{5.419965in}{1.097286in}}{\pgfqpoint{5.417131in}{1.090445in}}{\pgfqpoint{5.417131in}{1.083312in}}%
\pgfpathcurveto{\pgfqpoint{5.417131in}{1.076179in}}{\pgfqpoint{5.419965in}{1.069337in}}{\pgfqpoint{5.425008in}{1.064294in}}%
\pgfpathcurveto{\pgfqpoint{5.430052in}{1.059250in}}{\pgfqpoint{5.436894in}{1.056416in}}{\pgfqpoint{5.444026in}{1.056416in}}%
\pgfpathclose%
\pgfusepath{stroke,fill}%
\end{pgfscope}%
\begin{pgfscope}%
\pgfpathrectangle{\pgfqpoint{4.985294in}{0.500000in}}{\pgfqpoint{1.764706in}{1.700000in}}%
\pgfusepath{clip}%
\pgfsetbuttcap%
\pgfsetroundjoin%
\definecolor{currentfill}{rgb}{0.965302,0.713942,0.568499}%
\pgfsetfillcolor{currentfill}%
\pgfsetlinewidth{0.311001pt}%
\definecolor{currentstroke}{rgb}{1.000000,1.000000,1.000000}%
\pgfsetstrokecolor{currentstroke}%
\pgfsetdash{}{0pt}%
\pgfpathmoveto{\pgfqpoint{6.198037in}{1.303241in}}%
\pgfpathcurveto{\pgfqpoint{6.205170in}{1.303241in}}{\pgfqpoint{6.212011in}{1.306075in}}{\pgfqpoint{6.217055in}{1.311118in}}%
\pgfpathcurveto{\pgfqpoint{6.222099in}{1.316162in}}{\pgfqpoint{6.224933in}{1.323004in}}{\pgfqpoint{6.224933in}{1.330137in}}%
\pgfpathcurveto{\pgfqpoint{6.224933in}{1.337269in}}{\pgfqpoint{6.222099in}{1.344111in}}{\pgfqpoint{6.217055in}{1.349155in}}%
\pgfpathcurveto{\pgfqpoint{6.212011in}{1.354198in}}{\pgfqpoint{6.205170in}{1.357032in}}{\pgfqpoint{6.198037in}{1.357032in}}%
\pgfpathcurveto{\pgfqpoint{6.190904in}{1.357032in}}{\pgfqpoint{6.184063in}{1.354198in}}{\pgfqpoint{6.179019in}{1.349155in}}%
\pgfpathcurveto{\pgfqpoint{6.173975in}{1.344111in}}{\pgfqpoint{6.171141in}{1.337269in}}{\pgfqpoint{6.171141in}{1.330137in}}%
\pgfpathcurveto{\pgfqpoint{6.171141in}{1.323004in}}{\pgfqpoint{6.173975in}{1.316162in}}{\pgfqpoint{6.179019in}{1.311118in}}%
\pgfpathcurveto{\pgfqpoint{6.184063in}{1.306075in}}{\pgfqpoint{6.190904in}{1.303241in}}{\pgfqpoint{6.198037in}{1.303241in}}%
\pgfpathclose%
\pgfusepath{stroke,fill}%
\end{pgfscope}%
\begin{pgfscope}%
\pgfpathrectangle{\pgfqpoint{4.985294in}{0.500000in}}{\pgfqpoint{1.764706in}{1.700000in}}%
\pgfusepath{clip}%
\pgfsetbuttcap%
\pgfsetroundjoin%
\definecolor{currentfill}{rgb}{0.969803,0.809811,0.702523}%
\pgfsetfillcolor{currentfill}%
\pgfsetlinewidth{0.311001pt}%
\definecolor{currentstroke}{rgb}{1.000000,1.000000,1.000000}%
\pgfsetstrokecolor{currentstroke}%
\pgfsetdash{}{0pt}%
\pgfpathmoveto{\pgfqpoint{6.209958in}{1.175170in}}%
\pgfpathcurveto{\pgfqpoint{6.217091in}{1.175170in}}{\pgfqpoint{6.223932in}{1.178004in}}{\pgfqpoint{6.228976in}{1.183047in}}%
\pgfpathcurveto{\pgfqpoint{6.234020in}{1.188091in}}{\pgfqpoint{6.236853in}{1.194933in}}{\pgfqpoint{6.236853in}{1.202065in}}%
\pgfpathcurveto{\pgfqpoint{6.236853in}{1.209198in}}{\pgfqpoint{6.234020in}{1.216040in}}{\pgfqpoint{6.228976in}{1.221083in}}%
\pgfpathcurveto{\pgfqpoint{6.223932in}{1.226127in}}{\pgfqpoint{6.217091in}{1.228961in}}{\pgfqpoint{6.209958in}{1.228961in}}%
\pgfpathcurveto{\pgfqpoint{6.202825in}{1.228961in}}{\pgfqpoint{6.195983in}{1.226127in}}{\pgfqpoint{6.190940in}{1.221083in}}%
\pgfpathcurveto{\pgfqpoint{6.185896in}{1.216040in}}{\pgfqpoint{6.183062in}{1.209198in}}{\pgfqpoint{6.183062in}{1.202065in}}%
\pgfpathcurveto{\pgfqpoint{6.183062in}{1.194933in}}{\pgfqpoint{6.185896in}{1.188091in}}{\pgfqpoint{6.190940in}{1.183047in}}%
\pgfpathcurveto{\pgfqpoint{6.195983in}{1.178004in}}{\pgfqpoint{6.202825in}{1.175170in}}{\pgfqpoint{6.209958in}{1.175170in}}%
\pgfpathclose%
\pgfusepath{stroke,fill}%
\end{pgfscope}%
\begin{pgfscope}%
\pgfpathrectangle{\pgfqpoint{4.985294in}{0.500000in}}{\pgfqpoint{1.764706in}{1.700000in}}%
\pgfusepath{clip}%
\pgfsetbuttcap%
\pgfsetroundjoin%
\definecolor{currentfill}{rgb}{0.966120,0.744512,0.608720}%
\pgfsetfillcolor{currentfill}%
\pgfsetlinewidth{0.311001pt}%
\definecolor{currentstroke}{rgb}{1.000000,1.000000,1.000000}%
\pgfsetstrokecolor{currentstroke}%
\pgfsetdash{}{0pt}%
\pgfpathmoveto{\pgfqpoint{5.597473in}{0.974133in}}%
\pgfpathcurveto{\pgfqpoint{5.604606in}{0.974133in}}{\pgfqpoint{5.611448in}{0.976967in}}{\pgfqpoint{5.616491in}{0.982011in}}%
\pgfpathcurveto{\pgfqpoint{5.621535in}{0.987055in}}{\pgfqpoint{5.624369in}{0.993896in}}{\pgfqpoint{5.624369in}{1.001029in}}%
\pgfpathcurveto{\pgfqpoint{5.624369in}{1.008162in}}{\pgfqpoint{5.621535in}{1.015004in}}{\pgfqpoint{5.616491in}{1.020047in}}%
\pgfpathcurveto{\pgfqpoint{5.611448in}{1.025091in}}{\pgfqpoint{5.604606in}{1.027925in}}{\pgfqpoint{5.597473in}{1.027925in}}%
\pgfpathcurveto{\pgfqpoint{5.590340in}{1.027925in}}{\pgfqpoint{5.583499in}{1.025091in}}{\pgfqpoint{5.578455in}{1.020047in}}%
\pgfpathcurveto{\pgfqpoint{5.573411in}{1.015004in}}{\pgfqpoint{5.570578in}{1.008162in}}{\pgfqpoint{5.570578in}{1.001029in}}%
\pgfpathcurveto{\pgfqpoint{5.570578in}{0.993896in}}{\pgfqpoint{5.573411in}{0.987055in}}{\pgfqpoint{5.578455in}{0.982011in}}%
\pgfpathcurveto{\pgfqpoint{5.583499in}{0.976967in}}{\pgfqpoint{5.590340in}{0.974133in}}{\pgfqpoint{5.597473in}{0.974133in}}%
\pgfpathclose%
\pgfusepath{stroke,fill}%
\end{pgfscope}%
\begin{pgfscope}%
\pgfpathrectangle{\pgfqpoint{4.985294in}{0.500000in}}{\pgfqpoint{1.764706in}{1.700000in}}%
\pgfusepath{clip}%
\pgfsetbuttcap%
\pgfsetroundjoin%
\definecolor{currentfill}{rgb}{0.966812,0.762584,0.633643}%
\pgfsetfillcolor{currentfill}%
\pgfsetlinewidth{0.311001pt}%
\definecolor{currentstroke}{rgb}{1.000000,1.000000,1.000000}%
\pgfsetstrokecolor{currentstroke}%
\pgfsetdash{}{0pt}%
\pgfpathmoveto{\pgfqpoint{5.514031in}{1.202247in}}%
\pgfpathcurveto{\pgfqpoint{5.521164in}{1.202247in}}{\pgfqpoint{5.528005in}{1.205081in}}{\pgfqpoint{5.533049in}{1.210124in}}%
\pgfpathcurveto{\pgfqpoint{5.538093in}{1.215168in}}{\pgfqpoint{5.540927in}{1.222010in}}{\pgfqpoint{5.540927in}{1.229143in}}%
\pgfpathcurveto{\pgfqpoint{5.540927in}{1.236275in}}{\pgfqpoint{5.538093in}{1.243117in}}{\pgfqpoint{5.533049in}{1.248161in}}%
\pgfpathcurveto{\pgfqpoint{5.528005in}{1.253204in}}{\pgfqpoint{5.521164in}{1.256038in}}{\pgfqpoint{5.514031in}{1.256038in}}%
\pgfpathcurveto{\pgfqpoint{5.506898in}{1.256038in}}{\pgfqpoint{5.500056in}{1.253204in}}{\pgfqpoint{5.495013in}{1.248161in}}%
\pgfpathcurveto{\pgfqpoint{5.489969in}{1.243117in}}{\pgfqpoint{5.487135in}{1.236275in}}{\pgfqpoint{5.487135in}{1.229143in}}%
\pgfpathcurveto{\pgfqpoint{5.487135in}{1.222010in}}{\pgfqpoint{5.489969in}{1.215168in}}{\pgfqpoint{5.495013in}{1.210124in}}%
\pgfpathcurveto{\pgfqpoint{5.500056in}{1.205081in}}{\pgfqpoint{5.506898in}{1.202247in}}{\pgfqpoint{5.514031in}{1.202247in}}%
\pgfpathclose%
\pgfusepath{stroke,fill}%
\end{pgfscope}%
\begin{pgfscope}%
\pgfpathrectangle{\pgfqpoint{4.985294in}{0.500000in}}{\pgfqpoint{1.764706in}{1.700000in}}%
\pgfusepath{clip}%
\pgfsetbuttcap%
\pgfsetroundjoin%
\definecolor{currentfill}{rgb}{0.976287,0.879862,0.805788}%
\pgfsetfillcolor{currentfill}%
\pgfsetlinewidth{0.311001pt}%
\definecolor{currentstroke}{rgb}{1.000000,1.000000,1.000000}%
\pgfsetstrokecolor{currentstroke}%
\pgfsetdash{}{0pt}%
\pgfpathmoveto{\pgfqpoint{6.276399in}{1.325424in}}%
\pgfpathcurveto{\pgfqpoint{6.283532in}{1.325424in}}{\pgfqpoint{6.290374in}{1.328258in}}{\pgfqpoint{6.295417in}{1.333302in}}%
\pgfpathcurveto{\pgfqpoint{6.300461in}{1.338345in}}{\pgfqpoint{6.303295in}{1.345187in}}{\pgfqpoint{6.303295in}{1.352320in}}%
\pgfpathcurveto{\pgfqpoint{6.303295in}{1.359453in}}{\pgfqpoint{6.300461in}{1.366294in}}{\pgfqpoint{6.295417in}{1.371338in}}%
\pgfpathcurveto{\pgfqpoint{6.290374in}{1.376382in}}{\pgfqpoint{6.283532in}{1.379216in}}{\pgfqpoint{6.276399in}{1.379216in}}%
\pgfpathcurveto{\pgfqpoint{6.269266in}{1.379216in}}{\pgfqpoint{6.262425in}{1.376382in}}{\pgfqpoint{6.257381in}{1.371338in}}%
\pgfpathcurveto{\pgfqpoint{6.252337in}{1.366294in}}{\pgfqpoint{6.249503in}{1.359453in}}{\pgfqpoint{6.249503in}{1.352320in}}%
\pgfpathcurveto{\pgfqpoint{6.249503in}{1.345187in}}{\pgfqpoint{6.252337in}{1.338345in}}{\pgfqpoint{6.257381in}{1.333302in}}%
\pgfpathcurveto{\pgfqpoint{6.262425in}{1.328258in}}{\pgfqpoint{6.269266in}{1.325424in}}{\pgfqpoint{6.276399in}{1.325424in}}%
\pgfpathclose%
\pgfusepath{stroke,fill}%
\end{pgfscope}%
\begin{pgfscope}%
\pgfpathrectangle{\pgfqpoint{4.985294in}{0.500000in}}{\pgfqpoint{1.764706in}{1.700000in}}%
\pgfusepath{clip}%
\pgfsetbuttcap%
\pgfsetroundjoin%
\definecolor{currentfill}{rgb}{0.969803,0.809811,0.702523}%
\pgfsetfillcolor{currentfill}%
\pgfsetlinewidth{0.311001pt}%
\definecolor{currentstroke}{rgb}{1.000000,1.000000,1.000000}%
\pgfsetstrokecolor{currentstroke}%
\pgfsetdash{}{0pt}%
\pgfpathmoveto{\pgfqpoint{6.259866in}{1.009534in}}%
\pgfpathcurveto{\pgfqpoint{6.266999in}{1.009534in}}{\pgfqpoint{6.273840in}{1.012368in}}{\pgfqpoint{6.278884in}{1.017411in}}%
\pgfpathcurveto{\pgfqpoint{6.283928in}{1.022455in}}{\pgfqpoint{6.286762in}{1.029297in}}{\pgfqpoint{6.286762in}{1.036429in}}%
\pgfpathcurveto{\pgfqpoint{6.286762in}{1.043562in}}{\pgfqpoint{6.283928in}{1.050404in}}{\pgfqpoint{6.278884in}{1.055447in}}%
\pgfpathcurveto{\pgfqpoint{6.273840in}{1.060491in}}{\pgfqpoint{6.266999in}{1.063325in}}{\pgfqpoint{6.259866in}{1.063325in}}%
\pgfpathcurveto{\pgfqpoint{6.252733in}{1.063325in}}{\pgfqpoint{6.245891in}{1.060491in}}{\pgfqpoint{6.240848in}{1.055447in}}%
\pgfpathcurveto{\pgfqpoint{6.235804in}{1.050404in}}{\pgfqpoint{6.232970in}{1.043562in}}{\pgfqpoint{6.232970in}{1.036429in}}%
\pgfpathcurveto{\pgfqpoint{6.232970in}{1.029297in}}{\pgfqpoint{6.235804in}{1.022455in}}{\pgfqpoint{6.240848in}{1.017411in}}%
\pgfpathcurveto{\pgfqpoint{6.245891in}{1.012368in}}{\pgfqpoint{6.252733in}{1.009534in}}{\pgfqpoint{6.259866in}{1.009534in}}%
\pgfpathclose%
\pgfusepath{stroke,fill}%
\end{pgfscope}%
\begin{pgfscope}%
\pgfpathrectangle{\pgfqpoint{4.985294in}{0.500000in}}{\pgfqpoint{1.764706in}{1.700000in}}%
\pgfusepath{clip}%
\pgfsetbuttcap%
\pgfsetroundjoin%
\definecolor{currentfill}{rgb}{0.924566,0.290534,0.242426}%
\pgfsetfillcolor{currentfill}%
\pgfsetlinewidth{0.311001pt}%
\definecolor{currentstroke}{rgb}{1.000000,1.000000,1.000000}%
\pgfsetstrokecolor{currentstroke}%
\pgfsetdash{}{0pt}%
\pgfpathmoveto{\pgfqpoint{6.119218in}{1.424566in}}%
\pgfpathcurveto{\pgfqpoint{6.126351in}{1.424566in}}{\pgfqpoint{6.133193in}{1.427400in}}{\pgfqpoint{6.138236in}{1.432444in}}%
\pgfpathcurveto{\pgfqpoint{6.143280in}{1.437487in}}{\pgfqpoint{6.146114in}{1.444329in}}{\pgfqpoint{6.146114in}{1.451462in}}%
\pgfpathcurveto{\pgfqpoint{6.146114in}{1.458595in}}{\pgfqpoint{6.143280in}{1.465436in}}{\pgfqpoint{6.138236in}{1.470480in}}%
\pgfpathcurveto{\pgfqpoint{6.133193in}{1.475524in}}{\pgfqpoint{6.126351in}{1.478357in}}{\pgfqpoint{6.119218in}{1.478357in}}%
\pgfpathcurveto{\pgfqpoint{6.112086in}{1.478357in}}{\pgfqpoint{6.105244in}{1.475524in}}{\pgfqpoint{6.100200in}{1.470480in}}%
\pgfpathcurveto{\pgfqpoint{6.095157in}{1.465436in}}{\pgfqpoint{6.092323in}{1.458595in}}{\pgfqpoint{6.092323in}{1.451462in}}%
\pgfpathcurveto{\pgfqpoint{6.092323in}{1.444329in}}{\pgfqpoint{6.095157in}{1.437487in}}{\pgfqpoint{6.100200in}{1.432444in}}%
\pgfpathcurveto{\pgfqpoint{6.105244in}{1.427400in}}{\pgfqpoint{6.112086in}{1.424566in}}{\pgfqpoint{6.119218in}{1.424566in}}%
\pgfpathclose%
\pgfusepath{stroke,fill}%
\end{pgfscope}%
\begin{pgfscope}%
\pgfpathrectangle{\pgfqpoint{4.985294in}{0.500000in}}{\pgfqpoint{1.764706in}{1.700000in}}%
\pgfusepath{clip}%
\pgfsetbuttcap%
\pgfsetroundjoin%
\definecolor{currentfill}{rgb}{0.979124,0.903132,0.839793}%
\pgfsetfillcolor{currentfill}%
\pgfsetlinewidth{0.311001pt}%
\definecolor{currentstroke}{rgb}{1.000000,1.000000,1.000000}%
\pgfsetstrokecolor{currentstroke}%
\pgfsetdash{}{0pt}%
\pgfpathmoveto{\pgfqpoint{5.395995in}{1.283215in}}%
\pgfpathcurveto{\pgfqpoint{5.403128in}{1.283215in}}{\pgfqpoint{5.409970in}{1.286049in}}{\pgfqpoint{5.415014in}{1.291092in}}%
\pgfpathcurveto{\pgfqpoint{5.420057in}{1.296136in}}{\pgfqpoint{5.422891in}{1.302978in}}{\pgfqpoint{5.422891in}{1.310110in}}%
\pgfpathcurveto{\pgfqpoint{5.422891in}{1.317243in}}{\pgfqpoint{5.420057in}{1.324085in}}{\pgfqpoint{5.415014in}{1.329129in}}%
\pgfpathcurveto{\pgfqpoint{5.409970in}{1.334172in}}{\pgfqpoint{5.403128in}{1.337006in}}{\pgfqpoint{5.395995in}{1.337006in}}%
\pgfpathcurveto{\pgfqpoint{5.388863in}{1.337006in}}{\pgfqpoint{5.382021in}{1.334172in}}{\pgfqpoint{5.376977in}{1.329129in}}%
\pgfpathcurveto{\pgfqpoint{5.371934in}{1.324085in}}{\pgfqpoint{5.369100in}{1.317243in}}{\pgfqpoint{5.369100in}{1.310110in}}%
\pgfpathcurveto{\pgfqpoint{5.369100in}{1.302978in}}{\pgfqpoint{5.371934in}{1.296136in}}{\pgfqpoint{5.376977in}{1.291092in}}%
\pgfpathcurveto{\pgfqpoint{5.382021in}{1.286049in}}{\pgfqpoint{5.388863in}{1.283215in}}{\pgfqpoint{5.395995in}{1.283215in}}%
\pgfpathclose%
\pgfusepath{stroke,fill}%
\end{pgfscope}%
\begin{pgfscope}%
\pgfpathrectangle{\pgfqpoint{4.985294in}{0.500000in}}{\pgfqpoint{1.764706in}{1.700000in}}%
\pgfusepath{clip}%
\pgfsetbuttcap%
\pgfsetroundjoin%
\definecolor{currentfill}{rgb}{0.973832,0.856556,0.771584}%
\pgfsetfillcolor{currentfill}%
\pgfsetlinewidth{0.311001pt}%
\definecolor{currentstroke}{rgb}{1.000000,1.000000,1.000000}%
\pgfsetstrokecolor{currentstroke}%
\pgfsetdash{}{0pt}%
\pgfpathmoveto{\pgfqpoint{6.205329in}{1.626373in}}%
\pgfpathcurveto{\pgfqpoint{6.212462in}{1.626373in}}{\pgfqpoint{6.219304in}{1.629207in}}{\pgfqpoint{6.224347in}{1.634251in}}%
\pgfpathcurveto{\pgfqpoint{6.229391in}{1.639294in}}{\pgfqpoint{6.232225in}{1.646136in}}{\pgfqpoint{6.232225in}{1.653269in}}%
\pgfpathcurveto{\pgfqpoint{6.232225in}{1.660402in}}{\pgfqpoint{6.229391in}{1.667243in}}{\pgfqpoint{6.224347in}{1.672287in}}%
\pgfpathcurveto{\pgfqpoint{6.219304in}{1.677331in}}{\pgfqpoint{6.212462in}{1.680164in}}{\pgfqpoint{6.205329in}{1.680164in}}%
\pgfpathcurveto{\pgfqpoint{6.198197in}{1.680164in}}{\pgfqpoint{6.191355in}{1.677331in}}{\pgfqpoint{6.186311in}{1.672287in}}%
\pgfpathcurveto{\pgfqpoint{6.181268in}{1.667243in}}{\pgfqpoint{6.178434in}{1.660402in}}{\pgfqpoint{6.178434in}{1.653269in}}%
\pgfpathcurveto{\pgfqpoint{6.178434in}{1.646136in}}{\pgfqpoint{6.181268in}{1.639294in}}{\pgfqpoint{6.186311in}{1.634251in}}%
\pgfpathcurveto{\pgfqpoint{6.191355in}{1.629207in}}{\pgfqpoint{6.198197in}{1.626373in}}{\pgfqpoint{6.205329in}{1.626373in}}%
\pgfpathclose%
\pgfusepath{stroke,fill}%
\end{pgfscope}%
\begin{pgfscope}%
\pgfpathrectangle{\pgfqpoint{4.985294in}{0.500000in}}{\pgfqpoint{1.764706in}{1.700000in}}%
\pgfusepath{clip}%
\pgfsetbuttcap%
\pgfsetroundjoin%
\definecolor{currentfill}{rgb}{0.966120,0.744512,0.608720}%
\pgfsetfillcolor{currentfill}%
\pgfsetlinewidth{0.311001pt}%
\definecolor{currentstroke}{rgb}{1.000000,1.000000,1.000000}%
\pgfsetstrokecolor{currentstroke}%
\pgfsetdash{}{0pt}%
\pgfpathmoveto{\pgfqpoint{6.141873in}{1.614025in}}%
\pgfpathcurveto{\pgfqpoint{6.149006in}{1.614025in}}{\pgfqpoint{6.155847in}{1.616859in}}{\pgfqpoint{6.160891in}{1.621903in}}%
\pgfpathcurveto{\pgfqpoint{6.165935in}{1.626947in}}{\pgfqpoint{6.168769in}{1.633788in}}{\pgfqpoint{6.168769in}{1.640921in}}%
\pgfpathcurveto{\pgfqpoint{6.168769in}{1.648054in}}{\pgfqpoint{6.165935in}{1.654896in}}{\pgfqpoint{6.160891in}{1.659939in}}%
\pgfpathcurveto{\pgfqpoint{6.155847in}{1.664983in}}{\pgfqpoint{6.149006in}{1.667817in}}{\pgfqpoint{6.141873in}{1.667817in}}%
\pgfpathcurveto{\pgfqpoint{6.134740in}{1.667817in}}{\pgfqpoint{6.127898in}{1.664983in}}{\pgfqpoint{6.122855in}{1.659939in}}%
\pgfpathcurveto{\pgfqpoint{6.117811in}{1.654896in}}{\pgfqpoint{6.114977in}{1.648054in}}{\pgfqpoint{6.114977in}{1.640921in}}%
\pgfpathcurveto{\pgfqpoint{6.114977in}{1.633788in}}{\pgfqpoint{6.117811in}{1.626947in}}{\pgfqpoint{6.122855in}{1.621903in}}%
\pgfpathcurveto{\pgfqpoint{6.127898in}{1.616859in}}{\pgfqpoint{6.134740in}{1.614025in}}{\pgfqpoint{6.141873in}{1.614025in}}%
\pgfpathclose%
\pgfusepath{stroke,fill}%
\end{pgfscope}%
\begin{pgfscope}%
\pgfpathrectangle{\pgfqpoint{4.985294in}{0.500000in}}{\pgfqpoint{1.764706in}{1.700000in}}%
\pgfusepath{clip}%
\pgfsetbuttcap%
\pgfsetroundjoin%
\definecolor{currentfill}{rgb}{0.965440,0.720101,0.576404}%
\pgfsetfillcolor{currentfill}%
\pgfsetlinewidth{0.311001pt}%
\definecolor{currentstroke}{rgb}{1.000000,1.000000,1.000000}%
\pgfsetstrokecolor{currentstroke}%
\pgfsetdash{}{0pt}%
\pgfpathmoveto{\pgfqpoint{5.550419in}{1.119344in}}%
\pgfpathcurveto{\pgfqpoint{5.557552in}{1.119344in}}{\pgfqpoint{5.564394in}{1.122178in}}{\pgfqpoint{5.569438in}{1.127222in}}%
\pgfpathcurveto{\pgfqpoint{5.574481in}{1.132265in}}{\pgfqpoint{5.577315in}{1.139107in}}{\pgfqpoint{5.577315in}{1.146240in}}%
\pgfpathcurveto{\pgfqpoint{5.577315in}{1.153373in}}{\pgfqpoint{5.574481in}{1.160214in}}{\pgfqpoint{5.569438in}{1.165258in}}%
\pgfpathcurveto{\pgfqpoint{5.564394in}{1.170302in}}{\pgfqpoint{5.557552in}{1.173136in}}{\pgfqpoint{5.550419in}{1.173136in}}%
\pgfpathcurveto{\pgfqpoint{5.543287in}{1.173136in}}{\pgfqpoint{5.536445in}{1.170302in}}{\pgfqpoint{5.531401in}{1.165258in}}%
\pgfpathcurveto{\pgfqpoint{5.526358in}{1.160214in}}{\pgfqpoint{5.523524in}{1.153373in}}{\pgfqpoint{5.523524in}{1.146240in}}%
\pgfpathcurveto{\pgfqpoint{5.523524in}{1.139107in}}{\pgfqpoint{5.526358in}{1.132265in}}{\pgfqpoint{5.531401in}{1.127222in}}%
\pgfpathcurveto{\pgfqpoint{5.536445in}{1.122178in}}{\pgfqpoint{5.543287in}{1.119344in}}{\pgfqpoint{5.550419in}{1.119344in}}%
\pgfpathclose%
\pgfusepath{stroke,fill}%
\end{pgfscope}%
\begin{pgfscope}%
\pgfpathrectangle{\pgfqpoint{4.985294in}{0.500000in}}{\pgfqpoint{1.764706in}{1.700000in}}%
\pgfusepath{clip}%
\pgfsetbuttcap%
\pgfsetroundjoin%
\definecolor{currentfill}{rgb}{0.976961,0.885681,0.814303}%
\pgfsetfillcolor{currentfill}%
\pgfsetlinewidth{0.311001pt}%
\definecolor{currentstroke}{rgb}{1.000000,1.000000,1.000000}%
\pgfsetstrokecolor{currentstroke}%
\pgfsetdash{}{0pt}%
\pgfpathmoveto{\pgfqpoint{5.462421in}{1.421390in}}%
\pgfpathcurveto{\pgfqpoint{5.469554in}{1.421390in}}{\pgfqpoint{5.476395in}{1.424224in}}{\pgfqpoint{5.481439in}{1.429268in}}%
\pgfpathcurveto{\pgfqpoint{5.486483in}{1.434312in}}{\pgfqpoint{5.489316in}{1.441153in}}{\pgfqpoint{5.489316in}{1.448286in}}%
\pgfpathcurveto{\pgfqpoint{5.489316in}{1.455419in}}{\pgfqpoint{5.486483in}{1.462261in}}{\pgfqpoint{5.481439in}{1.467304in}}%
\pgfpathcurveto{\pgfqpoint{5.476395in}{1.472348in}}{\pgfqpoint{5.469554in}{1.475182in}}{\pgfqpoint{5.462421in}{1.475182in}}%
\pgfpathcurveto{\pgfqpoint{5.455288in}{1.475182in}}{\pgfqpoint{5.448446in}{1.472348in}}{\pgfqpoint{5.443403in}{1.467304in}}%
\pgfpathcurveto{\pgfqpoint{5.438359in}{1.462261in}}{\pgfqpoint{5.435525in}{1.455419in}}{\pgfqpoint{5.435525in}{1.448286in}}%
\pgfpathcurveto{\pgfqpoint{5.435525in}{1.441153in}}{\pgfqpoint{5.438359in}{1.434312in}}{\pgfqpoint{5.443403in}{1.429268in}}%
\pgfpathcurveto{\pgfqpoint{5.448446in}{1.424224in}}{\pgfqpoint{5.455288in}{1.421390in}}{\pgfqpoint{5.462421in}{1.421390in}}%
\pgfpathclose%
\pgfusepath{stroke,fill}%
\end{pgfscope}%
\begin{pgfscope}%
\pgfpathrectangle{\pgfqpoint{4.985294in}{0.500000in}}{\pgfqpoint{1.764706in}{1.700000in}}%
\pgfusepath{clip}%
\pgfsetbuttcap%
\pgfsetroundjoin%
\definecolor{currentfill}{rgb}{0.965042,0.701564,0.552889}%
\pgfsetfillcolor{currentfill}%
\pgfsetlinewidth{0.311001pt}%
\definecolor{currentstroke}{rgb}{1.000000,1.000000,1.000000}%
\pgfsetstrokecolor{currentstroke}%
\pgfsetdash{}{0pt}%
\pgfpathmoveto{\pgfqpoint{5.585674in}{1.599159in}}%
\pgfpathcurveto{\pgfqpoint{5.592807in}{1.599159in}}{\pgfqpoint{5.599649in}{1.601993in}}{\pgfqpoint{5.604692in}{1.607036in}}%
\pgfpathcurveto{\pgfqpoint{5.609736in}{1.612080in}}{\pgfqpoint{5.612570in}{1.618922in}}{\pgfqpoint{5.612570in}{1.626054in}}%
\pgfpathcurveto{\pgfqpoint{5.612570in}{1.633187in}}{\pgfqpoint{5.609736in}{1.640029in}}{\pgfqpoint{5.604692in}{1.645073in}}%
\pgfpathcurveto{\pgfqpoint{5.599649in}{1.650116in}}{\pgfqpoint{5.592807in}{1.652950in}}{\pgfqpoint{5.585674in}{1.652950in}}%
\pgfpathcurveto{\pgfqpoint{5.578541in}{1.652950in}}{\pgfqpoint{5.571700in}{1.650116in}}{\pgfqpoint{5.566656in}{1.645073in}}%
\pgfpathcurveto{\pgfqpoint{5.561612in}{1.640029in}}{\pgfqpoint{5.558779in}{1.633187in}}{\pgfqpoint{5.558779in}{1.626054in}}%
\pgfpathcurveto{\pgfqpoint{5.558779in}{1.618922in}}{\pgfqpoint{5.561612in}{1.612080in}}{\pgfqpoint{5.566656in}{1.607036in}}%
\pgfpathcurveto{\pgfqpoint{5.571700in}{1.601993in}}{\pgfqpoint{5.578541in}{1.599159in}}{\pgfqpoint{5.585674in}{1.599159in}}%
\pgfpathclose%
\pgfusepath{stroke,fill}%
\end{pgfscope}%
\begin{pgfscope}%
\pgfpathrectangle{\pgfqpoint{4.985294in}{0.500000in}}{\pgfqpoint{1.764706in}{1.700000in}}%
\pgfusepath{clip}%
\pgfsetbuttcap%
\pgfsetroundjoin%
\definecolor{currentfill}{rgb}{0.980678,0.914765,0.856766}%
\pgfsetfillcolor{currentfill}%
\pgfsetlinewidth{0.311001pt}%
\definecolor{currentstroke}{rgb}{1.000000,1.000000,1.000000}%
\pgfsetstrokecolor{currentstroke}%
\pgfsetdash{}{0pt}%
\pgfpathmoveto{\pgfqpoint{5.406342in}{1.330383in}}%
\pgfpathcurveto{\pgfqpoint{5.413475in}{1.330383in}}{\pgfqpoint{5.420317in}{1.333217in}}{\pgfqpoint{5.425360in}{1.338261in}}%
\pgfpathcurveto{\pgfqpoint{5.430404in}{1.343304in}}{\pgfqpoint{5.433238in}{1.350146in}}{\pgfqpoint{5.433238in}{1.357279in}}%
\pgfpathcurveto{\pgfqpoint{5.433238in}{1.364412in}}{\pgfqpoint{5.430404in}{1.371253in}}{\pgfqpoint{5.425360in}{1.376297in}}%
\pgfpathcurveto{\pgfqpoint{5.420317in}{1.381341in}}{\pgfqpoint{5.413475in}{1.384174in}}{\pgfqpoint{5.406342in}{1.384174in}}%
\pgfpathcurveto{\pgfqpoint{5.399209in}{1.384174in}}{\pgfqpoint{5.392368in}{1.381341in}}{\pgfqpoint{5.387324in}{1.376297in}}%
\pgfpathcurveto{\pgfqpoint{5.382280in}{1.371253in}}{\pgfqpoint{5.379446in}{1.364412in}}{\pgfqpoint{5.379446in}{1.357279in}}%
\pgfpathcurveto{\pgfqpoint{5.379446in}{1.350146in}}{\pgfqpoint{5.382280in}{1.343304in}}{\pgfqpoint{5.387324in}{1.338261in}}%
\pgfpathcurveto{\pgfqpoint{5.392368in}{1.333217in}}{\pgfqpoint{5.399209in}{1.330383in}}{\pgfqpoint{5.406342in}{1.330383in}}%
\pgfpathclose%
\pgfusepath{stroke,fill}%
\end{pgfscope}%
\begin{pgfscope}%
\pgfpathrectangle{\pgfqpoint{4.985294in}{0.500000in}}{\pgfqpoint{1.764706in}{1.700000in}}%
\pgfusepath{clip}%
\pgfsetbuttcap%
\pgfsetroundjoin%
\definecolor{currentfill}{rgb}{0.966120,0.744512,0.608720}%
\pgfsetfillcolor{currentfill}%
\pgfsetlinewidth{0.311001pt}%
\definecolor{currentstroke}{rgb}{1.000000,1.000000,1.000000}%
\pgfsetstrokecolor{currentstroke}%
\pgfsetdash{}{0pt}%
\pgfpathmoveto{\pgfqpoint{6.282043in}{0.990537in}}%
\pgfpathcurveto{\pgfqpoint{6.289176in}{0.990537in}}{\pgfqpoint{6.296017in}{0.993371in}}{\pgfqpoint{6.301061in}{0.998414in}}%
\pgfpathcurveto{\pgfqpoint{6.306105in}{1.003458in}}{\pgfqpoint{6.308938in}{1.010300in}}{\pgfqpoint{6.308938in}{1.017433in}}%
\pgfpathcurveto{\pgfqpoint{6.308938in}{1.024565in}}{\pgfqpoint{6.306105in}{1.031407in}}{\pgfqpoint{6.301061in}{1.036451in}}%
\pgfpathcurveto{\pgfqpoint{6.296017in}{1.041494in}}{\pgfqpoint{6.289176in}{1.044328in}}{\pgfqpoint{6.282043in}{1.044328in}}%
\pgfpathcurveto{\pgfqpoint{6.274910in}{1.044328in}}{\pgfqpoint{6.268068in}{1.041494in}}{\pgfqpoint{6.263025in}{1.036451in}}%
\pgfpathcurveto{\pgfqpoint{6.257981in}{1.031407in}}{\pgfqpoint{6.255147in}{1.024565in}}{\pgfqpoint{6.255147in}{1.017433in}}%
\pgfpathcurveto{\pgfqpoint{6.255147in}{1.010300in}}{\pgfqpoint{6.257981in}{1.003458in}}{\pgfqpoint{6.263025in}{0.998414in}}%
\pgfpathcurveto{\pgfqpoint{6.268068in}{0.993371in}}{\pgfqpoint{6.274910in}{0.990537in}}{\pgfqpoint{6.282043in}{0.990537in}}%
\pgfpathclose%
\pgfusepath{stroke,fill}%
\end{pgfscope}%
\begin{pgfscope}%
\pgfpathrectangle{\pgfqpoint{4.985294in}{0.500000in}}{\pgfqpoint{1.764706in}{1.700000in}}%
\pgfusepath{clip}%
\pgfsetbuttcap%
\pgfsetroundjoin%
\definecolor{currentfill}{rgb}{0.980678,0.914765,0.856766}%
\pgfsetfillcolor{currentfill}%
\pgfsetlinewidth{0.311001pt}%
\definecolor{currentstroke}{rgb}{1.000000,1.000000,1.000000}%
\pgfsetstrokecolor{currentstroke}%
\pgfsetdash{}{0pt}%
\pgfpathmoveto{\pgfqpoint{6.310395in}{1.229204in}}%
\pgfpathcurveto{\pgfqpoint{6.317528in}{1.229204in}}{\pgfqpoint{6.324369in}{1.232038in}}{\pgfqpoint{6.329413in}{1.237082in}}%
\pgfpathcurveto{\pgfqpoint{6.334457in}{1.242126in}}{\pgfqpoint{6.337290in}{1.248967in}}{\pgfqpoint{6.337290in}{1.256100in}}%
\pgfpathcurveto{\pgfqpoint{6.337290in}{1.263233in}}{\pgfqpoint{6.334457in}{1.270075in}}{\pgfqpoint{6.329413in}{1.275118in}}%
\pgfpathcurveto{\pgfqpoint{6.324369in}{1.280162in}}{\pgfqpoint{6.317528in}{1.282996in}}{\pgfqpoint{6.310395in}{1.282996in}}%
\pgfpathcurveto{\pgfqpoint{6.303262in}{1.282996in}}{\pgfqpoint{6.296420in}{1.280162in}}{\pgfqpoint{6.291377in}{1.275118in}}%
\pgfpathcurveto{\pgfqpoint{6.286333in}{1.270075in}}{\pgfqpoint{6.283499in}{1.263233in}}{\pgfqpoint{6.283499in}{1.256100in}}%
\pgfpathcurveto{\pgfqpoint{6.283499in}{1.248967in}}{\pgfqpoint{6.286333in}{1.242126in}}{\pgfqpoint{6.291377in}{1.237082in}}%
\pgfpathcurveto{\pgfqpoint{6.296420in}{1.232038in}}{\pgfqpoint{6.303262in}{1.229204in}}{\pgfqpoint{6.310395in}{1.229204in}}%
\pgfpathclose%
\pgfusepath{stroke,fill}%
\end{pgfscope}%
\begin{pgfscope}%
\pgfpathrectangle{\pgfqpoint{4.985294in}{0.500000in}}{\pgfqpoint{1.764706in}{1.700000in}}%
\pgfusepath{clip}%
\pgfsetbuttcap%
\pgfsetroundjoin%
\definecolor{currentfill}{rgb}{0.924566,0.290534,0.242426}%
\pgfsetfillcolor{currentfill}%
\pgfsetlinewidth{0.311001pt}%
\definecolor{currentstroke}{rgb}{1.000000,1.000000,1.000000}%
\pgfsetstrokecolor{currentstroke}%
\pgfsetdash{}{0pt}%
\pgfpathmoveto{\pgfqpoint{5.698817in}{1.003893in}}%
\pgfpathcurveto{\pgfqpoint{5.705949in}{1.003893in}}{\pgfqpoint{5.712791in}{1.006727in}}{\pgfqpoint{5.717835in}{1.011771in}}%
\pgfpathcurveto{\pgfqpoint{5.722878in}{1.016815in}}{\pgfqpoint{5.725712in}{1.023656in}}{\pgfqpoint{5.725712in}{1.030789in}}%
\pgfpathcurveto{\pgfqpoint{5.725712in}{1.037922in}}{\pgfqpoint{5.722878in}{1.044764in}}{\pgfqpoint{5.717835in}{1.049807in}}%
\pgfpathcurveto{\pgfqpoint{5.712791in}{1.054851in}}{\pgfqpoint{5.705949in}{1.057685in}}{\pgfqpoint{5.698817in}{1.057685in}}%
\pgfpathcurveto{\pgfqpoint{5.691684in}{1.057685in}}{\pgfqpoint{5.684842in}{1.054851in}}{\pgfqpoint{5.679798in}{1.049807in}}%
\pgfpathcurveto{\pgfqpoint{5.674755in}{1.044764in}}{\pgfqpoint{5.671921in}{1.037922in}}{\pgfqpoint{5.671921in}{1.030789in}}%
\pgfpathcurveto{\pgfqpoint{5.671921in}{1.023656in}}{\pgfqpoint{5.674755in}{1.016815in}}{\pgfqpoint{5.679798in}{1.011771in}}%
\pgfpathcurveto{\pgfqpoint{5.684842in}{1.006727in}}{\pgfqpoint{5.691684in}{1.003893in}}{\pgfqpoint{5.698817in}{1.003893in}}%
\pgfpathclose%
\pgfusepath{stroke,fill}%
\end{pgfscope}%
\begin{pgfscope}%
\pgfpathrectangle{\pgfqpoint{4.985294in}{0.500000in}}{\pgfqpoint{1.764706in}{1.700000in}}%
\pgfusepath{clip}%
\pgfsetbuttcap%
\pgfsetroundjoin%
\definecolor{currentfill}{rgb}{0.965592,0.726236,0.584384}%
\pgfsetfillcolor{currentfill}%
\pgfsetlinewidth{0.311001pt}%
\definecolor{currentstroke}{rgb}{1.000000,1.000000,1.000000}%
\pgfsetstrokecolor{currentstroke}%
\pgfsetdash{}{0pt}%
\pgfpathmoveto{\pgfqpoint{6.159548in}{1.550533in}}%
\pgfpathcurveto{\pgfqpoint{6.166681in}{1.550533in}}{\pgfqpoint{6.173522in}{1.553367in}}{\pgfqpoint{6.178566in}{1.558411in}}%
\pgfpathcurveto{\pgfqpoint{6.183610in}{1.563454in}}{\pgfqpoint{6.186443in}{1.570296in}}{\pgfqpoint{6.186443in}{1.577429in}}%
\pgfpathcurveto{\pgfqpoint{6.186443in}{1.584562in}}{\pgfqpoint{6.183610in}{1.591403in}}{\pgfqpoint{6.178566in}{1.596447in}}%
\pgfpathcurveto{\pgfqpoint{6.173522in}{1.601491in}}{\pgfqpoint{6.166681in}{1.604324in}}{\pgfqpoint{6.159548in}{1.604324in}}%
\pgfpathcurveto{\pgfqpoint{6.152415in}{1.604324in}}{\pgfqpoint{6.145573in}{1.601491in}}{\pgfqpoint{6.140530in}{1.596447in}}%
\pgfpathcurveto{\pgfqpoint{6.135486in}{1.591403in}}{\pgfqpoint{6.132652in}{1.584562in}}{\pgfqpoint{6.132652in}{1.577429in}}%
\pgfpathcurveto{\pgfqpoint{6.132652in}{1.570296in}}{\pgfqpoint{6.135486in}{1.563454in}}{\pgfqpoint{6.140530in}{1.558411in}}%
\pgfpathcurveto{\pgfqpoint{6.145573in}{1.553367in}}{\pgfqpoint{6.152415in}{1.550533in}}{\pgfqpoint{6.159548in}{1.550533in}}%
\pgfpathclose%
\pgfusepath{stroke,fill}%
\end{pgfscope}%
\begin{pgfscope}%
\pgfpathrectangle{\pgfqpoint{4.985294in}{0.500000in}}{\pgfqpoint{1.764706in}{1.700000in}}%
\pgfusepath{clip}%
\pgfsetbuttcap%
\pgfsetroundjoin%
\definecolor{currentfill}{rgb}{0.975644,0.874038,0.797253}%
\pgfsetfillcolor{currentfill}%
\pgfsetlinewidth{0.311001pt}%
\definecolor{currentstroke}{rgb}{1.000000,1.000000,1.000000}%
\pgfsetstrokecolor{currentstroke}%
\pgfsetdash{}{0pt}%
\pgfpathmoveto{\pgfqpoint{5.477237in}{1.461212in}}%
\pgfpathcurveto{\pgfqpoint{5.484370in}{1.461212in}}{\pgfqpoint{5.491212in}{1.464046in}}{\pgfqpoint{5.496255in}{1.469090in}}%
\pgfpathcurveto{\pgfqpoint{5.501299in}{1.474133in}}{\pgfqpoint{5.504133in}{1.480975in}}{\pgfqpoint{5.504133in}{1.488108in}}%
\pgfpathcurveto{\pgfqpoint{5.504133in}{1.495241in}}{\pgfqpoint{5.501299in}{1.502082in}}{\pgfqpoint{5.496255in}{1.507126in}}%
\pgfpathcurveto{\pgfqpoint{5.491212in}{1.512170in}}{\pgfqpoint{5.484370in}{1.515004in}}{\pgfqpoint{5.477237in}{1.515004in}}%
\pgfpathcurveto{\pgfqpoint{5.470104in}{1.515004in}}{\pgfqpoint{5.463263in}{1.512170in}}{\pgfqpoint{5.458219in}{1.507126in}}%
\pgfpathcurveto{\pgfqpoint{5.453175in}{1.502082in}}{\pgfqpoint{5.450342in}{1.495241in}}{\pgfqpoint{5.450342in}{1.488108in}}%
\pgfpathcurveto{\pgfqpoint{5.450342in}{1.480975in}}{\pgfqpoint{5.453175in}{1.474133in}}{\pgfqpoint{5.458219in}{1.469090in}}%
\pgfpathcurveto{\pgfqpoint{5.463263in}{1.464046in}}{\pgfqpoint{5.470104in}{1.461212in}}{\pgfqpoint{5.477237in}{1.461212in}}%
\pgfpathclose%
\pgfusepath{stroke,fill}%
\end{pgfscope}%
\begin{pgfscope}%
\pgfpathrectangle{\pgfqpoint{4.985294in}{0.500000in}}{\pgfqpoint{1.764706in}{1.700000in}}%
\pgfusepath{clip}%
\pgfsetbuttcap%
\pgfsetroundjoin%
\definecolor{currentfill}{rgb}{0.975644,0.874038,0.797253}%
\pgfsetfillcolor{currentfill}%
\pgfsetlinewidth{0.311001pt}%
\definecolor{currentstroke}{rgb}{1.000000,1.000000,1.000000}%
\pgfsetstrokecolor{currentstroke}%
\pgfsetdash{}{0pt}%
\pgfpathmoveto{\pgfqpoint{5.474057in}{1.156828in}}%
\pgfpathcurveto{\pgfqpoint{5.481189in}{1.156828in}}{\pgfqpoint{5.488031in}{1.159661in}}{\pgfqpoint{5.493075in}{1.164705in}}%
\pgfpathcurveto{\pgfqpoint{5.498118in}{1.169749in}}{\pgfqpoint{5.500952in}{1.176590in}}{\pgfqpoint{5.500952in}{1.183723in}}%
\pgfpathcurveto{\pgfqpoint{5.500952in}{1.190856in}}{\pgfqpoint{5.498118in}{1.197698in}}{\pgfqpoint{5.493075in}{1.202741in}}%
\pgfpathcurveto{\pgfqpoint{5.488031in}{1.207785in}}{\pgfqpoint{5.481189in}{1.210619in}}{\pgfqpoint{5.474057in}{1.210619in}}%
\pgfpathcurveto{\pgfqpoint{5.466924in}{1.210619in}}{\pgfqpoint{5.460082in}{1.207785in}}{\pgfqpoint{5.455038in}{1.202741in}}%
\pgfpathcurveto{\pgfqpoint{5.449995in}{1.197698in}}{\pgfqpoint{5.447161in}{1.190856in}}{\pgfqpoint{5.447161in}{1.183723in}}%
\pgfpathcurveto{\pgfqpoint{5.447161in}{1.176590in}}{\pgfqpoint{5.449995in}{1.169749in}}{\pgfqpoint{5.455038in}{1.164705in}}%
\pgfpathcurveto{\pgfqpoint{5.460082in}{1.159661in}}{\pgfqpoint{5.466924in}{1.156828in}}{\pgfqpoint{5.474057in}{1.156828in}}%
\pgfpathclose%
\pgfusepath{stroke,fill}%
\end{pgfscope}%
\begin{pgfscope}%
\pgfpathrectangle{\pgfqpoint{4.985294in}{0.500000in}}{\pgfqpoint{1.764706in}{1.700000in}}%
\pgfusepath{clip}%
\pgfsetbuttcap%
\pgfsetroundjoin%
\definecolor{currentfill}{rgb}{0.962283,0.593046,0.431453}%
\pgfsetfillcolor{currentfill}%
\pgfsetlinewidth{0.311001pt}%
\definecolor{currentstroke}{rgb}{1.000000,1.000000,1.000000}%
\pgfsetstrokecolor{currentstroke}%
\pgfsetdash{}{0pt}%
\pgfpathmoveto{\pgfqpoint{5.328782in}{1.494704in}}%
\pgfpathcurveto{\pgfqpoint{5.335915in}{1.494704in}}{\pgfqpoint{5.342756in}{1.497537in}}{\pgfqpoint{5.347800in}{1.502581in}}%
\pgfpathcurveto{\pgfqpoint{5.352844in}{1.507625in}}{\pgfqpoint{5.355678in}{1.514466in}}{\pgfqpoint{5.355678in}{1.521599in}}%
\pgfpathcurveto{\pgfqpoint{5.355678in}{1.528732in}}{\pgfqpoint{5.352844in}{1.535574in}}{\pgfqpoint{5.347800in}{1.540617in}}%
\pgfpathcurveto{\pgfqpoint{5.342756in}{1.545661in}}{\pgfqpoint{5.335915in}{1.548495in}}{\pgfqpoint{5.328782in}{1.548495in}}%
\pgfpathcurveto{\pgfqpoint{5.321649in}{1.548495in}}{\pgfqpoint{5.314807in}{1.545661in}}{\pgfqpoint{5.309764in}{1.540617in}}%
\pgfpathcurveto{\pgfqpoint{5.304720in}{1.535574in}}{\pgfqpoint{5.301886in}{1.528732in}}{\pgfqpoint{5.301886in}{1.521599in}}%
\pgfpathcurveto{\pgfqpoint{5.301886in}{1.514466in}}{\pgfqpoint{5.304720in}{1.507625in}}{\pgfqpoint{5.309764in}{1.502581in}}%
\pgfpathcurveto{\pgfqpoint{5.314807in}{1.497537in}}{\pgfqpoint{5.321649in}{1.494704in}}{\pgfqpoint{5.328782in}{1.494704in}}%
\pgfpathclose%
\pgfusepath{stroke,fill}%
\end{pgfscope}%
\begin{pgfscope}%
\pgfpathrectangle{\pgfqpoint{4.985294in}{0.500000in}}{\pgfqpoint{1.764706in}{1.700000in}}%
\pgfusepath{clip}%
\pgfsetbuttcap%
\pgfsetroundjoin%
\definecolor{currentfill}{rgb}{0.964173,0.657587,0.500469}%
\pgfsetfillcolor{currentfill}%
\pgfsetlinewidth{0.311001pt}%
\definecolor{currentstroke}{rgb}{1.000000,1.000000,1.000000}%
\pgfsetstrokecolor{currentstroke}%
\pgfsetdash{}{0pt}%
\pgfpathmoveto{\pgfqpoint{6.382507in}{1.533471in}}%
\pgfpathcurveto{\pgfqpoint{6.389639in}{1.533471in}}{\pgfqpoint{6.396481in}{1.536305in}}{\pgfqpoint{6.401525in}{1.541349in}}%
\pgfpathcurveto{\pgfqpoint{6.406568in}{1.546392in}}{\pgfqpoint{6.409402in}{1.553234in}}{\pgfqpoint{6.409402in}{1.560367in}}%
\pgfpathcurveto{\pgfqpoint{6.409402in}{1.567500in}}{\pgfqpoint{6.406568in}{1.574341in}}{\pgfqpoint{6.401525in}{1.579385in}}%
\pgfpathcurveto{\pgfqpoint{6.396481in}{1.584429in}}{\pgfqpoint{6.389639in}{1.587262in}}{\pgfqpoint{6.382507in}{1.587262in}}%
\pgfpathcurveto{\pgfqpoint{6.375374in}{1.587262in}}{\pgfqpoint{6.368532in}{1.584429in}}{\pgfqpoint{6.363488in}{1.579385in}}%
\pgfpathcurveto{\pgfqpoint{6.358445in}{1.574341in}}{\pgfqpoint{6.355611in}{1.567500in}}{\pgfqpoint{6.355611in}{1.560367in}}%
\pgfpathcurveto{\pgfqpoint{6.355611in}{1.553234in}}{\pgfqpoint{6.358445in}{1.546392in}}{\pgfqpoint{6.363488in}{1.541349in}}%
\pgfpathcurveto{\pgfqpoint{6.368532in}{1.536305in}}{\pgfqpoint{6.375374in}{1.533471in}}{\pgfqpoint{6.382507in}{1.533471in}}%
\pgfpathclose%
\pgfusepath{stroke,fill}%
\end{pgfscope}%
\begin{pgfscope}%
\pgfpathrectangle{\pgfqpoint{4.985294in}{0.500000in}}{\pgfqpoint{1.764706in}{1.700000in}}%
\pgfusepath{clip}%
\pgfsetbuttcap%
\pgfsetroundjoin%
\definecolor{currentfill}{rgb}{0.962765,0.606121,0.444717}%
\pgfsetfillcolor{currentfill}%
\pgfsetlinewidth{0.311001pt}%
\definecolor{currentstroke}{rgb}{1.000000,1.000000,1.000000}%
\pgfsetstrokecolor{currentstroke}%
\pgfsetdash{}{0pt}%
\pgfpathmoveto{\pgfqpoint{5.513239in}{0.874053in}}%
\pgfpathcurveto{\pgfqpoint{5.520371in}{0.874053in}}{\pgfqpoint{5.527213in}{0.876887in}}{\pgfqpoint{5.532257in}{0.881931in}}%
\pgfpathcurveto{\pgfqpoint{5.537300in}{0.886975in}}{\pgfqpoint{5.540134in}{0.893816in}}{\pgfqpoint{5.540134in}{0.900949in}}%
\pgfpathcurveto{\pgfqpoint{5.540134in}{0.908082in}}{\pgfqpoint{5.537300in}{0.914924in}}{\pgfqpoint{5.532257in}{0.919967in}}%
\pgfpathcurveto{\pgfqpoint{5.527213in}{0.925011in}}{\pgfqpoint{5.520371in}{0.927845in}}{\pgfqpoint{5.513239in}{0.927845in}}%
\pgfpathcurveto{\pgfqpoint{5.506106in}{0.927845in}}{\pgfqpoint{5.499264in}{0.925011in}}{\pgfqpoint{5.494220in}{0.919967in}}%
\pgfpathcurveto{\pgfqpoint{5.489177in}{0.914924in}}{\pgfqpoint{5.486343in}{0.908082in}}{\pgfqpoint{5.486343in}{0.900949in}}%
\pgfpathcurveto{\pgfqpoint{5.486343in}{0.893816in}}{\pgfqpoint{5.489177in}{0.886975in}}{\pgfqpoint{5.494220in}{0.881931in}}%
\pgfpathcurveto{\pgfqpoint{5.499264in}{0.876887in}}{\pgfqpoint{5.506106in}{0.874053in}}{\pgfqpoint{5.513239in}{0.874053in}}%
\pgfpathclose%
\pgfusepath{stroke,fill}%
\end{pgfscope}%
\begin{pgfscope}%
\pgfpathrectangle{\pgfqpoint{4.985294in}{0.500000in}}{\pgfqpoint{1.764706in}{1.700000in}}%
\pgfusepath{clip}%
\pgfsetbuttcap%
\pgfsetroundjoin%
\definecolor{currentfill}{rgb}{0.962985,0.612625,0.451451}%
\pgfsetfillcolor{currentfill}%
\pgfsetlinewidth{0.311001pt}%
\definecolor{currentstroke}{rgb}{1.000000,1.000000,1.000000}%
\pgfsetstrokecolor{currentstroke}%
\pgfsetdash{}{0pt}%
\pgfpathmoveto{\pgfqpoint{6.153203in}{1.750951in}}%
\pgfpathcurveto{\pgfqpoint{6.160336in}{1.750951in}}{\pgfqpoint{6.167177in}{1.753785in}}{\pgfqpoint{6.172221in}{1.758828in}}%
\pgfpathcurveto{\pgfqpoint{6.177265in}{1.763872in}}{\pgfqpoint{6.180099in}{1.770714in}}{\pgfqpoint{6.180099in}{1.777847in}}%
\pgfpathcurveto{\pgfqpoint{6.180099in}{1.784979in}}{\pgfqpoint{6.177265in}{1.791821in}}{\pgfqpoint{6.172221in}{1.796865in}}%
\pgfpathcurveto{\pgfqpoint{6.167177in}{1.801908in}}{\pgfqpoint{6.160336in}{1.804742in}}{\pgfqpoint{6.153203in}{1.804742in}}%
\pgfpathcurveto{\pgfqpoint{6.146070in}{1.804742in}}{\pgfqpoint{6.139228in}{1.801908in}}{\pgfqpoint{6.134185in}{1.796865in}}%
\pgfpathcurveto{\pgfqpoint{6.129141in}{1.791821in}}{\pgfqpoint{6.126307in}{1.784979in}}{\pgfqpoint{6.126307in}{1.777847in}}%
\pgfpathcurveto{\pgfqpoint{6.126307in}{1.770714in}}{\pgfqpoint{6.129141in}{1.763872in}}{\pgfqpoint{6.134185in}{1.758828in}}%
\pgfpathcurveto{\pgfqpoint{6.139228in}{1.753785in}}{\pgfqpoint{6.146070in}{1.750951in}}{\pgfqpoint{6.153203in}{1.750951in}}%
\pgfpathclose%
\pgfusepath{stroke,fill}%
\end{pgfscope}%
\begin{pgfscope}%
\pgfpathrectangle{\pgfqpoint{4.985294in}{0.500000in}}{\pgfqpoint{1.764706in}{1.700000in}}%
\pgfusepath{clip}%
\pgfsetbuttcap%
\pgfsetroundjoin%
\definecolor{currentfill}{rgb}{0.969359,0.803954,0.693832}%
\pgfsetfillcolor{currentfill}%
\pgfsetlinewidth{0.311001pt}%
\definecolor{currentstroke}{rgb}{1.000000,1.000000,1.000000}%
\pgfsetstrokecolor{currentstroke}%
\pgfsetdash{}{0pt}%
\pgfpathmoveto{\pgfqpoint{6.195679in}{1.683871in}}%
\pgfpathcurveto{\pgfqpoint{6.202812in}{1.683871in}}{\pgfqpoint{6.209653in}{1.686705in}}{\pgfqpoint{6.214697in}{1.691749in}}%
\pgfpathcurveto{\pgfqpoint{6.219741in}{1.696792in}}{\pgfqpoint{6.222575in}{1.703634in}}{\pgfqpoint{6.222575in}{1.710767in}}%
\pgfpathcurveto{\pgfqpoint{6.222575in}{1.717900in}}{\pgfqpoint{6.219741in}{1.724741in}}{\pgfqpoint{6.214697in}{1.729785in}}%
\pgfpathcurveto{\pgfqpoint{6.209653in}{1.734829in}}{\pgfqpoint{6.202812in}{1.737663in}}{\pgfqpoint{6.195679in}{1.737663in}}%
\pgfpathcurveto{\pgfqpoint{6.188546in}{1.737663in}}{\pgfqpoint{6.181704in}{1.734829in}}{\pgfqpoint{6.176661in}{1.729785in}}%
\pgfpathcurveto{\pgfqpoint{6.171617in}{1.724741in}}{\pgfqpoint{6.168783in}{1.717900in}}{\pgfqpoint{6.168783in}{1.710767in}}%
\pgfpathcurveto{\pgfqpoint{6.168783in}{1.703634in}}{\pgfqpoint{6.171617in}{1.696792in}}{\pgfqpoint{6.176661in}{1.691749in}}%
\pgfpathcurveto{\pgfqpoint{6.181704in}{1.686705in}}{\pgfqpoint{6.188546in}{1.683871in}}{\pgfqpoint{6.195679in}{1.683871in}}%
\pgfpathclose%
\pgfusepath{stroke,fill}%
\end{pgfscope}%
\begin{pgfscope}%
\pgfpathrectangle{\pgfqpoint{4.985294in}{0.500000in}}{\pgfqpoint{1.764706in}{1.700000in}}%
\pgfusepath{clip}%
\pgfsetbuttcap%
\pgfsetroundjoin%
\definecolor{currentfill}{rgb}{0.964679,0.682838,0.530002}%
\pgfsetfillcolor{currentfill}%
\pgfsetlinewidth{0.311001pt}%
\definecolor{currentstroke}{rgb}{1.000000,1.000000,1.000000}%
\pgfsetstrokecolor{currentstroke}%
\pgfsetdash{}{0pt}%
\pgfpathmoveto{\pgfqpoint{5.528789in}{1.335958in}}%
\pgfpathcurveto{\pgfqpoint{5.535922in}{1.335958in}}{\pgfqpoint{5.542764in}{1.338792in}}{\pgfqpoint{5.547807in}{1.343836in}}%
\pgfpathcurveto{\pgfqpoint{5.552851in}{1.348879in}}{\pgfqpoint{5.555685in}{1.355721in}}{\pgfqpoint{5.555685in}{1.362854in}}%
\pgfpathcurveto{\pgfqpoint{5.555685in}{1.369987in}}{\pgfqpoint{5.552851in}{1.376828in}}{\pgfqpoint{5.547807in}{1.381872in}}%
\pgfpathcurveto{\pgfqpoint{5.542764in}{1.386916in}}{\pgfqpoint{5.535922in}{1.389750in}}{\pgfqpoint{5.528789in}{1.389750in}}%
\pgfpathcurveto{\pgfqpoint{5.521656in}{1.389750in}}{\pgfqpoint{5.514815in}{1.386916in}}{\pgfqpoint{5.509771in}{1.381872in}}%
\pgfpathcurveto{\pgfqpoint{5.504727in}{1.376828in}}{\pgfqpoint{5.501893in}{1.369987in}}{\pgfqpoint{5.501893in}{1.362854in}}%
\pgfpathcurveto{\pgfqpoint{5.501893in}{1.355721in}}{\pgfqpoint{5.504727in}{1.348879in}}{\pgfqpoint{5.509771in}{1.343836in}}%
\pgfpathcurveto{\pgfqpoint{5.514815in}{1.338792in}}{\pgfqpoint{5.521656in}{1.335958in}}{\pgfqpoint{5.528789in}{1.335958in}}%
\pgfpathclose%
\pgfusepath{stroke,fill}%
\end{pgfscope}%
\begin{pgfscope}%
\pgfpathrectangle{\pgfqpoint{4.985294in}{0.500000in}}{\pgfqpoint{1.764706in}{1.700000in}}%
\pgfusepath{clip}%
\pgfsetbuttcap%
\pgfsetroundjoin%
\definecolor{currentfill}{rgb}{0.975644,0.874038,0.797253}%
\pgfsetfillcolor{currentfill}%
\pgfsetlinewidth{0.311001pt}%
\definecolor{currentstroke}{rgb}{1.000000,1.000000,1.000000}%
\pgfsetstrokecolor{currentstroke}%
\pgfsetdash{}{0pt}%
\pgfpathmoveto{\pgfqpoint{6.349275in}{1.420637in}}%
\pgfpathcurveto{\pgfqpoint{6.356408in}{1.420637in}}{\pgfqpoint{6.363250in}{1.423471in}}{\pgfqpoint{6.368293in}{1.428515in}}%
\pgfpathcurveto{\pgfqpoint{6.373337in}{1.433558in}}{\pgfqpoint{6.376171in}{1.440400in}}{\pgfqpoint{6.376171in}{1.447533in}}%
\pgfpathcurveto{\pgfqpoint{6.376171in}{1.454665in}}{\pgfqpoint{6.373337in}{1.461507in}}{\pgfqpoint{6.368293in}{1.466551in}}%
\pgfpathcurveto{\pgfqpoint{6.363250in}{1.471594in}}{\pgfqpoint{6.356408in}{1.474428in}}{\pgfqpoint{6.349275in}{1.474428in}}%
\pgfpathcurveto{\pgfqpoint{6.342143in}{1.474428in}}{\pgfqpoint{6.335301in}{1.471594in}}{\pgfqpoint{6.330257in}{1.466551in}}%
\pgfpathcurveto{\pgfqpoint{6.325214in}{1.461507in}}{\pgfqpoint{6.322380in}{1.454665in}}{\pgfqpoint{6.322380in}{1.447533in}}%
\pgfpathcurveto{\pgfqpoint{6.322380in}{1.440400in}}{\pgfqpoint{6.325214in}{1.433558in}}{\pgfqpoint{6.330257in}{1.428515in}}%
\pgfpathcurveto{\pgfqpoint{6.335301in}{1.423471in}}{\pgfqpoint{6.342143in}{1.420637in}}{\pgfqpoint{6.349275in}{1.420637in}}%
\pgfpathclose%
\pgfusepath{stroke,fill}%
\end{pgfscope}%
\begin{pgfscope}%
\pgfpathrectangle{\pgfqpoint{4.985294in}{0.500000in}}{\pgfqpoint{1.764706in}{1.700000in}}%
\pgfusepath{clip}%
\pgfsetbuttcap%
\pgfsetroundjoin%
\definecolor{currentfill}{rgb}{0.959229,0.533075,0.374889}%
\pgfsetfillcolor{currentfill}%
\pgfsetlinewidth{0.311001pt}%
\definecolor{currentstroke}{rgb}{1.000000,1.000000,1.000000}%
\pgfsetstrokecolor{currentstroke}%
\pgfsetdash{}{0pt}%
\pgfpathmoveto{\pgfqpoint{6.416531in}{1.494321in}}%
\pgfpathcurveto{\pgfqpoint{6.423663in}{1.494321in}}{\pgfqpoint{6.430505in}{1.497155in}}{\pgfqpoint{6.435549in}{1.502198in}}%
\pgfpathcurveto{\pgfqpoint{6.440592in}{1.507242in}}{\pgfqpoint{6.443426in}{1.514084in}}{\pgfqpoint{6.443426in}{1.521217in}}%
\pgfpathcurveto{\pgfqpoint{6.443426in}{1.528349in}}{\pgfqpoint{6.440592in}{1.535191in}}{\pgfqpoint{6.435549in}{1.540235in}}%
\pgfpathcurveto{\pgfqpoint{6.430505in}{1.545278in}}{\pgfqpoint{6.423663in}{1.548112in}}{\pgfqpoint{6.416531in}{1.548112in}}%
\pgfpathcurveto{\pgfqpoint{6.409398in}{1.548112in}}{\pgfqpoint{6.402556in}{1.545278in}}{\pgfqpoint{6.397512in}{1.540235in}}%
\pgfpathcurveto{\pgfqpoint{6.392469in}{1.535191in}}{\pgfqpoint{6.389635in}{1.528349in}}{\pgfqpoint{6.389635in}{1.521217in}}%
\pgfpathcurveto{\pgfqpoint{6.389635in}{1.514084in}}{\pgfqpoint{6.392469in}{1.507242in}}{\pgfqpoint{6.397512in}{1.502198in}}%
\pgfpathcurveto{\pgfqpoint{6.402556in}{1.497155in}}{\pgfqpoint{6.409398in}{1.494321in}}{\pgfqpoint{6.416531in}{1.494321in}}%
\pgfpathclose%
\pgfusepath{stroke,fill}%
\end{pgfscope}%
\begin{pgfscope}%
\pgfpathrectangle{\pgfqpoint{4.985294in}{0.500000in}}{\pgfqpoint{1.764706in}{1.700000in}}%
\pgfusepath{clip}%
\pgfsetbuttcap%
\pgfsetroundjoin%
\definecolor{currentfill}{rgb}{0.966120,0.744512,0.608720}%
\pgfsetfillcolor{currentfill}%
\pgfsetlinewidth{0.311001pt}%
\definecolor{currentstroke}{rgb}{1.000000,1.000000,1.000000}%
\pgfsetstrokecolor{currentstroke}%
\pgfsetdash{}{0pt}%
\pgfpathmoveto{\pgfqpoint{5.571620in}{1.600180in}}%
\pgfpathcurveto{\pgfqpoint{5.578752in}{1.600180in}}{\pgfqpoint{5.585594in}{1.603014in}}{\pgfqpoint{5.590638in}{1.608058in}}%
\pgfpathcurveto{\pgfqpoint{5.595681in}{1.613102in}}{\pgfqpoint{5.598515in}{1.619943in}}{\pgfqpoint{5.598515in}{1.627076in}}%
\pgfpathcurveto{\pgfqpoint{5.598515in}{1.634209in}}{\pgfqpoint{5.595681in}{1.641050in}}{\pgfqpoint{5.590638in}{1.646094in}}%
\pgfpathcurveto{\pgfqpoint{5.585594in}{1.651138in}}{\pgfqpoint{5.578752in}{1.653972in}}{\pgfqpoint{5.571620in}{1.653972in}}%
\pgfpathcurveto{\pgfqpoint{5.564487in}{1.653972in}}{\pgfqpoint{5.557645in}{1.651138in}}{\pgfqpoint{5.552601in}{1.646094in}}%
\pgfpathcurveto{\pgfqpoint{5.547558in}{1.641050in}}{\pgfqpoint{5.544724in}{1.634209in}}{\pgfqpoint{5.544724in}{1.627076in}}%
\pgfpathcurveto{\pgfqpoint{5.544724in}{1.619943in}}{\pgfqpoint{5.547558in}{1.613102in}}{\pgfqpoint{5.552601in}{1.608058in}}%
\pgfpathcurveto{\pgfqpoint{5.557645in}{1.603014in}}{\pgfqpoint{5.564487in}{1.600180in}}{\pgfqpoint{5.571620in}{1.600180in}}%
\pgfpathclose%
\pgfusepath{stroke,fill}%
\end{pgfscope}%
\begin{pgfscope}%
\pgfpathrectangle{\pgfqpoint{4.985294in}{0.500000in}}{\pgfqpoint{1.764706in}{1.700000in}}%
\pgfusepath{clip}%
\pgfsetbuttcap%
\pgfsetroundjoin%
\definecolor{currentfill}{rgb}{0.966812,0.762584,0.633643}%
\pgfsetfillcolor{currentfill}%
\pgfsetlinewidth{0.311001pt}%
\definecolor{currentstroke}{rgb}{1.000000,1.000000,1.000000}%
\pgfsetstrokecolor{currentstroke}%
\pgfsetdash{}{0pt}%
\pgfpathmoveto{\pgfqpoint{5.470802in}{0.953445in}}%
\pgfpathcurveto{\pgfqpoint{5.477934in}{0.953445in}}{\pgfqpoint{5.484776in}{0.956278in}}{\pgfqpoint{5.489820in}{0.961322in}}%
\pgfpathcurveto{\pgfqpoint{5.494863in}{0.966366in}}{\pgfqpoint{5.497697in}{0.973207in}}{\pgfqpoint{5.497697in}{0.980340in}}%
\pgfpathcurveto{\pgfqpoint{5.497697in}{0.987473in}}{\pgfqpoint{5.494863in}{0.994315in}}{\pgfqpoint{5.489820in}{0.999358in}}%
\pgfpathcurveto{\pgfqpoint{5.484776in}{1.004402in}}{\pgfqpoint{5.477934in}{1.007236in}}{\pgfqpoint{5.470802in}{1.007236in}}%
\pgfpathcurveto{\pgfqpoint{5.463669in}{1.007236in}}{\pgfqpoint{5.456827in}{1.004402in}}{\pgfqpoint{5.451784in}{0.999358in}}%
\pgfpathcurveto{\pgfqpoint{5.446740in}{0.994315in}}{\pgfqpoint{5.443906in}{0.987473in}}{\pgfqpoint{5.443906in}{0.980340in}}%
\pgfpathcurveto{\pgfqpoint{5.443906in}{0.973207in}}{\pgfqpoint{5.446740in}{0.966366in}}{\pgfqpoint{5.451784in}{0.961322in}}%
\pgfpathcurveto{\pgfqpoint{5.456827in}{0.956278in}}{\pgfqpoint{5.463669in}{0.953445in}}{\pgfqpoint{5.470802in}{0.953445in}}%
\pgfpathclose%
\pgfusepath{stroke,fill}%
\end{pgfscope}%
\begin{pgfscope}%
\pgfpathrectangle{\pgfqpoint{4.985294in}{0.500000in}}{\pgfqpoint{1.764706in}{1.700000in}}%
\pgfusepath{clip}%
\pgfsetbuttcap%
\pgfsetroundjoin%
\definecolor{currentfill}{rgb}{0.975644,0.874038,0.797253}%
\pgfsetfillcolor{currentfill}%
\pgfsetlinewidth{0.311001pt}%
\definecolor{currentstroke}{rgb}{1.000000,1.000000,1.000000}%
\pgfsetstrokecolor{currentstroke}%
\pgfsetdash{}{0pt}%
\pgfpathmoveto{\pgfqpoint{6.265476in}{1.473108in}}%
\pgfpathcurveto{\pgfqpoint{6.272608in}{1.473108in}}{\pgfqpoint{6.279450in}{1.475942in}}{\pgfqpoint{6.284494in}{1.480985in}}%
\pgfpathcurveto{\pgfqpoint{6.289537in}{1.486029in}}{\pgfqpoint{6.292371in}{1.492871in}}{\pgfqpoint{6.292371in}{1.500003in}}%
\pgfpathcurveto{\pgfqpoint{6.292371in}{1.507136in}}{\pgfqpoint{6.289537in}{1.513978in}}{\pgfqpoint{6.284494in}{1.519022in}}%
\pgfpathcurveto{\pgfqpoint{6.279450in}{1.524065in}}{\pgfqpoint{6.272608in}{1.526899in}}{\pgfqpoint{6.265476in}{1.526899in}}%
\pgfpathcurveto{\pgfqpoint{6.258343in}{1.526899in}}{\pgfqpoint{6.251501in}{1.524065in}}{\pgfqpoint{6.246457in}{1.519022in}}%
\pgfpathcurveto{\pgfqpoint{6.241414in}{1.513978in}}{\pgfqpoint{6.238580in}{1.507136in}}{\pgfqpoint{6.238580in}{1.500003in}}%
\pgfpathcurveto{\pgfqpoint{6.238580in}{1.492871in}}{\pgfqpoint{6.241414in}{1.486029in}}{\pgfqpoint{6.246457in}{1.480985in}}%
\pgfpathcurveto{\pgfqpoint{6.251501in}{1.475942in}}{\pgfqpoint{6.258343in}{1.473108in}}{\pgfqpoint{6.265476in}{1.473108in}}%
\pgfpathclose%
\pgfusepath{stroke,fill}%
\end{pgfscope}%
\begin{pgfscope}%
\pgfpathrectangle{\pgfqpoint{4.985294in}{0.500000in}}{\pgfqpoint{1.764706in}{1.700000in}}%
\pgfusepath{clip}%
\pgfsetbuttcap%
\pgfsetroundjoin%
\definecolor{currentfill}{rgb}{0.978376,0.897317,0.831308}%
\pgfsetfillcolor{currentfill}%
\pgfsetlinewidth{0.311001pt}%
\definecolor{currentstroke}{rgb}{1.000000,1.000000,1.000000}%
\pgfsetstrokecolor{currentstroke}%
\pgfsetdash{}{0pt}%
\pgfpathmoveto{\pgfqpoint{6.336360in}{1.416217in}}%
\pgfpathcurveto{\pgfqpoint{6.343493in}{1.416217in}}{\pgfqpoint{6.350334in}{1.419051in}}{\pgfqpoint{6.355378in}{1.424094in}}%
\pgfpathcurveto{\pgfqpoint{6.360422in}{1.429138in}}{\pgfqpoint{6.363256in}{1.435980in}}{\pgfqpoint{6.363256in}{1.443112in}}%
\pgfpathcurveto{\pgfqpoint{6.363256in}{1.450245in}}{\pgfqpoint{6.360422in}{1.457087in}}{\pgfqpoint{6.355378in}{1.462131in}}%
\pgfpathcurveto{\pgfqpoint{6.350334in}{1.467174in}}{\pgfqpoint{6.343493in}{1.470008in}}{\pgfqpoint{6.336360in}{1.470008in}}%
\pgfpathcurveto{\pgfqpoint{6.329227in}{1.470008in}}{\pgfqpoint{6.322385in}{1.467174in}}{\pgfqpoint{6.317342in}{1.462131in}}%
\pgfpathcurveto{\pgfqpoint{6.312298in}{1.457087in}}{\pgfqpoint{6.309464in}{1.450245in}}{\pgfqpoint{6.309464in}{1.443112in}}%
\pgfpathcurveto{\pgfqpoint{6.309464in}{1.435980in}}{\pgfqpoint{6.312298in}{1.429138in}}{\pgfqpoint{6.317342in}{1.424094in}}%
\pgfpathcurveto{\pgfqpoint{6.322385in}{1.419051in}}{\pgfqpoint{6.329227in}{1.416217in}}{\pgfqpoint{6.336360in}{1.416217in}}%
\pgfpathclose%
\pgfusepath{stroke,fill}%
\end{pgfscope}%
\begin{pgfscope}%
\pgfpathrectangle{\pgfqpoint{4.985294in}{0.500000in}}{\pgfqpoint{1.764706in}{1.700000in}}%
\pgfusepath{clip}%
\pgfsetbuttcap%
\pgfsetroundjoin%
\definecolor{currentfill}{rgb}{0.974412,0.862387,0.780156}%
\pgfsetfillcolor{currentfill}%
\pgfsetlinewidth{0.311001pt}%
\definecolor{currentstroke}{rgb}{1.000000,1.000000,1.000000}%
\pgfsetstrokecolor{currentstroke}%
\pgfsetdash{}{0pt}%
\pgfpathmoveto{\pgfqpoint{5.493139in}{1.513955in}}%
\pgfpathcurveto{\pgfqpoint{5.500272in}{1.513955in}}{\pgfqpoint{5.507114in}{1.516788in}}{\pgfqpoint{5.512157in}{1.521832in}}%
\pgfpathcurveto{\pgfqpoint{5.517201in}{1.526876in}}{\pgfqpoint{5.520035in}{1.533717in}}{\pgfqpoint{5.520035in}{1.540850in}}%
\pgfpathcurveto{\pgfqpoint{5.520035in}{1.547983in}}{\pgfqpoint{5.517201in}{1.554825in}}{\pgfqpoint{5.512157in}{1.559868in}}%
\pgfpathcurveto{\pgfqpoint{5.507114in}{1.564912in}}{\pgfqpoint{5.500272in}{1.567746in}}{\pgfqpoint{5.493139in}{1.567746in}}%
\pgfpathcurveto{\pgfqpoint{5.486006in}{1.567746in}}{\pgfqpoint{5.479165in}{1.564912in}}{\pgfqpoint{5.474121in}{1.559868in}}%
\pgfpathcurveto{\pgfqpoint{5.469077in}{1.554825in}}{\pgfqpoint{5.466244in}{1.547983in}}{\pgfqpoint{5.466244in}{1.540850in}}%
\pgfpathcurveto{\pgfqpoint{5.466244in}{1.533717in}}{\pgfqpoint{5.469077in}{1.526876in}}{\pgfqpoint{5.474121in}{1.521832in}}%
\pgfpathcurveto{\pgfqpoint{5.479165in}{1.516788in}}{\pgfqpoint{5.486006in}{1.513955in}}{\pgfqpoint{5.493139in}{1.513955in}}%
\pgfpathclose%
\pgfusepath{stroke,fill}%
\end{pgfscope}%
\begin{pgfscope}%
\pgfpathrectangle{\pgfqpoint{4.985294in}{0.500000in}}{\pgfqpoint{1.764706in}{1.700000in}}%
\pgfusepath{clip}%
\pgfsetbuttcap%
\pgfsetroundjoin%
\definecolor{currentfill}{rgb}{0.964433,0.670254,0.515093}%
\pgfsetfillcolor{currentfill}%
\pgfsetlinewidth{0.311001pt}%
\definecolor{currentstroke}{rgb}{1.000000,1.000000,1.000000}%
\pgfsetstrokecolor{currentstroke}%
\pgfsetdash{}{0pt}%
\pgfpathmoveto{\pgfqpoint{6.358459in}{1.583327in}}%
\pgfpathcurveto{\pgfqpoint{6.365592in}{1.583327in}}{\pgfqpoint{6.372433in}{1.586161in}}{\pgfqpoint{6.377477in}{1.591205in}}%
\pgfpathcurveto{\pgfqpoint{6.382521in}{1.596248in}}{\pgfqpoint{6.385354in}{1.603090in}}{\pgfqpoint{6.385354in}{1.610223in}}%
\pgfpathcurveto{\pgfqpoint{6.385354in}{1.617356in}}{\pgfqpoint{6.382521in}{1.624197in}}{\pgfqpoint{6.377477in}{1.629241in}}%
\pgfpathcurveto{\pgfqpoint{6.372433in}{1.634285in}}{\pgfqpoint{6.365592in}{1.637118in}}{\pgfqpoint{6.358459in}{1.637118in}}%
\pgfpathcurveto{\pgfqpoint{6.351326in}{1.637118in}}{\pgfqpoint{6.344484in}{1.634285in}}{\pgfqpoint{6.339441in}{1.629241in}}%
\pgfpathcurveto{\pgfqpoint{6.334397in}{1.624197in}}{\pgfqpoint{6.331563in}{1.617356in}}{\pgfqpoint{6.331563in}{1.610223in}}%
\pgfpathcurveto{\pgfqpoint{6.331563in}{1.603090in}}{\pgfqpoint{6.334397in}{1.596248in}}{\pgfqpoint{6.339441in}{1.591205in}}%
\pgfpathcurveto{\pgfqpoint{6.344484in}{1.586161in}}{\pgfqpoint{6.351326in}{1.583327in}}{\pgfqpoint{6.358459in}{1.583327in}}%
\pgfpathclose%
\pgfusepath{stroke,fill}%
\end{pgfscope}%
\begin{pgfscope}%
\pgfpathrectangle{\pgfqpoint{4.985294in}{0.500000in}}{\pgfqpoint{1.764706in}{1.700000in}}%
\pgfusepath{clip}%
\pgfsetbuttcap%
\pgfsetroundjoin%
\definecolor{currentfill}{rgb}{0.968105,0.786346,0.667739}%
\pgfsetfillcolor{currentfill}%
\pgfsetlinewidth{0.311001pt}%
\definecolor{currentstroke}{rgb}{1.000000,1.000000,1.000000}%
\pgfsetstrokecolor{currentstroke}%
\pgfsetdash{}{0pt}%
\pgfpathmoveto{\pgfqpoint{5.525704in}{1.664115in}}%
\pgfpathcurveto{\pgfqpoint{5.532837in}{1.664115in}}{\pgfqpoint{5.539679in}{1.666948in}}{\pgfqpoint{5.544723in}{1.671992in}}%
\pgfpathcurveto{\pgfqpoint{5.549766in}{1.677036in}}{\pgfqpoint{5.552600in}{1.683877in}}{\pgfqpoint{5.552600in}{1.691010in}}%
\pgfpathcurveto{\pgfqpoint{5.552600in}{1.698143in}}{\pgfqpoint{5.549766in}{1.704985in}}{\pgfqpoint{5.544723in}{1.710028in}}%
\pgfpathcurveto{\pgfqpoint{5.539679in}{1.715072in}}{\pgfqpoint{5.532837in}{1.717906in}}{\pgfqpoint{5.525704in}{1.717906in}}%
\pgfpathcurveto{\pgfqpoint{5.518572in}{1.717906in}}{\pgfqpoint{5.511730in}{1.715072in}}{\pgfqpoint{5.506686in}{1.710028in}}%
\pgfpathcurveto{\pgfqpoint{5.501643in}{1.704985in}}{\pgfqpoint{5.498809in}{1.698143in}}{\pgfqpoint{5.498809in}{1.691010in}}%
\pgfpathcurveto{\pgfqpoint{5.498809in}{1.683877in}}{\pgfqpoint{5.501643in}{1.677036in}}{\pgfqpoint{5.506686in}{1.671992in}}%
\pgfpathcurveto{\pgfqpoint{5.511730in}{1.666948in}}{\pgfqpoint{5.518572in}{1.664115in}}{\pgfqpoint{5.525704in}{1.664115in}}%
\pgfpathclose%
\pgfusepath{stroke,fill}%
\end{pgfscope}%
\begin{pgfscope}%
\pgfpathrectangle{\pgfqpoint{4.985294in}{0.500000in}}{\pgfqpoint{1.764706in}{1.700000in}}%
\pgfusepath{clip}%
\pgfsetbuttcap%
\pgfsetroundjoin%
\definecolor{currentfill}{rgb}{0.969803,0.809811,0.702523}%
\pgfsetfillcolor{currentfill}%
\pgfsetlinewidth{0.311001pt}%
\definecolor{currentstroke}{rgb}{1.000000,1.000000,1.000000}%
\pgfsetstrokecolor{currentstroke}%
\pgfsetdash{}{0pt}%
\pgfpathmoveto{\pgfqpoint{6.186751in}{1.087801in}}%
\pgfpathcurveto{\pgfqpoint{6.193884in}{1.087801in}}{\pgfqpoint{6.200726in}{1.090635in}}{\pgfqpoint{6.205770in}{1.095678in}}%
\pgfpathcurveto{\pgfqpoint{6.210813in}{1.100722in}}{\pgfqpoint{6.213647in}{1.107564in}}{\pgfqpoint{6.213647in}{1.114696in}}%
\pgfpathcurveto{\pgfqpoint{6.213647in}{1.121829in}}{\pgfqpoint{6.210813in}{1.128671in}}{\pgfqpoint{6.205770in}{1.133715in}}%
\pgfpathcurveto{\pgfqpoint{6.200726in}{1.138758in}}{\pgfqpoint{6.193884in}{1.141592in}}{\pgfqpoint{6.186751in}{1.141592in}}%
\pgfpathcurveto{\pgfqpoint{6.179619in}{1.141592in}}{\pgfqpoint{6.172777in}{1.138758in}}{\pgfqpoint{6.167733in}{1.133715in}}%
\pgfpathcurveto{\pgfqpoint{6.162690in}{1.128671in}}{\pgfqpoint{6.159856in}{1.121829in}}{\pgfqpoint{6.159856in}{1.114696in}}%
\pgfpathcurveto{\pgfqpoint{6.159856in}{1.107564in}}{\pgfqpoint{6.162690in}{1.100722in}}{\pgfqpoint{6.167733in}{1.095678in}}%
\pgfpathcurveto{\pgfqpoint{6.172777in}{1.090635in}}{\pgfqpoint{6.179619in}{1.087801in}}{\pgfqpoint{6.186751in}{1.087801in}}%
\pgfpathclose%
\pgfusepath{stroke,fill}%
\end{pgfscope}%
\begin{pgfscope}%
\pgfpathrectangle{\pgfqpoint{4.985294in}{0.500000in}}{\pgfqpoint{1.764706in}{1.700000in}}%
\pgfusepath{clip}%
\pgfsetbuttcap%
\pgfsetroundjoin%
\definecolor{currentfill}{rgb}{0.966812,0.762584,0.633643}%
\pgfsetfillcolor{currentfill}%
\pgfsetlinewidth{0.311001pt}%
\definecolor{currentstroke}{rgb}{1.000000,1.000000,1.000000}%
\pgfsetstrokecolor{currentstroke}%
\pgfsetdash{}{0pt}%
\pgfpathmoveto{\pgfqpoint{5.539863in}{1.676056in}}%
\pgfpathcurveto{\pgfqpoint{5.546995in}{1.676056in}}{\pgfqpoint{5.553837in}{1.678890in}}{\pgfqpoint{5.558881in}{1.683934in}}%
\pgfpathcurveto{\pgfqpoint{5.563924in}{1.688977in}}{\pgfqpoint{5.566758in}{1.695819in}}{\pgfqpoint{5.566758in}{1.702952in}}%
\pgfpathcurveto{\pgfqpoint{5.566758in}{1.710085in}}{\pgfqpoint{5.563924in}{1.716926in}}{\pgfqpoint{5.558881in}{1.721970in}}%
\pgfpathcurveto{\pgfqpoint{5.553837in}{1.727014in}}{\pgfqpoint{5.546995in}{1.729847in}}{\pgfqpoint{5.539863in}{1.729847in}}%
\pgfpathcurveto{\pgfqpoint{5.532730in}{1.729847in}}{\pgfqpoint{5.525888in}{1.727014in}}{\pgfqpoint{5.520844in}{1.721970in}}%
\pgfpathcurveto{\pgfqpoint{5.515801in}{1.716926in}}{\pgfqpoint{5.512967in}{1.710085in}}{\pgfqpoint{5.512967in}{1.702952in}}%
\pgfpathcurveto{\pgfqpoint{5.512967in}{1.695819in}}{\pgfqpoint{5.515801in}{1.688977in}}{\pgfqpoint{5.520844in}{1.683934in}}%
\pgfpathcurveto{\pgfqpoint{5.525888in}{1.678890in}}{\pgfqpoint{5.532730in}{1.676056in}}{\pgfqpoint{5.539863in}{1.676056in}}%
\pgfpathclose%
\pgfusepath{stroke,fill}%
\end{pgfscope}%
\begin{pgfscope}%
\pgfpathrectangle{\pgfqpoint{4.985294in}{0.500000in}}{\pgfqpoint{1.764706in}{1.700000in}}%
\pgfusepath{clip}%
\pgfsetbuttcap%
\pgfsetroundjoin%
\definecolor{currentfill}{rgb}{0.953126,0.456614,0.312398}%
\pgfsetfillcolor{currentfill}%
\pgfsetlinewidth{0.311001pt}%
\definecolor{currentstroke}{rgb}{1.000000,1.000000,1.000000}%
\pgfsetstrokecolor{currentstroke}%
\pgfsetdash{}{0pt}%
\pgfpathmoveto{\pgfqpoint{6.177089in}{0.833864in}}%
\pgfpathcurveto{\pgfqpoint{6.184222in}{0.833864in}}{\pgfqpoint{6.191063in}{0.836698in}}{\pgfqpoint{6.196107in}{0.841742in}}%
\pgfpathcurveto{\pgfqpoint{6.201151in}{0.846786in}}{\pgfqpoint{6.203985in}{0.853627in}}{\pgfqpoint{6.203985in}{0.860760in}}%
\pgfpathcurveto{\pgfqpoint{6.203985in}{0.867893in}}{\pgfqpoint{6.201151in}{0.874735in}}{\pgfqpoint{6.196107in}{0.879778in}}%
\pgfpathcurveto{\pgfqpoint{6.191063in}{0.884822in}}{\pgfqpoint{6.184222in}{0.887656in}}{\pgfqpoint{6.177089in}{0.887656in}}%
\pgfpathcurveto{\pgfqpoint{6.169956in}{0.887656in}}{\pgfqpoint{6.163114in}{0.884822in}}{\pgfqpoint{6.158071in}{0.879778in}}%
\pgfpathcurveto{\pgfqpoint{6.153027in}{0.874735in}}{\pgfqpoint{6.150193in}{0.867893in}}{\pgfqpoint{6.150193in}{0.860760in}}%
\pgfpathcurveto{\pgfqpoint{6.150193in}{0.853627in}}{\pgfqpoint{6.153027in}{0.846786in}}{\pgfqpoint{6.158071in}{0.841742in}}%
\pgfpathcurveto{\pgfqpoint{6.163114in}{0.836698in}}{\pgfqpoint{6.169956in}{0.833864in}}{\pgfqpoint{6.177089in}{0.833864in}}%
\pgfpathclose%
\pgfusepath{stroke,fill}%
\end{pgfscope}%
\begin{pgfscope}%
\pgfpathrectangle{\pgfqpoint{4.985294in}{0.500000in}}{\pgfqpoint{1.764706in}{1.700000in}}%
\pgfusepath{clip}%
\pgfsetbuttcap%
\pgfsetroundjoin%
\definecolor{currentfill}{rgb}{0.968931,0.798091,0.685123}%
\pgfsetfillcolor{currentfill}%
\pgfsetlinewidth{0.311001pt}%
\definecolor{currentstroke}{rgb}{1.000000,1.000000,1.000000}%
\pgfsetstrokecolor{currentstroke}%
\pgfsetdash{}{0pt}%
\pgfpathmoveto{\pgfqpoint{5.457286in}{1.640228in}}%
\pgfpathcurveto{\pgfqpoint{5.464419in}{1.640228in}}{\pgfqpoint{5.471261in}{1.643062in}}{\pgfqpoint{5.476305in}{1.648106in}}%
\pgfpathcurveto{\pgfqpoint{5.481348in}{1.653149in}}{\pgfqpoint{5.484182in}{1.659991in}}{\pgfqpoint{5.484182in}{1.667124in}}%
\pgfpathcurveto{\pgfqpoint{5.484182in}{1.674257in}}{\pgfqpoint{5.481348in}{1.681098in}}{\pgfqpoint{5.476305in}{1.686142in}}%
\pgfpathcurveto{\pgfqpoint{5.471261in}{1.691186in}}{\pgfqpoint{5.464419in}{1.694020in}}{\pgfqpoint{5.457286in}{1.694020in}}%
\pgfpathcurveto{\pgfqpoint{5.450154in}{1.694020in}}{\pgfqpoint{5.443312in}{1.691186in}}{\pgfqpoint{5.438268in}{1.686142in}}%
\pgfpathcurveto{\pgfqpoint{5.433225in}{1.681098in}}{\pgfqpoint{5.430391in}{1.674257in}}{\pgfqpoint{5.430391in}{1.667124in}}%
\pgfpathcurveto{\pgfqpoint{5.430391in}{1.659991in}}{\pgfqpoint{5.433225in}{1.653149in}}{\pgfqpoint{5.438268in}{1.648106in}}%
\pgfpathcurveto{\pgfqpoint{5.443312in}{1.643062in}}{\pgfqpoint{5.450154in}{1.640228in}}{\pgfqpoint{5.457286in}{1.640228in}}%
\pgfpathclose%
\pgfusepath{stroke,fill}%
\end{pgfscope}%
\begin{pgfscope}%
\pgfpathrectangle{\pgfqpoint{4.985294in}{0.500000in}}{\pgfqpoint{1.764706in}{1.700000in}}%
\pgfusepath{clip}%
\pgfsetbuttcap%
\pgfsetroundjoin%
\definecolor{currentfill}{rgb}{0.965169,0.707764,0.560659}%
\pgfsetfillcolor{currentfill}%
\pgfsetlinewidth{0.311001pt}%
\definecolor{currentstroke}{rgb}{1.000000,1.000000,1.000000}%
\pgfsetstrokecolor{currentstroke}%
\pgfsetdash{}{0pt}%
\pgfpathmoveto{\pgfqpoint{5.323723in}{1.299244in}}%
\pgfpathcurveto{\pgfqpoint{5.330856in}{1.299244in}}{\pgfqpoint{5.337697in}{1.302078in}}{\pgfqpoint{5.342741in}{1.307122in}}%
\pgfpathcurveto{\pgfqpoint{5.347784in}{1.312166in}}{\pgfqpoint{5.350618in}{1.319007in}}{\pgfqpoint{5.350618in}{1.326140in}}%
\pgfpathcurveto{\pgfqpoint{5.350618in}{1.333273in}}{\pgfqpoint{5.347784in}{1.340115in}}{\pgfqpoint{5.342741in}{1.345158in}}%
\pgfpathcurveto{\pgfqpoint{5.337697in}{1.350202in}}{\pgfqpoint{5.330856in}{1.353036in}}{\pgfqpoint{5.323723in}{1.353036in}}%
\pgfpathcurveto{\pgfqpoint{5.316590in}{1.353036in}}{\pgfqpoint{5.309748in}{1.350202in}}{\pgfqpoint{5.304705in}{1.345158in}}%
\pgfpathcurveto{\pgfqpoint{5.299661in}{1.340115in}}{\pgfqpoint{5.296827in}{1.333273in}}{\pgfqpoint{5.296827in}{1.326140in}}%
\pgfpathcurveto{\pgfqpoint{5.296827in}{1.319007in}}{\pgfqpoint{5.299661in}{1.312166in}}{\pgfqpoint{5.304705in}{1.307122in}}%
\pgfpathcurveto{\pgfqpoint{5.309748in}{1.302078in}}{\pgfqpoint{5.316590in}{1.299244in}}{\pgfqpoint{5.323723in}{1.299244in}}%
\pgfpathclose%
\pgfusepath{stroke,fill}%
\end{pgfscope}%
\begin{pgfscope}%
\pgfpathrectangle{\pgfqpoint{4.985294in}{0.500000in}}{\pgfqpoint{1.764706in}{1.700000in}}%
\pgfusepath{clip}%
\pgfsetbuttcap%
\pgfsetroundjoin%
\definecolor{currentfill}{rgb}{0.961433,0.573272,0.412036}%
\pgfsetfillcolor{currentfill}%
\pgfsetlinewidth{0.311001pt}%
\definecolor{currentstroke}{rgb}{1.000000,1.000000,1.000000}%
\pgfsetstrokecolor{currentstroke}%
\pgfsetdash{}{0pt}%
\pgfpathmoveto{\pgfqpoint{5.659455in}{0.967652in}}%
\pgfpathcurveto{\pgfqpoint{5.666588in}{0.967652in}}{\pgfqpoint{5.673429in}{0.970486in}}{\pgfqpoint{5.678473in}{0.975530in}}%
\pgfpathcurveto{\pgfqpoint{5.683517in}{0.980573in}}{\pgfqpoint{5.686350in}{0.987415in}}{\pgfqpoint{5.686350in}{0.994548in}}%
\pgfpathcurveto{\pgfqpoint{5.686350in}{1.001681in}}{\pgfqpoint{5.683517in}{1.008522in}}{\pgfqpoint{5.678473in}{1.013566in}}%
\pgfpathcurveto{\pgfqpoint{5.673429in}{1.018610in}}{\pgfqpoint{5.666588in}{1.021443in}}{\pgfqpoint{5.659455in}{1.021443in}}%
\pgfpathcurveto{\pgfqpoint{5.652322in}{1.021443in}}{\pgfqpoint{5.645480in}{1.018610in}}{\pgfqpoint{5.640437in}{1.013566in}}%
\pgfpathcurveto{\pgfqpoint{5.635393in}{1.008522in}}{\pgfqpoint{5.632559in}{1.001681in}}{\pgfqpoint{5.632559in}{0.994548in}}%
\pgfpathcurveto{\pgfqpoint{5.632559in}{0.987415in}}{\pgfqpoint{5.635393in}{0.980573in}}{\pgfqpoint{5.640437in}{0.975530in}}%
\pgfpathcurveto{\pgfqpoint{5.645480in}{0.970486in}}{\pgfqpoint{5.652322in}{0.967652in}}{\pgfqpoint{5.659455in}{0.967652in}}%
\pgfpathclose%
\pgfusepath{stroke,fill}%
\end{pgfscope}%
\begin{pgfscope}%
\pgfpathrectangle{\pgfqpoint{4.985294in}{0.500000in}}{\pgfqpoint{1.764706in}{1.700000in}}%
\pgfusepath{clip}%
\pgfsetbuttcap%
\pgfsetroundjoin%
\definecolor{currentfill}{rgb}{0.973271,0.850724,0.762998}%
\pgfsetfillcolor{currentfill}%
\pgfsetlinewidth{0.311001pt}%
\definecolor{currentstroke}{rgb}{1.000000,1.000000,1.000000}%
\pgfsetstrokecolor{currentstroke}%
\pgfsetdash{}{0pt}%
\pgfpathmoveto{\pgfqpoint{6.190203in}{1.633008in}}%
\pgfpathcurveto{\pgfqpoint{6.197336in}{1.633008in}}{\pgfqpoint{6.204178in}{1.635842in}}{\pgfqpoint{6.209222in}{1.640886in}}%
\pgfpathcurveto{\pgfqpoint{6.214265in}{1.645930in}}{\pgfqpoint{6.217099in}{1.652771in}}{\pgfqpoint{6.217099in}{1.659904in}}%
\pgfpathcurveto{\pgfqpoint{6.217099in}{1.667037in}}{\pgfqpoint{6.214265in}{1.673879in}}{\pgfqpoint{6.209222in}{1.678922in}}%
\pgfpathcurveto{\pgfqpoint{6.204178in}{1.683966in}}{\pgfqpoint{6.197336in}{1.686800in}}{\pgfqpoint{6.190203in}{1.686800in}}%
\pgfpathcurveto{\pgfqpoint{6.183071in}{1.686800in}}{\pgfqpoint{6.176229in}{1.683966in}}{\pgfqpoint{6.171185in}{1.678922in}}%
\pgfpathcurveto{\pgfqpoint{6.166142in}{1.673879in}}{\pgfqpoint{6.163308in}{1.667037in}}{\pgfqpoint{6.163308in}{1.659904in}}%
\pgfpathcurveto{\pgfqpoint{6.163308in}{1.652771in}}{\pgfqpoint{6.166142in}{1.645930in}}{\pgfqpoint{6.171185in}{1.640886in}}%
\pgfpathcurveto{\pgfqpoint{6.176229in}{1.635842in}}{\pgfqpoint{6.183071in}{1.633008in}}{\pgfqpoint{6.190203in}{1.633008in}}%
\pgfpathclose%
\pgfusepath{stroke,fill}%
\end{pgfscope}%
\begin{pgfscope}%
\pgfpathrectangle{\pgfqpoint{4.985294in}{0.500000in}}{\pgfqpoint{1.764706in}{1.700000in}}%
\pgfusepath{clip}%
\pgfsetbuttcap%
\pgfsetroundjoin%
\definecolor{currentfill}{rgb}{0.980678,0.914765,0.856766}%
\pgfsetfillcolor{currentfill}%
\pgfsetlinewidth{0.311001pt}%
\definecolor{currentstroke}{rgb}{1.000000,1.000000,1.000000}%
\pgfsetstrokecolor{currentstroke}%
\pgfsetdash{}{0pt}%
\pgfpathmoveto{\pgfqpoint{5.424069in}{1.289750in}}%
\pgfpathcurveto{\pgfqpoint{5.431202in}{1.289750in}}{\pgfqpoint{5.438044in}{1.292584in}}{\pgfqpoint{5.443087in}{1.297628in}}%
\pgfpathcurveto{\pgfqpoint{5.448131in}{1.302671in}}{\pgfqpoint{5.450965in}{1.309513in}}{\pgfqpoint{5.450965in}{1.316646in}}%
\pgfpathcurveto{\pgfqpoint{5.450965in}{1.323779in}}{\pgfqpoint{5.448131in}{1.330620in}}{\pgfqpoint{5.443087in}{1.335664in}}%
\pgfpathcurveto{\pgfqpoint{5.438044in}{1.340708in}}{\pgfqpoint{5.431202in}{1.343542in}}{\pgfqpoint{5.424069in}{1.343542in}}%
\pgfpathcurveto{\pgfqpoint{5.416936in}{1.343542in}}{\pgfqpoint{5.410095in}{1.340708in}}{\pgfqpoint{5.405051in}{1.335664in}}%
\pgfpathcurveto{\pgfqpoint{5.400007in}{1.330620in}}{\pgfqpoint{5.397174in}{1.323779in}}{\pgfqpoint{5.397174in}{1.316646in}}%
\pgfpathcurveto{\pgfqpoint{5.397174in}{1.309513in}}{\pgfqpoint{5.400007in}{1.302671in}}{\pgfqpoint{5.405051in}{1.297628in}}%
\pgfpathcurveto{\pgfqpoint{5.410095in}{1.292584in}}{\pgfqpoint{5.416936in}{1.289750in}}{\pgfqpoint{5.424069in}{1.289750in}}%
\pgfpathclose%
\pgfusepath{stroke,fill}%
\end{pgfscope}%
\begin{pgfscope}%
\pgfpathrectangle{\pgfqpoint{4.985294in}{0.500000in}}{\pgfqpoint{1.764706in}{1.700000in}}%
\pgfusepath{clip}%
\pgfsetbuttcap%
\pgfsetroundjoin%
\definecolor{currentfill}{rgb}{0.972201,0.839051,0.745789}%
\pgfsetfillcolor{currentfill}%
\pgfsetlinewidth{0.311001pt}%
\definecolor{currentstroke}{rgb}{1.000000,1.000000,1.000000}%
\pgfsetstrokecolor{currentstroke}%
\pgfsetdash{}{0pt}%
\pgfpathmoveto{\pgfqpoint{6.212620in}{1.126099in}}%
\pgfpathcurveto{\pgfqpoint{6.219753in}{1.126099in}}{\pgfqpoint{6.226595in}{1.128933in}}{\pgfqpoint{6.231638in}{1.133977in}}%
\pgfpathcurveto{\pgfqpoint{6.236682in}{1.139020in}}{\pgfqpoint{6.239516in}{1.145862in}}{\pgfqpoint{6.239516in}{1.152995in}}%
\pgfpathcurveto{\pgfqpoint{6.239516in}{1.160127in}}{\pgfqpoint{6.236682in}{1.166969in}}{\pgfqpoint{6.231638in}{1.172013in}}%
\pgfpathcurveto{\pgfqpoint{6.226595in}{1.177056in}}{\pgfqpoint{6.219753in}{1.179890in}}{\pgfqpoint{6.212620in}{1.179890in}}%
\pgfpathcurveto{\pgfqpoint{6.205487in}{1.179890in}}{\pgfqpoint{6.198646in}{1.177056in}}{\pgfqpoint{6.193602in}{1.172013in}}%
\pgfpathcurveto{\pgfqpoint{6.188558in}{1.166969in}}{\pgfqpoint{6.185724in}{1.160127in}}{\pgfqpoint{6.185724in}{1.152995in}}%
\pgfpathcurveto{\pgfqpoint{6.185724in}{1.145862in}}{\pgfqpoint{6.188558in}{1.139020in}}{\pgfqpoint{6.193602in}{1.133977in}}%
\pgfpathcurveto{\pgfqpoint{6.198646in}{1.128933in}}{\pgfqpoint{6.205487in}{1.126099in}}{\pgfqpoint{6.212620in}{1.126099in}}%
\pgfpathclose%
\pgfusepath{stroke,fill}%
\end{pgfscope}%
\begin{pgfscope}%
\pgfpathrectangle{\pgfqpoint{4.985294in}{0.500000in}}{\pgfqpoint{1.764706in}{1.700000in}}%
\pgfusepath{clip}%
\pgfsetbuttcap%
\pgfsetroundjoin%
\definecolor{currentfill}{rgb}{0.977657,0.891500,0.822809}%
\pgfsetfillcolor{currentfill}%
\pgfsetlinewidth{0.311001pt}%
\definecolor{currentstroke}{rgb}{1.000000,1.000000,1.000000}%
\pgfsetstrokecolor{currentstroke}%
\pgfsetdash{}{0pt}%
\pgfpathmoveto{\pgfqpoint{5.384439in}{1.351328in}}%
\pgfpathcurveto{\pgfqpoint{5.391572in}{1.351328in}}{\pgfqpoint{5.398413in}{1.354162in}}{\pgfqpoint{5.403457in}{1.359206in}}%
\pgfpathcurveto{\pgfqpoint{5.408501in}{1.364249in}}{\pgfqpoint{5.411334in}{1.371091in}}{\pgfqpoint{5.411334in}{1.378224in}}%
\pgfpathcurveto{\pgfqpoint{5.411334in}{1.385357in}}{\pgfqpoint{5.408501in}{1.392198in}}{\pgfqpoint{5.403457in}{1.397242in}}%
\pgfpathcurveto{\pgfqpoint{5.398413in}{1.402286in}}{\pgfqpoint{5.391572in}{1.405119in}}{\pgfqpoint{5.384439in}{1.405119in}}%
\pgfpathcurveto{\pgfqpoint{5.377306in}{1.405119in}}{\pgfqpoint{5.370464in}{1.402286in}}{\pgfqpoint{5.365421in}{1.397242in}}%
\pgfpathcurveto{\pgfqpoint{5.360377in}{1.392198in}}{\pgfqpoint{5.357543in}{1.385357in}}{\pgfqpoint{5.357543in}{1.378224in}}%
\pgfpathcurveto{\pgfqpoint{5.357543in}{1.371091in}}{\pgfqpoint{5.360377in}{1.364249in}}{\pgfqpoint{5.365421in}{1.359206in}}%
\pgfpathcurveto{\pgfqpoint{5.370464in}{1.354162in}}{\pgfqpoint{5.377306in}{1.351328in}}{\pgfqpoint{5.384439in}{1.351328in}}%
\pgfpathclose%
\pgfusepath{stroke,fill}%
\end{pgfscope}%
\begin{pgfscope}%
\pgfpathrectangle{\pgfqpoint{4.985294in}{0.500000in}}{\pgfqpoint{1.764706in}{1.700000in}}%
\pgfusepath{clip}%
\pgfsetbuttcap%
\pgfsetroundjoin%
\definecolor{currentfill}{rgb}{0.971202,0.827364,0.728520}%
\pgfsetfillcolor{currentfill}%
\pgfsetlinewidth{0.311001pt}%
\definecolor{currentstroke}{rgb}{1.000000,1.000000,1.000000}%
\pgfsetstrokecolor{currentstroke}%
\pgfsetdash{}{0pt}%
\pgfpathmoveto{\pgfqpoint{5.525787in}{1.599317in}}%
\pgfpathcurveto{\pgfqpoint{5.532920in}{1.599317in}}{\pgfqpoint{5.539762in}{1.602151in}}{\pgfqpoint{5.544805in}{1.607195in}}%
\pgfpathcurveto{\pgfqpoint{5.549849in}{1.612238in}}{\pgfqpoint{5.552683in}{1.619080in}}{\pgfqpoint{5.552683in}{1.626213in}}%
\pgfpathcurveto{\pgfqpoint{5.552683in}{1.633346in}}{\pgfqpoint{5.549849in}{1.640187in}}{\pgfqpoint{5.544805in}{1.645231in}}%
\pgfpathcurveto{\pgfqpoint{5.539762in}{1.650274in}}{\pgfqpoint{5.532920in}{1.653108in}}{\pgfqpoint{5.525787in}{1.653108in}}%
\pgfpathcurveto{\pgfqpoint{5.518654in}{1.653108in}}{\pgfqpoint{5.511813in}{1.650274in}}{\pgfqpoint{5.506769in}{1.645231in}}%
\pgfpathcurveto{\pgfqpoint{5.501725in}{1.640187in}}{\pgfqpoint{5.498892in}{1.633346in}}{\pgfqpoint{5.498892in}{1.626213in}}%
\pgfpathcurveto{\pgfqpoint{5.498892in}{1.619080in}}{\pgfqpoint{5.501725in}{1.612238in}}{\pgfqpoint{5.506769in}{1.607195in}}%
\pgfpathcurveto{\pgfqpoint{5.511813in}{1.602151in}}{\pgfqpoint{5.518654in}{1.599317in}}{\pgfqpoint{5.525787in}{1.599317in}}%
\pgfpathclose%
\pgfusepath{stroke,fill}%
\end{pgfscope}%
\begin{pgfscope}%
\pgfpathrectangle{\pgfqpoint{4.985294in}{0.500000in}}{\pgfqpoint{1.764706in}{1.700000in}}%
\pgfusepath{clip}%
\pgfsetbuttcap%
\pgfsetroundjoin%
\definecolor{currentfill}{rgb}{0.963728,0.638439,0.479050}%
\pgfsetfillcolor{currentfill}%
\pgfsetlinewidth{0.311001pt}%
\definecolor{currentstroke}{rgb}{1.000000,1.000000,1.000000}%
\pgfsetstrokecolor{currentstroke}%
\pgfsetdash{}{0pt}%
\pgfpathmoveto{\pgfqpoint{6.335105in}{1.639051in}}%
\pgfpathcurveto{\pgfqpoint{6.342238in}{1.639051in}}{\pgfqpoint{6.349079in}{1.641885in}}{\pgfqpoint{6.354123in}{1.646928in}}%
\pgfpathcurveto{\pgfqpoint{6.359167in}{1.651972in}}{\pgfqpoint{6.362000in}{1.658814in}}{\pgfqpoint{6.362000in}{1.665946in}}%
\pgfpathcurveto{\pgfqpoint{6.362000in}{1.673079in}}{\pgfqpoint{6.359167in}{1.679921in}}{\pgfqpoint{6.354123in}{1.684965in}}%
\pgfpathcurveto{\pgfqpoint{6.349079in}{1.690008in}}{\pgfqpoint{6.342238in}{1.692842in}}{\pgfqpoint{6.335105in}{1.692842in}}%
\pgfpathcurveto{\pgfqpoint{6.327972in}{1.692842in}}{\pgfqpoint{6.321130in}{1.690008in}}{\pgfqpoint{6.316087in}{1.684965in}}%
\pgfpathcurveto{\pgfqpoint{6.311043in}{1.679921in}}{\pgfqpoint{6.308209in}{1.673079in}}{\pgfqpoint{6.308209in}{1.665946in}}%
\pgfpathcurveto{\pgfqpoint{6.308209in}{1.658814in}}{\pgfqpoint{6.311043in}{1.651972in}}{\pgfqpoint{6.316087in}{1.646928in}}%
\pgfpathcurveto{\pgfqpoint{6.321130in}{1.641885in}}{\pgfqpoint{6.327972in}{1.639051in}}{\pgfqpoint{6.335105in}{1.639051in}}%
\pgfpathclose%
\pgfusepath{stroke,fill}%
\end{pgfscope}%
\begin{pgfscope}%
\pgfpathrectangle{\pgfqpoint{4.985294in}{0.500000in}}{\pgfqpoint{1.764706in}{1.700000in}}%
\pgfusepath{clip}%
\pgfsetbuttcap%
\pgfsetroundjoin%
\definecolor{currentfill}{rgb}{0.961115,0.566634,0.405693}%
\pgfsetfillcolor{currentfill}%
\pgfsetlinewidth{0.311001pt}%
\definecolor{currentstroke}{rgb}{1.000000,1.000000,1.000000}%
\pgfsetstrokecolor{currentstroke}%
\pgfsetdash{}{0pt}%
\pgfpathmoveto{\pgfqpoint{5.402942in}{0.942456in}}%
\pgfpathcurveto{\pgfqpoint{5.410075in}{0.942456in}}{\pgfqpoint{5.416916in}{0.945290in}}{\pgfqpoint{5.421960in}{0.950334in}}%
\pgfpathcurveto{\pgfqpoint{5.427004in}{0.955378in}}{\pgfqpoint{5.429838in}{0.962219in}}{\pgfqpoint{5.429838in}{0.969352in}}%
\pgfpathcurveto{\pgfqpoint{5.429838in}{0.976485in}}{\pgfqpoint{5.427004in}{0.983327in}}{\pgfqpoint{5.421960in}{0.988370in}}%
\pgfpathcurveto{\pgfqpoint{5.416916in}{0.993414in}}{\pgfqpoint{5.410075in}{0.996248in}}{\pgfqpoint{5.402942in}{0.996248in}}%
\pgfpathcurveto{\pgfqpoint{5.395809in}{0.996248in}}{\pgfqpoint{5.388967in}{0.993414in}}{\pgfqpoint{5.383924in}{0.988370in}}%
\pgfpathcurveto{\pgfqpoint{5.378880in}{0.983327in}}{\pgfqpoint{5.376046in}{0.976485in}}{\pgfqpoint{5.376046in}{0.969352in}}%
\pgfpathcurveto{\pgfqpoint{5.376046in}{0.962219in}}{\pgfqpoint{5.378880in}{0.955378in}}{\pgfqpoint{5.383924in}{0.950334in}}%
\pgfpathcurveto{\pgfqpoint{5.388967in}{0.945290in}}{\pgfqpoint{5.395809in}{0.942456in}}{\pgfqpoint{5.402942in}{0.942456in}}%
\pgfpathclose%
\pgfusepath{stroke,fill}%
\end{pgfscope}%
\begin{pgfscope}%
\pgfpathrectangle{\pgfqpoint{4.985294in}{0.500000in}}{\pgfqpoint{1.764706in}{1.700000in}}%
\pgfusepath{clip}%
\pgfsetbuttcap%
\pgfsetroundjoin%
\definecolor{currentfill}{rgb}{0.967398,0.774513,0.650573}%
\pgfsetfillcolor{currentfill}%
\pgfsetlinewidth{0.311001pt}%
\definecolor{currentstroke}{rgb}{1.000000,1.000000,1.000000}%
\pgfsetstrokecolor{currentstroke}%
\pgfsetdash{}{0pt}%
\pgfpathmoveto{\pgfqpoint{5.527617in}{1.132941in}}%
\pgfpathcurveto{\pgfqpoint{5.534750in}{1.132941in}}{\pgfqpoint{5.541592in}{1.135775in}}{\pgfqpoint{5.546636in}{1.140819in}}%
\pgfpathcurveto{\pgfqpoint{5.551679in}{1.145862in}}{\pgfqpoint{5.554513in}{1.152704in}}{\pgfqpoint{5.554513in}{1.159837in}}%
\pgfpathcurveto{\pgfqpoint{5.554513in}{1.166970in}}{\pgfqpoint{5.551679in}{1.173811in}}{\pgfqpoint{5.546636in}{1.178855in}}%
\pgfpathcurveto{\pgfqpoint{5.541592in}{1.183899in}}{\pgfqpoint{5.534750in}{1.186733in}}{\pgfqpoint{5.527617in}{1.186733in}}%
\pgfpathcurveto{\pgfqpoint{5.520485in}{1.186733in}}{\pgfqpoint{5.513643in}{1.183899in}}{\pgfqpoint{5.508599in}{1.178855in}}%
\pgfpathcurveto{\pgfqpoint{5.503556in}{1.173811in}}{\pgfqpoint{5.500722in}{1.166970in}}{\pgfqpoint{5.500722in}{1.159837in}}%
\pgfpathcurveto{\pgfqpoint{5.500722in}{1.152704in}}{\pgfqpoint{5.503556in}{1.145862in}}{\pgfqpoint{5.508599in}{1.140819in}}%
\pgfpathcurveto{\pgfqpoint{5.513643in}{1.135775in}}{\pgfqpoint{5.520485in}{1.132941in}}{\pgfqpoint{5.527617in}{1.132941in}}%
\pgfpathclose%
\pgfusepath{stroke,fill}%
\end{pgfscope}%
\begin{pgfscope}%
\pgfpathrectangle{\pgfqpoint{4.985294in}{0.500000in}}{\pgfqpoint{1.764706in}{1.700000in}}%
\pgfusepath{clip}%
\pgfsetbuttcap%
\pgfsetroundjoin%
\definecolor{currentfill}{rgb}{0.977657,0.891500,0.822809}%
\pgfsetfillcolor{currentfill}%
\pgfsetlinewidth{0.311001pt}%
\definecolor{currentstroke}{rgb}{1.000000,1.000000,1.000000}%
\pgfsetstrokecolor{currentstroke}%
\pgfsetdash{}{0pt}%
\pgfpathmoveto{\pgfqpoint{6.269485in}{1.510282in}}%
\pgfpathcurveto{\pgfqpoint{6.276618in}{1.510282in}}{\pgfqpoint{6.283460in}{1.513116in}}{\pgfqpoint{6.288503in}{1.518160in}}%
\pgfpathcurveto{\pgfqpoint{6.293547in}{1.523204in}}{\pgfqpoint{6.296381in}{1.530045in}}{\pgfqpoint{6.296381in}{1.537178in}}%
\pgfpathcurveto{\pgfqpoint{6.296381in}{1.544311in}}{\pgfqpoint{6.293547in}{1.551153in}}{\pgfqpoint{6.288503in}{1.556196in}}%
\pgfpathcurveto{\pgfqpoint{6.283460in}{1.561240in}}{\pgfqpoint{6.276618in}{1.564074in}}{\pgfqpoint{6.269485in}{1.564074in}}%
\pgfpathcurveto{\pgfqpoint{6.262352in}{1.564074in}}{\pgfqpoint{6.255511in}{1.561240in}}{\pgfqpoint{6.250467in}{1.556196in}}%
\pgfpathcurveto{\pgfqpoint{6.245423in}{1.551153in}}{\pgfqpoint{6.242589in}{1.544311in}}{\pgfqpoint{6.242589in}{1.537178in}}%
\pgfpathcurveto{\pgfqpoint{6.242589in}{1.530045in}}{\pgfqpoint{6.245423in}{1.523204in}}{\pgfqpoint{6.250467in}{1.518160in}}%
\pgfpathcurveto{\pgfqpoint{6.255511in}{1.513116in}}{\pgfqpoint{6.262352in}{1.510282in}}{\pgfqpoint{6.269485in}{1.510282in}}%
\pgfpathclose%
\pgfusepath{stroke,fill}%
\end{pgfscope}%
\begin{pgfscope}%
\pgfpathrectangle{\pgfqpoint{4.985294in}{0.500000in}}{\pgfqpoint{1.764706in}{1.700000in}}%
\pgfusepath{clip}%
\pgfsetbuttcap%
\pgfsetroundjoin%
\definecolor{currentfill}{rgb}{0.968105,0.786346,0.667739}%
\pgfsetfillcolor{currentfill}%
\pgfsetlinewidth{0.311001pt}%
\definecolor{currentstroke}{rgb}{1.000000,1.000000,1.000000}%
\pgfsetstrokecolor{currentstroke}%
\pgfsetdash{}{0pt}%
\pgfpathmoveto{\pgfqpoint{6.245099in}{1.679136in}}%
\pgfpathcurveto{\pgfqpoint{6.252232in}{1.679136in}}{\pgfqpoint{6.259074in}{1.681970in}}{\pgfqpoint{6.264117in}{1.687013in}}%
\pgfpathcurveto{\pgfqpoint{6.269161in}{1.692057in}}{\pgfqpoint{6.271995in}{1.698899in}}{\pgfqpoint{6.271995in}{1.706031in}}%
\pgfpathcurveto{\pgfqpoint{6.271995in}{1.713164in}}{\pgfqpoint{6.269161in}{1.720006in}}{\pgfqpoint{6.264117in}{1.725050in}}%
\pgfpathcurveto{\pgfqpoint{6.259074in}{1.730093in}}{\pgfqpoint{6.252232in}{1.732927in}}{\pgfqpoint{6.245099in}{1.732927in}}%
\pgfpathcurveto{\pgfqpoint{6.237966in}{1.732927in}}{\pgfqpoint{6.231125in}{1.730093in}}{\pgfqpoint{6.226081in}{1.725050in}}%
\pgfpathcurveto{\pgfqpoint{6.221037in}{1.720006in}}{\pgfqpoint{6.218203in}{1.713164in}}{\pgfqpoint{6.218203in}{1.706031in}}%
\pgfpathcurveto{\pgfqpoint{6.218203in}{1.698899in}}{\pgfqpoint{6.221037in}{1.692057in}}{\pgfqpoint{6.226081in}{1.687013in}}%
\pgfpathcurveto{\pgfqpoint{6.231125in}{1.681970in}}{\pgfqpoint{6.237966in}{1.679136in}}{\pgfqpoint{6.245099in}{1.679136in}}%
\pgfpathclose%
\pgfusepath{stroke,fill}%
\end{pgfscope}%
\begin{pgfscope}%
\pgfpathrectangle{\pgfqpoint{4.985294in}{0.500000in}}{\pgfqpoint{1.764706in}{1.700000in}}%
\pgfusepath{clip}%
\pgfsetbuttcap%
\pgfsetroundjoin%
\definecolor{currentfill}{rgb}{0.970255,0.815666,0.711203}%
\pgfsetfillcolor{currentfill}%
\pgfsetlinewidth{0.311001pt}%
\definecolor{currentstroke}{rgb}{1.000000,1.000000,1.000000}%
\pgfsetstrokecolor{currentstroke}%
\pgfsetdash{}{0pt}%
\pgfpathmoveto{\pgfqpoint{5.396949in}{1.103613in}}%
\pgfpathcurveto{\pgfqpoint{5.404081in}{1.103613in}}{\pgfqpoint{5.410923in}{1.106447in}}{\pgfqpoint{5.415967in}{1.111491in}}%
\pgfpathcurveto{\pgfqpoint{5.421010in}{1.116535in}}{\pgfqpoint{5.423844in}{1.123376in}}{\pgfqpoint{5.423844in}{1.130509in}}%
\pgfpathcurveto{\pgfqpoint{5.423844in}{1.137642in}}{\pgfqpoint{5.421010in}{1.144484in}}{\pgfqpoint{5.415967in}{1.149527in}}%
\pgfpathcurveto{\pgfqpoint{5.410923in}{1.154571in}}{\pgfqpoint{5.404081in}{1.157405in}}{\pgfqpoint{5.396949in}{1.157405in}}%
\pgfpathcurveto{\pgfqpoint{5.389816in}{1.157405in}}{\pgfqpoint{5.382974in}{1.154571in}}{\pgfqpoint{5.377931in}{1.149527in}}%
\pgfpathcurveto{\pgfqpoint{5.372887in}{1.144484in}}{\pgfqpoint{5.370053in}{1.137642in}}{\pgfqpoint{5.370053in}{1.130509in}}%
\pgfpathcurveto{\pgfqpoint{5.370053in}{1.123376in}}{\pgfqpoint{5.372887in}{1.116535in}}{\pgfqpoint{5.377931in}{1.111491in}}%
\pgfpathcurveto{\pgfqpoint{5.382974in}{1.106447in}}{\pgfqpoint{5.389816in}{1.103613in}}{\pgfqpoint{5.396949in}{1.103613in}}%
\pgfpathclose%
\pgfusepath{stroke,fill}%
\end{pgfscope}%
\begin{pgfscope}%
\pgfpathrectangle{\pgfqpoint{4.985294in}{0.500000in}}{\pgfqpoint{1.764706in}{1.700000in}}%
\pgfusepath{clip}%
\pgfsetbuttcap%
\pgfsetroundjoin%
\definecolor{currentfill}{rgb}{0.975018,0.868213,0.788710}%
\pgfsetfillcolor{currentfill}%
\pgfsetlinewidth{0.311001pt}%
\definecolor{currentstroke}{rgb}{1.000000,1.000000,1.000000}%
\pgfsetstrokecolor{currentstroke}%
\pgfsetdash{}{0pt}%
\pgfpathmoveto{\pgfqpoint{5.482331in}{1.070528in}}%
\pgfpathcurveto{\pgfqpoint{5.489464in}{1.070528in}}{\pgfqpoint{5.496305in}{1.073362in}}{\pgfqpoint{5.501349in}{1.078405in}}%
\pgfpathcurveto{\pgfqpoint{5.506393in}{1.083449in}}{\pgfqpoint{5.509227in}{1.090291in}}{\pgfqpoint{5.509227in}{1.097423in}}%
\pgfpathcurveto{\pgfqpoint{5.509227in}{1.104556in}}{\pgfqpoint{5.506393in}{1.111398in}}{\pgfqpoint{5.501349in}{1.116442in}}%
\pgfpathcurveto{\pgfqpoint{5.496305in}{1.121485in}}{\pgfqpoint{5.489464in}{1.124319in}}{\pgfqpoint{5.482331in}{1.124319in}}%
\pgfpathcurveto{\pgfqpoint{5.475198in}{1.124319in}}{\pgfqpoint{5.468356in}{1.121485in}}{\pgfqpoint{5.463313in}{1.116442in}}%
\pgfpathcurveto{\pgfqpoint{5.458269in}{1.111398in}}{\pgfqpoint{5.455435in}{1.104556in}}{\pgfqpoint{5.455435in}{1.097423in}}%
\pgfpathcurveto{\pgfqpoint{5.455435in}{1.090291in}}{\pgfqpoint{5.458269in}{1.083449in}}{\pgfqpoint{5.463313in}{1.078405in}}%
\pgfpathcurveto{\pgfqpoint{5.468356in}{1.073362in}}{\pgfqpoint{5.475198in}{1.070528in}}{\pgfqpoint{5.482331in}{1.070528in}}%
\pgfpathclose%
\pgfusepath{stroke,fill}%
\end{pgfscope}%
\begin{pgfscope}%
\pgfpathrectangle{\pgfqpoint{4.985294in}{0.500000in}}{\pgfqpoint{1.764706in}{1.700000in}}%
\pgfusepath{clip}%
\pgfsetbuttcap%
\pgfsetroundjoin%
\definecolor{currentfill}{rgb}{0.979891,0.908948,0.848279}%
\pgfsetfillcolor{currentfill}%
\pgfsetlinewidth{0.311001pt}%
\definecolor{currentstroke}{rgb}{1.000000,1.000000,1.000000}%
\pgfsetstrokecolor{currentstroke}%
\pgfsetdash{}{0pt}%
\pgfpathmoveto{\pgfqpoint{6.334165in}{1.271437in}}%
\pgfpathcurveto{\pgfqpoint{6.341298in}{1.271437in}}{\pgfqpoint{6.348139in}{1.274271in}}{\pgfqpoint{6.353183in}{1.279315in}}%
\pgfpathcurveto{\pgfqpoint{6.358227in}{1.284358in}}{\pgfqpoint{6.361060in}{1.291200in}}{\pgfqpoint{6.361060in}{1.298333in}}%
\pgfpathcurveto{\pgfqpoint{6.361060in}{1.305466in}}{\pgfqpoint{6.358227in}{1.312307in}}{\pgfqpoint{6.353183in}{1.317351in}}%
\pgfpathcurveto{\pgfqpoint{6.348139in}{1.322395in}}{\pgfqpoint{6.341298in}{1.325229in}}{\pgfqpoint{6.334165in}{1.325229in}}%
\pgfpathcurveto{\pgfqpoint{6.327032in}{1.325229in}}{\pgfqpoint{6.320190in}{1.322395in}}{\pgfqpoint{6.315147in}{1.317351in}}%
\pgfpathcurveto{\pgfqpoint{6.310103in}{1.312307in}}{\pgfqpoint{6.307269in}{1.305466in}}{\pgfqpoint{6.307269in}{1.298333in}}%
\pgfpathcurveto{\pgfqpoint{6.307269in}{1.291200in}}{\pgfqpoint{6.310103in}{1.284358in}}{\pgfqpoint{6.315147in}{1.279315in}}%
\pgfpathcurveto{\pgfqpoint{6.320190in}{1.274271in}}{\pgfqpoint{6.327032in}{1.271437in}}{\pgfqpoint{6.334165in}{1.271437in}}%
\pgfpathclose%
\pgfusepath{stroke,fill}%
\end{pgfscope}%
\begin{pgfscope}%
\pgfpathrectangle{\pgfqpoint{4.985294in}{0.500000in}}{\pgfqpoint{1.764706in}{1.700000in}}%
\pgfusepath{clip}%
\pgfsetbuttcap%
\pgfsetroundjoin%
\definecolor{currentfill}{rgb}{0.960421,0.553286,0.393191}%
\pgfsetfillcolor{currentfill}%
\pgfsetlinewidth{0.311001pt}%
\definecolor{currentstroke}{rgb}{1.000000,1.000000,1.000000}%
\pgfsetstrokecolor{currentstroke}%
\pgfsetdash{}{0pt}%
\pgfpathmoveto{\pgfqpoint{6.112501in}{0.868566in}}%
\pgfpathcurveto{\pgfqpoint{6.119633in}{0.868566in}}{\pgfqpoint{6.126475in}{0.871399in}}{\pgfqpoint{6.131519in}{0.876443in}}%
\pgfpathcurveto{\pgfqpoint{6.136562in}{0.881487in}}{\pgfqpoint{6.139396in}{0.888328in}}{\pgfqpoint{6.139396in}{0.895461in}}%
\pgfpathcurveto{\pgfqpoint{6.139396in}{0.902594in}}{\pgfqpoint{6.136562in}{0.909436in}}{\pgfqpoint{6.131519in}{0.914479in}}%
\pgfpathcurveto{\pgfqpoint{6.126475in}{0.919523in}}{\pgfqpoint{6.119633in}{0.922357in}}{\pgfqpoint{6.112501in}{0.922357in}}%
\pgfpathcurveto{\pgfqpoint{6.105368in}{0.922357in}}{\pgfqpoint{6.098526in}{0.919523in}}{\pgfqpoint{6.093482in}{0.914479in}}%
\pgfpathcurveto{\pgfqpoint{6.088439in}{0.909436in}}{\pgfqpoint{6.085605in}{0.902594in}}{\pgfqpoint{6.085605in}{0.895461in}}%
\pgfpathcurveto{\pgfqpoint{6.085605in}{0.888328in}}{\pgfqpoint{6.088439in}{0.881487in}}{\pgfqpoint{6.093482in}{0.876443in}}%
\pgfpathcurveto{\pgfqpoint{6.098526in}{0.871399in}}{\pgfqpoint{6.105368in}{0.868566in}}{\pgfqpoint{6.112501in}{0.868566in}}%
\pgfpathclose%
\pgfusepath{stroke,fill}%
\end{pgfscope}%
\begin{pgfscope}%
\pgfpathrectangle{\pgfqpoint{4.985294in}{0.500000in}}{\pgfqpoint{1.764706in}{1.700000in}}%
\pgfusepath{clip}%
\pgfsetbuttcap%
\pgfsetroundjoin%
\definecolor{currentfill}{rgb}{0.979124,0.903132,0.839793}%
\pgfsetfillcolor{currentfill}%
\pgfsetlinewidth{0.311001pt}%
\definecolor{currentstroke}{rgb}{1.000000,1.000000,1.000000}%
\pgfsetstrokecolor{currentstroke}%
\pgfsetdash{}{0pt}%
\pgfpathmoveto{\pgfqpoint{5.443776in}{1.414344in}}%
\pgfpathcurveto{\pgfqpoint{5.450909in}{1.414344in}}{\pgfqpoint{5.457750in}{1.417178in}}{\pgfqpoint{5.462794in}{1.422222in}}%
\pgfpathcurveto{\pgfqpoint{5.467838in}{1.427265in}}{\pgfqpoint{5.470671in}{1.434107in}}{\pgfqpoint{5.470671in}{1.441240in}}%
\pgfpathcurveto{\pgfqpoint{5.470671in}{1.448373in}}{\pgfqpoint{5.467838in}{1.455214in}}{\pgfqpoint{5.462794in}{1.460258in}}%
\pgfpathcurveto{\pgfqpoint{5.457750in}{1.465302in}}{\pgfqpoint{5.450909in}{1.468136in}}{\pgfqpoint{5.443776in}{1.468136in}}%
\pgfpathcurveto{\pgfqpoint{5.436643in}{1.468136in}}{\pgfqpoint{5.429801in}{1.465302in}}{\pgfqpoint{5.424758in}{1.460258in}}%
\pgfpathcurveto{\pgfqpoint{5.419714in}{1.455214in}}{\pgfqpoint{5.416880in}{1.448373in}}{\pgfqpoint{5.416880in}{1.441240in}}%
\pgfpathcurveto{\pgfqpoint{5.416880in}{1.434107in}}{\pgfqpoint{5.419714in}{1.427265in}}{\pgfqpoint{5.424758in}{1.422222in}}%
\pgfpathcurveto{\pgfqpoint{5.429801in}{1.417178in}}{\pgfqpoint{5.436643in}{1.414344in}}{\pgfqpoint{5.443776in}{1.414344in}}%
\pgfpathclose%
\pgfusepath{stroke,fill}%
\end{pgfscope}%
\begin{pgfscope}%
\pgfpathrectangle{\pgfqpoint{4.985294in}{0.500000in}}{\pgfqpoint{1.764706in}{1.700000in}}%
\pgfusepath{clip}%
\pgfsetbuttcap%
\pgfsetroundjoin%
\definecolor{currentfill}{rgb}{0.968105,0.786346,0.667739}%
\pgfsetfillcolor{currentfill}%
\pgfsetlinewidth{0.311001pt}%
\definecolor{currentstroke}{rgb}{1.000000,1.000000,1.000000}%
\pgfsetstrokecolor{currentstroke}%
\pgfsetdash{}{0pt}%
\pgfpathmoveto{\pgfqpoint{6.348667in}{1.525137in}}%
\pgfpathcurveto{\pgfqpoint{6.355800in}{1.525137in}}{\pgfqpoint{6.362642in}{1.527971in}}{\pgfqpoint{6.367685in}{1.533015in}}%
\pgfpathcurveto{\pgfqpoint{6.372729in}{1.538059in}}{\pgfqpoint{6.375563in}{1.544900in}}{\pgfqpoint{6.375563in}{1.552033in}}%
\pgfpathcurveto{\pgfqpoint{6.375563in}{1.559166in}}{\pgfqpoint{6.372729in}{1.566008in}}{\pgfqpoint{6.367685in}{1.571051in}}%
\pgfpathcurveto{\pgfqpoint{6.362642in}{1.576095in}}{\pgfqpoint{6.355800in}{1.578929in}}{\pgfqpoint{6.348667in}{1.578929in}}%
\pgfpathcurveto{\pgfqpoint{6.341534in}{1.578929in}}{\pgfqpoint{6.334693in}{1.576095in}}{\pgfqpoint{6.329649in}{1.571051in}}%
\pgfpathcurveto{\pgfqpoint{6.324605in}{1.566008in}}{\pgfqpoint{6.321771in}{1.559166in}}{\pgfqpoint{6.321771in}{1.552033in}}%
\pgfpathcurveto{\pgfqpoint{6.321771in}{1.544900in}}{\pgfqpoint{6.324605in}{1.538059in}}{\pgfqpoint{6.329649in}{1.533015in}}%
\pgfpathcurveto{\pgfqpoint{6.334693in}{1.527971in}}{\pgfqpoint{6.341534in}{1.525137in}}{\pgfqpoint{6.348667in}{1.525137in}}%
\pgfpathclose%
\pgfusepath{stroke,fill}%
\end{pgfscope}%
\begin{pgfscope}%
\pgfpathrectangle{\pgfqpoint{4.985294in}{0.500000in}}{\pgfqpoint{1.764706in}{1.700000in}}%
\pgfusepath{clip}%
\pgfsetbuttcap%
\pgfsetroundjoin%
\definecolor{currentfill}{rgb}{0.961433,0.573272,0.412036}%
\pgfsetfillcolor{currentfill}%
\pgfsetlinewidth{0.311001pt}%
\definecolor{currentstroke}{rgb}{1.000000,1.000000,1.000000}%
\pgfsetstrokecolor{currentstroke}%
\pgfsetdash{}{0pt}%
\pgfpathmoveto{\pgfqpoint{6.127682in}{1.771965in}}%
\pgfpathcurveto{\pgfqpoint{6.134814in}{1.771965in}}{\pgfqpoint{6.141656in}{1.774799in}}{\pgfqpoint{6.146700in}{1.779843in}}%
\pgfpathcurveto{\pgfqpoint{6.151743in}{1.784886in}}{\pgfqpoint{6.154577in}{1.791728in}}{\pgfqpoint{6.154577in}{1.798861in}}%
\pgfpathcurveto{\pgfqpoint{6.154577in}{1.805994in}}{\pgfqpoint{6.151743in}{1.812835in}}{\pgfqpoint{6.146700in}{1.817879in}}%
\pgfpathcurveto{\pgfqpoint{6.141656in}{1.822923in}}{\pgfqpoint{6.134814in}{1.825756in}}{\pgfqpoint{6.127682in}{1.825756in}}%
\pgfpathcurveto{\pgfqpoint{6.120549in}{1.825756in}}{\pgfqpoint{6.113707in}{1.822923in}}{\pgfqpoint{6.108663in}{1.817879in}}%
\pgfpathcurveto{\pgfqpoint{6.103620in}{1.812835in}}{\pgfqpoint{6.100786in}{1.805994in}}{\pgfqpoint{6.100786in}{1.798861in}}%
\pgfpathcurveto{\pgfqpoint{6.100786in}{1.791728in}}{\pgfqpoint{6.103620in}{1.784886in}}{\pgfqpoint{6.108663in}{1.779843in}}%
\pgfpathcurveto{\pgfqpoint{6.113707in}{1.774799in}}{\pgfqpoint{6.120549in}{1.771965in}}{\pgfqpoint{6.127682in}{1.771965in}}%
\pgfpathclose%
\pgfusepath{stroke,fill}%
\end{pgfscope}%
\begin{pgfscope}%
\pgfpathrectangle{\pgfqpoint{4.985294in}{0.500000in}}{\pgfqpoint{1.764706in}{1.700000in}}%
\pgfusepath{clip}%
\pgfsetbuttcap%
\pgfsetroundjoin%
\definecolor{currentfill}{rgb}{0.977657,0.891500,0.822809}%
\pgfsetfillcolor{currentfill}%
\pgfsetlinewidth{0.311001pt}%
\definecolor{currentstroke}{rgb}{1.000000,1.000000,1.000000}%
\pgfsetstrokecolor{currentstroke}%
\pgfsetdash{}{0pt}%
\pgfpathmoveto{\pgfqpoint{5.396884in}{1.230553in}}%
\pgfpathcurveto{\pgfqpoint{5.404017in}{1.230553in}}{\pgfqpoint{5.410859in}{1.233387in}}{\pgfqpoint{5.415903in}{1.238431in}}%
\pgfpathcurveto{\pgfqpoint{5.420946in}{1.243474in}}{\pgfqpoint{5.423780in}{1.250316in}}{\pgfqpoint{5.423780in}{1.257449in}}%
\pgfpathcurveto{\pgfqpoint{5.423780in}{1.264582in}}{\pgfqpoint{5.420946in}{1.271423in}}{\pgfqpoint{5.415903in}{1.276467in}}%
\pgfpathcurveto{\pgfqpoint{5.410859in}{1.281511in}}{\pgfqpoint{5.404017in}{1.284345in}}{\pgfqpoint{5.396884in}{1.284345in}}%
\pgfpathcurveto{\pgfqpoint{5.389752in}{1.284345in}}{\pgfqpoint{5.382910in}{1.281511in}}{\pgfqpoint{5.377866in}{1.276467in}}%
\pgfpathcurveto{\pgfqpoint{5.372823in}{1.271423in}}{\pgfqpoint{5.369989in}{1.264582in}}{\pgfqpoint{5.369989in}{1.257449in}}%
\pgfpathcurveto{\pgfqpoint{5.369989in}{1.250316in}}{\pgfqpoint{5.372823in}{1.243474in}}{\pgfqpoint{5.377866in}{1.238431in}}%
\pgfpathcurveto{\pgfqpoint{5.382910in}{1.233387in}}{\pgfqpoint{5.389752in}{1.230553in}}{\pgfqpoint{5.396884in}{1.230553in}}%
\pgfpathclose%
\pgfusepath{stroke,fill}%
\end{pgfscope}%
\begin{pgfscope}%
\pgfpathrectangle{\pgfqpoint{4.985294in}{0.500000in}}{\pgfqpoint{1.764706in}{1.700000in}}%
\pgfusepath{clip}%
\pgfsetbuttcap%
\pgfsetroundjoin%
\definecolor{currentfill}{rgb}{0.971694,0.833208,0.737161}%
\pgfsetfillcolor{currentfill}%
\pgfsetlinewidth{0.311001pt}%
\definecolor{currentstroke}{rgb}{1.000000,1.000000,1.000000}%
\pgfsetstrokecolor{currentstroke}%
\pgfsetdash{}{0pt}%
\pgfpathmoveto{\pgfqpoint{5.401037in}{1.522756in}}%
\pgfpathcurveto{\pgfqpoint{5.408170in}{1.522756in}}{\pgfqpoint{5.415011in}{1.525590in}}{\pgfqpoint{5.420055in}{1.530633in}}%
\pgfpathcurveto{\pgfqpoint{5.425099in}{1.535677in}}{\pgfqpoint{5.427933in}{1.542519in}}{\pgfqpoint{5.427933in}{1.549651in}}%
\pgfpathcurveto{\pgfqpoint{5.427933in}{1.556784in}}{\pgfqpoint{5.425099in}{1.563626in}}{\pgfqpoint{5.420055in}{1.568670in}}%
\pgfpathcurveto{\pgfqpoint{5.415011in}{1.573713in}}{\pgfqpoint{5.408170in}{1.576547in}}{\pgfqpoint{5.401037in}{1.576547in}}%
\pgfpathcurveto{\pgfqpoint{5.393904in}{1.576547in}}{\pgfqpoint{5.387062in}{1.573713in}}{\pgfqpoint{5.382019in}{1.568670in}}%
\pgfpathcurveto{\pgfqpoint{5.376975in}{1.563626in}}{\pgfqpoint{5.374141in}{1.556784in}}{\pgfqpoint{5.374141in}{1.549651in}}%
\pgfpathcurveto{\pgfqpoint{5.374141in}{1.542519in}}{\pgfqpoint{5.376975in}{1.535677in}}{\pgfqpoint{5.382019in}{1.530633in}}%
\pgfpathcurveto{\pgfqpoint{5.387062in}{1.525590in}}{\pgfqpoint{5.393904in}{1.522756in}}{\pgfqpoint{5.401037in}{1.522756in}}%
\pgfpathclose%
\pgfusepath{stroke,fill}%
\end{pgfscope}%
\begin{pgfscope}%
\pgfpathrectangle{\pgfqpoint{4.985294in}{0.500000in}}{\pgfqpoint{1.764706in}{1.700000in}}%
\pgfusepath{clip}%
\pgfsetbuttcap%
\pgfsetroundjoin%
\definecolor{currentfill}{rgb}{0.948235,0.413004,0.283323}%
\pgfsetfillcolor{currentfill}%
\pgfsetlinewidth{0.311001pt}%
\definecolor{currentstroke}{rgb}{1.000000,1.000000,1.000000}%
\pgfsetstrokecolor{currentstroke}%
\pgfsetdash{}{0pt}%
\pgfpathmoveto{\pgfqpoint{6.113301in}{1.134486in}}%
\pgfpathcurveto{\pgfqpoint{6.120434in}{1.134486in}}{\pgfqpoint{6.127276in}{1.137320in}}{\pgfqpoint{6.132319in}{1.142364in}}%
\pgfpathcurveto{\pgfqpoint{6.137363in}{1.147408in}}{\pgfqpoint{6.140197in}{1.154249in}}{\pgfqpoint{6.140197in}{1.161382in}}%
\pgfpathcurveto{\pgfqpoint{6.140197in}{1.168515in}}{\pgfqpoint{6.137363in}{1.175357in}}{\pgfqpoint{6.132319in}{1.180400in}}%
\pgfpathcurveto{\pgfqpoint{6.127276in}{1.185444in}}{\pgfqpoint{6.120434in}{1.188278in}}{\pgfqpoint{6.113301in}{1.188278in}}%
\pgfpathcurveto{\pgfqpoint{6.106168in}{1.188278in}}{\pgfqpoint{6.099327in}{1.185444in}}{\pgfqpoint{6.094283in}{1.180400in}}%
\pgfpathcurveto{\pgfqpoint{6.089239in}{1.175357in}}{\pgfqpoint{6.086406in}{1.168515in}}{\pgfqpoint{6.086406in}{1.161382in}}%
\pgfpathcurveto{\pgfqpoint{6.086406in}{1.154249in}}{\pgfqpoint{6.089239in}{1.147408in}}{\pgfqpoint{6.094283in}{1.142364in}}%
\pgfpathcurveto{\pgfqpoint{6.099327in}{1.137320in}}{\pgfqpoint{6.106168in}{1.134486in}}{\pgfqpoint{6.113301in}{1.134486in}}%
\pgfpathclose%
\pgfusepath{stroke,fill}%
\end{pgfscope}%
\begin{pgfscope}%
\pgfpathrectangle{\pgfqpoint{4.985294in}{0.500000in}}{\pgfqpoint{1.764706in}{1.700000in}}%
\pgfusepath{clip}%
\pgfsetbuttcap%
\pgfsetroundjoin%
\definecolor{currentfill}{rgb}{0.964679,0.682838,0.530002}%
\pgfsetfillcolor{currentfill}%
\pgfsetlinewidth{0.311001pt}%
\definecolor{currentstroke}{rgb}{1.000000,1.000000,1.000000}%
\pgfsetstrokecolor{currentstroke}%
\pgfsetdash{}{0pt}%
\pgfpathmoveto{\pgfqpoint{6.138148in}{1.053871in}}%
\pgfpathcurveto{\pgfqpoint{6.145281in}{1.053871in}}{\pgfqpoint{6.152122in}{1.056705in}}{\pgfqpoint{6.157166in}{1.061748in}}%
\pgfpathcurveto{\pgfqpoint{6.162210in}{1.066792in}}{\pgfqpoint{6.165044in}{1.073634in}}{\pgfqpoint{6.165044in}{1.080767in}}%
\pgfpathcurveto{\pgfqpoint{6.165044in}{1.087899in}}{\pgfqpoint{6.162210in}{1.094741in}}{\pgfqpoint{6.157166in}{1.099785in}}%
\pgfpathcurveto{\pgfqpoint{6.152122in}{1.104828in}}{\pgfqpoint{6.145281in}{1.107662in}}{\pgfqpoint{6.138148in}{1.107662in}}%
\pgfpathcurveto{\pgfqpoint{6.131015in}{1.107662in}}{\pgfqpoint{6.124173in}{1.104828in}}{\pgfqpoint{6.119130in}{1.099785in}}%
\pgfpathcurveto{\pgfqpoint{6.114086in}{1.094741in}}{\pgfqpoint{6.111252in}{1.087899in}}{\pgfqpoint{6.111252in}{1.080767in}}%
\pgfpathcurveto{\pgfqpoint{6.111252in}{1.073634in}}{\pgfqpoint{6.114086in}{1.066792in}}{\pgfqpoint{6.119130in}{1.061748in}}%
\pgfpathcurveto{\pgfqpoint{6.124173in}{1.056705in}}{\pgfqpoint{6.131015in}{1.053871in}}{\pgfqpoint{6.138148in}{1.053871in}}%
\pgfpathclose%
\pgfusepath{stroke,fill}%
\end{pgfscope}%
\begin{pgfscope}%
\pgfpathrectangle{\pgfqpoint{4.985294in}{0.500000in}}{\pgfqpoint{1.764706in}{1.700000in}}%
\pgfusepath{clip}%
\pgfsetbuttcap%
\pgfsetroundjoin%
\definecolor{currentfill}{rgb}{0.965753,0.732351,0.592427}%
\pgfsetfillcolor{currentfill}%
\pgfsetlinewidth{0.311001pt}%
\definecolor{currentstroke}{rgb}{1.000000,1.000000,1.000000}%
\pgfsetstrokecolor{currentstroke}%
\pgfsetdash{}{0pt}%
\pgfpathmoveto{\pgfqpoint{5.589531in}{1.021775in}}%
\pgfpathcurveto{\pgfqpoint{5.596664in}{1.021775in}}{\pgfqpoint{5.603505in}{1.024608in}}{\pgfqpoint{5.608549in}{1.029652in}}%
\pgfpathcurveto{\pgfqpoint{5.613593in}{1.034696in}}{\pgfqpoint{5.616426in}{1.041537in}}{\pgfqpoint{5.616426in}{1.048670in}}%
\pgfpathcurveto{\pgfqpoint{5.616426in}{1.055803in}}{\pgfqpoint{5.613593in}{1.062645in}}{\pgfqpoint{5.608549in}{1.067688in}}%
\pgfpathcurveto{\pgfqpoint{5.603505in}{1.072732in}}{\pgfqpoint{5.596664in}{1.075566in}}{\pgfqpoint{5.589531in}{1.075566in}}%
\pgfpathcurveto{\pgfqpoint{5.582398in}{1.075566in}}{\pgfqpoint{5.575556in}{1.072732in}}{\pgfqpoint{5.570513in}{1.067688in}}%
\pgfpathcurveto{\pgfqpoint{5.565469in}{1.062645in}}{\pgfqpoint{5.562635in}{1.055803in}}{\pgfqpoint{5.562635in}{1.048670in}}%
\pgfpathcurveto{\pgfqpoint{5.562635in}{1.041537in}}{\pgfqpoint{5.565469in}{1.034696in}}{\pgfqpoint{5.570513in}{1.029652in}}%
\pgfpathcurveto{\pgfqpoint{5.575556in}{1.024608in}}{\pgfqpoint{5.582398in}{1.021775in}}{\pgfqpoint{5.589531in}{1.021775in}}%
\pgfpathclose%
\pgfusepath{stroke,fill}%
\end{pgfscope}%
\begin{pgfscope}%
\pgfpathrectangle{\pgfqpoint{4.985294in}{0.500000in}}{\pgfqpoint{1.764706in}{1.700000in}}%
\pgfusepath{clip}%
\pgfsetbuttcap%
\pgfsetroundjoin%
\definecolor{currentfill}{rgb}{0.968105,0.786346,0.667739}%
\pgfsetfillcolor{currentfill}%
\pgfsetlinewidth{0.311001pt}%
\definecolor{currentstroke}{rgb}{1.000000,1.000000,1.000000}%
\pgfsetstrokecolor{currentstroke}%
\pgfsetdash{}{0pt}%
\pgfpathmoveto{\pgfqpoint{6.213119in}{1.242165in}}%
\pgfpathcurveto{\pgfqpoint{6.220252in}{1.242165in}}{\pgfqpoint{6.227093in}{1.244999in}}{\pgfqpoint{6.232137in}{1.250042in}}%
\pgfpathcurveto{\pgfqpoint{6.237181in}{1.255086in}}{\pgfqpoint{6.240015in}{1.261928in}}{\pgfqpoint{6.240015in}{1.269061in}}%
\pgfpathcurveto{\pgfqpoint{6.240015in}{1.276193in}}{\pgfqpoint{6.237181in}{1.283035in}}{\pgfqpoint{6.232137in}{1.288079in}}%
\pgfpathcurveto{\pgfqpoint{6.227093in}{1.293122in}}{\pgfqpoint{6.220252in}{1.295956in}}{\pgfqpoint{6.213119in}{1.295956in}}%
\pgfpathcurveto{\pgfqpoint{6.205986in}{1.295956in}}{\pgfqpoint{6.199144in}{1.293122in}}{\pgfqpoint{6.194101in}{1.288079in}}%
\pgfpathcurveto{\pgfqpoint{6.189057in}{1.283035in}}{\pgfqpoint{6.186223in}{1.276193in}}{\pgfqpoint{6.186223in}{1.269061in}}%
\pgfpathcurveto{\pgfqpoint{6.186223in}{1.261928in}}{\pgfqpoint{6.189057in}{1.255086in}}{\pgfqpoint{6.194101in}{1.250042in}}%
\pgfpathcurveto{\pgfqpoint{6.199144in}{1.244999in}}{\pgfqpoint{6.205986in}{1.242165in}}{\pgfqpoint{6.213119in}{1.242165in}}%
\pgfpathclose%
\pgfusepath{stroke,fill}%
\end{pgfscope}%
\begin{pgfscope}%
\pgfpathrectangle{\pgfqpoint{4.985294in}{0.500000in}}{\pgfqpoint{1.764706in}{1.700000in}}%
\pgfusepath{clip}%
\pgfsetbuttcap%
\pgfsetroundjoin%
\definecolor{currentfill}{rgb}{0.975018,0.868213,0.788710}%
\pgfsetfillcolor{currentfill}%
\pgfsetlinewidth{0.311001pt}%
\definecolor{currentstroke}{rgb}{1.000000,1.000000,1.000000}%
\pgfsetstrokecolor{currentstroke}%
\pgfsetdash{}{0pt}%
\pgfpathmoveto{\pgfqpoint{5.474539in}{1.423664in}}%
\pgfpathcurveto{\pgfqpoint{5.481672in}{1.423664in}}{\pgfqpoint{5.488513in}{1.426498in}}{\pgfqpoint{5.493557in}{1.431542in}}%
\pgfpathcurveto{\pgfqpoint{5.498601in}{1.436585in}}{\pgfqpoint{5.501435in}{1.443427in}}{\pgfqpoint{5.501435in}{1.450560in}}%
\pgfpathcurveto{\pgfqpoint{5.501435in}{1.457693in}}{\pgfqpoint{5.498601in}{1.464534in}}{\pgfqpoint{5.493557in}{1.469578in}}%
\pgfpathcurveto{\pgfqpoint{5.488513in}{1.474622in}}{\pgfqpoint{5.481672in}{1.477455in}}{\pgfqpoint{5.474539in}{1.477455in}}%
\pgfpathcurveto{\pgfqpoint{5.467406in}{1.477455in}}{\pgfqpoint{5.460564in}{1.474622in}}{\pgfqpoint{5.455521in}{1.469578in}}%
\pgfpathcurveto{\pgfqpoint{5.450477in}{1.464534in}}{\pgfqpoint{5.447643in}{1.457693in}}{\pgfqpoint{5.447643in}{1.450560in}}%
\pgfpathcurveto{\pgfqpoint{5.447643in}{1.443427in}}{\pgfqpoint{5.450477in}{1.436585in}}{\pgfqpoint{5.455521in}{1.431542in}}%
\pgfpathcurveto{\pgfqpoint{5.460564in}{1.426498in}}{\pgfqpoint{5.467406in}{1.423664in}}{\pgfqpoint{5.474539in}{1.423664in}}%
\pgfpathclose%
\pgfusepath{stroke,fill}%
\end{pgfscope}%
\begin{pgfscope}%
\pgfpathrectangle{\pgfqpoint{4.985294in}{0.500000in}}{\pgfqpoint{1.764706in}{1.700000in}}%
\pgfusepath{clip}%
\pgfsetbuttcap%
\pgfsetroundjoin%
\definecolor{currentfill}{rgb}{0.980678,0.914765,0.856766}%
\pgfsetfillcolor{currentfill}%
\pgfsetlinewidth{0.311001pt}%
\definecolor{currentstroke}{rgb}{1.000000,1.000000,1.000000}%
\pgfsetstrokecolor{currentstroke}%
\pgfsetdash{}{0pt}%
\pgfpathmoveto{\pgfqpoint{6.299862in}{1.318039in}}%
\pgfpathcurveto{\pgfqpoint{6.306995in}{1.318039in}}{\pgfqpoint{6.313837in}{1.320873in}}{\pgfqpoint{6.318880in}{1.325916in}}%
\pgfpathcurveto{\pgfqpoint{6.323924in}{1.330960in}}{\pgfqpoint{6.326758in}{1.337802in}}{\pgfqpoint{6.326758in}{1.344934in}}%
\pgfpathcurveto{\pgfqpoint{6.326758in}{1.352067in}}{\pgfqpoint{6.323924in}{1.358909in}}{\pgfqpoint{6.318880in}{1.363953in}}%
\pgfpathcurveto{\pgfqpoint{6.313837in}{1.368996in}}{\pgfqpoint{6.306995in}{1.371830in}}{\pgfqpoint{6.299862in}{1.371830in}}%
\pgfpathcurveto{\pgfqpoint{6.292729in}{1.371830in}}{\pgfqpoint{6.285888in}{1.368996in}}{\pgfqpoint{6.280844in}{1.363953in}}%
\pgfpathcurveto{\pgfqpoint{6.275800in}{1.358909in}}{\pgfqpoint{6.272966in}{1.352067in}}{\pgfqpoint{6.272966in}{1.344934in}}%
\pgfpathcurveto{\pgfqpoint{6.272966in}{1.337802in}}{\pgfqpoint{6.275800in}{1.330960in}}{\pgfqpoint{6.280844in}{1.325916in}}%
\pgfpathcurveto{\pgfqpoint{6.285888in}{1.320873in}}{\pgfqpoint{6.292729in}{1.318039in}}{\pgfqpoint{6.299862in}{1.318039in}}%
\pgfpathclose%
\pgfusepath{stroke,fill}%
\end{pgfscope}%
\begin{pgfscope}%
\pgfpathrectangle{\pgfqpoint{4.985294in}{0.500000in}}{\pgfqpoint{1.764706in}{1.700000in}}%
\pgfusepath{clip}%
\pgfsetbuttcap%
\pgfsetroundjoin%
\definecolor{currentfill}{rgb}{0.969359,0.803954,0.693832}%
\pgfsetfillcolor{currentfill}%
\pgfsetlinewidth{0.311001pt}%
\definecolor{currentstroke}{rgb}{1.000000,1.000000,1.000000}%
\pgfsetstrokecolor{currentstroke}%
\pgfsetdash{}{0pt}%
\pgfpathmoveto{\pgfqpoint{6.213540in}{1.495135in}}%
\pgfpathcurveto{\pgfqpoint{6.220672in}{1.495135in}}{\pgfqpoint{6.227514in}{1.497969in}}{\pgfqpoint{6.232558in}{1.503013in}}%
\pgfpathcurveto{\pgfqpoint{6.237601in}{1.508057in}}{\pgfqpoint{6.240435in}{1.514898in}}{\pgfqpoint{6.240435in}{1.522031in}}%
\pgfpathcurveto{\pgfqpoint{6.240435in}{1.529164in}}{\pgfqpoint{6.237601in}{1.536006in}}{\pgfqpoint{6.232558in}{1.541049in}}%
\pgfpathcurveto{\pgfqpoint{6.227514in}{1.546093in}}{\pgfqpoint{6.220672in}{1.548927in}}{\pgfqpoint{6.213540in}{1.548927in}}%
\pgfpathcurveto{\pgfqpoint{6.206407in}{1.548927in}}{\pgfqpoint{6.199565in}{1.546093in}}{\pgfqpoint{6.194521in}{1.541049in}}%
\pgfpathcurveto{\pgfqpoint{6.189478in}{1.536006in}}{\pgfqpoint{6.186644in}{1.529164in}}{\pgfqpoint{6.186644in}{1.522031in}}%
\pgfpathcurveto{\pgfqpoint{6.186644in}{1.514898in}}{\pgfqpoint{6.189478in}{1.508057in}}{\pgfqpoint{6.194521in}{1.503013in}}%
\pgfpathcurveto{\pgfqpoint{6.199565in}{1.497969in}}{\pgfqpoint{6.206407in}{1.495135in}}{\pgfqpoint{6.213540in}{1.495135in}}%
\pgfpathclose%
\pgfusepath{stroke,fill}%
\end{pgfscope}%
\begin{pgfscope}%
\pgfpathrectangle{\pgfqpoint{4.985294in}{0.500000in}}{\pgfqpoint{1.764706in}{1.700000in}}%
\pgfusepath{clip}%
\pgfsetbuttcap%
\pgfsetroundjoin%
\definecolor{currentfill}{rgb}{0.955103,0.477872,0.328626}%
\pgfsetfillcolor{currentfill}%
\pgfsetlinewidth{0.311001pt}%
\definecolor{currentstroke}{rgb}{1.000000,1.000000,1.000000}%
\pgfsetstrokecolor{currentstroke}%
\pgfsetdash{}{0pt}%
\pgfpathmoveto{\pgfqpoint{6.402725in}{1.583919in}}%
\pgfpathcurveto{\pgfqpoint{6.409857in}{1.583919in}}{\pgfqpoint{6.416699in}{1.586753in}}{\pgfqpoint{6.421743in}{1.591797in}}%
\pgfpathcurveto{\pgfqpoint{6.426786in}{1.596841in}}{\pgfqpoint{6.429620in}{1.603682in}}{\pgfqpoint{6.429620in}{1.610815in}}%
\pgfpathcurveto{\pgfqpoint{6.429620in}{1.617948in}}{\pgfqpoint{6.426786in}{1.624790in}}{\pgfqpoint{6.421743in}{1.629833in}}%
\pgfpathcurveto{\pgfqpoint{6.416699in}{1.634877in}}{\pgfqpoint{6.409857in}{1.637711in}}{\pgfqpoint{6.402725in}{1.637711in}}%
\pgfpathcurveto{\pgfqpoint{6.395592in}{1.637711in}}{\pgfqpoint{6.388750in}{1.634877in}}{\pgfqpoint{6.383707in}{1.629833in}}%
\pgfpathcurveto{\pgfqpoint{6.378663in}{1.624790in}}{\pgfqpoint{6.375829in}{1.617948in}}{\pgfqpoint{6.375829in}{1.610815in}}%
\pgfpathcurveto{\pgfqpoint{6.375829in}{1.603682in}}{\pgfqpoint{6.378663in}{1.596841in}}{\pgfqpoint{6.383707in}{1.591797in}}%
\pgfpathcurveto{\pgfqpoint{6.388750in}{1.586753in}}{\pgfqpoint{6.395592in}{1.583919in}}{\pgfqpoint{6.402725in}{1.583919in}}%
\pgfpathclose%
\pgfusepath{stroke,fill}%
\end{pgfscope}%
\begin{pgfscope}%
\pgfpathrectangle{\pgfqpoint{4.985294in}{0.500000in}}{\pgfqpoint{1.764706in}{1.700000in}}%
\pgfusepath{clip}%
\pgfsetbuttcap%
\pgfsetroundjoin%
\definecolor{currentfill}{rgb}{0.972726,0.844889,0.754401}%
\pgfsetfillcolor{currentfill}%
\pgfsetlinewidth{0.311001pt}%
\definecolor{currentstroke}{rgb}{1.000000,1.000000,1.000000}%
\pgfsetstrokecolor{currentstroke}%
\pgfsetdash{}{0pt}%
\pgfpathmoveto{\pgfqpoint{5.438461in}{1.057764in}}%
\pgfpathcurveto{\pgfqpoint{5.445594in}{1.057764in}}{\pgfqpoint{5.452435in}{1.060598in}}{\pgfqpoint{5.457479in}{1.065642in}}%
\pgfpathcurveto{\pgfqpoint{5.462523in}{1.070685in}}{\pgfqpoint{5.465356in}{1.077527in}}{\pgfqpoint{5.465356in}{1.084660in}}%
\pgfpathcurveto{\pgfqpoint{5.465356in}{1.091793in}}{\pgfqpoint{5.462523in}{1.098634in}}{\pgfqpoint{5.457479in}{1.103678in}}%
\pgfpathcurveto{\pgfqpoint{5.452435in}{1.108722in}}{\pgfqpoint{5.445594in}{1.111556in}}{\pgfqpoint{5.438461in}{1.111556in}}%
\pgfpathcurveto{\pgfqpoint{5.431328in}{1.111556in}}{\pgfqpoint{5.424486in}{1.108722in}}{\pgfqpoint{5.419443in}{1.103678in}}%
\pgfpathcurveto{\pgfqpoint{5.414399in}{1.098634in}}{\pgfqpoint{5.411565in}{1.091793in}}{\pgfqpoint{5.411565in}{1.084660in}}%
\pgfpathcurveto{\pgfqpoint{5.411565in}{1.077527in}}{\pgfqpoint{5.414399in}{1.070685in}}{\pgfqpoint{5.419443in}{1.065642in}}%
\pgfpathcurveto{\pgfqpoint{5.424486in}{1.060598in}}{\pgfqpoint{5.431328in}{1.057764in}}{\pgfqpoint{5.438461in}{1.057764in}}%
\pgfpathclose%
\pgfusepath{stroke,fill}%
\end{pgfscope}%
\begin{pgfscope}%
\pgfpathrectangle{\pgfqpoint{4.985294in}{0.500000in}}{\pgfqpoint{1.764706in}{1.700000in}}%
\pgfusepath{clip}%
\pgfsetbuttcap%
\pgfsetroundjoin%
\definecolor{currentfill}{rgb}{0.979891,0.908948,0.848279}%
\pgfsetfillcolor{currentfill}%
\pgfsetlinewidth{0.311001pt}%
\definecolor{currentstroke}{rgb}{1.000000,1.000000,1.000000}%
\pgfsetstrokecolor{currentstroke}%
\pgfsetdash{}{0pt}%
\pgfpathmoveto{\pgfqpoint{6.291939in}{1.279348in}}%
\pgfpathcurveto{\pgfqpoint{6.299072in}{1.279348in}}{\pgfqpoint{6.305913in}{1.282181in}}{\pgfqpoint{6.310957in}{1.287225in}}%
\pgfpathcurveto{\pgfqpoint{6.316001in}{1.292269in}}{\pgfqpoint{6.318835in}{1.299110in}}{\pgfqpoint{6.318835in}{1.306243in}}%
\pgfpathcurveto{\pgfqpoint{6.318835in}{1.313376in}}{\pgfqpoint{6.316001in}{1.320218in}}{\pgfqpoint{6.310957in}{1.325261in}}%
\pgfpathcurveto{\pgfqpoint{6.305913in}{1.330305in}}{\pgfqpoint{6.299072in}{1.333139in}}{\pgfqpoint{6.291939in}{1.333139in}}%
\pgfpathcurveto{\pgfqpoint{6.284806in}{1.333139in}}{\pgfqpoint{6.277964in}{1.330305in}}{\pgfqpoint{6.272921in}{1.325261in}}%
\pgfpathcurveto{\pgfqpoint{6.267877in}{1.320218in}}{\pgfqpoint{6.265043in}{1.313376in}}{\pgfqpoint{6.265043in}{1.306243in}}%
\pgfpathcurveto{\pgfqpoint{6.265043in}{1.299110in}}{\pgfqpoint{6.267877in}{1.292269in}}{\pgfqpoint{6.272921in}{1.287225in}}%
\pgfpathcurveto{\pgfqpoint{6.277964in}{1.282181in}}{\pgfqpoint{6.284806in}{1.279348in}}{\pgfqpoint{6.291939in}{1.279348in}}%
\pgfpathclose%
\pgfusepath{stroke,fill}%
\end{pgfscope}%
\begin{pgfscope}%
\pgfpathrectangle{\pgfqpoint{4.985294in}{0.500000in}}{\pgfqpoint{1.764706in}{1.700000in}}%
\pgfusepath{clip}%
\pgfsetbuttcap%
\pgfsetroundjoin%
\definecolor{currentfill}{rgb}{0.979891,0.908948,0.848279}%
\pgfsetfillcolor{currentfill}%
\pgfsetlinewidth{0.311001pt}%
\definecolor{currentstroke}{rgb}{1.000000,1.000000,1.000000}%
\pgfsetstrokecolor{currentstroke}%
\pgfsetdash{}{0pt}%
\pgfpathmoveto{\pgfqpoint{5.427648in}{1.270486in}}%
\pgfpathcurveto{\pgfqpoint{5.434781in}{1.270486in}}{\pgfqpoint{5.441623in}{1.273320in}}{\pgfqpoint{5.446667in}{1.278364in}}%
\pgfpathcurveto{\pgfqpoint{5.451710in}{1.283408in}}{\pgfqpoint{5.454544in}{1.290249in}}{\pgfqpoint{5.454544in}{1.297382in}}%
\pgfpathcurveto{\pgfqpoint{5.454544in}{1.304515in}}{\pgfqpoint{5.451710in}{1.311357in}}{\pgfqpoint{5.446667in}{1.316400in}}%
\pgfpathcurveto{\pgfqpoint{5.441623in}{1.321444in}}{\pgfqpoint{5.434781in}{1.324278in}}{\pgfqpoint{5.427648in}{1.324278in}}%
\pgfpathcurveto{\pgfqpoint{5.420516in}{1.324278in}}{\pgfqpoint{5.413674in}{1.321444in}}{\pgfqpoint{5.408630in}{1.316400in}}%
\pgfpathcurveto{\pgfqpoint{5.403587in}{1.311357in}}{\pgfqpoint{5.400753in}{1.304515in}}{\pgfqpoint{5.400753in}{1.297382in}}%
\pgfpathcurveto{\pgfqpoint{5.400753in}{1.290249in}}{\pgfqpoint{5.403587in}{1.283408in}}{\pgfqpoint{5.408630in}{1.278364in}}%
\pgfpathcurveto{\pgfqpoint{5.413674in}{1.273320in}}{\pgfqpoint{5.420516in}{1.270486in}}{\pgfqpoint{5.427648in}{1.270486in}}%
\pgfpathclose%
\pgfusepath{stroke,fill}%
\end{pgfscope}%
\begin{pgfscope}%
\pgfpathrectangle{\pgfqpoint{4.985294in}{0.500000in}}{\pgfqpoint{1.764706in}{1.700000in}}%
\pgfusepath{clip}%
\pgfsetbuttcap%
\pgfsetroundjoin%
\definecolor{currentfill}{rgb}{0.971694,0.833208,0.737161}%
\pgfsetfillcolor{currentfill}%
\pgfsetlinewidth{0.311001pt}%
\definecolor{currentstroke}{rgb}{1.000000,1.000000,1.000000}%
\pgfsetstrokecolor{currentstroke}%
\pgfsetdash{}{0pt}%
\pgfpathmoveto{\pgfqpoint{5.387708in}{1.148807in}}%
\pgfpathcurveto{\pgfqpoint{5.394841in}{1.148807in}}{\pgfqpoint{5.401683in}{1.151641in}}{\pgfqpoint{5.406726in}{1.156684in}}%
\pgfpathcurveto{\pgfqpoint{5.411770in}{1.161728in}}{\pgfqpoint{5.414604in}{1.168570in}}{\pgfqpoint{5.414604in}{1.175702in}}%
\pgfpathcurveto{\pgfqpoint{5.414604in}{1.182835in}}{\pgfqpoint{5.411770in}{1.189677in}}{\pgfqpoint{5.406726in}{1.194721in}}%
\pgfpathcurveto{\pgfqpoint{5.401683in}{1.199764in}}{\pgfqpoint{5.394841in}{1.202598in}}{\pgfqpoint{5.387708in}{1.202598in}}%
\pgfpathcurveto{\pgfqpoint{5.380575in}{1.202598in}}{\pgfqpoint{5.373734in}{1.199764in}}{\pgfqpoint{5.368690in}{1.194721in}}%
\pgfpathcurveto{\pgfqpoint{5.363646in}{1.189677in}}{\pgfqpoint{5.360813in}{1.182835in}}{\pgfqpoint{5.360813in}{1.175702in}}%
\pgfpathcurveto{\pgfqpoint{5.360813in}{1.168570in}}{\pgfqpoint{5.363646in}{1.161728in}}{\pgfqpoint{5.368690in}{1.156684in}}%
\pgfpathcurveto{\pgfqpoint{5.373734in}{1.151641in}}{\pgfqpoint{5.380575in}{1.148807in}}{\pgfqpoint{5.387708in}{1.148807in}}%
\pgfpathclose%
\pgfusepath{stroke,fill}%
\end{pgfscope}%
\begin{pgfscope}%
\pgfpathrectangle{\pgfqpoint{4.985294in}{0.500000in}}{\pgfqpoint{1.764706in}{1.700000in}}%
\pgfusepath{clip}%
\pgfsetbuttcap%
\pgfsetroundjoin%
\definecolor{currentfill}{rgb}{0.965302,0.713942,0.568499}%
\pgfsetfillcolor{currentfill}%
\pgfsetlinewidth{0.311001pt}%
\definecolor{currentstroke}{rgb}{1.000000,1.000000,1.000000}%
\pgfsetstrokecolor{currentstroke}%
\pgfsetdash{}{0pt}%
\pgfpathmoveto{\pgfqpoint{5.487205in}{1.709049in}}%
\pgfpathcurveto{\pgfqpoint{5.494338in}{1.709049in}}{\pgfqpoint{5.501179in}{1.711883in}}{\pgfqpoint{5.506223in}{1.716927in}}%
\pgfpathcurveto{\pgfqpoint{5.511267in}{1.721971in}}{\pgfqpoint{5.514101in}{1.728812in}}{\pgfqpoint{5.514101in}{1.735945in}}%
\pgfpathcurveto{\pgfqpoint{5.514101in}{1.743078in}}{\pgfqpoint{5.511267in}{1.749920in}}{\pgfqpoint{5.506223in}{1.754963in}}%
\pgfpathcurveto{\pgfqpoint{5.501179in}{1.760007in}}{\pgfqpoint{5.494338in}{1.762841in}}{\pgfqpoint{5.487205in}{1.762841in}}%
\pgfpathcurveto{\pgfqpoint{5.480072in}{1.762841in}}{\pgfqpoint{5.473230in}{1.760007in}}{\pgfqpoint{5.468187in}{1.754963in}}%
\pgfpathcurveto{\pgfqpoint{5.463143in}{1.749920in}}{\pgfqpoint{5.460309in}{1.743078in}}{\pgfqpoint{5.460309in}{1.735945in}}%
\pgfpathcurveto{\pgfqpoint{5.460309in}{1.728812in}}{\pgfqpoint{5.463143in}{1.721971in}}{\pgfqpoint{5.468187in}{1.716927in}}%
\pgfpathcurveto{\pgfqpoint{5.473230in}{1.711883in}}{\pgfqpoint{5.480072in}{1.709049in}}{\pgfqpoint{5.487205in}{1.709049in}}%
\pgfpathclose%
\pgfusepath{stroke,fill}%
\end{pgfscope}%
\begin{pgfscope}%
\pgfpathrectangle{\pgfqpoint{4.985294in}{0.500000in}}{\pgfqpoint{1.764706in}{1.700000in}}%
\pgfusepath{clip}%
\pgfsetbuttcap%
\pgfsetroundjoin%
\definecolor{currentfill}{rgb}{0.975018,0.868213,0.788710}%
\pgfsetfillcolor{currentfill}%
\pgfsetlinewidth{0.311001pt}%
\definecolor{currentstroke}{rgb}{1.000000,1.000000,1.000000}%
\pgfsetstrokecolor{currentstroke}%
\pgfsetdash{}{0pt}%
\pgfpathmoveto{\pgfqpoint{6.251629in}{1.517438in}}%
\pgfpathcurveto{\pgfqpoint{6.258762in}{1.517438in}}{\pgfqpoint{6.265604in}{1.520272in}}{\pgfqpoint{6.270647in}{1.525315in}}%
\pgfpathcurveto{\pgfqpoint{6.275691in}{1.530359in}}{\pgfqpoint{6.278525in}{1.537201in}}{\pgfqpoint{6.278525in}{1.544334in}}%
\pgfpathcurveto{\pgfqpoint{6.278525in}{1.551466in}}{\pgfqpoint{6.275691in}{1.558308in}}{\pgfqpoint{6.270647in}{1.563352in}}%
\pgfpathcurveto{\pgfqpoint{6.265604in}{1.568395in}}{\pgfqpoint{6.258762in}{1.571229in}}{\pgfqpoint{6.251629in}{1.571229in}}%
\pgfpathcurveto{\pgfqpoint{6.244496in}{1.571229in}}{\pgfqpoint{6.237655in}{1.568395in}}{\pgfqpoint{6.232611in}{1.563352in}}%
\pgfpathcurveto{\pgfqpoint{6.227567in}{1.558308in}}{\pgfqpoint{6.224734in}{1.551466in}}{\pgfqpoint{6.224734in}{1.544334in}}%
\pgfpathcurveto{\pgfqpoint{6.224734in}{1.537201in}}{\pgfqpoint{6.227567in}{1.530359in}}{\pgfqpoint{6.232611in}{1.525315in}}%
\pgfpathcurveto{\pgfqpoint{6.237655in}{1.520272in}}{\pgfqpoint{6.244496in}{1.517438in}}{\pgfqpoint{6.251629in}{1.517438in}}%
\pgfpathclose%
\pgfusepath{stroke,fill}%
\end{pgfscope}%
\begin{pgfscope}%
\pgfpathrectangle{\pgfqpoint{4.985294in}{0.500000in}}{\pgfqpoint{1.764706in}{1.700000in}}%
\pgfusepath{clip}%
\pgfsetbuttcap%
\pgfsetroundjoin%
\definecolor{currentfill}{rgb}{0.905301,0.238545,0.247481}%
\pgfsetfillcolor{currentfill}%
\pgfsetlinewidth{0.311001pt}%
\definecolor{currentstroke}{rgb}{1.000000,1.000000,1.000000}%
\pgfsetstrokecolor{currentstroke}%
\pgfsetdash{}{0pt}%
\pgfpathmoveto{\pgfqpoint{6.379363in}{0.954240in}}%
\pgfpathcurveto{\pgfqpoint{6.386495in}{0.954240in}}{\pgfqpoint{6.393337in}{0.957074in}}{\pgfqpoint{6.398381in}{0.962117in}}%
\pgfpathcurveto{\pgfqpoint{6.403424in}{0.967161in}}{\pgfqpoint{6.406258in}{0.974003in}}{\pgfqpoint{6.406258in}{0.981136in}}%
\pgfpathcurveto{\pgfqpoint{6.406258in}{0.988268in}}{\pgfqpoint{6.403424in}{0.995110in}}{\pgfqpoint{6.398381in}{1.000154in}}%
\pgfpathcurveto{\pgfqpoint{6.393337in}{1.005197in}}{\pgfqpoint{6.386495in}{1.008031in}}{\pgfqpoint{6.379363in}{1.008031in}}%
\pgfpathcurveto{\pgfqpoint{6.372230in}{1.008031in}}{\pgfqpoint{6.365388in}{1.005197in}}{\pgfqpoint{6.360344in}{1.000154in}}%
\pgfpathcurveto{\pgfqpoint{6.355301in}{0.995110in}}{\pgfqpoint{6.352467in}{0.988268in}}{\pgfqpoint{6.352467in}{0.981136in}}%
\pgfpathcurveto{\pgfqpoint{6.352467in}{0.974003in}}{\pgfqpoint{6.355301in}{0.967161in}}{\pgfqpoint{6.360344in}{0.962117in}}%
\pgfpathcurveto{\pgfqpoint{6.365388in}{0.957074in}}{\pgfqpoint{6.372230in}{0.954240in}}{\pgfqpoint{6.379363in}{0.954240in}}%
\pgfpathclose%
\pgfusepath{stroke,fill}%
\end{pgfscope}%
\begin{pgfscope}%
\pgfpathrectangle{\pgfqpoint{4.985294in}{0.500000in}}{\pgfqpoint{1.764706in}{1.700000in}}%
\pgfusepath{clip}%
\pgfsetbuttcap%
\pgfsetroundjoin%
\definecolor{currentfill}{rgb}{0.965928,0.738443,0.600540}%
\pgfsetfillcolor{currentfill}%
\pgfsetlinewidth{0.311001pt}%
\definecolor{currentstroke}{rgb}{1.000000,1.000000,1.000000}%
\pgfsetstrokecolor{currentstroke}%
\pgfsetdash{}{0pt}%
\pgfpathmoveto{\pgfqpoint{6.180256in}{0.927858in}}%
\pgfpathcurveto{\pgfqpoint{6.187389in}{0.927858in}}{\pgfqpoint{6.194230in}{0.930692in}}{\pgfqpoint{6.199274in}{0.935736in}}%
\pgfpathcurveto{\pgfqpoint{6.204317in}{0.940779in}}{\pgfqpoint{6.207151in}{0.947621in}}{\pgfqpoint{6.207151in}{0.954754in}}%
\pgfpathcurveto{\pgfqpoint{6.207151in}{0.961887in}}{\pgfqpoint{6.204317in}{0.968728in}}{\pgfqpoint{6.199274in}{0.973772in}}%
\pgfpathcurveto{\pgfqpoint{6.194230in}{0.978816in}}{\pgfqpoint{6.187389in}{0.981650in}}{\pgfqpoint{6.180256in}{0.981650in}}%
\pgfpathcurveto{\pgfqpoint{6.173123in}{0.981650in}}{\pgfqpoint{6.166281in}{0.978816in}}{\pgfqpoint{6.161238in}{0.973772in}}%
\pgfpathcurveto{\pgfqpoint{6.156194in}{0.968728in}}{\pgfqpoint{6.153360in}{0.961887in}}{\pgfqpoint{6.153360in}{0.954754in}}%
\pgfpathcurveto{\pgfqpoint{6.153360in}{0.947621in}}{\pgfqpoint{6.156194in}{0.940779in}}{\pgfqpoint{6.161238in}{0.935736in}}%
\pgfpathcurveto{\pgfqpoint{6.166281in}{0.930692in}}{\pgfqpoint{6.173123in}{0.927858in}}{\pgfqpoint{6.180256in}{0.927858in}}%
\pgfpathclose%
\pgfusepath{stroke,fill}%
\end{pgfscope}%
\begin{pgfscope}%
\pgfpathrectangle{\pgfqpoint{4.985294in}{0.500000in}}{\pgfqpoint{1.764706in}{1.700000in}}%
\pgfusepath{clip}%
\pgfsetbuttcap%
\pgfsetroundjoin%
\definecolor{currentfill}{rgb}{0.958331,0.519463,0.362986}%
\pgfsetfillcolor{currentfill}%
\pgfsetlinewidth{0.311001pt}%
\definecolor{currentstroke}{rgb}{1.000000,1.000000,1.000000}%
\pgfsetstrokecolor{currentstroke}%
\pgfsetdash{}{0pt}%
\pgfpathmoveto{\pgfqpoint{5.290332in}{1.358448in}}%
\pgfpathcurveto{\pgfqpoint{5.297465in}{1.358448in}}{\pgfqpoint{5.304307in}{1.361282in}}{\pgfqpoint{5.309350in}{1.366326in}}%
\pgfpathcurveto{\pgfqpoint{5.314394in}{1.371369in}}{\pgfqpoint{5.317228in}{1.378211in}}{\pgfqpoint{5.317228in}{1.385344in}}%
\pgfpathcurveto{\pgfqpoint{5.317228in}{1.392477in}}{\pgfqpoint{5.314394in}{1.399318in}}{\pgfqpoint{5.309350in}{1.404362in}}%
\pgfpathcurveto{\pgfqpoint{5.304307in}{1.409406in}}{\pgfqpoint{5.297465in}{1.412240in}}{\pgfqpoint{5.290332in}{1.412240in}}%
\pgfpathcurveto{\pgfqpoint{5.283199in}{1.412240in}}{\pgfqpoint{5.276358in}{1.409406in}}{\pgfqpoint{5.271314in}{1.404362in}}%
\pgfpathcurveto{\pgfqpoint{5.266270in}{1.399318in}}{\pgfqpoint{5.263436in}{1.392477in}}{\pgfqpoint{5.263436in}{1.385344in}}%
\pgfpathcurveto{\pgfqpoint{5.263436in}{1.378211in}}{\pgfqpoint{5.266270in}{1.371369in}}{\pgfqpoint{5.271314in}{1.366326in}}%
\pgfpathcurveto{\pgfqpoint{5.276358in}{1.361282in}}{\pgfqpoint{5.283199in}{1.358448in}}{\pgfqpoint{5.290332in}{1.358448in}}%
\pgfpathclose%
\pgfusepath{stroke,fill}%
\end{pgfscope}%
\begin{pgfscope}%
\pgfpathrectangle{\pgfqpoint{4.985294in}{0.500000in}}{\pgfqpoint{1.764706in}{1.700000in}}%
\pgfusepath{clip}%
\pgfsetbuttcap%
\pgfsetroundjoin%
\definecolor{currentfill}{rgb}{0.981377,0.920617,0.865369}%
\pgfsetfillcolor{currentfill}%
\pgfsetlinewidth{0.311001pt}%
\definecolor{currentstroke}{rgb}{1.000000,1.000000,1.000000}%
\pgfsetstrokecolor{currentstroke}%
\pgfsetdash{}{0pt}%
\pgfpathmoveto{\pgfqpoint{6.304263in}{1.462937in}}%
\pgfpathcurveto{\pgfqpoint{6.311396in}{1.462937in}}{\pgfqpoint{6.318238in}{1.465771in}}{\pgfqpoint{6.323281in}{1.470815in}}%
\pgfpathcurveto{\pgfqpoint{6.328325in}{1.475858in}}{\pgfqpoint{6.331159in}{1.482700in}}{\pgfqpoint{6.331159in}{1.489833in}}%
\pgfpathcurveto{\pgfqpoint{6.331159in}{1.496966in}}{\pgfqpoint{6.328325in}{1.503807in}}{\pgfqpoint{6.323281in}{1.508851in}}%
\pgfpathcurveto{\pgfqpoint{6.318238in}{1.513895in}}{\pgfqpoint{6.311396in}{1.516729in}}{\pgfqpoint{6.304263in}{1.516729in}}%
\pgfpathcurveto{\pgfqpoint{6.297131in}{1.516729in}}{\pgfqpoint{6.290289in}{1.513895in}}{\pgfqpoint{6.285245in}{1.508851in}}%
\pgfpathcurveto{\pgfqpoint{6.280202in}{1.503807in}}{\pgfqpoint{6.277368in}{1.496966in}}{\pgfqpoint{6.277368in}{1.489833in}}%
\pgfpathcurveto{\pgfqpoint{6.277368in}{1.482700in}}{\pgfqpoint{6.280202in}{1.475858in}}{\pgfqpoint{6.285245in}{1.470815in}}%
\pgfpathcurveto{\pgfqpoint{6.290289in}{1.465771in}}{\pgfqpoint{6.297131in}{1.462937in}}{\pgfqpoint{6.304263in}{1.462937in}}%
\pgfpathclose%
\pgfusepath{stroke,fill}%
\end{pgfscope}%
\begin{pgfscope}%
\pgfpathrectangle{\pgfqpoint{4.985294in}{0.500000in}}{\pgfqpoint{1.764706in}{1.700000in}}%
\pgfusepath{clip}%
\pgfsetbuttcap%
\pgfsetroundjoin%
\definecolor{currentfill}{rgb}{0.977657,0.891500,0.822809}%
\pgfsetfillcolor{currentfill}%
\pgfsetlinewidth{0.311001pt}%
\definecolor{currentstroke}{rgb}{1.000000,1.000000,1.000000}%
\pgfsetstrokecolor{currentstroke}%
\pgfsetdash{}{0pt}%
\pgfpathmoveto{\pgfqpoint{5.433102in}{1.493239in}}%
\pgfpathcurveto{\pgfqpoint{5.440234in}{1.493239in}}{\pgfqpoint{5.447076in}{1.496073in}}{\pgfqpoint{5.452120in}{1.501117in}}%
\pgfpathcurveto{\pgfqpoint{5.457163in}{1.506161in}}{\pgfqpoint{5.459997in}{1.513002in}}{\pgfqpoint{5.459997in}{1.520135in}}%
\pgfpathcurveto{\pgfqpoint{5.459997in}{1.527268in}}{\pgfqpoint{5.457163in}{1.534110in}}{\pgfqpoint{5.452120in}{1.539153in}}%
\pgfpathcurveto{\pgfqpoint{5.447076in}{1.544197in}}{\pgfqpoint{5.440234in}{1.547031in}}{\pgfqpoint{5.433102in}{1.547031in}}%
\pgfpathcurveto{\pgfqpoint{5.425969in}{1.547031in}}{\pgfqpoint{5.419127in}{1.544197in}}{\pgfqpoint{5.414084in}{1.539153in}}%
\pgfpathcurveto{\pgfqpoint{5.409040in}{1.534110in}}{\pgfqpoint{5.406206in}{1.527268in}}{\pgfqpoint{5.406206in}{1.520135in}}%
\pgfpathcurveto{\pgfqpoint{5.406206in}{1.513002in}}{\pgfqpoint{5.409040in}{1.506161in}}{\pgfqpoint{5.414084in}{1.501117in}}%
\pgfpathcurveto{\pgfqpoint{5.419127in}{1.496073in}}{\pgfqpoint{5.425969in}{1.493239in}}{\pgfqpoint{5.433102in}{1.493239in}}%
\pgfpathclose%
\pgfusepath{stroke,fill}%
\end{pgfscope}%
\begin{pgfscope}%
\pgfpathrectangle{\pgfqpoint{4.985294in}{0.500000in}}{\pgfqpoint{1.764706in}{1.700000in}}%
\pgfusepath{clip}%
\pgfsetbuttcap%
\pgfsetroundjoin%
\definecolor{currentfill}{rgb}{0.968931,0.798091,0.685123}%
\pgfsetfillcolor{currentfill}%
\pgfsetlinewidth{0.311001pt}%
\definecolor{currentstroke}{rgb}{1.000000,1.000000,1.000000}%
\pgfsetstrokecolor{currentstroke}%
\pgfsetdash{}{0pt}%
\pgfpathmoveto{\pgfqpoint{6.284919in}{1.025435in}}%
\pgfpathcurveto{\pgfqpoint{6.292052in}{1.025435in}}{\pgfqpoint{6.298894in}{1.028269in}}{\pgfqpoint{6.303937in}{1.033313in}}%
\pgfpathcurveto{\pgfqpoint{6.308981in}{1.038356in}}{\pgfqpoint{6.311815in}{1.045198in}}{\pgfqpoint{6.311815in}{1.052331in}}%
\pgfpathcurveto{\pgfqpoint{6.311815in}{1.059464in}}{\pgfqpoint{6.308981in}{1.066305in}}{\pgfqpoint{6.303937in}{1.071349in}}%
\pgfpathcurveto{\pgfqpoint{6.298894in}{1.076393in}}{\pgfqpoint{6.292052in}{1.079226in}}{\pgfqpoint{6.284919in}{1.079226in}}%
\pgfpathcurveto{\pgfqpoint{6.277786in}{1.079226in}}{\pgfqpoint{6.270945in}{1.076393in}}{\pgfqpoint{6.265901in}{1.071349in}}%
\pgfpathcurveto{\pgfqpoint{6.260857in}{1.066305in}}{\pgfqpoint{6.258023in}{1.059464in}}{\pgfqpoint{6.258023in}{1.052331in}}%
\pgfpathcurveto{\pgfqpoint{6.258023in}{1.045198in}}{\pgfqpoint{6.260857in}{1.038356in}}{\pgfqpoint{6.265901in}{1.033313in}}%
\pgfpathcurveto{\pgfqpoint{6.270945in}{1.028269in}}{\pgfqpoint{6.277786in}{1.025435in}}{\pgfqpoint{6.284919in}{1.025435in}}%
\pgfpathclose%
\pgfusepath{stroke,fill}%
\end{pgfscope}%
\begin{pgfscope}%
\pgfpathrectangle{\pgfqpoint{4.985294in}{0.500000in}}{\pgfqpoint{1.764706in}{1.700000in}}%
\pgfusepath{clip}%
\pgfsetbuttcap%
\pgfsetroundjoin%
\definecolor{currentfill}{rgb}{0.973271,0.850724,0.762998}%
\pgfsetfillcolor{currentfill}%
\pgfsetlinewidth{0.311001pt}%
\definecolor{currentstroke}{rgb}{1.000000,1.000000,1.000000}%
\pgfsetstrokecolor{currentstroke}%
\pgfsetdash{}{0pt}%
\pgfpathmoveto{\pgfqpoint{6.323881in}{1.121250in}}%
\pgfpathcurveto{\pgfqpoint{6.331014in}{1.121250in}}{\pgfqpoint{6.337856in}{1.124084in}}{\pgfqpoint{6.342900in}{1.129128in}}%
\pgfpathcurveto{\pgfqpoint{6.347943in}{1.134171in}}{\pgfqpoint{6.350777in}{1.141013in}}{\pgfqpoint{6.350777in}{1.148146in}}%
\pgfpathcurveto{\pgfqpoint{6.350777in}{1.155279in}}{\pgfqpoint{6.347943in}{1.162120in}}{\pgfqpoint{6.342900in}{1.167164in}}%
\pgfpathcurveto{\pgfqpoint{6.337856in}{1.172208in}}{\pgfqpoint{6.331014in}{1.175042in}}{\pgfqpoint{6.323881in}{1.175042in}}%
\pgfpathcurveto{\pgfqpoint{6.316749in}{1.175042in}}{\pgfqpoint{6.309907in}{1.172208in}}{\pgfqpoint{6.304863in}{1.167164in}}%
\pgfpathcurveto{\pgfqpoint{6.299820in}{1.162120in}}{\pgfqpoint{6.296986in}{1.155279in}}{\pgfqpoint{6.296986in}{1.148146in}}%
\pgfpathcurveto{\pgfqpoint{6.296986in}{1.141013in}}{\pgfqpoint{6.299820in}{1.134171in}}{\pgfqpoint{6.304863in}{1.129128in}}%
\pgfpathcurveto{\pgfqpoint{6.309907in}{1.124084in}}{\pgfqpoint{6.316749in}{1.121250in}}{\pgfqpoint{6.323881in}{1.121250in}}%
\pgfpathclose%
\pgfusepath{stroke,fill}%
\end{pgfscope}%
\begin{pgfscope}%
\pgfpathrectangle{\pgfqpoint{4.985294in}{0.500000in}}{\pgfqpoint{1.764706in}{1.700000in}}%
\pgfusepath{clip}%
\pgfsetbuttcap%
\pgfsetroundjoin%
\definecolor{currentfill}{rgb}{0.973832,0.856556,0.771584}%
\pgfsetfillcolor{currentfill}%
\pgfsetlinewidth{0.311001pt}%
\definecolor{currentstroke}{rgb}{1.000000,1.000000,1.000000}%
\pgfsetstrokecolor{currentstroke}%
\pgfsetdash{}{0pt}%
\pgfpathmoveto{\pgfqpoint{6.304421in}{1.577877in}}%
\pgfpathcurveto{\pgfqpoint{6.311554in}{1.577877in}}{\pgfqpoint{6.318395in}{1.580711in}}{\pgfqpoint{6.323439in}{1.585755in}}%
\pgfpathcurveto{\pgfqpoint{6.328483in}{1.590799in}}{\pgfqpoint{6.331316in}{1.597640in}}{\pgfqpoint{6.331316in}{1.604773in}}%
\pgfpathcurveto{\pgfqpoint{6.331316in}{1.611906in}}{\pgfqpoint{6.328483in}{1.618748in}}{\pgfqpoint{6.323439in}{1.623791in}}%
\pgfpathcurveto{\pgfqpoint{6.318395in}{1.628835in}}{\pgfqpoint{6.311554in}{1.631669in}}{\pgfqpoint{6.304421in}{1.631669in}}%
\pgfpathcurveto{\pgfqpoint{6.297288in}{1.631669in}}{\pgfqpoint{6.290446in}{1.628835in}}{\pgfqpoint{6.285403in}{1.623791in}}%
\pgfpathcurveto{\pgfqpoint{6.280359in}{1.618748in}}{\pgfqpoint{6.277525in}{1.611906in}}{\pgfqpoint{6.277525in}{1.604773in}}%
\pgfpathcurveto{\pgfqpoint{6.277525in}{1.597640in}}{\pgfqpoint{6.280359in}{1.590799in}}{\pgfqpoint{6.285403in}{1.585755in}}%
\pgfpathcurveto{\pgfqpoint{6.290446in}{1.580711in}}{\pgfqpoint{6.297288in}{1.577877in}}{\pgfqpoint{6.304421in}{1.577877in}}%
\pgfpathclose%
\pgfusepath{stroke,fill}%
\end{pgfscope}%
\begin{pgfscope}%
\pgfpathrectangle{\pgfqpoint{4.985294in}{0.500000in}}{\pgfqpoint{1.764706in}{1.700000in}}%
\pgfusepath{clip}%
\pgfsetbuttcap%
\pgfsetroundjoin%
\definecolor{currentfill}{rgb}{0.963379,0.625574,0.465113}%
\pgfsetfillcolor{currentfill}%
\pgfsetlinewidth{0.311001pt}%
\definecolor{currentstroke}{rgb}{1.000000,1.000000,1.000000}%
\pgfsetstrokecolor{currentstroke}%
\pgfsetdash{}{0pt}%
\pgfpathmoveto{\pgfqpoint{5.547318in}{1.213632in}}%
\pgfpathcurveto{\pgfqpoint{5.554450in}{1.213632in}}{\pgfqpoint{5.561292in}{1.216466in}}{\pgfqpoint{5.566336in}{1.221510in}}%
\pgfpathcurveto{\pgfqpoint{5.571379in}{1.226554in}}{\pgfqpoint{5.574213in}{1.233395in}}{\pgfqpoint{5.574213in}{1.240528in}}%
\pgfpathcurveto{\pgfqpoint{5.574213in}{1.247661in}}{\pgfqpoint{5.571379in}{1.254503in}}{\pgfqpoint{5.566336in}{1.259546in}}%
\pgfpathcurveto{\pgfqpoint{5.561292in}{1.264590in}}{\pgfqpoint{5.554450in}{1.267424in}}{\pgfqpoint{5.547318in}{1.267424in}}%
\pgfpathcurveto{\pgfqpoint{5.540185in}{1.267424in}}{\pgfqpoint{5.533343in}{1.264590in}}{\pgfqpoint{5.528299in}{1.259546in}}%
\pgfpathcurveto{\pgfqpoint{5.523256in}{1.254503in}}{\pgfqpoint{5.520422in}{1.247661in}}{\pgfqpoint{5.520422in}{1.240528in}}%
\pgfpathcurveto{\pgfqpoint{5.520422in}{1.233395in}}{\pgfqpoint{5.523256in}{1.226554in}}{\pgfqpoint{5.528299in}{1.221510in}}%
\pgfpathcurveto{\pgfqpoint{5.533343in}{1.216466in}}{\pgfqpoint{5.540185in}{1.213632in}}{\pgfqpoint{5.547318in}{1.213632in}}%
\pgfpathclose%
\pgfusepath{stroke,fill}%
\end{pgfscope}%
\begin{pgfscope}%
\pgfpathrectangle{\pgfqpoint{4.985294in}{0.500000in}}{\pgfqpoint{1.764706in}{1.700000in}}%
\pgfusepath{clip}%
\pgfsetbuttcap%
\pgfsetroundjoin%
\definecolor{currentfill}{rgb}{0.965302,0.713942,0.568499}%
\pgfsetfillcolor{currentfill}%
\pgfsetlinewidth{0.311001pt}%
\definecolor{currentstroke}{rgb}{1.000000,1.000000,1.000000}%
\pgfsetstrokecolor{currentstroke}%
\pgfsetdash{}{0pt}%
\pgfpathmoveto{\pgfqpoint{5.407350in}{1.002087in}}%
\pgfpathcurveto{\pgfqpoint{5.414483in}{1.002087in}}{\pgfqpoint{5.421325in}{1.004920in}}{\pgfqpoint{5.426369in}{1.009964in}}%
\pgfpathcurveto{\pgfqpoint{5.431412in}{1.015008in}}{\pgfqpoint{5.434246in}{1.021849in}}{\pgfqpoint{5.434246in}{1.028982in}}%
\pgfpathcurveto{\pgfqpoint{5.434246in}{1.036115in}}{\pgfqpoint{5.431412in}{1.042957in}}{\pgfqpoint{5.426369in}{1.048000in}}%
\pgfpathcurveto{\pgfqpoint{5.421325in}{1.053044in}}{\pgfqpoint{5.414483in}{1.055878in}}{\pgfqpoint{5.407350in}{1.055878in}}%
\pgfpathcurveto{\pgfqpoint{5.400218in}{1.055878in}}{\pgfqpoint{5.393376in}{1.053044in}}{\pgfqpoint{5.388332in}{1.048000in}}%
\pgfpathcurveto{\pgfqpoint{5.383289in}{1.042957in}}{\pgfqpoint{5.380455in}{1.036115in}}{\pgfqpoint{5.380455in}{1.028982in}}%
\pgfpathcurveto{\pgfqpoint{5.380455in}{1.021849in}}{\pgfqpoint{5.383289in}{1.015008in}}{\pgfqpoint{5.388332in}{1.009964in}}%
\pgfpathcurveto{\pgfqpoint{5.393376in}{1.004920in}}{\pgfqpoint{5.400218in}{1.002087in}}{\pgfqpoint{5.407350in}{1.002087in}}%
\pgfpathclose%
\pgfusepath{stroke,fill}%
\end{pgfscope}%
\begin{pgfscope}%
\pgfpathrectangle{\pgfqpoint{4.985294in}{0.500000in}}{\pgfqpoint{1.764706in}{1.700000in}}%
\pgfusepath{clip}%
\pgfsetbuttcap%
\pgfsetroundjoin%
\definecolor{currentfill}{rgb}{0.979124,0.903132,0.839793}%
\pgfsetfillcolor{currentfill}%
\pgfsetlinewidth{0.311001pt}%
\definecolor{currentstroke}{rgb}{1.000000,1.000000,1.000000}%
\pgfsetstrokecolor{currentstroke}%
\pgfsetdash{}{0pt}%
\pgfpathmoveto{\pgfqpoint{6.284631in}{1.233317in}}%
\pgfpathcurveto{\pgfqpoint{6.291764in}{1.233317in}}{\pgfqpoint{6.298606in}{1.236151in}}{\pgfqpoint{6.303649in}{1.241194in}}%
\pgfpathcurveto{\pgfqpoint{6.308693in}{1.246238in}}{\pgfqpoint{6.311527in}{1.253080in}}{\pgfqpoint{6.311527in}{1.260212in}}%
\pgfpathcurveto{\pgfqpoint{6.311527in}{1.267345in}}{\pgfqpoint{6.308693in}{1.274187in}}{\pgfqpoint{6.303649in}{1.279231in}}%
\pgfpathcurveto{\pgfqpoint{6.298606in}{1.284274in}}{\pgfqpoint{6.291764in}{1.287108in}}{\pgfqpoint{6.284631in}{1.287108in}}%
\pgfpathcurveto{\pgfqpoint{6.277498in}{1.287108in}}{\pgfqpoint{6.270657in}{1.284274in}}{\pgfqpoint{6.265613in}{1.279231in}}%
\pgfpathcurveto{\pgfqpoint{6.260569in}{1.274187in}}{\pgfqpoint{6.257735in}{1.267345in}}{\pgfqpoint{6.257735in}{1.260212in}}%
\pgfpathcurveto{\pgfqpoint{6.257735in}{1.253080in}}{\pgfqpoint{6.260569in}{1.246238in}}{\pgfqpoint{6.265613in}{1.241194in}}%
\pgfpathcurveto{\pgfqpoint{6.270657in}{1.236151in}}{\pgfqpoint{6.277498in}{1.233317in}}{\pgfqpoint{6.284631in}{1.233317in}}%
\pgfpathclose%
\pgfusepath{stroke,fill}%
\end{pgfscope}%
\begin{pgfscope}%
\pgfpathrectangle{\pgfqpoint{4.985294in}{0.500000in}}{\pgfqpoint{1.764706in}{1.700000in}}%
\pgfusepath{clip}%
\pgfsetbuttcap%
\pgfsetroundjoin%
\definecolor{currentfill}{rgb}{0.968931,0.798091,0.685123}%
\pgfsetfillcolor{currentfill}%
\pgfsetlinewidth{0.311001pt}%
\definecolor{currentstroke}{rgb}{1.000000,1.000000,1.000000}%
\pgfsetstrokecolor{currentstroke}%
\pgfsetdash{}{0pt}%
\pgfpathmoveto{\pgfqpoint{6.183298in}{1.096010in}}%
\pgfpathcurveto{\pgfqpoint{6.190430in}{1.096010in}}{\pgfqpoint{6.197272in}{1.098844in}}{\pgfqpoint{6.202316in}{1.103887in}}%
\pgfpathcurveto{\pgfqpoint{6.207359in}{1.108931in}}{\pgfqpoint{6.210193in}{1.115773in}}{\pgfqpoint{6.210193in}{1.122905in}}%
\pgfpathcurveto{\pgfqpoint{6.210193in}{1.130038in}}{\pgfqpoint{6.207359in}{1.136880in}}{\pgfqpoint{6.202316in}{1.141924in}}%
\pgfpathcurveto{\pgfqpoint{6.197272in}{1.146967in}}{\pgfqpoint{6.190430in}{1.149801in}}{\pgfqpoint{6.183298in}{1.149801in}}%
\pgfpathcurveto{\pgfqpoint{6.176165in}{1.149801in}}{\pgfqpoint{6.169323in}{1.146967in}}{\pgfqpoint{6.164279in}{1.141924in}}%
\pgfpathcurveto{\pgfqpoint{6.159236in}{1.136880in}}{\pgfqpoint{6.156402in}{1.130038in}}{\pgfqpoint{6.156402in}{1.122905in}}%
\pgfpathcurveto{\pgfqpoint{6.156402in}{1.115773in}}{\pgfqpoint{6.159236in}{1.108931in}}{\pgfqpoint{6.164279in}{1.103887in}}%
\pgfpathcurveto{\pgfqpoint{6.169323in}{1.098844in}}{\pgfqpoint{6.176165in}{1.096010in}}{\pgfqpoint{6.183298in}{1.096010in}}%
\pgfpathclose%
\pgfusepath{stroke,fill}%
\end{pgfscope}%
\begin{pgfscope}%
\pgfpathrectangle{\pgfqpoint{4.985294in}{0.500000in}}{\pgfqpoint{1.764706in}{1.700000in}}%
\pgfusepath{clip}%
\pgfsetbuttcap%
\pgfsetroundjoin%
\definecolor{currentfill}{rgb}{0.979891,0.908948,0.848279}%
\pgfsetfillcolor{currentfill}%
\pgfsetlinewidth{0.311001pt}%
\definecolor{currentstroke}{rgb}{1.000000,1.000000,1.000000}%
\pgfsetstrokecolor{currentstroke}%
\pgfsetdash{}{0pt}%
\pgfpathmoveto{\pgfqpoint{5.428423in}{1.276757in}}%
\pgfpathcurveto{\pgfqpoint{5.435555in}{1.276757in}}{\pgfqpoint{5.442397in}{1.279591in}}{\pgfqpoint{5.447441in}{1.284635in}}%
\pgfpathcurveto{\pgfqpoint{5.452484in}{1.289679in}}{\pgfqpoint{5.455318in}{1.296520in}}{\pgfqpoint{5.455318in}{1.303653in}}%
\pgfpathcurveto{\pgfqpoint{5.455318in}{1.310786in}}{\pgfqpoint{5.452484in}{1.317627in}}{\pgfqpoint{5.447441in}{1.322671in}}%
\pgfpathcurveto{\pgfqpoint{5.442397in}{1.327715in}}{\pgfqpoint{5.435555in}{1.330549in}}{\pgfqpoint{5.428423in}{1.330549in}}%
\pgfpathcurveto{\pgfqpoint{5.421290in}{1.330549in}}{\pgfqpoint{5.414448in}{1.327715in}}{\pgfqpoint{5.409404in}{1.322671in}}%
\pgfpathcurveto{\pgfqpoint{5.404361in}{1.317627in}}{\pgfqpoint{5.401527in}{1.310786in}}{\pgfqpoint{5.401527in}{1.303653in}}%
\pgfpathcurveto{\pgfqpoint{5.401527in}{1.296520in}}{\pgfqpoint{5.404361in}{1.289679in}}{\pgfqpoint{5.409404in}{1.284635in}}%
\pgfpathcurveto{\pgfqpoint{5.414448in}{1.279591in}}{\pgfqpoint{5.421290in}{1.276757in}}{\pgfqpoint{5.428423in}{1.276757in}}%
\pgfpathclose%
\pgfusepath{stroke,fill}%
\end{pgfscope}%
\begin{pgfscope}%
\pgfpathrectangle{\pgfqpoint{4.985294in}{0.500000in}}{\pgfqpoint{1.764706in}{1.700000in}}%
\pgfusepath{clip}%
\pgfsetbuttcap%
\pgfsetroundjoin%
\definecolor{currentfill}{rgb}{0.973271,0.850724,0.762998}%
\pgfsetfillcolor{currentfill}%
\pgfsetlinewidth{0.311001pt}%
\definecolor{currentstroke}{rgb}{1.000000,1.000000,1.000000}%
\pgfsetstrokecolor{currentstroke}%
\pgfsetdash{}{0pt}%
\pgfpathmoveto{\pgfqpoint{6.265379in}{1.636990in}}%
\pgfpathcurveto{\pgfqpoint{6.272512in}{1.636990in}}{\pgfqpoint{6.279354in}{1.639824in}}{\pgfqpoint{6.284397in}{1.644868in}}%
\pgfpathcurveto{\pgfqpoint{6.289441in}{1.649912in}}{\pgfqpoint{6.292275in}{1.656753in}}{\pgfqpoint{6.292275in}{1.663886in}}%
\pgfpathcurveto{\pgfqpoint{6.292275in}{1.671019in}}{\pgfqpoint{6.289441in}{1.677860in}}{\pgfqpoint{6.284397in}{1.682904in}}%
\pgfpathcurveto{\pgfqpoint{6.279354in}{1.687948in}}{\pgfqpoint{6.272512in}{1.690782in}}{\pgfqpoint{6.265379in}{1.690782in}}%
\pgfpathcurveto{\pgfqpoint{6.258246in}{1.690782in}}{\pgfqpoint{6.251405in}{1.687948in}}{\pgfqpoint{6.246361in}{1.682904in}}%
\pgfpathcurveto{\pgfqpoint{6.241317in}{1.677860in}}{\pgfqpoint{6.238483in}{1.671019in}}{\pgfqpoint{6.238483in}{1.663886in}}%
\pgfpathcurveto{\pgfqpoint{6.238483in}{1.656753in}}{\pgfqpoint{6.241317in}{1.649912in}}{\pgfqpoint{6.246361in}{1.644868in}}%
\pgfpathcurveto{\pgfqpoint{6.251405in}{1.639824in}}{\pgfqpoint{6.258246in}{1.636990in}}{\pgfqpoint{6.265379in}{1.636990in}}%
\pgfpathclose%
\pgfusepath{stroke,fill}%
\end{pgfscope}%
\begin{pgfscope}%
\pgfpathrectangle{\pgfqpoint{4.985294in}{0.500000in}}{\pgfqpoint{1.764706in}{1.700000in}}%
\pgfusepath{clip}%
\pgfsetbuttcap%
\pgfsetroundjoin%
\definecolor{currentfill}{rgb}{0.962532,0.599594,0.438051}%
\pgfsetfillcolor{currentfill}%
\pgfsetlinewidth{0.311001pt}%
\definecolor{currentstroke}{rgb}{1.000000,1.000000,1.000000}%
\pgfsetstrokecolor{currentstroke}%
\pgfsetdash{}{0pt}%
\pgfpathmoveto{\pgfqpoint{6.312758in}{0.962668in}}%
\pgfpathcurveto{\pgfqpoint{6.319891in}{0.962668in}}{\pgfqpoint{6.326732in}{0.965502in}}{\pgfqpoint{6.331776in}{0.970546in}}%
\pgfpathcurveto{\pgfqpoint{6.336820in}{0.975589in}}{\pgfqpoint{6.339653in}{0.982431in}}{\pgfqpoint{6.339653in}{0.989564in}}%
\pgfpathcurveto{\pgfqpoint{6.339653in}{0.996697in}}{\pgfqpoint{6.336820in}{1.003538in}}{\pgfqpoint{6.331776in}{1.008582in}}%
\pgfpathcurveto{\pgfqpoint{6.326732in}{1.013626in}}{\pgfqpoint{6.319891in}{1.016459in}}{\pgfqpoint{6.312758in}{1.016459in}}%
\pgfpathcurveto{\pgfqpoint{6.305625in}{1.016459in}}{\pgfqpoint{6.298783in}{1.013626in}}{\pgfqpoint{6.293740in}{1.008582in}}%
\pgfpathcurveto{\pgfqpoint{6.288696in}{1.003538in}}{\pgfqpoint{6.285862in}{0.996697in}}{\pgfqpoint{6.285862in}{0.989564in}}%
\pgfpathcurveto{\pgfqpoint{6.285862in}{0.982431in}}{\pgfqpoint{6.288696in}{0.975589in}}{\pgfqpoint{6.293740in}{0.970546in}}%
\pgfpathcurveto{\pgfqpoint{6.298783in}{0.965502in}}{\pgfqpoint{6.305625in}{0.962668in}}{\pgfqpoint{6.312758in}{0.962668in}}%
\pgfpathclose%
\pgfusepath{stroke,fill}%
\end{pgfscope}%
\begin{pgfscope}%
\pgfpathrectangle{\pgfqpoint{4.985294in}{0.500000in}}{\pgfqpoint{1.764706in}{1.700000in}}%
\pgfusepath{clip}%
\pgfsetbuttcap%
\pgfsetroundjoin%
\definecolor{currentfill}{rgb}{0.978376,0.897317,0.831308}%
\pgfsetfillcolor{currentfill}%
\pgfsetlinewidth{0.311001pt}%
\definecolor{currentstroke}{rgb}{1.000000,1.000000,1.000000}%
\pgfsetstrokecolor{currentstroke}%
\pgfsetdash{}{0pt}%
\pgfpathmoveto{\pgfqpoint{5.399284in}{1.414796in}}%
\pgfpathcurveto{\pgfqpoint{5.406417in}{1.414796in}}{\pgfqpoint{5.413259in}{1.417630in}}{\pgfqpoint{5.418302in}{1.422674in}}%
\pgfpathcurveto{\pgfqpoint{5.423346in}{1.427718in}}{\pgfqpoint{5.426180in}{1.434559in}}{\pgfqpoint{5.426180in}{1.441692in}}%
\pgfpathcurveto{\pgfqpoint{5.426180in}{1.448825in}}{\pgfqpoint{5.423346in}{1.455667in}}{\pgfqpoint{5.418302in}{1.460710in}}%
\pgfpathcurveto{\pgfqpoint{5.413259in}{1.465754in}}{\pgfqpoint{5.406417in}{1.468588in}}{\pgfqpoint{5.399284in}{1.468588in}}%
\pgfpathcurveto{\pgfqpoint{5.392151in}{1.468588in}}{\pgfqpoint{5.385310in}{1.465754in}}{\pgfqpoint{5.380266in}{1.460710in}}%
\pgfpathcurveto{\pgfqpoint{5.375222in}{1.455667in}}{\pgfqpoint{5.372388in}{1.448825in}}{\pgfqpoint{5.372388in}{1.441692in}}%
\pgfpathcurveto{\pgfqpoint{5.372388in}{1.434559in}}{\pgfqpoint{5.375222in}{1.427718in}}{\pgfqpoint{5.380266in}{1.422674in}}%
\pgfpathcurveto{\pgfqpoint{5.385310in}{1.417630in}}{\pgfqpoint{5.392151in}{1.414796in}}{\pgfqpoint{5.399284in}{1.414796in}}%
\pgfpathclose%
\pgfusepath{stroke,fill}%
\end{pgfscope}%
\begin{pgfscope}%
\pgfpathrectangle{\pgfqpoint{4.985294in}{0.500000in}}{\pgfqpoint{1.764706in}{1.700000in}}%
\pgfusepath{clip}%
\pgfsetbuttcap%
\pgfsetroundjoin%
\definecolor{currentfill}{rgb}{0.967092,0.768560,0.642079}%
\pgfsetfillcolor{currentfill}%
\pgfsetlinewidth{0.311001pt}%
\definecolor{currentstroke}{rgb}{1.000000,1.000000,1.000000}%
\pgfsetstrokecolor{currentstroke}%
\pgfsetdash{}{0pt}%
\pgfpathmoveto{\pgfqpoint{5.420198in}{1.609217in}}%
\pgfpathcurveto{\pgfqpoint{5.427331in}{1.609217in}}{\pgfqpoint{5.434173in}{1.612051in}}{\pgfqpoint{5.439217in}{1.617095in}}%
\pgfpathcurveto{\pgfqpoint{5.444260in}{1.622139in}}{\pgfqpoint{5.447094in}{1.628980in}}{\pgfqpoint{5.447094in}{1.636113in}}%
\pgfpathcurveto{\pgfqpoint{5.447094in}{1.643246in}}{\pgfqpoint{5.444260in}{1.650087in}}{\pgfqpoint{5.439217in}{1.655131in}}%
\pgfpathcurveto{\pgfqpoint{5.434173in}{1.660175in}}{\pgfqpoint{5.427331in}{1.663009in}}{\pgfqpoint{5.420198in}{1.663009in}}%
\pgfpathcurveto{\pgfqpoint{5.413066in}{1.663009in}}{\pgfqpoint{5.406224in}{1.660175in}}{\pgfqpoint{5.401180in}{1.655131in}}%
\pgfpathcurveto{\pgfqpoint{5.396137in}{1.650087in}}{\pgfqpoint{5.393303in}{1.643246in}}{\pgfqpoint{5.393303in}{1.636113in}}%
\pgfpathcurveto{\pgfqpoint{5.393303in}{1.628980in}}{\pgfqpoint{5.396137in}{1.622139in}}{\pgfqpoint{5.401180in}{1.617095in}}%
\pgfpathcurveto{\pgfqpoint{5.406224in}{1.612051in}}{\pgfqpoint{5.413066in}{1.609217in}}{\pgfqpoint{5.420198in}{1.609217in}}%
\pgfpathclose%
\pgfusepath{stroke,fill}%
\end{pgfscope}%
\begin{pgfscope}%
\pgfpathrectangle{\pgfqpoint{4.985294in}{0.500000in}}{\pgfqpoint{1.764706in}{1.700000in}}%
\pgfusepath{clip}%
\pgfsetbuttcap%
\pgfsetroundjoin%
\definecolor{currentfill}{rgb}{0.977657,0.891500,0.822809}%
\pgfsetfillcolor{currentfill}%
\pgfsetlinewidth{0.311001pt}%
\definecolor{currentstroke}{rgb}{1.000000,1.000000,1.000000}%
\pgfsetstrokecolor{currentstroke}%
\pgfsetdash{}{0pt}%
\pgfpathmoveto{\pgfqpoint{5.431500in}{1.158325in}}%
\pgfpathcurveto{\pgfqpoint{5.438633in}{1.158325in}}{\pgfqpoint{5.445474in}{1.161159in}}{\pgfqpoint{5.450518in}{1.166203in}}%
\pgfpathcurveto{\pgfqpoint{5.455562in}{1.171246in}}{\pgfqpoint{5.458396in}{1.178088in}}{\pgfqpoint{5.458396in}{1.185221in}}%
\pgfpathcurveto{\pgfqpoint{5.458396in}{1.192354in}}{\pgfqpoint{5.455562in}{1.199195in}}{\pgfqpoint{5.450518in}{1.204239in}}%
\pgfpathcurveto{\pgfqpoint{5.445474in}{1.209283in}}{\pgfqpoint{5.438633in}{1.212116in}}{\pgfqpoint{5.431500in}{1.212116in}}%
\pgfpathcurveto{\pgfqpoint{5.424367in}{1.212116in}}{\pgfqpoint{5.417525in}{1.209283in}}{\pgfqpoint{5.412482in}{1.204239in}}%
\pgfpathcurveto{\pgfqpoint{5.407438in}{1.199195in}}{\pgfqpoint{5.404604in}{1.192354in}}{\pgfqpoint{5.404604in}{1.185221in}}%
\pgfpathcurveto{\pgfqpoint{5.404604in}{1.178088in}}{\pgfqpoint{5.407438in}{1.171246in}}{\pgfqpoint{5.412482in}{1.166203in}}%
\pgfpathcurveto{\pgfqpoint{5.417525in}{1.161159in}}{\pgfqpoint{5.424367in}{1.158325in}}{\pgfqpoint{5.431500in}{1.158325in}}%
\pgfpathclose%
\pgfusepath{stroke,fill}%
\end{pgfscope}%
\begin{pgfscope}%
\pgfpathrectangle{\pgfqpoint{4.985294in}{0.500000in}}{\pgfqpoint{1.764706in}{1.700000in}}%
\pgfusepath{clip}%
\pgfsetbuttcap%
\pgfsetroundjoin%
\definecolor{currentfill}{rgb}{0.981377,0.920617,0.865369}%
\pgfsetfillcolor{currentfill}%
\pgfsetlinewidth{0.311001pt}%
\definecolor{currentstroke}{rgb}{1.000000,1.000000,1.000000}%
\pgfsetstrokecolor{currentstroke}%
\pgfsetdash{}{0pt}%
\pgfpathmoveto{\pgfqpoint{6.307696in}{1.473757in}}%
\pgfpathcurveto{\pgfqpoint{6.314829in}{1.473757in}}{\pgfqpoint{6.321671in}{1.476591in}}{\pgfqpoint{6.326715in}{1.481635in}}%
\pgfpathcurveto{\pgfqpoint{6.331758in}{1.486678in}}{\pgfqpoint{6.334592in}{1.493520in}}{\pgfqpoint{6.334592in}{1.500653in}}%
\pgfpathcurveto{\pgfqpoint{6.334592in}{1.507786in}}{\pgfqpoint{6.331758in}{1.514627in}}{\pgfqpoint{6.326715in}{1.519671in}}%
\pgfpathcurveto{\pgfqpoint{6.321671in}{1.524715in}}{\pgfqpoint{6.314829in}{1.527549in}}{\pgfqpoint{6.307696in}{1.527549in}}%
\pgfpathcurveto{\pgfqpoint{6.300564in}{1.527549in}}{\pgfqpoint{6.293722in}{1.524715in}}{\pgfqpoint{6.288678in}{1.519671in}}%
\pgfpathcurveto{\pgfqpoint{6.283635in}{1.514627in}}{\pgfqpoint{6.280801in}{1.507786in}}{\pgfqpoint{6.280801in}{1.500653in}}%
\pgfpathcurveto{\pgfqpoint{6.280801in}{1.493520in}}{\pgfqpoint{6.283635in}{1.486678in}}{\pgfqpoint{6.288678in}{1.481635in}}%
\pgfpathcurveto{\pgfqpoint{6.293722in}{1.476591in}}{\pgfqpoint{6.300564in}{1.473757in}}{\pgfqpoint{6.307696in}{1.473757in}}%
\pgfpathclose%
\pgfusepath{stroke,fill}%
\end{pgfscope}%
\begin{pgfscope}%
\pgfpathrectangle{\pgfqpoint{4.985294in}{0.500000in}}{\pgfqpoint{1.764706in}{1.700000in}}%
\pgfusepath{clip}%
\pgfsetbuttcap%
\pgfsetroundjoin%
\definecolor{currentfill}{rgb}{0.969359,0.803954,0.693832}%
\pgfsetfillcolor{currentfill}%
\pgfsetlinewidth{0.311001pt}%
\definecolor{currentstroke}{rgb}{1.000000,1.000000,1.000000}%
\pgfsetstrokecolor{currentstroke}%
\pgfsetdash{}{0pt}%
\pgfpathmoveto{\pgfqpoint{5.533558in}{1.082897in}}%
\pgfpathcurveto{\pgfqpoint{5.540691in}{1.082897in}}{\pgfqpoint{5.547533in}{1.085731in}}{\pgfqpoint{5.552576in}{1.090775in}}%
\pgfpathcurveto{\pgfqpoint{5.557620in}{1.095818in}}{\pgfqpoint{5.560454in}{1.102660in}}{\pgfqpoint{5.560454in}{1.109793in}}%
\pgfpathcurveto{\pgfqpoint{5.560454in}{1.116926in}}{\pgfqpoint{5.557620in}{1.123767in}}{\pgfqpoint{5.552576in}{1.128811in}}%
\pgfpathcurveto{\pgfqpoint{5.547533in}{1.133855in}}{\pgfqpoint{5.540691in}{1.136689in}}{\pgfqpoint{5.533558in}{1.136689in}}%
\pgfpathcurveto{\pgfqpoint{5.526425in}{1.136689in}}{\pgfqpoint{5.519584in}{1.133855in}}{\pgfqpoint{5.514540in}{1.128811in}}%
\pgfpathcurveto{\pgfqpoint{5.509496in}{1.123767in}}{\pgfqpoint{5.506663in}{1.116926in}}{\pgfqpoint{5.506663in}{1.109793in}}%
\pgfpathcurveto{\pgfqpoint{5.506663in}{1.102660in}}{\pgfqpoint{5.509496in}{1.095818in}}{\pgfqpoint{5.514540in}{1.090775in}}%
\pgfpathcurveto{\pgfqpoint{5.519584in}{1.085731in}}{\pgfqpoint{5.526425in}{1.082897in}}{\pgfqpoint{5.533558in}{1.082897in}}%
\pgfpathclose%
\pgfusepath{stroke,fill}%
\end{pgfscope}%
\begin{pgfscope}%
\pgfpathrectangle{\pgfqpoint{4.985294in}{0.500000in}}{\pgfqpoint{1.764706in}{1.700000in}}%
\pgfusepath{clip}%
\pgfsetbuttcap%
\pgfsetroundjoin%
\definecolor{currentfill}{rgb}{0.941676,0.367866,0.260395}%
\pgfsetfillcolor{currentfill}%
\pgfsetlinewidth{0.311001pt}%
\definecolor{currentstroke}{rgb}{1.000000,1.000000,1.000000}%
\pgfsetstrokecolor{currentstroke}%
\pgfsetdash{}{0pt}%
\pgfpathmoveto{\pgfqpoint{6.412349in}{1.056310in}}%
\pgfpathcurveto{\pgfqpoint{6.419482in}{1.056310in}}{\pgfqpoint{6.426324in}{1.059144in}}{\pgfqpoint{6.431368in}{1.064188in}}%
\pgfpathcurveto{\pgfqpoint{6.436411in}{1.069231in}}{\pgfqpoint{6.439245in}{1.076073in}}{\pgfqpoint{6.439245in}{1.083206in}}%
\pgfpathcurveto{\pgfqpoint{6.439245in}{1.090339in}}{\pgfqpoint{6.436411in}{1.097180in}}{\pgfqpoint{6.431368in}{1.102224in}}%
\pgfpathcurveto{\pgfqpoint{6.426324in}{1.107268in}}{\pgfqpoint{6.419482in}{1.110101in}}{\pgfqpoint{6.412349in}{1.110101in}}%
\pgfpathcurveto{\pgfqpoint{6.405217in}{1.110101in}}{\pgfqpoint{6.398375in}{1.107268in}}{\pgfqpoint{6.393331in}{1.102224in}}%
\pgfpathcurveto{\pgfqpoint{6.388288in}{1.097180in}}{\pgfqpoint{6.385454in}{1.090339in}}{\pgfqpoint{6.385454in}{1.083206in}}%
\pgfpathcurveto{\pgfqpoint{6.385454in}{1.076073in}}{\pgfqpoint{6.388288in}{1.069231in}}{\pgfqpoint{6.393331in}{1.064188in}}%
\pgfpathcurveto{\pgfqpoint{6.398375in}{1.059144in}}{\pgfqpoint{6.405217in}{1.056310in}}{\pgfqpoint{6.412349in}{1.056310in}}%
\pgfpathclose%
\pgfusepath{stroke,fill}%
\end{pgfscope}%
\begin{pgfscope}%
\pgfpathrectangle{\pgfqpoint{4.985294in}{0.500000in}}{\pgfqpoint{1.764706in}{1.700000in}}%
\pgfusepath{clip}%
\pgfsetbuttcap%
\pgfsetroundjoin%
\definecolor{currentfill}{rgb}{0.950851,0.435000,0.297228}%
\pgfsetfillcolor{currentfill}%
\pgfsetlinewidth{0.311001pt}%
\definecolor{currentstroke}{rgb}{1.000000,1.000000,1.000000}%
\pgfsetstrokecolor{currentstroke}%
\pgfsetdash{}{0pt}%
\pgfpathmoveto{\pgfqpoint{5.323554in}{1.049757in}}%
\pgfpathcurveto{\pgfqpoint{5.330687in}{1.049757in}}{\pgfqpoint{5.337528in}{1.052591in}}{\pgfqpoint{5.342572in}{1.057635in}}%
\pgfpathcurveto{\pgfqpoint{5.347616in}{1.062678in}}{\pgfqpoint{5.350450in}{1.069520in}}{\pgfqpoint{5.350450in}{1.076653in}}%
\pgfpathcurveto{\pgfqpoint{5.350450in}{1.083786in}}{\pgfqpoint{5.347616in}{1.090627in}}{\pgfqpoint{5.342572in}{1.095671in}}%
\pgfpathcurveto{\pgfqpoint{5.337528in}{1.100715in}}{\pgfqpoint{5.330687in}{1.103549in}}{\pgfqpoint{5.323554in}{1.103549in}}%
\pgfpathcurveto{\pgfqpoint{5.316421in}{1.103549in}}{\pgfqpoint{5.309579in}{1.100715in}}{\pgfqpoint{5.304536in}{1.095671in}}%
\pgfpathcurveto{\pgfqpoint{5.299492in}{1.090627in}}{\pgfqpoint{5.296658in}{1.083786in}}{\pgfqpoint{5.296658in}{1.076653in}}%
\pgfpathcurveto{\pgfqpoint{5.296658in}{1.069520in}}{\pgfqpoint{5.299492in}{1.062678in}}{\pgfqpoint{5.304536in}{1.057635in}}%
\pgfpathcurveto{\pgfqpoint{5.309579in}{1.052591in}}{\pgfqpoint{5.316421in}{1.049757in}}{\pgfqpoint{5.323554in}{1.049757in}}%
\pgfpathclose%
\pgfusepath{stroke,fill}%
\end{pgfscope}%
\begin{pgfscope}%
\pgfpathrectangle{\pgfqpoint{4.985294in}{0.500000in}}{\pgfqpoint{1.764706in}{1.700000in}}%
\pgfusepath{clip}%
\pgfsetbuttcap%
\pgfsetroundjoin%
\definecolor{currentfill}{rgb}{0.966328,0.750560,0.616961}%
\pgfsetfillcolor{currentfill}%
\pgfsetlinewidth{0.311001pt}%
\definecolor{currentstroke}{rgb}{1.000000,1.000000,1.000000}%
\pgfsetstrokecolor{currentstroke}%
\pgfsetdash{}{0pt}%
\pgfpathmoveto{\pgfqpoint{5.518651in}{1.197653in}}%
\pgfpathcurveto{\pgfqpoint{5.525783in}{1.197653in}}{\pgfqpoint{5.532625in}{1.200487in}}{\pgfqpoint{5.537669in}{1.205530in}}%
\pgfpathcurveto{\pgfqpoint{5.542712in}{1.210574in}}{\pgfqpoint{5.545546in}{1.217416in}}{\pgfqpoint{5.545546in}{1.224548in}}%
\pgfpathcurveto{\pgfqpoint{5.545546in}{1.231681in}}{\pgfqpoint{5.542712in}{1.238523in}}{\pgfqpoint{5.537669in}{1.243567in}}%
\pgfpathcurveto{\pgfqpoint{5.532625in}{1.248610in}}{\pgfqpoint{5.525783in}{1.251444in}}{\pgfqpoint{5.518651in}{1.251444in}}%
\pgfpathcurveto{\pgfqpoint{5.511518in}{1.251444in}}{\pgfqpoint{5.504676in}{1.248610in}}{\pgfqpoint{5.499632in}{1.243567in}}%
\pgfpathcurveto{\pgfqpoint{5.494589in}{1.238523in}}{\pgfqpoint{5.491755in}{1.231681in}}{\pgfqpoint{5.491755in}{1.224548in}}%
\pgfpathcurveto{\pgfqpoint{5.491755in}{1.217416in}}{\pgfqpoint{5.494589in}{1.210574in}}{\pgfqpoint{5.499632in}{1.205530in}}%
\pgfpathcurveto{\pgfqpoint{5.504676in}{1.200487in}}{\pgfqpoint{5.511518in}{1.197653in}}{\pgfqpoint{5.518651in}{1.197653in}}%
\pgfpathclose%
\pgfusepath{stroke,fill}%
\end{pgfscope}%
\begin{pgfscope}%
\pgfpathrectangle{\pgfqpoint{4.985294in}{0.500000in}}{\pgfqpoint{1.764706in}{1.700000in}}%
\pgfusepath{clip}%
\pgfsetbuttcap%
\pgfsetroundjoin%
\definecolor{currentfill}{rgb}{0.966120,0.744512,0.608720}%
\pgfsetfillcolor{currentfill}%
\pgfsetlinewidth{0.311001pt}%
\definecolor{currentstroke}{rgb}{1.000000,1.000000,1.000000}%
\pgfsetstrokecolor{currentstroke}%
\pgfsetdash{}{0pt}%
\pgfpathmoveto{\pgfqpoint{5.557238in}{1.539886in}}%
\pgfpathcurveto{\pgfqpoint{5.564371in}{1.539886in}}{\pgfqpoint{5.571213in}{1.542720in}}{\pgfqpoint{5.576256in}{1.547763in}}%
\pgfpathcurveto{\pgfqpoint{5.581300in}{1.552807in}}{\pgfqpoint{5.584134in}{1.559649in}}{\pgfqpoint{5.584134in}{1.566781in}}%
\pgfpathcurveto{\pgfqpoint{5.584134in}{1.573914in}}{\pgfqpoint{5.581300in}{1.580756in}}{\pgfqpoint{5.576256in}{1.585800in}}%
\pgfpathcurveto{\pgfqpoint{5.571213in}{1.590843in}}{\pgfqpoint{5.564371in}{1.593677in}}{\pgfqpoint{5.557238in}{1.593677in}}%
\pgfpathcurveto{\pgfqpoint{5.550105in}{1.593677in}}{\pgfqpoint{5.543264in}{1.590843in}}{\pgfqpoint{5.538220in}{1.585800in}}%
\pgfpathcurveto{\pgfqpoint{5.533176in}{1.580756in}}{\pgfqpoint{5.530342in}{1.573914in}}{\pgfqpoint{5.530342in}{1.566781in}}%
\pgfpathcurveto{\pgfqpoint{5.530342in}{1.559649in}}{\pgfqpoint{5.533176in}{1.552807in}}{\pgfqpoint{5.538220in}{1.547763in}}%
\pgfpathcurveto{\pgfqpoint{5.543264in}{1.542720in}}{\pgfqpoint{5.550105in}{1.539886in}}{\pgfqpoint{5.557238in}{1.539886in}}%
\pgfpathclose%
\pgfusepath{stroke,fill}%
\end{pgfscope}%
\begin{pgfscope}%
\pgfpathrectangle{\pgfqpoint{4.985294in}{0.500000in}}{\pgfqpoint{1.764706in}{1.700000in}}%
\pgfusepath{clip}%
\pgfsetbuttcap%
\pgfsetroundjoin%
\definecolor{currentfill}{rgb}{0.969359,0.803954,0.693832}%
\pgfsetfillcolor{currentfill}%
\pgfsetlinewidth{0.311001pt}%
\definecolor{currentstroke}{rgb}{1.000000,1.000000,1.000000}%
\pgfsetstrokecolor{currentstroke}%
\pgfsetdash{}{0pt}%
\pgfpathmoveto{\pgfqpoint{5.349371in}{1.355204in}}%
\pgfpathcurveto{\pgfqpoint{5.356503in}{1.355204in}}{\pgfqpoint{5.363345in}{1.358038in}}{\pgfqpoint{5.368389in}{1.363082in}}%
\pgfpathcurveto{\pgfqpoint{5.373432in}{1.368126in}}{\pgfqpoint{5.376266in}{1.374967in}}{\pgfqpoint{5.376266in}{1.382100in}}%
\pgfpathcurveto{\pgfqpoint{5.376266in}{1.389233in}}{\pgfqpoint{5.373432in}{1.396075in}}{\pgfqpoint{5.368389in}{1.401118in}}%
\pgfpathcurveto{\pgfqpoint{5.363345in}{1.406162in}}{\pgfqpoint{5.356503in}{1.408996in}}{\pgfqpoint{5.349371in}{1.408996in}}%
\pgfpathcurveto{\pgfqpoint{5.342238in}{1.408996in}}{\pgfqpoint{5.335396in}{1.406162in}}{\pgfqpoint{5.330352in}{1.401118in}}%
\pgfpathcurveto{\pgfqpoint{5.325309in}{1.396075in}}{\pgfqpoint{5.322475in}{1.389233in}}{\pgfqpoint{5.322475in}{1.382100in}}%
\pgfpathcurveto{\pgfqpoint{5.322475in}{1.374967in}}{\pgfqpoint{5.325309in}{1.368126in}}{\pgfqpoint{5.330352in}{1.363082in}}%
\pgfpathcurveto{\pgfqpoint{5.335396in}{1.358038in}}{\pgfqpoint{5.342238in}{1.355204in}}{\pgfqpoint{5.349371in}{1.355204in}}%
\pgfpathclose%
\pgfusepath{stroke,fill}%
\end{pgfscope}%
\begin{pgfscope}%
\pgfpathrectangle{\pgfqpoint{4.985294in}{0.500000in}}{\pgfqpoint{1.764706in}{1.700000in}}%
\pgfusepath{clip}%
\pgfsetbuttcap%
\pgfsetroundjoin%
\definecolor{currentfill}{rgb}{0.976287,0.879862,0.805788}%
\pgfsetfillcolor{currentfill}%
\pgfsetlinewidth{0.311001pt}%
\definecolor{currentstroke}{rgb}{1.000000,1.000000,1.000000}%
\pgfsetstrokecolor{currentstroke}%
\pgfsetdash{}{0pt}%
\pgfpathmoveto{\pgfqpoint{5.424504in}{1.505320in}}%
\pgfpathcurveto{\pgfqpoint{5.431637in}{1.505320in}}{\pgfqpoint{5.438479in}{1.508154in}}{\pgfqpoint{5.443522in}{1.513198in}}%
\pgfpathcurveto{\pgfqpoint{5.448566in}{1.518242in}}{\pgfqpoint{5.451400in}{1.525083in}}{\pgfqpoint{5.451400in}{1.532216in}}%
\pgfpathcurveto{\pgfqpoint{5.451400in}{1.539349in}}{\pgfqpoint{5.448566in}{1.546190in}}{\pgfqpoint{5.443522in}{1.551234in}}%
\pgfpathcurveto{\pgfqpoint{5.438479in}{1.556278in}}{\pgfqpoint{5.431637in}{1.559112in}}{\pgfqpoint{5.424504in}{1.559112in}}%
\pgfpathcurveto{\pgfqpoint{5.417371in}{1.559112in}}{\pgfqpoint{5.410530in}{1.556278in}}{\pgfqpoint{5.405486in}{1.551234in}}%
\pgfpathcurveto{\pgfqpoint{5.400442in}{1.546190in}}{\pgfqpoint{5.397609in}{1.539349in}}{\pgfqpoint{5.397609in}{1.532216in}}%
\pgfpathcurveto{\pgfqpoint{5.397609in}{1.525083in}}{\pgfqpoint{5.400442in}{1.518242in}}{\pgfqpoint{5.405486in}{1.513198in}}%
\pgfpathcurveto{\pgfqpoint{5.410530in}{1.508154in}}{\pgfqpoint{5.417371in}{1.505320in}}{\pgfqpoint{5.424504in}{1.505320in}}%
\pgfpathclose%
\pgfusepath{stroke,fill}%
\end{pgfscope}%
\begin{pgfscope}%
\pgfpathrectangle{\pgfqpoint{4.985294in}{0.500000in}}{\pgfqpoint{1.764706in}{1.700000in}}%
\pgfusepath{clip}%
\pgfsetbuttcap%
\pgfsetroundjoin%
\definecolor{currentfill}{rgb}{0.964032,0.651225,0.493258}%
\pgfsetfillcolor{currentfill}%
\pgfsetlinewidth{0.311001pt}%
\definecolor{currentstroke}{rgb}{1.000000,1.000000,1.000000}%
\pgfsetstrokecolor{currentstroke}%
\pgfsetdash{}{0pt}%
\pgfpathmoveto{\pgfqpoint{6.421152in}{1.304494in}}%
\pgfpathcurveto{\pgfqpoint{6.428284in}{1.304494in}}{\pgfqpoint{6.435126in}{1.307328in}}{\pgfqpoint{6.440170in}{1.312371in}}%
\pgfpathcurveto{\pgfqpoint{6.445213in}{1.317415in}}{\pgfqpoint{6.448047in}{1.324257in}}{\pgfqpoint{6.448047in}{1.331390in}}%
\pgfpathcurveto{\pgfqpoint{6.448047in}{1.338522in}}{\pgfqpoint{6.445213in}{1.345364in}}{\pgfqpoint{6.440170in}{1.350408in}}%
\pgfpathcurveto{\pgfqpoint{6.435126in}{1.355451in}}{\pgfqpoint{6.428284in}{1.358285in}}{\pgfqpoint{6.421152in}{1.358285in}}%
\pgfpathcurveto{\pgfqpoint{6.414019in}{1.358285in}}{\pgfqpoint{6.407177in}{1.355451in}}{\pgfqpoint{6.402134in}{1.350408in}}%
\pgfpathcurveto{\pgfqpoint{6.397090in}{1.345364in}}{\pgfqpoint{6.394256in}{1.338522in}}{\pgfqpoint{6.394256in}{1.331390in}}%
\pgfpathcurveto{\pgfqpoint{6.394256in}{1.324257in}}{\pgfqpoint{6.397090in}{1.317415in}}{\pgfqpoint{6.402134in}{1.312371in}}%
\pgfpathcurveto{\pgfqpoint{6.407177in}{1.307328in}}{\pgfqpoint{6.414019in}{1.304494in}}{\pgfqpoint{6.421152in}{1.304494in}}%
\pgfpathclose%
\pgfusepath{stroke,fill}%
\end{pgfscope}%
\begin{pgfscope}%
\pgfpathrectangle{\pgfqpoint{4.985294in}{0.500000in}}{\pgfqpoint{1.764706in}{1.700000in}}%
\pgfusepath{clip}%
\pgfsetbuttcap%
\pgfsetroundjoin%
\definecolor{currentfill}{rgb}{0.979891,0.908948,0.848279}%
\pgfsetfillcolor{currentfill}%
\pgfsetlinewidth{0.311001pt}%
\definecolor{currentstroke}{rgb}{1.000000,1.000000,1.000000}%
\pgfsetstrokecolor{currentstroke}%
\pgfsetdash{}{0pt}%
\pgfpathmoveto{\pgfqpoint{5.409766in}{1.251967in}}%
\pgfpathcurveto{\pgfqpoint{5.416899in}{1.251967in}}{\pgfqpoint{5.423741in}{1.254801in}}{\pgfqpoint{5.428784in}{1.259845in}}%
\pgfpathcurveto{\pgfqpoint{5.433828in}{1.264888in}}{\pgfqpoint{5.436662in}{1.271730in}}{\pgfqpoint{5.436662in}{1.278863in}}%
\pgfpathcurveto{\pgfqpoint{5.436662in}{1.285996in}}{\pgfqpoint{5.433828in}{1.292837in}}{\pgfqpoint{5.428784in}{1.297881in}}%
\pgfpathcurveto{\pgfqpoint{5.423741in}{1.302925in}}{\pgfqpoint{5.416899in}{1.305758in}}{\pgfqpoint{5.409766in}{1.305758in}}%
\pgfpathcurveto{\pgfqpoint{5.402633in}{1.305758in}}{\pgfqpoint{5.395792in}{1.302925in}}{\pgfqpoint{5.390748in}{1.297881in}}%
\pgfpathcurveto{\pgfqpoint{5.385704in}{1.292837in}}{\pgfqpoint{5.382870in}{1.285996in}}{\pgfqpoint{5.382870in}{1.278863in}}%
\pgfpathcurveto{\pgfqpoint{5.382870in}{1.271730in}}{\pgfqpoint{5.385704in}{1.264888in}}{\pgfqpoint{5.390748in}{1.259845in}}%
\pgfpathcurveto{\pgfqpoint{5.395792in}{1.254801in}}{\pgfqpoint{5.402633in}{1.251967in}}{\pgfqpoint{5.409766in}{1.251967in}}%
\pgfpathclose%
\pgfusepath{stroke,fill}%
\end{pgfscope}%
\begin{pgfscope}%
\pgfpathrectangle{\pgfqpoint{4.985294in}{0.500000in}}{\pgfqpoint{1.764706in}{1.700000in}}%
\pgfusepath{clip}%
\pgfsetbuttcap%
\pgfsetroundjoin%
\definecolor{currentfill}{rgb}{0.977657,0.891500,0.822809}%
\pgfsetfillcolor{currentfill}%
\pgfsetlinewidth{0.311001pt}%
\definecolor{currentstroke}{rgb}{1.000000,1.000000,1.000000}%
\pgfsetstrokecolor{currentstroke}%
\pgfsetdash{}{0pt}%
\pgfpathmoveto{\pgfqpoint{6.353332in}{1.309970in}}%
\pgfpathcurveto{\pgfqpoint{6.360465in}{1.309970in}}{\pgfqpoint{6.367307in}{1.312803in}}{\pgfqpoint{6.372351in}{1.317847in}}%
\pgfpathcurveto{\pgfqpoint{6.377394in}{1.322891in}}{\pgfqpoint{6.380228in}{1.329732in}}{\pgfqpoint{6.380228in}{1.336865in}}%
\pgfpathcurveto{\pgfqpoint{6.380228in}{1.343998in}}{\pgfqpoint{6.377394in}{1.350840in}}{\pgfqpoint{6.372351in}{1.355883in}}%
\pgfpathcurveto{\pgfqpoint{6.367307in}{1.360927in}}{\pgfqpoint{6.360465in}{1.363761in}}{\pgfqpoint{6.353332in}{1.363761in}}%
\pgfpathcurveto{\pgfqpoint{6.346200in}{1.363761in}}{\pgfqpoint{6.339358in}{1.360927in}}{\pgfqpoint{6.334314in}{1.355883in}}%
\pgfpathcurveto{\pgfqpoint{6.329271in}{1.350840in}}{\pgfqpoint{6.326437in}{1.343998in}}{\pgfqpoint{6.326437in}{1.336865in}}%
\pgfpathcurveto{\pgfqpoint{6.326437in}{1.329732in}}{\pgfqpoint{6.329271in}{1.322891in}}{\pgfqpoint{6.334314in}{1.317847in}}%
\pgfpathcurveto{\pgfqpoint{6.339358in}{1.312803in}}{\pgfqpoint{6.346200in}{1.309970in}}{\pgfqpoint{6.353332in}{1.309970in}}%
\pgfpathclose%
\pgfusepath{stroke,fill}%
\end{pgfscope}%
\begin{pgfscope}%
\pgfpathrectangle{\pgfqpoint{4.985294in}{0.500000in}}{\pgfqpoint{1.764706in}{1.700000in}}%
\pgfusepath{clip}%
\pgfsetbuttcap%
\pgfsetroundjoin%
\definecolor{currentfill}{rgb}{0.962765,0.606121,0.444717}%
\pgfsetfillcolor{currentfill}%
\pgfsetlinewidth{0.311001pt}%
\definecolor{currentstroke}{rgb}{1.000000,1.000000,1.000000}%
\pgfsetstrokecolor{currentstroke}%
\pgfsetdash{}{0pt}%
\pgfpathmoveto{\pgfqpoint{5.616586in}{1.047006in}}%
\pgfpathcurveto{\pgfqpoint{5.623719in}{1.047006in}}{\pgfqpoint{5.630561in}{1.049840in}}{\pgfqpoint{5.635604in}{1.054884in}}%
\pgfpathcurveto{\pgfqpoint{5.640648in}{1.059928in}}{\pgfqpoint{5.643482in}{1.066769in}}{\pgfqpoint{5.643482in}{1.073902in}}%
\pgfpathcurveto{\pgfqpoint{5.643482in}{1.081035in}}{\pgfqpoint{5.640648in}{1.087877in}}{\pgfqpoint{5.635604in}{1.092920in}}%
\pgfpathcurveto{\pgfqpoint{5.630561in}{1.097964in}}{\pgfqpoint{5.623719in}{1.100798in}}{\pgfqpoint{5.616586in}{1.100798in}}%
\pgfpathcurveto{\pgfqpoint{5.609453in}{1.100798in}}{\pgfqpoint{5.602612in}{1.097964in}}{\pgfqpoint{5.597568in}{1.092920in}}%
\pgfpathcurveto{\pgfqpoint{5.592524in}{1.087877in}}{\pgfqpoint{5.589690in}{1.081035in}}{\pgfqpoint{5.589690in}{1.073902in}}%
\pgfpathcurveto{\pgfqpoint{5.589690in}{1.066769in}}{\pgfqpoint{5.592524in}{1.059928in}}{\pgfqpoint{5.597568in}{1.054884in}}%
\pgfpathcurveto{\pgfqpoint{5.602612in}{1.049840in}}{\pgfqpoint{5.609453in}{1.047006in}}{\pgfqpoint{5.616586in}{1.047006in}}%
\pgfpathclose%
\pgfusepath{stroke,fill}%
\end{pgfscope}%
\begin{pgfscope}%
\pgfpathrectangle{\pgfqpoint{4.985294in}{0.500000in}}{\pgfqpoint{1.764706in}{1.700000in}}%
\pgfusepath{clip}%
\pgfsetbuttcap%
\pgfsetroundjoin%
\definecolor{currentfill}{rgb}{0.968931,0.798091,0.685123}%
\pgfsetfillcolor{currentfill}%
\pgfsetlinewidth{0.311001pt}%
\definecolor{currentstroke}{rgb}{1.000000,1.000000,1.000000}%
\pgfsetstrokecolor{currentstroke}%
\pgfsetdash{}{0pt}%
\pgfpathmoveto{\pgfqpoint{5.379352in}{1.129085in}}%
\pgfpathcurveto{\pgfqpoint{5.386484in}{1.129085in}}{\pgfqpoint{5.393326in}{1.131919in}}{\pgfqpoint{5.398370in}{1.136963in}}%
\pgfpathcurveto{\pgfqpoint{5.403413in}{1.142006in}}{\pgfqpoint{5.406247in}{1.148848in}}{\pgfqpoint{5.406247in}{1.155981in}}%
\pgfpathcurveto{\pgfqpoint{5.406247in}{1.163114in}}{\pgfqpoint{5.403413in}{1.169955in}}{\pgfqpoint{5.398370in}{1.174999in}}%
\pgfpathcurveto{\pgfqpoint{5.393326in}{1.180043in}}{\pgfqpoint{5.386484in}{1.182877in}}{\pgfqpoint{5.379352in}{1.182877in}}%
\pgfpathcurveto{\pgfqpoint{5.372219in}{1.182877in}}{\pgfqpoint{5.365377in}{1.180043in}}{\pgfqpoint{5.360333in}{1.174999in}}%
\pgfpathcurveto{\pgfqpoint{5.355290in}{1.169955in}}{\pgfqpoint{5.352456in}{1.163114in}}{\pgfqpoint{5.352456in}{1.155981in}}%
\pgfpathcurveto{\pgfqpoint{5.352456in}{1.148848in}}{\pgfqpoint{5.355290in}{1.142006in}}{\pgfqpoint{5.360333in}{1.136963in}}%
\pgfpathcurveto{\pgfqpoint{5.365377in}{1.131919in}}{\pgfqpoint{5.372219in}{1.129085in}}{\pgfqpoint{5.379352in}{1.129085in}}%
\pgfpathclose%
\pgfusepath{stroke,fill}%
\end{pgfscope}%
\begin{pgfscope}%
\pgfpathrectangle{\pgfqpoint{4.985294in}{0.500000in}}{\pgfqpoint{1.764706in}{1.700000in}}%
\pgfusepath{clip}%
\pgfsetbuttcap%
\pgfsetroundjoin%
\definecolor{currentfill}{rgb}{0.958791,0.526283,0.368909}%
\pgfsetfillcolor{currentfill}%
\pgfsetlinewidth{0.311001pt}%
\definecolor{currentstroke}{rgb}{1.000000,1.000000,1.000000}%
\pgfsetstrokecolor{currentstroke}%
\pgfsetdash{}{0pt}%
\pgfpathmoveto{\pgfqpoint{6.438934in}{1.245297in}}%
\pgfpathcurveto{\pgfqpoint{6.446067in}{1.245297in}}{\pgfqpoint{6.452908in}{1.248131in}}{\pgfqpoint{6.457952in}{1.253175in}}%
\pgfpathcurveto{\pgfqpoint{6.462996in}{1.258218in}}{\pgfqpoint{6.465830in}{1.265060in}}{\pgfqpoint{6.465830in}{1.272193in}}%
\pgfpathcurveto{\pgfqpoint{6.465830in}{1.279326in}}{\pgfqpoint{6.462996in}{1.286167in}}{\pgfqpoint{6.457952in}{1.291211in}}%
\pgfpathcurveto{\pgfqpoint{6.452908in}{1.296255in}}{\pgfqpoint{6.446067in}{1.299089in}}{\pgfqpoint{6.438934in}{1.299089in}}%
\pgfpathcurveto{\pgfqpoint{6.431801in}{1.299089in}}{\pgfqpoint{6.424960in}{1.296255in}}{\pgfqpoint{6.419916in}{1.291211in}}%
\pgfpathcurveto{\pgfqpoint{6.414872in}{1.286167in}}{\pgfqpoint{6.412038in}{1.279326in}}{\pgfqpoint{6.412038in}{1.272193in}}%
\pgfpathcurveto{\pgfqpoint{6.412038in}{1.265060in}}{\pgfqpoint{6.414872in}{1.258218in}}{\pgfqpoint{6.419916in}{1.253175in}}%
\pgfpathcurveto{\pgfqpoint{6.424960in}{1.248131in}}{\pgfqpoint{6.431801in}{1.245297in}}{\pgfqpoint{6.438934in}{1.245297in}}%
\pgfpathclose%
\pgfusepath{stroke,fill}%
\end{pgfscope}%
\begin{pgfscope}%
\pgfpathrectangle{\pgfqpoint{4.985294in}{0.500000in}}{\pgfqpoint{1.764706in}{1.700000in}}%
\pgfusepath{clip}%
\pgfsetbuttcap%
\pgfsetroundjoin%
\definecolor{currentfill}{rgb}{0.975018,0.868213,0.788710}%
\pgfsetfillcolor{currentfill}%
\pgfsetlinewidth{0.311001pt}%
\definecolor{currentstroke}{rgb}{1.000000,1.000000,1.000000}%
\pgfsetstrokecolor{currentstroke}%
\pgfsetdash{}{0pt}%
\pgfpathmoveto{\pgfqpoint{6.262817in}{1.276075in}}%
\pgfpathcurveto{\pgfqpoint{6.269950in}{1.276075in}}{\pgfqpoint{6.276792in}{1.278909in}}{\pgfqpoint{6.281835in}{1.283953in}}%
\pgfpathcurveto{\pgfqpoint{6.286879in}{1.288997in}}{\pgfqpoint{6.289713in}{1.295838in}}{\pgfqpoint{6.289713in}{1.302971in}}%
\pgfpathcurveto{\pgfqpoint{6.289713in}{1.310104in}}{\pgfqpoint{6.286879in}{1.316945in}}{\pgfqpoint{6.281835in}{1.321989in}}%
\pgfpathcurveto{\pgfqpoint{6.276792in}{1.327033in}}{\pgfqpoint{6.269950in}{1.329867in}}{\pgfqpoint{6.262817in}{1.329867in}}%
\pgfpathcurveto{\pgfqpoint{6.255684in}{1.329867in}}{\pgfqpoint{6.248843in}{1.327033in}}{\pgfqpoint{6.243799in}{1.321989in}}%
\pgfpathcurveto{\pgfqpoint{6.238755in}{1.316945in}}{\pgfqpoint{6.235922in}{1.310104in}}{\pgfqpoint{6.235922in}{1.302971in}}%
\pgfpathcurveto{\pgfqpoint{6.235922in}{1.295838in}}{\pgfqpoint{6.238755in}{1.288997in}}{\pgfqpoint{6.243799in}{1.283953in}}%
\pgfpathcurveto{\pgfqpoint{6.248843in}{1.278909in}}{\pgfqpoint{6.255684in}{1.276075in}}{\pgfqpoint{6.262817in}{1.276075in}}%
\pgfpathclose%
\pgfusepath{stroke,fill}%
\end{pgfscope}%
\begin{pgfscope}%
\pgfpathrectangle{\pgfqpoint{4.985294in}{0.500000in}}{\pgfqpoint{1.764706in}{1.700000in}}%
\pgfusepath{clip}%
\pgfsetbuttcap%
\pgfsetroundjoin%
\definecolor{currentfill}{rgb}{0.954476,0.470822,0.323110}%
\pgfsetfillcolor{currentfill}%
\pgfsetlinewidth{0.311001pt}%
\definecolor{currentstroke}{rgb}{1.000000,1.000000,1.000000}%
\pgfsetstrokecolor{currentstroke}%
\pgfsetdash{}{0pt}%
\pgfpathmoveto{\pgfqpoint{6.054725in}{1.734462in}}%
\pgfpathcurveto{\pgfqpoint{6.061858in}{1.734462in}}{\pgfqpoint{6.068699in}{1.737296in}}{\pgfqpoint{6.073743in}{1.742339in}}%
\pgfpathcurveto{\pgfqpoint{6.078787in}{1.747383in}}{\pgfqpoint{6.081621in}{1.754225in}}{\pgfqpoint{6.081621in}{1.761357in}}%
\pgfpathcurveto{\pgfqpoint{6.081621in}{1.768490in}}{\pgfqpoint{6.078787in}{1.775332in}}{\pgfqpoint{6.073743in}{1.780376in}}%
\pgfpathcurveto{\pgfqpoint{6.068699in}{1.785419in}}{\pgfqpoint{6.061858in}{1.788253in}}{\pgfqpoint{6.054725in}{1.788253in}}%
\pgfpathcurveto{\pgfqpoint{6.047592in}{1.788253in}}{\pgfqpoint{6.040750in}{1.785419in}}{\pgfqpoint{6.035707in}{1.780376in}}%
\pgfpathcurveto{\pgfqpoint{6.030663in}{1.775332in}}{\pgfqpoint{6.027829in}{1.768490in}}{\pgfqpoint{6.027829in}{1.761357in}}%
\pgfpathcurveto{\pgfqpoint{6.027829in}{1.754225in}}{\pgfqpoint{6.030663in}{1.747383in}}{\pgfqpoint{6.035707in}{1.742339in}}%
\pgfpathcurveto{\pgfqpoint{6.040750in}{1.737296in}}{\pgfqpoint{6.047592in}{1.734462in}}{\pgfqpoint{6.054725in}{1.734462in}}%
\pgfpathclose%
\pgfusepath{stroke,fill}%
\end{pgfscope}%
\begin{pgfscope}%
\pgfpathrectangle{\pgfqpoint{4.985294in}{0.500000in}}{\pgfqpoint{1.764706in}{1.700000in}}%
\pgfusepath{clip}%
\pgfsetbuttcap%
\pgfsetroundjoin%
\definecolor{currentfill}{rgb}{0.968105,0.786346,0.667739}%
\pgfsetfillcolor{currentfill}%
\pgfsetlinewidth{0.311001pt}%
\definecolor{currentstroke}{rgb}{1.000000,1.000000,1.000000}%
\pgfsetstrokecolor{currentstroke}%
\pgfsetdash{}{0pt}%
\pgfpathmoveto{\pgfqpoint{5.537076in}{0.945709in}}%
\pgfpathcurveto{\pgfqpoint{5.544209in}{0.945709in}}{\pgfqpoint{5.551051in}{0.948542in}}{\pgfqpoint{5.556094in}{0.953586in}}%
\pgfpathcurveto{\pgfqpoint{5.561138in}{0.958630in}}{\pgfqpoint{5.563972in}{0.965471in}}{\pgfqpoint{5.563972in}{0.972604in}}%
\pgfpathcurveto{\pgfqpoint{5.563972in}{0.979737in}}{\pgfqpoint{5.561138in}{0.986579in}}{\pgfqpoint{5.556094in}{0.991622in}}%
\pgfpathcurveto{\pgfqpoint{5.551051in}{0.996666in}}{\pgfqpoint{5.544209in}{0.999500in}}{\pgfqpoint{5.537076in}{0.999500in}}%
\pgfpathcurveto{\pgfqpoint{5.529943in}{0.999500in}}{\pgfqpoint{5.523102in}{0.996666in}}{\pgfqpoint{5.518058in}{0.991622in}}%
\pgfpathcurveto{\pgfqpoint{5.513014in}{0.986579in}}{\pgfqpoint{5.510180in}{0.979737in}}{\pgfqpoint{5.510180in}{0.972604in}}%
\pgfpathcurveto{\pgfqpoint{5.510180in}{0.965471in}}{\pgfqpoint{5.513014in}{0.958630in}}{\pgfqpoint{5.518058in}{0.953586in}}%
\pgfpathcurveto{\pgfqpoint{5.523102in}{0.948542in}}{\pgfqpoint{5.529943in}{0.945709in}}{\pgfqpoint{5.537076in}{0.945709in}}%
\pgfpathclose%
\pgfusepath{stroke,fill}%
\end{pgfscope}%
\begin{pgfscope}%
\pgfpathrectangle{\pgfqpoint{4.985294in}{0.500000in}}{\pgfqpoint{1.764706in}{1.700000in}}%
\pgfusepath{clip}%
\pgfsetbuttcap%
\pgfsetroundjoin%
\definecolor{currentfill}{rgb}{0.971202,0.827364,0.728520}%
\pgfsetfillcolor{currentfill}%
\pgfsetlinewidth{0.311001pt}%
\definecolor{currentstroke}{rgb}{1.000000,1.000000,1.000000}%
\pgfsetstrokecolor{currentstroke}%
\pgfsetdash{}{0pt}%
\pgfpathmoveto{\pgfqpoint{5.426655in}{1.056569in}}%
\pgfpathcurveto{\pgfqpoint{5.433787in}{1.056569in}}{\pgfqpoint{5.440629in}{1.059403in}}{\pgfqpoint{5.445673in}{1.064447in}}%
\pgfpathcurveto{\pgfqpoint{5.450716in}{1.069491in}}{\pgfqpoint{5.453550in}{1.076332in}}{\pgfqpoint{5.453550in}{1.083465in}}%
\pgfpathcurveto{\pgfqpoint{5.453550in}{1.090598in}}{\pgfqpoint{5.450716in}{1.097439in}}{\pgfqpoint{5.445673in}{1.102483in}}%
\pgfpathcurveto{\pgfqpoint{5.440629in}{1.107527in}}{\pgfqpoint{5.433787in}{1.110361in}}{\pgfqpoint{5.426655in}{1.110361in}}%
\pgfpathcurveto{\pgfqpoint{5.419522in}{1.110361in}}{\pgfqpoint{5.412680in}{1.107527in}}{\pgfqpoint{5.407636in}{1.102483in}}%
\pgfpathcurveto{\pgfqpoint{5.402593in}{1.097439in}}{\pgfqpoint{5.399759in}{1.090598in}}{\pgfqpoint{5.399759in}{1.083465in}}%
\pgfpathcurveto{\pgfqpoint{5.399759in}{1.076332in}}{\pgfqpoint{5.402593in}{1.069491in}}{\pgfqpoint{5.407636in}{1.064447in}}%
\pgfpathcurveto{\pgfqpoint{5.412680in}{1.059403in}}{\pgfqpoint{5.419522in}{1.056569in}}{\pgfqpoint{5.426655in}{1.056569in}}%
\pgfpathclose%
\pgfusepath{stroke,fill}%
\end{pgfscope}%
\begin{pgfscope}%
\pgfpathrectangle{\pgfqpoint{4.985294in}{0.500000in}}{\pgfqpoint{1.764706in}{1.700000in}}%
\pgfusepath{clip}%
\pgfsetbuttcap%
\pgfsetroundjoin%
\definecolor{currentfill}{rgb}{0.924566,0.290534,0.242426}%
\pgfsetfillcolor{currentfill}%
\pgfsetlinewidth{0.311001pt}%
\definecolor{currentstroke}{rgb}{1.000000,1.000000,1.000000}%
\pgfsetstrokecolor{currentstroke}%
\pgfsetdash{}{0pt}%
\pgfpathmoveto{\pgfqpoint{5.640932in}{1.115449in}}%
\pgfpathcurveto{\pgfqpoint{5.648065in}{1.115449in}}{\pgfqpoint{5.654906in}{1.118283in}}{\pgfqpoint{5.659950in}{1.123327in}}%
\pgfpathcurveto{\pgfqpoint{5.664994in}{1.128370in}}{\pgfqpoint{5.667828in}{1.135212in}}{\pgfqpoint{5.667828in}{1.142345in}}%
\pgfpathcurveto{\pgfqpoint{5.667828in}{1.149478in}}{\pgfqpoint{5.664994in}{1.156319in}}{\pgfqpoint{5.659950in}{1.161363in}}%
\pgfpathcurveto{\pgfqpoint{5.654906in}{1.166407in}}{\pgfqpoint{5.648065in}{1.169240in}}{\pgfqpoint{5.640932in}{1.169240in}}%
\pgfpathcurveto{\pgfqpoint{5.633799in}{1.169240in}}{\pgfqpoint{5.626957in}{1.166407in}}{\pgfqpoint{5.621914in}{1.161363in}}%
\pgfpathcurveto{\pgfqpoint{5.616870in}{1.156319in}}{\pgfqpoint{5.614036in}{1.149478in}}{\pgfqpoint{5.614036in}{1.142345in}}%
\pgfpathcurveto{\pgfqpoint{5.614036in}{1.135212in}}{\pgfqpoint{5.616870in}{1.128370in}}{\pgfqpoint{5.621914in}{1.123327in}}%
\pgfpathcurveto{\pgfqpoint{5.626957in}{1.118283in}}{\pgfqpoint{5.633799in}{1.115449in}}{\pgfqpoint{5.640932in}{1.115449in}}%
\pgfpathclose%
\pgfusepath{stroke,fill}%
\end{pgfscope}%
\begin{pgfscope}%
\pgfpathrectangle{\pgfqpoint{4.985294in}{0.500000in}}{\pgfqpoint{1.764706in}{1.700000in}}%
\pgfusepath{clip}%
\pgfsetbuttcap%
\pgfsetroundjoin%
\definecolor{currentfill}{rgb}{0.964920,0.695342,0.545192}%
\pgfsetfillcolor{currentfill}%
\pgfsetlinewidth{0.311001pt}%
\definecolor{currentstroke}{rgb}{1.000000,1.000000,1.000000}%
\pgfsetstrokecolor{currentstroke}%
\pgfsetdash{}{0pt}%
\pgfpathmoveto{\pgfqpoint{6.335016in}{1.028820in}}%
\pgfpathcurveto{\pgfqpoint{6.342149in}{1.028820in}}{\pgfqpoint{6.348991in}{1.031653in}}{\pgfqpoint{6.354034in}{1.036697in}}%
\pgfpathcurveto{\pgfqpoint{6.359078in}{1.041741in}}{\pgfqpoint{6.361912in}{1.048582in}}{\pgfqpoint{6.361912in}{1.055715in}}%
\pgfpathcurveto{\pgfqpoint{6.361912in}{1.062848in}}{\pgfqpoint{6.359078in}{1.069690in}}{\pgfqpoint{6.354034in}{1.074733in}}%
\pgfpathcurveto{\pgfqpoint{6.348991in}{1.079777in}}{\pgfqpoint{6.342149in}{1.082611in}}{\pgfqpoint{6.335016in}{1.082611in}}%
\pgfpathcurveto{\pgfqpoint{6.327884in}{1.082611in}}{\pgfqpoint{6.321042in}{1.079777in}}{\pgfqpoint{6.315998in}{1.074733in}}%
\pgfpathcurveto{\pgfqpoint{6.310955in}{1.069690in}}{\pgfqpoint{6.308121in}{1.062848in}}{\pgfqpoint{6.308121in}{1.055715in}}%
\pgfpathcurveto{\pgfqpoint{6.308121in}{1.048582in}}{\pgfqpoint{6.310955in}{1.041741in}}{\pgfqpoint{6.315998in}{1.036697in}}%
\pgfpathcurveto{\pgfqpoint{6.321042in}{1.031653in}}{\pgfqpoint{6.327884in}{1.028820in}}{\pgfqpoint{6.335016in}{1.028820in}}%
\pgfpathclose%
\pgfusepath{stroke,fill}%
\end{pgfscope}%
\begin{pgfscope}%
\pgfpathrectangle{\pgfqpoint{4.985294in}{0.500000in}}{\pgfqpoint{1.764706in}{1.700000in}}%
\pgfusepath{clip}%
\pgfsetbuttcap%
\pgfsetroundjoin%
\definecolor{currentfill}{rgb}{0.959645,0.539840,0.380928}%
\pgfsetfillcolor{currentfill}%
\pgfsetlinewidth{0.311001pt}%
\definecolor{currentstroke}{rgb}{1.000000,1.000000,1.000000}%
\pgfsetstrokecolor{currentstroke}%
\pgfsetdash{}{0pt}%
\pgfpathmoveto{\pgfqpoint{6.438182in}{1.333965in}}%
\pgfpathcurveto{\pgfqpoint{6.445315in}{1.333965in}}{\pgfqpoint{6.452157in}{1.336799in}}{\pgfqpoint{6.457201in}{1.341843in}}%
\pgfpathcurveto{\pgfqpoint{6.462244in}{1.346886in}}{\pgfqpoint{6.465078in}{1.353728in}}{\pgfqpoint{6.465078in}{1.360861in}}%
\pgfpathcurveto{\pgfqpoint{6.465078in}{1.367994in}}{\pgfqpoint{6.462244in}{1.374835in}}{\pgfqpoint{6.457201in}{1.379879in}}%
\pgfpathcurveto{\pgfqpoint{6.452157in}{1.384923in}}{\pgfqpoint{6.445315in}{1.387757in}}{\pgfqpoint{6.438182in}{1.387757in}}%
\pgfpathcurveto{\pgfqpoint{6.431050in}{1.387757in}}{\pgfqpoint{6.424208in}{1.384923in}}{\pgfqpoint{6.419164in}{1.379879in}}%
\pgfpathcurveto{\pgfqpoint{6.414121in}{1.374835in}}{\pgfqpoint{6.411287in}{1.367994in}}{\pgfqpoint{6.411287in}{1.360861in}}%
\pgfpathcurveto{\pgfqpoint{6.411287in}{1.353728in}}{\pgfqpoint{6.414121in}{1.346886in}}{\pgfqpoint{6.419164in}{1.341843in}}%
\pgfpathcurveto{\pgfqpoint{6.424208in}{1.336799in}}{\pgfqpoint{6.431050in}{1.333965in}}{\pgfqpoint{6.438182in}{1.333965in}}%
\pgfpathclose%
\pgfusepath{stroke,fill}%
\end{pgfscope}%
\begin{pgfscope}%
\pgfpathrectangle{\pgfqpoint{4.985294in}{0.500000in}}{\pgfqpoint{1.764706in}{1.700000in}}%
\pgfusepath{clip}%
\pgfsetbuttcap%
\pgfsetroundjoin%
\definecolor{currentfill}{rgb}{0.971694,0.833208,0.737161}%
\pgfsetfillcolor{currentfill}%
\pgfsetlinewidth{0.311001pt}%
\definecolor{currentstroke}{rgb}{1.000000,1.000000,1.000000}%
\pgfsetstrokecolor{currentstroke}%
\pgfsetdash{}{0pt}%
\pgfpathmoveto{\pgfqpoint{6.225853in}{1.516426in}}%
\pgfpathcurveto{\pgfqpoint{6.232986in}{1.516426in}}{\pgfqpoint{6.239827in}{1.519259in}}{\pgfqpoint{6.244871in}{1.524303in}}%
\pgfpathcurveto{\pgfqpoint{6.249915in}{1.529347in}}{\pgfqpoint{6.252748in}{1.536188in}}{\pgfqpoint{6.252748in}{1.543321in}}%
\pgfpathcurveto{\pgfqpoint{6.252748in}{1.550454in}}{\pgfqpoint{6.249915in}{1.557296in}}{\pgfqpoint{6.244871in}{1.562339in}}%
\pgfpathcurveto{\pgfqpoint{6.239827in}{1.567383in}}{\pgfqpoint{6.232986in}{1.570217in}}{\pgfqpoint{6.225853in}{1.570217in}}%
\pgfpathcurveto{\pgfqpoint{6.218720in}{1.570217in}}{\pgfqpoint{6.211878in}{1.567383in}}{\pgfqpoint{6.206835in}{1.562339in}}%
\pgfpathcurveto{\pgfqpoint{6.201791in}{1.557296in}}{\pgfqpoint{6.198957in}{1.550454in}}{\pgfqpoint{6.198957in}{1.543321in}}%
\pgfpathcurveto{\pgfqpoint{6.198957in}{1.536188in}}{\pgfqpoint{6.201791in}{1.529347in}}{\pgfqpoint{6.206835in}{1.524303in}}%
\pgfpathcurveto{\pgfqpoint{6.211878in}{1.519259in}}{\pgfqpoint{6.218720in}{1.516426in}}{\pgfqpoint{6.225853in}{1.516426in}}%
\pgfpathclose%
\pgfusepath{stroke,fill}%
\end{pgfscope}%
\begin{pgfscope}%
\pgfpathrectangle{\pgfqpoint{4.985294in}{0.500000in}}{\pgfqpoint{1.764706in}{1.700000in}}%
\pgfusepath{clip}%
\pgfsetbuttcap%
\pgfsetroundjoin%
\definecolor{currentfill}{rgb}{0.964173,0.657587,0.500469}%
\pgfsetfillcolor{currentfill}%
\pgfsetlinewidth{0.311001pt}%
\definecolor{currentstroke}{rgb}{1.000000,1.000000,1.000000}%
\pgfsetstrokecolor{currentstroke}%
\pgfsetdash{}{0pt}%
\pgfpathmoveto{\pgfqpoint{6.119802in}{1.010590in}}%
\pgfpathcurveto{\pgfqpoint{6.126934in}{1.010590in}}{\pgfqpoint{6.133776in}{1.013424in}}{\pgfqpoint{6.138820in}{1.018468in}}%
\pgfpathcurveto{\pgfqpoint{6.143863in}{1.023511in}}{\pgfqpoint{6.146697in}{1.030353in}}{\pgfqpoint{6.146697in}{1.037486in}}%
\pgfpathcurveto{\pgfqpoint{6.146697in}{1.044619in}}{\pgfqpoint{6.143863in}{1.051460in}}{\pgfqpoint{6.138820in}{1.056504in}}%
\pgfpathcurveto{\pgfqpoint{6.133776in}{1.061548in}}{\pgfqpoint{6.126934in}{1.064382in}}{\pgfqpoint{6.119802in}{1.064382in}}%
\pgfpathcurveto{\pgfqpoint{6.112669in}{1.064382in}}{\pgfqpoint{6.105827in}{1.061548in}}{\pgfqpoint{6.100783in}{1.056504in}}%
\pgfpathcurveto{\pgfqpoint{6.095740in}{1.051460in}}{\pgfqpoint{6.092906in}{1.044619in}}{\pgfqpoint{6.092906in}{1.037486in}}%
\pgfpathcurveto{\pgfqpoint{6.092906in}{1.030353in}}{\pgfqpoint{6.095740in}{1.023511in}}{\pgfqpoint{6.100783in}{1.018468in}}%
\pgfpathcurveto{\pgfqpoint{6.105827in}{1.013424in}}{\pgfqpoint{6.112669in}{1.010590in}}{\pgfqpoint{6.119802in}{1.010590in}}%
\pgfpathclose%
\pgfusepath{stroke,fill}%
\end{pgfscope}%
\begin{pgfscope}%
\pgfpathrectangle{\pgfqpoint{4.985294in}{0.500000in}}{\pgfqpoint{1.764706in}{1.700000in}}%
\pgfusepath{clip}%
\pgfsetbuttcap%
\pgfsetroundjoin%
\definecolor{currentfill}{rgb}{0.962283,0.593046,0.431453}%
\pgfsetfillcolor{currentfill}%
\pgfsetlinewidth{0.311001pt}%
\definecolor{currentstroke}{rgb}{1.000000,1.000000,1.000000}%
\pgfsetstrokecolor{currentstroke}%
\pgfsetdash{}{0pt}%
\pgfpathmoveto{\pgfqpoint{6.346938in}{1.004890in}}%
\pgfpathcurveto{\pgfqpoint{6.354071in}{1.004890in}}{\pgfqpoint{6.360912in}{1.007723in}}{\pgfqpoint{6.365956in}{1.012767in}}%
\pgfpathcurveto{\pgfqpoint{6.371000in}{1.017811in}}{\pgfqpoint{6.373834in}{1.024652in}}{\pgfqpoint{6.373834in}{1.031785in}}%
\pgfpathcurveto{\pgfqpoint{6.373834in}{1.038918in}}{\pgfqpoint{6.371000in}{1.045760in}}{\pgfqpoint{6.365956in}{1.050803in}}%
\pgfpathcurveto{\pgfqpoint{6.360912in}{1.055847in}}{\pgfqpoint{6.354071in}{1.058681in}}{\pgfqpoint{6.346938in}{1.058681in}}%
\pgfpathcurveto{\pgfqpoint{6.339805in}{1.058681in}}{\pgfqpoint{6.332963in}{1.055847in}}{\pgfqpoint{6.327920in}{1.050803in}}%
\pgfpathcurveto{\pgfqpoint{6.322876in}{1.045760in}}{\pgfqpoint{6.320042in}{1.038918in}}{\pgfqpoint{6.320042in}{1.031785in}}%
\pgfpathcurveto{\pgfqpoint{6.320042in}{1.024652in}}{\pgfqpoint{6.322876in}{1.017811in}}{\pgfqpoint{6.327920in}{1.012767in}}%
\pgfpathcurveto{\pgfqpoint{6.332963in}{1.007723in}}{\pgfqpoint{6.339805in}{1.004890in}}{\pgfqpoint{6.346938in}{1.004890in}}%
\pgfpathclose%
\pgfusepath{stroke,fill}%
\end{pgfscope}%
\begin{pgfscope}%
\pgfpathrectangle{\pgfqpoint{4.985294in}{0.500000in}}{\pgfqpoint{1.764706in}{1.700000in}}%
\pgfusepath{clip}%
\pgfsetbuttcap%
\pgfsetroundjoin%
\definecolor{currentfill}{rgb}{0.964799,0.689101,0.537560}%
\pgfsetfillcolor{currentfill}%
\pgfsetlinewidth{0.311001pt}%
\definecolor{currentstroke}{rgb}{1.000000,1.000000,1.000000}%
\pgfsetstrokecolor{currentstroke}%
\pgfsetdash{}{0pt}%
\pgfpathmoveto{\pgfqpoint{6.345577in}{1.599717in}}%
\pgfpathcurveto{\pgfqpoint{6.352710in}{1.599717in}}{\pgfqpoint{6.359552in}{1.602551in}}{\pgfqpoint{6.364596in}{1.607594in}}%
\pgfpathcurveto{\pgfqpoint{6.369639in}{1.612638in}}{\pgfqpoint{6.372473in}{1.619480in}}{\pgfqpoint{6.372473in}{1.626613in}}%
\pgfpathcurveto{\pgfqpoint{6.372473in}{1.633745in}}{\pgfqpoint{6.369639in}{1.640587in}}{\pgfqpoint{6.364596in}{1.645631in}}%
\pgfpathcurveto{\pgfqpoint{6.359552in}{1.650674in}}{\pgfqpoint{6.352710in}{1.653508in}}{\pgfqpoint{6.345577in}{1.653508in}}%
\pgfpathcurveto{\pgfqpoint{6.338445in}{1.653508in}}{\pgfqpoint{6.331603in}{1.650674in}}{\pgfqpoint{6.326559in}{1.645631in}}%
\pgfpathcurveto{\pgfqpoint{6.321516in}{1.640587in}}{\pgfqpoint{6.318682in}{1.633745in}}{\pgfqpoint{6.318682in}{1.626613in}}%
\pgfpathcurveto{\pgfqpoint{6.318682in}{1.619480in}}{\pgfqpoint{6.321516in}{1.612638in}}{\pgfqpoint{6.326559in}{1.607594in}}%
\pgfpathcurveto{\pgfqpoint{6.331603in}{1.602551in}}{\pgfqpoint{6.338445in}{1.599717in}}{\pgfqpoint{6.345577in}{1.599717in}}%
\pgfpathclose%
\pgfusepath{stroke,fill}%
\end{pgfscope}%
\begin{pgfscope}%
\pgfpathrectangle{\pgfqpoint{4.985294in}{0.500000in}}{\pgfqpoint{1.764706in}{1.700000in}}%
\pgfusepath{clip}%
\pgfsetbuttcap%
\pgfsetroundjoin%
\definecolor{currentfill}{rgb}{0.965440,0.720101,0.576404}%
\pgfsetfillcolor{currentfill}%
\pgfsetlinewidth{0.311001pt}%
\definecolor{currentstroke}{rgb}{1.000000,1.000000,1.000000}%
\pgfsetstrokecolor{currentstroke}%
\pgfsetdash{}{0pt}%
\pgfpathmoveto{\pgfqpoint{6.132427in}{1.682604in}}%
\pgfpathcurveto{\pgfqpoint{6.139560in}{1.682604in}}{\pgfqpoint{6.146402in}{1.685438in}}{\pgfqpoint{6.151446in}{1.690482in}}%
\pgfpathcurveto{\pgfqpoint{6.156489in}{1.695526in}}{\pgfqpoint{6.159323in}{1.702367in}}{\pgfqpoint{6.159323in}{1.709500in}}%
\pgfpathcurveto{\pgfqpoint{6.159323in}{1.716633in}}{\pgfqpoint{6.156489in}{1.723474in}}{\pgfqpoint{6.151446in}{1.728518in}}%
\pgfpathcurveto{\pgfqpoint{6.146402in}{1.733562in}}{\pgfqpoint{6.139560in}{1.736396in}}{\pgfqpoint{6.132427in}{1.736396in}}%
\pgfpathcurveto{\pgfqpoint{6.125295in}{1.736396in}}{\pgfqpoint{6.118453in}{1.733562in}}{\pgfqpoint{6.113409in}{1.728518in}}%
\pgfpathcurveto{\pgfqpoint{6.108366in}{1.723474in}}{\pgfqpoint{6.105532in}{1.716633in}}{\pgfqpoint{6.105532in}{1.709500in}}%
\pgfpathcurveto{\pgfqpoint{6.105532in}{1.702367in}}{\pgfqpoint{6.108366in}{1.695526in}}{\pgfqpoint{6.113409in}{1.690482in}}%
\pgfpathcurveto{\pgfqpoint{6.118453in}{1.685438in}}{\pgfqpoint{6.125295in}{1.682604in}}{\pgfqpoint{6.132427in}{1.682604in}}%
\pgfpathclose%
\pgfusepath{stroke,fill}%
\end{pgfscope}%
\begin{pgfscope}%
\pgfpathrectangle{\pgfqpoint{4.985294in}{0.500000in}}{\pgfqpoint{1.764706in}{1.700000in}}%
\pgfusepath{clip}%
\pgfsetbuttcap%
\pgfsetroundjoin%
\definecolor{currentfill}{rgb}{0.969803,0.809811,0.702523}%
\pgfsetfillcolor{currentfill}%
\pgfsetlinewidth{0.311001pt}%
\definecolor{currentstroke}{rgb}{1.000000,1.000000,1.000000}%
\pgfsetstrokecolor{currentstroke}%
\pgfsetdash{}{0pt}%
\pgfpathmoveto{\pgfqpoint{5.352711in}{1.258156in}}%
\pgfpathcurveto{\pgfqpoint{5.359844in}{1.258156in}}{\pgfqpoint{5.366686in}{1.260990in}}{\pgfqpoint{5.371729in}{1.266034in}}%
\pgfpathcurveto{\pgfqpoint{5.376773in}{1.271078in}}{\pgfqpoint{5.379607in}{1.277919in}}{\pgfqpoint{5.379607in}{1.285052in}}%
\pgfpathcurveto{\pgfqpoint{5.379607in}{1.292185in}}{\pgfqpoint{5.376773in}{1.299027in}}{\pgfqpoint{5.371729in}{1.304070in}}%
\pgfpathcurveto{\pgfqpoint{5.366686in}{1.309114in}}{\pgfqpoint{5.359844in}{1.311948in}}{\pgfqpoint{5.352711in}{1.311948in}}%
\pgfpathcurveto{\pgfqpoint{5.345578in}{1.311948in}}{\pgfqpoint{5.338737in}{1.309114in}}{\pgfqpoint{5.333693in}{1.304070in}}%
\pgfpathcurveto{\pgfqpoint{5.328649in}{1.299027in}}{\pgfqpoint{5.325815in}{1.292185in}}{\pgfqpoint{5.325815in}{1.285052in}}%
\pgfpathcurveto{\pgfqpoint{5.325815in}{1.277919in}}{\pgfqpoint{5.328649in}{1.271078in}}{\pgfqpoint{5.333693in}{1.266034in}}%
\pgfpathcurveto{\pgfqpoint{5.338737in}{1.260990in}}{\pgfqpoint{5.345578in}{1.258156in}}{\pgfqpoint{5.352711in}{1.258156in}}%
\pgfpathclose%
\pgfusepath{stroke,fill}%
\end{pgfscope}%
\begin{pgfscope}%
\pgfpathrectangle{\pgfqpoint{4.985294in}{0.500000in}}{\pgfqpoint{1.764706in}{1.700000in}}%
\pgfusepath{clip}%
\pgfsetbuttcap%
\pgfsetroundjoin%
\definecolor{currentfill}{rgb}{0.947270,0.405591,0.279023}%
\pgfsetfillcolor{currentfill}%
\pgfsetlinewidth{0.311001pt}%
\definecolor{currentstroke}{rgb}{1.000000,1.000000,1.000000}%
\pgfsetstrokecolor{currentstroke}%
\pgfsetdash{}{0pt}%
\pgfpathmoveto{\pgfqpoint{6.454795in}{1.349071in}}%
\pgfpathcurveto{\pgfqpoint{6.461928in}{1.349071in}}{\pgfqpoint{6.468770in}{1.351905in}}{\pgfqpoint{6.473813in}{1.356948in}}%
\pgfpathcurveto{\pgfqpoint{6.478857in}{1.361992in}}{\pgfqpoint{6.481691in}{1.368834in}}{\pgfqpoint{6.481691in}{1.375966in}}%
\pgfpathcurveto{\pgfqpoint{6.481691in}{1.383099in}}{\pgfqpoint{6.478857in}{1.389941in}}{\pgfqpoint{6.473813in}{1.394985in}}%
\pgfpathcurveto{\pgfqpoint{6.468770in}{1.400028in}}{\pgfqpoint{6.461928in}{1.402862in}}{\pgfqpoint{6.454795in}{1.402862in}}%
\pgfpathcurveto{\pgfqpoint{6.447662in}{1.402862in}}{\pgfqpoint{6.440821in}{1.400028in}}{\pgfqpoint{6.435777in}{1.394985in}}%
\pgfpathcurveto{\pgfqpoint{6.430733in}{1.389941in}}{\pgfqpoint{6.427899in}{1.383099in}}{\pgfqpoint{6.427899in}{1.375966in}}%
\pgfpathcurveto{\pgfqpoint{6.427899in}{1.368834in}}{\pgfqpoint{6.430733in}{1.361992in}}{\pgfqpoint{6.435777in}{1.356948in}}%
\pgfpathcurveto{\pgfqpoint{6.440821in}{1.351905in}}{\pgfqpoint{6.447662in}{1.349071in}}{\pgfqpoint{6.454795in}{1.349071in}}%
\pgfpathclose%
\pgfusepath{stroke,fill}%
\end{pgfscope}%
\begin{pgfscope}%
\pgfpathrectangle{\pgfqpoint{4.985294in}{0.500000in}}{\pgfqpoint{1.764706in}{1.700000in}}%
\pgfusepath{clip}%
\pgfsetbuttcap%
\pgfsetroundjoin%
\definecolor{currentfill}{rgb}{0.970718,0.821518,0.719872}%
\pgfsetfillcolor{currentfill}%
\pgfsetlinewidth{0.311001pt}%
\definecolor{currentstroke}{rgb}{1.000000,1.000000,1.000000}%
\pgfsetstrokecolor{currentstroke}%
\pgfsetdash{}{0pt}%
\pgfpathmoveto{\pgfqpoint{6.236663in}{1.447171in}}%
\pgfpathcurveto{\pgfqpoint{6.243796in}{1.447171in}}{\pgfqpoint{6.250637in}{1.450005in}}{\pgfqpoint{6.255681in}{1.455049in}}%
\pgfpathcurveto{\pgfqpoint{6.260725in}{1.460092in}}{\pgfqpoint{6.263559in}{1.466934in}}{\pgfqpoint{6.263559in}{1.474067in}}%
\pgfpathcurveto{\pgfqpoint{6.263559in}{1.481200in}}{\pgfqpoint{6.260725in}{1.488041in}}{\pgfqpoint{6.255681in}{1.493085in}}%
\pgfpathcurveto{\pgfqpoint{6.250637in}{1.498129in}}{\pgfqpoint{6.243796in}{1.500962in}}{\pgfqpoint{6.236663in}{1.500962in}}%
\pgfpathcurveto{\pgfqpoint{6.229530in}{1.500962in}}{\pgfqpoint{6.222688in}{1.498129in}}{\pgfqpoint{6.217645in}{1.493085in}}%
\pgfpathcurveto{\pgfqpoint{6.212601in}{1.488041in}}{\pgfqpoint{6.209767in}{1.481200in}}{\pgfqpoint{6.209767in}{1.474067in}}%
\pgfpathcurveto{\pgfqpoint{6.209767in}{1.466934in}}{\pgfqpoint{6.212601in}{1.460092in}}{\pgfqpoint{6.217645in}{1.455049in}}%
\pgfpathcurveto{\pgfqpoint{6.222688in}{1.450005in}}{\pgfqpoint{6.229530in}{1.447171in}}{\pgfqpoint{6.236663in}{1.447171in}}%
\pgfpathclose%
\pgfusepath{stroke,fill}%
\end{pgfscope}%
\begin{pgfscope}%
\pgfpathrectangle{\pgfqpoint{4.985294in}{0.500000in}}{\pgfqpoint{1.764706in}{1.700000in}}%
\pgfusepath{clip}%
\pgfsetbuttcap%
\pgfsetroundjoin%
\definecolor{currentfill}{rgb}{0.949145,0.420383,0.287810}%
\pgfsetfillcolor{currentfill}%
\pgfsetlinewidth{0.311001pt}%
\definecolor{currentstroke}{rgb}{1.000000,1.000000,1.000000}%
\pgfsetstrokecolor{currentstroke}%
\pgfsetdash{}{0pt}%
\pgfpathmoveto{\pgfqpoint{6.144418in}{1.285883in}}%
\pgfpathcurveto{\pgfqpoint{6.151551in}{1.285883in}}{\pgfqpoint{6.158392in}{1.288717in}}{\pgfqpoint{6.163436in}{1.293760in}}%
\pgfpathcurveto{\pgfqpoint{6.168480in}{1.298804in}}{\pgfqpoint{6.171314in}{1.305646in}}{\pgfqpoint{6.171314in}{1.312778in}}%
\pgfpathcurveto{\pgfqpoint{6.171314in}{1.319911in}}{\pgfqpoint{6.168480in}{1.326753in}}{\pgfqpoint{6.163436in}{1.331796in}}%
\pgfpathcurveto{\pgfqpoint{6.158392in}{1.336840in}}{\pgfqpoint{6.151551in}{1.339674in}}{\pgfqpoint{6.144418in}{1.339674in}}%
\pgfpathcurveto{\pgfqpoint{6.137285in}{1.339674in}}{\pgfqpoint{6.130444in}{1.336840in}}{\pgfqpoint{6.125400in}{1.331796in}}%
\pgfpathcurveto{\pgfqpoint{6.120356in}{1.326753in}}{\pgfqpoint{6.117522in}{1.319911in}}{\pgfqpoint{6.117522in}{1.312778in}}%
\pgfpathcurveto{\pgfqpoint{6.117522in}{1.305646in}}{\pgfqpoint{6.120356in}{1.298804in}}{\pgfqpoint{6.125400in}{1.293760in}}%
\pgfpathcurveto{\pgfqpoint{6.130444in}{1.288717in}}{\pgfqpoint{6.137285in}{1.285883in}}{\pgfqpoint{6.144418in}{1.285883in}}%
\pgfpathclose%
\pgfusepath{stroke,fill}%
\end{pgfscope}%
\begin{pgfscope}%
\pgfpathrectangle{\pgfqpoint{4.985294in}{0.500000in}}{\pgfqpoint{1.764706in}{1.700000in}}%
\pgfusepath{clip}%
\pgfsetbuttcap%
\pgfsetroundjoin%
\definecolor{currentfill}{rgb}{0.962283,0.593046,0.431453}%
\pgfsetfillcolor{currentfill}%
\pgfsetlinewidth{0.311001pt}%
\definecolor{currentstroke}{rgb}{1.000000,1.000000,1.000000}%
\pgfsetstrokecolor{currentstroke}%
\pgfsetdash{}{0pt}%
\pgfpathmoveto{\pgfqpoint{6.138456in}{1.120363in}}%
\pgfpathcurveto{\pgfqpoint{6.145589in}{1.120363in}}{\pgfqpoint{6.152430in}{1.123197in}}{\pgfqpoint{6.157474in}{1.128241in}}%
\pgfpathcurveto{\pgfqpoint{6.162518in}{1.133285in}}{\pgfqpoint{6.165352in}{1.140126in}}{\pgfqpoint{6.165352in}{1.147259in}}%
\pgfpathcurveto{\pgfqpoint{6.165352in}{1.154392in}}{\pgfqpoint{6.162518in}{1.161234in}}{\pgfqpoint{6.157474in}{1.166277in}}%
\pgfpathcurveto{\pgfqpoint{6.152430in}{1.171321in}}{\pgfqpoint{6.145589in}{1.174155in}}{\pgfqpoint{6.138456in}{1.174155in}}%
\pgfpathcurveto{\pgfqpoint{6.131323in}{1.174155in}}{\pgfqpoint{6.124481in}{1.171321in}}{\pgfqpoint{6.119438in}{1.166277in}}%
\pgfpathcurveto{\pgfqpoint{6.114394in}{1.161234in}}{\pgfqpoint{6.111560in}{1.154392in}}{\pgfqpoint{6.111560in}{1.147259in}}%
\pgfpathcurveto{\pgfqpoint{6.111560in}{1.140126in}}{\pgfqpoint{6.114394in}{1.133285in}}{\pgfqpoint{6.119438in}{1.128241in}}%
\pgfpathcurveto{\pgfqpoint{6.124481in}{1.123197in}}{\pgfqpoint{6.131323in}{1.120363in}}{\pgfqpoint{6.138456in}{1.120363in}}%
\pgfpathclose%
\pgfusepath{stroke,fill}%
\end{pgfscope}%
\begin{pgfscope}%
\pgfpathrectangle{\pgfqpoint{4.985294in}{0.500000in}}{\pgfqpoint{1.764706in}{1.700000in}}%
\pgfusepath{clip}%
\pgfsetbuttcap%
\pgfsetroundjoin%
\definecolor{currentfill}{rgb}{0.973832,0.856556,0.771584}%
\pgfsetfillcolor{currentfill}%
\pgfsetlinewidth{0.311001pt}%
\definecolor{currentstroke}{rgb}{1.000000,1.000000,1.000000}%
\pgfsetstrokecolor{currentstroke}%
\pgfsetdash{}{0pt}%
\pgfpathmoveto{\pgfqpoint{5.472315in}{1.345716in}}%
\pgfpathcurveto{\pgfqpoint{5.479448in}{1.345716in}}{\pgfqpoint{5.486290in}{1.348550in}}{\pgfqpoint{5.491333in}{1.353594in}}%
\pgfpathcurveto{\pgfqpoint{5.496377in}{1.358638in}}{\pgfqpoint{5.499211in}{1.365479in}}{\pgfqpoint{5.499211in}{1.372612in}}%
\pgfpathcurveto{\pgfqpoint{5.499211in}{1.379745in}}{\pgfqpoint{5.496377in}{1.386587in}}{\pgfqpoint{5.491333in}{1.391630in}}%
\pgfpathcurveto{\pgfqpoint{5.486290in}{1.396674in}}{\pgfqpoint{5.479448in}{1.399508in}}{\pgfqpoint{5.472315in}{1.399508in}}%
\pgfpathcurveto{\pgfqpoint{5.465182in}{1.399508in}}{\pgfqpoint{5.458341in}{1.396674in}}{\pgfqpoint{5.453297in}{1.391630in}}%
\pgfpathcurveto{\pgfqpoint{5.448253in}{1.386587in}}{\pgfqpoint{5.445420in}{1.379745in}}{\pgfqpoint{5.445420in}{1.372612in}}%
\pgfpathcurveto{\pgfqpoint{5.445420in}{1.365479in}}{\pgfqpoint{5.448253in}{1.358638in}}{\pgfqpoint{5.453297in}{1.353594in}}%
\pgfpathcurveto{\pgfqpoint{5.458341in}{1.348550in}}{\pgfqpoint{5.465182in}{1.345716in}}{\pgfqpoint{5.472315in}{1.345716in}}%
\pgfpathclose%
\pgfusepath{stroke,fill}%
\end{pgfscope}%
\begin{pgfscope}%
\pgfpathrectangle{\pgfqpoint{4.985294in}{0.500000in}}{\pgfqpoint{1.764706in}{1.700000in}}%
\pgfusepath{clip}%
\pgfsetbuttcap%
\pgfsetroundjoin%
\definecolor{currentfill}{rgb}{0.965928,0.738443,0.600540}%
\pgfsetfillcolor{currentfill}%
\pgfsetlinewidth{0.311001pt}%
\definecolor{currentstroke}{rgb}{1.000000,1.000000,1.000000}%
\pgfsetstrokecolor{currentstroke}%
\pgfsetdash{}{0pt}%
\pgfpathmoveto{\pgfqpoint{6.190126in}{1.201242in}}%
\pgfpathcurveto{\pgfqpoint{6.197259in}{1.201242in}}{\pgfqpoint{6.204101in}{1.204075in}}{\pgfqpoint{6.209145in}{1.209119in}}%
\pgfpathcurveto{\pgfqpoint{6.214188in}{1.214163in}}{\pgfqpoint{6.217022in}{1.221004in}}{\pgfqpoint{6.217022in}{1.228137in}}%
\pgfpathcurveto{\pgfqpoint{6.217022in}{1.235270in}}{\pgfqpoint{6.214188in}{1.242112in}}{\pgfqpoint{6.209145in}{1.247155in}}%
\pgfpathcurveto{\pgfqpoint{6.204101in}{1.252199in}}{\pgfqpoint{6.197259in}{1.255033in}}{\pgfqpoint{6.190126in}{1.255033in}}%
\pgfpathcurveto{\pgfqpoint{6.182994in}{1.255033in}}{\pgfqpoint{6.176152in}{1.252199in}}{\pgfqpoint{6.171108in}{1.247155in}}%
\pgfpathcurveto{\pgfqpoint{6.166065in}{1.242112in}}{\pgfqpoint{6.163231in}{1.235270in}}{\pgfqpoint{6.163231in}{1.228137in}}%
\pgfpathcurveto{\pgfqpoint{6.163231in}{1.221004in}}{\pgfqpoint{6.166065in}{1.214163in}}{\pgfqpoint{6.171108in}{1.209119in}}%
\pgfpathcurveto{\pgfqpoint{6.176152in}{1.204075in}}{\pgfqpoint{6.182994in}{1.201242in}}{\pgfqpoint{6.190126in}{1.201242in}}%
\pgfpathclose%
\pgfusepath{stroke,fill}%
\end{pgfscope}%
\begin{pgfscope}%
\pgfpathrectangle{\pgfqpoint{4.985294in}{0.500000in}}{\pgfqpoint{1.764706in}{1.700000in}}%
\pgfusepath{clip}%
\pgfsetbuttcap%
\pgfsetroundjoin%
\definecolor{currentfill}{rgb}{0.970255,0.815666,0.711203}%
\pgfsetfillcolor{currentfill}%
\pgfsetlinewidth{0.311001pt}%
\definecolor{currentstroke}{rgb}{1.000000,1.000000,1.000000}%
\pgfsetstrokecolor{currentstroke}%
\pgfsetdash{}{0pt}%
\pgfpathmoveto{\pgfqpoint{6.205819in}{1.528097in}}%
\pgfpathcurveto{\pgfqpoint{6.212952in}{1.528097in}}{\pgfqpoint{6.219793in}{1.530931in}}{\pgfqpoint{6.224837in}{1.535975in}}%
\pgfpathcurveto{\pgfqpoint{6.229881in}{1.541019in}}{\pgfqpoint{6.232715in}{1.547860in}}{\pgfqpoint{6.232715in}{1.554993in}}%
\pgfpathcurveto{\pgfqpoint{6.232715in}{1.562126in}}{\pgfqpoint{6.229881in}{1.568968in}}{\pgfqpoint{6.224837in}{1.574011in}}%
\pgfpathcurveto{\pgfqpoint{6.219793in}{1.579055in}}{\pgfqpoint{6.212952in}{1.581889in}}{\pgfqpoint{6.205819in}{1.581889in}}%
\pgfpathcurveto{\pgfqpoint{6.198686in}{1.581889in}}{\pgfqpoint{6.191845in}{1.579055in}}{\pgfqpoint{6.186801in}{1.574011in}}%
\pgfpathcurveto{\pgfqpoint{6.181757in}{1.568968in}}{\pgfqpoint{6.178923in}{1.562126in}}{\pgfqpoint{6.178923in}{1.554993in}}%
\pgfpathcurveto{\pgfqpoint{6.178923in}{1.547860in}}{\pgfqpoint{6.181757in}{1.541019in}}{\pgfqpoint{6.186801in}{1.535975in}}%
\pgfpathcurveto{\pgfqpoint{6.191845in}{1.530931in}}{\pgfqpoint{6.198686in}{1.528097in}}{\pgfqpoint{6.205819in}{1.528097in}}%
\pgfpathclose%
\pgfusepath{stroke,fill}%
\end{pgfscope}%
\begin{pgfscope}%
\pgfpathrectangle{\pgfqpoint{4.985294in}{0.500000in}}{\pgfqpoint{1.764706in}{1.700000in}}%
\pgfusepath{clip}%
\pgfsetbuttcap%
\pgfsetroundjoin%
\definecolor{currentfill}{rgb}{0.974412,0.862387,0.780156}%
\pgfsetfillcolor{currentfill}%
\pgfsetlinewidth{0.311001pt}%
\definecolor{currentstroke}{rgb}{1.000000,1.000000,1.000000}%
\pgfsetstrokecolor{currentstroke}%
\pgfsetdash{}{0pt}%
\pgfpathmoveto{\pgfqpoint{6.263309in}{1.304255in}}%
\pgfpathcurveto{\pgfqpoint{6.270442in}{1.304255in}}{\pgfqpoint{6.277283in}{1.307089in}}{\pgfqpoint{6.282327in}{1.312132in}}%
\pgfpathcurveto{\pgfqpoint{6.287371in}{1.317176in}}{\pgfqpoint{6.290205in}{1.324018in}}{\pgfqpoint{6.290205in}{1.331151in}}%
\pgfpathcurveto{\pgfqpoint{6.290205in}{1.338283in}}{\pgfqpoint{6.287371in}{1.345125in}}{\pgfqpoint{6.282327in}{1.350169in}}%
\pgfpathcurveto{\pgfqpoint{6.277283in}{1.355212in}}{\pgfqpoint{6.270442in}{1.358046in}}{\pgfqpoint{6.263309in}{1.358046in}}%
\pgfpathcurveto{\pgfqpoint{6.256176in}{1.358046in}}{\pgfqpoint{6.249335in}{1.355212in}}{\pgfqpoint{6.244291in}{1.350169in}}%
\pgfpathcurveto{\pgfqpoint{6.239247in}{1.345125in}}{\pgfqpoint{6.236413in}{1.338283in}}{\pgfqpoint{6.236413in}{1.331151in}}%
\pgfpathcurveto{\pgfqpoint{6.236413in}{1.324018in}}{\pgfqpoint{6.239247in}{1.317176in}}{\pgfqpoint{6.244291in}{1.312132in}}%
\pgfpathcurveto{\pgfqpoint{6.249335in}{1.307089in}}{\pgfqpoint{6.256176in}{1.304255in}}{\pgfqpoint{6.263309in}{1.304255in}}%
\pgfpathclose%
\pgfusepath{stroke,fill}%
\end{pgfscope}%
\begin{pgfscope}%
\pgfpathrectangle{\pgfqpoint{4.985294in}{0.500000in}}{\pgfqpoint{1.764706in}{1.700000in}}%
\pgfusepath{clip}%
\pgfsetbuttcap%
\pgfsetroundjoin%
\definecolor{currentfill}{rgb}{0.973832,0.856556,0.771584}%
\pgfsetfillcolor{currentfill}%
\pgfsetlinewidth{0.311001pt}%
\definecolor{currentstroke}{rgb}{1.000000,1.000000,1.000000}%
\pgfsetstrokecolor{currentstroke}%
\pgfsetdash{}{0pt}%
\pgfpathmoveto{\pgfqpoint{5.433197in}{1.551679in}}%
\pgfpathcurveto{\pgfqpoint{5.440330in}{1.551679in}}{\pgfqpoint{5.447172in}{1.554513in}}{\pgfqpoint{5.452215in}{1.559557in}}%
\pgfpathcurveto{\pgfqpoint{5.457259in}{1.564601in}}{\pgfqpoint{5.460093in}{1.571442in}}{\pgfqpoint{5.460093in}{1.578575in}}%
\pgfpathcurveto{\pgfqpoint{5.460093in}{1.585708in}}{\pgfqpoint{5.457259in}{1.592550in}}{\pgfqpoint{5.452215in}{1.597593in}}%
\pgfpathcurveto{\pgfqpoint{5.447172in}{1.602637in}}{\pgfqpoint{5.440330in}{1.605471in}}{\pgfqpoint{5.433197in}{1.605471in}}%
\pgfpathcurveto{\pgfqpoint{5.426064in}{1.605471in}}{\pgfqpoint{5.419223in}{1.602637in}}{\pgfqpoint{5.414179in}{1.597593in}}%
\pgfpathcurveto{\pgfqpoint{5.409135in}{1.592550in}}{\pgfqpoint{5.406301in}{1.585708in}}{\pgfqpoint{5.406301in}{1.578575in}}%
\pgfpathcurveto{\pgfqpoint{5.406301in}{1.571442in}}{\pgfqpoint{5.409135in}{1.564601in}}{\pgfqpoint{5.414179in}{1.559557in}}%
\pgfpathcurveto{\pgfqpoint{5.419223in}{1.554513in}}{\pgfqpoint{5.426064in}{1.551679in}}{\pgfqpoint{5.433197in}{1.551679in}}%
\pgfpathclose%
\pgfusepath{stroke,fill}%
\end{pgfscope}%
\begin{pgfscope}%
\pgfpathrectangle{\pgfqpoint{4.985294in}{0.500000in}}{\pgfqpoint{1.764706in}{1.700000in}}%
\pgfusepath{clip}%
\pgfsetbuttcap%
\pgfsetroundjoin%
\definecolor{currentfill}{rgb}{0.955697,0.484891,0.334214}%
\pgfsetfillcolor{currentfill}%
\pgfsetlinewidth{0.311001pt}%
\definecolor{currentstroke}{rgb}{1.000000,1.000000,1.000000}%
\pgfsetstrokecolor{currentstroke}%
\pgfsetdash{}{0pt}%
\pgfpathmoveto{\pgfqpoint{5.292734in}{1.412197in}}%
\pgfpathcurveto{\pgfqpoint{5.299866in}{1.412197in}}{\pgfqpoint{5.306708in}{1.415031in}}{\pgfqpoint{5.311752in}{1.420075in}}%
\pgfpathcurveto{\pgfqpoint{5.316795in}{1.425119in}}{\pgfqpoint{5.319629in}{1.431960in}}{\pgfqpoint{5.319629in}{1.439093in}}%
\pgfpathcurveto{\pgfqpoint{5.319629in}{1.446226in}}{\pgfqpoint{5.316795in}{1.453067in}}{\pgfqpoint{5.311752in}{1.458111in}}%
\pgfpathcurveto{\pgfqpoint{5.306708in}{1.463155in}}{\pgfqpoint{5.299866in}{1.465989in}}{\pgfqpoint{5.292734in}{1.465989in}}%
\pgfpathcurveto{\pgfqpoint{5.285601in}{1.465989in}}{\pgfqpoint{5.278759in}{1.463155in}}{\pgfqpoint{5.273716in}{1.458111in}}%
\pgfpathcurveto{\pgfqpoint{5.268672in}{1.453067in}}{\pgfqpoint{5.265838in}{1.446226in}}{\pgfqpoint{5.265838in}{1.439093in}}%
\pgfpathcurveto{\pgfqpoint{5.265838in}{1.431960in}}{\pgfqpoint{5.268672in}{1.425119in}}{\pgfqpoint{5.273716in}{1.420075in}}%
\pgfpathcurveto{\pgfqpoint{5.278759in}{1.415031in}}{\pgfqpoint{5.285601in}{1.412197in}}{\pgfqpoint{5.292734in}{1.412197in}}%
\pgfpathclose%
\pgfusepath{stroke,fill}%
\end{pgfscope}%
\begin{pgfscope}%
\pgfpathrectangle{\pgfqpoint{4.985294in}{0.500000in}}{\pgfqpoint{1.764706in}{1.700000in}}%
\pgfusepath{clip}%
\pgfsetbuttcap%
\pgfsetroundjoin%
\definecolor{currentfill}{rgb}{0.969803,0.809811,0.702523}%
\pgfsetfillcolor{currentfill}%
\pgfsetlinewidth{0.311001pt}%
\definecolor{currentstroke}{rgb}{1.000000,1.000000,1.000000}%
\pgfsetstrokecolor{currentstroke}%
\pgfsetdash{}{0pt}%
\pgfpathmoveto{\pgfqpoint{5.423850in}{1.585567in}}%
\pgfpathcurveto{\pgfqpoint{5.430983in}{1.585567in}}{\pgfqpoint{5.437825in}{1.588401in}}{\pgfqpoint{5.442868in}{1.593445in}}%
\pgfpathcurveto{\pgfqpoint{5.447912in}{1.598488in}}{\pgfqpoint{5.450746in}{1.605330in}}{\pgfqpoint{5.450746in}{1.612463in}}%
\pgfpathcurveto{\pgfqpoint{5.450746in}{1.619596in}}{\pgfqpoint{5.447912in}{1.626437in}}{\pgfqpoint{5.442868in}{1.631481in}}%
\pgfpathcurveto{\pgfqpoint{5.437825in}{1.636525in}}{\pgfqpoint{5.430983in}{1.639358in}}{\pgfqpoint{5.423850in}{1.639358in}}%
\pgfpathcurveto{\pgfqpoint{5.416718in}{1.639358in}}{\pgfqpoint{5.409876in}{1.636525in}}{\pgfqpoint{5.404832in}{1.631481in}}%
\pgfpathcurveto{\pgfqpoint{5.399789in}{1.626437in}}{\pgfqpoint{5.396955in}{1.619596in}}{\pgfqpoint{5.396955in}{1.612463in}}%
\pgfpathcurveto{\pgfqpoint{5.396955in}{1.605330in}}{\pgfqpoint{5.399789in}{1.598488in}}{\pgfqpoint{5.404832in}{1.593445in}}%
\pgfpathcurveto{\pgfqpoint{5.409876in}{1.588401in}}{\pgfqpoint{5.416718in}{1.585567in}}{\pgfqpoint{5.423850in}{1.585567in}}%
\pgfpathclose%
\pgfusepath{stroke,fill}%
\end{pgfscope}%
\begin{pgfscope}%
\pgfpathrectangle{\pgfqpoint{4.985294in}{0.500000in}}{\pgfqpoint{1.764706in}{1.700000in}}%
\pgfusepath{clip}%
\pgfsetbuttcap%
\pgfsetroundjoin%
\definecolor{currentfill}{rgb}{0.971202,0.827364,0.728520}%
\pgfsetfillcolor{currentfill}%
\pgfsetlinewidth{0.311001pt}%
\definecolor{currentstroke}{rgb}{1.000000,1.000000,1.000000}%
\pgfsetstrokecolor{currentstroke}%
\pgfsetdash{}{0pt}%
\pgfpathmoveto{\pgfqpoint{5.402993in}{1.531822in}}%
\pgfpathcurveto{\pgfqpoint{5.410126in}{1.531822in}}{\pgfqpoint{5.416967in}{1.534655in}}{\pgfqpoint{5.422011in}{1.539699in}}%
\pgfpathcurveto{\pgfqpoint{5.427055in}{1.544743in}}{\pgfqpoint{5.429889in}{1.551584in}}{\pgfqpoint{5.429889in}{1.558717in}}%
\pgfpathcurveto{\pgfqpoint{5.429889in}{1.565850in}}{\pgfqpoint{5.427055in}{1.572692in}}{\pgfqpoint{5.422011in}{1.577735in}}%
\pgfpathcurveto{\pgfqpoint{5.416967in}{1.582779in}}{\pgfqpoint{5.410126in}{1.585613in}}{\pgfqpoint{5.402993in}{1.585613in}}%
\pgfpathcurveto{\pgfqpoint{5.395860in}{1.585613in}}{\pgfqpoint{5.389018in}{1.582779in}}{\pgfqpoint{5.383975in}{1.577735in}}%
\pgfpathcurveto{\pgfqpoint{5.378931in}{1.572692in}}{\pgfqpoint{5.376097in}{1.565850in}}{\pgfqpoint{5.376097in}{1.558717in}}%
\pgfpathcurveto{\pgfqpoint{5.376097in}{1.551584in}}{\pgfqpoint{5.378931in}{1.544743in}}{\pgfqpoint{5.383975in}{1.539699in}}%
\pgfpathcurveto{\pgfqpoint{5.389018in}{1.534655in}}{\pgfqpoint{5.395860in}{1.531822in}}{\pgfqpoint{5.402993in}{1.531822in}}%
\pgfpathclose%
\pgfusepath{stroke,fill}%
\end{pgfscope}%
\begin{pgfscope}%
\pgfpathrectangle{\pgfqpoint{4.985294in}{0.500000in}}{\pgfqpoint{1.764706in}{1.700000in}}%
\pgfusepath{clip}%
\pgfsetbuttcap%
\pgfsetroundjoin%
\definecolor{currentfill}{rgb}{0.975644,0.874038,0.797253}%
\pgfsetfillcolor{currentfill}%
\pgfsetlinewidth{0.311001pt}%
\definecolor{currentstroke}{rgb}{1.000000,1.000000,1.000000}%
\pgfsetstrokecolor{currentstroke}%
\pgfsetdash{}{0pt}%
\pgfpathmoveto{\pgfqpoint{5.468940in}{1.180997in}}%
\pgfpathcurveto{\pgfqpoint{5.476073in}{1.180997in}}{\pgfqpoint{5.482914in}{1.183831in}}{\pgfqpoint{5.487958in}{1.188874in}}%
\pgfpathcurveto{\pgfqpoint{5.493002in}{1.193918in}}{\pgfqpoint{5.495835in}{1.200760in}}{\pgfqpoint{5.495835in}{1.207893in}}%
\pgfpathcurveto{\pgfqpoint{5.495835in}{1.215025in}}{\pgfqpoint{5.493002in}{1.221867in}}{\pgfqpoint{5.487958in}{1.226911in}}%
\pgfpathcurveto{\pgfqpoint{5.482914in}{1.231954in}}{\pgfqpoint{5.476073in}{1.234788in}}{\pgfqpoint{5.468940in}{1.234788in}}%
\pgfpathcurveto{\pgfqpoint{5.461807in}{1.234788in}}{\pgfqpoint{5.454965in}{1.231954in}}{\pgfqpoint{5.449922in}{1.226911in}}%
\pgfpathcurveto{\pgfqpoint{5.444878in}{1.221867in}}{\pgfqpoint{5.442044in}{1.215025in}}{\pgfqpoint{5.442044in}{1.207893in}}%
\pgfpathcurveto{\pgfqpoint{5.442044in}{1.200760in}}{\pgfqpoint{5.444878in}{1.193918in}}{\pgfqpoint{5.449922in}{1.188874in}}%
\pgfpathcurveto{\pgfqpoint{5.454965in}{1.183831in}}{\pgfqpoint{5.461807in}{1.180997in}}{\pgfqpoint{5.468940in}{1.180997in}}%
\pgfpathclose%
\pgfusepath{stroke,fill}%
\end{pgfscope}%
\begin{pgfscope}%
\pgfpathrectangle{\pgfqpoint{4.985294in}{0.500000in}}{\pgfqpoint{1.764706in}{1.700000in}}%
\pgfusepath{clip}%
\pgfsetbuttcap%
\pgfsetroundjoin%
\definecolor{currentfill}{rgb}{0.970255,0.815666,0.711203}%
\pgfsetfillcolor{currentfill}%
\pgfsetlinewidth{0.311001pt}%
\definecolor{currentstroke}{rgb}{1.000000,1.000000,1.000000}%
\pgfsetstrokecolor{currentstroke}%
\pgfsetdash{}{0pt}%
\pgfpathmoveto{\pgfqpoint{5.410092in}{1.070944in}}%
\pgfpathcurveto{\pgfqpoint{5.417225in}{1.070944in}}{\pgfqpoint{5.424067in}{1.073778in}}{\pgfqpoint{5.429110in}{1.078822in}}%
\pgfpathcurveto{\pgfqpoint{5.434154in}{1.083866in}}{\pgfqpoint{5.436988in}{1.090707in}}{\pgfqpoint{5.436988in}{1.097840in}}%
\pgfpathcurveto{\pgfqpoint{5.436988in}{1.104973in}}{\pgfqpoint{5.434154in}{1.111815in}}{\pgfqpoint{5.429110in}{1.116858in}}%
\pgfpathcurveto{\pgfqpoint{5.424067in}{1.121902in}}{\pgfqpoint{5.417225in}{1.124736in}}{\pgfqpoint{5.410092in}{1.124736in}}%
\pgfpathcurveto{\pgfqpoint{5.402960in}{1.124736in}}{\pgfqpoint{5.396118in}{1.121902in}}{\pgfqpoint{5.391074in}{1.116858in}}%
\pgfpathcurveto{\pgfqpoint{5.386031in}{1.111815in}}{\pgfqpoint{5.383197in}{1.104973in}}{\pgfqpoint{5.383197in}{1.097840in}}%
\pgfpathcurveto{\pgfqpoint{5.383197in}{1.090707in}}{\pgfqpoint{5.386031in}{1.083866in}}{\pgfqpoint{5.391074in}{1.078822in}}%
\pgfpathcurveto{\pgfqpoint{5.396118in}{1.073778in}}{\pgfqpoint{5.402960in}{1.070944in}}{\pgfqpoint{5.410092in}{1.070944in}}%
\pgfpathclose%
\pgfusepath{stroke,fill}%
\end{pgfscope}%
\begin{pgfscope}%
\pgfpathrectangle{\pgfqpoint{4.985294in}{0.500000in}}{\pgfqpoint{1.764706in}{1.700000in}}%
\pgfusepath{clip}%
\pgfsetbuttcap%
\pgfsetroundjoin%
\definecolor{currentfill}{rgb}{0.973832,0.856556,0.771584}%
\pgfsetfillcolor{currentfill}%
\pgfsetlinewidth{0.311001pt}%
\definecolor{currentstroke}{rgb}{1.000000,1.000000,1.000000}%
\pgfsetstrokecolor{currentstroke}%
\pgfsetdash{}{0pt}%
\pgfpathmoveto{\pgfqpoint{5.472905in}{1.349924in}}%
\pgfpathcurveto{\pgfqpoint{5.480038in}{1.349924in}}{\pgfqpoint{5.486880in}{1.352757in}}{\pgfqpoint{5.491923in}{1.357801in}}%
\pgfpathcurveto{\pgfqpoint{5.496967in}{1.362845in}}{\pgfqpoint{5.499801in}{1.369686in}}{\pgfqpoint{5.499801in}{1.376819in}}%
\pgfpathcurveto{\pgfqpoint{5.499801in}{1.383952in}}{\pgfqpoint{5.496967in}{1.390794in}}{\pgfqpoint{5.491923in}{1.395837in}}%
\pgfpathcurveto{\pgfqpoint{5.486880in}{1.400881in}}{\pgfqpoint{5.480038in}{1.403715in}}{\pgfqpoint{5.472905in}{1.403715in}}%
\pgfpathcurveto{\pgfqpoint{5.465772in}{1.403715in}}{\pgfqpoint{5.458931in}{1.400881in}}{\pgfqpoint{5.453887in}{1.395837in}}%
\pgfpathcurveto{\pgfqpoint{5.448843in}{1.390794in}}{\pgfqpoint{5.446009in}{1.383952in}}{\pgfqpoint{5.446009in}{1.376819in}}%
\pgfpathcurveto{\pgfqpoint{5.446009in}{1.369686in}}{\pgfqpoint{5.448843in}{1.362845in}}{\pgfqpoint{5.453887in}{1.357801in}}%
\pgfpathcurveto{\pgfqpoint{5.458931in}{1.352757in}}{\pgfqpoint{5.465772in}{1.349924in}}{\pgfqpoint{5.472905in}{1.349924in}}%
\pgfpathclose%
\pgfusepath{stroke,fill}%
\end{pgfscope}%
\begin{pgfscope}%
\pgfpathrectangle{\pgfqpoint{4.985294in}{0.500000in}}{\pgfqpoint{1.764706in}{1.700000in}}%
\pgfusepath{clip}%
\pgfsetbuttcap%
\pgfsetroundjoin%
\definecolor{currentfill}{rgb}{0.968931,0.798091,0.685123}%
\pgfsetfillcolor{currentfill}%
\pgfsetlinewidth{0.311001pt}%
\definecolor{currentstroke}{rgb}{1.000000,1.000000,1.000000}%
\pgfsetstrokecolor{currentstroke}%
\pgfsetdash{}{0pt}%
\pgfpathmoveto{\pgfqpoint{6.207638in}{1.499033in}}%
\pgfpathcurveto{\pgfqpoint{6.214771in}{1.499033in}}{\pgfqpoint{6.221612in}{1.501867in}}{\pgfqpoint{6.226656in}{1.506911in}}%
\pgfpathcurveto{\pgfqpoint{6.231700in}{1.511954in}}{\pgfqpoint{6.234534in}{1.518796in}}{\pgfqpoint{6.234534in}{1.525929in}}%
\pgfpathcurveto{\pgfqpoint{6.234534in}{1.533062in}}{\pgfqpoint{6.231700in}{1.539903in}}{\pgfqpoint{6.226656in}{1.544947in}}%
\pgfpathcurveto{\pgfqpoint{6.221612in}{1.549991in}}{\pgfqpoint{6.214771in}{1.552824in}}{\pgfqpoint{6.207638in}{1.552824in}}%
\pgfpathcurveto{\pgfqpoint{6.200505in}{1.552824in}}{\pgfqpoint{6.193663in}{1.549991in}}{\pgfqpoint{6.188620in}{1.544947in}}%
\pgfpathcurveto{\pgfqpoint{6.183576in}{1.539903in}}{\pgfqpoint{6.180742in}{1.533062in}}{\pgfqpoint{6.180742in}{1.525929in}}%
\pgfpathcurveto{\pgfqpoint{6.180742in}{1.518796in}}{\pgfqpoint{6.183576in}{1.511954in}}{\pgfqpoint{6.188620in}{1.506911in}}%
\pgfpathcurveto{\pgfqpoint{6.193663in}{1.501867in}}{\pgfqpoint{6.200505in}{1.499033in}}{\pgfqpoint{6.207638in}{1.499033in}}%
\pgfpathclose%
\pgfusepath{stroke,fill}%
\end{pgfscope}%
\begin{pgfscope}%
\pgfpathrectangle{\pgfqpoint{4.985294in}{0.500000in}}{\pgfqpoint{1.764706in}{1.700000in}}%
\pgfusepath{clip}%
\pgfsetbuttcap%
\pgfsetroundjoin%
\definecolor{currentfill}{rgb}{0.973832,0.856556,0.771584}%
\pgfsetfillcolor{currentfill}%
\pgfsetlinewidth{0.311001pt}%
\definecolor{currentstroke}{rgb}{1.000000,1.000000,1.000000}%
\pgfsetstrokecolor{currentstroke}%
\pgfsetdash{}{0pt}%
\pgfpathmoveto{\pgfqpoint{6.364224in}{1.232774in}}%
\pgfpathcurveto{\pgfqpoint{6.371357in}{1.232774in}}{\pgfqpoint{6.378199in}{1.235608in}}{\pgfqpoint{6.383242in}{1.240652in}}%
\pgfpathcurveto{\pgfqpoint{6.388286in}{1.245695in}}{\pgfqpoint{6.391120in}{1.252537in}}{\pgfqpoint{6.391120in}{1.259670in}}%
\pgfpathcurveto{\pgfqpoint{6.391120in}{1.266803in}}{\pgfqpoint{6.388286in}{1.273644in}}{\pgfqpoint{6.383242in}{1.278688in}}%
\pgfpathcurveto{\pgfqpoint{6.378199in}{1.283732in}}{\pgfqpoint{6.371357in}{1.286565in}}{\pgfqpoint{6.364224in}{1.286565in}}%
\pgfpathcurveto{\pgfqpoint{6.357091in}{1.286565in}}{\pgfqpoint{6.350250in}{1.283732in}}{\pgfqpoint{6.345206in}{1.278688in}}%
\pgfpathcurveto{\pgfqpoint{6.340162in}{1.273644in}}{\pgfqpoint{6.337329in}{1.266803in}}{\pgfqpoint{6.337329in}{1.259670in}}%
\pgfpathcurveto{\pgfqpoint{6.337329in}{1.252537in}}{\pgfqpoint{6.340162in}{1.245695in}}{\pgfqpoint{6.345206in}{1.240652in}}%
\pgfpathcurveto{\pgfqpoint{6.350250in}{1.235608in}}{\pgfqpoint{6.357091in}{1.232774in}}{\pgfqpoint{6.364224in}{1.232774in}}%
\pgfpathclose%
\pgfusepath{stroke,fill}%
\end{pgfscope}%
\begin{pgfscope}%
\pgfpathrectangle{\pgfqpoint{4.985294in}{0.500000in}}{\pgfqpoint{1.764706in}{1.700000in}}%
\pgfusepath{clip}%
\pgfsetbuttcap%
\pgfsetroundjoin%
\definecolor{currentfill}{rgb}{0.852817,0.156578,0.279098}%
\pgfsetfillcolor{currentfill}%
\pgfsetlinewidth{0.311001pt}%
\definecolor{currentstroke}{rgb}{1.000000,1.000000,1.000000}%
\pgfsetstrokecolor{currentstroke}%
\pgfsetdash{}{0pt}%
\pgfpathmoveto{\pgfqpoint{5.859838in}{1.779328in}}%
\pgfpathcurveto{\pgfqpoint{5.866970in}{1.779328in}}{\pgfqpoint{5.873812in}{1.782161in}}{\pgfqpoint{5.878856in}{1.787205in}}%
\pgfpathcurveto{\pgfqpoint{5.883899in}{1.792249in}}{\pgfqpoint{5.886733in}{1.799090in}}{\pgfqpoint{5.886733in}{1.806223in}}%
\pgfpathcurveto{\pgfqpoint{5.886733in}{1.813356in}}{\pgfqpoint{5.883899in}{1.820198in}}{\pgfqpoint{5.878856in}{1.825241in}}%
\pgfpathcurveto{\pgfqpoint{5.873812in}{1.830285in}}{\pgfqpoint{5.866970in}{1.833119in}}{\pgfqpoint{5.859838in}{1.833119in}}%
\pgfpathcurveto{\pgfqpoint{5.852705in}{1.833119in}}{\pgfqpoint{5.845863in}{1.830285in}}{\pgfqpoint{5.840819in}{1.825241in}}%
\pgfpathcurveto{\pgfqpoint{5.835776in}{1.820198in}}{\pgfqpoint{5.832942in}{1.813356in}}{\pgfqpoint{5.832942in}{1.806223in}}%
\pgfpathcurveto{\pgfqpoint{5.832942in}{1.799090in}}{\pgfqpoint{5.835776in}{1.792249in}}{\pgfqpoint{5.840819in}{1.787205in}}%
\pgfpathcurveto{\pgfqpoint{5.845863in}{1.782161in}}{\pgfqpoint{5.852705in}{1.779328in}}{\pgfqpoint{5.859838in}{1.779328in}}%
\pgfpathclose%
\pgfusepath{stroke,fill}%
\end{pgfscope}%
\begin{pgfscope}%
\pgfpathrectangle{\pgfqpoint{4.985294in}{0.500000in}}{\pgfqpoint{1.764706in}{1.700000in}}%
\pgfusepath{clip}%
\pgfsetbuttcap%
\pgfsetroundjoin%
\definecolor{currentfill}{rgb}{0.973271,0.850724,0.762998}%
\pgfsetfillcolor{currentfill}%
\pgfsetlinewidth{0.311001pt}%
\definecolor{currentstroke}{rgb}{1.000000,1.000000,1.000000}%
\pgfsetstrokecolor{currentstroke}%
\pgfsetdash{}{0pt}%
\pgfpathmoveto{\pgfqpoint{5.486415in}{1.169275in}}%
\pgfpathcurveto{\pgfqpoint{5.493548in}{1.169275in}}{\pgfqpoint{5.500389in}{1.172109in}}{\pgfqpoint{5.505433in}{1.177153in}}%
\pgfpathcurveto{\pgfqpoint{5.510477in}{1.182197in}}{\pgfqpoint{5.513311in}{1.189038in}}{\pgfqpoint{5.513311in}{1.196171in}}%
\pgfpathcurveto{\pgfqpoint{5.513311in}{1.203304in}}{\pgfqpoint{5.510477in}{1.210146in}}{\pgfqpoint{5.505433in}{1.215189in}}%
\pgfpathcurveto{\pgfqpoint{5.500389in}{1.220233in}}{\pgfqpoint{5.493548in}{1.223067in}}{\pgfqpoint{5.486415in}{1.223067in}}%
\pgfpathcurveto{\pgfqpoint{5.479282in}{1.223067in}}{\pgfqpoint{5.472440in}{1.220233in}}{\pgfqpoint{5.467397in}{1.215189in}}%
\pgfpathcurveto{\pgfqpoint{5.462353in}{1.210146in}}{\pgfqpoint{5.459519in}{1.203304in}}{\pgfqpoint{5.459519in}{1.196171in}}%
\pgfpathcurveto{\pgfqpoint{5.459519in}{1.189038in}}{\pgfqpoint{5.462353in}{1.182197in}}{\pgfqpoint{5.467397in}{1.177153in}}%
\pgfpathcurveto{\pgfqpoint{5.472440in}{1.172109in}}{\pgfqpoint{5.479282in}{1.169275in}}{\pgfqpoint{5.486415in}{1.169275in}}%
\pgfpathclose%
\pgfusepath{stroke,fill}%
\end{pgfscope}%
\begin{pgfscope}%
\pgfpathrectangle{\pgfqpoint{4.985294in}{0.500000in}}{\pgfqpoint{1.764706in}{1.700000in}}%
\pgfusepath{clip}%
\pgfsetbuttcap%
\pgfsetroundjoin%
\definecolor{currentfill}{rgb}{0.975644,0.874038,0.797253}%
\pgfsetfillcolor{currentfill}%
\pgfsetlinewidth{0.311001pt}%
\definecolor{currentstroke}{rgb}{1.000000,1.000000,1.000000}%
\pgfsetstrokecolor{currentstroke}%
\pgfsetdash{}{0pt}%
\pgfpathmoveto{\pgfqpoint{6.300152in}{1.112229in}}%
\pgfpathcurveto{\pgfqpoint{6.307284in}{1.112229in}}{\pgfqpoint{6.314126in}{1.115063in}}{\pgfqpoint{6.319170in}{1.120107in}}%
\pgfpathcurveto{\pgfqpoint{6.324213in}{1.125150in}}{\pgfqpoint{6.327047in}{1.131992in}}{\pgfqpoint{6.327047in}{1.139125in}}%
\pgfpathcurveto{\pgfqpoint{6.327047in}{1.146258in}}{\pgfqpoint{6.324213in}{1.153099in}}{\pgfqpoint{6.319170in}{1.158143in}}%
\pgfpathcurveto{\pgfqpoint{6.314126in}{1.163187in}}{\pgfqpoint{6.307284in}{1.166021in}}{\pgfqpoint{6.300152in}{1.166021in}}%
\pgfpathcurveto{\pgfqpoint{6.293019in}{1.166021in}}{\pgfqpoint{6.286177in}{1.163187in}}{\pgfqpoint{6.281134in}{1.158143in}}%
\pgfpathcurveto{\pgfqpoint{6.276090in}{1.153099in}}{\pgfqpoint{6.273256in}{1.146258in}}{\pgfqpoint{6.273256in}{1.139125in}}%
\pgfpathcurveto{\pgfqpoint{6.273256in}{1.131992in}}{\pgfqpoint{6.276090in}{1.125150in}}{\pgfqpoint{6.281134in}{1.120107in}}%
\pgfpathcurveto{\pgfqpoint{6.286177in}{1.115063in}}{\pgfqpoint{6.293019in}{1.112229in}}{\pgfqpoint{6.300152in}{1.112229in}}%
\pgfpathclose%
\pgfusepath{stroke,fill}%
\end{pgfscope}%
\begin{pgfscope}%
\pgfpathrectangle{\pgfqpoint{4.985294in}{0.500000in}}{\pgfqpoint{1.764706in}{1.700000in}}%
\pgfusepath{clip}%
\pgfsetbuttcap%
\pgfsetroundjoin%
\definecolor{currentfill}{rgb}{0.972726,0.844889,0.754401}%
\pgfsetfillcolor{currentfill}%
\pgfsetlinewidth{0.311001pt}%
\definecolor{currentstroke}{rgb}{1.000000,1.000000,1.000000}%
\pgfsetstrokecolor{currentstroke}%
\pgfsetdash{}{0pt}%
\pgfpathmoveto{\pgfqpoint{5.488736in}{1.422540in}}%
\pgfpathcurveto{\pgfqpoint{5.495868in}{1.422540in}}{\pgfqpoint{5.502710in}{1.425374in}}{\pgfqpoint{5.507754in}{1.430418in}}%
\pgfpathcurveto{\pgfqpoint{5.512797in}{1.435462in}}{\pgfqpoint{5.515631in}{1.442303in}}{\pgfqpoint{5.515631in}{1.449436in}}%
\pgfpathcurveto{\pgfqpoint{5.515631in}{1.456569in}}{\pgfqpoint{5.512797in}{1.463411in}}{\pgfqpoint{5.507754in}{1.468454in}}%
\pgfpathcurveto{\pgfqpoint{5.502710in}{1.473498in}}{\pgfqpoint{5.495868in}{1.476332in}}{\pgfqpoint{5.488736in}{1.476332in}}%
\pgfpathcurveto{\pgfqpoint{5.481603in}{1.476332in}}{\pgfqpoint{5.474761in}{1.473498in}}{\pgfqpoint{5.469717in}{1.468454in}}%
\pgfpathcurveto{\pgfqpoint{5.464674in}{1.463411in}}{\pgfqpoint{5.461840in}{1.456569in}}{\pgfqpoint{5.461840in}{1.449436in}}%
\pgfpathcurveto{\pgfqpoint{5.461840in}{1.442303in}}{\pgfqpoint{5.464674in}{1.435462in}}{\pgfqpoint{5.469717in}{1.430418in}}%
\pgfpathcurveto{\pgfqpoint{5.474761in}{1.425374in}}{\pgfqpoint{5.481603in}{1.422540in}}{\pgfqpoint{5.488736in}{1.422540in}}%
\pgfpathclose%
\pgfusepath{stroke,fill}%
\end{pgfscope}%
\begin{pgfscope}%
\pgfpathrectangle{\pgfqpoint{4.985294in}{0.500000in}}{\pgfqpoint{1.764706in}{1.700000in}}%
\pgfusepath{clip}%
\pgfsetbuttcap%
\pgfsetroundjoin%
\definecolor{currentfill}{rgb}{0.964032,0.651225,0.493258}%
\pgfsetfillcolor{currentfill}%
\pgfsetlinewidth{0.311001pt}%
\definecolor{currentstroke}{rgb}{1.000000,1.000000,1.000000}%
\pgfsetstrokecolor{currentstroke}%
\pgfsetdash{}{0pt}%
\pgfpathmoveto{\pgfqpoint{5.361444in}{1.556103in}}%
\pgfpathcurveto{\pgfqpoint{5.368577in}{1.556103in}}{\pgfqpoint{5.375419in}{1.558937in}}{\pgfqpoint{5.380462in}{1.563981in}}%
\pgfpathcurveto{\pgfqpoint{5.385506in}{1.569024in}}{\pgfqpoint{5.388340in}{1.575866in}}{\pgfqpoint{5.388340in}{1.582999in}}%
\pgfpathcurveto{\pgfqpoint{5.388340in}{1.590132in}}{\pgfqpoint{5.385506in}{1.596973in}}{\pgfqpoint{5.380462in}{1.602017in}}%
\pgfpathcurveto{\pgfqpoint{5.375419in}{1.607061in}}{\pgfqpoint{5.368577in}{1.609895in}}{\pgfqpoint{5.361444in}{1.609895in}}%
\pgfpathcurveto{\pgfqpoint{5.354311in}{1.609895in}}{\pgfqpoint{5.347470in}{1.607061in}}{\pgfqpoint{5.342426in}{1.602017in}}%
\pgfpathcurveto{\pgfqpoint{5.337382in}{1.596973in}}{\pgfqpoint{5.334548in}{1.590132in}}{\pgfqpoint{5.334548in}{1.582999in}}%
\pgfpathcurveto{\pgfqpoint{5.334548in}{1.575866in}}{\pgfqpoint{5.337382in}{1.569024in}}{\pgfqpoint{5.342426in}{1.563981in}}%
\pgfpathcurveto{\pgfqpoint{5.347470in}{1.558937in}}{\pgfqpoint{5.354311in}{1.556103in}}{\pgfqpoint{5.361444in}{1.556103in}}%
\pgfpathclose%
\pgfusepath{stroke,fill}%
\end{pgfscope}%
\begin{pgfscope}%
\pgfpathrectangle{\pgfqpoint{4.985294in}{0.500000in}}{\pgfqpoint{1.764706in}{1.700000in}}%
\pgfusepath{clip}%
\pgfsetbuttcap%
\pgfsetroundjoin%
\definecolor{currentfill}{rgb}{0.965302,0.713942,0.568499}%
\pgfsetfillcolor{currentfill}%
\pgfsetlinewidth{0.311001pt}%
\definecolor{currentstroke}{rgb}{1.000000,1.000000,1.000000}%
\pgfsetstrokecolor{currentstroke}%
\pgfsetdash{}{0pt}%
\pgfpathmoveto{\pgfqpoint{5.551585in}{1.492722in}}%
\pgfpathcurveto{\pgfqpoint{5.558718in}{1.492722in}}{\pgfqpoint{5.565559in}{1.495556in}}{\pgfqpoint{5.570603in}{1.500599in}}%
\pgfpathcurveto{\pgfqpoint{5.575647in}{1.505643in}}{\pgfqpoint{5.578481in}{1.512485in}}{\pgfqpoint{5.578481in}{1.519617in}}%
\pgfpathcurveto{\pgfqpoint{5.578481in}{1.526750in}}{\pgfqpoint{5.575647in}{1.533592in}}{\pgfqpoint{5.570603in}{1.538636in}}%
\pgfpathcurveto{\pgfqpoint{5.565559in}{1.543679in}}{\pgfqpoint{5.558718in}{1.546513in}}{\pgfqpoint{5.551585in}{1.546513in}}%
\pgfpathcurveto{\pgfqpoint{5.544452in}{1.546513in}}{\pgfqpoint{5.537610in}{1.543679in}}{\pgfqpoint{5.532567in}{1.538636in}}%
\pgfpathcurveto{\pgfqpoint{5.527523in}{1.533592in}}{\pgfqpoint{5.524689in}{1.526750in}}{\pgfqpoint{5.524689in}{1.519617in}}%
\pgfpathcurveto{\pgfqpoint{5.524689in}{1.512485in}}{\pgfqpoint{5.527523in}{1.505643in}}{\pgfqpoint{5.532567in}{1.500599in}}%
\pgfpathcurveto{\pgfqpoint{5.537610in}{1.495556in}}{\pgfqpoint{5.544452in}{1.492722in}}{\pgfqpoint{5.551585in}{1.492722in}}%
\pgfpathclose%
\pgfusepath{stroke,fill}%
\end{pgfscope}%
\begin{pgfscope}%
\pgfpathrectangle{\pgfqpoint{4.985294in}{0.500000in}}{\pgfqpoint{1.764706in}{1.700000in}}%
\pgfusepath{clip}%
\pgfsetbuttcap%
\pgfsetroundjoin%
\definecolor{currentfill}{rgb}{0.966560,0.756582,0.625273}%
\pgfsetfillcolor{currentfill}%
\pgfsetlinewidth{0.311001pt}%
\definecolor{currentstroke}{rgb}{1.000000,1.000000,1.000000}%
\pgfsetstrokecolor{currentstroke}%
\pgfsetdash{}{0pt}%
\pgfpathmoveto{\pgfqpoint{6.151572in}{1.601598in}}%
\pgfpathcurveto{\pgfqpoint{6.158705in}{1.601598in}}{\pgfqpoint{6.165547in}{1.604432in}}{\pgfqpoint{6.170590in}{1.609475in}}%
\pgfpathcurveto{\pgfqpoint{6.175634in}{1.614519in}}{\pgfqpoint{6.178468in}{1.621361in}}{\pgfqpoint{6.178468in}{1.628493in}}%
\pgfpathcurveto{\pgfqpoint{6.178468in}{1.635626in}}{\pgfqpoint{6.175634in}{1.642468in}}{\pgfqpoint{6.170590in}{1.647512in}}%
\pgfpathcurveto{\pgfqpoint{6.165547in}{1.652555in}}{\pgfqpoint{6.158705in}{1.655389in}}{\pgfqpoint{6.151572in}{1.655389in}}%
\pgfpathcurveto{\pgfqpoint{6.144439in}{1.655389in}}{\pgfqpoint{6.137598in}{1.652555in}}{\pgfqpoint{6.132554in}{1.647512in}}%
\pgfpathcurveto{\pgfqpoint{6.127510in}{1.642468in}}{\pgfqpoint{6.124677in}{1.635626in}}{\pgfqpoint{6.124677in}{1.628493in}}%
\pgfpathcurveto{\pgfqpoint{6.124677in}{1.621361in}}{\pgfqpoint{6.127510in}{1.614519in}}{\pgfqpoint{6.132554in}{1.609475in}}%
\pgfpathcurveto{\pgfqpoint{6.137598in}{1.604432in}}{\pgfqpoint{6.144439in}{1.601598in}}{\pgfqpoint{6.151572in}{1.601598in}}%
\pgfpathclose%
\pgfusepath{stroke,fill}%
\end{pgfscope}%
\begin{pgfscope}%
\pgfpathrectangle{\pgfqpoint{4.985294in}{0.500000in}}{\pgfqpoint{1.764706in}{1.700000in}}%
\pgfusepath{clip}%
\pgfsetbuttcap%
\pgfsetroundjoin%
\definecolor{currentfill}{rgb}{0.968509,0.792226,0.676405}%
\pgfsetfillcolor{currentfill}%
\pgfsetlinewidth{0.311001pt}%
\definecolor{currentstroke}{rgb}{1.000000,1.000000,1.000000}%
\pgfsetstrokecolor{currentstroke}%
\pgfsetdash{}{0pt}%
\pgfpathmoveto{\pgfqpoint{5.499721in}{1.339995in}}%
\pgfpathcurveto{\pgfqpoint{5.506854in}{1.339995in}}{\pgfqpoint{5.513696in}{1.342829in}}{\pgfqpoint{5.518739in}{1.347873in}}%
\pgfpathcurveto{\pgfqpoint{5.523783in}{1.352916in}}{\pgfqpoint{5.526617in}{1.359758in}}{\pgfqpoint{5.526617in}{1.366891in}}%
\pgfpathcurveto{\pgfqpoint{5.526617in}{1.374024in}}{\pgfqpoint{5.523783in}{1.380865in}}{\pgfqpoint{5.518739in}{1.385909in}}%
\pgfpathcurveto{\pgfqpoint{5.513696in}{1.390953in}}{\pgfqpoint{5.506854in}{1.393787in}}{\pgfqpoint{5.499721in}{1.393787in}}%
\pgfpathcurveto{\pgfqpoint{5.492588in}{1.393787in}}{\pgfqpoint{5.485747in}{1.390953in}}{\pgfqpoint{5.480703in}{1.385909in}}%
\pgfpathcurveto{\pgfqpoint{5.475659in}{1.380865in}}{\pgfqpoint{5.472826in}{1.374024in}}{\pgfqpoint{5.472826in}{1.366891in}}%
\pgfpathcurveto{\pgfqpoint{5.472826in}{1.359758in}}{\pgfqpoint{5.475659in}{1.352916in}}{\pgfqpoint{5.480703in}{1.347873in}}%
\pgfpathcurveto{\pgfqpoint{5.485747in}{1.342829in}}{\pgfqpoint{5.492588in}{1.339995in}}{\pgfqpoint{5.499721in}{1.339995in}}%
\pgfpathclose%
\pgfusepath{stroke,fill}%
\end{pgfscope}%
\begin{pgfscope}%
\pgfpathrectangle{\pgfqpoint{4.985294in}{0.500000in}}{\pgfqpoint{1.764706in}{1.700000in}}%
\pgfusepath{clip}%
\pgfsetbuttcap%
\pgfsetroundjoin%
\definecolor{currentfill}{rgb}{0.974412,0.862387,0.780156}%
\pgfsetfillcolor{currentfill}%
\pgfsetlinewidth{0.311001pt}%
\definecolor{currentstroke}{rgb}{1.000000,1.000000,1.000000}%
\pgfsetstrokecolor{currentstroke}%
\pgfsetdash{}{0pt}%
\pgfpathmoveto{\pgfqpoint{5.479266in}{1.164400in}}%
\pgfpathcurveto{\pgfqpoint{5.486399in}{1.164400in}}{\pgfqpoint{5.493240in}{1.167234in}}{\pgfqpoint{5.498284in}{1.172277in}}%
\pgfpathcurveto{\pgfqpoint{5.503328in}{1.177321in}}{\pgfqpoint{5.506162in}{1.184163in}}{\pgfqpoint{5.506162in}{1.191296in}}%
\pgfpathcurveto{\pgfqpoint{5.506162in}{1.198428in}}{\pgfqpoint{5.503328in}{1.205270in}}{\pgfqpoint{5.498284in}{1.210314in}}%
\pgfpathcurveto{\pgfqpoint{5.493240in}{1.215357in}}{\pgfqpoint{5.486399in}{1.218191in}}{\pgfqpoint{5.479266in}{1.218191in}}%
\pgfpathcurveto{\pgfqpoint{5.472133in}{1.218191in}}{\pgfqpoint{5.465292in}{1.215357in}}{\pgfqpoint{5.460248in}{1.210314in}}%
\pgfpathcurveto{\pgfqpoint{5.455204in}{1.205270in}}{\pgfqpoint{5.452370in}{1.198428in}}{\pgfqpoint{5.452370in}{1.191296in}}%
\pgfpathcurveto{\pgfqpoint{5.452370in}{1.184163in}}{\pgfqpoint{5.455204in}{1.177321in}}{\pgfqpoint{5.460248in}{1.172277in}}%
\pgfpathcurveto{\pgfqpoint{5.465292in}{1.167234in}}{\pgfqpoint{5.472133in}{1.164400in}}{\pgfqpoint{5.479266in}{1.164400in}}%
\pgfpathclose%
\pgfusepath{stroke,fill}%
\end{pgfscope}%
\begin{pgfscope}%
\pgfpathrectangle{\pgfqpoint{4.985294in}{0.500000in}}{\pgfqpoint{1.764706in}{1.700000in}}%
\pgfusepath{clip}%
\pgfsetbuttcap%
\pgfsetroundjoin%
\definecolor{currentfill}{rgb}{0.967092,0.768560,0.642079}%
\pgfsetfillcolor{currentfill}%
\pgfsetlinewidth{0.311001pt}%
\definecolor{currentstroke}{rgb}{1.000000,1.000000,1.000000}%
\pgfsetstrokecolor{currentstroke}%
\pgfsetdash{}{0pt}%
\pgfpathmoveto{\pgfqpoint{5.511570in}{0.935507in}}%
\pgfpathcurveto{\pgfqpoint{5.518703in}{0.935507in}}{\pgfqpoint{5.525544in}{0.938341in}}{\pgfqpoint{5.530588in}{0.943384in}}%
\pgfpathcurveto{\pgfqpoint{5.535632in}{0.948428in}}{\pgfqpoint{5.538466in}{0.955270in}}{\pgfqpoint{5.538466in}{0.962402in}}%
\pgfpathcurveto{\pgfqpoint{5.538466in}{0.969535in}}{\pgfqpoint{5.535632in}{0.976377in}}{\pgfqpoint{5.530588in}{0.981421in}}%
\pgfpathcurveto{\pgfqpoint{5.525544in}{0.986464in}}{\pgfqpoint{5.518703in}{0.989298in}}{\pgfqpoint{5.511570in}{0.989298in}}%
\pgfpathcurveto{\pgfqpoint{5.504437in}{0.989298in}}{\pgfqpoint{5.497595in}{0.986464in}}{\pgfqpoint{5.492552in}{0.981421in}}%
\pgfpathcurveto{\pgfqpoint{5.487508in}{0.976377in}}{\pgfqpoint{5.484674in}{0.969535in}}{\pgfqpoint{5.484674in}{0.962402in}}%
\pgfpathcurveto{\pgfqpoint{5.484674in}{0.955270in}}{\pgfqpoint{5.487508in}{0.948428in}}{\pgfqpoint{5.492552in}{0.943384in}}%
\pgfpathcurveto{\pgfqpoint{5.497595in}{0.938341in}}{\pgfqpoint{5.504437in}{0.935507in}}{\pgfqpoint{5.511570in}{0.935507in}}%
\pgfpathclose%
\pgfusepath{stroke,fill}%
\end{pgfscope}%
\begin{pgfscope}%
\pgfpathrectangle{\pgfqpoint{4.985294in}{0.500000in}}{\pgfqpoint{1.764706in}{1.700000in}}%
\pgfusepath{clip}%
\pgfsetbuttcap%
\pgfsetroundjoin%
\definecolor{currentfill}{rgb}{0.972201,0.839051,0.745789}%
\pgfsetfillcolor{currentfill}%
\pgfsetlinewidth{0.311001pt}%
\definecolor{currentstroke}{rgb}{1.000000,1.000000,1.000000}%
\pgfsetstrokecolor{currentstroke}%
\pgfsetdash{}{0pt}%
\pgfpathmoveto{\pgfqpoint{6.244263in}{1.255375in}}%
\pgfpathcurveto{\pgfqpoint{6.251396in}{1.255375in}}{\pgfqpoint{6.258238in}{1.258209in}}{\pgfqpoint{6.263281in}{1.263253in}}%
\pgfpathcurveto{\pgfqpoint{6.268325in}{1.268297in}}{\pgfqpoint{6.271159in}{1.275138in}}{\pgfqpoint{6.271159in}{1.282271in}}%
\pgfpathcurveto{\pgfqpoint{6.271159in}{1.289404in}}{\pgfqpoint{6.268325in}{1.296246in}}{\pgfqpoint{6.263281in}{1.301289in}}%
\pgfpathcurveto{\pgfqpoint{6.258238in}{1.306333in}}{\pgfqpoint{6.251396in}{1.309167in}}{\pgfqpoint{6.244263in}{1.309167in}}%
\pgfpathcurveto{\pgfqpoint{6.237130in}{1.309167in}}{\pgfqpoint{6.230289in}{1.306333in}}{\pgfqpoint{6.225245in}{1.301289in}}%
\pgfpathcurveto{\pgfqpoint{6.220201in}{1.296246in}}{\pgfqpoint{6.217368in}{1.289404in}}{\pgfqpoint{6.217368in}{1.282271in}}%
\pgfpathcurveto{\pgfqpoint{6.217368in}{1.275138in}}{\pgfqpoint{6.220201in}{1.268297in}}{\pgfqpoint{6.225245in}{1.263253in}}%
\pgfpathcurveto{\pgfqpoint{6.230289in}{1.258209in}}{\pgfqpoint{6.237130in}{1.255375in}}{\pgfqpoint{6.244263in}{1.255375in}}%
\pgfpathclose%
\pgfusepath{stroke,fill}%
\end{pgfscope}%
\begin{pgfscope}%
\pgfpathrectangle{\pgfqpoint{4.985294in}{0.500000in}}{\pgfqpoint{1.764706in}{1.700000in}}%
\pgfusepath{clip}%
\pgfsetbuttcap%
\pgfsetroundjoin%
\definecolor{currentfill}{rgb}{0.962283,0.593046,0.431453}%
\pgfsetfillcolor{currentfill}%
\pgfsetlinewidth{0.311001pt}%
\definecolor{currentstroke}{rgb}{1.000000,1.000000,1.000000}%
\pgfsetstrokecolor{currentstroke}%
\pgfsetdash{}{0pt}%
\pgfpathmoveto{\pgfqpoint{6.392188in}{1.548799in}}%
\pgfpathcurveto{\pgfqpoint{6.399321in}{1.548799in}}{\pgfqpoint{6.406163in}{1.551633in}}{\pgfqpoint{6.411207in}{1.556676in}}%
\pgfpathcurveto{\pgfqpoint{6.416250in}{1.561720in}}{\pgfqpoint{6.419084in}{1.568562in}}{\pgfqpoint{6.419084in}{1.575694in}}%
\pgfpathcurveto{\pgfqpoint{6.419084in}{1.582827in}}{\pgfqpoint{6.416250in}{1.589669in}}{\pgfqpoint{6.411207in}{1.594713in}}%
\pgfpathcurveto{\pgfqpoint{6.406163in}{1.599756in}}{\pgfqpoint{6.399321in}{1.602590in}}{\pgfqpoint{6.392188in}{1.602590in}}%
\pgfpathcurveto{\pgfqpoint{6.385056in}{1.602590in}}{\pgfqpoint{6.378214in}{1.599756in}}{\pgfqpoint{6.373170in}{1.594713in}}%
\pgfpathcurveto{\pgfqpoint{6.368127in}{1.589669in}}{\pgfqpoint{6.365293in}{1.582827in}}{\pgfqpoint{6.365293in}{1.575694in}}%
\pgfpathcurveto{\pgfqpoint{6.365293in}{1.568562in}}{\pgfqpoint{6.368127in}{1.561720in}}{\pgfqpoint{6.373170in}{1.556676in}}%
\pgfpathcurveto{\pgfqpoint{6.378214in}{1.551633in}}{\pgfqpoint{6.385056in}{1.548799in}}{\pgfqpoint{6.392188in}{1.548799in}}%
\pgfpathclose%
\pgfusepath{stroke,fill}%
\end{pgfscope}%
\begin{pgfscope}%
\pgfpathrectangle{\pgfqpoint{4.985294in}{0.500000in}}{\pgfqpoint{1.764706in}{1.700000in}}%
\pgfusepath{clip}%
\pgfsetbuttcap%
\pgfsetroundjoin%
\definecolor{currentfill}{rgb}{0.962532,0.599594,0.438051}%
\pgfsetfillcolor{currentfill}%
\pgfsetlinewidth{0.311001pt}%
\definecolor{currentstroke}{rgb}{1.000000,1.000000,1.000000}%
\pgfsetstrokecolor{currentstroke}%
\pgfsetdash{}{0pt}%
\pgfpathmoveto{\pgfqpoint{6.141798in}{1.764824in}}%
\pgfpathcurveto{\pgfqpoint{6.148930in}{1.764824in}}{\pgfqpoint{6.155772in}{1.767658in}}{\pgfqpoint{6.160816in}{1.772702in}}%
\pgfpathcurveto{\pgfqpoint{6.165859in}{1.777746in}}{\pgfqpoint{6.168693in}{1.784587in}}{\pgfqpoint{6.168693in}{1.791720in}}%
\pgfpathcurveto{\pgfqpoint{6.168693in}{1.798853in}}{\pgfqpoint{6.165859in}{1.805694in}}{\pgfqpoint{6.160816in}{1.810738in}}%
\pgfpathcurveto{\pgfqpoint{6.155772in}{1.815782in}}{\pgfqpoint{6.148930in}{1.818616in}}{\pgfqpoint{6.141798in}{1.818616in}}%
\pgfpathcurveto{\pgfqpoint{6.134665in}{1.818616in}}{\pgfqpoint{6.127823in}{1.815782in}}{\pgfqpoint{6.122779in}{1.810738in}}%
\pgfpathcurveto{\pgfqpoint{6.117736in}{1.805694in}}{\pgfqpoint{6.114902in}{1.798853in}}{\pgfqpoint{6.114902in}{1.791720in}}%
\pgfpathcurveto{\pgfqpoint{6.114902in}{1.784587in}}{\pgfqpoint{6.117736in}{1.777746in}}{\pgfqpoint{6.122779in}{1.772702in}}%
\pgfpathcurveto{\pgfqpoint{6.127823in}{1.767658in}}{\pgfqpoint{6.134665in}{1.764824in}}{\pgfqpoint{6.141798in}{1.764824in}}%
\pgfpathclose%
\pgfusepath{stroke,fill}%
\end{pgfscope}%
\begin{pgfscope}%
\pgfpathrectangle{\pgfqpoint{4.985294in}{0.500000in}}{\pgfqpoint{1.764706in}{1.700000in}}%
\pgfusepath{clip}%
\pgfsetbuttcap%
\pgfsetroundjoin%
\definecolor{currentfill}{rgb}{0.980678,0.914765,0.856766}%
\pgfsetfillcolor{currentfill}%
\pgfsetlinewidth{0.311001pt}%
\definecolor{currentstroke}{rgb}{1.000000,1.000000,1.000000}%
\pgfsetstrokecolor{currentstroke}%
\pgfsetdash{}{0pt}%
\pgfpathmoveto{\pgfqpoint{6.302093in}{1.506221in}}%
\pgfpathcurveto{\pgfqpoint{6.309226in}{1.506221in}}{\pgfqpoint{6.316068in}{1.509055in}}{\pgfqpoint{6.321111in}{1.514099in}}%
\pgfpathcurveto{\pgfqpoint{6.326155in}{1.519142in}}{\pgfqpoint{6.328989in}{1.525984in}}{\pgfqpoint{6.328989in}{1.533117in}}%
\pgfpathcurveto{\pgfqpoint{6.328989in}{1.540250in}}{\pgfqpoint{6.326155in}{1.547091in}}{\pgfqpoint{6.321111in}{1.552135in}}%
\pgfpathcurveto{\pgfqpoint{6.316068in}{1.557179in}}{\pgfqpoint{6.309226in}{1.560013in}}{\pgfqpoint{6.302093in}{1.560013in}}%
\pgfpathcurveto{\pgfqpoint{6.294961in}{1.560013in}}{\pgfqpoint{6.288119in}{1.557179in}}{\pgfqpoint{6.283075in}{1.552135in}}%
\pgfpathcurveto{\pgfqpoint{6.278032in}{1.547091in}}{\pgfqpoint{6.275198in}{1.540250in}}{\pgfqpoint{6.275198in}{1.533117in}}%
\pgfpathcurveto{\pgfqpoint{6.275198in}{1.525984in}}{\pgfqpoint{6.278032in}{1.519142in}}{\pgfqpoint{6.283075in}{1.514099in}}%
\pgfpathcurveto{\pgfqpoint{6.288119in}{1.509055in}}{\pgfqpoint{6.294961in}{1.506221in}}{\pgfqpoint{6.302093in}{1.506221in}}%
\pgfpathclose%
\pgfusepath{stroke,fill}%
\end{pgfscope}%
\begin{pgfscope}%
\pgfpathrectangle{\pgfqpoint{4.985294in}{0.500000in}}{\pgfqpoint{1.764706in}{1.700000in}}%
\pgfusepath{clip}%
\pgfsetbuttcap%
\pgfsetroundjoin%
\definecolor{currentfill}{rgb}{0.978376,0.897317,0.831308}%
\pgfsetfillcolor{currentfill}%
\pgfsetlinewidth{0.311001pt}%
\definecolor{currentstroke}{rgb}{1.000000,1.000000,1.000000}%
\pgfsetstrokecolor{currentstroke}%
\pgfsetdash{}{0pt}%
\pgfpathmoveto{\pgfqpoint{6.329894in}{1.450162in}}%
\pgfpathcurveto{\pgfqpoint{6.337027in}{1.450162in}}{\pgfqpoint{6.343869in}{1.452996in}}{\pgfqpoint{6.348913in}{1.458040in}}%
\pgfpathcurveto{\pgfqpoint{6.353956in}{1.463084in}}{\pgfqpoint{6.356790in}{1.469925in}}{\pgfqpoint{6.356790in}{1.477058in}}%
\pgfpathcurveto{\pgfqpoint{6.356790in}{1.484191in}}{\pgfqpoint{6.353956in}{1.491033in}}{\pgfqpoint{6.348913in}{1.496076in}}%
\pgfpathcurveto{\pgfqpoint{6.343869in}{1.501120in}}{\pgfqpoint{6.337027in}{1.503954in}}{\pgfqpoint{6.329894in}{1.503954in}}%
\pgfpathcurveto{\pgfqpoint{6.322762in}{1.503954in}}{\pgfqpoint{6.315920in}{1.501120in}}{\pgfqpoint{6.310876in}{1.496076in}}%
\pgfpathcurveto{\pgfqpoint{6.305833in}{1.491033in}}{\pgfqpoint{6.302999in}{1.484191in}}{\pgfqpoint{6.302999in}{1.477058in}}%
\pgfpathcurveto{\pgfqpoint{6.302999in}{1.469925in}}{\pgfqpoint{6.305833in}{1.463084in}}{\pgfqpoint{6.310876in}{1.458040in}}%
\pgfpathcurveto{\pgfqpoint{6.315920in}{1.452996in}}{\pgfqpoint{6.322762in}{1.450162in}}{\pgfqpoint{6.329894in}{1.450162in}}%
\pgfpathclose%
\pgfusepath{stroke,fill}%
\end{pgfscope}%
\begin{pgfscope}%
\pgfpathrectangle{\pgfqpoint{4.985294in}{0.500000in}}{\pgfqpoint{1.764706in}{1.700000in}}%
\pgfusepath{clip}%
\pgfsetbuttcap%
\pgfsetroundjoin%
\definecolor{currentfill}{rgb}{0.941676,0.367866,0.260395}%
\pgfsetfillcolor{currentfill}%
\pgfsetlinewidth{0.311001pt}%
\definecolor{currentstroke}{rgb}{1.000000,1.000000,1.000000}%
\pgfsetstrokecolor{currentstroke}%
\pgfsetdash{}{0pt}%
\pgfpathmoveto{\pgfqpoint{6.461669in}{1.270907in}}%
\pgfpathcurveto{\pgfqpoint{6.468802in}{1.270907in}}{\pgfqpoint{6.475644in}{1.273741in}}{\pgfqpoint{6.480687in}{1.278785in}}%
\pgfpathcurveto{\pgfqpoint{6.485731in}{1.283828in}}{\pgfqpoint{6.488565in}{1.290670in}}{\pgfqpoint{6.488565in}{1.297803in}}%
\pgfpathcurveto{\pgfqpoint{6.488565in}{1.304936in}}{\pgfqpoint{6.485731in}{1.311777in}}{\pgfqpoint{6.480687in}{1.316821in}}%
\pgfpathcurveto{\pgfqpoint{6.475644in}{1.321865in}}{\pgfqpoint{6.468802in}{1.324698in}}{\pgfqpoint{6.461669in}{1.324698in}}%
\pgfpathcurveto{\pgfqpoint{6.454536in}{1.324698in}}{\pgfqpoint{6.447695in}{1.321865in}}{\pgfqpoint{6.442651in}{1.316821in}}%
\pgfpathcurveto{\pgfqpoint{6.437607in}{1.311777in}}{\pgfqpoint{6.434773in}{1.304936in}}{\pgfqpoint{6.434773in}{1.297803in}}%
\pgfpathcurveto{\pgfqpoint{6.434773in}{1.290670in}}{\pgfqpoint{6.437607in}{1.283828in}}{\pgfqpoint{6.442651in}{1.278785in}}%
\pgfpathcurveto{\pgfqpoint{6.447695in}{1.273741in}}{\pgfqpoint{6.454536in}{1.270907in}}{\pgfqpoint{6.461669in}{1.270907in}}%
\pgfpathclose%
\pgfusepath{stroke,fill}%
\end{pgfscope}%
\begin{pgfscope}%
\pgfpathrectangle{\pgfqpoint{4.985294in}{0.500000in}}{\pgfqpoint{1.764706in}{1.700000in}}%
\pgfusepath{clip}%
\pgfsetbuttcap%
\pgfsetroundjoin%
\definecolor{currentfill}{rgb}{0.979124,0.903132,0.839793}%
\pgfsetfillcolor{currentfill}%
\pgfsetlinewidth{0.311001pt}%
\definecolor{currentstroke}{rgb}{1.000000,1.000000,1.000000}%
\pgfsetstrokecolor{currentstroke}%
\pgfsetdash{}{0pt}%
\pgfpathmoveto{\pgfqpoint{5.439807in}{1.310889in}}%
\pgfpathcurveto{\pgfqpoint{5.446940in}{1.310889in}}{\pgfqpoint{5.453782in}{1.313723in}}{\pgfqpoint{5.458825in}{1.318767in}}%
\pgfpathcurveto{\pgfqpoint{5.463869in}{1.323811in}}{\pgfqpoint{5.466703in}{1.330652in}}{\pgfqpoint{5.466703in}{1.337785in}}%
\pgfpathcurveto{\pgfqpoint{5.466703in}{1.344918in}}{\pgfqpoint{5.463869in}{1.351760in}}{\pgfqpoint{5.458825in}{1.356803in}}%
\pgfpathcurveto{\pgfqpoint{5.453782in}{1.361847in}}{\pgfqpoint{5.446940in}{1.364681in}}{\pgfqpoint{5.439807in}{1.364681in}}%
\pgfpathcurveto{\pgfqpoint{5.432674in}{1.364681in}}{\pgfqpoint{5.425833in}{1.361847in}}{\pgfqpoint{5.420789in}{1.356803in}}%
\pgfpathcurveto{\pgfqpoint{5.415745in}{1.351760in}}{\pgfqpoint{5.412912in}{1.344918in}}{\pgfqpoint{5.412912in}{1.337785in}}%
\pgfpathcurveto{\pgfqpoint{5.412912in}{1.330652in}}{\pgfqpoint{5.415745in}{1.323811in}}{\pgfqpoint{5.420789in}{1.318767in}}%
\pgfpathcurveto{\pgfqpoint{5.425833in}{1.313723in}}{\pgfqpoint{5.432674in}{1.310889in}}{\pgfqpoint{5.439807in}{1.310889in}}%
\pgfpathclose%
\pgfusepath{stroke,fill}%
\end{pgfscope}%
\begin{pgfscope}%
\pgfpathrectangle{\pgfqpoint{4.985294in}{0.500000in}}{\pgfqpoint{1.764706in}{1.700000in}}%
\pgfusepath{clip}%
\pgfsetbuttcap%
\pgfsetroundjoin%
\definecolor{currentfill}{rgb}{0.946260,0.398132,0.274897}%
\pgfsetfillcolor{currentfill}%
\pgfsetlinewidth{0.311001pt}%
\definecolor{currentstroke}{rgb}{1.000000,1.000000,1.000000}%
\pgfsetstrokecolor{currentstroke}%
\pgfsetdash{}{0pt}%
\pgfpathmoveto{\pgfqpoint{5.574734in}{1.327881in}}%
\pgfpathcurveto{\pgfqpoint{5.581867in}{1.327881in}}{\pgfqpoint{5.588708in}{1.330715in}}{\pgfqpoint{5.593752in}{1.335759in}}%
\pgfpathcurveto{\pgfqpoint{5.598796in}{1.340803in}}{\pgfqpoint{5.601630in}{1.347644in}}{\pgfqpoint{5.601630in}{1.354777in}}%
\pgfpathcurveto{\pgfqpoint{5.601630in}{1.361910in}}{\pgfqpoint{5.598796in}{1.368752in}}{\pgfqpoint{5.593752in}{1.373795in}}%
\pgfpathcurveto{\pgfqpoint{5.588708in}{1.378839in}}{\pgfqpoint{5.581867in}{1.381673in}}{\pgfqpoint{5.574734in}{1.381673in}}%
\pgfpathcurveto{\pgfqpoint{5.567601in}{1.381673in}}{\pgfqpoint{5.560759in}{1.378839in}}{\pgfqpoint{5.555716in}{1.373795in}}%
\pgfpathcurveto{\pgfqpoint{5.550672in}{1.368752in}}{\pgfqpoint{5.547838in}{1.361910in}}{\pgfqpoint{5.547838in}{1.354777in}}%
\pgfpathcurveto{\pgfqpoint{5.547838in}{1.347644in}}{\pgfqpoint{5.550672in}{1.340803in}}{\pgfqpoint{5.555716in}{1.335759in}}%
\pgfpathcurveto{\pgfqpoint{5.560759in}{1.330715in}}{\pgfqpoint{5.567601in}{1.327881in}}{\pgfqpoint{5.574734in}{1.327881in}}%
\pgfpathclose%
\pgfusepath{stroke,fill}%
\end{pgfscope}%
\begin{pgfscope}%
\pgfpathrectangle{\pgfqpoint{4.985294in}{0.500000in}}{\pgfqpoint{1.764706in}{1.700000in}}%
\pgfusepath{clip}%
\pgfsetbuttcap%
\pgfsetroundjoin%
\definecolor{currentfill}{rgb}{0.972726,0.844889,0.754401}%
\pgfsetfillcolor{currentfill}%
\pgfsetlinewidth{0.311001pt}%
\definecolor{currentstroke}{rgb}{1.000000,1.000000,1.000000}%
\pgfsetstrokecolor{currentstroke}%
\pgfsetdash{}{0pt}%
\pgfpathmoveto{\pgfqpoint{6.263970in}{1.640293in}}%
\pgfpathcurveto{\pgfqpoint{6.271103in}{1.640293in}}{\pgfqpoint{6.277944in}{1.643127in}}{\pgfqpoint{6.282988in}{1.648170in}}%
\pgfpathcurveto{\pgfqpoint{6.288031in}{1.653214in}}{\pgfqpoint{6.290865in}{1.660056in}}{\pgfqpoint{6.290865in}{1.667189in}}%
\pgfpathcurveto{\pgfqpoint{6.290865in}{1.674321in}}{\pgfqpoint{6.288031in}{1.681163in}}{\pgfqpoint{6.282988in}{1.686207in}}%
\pgfpathcurveto{\pgfqpoint{6.277944in}{1.691250in}}{\pgfqpoint{6.271103in}{1.694084in}}{\pgfqpoint{6.263970in}{1.694084in}}%
\pgfpathcurveto{\pgfqpoint{6.256837in}{1.694084in}}{\pgfqpoint{6.249995in}{1.691250in}}{\pgfqpoint{6.244952in}{1.686207in}}%
\pgfpathcurveto{\pgfqpoint{6.239908in}{1.681163in}}{\pgfqpoint{6.237074in}{1.674321in}}{\pgfqpoint{6.237074in}{1.667189in}}%
\pgfpathcurveto{\pgfqpoint{6.237074in}{1.660056in}}{\pgfqpoint{6.239908in}{1.653214in}}{\pgfqpoint{6.244952in}{1.648170in}}%
\pgfpathcurveto{\pgfqpoint{6.249995in}{1.643127in}}{\pgfqpoint{6.256837in}{1.640293in}}{\pgfqpoint{6.263970in}{1.640293in}}%
\pgfpathclose%
\pgfusepath{stroke,fill}%
\end{pgfscope}%
\begin{pgfscope}%
\pgfpathrectangle{\pgfqpoint{4.985294in}{0.500000in}}{\pgfqpoint{1.764706in}{1.700000in}}%
\pgfusepath{clip}%
\pgfsetbuttcap%
\pgfsetroundjoin%
\definecolor{currentfill}{rgb}{0.972201,0.839051,0.745789}%
\pgfsetfillcolor{currentfill}%
\pgfsetlinewidth{0.311001pt}%
\definecolor{currentstroke}{rgb}{1.000000,1.000000,1.000000}%
\pgfsetstrokecolor{currentstroke}%
\pgfsetdash{}{0pt}%
\pgfpathmoveto{\pgfqpoint{5.418832in}{1.553201in}}%
\pgfpathcurveto{\pgfqpoint{5.425965in}{1.553201in}}{\pgfqpoint{5.432807in}{1.556035in}}{\pgfqpoint{5.437850in}{1.561079in}}%
\pgfpathcurveto{\pgfqpoint{5.442894in}{1.566122in}}{\pgfqpoint{5.445728in}{1.572964in}}{\pgfqpoint{5.445728in}{1.580097in}}%
\pgfpathcurveto{\pgfqpoint{5.445728in}{1.587230in}}{\pgfqpoint{5.442894in}{1.594071in}}{\pgfqpoint{5.437850in}{1.599115in}}%
\pgfpathcurveto{\pgfqpoint{5.432807in}{1.604159in}}{\pgfqpoint{5.425965in}{1.606993in}}{\pgfqpoint{5.418832in}{1.606993in}}%
\pgfpathcurveto{\pgfqpoint{5.411699in}{1.606993in}}{\pgfqpoint{5.404858in}{1.604159in}}{\pgfqpoint{5.399814in}{1.599115in}}%
\pgfpathcurveto{\pgfqpoint{5.394770in}{1.594071in}}{\pgfqpoint{5.391937in}{1.587230in}}{\pgfqpoint{5.391937in}{1.580097in}}%
\pgfpathcurveto{\pgfqpoint{5.391937in}{1.572964in}}{\pgfqpoint{5.394770in}{1.566122in}}{\pgfqpoint{5.399814in}{1.561079in}}%
\pgfpathcurveto{\pgfqpoint{5.404858in}{1.556035in}}{\pgfqpoint{5.411699in}{1.553201in}}{\pgfqpoint{5.418832in}{1.553201in}}%
\pgfpathclose%
\pgfusepath{stroke,fill}%
\end{pgfscope}%
\begin{pgfscope}%
\pgfpathrectangle{\pgfqpoint{4.985294in}{0.500000in}}{\pgfqpoint{1.764706in}{1.700000in}}%
\pgfusepath{clip}%
\pgfsetbuttcap%
\pgfsetroundjoin%
\definecolor{currentfill}{rgb}{0.964799,0.689101,0.537560}%
\pgfsetfillcolor{currentfill}%
\pgfsetlinewidth{0.311001pt}%
\definecolor{currentstroke}{rgb}{1.000000,1.000000,1.000000}%
\pgfsetstrokecolor{currentstroke}%
\pgfsetdash{}{0pt}%
\pgfpathmoveto{\pgfqpoint{6.156363in}{0.904397in}}%
\pgfpathcurveto{\pgfqpoint{6.163496in}{0.904397in}}{\pgfqpoint{6.170338in}{0.907231in}}{\pgfqpoint{6.175382in}{0.912275in}}%
\pgfpathcurveto{\pgfqpoint{6.180425in}{0.917318in}}{\pgfqpoint{6.183259in}{0.924160in}}{\pgfqpoint{6.183259in}{0.931293in}}%
\pgfpathcurveto{\pgfqpoint{6.183259in}{0.938426in}}{\pgfqpoint{6.180425in}{0.945267in}}{\pgfqpoint{6.175382in}{0.950311in}}%
\pgfpathcurveto{\pgfqpoint{6.170338in}{0.955355in}}{\pgfqpoint{6.163496in}{0.958189in}}{\pgfqpoint{6.156363in}{0.958189in}}%
\pgfpathcurveto{\pgfqpoint{6.149231in}{0.958189in}}{\pgfqpoint{6.142389in}{0.955355in}}{\pgfqpoint{6.137345in}{0.950311in}}%
\pgfpathcurveto{\pgfqpoint{6.132302in}{0.945267in}}{\pgfqpoint{6.129468in}{0.938426in}}{\pgfqpoint{6.129468in}{0.931293in}}%
\pgfpathcurveto{\pgfqpoint{6.129468in}{0.924160in}}{\pgfqpoint{6.132302in}{0.917318in}}{\pgfqpoint{6.137345in}{0.912275in}}%
\pgfpathcurveto{\pgfqpoint{6.142389in}{0.907231in}}{\pgfqpoint{6.149231in}{0.904397in}}{\pgfqpoint{6.156363in}{0.904397in}}%
\pgfpathclose%
\pgfusepath{stroke,fill}%
\end{pgfscope}%
\begin{pgfscope}%
\pgfpathrectangle{\pgfqpoint{4.985294in}{0.500000in}}{\pgfqpoint{1.764706in}{1.700000in}}%
\pgfusepath{clip}%
\pgfsetbuttcap%
\pgfsetroundjoin%
\definecolor{currentfill}{rgb}{0.959229,0.533075,0.374889}%
\pgfsetfillcolor{currentfill}%
\pgfsetlinewidth{0.311001pt}%
\definecolor{currentstroke}{rgb}{1.000000,1.000000,1.000000}%
\pgfsetstrokecolor{currentstroke}%
\pgfsetdash{}{0pt}%
\pgfpathmoveto{\pgfqpoint{6.131433in}{1.786489in}}%
\pgfpathcurveto{\pgfqpoint{6.138566in}{1.786489in}}{\pgfqpoint{6.145408in}{1.789323in}}{\pgfqpoint{6.150451in}{1.794367in}}%
\pgfpathcurveto{\pgfqpoint{6.155495in}{1.799410in}}{\pgfqpoint{6.158329in}{1.806252in}}{\pgfqpoint{6.158329in}{1.813385in}}%
\pgfpathcurveto{\pgfqpoint{6.158329in}{1.820518in}}{\pgfqpoint{6.155495in}{1.827359in}}{\pgfqpoint{6.150451in}{1.832403in}}%
\pgfpathcurveto{\pgfqpoint{6.145408in}{1.837447in}}{\pgfqpoint{6.138566in}{1.840280in}}{\pgfqpoint{6.131433in}{1.840280in}}%
\pgfpathcurveto{\pgfqpoint{6.124300in}{1.840280in}}{\pgfqpoint{6.117459in}{1.837447in}}{\pgfqpoint{6.112415in}{1.832403in}}%
\pgfpathcurveto{\pgfqpoint{6.107371in}{1.827359in}}{\pgfqpoint{6.104537in}{1.820518in}}{\pgfqpoint{6.104537in}{1.813385in}}%
\pgfpathcurveto{\pgfqpoint{6.104537in}{1.806252in}}{\pgfqpoint{6.107371in}{1.799410in}}{\pgfqpoint{6.112415in}{1.794367in}}%
\pgfpathcurveto{\pgfqpoint{6.117459in}{1.789323in}}{\pgfqpoint{6.124300in}{1.786489in}}{\pgfqpoint{6.131433in}{1.786489in}}%
\pgfpathclose%
\pgfusepath{stroke,fill}%
\end{pgfscope}%
\begin{pgfscope}%
\pgfpathrectangle{\pgfqpoint{4.985294in}{0.500000in}}{\pgfqpoint{1.764706in}{1.700000in}}%
\pgfusepath{clip}%
\pgfsetbuttcap%
\pgfsetroundjoin%
\definecolor{currentfill}{rgb}{0.942910,0.375495,0.263698}%
\pgfsetfillcolor{currentfill}%
\pgfsetlinewidth{0.311001pt}%
\definecolor{currentstroke}{rgb}{1.000000,1.000000,1.000000}%
\pgfsetstrokecolor{currentstroke}%
\pgfsetdash{}{0pt}%
\pgfpathmoveto{\pgfqpoint{5.587847in}{1.224362in}}%
\pgfpathcurveto{\pgfqpoint{5.594979in}{1.224362in}}{\pgfqpoint{5.601821in}{1.227196in}}{\pgfqpoint{5.606865in}{1.232239in}}%
\pgfpathcurveto{\pgfqpoint{5.611908in}{1.237283in}}{\pgfqpoint{5.614742in}{1.244125in}}{\pgfqpoint{5.614742in}{1.251258in}}%
\pgfpathcurveto{\pgfqpoint{5.614742in}{1.258390in}}{\pgfqpoint{5.611908in}{1.265232in}}{\pgfqpoint{5.606865in}{1.270276in}}%
\pgfpathcurveto{\pgfqpoint{5.601821in}{1.275319in}}{\pgfqpoint{5.594979in}{1.278153in}}{\pgfqpoint{5.587847in}{1.278153in}}%
\pgfpathcurveto{\pgfqpoint{5.580714in}{1.278153in}}{\pgfqpoint{5.573872in}{1.275319in}}{\pgfqpoint{5.568828in}{1.270276in}}%
\pgfpathcurveto{\pgfqpoint{5.563785in}{1.265232in}}{\pgfqpoint{5.560951in}{1.258390in}}{\pgfqpoint{5.560951in}{1.251258in}}%
\pgfpathcurveto{\pgfqpoint{5.560951in}{1.244125in}}{\pgfqpoint{5.563785in}{1.237283in}}{\pgfqpoint{5.568828in}{1.232239in}}%
\pgfpathcurveto{\pgfqpoint{5.573872in}{1.227196in}}{\pgfqpoint{5.580714in}{1.224362in}}{\pgfqpoint{5.587847in}{1.224362in}}%
\pgfpathclose%
\pgfusepath{stroke,fill}%
\end{pgfscope}%
\begin{pgfscope}%
\pgfpathrectangle{\pgfqpoint{4.985294in}{0.500000in}}{\pgfqpoint{1.764706in}{1.700000in}}%
\pgfusepath{clip}%
\pgfsetbuttcap%
\pgfsetroundjoin%
\definecolor{currentfill}{rgb}{0.977657,0.891500,0.822809}%
\pgfsetfillcolor{currentfill}%
\pgfsetlinewidth{0.311001pt}%
\definecolor{currentstroke}{rgb}{1.000000,1.000000,1.000000}%
\pgfsetstrokecolor{currentstroke}%
\pgfsetdash{}{0pt}%
\pgfpathmoveto{\pgfqpoint{5.453306in}{1.156883in}}%
\pgfpathcurveto{\pgfqpoint{5.460439in}{1.156883in}}{\pgfqpoint{5.467280in}{1.159717in}}{\pgfqpoint{5.472324in}{1.164760in}}%
\pgfpathcurveto{\pgfqpoint{5.477368in}{1.169804in}}{\pgfqpoint{5.480201in}{1.176646in}}{\pgfqpoint{5.480201in}{1.183779in}}%
\pgfpathcurveto{\pgfqpoint{5.480201in}{1.190911in}}{\pgfqpoint{5.477368in}{1.197753in}}{\pgfqpoint{5.472324in}{1.202797in}}%
\pgfpathcurveto{\pgfqpoint{5.467280in}{1.207840in}}{\pgfqpoint{5.460439in}{1.210674in}}{\pgfqpoint{5.453306in}{1.210674in}}%
\pgfpathcurveto{\pgfqpoint{5.446173in}{1.210674in}}{\pgfqpoint{5.439331in}{1.207840in}}{\pgfqpoint{5.434288in}{1.202797in}}%
\pgfpathcurveto{\pgfqpoint{5.429244in}{1.197753in}}{\pgfqpoint{5.426410in}{1.190911in}}{\pgfqpoint{5.426410in}{1.183779in}}%
\pgfpathcurveto{\pgfqpoint{5.426410in}{1.176646in}}{\pgfqpoint{5.429244in}{1.169804in}}{\pgfqpoint{5.434288in}{1.164760in}}%
\pgfpathcurveto{\pgfqpoint{5.439331in}{1.159717in}}{\pgfqpoint{5.446173in}{1.156883in}}{\pgfqpoint{5.453306in}{1.156883in}}%
\pgfpathclose%
\pgfusepath{stroke,fill}%
\end{pgfscope}%
\begin{pgfscope}%
\pgfpathrectangle{\pgfqpoint{4.985294in}{0.500000in}}{\pgfqpoint{1.764706in}{1.700000in}}%
\pgfusepath{clip}%
\pgfsetbuttcap%
\pgfsetroundjoin%
\definecolor{currentfill}{rgb}{0.967398,0.774513,0.650573}%
\pgfsetfillcolor{currentfill}%
\pgfsetlinewidth{0.311001pt}%
\definecolor{currentstroke}{rgb}{1.000000,1.000000,1.000000}%
\pgfsetstrokecolor{currentstroke}%
\pgfsetdash{}{0pt}%
\pgfpathmoveto{\pgfqpoint{5.406618in}{1.041850in}}%
\pgfpathcurveto{\pgfqpoint{5.413751in}{1.041850in}}{\pgfqpoint{5.420593in}{1.044684in}}{\pgfqpoint{5.425637in}{1.049728in}}%
\pgfpathcurveto{\pgfqpoint{5.430680in}{1.054771in}}{\pgfqpoint{5.433514in}{1.061613in}}{\pgfqpoint{5.433514in}{1.068746in}}%
\pgfpathcurveto{\pgfqpoint{5.433514in}{1.075879in}}{\pgfqpoint{5.430680in}{1.082720in}}{\pgfqpoint{5.425637in}{1.087764in}}%
\pgfpathcurveto{\pgfqpoint{5.420593in}{1.092808in}}{\pgfqpoint{5.413751in}{1.095642in}}{\pgfqpoint{5.406618in}{1.095642in}}%
\pgfpathcurveto{\pgfqpoint{5.399486in}{1.095642in}}{\pgfqpoint{5.392644in}{1.092808in}}{\pgfqpoint{5.387600in}{1.087764in}}%
\pgfpathcurveto{\pgfqpoint{5.382557in}{1.082720in}}{\pgfqpoint{5.379723in}{1.075879in}}{\pgfqpoint{5.379723in}{1.068746in}}%
\pgfpathcurveto{\pgfqpoint{5.379723in}{1.061613in}}{\pgfqpoint{5.382557in}{1.054771in}}{\pgfqpoint{5.387600in}{1.049728in}}%
\pgfpathcurveto{\pgfqpoint{5.392644in}{1.044684in}}{\pgfqpoint{5.399486in}{1.041850in}}{\pgfqpoint{5.406618in}{1.041850in}}%
\pgfpathclose%
\pgfusepath{stroke,fill}%
\end{pgfscope}%
\begin{pgfscope}%
\pgfpathrectangle{\pgfqpoint{4.985294in}{0.500000in}}{\pgfqpoint{1.764706in}{1.700000in}}%
\pgfusepath{clip}%
\pgfsetbuttcap%
\pgfsetroundjoin%
\definecolor{currentfill}{rgb}{0.979124,0.903132,0.839793}%
\pgfsetfillcolor{currentfill}%
\pgfsetlinewidth{0.311001pt}%
\definecolor{currentstroke}{rgb}{1.000000,1.000000,1.000000}%
\pgfsetstrokecolor{currentstroke}%
\pgfsetdash{}{0pt}%
\pgfpathmoveto{\pgfqpoint{6.290060in}{1.388620in}}%
\pgfpathcurveto{\pgfqpoint{6.297192in}{1.388620in}}{\pgfqpoint{6.304034in}{1.391454in}}{\pgfqpoint{6.309078in}{1.396498in}}%
\pgfpathcurveto{\pgfqpoint{6.314121in}{1.401541in}}{\pgfqpoint{6.316955in}{1.408383in}}{\pgfqpoint{6.316955in}{1.415516in}}%
\pgfpathcurveto{\pgfqpoint{6.316955in}{1.422649in}}{\pgfqpoint{6.314121in}{1.429490in}}{\pgfqpoint{6.309078in}{1.434534in}}%
\pgfpathcurveto{\pgfqpoint{6.304034in}{1.439578in}}{\pgfqpoint{6.297192in}{1.442411in}}{\pgfqpoint{6.290060in}{1.442411in}}%
\pgfpathcurveto{\pgfqpoint{6.282927in}{1.442411in}}{\pgfqpoint{6.276085in}{1.439578in}}{\pgfqpoint{6.271041in}{1.434534in}}%
\pgfpathcurveto{\pgfqpoint{6.265998in}{1.429490in}}{\pgfqpoint{6.263164in}{1.422649in}}{\pgfqpoint{6.263164in}{1.415516in}}%
\pgfpathcurveto{\pgfqpoint{6.263164in}{1.408383in}}{\pgfqpoint{6.265998in}{1.401541in}}{\pgfqpoint{6.271041in}{1.396498in}}%
\pgfpathcurveto{\pgfqpoint{6.276085in}{1.391454in}}{\pgfqpoint{6.282927in}{1.388620in}}{\pgfqpoint{6.290060in}{1.388620in}}%
\pgfpathclose%
\pgfusepath{stroke,fill}%
\end{pgfscope}%
\begin{pgfscope}%
\pgfpathrectangle{\pgfqpoint{4.985294in}{0.500000in}}{\pgfqpoint{1.764706in}{1.700000in}}%
\pgfusepath{clip}%
\pgfsetbuttcap%
\pgfsetroundjoin%
\definecolor{currentfill}{rgb}{0.967735,0.780441,0.659127}%
\pgfsetfillcolor{currentfill}%
\pgfsetlinewidth{0.311001pt}%
\definecolor{currentstroke}{rgb}{1.000000,1.000000,1.000000}%
\pgfsetstrokecolor{currentstroke}%
\pgfsetdash{}{0pt}%
\pgfpathmoveto{\pgfqpoint{5.502778in}{1.677379in}}%
\pgfpathcurveto{\pgfqpoint{5.509911in}{1.677379in}}{\pgfqpoint{5.516753in}{1.680213in}}{\pgfqpoint{5.521797in}{1.685257in}}%
\pgfpathcurveto{\pgfqpoint{5.526840in}{1.690301in}}{\pgfqpoint{5.529674in}{1.697142in}}{\pgfqpoint{5.529674in}{1.704275in}}%
\pgfpathcurveto{\pgfqpoint{5.529674in}{1.711408in}}{\pgfqpoint{5.526840in}{1.718250in}}{\pgfqpoint{5.521797in}{1.723293in}}%
\pgfpathcurveto{\pgfqpoint{5.516753in}{1.728337in}}{\pgfqpoint{5.509911in}{1.731171in}}{\pgfqpoint{5.502778in}{1.731171in}}%
\pgfpathcurveto{\pgfqpoint{5.495646in}{1.731171in}}{\pgfqpoint{5.488804in}{1.728337in}}{\pgfqpoint{5.483760in}{1.723293in}}%
\pgfpathcurveto{\pgfqpoint{5.478717in}{1.718250in}}{\pgfqpoint{5.475883in}{1.711408in}}{\pgfqpoint{5.475883in}{1.704275in}}%
\pgfpathcurveto{\pgfqpoint{5.475883in}{1.697142in}}{\pgfqpoint{5.478717in}{1.690301in}}{\pgfqpoint{5.483760in}{1.685257in}}%
\pgfpathcurveto{\pgfqpoint{5.488804in}{1.680213in}}{\pgfqpoint{5.495646in}{1.677379in}}{\pgfqpoint{5.502778in}{1.677379in}}%
\pgfpathclose%
\pgfusepath{stroke,fill}%
\end{pgfscope}%
\begin{pgfscope}%
\pgfpathrectangle{\pgfqpoint{4.985294in}{0.500000in}}{\pgfqpoint{1.764706in}{1.700000in}}%
\pgfusepath{clip}%
\pgfsetbuttcap%
\pgfsetroundjoin%
\definecolor{currentfill}{rgb}{0.968509,0.792226,0.676405}%
\pgfsetfillcolor{currentfill}%
\pgfsetlinewidth{0.311001pt}%
\definecolor{currentstroke}{rgb}{1.000000,1.000000,1.000000}%
\pgfsetstrokecolor{currentstroke}%
\pgfsetdash{}{0pt}%
\pgfpathmoveto{\pgfqpoint{6.355042in}{1.123249in}}%
\pgfpathcurveto{\pgfqpoint{6.362175in}{1.123249in}}{\pgfqpoint{6.369016in}{1.126083in}}{\pgfqpoint{6.374060in}{1.131127in}}%
\pgfpathcurveto{\pgfqpoint{6.379104in}{1.136170in}}{\pgfqpoint{6.381938in}{1.143012in}}{\pgfqpoint{6.381938in}{1.150145in}}%
\pgfpathcurveto{\pgfqpoint{6.381938in}{1.157278in}}{\pgfqpoint{6.379104in}{1.164119in}}{\pgfqpoint{6.374060in}{1.169163in}}%
\pgfpathcurveto{\pgfqpoint{6.369016in}{1.174207in}}{\pgfqpoint{6.362175in}{1.177041in}}{\pgfqpoint{6.355042in}{1.177041in}}%
\pgfpathcurveto{\pgfqpoint{6.347909in}{1.177041in}}{\pgfqpoint{6.341067in}{1.174207in}}{\pgfqpoint{6.336024in}{1.169163in}}%
\pgfpathcurveto{\pgfqpoint{6.330980in}{1.164119in}}{\pgfqpoint{6.328146in}{1.157278in}}{\pgfqpoint{6.328146in}{1.150145in}}%
\pgfpathcurveto{\pgfqpoint{6.328146in}{1.143012in}}{\pgfqpoint{6.330980in}{1.136170in}}{\pgfqpoint{6.336024in}{1.131127in}}%
\pgfpathcurveto{\pgfqpoint{6.341067in}{1.126083in}}{\pgfqpoint{6.347909in}{1.123249in}}{\pgfqpoint{6.355042in}{1.123249in}}%
\pgfpathclose%
\pgfusepath{stroke,fill}%
\end{pgfscope}%
\begin{pgfscope}%
\pgfpathrectangle{\pgfqpoint{4.985294in}{0.500000in}}{\pgfqpoint{1.764706in}{1.700000in}}%
\pgfusepath{clip}%
\pgfsetbuttcap%
\pgfsetroundjoin%
\definecolor{currentfill}{rgb}{0.975644,0.874038,0.797253}%
\pgfsetfillcolor{currentfill}%
\pgfsetlinewidth{0.311001pt}%
\definecolor{currentstroke}{rgb}{1.000000,1.000000,1.000000}%
\pgfsetstrokecolor{currentstroke}%
\pgfsetdash{}{0pt}%
\pgfpathmoveto{\pgfqpoint{5.434729in}{1.528006in}}%
\pgfpathcurveto{\pgfqpoint{5.441862in}{1.528006in}}{\pgfqpoint{5.448704in}{1.530840in}}{\pgfqpoint{5.453747in}{1.535884in}}%
\pgfpathcurveto{\pgfqpoint{5.458791in}{1.540927in}}{\pgfqpoint{5.461625in}{1.547769in}}{\pgfqpoint{5.461625in}{1.554902in}}%
\pgfpathcurveto{\pgfqpoint{5.461625in}{1.562035in}}{\pgfqpoint{5.458791in}{1.568876in}}{\pgfqpoint{5.453747in}{1.573920in}}%
\pgfpathcurveto{\pgfqpoint{5.448704in}{1.578964in}}{\pgfqpoint{5.441862in}{1.581797in}}{\pgfqpoint{5.434729in}{1.581797in}}%
\pgfpathcurveto{\pgfqpoint{5.427596in}{1.581797in}}{\pgfqpoint{5.420755in}{1.578964in}}{\pgfqpoint{5.415711in}{1.573920in}}%
\pgfpathcurveto{\pgfqpoint{5.410667in}{1.568876in}}{\pgfqpoint{5.407833in}{1.562035in}}{\pgfqpoint{5.407833in}{1.554902in}}%
\pgfpathcurveto{\pgfqpoint{5.407833in}{1.547769in}}{\pgfqpoint{5.410667in}{1.540927in}}{\pgfqpoint{5.415711in}{1.535884in}}%
\pgfpathcurveto{\pgfqpoint{5.420755in}{1.530840in}}{\pgfqpoint{5.427596in}{1.528006in}}{\pgfqpoint{5.434729in}{1.528006in}}%
\pgfpathclose%
\pgfusepath{stroke,fill}%
\end{pgfscope}%
\begin{pgfscope}%
\pgfpathrectangle{\pgfqpoint{4.985294in}{0.500000in}}{\pgfqpoint{1.764706in}{1.700000in}}%
\pgfusepath{clip}%
\pgfsetbuttcap%
\pgfsetroundjoin%
\definecolor{currentfill}{rgb}{0.965440,0.720101,0.576404}%
\pgfsetfillcolor{currentfill}%
\pgfsetlinewidth{0.311001pt}%
\definecolor{currentstroke}{rgb}{1.000000,1.000000,1.000000}%
\pgfsetstrokecolor{currentstroke}%
\pgfsetdash{}{0pt}%
\pgfpathmoveto{\pgfqpoint{5.580514in}{1.652994in}}%
\pgfpathcurveto{\pgfqpoint{5.587647in}{1.652994in}}{\pgfqpoint{5.594489in}{1.655828in}}{\pgfqpoint{5.599532in}{1.660872in}}%
\pgfpathcurveto{\pgfqpoint{5.604576in}{1.665916in}}{\pgfqpoint{5.607410in}{1.672757in}}{\pgfqpoint{5.607410in}{1.679890in}}%
\pgfpathcurveto{\pgfqpoint{5.607410in}{1.687023in}}{\pgfqpoint{5.604576in}{1.693865in}}{\pgfqpoint{5.599532in}{1.698908in}}%
\pgfpathcurveto{\pgfqpoint{5.594489in}{1.703952in}}{\pgfqpoint{5.587647in}{1.706786in}}{\pgfqpoint{5.580514in}{1.706786in}}%
\pgfpathcurveto{\pgfqpoint{5.573381in}{1.706786in}}{\pgfqpoint{5.566540in}{1.703952in}}{\pgfqpoint{5.561496in}{1.698908in}}%
\pgfpathcurveto{\pgfqpoint{5.556452in}{1.693865in}}{\pgfqpoint{5.553618in}{1.687023in}}{\pgfqpoint{5.553618in}{1.679890in}}%
\pgfpathcurveto{\pgfqpoint{5.553618in}{1.672757in}}{\pgfqpoint{5.556452in}{1.665916in}}{\pgfqpoint{5.561496in}{1.660872in}}%
\pgfpathcurveto{\pgfqpoint{5.566540in}{1.655828in}}{\pgfqpoint{5.573381in}{1.652994in}}{\pgfqpoint{5.580514in}{1.652994in}}%
\pgfpathclose%
\pgfusepath{stroke,fill}%
\end{pgfscope}%
\begin{pgfscope}%
\pgfpathrectangle{\pgfqpoint{4.985294in}{0.500000in}}{\pgfqpoint{1.764706in}{1.700000in}}%
\pgfusepath{clip}%
\pgfsetbuttcap%
\pgfsetroundjoin%
\definecolor{currentfill}{rgb}{0.973832,0.856556,0.771584}%
\pgfsetfillcolor{currentfill}%
\pgfsetlinewidth{0.311001pt}%
\definecolor{currentstroke}{rgb}{1.000000,1.000000,1.000000}%
\pgfsetstrokecolor{currentstroke}%
\pgfsetdash{}{0pt}%
\pgfpathmoveto{\pgfqpoint{5.485292in}{1.440630in}}%
\pgfpathcurveto{\pgfqpoint{5.492424in}{1.440630in}}{\pgfqpoint{5.499266in}{1.443463in}}{\pgfqpoint{5.504310in}{1.448507in}}%
\pgfpathcurveto{\pgfqpoint{5.509353in}{1.453551in}}{\pgfqpoint{5.512187in}{1.460392in}}{\pgfqpoint{5.512187in}{1.467525in}}%
\pgfpathcurveto{\pgfqpoint{5.512187in}{1.474658in}}{\pgfqpoint{5.509353in}{1.481500in}}{\pgfqpoint{5.504310in}{1.486543in}}%
\pgfpathcurveto{\pgfqpoint{5.499266in}{1.491587in}}{\pgfqpoint{5.492424in}{1.494421in}}{\pgfqpoint{5.485292in}{1.494421in}}%
\pgfpathcurveto{\pgfqpoint{5.478159in}{1.494421in}}{\pgfqpoint{5.471317in}{1.491587in}}{\pgfqpoint{5.466273in}{1.486543in}}%
\pgfpathcurveto{\pgfqpoint{5.461230in}{1.481500in}}{\pgfqpoint{5.458396in}{1.474658in}}{\pgfqpoint{5.458396in}{1.467525in}}%
\pgfpathcurveto{\pgfqpoint{5.458396in}{1.460392in}}{\pgfqpoint{5.461230in}{1.453551in}}{\pgfqpoint{5.466273in}{1.448507in}}%
\pgfpathcurveto{\pgfqpoint{5.471317in}{1.443463in}}{\pgfqpoint{5.478159in}{1.440630in}}{\pgfqpoint{5.485292in}{1.440630in}}%
\pgfpathclose%
\pgfusepath{stroke,fill}%
\end{pgfscope}%
\begin{pgfscope}%
\pgfpathrectangle{\pgfqpoint{4.985294in}{0.500000in}}{\pgfqpoint{1.764706in}{1.700000in}}%
\pgfusepath{clip}%
\pgfsetbuttcap%
\pgfsetroundjoin%
\definecolor{currentfill}{rgb}{0.975644,0.874038,0.797253}%
\pgfsetfillcolor{currentfill}%
\pgfsetlinewidth{0.311001pt}%
\definecolor{currentstroke}{rgb}{1.000000,1.000000,1.000000}%
\pgfsetstrokecolor{currentstroke}%
\pgfsetdash{}{0pt}%
\pgfpathmoveto{\pgfqpoint{5.443568in}{1.090586in}}%
\pgfpathcurveto{\pgfqpoint{5.450700in}{1.090586in}}{\pgfqpoint{5.457542in}{1.093420in}}{\pgfqpoint{5.462586in}{1.098463in}}%
\pgfpathcurveto{\pgfqpoint{5.467629in}{1.103507in}}{\pgfqpoint{5.470463in}{1.110349in}}{\pgfqpoint{5.470463in}{1.117481in}}%
\pgfpathcurveto{\pgfqpoint{5.470463in}{1.124614in}}{\pgfqpoint{5.467629in}{1.131456in}}{\pgfqpoint{5.462586in}{1.136500in}}%
\pgfpathcurveto{\pgfqpoint{5.457542in}{1.141543in}}{\pgfqpoint{5.450700in}{1.144377in}}{\pgfqpoint{5.443568in}{1.144377in}}%
\pgfpathcurveto{\pgfqpoint{5.436435in}{1.144377in}}{\pgfqpoint{5.429593in}{1.141543in}}{\pgfqpoint{5.424549in}{1.136500in}}%
\pgfpathcurveto{\pgfqpoint{5.419506in}{1.131456in}}{\pgfqpoint{5.416672in}{1.124614in}}{\pgfqpoint{5.416672in}{1.117481in}}%
\pgfpathcurveto{\pgfqpoint{5.416672in}{1.110349in}}{\pgfqpoint{5.419506in}{1.103507in}}{\pgfqpoint{5.424549in}{1.098463in}}%
\pgfpathcurveto{\pgfqpoint{5.429593in}{1.093420in}}{\pgfqpoint{5.436435in}{1.090586in}}{\pgfqpoint{5.443568in}{1.090586in}}%
\pgfpathclose%
\pgfusepath{stroke,fill}%
\end{pgfscope}%
\begin{pgfscope}%
\pgfpathrectangle{\pgfqpoint{4.985294in}{0.500000in}}{\pgfqpoint{1.764706in}{1.700000in}}%
\pgfusepath{clip}%
\pgfsetbuttcap%
\pgfsetroundjoin%
\definecolor{currentfill}{rgb}{0.962283,0.593046,0.431453}%
\pgfsetfillcolor{currentfill}%
\pgfsetlinewidth{0.311001pt}%
\definecolor{currentstroke}{rgb}{1.000000,1.000000,1.000000}%
\pgfsetstrokecolor{currentstroke}%
\pgfsetdash{}{0pt}%
\pgfpathmoveto{\pgfqpoint{6.084836in}{1.698191in}}%
\pgfpathcurveto{\pgfqpoint{6.091969in}{1.698191in}}{\pgfqpoint{6.098811in}{1.701024in}}{\pgfqpoint{6.103854in}{1.706068in}}%
\pgfpathcurveto{\pgfqpoint{6.108898in}{1.711112in}}{\pgfqpoint{6.111732in}{1.717953in}}{\pgfqpoint{6.111732in}{1.725086in}}%
\pgfpathcurveto{\pgfqpoint{6.111732in}{1.732219in}}{\pgfqpoint{6.108898in}{1.739061in}}{\pgfqpoint{6.103854in}{1.744104in}}%
\pgfpathcurveto{\pgfqpoint{6.098811in}{1.749148in}}{\pgfqpoint{6.091969in}{1.751982in}}{\pgfqpoint{6.084836in}{1.751982in}}%
\pgfpathcurveto{\pgfqpoint{6.077704in}{1.751982in}}{\pgfqpoint{6.070862in}{1.749148in}}{\pgfqpoint{6.065818in}{1.744104in}}%
\pgfpathcurveto{\pgfqpoint{6.060775in}{1.739061in}}{\pgfqpoint{6.057941in}{1.732219in}}{\pgfqpoint{6.057941in}{1.725086in}}%
\pgfpathcurveto{\pgfqpoint{6.057941in}{1.717953in}}{\pgfqpoint{6.060775in}{1.711112in}}{\pgfqpoint{6.065818in}{1.706068in}}%
\pgfpathcurveto{\pgfqpoint{6.070862in}{1.701024in}}{\pgfqpoint{6.077704in}{1.698191in}}{\pgfqpoint{6.084836in}{1.698191in}}%
\pgfpathclose%
\pgfusepath{stroke,fill}%
\end{pgfscope}%
\begin{pgfscope}%
\pgfpathrectangle{\pgfqpoint{4.985294in}{0.500000in}}{\pgfqpoint{1.764706in}{1.700000in}}%
\pgfusepath{clip}%
\pgfsetbuttcap%
\pgfsetroundjoin%
\definecolor{currentfill}{rgb}{0.934351,0.329284,0.247753}%
\pgfsetfillcolor{currentfill}%
\pgfsetlinewidth{0.311001pt}%
\definecolor{currentstroke}{rgb}{1.000000,1.000000,1.000000}%
\pgfsetstrokecolor{currentstroke}%
\pgfsetdash{}{0pt}%
\pgfpathmoveto{\pgfqpoint{5.473922in}{1.794706in}}%
\pgfpathcurveto{\pgfqpoint{5.481055in}{1.794706in}}{\pgfqpoint{5.487897in}{1.797540in}}{\pgfqpoint{5.492940in}{1.802584in}}%
\pgfpathcurveto{\pgfqpoint{5.497984in}{1.807628in}}{\pgfqpoint{5.500818in}{1.814469in}}{\pgfqpoint{5.500818in}{1.821602in}}%
\pgfpathcurveto{\pgfqpoint{5.500818in}{1.828735in}}{\pgfqpoint{5.497984in}{1.835577in}}{\pgfqpoint{5.492940in}{1.840620in}}%
\pgfpathcurveto{\pgfqpoint{5.487897in}{1.845664in}}{\pgfqpoint{5.481055in}{1.848498in}}{\pgfqpoint{5.473922in}{1.848498in}}%
\pgfpathcurveto{\pgfqpoint{5.466789in}{1.848498in}}{\pgfqpoint{5.459948in}{1.845664in}}{\pgfqpoint{5.454904in}{1.840620in}}%
\pgfpathcurveto{\pgfqpoint{5.449860in}{1.835577in}}{\pgfqpoint{5.447026in}{1.828735in}}{\pgfqpoint{5.447026in}{1.821602in}}%
\pgfpathcurveto{\pgfqpoint{5.447026in}{1.814469in}}{\pgfqpoint{5.449860in}{1.807628in}}{\pgfqpoint{5.454904in}{1.802584in}}%
\pgfpathcurveto{\pgfqpoint{5.459948in}{1.797540in}}{\pgfqpoint{5.466789in}{1.794706in}}{\pgfqpoint{5.473922in}{1.794706in}}%
\pgfpathclose%
\pgfusepath{stroke,fill}%
\end{pgfscope}%
\begin{pgfscope}%
\pgfpathrectangle{\pgfqpoint{4.985294in}{0.500000in}}{\pgfqpoint{1.764706in}{1.700000in}}%
\pgfusepath{clip}%
\pgfsetbuttcap%
\pgfsetroundjoin%
\definecolor{currentfill}{rgb}{0.977657,0.891500,0.822809}%
\pgfsetfillcolor{currentfill}%
\pgfsetlinewidth{0.311001pt}%
\definecolor{currentstroke}{rgb}{1.000000,1.000000,1.000000}%
\pgfsetstrokecolor{currentstroke}%
\pgfsetdash{}{0pt}%
\pgfpathmoveto{\pgfqpoint{6.310450in}{1.529328in}}%
\pgfpathcurveto{\pgfqpoint{6.317583in}{1.529328in}}{\pgfqpoint{6.324424in}{1.532162in}}{\pgfqpoint{6.329468in}{1.537206in}}%
\pgfpathcurveto{\pgfqpoint{6.334512in}{1.542249in}}{\pgfqpoint{6.337346in}{1.549091in}}{\pgfqpoint{6.337346in}{1.556224in}}%
\pgfpathcurveto{\pgfqpoint{6.337346in}{1.563357in}}{\pgfqpoint{6.334512in}{1.570198in}}{\pgfqpoint{6.329468in}{1.575242in}}%
\pgfpathcurveto{\pgfqpoint{6.324424in}{1.580286in}}{\pgfqpoint{6.317583in}{1.583120in}}{\pgfqpoint{6.310450in}{1.583120in}}%
\pgfpathcurveto{\pgfqpoint{6.303317in}{1.583120in}}{\pgfqpoint{6.296475in}{1.580286in}}{\pgfqpoint{6.291432in}{1.575242in}}%
\pgfpathcurveto{\pgfqpoint{6.286388in}{1.570198in}}{\pgfqpoint{6.283554in}{1.563357in}}{\pgfqpoint{6.283554in}{1.556224in}}%
\pgfpathcurveto{\pgfqpoint{6.283554in}{1.549091in}}{\pgfqpoint{6.286388in}{1.542249in}}{\pgfqpoint{6.291432in}{1.537206in}}%
\pgfpathcurveto{\pgfqpoint{6.296475in}{1.532162in}}{\pgfqpoint{6.303317in}{1.529328in}}{\pgfqpoint{6.310450in}{1.529328in}}%
\pgfpathclose%
\pgfusepath{stroke,fill}%
\end{pgfscope}%
\begin{pgfscope}%
\pgfpathrectangle{\pgfqpoint{4.985294in}{0.500000in}}{\pgfqpoint{1.764706in}{1.700000in}}%
\pgfusepath{clip}%
\pgfsetbuttcap%
\pgfsetroundjoin%
\definecolor{currentfill}{rgb}{0.978376,0.897317,0.831308}%
\pgfsetfillcolor{currentfill}%
\pgfsetlinewidth{0.311001pt}%
\definecolor{currentstroke}{rgb}{1.000000,1.000000,1.000000}%
\pgfsetstrokecolor{currentstroke}%
\pgfsetdash{}{0pt}%
\pgfpathmoveto{\pgfqpoint{6.272053in}{1.571086in}}%
\pgfpathcurveto{\pgfqpoint{6.279186in}{1.571086in}}{\pgfqpoint{6.286027in}{1.573919in}}{\pgfqpoint{6.291071in}{1.578963in}}%
\pgfpathcurveto{\pgfqpoint{6.296115in}{1.584007in}}{\pgfqpoint{6.298949in}{1.590848in}}{\pgfqpoint{6.298949in}{1.597981in}}%
\pgfpathcurveto{\pgfqpoint{6.298949in}{1.605114in}}{\pgfqpoint{6.296115in}{1.611956in}}{\pgfqpoint{6.291071in}{1.616999in}}%
\pgfpathcurveto{\pgfqpoint{6.286027in}{1.622043in}}{\pgfqpoint{6.279186in}{1.624877in}}{\pgfqpoint{6.272053in}{1.624877in}}%
\pgfpathcurveto{\pgfqpoint{6.264920in}{1.624877in}}{\pgfqpoint{6.258078in}{1.622043in}}{\pgfqpoint{6.253035in}{1.616999in}}%
\pgfpathcurveto{\pgfqpoint{6.247991in}{1.611956in}}{\pgfqpoint{6.245157in}{1.605114in}}{\pgfqpoint{6.245157in}{1.597981in}}%
\pgfpathcurveto{\pgfqpoint{6.245157in}{1.590848in}}{\pgfqpoint{6.247991in}{1.584007in}}{\pgfqpoint{6.253035in}{1.578963in}}%
\pgfpathcurveto{\pgfqpoint{6.258078in}{1.573919in}}{\pgfqpoint{6.264920in}{1.571086in}}{\pgfqpoint{6.272053in}{1.571086in}}%
\pgfpathclose%
\pgfusepath{stroke,fill}%
\end{pgfscope}%
\begin{pgfscope}%
\pgfpathrectangle{\pgfqpoint{4.985294in}{0.500000in}}{\pgfqpoint{1.764706in}{1.700000in}}%
\pgfusepath{clip}%
\pgfsetbuttcap%
\pgfsetroundjoin%
\definecolor{currentfill}{rgb}{0.959645,0.539840,0.380928}%
\pgfsetfillcolor{currentfill}%
\pgfsetlinewidth{0.311001pt}%
\definecolor{currentstroke}{rgb}{1.000000,1.000000,1.000000}%
\pgfsetstrokecolor{currentstroke}%
\pgfsetdash{}{0pt}%
\pgfpathmoveto{\pgfqpoint{5.331358in}{1.078871in}}%
\pgfpathcurveto{\pgfqpoint{5.338491in}{1.078871in}}{\pgfqpoint{5.345332in}{1.081705in}}{\pgfqpoint{5.350376in}{1.086748in}}%
\pgfpathcurveto{\pgfqpoint{5.355420in}{1.091792in}}{\pgfqpoint{5.358254in}{1.098634in}}{\pgfqpoint{5.358254in}{1.105766in}}%
\pgfpathcurveto{\pgfqpoint{5.358254in}{1.112899in}}{\pgfqpoint{5.355420in}{1.119741in}}{\pgfqpoint{5.350376in}{1.124785in}}%
\pgfpathcurveto{\pgfqpoint{5.345332in}{1.129828in}}{\pgfqpoint{5.338491in}{1.132662in}}{\pgfqpoint{5.331358in}{1.132662in}}%
\pgfpathcurveto{\pgfqpoint{5.324225in}{1.132662in}}{\pgfqpoint{5.317383in}{1.129828in}}{\pgfqpoint{5.312340in}{1.124785in}}%
\pgfpathcurveto{\pgfqpoint{5.307296in}{1.119741in}}{\pgfqpoint{5.304462in}{1.112899in}}{\pgfqpoint{5.304462in}{1.105766in}}%
\pgfpathcurveto{\pgfqpoint{5.304462in}{1.098634in}}{\pgfqpoint{5.307296in}{1.091792in}}{\pgfqpoint{5.312340in}{1.086748in}}%
\pgfpathcurveto{\pgfqpoint{5.317383in}{1.081705in}}{\pgfqpoint{5.324225in}{1.078871in}}{\pgfqpoint{5.331358in}{1.078871in}}%
\pgfpathclose%
\pgfusepath{stroke,fill}%
\end{pgfscope}%
\begin{pgfscope}%
\pgfpathrectangle{\pgfqpoint{4.985294in}{0.500000in}}{\pgfqpoint{1.764706in}{1.700000in}}%
\pgfusepath{clip}%
\pgfsetbuttcap%
\pgfsetroundjoin%
\definecolor{currentfill}{rgb}{0.979891,0.908948,0.848279}%
\pgfsetfillcolor{currentfill}%
\pgfsetlinewidth{0.311001pt}%
\definecolor{currentstroke}{rgb}{1.000000,1.000000,1.000000}%
\pgfsetstrokecolor{currentstroke}%
\pgfsetdash{}{0pt}%
\pgfpathmoveto{\pgfqpoint{5.405495in}{1.393142in}}%
\pgfpathcurveto{\pgfqpoint{5.412628in}{1.393142in}}{\pgfqpoint{5.419470in}{1.395976in}}{\pgfqpoint{5.424513in}{1.401020in}}%
\pgfpathcurveto{\pgfqpoint{5.429557in}{1.406063in}}{\pgfqpoint{5.432391in}{1.412905in}}{\pgfqpoint{5.432391in}{1.420038in}}%
\pgfpathcurveto{\pgfqpoint{5.432391in}{1.427171in}}{\pgfqpoint{5.429557in}{1.434012in}}{\pgfqpoint{5.424513in}{1.439056in}}%
\pgfpathcurveto{\pgfqpoint{5.419470in}{1.444100in}}{\pgfqpoint{5.412628in}{1.446933in}}{\pgfqpoint{5.405495in}{1.446933in}}%
\pgfpathcurveto{\pgfqpoint{5.398362in}{1.446933in}}{\pgfqpoint{5.391521in}{1.444100in}}{\pgfqpoint{5.386477in}{1.439056in}}%
\pgfpathcurveto{\pgfqpoint{5.381433in}{1.434012in}}{\pgfqpoint{5.378599in}{1.427171in}}{\pgfqpoint{5.378599in}{1.420038in}}%
\pgfpathcurveto{\pgfqpoint{5.378599in}{1.412905in}}{\pgfqpoint{5.381433in}{1.406063in}}{\pgfqpoint{5.386477in}{1.401020in}}%
\pgfpathcurveto{\pgfqpoint{5.391521in}{1.395976in}}{\pgfqpoint{5.398362in}{1.393142in}}{\pgfqpoint{5.405495in}{1.393142in}}%
\pgfpathclose%
\pgfusepath{stroke,fill}%
\end{pgfscope}%
\begin{pgfscope}%
\pgfpathrectangle{\pgfqpoint{4.985294in}{0.500000in}}{\pgfqpoint{1.764706in}{1.700000in}}%
\pgfusepath{clip}%
\pgfsetbuttcap%
\pgfsetroundjoin%
\definecolor{currentfill}{rgb}{0.937528,0.344792,0.251999}%
\pgfsetfillcolor{currentfill}%
\pgfsetlinewidth{0.311001pt}%
\definecolor{currentstroke}{rgb}{1.000000,1.000000,1.000000}%
\pgfsetstrokecolor{currentstroke}%
\pgfsetdash{}{0pt}%
\pgfpathmoveto{\pgfqpoint{5.588822in}{1.243008in}}%
\pgfpathcurveto{\pgfqpoint{5.595955in}{1.243008in}}{\pgfqpoint{5.602796in}{1.245842in}}{\pgfqpoint{5.607840in}{1.250886in}}%
\pgfpathcurveto{\pgfqpoint{5.612884in}{1.255930in}}{\pgfqpoint{5.615718in}{1.262771in}}{\pgfqpoint{5.615718in}{1.269904in}}%
\pgfpathcurveto{\pgfqpoint{5.615718in}{1.277037in}}{\pgfqpoint{5.612884in}{1.283879in}}{\pgfqpoint{5.607840in}{1.288922in}}%
\pgfpathcurveto{\pgfqpoint{5.602796in}{1.293966in}}{\pgfqpoint{5.595955in}{1.296800in}}{\pgfqpoint{5.588822in}{1.296800in}}%
\pgfpathcurveto{\pgfqpoint{5.581689in}{1.296800in}}{\pgfqpoint{5.574847in}{1.293966in}}{\pgfqpoint{5.569804in}{1.288922in}}%
\pgfpathcurveto{\pgfqpoint{5.564760in}{1.283879in}}{\pgfqpoint{5.561926in}{1.277037in}}{\pgfqpoint{5.561926in}{1.269904in}}%
\pgfpathcurveto{\pgfqpoint{5.561926in}{1.262771in}}{\pgfqpoint{5.564760in}{1.255930in}}{\pgfqpoint{5.569804in}{1.250886in}}%
\pgfpathcurveto{\pgfqpoint{5.574847in}{1.245842in}}{\pgfqpoint{5.581689in}{1.243008in}}{\pgfqpoint{5.588822in}{1.243008in}}%
\pgfpathclose%
\pgfusepath{stroke,fill}%
\end{pgfscope}%
\begin{pgfscope}%
\pgfpathrectangle{\pgfqpoint{4.985294in}{0.500000in}}{\pgfqpoint{1.764706in}{1.700000in}}%
\pgfusepath{clip}%
\pgfsetbuttcap%
\pgfsetroundjoin%
\definecolor{currentfill}{rgb}{0.976961,0.885681,0.814303}%
\pgfsetfillcolor{currentfill}%
\pgfsetlinewidth{0.311001pt}%
\definecolor{currentstroke}{rgb}{1.000000,1.000000,1.000000}%
\pgfsetstrokecolor{currentstroke}%
\pgfsetdash{}{0pt}%
\pgfpathmoveto{\pgfqpoint{5.449972in}{1.510542in}}%
\pgfpathcurveto{\pgfqpoint{5.457105in}{1.510542in}}{\pgfqpoint{5.463947in}{1.513376in}}{\pgfqpoint{5.468991in}{1.518420in}}%
\pgfpathcurveto{\pgfqpoint{5.474034in}{1.523463in}}{\pgfqpoint{5.476868in}{1.530305in}}{\pgfqpoint{5.476868in}{1.537438in}}%
\pgfpathcurveto{\pgfqpoint{5.476868in}{1.544571in}}{\pgfqpoint{5.474034in}{1.551412in}}{\pgfqpoint{5.468991in}{1.556456in}}%
\pgfpathcurveto{\pgfqpoint{5.463947in}{1.561500in}}{\pgfqpoint{5.457105in}{1.564334in}}{\pgfqpoint{5.449972in}{1.564334in}}%
\pgfpathcurveto{\pgfqpoint{5.442840in}{1.564334in}}{\pgfqpoint{5.435998in}{1.561500in}}{\pgfqpoint{5.430954in}{1.556456in}}%
\pgfpathcurveto{\pgfqpoint{5.425911in}{1.551412in}}{\pgfqpoint{5.423077in}{1.544571in}}{\pgfqpoint{5.423077in}{1.537438in}}%
\pgfpathcurveto{\pgfqpoint{5.423077in}{1.530305in}}{\pgfqpoint{5.425911in}{1.523463in}}{\pgfqpoint{5.430954in}{1.518420in}}%
\pgfpathcurveto{\pgfqpoint{5.435998in}{1.513376in}}{\pgfqpoint{5.442840in}{1.510542in}}{\pgfqpoint{5.449972in}{1.510542in}}%
\pgfpathclose%
\pgfusepath{stroke,fill}%
\end{pgfscope}%
\begin{pgfscope}%
\pgfpathrectangle{\pgfqpoint{4.985294in}{0.500000in}}{\pgfqpoint{1.764706in}{1.700000in}}%
\pgfusepath{clip}%
\pgfsetbuttcap%
\pgfsetroundjoin%
\definecolor{currentfill}{rgb}{0.908486,0.245685,0.245983}%
\pgfsetfillcolor{currentfill}%
\pgfsetlinewidth{0.311001pt}%
\definecolor{currentstroke}{rgb}{1.000000,1.000000,1.000000}%
\pgfsetstrokecolor{currentstroke}%
\pgfsetdash{}{0pt}%
\pgfpathmoveto{\pgfqpoint{6.400637in}{1.684319in}}%
\pgfpathcurveto{\pgfqpoint{6.407770in}{1.684319in}}{\pgfqpoint{6.414611in}{1.687152in}}{\pgfqpoint{6.419655in}{1.692196in}}%
\pgfpathcurveto{\pgfqpoint{6.424699in}{1.697240in}}{\pgfqpoint{6.427532in}{1.704081in}}{\pgfqpoint{6.427532in}{1.711214in}}%
\pgfpathcurveto{\pgfqpoint{6.427532in}{1.718347in}}{\pgfqpoint{6.424699in}{1.725189in}}{\pgfqpoint{6.419655in}{1.730232in}}%
\pgfpathcurveto{\pgfqpoint{6.414611in}{1.735276in}}{\pgfqpoint{6.407770in}{1.738110in}}{\pgfqpoint{6.400637in}{1.738110in}}%
\pgfpathcurveto{\pgfqpoint{6.393504in}{1.738110in}}{\pgfqpoint{6.386662in}{1.735276in}}{\pgfqpoint{6.381619in}{1.730232in}}%
\pgfpathcurveto{\pgfqpoint{6.376575in}{1.725189in}}{\pgfqpoint{6.373741in}{1.718347in}}{\pgfqpoint{6.373741in}{1.711214in}}%
\pgfpathcurveto{\pgfqpoint{6.373741in}{1.704081in}}{\pgfqpoint{6.376575in}{1.697240in}}{\pgfqpoint{6.381619in}{1.692196in}}%
\pgfpathcurveto{\pgfqpoint{6.386662in}{1.687152in}}{\pgfqpoint{6.393504in}{1.684319in}}{\pgfqpoint{6.400637in}{1.684319in}}%
\pgfpathclose%
\pgfusepath{stroke,fill}%
\end{pgfscope}%
\begin{pgfscope}%
\pgfpathrectangle{\pgfqpoint{4.985294in}{0.500000in}}{\pgfqpoint{1.764706in}{1.700000in}}%
\pgfusepath{clip}%
\pgfsetbuttcap%
\pgfsetroundjoin%
\definecolor{currentfill}{rgb}{0.973271,0.850724,0.762998}%
\pgfsetfillcolor{currentfill}%
\pgfsetlinewidth{0.311001pt}%
\definecolor{currentstroke}{rgb}{1.000000,1.000000,1.000000}%
\pgfsetstrokecolor{currentstroke}%
\pgfsetdash{}{0pt}%
\pgfpathmoveto{\pgfqpoint{6.305512in}{1.582406in}}%
\pgfpathcurveto{\pgfqpoint{6.312644in}{1.582406in}}{\pgfqpoint{6.319486in}{1.585240in}}{\pgfqpoint{6.324530in}{1.590284in}}%
\pgfpathcurveto{\pgfqpoint{6.329573in}{1.595327in}}{\pgfqpoint{6.332407in}{1.602169in}}{\pgfqpoint{6.332407in}{1.609302in}}%
\pgfpathcurveto{\pgfqpoint{6.332407in}{1.616435in}}{\pgfqpoint{6.329573in}{1.623276in}}{\pgfqpoint{6.324530in}{1.628320in}}%
\pgfpathcurveto{\pgfqpoint{6.319486in}{1.633364in}}{\pgfqpoint{6.312644in}{1.636198in}}{\pgfqpoint{6.305512in}{1.636198in}}%
\pgfpathcurveto{\pgfqpoint{6.298379in}{1.636198in}}{\pgfqpoint{6.291537in}{1.633364in}}{\pgfqpoint{6.286493in}{1.628320in}}%
\pgfpathcurveto{\pgfqpoint{6.281450in}{1.623276in}}{\pgfqpoint{6.278616in}{1.616435in}}{\pgfqpoint{6.278616in}{1.609302in}}%
\pgfpathcurveto{\pgfqpoint{6.278616in}{1.602169in}}{\pgfqpoint{6.281450in}{1.595327in}}{\pgfqpoint{6.286493in}{1.590284in}}%
\pgfpathcurveto{\pgfqpoint{6.291537in}{1.585240in}}{\pgfqpoint{6.298379in}{1.582406in}}{\pgfqpoint{6.305512in}{1.582406in}}%
\pgfpathclose%
\pgfusepath{stroke,fill}%
\end{pgfscope}%
\begin{pgfscope}%
\pgfpathrectangle{\pgfqpoint{4.985294in}{0.500000in}}{\pgfqpoint{1.764706in}{1.700000in}}%
\pgfusepath{clip}%
\pgfsetbuttcap%
\pgfsetroundjoin%
\definecolor{currentfill}{rgb}{0.972201,0.839051,0.745789}%
\pgfsetfillcolor{currentfill}%
\pgfsetlinewidth{0.311001pt}%
\definecolor{currentstroke}{rgb}{1.000000,1.000000,1.000000}%
\pgfsetstrokecolor{currentstroke}%
\pgfsetdash{}{0pt}%
\pgfpathmoveto{\pgfqpoint{5.359223in}{1.300707in}}%
\pgfpathcurveto{\pgfqpoint{5.366356in}{1.300707in}}{\pgfqpoint{5.373198in}{1.303541in}}{\pgfqpoint{5.378241in}{1.308585in}}%
\pgfpathcurveto{\pgfqpoint{5.383285in}{1.313628in}}{\pgfqpoint{5.386119in}{1.320470in}}{\pgfqpoint{5.386119in}{1.327603in}}%
\pgfpathcurveto{\pgfqpoint{5.386119in}{1.334736in}}{\pgfqpoint{5.383285in}{1.341577in}}{\pgfqpoint{5.378241in}{1.346621in}}%
\pgfpathcurveto{\pgfqpoint{5.373198in}{1.351665in}}{\pgfqpoint{5.366356in}{1.354498in}}{\pgfqpoint{5.359223in}{1.354498in}}%
\pgfpathcurveto{\pgfqpoint{5.352090in}{1.354498in}}{\pgfqpoint{5.345249in}{1.351665in}}{\pgfqpoint{5.340205in}{1.346621in}}%
\pgfpathcurveto{\pgfqpoint{5.335161in}{1.341577in}}{\pgfqpoint{5.332327in}{1.334736in}}{\pgfqpoint{5.332327in}{1.327603in}}%
\pgfpathcurveto{\pgfqpoint{5.332327in}{1.320470in}}{\pgfqpoint{5.335161in}{1.313628in}}{\pgfqpoint{5.340205in}{1.308585in}}%
\pgfpathcurveto{\pgfqpoint{5.345249in}{1.303541in}}{\pgfqpoint{5.352090in}{1.300707in}}{\pgfqpoint{5.359223in}{1.300707in}}%
\pgfpathclose%
\pgfusepath{stroke,fill}%
\end{pgfscope}%
\begin{pgfscope}%
\pgfpathrectangle{\pgfqpoint{4.985294in}{0.500000in}}{\pgfqpoint{1.764706in}{1.700000in}}%
\pgfusepath{clip}%
\pgfsetbuttcap%
\pgfsetroundjoin%
\definecolor{currentfill}{rgb}{0.963884,0.644842,0.486120}%
\pgfsetfillcolor{currentfill}%
\pgfsetlinewidth{0.311001pt}%
\definecolor{currentstroke}{rgb}{1.000000,1.000000,1.000000}%
\pgfsetstrokecolor{currentstroke}%
\pgfsetdash{}{0pt}%
\pgfpathmoveto{\pgfqpoint{5.326666in}{1.173292in}}%
\pgfpathcurveto{\pgfqpoint{5.333799in}{1.173292in}}{\pgfqpoint{5.340641in}{1.176126in}}{\pgfqpoint{5.345685in}{1.181170in}}%
\pgfpathcurveto{\pgfqpoint{5.350728in}{1.186213in}}{\pgfqpoint{5.353562in}{1.193055in}}{\pgfqpoint{5.353562in}{1.200188in}}%
\pgfpathcurveto{\pgfqpoint{5.353562in}{1.207321in}}{\pgfqpoint{5.350728in}{1.214162in}}{\pgfqpoint{5.345685in}{1.219206in}}%
\pgfpathcurveto{\pgfqpoint{5.340641in}{1.224250in}}{\pgfqpoint{5.333799in}{1.227084in}}{\pgfqpoint{5.326666in}{1.227084in}}%
\pgfpathcurveto{\pgfqpoint{5.319534in}{1.227084in}}{\pgfqpoint{5.312692in}{1.224250in}}{\pgfqpoint{5.307648in}{1.219206in}}%
\pgfpathcurveto{\pgfqpoint{5.302605in}{1.214162in}}{\pgfqpoint{5.299771in}{1.207321in}}{\pgfqpoint{5.299771in}{1.200188in}}%
\pgfpathcurveto{\pgfqpoint{5.299771in}{1.193055in}}{\pgfqpoint{5.302605in}{1.186213in}}{\pgfqpoint{5.307648in}{1.181170in}}%
\pgfpathcurveto{\pgfqpoint{5.312692in}{1.176126in}}{\pgfqpoint{5.319534in}{1.173292in}}{\pgfqpoint{5.326666in}{1.173292in}}%
\pgfpathclose%
\pgfusepath{stroke,fill}%
\end{pgfscope}%
\begin{pgfscope}%
\pgfpathrectangle{\pgfqpoint{4.985294in}{0.500000in}}{\pgfqpoint{1.764706in}{1.700000in}}%
\pgfusepath{clip}%
\pgfsetbuttcap%
\pgfsetroundjoin%
\definecolor{currentfill}{rgb}{0.965753,0.732351,0.592427}%
\pgfsetfillcolor{currentfill}%
\pgfsetlinewidth{0.311001pt}%
\definecolor{currentstroke}{rgb}{1.000000,1.000000,1.000000}%
\pgfsetstrokecolor{currentstroke}%
\pgfsetdash{}{0pt}%
\pgfpathmoveto{\pgfqpoint{6.202406in}{1.414745in}}%
\pgfpathcurveto{\pgfqpoint{6.209539in}{1.414745in}}{\pgfqpoint{6.216380in}{1.417579in}}{\pgfqpoint{6.221424in}{1.422622in}}%
\pgfpathcurveto{\pgfqpoint{6.226468in}{1.427666in}}{\pgfqpoint{6.229302in}{1.434508in}}{\pgfqpoint{6.229302in}{1.441640in}}%
\pgfpathcurveto{\pgfqpoint{6.229302in}{1.448773in}}{\pgfqpoint{6.226468in}{1.455615in}}{\pgfqpoint{6.221424in}{1.460658in}}%
\pgfpathcurveto{\pgfqpoint{6.216380in}{1.465702in}}{\pgfqpoint{6.209539in}{1.468536in}}{\pgfqpoint{6.202406in}{1.468536in}}%
\pgfpathcurveto{\pgfqpoint{6.195273in}{1.468536in}}{\pgfqpoint{6.188432in}{1.465702in}}{\pgfqpoint{6.183388in}{1.460658in}}%
\pgfpathcurveto{\pgfqpoint{6.178344in}{1.455615in}}{\pgfqpoint{6.175510in}{1.448773in}}{\pgfqpoint{6.175510in}{1.441640in}}%
\pgfpathcurveto{\pgfqpoint{6.175510in}{1.434508in}}{\pgfqpoint{6.178344in}{1.427666in}}{\pgfqpoint{6.183388in}{1.422622in}}%
\pgfpathcurveto{\pgfqpoint{6.188432in}{1.417579in}}{\pgfqpoint{6.195273in}{1.414745in}}{\pgfqpoint{6.202406in}{1.414745in}}%
\pgfpathclose%
\pgfusepath{stroke,fill}%
\end{pgfscope}%
\begin{pgfscope}%
\pgfpathrectangle{\pgfqpoint{4.985294in}{0.500000in}}{\pgfqpoint{1.764706in}{1.700000in}}%
\pgfusepath{clip}%
\pgfsetbuttcap%
\pgfsetroundjoin%
\definecolor{currentfill}{rgb}{0.978376,0.897317,0.831308}%
\pgfsetfillcolor{currentfill}%
\pgfsetlinewidth{0.311001pt}%
\definecolor{currentstroke}{rgb}{1.000000,1.000000,1.000000}%
\pgfsetstrokecolor{currentstroke}%
\pgfsetdash{}{0pt}%
\pgfpathmoveto{\pgfqpoint{5.421189in}{1.185948in}}%
\pgfpathcurveto{\pgfqpoint{5.428322in}{1.185948in}}{\pgfqpoint{5.435164in}{1.188782in}}{\pgfqpoint{5.440207in}{1.193826in}}%
\pgfpathcurveto{\pgfqpoint{5.445251in}{1.198869in}}{\pgfqpoint{5.448085in}{1.205711in}}{\pgfqpoint{5.448085in}{1.212844in}}%
\pgfpathcurveto{\pgfqpoint{5.448085in}{1.219977in}}{\pgfqpoint{5.445251in}{1.226818in}}{\pgfqpoint{5.440207in}{1.231862in}}%
\pgfpathcurveto{\pgfqpoint{5.435164in}{1.236906in}}{\pgfqpoint{5.428322in}{1.239739in}}{\pgfqpoint{5.421189in}{1.239739in}}%
\pgfpathcurveto{\pgfqpoint{5.414056in}{1.239739in}}{\pgfqpoint{5.407215in}{1.236906in}}{\pgfqpoint{5.402171in}{1.231862in}}%
\pgfpathcurveto{\pgfqpoint{5.397128in}{1.226818in}}{\pgfqpoint{5.394294in}{1.219977in}}{\pgfqpoint{5.394294in}{1.212844in}}%
\pgfpathcurveto{\pgfqpoint{5.394294in}{1.205711in}}{\pgfqpoint{5.397128in}{1.198869in}}{\pgfqpoint{5.402171in}{1.193826in}}%
\pgfpathcurveto{\pgfqpoint{5.407215in}{1.188782in}}{\pgfqpoint{5.414056in}{1.185948in}}{\pgfqpoint{5.421189in}{1.185948in}}%
\pgfpathclose%
\pgfusepath{stroke,fill}%
\end{pgfscope}%
\begin{pgfscope}%
\pgfpathrectangle{\pgfqpoint{4.985294in}{0.500000in}}{\pgfqpoint{1.764706in}{1.700000in}}%
\pgfusepath{clip}%
\pgfsetbuttcap%
\pgfsetroundjoin%
\definecolor{currentfill}{rgb}{0.971202,0.827364,0.728520}%
\pgfsetfillcolor{currentfill}%
\pgfsetlinewidth{0.311001pt}%
\definecolor{currentstroke}{rgb}{1.000000,1.000000,1.000000}%
\pgfsetstrokecolor{currentstroke}%
\pgfsetdash{}{0pt}%
\pgfpathmoveto{\pgfqpoint{6.203443in}{1.552930in}}%
\pgfpathcurveto{\pgfqpoint{6.210575in}{1.552930in}}{\pgfqpoint{6.217417in}{1.555764in}}{\pgfqpoint{6.222461in}{1.560808in}}%
\pgfpathcurveto{\pgfqpoint{6.227504in}{1.565852in}}{\pgfqpoint{6.230338in}{1.572693in}}{\pgfqpoint{6.230338in}{1.579826in}}%
\pgfpathcurveto{\pgfqpoint{6.230338in}{1.586959in}}{\pgfqpoint{6.227504in}{1.593801in}}{\pgfqpoint{6.222461in}{1.598844in}}%
\pgfpathcurveto{\pgfqpoint{6.217417in}{1.603888in}}{\pgfqpoint{6.210575in}{1.606722in}}{\pgfqpoint{6.203443in}{1.606722in}}%
\pgfpathcurveto{\pgfqpoint{6.196310in}{1.606722in}}{\pgfqpoint{6.189468in}{1.603888in}}{\pgfqpoint{6.184425in}{1.598844in}}%
\pgfpathcurveto{\pgfqpoint{6.179381in}{1.593801in}}{\pgfqpoint{6.176547in}{1.586959in}}{\pgfqpoint{6.176547in}{1.579826in}}%
\pgfpathcurveto{\pgfqpoint{6.176547in}{1.572693in}}{\pgfqpoint{6.179381in}{1.565852in}}{\pgfqpoint{6.184425in}{1.560808in}}%
\pgfpathcurveto{\pgfqpoint{6.189468in}{1.555764in}}{\pgfqpoint{6.196310in}{1.552930in}}{\pgfqpoint{6.203443in}{1.552930in}}%
\pgfpathclose%
\pgfusepath{stroke,fill}%
\end{pgfscope}%
\begin{pgfscope}%
\pgfpathrectangle{\pgfqpoint{4.985294in}{0.500000in}}{\pgfqpoint{1.764706in}{1.700000in}}%
\pgfusepath{clip}%
\pgfsetbuttcap%
\pgfsetroundjoin%
\definecolor{currentfill}{rgb}{0.550643,0.114208,0.357379}%
\pgfsetfillcolor{currentfill}%
\pgfsetlinewidth{0.311001pt}%
\definecolor{currentstroke}{rgb}{1.000000,1.000000,1.000000}%
\pgfsetstrokecolor{currentstroke}%
\pgfsetdash{}{0pt}%
\pgfpathmoveto{\pgfqpoint{5.949719in}{0.877492in}}%
\pgfpathcurveto{\pgfqpoint{5.956852in}{0.877492in}}{\pgfqpoint{5.963693in}{0.880326in}}{\pgfqpoint{5.968737in}{0.885369in}}%
\pgfpathcurveto{\pgfqpoint{5.973781in}{0.890413in}}{\pgfqpoint{5.976614in}{0.897255in}}{\pgfqpoint{5.976614in}{0.904387in}}%
\pgfpathcurveto{\pgfqpoint{5.976614in}{0.911520in}}{\pgfqpoint{5.973781in}{0.918362in}}{\pgfqpoint{5.968737in}{0.923406in}}%
\pgfpathcurveto{\pgfqpoint{5.963693in}{0.928449in}}{\pgfqpoint{5.956852in}{0.931283in}}{\pgfqpoint{5.949719in}{0.931283in}}%
\pgfpathcurveto{\pgfqpoint{5.942586in}{0.931283in}}{\pgfqpoint{5.935744in}{0.928449in}}{\pgfqpoint{5.930701in}{0.923406in}}%
\pgfpathcurveto{\pgfqpoint{5.925657in}{0.918362in}}{\pgfqpoint{5.922823in}{0.911520in}}{\pgfqpoint{5.922823in}{0.904387in}}%
\pgfpathcurveto{\pgfqpoint{5.922823in}{0.897255in}}{\pgfqpoint{5.925657in}{0.890413in}}{\pgfqpoint{5.930701in}{0.885369in}}%
\pgfpathcurveto{\pgfqpoint{5.935744in}{0.880326in}}{\pgfqpoint{5.942586in}{0.877492in}}{\pgfqpoint{5.949719in}{0.877492in}}%
\pgfpathclose%
\pgfusepath{stroke,fill}%
\end{pgfscope}%
\begin{pgfscope}%
\pgfpathrectangle{\pgfqpoint{4.985294in}{0.500000in}}{\pgfqpoint{1.764706in}{1.700000in}}%
\pgfusepath{clip}%
\pgfsetbuttcap%
\pgfsetroundjoin%
\definecolor{currentfill}{rgb}{0.975644,0.874038,0.797253}%
\pgfsetfillcolor{currentfill}%
\pgfsetlinewidth{0.311001pt}%
\definecolor{currentstroke}{rgb}{1.000000,1.000000,1.000000}%
\pgfsetstrokecolor{currentstroke}%
\pgfsetdash{}{0pt}%
\pgfpathmoveto{\pgfqpoint{5.458226in}{1.547007in}}%
\pgfpathcurveto{\pgfqpoint{5.465359in}{1.547007in}}{\pgfqpoint{5.472200in}{1.549841in}}{\pgfqpoint{5.477244in}{1.554885in}}%
\pgfpathcurveto{\pgfqpoint{5.482288in}{1.559929in}}{\pgfqpoint{5.485122in}{1.566770in}}{\pgfqpoint{5.485122in}{1.573903in}}%
\pgfpathcurveto{\pgfqpoint{5.485122in}{1.581036in}}{\pgfqpoint{5.482288in}{1.587877in}}{\pgfqpoint{5.477244in}{1.592921in}}%
\pgfpathcurveto{\pgfqpoint{5.472200in}{1.597965in}}{\pgfqpoint{5.465359in}{1.600799in}}{\pgfqpoint{5.458226in}{1.600799in}}%
\pgfpathcurveto{\pgfqpoint{5.451093in}{1.600799in}}{\pgfqpoint{5.444251in}{1.597965in}}{\pgfqpoint{5.439208in}{1.592921in}}%
\pgfpathcurveto{\pgfqpoint{5.434164in}{1.587877in}}{\pgfqpoint{5.431330in}{1.581036in}}{\pgfqpoint{5.431330in}{1.573903in}}%
\pgfpathcurveto{\pgfqpoint{5.431330in}{1.566770in}}{\pgfqpoint{5.434164in}{1.559929in}}{\pgfqpoint{5.439208in}{1.554885in}}%
\pgfpathcurveto{\pgfqpoint{5.444251in}{1.549841in}}{\pgfqpoint{5.451093in}{1.547007in}}{\pgfqpoint{5.458226in}{1.547007in}}%
\pgfpathclose%
\pgfusepath{stroke,fill}%
\end{pgfscope}%
\begin{pgfscope}%
\pgfpathrectangle{\pgfqpoint{4.985294in}{0.500000in}}{\pgfqpoint{1.764706in}{1.700000in}}%
\pgfusepath{clip}%
\pgfsetbuttcap%
\pgfsetroundjoin%
\definecolor{currentfill}{rgb}{0.960421,0.553286,0.393191}%
\pgfsetfillcolor{currentfill}%
\pgfsetlinewidth{0.311001pt}%
\definecolor{currentstroke}{rgb}{1.000000,1.000000,1.000000}%
\pgfsetstrokecolor{currentstroke}%
\pgfsetdash{}{0pt}%
\pgfpathmoveto{\pgfqpoint{6.113747in}{1.071955in}}%
\pgfpathcurveto{\pgfqpoint{6.120880in}{1.071955in}}{\pgfqpoint{6.127721in}{1.074789in}}{\pgfqpoint{6.132765in}{1.079833in}}%
\pgfpathcurveto{\pgfqpoint{6.137808in}{1.084876in}}{\pgfqpoint{6.140642in}{1.091718in}}{\pgfqpoint{6.140642in}{1.098851in}}%
\pgfpathcurveto{\pgfqpoint{6.140642in}{1.105984in}}{\pgfqpoint{6.137808in}{1.112825in}}{\pgfqpoint{6.132765in}{1.117869in}}%
\pgfpathcurveto{\pgfqpoint{6.127721in}{1.122913in}}{\pgfqpoint{6.120880in}{1.125747in}}{\pgfqpoint{6.113747in}{1.125747in}}%
\pgfpathcurveto{\pgfqpoint{6.106614in}{1.125747in}}{\pgfqpoint{6.099772in}{1.122913in}}{\pgfqpoint{6.094729in}{1.117869in}}%
\pgfpathcurveto{\pgfqpoint{6.089685in}{1.112825in}}{\pgfqpoint{6.086851in}{1.105984in}}{\pgfqpoint{6.086851in}{1.098851in}}%
\pgfpathcurveto{\pgfqpoint{6.086851in}{1.091718in}}{\pgfqpoint{6.089685in}{1.084876in}}{\pgfqpoint{6.094729in}{1.079833in}}%
\pgfpathcurveto{\pgfqpoint{6.099772in}{1.074789in}}{\pgfqpoint{6.106614in}{1.071955in}}{\pgfqpoint{6.113747in}{1.071955in}}%
\pgfpathclose%
\pgfusepath{stroke,fill}%
\end{pgfscope}%
\begin{pgfscope}%
\pgfpathrectangle{\pgfqpoint{4.985294in}{0.500000in}}{\pgfqpoint{1.764706in}{1.700000in}}%
\pgfusepath{clip}%
\pgfsetbuttcap%
\pgfsetroundjoin%
\definecolor{currentfill}{rgb}{0.964920,0.695342,0.545192}%
\pgfsetfillcolor{currentfill}%
\pgfsetlinewidth{0.311001pt}%
\definecolor{currentstroke}{rgb}{1.000000,1.000000,1.000000}%
\pgfsetstrokecolor{currentstroke}%
\pgfsetdash{}{0pt}%
\pgfpathmoveto{\pgfqpoint{6.398303in}{1.440018in}}%
\pgfpathcurveto{\pgfqpoint{6.405436in}{1.440018in}}{\pgfqpoint{6.412278in}{1.442851in}}{\pgfqpoint{6.417321in}{1.447895in}}%
\pgfpathcurveto{\pgfqpoint{6.422365in}{1.452939in}}{\pgfqpoint{6.425199in}{1.459780in}}{\pgfqpoint{6.425199in}{1.466913in}}%
\pgfpathcurveto{\pgfqpoint{6.425199in}{1.474046in}}{\pgfqpoint{6.422365in}{1.480888in}}{\pgfqpoint{6.417321in}{1.485931in}}%
\pgfpathcurveto{\pgfqpoint{6.412278in}{1.490975in}}{\pgfqpoint{6.405436in}{1.493809in}}{\pgfqpoint{6.398303in}{1.493809in}}%
\pgfpathcurveto{\pgfqpoint{6.391170in}{1.493809in}}{\pgfqpoint{6.384329in}{1.490975in}}{\pgfqpoint{6.379285in}{1.485931in}}%
\pgfpathcurveto{\pgfqpoint{6.374241in}{1.480888in}}{\pgfqpoint{6.371407in}{1.474046in}}{\pgfqpoint{6.371407in}{1.466913in}}%
\pgfpathcurveto{\pgfqpoint{6.371407in}{1.459780in}}{\pgfqpoint{6.374241in}{1.452939in}}{\pgfqpoint{6.379285in}{1.447895in}}%
\pgfpathcurveto{\pgfqpoint{6.384329in}{1.442851in}}{\pgfqpoint{6.391170in}{1.440018in}}{\pgfqpoint{6.398303in}{1.440018in}}%
\pgfpathclose%
\pgfusepath{stroke,fill}%
\end{pgfscope}%
\begin{pgfscope}%
\pgfpathrectangle{\pgfqpoint{4.985294in}{0.500000in}}{\pgfqpoint{1.764706in}{1.700000in}}%
\pgfusepath{clip}%
\pgfsetbuttcap%
\pgfsetroundjoin%
\definecolor{currentfill}{rgb}{0.975018,0.868213,0.788710}%
\pgfsetfillcolor{currentfill}%
\pgfsetlinewidth{0.311001pt}%
\definecolor{currentstroke}{rgb}{1.000000,1.000000,1.000000}%
\pgfsetstrokecolor{currentstroke}%
\pgfsetdash{}{0pt}%
\pgfpathmoveto{\pgfqpoint{5.373150in}{1.320378in}}%
\pgfpathcurveto{\pgfqpoint{5.380283in}{1.320378in}}{\pgfqpoint{5.387124in}{1.323212in}}{\pgfqpoint{5.392168in}{1.328256in}}%
\pgfpathcurveto{\pgfqpoint{5.397212in}{1.333299in}}{\pgfqpoint{5.400046in}{1.340141in}}{\pgfqpoint{5.400046in}{1.347274in}}%
\pgfpathcurveto{\pgfqpoint{5.400046in}{1.354406in}}{\pgfqpoint{5.397212in}{1.361248in}}{\pgfqpoint{5.392168in}{1.366292in}}%
\pgfpathcurveto{\pgfqpoint{5.387124in}{1.371335in}}{\pgfqpoint{5.380283in}{1.374169in}}{\pgfqpoint{5.373150in}{1.374169in}}%
\pgfpathcurveto{\pgfqpoint{5.366017in}{1.374169in}}{\pgfqpoint{5.359175in}{1.371335in}}{\pgfqpoint{5.354132in}{1.366292in}}%
\pgfpathcurveto{\pgfqpoint{5.349088in}{1.361248in}}{\pgfqpoint{5.346254in}{1.354406in}}{\pgfqpoint{5.346254in}{1.347274in}}%
\pgfpathcurveto{\pgfqpoint{5.346254in}{1.340141in}}{\pgfqpoint{5.349088in}{1.333299in}}{\pgfqpoint{5.354132in}{1.328256in}}%
\pgfpathcurveto{\pgfqpoint{5.359175in}{1.323212in}}{\pgfqpoint{5.366017in}{1.320378in}}{\pgfqpoint{5.373150in}{1.320378in}}%
\pgfpathclose%
\pgfusepath{stroke,fill}%
\end{pgfscope}%
\begin{pgfscope}%
\pgfpathrectangle{\pgfqpoint{4.985294in}{0.500000in}}{\pgfqpoint{1.764706in}{1.700000in}}%
\pgfusepath{clip}%
\pgfsetbuttcap%
\pgfsetroundjoin%
\definecolor{currentfill}{rgb}{0.977657,0.891500,0.822809}%
\pgfsetfillcolor{currentfill}%
\pgfsetlinewidth{0.311001pt}%
\definecolor{currentstroke}{rgb}{1.000000,1.000000,1.000000}%
\pgfsetstrokecolor{currentstroke}%
\pgfsetdash{}{0pt}%
\pgfpathmoveto{\pgfqpoint{5.447530in}{1.475774in}}%
\pgfpathcurveto{\pgfqpoint{5.454663in}{1.475774in}}{\pgfqpoint{5.461505in}{1.478608in}}{\pgfqpoint{5.466549in}{1.483652in}}%
\pgfpathcurveto{\pgfqpoint{5.471592in}{1.488696in}}{\pgfqpoint{5.474426in}{1.495537in}}{\pgfqpoint{5.474426in}{1.502670in}}%
\pgfpathcurveto{\pgfqpoint{5.474426in}{1.509803in}}{\pgfqpoint{5.471592in}{1.516645in}}{\pgfqpoint{5.466549in}{1.521688in}}%
\pgfpathcurveto{\pgfqpoint{5.461505in}{1.526732in}}{\pgfqpoint{5.454663in}{1.529566in}}{\pgfqpoint{5.447530in}{1.529566in}}%
\pgfpathcurveto{\pgfqpoint{5.440398in}{1.529566in}}{\pgfqpoint{5.433556in}{1.526732in}}{\pgfqpoint{5.428512in}{1.521688in}}%
\pgfpathcurveto{\pgfqpoint{5.423469in}{1.516645in}}{\pgfqpoint{5.420635in}{1.509803in}}{\pgfqpoint{5.420635in}{1.502670in}}%
\pgfpathcurveto{\pgfqpoint{5.420635in}{1.495537in}}{\pgfqpoint{5.423469in}{1.488696in}}{\pgfqpoint{5.428512in}{1.483652in}}%
\pgfpathcurveto{\pgfqpoint{5.433556in}{1.478608in}}{\pgfqpoint{5.440398in}{1.475774in}}{\pgfqpoint{5.447530in}{1.475774in}}%
\pgfpathclose%
\pgfusepath{stroke,fill}%
\end{pgfscope}%
\begin{pgfscope}%
\pgfpathrectangle{\pgfqpoint{4.985294in}{0.500000in}}{\pgfqpoint{1.764706in}{1.700000in}}%
\pgfusepath{clip}%
\pgfsetbuttcap%
\pgfsetroundjoin%
\definecolor{currentfill}{rgb}{0.940366,0.360209,0.257347}%
\pgfsetfillcolor{currentfill}%
\pgfsetlinewidth{0.311001pt}%
\definecolor{currentstroke}{rgb}{1.000000,1.000000,1.000000}%
\pgfsetstrokecolor{currentstroke}%
\pgfsetdash{}{0pt}%
\pgfpathmoveto{\pgfqpoint{5.569133in}{1.821031in}}%
\pgfpathcurveto{\pgfqpoint{5.576266in}{1.821031in}}{\pgfqpoint{5.583107in}{1.823865in}}{\pgfqpoint{5.588151in}{1.828909in}}%
\pgfpathcurveto{\pgfqpoint{5.593194in}{1.833952in}}{\pgfqpoint{5.596028in}{1.840794in}}{\pgfqpoint{5.596028in}{1.847927in}}%
\pgfpathcurveto{\pgfqpoint{5.596028in}{1.855059in}}{\pgfqpoint{5.593194in}{1.861901in}}{\pgfqpoint{5.588151in}{1.866945in}}%
\pgfpathcurveto{\pgfqpoint{5.583107in}{1.871988in}}{\pgfqpoint{5.576266in}{1.874822in}}{\pgfqpoint{5.569133in}{1.874822in}}%
\pgfpathcurveto{\pgfqpoint{5.562000in}{1.874822in}}{\pgfqpoint{5.555158in}{1.871988in}}{\pgfqpoint{5.550115in}{1.866945in}}%
\pgfpathcurveto{\pgfqpoint{5.545071in}{1.861901in}}{\pgfqpoint{5.542237in}{1.855059in}}{\pgfqpoint{5.542237in}{1.847927in}}%
\pgfpathcurveto{\pgfqpoint{5.542237in}{1.840794in}}{\pgfqpoint{5.545071in}{1.833952in}}{\pgfqpoint{5.550115in}{1.828909in}}%
\pgfpathcurveto{\pgfqpoint{5.555158in}{1.823865in}}{\pgfqpoint{5.562000in}{1.821031in}}{\pgfqpoint{5.569133in}{1.821031in}}%
\pgfpathclose%
\pgfusepath{stroke,fill}%
\end{pgfscope}%
\begin{pgfscope}%
\pgfpathrectangle{\pgfqpoint{4.985294in}{0.500000in}}{\pgfqpoint{1.764706in}{1.700000in}}%
\pgfusepath{clip}%
\pgfsetbuttcap%
\pgfsetroundjoin%
\definecolor{currentfill}{rgb}{0.972201,0.839051,0.745789}%
\pgfsetfillcolor{currentfill}%
\pgfsetlinewidth{0.311001pt}%
\definecolor{currentstroke}{rgb}{1.000000,1.000000,1.000000}%
\pgfsetstrokecolor{currentstroke}%
\pgfsetdash{}{0pt}%
\pgfpathmoveto{\pgfqpoint{5.374162in}{1.437717in}}%
\pgfpathcurveto{\pgfqpoint{5.381295in}{1.437717in}}{\pgfqpoint{5.388137in}{1.440551in}}{\pgfqpoint{5.393181in}{1.445595in}}%
\pgfpathcurveto{\pgfqpoint{5.398224in}{1.450638in}}{\pgfqpoint{5.401058in}{1.457480in}}{\pgfqpoint{5.401058in}{1.464613in}}%
\pgfpathcurveto{\pgfqpoint{5.401058in}{1.471746in}}{\pgfqpoint{5.398224in}{1.478587in}}{\pgfqpoint{5.393181in}{1.483631in}}%
\pgfpathcurveto{\pgfqpoint{5.388137in}{1.488675in}}{\pgfqpoint{5.381295in}{1.491508in}}{\pgfqpoint{5.374162in}{1.491508in}}%
\pgfpathcurveto{\pgfqpoint{5.367030in}{1.491508in}}{\pgfqpoint{5.360188in}{1.488675in}}{\pgfqpoint{5.355144in}{1.483631in}}%
\pgfpathcurveto{\pgfqpoint{5.350101in}{1.478587in}}{\pgfqpoint{5.347267in}{1.471746in}}{\pgfqpoint{5.347267in}{1.464613in}}%
\pgfpathcurveto{\pgfqpoint{5.347267in}{1.457480in}}{\pgfqpoint{5.350101in}{1.450638in}}{\pgfqpoint{5.355144in}{1.445595in}}%
\pgfpathcurveto{\pgfqpoint{5.360188in}{1.440551in}}{\pgfqpoint{5.367030in}{1.437717in}}{\pgfqpoint{5.374162in}{1.437717in}}%
\pgfpathclose%
\pgfusepath{stroke,fill}%
\end{pgfscope}%
\begin{pgfscope}%
\pgfpathrectangle{\pgfqpoint{4.985294in}{0.500000in}}{\pgfqpoint{1.764706in}{1.700000in}}%
\pgfusepath{clip}%
\pgfsetbuttcap%
\pgfsetroundjoin%
\definecolor{currentfill}{rgb}{0.979124,0.903132,0.839793}%
\pgfsetfillcolor{currentfill}%
\pgfsetlinewidth{0.311001pt}%
\definecolor{currentstroke}{rgb}{1.000000,1.000000,1.000000}%
\pgfsetstrokecolor{currentstroke}%
\pgfsetdash{}{0pt}%
\pgfpathmoveto{\pgfqpoint{6.321003in}{1.201334in}}%
\pgfpathcurveto{\pgfqpoint{6.328136in}{1.201334in}}{\pgfqpoint{6.334978in}{1.204168in}}{\pgfqpoint{6.340021in}{1.209212in}}%
\pgfpathcurveto{\pgfqpoint{6.345065in}{1.214255in}}{\pgfqpoint{6.347899in}{1.221097in}}{\pgfqpoint{6.347899in}{1.228230in}}%
\pgfpathcurveto{\pgfqpoint{6.347899in}{1.235363in}}{\pgfqpoint{6.345065in}{1.242204in}}{\pgfqpoint{6.340021in}{1.247248in}}%
\pgfpathcurveto{\pgfqpoint{6.334978in}{1.252292in}}{\pgfqpoint{6.328136in}{1.255126in}}{\pgfqpoint{6.321003in}{1.255126in}}%
\pgfpathcurveto{\pgfqpoint{6.313870in}{1.255126in}}{\pgfqpoint{6.307029in}{1.252292in}}{\pgfqpoint{6.301985in}{1.247248in}}%
\pgfpathcurveto{\pgfqpoint{6.296941in}{1.242204in}}{\pgfqpoint{6.294108in}{1.235363in}}{\pgfqpoint{6.294108in}{1.228230in}}%
\pgfpathcurveto{\pgfqpoint{6.294108in}{1.221097in}}{\pgfqpoint{6.296941in}{1.214255in}}{\pgfqpoint{6.301985in}{1.209212in}}%
\pgfpathcurveto{\pgfqpoint{6.307029in}{1.204168in}}{\pgfqpoint{6.313870in}{1.201334in}}{\pgfqpoint{6.321003in}{1.201334in}}%
\pgfpathclose%
\pgfusepath{stroke,fill}%
\end{pgfscope}%
\begin{pgfscope}%
\pgfpathrectangle{\pgfqpoint{4.985294in}{0.500000in}}{\pgfqpoint{1.764706in}{1.700000in}}%
\pgfusepath{clip}%
\pgfsetbuttcap%
\pgfsetroundjoin%
\definecolor{currentfill}{rgb}{0.938993,0.352507,0.254528}%
\pgfsetfillcolor{currentfill}%
\pgfsetlinewidth{0.311001pt}%
\definecolor{currentstroke}{rgb}{1.000000,1.000000,1.000000}%
\pgfsetstrokecolor{currentstroke}%
\pgfsetdash{}{0pt}%
\pgfpathmoveto{\pgfqpoint{5.688361in}{0.861641in}}%
\pgfpathcurveto{\pgfqpoint{5.695494in}{0.861641in}}{\pgfqpoint{5.702336in}{0.864475in}}{\pgfqpoint{5.707380in}{0.869519in}}%
\pgfpathcurveto{\pgfqpoint{5.712423in}{0.874563in}}{\pgfqpoint{5.715257in}{0.881404in}}{\pgfqpoint{5.715257in}{0.888537in}}%
\pgfpathcurveto{\pgfqpoint{5.715257in}{0.895670in}}{\pgfqpoint{5.712423in}{0.902512in}}{\pgfqpoint{5.707380in}{0.907555in}}%
\pgfpathcurveto{\pgfqpoint{5.702336in}{0.912599in}}{\pgfqpoint{5.695494in}{0.915433in}}{\pgfqpoint{5.688361in}{0.915433in}}%
\pgfpathcurveto{\pgfqpoint{5.681229in}{0.915433in}}{\pgfqpoint{5.674387in}{0.912599in}}{\pgfqpoint{5.669343in}{0.907555in}}%
\pgfpathcurveto{\pgfqpoint{5.664300in}{0.902512in}}{\pgfqpoint{5.661466in}{0.895670in}}{\pgfqpoint{5.661466in}{0.888537in}}%
\pgfpathcurveto{\pgfqpoint{5.661466in}{0.881404in}}{\pgfqpoint{5.664300in}{0.874563in}}{\pgfqpoint{5.669343in}{0.869519in}}%
\pgfpathcurveto{\pgfqpoint{5.674387in}{0.864475in}}{\pgfqpoint{5.681229in}{0.861641in}}{\pgfqpoint{5.688361in}{0.861641in}}%
\pgfpathclose%
\pgfusepath{stroke,fill}%
\end{pgfscope}%
\begin{pgfscope}%
\pgfpathrectangle{\pgfqpoint{4.985294in}{0.500000in}}{\pgfqpoint{1.764706in}{1.700000in}}%
\pgfusepath{clip}%
\pgfsetbuttcap%
\pgfsetroundjoin%
\definecolor{currentfill}{rgb}{0.796501,0.105066,0.310630}%
\pgfsetfillcolor{currentfill}%
\pgfsetlinewidth{0.311001pt}%
\definecolor{currentstroke}{rgb}{1.000000,1.000000,1.000000}%
\pgfsetstrokecolor{currentstroke}%
\pgfsetdash{}{0pt}%
\pgfpathmoveto{\pgfqpoint{5.502368in}{1.846642in}}%
\pgfpathcurveto{\pgfqpoint{5.509501in}{1.846642in}}{\pgfqpoint{5.516343in}{1.849476in}}{\pgfqpoint{5.521387in}{1.854520in}}%
\pgfpathcurveto{\pgfqpoint{5.526430in}{1.859564in}}{\pgfqpoint{5.529264in}{1.866405in}}{\pgfqpoint{5.529264in}{1.873538in}}%
\pgfpathcurveto{\pgfqpoint{5.529264in}{1.880671in}}{\pgfqpoint{5.526430in}{1.887513in}}{\pgfqpoint{5.521387in}{1.892556in}}%
\pgfpathcurveto{\pgfqpoint{5.516343in}{1.897600in}}{\pgfqpoint{5.509501in}{1.900434in}}{\pgfqpoint{5.502368in}{1.900434in}}%
\pgfpathcurveto{\pgfqpoint{5.495236in}{1.900434in}}{\pgfqpoint{5.488394in}{1.897600in}}{\pgfqpoint{5.483350in}{1.892556in}}%
\pgfpathcurveto{\pgfqpoint{5.478307in}{1.887513in}}{\pgfqpoint{5.475473in}{1.880671in}}{\pgfqpoint{5.475473in}{1.873538in}}%
\pgfpathcurveto{\pgfqpoint{5.475473in}{1.866405in}}{\pgfqpoint{5.478307in}{1.859564in}}{\pgfqpoint{5.483350in}{1.854520in}}%
\pgfpathcurveto{\pgfqpoint{5.488394in}{1.849476in}}{\pgfqpoint{5.495236in}{1.846642in}}{\pgfqpoint{5.502368in}{1.846642in}}%
\pgfpathclose%
\pgfusepath{stroke,fill}%
\end{pgfscope}%
\begin{pgfscope}%
\pgfpathrectangle{\pgfqpoint{4.985294in}{0.500000in}}{\pgfqpoint{1.764706in}{1.700000in}}%
\pgfusepath{clip}%
\pgfsetbuttcap%
\pgfsetroundjoin%
\definecolor{currentfill}{rgb}{0.966328,0.750560,0.616961}%
\pgfsetfillcolor{currentfill}%
\pgfsetlinewidth{0.311001pt}%
\definecolor{currentstroke}{rgb}{1.000000,1.000000,1.000000}%
\pgfsetstrokecolor{currentstroke}%
\pgfsetdash{}{0pt}%
\pgfpathmoveto{\pgfqpoint{5.516652in}{1.216287in}}%
\pgfpathcurveto{\pgfqpoint{5.523785in}{1.216287in}}{\pgfqpoint{5.530626in}{1.219121in}}{\pgfqpoint{5.535670in}{1.224165in}}%
\pgfpathcurveto{\pgfqpoint{5.540714in}{1.229208in}}{\pgfqpoint{5.543548in}{1.236050in}}{\pgfqpoint{5.543548in}{1.243183in}}%
\pgfpathcurveto{\pgfqpoint{5.543548in}{1.250316in}}{\pgfqpoint{5.540714in}{1.257157in}}{\pgfqpoint{5.535670in}{1.262201in}}%
\pgfpathcurveto{\pgfqpoint{5.530626in}{1.267245in}}{\pgfqpoint{5.523785in}{1.270079in}}{\pgfqpoint{5.516652in}{1.270079in}}%
\pgfpathcurveto{\pgfqpoint{5.509519in}{1.270079in}}{\pgfqpoint{5.502677in}{1.267245in}}{\pgfqpoint{5.497634in}{1.262201in}}%
\pgfpathcurveto{\pgfqpoint{5.492590in}{1.257157in}}{\pgfqpoint{5.489756in}{1.250316in}}{\pgfqpoint{5.489756in}{1.243183in}}%
\pgfpathcurveto{\pgfqpoint{5.489756in}{1.236050in}}{\pgfqpoint{5.492590in}{1.229208in}}{\pgfqpoint{5.497634in}{1.224165in}}%
\pgfpathcurveto{\pgfqpoint{5.502677in}{1.219121in}}{\pgfqpoint{5.509519in}{1.216287in}}{\pgfqpoint{5.516652in}{1.216287in}}%
\pgfpathclose%
\pgfusepath{stroke,fill}%
\end{pgfscope}%
\begin{pgfscope}%
\pgfpathrectangle{\pgfqpoint{4.985294in}{0.500000in}}{\pgfqpoint{1.764706in}{1.700000in}}%
\pgfusepath{clip}%
\pgfsetbuttcap%
\pgfsetroundjoin%
\definecolor{currentfill}{rgb}{0.964799,0.689101,0.537560}%
\pgfsetfillcolor{currentfill}%
\pgfsetlinewidth{0.311001pt}%
\definecolor{currentstroke}{rgb}{1.000000,1.000000,1.000000}%
\pgfsetstrokecolor{currentstroke}%
\pgfsetdash{}{0pt}%
\pgfpathmoveto{\pgfqpoint{6.157834in}{1.113220in}}%
\pgfpathcurveto{\pgfqpoint{6.164967in}{1.113220in}}{\pgfqpoint{6.171809in}{1.116054in}}{\pgfqpoint{6.176853in}{1.121097in}}%
\pgfpathcurveto{\pgfqpoint{6.181896in}{1.126141in}}{\pgfqpoint{6.184730in}{1.132983in}}{\pgfqpoint{6.184730in}{1.140115in}}%
\pgfpathcurveto{\pgfqpoint{6.184730in}{1.147248in}}{\pgfqpoint{6.181896in}{1.154090in}}{\pgfqpoint{6.176853in}{1.159134in}}%
\pgfpathcurveto{\pgfqpoint{6.171809in}{1.164177in}}{\pgfqpoint{6.164967in}{1.167011in}}{\pgfqpoint{6.157834in}{1.167011in}}%
\pgfpathcurveto{\pgfqpoint{6.150702in}{1.167011in}}{\pgfqpoint{6.143860in}{1.164177in}}{\pgfqpoint{6.138816in}{1.159134in}}%
\pgfpathcurveto{\pgfqpoint{6.133773in}{1.154090in}}{\pgfqpoint{6.130939in}{1.147248in}}{\pgfqpoint{6.130939in}{1.140115in}}%
\pgfpathcurveto{\pgfqpoint{6.130939in}{1.132983in}}{\pgfqpoint{6.133773in}{1.126141in}}{\pgfqpoint{6.138816in}{1.121097in}}%
\pgfpathcurveto{\pgfqpoint{6.143860in}{1.116054in}}{\pgfqpoint{6.150702in}{1.113220in}}{\pgfqpoint{6.157834in}{1.113220in}}%
\pgfpathclose%
\pgfusepath{stroke,fill}%
\end{pgfscope}%
\begin{pgfscope}%
\pgfpathrectangle{\pgfqpoint{4.985294in}{0.500000in}}{\pgfqpoint{1.764706in}{1.700000in}}%
\pgfusepath{clip}%
\pgfsetbuttcap%
\pgfsetroundjoin%
\definecolor{currentfill}{rgb}{0.966120,0.744512,0.608720}%
\pgfsetfillcolor{currentfill}%
\pgfsetlinewidth{0.311001pt}%
\definecolor{currentstroke}{rgb}{1.000000,1.000000,1.000000}%
\pgfsetstrokecolor{currentstroke}%
\pgfsetdash{}{0pt}%
\pgfpathmoveto{\pgfqpoint{5.524606in}{1.183088in}}%
\pgfpathcurveto{\pgfqpoint{5.531739in}{1.183088in}}{\pgfqpoint{5.538580in}{1.185922in}}{\pgfqpoint{5.543624in}{1.190966in}}%
\pgfpathcurveto{\pgfqpoint{5.548668in}{1.196009in}}{\pgfqpoint{5.551501in}{1.202851in}}{\pgfqpoint{5.551501in}{1.209984in}}%
\pgfpathcurveto{\pgfqpoint{5.551501in}{1.217117in}}{\pgfqpoint{5.548668in}{1.223958in}}{\pgfqpoint{5.543624in}{1.229002in}}%
\pgfpathcurveto{\pgfqpoint{5.538580in}{1.234046in}}{\pgfqpoint{5.531739in}{1.236880in}}{\pgfqpoint{5.524606in}{1.236880in}}%
\pgfpathcurveto{\pgfqpoint{5.517473in}{1.236880in}}{\pgfqpoint{5.510631in}{1.234046in}}{\pgfqpoint{5.505588in}{1.229002in}}%
\pgfpathcurveto{\pgfqpoint{5.500544in}{1.223958in}}{\pgfqpoint{5.497710in}{1.217117in}}{\pgfqpoint{5.497710in}{1.209984in}}%
\pgfpathcurveto{\pgfqpoint{5.497710in}{1.202851in}}{\pgfqpoint{5.500544in}{1.196009in}}{\pgfqpoint{5.505588in}{1.190966in}}%
\pgfpathcurveto{\pgfqpoint{5.510631in}{1.185922in}}{\pgfqpoint{5.517473in}{1.183088in}}{\pgfqpoint{5.524606in}{1.183088in}}%
\pgfpathclose%
\pgfusepath{stroke,fill}%
\end{pgfscope}%
\begin{pgfscope}%
\pgfpathrectangle{\pgfqpoint{4.985294in}{0.500000in}}{\pgfqpoint{1.764706in}{1.700000in}}%
\pgfusepath{clip}%
\pgfsetbuttcap%
\pgfsetroundjoin%
\definecolor{currentfill}{rgb}{0.898503,0.224633,0.251087}%
\pgfsetfillcolor{currentfill}%
\pgfsetlinewidth{0.311001pt}%
\definecolor{currentstroke}{rgb}{1.000000,1.000000,1.000000}%
\pgfsetstrokecolor{currentstroke}%
\pgfsetdash{}{0pt}%
\pgfpathmoveto{\pgfqpoint{6.321930in}{0.881797in}}%
\pgfpathcurveto{\pgfqpoint{6.329062in}{0.881797in}}{\pgfqpoint{6.335904in}{0.884631in}}{\pgfqpoint{6.340948in}{0.889674in}}%
\pgfpathcurveto{\pgfqpoint{6.345991in}{0.894718in}}{\pgfqpoint{6.348825in}{0.901560in}}{\pgfqpoint{6.348825in}{0.908692in}}%
\pgfpathcurveto{\pgfqpoint{6.348825in}{0.915825in}}{\pgfqpoint{6.345991in}{0.922667in}}{\pgfqpoint{6.340948in}{0.927711in}}%
\pgfpathcurveto{\pgfqpoint{6.335904in}{0.932754in}}{\pgfqpoint{6.329062in}{0.935588in}}{\pgfqpoint{6.321930in}{0.935588in}}%
\pgfpathcurveto{\pgfqpoint{6.314797in}{0.935588in}}{\pgfqpoint{6.307955in}{0.932754in}}{\pgfqpoint{6.302911in}{0.927711in}}%
\pgfpathcurveto{\pgfqpoint{6.297868in}{0.922667in}}{\pgfqpoint{6.295034in}{0.915825in}}{\pgfqpoint{6.295034in}{0.908692in}}%
\pgfpathcurveto{\pgfqpoint{6.295034in}{0.901560in}}{\pgfqpoint{6.297868in}{0.894718in}}{\pgfqpoint{6.302911in}{0.889674in}}%
\pgfpathcurveto{\pgfqpoint{6.307955in}{0.884631in}}{\pgfqpoint{6.314797in}{0.881797in}}{\pgfqpoint{6.321930in}{0.881797in}}%
\pgfpathclose%
\pgfusepath{stroke,fill}%
\end{pgfscope}%
\begin{pgfscope}%
\pgfpathrectangle{\pgfqpoint{4.985294in}{0.500000in}}{\pgfqpoint{1.764706in}{1.700000in}}%
\pgfusepath{clip}%
\pgfsetbuttcap%
\pgfsetroundjoin%
\definecolor{currentfill}{rgb}{0.972201,0.839051,0.745789}%
\pgfsetfillcolor{currentfill}%
\pgfsetlinewidth{0.311001pt}%
\definecolor{currentstroke}{rgb}{1.000000,1.000000,1.000000}%
\pgfsetstrokecolor{currentstroke}%
\pgfsetdash{}{0pt}%
\pgfpathmoveto{\pgfqpoint{5.472815in}{1.609262in}}%
\pgfpathcurveto{\pgfqpoint{5.479948in}{1.609262in}}{\pgfqpoint{5.486789in}{1.612096in}}{\pgfqpoint{5.491833in}{1.617140in}}%
\pgfpathcurveto{\pgfqpoint{5.496877in}{1.622183in}}{\pgfqpoint{5.499711in}{1.629025in}}{\pgfqpoint{5.499711in}{1.636158in}}%
\pgfpathcurveto{\pgfqpoint{5.499711in}{1.643290in}}{\pgfqpoint{5.496877in}{1.650132in}}{\pgfqpoint{5.491833in}{1.655176in}}%
\pgfpathcurveto{\pgfqpoint{5.486789in}{1.660219in}}{\pgfqpoint{5.479948in}{1.663053in}}{\pgfqpoint{5.472815in}{1.663053in}}%
\pgfpathcurveto{\pgfqpoint{5.465682in}{1.663053in}}{\pgfqpoint{5.458840in}{1.660219in}}{\pgfqpoint{5.453797in}{1.655176in}}%
\pgfpathcurveto{\pgfqpoint{5.448753in}{1.650132in}}{\pgfqpoint{5.445919in}{1.643290in}}{\pgfqpoint{5.445919in}{1.636158in}}%
\pgfpathcurveto{\pgfqpoint{5.445919in}{1.629025in}}{\pgfqpoint{5.448753in}{1.622183in}}{\pgfqpoint{5.453797in}{1.617140in}}%
\pgfpathcurveto{\pgfqpoint{5.458840in}{1.612096in}}{\pgfqpoint{5.465682in}{1.609262in}}{\pgfqpoint{5.472815in}{1.609262in}}%
\pgfpathclose%
\pgfusepath{stroke,fill}%
\end{pgfscope}%
\begin{pgfscope}%
\pgfpathrectangle{\pgfqpoint{4.985294in}{0.500000in}}{\pgfqpoint{1.764706in}{1.700000in}}%
\pgfusepath{clip}%
\pgfsetbuttcap%
\pgfsetroundjoin%
\definecolor{currentfill}{rgb}{0.975018,0.868213,0.788710}%
\pgfsetfillcolor{currentfill}%
\pgfsetlinewidth{0.311001pt}%
\definecolor{currentstroke}{rgb}{1.000000,1.000000,1.000000}%
\pgfsetstrokecolor{currentstroke}%
\pgfsetdash{}{0pt}%
\pgfpathmoveto{\pgfqpoint{5.468870in}{1.065375in}}%
\pgfpathcurveto{\pgfqpoint{5.476003in}{1.065375in}}{\pgfqpoint{5.482845in}{1.068209in}}{\pgfqpoint{5.487888in}{1.073253in}}%
\pgfpathcurveto{\pgfqpoint{5.492932in}{1.078296in}}{\pgfqpoint{5.495766in}{1.085138in}}{\pgfqpoint{5.495766in}{1.092271in}}%
\pgfpathcurveto{\pgfqpoint{5.495766in}{1.099404in}}{\pgfqpoint{5.492932in}{1.106245in}}{\pgfqpoint{5.487888in}{1.111289in}}%
\pgfpathcurveto{\pgfqpoint{5.482845in}{1.116333in}}{\pgfqpoint{5.476003in}{1.119166in}}{\pgfqpoint{5.468870in}{1.119166in}}%
\pgfpathcurveto{\pgfqpoint{5.461737in}{1.119166in}}{\pgfqpoint{5.454896in}{1.116333in}}{\pgfqpoint{5.449852in}{1.111289in}}%
\pgfpathcurveto{\pgfqpoint{5.444808in}{1.106245in}}{\pgfqpoint{5.441974in}{1.099404in}}{\pgfqpoint{5.441974in}{1.092271in}}%
\pgfpathcurveto{\pgfqpoint{5.441974in}{1.085138in}}{\pgfqpoint{5.444808in}{1.078296in}}{\pgfqpoint{5.449852in}{1.073253in}}%
\pgfpathcurveto{\pgfqpoint{5.454896in}{1.068209in}}{\pgfqpoint{5.461737in}{1.065375in}}{\pgfqpoint{5.468870in}{1.065375in}}%
\pgfpathclose%
\pgfusepath{stroke,fill}%
\end{pgfscope}%
\begin{pgfscope}%
\pgfpathrectangle{\pgfqpoint{4.985294in}{0.500000in}}{\pgfqpoint{1.764706in}{1.700000in}}%
\pgfusepath{clip}%
\pgfsetbuttcap%
\pgfsetroundjoin%
\definecolor{currentfill}{rgb}{0.972726,0.844889,0.754401}%
\pgfsetfillcolor{currentfill}%
\pgfsetlinewidth{0.311001pt}%
\definecolor{currentstroke}{rgb}{1.000000,1.000000,1.000000}%
\pgfsetstrokecolor{currentstroke}%
\pgfsetdash{}{0pt}%
\pgfpathmoveto{\pgfqpoint{5.512475in}{1.570501in}}%
\pgfpathcurveto{\pgfqpoint{5.519608in}{1.570501in}}{\pgfqpoint{5.526449in}{1.573335in}}{\pgfqpoint{5.531493in}{1.578379in}}%
\pgfpathcurveto{\pgfqpoint{5.536537in}{1.583423in}}{\pgfqpoint{5.539371in}{1.590264in}}{\pgfqpoint{5.539371in}{1.597397in}}%
\pgfpathcurveto{\pgfqpoint{5.539371in}{1.604530in}}{\pgfqpoint{5.536537in}{1.611371in}}{\pgfqpoint{5.531493in}{1.616415in}}%
\pgfpathcurveto{\pgfqpoint{5.526449in}{1.621459in}}{\pgfqpoint{5.519608in}{1.624293in}}{\pgfqpoint{5.512475in}{1.624293in}}%
\pgfpathcurveto{\pgfqpoint{5.505342in}{1.624293in}}{\pgfqpoint{5.498501in}{1.621459in}}{\pgfqpoint{5.493457in}{1.616415in}}%
\pgfpathcurveto{\pgfqpoint{5.488413in}{1.611371in}}{\pgfqpoint{5.485579in}{1.604530in}}{\pgfqpoint{5.485579in}{1.597397in}}%
\pgfpathcurveto{\pgfqpoint{5.485579in}{1.590264in}}{\pgfqpoint{5.488413in}{1.583423in}}{\pgfqpoint{5.493457in}{1.578379in}}%
\pgfpathcurveto{\pgfqpoint{5.498501in}{1.573335in}}{\pgfqpoint{5.505342in}{1.570501in}}{\pgfqpoint{5.512475in}{1.570501in}}%
\pgfpathclose%
\pgfusepath{stroke,fill}%
\end{pgfscope}%
\begin{pgfscope}%
\pgfpathrectangle{\pgfqpoint{4.985294in}{0.500000in}}{\pgfqpoint{1.764706in}{1.700000in}}%
\pgfusepath{clip}%
\pgfsetbuttcap%
\pgfsetroundjoin%
\definecolor{currentfill}{rgb}{0.959229,0.533075,0.374889}%
\pgfsetfillcolor{currentfill}%
\pgfsetlinewidth{0.311001pt}%
\definecolor{currentstroke}{rgb}{1.000000,1.000000,1.000000}%
\pgfsetstrokecolor{currentstroke}%
\pgfsetdash{}{0pt}%
\pgfpathmoveto{\pgfqpoint{5.319094in}{1.117065in}}%
\pgfpathcurveto{\pgfqpoint{5.326227in}{1.117065in}}{\pgfqpoint{5.333069in}{1.119898in}}{\pgfqpoint{5.338112in}{1.124942in}}%
\pgfpathcurveto{\pgfqpoint{5.343156in}{1.129986in}}{\pgfqpoint{5.345990in}{1.136827in}}{\pgfqpoint{5.345990in}{1.143960in}}%
\pgfpathcurveto{\pgfqpoint{5.345990in}{1.151093in}}{\pgfqpoint{5.343156in}{1.157935in}}{\pgfqpoint{5.338112in}{1.162978in}}%
\pgfpathcurveto{\pgfqpoint{5.333069in}{1.168022in}}{\pgfqpoint{5.326227in}{1.170856in}}{\pgfqpoint{5.319094in}{1.170856in}}%
\pgfpathcurveto{\pgfqpoint{5.311962in}{1.170856in}}{\pgfqpoint{5.305120in}{1.168022in}}{\pgfqpoint{5.300076in}{1.162978in}}%
\pgfpathcurveto{\pgfqpoint{5.295033in}{1.157935in}}{\pgfqpoint{5.292199in}{1.151093in}}{\pgfqpoint{5.292199in}{1.143960in}}%
\pgfpathcurveto{\pgfqpoint{5.292199in}{1.136827in}}{\pgfqpoint{5.295033in}{1.129986in}}{\pgfqpoint{5.300076in}{1.124942in}}%
\pgfpathcurveto{\pgfqpoint{5.305120in}{1.119898in}}{\pgfqpoint{5.311962in}{1.117065in}}{\pgfqpoint{5.319094in}{1.117065in}}%
\pgfpathclose%
\pgfusepath{stroke,fill}%
\end{pgfscope}%
\begin{pgfscope}%
\pgfpathrectangle{\pgfqpoint{4.985294in}{0.500000in}}{\pgfqpoint{1.764706in}{1.700000in}}%
\pgfusepath{clip}%
\pgfsetbuttcap%
\pgfsetroundjoin%
\definecolor{currentfill}{rgb}{0.979124,0.903132,0.839793}%
\pgfsetfillcolor{currentfill}%
\pgfsetlinewidth{0.311001pt}%
\definecolor{currentstroke}{rgb}{1.000000,1.000000,1.000000}%
\pgfsetstrokecolor{currentstroke}%
\pgfsetdash{}{0pt}%
\pgfpathmoveto{\pgfqpoint{6.287544in}{1.429559in}}%
\pgfpathcurveto{\pgfqpoint{6.294677in}{1.429559in}}{\pgfqpoint{6.301518in}{1.432393in}}{\pgfqpoint{6.306562in}{1.437437in}}%
\pgfpathcurveto{\pgfqpoint{6.311606in}{1.442481in}}{\pgfqpoint{6.314440in}{1.449322in}}{\pgfqpoint{6.314440in}{1.456455in}}%
\pgfpathcurveto{\pgfqpoint{6.314440in}{1.463588in}}{\pgfqpoint{6.311606in}{1.470430in}}{\pgfqpoint{6.306562in}{1.475473in}}%
\pgfpathcurveto{\pgfqpoint{6.301518in}{1.480517in}}{\pgfqpoint{6.294677in}{1.483351in}}{\pgfqpoint{6.287544in}{1.483351in}}%
\pgfpathcurveto{\pgfqpoint{6.280411in}{1.483351in}}{\pgfqpoint{6.273569in}{1.480517in}}{\pgfqpoint{6.268526in}{1.475473in}}%
\pgfpathcurveto{\pgfqpoint{6.263482in}{1.470430in}}{\pgfqpoint{6.260648in}{1.463588in}}{\pgfqpoint{6.260648in}{1.456455in}}%
\pgfpathcurveto{\pgfqpoint{6.260648in}{1.449322in}}{\pgfqpoint{6.263482in}{1.442481in}}{\pgfqpoint{6.268526in}{1.437437in}}%
\pgfpathcurveto{\pgfqpoint{6.273569in}{1.432393in}}{\pgfqpoint{6.280411in}{1.429559in}}{\pgfqpoint{6.287544in}{1.429559in}}%
\pgfpathclose%
\pgfusepath{stroke,fill}%
\end{pgfscope}%
\begin{pgfscope}%
\pgfpathrectangle{\pgfqpoint{4.985294in}{0.500000in}}{\pgfqpoint{1.764706in}{1.700000in}}%
\pgfusepath{clip}%
\pgfsetbuttcap%
\pgfsetroundjoin%
\definecolor{currentfill}{rgb}{0.967735,0.780441,0.659127}%
\pgfsetfillcolor{currentfill}%
\pgfsetlinewidth{0.311001pt}%
\definecolor{currentstroke}{rgb}{1.000000,1.000000,1.000000}%
\pgfsetstrokecolor{currentstroke}%
\pgfsetdash{}{0pt}%
\pgfpathmoveto{\pgfqpoint{6.233112in}{1.682881in}}%
\pgfpathcurveto{\pgfqpoint{6.240245in}{1.682881in}}{\pgfqpoint{6.247087in}{1.685715in}}{\pgfqpoint{6.252130in}{1.690759in}}%
\pgfpathcurveto{\pgfqpoint{6.257174in}{1.695802in}}{\pgfqpoint{6.260008in}{1.702644in}}{\pgfqpoint{6.260008in}{1.709777in}}%
\pgfpathcurveto{\pgfqpoint{6.260008in}{1.716910in}}{\pgfqpoint{6.257174in}{1.723751in}}{\pgfqpoint{6.252130in}{1.728795in}}%
\pgfpathcurveto{\pgfqpoint{6.247087in}{1.733839in}}{\pgfqpoint{6.240245in}{1.736673in}}{\pgfqpoint{6.233112in}{1.736673in}}%
\pgfpathcurveto{\pgfqpoint{6.225979in}{1.736673in}}{\pgfqpoint{6.219138in}{1.733839in}}{\pgfqpoint{6.214094in}{1.728795in}}%
\pgfpathcurveto{\pgfqpoint{6.209050in}{1.723751in}}{\pgfqpoint{6.206216in}{1.716910in}}{\pgfqpoint{6.206216in}{1.709777in}}%
\pgfpathcurveto{\pgfqpoint{6.206216in}{1.702644in}}{\pgfqpoint{6.209050in}{1.695802in}}{\pgfqpoint{6.214094in}{1.690759in}}%
\pgfpathcurveto{\pgfqpoint{6.219138in}{1.685715in}}{\pgfqpoint{6.225979in}{1.682881in}}{\pgfqpoint{6.233112in}{1.682881in}}%
\pgfpathclose%
\pgfusepath{stroke,fill}%
\end{pgfscope}%
\begin{pgfscope}%
\pgfpathrectangle{\pgfqpoint{4.985294in}{0.500000in}}{\pgfqpoint{1.764706in}{1.700000in}}%
\pgfusepath{clip}%
\pgfsetbuttcap%
\pgfsetroundjoin%
\definecolor{currentfill}{rgb}{0.961433,0.573272,0.412036}%
\pgfsetfillcolor{currentfill}%
\pgfsetlinewidth{0.311001pt}%
\definecolor{currentstroke}{rgb}{1.000000,1.000000,1.000000}%
\pgfsetstrokecolor{currentstroke}%
\pgfsetdash{}{0pt}%
\pgfpathmoveto{\pgfqpoint{6.297246in}{0.937645in}}%
\pgfpathcurveto{\pgfqpoint{6.304379in}{0.937645in}}{\pgfqpoint{6.311221in}{0.940479in}}{\pgfqpoint{6.316264in}{0.945522in}}%
\pgfpathcurveto{\pgfqpoint{6.321308in}{0.950566in}}{\pgfqpoint{6.324142in}{0.957408in}}{\pgfqpoint{6.324142in}{0.964540in}}%
\pgfpathcurveto{\pgfqpoint{6.324142in}{0.971673in}}{\pgfqpoint{6.321308in}{0.978515in}}{\pgfqpoint{6.316264in}{0.983559in}}%
\pgfpathcurveto{\pgfqpoint{6.311221in}{0.988602in}}{\pgfqpoint{6.304379in}{0.991436in}}{\pgfqpoint{6.297246in}{0.991436in}}%
\pgfpathcurveto{\pgfqpoint{6.290113in}{0.991436in}}{\pgfqpoint{6.283272in}{0.988602in}}{\pgfqpoint{6.278228in}{0.983559in}}%
\pgfpathcurveto{\pgfqpoint{6.273184in}{0.978515in}}{\pgfqpoint{6.270351in}{0.971673in}}{\pgfqpoint{6.270351in}{0.964540in}}%
\pgfpathcurveto{\pgfqpoint{6.270351in}{0.957408in}}{\pgfqpoint{6.273184in}{0.950566in}}{\pgfqpoint{6.278228in}{0.945522in}}%
\pgfpathcurveto{\pgfqpoint{6.283272in}{0.940479in}}{\pgfqpoint{6.290113in}{0.937645in}}{\pgfqpoint{6.297246in}{0.937645in}}%
\pgfpathclose%
\pgfusepath{stroke,fill}%
\end{pgfscope}%
\begin{pgfscope}%
\pgfpathrectangle{\pgfqpoint{4.985294in}{0.500000in}}{\pgfqpoint{1.764706in}{1.700000in}}%
\pgfusepath{clip}%
\pgfsetbuttcap%
\pgfsetroundjoin%
\definecolor{currentfill}{rgb}{0.973271,0.850724,0.762998}%
\pgfsetfillcolor{currentfill}%
\pgfsetlinewidth{0.311001pt}%
\definecolor{currentstroke}{rgb}{1.000000,1.000000,1.000000}%
\pgfsetstrokecolor{currentstroke}%
\pgfsetdash{}{0pt}%
\pgfpathmoveto{\pgfqpoint{6.258690in}{1.424683in}}%
\pgfpathcurveto{\pgfqpoint{6.265823in}{1.424683in}}{\pgfqpoint{6.272665in}{1.427517in}}{\pgfqpoint{6.277708in}{1.432561in}}%
\pgfpathcurveto{\pgfqpoint{6.282752in}{1.437605in}}{\pgfqpoint{6.285586in}{1.444446in}}{\pgfqpoint{6.285586in}{1.451579in}}%
\pgfpathcurveto{\pgfqpoint{6.285586in}{1.458712in}}{\pgfqpoint{6.282752in}{1.465554in}}{\pgfqpoint{6.277708in}{1.470597in}}%
\pgfpathcurveto{\pgfqpoint{6.272665in}{1.475641in}}{\pgfqpoint{6.265823in}{1.478475in}}{\pgfqpoint{6.258690in}{1.478475in}}%
\pgfpathcurveto{\pgfqpoint{6.251557in}{1.478475in}}{\pgfqpoint{6.244716in}{1.475641in}}{\pgfqpoint{6.239672in}{1.470597in}}%
\pgfpathcurveto{\pgfqpoint{6.234628in}{1.465554in}}{\pgfqpoint{6.231794in}{1.458712in}}{\pgfqpoint{6.231794in}{1.451579in}}%
\pgfpathcurveto{\pgfqpoint{6.231794in}{1.444446in}}{\pgfqpoint{6.234628in}{1.437605in}}{\pgfqpoint{6.239672in}{1.432561in}}%
\pgfpathcurveto{\pgfqpoint{6.244716in}{1.427517in}}{\pgfqpoint{6.251557in}{1.424683in}}{\pgfqpoint{6.258690in}{1.424683in}}%
\pgfpathclose%
\pgfusepath{stroke,fill}%
\end{pgfscope}%
\begin{pgfscope}%
\pgfpathrectangle{\pgfqpoint{4.985294in}{0.500000in}}{\pgfqpoint{1.764706in}{1.700000in}}%
\pgfusepath{clip}%
\pgfsetbuttcap%
\pgfsetroundjoin%
\definecolor{currentfill}{rgb}{0.950017,0.427714,0.292447}%
\pgfsetfillcolor{currentfill}%
\pgfsetlinewidth{0.311001pt}%
\definecolor{currentstroke}{rgb}{1.000000,1.000000,1.000000}%
\pgfsetstrokecolor{currentstroke}%
\pgfsetdash{}{0pt}%
\pgfpathmoveto{\pgfqpoint{5.325370in}{1.560202in}}%
\pgfpathcurveto{\pgfqpoint{5.332503in}{1.560202in}}{\pgfqpoint{5.339344in}{1.563036in}}{\pgfqpoint{5.344388in}{1.568079in}}%
\pgfpathcurveto{\pgfqpoint{5.349432in}{1.573123in}}{\pgfqpoint{5.352266in}{1.579965in}}{\pgfqpoint{5.352266in}{1.587097in}}%
\pgfpathcurveto{\pgfqpoint{5.352266in}{1.594230in}}{\pgfqpoint{5.349432in}{1.601072in}}{\pgfqpoint{5.344388in}{1.606116in}}%
\pgfpathcurveto{\pgfqpoint{5.339344in}{1.611159in}}{\pgfqpoint{5.332503in}{1.613993in}}{\pgfqpoint{5.325370in}{1.613993in}}%
\pgfpathcurveto{\pgfqpoint{5.318237in}{1.613993in}}{\pgfqpoint{5.311395in}{1.611159in}}{\pgfqpoint{5.306352in}{1.606116in}}%
\pgfpathcurveto{\pgfqpoint{5.301308in}{1.601072in}}{\pgfqpoint{5.298474in}{1.594230in}}{\pgfqpoint{5.298474in}{1.587097in}}%
\pgfpathcurveto{\pgfqpoint{5.298474in}{1.579965in}}{\pgfqpoint{5.301308in}{1.573123in}}{\pgfqpoint{5.306352in}{1.568079in}}%
\pgfpathcurveto{\pgfqpoint{5.311395in}{1.563036in}}{\pgfqpoint{5.318237in}{1.560202in}}{\pgfqpoint{5.325370in}{1.560202in}}%
\pgfpathclose%
\pgfusepath{stroke,fill}%
\end{pgfscope}%
\begin{pgfscope}%
\pgfpathrectangle{\pgfqpoint{4.985294in}{0.500000in}}{\pgfqpoint{1.764706in}{1.700000in}}%
\pgfusepath{clip}%
\pgfsetbuttcap%
\pgfsetroundjoin%
\definecolor{currentfill}{rgb}{0.979891,0.908948,0.848279}%
\pgfsetfillcolor{currentfill}%
\pgfsetlinewidth{0.311001pt}%
\definecolor{currentstroke}{rgb}{1.000000,1.000000,1.000000}%
\pgfsetstrokecolor{currentstroke}%
\pgfsetdash{}{0pt}%
\pgfpathmoveto{\pgfqpoint{5.419182in}{1.257478in}}%
\pgfpathcurveto{\pgfqpoint{5.426315in}{1.257478in}}{\pgfqpoint{5.433157in}{1.260311in}}{\pgfqpoint{5.438200in}{1.265355in}}%
\pgfpathcurveto{\pgfqpoint{5.443244in}{1.270399in}}{\pgfqpoint{5.446078in}{1.277240in}}{\pgfqpoint{5.446078in}{1.284373in}}%
\pgfpathcurveto{\pgfqpoint{5.446078in}{1.291506in}}{\pgfqpoint{5.443244in}{1.298348in}}{\pgfqpoint{5.438200in}{1.303391in}}%
\pgfpathcurveto{\pgfqpoint{5.433157in}{1.308435in}}{\pgfqpoint{5.426315in}{1.311269in}}{\pgfqpoint{5.419182in}{1.311269in}}%
\pgfpathcurveto{\pgfqpoint{5.412049in}{1.311269in}}{\pgfqpoint{5.405208in}{1.308435in}}{\pgfqpoint{5.400164in}{1.303391in}}%
\pgfpathcurveto{\pgfqpoint{5.395120in}{1.298348in}}{\pgfqpoint{5.392287in}{1.291506in}}{\pgfqpoint{5.392287in}{1.284373in}}%
\pgfpathcurveto{\pgfqpoint{5.392287in}{1.277240in}}{\pgfqpoint{5.395120in}{1.270399in}}{\pgfqpoint{5.400164in}{1.265355in}}%
\pgfpathcurveto{\pgfqpoint{5.405208in}{1.260311in}}{\pgfqpoint{5.412049in}{1.257478in}}{\pgfqpoint{5.419182in}{1.257478in}}%
\pgfpathclose%
\pgfusepath{stroke,fill}%
\end{pgfscope}%
\begin{pgfscope}%
\pgfpathrectangle{\pgfqpoint{4.985294in}{0.500000in}}{\pgfqpoint{1.764706in}{1.700000in}}%
\pgfusepath{clip}%
\pgfsetbuttcap%
\pgfsetroundjoin%
\definecolor{currentfill}{rgb}{0.974412,0.862387,0.780156}%
\pgfsetfillcolor{currentfill}%
\pgfsetlinewidth{0.311001pt}%
\definecolor{currentstroke}{rgb}{1.000000,1.000000,1.000000}%
\pgfsetstrokecolor{currentstroke}%
\pgfsetdash{}{0pt}%
\pgfpathmoveto{\pgfqpoint{6.235881in}{1.549933in}}%
\pgfpathcurveto{\pgfqpoint{6.243014in}{1.549933in}}{\pgfqpoint{6.249856in}{1.552767in}}{\pgfqpoint{6.254899in}{1.557810in}}%
\pgfpathcurveto{\pgfqpoint{6.259943in}{1.562854in}}{\pgfqpoint{6.262777in}{1.569696in}}{\pgfqpoint{6.262777in}{1.576828in}}%
\pgfpathcurveto{\pgfqpoint{6.262777in}{1.583961in}}{\pgfqpoint{6.259943in}{1.590803in}}{\pgfqpoint{6.254899in}{1.595847in}}%
\pgfpathcurveto{\pgfqpoint{6.249856in}{1.600890in}}{\pgfqpoint{6.243014in}{1.603724in}}{\pgfqpoint{6.235881in}{1.603724in}}%
\pgfpathcurveto{\pgfqpoint{6.228748in}{1.603724in}}{\pgfqpoint{6.221907in}{1.600890in}}{\pgfqpoint{6.216863in}{1.595847in}}%
\pgfpathcurveto{\pgfqpoint{6.211819in}{1.590803in}}{\pgfqpoint{6.208985in}{1.583961in}}{\pgfqpoint{6.208985in}{1.576828in}}%
\pgfpathcurveto{\pgfqpoint{6.208985in}{1.569696in}}{\pgfqpoint{6.211819in}{1.562854in}}{\pgfqpoint{6.216863in}{1.557810in}}%
\pgfpathcurveto{\pgfqpoint{6.221907in}{1.552767in}}{\pgfqpoint{6.228748in}{1.549933in}}{\pgfqpoint{6.235881in}{1.549933in}}%
\pgfpathclose%
\pgfusepath{stroke,fill}%
\end{pgfscope}%
\begin{pgfscope}%
\pgfpathrectangle{\pgfqpoint{4.985294in}{0.500000in}}{\pgfqpoint{1.764706in}{1.700000in}}%
\pgfusepath{clip}%
\pgfsetbuttcap%
\pgfsetroundjoin%
\definecolor{currentfill}{rgb}{0.818205,0.120806,0.299261}%
\pgfsetfillcolor{currentfill}%
\pgfsetlinewidth{0.311001pt}%
\definecolor{currentstroke}{rgb}{1.000000,1.000000,1.000000}%
\pgfsetstrokecolor{currentstroke}%
\pgfsetdash{}{0pt}%
\pgfpathmoveto{\pgfqpoint{5.610041in}{1.864441in}}%
\pgfpathcurveto{\pgfqpoint{5.617174in}{1.864441in}}{\pgfqpoint{5.624015in}{1.867274in}}{\pgfqpoint{5.629059in}{1.872318in}}%
\pgfpathcurveto{\pgfqpoint{5.634103in}{1.877362in}}{\pgfqpoint{5.636937in}{1.884203in}}{\pgfqpoint{5.636937in}{1.891336in}}%
\pgfpathcurveto{\pgfqpoint{5.636937in}{1.898469in}}{\pgfqpoint{5.634103in}{1.905311in}}{\pgfqpoint{5.629059in}{1.910354in}}%
\pgfpathcurveto{\pgfqpoint{5.624015in}{1.915398in}}{\pgfqpoint{5.617174in}{1.918232in}}{\pgfqpoint{5.610041in}{1.918232in}}%
\pgfpathcurveto{\pgfqpoint{5.602908in}{1.918232in}}{\pgfqpoint{5.596066in}{1.915398in}}{\pgfqpoint{5.591023in}{1.910354in}}%
\pgfpathcurveto{\pgfqpoint{5.585979in}{1.905311in}}{\pgfqpoint{5.583145in}{1.898469in}}{\pgfqpoint{5.583145in}{1.891336in}}%
\pgfpathcurveto{\pgfqpoint{5.583145in}{1.884203in}}{\pgfqpoint{5.585979in}{1.877362in}}{\pgfqpoint{5.591023in}{1.872318in}}%
\pgfpathcurveto{\pgfqpoint{5.596066in}{1.867274in}}{\pgfqpoint{5.602908in}{1.864441in}}{\pgfqpoint{5.610041in}{1.864441in}}%
\pgfpathclose%
\pgfusepath{stroke,fill}%
\end{pgfscope}%
\begin{pgfscope}%
\pgfpathrectangle{\pgfqpoint{4.985294in}{0.500000in}}{\pgfqpoint{1.764706in}{1.700000in}}%
\pgfusepath{clip}%
\pgfsetbuttcap%
\pgfsetroundjoin%
\definecolor{currentfill}{rgb}{0.971694,0.833208,0.737161}%
\pgfsetfillcolor{currentfill}%
\pgfsetlinewidth{0.311001pt}%
\definecolor{currentstroke}{rgb}{1.000000,1.000000,1.000000}%
\pgfsetstrokecolor{currentstroke}%
\pgfsetdash{}{0pt}%
\pgfpathmoveto{\pgfqpoint{5.506570in}{1.617980in}}%
\pgfpathcurveto{\pgfqpoint{5.513703in}{1.617980in}}{\pgfqpoint{5.520545in}{1.620814in}}{\pgfqpoint{5.525589in}{1.625857in}}%
\pgfpathcurveto{\pgfqpoint{5.530632in}{1.630901in}}{\pgfqpoint{5.533466in}{1.637743in}}{\pgfqpoint{5.533466in}{1.644875in}}%
\pgfpathcurveto{\pgfqpoint{5.533466in}{1.652008in}}{\pgfqpoint{5.530632in}{1.658850in}}{\pgfqpoint{5.525589in}{1.663894in}}%
\pgfpathcurveto{\pgfqpoint{5.520545in}{1.668937in}}{\pgfqpoint{5.513703in}{1.671771in}}{\pgfqpoint{5.506570in}{1.671771in}}%
\pgfpathcurveto{\pgfqpoint{5.499438in}{1.671771in}}{\pgfqpoint{5.492596in}{1.668937in}}{\pgfqpoint{5.487552in}{1.663894in}}%
\pgfpathcurveto{\pgfqpoint{5.482509in}{1.658850in}}{\pgfqpoint{5.479675in}{1.652008in}}{\pgfqpoint{5.479675in}{1.644875in}}%
\pgfpathcurveto{\pgfqpoint{5.479675in}{1.637743in}}{\pgfqpoint{5.482509in}{1.630901in}}{\pgfqpoint{5.487552in}{1.625857in}}%
\pgfpathcurveto{\pgfqpoint{5.492596in}{1.620814in}}{\pgfqpoint{5.499438in}{1.617980in}}{\pgfqpoint{5.506570in}{1.617980in}}%
\pgfpathclose%
\pgfusepath{stroke,fill}%
\end{pgfscope}%
\begin{pgfscope}%
\pgfpathrectangle{\pgfqpoint{4.985294in}{0.500000in}}{\pgfqpoint{1.764706in}{1.700000in}}%
\pgfusepath{clip}%
\pgfsetbuttcap%
\pgfsetroundjoin%
\definecolor{currentfill}{rgb}{0.979891,0.908948,0.848279}%
\pgfsetfillcolor{currentfill}%
\pgfsetlinewidth{0.311001pt}%
\definecolor{currentstroke}{rgb}{1.000000,1.000000,1.000000}%
\pgfsetstrokecolor{currentstroke}%
\pgfsetdash{}{0pt}%
\pgfpathmoveto{\pgfqpoint{6.323673in}{1.446627in}}%
\pgfpathcurveto{\pgfqpoint{6.330806in}{1.446627in}}{\pgfqpoint{6.337647in}{1.449461in}}{\pgfqpoint{6.342691in}{1.454505in}}%
\pgfpathcurveto{\pgfqpoint{6.347735in}{1.459549in}}{\pgfqpoint{6.350568in}{1.466390in}}{\pgfqpoint{6.350568in}{1.473523in}}%
\pgfpathcurveto{\pgfqpoint{6.350568in}{1.480656in}}{\pgfqpoint{6.347735in}{1.487498in}}{\pgfqpoint{6.342691in}{1.492541in}}%
\pgfpathcurveto{\pgfqpoint{6.337647in}{1.497585in}}{\pgfqpoint{6.330806in}{1.500419in}}{\pgfqpoint{6.323673in}{1.500419in}}%
\pgfpathcurveto{\pgfqpoint{6.316540in}{1.500419in}}{\pgfqpoint{6.309698in}{1.497585in}}{\pgfqpoint{6.304655in}{1.492541in}}%
\pgfpathcurveto{\pgfqpoint{6.299611in}{1.487498in}}{\pgfqpoint{6.296777in}{1.480656in}}{\pgfqpoint{6.296777in}{1.473523in}}%
\pgfpathcurveto{\pgfqpoint{6.296777in}{1.466390in}}{\pgfqpoint{6.299611in}{1.459549in}}{\pgfqpoint{6.304655in}{1.454505in}}%
\pgfpathcurveto{\pgfqpoint{6.309698in}{1.449461in}}{\pgfqpoint{6.316540in}{1.446627in}}{\pgfqpoint{6.323673in}{1.446627in}}%
\pgfpathclose%
\pgfusepath{stroke,fill}%
\end{pgfscope}%
\begin{pgfscope}%
\pgfpathrectangle{\pgfqpoint{4.985294in}{0.500000in}}{\pgfqpoint{1.764706in}{1.700000in}}%
\pgfusepath{clip}%
\pgfsetbuttcap%
\pgfsetroundjoin%
\definecolor{currentfill}{rgb}{0.969803,0.809811,0.702523}%
\pgfsetfillcolor{currentfill}%
\pgfsetlinewidth{0.311001pt}%
\definecolor{currentstroke}{rgb}{1.000000,1.000000,1.000000}%
\pgfsetstrokecolor{currentstroke}%
\pgfsetdash{}{0pt}%
\pgfpathmoveto{\pgfqpoint{6.192956in}{1.110402in}}%
\pgfpathcurveto{\pgfqpoint{6.200089in}{1.110402in}}{\pgfqpoint{6.206931in}{1.113236in}}{\pgfqpoint{6.211974in}{1.118279in}}%
\pgfpathcurveto{\pgfqpoint{6.217018in}{1.123323in}}{\pgfqpoint{6.219852in}{1.130164in}}{\pgfqpoint{6.219852in}{1.137297in}}%
\pgfpathcurveto{\pgfqpoint{6.219852in}{1.144430in}}{\pgfqpoint{6.217018in}{1.151272in}}{\pgfqpoint{6.211974in}{1.156315in}}%
\pgfpathcurveto{\pgfqpoint{6.206931in}{1.161359in}}{\pgfqpoint{6.200089in}{1.164193in}}{\pgfqpoint{6.192956in}{1.164193in}}%
\pgfpathcurveto{\pgfqpoint{6.185823in}{1.164193in}}{\pgfqpoint{6.178982in}{1.161359in}}{\pgfqpoint{6.173938in}{1.156315in}}%
\pgfpathcurveto{\pgfqpoint{6.168894in}{1.151272in}}{\pgfqpoint{6.166060in}{1.144430in}}{\pgfqpoint{6.166060in}{1.137297in}}%
\pgfpathcurveto{\pgfqpoint{6.166060in}{1.130164in}}{\pgfqpoint{6.168894in}{1.123323in}}{\pgfqpoint{6.173938in}{1.118279in}}%
\pgfpathcurveto{\pgfqpoint{6.178982in}{1.113236in}}{\pgfqpoint{6.185823in}{1.110402in}}{\pgfqpoint{6.192956in}{1.110402in}}%
\pgfpathclose%
\pgfusepath{stroke,fill}%
\end{pgfscope}%
\begin{pgfscope}%
\pgfpathrectangle{\pgfqpoint{4.985294in}{0.500000in}}{\pgfqpoint{1.764706in}{1.700000in}}%
\pgfusepath{clip}%
\pgfsetbuttcap%
\pgfsetroundjoin%
\definecolor{currentfill}{rgb}{0.977657,0.891500,0.822809}%
\pgfsetfillcolor{currentfill}%
\pgfsetlinewidth{0.311001pt}%
\definecolor{currentstroke}{rgb}{1.000000,1.000000,1.000000}%
\pgfsetstrokecolor{currentstroke}%
\pgfsetdash{}{0pt}%
\pgfpathmoveto{\pgfqpoint{5.451558in}{1.444602in}}%
\pgfpathcurveto{\pgfqpoint{5.458691in}{1.444602in}}{\pgfqpoint{5.465532in}{1.447436in}}{\pgfqpoint{5.470576in}{1.452480in}}%
\pgfpathcurveto{\pgfqpoint{5.475620in}{1.457524in}}{\pgfqpoint{5.478454in}{1.464365in}}{\pgfqpoint{5.478454in}{1.471498in}}%
\pgfpathcurveto{\pgfqpoint{5.478454in}{1.478631in}}{\pgfqpoint{5.475620in}{1.485473in}}{\pgfqpoint{5.470576in}{1.490516in}}%
\pgfpathcurveto{\pgfqpoint{5.465532in}{1.495560in}}{\pgfqpoint{5.458691in}{1.498394in}}{\pgfqpoint{5.451558in}{1.498394in}}%
\pgfpathcurveto{\pgfqpoint{5.444425in}{1.498394in}}{\pgfqpoint{5.437583in}{1.495560in}}{\pgfqpoint{5.432540in}{1.490516in}}%
\pgfpathcurveto{\pgfqpoint{5.427496in}{1.485473in}}{\pgfqpoint{5.424662in}{1.478631in}}{\pgfqpoint{5.424662in}{1.471498in}}%
\pgfpathcurveto{\pgfqpoint{5.424662in}{1.464365in}}{\pgfqpoint{5.427496in}{1.457524in}}{\pgfqpoint{5.432540in}{1.452480in}}%
\pgfpathcurveto{\pgfqpoint{5.437583in}{1.447436in}}{\pgfqpoint{5.444425in}{1.444602in}}{\pgfqpoint{5.451558in}{1.444602in}}%
\pgfpathclose%
\pgfusepath{stroke,fill}%
\end{pgfscope}%
\begin{pgfscope}%
\pgfpathrectangle{\pgfqpoint{4.985294in}{0.500000in}}{\pgfqpoint{1.764706in}{1.700000in}}%
\pgfusepath{clip}%
\pgfsetbuttcap%
\pgfsetroundjoin%
\definecolor{currentfill}{rgb}{0.965753,0.732351,0.592427}%
\pgfsetfillcolor{currentfill}%
\pgfsetlinewidth{0.311001pt}%
\definecolor{currentstroke}{rgb}{1.000000,1.000000,1.000000}%
\pgfsetstrokecolor{currentstroke}%
\pgfsetdash{}{0pt}%
\pgfpathmoveto{\pgfqpoint{6.335867in}{1.592208in}}%
\pgfpathcurveto{\pgfqpoint{6.343000in}{1.592208in}}{\pgfqpoint{6.349841in}{1.595041in}}{\pgfqpoint{6.354885in}{1.600085in}}%
\pgfpathcurveto{\pgfqpoint{6.359929in}{1.605129in}}{\pgfqpoint{6.362762in}{1.611970in}}{\pgfqpoint{6.362762in}{1.619103in}}%
\pgfpathcurveto{\pgfqpoint{6.362762in}{1.626236in}}{\pgfqpoint{6.359929in}{1.633078in}}{\pgfqpoint{6.354885in}{1.638121in}}%
\pgfpathcurveto{\pgfqpoint{6.349841in}{1.643165in}}{\pgfqpoint{6.343000in}{1.645999in}}{\pgfqpoint{6.335867in}{1.645999in}}%
\pgfpathcurveto{\pgfqpoint{6.328734in}{1.645999in}}{\pgfqpoint{6.321892in}{1.643165in}}{\pgfqpoint{6.316849in}{1.638121in}}%
\pgfpathcurveto{\pgfqpoint{6.311805in}{1.633078in}}{\pgfqpoint{6.308971in}{1.626236in}}{\pgfqpoint{6.308971in}{1.619103in}}%
\pgfpathcurveto{\pgfqpoint{6.308971in}{1.611970in}}{\pgfqpoint{6.311805in}{1.605129in}}{\pgfqpoint{6.316849in}{1.600085in}}%
\pgfpathcurveto{\pgfqpoint{6.321892in}{1.595041in}}{\pgfqpoint{6.328734in}{1.592208in}}{\pgfqpoint{6.335867in}{1.592208in}}%
\pgfpathclose%
\pgfusepath{stroke,fill}%
\end{pgfscope}%
\begin{pgfscope}%
\pgfpathrectangle{\pgfqpoint{4.985294in}{0.500000in}}{\pgfqpoint{1.764706in}{1.700000in}}%
\pgfusepath{clip}%
\pgfsetbuttcap%
\pgfsetroundjoin%
\definecolor{currentfill}{rgb}{0.980678,0.914765,0.856766}%
\pgfsetfillcolor{currentfill}%
\pgfsetlinewidth{0.311001pt}%
\definecolor{currentstroke}{rgb}{1.000000,1.000000,1.000000}%
\pgfsetstrokecolor{currentstroke}%
\pgfsetdash{}{0pt}%
\pgfpathmoveto{\pgfqpoint{5.422414in}{1.280189in}}%
\pgfpathcurveto{\pgfqpoint{5.429547in}{1.280189in}}{\pgfqpoint{5.436389in}{1.283023in}}{\pgfqpoint{5.441432in}{1.288067in}}%
\pgfpathcurveto{\pgfqpoint{5.446476in}{1.293111in}}{\pgfqpoint{5.449310in}{1.299952in}}{\pgfqpoint{5.449310in}{1.307085in}}%
\pgfpathcurveto{\pgfqpoint{5.449310in}{1.314218in}}{\pgfqpoint{5.446476in}{1.321060in}}{\pgfqpoint{5.441432in}{1.326103in}}%
\pgfpathcurveto{\pgfqpoint{5.436389in}{1.331147in}}{\pgfqpoint{5.429547in}{1.333981in}}{\pgfqpoint{5.422414in}{1.333981in}}%
\pgfpathcurveto{\pgfqpoint{5.415281in}{1.333981in}}{\pgfqpoint{5.408440in}{1.331147in}}{\pgfqpoint{5.403396in}{1.326103in}}%
\pgfpathcurveto{\pgfqpoint{5.398352in}{1.321060in}}{\pgfqpoint{5.395518in}{1.314218in}}{\pgfqpoint{5.395518in}{1.307085in}}%
\pgfpathcurveto{\pgfqpoint{5.395518in}{1.299952in}}{\pgfqpoint{5.398352in}{1.293111in}}{\pgfqpoint{5.403396in}{1.288067in}}%
\pgfpathcurveto{\pgfqpoint{5.408440in}{1.283023in}}{\pgfqpoint{5.415281in}{1.280189in}}{\pgfqpoint{5.422414in}{1.280189in}}%
\pgfpathclose%
\pgfusepath{stroke,fill}%
\end{pgfscope}%
\begin{pgfscope}%
\pgfpathrectangle{\pgfqpoint{4.985294in}{0.500000in}}{\pgfqpoint{1.764706in}{1.700000in}}%
\pgfusepath{clip}%
\pgfsetbuttcap%
\pgfsetroundjoin%
\definecolor{currentfill}{rgb}{0.968931,0.798091,0.685123}%
\pgfsetfillcolor{currentfill}%
\pgfsetlinewidth{0.311001pt}%
\definecolor{currentstroke}{rgb}{1.000000,1.000000,1.000000}%
\pgfsetstrokecolor{currentstroke}%
\pgfsetdash{}{0pt}%
\pgfpathmoveto{\pgfqpoint{6.376910in}{1.188771in}}%
\pgfpathcurveto{\pgfqpoint{6.384043in}{1.188771in}}{\pgfqpoint{6.390885in}{1.191605in}}{\pgfqpoint{6.395928in}{1.196648in}}%
\pgfpathcurveto{\pgfqpoint{6.400972in}{1.201692in}}{\pgfqpoint{6.403806in}{1.208534in}}{\pgfqpoint{6.403806in}{1.215667in}}%
\pgfpathcurveto{\pgfqpoint{6.403806in}{1.222799in}}{\pgfqpoint{6.400972in}{1.229641in}}{\pgfqpoint{6.395928in}{1.234685in}}%
\pgfpathcurveto{\pgfqpoint{6.390885in}{1.239728in}}{\pgfqpoint{6.384043in}{1.242562in}}{\pgfqpoint{6.376910in}{1.242562in}}%
\pgfpathcurveto{\pgfqpoint{6.369777in}{1.242562in}}{\pgfqpoint{6.362936in}{1.239728in}}{\pgfqpoint{6.357892in}{1.234685in}}%
\pgfpathcurveto{\pgfqpoint{6.352848in}{1.229641in}}{\pgfqpoint{6.350015in}{1.222799in}}{\pgfqpoint{6.350015in}{1.215667in}}%
\pgfpathcurveto{\pgfqpoint{6.350015in}{1.208534in}}{\pgfqpoint{6.352848in}{1.201692in}}{\pgfqpoint{6.357892in}{1.196648in}}%
\pgfpathcurveto{\pgfqpoint{6.362936in}{1.191605in}}{\pgfqpoint{6.369777in}{1.188771in}}{\pgfqpoint{6.376910in}{1.188771in}}%
\pgfpathclose%
\pgfusepath{stroke,fill}%
\end{pgfscope}%
\begin{pgfscope}%
\pgfpathrectangle{\pgfqpoint{4.985294in}{0.500000in}}{\pgfqpoint{1.764706in}{1.700000in}}%
\pgfusepath{clip}%
\pgfsetbuttcap%
\pgfsetroundjoin%
\definecolor{currentfill}{rgb}{0.975018,0.868213,0.788710}%
\pgfsetfillcolor{currentfill}%
\pgfsetlinewidth{0.311001pt}%
\definecolor{currentstroke}{rgb}{1.000000,1.000000,1.000000}%
\pgfsetstrokecolor{currentstroke}%
\pgfsetdash{}{0pt}%
\pgfpathmoveto{\pgfqpoint{6.231273in}{1.602317in}}%
\pgfpathcurveto{\pgfqpoint{6.238406in}{1.602317in}}{\pgfqpoint{6.245248in}{1.605151in}}{\pgfqpoint{6.250292in}{1.610195in}}%
\pgfpathcurveto{\pgfqpoint{6.255335in}{1.615239in}}{\pgfqpoint{6.258169in}{1.622080in}}{\pgfqpoint{6.258169in}{1.629213in}}%
\pgfpathcurveto{\pgfqpoint{6.258169in}{1.636346in}}{\pgfqpoint{6.255335in}{1.643188in}}{\pgfqpoint{6.250292in}{1.648231in}}%
\pgfpathcurveto{\pgfqpoint{6.245248in}{1.653275in}}{\pgfqpoint{6.238406in}{1.656109in}}{\pgfqpoint{6.231273in}{1.656109in}}%
\pgfpathcurveto{\pgfqpoint{6.224141in}{1.656109in}}{\pgfqpoint{6.217299in}{1.653275in}}{\pgfqpoint{6.212255in}{1.648231in}}%
\pgfpathcurveto{\pgfqpoint{6.207212in}{1.643188in}}{\pgfqpoint{6.204378in}{1.636346in}}{\pgfqpoint{6.204378in}{1.629213in}}%
\pgfpathcurveto{\pgfqpoint{6.204378in}{1.622080in}}{\pgfqpoint{6.207212in}{1.615239in}}{\pgfqpoint{6.212255in}{1.610195in}}%
\pgfpathcurveto{\pgfqpoint{6.217299in}{1.605151in}}{\pgfqpoint{6.224141in}{1.602317in}}{\pgfqpoint{6.231273in}{1.602317in}}%
\pgfpathclose%
\pgfusepath{stroke,fill}%
\end{pgfscope}%
\begin{pgfscope}%
\pgfpathrectangle{\pgfqpoint{4.985294in}{0.500000in}}{\pgfqpoint{1.764706in}{1.700000in}}%
\pgfusepath{clip}%
\pgfsetbuttcap%
\pgfsetroundjoin%
\definecolor{currentfill}{rgb}{0.965928,0.738443,0.600540}%
\pgfsetfillcolor{currentfill}%
\pgfsetlinewidth{0.311001pt}%
\definecolor{currentstroke}{rgb}{1.000000,1.000000,1.000000}%
\pgfsetstrokecolor{currentstroke}%
\pgfsetdash{}{0pt}%
\pgfpathmoveto{\pgfqpoint{6.206594in}{1.365906in}}%
\pgfpathcurveto{\pgfqpoint{6.213726in}{1.365906in}}{\pgfqpoint{6.220568in}{1.368740in}}{\pgfqpoint{6.225612in}{1.373784in}}%
\pgfpathcurveto{\pgfqpoint{6.230655in}{1.378827in}}{\pgfqpoint{6.233489in}{1.385669in}}{\pgfqpoint{6.233489in}{1.392802in}}%
\pgfpathcurveto{\pgfqpoint{6.233489in}{1.399935in}}{\pgfqpoint{6.230655in}{1.406776in}}{\pgfqpoint{6.225612in}{1.411820in}}%
\pgfpathcurveto{\pgfqpoint{6.220568in}{1.416864in}}{\pgfqpoint{6.213726in}{1.419698in}}{\pgfqpoint{6.206594in}{1.419698in}}%
\pgfpathcurveto{\pgfqpoint{6.199461in}{1.419698in}}{\pgfqpoint{6.192619in}{1.416864in}}{\pgfqpoint{6.187576in}{1.411820in}}%
\pgfpathcurveto{\pgfqpoint{6.182532in}{1.406776in}}{\pgfqpoint{6.179698in}{1.399935in}}{\pgfqpoint{6.179698in}{1.392802in}}%
\pgfpathcurveto{\pgfqpoint{6.179698in}{1.385669in}}{\pgfqpoint{6.182532in}{1.378827in}}{\pgfqpoint{6.187576in}{1.373784in}}%
\pgfpathcurveto{\pgfqpoint{6.192619in}{1.368740in}}{\pgfqpoint{6.199461in}{1.365906in}}{\pgfqpoint{6.206594in}{1.365906in}}%
\pgfpathclose%
\pgfusepath{stroke,fill}%
\end{pgfscope}%
\begin{pgfscope}%
\pgfpathrectangle{\pgfqpoint{4.985294in}{0.500000in}}{\pgfqpoint{1.764706in}{1.700000in}}%
\pgfusepath{clip}%
\pgfsetbuttcap%
\pgfsetroundjoin%
\definecolor{currentfill}{rgb}{0.976287,0.879862,0.805788}%
\pgfsetfillcolor{currentfill}%
\pgfsetlinewidth{0.311001pt}%
\definecolor{currentstroke}{rgb}{1.000000,1.000000,1.000000}%
\pgfsetstrokecolor{currentstroke}%
\pgfsetdash{}{0pt}%
\pgfpathmoveto{\pgfqpoint{6.277268in}{1.603863in}}%
\pgfpathcurveto{\pgfqpoint{6.284401in}{1.603863in}}{\pgfqpoint{6.291243in}{1.606697in}}{\pgfqpoint{6.296287in}{1.611740in}}%
\pgfpathcurveto{\pgfqpoint{6.301330in}{1.616784in}}{\pgfqpoint{6.304164in}{1.623626in}}{\pgfqpoint{6.304164in}{1.630758in}}%
\pgfpathcurveto{\pgfqpoint{6.304164in}{1.637891in}}{\pgfqpoint{6.301330in}{1.644733in}}{\pgfqpoint{6.296287in}{1.649777in}}%
\pgfpathcurveto{\pgfqpoint{6.291243in}{1.654820in}}{\pgfqpoint{6.284401in}{1.657654in}}{\pgfqpoint{6.277268in}{1.657654in}}%
\pgfpathcurveto{\pgfqpoint{6.270136in}{1.657654in}}{\pgfqpoint{6.263294in}{1.654820in}}{\pgfqpoint{6.258250in}{1.649777in}}%
\pgfpathcurveto{\pgfqpoint{6.253207in}{1.644733in}}{\pgfqpoint{6.250373in}{1.637891in}}{\pgfqpoint{6.250373in}{1.630758in}}%
\pgfpathcurveto{\pgfqpoint{6.250373in}{1.623626in}}{\pgfqpoint{6.253207in}{1.616784in}}{\pgfqpoint{6.258250in}{1.611740in}}%
\pgfpathcurveto{\pgfqpoint{6.263294in}{1.606697in}}{\pgfqpoint{6.270136in}{1.603863in}}{\pgfqpoint{6.277268in}{1.603863in}}%
\pgfpathclose%
\pgfusepath{stroke,fill}%
\end{pgfscope}%
\begin{pgfscope}%
\pgfpathrectangle{\pgfqpoint{4.985294in}{0.500000in}}{\pgfqpoint{1.764706in}{1.700000in}}%
\pgfusepath{clip}%
\pgfsetbuttcap%
\pgfsetroundjoin%
\definecolor{currentfill}{rgb}{0.972201,0.839051,0.745789}%
\pgfsetfillcolor{currentfill}%
\pgfsetlinewidth{0.311001pt}%
\definecolor{currentstroke}{rgb}{1.000000,1.000000,1.000000}%
\pgfsetstrokecolor{currentstroke}%
\pgfsetdash{}{0pt}%
\pgfpathmoveto{\pgfqpoint{6.203056in}{1.057898in}}%
\pgfpathcurveto{\pgfqpoint{6.210188in}{1.057898in}}{\pgfqpoint{6.217030in}{1.060732in}}{\pgfqpoint{6.222074in}{1.065776in}}%
\pgfpathcurveto{\pgfqpoint{6.227117in}{1.070820in}}{\pgfqpoint{6.229951in}{1.077661in}}{\pgfqpoint{6.229951in}{1.084794in}}%
\pgfpathcurveto{\pgfqpoint{6.229951in}{1.091927in}}{\pgfqpoint{6.227117in}{1.098769in}}{\pgfqpoint{6.222074in}{1.103812in}}%
\pgfpathcurveto{\pgfqpoint{6.217030in}{1.108856in}}{\pgfqpoint{6.210188in}{1.111690in}}{\pgfqpoint{6.203056in}{1.111690in}}%
\pgfpathcurveto{\pgfqpoint{6.195923in}{1.111690in}}{\pgfqpoint{6.189081in}{1.108856in}}{\pgfqpoint{6.184037in}{1.103812in}}%
\pgfpathcurveto{\pgfqpoint{6.178994in}{1.098769in}}{\pgfqpoint{6.176160in}{1.091927in}}{\pgfqpoint{6.176160in}{1.084794in}}%
\pgfpathcurveto{\pgfqpoint{6.176160in}{1.077661in}}{\pgfqpoint{6.178994in}{1.070820in}}{\pgfqpoint{6.184037in}{1.065776in}}%
\pgfpathcurveto{\pgfqpoint{6.189081in}{1.060732in}}{\pgfqpoint{6.195923in}{1.057898in}}{\pgfqpoint{6.203056in}{1.057898in}}%
\pgfpathclose%
\pgfusepath{stroke,fill}%
\end{pgfscope}%
\begin{pgfscope}%
\pgfpathrectangle{\pgfqpoint{4.985294in}{0.500000in}}{\pgfqpoint{1.764706in}{1.700000in}}%
\pgfusepath{clip}%
\pgfsetbuttcap%
\pgfsetroundjoin%
\definecolor{currentfill}{rgb}{0.974412,0.862387,0.780156}%
\pgfsetfillcolor{currentfill}%
\pgfsetlinewidth{0.311001pt}%
\definecolor{currentstroke}{rgb}{1.000000,1.000000,1.000000}%
\pgfsetstrokecolor{currentstroke}%
\pgfsetdash{}{0pt}%
\pgfpathmoveto{\pgfqpoint{6.311267in}{1.110224in}}%
\pgfpathcurveto{\pgfqpoint{6.318400in}{1.110224in}}{\pgfqpoint{6.325242in}{1.113058in}}{\pgfqpoint{6.330286in}{1.118102in}}%
\pgfpathcurveto{\pgfqpoint{6.335329in}{1.123145in}}{\pgfqpoint{6.338163in}{1.129987in}}{\pgfqpoint{6.338163in}{1.137120in}}%
\pgfpathcurveto{\pgfqpoint{6.338163in}{1.144252in}}{\pgfqpoint{6.335329in}{1.151094in}}{\pgfqpoint{6.330286in}{1.156138in}}%
\pgfpathcurveto{\pgfqpoint{6.325242in}{1.161181in}}{\pgfqpoint{6.318400in}{1.164015in}}{\pgfqpoint{6.311267in}{1.164015in}}%
\pgfpathcurveto{\pgfqpoint{6.304135in}{1.164015in}}{\pgfqpoint{6.297293in}{1.161181in}}{\pgfqpoint{6.292249in}{1.156138in}}%
\pgfpathcurveto{\pgfqpoint{6.287206in}{1.151094in}}{\pgfqpoint{6.284372in}{1.144252in}}{\pgfqpoint{6.284372in}{1.137120in}}%
\pgfpathcurveto{\pgfqpoint{6.284372in}{1.129987in}}{\pgfqpoint{6.287206in}{1.123145in}}{\pgfqpoint{6.292249in}{1.118102in}}%
\pgfpathcurveto{\pgfqpoint{6.297293in}{1.113058in}}{\pgfqpoint{6.304135in}{1.110224in}}{\pgfqpoint{6.311267in}{1.110224in}}%
\pgfpathclose%
\pgfusepath{stroke,fill}%
\end{pgfscope}%
\begin{pgfscope}%
\pgfpathrectangle{\pgfqpoint{4.985294in}{0.500000in}}{\pgfqpoint{1.764706in}{1.700000in}}%
\pgfusepath{clip}%
\pgfsetbuttcap%
\pgfsetroundjoin%
\definecolor{currentfill}{rgb}{0.960043,0.546576,0.387029}%
\pgfsetfillcolor{currentfill}%
\pgfsetlinewidth{0.311001pt}%
\definecolor{currentstroke}{rgb}{1.000000,1.000000,1.000000}%
\pgfsetstrokecolor{currentstroke}%
\pgfsetdash{}{0pt}%
\pgfpathmoveto{\pgfqpoint{5.603259in}{1.763072in}}%
\pgfpathcurveto{\pgfqpoint{5.610392in}{1.763072in}}{\pgfqpoint{5.617234in}{1.765906in}}{\pgfqpoint{5.622277in}{1.770950in}}%
\pgfpathcurveto{\pgfqpoint{5.627321in}{1.775994in}}{\pgfqpoint{5.630155in}{1.782835in}}{\pgfqpoint{5.630155in}{1.789968in}}%
\pgfpathcurveto{\pgfqpoint{5.630155in}{1.797101in}}{\pgfqpoint{5.627321in}{1.803943in}}{\pgfqpoint{5.622277in}{1.808986in}}%
\pgfpathcurveto{\pgfqpoint{5.617234in}{1.814030in}}{\pgfqpoint{5.610392in}{1.816864in}}{\pgfqpoint{5.603259in}{1.816864in}}%
\pgfpathcurveto{\pgfqpoint{5.596126in}{1.816864in}}{\pgfqpoint{5.589285in}{1.814030in}}{\pgfqpoint{5.584241in}{1.808986in}}%
\pgfpathcurveto{\pgfqpoint{5.579197in}{1.803943in}}{\pgfqpoint{5.576364in}{1.797101in}}{\pgfqpoint{5.576364in}{1.789968in}}%
\pgfpathcurveto{\pgfqpoint{5.576364in}{1.782835in}}{\pgfqpoint{5.579197in}{1.775994in}}{\pgfqpoint{5.584241in}{1.770950in}}%
\pgfpathcurveto{\pgfqpoint{5.589285in}{1.765906in}}{\pgfqpoint{5.596126in}{1.763072in}}{\pgfqpoint{5.603259in}{1.763072in}}%
\pgfpathclose%
\pgfusepath{stroke,fill}%
\end{pgfscope}%
\begin{pgfscope}%
\pgfpathrectangle{\pgfqpoint{4.985294in}{0.500000in}}{\pgfqpoint{1.764706in}{1.700000in}}%
\pgfusepath{clip}%
\pgfsetbuttcap%
\pgfsetroundjoin%
\definecolor{currentfill}{rgb}{0.971694,0.833208,0.737161}%
\pgfsetfillcolor{currentfill}%
\pgfsetlinewidth{0.311001pt}%
\definecolor{currentstroke}{rgb}{1.000000,1.000000,1.000000}%
\pgfsetstrokecolor{currentstroke}%
\pgfsetdash{}{0pt}%
\pgfpathmoveto{\pgfqpoint{6.352033in}{1.151797in}}%
\pgfpathcurveto{\pgfqpoint{6.359166in}{1.151797in}}{\pgfqpoint{6.366007in}{1.154631in}}{\pgfqpoint{6.371051in}{1.159675in}}%
\pgfpathcurveto{\pgfqpoint{6.376095in}{1.164718in}}{\pgfqpoint{6.378929in}{1.171560in}}{\pgfqpoint{6.378929in}{1.178693in}}%
\pgfpathcurveto{\pgfqpoint{6.378929in}{1.185826in}}{\pgfqpoint{6.376095in}{1.192667in}}{\pgfqpoint{6.371051in}{1.197711in}}%
\pgfpathcurveto{\pgfqpoint{6.366007in}{1.202755in}}{\pgfqpoint{6.359166in}{1.205589in}}{\pgfqpoint{6.352033in}{1.205589in}}%
\pgfpathcurveto{\pgfqpoint{6.344900in}{1.205589in}}{\pgfqpoint{6.338058in}{1.202755in}}{\pgfqpoint{6.333015in}{1.197711in}}%
\pgfpathcurveto{\pgfqpoint{6.327971in}{1.192667in}}{\pgfqpoint{6.325137in}{1.185826in}}{\pgfqpoint{6.325137in}{1.178693in}}%
\pgfpathcurveto{\pgfqpoint{6.325137in}{1.171560in}}{\pgfqpoint{6.327971in}{1.164718in}}{\pgfqpoint{6.333015in}{1.159675in}}%
\pgfpathcurveto{\pgfqpoint{6.338058in}{1.154631in}}{\pgfqpoint{6.344900in}{1.151797in}}{\pgfqpoint{6.352033in}{1.151797in}}%
\pgfpathclose%
\pgfusepath{stroke,fill}%
\end{pgfscope}%
\begin{pgfscope}%
\pgfpathrectangle{\pgfqpoint{4.985294in}{0.500000in}}{\pgfqpoint{1.764706in}{1.700000in}}%
\pgfusepath{clip}%
\pgfsetbuttcap%
\pgfsetroundjoin%
\definecolor{currentfill}{rgb}{0.941676,0.367866,0.260395}%
\pgfsetfillcolor{currentfill}%
\pgfsetlinewidth{0.311001pt}%
\definecolor{currentstroke}{rgb}{1.000000,1.000000,1.000000}%
\pgfsetstrokecolor{currentstroke}%
\pgfsetdash{}{0pt}%
\pgfpathmoveto{\pgfqpoint{6.379480in}{1.678910in}}%
\pgfpathcurveto{\pgfqpoint{6.386613in}{1.678910in}}{\pgfqpoint{6.393455in}{1.681744in}}{\pgfqpoint{6.398498in}{1.686787in}}%
\pgfpathcurveto{\pgfqpoint{6.403542in}{1.691831in}}{\pgfqpoint{6.406376in}{1.698673in}}{\pgfqpoint{6.406376in}{1.705805in}}%
\pgfpathcurveto{\pgfqpoint{6.406376in}{1.712938in}}{\pgfqpoint{6.403542in}{1.719780in}}{\pgfqpoint{6.398498in}{1.724824in}}%
\pgfpathcurveto{\pgfqpoint{6.393455in}{1.729867in}}{\pgfqpoint{6.386613in}{1.732701in}}{\pgfqpoint{6.379480in}{1.732701in}}%
\pgfpathcurveto{\pgfqpoint{6.372347in}{1.732701in}}{\pgfqpoint{6.365506in}{1.729867in}}{\pgfqpoint{6.360462in}{1.724824in}}%
\pgfpathcurveto{\pgfqpoint{6.355418in}{1.719780in}}{\pgfqpoint{6.352584in}{1.712938in}}{\pgfqpoint{6.352584in}{1.705805in}}%
\pgfpathcurveto{\pgfqpoint{6.352584in}{1.698673in}}{\pgfqpoint{6.355418in}{1.691831in}}{\pgfqpoint{6.360462in}{1.686787in}}%
\pgfpathcurveto{\pgfqpoint{6.365506in}{1.681744in}}{\pgfqpoint{6.372347in}{1.678910in}}{\pgfqpoint{6.379480in}{1.678910in}}%
\pgfpathclose%
\pgfusepath{stroke,fill}%
\end{pgfscope}%
\begin{pgfscope}%
\pgfpathrectangle{\pgfqpoint{4.985294in}{0.500000in}}{\pgfqpoint{1.764706in}{1.700000in}}%
\pgfusepath{clip}%
\pgfsetbuttcap%
\pgfsetroundjoin%
\definecolor{currentfill}{rgb}{0.911533,0.252926,0.244703}%
\pgfsetfillcolor{currentfill}%
\pgfsetlinewidth{0.311001pt}%
\definecolor{currentstroke}{rgb}{1.000000,1.000000,1.000000}%
\pgfsetstrokecolor{currentstroke}%
\pgfsetdash{}{0pt}%
\pgfpathmoveto{\pgfqpoint{6.401496in}{1.679290in}}%
\pgfpathcurveto{\pgfqpoint{6.408629in}{1.679290in}}{\pgfqpoint{6.415471in}{1.682124in}}{\pgfqpoint{6.420515in}{1.687168in}}%
\pgfpathcurveto{\pgfqpoint{6.425558in}{1.692211in}}{\pgfqpoint{6.428392in}{1.699053in}}{\pgfqpoint{6.428392in}{1.706186in}}%
\pgfpathcurveto{\pgfqpoint{6.428392in}{1.713318in}}{\pgfqpoint{6.425558in}{1.720160in}}{\pgfqpoint{6.420515in}{1.725204in}}%
\pgfpathcurveto{\pgfqpoint{6.415471in}{1.730247in}}{\pgfqpoint{6.408629in}{1.733081in}}{\pgfqpoint{6.401496in}{1.733081in}}%
\pgfpathcurveto{\pgfqpoint{6.394364in}{1.733081in}}{\pgfqpoint{6.387522in}{1.730247in}}{\pgfqpoint{6.382478in}{1.725204in}}%
\pgfpathcurveto{\pgfqpoint{6.377435in}{1.720160in}}{\pgfqpoint{6.374601in}{1.713318in}}{\pgfqpoint{6.374601in}{1.706186in}}%
\pgfpathcurveto{\pgfqpoint{6.374601in}{1.699053in}}{\pgfqpoint{6.377435in}{1.692211in}}{\pgfqpoint{6.382478in}{1.687168in}}%
\pgfpathcurveto{\pgfqpoint{6.387522in}{1.682124in}}{\pgfqpoint{6.394364in}{1.679290in}}{\pgfqpoint{6.401496in}{1.679290in}}%
\pgfpathclose%
\pgfusepath{stroke,fill}%
\end{pgfscope}%
\begin{pgfscope}%
\pgfpathrectangle{\pgfqpoint{4.985294in}{0.500000in}}{\pgfqpoint{1.764706in}{1.700000in}}%
\pgfusepath{clip}%
\pgfsetbuttcap%
\pgfsetroundjoin%
\definecolor{currentfill}{rgb}{0.018319,0.022977,0.107385}%
\pgfsetfillcolor{currentfill}%
\pgfsetlinewidth{0.311001pt}%
\definecolor{currentstroke}{rgb}{1.000000,1.000000,1.000000}%
\pgfsetstrokecolor{currentstroke}%
\pgfsetdash{}{0pt}%
\pgfpathmoveto{\pgfqpoint{5.612272in}{0.609349in}}%
\pgfpathcurveto{\pgfqpoint{5.619405in}{0.609349in}}{\pgfqpoint{5.626246in}{0.612183in}}{\pgfqpoint{5.631290in}{0.617226in}}%
\pgfpathcurveto{\pgfqpoint{5.636334in}{0.622270in}}{\pgfqpoint{5.639168in}{0.629112in}}{\pgfqpoint{5.639168in}{0.636244in}}%
\pgfpathcurveto{\pgfqpoint{5.639168in}{0.643377in}}{\pgfqpoint{5.636334in}{0.650219in}}{\pgfqpoint{5.631290in}{0.655263in}}%
\pgfpathcurveto{\pgfqpoint{5.626246in}{0.660306in}}{\pgfqpoint{5.619405in}{0.663140in}}{\pgfqpoint{5.612272in}{0.663140in}}%
\pgfpathcurveto{\pgfqpoint{5.605139in}{0.663140in}}{\pgfqpoint{5.598297in}{0.660306in}}{\pgfqpoint{5.593254in}{0.655263in}}%
\pgfpathcurveto{\pgfqpoint{5.588210in}{0.650219in}}{\pgfqpoint{5.585376in}{0.643377in}}{\pgfqpoint{5.585376in}{0.636244in}}%
\pgfpathcurveto{\pgfqpoint{5.585376in}{0.629112in}}{\pgfqpoint{5.588210in}{0.622270in}}{\pgfqpoint{5.593254in}{0.617226in}}%
\pgfpathcurveto{\pgfqpoint{5.598297in}{0.612183in}}{\pgfqpoint{5.605139in}{0.609349in}}{\pgfqpoint{5.612272in}{0.609349in}}%
\pgfpathclose%
\pgfusepath{stroke,fill}%
\end{pgfscope}%
\begin{pgfscope}%
\pgfpathrectangle{\pgfqpoint{4.985294in}{0.500000in}}{\pgfqpoint{1.764706in}{1.700000in}}%
\pgfusepath{clip}%
\pgfsetbuttcap%
\pgfsetroundjoin%
\definecolor{currentfill}{rgb}{0.964920,0.695342,0.545192}%
\pgfsetfillcolor{currentfill}%
\pgfsetlinewidth{0.311001pt}%
\definecolor{currentstroke}{rgb}{1.000000,1.000000,1.000000}%
\pgfsetstrokecolor{currentstroke}%
\pgfsetdash{}{0pt}%
\pgfpathmoveto{\pgfqpoint{5.604069in}{1.019714in}}%
\pgfpathcurveto{\pgfqpoint{5.611202in}{1.019714in}}{\pgfqpoint{5.618044in}{1.022547in}}{\pgfqpoint{5.623087in}{1.027591in}}%
\pgfpathcurveto{\pgfqpoint{5.628131in}{1.032635in}}{\pgfqpoint{5.630965in}{1.039476in}}{\pgfqpoint{5.630965in}{1.046609in}}%
\pgfpathcurveto{\pgfqpoint{5.630965in}{1.053742in}}{\pgfqpoint{5.628131in}{1.060584in}}{\pgfqpoint{5.623087in}{1.065627in}}%
\pgfpathcurveto{\pgfqpoint{5.618044in}{1.070671in}}{\pgfqpoint{5.611202in}{1.073505in}}{\pgfqpoint{5.604069in}{1.073505in}}%
\pgfpathcurveto{\pgfqpoint{5.596936in}{1.073505in}}{\pgfqpoint{5.590095in}{1.070671in}}{\pgfqpoint{5.585051in}{1.065627in}}%
\pgfpathcurveto{\pgfqpoint{5.580007in}{1.060584in}}{\pgfqpoint{5.577174in}{1.053742in}}{\pgfqpoint{5.577174in}{1.046609in}}%
\pgfpathcurveto{\pgfqpoint{5.577174in}{1.039476in}}{\pgfqpoint{5.580007in}{1.032635in}}{\pgfqpoint{5.585051in}{1.027591in}}%
\pgfpathcurveto{\pgfqpoint{5.590095in}{1.022547in}}{\pgfqpoint{5.596936in}{1.019714in}}{\pgfqpoint{5.604069in}{1.019714in}}%
\pgfpathclose%
\pgfusepath{stroke,fill}%
\end{pgfscope}%
\begin{pgfscope}%
\pgfpathrectangle{\pgfqpoint{4.985294in}{0.500000in}}{\pgfqpoint{1.764706in}{1.700000in}}%
\pgfusepath{clip}%
\pgfsetbuttcap%
\pgfsetroundjoin%
\definecolor{currentfill}{rgb}{0.981377,0.920617,0.865369}%
\pgfsetfillcolor{currentfill}%
\pgfsetlinewidth{0.311001pt}%
\definecolor{currentstroke}{rgb}{1.000000,1.000000,1.000000}%
\pgfsetstrokecolor{currentstroke}%
\pgfsetdash{}{0pt}%
\pgfpathmoveto{\pgfqpoint{6.323015in}{1.337147in}}%
\pgfpathcurveto{\pgfqpoint{6.330148in}{1.337147in}}{\pgfqpoint{6.336990in}{1.339981in}}{\pgfqpoint{6.342034in}{1.345024in}}%
\pgfpathcurveto{\pgfqpoint{6.347077in}{1.350068in}}{\pgfqpoint{6.349911in}{1.356910in}}{\pgfqpoint{6.349911in}{1.364043in}}%
\pgfpathcurveto{\pgfqpoint{6.349911in}{1.371175in}}{\pgfqpoint{6.347077in}{1.378017in}}{\pgfqpoint{6.342034in}{1.383061in}}%
\pgfpathcurveto{\pgfqpoint{6.336990in}{1.388104in}}{\pgfqpoint{6.330148in}{1.390938in}}{\pgfqpoint{6.323015in}{1.390938in}}%
\pgfpathcurveto{\pgfqpoint{6.315883in}{1.390938in}}{\pgfqpoint{6.309041in}{1.388104in}}{\pgfqpoint{6.303997in}{1.383061in}}%
\pgfpathcurveto{\pgfqpoint{6.298954in}{1.378017in}}{\pgfqpoint{6.296120in}{1.371175in}}{\pgfqpoint{6.296120in}{1.364043in}}%
\pgfpathcurveto{\pgfqpoint{6.296120in}{1.356910in}}{\pgfqpoint{6.298954in}{1.350068in}}{\pgfqpoint{6.303997in}{1.345024in}}%
\pgfpathcurveto{\pgfqpoint{6.309041in}{1.339981in}}{\pgfqpoint{6.315883in}{1.337147in}}{\pgfqpoint{6.323015in}{1.337147in}}%
\pgfpathclose%
\pgfusepath{stroke,fill}%
\end{pgfscope}%
\begin{pgfscope}%
\pgfpathrectangle{\pgfqpoint{4.985294in}{0.500000in}}{\pgfqpoint{1.764706in}{1.700000in}}%
\pgfusepath{clip}%
\pgfsetbuttcap%
\pgfsetroundjoin%
\definecolor{currentfill}{rgb}{0.922239,0.282873,0.242296}%
\pgfsetfillcolor{currentfill}%
\pgfsetlinewidth{0.311001pt}%
\definecolor{currentstroke}{rgb}{1.000000,1.000000,1.000000}%
\pgfsetstrokecolor{currentstroke}%
\pgfsetdash{}{0pt}%
\pgfpathmoveto{\pgfqpoint{5.537139in}{1.828424in}}%
\pgfpathcurveto{\pgfqpoint{5.544272in}{1.828424in}}{\pgfqpoint{5.551113in}{1.831258in}}{\pgfqpoint{5.556157in}{1.836302in}}%
\pgfpathcurveto{\pgfqpoint{5.561201in}{1.841346in}}{\pgfqpoint{5.564035in}{1.848187in}}{\pgfqpoint{5.564035in}{1.855320in}}%
\pgfpathcurveto{\pgfqpoint{5.564035in}{1.862453in}}{\pgfqpoint{5.561201in}{1.869295in}}{\pgfqpoint{5.556157in}{1.874338in}}%
\pgfpathcurveto{\pgfqpoint{5.551113in}{1.879382in}}{\pgfqpoint{5.544272in}{1.882216in}}{\pgfqpoint{5.537139in}{1.882216in}}%
\pgfpathcurveto{\pgfqpoint{5.530006in}{1.882216in}}{\pgfqpoint{5.523164in}{1.879382in}}{\pgfqpoint{5.518121in}{1.874338in}}%
\pgfpathcurveto{\pgfqpoint{5.513077in}{1.869295in}}{\pgfqpoint{5.510243in}{1.862453in}}{\pgfqpoint{5.510243in}{1.855320in}}%
\pgfpathcurveto{\pgfqpoint{5.510243in}{1.848187in}}{\pgfqpoint{5.513077in}{1.841346in}}{\pgfqpoint{5.518121in}{1.836302in}}%
\pgfpathcurveto{\pgfqpoint{5.523164in}{1.831258in}}{\pgfqpoint{5.530006in}{1.828424in}}{\pgfqpoint{5.537139in}{1.828424in}}%
\pgfpathclose%
\pgfusepath{stroke,fill}%
\end{pgfscope}%
\begin{pgfscope}%
\pgfpathrectangle{\pgfqpoint{4.985294in}{0.500000in}}{\pgfqpoint{1.764706in}{1.700000in}}%
\pgfusepath{clip}%
\pgfsetbuttcap%
\pgfsetroundjoin%
\definecolor{currentfill}{rgb}{0.965753,0.732351,0.592427}%
\pgfsetfillcolor{currentfill}%
\pgfsetlinewidth{0.311001pt}%
\definecolor{currentstroke}{rgb}{1.000000,1.000000,1.000000}%
\pgfsetstrokecolor{currentstroke}%
\pgfsetdash{}{0pt}%
\pgfpathmoveto{\pgfqpoint{5.583680in}{1.034765in}}%
\pgfpathcurveto{\pgfqpoint{5.590813in}{1.034765in}}{\pgfqpoint{5.597655in}{1.037598in}}{\pgfqpoint{5.602698in}{1.042642in}}%
\pgfpathcurveto{\pgfqpoint{5.607742in}{1.047686in}}{\pgfqpoint{5.610576in}{1.054527in}}{\pgfqpoint{5.610576in}{1.061660in}}%
\pgfpathcurveto{\pgfqpoint{5.610576in}{1.068793in}}{\pgfqpoint{5.607742in}{1.075635in}}{\pgfqpoint{5.602698in}{1.080678in}}%
\pgfpathcurveto{\pgfqpoint{5.597655in}{1.085722in}}{\pgfqpoint{5.590813in}{1.088556in}}{\pgfqpoint{5.583680in}{1.088556in}}%
\pgfpathcurveto{\pgfqpoint{5.576547in}{1.088556in}}{\pgfqpoint{5.569706in}{1.085722in}}{\pgfqpoint{5.564662in}{1.080678in}}%
\pgfpathcurveto{\pgfqpoint{5.559618in}{1.075635in}}{\pgfqpoint{5.556785in}{1.068793in}}{\pgfqpoint{5.556785in}{1.061660in}}%
\pgfpathcurveto{\pgfqpoint{5.556785in}{1.054527in}}{\pgfqpoint{5.559618in}{1.047686in}}{\pgfqpoint{5.564662in}{1.042642in}}%
\pgfpathcurveto{\pgfqpoint{5.569706in}{1.037598in}}{\pgfqpoint{5.576547in}{1.034765in}}{\pgfqpoint{5.583680in}{1.034765in}}%
\pgfpathclose%
\pgfusepath{stroke,fill}%
\end{pgfscope}%
\begin{pgfscope}%
\pgfpathrectangle{\pgfqpoint{4.985294in}{0.500000in}}{\pgfqpoint{1.764706in}{1.700000in}}%
\pgfusepath{clip}%
\pgfsetbuttcap%
\pgfsetroundjoin%
\definecolor{currentfill}{rgb}{0.963190,0.619109,0.458249}%
\pgfsetfillcolor{currentfill}%
\pgfsetlinewidth{0.311001pt}%
\definecolor{currentstroke}{rgb}{1.000000,1.000000,1.000000}%
\pgfsetstrokecolor{currentstroke}%
\pgfsetdash{}{0pt}%
\pgfpathmoveto{\pgfqpoint{6.409175in}{1.160139in}}%
\pgfpathcurveto{\pgfqpoint{6.416307in}{1.160139in}}{\pgfqpoint{6.423149in}{1.162973in}}{\pgfqpoint{6.428193in}{1.168017in}}%
\pgfpathcurveto{\pgfqpoint{6.433236in}{1.173061in}}{\pgfqpoint{6.436070in}{1.179902in}}{\pgfqpoint{6.436070in}{1.187035in}}%
\pgfpathcurveto{\pgfqpoint{6.436070in}{1.194168in}}{\pgfqpoint{6.433236in}{1.201010in}}{\pgfqpoint{6.428193in}{1.206053in}}%
\pgfpathcurveto{\pgfqpoint{6.423149in}{1.211097in}}{\pgfqpoint{6.416307in}{1.213931in}}{\pgfqpoint{6.409175in}{1.213931in}}%
\pgfpathcurveto{\pgfqpoint{6.402042in}{1.213931in}}{\pgfqpoint{6.395200in}{1.211097in}}{\pgfqpoint{6.390156in}{1.206053in}}%
\pgfpathcurveto{\pgfqpoint{6.385113in}{1.201010in}}{\pgfqpoint{6.382279in}{1.194168in}}{\pgfqpoint{6.382279in}{1.187035in}}%
\pgfpathcurveto{\pgfqpoint{6.382279in}{1.179902in}}{\pgfqpoint{6.385113in}{1.173061in}}{\pgfqpoint{6.390156in}{1.168017in}}%
\pgfpathcurveto{\pgfqpoint{6.395200in}{1.162973in}}{\pgfqpoint{6.402042in}{1.160139in}}{\pgfqpoint{6.409175in}{1.160139in}}%
\pgfpathclose%
\pgfusepath{stroke,fill}%
\end{pgfscope}%
\begin{pgfscope}%
\pgfpathrectangle{\pgfqpoint{4.985294in}{0.500000in}}{\pgfqpoint{1.764706in}{1.700000in}}%
\pgfusepath{clip}%
\pgfsetbuttcap%
\pgfsetroundjoin%
\definecolor{currentfill}{rgb}{0.973832,0.856556,0.771584}%
\pgfsetfillcolor{currentfill}%
\pgfsetlinewidth{0.311001pt}%
\definecolor{currentstroke}{rgb}{1.000000,1.000000,1.000000}%
\pgfsetstrokecolor{currentstroke}%
\pgfsetdash{}{0pt}%
\pgfpathmoveto{\pgfqpoint{6.251085in}{1.235284in}}%
\pgfpathcurveto{\pgfqpoint{6.258218in}{1.235284in}}{\pgfqpoint{6.265059in}{1.238117in}}{\pgfqpoint{6.270103in}{1.243161in}}%
\pgfpathcurveto{\pgfqpoint{6.275147in}{1.248205in}}{\pgfqpoint{6.277981in}{1.255046in}}{\pgfqpoint{6.277981in}{1.262179in}}%
\pgfpathcurveto{\pgfqpoint{6.277981in}{1.269312in}}{\pgfqpoint{6.275147in}{1.276154in}}{\pgfqpoint{6.270103in}{1.281197in}}%
\pgfpathcurveto{\pgfqpoint{6.265059in}{1.286241in}}{\pgfqpoint{6.258218in}{1.289075in}}{\pgfqpoint{6.251085in}{1.289075in}}%
\pgfpathcurveto{\pgfqpoint{6.243952in}{1.289075in}}{\pgfqpoint{6.237111in}{1.286241in}}{\pgfqpoint{6.232067in}{1.281197in}}%
\pgfpathcurveto{\pgfqpoint{6.227023in}{1.276154in}}{\pgfqpoint{6.224189in}{1.269312in}}{\pgfqpoint{6.224189in}{1.262179in}}%
\pgfpathcurveto{\pgfqpoint{6.224189in}{1.255046in}}{\pgfqpoint{6.227023in}{1.248205in}}{\pgfqpoint{6.232067in}{1.243161in}}%
\pgfpathcurveto{\pgfqpoint{6.237111in}{1.238117in}}{\pgfqpoint{6.243952in}{1.235284in}}{\pgfqpoint{6.251085in}{1.235284in}}%
\pgfpathclose%
\pgfusepath{stroke,fill}%
\end{pgfscope}%
\begin{pgfscope}%
\pgfpathrectangle{\pgfqpoint{4.985294in}{0.500000in}}{\pgfqpoint{1.764706in}{1.700000in}}%
\pgfusepath{clip}%
\pgfsetbuttcap%
\pgfsetroundjoin%
\definecolor{currentfill}{rgb}{0.965753,0.732351,0.592427}%
\pgfsetfillcolor{currentfill}%
\pgfsetlinewidth{0.311001pt}%
\definecolor{currentstroke}{rgb}{1.000000,1.000000,1.000000}%
\pgfsetstrokecolor{currentstroke}%
\pgfsetdash{}{0pt}%
\pgfpathmoveto{\pgfqpoint{6.203202in}{1.395787in}}%
\pgfpathcurveto{\pgfqpoint{6.210334in}{1.395787in}}{\pgfqpoint{6.217176in}{1.398621in}}{\pgfqpoint{6.222220in}{1.403665in}}%
\pgfpathcurveto{\pgfqpoint{6.227263in}{1.408708in}}{\pgfqpoint{6.230097in}{1.415550in}}{\pgfqpoint{6.230097in}{1.422683in}}%
\pgfpathcurveto{\pgfqpoint{6.230097in}{1.429816in}}{\pgfqpoint{6.227263in}{1.436657in}}{\pgfqpoint{6.222220in}{1.441701in}}%
\pgfpathcurveto{\pgfqpoint{6.217176in}{1.446745in}}{\pgfqpoint{6.210334in}{1.449578in}}{\pgfqpoint{6.203202in}{1.449578in}}%
\pgfpathcurveto{\pgfqpoint{6.196069in}{1.449578in}}{\pgfqpoint{6.189227in}{1.446745in}}{\pgfqpoint{6.184183in}{1.441701in}}%
\pgfpathcurveto{\pgfqpoint{6.179140in}{1.436657in}}{\pgfqpoint{6.176306in}{1.429816in}}{\pgfqpoint{6.176306in}{1.422683in}}%
\pgfpathcurveto{\pgfqpoint{6.176306in}{1.415550in}}{\pgfqpoint{6.179140in}{1.408708in}}{\pgfqpoint{6.184183in}{1.403665in}}%
\pgfpathcurveto{\pgfqpoint{6.189227in}{1.398621in}}{\pgfqpoint{6.196069in}{1.395787in}}{\pgfqpoint{6.203202in}{1.395787in}}%
\pgfpathclose%
\pgfusepath{stroke,fill}%
\end{pgfscope}%
\begin{pgfscope}%
\pgfpathrectangle{\pgfqpoint{4.985294in}{0.500000in}}{\pgfqpoint{1.764706in}{1.700000in}}%
\pgfusepath{clip}%
\pgfsetbuttcap%
\pgfsetroundjoin%
\definecolor{currentfill}{rgb}{0.979891,0.908948,0.848279}%
\pgfsetfillcolor{currentfill}%
\pgfsetlinewidth{0.311001pt}%
\definecolor{currentstroke}{rgb}{1.000000,1.000000,1.000000}%
\pgfsetstrokecolor{currentstroke}%
\pgfsetdash{}{0pt}%
\pgfpathmoveto{\pgfqpoint{5.407379in}{1.371429in}}%
\pgfpathcurveto{\pgfqpoint{5.414512in}{1.371429in}}{\pgfqpoint{5.421354in}{1.374263in}}{\pgfqpoint{5.426397in}{1.379306in}}%
\pgfpathcurveto{\pgfqpoint{5.431441in}{1.384350in}}{\pgfqpoint{5.434275in}{1.391192in}}{\pgfqpoint{5.434275in}{1.398324in}}%
\pgfpathcurveto{\pgfqpoint{5.434275in}{1.405457in}}{\pgfqpoint{5.431441in}{1.412299in}}{\pgfqpoint{5.426397in}{1.417343in}}%
\pgfpathcurveto{\pgfqpoint{5.421354in}{1.422386in}}{\pgfqpoint{5.414512in}{1.425220in}}{\pgfqpoint{5.407379in}{1.425220in}}%
\pgfpathcurveto{\pgfqpoint{5.400246in}{1.425220in}}{\pgfqpoint{5.393405in}{1.422386in}}{\pgfqpoint{5.388361in}{1.417343in}}%
\pgfpathcurveto{\pgfqpoint{5.383317in}{1.412299in}}{\pgfqpoint{5.380483in}{1.405457in}}{\pgfqpoint{5.380483in}{1.398324in}}%
\pgfpathcurveto{\pgfqpoint{5.380483in}{1.391192in}}{\pgfqpoint{5.383317in}{1.384350in}}{\pgfqpoint{5.388361in}{1.379306in}}%
\pgfpathcurveto{\pgfqpoint{5.393405in}{1.374263in}}{\pgfqpoint{5.400246in}{1.371429in}}{\pgfqpoint{5.407379in}{1.371429in}}%
\pgfpathclose%
\pgfusepath{stroke,fill}%
\end{pgfscope}%
\begin{pgfscope}%
\pgfpathrectangle{\pgfqpoint{4.985294in}{0.500000in}}{\pgfqpoint{1.764706in}{1.700000in}}%
\pgfusepath{clip}%
\pgfsetbuttcap%
\pgfsetroundjoin%
\definecolor{currentfill}{rgb}{0.965928,0.738443,0.600540}%
\pgfsetfillcolor{currentfill}%
\pgfsetlinewidth{0.311001pt}%
\definecolor{currentstroke}{rgb}{1.000000,1.000000,1.000000}%
\pgfsetstrokecolor{currentstroke}%
\pgfsetdash{}{0pt}%
\pgfpathmoveto{\pgfqpoint{6.168304in}{1.541686in}}%
\pgfpathcurveto{\pgfqpoint{6.175437in}{1.541686in}}{\pgfqpoint{6.182278in}{1.544520in}}{\pgfqpoint{6.187322in}{1.549563in}}%
\pgfpathcurveto{\pgfqpoint{6.192366in}{1.554607in}}{\pgfqpoint{6.195200in}{1.561449in}}{\pgfqpoint{6.195200in}{1.568582in}}%
\pgfpathcurveto{\pgfqpoint{6.195200in}{1.575714in}}{\pgfqpoint{6.192366in}{1.582556in}}{\pgfqpoint{6.187322in}{1.587600in}}%
\pgfpathcurveto{\pgfqpoint{6.182278in}{1.592643in}}{\pgfqpoint{6.175437in}{1.595477in}}{\pgfqpoint{6.168304in}{1.595477in}}%
\pgfpathcurveto{\pgfqpoint{6.161171in}{1.595477in}}{\pgfqpoint{6.154330in}{1.592643in}}{\pgfqpoint{6.149286in}{1.587600in}}%
\pgfpathcurveto{\pgfqpoint{6.144242in}{1.582556in}}{\pgfqpoint{6.141408in}{1.575714in}}{\pgfqpoint{6.141408in}{1.568582in}}%
\pgfpathcurveto{\pgfqpoint{6.141408in}{1.561449in}}{\pgfqpoint{6.144242in}{1.554607in}}{\pgfqpoint{6.149286in}{1.549563in}}%
\pgfpathcurveto{\pgfqpoint{6.154330in}{1.544520in}}{\pgfqpoint{6.161171in}{1.541686in}}{\pgfqpoint{6.168304in}{1.541686in}}%
\pgfpathclose%
\pgfusepath{stroke,fill}%
\end{pgfscope}%
\begin{pgfscope}%
\pgfpathrectangle{\pgfqpoint{4.985294in}{0.500000in}}{\pgfqpoint{1.764706in}{1.700000in}}%
\pgfusepath{clip}%
\pgfsetbuttcap%
\pgfsetroundjoin%
\definecolor{currentfill}{rgb}{0.979891,0.908948,0.848279}%
\pgfsetfillcolor{currentfill}%
\pgfsetlinewidth{0.311001pt}%
\definecolor{currentstroke}{rgb}{1.000000,1.000000,1.000000}%
\pgfsetstrokecolor{currentstroke}%
\pgfsetdash{}{0pt}%
\pgfpathmoveto{\pgfqpoint{6.299025in}{1.531893in}}%
\pgfpathcurveto{\pgfqpoint{6.306158in}{1.531893in}}{\pgfqpoint{6.313000in}{1.534726in}}{\pgfqpoint{6.318044in}{1.539770in}}%
\pgfpathcurveto{\pgfqpoint{6.323087in}{1.544814in}}{\pgfqpoint{6.325921in}{1.551655in}}{\pgfqpoint{6.325921in}{1.558788in}}%
\pgfpathcurveto{\pgfqpoint{6.325921in}{1.565921in}}{\pgfqpoint{6.323087in}{1.572763in}}{\pgfqpoint{6.318044in}{1.577806in}}%
\pgfpathcurveto{\pgfqpoint{6.313000in}{1.582850in}}{\pgfqpoint{6.306158in}{1.585684in}}{\pgfqpoint{6.299025in}{1.585684in}}%
\pgfpathcurveto{\pgfqpoint{6.291893in}{1.585684in}}{\pgfqpoint{6.285051in}{1.582850in}}{\pgfqpoint{6.280007in}{1.577806in}}%
\pgfpathcurveto{\pgfqpoint{6.274964in}{1.572763in}}{\pgfqpoint{6.272130in}{1.565921in}}{\pgfqpoint{6.272130in}{1.558788in}}%
\pgfpathcurveto{\pgfqpoint{6.272130in}{1.551655in}}{\pgfqpoint{6.274964in}{1.544814in}}{\pgfqpoint{6.280007in}{1.539770in}}%
\pgfpathcurveto{\pgfqpoint{6.285051in}{1.534726in}}{\pgfqpoint{6.291893in}{1.531893in}}{\pgfqpoint{6.299025in}{1.531893in}}%
\pgfpathclose%
\pgfusepath{stroke,fill}%
\end{pgfscope}%
\begin{pgfscope}%
\pgfpathrectangle{\pgfqpoint{4.985294in}{0.500000in}}{\pgfqpoint{1.764706in}{1.700000in}}%
\pgfusepath{clip}%
\pgfsetbuttcap%
\pgfsetroundjoin%
\definecolor{currentfill}{rgb}{0.965302,0.713942,0.568499}%
\pgfsetfillcolor{currentfill}%
\pgfsetlinewidth{0.311001pt}%
\definecolor{currentstroke}{rgb}{1.000000,1.000000,1.000000}%
\pgfsetstrokecolor{currentstroke}%
\pgfsetdash{}{0pt}%
\pgfpathmoveto{\pgfqpoint{6.139295in}{1.581205in}}%
\pgfpathcurveto{\pgfqpoint{6.146428in}{1.581205in}}{\pgfqpoint{6.153270in}{1.584039in}}{\pgfqpoint{6.158313in}{1.589083in}}%
\pgfpathcurveto{\pgfqpoint{6.163357in}{1.594127in}}{\pgfqpoint{6.166191in}{1.600968in}}{\pgfqpoint{6.166191in}{1.608101in}}%
\pgfpathcurveto{\pgfqpoint{6.166191in}{1.615234in}}{\pgfqpoint{6.163357in}{1.622076in}}{\pgfqpoint{6.158313in}{1.627119in}}%
\pgfpathcurveto{\pgfqpoint{6.153270in}{1.632163in}}{\pgfqpoint{6.146428in}{1.634997in}}{\pgfqpoint{6.139295in}{1.634997in}}%
\pgfpathcurveto{\pgfqpoint{6.132162in}{1.634997in}}{\pgfqpoint{6.125321in}{1.632163in}}{\pgfqpoint{6.120277in}{1.627119in}}%
\pgfpathcurveto{\pgfqpoint{6.115233in}{1.622076in}}{\pgfqpoint{6.112399in}{1.615234in}}{\pgfqpoint{6.112399in}{1.608101in}}%
\pgfpathcurveto{\pgfqpoint{6.112399in}{1.600968in}}{\pgfqpoint{6.115233in}{1.594127in}}{\pgfqpoint{6.120277in}{1.589083in}}%
\pgfpathcurveto{\pgfqpoint{6.125321in}{1.584039in}}{\pgfqpoint{6.132162in}{1.581205in}}{\pgfqpoint{6.139295in}{1.581205in}}%
\pgfpathclose%
\pgfusepath{stroke,fill}%
\end{pgfscope}%
\begin{pgfscope}%
\pgfpathrectangle{\pgfqpoint{4.985294in}{0.500000in}}{\pgfqpoint{1.764706in}{1.700000in}}%
\pgfusepath{clip}%
\pgfsetbuttcap%
\pgfsetroundjoin%
\definecolor{currentfill}{rgb}{0.966120,0.744512,0.608720}%
\pgfsetfillcolor{currentfill}%
\pgfsetlinewidth{0.311001pt}%
\definecolor{currentstroke}{rgb}{1.000000,1.000000,1.000000}%
\pgfsetstrokecolor{currentstroke}%
\pgfsetdash{}{0pt}%
\pgfpathmoveto{\pgfqpoint{5.521094in}{1.195379in}}%
\pgfpathcurveto{\pgfqpoint{5.528227in}{1.195379in}}{\pgfqpoint{5.535068in}{1.198213in}}{\pgfqpoint{5.540112in}{1.203256in}}%
\pgfpathcurveto{\pgfqpoint{5.545156in}{1.208300in}}{\pgfqpoint{5.547990in}{1.215142in}}{\pgfqpoint{5.547990in}{1.222274in}}%
\pgfpathcurveto{\pgfqpoint{5.547990in}{1.229407in}}{\pgfqpoint{5.545156in}{1.236249in}}{\pgfqpoint{5.540112in}{1.241293in}}%
\pgfpathcurveto{\pgfqpoint{5.535068in}{1.246336in}}{\pgfqpoint{5.528227in}{1.249170in}}{\pgfqpoint{5.521094in}{1.249170in}}%
\pgfpathcurveto{\pgfqpoint{5.513961in}{1.249170in}}{\pgfqpoint{5.507119in}{1.246336in}}{\pgfqpoint{5.502076in}{1.241293in}}%
\pgfpathcurveto{\pgfqpoint{5.497032in}{1.236249in}}{\pgfqpoint{5.494198in}{1.229407in}}{\pgfqpoint{5.494198in}{1.222274in}}%
\pgfpathcurveto{\pgfqpoint{5.494198in}{1.215142in}}{\pgfqpoint{5.497032in}{1.208300in}}{\pgfqpoint{5.502076in}{1.203256in}}%
\pgfpathcurveto{\pgfqpoint{5.507119in}{1.198213in}}{\pgfqpoint{5.513961in}{1.195379in}}{\pgfqpoint{5.521094in}{1.195379in}}%
\pgfpathclose%
\pgfusepath{stroke,fill}%
\end{pgfscope}%
\begin{pgfscope}%
\pgfpathrectangle{\pgfqpoint{4.985294in}{0.500000in}}{\pgfqpoint{1.764706in}{1.700000in}}%
\pgfusepath{clip}%
\pgfsetbuttcap%
\pgfsetroundjoin%
\definecolor{currentfill}{rgb}{0.976961,0.885681,0.814303}%
\pgfsetfillcolor{currentfill}%
\pgfsetlinewidth{0.311001pt}%
\definecolor{currentstroke}{rgb}{1.000000,1.000000,1.000000}%
\pgfsetstrokecolor{currentstroke}%
\pgfsetdash{}{0pt}%
\pgfpathmoveto{\pgfqpoint{5.435847in}{1.130568in}}%
\pgfpathcurveto{\pgfqpoint{5.442980in}{1.130568in}}{\pgfqpoint{5.449822in}{1.133402in}}{\pgfqpoint{5.454865in}{1.138445in}}%
\pgfpathcurveto{\pgfqpoint{5.459909in}{1.143489in}}{\pgfqpoint{5.462743in}{1.150330in}}{\pgfqpoint{5.462743in}{1.157463in}}%
\pgfpathcurveto{\pgfqpoint{5.462743in}{1.164596in}}{\pgfqpoint{5.459909in}{1.171438in}}{\pgfqpoint{5.454865in}{1.176481in}}%
\pgfpathcurveto{\pgfqpoint{5.449822in}{1.181525in}}{\pgfqpoint{5.442980in}{1.184359in}}{\pgfqpoint{5.435847in}{1.184359in}}%
\pgfpathcurveto{\pgfqpoint{5.428714in}{1.184359in}}{\pgfqpoint{5.421873in}{1.181525in}}{\pgfqpoint{5.416829in}{1.176481in}}%
\pgfpathcurveto{\pgfqpoint{5.411785in}{1.171438in}}{\pgfqpoint{5.408952in}{1.164596in}}{\pgfqpoint{5.408952in}{1.157463in}}%
\pgfpathcurveto{\pgfqpoint{5.408952in}{1.150330in}}{\pgfqpoint{5.411785in}{1.143489in}}{\pgfqpoint{5.416829in}{1.138445in}}%
\pgfpathcurveto{\pgfqpoint{5.421873in}{1.133402in}}{\pgfqpoint{5.428714in}{1.130568in}}{\pgfqpoint{5.435847in}{1.130568in}}%
\pgfpathclose%
\pgfusepath{stroke,fill}%
\end{pgfscope}%
\begin{pgfscope}%
\pgfpathrectangle{\pgfqpoint{4.985294in}{0.500000in}}{\pgfqpoint{1.764706in}{1.700000in}}%
\pgfusepath{clip}%
\pgfsetbuttcap%
\pgfsetroundjoin%
\definecolor{currentfill}{rgb}{0.972726,0.844889,0.754401}%
\pgfsetfillcolor{currentfill}%
\pgfsetlinewidth{0.311001pt}%
\definecolor{currentstroke}{rgb}{1.000000,1.000000,1.000000}%
\pgfsetstrokecolor{currentstroke}%
\pgfsetdash{}{0pt}%
\pgfpathmoveto{\pgfqpoint{6.358074in}{1.182324in}}%
\pgfpathcurveto{\pgfqpoint{6.365207in}{1.182324in}}{\pgfqpoint{6.372049in}{1.185158in}}{\pgfqpoint{6.377093in}{1.190201in}}%
\pgfpathcurveto{\pgfqpoint{6.382136in}{1.195245in}}{\pgfqpoint{6.384970in}{1.202087in}}{\pgfqpoint{6.384970in}{1.209219in}}%
\pgfpathcurveto{\pgfqpoint{6.384970in}{1.216352in}}{\pgfqpoint{6.382136in}{1.223194in}}{\pgfqpoint{6.377093in}{1.228237in}}%
\pgfpathcurveto{\pgfqpoint{6.372049in}{1.233281in}}{\pgfqpoint{6.365207in}{1.236115in}}{\pgfqpoint{6.358074in}{1.236115in}}%
\pgfpathcurveto{\pgfqpoint{6.350942in}{1.236115in}}{\pgfqpoint{6.344100in}{1.233281in}}{\pgfqpoint{6.339056in}{1.228237in}}%
\pgfpathcurveto{\pgfqpoint{6.334013in}{1.223194in}}{\pgfqpoint{6.331179in}{1.216352in}}{\pgfqpoint{6.331179in}{1.209219in}}%
\pgfpathcurveto{\pgfqpoint{6.331179in}{1.202087in}}{\pgfqpoint{6.334013in}{1.195245in}}{\pgfqpoint{6.339056in}{1.190201in}}%
\pgfpathcurveto{\pgfqpoint{6.344100in}{1.185158in}}{\pgfqpoint{6.350942in}{1.182324in}}{\pgfqpoint{6.358074in}{1.182324in}}%
\pgfpathclose%
\pgfusepath{stroke,fill}%
\end{pgfscope}%
\begin{pgfscope}%
\pgfpathrectangle{\pgfqpoint{4.985294in}{0.500000in}}{\pgfqpoint{1.764706in}{1.700000in}}%
\pgfusepath{clip}%
\pgfsetbuttcap%
\pgfsetroundjoin%
\definecolor{currentfill}{rgb}{0.964799,0.689101,0.537560}%
\pgfsetfillcolor{currentfill}%
\pgfsetlinewidth{0.311001pt}%
\definecolor{currentstroke}{rgb}{1.000000,1.000000,1.000000}%
\pgfsetstrokecolor{currentstroke}%
\pgfsetdash{}{0pt}%
\pgfpathmoveto{\pgfqpoint{5.571042in}{1.537812in}}%
\pgfpathcurveto{\pgfqpoint{5.578175in}{1.537812in}}{\pgfqpoint{5.585016in}{1.540646in}}{\pgfqpoint{5.590060in}{1.545690in}}%
\pgfpathcurveto{\pgfqpoint{5.595104in}{1.550734in}}{\pgfqpoint{5.597937in}{1.557575in}}{\pgfqpoint{5.597937in}{1.564708in}}%
\pgfpathcurveto{\pgfqpoint{5.597937in}{1.571841in}}{\pgfqpoint{5.595104in}{1.578682in}}{\pgfqpoint{5.590060in}{1.583726in}}%
\pgfpathcurveto{\pgfqpoint{5.585016in}{1.588770in}}{\pgfqpoint{5.578175in}{1.591604in}}{\pgfqpoint{5.571042in}{1.591604in}}%
\pgfpathcurveto{\pgfqpoint{5.563909in}{1.591604in}}{\pgfqpoint{5.557067in}{1.588770in}}{\pgfqpoint{5.552024in}{1.583726in}}%
\pgfpathcurveto{\pgfqpoint{5.546980in}{1.578682in}}{\pgfqpoint{5.544146in}{1.571841in}}{\pgfqpoint{5.544146in}{1.564708in}}%
\pgfpathcurveto{\pgfqpoint{5.544146in}{1.557575in}}{\pgfqpoint{5.546980in}{1.550734in}}{\pgfqpoint{5.552024in}{1.545690in}}%
\pgfpathcurveto{\pgfqpoint{5.557067in}{1.540646in}}{\pgfqpoint{5.563909in}{1.537812in}}{\pgfqpoint{5.571042in}{1.537812in}}%
\pgfpathclose%
\pgfusepath{stroke,fill}%
\end{pgfscope}%
\begin{pgfscope}%
\pgfpathrectangle{\pgfqpoint{4.985294in}{0.500000in}}{\pgfqpoint{1.764706in}{1.700000in}}%
\pgfusepath{clip}%
\pgfsetbuttcap%
\pgfsetroundjoin%
\definecolor{currentfill}{rgb}{0.961433,0.573272,0.412036}%
\pgfsetfillcolor{currentfill}%
\pgfsetlinewidth{0.311001pt}%
\definecolor{currentstroke}{rgb}{1.000000,1.000000,1.000000}%
\pgfsetstrokecolor{currentstroke}%
\pgfsetdash{}{0pt}%
\pgfpathmoveto{\pgfqpoint{6.233081in}{1.750979in}}%
\pgfpathcurveto{\pgfqpoint{6.240213in}{1.750979in}}{\pgfqpoint{6.247055in}{1.753813in}}{\pgfqpoint{6.252099in}{1.758857in}}%
\pgfpathcurveto{\pgfqpoint{6.257142in}{1.763900in}}{\pgfqpoint{6.259976in}{1.770742in}}{\pgfqpoint{6.259976in}{1.777875in}}%
\pgfpathcurveto{\pgfqpoint{6.259976in}{1.785008in}}{\pgfqpoint{6.257142in}{1.791849in}}{\pgfqpoint{6.252099in}{1.796893in}}%
\pgfpathcurveto{\pgfqpoint{6.247055in}{1.801937in}}{\pgfqpoint{6.240213in}{1.804770in}}{\pgfqpoint{6.233081in}{1.804770in}}%
\pgfpathcurveto{\pgfqpoint{6.225948in}{1.804770in}}{\pgfqpoint{6.219106in}{1.801937in}}{\pgfqpoint{6.214062in}{1.796893in}}%
\pgfpathcurveto{\pgfqpoint{6.209019in}{1.791849in}}{\pgfqpoint{6.206185in}{1.785008in}}{\pgfqpoint{6.206185in}{1.777875in}}%
\pgfpathcurveto{\pgfqpoint{6.206185in}{1.770742in}}{\pgfqpoint{6.209019in}{1.763900in}}{\pgfqpoint{6.214062in}{1.758857in}}%
\pgfpathcurveto{\pgfqpoint{6.219106in}{1.753813in}}{\pgfqpoint{6.225948in}{1.750979in}}{\pgfqpoint{6.233081in}{1.750979in}}%
\pgfpathclose%
\pgfusepath{stroke,fill}%
\end{pgfscope}%
\begin{pgfscope}%
\pgfpathrectangle{\pgfqpoint{4.985294in}{0.500000in}}{\pgfqpoint{1.764706in}{1.700000in}}%
\pgfusepath{clip}%
\pgfsetbuttcap%
\pgfsetroundjoin%
\definecolor{currentfill}{rgb}{0.976961,0.885681,0.814303}%
\pgfsetfillcolor{currentfill}%
\pgfsetlinewidth{0.311001pt}%
\definecolor{currentstroke}{rgb}{1.000000,1.000000,1.000000}%
\pgfsetstrokecolor{currentstroke}%
\pgfsetdash{}{0pt}%
\pgfpathmoveto{\pgfqpoint{6.280612in}{1.373901in}}%
\pgfpathcurveto{\pgfqpoint{6.287745in}{1.373901in}}{\pgfqpoint{6.294586in}{1.376734in}}{\pgfqpoint{6.299630in}{1.381778in}}%
\pgfpathcurveto{\pgfqpoint{6.304674in}{1.386822in}}{\pgfqpoint{6.307508in}{1.393663in}}{\pgfqpoint{6.307508in}{1.400796in}}%
\pgfpathcurveto{\pgfqpoint{6.307508in}{1.407929in}}{\pgfqpoint{6.304674in}{1.414771in}}{\pgfqpoint{6.299630in}{1.419814in}}%
\pgfpathcurveto{\pgfqpoint{6.294586in}{1.424858in}}{\pgfqpoint{6.287745in}{1.427692in}}{\pgfqpoint{6.280612in}{1.427692in}}%
\pgfpathcurveto{\pgfqpoint{6.273479in}{1.427692in}}{\pgfqpoint{6.266638in}{1.424858in}}{\pgfqpoint{6.261594in}{1.419814in}}%
\pgfpathcurveto{\pgfqpoint{6.256550in}{1.414771in}}{\pgfqpoint{6.253716in}{1.407929in}}{\pgfqpoint{6.253716in}{1.400796in}}%
\pgfpathcurveto{\pgfqpoint{6.253716in}{1.393663in}}{\pgfqpoint{6.256550in}{1.386822in}}{\pgfqpoint{6.261594in}{1.381778in}}%
\pgfpathcurveto{\pgfqpoint{6.266638in}{1.376734in}}{\pgfqpoint{6.273479in}{1.373901in}}{\pgfqpoint{6.280612in}{1.373901in}}%
\pgfpathclose%
\pgfusepath{stroke,fill}%
\end{pgfscope}%
\begin{pgfscope}%
\pgfpathrectangle{\pgfqpoint{4.985294in}{0.500000in}}{\pgfqpoint{1.764706in}{1.700000in}}%
\pgfusepath{clip}%
\pgfsetbuttcap%
\pgfsetroundjoin%
\definecolor{currentfill}{rgb}{0.963190,0.619109,0.458249}%
\pgfsetfillcolor{currentfill}%
\pgfsetlinewidth{0.311001pt}%
\definecolor{currentstroke}{rgb}{1.000000,1.000000,1.000000}%
\pgfsetstrokecolor{currentstroke}%
\pgfsetdash{}{0pt}%
\pgfpathmoveto{\pgfqpoint{5.568931in}{1.752969in}}%
\pgfpathcurveto{\pgfqpoint{5.576064in}{1.752969in}}{\pgfqpoint{5.582905in}{1.755802in}}{\pgfqpoint{5.587949in}{1.760846in}}%
\pgfpathcurveto{\pgfqpoint{5.592993in}{1.765890in}}{\pgfqpoint{5.595827in}{1.772731in}}{\pgfqpoint{5.595827in}{1.779864in}}%
\pgfpathcurveto{\pgfqpoint{5.595827in}{1.786997in}}{\pgfqpoint{5.592993in}{1.793839in}}{\pgfqpoint{5.587949in}{1.798882in}}%
\pgfpathcurveto{\pgfqpoint{5.582905in}{1.803926in}}{\pgfqpoint{5.576064in}{1.806760in}}{\pgfqpoint{5.568931in}{1.806760in}}%
\pgfpathcurveto{\pgfqpoint{5.561798in}{1.806760in}}{\pgfqpoint{5.554956in}{1.803926in}}{\pgfqpoint{5.549913in}{1.798882in}}%
\pgfpathcurveto{\pgfqpoint{5.544869in}{1.793839in}}{\pgfqpoint{5.542035in}{1.786997in}}{\pgfqpoint{5.542035in}{1.779864in}}%
\pgfpathcurveto{\pgfqpoint{5.542035in}{1.772731in}}{\pgfqpoint{5.544869in}{1.765890in}}{\pgfqpoint{5.549913in}{1.760846in}}%
\pgfpathcurveto{\pgfqpoint{5.554956in}{1.755802in}}{\pgfqpoint{5.561798in}{1.752969in}}{\pgfqpoint{5.568931in}{1.752969in}}%
\pgfpathclose%
\pgfusepath{stroke,fill}%
\end{pgfscope}%
\begin{pgfscope}%
\pgfpathrectangle{\pgfqpoint{4.985294in}{0.500000in}}{\pgfqpoint{1.764706in}{1.700000in}}%
\pgfusepath{clip}%
\pgfsetbuttcap%
\pgfsetroundjoin%
\definecolor{currentfill}{rgb}{0.967398,0.774513,0.650573}%
\pgfsetfillcolor{currentfill}%
\pgfsetlinewidth{0.311001pt}%
\definecolor{currentstroke}{rgb}{1.000000,1.000000,1.000000}%
\pgfsetstrokecolor{currentstroke}%
\pgfsetdash{}{0pt}%
\pgfpathmoveto{\pgfqpoint{5.371306in}{1.499838in}}%
\pgfpathcurveto{\pgfqpoint{5.378439in}{1.499838in}}{\pgfqpoint{5.385280in}{1.502672in}}{\pgfqpoint{5.390324in}{1.507715in}}%
\pgfpathcurveto{\pgfqpoint{5.395368in}{1.512759in}}{\pgfqpoint{5.398202in}{1.519601in}}{\pgfqpoint{5.398202in}{1.526734in}}%
\pgfpathcurveto{\pgfqpoint{5.398202in}{1.533866in}}{\pgfqpoint{5.395368in}{1.540708in}}{\pgfqpoint{5.390324in}{1.545752in}}%
\pgfpathcurveto{\pgfqpoint{5.385280in}{1.550795in}}{\pgfqpoint{5.378439in}{1.553629in}}{\pgfqpoint{5.371306in}{1.553629in}}%
\pgfpathcurveto{\pgfqpoint{5.364173in}{1.553629in}}{\pgfqpoint{5.357332in}{1.550795in}}{\pgfqpoint{5.352288in}{1.545752in}}%
\pgfpathcurveto{\pgfqpoint{5.347244in}{1.540708in}}{\pgfqpoint{5.344410in}{1.533866in}}{\pgfqpoint{5.344410in}{1.526734in}}%
\pgfpathcurveto{\pgfqpoint{5.344410in}{1.519601in}}{\pgfqpoint{5.347244in}{1.512759in}}{\pgfqpoint{5.352288in}{1.507715in}}%
\pgfpathcurveto{\pgfqpoint{5.357332in}{1.502672in}}{\pgfqpoint{5.364173in}{1.499838in}}{\pgfqpoint{5.371306in}{1.499838in}}%
\pgfpathclose%
\pgfusepath{stroke,fill}%
\end{pgfscope}%
\begin{pgfscope}%
\pgfpathrectangle{\pgfqpoint{4.985294in}{0.500000in}}{\pgfqpoint{1.764706in}{1.700000in}}%
\pgfusepath{clip}%
\pgfsetbuttcap%
\pgfsetroundjoin%
\definecolor{currentfill}{rgb}{0.964679,0.682838,0.530002}%
\pgfsetfillcolor{currentfill}%
\pgfsetlinewidth{0.311001pt}%
\definecolor{currentstroke}{rgb}{1.000000,1.000000,1.000000}%
\pgfsetstrokecolor{currentstroke}%
\pgfsetdash{}{0pt}%
\pgfpathmoveto{\pgfqpoint{5.616624in}{0.920314in}}%
\pgfpathcurveto{\pgfqpoint{5.623757in}{0.920314in}}{\pgfqpoint{5.630599in}{0.923148in}}{\pgfqpoint{5.635642in}{0.928191in}}%
\pgfpathcurveto{\pgfqpoint{5.640686in}{0.933235in}}{\pgfqpoint{5.643520in}{0.940077in}}{\pgfqpoint{5.643520in}{0.947210in}}%
\pgfpathcurveto{\pgfqpoint{5.643520in}{0.954342in}}{\pgfqpoint{5.640686in}{0.961184in}}{\pgfqpoint{5.635642in}{0.966228in}}%
\pgfpathcurveto{\pgfqpoint{5.630599in}{0.971271in}}{\pgfqpoint{5.623757in}{0.974105in}}{\pgfqpoint{5.616624in}{0.974105in}}%
\pgfpathcurveto{\pgfqpoint{5.609491in}{0.974105in}}{\pgfqpoint{5.602650in}{0.971271in}}{\pgfqpoint{5.597606in}{0.966228in}}%
\pgfpathcurveto{\pgfqpoint{5.592562in}{0.961184in}}{\pgfqpoint{5.589728in}{0.954342in}}{\pgfqpoint{5.589728in}{0.947210in}}%
\pgfpathcurveto{\pgfqpoint{5.589728in}{0.940077in}}{\pgfqpoint{5.592562in}{0.933235in}}{\pgfqpoint{5.597606in}{0.928191in}}%
\pgfpathcurveto{\pgfqpoint{5.602650in}{0.923148in}}{\pgfqpoint{5.609491in}{0.920314in}}{\pgfqpoint{5.616624in}{0.920314in}}%
\pgfpathclose%
\pgfusepath{stroke,fill}%
\end{pgfscope}%
\begin{pgfscope}%
\pgfpathrectangle{\pgfqpoint{4.985294in}{0.500000in}}{\pgfqpoint{1.764706in}{1.700000in}}%
\pgfusepath{clip}%
\pgfsetbuttcap%
\pgfsetroundjoin%
\definecolor{currentfill}{rgb}{0.807528,0.112400,0.304997}%
\pgfsetfillcolor{currentfill}%
\pgfsetlinewidth{0.311001pt}%
\definecolor{currentstroke}{rgb}{1.000000,1.000000,1.000000}%
\pgfsetstrokecolor{currentstroke}%
\pgfsetdash{}{0pt}%
\pgfpathmoveto{\pgfqpoint{6.260121in}{1.842473in}}%
\pgfpathcurveto{\pgfqpoint{6.267253in}{1.842473in}}{\pgfqpoint{6.274095in}{1.845306in}}{\pgfqpoint{6.279139in}{1.850350in}}%
\pgfpathcurveto{\pgfqpoint{6.284182in}{1.855394in}}{\pgfqpoint{6.287016in}{1.862235in}}{\pgfqpoint{6.287016in}{1.869368in}}%
\pgfpathcurveto{\pgfqpoint{6.287016in}{1.876501in}}{\pgfqpoint{6.284182in}{1.883343in}}{\pgfqpoint{6.279139in}{1.888386in}}%
\pgfpathcurveto{\pgfqpoint{6.274095in}{1.893430in}}{\pgfqpoint{6.267253in}{1.896264in}}{\pgfqpoint{6.260121in}{1.896264in}}%
\pgfpathcurveto{\pgfqpoint{6.252988in}{1.896264in}}{\pgfqpoint{6.246146in}{1.893430in}}{\pgfqpoint{6.241102in}{1.888386in}}%
\pgfpathcurveto{\pgfqpoint{6.236059in}{1.883343in}}{\pgfqpoint{6.233225in}{1.876501in}}{\pgfqpoint{6.233225in}{1.869368in}}%
\pgfpathcurveto{\pgfqpoint{6.233225in}{1.862235in}}{\pgfqpoint{6.236059in}{1.855394in}}{\pgfqpoint{6.241102in}{1.850350in}}%
\pgfpathcurveto{\pgfqpoint{6.246146in}{1.845306in}}{\pgfqpoint{6.252988in}{1.842473in}}{\pgfqpoint{6.260121in}{1.842473in}}%
\pgfpathclose%
\pgfusepath{stroke,fill}%
\end{pgfscope}%
\begin{pgfscope}%
\pgfpathrectangle{\pgfqpoint{4.985294in}{0.500000in}}{\pgfqpoint{1.764706in}{1.700000in}}%
\pgfusepath{clip}%
\pgfsetbuttcap%
\pgfsetroundjoin%
\definecolor{currentfill}{rgb}{0.975018,0.868213,0.788710}%
\pgfsetfillcolor{currentfill}%
\pgfsetlinewidth{0.311001pt}%
\definecolor{currentstroke}{rgb}{1.000000,1.000000,1.000000}%
\pgfsetstrokecolor{currentstroke}%
\pgfsetdash{}{0pt}%
\pgfpathmoveto{\pgfqpoint{5.469528in}{1.236605in}}%
\pgfpathcurveto{\pgfqpoint{5.476661in}{1.236605in}}{\pgfqpoint{5.483503in}{1.239439in}}{\pgfqpoint{5.488547in}{1.244482in}}%
\pgfpathcurveto{\pgfqpoint{5.493590in}{1.249526in}}{\pgfqpoint{5.496424in}{1.256368in}}{\pgfqpoint{5.496424in}{1.263501in}}%
\pgfpathcurveto{\pgfqpoint{5.496424in}{1.270633in}}{\pgfqpoint{5.493590in}{1.277475in}}{\pgfqpoint{5.488547in}{1.282519in}}%
\pgfpathcurveto{\pgfqpoint{5.483503in}{1.287562in}}{\pgfqpoint{5.476661in}{1.290396in}}{\pgfqpoint{5.469528in}{1.290396in}}%
\pgfpathcurveto{\pgfqpoint{5.462396in}{1.290396in}}{\pgfqpoint{5.455554in}{1.287562in}}{\pgfqpoint{5.450510in}{1.282519in}}%
\pgfpathcurveto{\pgfqpoint{5.445467in}{1.277475in}}{\pgfqpoint{5.442633in}{1.270633in}}{\pgfqpoint{5.442633in}{1.263501in}}%
\pgfpathcurveto{\pgfqpoint{5.442633in}{1.256368in}}{\pgfqpoint{5.445467in}{1.249526in}}{\pgfqpoint{5.450510in}{1.244482in}}%
\pgfpathcurveto{\pgfqpoint{5.455554in}{1.239439in}}{\pgfqpoint{5.462396in}{1.236605in}}{\pgfqpoint{5.469528in}{1.236605in}}%
\pgfpathclose%
\pgfusepath{stroke,fill}%
\end{pgfscope}%
\begin{pgfscope}%
\pgfpathrectangle{\pgfqpoint{4.985294in}{0.500000in}}{\pgfqpoint{1.764706in}{1.700000in}}%
\pgfusepath{clip}%
\pgfsetbuttcap%
\pgfsetroundjoin%
\definecolor{currentfill}{rgb}{0.949145,0.420383,0.287810}%
\pgfsetfillcolor{currentfill}%
\pgfsetlinewidth{0.311001pt}%
\definecolor{currentstroke}{rgb}{1.000000,1.000000,1.000000}%
\pgfsetstrokecolor{currentstroke}%
\pgfsetdash{}{0pt}%
\pgfpathmoveto{\pgfqpoint{5.291639in}{1.152988in}}%
\pgfpathcurveto{\pgfqpoint{5.298772in}{1.152988in}}{\pgfqpoint{5.305614in}{1.155822in}}{\pgfqpoint{5.310657in}{1.160866in}}%
\pgfpathcurveto{\pgfqpoint{5.315701in}{1.165909in}}{\pgfqpoint{5.318535in}{1.172751in}}{\pgfqpoint{5.318535in}{1.179884in}}%
\pgfpathcurveto{\pgfqpoint{5.318535in}{1.187017in}}{\pgfqpoint{5.315701in}{1.193858in}}{\pgfqpoint{5.310657in}{1.198902in}}%
\pgfpathcurveto{\pgfqpoint{5.305614in}{1.203946in}}{\pgfqpoint{5.298772in}{1.206780in}}{\pgfqpoint{5.291639in}{1.206780in}}%
\pgfpathcurveto{\pgfqpoint{5.284506in}{1.206780in}}{\pgfqpoint{5.277665in}{1.203946in}}{\pgfqpoint{5.272621in}{1.198902in}}%
\pgfpathcurveto{\pgfqpoint{5.267577in}{1.193858in}}{\pgfqpoint{5.264744in}{1.187017in}}{\pgfqpoint{5.264744in}{1.179884in}}%
\pgfpathcurveto{\pgfqpoint{5.264744in}{1.172751in}}{\pgfqpoint{5.267577in}{1.165909in}}{\pgfqpoint{5.272621in}{1.160866in}}%
\pgfpathcurveto{\pgfqpoint{5.277665in}{1.155822in}}{\pgfqpoint{5.284506in}{1.152988in}}{\pgfqpoint{5.291639in}{1.152988in}}%
\pgfpathclose%
\pgfusepath{stroke,fill}%
\end{pgfscope}%
\begin{pgfscope}%
\pgfpathrectangle{\pgfqpoint{4.985294in}{0.500000in}}{\pgfqpoint{1.764706in}{1.700000in}}%
\pgfusepath{clip}%
\pgfsetbuttcap%
\pgfsetroundjoin%
\definecolor{currentfill}{rgb}{0.973832,0.856556,0.771584}%
\pgfsetfillcolor{currentfill}%
\pgfsetlinewidth{0.311001pt}%
\definecolor{currentstroke}{rgb}{1.000000,1.000000,1.000000}%
\pgfsetstrokecolor{currentstroke}%
\pgfsetdash{}{0pt}%
\pgfpathmoveto{\pgfqpoint{6.319337in}{1.119877in}}%
\pgfpathcurveto{\pgfqpoint{6.326470in}{1.119877in}}{\pgfqpoint{6.333311in}{1.122711in}}{\pgfqpoint{6.338355in}{1.127755in}}%
\pgfpathcurveto{\pgfqpoint{6.343399in}{1.132799in}}{\pgfqpoint{6.346233in}{1.139640in}}{\pgfqpoint{6.346233in}{1.146773in}}%
\pgfpathcurveto{\pgfqpoint{6.346233in}{1.153906in}}{\pgfqpoint{6.343399in}{1.160748in}}{\pgfqpoint{6.338355in}{1.165791in}}%
\pgfpathcurveto{\pgfqpoint{6.333311in}{1.170835in}}{\pgfqpoint{6.326470in}{1.173669in}}{\pgfqpoint{6.319337in}{1.173669in}}%
\pgfpathcurveto{\pgfqpoint{6.312204in}{1.173669in}}{\pgfqpoint{6.305362in}{1.170835in}}{\pgfqpoint{6.300319in}{1.165791in}}%
\pgfpathcurveto{\pgfqpoint{6.295275in}{1.160748in}}{\pgfqpoint{6.292441in}{1.153906in}}{\pgfqpoint{6.292441in}{1.146773in}}%
\pgfpathcurveto{\pgfqpoint{6.292441in}{1.139640in}}{\pgfqpoint{6.295275in}{1.132799in}}{\pgfqpoint{6.300319in}{1.127755in}}%
\pgfpathcurveto{\pgfqpoint{6.305362in}{1.122711in}}{\pgfqpoint{6.312204in}{1.119877in}}{\pgfqpoint{6.319337in}{1.119877in}}%
\pgfpathclose%
\pgfusepath{stroke,fill}%
\end{pgfscope}%
\begin{pgfscope}%
\pgfpathrectangle{\pgfqpoint{4.985294in}{0.500000in}}{\pgfqpoint{1.764706in}{1.700000in}}%
\pgfusepath{clip}%
\pgfsetbuttcap%
\pgfsetroundjoin%
\definecolor{currentfill}{rgb}{0.961115,0.566634,0.405693}%
\pgfsetfillcolor{currentfill}%
\pgfsetlinewidth{0.311001pt}%
\definecolor{currentstroke}{rgb}{1.000000,1.000000,1.000000}%
\pgfsetstrokecolor{currentstroke}%
\pgfsetdash{}{0pt}%
\pgfpathmoveto{\pgfqpoint{5.603859in}{1.090919in}}%
\pgfpathcurveto{\pgfqpoint{5.610991in}{1.090919in}}{\pgfqpoint{5.617833in}{1.093753in}}{\pgfqpoint{5.622877in}{1.098796in}}%
\pgfpathcurveto{\pgfqpoint{5.627920in}{1.103840in}}{\pgfqpoint{5.630754in}{1.110681in}}{\pgfqpoint{5.630754in}{1.117814in}}%
\pgfpathcurveto{\pgfqpoint{5.630754in}{1.124947in}}{\pgfqpoint{5.627920in}{1.131789in}}{\pgfqpoint{5.622877in}{1.136832in}}%
\pgfpathcurveto{\pgfqpoint{5.617833in}{1.141876in}}{\pgfqpoint{5.610991in}{1.144710in}}{\pgfqpoint{5.603859in}{1.144710in}}%
\pgfpathcurveto{\pgfqpoint{5.596726in}{1.144710in}}{\pgfqpoint{5.589884in}{1.141876in}}{\pgfqpoint{5.584840in}{1.136832in}}%
\pgfpathcurveto{\pgfqpoint{5.579797in}{1.131789in}}{\pgfqpoint{5.576963in}{1.124947in}}{\pgfqpoint{5.576963in}{1.117814in}}%
\pgfpathcurveto{\pgfqpoint{5.576963in}{1.110681in}}{\pgfqpoint{5.579797in}{1.103840in}}{\pgfqpoint{5.584840in}{1.098796in}}%
\pgfpathcurveto{\pgfqpoint{5.589884in}{1.093753in}}{\pgfqpoint{5.596726in}{1.090919in}}{\pgfqpoint{5.603859in}{1.090919in}}%
\pgfpathclose%
\pgfusepath{stroke,fill}%
\end{pgfscope}%
\begin{pgfscope}%
\pgfpathrectangle{\pgfqpoint{4.985294in}{0.500000in}}{\pgfqpoint{1.764706in}{1.700000in}}%
\pgfusepath{clip}%
\pgfsetbuttcap%
\pgfsetroundjoin%
\definecolor{currentfill}{rgb}{0.966120,0.744512,0.608720}%
\pgfsetfillcolor{currentfill}%
\pgfsetlinewidth{0.311001pt}%
\definecolor{currentstroke}{rgb}{1.000000,1.000000,1.000000}%
\pgfsetstrokecolor{currentstroke}%
\pgfsetdash{}{0pt}%
\pgfpathmoveto{\pgfqpoint{5.549928in}{1.100346in}}%
\pgfpathcurveto{\pgfqpoint{5.557061in}{1.100346in}}{\pgfqpoint{5.563902in}{1.103180in}}{\pgfqpoint{5.568946in}{1.108224in}}%
\pgfpathcurveto{\pgfqpoint{5.573990in}{1.113268in}}{\pgfqpoint{5.576823in}{1.120109in}}{\pgfqpoint{5.576823in}{1.127242in}}%
\pgfpathcurveto{\pgfqpoint{5.576823in}{1.134375in}}{\pgfqpoint{5.573990in}{1.141217in}}{\pgfqpoint{5.568946in}{1.146260in}}%
\pgfpathcurveto{\pgfqpoint{5.563902in}{1.151304in}}{\pgfqpoint{5.557061in}{1.154138in}}{\pgfqpoint{5.549928in}{1.154138in}}%
\pgfpathcurveto{\pgfqpoint{5.542795in}{1.154138in}}{\pgfqpoint{5.535953in}{1.151304in}}{\pgfqpoint{5.530910in}{1.146260in}}%
\pgfpathcurveto{\pgfqpoint{5.525866in}{1.141217in}}{\pgfqpoint{5.523032in}{1.134375in}}{\pgfqpoint{5.523032in}{1.127242in}}%
\pgfpathcurveto{\pgfqpoint{5.523032in}{1.120109in}}{\pgfqpoint{5.525866in}{1.113268in}}{\pgfqpoint{5.530910in}{1.108224in}}%
\pgfpathcurveto{\pgfqpoint{5.535953in}{1.103180in}}{\pgfqpoint{5.542795in}{1.100346in}}{\pgfqpoint{5.549928in}{1.100346in}}%
\pgfpathclose%
\pgfusepath{stroke,fill}%
\end{pgfscope}%
\begin{pgfscope}%
\pgfpathrectangle{\pgfqpoint{4.985294in}{0.500000in}}{\pgfqpoint{1.764706in}{1.700000in}}%
\pgfusepath{clip}%
\pgfsetbuttcap%
\pgfsetroundjoin%
\definecolor{currentfill}{rgb}{0.962985,0.612625,0.451451}%
\pgfsetfillcolor{currentfill}%
\pgfsetlinewidth{0.311001pt}%
\definecolor{currentstroke}{rgb}{1.000000,1.000000,1.000000}%
\pgfsetstrokecolor{currentstroke}%
\pgfsetdash{}{0pt}%
\pgfpathmoveto{\pgfqpoint{6.111063in}{1.026831in}}%
\pgfpathcurveto{\pgfqpoint{6.118196in}{1.026831in}}{\pgfqpoint{6.125037in}{1.029665in}}{\pgfqpoint{6.130081in}{1.034708in}}%
\pgfpathcurveto{\pgfqpoint{6.135125in}{1.039752in}}{\pgfqpoint{6.137959in}{1.046594in}}{\pgfqpoint{6.137959in}{1.053727in}}%
\pgfpathcurveto{\pgfqpoint{6.137959in}{1.060859in}}{\pgfqpoint{6.135125in}{1.067701in}}{\pgfqpoint{6.130081in}{1.072745in}}%
\pgfpathcurveto{\pgfqpoint{6.125037in}{1.077788in}}{\pgfqpoint{6.118196in}{1.080622in}}{\pgfqpoint{6.111063in}{1.080622in}}%
\pgfpathcurveto{\pgfqpoint{6.103930in}{1.080622in}}{\pgfqpoint{6.097089in}{1.077788in}}{\pgfqpoint{6.092045in}{1.072745in}}%
\pgfpathcurveto{\pgfqpoint{6.087001in}{1.067701in}}{\pgfqpoint{6.084167in}{1.060859in}}{\pgfqpoint{6.084167in}{1.053727in}}%
\pgfpathcurveto{\pgfqpoint{6.084167in}{1.046594in}}{\pgfqpoint{6.087001in}{1.039752in}}{\pgfqpoint{6.092045in}{1.034708in}}%
\pgfpathcurveto{\pgfqpoint{6.097089in}{1.029665in}}{\pgfqpoint{6.103930in}{1.026831in}}{\pgfqpoint{6.111063in}{1.026831in}}%
\pgfpathclose%
\pgfusepath{stroke,fill}%
\end{pgfscope}%
\begin{pgfscope}%
\pgfpathrectangle{\pgfqpoint{4.985294in}{0.500000in}}{\pgfqpoint{1.764706in}{1.700000in}}%
\pgfusepath{clip}%
\pgfsetbuttcap%
\pgfsetroundjoin%
\definecolor{currentfill}{rgb}{0.976287,0.879862,0.805788}%
\pgfsetfillcolor{currentfill}%
\pgfsetlinewidth{0.311001pt}%
\definecolor{currentstroke}{rgb}{1.000000,1.000000,1.000000}%
\pgfsetstrokecolor{currentstroke}%
\pgfsetdash{}{0pt}%
\pgfpathmoveto{\pgfqpoint{5.458180in}{1.532953in}}%
\pgfpathcurveto{\pgfqpoint{5.465313in}{1.532953in}}{\pgfqpoint{5.472155in}{1.535787in}}{\pgfqpoint{5.477198in}{1.540830in}}%
\pgfpathcurveto{\pgfqpoint{5.482242in}{1.545874in}}{\pgfqpoint{5.485076in}{1.552716in}}{\pgfqpoint{5.485076in}{1.559848in}}%
\pgfpathcurveto{\pgfqpoint{5.485076in}{1.566981in}}{\pgfqpoint{5.482242in}{1.573823in}}{\pgfqpoint{5.477198in}{1.578867in}}%
\pgfpathcurveto{\pgfqpoint{5.472155in}{1.583910in}}{\pgfqpoint{5.465313in}{1.586744in}}{\pgfqpoint{5.458180in}{1.586744in}}%
\pgfpathcurveto{\pgfqpoint{5.451047in}{1.586744in}}{\pgfqpoint{5.444206in}{1.583910in}}{\pgfqpoint{5.439162in}{1.578867in}}%
\pgfpathcurveto{\pgfqpoint{5.434118in}{1.573823in}}{\pgfqpoint{5.431284in}{1.566981in}}{\pgfqpoint{5.431284in}{1.559848in}}%
\pgfpathcurveto{\pgfqpoint{5.431284in}{1.552716in}}{\pgfqpoint{5.434118in}{1.545874in}}{\pgfqpoint{5.439162in}{1.540830in}}%
\pgfpathcurveto{\pgfqpoint{5.444206in}{1.535787in}}{\pgfqpoint{5.451047in}{1.532953in}}{\pgfqpoint{5.458180in}{1.532953in}}%
\pgfpathclose%
\pgfusepath{stroke,fill}%
\end{pgfscope}%
\begin{pgfscope}%
\pgfpathrectangle{\pgfqpoint{4.985294in}{0.500000in}}{\pgfqpoint{1.764706in}{1.700000in}}%
\pgfusepath{clip}%
\pgfsetbuttcap%
\pgfsetroundjoin%
\definecolor{currentfill}{rgb}{0.969359,0.803954,0.693832}%
\pgfsetfillcolor{currentfill}%
\pgfsetlinewidth{0.311001pt}%
\definecolor{currentstroke}{rgb}{1.000000,1.000000,1.000000}%
\pgfsetstrokecolor{currentstroke}%
\pgfsetdash{}{0pt}%
\pgfpathmoveto{\pgfqpoint{5.351023in}{1.262862in}}%
\pgfpathcurveto{\pgfqpoint{5.358156in}{1.262862in}}{\pgfqpoint{5.364998in}{1.265696in}}{\pgfqpoint{5.370041in}{1.270740in}}%
\pgfpathcurveto{\pgfqpoint{5.375085in}{1.275783in}}{\pgfqpoint{5.377919in}{1.282625in}}{\pgfqpoint{5.377919in}{1.289758in}}%
\pgfpathcurveto{\pgfqpoint{5.377919in}{1.296891in}}{\pgfqpoint{5.375085in}{1.303732in}}{\pgfqpoint{5.370041in}{1.308776in}}%
\pgfpathcurveto{\pgfqpoint{5.364998in}{1.313820in}}{\pgfqpoint{5.358156in}{1.316654in}}{\pgfqpoint{5.351023in}{1.316654in}}%
\pgfpathcurveto{\pgfqpoint{5.343890in}{1.316654in}}{\pgfqpoint{5.337049in}{1.313820in}}{\pgfqpoint{5.332005in}{1.308776in}}%
\pgfpathcurveto{\pgfqpoint{5.326961in}{1.303732in}}{\pgfqpoint{5.324128in}{1.296891in}}{\pgfqpoint{5.324128in}{1.289758in}}%
\pgfpathcurveto{\pgfqpoint{5.324128in}{1.282625in}}{\pgfqpoint{5.326961in}{1.275783in}}{\pgfqpoint{5.332005in}{1.270740in}}%
\pgfpathcurveto{\pgfqpoint{5.337049in}{1.265696in}}{\pgfqpoint{5.343890in}{1.262862in}}{\pgfqpoint{5.351023in}{1.262862in}}%
\pgfpathclose%
\pgfusepath{stroke,fill}%
\end{pgfscope}%
\begin{pgfscope}%
\pgfpathrectangle{\pgfqpoint{4.985294in}{0.500000in}}{\pgfqpoint{1.764706in}{1.700000in}}%
\pgfusepath{clip}%
\pgfsetbuttcap%
\pgfsetroundjoin%
\definecolor{currentfill}{rgb}{0.887314,0.204699,0.257695}%
\pgfsetfillcolor{currentfill}%
\pgfsetlinewidth{0.311001pt}%
\definecolor{currentstroke}{rgb}{1.000000,1.000000,1.000000}%
\pgfsetstrokecolor{currentstroke}%
\pgfsetdash{}{0pt}%
\pgfpathmoveto{\pgfqpoint{5.247176in}{1.280337in}}%
\pgfpathcurveto{\pgfqpoint{5.254309in}{1.280337in}}{\pgfqpoint{5.261150in}{1.283171in}}{\pgfqpoint{5.266194in}{1.288214in}}%
\pgfpathcurveto{\pgfqpoint{5.271238in}{1.293258in}}{\pgfqpoint{5.274071in}{1.300100in}}{\pgfqpoint{5.274071in}{1.307233in}}%
\pgfpathcurveto{\pgfqpoint{5.274071in}{1.314365in}}{\pgfqpoint{5.271238in}{1.321207in}}{\pgfqpoint{5.266194in}{1.326251in}}%
\pgfpathcurveto{\pgfqpoint{5.261150in}{1.331294in}}{\pgfqpoint{5.254309in}{1.334128in}}{\pgfqpoint{5.247176in}{1.334128in}}%
\pgfpathcurveto{\pgfqpoint{5.240043in}{1.334128in}}{\pgfqpoint{5.233201in}{1.331294in}}{\pgfqpoint{5.228158in}{1.326251in}}%
\pgfpathcurveto{\pgfqpoint{5.223114in}{1.321207in}}{\pgfqpoint{5.220280in}{1.314365in}}{\pgfqpoint{5.220280in}{1.307233in}}%
\pgfpathcurveto{\pgfqpoint{5.220280in}{1.300100in}}{\pgfqpoint{5.223114in}{1.293258in}}{\pgfqpoint{5.228158in}{1.288214in}}%
\pgfpathcurveto{\pgfqpoint{5.233201in}{1.283171in}}{\pgfqpoint{5.240043in}{1.280337in}}{\pgfqpoint{5.247176in}{1.280337in}}%
\pgfpathclose%
\pgfusepath{stroke,fill}%
\end{pgfscope}%
\begin{pgfscope}%
\pgfpathrectangle{\pgfqpoint{4.985294in}{0.500000in}}{\pgfqpoint{1.764706in}{1.700000in}}%
\pgfusepath{clip}%
\pgfsetbuttcap%
\pgfsetroundjoin%
\definecolor{currentfill}{rgb}{0.870791,0.179821,0.267974}%
\pgfsetfillcolor{currentfill}%
\pgfsetlinewidth{0.311001pt}%
\definecolor{currentstroke}{rgb}{1.000000,1.000000,1.000000}%
\pgfsetstrokecolor{currentstroke}%
\pgfsetdash{}{0pt}%
\pgfpathmoveto{\pgfqpoint{5.329648in}{0.941074in}}%
\pgfpathcurveto{\pgfqpoint{5.336781in}{0.941074in}}{\pgfqpoint{5.343622in}{0.943908in}}{\pgfqpoint{5.348666in}{0.948952in}}%
\pgfpathcurveto{\pgfqpoint{5.353710in}{0.953996in}}{\pgfqpoint{5.356543in}{0.960837in}}{\pgfqpoint{5.356543in}{0.967970in}}%
\pgfpathcurveto{\pgfqpoint{5.356543in}{0.975103in}}{\pgfqpoint{5.353710in}{0.981945in}}{\pgfqpoint{5.348666in}{0.986988in}}%
\pgfpathcurveto{\pgfqpoint{5.343622in}{0.992032in}}{\pgfqpoint{5.336781in}{0.994866in}}{\pgfqpoint{5.329648in}{0.994866in}}%
\pgfpathcurveto{\pgfqpoint{5.322515in}{0.994866in}}{\pgfqpoint{5.315673in}{0.992032in}}{\pgfqpoint{5.310630in}{0.986988in}}%
\pgfpathcurveto{\pgfqpoint{5.305586in}{0.981945in}}{\pgfqpoint{5.302752in}{0.975103in}}{\pgfqpoint{5.302752in}{0.967970in}}%
\pgfpathcurveto{\pgfqpoint{5.302752in}{0.960837in}}{\pgfqpoint{5.305586in}{0.953996in}}{\pgfqpoint{5.310630in}{0.948952in}}%
\pgfpathcurveto{\pgfqpoint{5.315673in}{0.943908in}}{\pgfqpoint{5.322515in}{0.941074in}}{\pgfqpoint{5.329648in}{0.941074in}}%
\pgfpathclose%
\pgfusepath{stroke,fill}%
\end{pgfscope}%
\begin{pgfscope}%
\pgfpathrectangle{\pgfqpoint{4.985294in}{0.500000in}}{\pgfqpoint{1.764706in}{1.700000in}}%
\pgfusepath{clip}%
\pgfsetbuttcap%
\pgfsetroundjoin%
\definecolor{currentfill}{rgb}{0.977657,0.891500,0.822809}%
\pgfsetfillcolor{currentfill}%
\pgfsetlinewidth{0.311001pt}%
\definecolor{currentstroke}{rgb}{1.000000,1.000000,1.000000}%
\pgfsetstrokecolor{currentstroke}%
\pgfsetdash{}{0pt}%
\pgfpathmoveto{\pgfqpoint{5.417940in}{1.475370in}}%
\pgfpathcurveto{\pgfqpoint{5.425073in}{1.475370in}}{\pgfqpoint{5.431915in}{1.478204in}}{\pgfqpoint{5.436958in}{1.483247in}}%
\pgfpathcurveto{\pgfqpoint{5.442002in}{1.488291in}}{\pgfqpoint{5.444836in}{1.495133in}}{\pgfqpoint{5.444836in}{1.502265in}}%
\pgfpathcurveto{\pgfqpoint{5.444836in}{1.509398in}}{\pgfqpoint{5.442002in}{1.516240in}}{\pgfqpoint{5.436958in}{1.521284in}}%
\pgfpathcurveto{\pgfqpoint{5.431915in}{1.526327in}}{\pgfqpoint{5.425073in}{1.529161in}}{\pgfqpoint{5.417940in}{1.529161in}}%
\pgfpathcurveto{\pgfqpoint{5.410807in}{1.529161in}}{\pgfqpoint{5.403966in}{1.526327in}}{\pgfqpoint{5.398922in}{1.521284in}}%
\pgfpathcurveto{\pgfqpoint{5.393878in}{1.516240in}}{\pgfqpoint{5.391045in}{1.509398in}}{\pgfqpoint{5.391045in}{1.502265in}}%
\pgfpathcurveto{\pgfqpoint{5.391045in}{1.495133in}}{\pgfqpoint{5.393878in}{1.488291in}}{\pgfqpoint{5.398922in}{1.483247in}}%
\pgfpathcurveto{\pgfqpoint{5.403966in}{1.478204in}}{\pgfqpoint{5.410807in}{1.475370in}}{\pgfqpoint{5.417940in}{1.475370in}}%
\pgfpathclose%
\pgfusepath{stroke,fill}%
\end{pgfscope}%
\begin{pgfscope}%
\pgfpathrectangle{\pgfqpoint{4.985294in}{0.500000in}}{\pgfqpoint{1.764706in}{1.700000in}}%
\pgfusepath{clip}%
\pgfsetbuttcap%
\pgfsetroundjoin%
\definecolor{currentfill}{rgb}{0.963728,0.638439,0.479050}%
\pgfsetfillcolor{currentfill}%
\pgfsetlinewidth{0.311001pt}%
\definecolor{currentstroke}{rgb}{1.000000,1.000000,1.000000}%
\pgfsetstrokecolor{currentstroke}%
\pgfsetdash{}{0pt}%
\pgfpathmoveto{\pgfqpoint{5.567804in}{1.497103in}}%
\pgfpathcurveto{\pgfqpoint{5.574937in}{1.497103in}}{\pgfqpoint{5.581778in}{1.499936in}}{\pgfqpoint{5.586822in}{1.504980in}}%
\pgfpathcurveto{\pgfqpoint{5.591866in}{1.510024in}}{\pgfqpoint{5.594699in}{1.516865in}}{\pgfqpoint{5.594699in}{1.523998in}}%
\pgfpathcurveto{\pgfqpoint{5.594699in}{1.531131in}}{\pgfqpoint{5.591866in}{1.537973in}}{\pgfqpoint{5.586822in}{1.543016in}}%
\pgfpathcurveto{\pgfqpoint{5.581778in}{1.548060in}}{\pgfqpoint{5.574937in}{1.550894in}}{\pgfqpoint{5.567804in}{1.550894in}}%
\pgfpathcurveto{\pgfqpoint{5.560671in}{1.550894in}}{\pgfqpoint{5.553829in}{1.548060in}}{\pgfqpoint{5.548786in}{1.543016in}}%
\pgfpathcurveto{\pgfqpoint{5.543742in}{1.537973in}}{\pgfqpoint{5.540908in}{1.531131in}}{\pgfqpoint{5.540908in}{1.523998in}}%
\pgfpathcurveto{\pgfqpoint{5.540908in}{1.516865in}}{\pgfqpoint{5.543742in}{1.510024in}}{\pgfqpoint{5.548786in}{1.504980in}}%
\pgfpathcurveto{\pgfqpoint{5.553829in}{1.499936in}}{\pgfqpoint{5.560671in}{1.497103in}}{\pgfqpoint{5.567804in}{1.497103in}}%
\pgfpathclose%
\pgfusepath{stroke,fill}%
\end{pgfscope}%
\begin{pgfscope}%
\pgfpathrectangle{\pgfqpoint{4.985294in}{0.500000in}}{\pgfqpoint{1.764706in}{1.700000in}}%
\pgfusepath{clip}%
\pgfsetbuttcap%
\pgfsetroundjoin%
\definecolor{currentfill}{rgb}{0.857426,0.162258,0.276275}%
\pgfsetfillcolor{currentfill}%
\pgfsetlinewidth{0.311001pt}%
\definecolor{currentstroke}{rgb}{1.000000,1.000000,1.000000}%
\pgfsetstrokecolor{currentstroke}%
\pgfsetdash{}{0pt}%
\pgfpathmoveto{\pgfqpoint{5.736205in}{1.647760in}}%
\pgfpathcurveto{\pgfqpoint{5.743338in}{1.647760in}}{\pgfqpoint{5.750179in}{1.650594in}}{\pgfqpoint{5.755223in}{1.655638in}}%
\pgfpathcurveto{\pgfqpoint{5.760267in}{1.660682in}}{\pgfqpoint{5.763100in}{1.667523in}}{\pgfqpoint{5.763100in}{1.674656in}}%
\pgfpathcurveto{\pgfqpoint{5.763100in}{1.681789in}}{\pgfqpoint{5.760267in}{1.688631in}}{\pgfqpoint{5.755223in}{1.693674in}}%
\pgfpathcurveto{\pgfqpoint{5.750179in}{1.698718in}}{\pgfqpoint{5.743338in}{1.701552in}}{\pgfqpoint{5.736205in}{1.701552in}}%
\pgfpathcurveto{\pgfqpoint{5.729072in}{1.701552in}}{\pgfqpoint{5.722230in}{1.698718in}}{\pgfqpoint{5.717187in}{1.693674in}}%
\pgfpathcurveto{\pgfqpoint{5.712143in}{1.688631in}}{\pgfqpoint{5.709309in}{1.681789in}}{\pgfqpoint{5.709309in}{1.674656in}}%
\pgfpathcurveto{\pgfqpoint{5.709309in}{1.667523in}}{\pgfqpoint{5.712143in}{1.660682in}}{\pgfqpoint{5.717187in}{1.655638in}}%
\pgfpathcurveto{\pgfqpoint{5.722230in}{1.650594in}}{\pgfqpoint{5.729072in}{1.647760in}}{\pgfqpoint{5.736205in}{1.647760in}}%
\pgfpathclose%
\pgfusepath{stroke,fill}%
\end{pgfscope}%
\begin{pgfscope}%
\pgfpathrectangle{\pgfqpoint{4.985294in}{0.500000in}}{\pgfqpoint{1.764706in}{1.700000in}}%
\pgfusepath{clip}%
\pgfsetbuttcap%
\pgfsetroundjoin%
\definecolor{currentfill}{rgb}{0.974412,0.862387,0.780156}%
\pgfsetfillcolor{currentfill}%
\pgfsetlinewidth{0.311001pt}%
\definecolor{currentstroke}{rgb}{1.000000,1.000000,1.000000}%
\pgfsetstrokecolor{currentstroke}%
\pgfsetdash{}{0pt}%
\pgfpathmoveto{\pgfqpoint{6.249805in}{1.220242in}}%
\pgfpathcurveto{\pgfqpoint{6.256937in}{1.220242in}}{\pgfqpoint{6.263779in}{1.223076in}}{\pgfqpoint{6.268823in}{1.228120in}}%
\pgfpathcurveto{\pgfqpoint{6.273866in}{1.233163in}}{\pgfqpoint{6.276700in}{1.240005in}}{\pgfqpoint{6.276700in}{1.247138in}}%
\pgfpathcurveto{\pgfqpoint{6.276700in}{1.254271in}}{\pgfqpoint{6.273866in}{1.261112in}}{\pgfqpoint{6.268823in}{1.266156in}}%
\pgfpathcurveto{\pgfqpoint{6.263779in}{1.271200in}}{\pgfqpoint{6.256937in}{1.274034in}}{\pgfqpoint{6.249805in}{1.274034in}}%
\pgfpathcurveto{\pgfqpoint{6.242672in}{1.274034in}}{\pgfqpoint{6.235830in}{1.271200in}}{\pgfqpoint{6.230786in}{1.266156in}}%
\pgfpathcurveto{\pgfqpoint{6.225743in}{1.261112in}}{\pgfqpoint{6.222909in}{1.254271in}}{\pgfqpoint{6.222909in}{1.247138in}}%
\pgfpathcurveto{\pgfqpoint{6.222909in}{1.240005in}}{\pgfqpoint{6.225743in}{1.233163in}}{\pgfqpoint{6.230786in}{1.228120in}}%
\pgfpathcurveto{\pgfqpoint{6.235830in}{1.223076in}}{\pgfqpoint{6.242672in}{1.220242in}}{\pgfqpoint{6.249805in}{1.220242in}}%
\pgfpathclose%
\pgfusepath{stroke,fill}%
\end{pgfscope}%
\begin{pgfscope}%
\pgfpathrectangle{\pgfqpoint{4.985294in}{0.500000in}}{\pgfqpoint{1.764706in}{1.700000in}}%
\pgfusepath{clip}%
\pgfsetbuttcap%
\pgfsetroundjoin%
\definecolor{currentfill}{rgb}{0.979124,0.903132,0.839793}%
\pgfsetfillcolor{currentfill}%
\pgfsetlinewidth{0.311001pt}%
\definecolor{currentstroke}{rgb}{1.000000,1.000000,1.000000}%
\pgfsetstrokecolor{currentstroke}%
\pgfsetdash{}{0pt}%
\pgfpathmoveto{\pgfqpoint{5.406864in}{1.405437in}}%
\pgfpathcurveto{\pgfqpoint{5.413997in}{1.405437in}}{\pgfqpoint{5.420838in}{1.408271in}}{\pgfqpoint{5.425882in}{1.413314in}}%
\pgfpathcurveto{\pgfqpoint{5.430926in}{1.418358in}}{\pgfqpoint{5.433760in}{1.425200in}}{\pgfqpoint{5.433760in}{1.432333in}}%
\pgfpathcurveto{\pgfqpoint{5.433760in}{1.439465in}}{\pgfqpoint{5.430926in}{1.446307in}}{\pgfqpoint{5.425882in}{1.451351in}}%
\pgfpathcurveto{\pgfqpoint{5.420838in}{1.456394in}}{\pgfqpoint{5.413997in}{1.459228in}}{\pgfqpoint{5.406864in}{1.459228in}}%
\pgfpathcurveto{\pgfqpoint{5.399731in}{1.459228in}}{\pgfqpoint{5.392890in}{1.456394in}}{\pgfqpoint{5.387846in}{1.451351in}}%
\pgfpathcurveto{\pgfqpoint{5.382802in}{1.446307in}}{\pgfqpoint{5.379968in}{1.439465in}}{\pgfqpoint{5.379968in}{1.432333in}}%
\pgfpathcurveto{\pgfqpoint{5.379968in}{1.425200in}}{\pgfqpoint{5.382802in}{1.418358in}}{\pgfqpoint{5.387846in}{1.413314in}}%
\pgfpathcurveto{\pgfqpoint{5.392890in}{1.408271in}}{\pgfqpoint{5.399731in}{1.405437in}}{\pgfqpoint{5.406864in}{1.405437in}}%
\pgfpathclose%
\pgfusepath{stroke,fill}%
\end{pgfscope}%
\begin{pgfscope}%
\pgfpathrectangle{\pgfqpoint{4.985294in}{0.500000in}}{\pgfqpoint{1.764706in}{1.700000in}}%
\pgfusepath{clip}%
\pgfsetbuttcap%
\pgfsetroundjoin%
\definecolor{currentfill}{rgb}{0.095724,0.060501,0.162005}%
\pgfsetfillcolor{currentfill}%
\pgfsetlinewidth{0.311001pt}%
\definecolor{currentstroke}{rgb}{1.000000,1.000000,1.000000}%
\pgfsetstrokecolor{currentstroke}%
\pgfsetdash{}{0pt}%
\pgfpathmoveto{\pgfqpoint{5.778726in}{1.063839in}}%
\pgfpathcurveto{\pgfqpoint{5.785859in}{1.063839in}}{\pgfqpoint{5.792701in}{1.066673in}}{\pgfqpoint{5.797744in}{1.071717in}}%
\pgfpathcurveto{\pgfqpoint{5.802788in}{1.076760in}}{\pgfqpoint{5.805622in}{1.083602in}}{\pgfqpoint{5.805622in}{1.090735in}}%
\pgfpathcurveto{\pgfqpoint{5.805622in}{1.097868in}}{\pgfqpoint{5.802788in}{1.104709in}}{\pgfqpoint{5.797744in}{1.109753in}}%
\pgfpathcurveto{\pgfqpoint{5.792701in}{1.114797in}}{\pgfqpoint{5.785859in}{1.117630in}}{\pgfqpoint{5.778726in}{1.117630in}}%
\pgfpathcurveto{\pgfqpoint{5.771593in}{1.117630in}}{\pgfqpoint{5.764752in}{1.114797in}}{\pgfqpoint{5.759708in}{1.109753in}}%
\pgfpathcurveto{\pgfqpoint{5.754664in}{1.104709in}}{\pgfqpoint{5.751830in}{1.097868in}}{\pgfqpoint{5.751830in}{1.090735in}}%
\pgfpathcurveto{\pgfqpoint{5.751830in}{1.083602in}}{\pgfqpoint{5.754664in}{1.076760in}}{\pgfqpoint{5.759708in}{1.071717in}}%
\pgfpathcurveto{\pgfqpoint{5.764752in}{1.066673in}}{\pgfqpoint{5.771593in}{1.063839in}}{\pgfqpoint{5.778726in}{1.063839in}}%
\pgfpathclose%
\pgfusepath{stroke,fill}%
\end{pgfscope}%
\begin{pgfscope}%
\pgfpathrectangle{\pgfqpoint{4.985294in}{0.500000in}}{\pgfqpoint{1.764706in}{1.700000in}}%
\pgfusepath{clip}%
\pgfsetbuttcap%
\pgfsetroundjoin%
\definecolor{currentfill}{rgb}{0.938993,0.352507,0.254528}%
\pgfsetfillcolor{currentfill}%
\pgfsetlinewidth{0.311001pt}%
\definecolor{currentstroke}{rgb}{1.000000,1.000000,1.000000}%
\pgfsetstrokecolor{currentstroke}%
\pgfsetdash{}{0pt}%
\pgfpathmoveto{\pgfqpoint{5.488744in}{1.799366in}}%
\pgfpathcurveto{\pgfqpoint{5.495877in}{1.799366in}}{\pgfqpoint{5.502719in}{1.802200in}}{\pgfqpoint{5.507762in}{1.807243in}}%
\pgfpathcurveto{\pgfqpoint{5.512806in}{1.812287in}}{\pgfqpoint{5.515640in}{1.819128in}}{\pgfqpoint{5.515640in}{1.826261in}}%
\pgfpathcurveto{\pgfqpoint{5.515640in}{1.833394in}}{\pgfqpoint{5.512806in}{1.840236in}}{\pgfqpoint{5.507762in}{1.845279in}}%
\pgfpathcurveto{\pgfqpoint{5.502719in}{1.850323in}}{\pgfqpoint{5.495877in}{1.853157in}}{\pgfqpoint{5.488744in}{1.853157in}}%
\pgfpathcurveto{\pgfqpoint{5.481611in}{1.853157in}}{\pgfqpoint{5.474770in}{1.850323in}}{\pgfqpoint{5.469726in}{1.845279in}}%
\pgfpathcurveto{\pgfqpoint{5.464682in}{1.840236in}}{\pgfqpoint{5.461848in}{1.833394in}}{\pgfqpoint{5.461848in}{1.826261in}}%
\pgfpathcurveto{\pgfqpoint{5.461848in}{1.819128in}}{\pgfqpoint{5.464682in}{1.812287in}}{\pgfqpoint{5.469726in}{1.807243in}}%
\pgfpathcurveto{\pgfqpoint{5.474770in}{1.802200in}}{\pgfqpoint{5.481611in}{1.799366in}}{\pgfqpoint{5.488744in}{1.799366in}}%
\pgfpathclose%
\pgfusepath{stroke,fill}%
\end{pgfscope}%
\begin{pgfscope}%
\pgfpathrectangle{\pgfqpoint{4.985294in}{0.500000in}}{\pgfqpoint{1.764706in}{1.700000in}}%
\pgfusepath{clip}%
\pgfsetbuttcap%
\pgfsetroundjoin%
\definecolor{currentfill}{rgb}{0.950017,0.427714,0.292447}%
\pgfsetfillcolor{currentfill}%
\pgfsetlinewidth{0.311001pt}%
\definecolor{currentstroke}{rgb}{1.000000,1.000000,1.000000}%
\pgfsetstrokecolor{currentstroke}%
\pgfsetdash{}{0pt}%
\pgfpathmoveto{\pgfqpoint{6.203933in}{1.794108in}}%
\pgfpathcurveto{\pgfqpoint{6.211066in}{1.794108in}}{\pgfqpoint{6.217908in}{1.796942in}}{\pgfqpoint{6.222951in}{1.801985in}}%
\pgfpathcurveto{\pgfqpoint{6.227995in}{1.807029in}}{\pgfqpoint{6.230829in}{1.813871in}}{\pgfqpoint{6.230829in}{1.821003in}}%
\pgfpathcurveto{\pgfqpoint{6.230829in}{1.828136in}}{\pgfqpoint{6.227995in}{1.834978in}}{\pgfqpoint{6.222951in}{1.840022in}}%
\pgfpathcurveto{\pgfqpoint{6.217908in}{1.845065in}}{\pgfqpoint{6.211066in}{1.847899in}}{\pgfqpoint{6.203933in}{1.847899in}}%
\pgfpathcurveto{\pgfqpoint{6.196800in}{1.847899in}}{\pgfqpoint{6.189959in}{1.845065in}}{\pgfqpoint{6.184915in}{1.840022in}}%
\pgfpathcurveto{\pgfqpoint{6.179872in}{1.834978in}}{\pgfqpoint{6.177038in}{1.828136in}}{\pgfqpoint{6.177038in}{1.821003in}}%
\pgfpathcurveto{\pgfqpoint{6.177038in}{1.813871in}}{\pgfqpoint{6.179872in}{1.807029in}}{\pgfqpoint{6.184915in}{1.801985in}}%
\pgfpathcurveto{\pgfqpoint{6.189959in}{1.796942in}}{\pgfqpoint{6.196800in}{1.794108in}}{\pgfqpoint{6.203933in}{1.794108in}}%
\pgfpathclose%
\pgfusepath{stroke,fill}%
\end{pgfscope}%
\begin{pgfscope}%
\pgfpathrectangle{\pgfqpoint{4.985294in}{0.500000in}}{\pgfqpoint{1.764706in}{1.700000in}}%
\pgfusepath{clip}%
\pgfsetbuttcap%
\pgfsetroundjoin%
\definecolor{currentfill}{rgb}{0.979891,0.908948,0.848279}%
\pgfsetfillcolor{currentfill}%
\pgfsetlinewidth{0.311001pt}%
\definecolor{currentstroke}{rgb}{1.000000,1.000000,1.000000}%
\pgfsetstrokecolor{currentstroke}%
\pgfsetdash{}{0pt}%
\pgfpathmoveto{\pgfqpoint{5.435037in}{1.336549in}}%
\pgfpathcurveto{\pgfqpoint{5.442170in}{1.336549in}}{\pgfqpoint{5.449011in}{1.339383in}}{\pgfqpoint{5.454055in}{1.344427in}}%
\pgfpathcurveto{\pgfqpoint{5.459099in}{1.349471in}}{\pgfqpoint{5.461933in}{1.356312in}}{\pgfqpoint{5.461933in}{1.363445in}}%
\pgfpathcurveto{\pgfqpoint{5.461933in}{1.370578in}}{\pgfqpoint{5.459099in}{1.377420in}}{\pgfqpoint{5.454055in}{1.382463in}}%
\pgfpathcurveto{\pgfqpoint{5.449011in}{1.387507in}}{\pgfqpoint{5.442170in}{1.390341in}}{\pgfqpoint{5.435037in}{1.390341in}}%
\pgfpathcurveto{\pgfqpoint{5.427904in}{1.390341in}}{\pgfqpoint{5.421062in}{1.387507in}}{\pgfqpoint{5.416019in}{1.382463in}}%
\pgfpathcurveto{\pgfqpoint{5.410975in}{1.377420in}}{\pgfqpoint{5.408141in}{1.370578in}}{\pgfqpoint{5.408141in}{1.363445in}}%
\pgfpathcurveto{\pgfqpoint{5.408141in}{1.356312in}}{\pgfqpoint{5.410975in}{1.349471in}}{\pgfqpoint{5.416019in}{1.344427in}}%
\pgfpathcurveto{\pgfqpoint{5.421062in}{1.339383in}}{\pgfqpoint{5.427904in}{1.336549in}}{\pgfqpoint{5.435037in}{1.336549in}}%
\pgfpathclose%
\pgfusepath{stroke,fill}%
\end{pgfscope}%
\begin{pgfscope}%
\pgfpathrectangle{\pgfqpoint{4.985294in}{0.500000in}}{\pgfqpoint{1.764706in}{1.700000in}}%
\pgfusepath{clip}%
\pgfsetbuttcap%
\pgfsetroundjoin%
\definecolor{currentfill}{rgb}{0.965042,0.701564,0.552889}%
\pgfsetfillcolor{currentfill}%
\pgfsetlinewidth{0.311001pt}%
\definecolor{currentstroke}{rgb}{1.000000,1.000000,1.000000}%
\pgfsetstrokecolor{currentstroke}%
\pgfsetdash{}{0pt}%
\pgfpathmoveto{\pgfqpoint{6.123812in}{1.608981in}}%
\pgfpathcurveto{\pgfqpoint{6.130945in}{1.608981in}}{\pgfqpoint{6.137786in}{1.611815in}}{\pgfqpoint{6.142830in}{1.616859in}}%
\pgfpathcurveto{\pgfqpoint{6.147874in}{1.621903in}}{\pgfqpoint{6.150708in}{1.628744in}}{\pgfqpoint{6.150708in}{1.635877in}}%
\pgfpathcurveto{\pgfqpoint{6.150708in}{1.643010in}}{\pgfqpoint{6.147874in}{1.649852in}}{\pgfqpoint{6.142830in}{1.654895in}}%
\pgfpathcurveto{\pgfqpoint{6.137786in}{1.659939in}}{\pgfqpoint{6.130945in}{1.662773in}}{\pgfqpoint{6.123812in}{1.662773in}}%
\pgfpathcurveto{\pgfqpoint{6.116679in}{1.662773in}}{\pgfqpoint{6.109837in}{1.659939in}}{\pgfqpoint{6.104794in}{1.654895in}}%
\pgfpathcurveto{\pgfqpoint{6.099750in}{1.649852in}}{\pgfqpoint{6.096916in}{1.643010in}}{\pgfqpoint{6.096916in}{1.635877in}}%
\pgfpathcurveto{\pgfqpoint{6.096916in}{1.628744in}}{\pgfqpoint{6.099750in}{1.621903in}}{\pgfqpoint{6.104794in}{1.616859in}}%
\pgfpathcurveto{\pgfqpoint{6.109837in}{1.611815in}}{\pgfqpoint{6.116679in}{1.608981in}}{\pgfqpoint{6.123812in}{1.608981in}}%
\pgfpathclose%
\pgfusepath{stroke,fill}%
\end{pgfscope}%
\begin{pgfscope}%
\pgfpathrectangle{\pgfqpoint{4.985294in}{0.500000in}}{\pgfqpoint{1.764706in}{1.700000in}}%
\pgfusepath{clip}%
\pgfsetbuttcap%
\pgfsetroundjoin%
\definecolor{currentfill}{rgb}{0.973271,0.850724,0.762998}%
\pgfsetfillcolor{currentfill}%
\pgfsetlinewidth{0.311001pt}%
\definecolor{currentstroke}{rgb}{1.000000,1.000000,1.000000}%
\pgfsetstrokecolor{currentstroke}%
\pgfsetdash{}{0pt}%
\pgfpathmoveto{\pgfqpoint{5.365487in}{1.348693in}}%
\pgfpathcurveto{\pgfqpoint{5.372619in}{1.348693in}}{\pgfqpoint{5.379461in}{1.351527in}}{\pgfqpoint{5.384505in}{1.356571in}}%
\pgfpathcurveto{\pgfqpoint{5.389548in}{1.361615in}}{\pgfqpoint{5.392382in}{1.368456in}}{\pgfqpoint{5.392382in}{1.375589in}}%
\pgfpathcurveto{\pgfqpoint{5.392382in}{1.382722in}}{\pgfqpoint{5.389548in}{1.389564in}}{\pgfqpoint{5.384505in}{1.394607in}}%
\pgfpathcurveto{\pgfqpoint{5.379461in}{1.399651in}}{\pgfqpoint{5.372619in}{1.402485in}}{\pgfqpoint{5.365487in}{1.402485in}}%
\pgfpathcurveto{\pgfqpoint{5.358354in}{1.402485in}}{\pgfqpoint{5.351512in}{1.399651in}}{\pgfqpoint{5.346468in}{1.394607in}}%
\pgfpathcurveto{\pgfqpoint{5.341425in}{1.389564in}}{\pgfqpoint{5.338591in}{1.382722in}}{\pgfqpoint{5.338591in}{1.375589in}}%
\pgfpathcurveto{\pgfqpoint{5.338591in}{1.368456in}}{\pgfqpoint{5.341425in}{1.361615in}}{\pgfqpoint{5.346468in}{1.356571in}}%
\pgfpathcurveto{\pgfqpoint{5.351512in}{1.351527in}}{\pgfqpoint{5.358354in}{1.348693in}}{\pgfqpoint{5.365487in}{1.348693in}}%
\pgfpathclose%
\pgfusepath{stroke,fill}%
\end{pgfscope}%
\begin{pgfscope}%
\pgfpathrectangle{\pgfqpoint{4.985294in}{0.500000in}}{\pgfqpoint{1.764706in}{1.700000in}}%
\pgfusepath{clip}%
\pgfsetbuttcap%
\pgfsetroundjoin%
\definecolor{currentfill}{rgb}{0.966328,0.750560,0.616961}%
\pgfsetfillcolor{currentfill}%
\pgfsetlinewidth{0.311001pt}%
\definecolor{currentstroke}{rgb}{1.000000,1.000000,1.000000}%
\pgfsetstrokecolor{currentstroke}%
\pgfsetdash{}{0pt}%
\pgfpathmoveto{\pgfqpoint{6.225802in}{0.949354in}}%
\pgfpathcurveto{\pgfqpoint{6.232935in}{0.949354in}}{\pgfqpoint{6.239776in}{0.952188in}}{\pgfqpoint{6.244820in}{0.957231in}}%
\pgfpathcurveto{\pgfqpoint{6.249864in}{0.962275in}}{\pgfqpoint{6.252697in}{0.969117in}}{\pgfqpoint{6.252697in}{0.976250in}}%
\pgfpathcurveto{\pgfqpoint{6.252697in}{0.983382in}}{\pgfqpoint{6.249864in}{0.990224in}}{\pgfqpoint{6.244820in}{0.995268in}}%
\pgfpathcurveto{\pgfqpoint{6.239776in}{1.000311in}}{\pgfqpoint{6.232935in}{1.003145in}}{\pgfqpoint{6.225802in}{1.003145in}}%
\pgfpathcurveto{\pgfqpoint{6.218669in}{1.003145in}}{\pgfqpoint{6.211827in}{1.000311in}}{\pgfqpoint{6.206784in}{0.995268in}}%
\pgfpathcurveto{\pgfqpoint{6.201740in}{0.990224in}}{\pgfqpoint{6.198906in}{0.983382in}}{\pgfqpoint{6.198906in}{0.976250in}}%
\pgfpathcurveto{\pgfqpoint{6.198906in}{0.969117in}}{\pgfqpoint{6.201740in}{0.962275in}}{\pgfqpoint{6.206784in}{0.957231in}}%
\pgfpathcurveto{\pgfqpoint{6.211827in}{0.952188in}}{\pgfqpoint{6.218669in}{0.949354in}}{\pgfqpoint{6.225802in}{0.949354in}}%
\pgfpathclose%
\pgfusepath{stroke,fill}%
\end{pgfscope}%
\begin{pgfscope}%
\pgfpathrectangle{\pgfqpoint{4.985294in}{0.500000in}}{\pgfqpoint{1.764706in}{1.700000in}}%
\pgfusepath{clip}%
\pgfsetbuttcap%
\pgfsetroundjoin%
\definecolor{currentfill}{rgb}{0.967735,0.780441,0.659127}%
\pgfsetfillcolor{currentfill}%
\pgfsetlinewidth{0.311001pt}%
\definecolor{currentstroke}{rgb}{1.000000,1.000000,1.000000}%
\pgfsetstrokecolor{currentstroke}%
\pgfsetdash{}{0pt}%
\pgfpathmoveto{\pgfqpoint{6.169816in}{1.584183in}}%
\pgfpathcurveto{\pgfqpoint{6.176949in}{1.584183in}}{\pgfqpoint{6.183790in}{1.587017in}}{\pgfqpoint{6.188834in}{1.592060in}}%
\pgfpathcurveto{\pgfqpoint{6.193878in}{1.597104in}}{\pgfqpoint{6.196712in}{1.603946in}}{\pgfqpoint{6.196712in}{1.611078in}}%
\pgfpathcurveto{\pgfqpoint{6.196712in}{1.618211in}}{\pgfqpoint{6.193878in}{1.625053in}}{\pgfqpoint{6.188834in}{1.630097in}}%
\pgfpathcurveto{\pgfqpoint{6.183790in}{1.635140in}}{\pgfqpoint{6.176949in}{1.637974in}}{\pgfqpoint{6.169816in}{1.637974in}}%
\pgfpathcurveto{\pgfqpoint{6.162683in}{1.637974in}}{\pgfqpoint{6.155842in}{1.635140in}}{\pgfqpoint{6.150798in}{1.630097in}}%
\pgfpathcurveto{\pgfqpoint{6.145754in}{1.625053in}}{\pgfqpoint{6.142920in}{1.618211in}}{\pgfqpoint{6.142920in}{1.611078in}}%
\pgfpathcurveto{\pgfqpoint{6.142920in}{1.603946in}}{\pgfqpoint{6.145754in}{1.597104in}}{\pgfqpoint{6.150798in}{1.592060in}}%
\pgfpathcurveto{\pgfqpoint{6.155842in}{1.587017in}}{\pgfqpoint{6.162683in}{1.584183in}}{\pgfqpoint{6.169816in}{1.584183in}}%
\pgfpathclose%
\pgfusepath{stroke,fill}%
\end{pgfscope}%
\begin{pgfscope}%
\pgfpathrectangle{\pgfqpoint{4.985294in}{0.500000in}}{\pgfqpoint{1.764706in}{1.700000in}}%
\pgfusepath{clip}%
\pgfsetbuttcap%
\pgfsetroundjoin%
\definecolor{currentfill}{rgb}{0.969803,0.809811,0.702523}%
\pgfsetfillcolor{currentfill}%
\pgfsetlinewidth{0.311001pt}%
\definecolor{currentstroke}{rgb}{1.000000,1.000000,1.000000}%
\pgfsetstrokecolor{currentstroke}%
\pgfsetdash{}{0pt}%
\pgfpathmoveto{\pgfqpoint{6.370479in}{1.182526in}}%
\pgfpathcurveto{\pgfqpoint{6.377612in}{1.182526in}}{\pgfqpoint{6.384453in}{1.185360in}}{\pgfqpoint{6.389497in}{1.190404in}}%
\pgfpathcurveto{\pgfqpoint{6.394541in}{1.195448in}}{\pgfqpoint{6.397375in}{1.202289in}}{\pgfqpoint{6.397375in}{1.209422in}}%
\pgfpathcurveto{\pgfqpoint{6.397375in}{1.216555in}}{\pgfqpoint{6.394541in}{1.223396in}}{\pgfqpoint{6.389497in}{1.228440in}}%
\pgfpathcurveto{\pgfqpoint{6.384453in}{1.233484in}}{\pgfqpoint{6.377612in}{1.236318in}}{\pgfqpoint{6.370479in}{1.236318in}}%
\pgfpathcurveto{\pgfqpoint{6.363346in}{1.236318in}}{\pgfqpoint{6.356504in}{1.233484in}}{\pgfqpoint{6.351461in}{1.228440in}}%
\pgfpathcurveto{\pgfqpoint{6.346417in}{1.223396in}}{\pgfqpoint{6.343583in}{1.216555in}}{\pgfqpoint{6.343583in}{1.209422in}}%
\pgfpathcurveto{\pgfqpoint{6.343583in}{1.202289in}}{\pgfqpoint{6.346417in}{1.195448in}}{\pgfqpoint{6.351461in}{1.190404in}}%
\pgfpathcurveto{\pgfqpoint{6.356504in}{1.185360in}}{\pgfqpoint{6.363346in}{1.182526in}}{\pgfqpoint{6.370479in}{1.182526in}}%
\pgfpathclose%
\pgfusepath{stroke,fill}%
\end{pgfscope}%
\begin{pgfscope}%
\pgfpathrectangle{\pgfqpoint{4.985294in}{0.500000in}}{\pgfqpoint{1.764706in}{1.700000in}}%
\pgfusepath{clip}%
\pgfsetbuttcap%
\pgfsetroundjoin%
\definecolor{currentfill}{rgb}{0.959229,0.533075,0.374889}%
\pgfsetfillcolor{currentfill}%
\pgfsetlinewidth{0.311001pt}%
\definecolor{currentstroke}{rgb}{1.000000,1.000000,1.000000}%
\pgfsetstrokecolor{currentstroke}%
\pgfsetdash{}{0pt}%
\pgfpathmoveto{\pgfqpoint{6.438198in}{1.346844in}}%
\pgfpathcurveto{\pgfqpoint{6.445331in}{1.346844in}}{\pgfqpoint{6.452173in}{1.349678in}}{\pgfqpoint{6.457216in}{1.354722in}}%
\pgfpathcurveto{\pgfqpoint{6.462260in}{1.359765in}}{\pgfqpoint{6.465094in}{1.366607in}}{\pgfqpoint{6.465094in}{1.373740in}}%
\pgfpathcurveto{\pgfqpoint{6.465094in}{1.380873in}}{\pgfqpoint{6.462260in}{1.387714in}}{\pgfqpoint{6.457216in}{1.392758in}}%
\pgfpathcurveto{\pgfqpoint{6.452173in}{1.397801in}}{\pgfqpoint{6.445331in}{1.400635in}}{\pgfqpoint{6.438198in}{1.400635in}}%
\pgfpathcurveto{\pgfqpoint{6.431065in}{1.400635in}}{\pgfqpoint{6.424224in}{1.397801in}}{\pgfqpoint{6.419180in}{1.392758in}}%
\pgfpathcurveto{\pgfqpoint{6.414136in}{1.387714in}}{\pgfqpoint{6.411302in}{1.380873in}}{\pgfqpoint{6.411302in}{1.373740in}}%
\pgfpathcurveto{\pgfqpoint{6.411302in}{1.366607in}}{\pgfqpoint{6.414136in}{1.359765in}}{\pgfqpoint{6.419180in}{1.354722in}}%
\pgfpathcurveto{\pgfqpoint{6.424224in}{1.349678in}}{\pgfqpoint{6.431065in}{1.346844in}}{\pgfqpoint{6.438198in}{1.346844in}}%
\pgfpathclose%
\pgfusepath{stroke,fill}%
\end{pgfscope}%
\begin{pgfscope}%
\pgfpathrectangle{\pgfqpoint{4.985294in}{0.500000in}}{\pgfqpoint{1.764706in}{1.700000in}}%
\pgfusepath{clip}%
\pgfsetbuttcap%
\pgfsetroundjoin%
\definecolor{currentfill}{rgb}{0.964799,0.689101,0.537560}%
\pgfsetfillcolor{currentfill}%
\pgfsetlinewidth{0.311001pt}%
\definecolor{currentstroke}{rgb}{1.000000,1.000000,1.000000}%
\pgfsetstrokecolor{currentstroke}%
\pgfsetdash{}{0pt}%
\pgfpathmoveto{\pgfqpoint{6.148998in}{1.085106in}}%
\pgfpathcurveto{\pgfqpoint{6.156131in}{1.085106in}}{\pgfqpoint{6.162972in}{1.087939in}}{\pgfqpoint{6.168016in}{1.092983in}}%
\pgfpathcurveto{\pgfqpoint{6.173060in}{1.098027in}}{\pgfqpoint{6.175894in}{1.104868in}}{\pgfqpoint{6.175894in}{1.112001in}}%
\pgfpathcurveto{\pgfqpoint{6.175894in}{1.119134in}}{\pgfqpoint{6.173060in}{1.125976in}}{\pgfqpoint{6.168016in}{1.131019in}}%
\pgfpathcurveto{\pgfqpoint{6.162972in}{1.136063in}}{\pgfqpoint{6.156131in}{1.138897in}}{\pgfqpoint{6.148998in}{1.138897in}}%
\pgfpathcurveto{\pgfqpoint{6.141865in}{1.138897in}}{\pgfqpoint{6.135023in}{1.136063in}}{\pgfqpoint{6.129980in}{1.131019in}}%
\pgfpathcurveto{\pgfqpoint{6.124936in}{1.125976in}}{\pgfqpoint{6.122102in}{1.119134in}}{\pgfqpoint{6.122102in}{1.112001in}}%
\pgfpathcurveto{\pgfqpoint{6.122102in}{1.104868in}}{\pgfqpoint{6.124936in}{1.098027in}}{\pgfqpoint{6.129980in}{1.092983in}}%
\pgfpathcurveto{\pgfqpoint{6.135023in}{1.087939in}}{\pgfqpoint{6.141865in}{1.085106in}}{\pgfqpoint{6.148998in}{1.085106in}}%
\pgfpathclose%
\pgfusepath{stroke,fill}%
\end{pgfscope}%
\begin{pgfscope}%
\pgfpathrectangle{\pgfqpoint{4.985294in}{0.500000in}}{\pgfqpoint{1.764706in}{1.700000in}}%
\pgfusepath{clip}%
\pgfsetbuttcap%
\pgfsetroundjoin%
\definecolor{currentfill}{rgb}{0.972726,0.844889,0.754401}%
\pgfsetfillcolor{currentfill}%
\pgfsetlinewidth{0.311001pt}%
\definecolor{currentstroke}{rgb}{1.000000,1.000000,1.000000}%
\pgfsetstrokecolor{currentstroke}%
\pgfsetdash{}{0pt}%
\pgfpathmoveto{\pgfqpoint{6.252148in}{1.293634in}}%
\pgfpathcurveto{\pgfqpoint{6.259281in}{1.293634in}}{\pgfqpoint{6.266123in}{1.296468in}}{\pgfqpoint{6.271167in}{1.301511in}}%
\pgfpathcurveto{\pgfqpoint{6.276210in}{1.306555in}}{\pgfqpoint{6.279044in}{1.313397in}}{\pgfqpoint{6.279044in}{1.320529in}}%
\pgfpathcurveto{\pgfqpoint{6.279044in}{1.327662in}}{\pgfqpoint{6.276210in}{1.334504in}}{\pgfqpoint{6.271167in}{1.339548in}}%
\pgfpathcurveto{\pgfqpoint{6.266123in}{1.344591in}}{\pgfqpoint{6.259281in}{1.347425in}}{\pgfqpoint{6.252148in}{1.347425in}}%
\pgfpathcurveto{\pgfqpoint{6.245016in}{1.347425in}}{\pgfqpoint{6.238174in}{1.344591in}}{\pgfqpoint{6.233130in}{1.339548in}}%
\pgfpathcurveto{\pgfqpoint{6.228087in}{1.334504in}}{\pgfqpoint{6.225253in}{1.327662in}}{\pgfqpoint{6.225253in}{1.320529in}}%
\pgfpathcurveto{\pgfqpoint{6.225253in}{1.313397in}}{\pgfqpoint{6.228087in}{1.306555in}}{\pgfqpoint{6.233130in}{1.301511in}}%
\pgfpathcurveto{\pgfqpoint{6.238174in}{1.296468in}}{\pgfqpoint{6.245016in}{1.293634in}}{\pgfqpoint{6.252148in}{1.293634in}}%
\pgfpathclose%
\pgfusepath{stroke,fill}%
\end{pgfscope}%
\begin{pgfscope}%
\pgfpathrectangle{\pgfqpoint{4.985294in}{0.500000in}}{\pgfqpoint{1.764706in}{1.700000in}}%
\pgfusepath{clip}%
\pgfsetbuttcap%
\pgfsetroundjoin%
\definecolor{currentfill}{rgb}{0.968931,0.798091,0.685123}%
\pgfsetfillcolor{currentfill}%
\pgfsetlinewidth{0.311001pt}%
\definecolor{currentstroke}{rgb}{1.000000,1.000000,1.000000}%
\pgfsetstrokecolor{currentstroke}%
\pgfsetdash{}{0pt}%
\pgfpathmoveto{\pgfqpoint{5.470246in}{0.976279in}}%
\pgfpathcurveto{\pgfqpoint{5.477379in}{0.976279in}}{\pgfqpoint{5.484220in}{0.979113in}}{\pgfqpoint{5.489264in}{0.984156in}}%
\pgfpathcurveto{\pgfqpoint{5.494308in}{0.989200in}}{\pgfqpoint{5.497141in}{0.996041in}}{\pgfqpoint{5.497141in}{1.003174in}}%
\pgfpathcurveto{\pgfqpoint{5.497141in}{1.010307in}}{\pgfqpoint{5.494308in}{1.017149in}}{\pgfqpoint{5.489264in}{1.022192in}}%
\pgfpathcurveto{\pgfqpoint{5.484220in}{1.027236in}}{\pgfqpoint{5.477379in}{1.030070in}}{\pgfqpoint{5.470246in}{1.030070in}}%
\pgfpathcurveto{\pgfqpoint{5.463113in}{1.030070in}}{\pgfqpoint{5.456271in}{1.027236in}}{\pgfqpoint{5.451228in}{1.022192in}}%
\pgfpathcurveto{\pgfqpoint{5.446184in}{1.017149in}}{\pgfqpoint{5.443350in}{1.010307in}}{\pgfqpoint{5.443350in}{1.003174in}}%
\pgfpathcurveto{\pgfqpoint{5.443350in}{0.996041in}}{\pgfqpoint{5.446184in}{0.989200in}}{\pgfqpoint{5.451228in}{0.984156in}}%
\pgfpathcurveto{\pgfqpoint{5.456271in}{0.979113in}}{\pgfqpoint{5.463113in}{0.976279in}}{\pgfqpoint{5.470246in}{0.976279in}}%
\pgfpathclose%
\pgfusepath{stroke,fill}%
\end{pgfscope}%
\begin{pgfscope}%
\pgfpathrectangle{\pgfqpoint{4.985294in}{0.500000in}}{\pgfqpoint{1.764706in}{1.700000in}}%
\pgfusepath{clip}%
\pgfsetbuttcap%
\pgfsetroundjoin%
\definecolor{currentfill}{rgb}{0.979891,0.908948,0.848279}%
\pgfsetfillcolor{currentfill}%
\pgfsetlinewidth{0.311001pt}%
\definecolor{currentstroke}{rgb}{1.000000,1.000000,1.000000}%
\pgfsetstrokecolor{currentstroke}%
\pgfsetdash{}{0pt}%
\pgfpathmoveto{\pgfqpoint{6.294718in}{1.302453in}}%
\pgfpathcurveto{\pgfqpoint{6.301851in}{1.302453in}}{\pgfqpoint{6.308692in}{1.305287in}}{\pgfqpoint{6.313736in}{1.310331in}}%
\pgfpathcurveto{\pgfqpoint{6.318780in}{1.315374in}}{\pgfqpoint{6.321614in}{1.322216in}}{\pgfqpoint{6.321614in}{1.329349in}}%
\pgfpathcurveto{\pgfqpoint{6.321614in}{1.336482in}}{\pgfqpoint{6.318780in}{1.343323in}}{\pgfqpoint{6.313736in}{1.348367in}}%
\pgfpathcurveto{\pgfqpoint{6.308692in}{1.353411in}}{\pgfqpoint{6.301851in}{1.356245in}}{\pgfqpoint{6.294718in}{1.356245in}}%
\pgfpathcurveto{\pgfqpoint{6.287585in}{1.356245in}}{\pgfqpoint{6.280743in}{1.353411in}}{\pgfqpoint{6.275700in}{1.348367in}}%
\pgfpathcurveto{\pgfqpoint{6.270656in}{1.343323in}}{\pgfqpoint{6.267822in}{1.336482in}}{\pgfqpoint{6.267822in}{1.329349in}}%
\pgfpathcurveto{\pgfqpoint{6.267822in}{1.322216in}}{\pgfqpoint{6.270656in}{1.315374in}}{\pgfqpoint{6.275700in}{1.310331in}}%
\pgfpathcurveto{\pgfqpoint{6.280743in}{1.305287in}}{\pgfqpoint{6.287585in}{1.302453in}}{\pgfqpoint{6.294718in}{1.302453in}}%
\pgfpathclose%
\pgfusepath{stroke,fill}%
\end{pgfscope}%
\begin{pgfscope}%
\pgfpathrectangle{\pgfqpoint{4.985294in}{0.500000in}}{\pgfqpoint{1.764706in}{1.700000in}}%
\pgfusepath{clip}%
\pgfsetbuttcap%
\pgfsetroundjoin%
\definecolor{currentfill}{rgb}{0.975644,0.874038,0.797253}%
\pgfsetfillcolor{currentfill}%
\pgfsetlinewidth{0.311001pt}%
\definecolor{currentstroke}{rgb}{1.000000,1.000000,1.000000}%
\pgfsetstrokecolor{currentstroke}%
\pgfsetdash{}{0pt}%
\pgfpathmoveto{\pgfqpoint{6.235346in}{1.108904in}}%
\pgfpathcurveto{\pgfqpoint{6.242479in}{1.108904in}}{\pgfqpoint{6.249320in}{1.111737in}}{\pgfqpoint{6.254364in}{1.116781in}}%
\pgfpathcurveto{\pgfqpoint{6.259408in}{1.121825in}}{\pgfqpoint{6.262242in}{1.128666in}}{\pgfqpoint{6.262242in}{1.135799in}}%
\pgfpathcurveto{\pgfqpoint{6.262242in}{1.142932in}}{\pgfqpoint{6.259408in}{1.149774in}}{\pgfqpoint{6.254364in}{1.154817in}}%
\pgfpathcurveto{\pgfqpoint{6.249320in}{1.159861in}}{\pgfqpoint{6.242479in}{1.162695in}}{\pgfqpoint{6.235346in}{1.162695in}}%
\pgfpathcurveto{\pgfqpoint{6.228213in}{1.162695in}}{\pgfqpoint{6.221371in}{1.159861in}}{\pgfqpoint{6.216328in}{1.154817in}}%
\pgfpathcurveto{\pgfqpoint{6.211284in}{1.149774in}}{\pgfqpoint{6.208450in}{1.142932in}}{\pgfqpoint{6.208450in}{1.135799in}}%
\pgfpathcurveto{\pgfqpoint{6.208450in}{1.128666in}}{\pgfqpoint{6.211284in}{1.121825in}}{\pgfqpoint{6.216328in}{1.116781in}}%
\pgfpathcurveto{\pgfqpoint{6.221371in}{1.111737in}}{\pgfqpoint{6.228213in}{1.108904in}}{\pgfqpoint{6.235346in}{1.108904in}}%
\pgfpathclose%
\pgfusepath{stroke,fill}%
\end{pgfscope}%
\begin{pgfscope}%
\pgfpathrectangle{\pgfqpoint{4.985294in}{0.500000in}}{\pgfqpoint{1.764706in}{1.700000in}}%
\pgfusepath{clip}%
\pgfsetbuttcap%
\pgfsetroundjoin%
\definecolor{currentfill}{rgb}{0.970718,0.821518,0.719872}%
\pgfsetfillcolor{currentfill}%
\pgfsetlinewidth{0.311001pt}%
\definecolor{currentstroke}{rgb}{1.000000,1.000000,1.000000}%
\pgfsetstrokecolor{currentstroke}%
\pgfsetdash{}{0pt}%
\pgfpathmoveto{\pgfqpoint{5.492521in}{1.210545in}}%
\pgfpathcurveto{\pgfqpoint{5.499654in}{1.210545in}}{\pgfqpoint{5.506495in}{1.213378in}}{\pgfqpoint{5.511539in}{1.218422in}}%
\pgfpathcurveto{\pgfqpoint{5.516583in}{1.223466in}}{\pgfqpoint{5.519417in}{1.230307in}}{\pgfqpoint{5.519417in}{1.237440in}}%
\pgfpathcurveto{\pgfqpoint{5.519417in}{1.244573in}}{\pgfqpoint{5.516583in}{1.251415in}}{\pgfqpoint{5.511539in}{1.256458in}}%
\pgfpathcurveto{\pgfqpoint{5.506495in}{1.261502in}}{\pgfqpoint{5.499654in}{1.264336in}}{\pgfqpoint{5.492521in}{1.264336in}}%
\pgfpathcurveto{\pgfqpoint{5.485388in}{1.264336in}}{\pgfqpoint{5.478546in}{1.261502in}}{\pgfqpoint{5.473503in}{1.256458in}}%
\pgfpathcurveto{\pgfqpoint{5.468459in}{1.251415in}}{\pgfqpoint{5.465625in}{1.244573in}}{\pgfqpoint{5.465625in}{1.237440in}}%
\pgfpathcurveto{\pgfqpoint{5.465625in}{1.230307in}}{\pgfqpoint{5.468459in}{1.223466in}}{\pgfqpoint{5.473503in}{1.218422in}}%
\pgfpathcurveto{\pgfqpoint{5.478546in}{1.213378in}}{\pgfqpoint{5.485388in}{1.210545in}}{\pgfqpoint{5.492521in}{1.210545in}}%
\pgfpathclose%
\pgfusepath{stroke,fill}%
\end{pgfscope}%
\begin{pgfscope}%
\pgfpathrectangle{\pgfqpoint{4.985294in}{0.500000in}}{\pgfqpoint{1.764706in}{1.700000in}}%
\pgfusepath{clip}%
\pgfsetbuttcap%
\pgfsetroundjoin%
\definecolor{currentfill}{rgb}{0.974412,0.862387,0.780156}%
\pgfsetfillcolor{currentfill}%
\pgfsetlinewidth{0.311001pt}%
\definecolor{currentstroke}{rgb}{1.000000,1.000000,1.000000}%
\pgfsetstrokecolor{currentstroke}%
\pgfsetdash{}{0pt}%
\pgfpathmoveto{\pgfqpoint{6.247560in}{1.521160in}}%
\pgfpathcurveto{\pgfqpoint{6.254693in}{1.521160in}}{\pgfqpoint{6.261535in}{1.523994in}}{\pgfqpoint{6.266579in}{1.529038in}}%
\pgfpathcurveto{\pgfqpoint{6.271622in}{1.534081in}}{\pgfqpoint{6.274456in}{1.540923in}}{\pgfqpoint{6.274456in}{1.548056in}}%
\pgfpathcurveto{\pgfqpoint{6.274456in}{1.555188in}}{\pgfqpoint{6.271622in}{1.562030in}}{\pgfqpoint{6.266579in}{1.567074in}}%
\pgfpathcurveto{\pgfqpoint{6.261535in}{1.572117in}}{\pgfqpoint{6.254693in}{1.574951in}}{\pgfqpoint{6.247560in}{1.574951in}}%
\pgfpathcurveto{\pgfqpoint{6.240428in}{1.574951in}}{\pgfqpoint{6.233586in}{1.572117in}}{\pgfqpoint{6.228542in}{1.567074in}}%
\pgfpathcurveto{\pgfqpoint{6.223499in}{1.562030in}}{\pgfqpoint{6.220665in}{1.555188in}}{\pgfqpoint{6.220665in}{1.548056in}}%
\pgfpathcurveto{\pgfqpoint{6.220665in}{1.540923in}}{\pgfqpoint{6.223499in}{1.534081in}}{\pgfqpoint{6.228542in}{1.529038in}}%
\pgfpathcurveto{\pgfqpoint{6.233586in}{1.523994in}}{\pgfqpoint{6.240428in}{1.521160in}}{\pgfqpoint{6.247560in}{1.521160in}}%
\pgfpathclose%
\pgfusepath{stroke,fill}%
\end{pgfscope}%
\begin{pgfscope}%
\pgfpathrectangle{\pgfqpoint{4.985294in}{0.500000in}}{\pgfqpoint{1.764706in}{1.700000in}}%
\pgfusepath{clip}%
\pgfsetbuttcap%
\pgfsetroundjoin%
\definecolor{currentfill}{rgb}{0.971202,0.827364,0.728520}%
\pgfsetfillcolor{currentfill}%
\pgfsetlinewidth{0.311001pt}%
\definecolor{currentstroke}{rgb}{1.000000,1.000000,1.000000}%
\pgfsetstrokecolor{currentstroke}%
\pgfsetdash{}{0pt}%
\pgfpathmoveto{\pgfqpoint{6.187543in}{1.665731in}}%
\pgfpathcurveto{\pgfqpoint{6.194675in}{1.665731in}}{\pgfqpoint{6.201517in}{1.668565in}}{\pgfqpoint{6.206561in}{1.673609in}}%
\pgfpathcurveto{\pgfqpoint{6.211604in}{1.678652in}}{\pgfqpoint{6.214438in}{1.685494in}}{\pgfqpoint{6.214438in}{1.692627in}}%
\pgfpathcurveto{\pgfqpoint{6.214438in}{1.699759in}}{\pgfqpoint{6.211604in}{1.706601in}}{\pgfqpoint{6.206561in}{1.711645in}}%
\pgfpathcurveto{\pgfqpoint{6.201517in}{1.716688in}}{\pgfqpoint{6.194675in}{1.719522in}}{\pgfqpoint{6.187543in}{1.719522in}}%
\pgfpathcurveto{\pgfqpoint{6.180410in}{1.719522in}}{\pgfqpoint{6.173568in}{1.716688in}}{\pgfqpoint{6.168524in}{1.711645in}}%
\pgfpathcurveto{\pgfqpoint{6.163481in}{1.706601in}}{\pgfqpoint{6.160647in}{1.699759in}}{\pgfqpoint{6.160647in}{1.692627in}}%
\pgfpathcurveto{\pgfqpoint{6.160647in}{1.685494in}}{\pgfqpoint{6.163481in}{1.678652in}}{\pgfqpoint{6.168524in}{1.673609in}}%
\pgfpathcurveto{\pgfqpoint{6.173568in}{1.668565in}}{\pgfqpoint{6.180410in}{1.665731in}}{\pgfqpoint{6.187543in}{1.665731in}}%
\pgfpathclose%
\pgfusepath{stroke,fill}%
\end{pgfscope}%
\begin{pgfscope}%
\pgfpathrectangle{\pgfqpoint{4.985294in}{0.500000in}}{\pgfqpoint{1.764706in}{1.700000in}}%
\pgfusepath{clip}%
\pgfsetbuttcap%
\pgfsetroundjoin%
\definecolor{currentfill}{rgb}{0.973832,0.856556,0.771584}%
\pgfsetfillcolor{currentfill}%
\pgfsetlinewidth{0.311001pt}%
\definecolor{currentstroke}{rgb}{1.000000,1.000000,1.000000}%
\pgfsetstrokecolor{currentstroke}%
\pgfsetdash{}{0pt}%
\pgfpathmoveto{\pgfqpoint{6.350642in}{1.182826in}}%
\pgfpathcurveto{\pgfqpoint{6.357775in}{1.182826in}}{\pgfqpoint{6.364616in}{1.185660in}}{\pgfqpoint{6.369660in}{1.190703in}}%
\pgfpathcurveto{\pgfqpoint{6.374704in}{1.195747in}}{\pgfqpoint{6.377537in}{1.202589in}}{\pgfqpoint{6.377537in}{1.209722in}}%
\pgfpathcurveto{\pgfqpoint{6.377537in}{1.216854in}}{\pgfqpoint{6.374704in}{1.223696in}}{\pgfqpoint{6.369660in}{1.228740in}}%
\pgfpathcurveto{\pgfqpoint{6.364616in}{1.233783in}}{\pgfqpoint{6.357775in}{1.236617in}}{\pgfqpoint{6.350642in}{1.236617in}}%
\pgfpathcurveto{\pgfqpoint{6.343509in}{1.236617in}}{\pgfqpoint{6.336667in}{1.233783in}}{\pgfqpoint{6.331624in}{1.228740in}}%
\pgfpathcurveto{\pgfqpoint{6.326580in}{1.223696in}}{\pgfqpoint{6.323746in}{1.216854in}}{\pgfqpoint{6.323746in}{1.209722in}}%
\pgfpathcurveto{\pgfqpoint{6.323746in}{1.202589in}}{\pgfqpoint{6.326580in}{1.195747in}}{\pgfqpoint{6.331624in}{1.190703in}}%
\pgfpathcurveto{\pgfqpoint{6.336667in}{1.185660in}}{\pgfqpoint{6.343509in}{1.182826in}}{\pgfqpoint{6.350642in}{1.182826in}}%
\pgfpathclose%
\pgfusepath{stroke,fill}%
\end{pgfscope}%
\begin{pgfscope}%
\pgfpathrectangle{\pgfqpoint{4.985294in}{0.500000in}}{\pgfqpoint{1.764706in}{1.700000in}}%
\pgfusepath{clip}%
\pgfsetbuttcap%
\pgfsetroundjoin%
\definecolor{currentfill}{rgb}{0.963379,0.625574,0.465113}%
\pgfsetfillcolor{currentfill}%
\pgfsetlinewidth{0.311001pt}%
\definecolor{currentstroke}{rgb}{1.000000,1.000000,1.000000}%
\pgfsetstrokecolor{currentstroke}%
\pgfsetdash{}{0pt}%
\pgfpathmoveto{\pgfqpoint{6.181992in}{1.327428in}}%
\pgfpathcurveto{\pgfqpoint{6.189125in}{1.327428in}}{\pgfqpoint{6.195966in}{1.330261in}}{\pgfqpoint{6.201010in}{1.335305in}}%
\pgfpathcurveto{\pgfqpoint{6.206053in}{1.340349in}}{\pgfqpoint{6.208887in}{1.347190in}}{\pgfqpoint{6.208887in}{1.354323in}}%
\pgfpathcurveto{\pgfqpoint{6.208887in}{1.361456in}}{\pgfqpoint{6.206053in}{1.368298in}}{\pgfqpoint{6.201010in}{1.373341in}}%
\pgfpathcurveto{\pgfqpoint{6.195966in}{1.378385in}}{\pgfqpoint{6.189125in}{1.381219in}}{\pgfqpoint{6.181992in}{1.381219in}}%
\pgfpathcurveto{\pgfqpoint{6.174859in}{1.381219in}}{\pgfqpoint{6.168017in}{1.378385in}}{\pgfqpoint{6.162974in}{1.373341in}}%
\pgfpathcurveto{\pgfqpoint{6.157930in}{1.368298in}}{\pgfqpoint{6.155096in}{1.361456in}}{\pgfqpoint{6.155096in}{1.354323in}}%
\pgfpathcurveto{\pgfqpoint{6.155096in}{1.347190in}}{\pgfqpoint{6.157930in}{1.340349in}}{\pgfqpoint{6.162974in}{1.335305in}}%
\pgfpathcurveto{\pgfqpoint{6.168017in}{1.330261in}}{\pgfqpoint{6.174859in}{1.327428in}}{\pgfqpoint{6.181992in}{1.327428in}}%
\pgfpathclose%
\pgfusepath{stroke,fill}%
\end{pgfscope}%
\begin{pgfscope}%
\pgfpathrectangle{\pgfqpoint{4.985294in}{0.500000in}}{\pgfqpoint{1.764706in}{1.700000in}}%
\pgfusepath{clip}%
\pgfsetbuttcap%
\pgfsetroundjoin%
\definecolor{currentfill}{rgb}{0.967735,0.780441,0.659127}%
\pgfsetfillcolor{currentfill}%
\pgfsetlinewidth{0.311001pt}%
\definecolor{currentstroke}{rgb}{1.000000,1.000000,1.000000}%
\pgfsetstrokecolor{currentstroke}%
\pgfsetdash{}{0pt}%
\pgfpathmoveto{\pgfqpoint{6.166498in}{0.981928in}}%
\pgfpathcurveto{\pgfqpoint{6.173631in}{0.981928in}}{\pgfqpoint{6.180473in}{0.984762in}}{\pgfqpoint{6.185517in}{0.989806in}}%
\pgfpathcurveto{\pgfqpoint{6.190560in}{0.994850in}}{\pgfqpoint{6.193394in}{1.001691in}}{\pgfqpoint{6.193394in}{1.008824in}}%
\pgfpathcurveto{\pgfqpoint{6.193394in}{1.015957in}}{\pgfqpoint{6.190560in}{1.022798in}}{\pgfqpoint{6.185517in}{1.027842in}}%
\pgfpathcurveto{\pgfqpoint{6.180473in}{1.032886in}}{\pgfqpoint{6.173631in}{1.035720in}}{\pgfqpoint{6.166498in}{1.035720in}}%
\pgfpathcurveto{\pgfqpoint{6.159366in}{1.035720in}}{\pgfqpoint{6.152524in}{1.032886in}}{\pgfqpoint{6.147480in}{1.027842in}}%
\pgfpathcurveto{\pgfqpoint{6.142437in}{1.022798in}}{\pgfqpoint{6.139603in}{1.015957in}}{\pgfqpoint{6.139603in}{1.008824in}}%
\pgfpathcurveto{\pgfqpoint{6.139603in}{1.001691in}}{\pgfqpoint{6.142437in}{0.994850in}}{\pgfqpoint{6.147480in}{0.989806in}}%
\pgfpathcurveto{\pgfqpoint{6.152524in}{0.984762in}}{\pgfqpoint{6.159366in}{0.981928in}}{\pgfqpoint{6.166498in}{0.981928in}}%
\pgfpathclose%
\pgfusepath{stroke,fill}%
\end{pgfscope}%
\begin{pgfscope}%
\pgfpathrectangle{\pgfqpoint{4.985294in}{0.500000in}}{\pgfqpoint{1.764706in}{1.700000in}}%
\pgfusepath{clip}%
\pgfsetbuttcap%
\pgfsetroundjoin%
\definecolor{currentfill}{rgb}{0.730358,0.086862,0.337485}%
\pgfsetfillcolor{currentfill}%
\pgfsetlinewidth{0.311001pt}%
\definecolor{currentstroke}{rgb}{1.000000,1.000000,1.000000}%
\pgfsetstrokecolor{currentstroke}%
\pgfsetdash{}{0pt}%
\pgfpathmoveto{\pgfqpoint{5.226004in}{1.395190in}}%
\pgfpathcurveto{\pgfqpoint{5.233137in}{1.395190in}}{\pgfqpoint{5.239979in}{1.398023in}}{\pgfqpoint{5.245023in}{1.403067in}}%
\pgfpathcurveto{\pgfqpoint{5.250066in}{1.408111in}}{\pgfqpoint{5.252900in}{1.414952in}}{\pgfqpoint{5.252900in}{1.422085in}}%
\pgfpathcurveto{\pgfqpoint{5.252900in}{1.429218in}}{\pgfqpoint{5.250066in}{1.436060in}}{\pgfqpoint{5.245023in}{1.441103in}}%
\pgfpathcurveto{\pgfqpoint{5.239979in}{1.446147in}}{\pgfqpoint{5.233137in}{1.448981in}}{\pgfqpoint{5.226004in}{1.448981in}}%
\pgfpathcurveto{\pgfqpoint{5.218872in}{1.448981in}}{\pgfqpoint{5.212030in}{1.446147in}}{\pgfqpoint{5.206986in}{1.441103in}}%
\pgfpathcurveto{\pgfqpoint{5.201943in}{1.436060in}}{\pgfqpoint{5.199109in}{1.429218in}}{\pgfqpoint{5.199109in}{1.422085in}}%
\pgfpathcurveto{\pgfqpoint{5.199109in}{1.414952in}}{\pgfqpoint{5.201943in}{1.408111in}}{\pgfqpoint{5.206986in}{1.403067in}}%
\pgfpathcurveto{\pgfqpoint{5.212030in}{1.398023in}}{\pgfqpoint{5.218872in}{1.395190in}}{\pgfqpoint{5.226004in}{1.395190in}}%
\pgfpathclose%
\pgfusepath{stroke,fill}%
\end{pgfscope}%
\begin{pgfscope}%
\pgfpathrectangle{\pgfqpoint{4.985294in}{0.500000in}}{\pgfqpoint{1.764706in}{1.700000in}}%
\pgfusepath{clip}%
\pgfsetbuttcap%
\pgfsetroundjoin%
\definecolor{currentfill}{rgb}{0.972201,0.839051,0.745789}%
\pgfsetfillcolor{currentfill}%
\pgfsetlinewidth{0.311001pt}%
\definecolor{currentstroke}{rgb}{1.000000,1.000000,1.000000}%
\pgfsetstrokecolor{currentstroke}%
\pgfsetdash{}{0pt}%
\pgfpathmoveto{\pgfqpoint{6.214940in}{1.542787in}}%
\pgfpathcurveto{\pgfqpoint{6.222073in}{1.542787in}}{\pgfqpoint{6.228914in}{1.545621in}}{\pgfqpoint{6.233958in}{1.550665in}}%
\pgfpathcurveto{\pgfqpoint{6.239002in}{1.555709in}}{\pgfqpoint{6.241836in}{1.562550in}}{\pgfqpoint{6.241836in}{1.569683in}}%
\pgfpathcurveto{\pgfqpoint{6.241836in}{1.576816in}}{\pgfqpoint{6.239002in}{1.583658in}}{\pgfqpoint{6.233958in}{1.588701in}}%
\pgfpathcurveto{\pgfqpoint{6.228914in}{1.593745in}}{\pgfqpoint{6.222073in}{1.596579in}}{\pgfqpoint{6.214940in}{1.596579in}}%
\pgfpathcurveto{\pgfqpoint{6.207807in}{1.596579in}}{\pgfqpoint{6.200965in}{1.593745in}}{\pgfqpoint{6.195922in}{1.588701in}}%
\pgfpathcurveto{\pgfqpoint{6.190878in}{1.583658in}}{\pgfqpoint{6.188044in}{1.576816in}}{\pgfqpoint{6.188044in}{1.569683in}}%
\pgfpathcurveto{\pgfqpoint{6.188044in}{1.562550in}}{\pgfqpoint{6.190878in}{1.555709in}}{\pgfqpoint{6.195922in}{1.550665in}}%
\pgfpathcurveto{\pgfqpoint{6.200965in}{1.545621in}}{\pgfqpoint{6.207807in}{1.542787in}}{\pgfqpoint{6.214940in}{1.542787in}}%
\pgfpathclose%
\pgfusepath{stroke,fill}%
\end{pgfscope}%
\begin{pgfscope}%
\pgfpathrectangle{\pgfqpoint{4.985294in}{0.500000in}}{\pgfqpoint{1.764706in}{1.700000in}}%
\pgfusepath{clip}%
\pgfsetbuttcap%
\pgfsetroundjoin%
\definecolor{currentfill}{rgb}{0.964679,0.682838,0.530002}%
\pgfsetfillcolor{currentfill}%
\pgfsetlinewidth{0.311001pt}%
\definecolor{currentstroke}{rgb}{1.000000,1.000000,1.000000}%
\pgfsetstrokecolor{currentstroke}%
\pgfsetdash{}{0pt}%
\pgfpathmoveto{\pgfqpoint{6.413727in}{1.254417in}}%
\pgfpathcurveto{\pgfqpoint{6.420860in}{1.254417in}}{\pgfqpoint{6.427701in}{1.257251in}}{\pgfqpoint{6.432745in}{1.262295in}}%
\pgfpathcurveto{\pgfqpoint{6.437789in}{1.267339in}}{\pgfqpoint{6.440623in}{1.274180in}}{\pgfqpoint{6.440623in}{1.281313in}}%
\pgfpathcurveto{\pgfqpoint{6.440623in}{1.288446in}}{\pgfqpoint{6.437789in}{1.295288in}}{\pgfqpoint{6.432745in}{1.300331in}}%
\pgfpathcurveto{\pgfqpoint{6.427701in}{1.305375in}}{\pgfqpoint{6.420860in}{1.308209in}}{\pgfqpoint{6.413727in}{1.308209in}}%
\pgfpathcurveto{\pgfqpoint{6.406594in}{1.308209in}}{\pgfqpoint{6.399753in}{1.305375in}}{\pgfqpoint{6.394709in}{1.300331in}}%
\pgfpathcurveto{\pgfqpoint{6.389665in}{1.295288in}}{\pgfqpoint{6.386831in}{1.288446in}}{\pgfqpoint{6.386831in}{1.281313in}}%
\pgfpathcurveto{\pgfqpoint{6.386831in}{1.274180in}}{\pgfqpoint{6.389665in}{1.267339in}}{\pgfqpoint{6.394709in}{1.262295in}}%
\pgfpathcurveto{\pgfqpoint{6.399753in}{1.257251in}}{\pgfqpoint{6.406594in}{1.254417in}}{\pgfqpoint{6.413727in}{1.254417in}}%
\pgfpathclose%
\pgfusepath{stroke,fill}%
\end{pgfscope}%
\begin{pgfscope}%
\pgfpathrectangle{\pgfqpoint{4.985294in}{0.500000in}}{\pgfqpoint{1.764706in}{1.700000in}}%
\pgfusepath{clip}%
\pgfsetbuttcap%
\pgfsetroundjoin%
\definecolor{currentfill}{rgb}{0.974412,0.862387,0.780156}%
\pgfsetfillcolor{currentfill}%
\pgfsetlinewidth{0.311001pt}%
\definecolor{currentstroke}{rgb}{1.000000,1.000000,1.000000}%
\pgfsetstrokecolor{currentstroke}%
\pgfsetdash{}{0pt}%
\pgfpathmoveto{\pgfqpoint{6.265648in}{1.342118in}}%
\pgfpathcurveto{\pgfqpoint{6.272781in}{1.342118in}}{\pgfqpoint{6.279622in}{1.344952in}}{\pgfqpoint{6.284666in}{1.349996in}}%
\pgfpathcurveto{\pgfqpoint{6.289710in}{1.355039in}}{\pgfqpoint{6.292544in}{1.361881in}}{\pgfqpoint{6.292544in}{1.369014in}}%
\pgfpathcurveto{\pgfqpoint{6.292544in}{1.376147in}}{\pgfqpoint{6.289710in}{1.382988in}}{\pgfqpoint{6.284666in}{1.388032in}}%
\pgfpathcurveto{\pgfqpoint{6.279622in}{1.393076in}}{\pgfqpoint{6.272781in}{1.395910in}}{\pgfqpoint{6.265648in}{1.395910in}}%
\pgfpathcurveto{\pgfqpoint{6.258515in}{1.395910in}}{\pgfqpoint{6.251673in}{1.393076in}}{\pgfqpoint{6.246630in}{1.388032in}}%
\pgfpathcurveto{\pgfqpoint{6.241586in}{1.382988in}}{\pgfqpoint{6.238752in}{1.376147in}}{\pgfqpoint{6.238752in}{1.369014in}}%
\pgfpathcurveto{\pgfqpoint{6.238752in}{1.361881in}}{\pgfqpoint{6.241586in}{1.355039in}}{\pgfqpoint{6.246630in}{1.349996in}}%
\pgfpathcurveto{\pgfqpoint{6.251673in}{1.344952in}}{\pgfqpoint{6.258515in}{1.342118in}}{\pgfqpoint{6.265648in}{1.342118in}}%
\pgfpathclose%
\pgfusepath{stroke,fill}%
\end{pgfscope}%
\begin{pgfscope}%
\pgfpathrectangle{\pgfqpoint{4.985294in}{0.500000in}}{\pgfqpoint{1.764706in}{1.700000in}}%
\pgfusepath{clip}%
\pgfsetbuttcap%
\pgfsetroundjoin%
\definecolor{currentfill}{rgb}{0.971694,0.833208,0.737161}%
\pgfsetfillcolor{currentfill}%
\pgfsetlinewidth{0.311001pt}%
\definecolor{currentstroke}{rgb}{1.000000,1.000000,1.000000}%
\pgfsetstrokecolor{currentstroke}%
\pgfsetdash{}{0pt}%
\pgfpathmoveto{\pgfqpoint{5.478652in}{1.004260in}}%
\pgfpathcurveto{\pgfqpoint{5.485784in}{1.004260in}}{\pgfqpoint{5.492626in}{1.007094in}}{\pgfqpoint{5.497670in}{1.012138in}}%
\pgfpathcurveto{\pgfqpoint{5.502713in}{1.017181in}}{\pgfqpoint{5.505547in}{1.024023in}}{\pgfqpoint{5.505547in}{1.031156in}}%
\pgfpathcurveto{\pgfqpoint{5.505547in}{1.038289in}}{\pgfqpoint{5.502713in}{1.045130in}}{\pgfqpoint{5.497670in}{1.050174in}}%
\pgfpathcurveto{\pgfqpoint{5.492626in}{1.055218in}}{\pgfqpoint{5.485784in}{1.058052in}}{\pgfqpoint{5.478652in}{1.058052in}}%
\pgfpathcurveto{\pgfqpoint{5.471519in}{1.058052in}}{\pgfqpoint{5.464677in}{1.055218in}}{\pgfqpoint{5.459634in}{1.050174in}}%
\pgfpathcurveto{\pgfqpoint{5.454590in}{1.045130in}}{\pgfqpoint{5.451756in}{1.038289in}}{\pgfqpoint{5.451756in}{1.031156in}}%
\pgfpathcurveto{\pgfqpoint{5.451756in}{1.024023in}}{\pgfqpoint{5.454590in}{1.017181in}}{\pgfqpoint{5.459634in}{1.012138in}}%
\pgfpathcurveto{\pgfqpoint{5.464677in}{1.007094in}}{\pgfqpoint{5.471519in}{1.004260in}}{\pgfqpoint{5.478652in}{1.004260in}}%
\pgfpathclose%
\pgfusepath{stroke,fill}%
\end{pgfscope}%
\begin{pgfscope}%
\pgfpathrectangle{\pgfqpoint{4.985294in}{0.500000in}}{\pgfqpoint{1.764706in}{1.700000in}}%
\pgfusepath{clip}%
\pgfsetbuttcap%
\pgfsetroundjoin%
\definecolor{currentfill}{rgb}{0.966560,0.756582,0.625273}%
\pgfsetfillcolor{currentfill}%
\pgfsetlinewidth{0.311001pt}%
\definecolor{currentstroke}{rgb}{1.000000,1.000000,1.000000}%
\pgfsetstrokecolor{currentstroke}%
\pgfsetdash{}{0pt}%
\pgfpathmoveto{\pgfqpoint{6.156837in}{0.959362in}}%
\pgfpathcurveto{\pgfqpoint{6.163970in}{0.959362in}}{\pgfqpoint{6.170812in}{0.962196in}}{\pgfqpoint{6.175855in}{0.967239in}}%
\pgfpathcurveto{\pgfqpoint{6.180899in}{0.972283in}}{\pgfqpoint{6.183733in}{0.979125in}}{\pgfqpoint{6.183733in}{0.986258in}}%
\pgfpathcurveto{\pgfqpoint{6.183733in}{0.993390in}}{\pgfqpoint{6.180899in}{1.000232in}}{\pgfqpoint{6.175855in}{1.005276in}}%
\pgfpathcurveto{\pgfqpoint{6.170812in}{1.010319in}}{\pgfqpoint{6.163970in}{1.013153in}}{\pgfqpoint{6.156837in}{1.013153in}}%
\pgfpathcurveto{\pgfqpoint{6.149704in}{1.013153in}}{\pgfqpoint{6.142863in}{1.010319in}}{\pgfqpoint{6.137819in}{1.005276in}}%
\pgfpathcurveto{\pgfqpoint{6.132775in}{1.000232in}}{\pgfqpoint{6.129942in}{0.993390in}}{\pgfqpoint{6.129942in}{0.986258in}}%
\pgfpathcurveto{\pgfqpoint{6.129942in}{0.979125in}}{\pgfqpoint{6.132775in}{0.972283in}}{\pgfqpoint{6.137819in}{0.967239in}}%
\pgfpathcurveto{\pgfqpoint{6.142863in}{0.962196in}}{\pgfqpoint{6.149704in}{0.959362in}}{\pgfqpoint{6.156837in}{0.959362in}}%
\pgfpathclose%
\pgfusepath{stroke,fill}%
\end{pgfscope}%
\begin{pgfscope}%
\pgfpathrectangle{\pgfqpoint{4.985294in}{0.500000in}}{\pgfqpoint{1.764706in}{1.700000in}}%
\pgfusepath{clip}%
\pgfsetbuttcap%
\pgfsetroundjoin%
\definecolor{currentfill}{rgb}{0.973832,0.856556,0.771584}%
\pgfsetfillcolor{currentfill}%
\pgfsetlinewidth{0.311001pt}%
\definecolor{currentstroke}{rgb}{1.000000,1.000000,1.000000}%
\pgfsetstrokecolor{currentstroke}%
\pgfsetdash{}{0pt}%
\pgfpathmoveto{\pgfqpoint{6.284174in}{1.612045in}}%
\pgfpathcurveto{\pgfqpoint{6.291307in}{1.612045in}}{\pgfqpoint{6.298149in}{1.614879in}}{\pgfqpoint{6.303192in}{1.619923in}}%
\pgfpathcurveto{\pgfqpoint{6.308236in}{1.624966in}}{\pgfqpoint{6.311070in}{1.631808in}}{\pgfqpoint{6.311070in}{1.638941in}}%
\pgfpathcurveto{\pgfqpoint{6.311070in}{1.646074in}}{\pgfqpoint{6.308236in}{1.652915in}}{\pgfqpoint{6.303192in}{1.657959in}}%
\pgfpathcurveto{\pgfqpoint{6.298149in}{1.663003in}}{\pgfqpoint{6.291307in}{1.665836in}}{\pgfqpoint{6.284174in}{1.665836in}}%
\pgfpathcurveto{\pgfqpoint{6.277041in}{1.665836in}}{\pgfqpoint{6.270200in}{1.663003in}}{\pgfqpoint{6.265156in}{1.657959in}}%
\pgfpathcurveto{\pgfqpoint{6.260112in}{1.652915in}}{\pgfqpoint{6.257278in}{1.646074in}}{\pgfqpoint{6.257278in}{1.638941in}}%
\pgfpathcurveto{\pgfqpoint{6.257278in}{1.631808in}}{\pgfqpoint{6.260112in}{1.624966in}}{\pgfqpoint{6.265156in}{1.619923in}}%
\pgfpathcurveto{\pgfqpoint{6.270200in}{1.614879in}}{\pgfqpoint{6.277041in}{1.612045in}}{\pgfqpoint{6.284174in}{1.612045in}}%
\pgfpathclose%
\pgfusepath{stroke,fill}%
\end{pgfscope}%
\begin{pgfscope}%
\pgfpathrectangle{\pgfqpoint{4.985294in}{0.500000in}}{\pgfqpoint{1.764706in}{1.700000in}}%
\pgfusepath{clip}%
\pgfsetbuttcap%
\pgfsetroundjoin%
\definecolor{currentfill}{rgb}{0.977657,0.891500,0.822809}%
\pgfsetfillcolor{currentfill}%
\pgfsetlinewidth{0.311001pt}%
\definecolor{currentstroke}{rgb}{1.000000,1.000000,1.000000}%
\pgfsetstrokecolor{currentstroke}%
\pgfsetdash{}{0pt}%
\pgfpathmoveto{\pgfqpoint{5.430942in}{1.489478in}}%
\pgfpathcurveto{\pgfqpoint{5.438075in}{1.489478in}}{\pgfqpoint{5.444917in}{1.492312in}}{\pgfqpoint{5.449960in}{1.497356in}}%
\pgfpathcurveto{\pgfqpoint{5.455004in}{1.502400in}}{\pgfqpoint{5.457838in}{1.509241in}}{\pgfqpoint{5.457838in}{1.516374in}}%
\pgfpathcurveto{\pgfqpoint{5.457838in}{1.523507in}}{\pgfqpoint{5.455004in}{1.530349in}}{\pgfqpoint{5.449960in}{1.535392in}}%
\pgfpathcurveto{\pgfqpoint{5.444917in}{1.540436in}}{\pgfqpoint{5.438075in}{1.543270in}}{\pgfqpoint{5.430942in}{1.543270in}}%
\pgfpathcurveto{\pgfqpoint{5.423809in}{1.543270in}}{\pgfqpoint{5.416968in}{1.540436in}}{\pgfqpoint{5.411924in}{1.535392in}}%
\pgfpathcurveto{\pgfqpoint{5.406880in}{1.530349in}}{\pgfqpoint{5.404046in}{1.523507in}}{\pgfqpoint{5.404046in}{1.516374in}}%
\pgfpathcurveto{\pgfqpoint{5.404046in}{1.509241in}}{\pgfqpoint{5.406880in}{1.502400in}}{\pgfqpoint{5.411924in}{1.497356in}}%
\pgfpathcurveto{\pgfqpoint{5.416968in}{1.492312in}}{\pgfqpoint{5.423809in}{1.489478in}}{\pgfqpoint{5.430942in}{1.489478in}}%
\pgfpathclose%
\pgfusepath{stroke,fill}%
\end{pgfscope}%
\begin{pgfscope}%
\pgfpathrectangle{\pgfqpoint{4.985294in}{0.500000in}}{\pgfqpoint{1.764706in}{1.700000in}}%
\pgfusepath{clip}%
\pgfsetbuttcap%
\pgfsetroundjoin%
\definecolor{currentfill}{rgb}{0.971694,0.833208,0.737161}%
\pgfsetfillcolor{currentfill}%
\pgfsetlinewidth{0.311001pt}%
\definecolor{currentstroke}{rgb}{1.000000,1.000000,1.000000}%
\pgfsetstrokecolor{currentstroke}%
\pgfsetdash{}{0pt}%
\pgfpathmoveto{\pgfqpoint{5.515353in}{1.506489in}}%
\pgfpathcurveto{\pgfqpoint{5.522486in}{1.506489in}}{\pgfqpoint{5.529327in}{1.509323in}}{\pgfqpoint{5.534371in}{1.514367in}}%
\pgfpathcurveto{\pgfqpoint{5.539415in}{1.519410in}}{\pgfqpoint{5.542248in}{1.526252in}}{\pgfqpoint{5.542248in}{1.533385in}}%
\pgfpathcurveto{\pgfqpoint{5.542248in}{1.540518in}}{\pgfqpoint{5.539415in}{1.547359in}}{\pgfqpoint{5.534371in}{1.552403in}}%
\pgfpathcurveto{\pgfqpoint{5.529327in}{1.557447in}}{\pgfqpoint{5.522486in}{1.560281in}}{\pgfqpoint{5.515353in}{1.560281in}}%
\pgfpathcurveto{\pgfqpoint{5.508220in}{1.560281in}}{\pgfqpoint{5.501378in}{1.557447in}}{\pgfqpoint{5.496335in}{1.552403in}}%
\pgfpathcurveto{\pgfqpoint{5.491291in}{1.547359in}}{\pgfqpoint{5.488457in}{1.540518in}}{\pgfqpoint{5.488457in}{1.533385in}}%
\pgfpathcurveto{\pgfqpoint{5.488457in}{1.526252in}}{\pgfqpoint{5.491291in}{1.519410in}}{\pgfqpoint{5.496335in}{1.514367in}}%
\pgfpathcurveto{\pgfqpoint{5.501378in}{1.509323in}}{\pgfqpoint{5.508220in}{1.506489in}}{\pgfqpoint{5.515353in}{1.506489in}}%
\pgfpathclose%
\pgfusepath{stroke,fill}%
\end{pgfscope}%
\begin{pgfscope}%
\pgfpathrectangle{\pgfqpoint{4.985294in}{0.500000in}}{\pgfqpoint{1.764706in}{1.700000in}}%
\pgfusepath{clip}%
\pgfsetbuttcap%
\pgfsetroundjoin%
\definecolor{currentfill}{rgb}{0.972201,0.839051,0.745789}%
\pgfsetfillcolor{currentfill}%
\pgfsetlinewidth{0.311001pt}%
\definecolor{currentstroke}{rgb}{1.000000,1.000000,1.000000}%
\pgfsetstrokecolor{currentstroke}%
\pgfsetdash{}{0pt}%
\pgfpathmoveto{\pgfqpoint{5.518864in}{1.576282in}}%
\pgfpathcurveto{\pgfqpoint{5.525997in}{1.576282in}}{\pgfqpoint{5.532838in}{1.579116in}}{\pgfqpoint{5.537882in}{1.584160in}}%
\pgfpathcurveto{\pgfqpoint{5.542926in}{1.589204in}}{\pgfqpoint{5.545760in}{1.596045in}}{\pgfqpoint{5.545760in}{1.603178in}}%
\pgfpathcurveto{\pgfqpoint{5.545760in}{1.610311in}}{\pgfqpoint{5.542926in}{1.617153in}}{\pgfqpoint{5.537882in}{1.622196in}}%
\pgfpathcurveto{\pgfqpoint{5.532838in}{1.627240in}}{\pgfqpoint{5.525997in}{1.630074in}}{\pgfqpoint{5.518864in}{1.630074in}}%
\pgfpathcurveto{\pgfqpoint{5.511731in}{1.630074in}}{\pgfqpoint{5.504889in}{1.627240in}}{\pgfqpoint{5.499846in}{1.622196in}}%
\pgfpathcurveto{\pgfqpoint{5.494802in}{1.617153in}}{\pgfqpoint{5.491968in}{1.610311in}}{\pgfqpoint{5.491968in}{1.603178in}}%
\pgfpathcurveto{\pgfqpoint{5.491968in}{1.596045in}}{\pgfqpoint{5.494802in}{1.589204in}}{\pgfqpoint{5.499846in}{1.584160in}}%
\pgfpathcurveto{\pgfqpoint{5.504889in}{1.579116in}}{\pgfqpoint{5.511731in}{1.576282in}}{\pgfqpoint{5.518864in}{1.576282in}}%
\pgfpathclose%
\pgfusepath{stroke,fill}%
\end{pgfscope}%
\begin{pgfscope}%
\pgfpathrectangle{\pgfqpoint{4.985294in}{0.500000in}}{\pgfqpoint{1.764706in}{1.700000in}}%
\pgfusepath{clip}%
\pgfsetbuttcap%
\pgfsetroundjoin%
\definecolor{currentfill}{rgb}{0.971202,0.827364,0.728520}%
\pgfsetfillcolor{currentfill}%
\pgfsetlinewidth{0.311001pt}%
\definecolor{currentstroke}{rgb}{1.000000,1.000000,1.000000}%
\pgfsetstrokecolor{currentstroke}%
\pgfsetdash{}{0pt}%
\pgfpathmoveto{\pgfqpoint{5.356393in}{1.343419in}}%
\pgfpathcurveto{\pgfqpoint{5.363526in}{1.343419in}}{\pgfqpoint{5.370367in}{1.346253in}}{\pgfqpoint{5.375411in}{1.351297in}}%
\pgfpathcurveto{\pgfqpoint{5.380455in}{1.356341in}}{\pgfqpoint{5.383288in}{1.363182in}}{\pgfqpoint{5.383288in}{1.370315in}}%
\pgfpathcurveto{\pgfqpoint{5.383288in}{1.377448in}}{\pgfqpoint{5.380455in}{1.384290in}}{\pgfqpoint{5.375411in}{1.389333in}}%
\pgfpathcurveto{\pgfqpoint{5.370367in}{1.394377in}}{\pgfqpoint{5.363526in}{1.397211in}}{\pgfqpoint{5.356393in}{1.397211in}}%
\pgfpathcurveto{\pgfqpoint{5.349260in}{1.397211in}}{\pgfqpoint{5.342418in}{1.394377in}}{\pgfqpoint{5.337375in}{1.389333in}}%
\pgfpathcurveto{\pgfqpoint{5.332331in}{1.384290in}}{\pgfqpoint{5.329497in}{1.377448in}}{\pgfqpoint{5.329497in}{1.370315in}}%
\pgfpathcurveto{\pgfqpoint{5.329497in}{1.363182in}}{\pgfqpoint{5.332331in}{1.356341in}}{\pgfqpoint{5.337375in}{1.351297in}}%
\pgfpathcurveto{\pgfqpoint{5.342418in}{1.346253in}}{\pgfqpoint{5.349260in}{1.343419in}}{\pgfqpoint{5.356393in}{1.343419in}}%
\pgfpathclose%
\pgfusepath{stroke,fill}%
\end{pgfscope}%
\begin{pgfscope}%
\pgfpathrectangle{\pgfqpoint{4.985294in}{0.500000in}}{\pgfqpoint{1.764706in}{1.700000in}}%
\pgfusepath{clip}%
\pgfsetbuttcap%
\pgfsetroundjoin%
\definecolor{currentfill}{rgb}{0.966328,0.750560,0.616961}%
\pgfsetfillcolor{currentfill}%
\pgfsetlinewidth{0.311001pt}%
\definecolor{currentstroke}{rgb}{1.000000,1.000000,1.000000}%
\pgfsetstrokecolor{currentstroke}%
\pgfsetdash{}{0pt}%
\pgfpathmoveto{\pgfqpoint{6.149615in}{1.606756in}}%
\pgfpathcurveto{\pgfqpoint{6.156748in}{1.606756in}}{\pgfqpoint{6.163590in}{1.609590in}}{\pgfqpoint{6.168634in}{1.614634in}}%
\pgfpathcurveto{\pgfqpoint{6.173677in}{1.619677in}}{\pgfqpoint{6.176511in}{1.626519in}}{\pgfqpoint{6.176511in}{1.633652in}}%
\pgfpathcurveto{\pgfqpoint{6.176511in}{1.640785in}}{\pgfqpoint{6.173677in}{1.647626in}}{\pgfqpoint{6.168634in}{1.652670in}}%
\pgfpathcurveto{\pgfqpoint{6.163590in}{1.657714in}}{\pgfqpoint{6.156748in}{1.660548in}}{\pgfqpoint{6.149615in}{1.660548in}}%
\pgfpathcurveto{\pgfqpoint{6.142483in}{1.660548in}}{\pgfqpoint{6.135641in}{1.657714in}}{\pgfqpoint{6.130597in}{1.652670in}}%
\pgfpathcurveto{\pgfqpoint{6.125554in}{1.647626in}}{\pgfqpoint{6.122720in}{1.640785in}}{\pgfqpoint{6.122720in}{1.633652in}}%
\pgfpathcurveto{\pgfqpoint{6.122720in}{1.626519in}}{\pgfqpoint{6.125554in}{1.619677in}}{\pgfqpoint{6.130597in}{1.614634in}}%
\pgfpathcurveto{\pgfqpoint{6.135641in}{1.609590in}}{\pgfqpoint{6.142483in}{1.606756in}}{\pgfqpoint{6.149615in}{1.606756in}}%
\pgfpathclose%
\pgfusepath{stroke,fill}%
\end{pgfscope}%
\begin{pgfscope}%
\pgfpathrectangle{\pgfqpoint{4.985294in}{0.500000in}}{\pgfqpoint{1.764706in}{1.700000in}}%
\pgfusepath{clip}%
\pgfsetbuttcap%
\pgfsetroundjoin%
\definecolor{currentfill}{rgb}{0.828528,0.130141,0.293475}%
\pgfsetfillcolor{currentfill}%
\pgfsetlinewidth{0.311001pt}%
\definecolor{currentstroke}{rgb}{1.000000,1.000000,1.000000}%
\pgfsetstrokecolor{currentstroke}%
\pgfsetdash{}{0pt}%
\pgfpathmoveto{\pgfqpoint{5.418228in}{1.789525in}}%
\pgfpathcurveto{\pgfqpoint{5.425361in}{1.789525in}}{\pgfqpoint{5.432203in}{1.792358in}}{\pgfqpoint{5.437246in}{1.797402in}}%
\pgfpathcurveto{\pgfqpoint{5.442290in}{1.802446in}}{\pgfqpoint{5.445124in}{1.809287in}}{\pgfqpoint{5.445124in}{1.816420in}}%
\pgfpathcurveto{\pgfqpoint{5.445124in}{1.823553in}}{\pgfqpoint{5.442290in}{1.830395in}}{\pgfqpoint{5.437246in}{1.835438in}}%
\pgfpathcurveto{\pgfqpoint{5.432203in}{1.840482in}}{\pgfqpoint{5.425361in}{1.843316in}}{\pgfqpoint{5.418228in}{1.843316in}}%
\pgfpathcurveto{\pgfqpoint{5.411095in}{1.843316in}}{\pgfqpoint{5.404254in}{1.840482in}}{\pgfqpoint{5.399210in}{1.835438in}}%
\pgfpathcurveto{\pgfqpoint{5.394166in}{1.830395in}}{\pgfqpoint{5.391332in}{1.823553in}}{\pgfqpoint{5.391332in}{1.816420in}}%
\pgfpathcurveto{\pgfqpoint{5.391332in}{1.809287in}}{\pgfqpoint{5.394166in}{1.802446in}}{\pgfqpoint{5.399210in}{1.797402in}}%
\pgfpathcurveto{\pgfqpoint{5.404254in}{1.792358in}}{\pgfqpoint{5.411095in}{1.789525in}}{\pgfqpoint{5.418228in}{1.789525in}}%
\pgfpathclose%
\pgfusepath{stroke,fill}%
\end{pgfscope}%
\begin{pgfscope}%
\pgfpathrectangle{\pgfqpoint{4.985294in}{0.500000in}}{\pgfqpoint{1.764706in}{1.700000in}}%
\pgfusepath{clip}%
\pgfsetbuttcap%
\pgfsetroundjoin%
\definecolor{currentfill}{rgb}{0.979124,0.903132,0.839793}%
\pgfsetfillcolor{currentfill}%
\pgfsetlinewidth{0.311001pt}%
\definecolor{currentstroke}{rgb}{1.000000,1.000000,1.000000}%
\pgfsetstrokecolor{currentstroke}%
\pgfsetdash{}{0pt}%
\pgfpathmoveto{\pgfqpoint{6.318430in}{1.188731in}}%
\pgfpathcurveto{\pgfqpoint{6.325563in}{1.188731in}}{\pgfqpoint{6.332404in}{1.191565in}}{\pgfqpoint{6.337448in}{1.196609in}}%
\pgfpathcurveto{\pgfqpoint{6.342492in}{1.201652in}}{\pgfqpoint{6.345326in}{1.208494in}}{\pgfqpoint{6.345326in}{1.215627in}}%
\pgfpathcurveto{\pgfqpoint{6.345326in}{1.222760in}}{\pgfqpoint{6.342492in}{1.229601in}}{\pgfqpoint{6.337448in}{1.234645in}}%
\pgfpathcurveto{\pgfqpoint{6.332404in}{1.239689in}}{\pgfqpoint{6.325563in}{1.242522in}}{\pgfqpoint{6.318430in}{1.242522in}}%
\pgfpathcurveto{\pgfqpoint{6.311297in}{1.242522in}}{\pgfqpoint{6.304455in}{1.239689in}}{\pgfqpoint{6.299412in}{1.234645in}}%
\pgfpathcurveto{\pgfqpoint{6.294368in}{1.229601in}}{\pgfqpoint{6.291534in}{1.222760in}}{\pgfqpoint{6.291534in}{1.215627in}}%
\pgfpathcurveto{\pgfqpoint{6.291534in}{1.208494in}}{\pgfqpoint{6.294368in}{1.201652in}}{\pgfqpoint{6.299412in}{1.196609in}}%
\pgfpathcurveto{\pgfqpoint{6.304455in}{1.191565in}}{\pgfqpoint{6.311297in}{1.188731in}}{\pgfqpoint{6.318430in}{1.188731in}}%
\pgfpathclose%
\pgfusepath{stroke,fill}%
\end{pgfscope}%
\begin{pgfscope}%
\pgfpathrectangle{\pgfqpoint{4.985294in}{0.500000in}}{\pgfqpoint{1.764706in}{1.700000in}}%
\pgfusepath{clip}%
\pgfsetbuttcap%
\pgfsetroundjoin%
\definecolor{currentfill}{rgb}{0.852817,0.156578,0.279098}%
\pgfsetfillcolor{currentfill}%
\pgfsetlinewidth{0.311001pt}%
\definecolor{currentstroke}{rgb}{1.000000,1.000000,1.000000}%
\pgfsetstrokecolor{currentstroke}%
\pgfsetdash{}{0pt}%
\pgfpathmoveto{\pgfqpoint{5.805465in}{1.712792in}}%
\pgfpathcurveto{\pgfqpoint{5.812598in}{1.712792in}}{\pgfqpoint{5.819440in}{1.715626in}}{\pgfqpoint{5.824483in}{1.720670in}}%
\pgfpathcurveto{\pgfqpoint{5.829527in}{1.725714in}}{\pgfqpoint{5.832361in}{1.732555in}}{\pgfqpoint{5.832361in}{1.739688in}}%
\pgfpathcurveto{\pgfqpoint{5.832361in}{1.746821in}}{\pgfqpoint{5.829527in}{1.753663in}}{\pgfqpoint{5.824483in}{1.758706in}}%
\pgfpathcurveto{\pgfqpoint{5.819440in}{1.763750in}}{\pgfqpoint{5.812598in}{1.766584in}}{\pgfqpoint{5.805465in}{1.766584in}}%
\pgfpathcurveto{\pgfqpoint{5.798332in}{1.766584in}}{\pgfqpoint{5.791491in}{1.763750in}}{\pgfqpoint{5.786447in}{1.758706in}}%
\pgfpathcurveto{\pgfqpoint{5.781403in}{1.753663in}}{\pgfqpoint{5.778570in}{1.746821in}}{\pgfqpoint{5.778570in}{1.739688in}}%
\pgfpathcurveto{\pgfqpoint{5.778570in}{1.732555in}}{\pgfqpoint{5.781403in}{1.725714in}}{\pgfqpoint{5.786447in}{1.720670in}}%
\pgfpathcurveto{\pgfqpoint{5.791491in}{1.715626in}}{\pgfqpoint{5.798332in}{1.712792in}}{\pgfqpoint{5.805465in}{1.712792in}}%
\pgfpathclose%
\pgfusepath{stroke,fill}%
\end{pgfscope}%
\begin{pgfscope}%
\pgfpathrectangle{\pgfqpoint{4.985294in}{0.500000in}}{\pgfqpoint{1.764706in}{1.700000in}}%
\pgfusepath{clip}%
\pgfsetbuttcap%
\pgfsetroundjoin%
\definecolor{currentfill}{rgb}{0.945204,0.390623,0.270949}%
\pgfsetfillcolor{currentfill}%
\pgfsetlinewidth{0.311001pt}%
\definecolor{currentstroke}{rgb}{1.000000,1.000000,1.000000}%
\pgfsetstrokecolor{currentstroke}%
\pgfsetdash{}{0pt}%
\pgfpathmoveto{\pgfqpoint{5.708017in}{1.732652in}}%
\pgfpathcurveto{\pgfqpoint{5.715150in}{1.732652in}}{\pgfqpoint{5.721991in}{1.735486in}}{\pgfqpoint{5.727035in}{1.740530in}}%
\pgfpathcurveto{\pgfqpoint{5.732079in}{1.745574in}}{\pgfqpoint{5.734913in}{1.752415in}}{\pgfqpoint{5.734913in}{1.759548in}}%
\pgfpathcurveto{\pgfqpoint{5.734913in}{1.766681in}}{\pgfqpoint{5.732079in}{1.773523in}}{\pgfqpoint{5.727035in}{1.778566in}}%
\pgfpathcurveto{\pgfqpoint{5.721991in}{1.783610in}}{\pgfqpoint{5.715150in}{1.786444in}}{\pgfqpoint{5.708017in}{1.786444in}}%
\pgfpathcurveto{\pgfqpoint{5.700884in}{1.786444in}}{\pgfqpoint{5.694042in}{1.783610in}}{\pgfqpoint{5.688999in}{1.778566in}}%
\pgfpathcurveto{\pgfqpoint{5.683955in}{1.773523in}}{\pgfqpoint{5.681121in}{1.766681in}}{\pgfqpoint{5.681121in}{1.759548in}}%
\pgfpathcurveto{\pgfqpoint{5.681121in}{1.752415in}}{\pgfqpoint{5.683955in}{1.745574in}}{\pgfqpoint{5.688999in}{1.740530in}}%
\pgfpathcurveto{\pgfqpoint{5.694042in}{1.735486in}}{\pgfqpoint{5.700884in}{1.732652in}}{\pgfqpoint{5.708017in}{1.732652in}}%
\pgfpathclose%
\pgfusepath{stroke,fill}%
\end{pgfscope}%
\begin{pgfscope}%
\pgfpathrectangle{\pgfqpoint{4.985294in}{0.500000in}}{\pgfqpoint{1.764706in}{1.700000in}}%
\pgfusepath{clip}%
\pgfsetbuttcap%
\pgfsetroundjoin%
\definecolor{currentfill}{rgb}{0.965753,0.732351,0.592427}%
\pgfsetfillcolor{currentfill}%
\pgfsetlinewidth{0.311001pt}%
\definecolor{currentstroke}{rgb}{1.000000,1.000000,1.000000}%
\pgfsetstrokecolor{currentstroke}%
\pgfsetdash{}{0pt}%
\pgfpathmoveto{\pgfqpoint{6.156634in}{0.931815in}}%
\pgfpathcurveto{\pgfqpoint{6.163767in}{0.931815in}}{\pgfqpoint{6.170608in}{0.934649in}}{\pgfqpoint{6.175652in}{0.939692in}}%
\pgfpathcurveto{\pgfqpoint{6.180696in}{0.944736in}}{\pgfqpoint{6.183530in}{0.951578in}}{\pgfqpoint{6.183530in}{0.958711in}}%
\pgfpathcurveto{\pgfqpoint{6.183530in}{0.965843in}}{\pgfqpoint{6.180696in}{0.972685in}}{\pgfqpoint{6.175652in}{0.977729in}}%
\pgfpathcurveto{\pgfqpoint{6.170608in}{0.982772in}}{\pgfqpoint{6.163767in}{0.985606in}}{\pgfqpoint{6.156634in}{0.985606in}}%
\pgfpathcurveto{\pgfqpoint{6.149501in}{0.985606in}}{\pgfqpoint{6.142659in}{0.982772in}}{\pgfqpoint{6.137616in}{0.977729in}}%
\pgfpathcurveto{\pgfqpoint{6.132572in}{0.972685in}}{\pgfqpoint{6.129738in}{0.965843in}}{\pgfqpoint{6.129738in}{0.958711in}}%
\pgfpathcurveto{\pgfqpoint{6.129738in}{0.951578in}}{\pgfqpoint{6.132572in}{0.944736in}}{\pgfqpoint{6.137616in}{0.939692in}}%
\pgfpathcurveto{\pgfqpoint{6.142659in}{0.934649in}}{\pgfqpoint{6.149501in}{0.931815in}}{\pgfqpoint{6.156634in}{0.931815in}}%
\pgfpathclose%
\pgfusepath{stroke,fill}%
\end{pgfscope}%
\begin{pgfscope}%
\pgfpathrectangle{\pgfqpoint{4.985294in}{0.500000in}}{\pgfqpoint{1.764706in}{1.700000in}}%
\pgfusepath{clip}%
\pgfsetbuttcap%
\pgfsetroundjoin%
\definecolor{currentfill}{rgb}{0.975644,0.874038,0.797253}%
\pgfsetfillcolor{currentfill}%
\pgfsetlinewidth{0.311001pt}%
\definecolor{currentstroke}{rgb}{1.000000,1.000000,1.000000}%
\pgfsetstrokecolor{currentstroke}%
\pgfsetdash{}{0pt}%
\pgfpathmoveto{\pgfqpoint{5.375981in}{1.322316in}}%
\pgfpathcurveto{\pgfqpoint{5.383114in}{1.322316in}}{\pgfqpoint{5.389955in}{1.325150in}}{\pgfqpoint{5.394999in}{1.330194in}}%
\pgfpathcurveto{\pgfqpoint{5.400043in}{1.335238in}}{\pgfqpoint{5.402877in}{1.342079in}}{\pgfqpoint{5.402877in}{1.349212in}}%
\pgfpathcurveto{\pgfqpoint{5.402877in}{1.356345in}}{\pgfqpoint{5.400043in}{1.363186in}}{\pgfqpoint{5.394999in}{1.368230in}}%
\pgfpathcurveto{\pgfqpoint{5.389955in}{1.373274in}}{\pgfqpoint{5.383114in}{1.376108in}}{\pgfqpoint{5.375981in}{1.376108in}}%
\pgfpathcurveto{\pgfqpoint{5.368848in}{1.376108in}}{\pgfqpoint{5.362007in}{1.373274in}}{\pgfqpoint{5.356963in}{1.368230in}}%
\pgfpathcurveto{\pgfqpoint{5.351919in}{1.363186in}}{\pgfqpoint{5.349085in}{1.356345in}}{\pgfqpoint{5.349085in}{1.349212in}}%
\pgfpathcurveto{\pgfqpoint{5.349085in}{1.342079in}}{\pgfqpoint{5.351919in}{1.335238in}}{\pgfqpoint{5.356963in}{1.330194in}}%
\pgfpathcurveto{\pgfqpoint{5.362007in}{1.325150in}}{\pgfqpoint{5.368848in}{1.322316in}}{\pgfqpoint{5.375981in}{1.322316in}}%
\pgfpathclose%
\pgfusepath{stroke,fill}%
\end{pgfscope}%
\begin{pgfscope}%
\pgfpathrectangle{\pgfqpoint{4.985294in}{0.500000in}}{\pgfqpoint{1.764706in}{1.700000in}}%
\pgfusepath{clip}%
\pgfsetbuttcap%
\pgfsetroundjoin%
\definecolor{currentfill}{rgb}{0.891169,0.211218,0.255359}%
\pgfsetfillcolor{currentfill}%
\pgfsetlinewidth{0.311001pt}%
\definecolor{currentstroke}{rgb}{1.000000,1.000000,1.000000}%
\pgfsetstrokecolor{currentstroke}%
\pgfsetdash{}{0pt}%
\pgfpathmoveto{\pgfqpoint{5.721898in}{0.894902in}}%
\pgfpathcurveto{\pgfqpoint{5.729031in}{0.894902in}}{\pgfqpoint{5.735872in}{0.897736in}}{\pgfqpoint{5.740916in}{0.902779in}}%
\pgfpathcurveto{\pgfqpoint{5.745960in}{0.907823in}}{\pgfqpoint{5.748794in}{0.914665in}}{\pgfqpoint{5.748794in}{0.921797in}}%
\pgfpathcurveto{\pgfqpoint{5.748794in}{0.928930in}}{\pgfqpoint{5.745960in}{0.935772in}}{\pgfqpoint{5.740916in}{0.940816in}}%
\pgfpathcurveto{\pgfqpoint{5.735872in}{0.945859in}}{\pgfqpoint{5.729031in}{0.948693in}}{\pgfqpoint{5.721898in}{0.948693in}}%
\pgfpathcurveto{\pgfqpoint{5.714765in}{0.948693in}}{\pgfqpoint{5.707924in}{0.945859in}}{\pgfqpoint{5.702880in}{0.940816in}}%
\pgfpathcurveto{\pgfqpoint{5.697836in}{0.935772in}}{\pgfqpoint{5.695002in}{0.928930in}}{\pgfqpoint{5.695002in}{0.921797in}}%
\pgfpathcurveto{\pgfqpoint{5.695002in}{0.914665in}}{\pgfqpoint{5.697836in}{0.907823in}}{\pgfqpoint{5.702880in}{0.902779in}}%
\pgfpathcurveto{\pgfqpoint{5.707924in}{0.897736in}}{\pgfqpoint{5.714765in}{0.894902in}}{\pgfqpoint{5.721898in}{0.894902in}}%
\pgfpathclose%
\pgfusepath{stroke,fill}%
\end{pgfscope}%
\begin{pgfscope}%
\pgfpathrectangle{\pgfqpoint{4.985294in}{0.500000in}}{\pgfqpoint{1.764706in}{1.700000in}}%
\pgfusepath{clip}%
\pgfsetbuttcap%
\pgfsetroundjoin%
\definecolor{currentfill}{rgb}{0.978376,0.897317,0.831308}%
\pgfsetfillcolor{currentfill}%
\pgfsetlinewidth{0.311001pt}%
\definecolor{currentstroke}{rgb}{1.000000,1.000000,1.000000}%
\pgfsetstrokecolor{currentstroke}%
\pgfsetdash{}{0pt}%
\pgfpathmoveto{\pgfqpoint{6.287191in}{1.354211in}}%
\pgfpathcurveto{\pgfqpoint{6.294324in}{1.354211in}}{\pgfqpoint{6.301166in}{1.357044in}}{\pgfqpoint{6.306210in}{1.362088in}}%
\pgfpathcurveto{\pgfqpoint{6.311253in}{1.367132in}}{\pgfqpoint{6.314087in}{1.373973in}}{\pgfqpoint{6.314087in}{1.381106in}}%
\pgfpathcurveto{\pgfqpoint{6.314087in}{1.388239in}}{\pgfqpoint{6.311253in}{1.395081in}}{\pgfqpoint{6.306210in}{1.400124in}}%
\pgfpathcurveto{\pgfqpoint{6.301166in}{1.405168in}}{\pgfqpoint{6.294324in}{1.408002in}}{\pgfqpoint{6.287191in}{1.408002in}}%
\pgfpathcurveto{\pgfqpoint{6.280059in}{1.408002in}}{\pgfqpoint{6.273217in}{1.405168in}}{\pgfqpoint{6.268173in}{1.400124in}}%
\pgfpathcurveto{\pgfqpoint{6.263130in}{1.395081in}}{\pgfqpoint{6.260296in}{1.388239in}}{\pgfqpoint{6.260296in}{1.381106in}}%
\pgfpathcurveto{\pgfqpoint{6.260296in}{1.373973in}}{\pgfqpoint{6.263130in}{1.367132in}}{\pgfqpoint{6.268173in}{1.362088in}}%
\pgfpathcurveto{\pgfqpoint{6.273217in}{1.357044in}}{\pgfqpoint{6.280059in}{1.354211in}}{\pgfqpoint{6.287191in}{1.354211in}}%
\pgfpathclose%
\pgfusepath{stroke,fill}%
\end{pgfscope}%
\begin{pgfscope}%
\pgfpathrectangle{\pgfqpoint{4.985294in}{0.500000in}}{\pgfqpoint{1.764706in}{1.700000in}}%
\pgfusepath{clip}%
\pgfsetbuttcap%
\pgfsetroundjoin%
\definecolor{currentfill}{rgb}{0.972726,0.844889,0.754401}%
\pgfsetfillcolor{currentfill}%
\pgfsetlinewidth{0.311001pt}%
\definecolor{currentstroke}{rgb}{1.000000,1.000000,1.000000}%
\pgfsetstrokecolor{currentstroke}%
\pgfsetdash{}{0pt}%
\pgfpathmoveto{\pgfqpoint{5.405301in}{1.119483in}}%
\pgfpathcurveto{\pgfqpoint{5.412434in}{1.119483in}}{\pgfqpoint{5.419276in}{1.122317in}}{\pgfqpoint{5.424319in}{1.127360in}}%
\pgfpathcurveto{\pgfqpoint{5.429363in}{1.132404in}}{\pgfqpoint{5.432197in}{1.139246in}}{\pgfqpoint{5.432197in}{1.146378in}}%
\pgfpathcurveto{\pgfqpoint{5.432197in}{1.153511in}}{\pgfqpoint{5.429363in}{1.160353in}}{\pgfqpoint{5.424319in}{1.165397in}}%
\pgfpathcurveto{\pgfqpoint{5.419276in}{1.170440in}}{\pgfqpoint{5.412434in}{1.173274in}}{\pgfqpoint{5.405301in}{1.173274in}}%
\pgfpathcurveto{\pgfqpoint{5.398168in}{1.173274in}}{\pgfqpoint{5.391327in}{1.170440in}}{\pgfqpoint{5.386283in}{1.165397in}}%
\pgfpathcurveto{\pgfqpoint{5.381239in}{1.160353in}}{\pgfqpoint{5.378406in}{1.153511in}}{\pgfqpoint{5.378406in}{1.146378in}}%
\pgfpathcurveto{\pgfqpoint{5.378406in}{1.139246in}}{\pgfqpoint{5.381239in}{1.132404in}}{\pgfqpoint{5.386283in}{1.127360in}}%
\pgfpathcurveto{\pgfqpoint{5.391327in}{1.122317in}}{\pgfqpoint{5.398168in}{1.119483in}}{\pgfqpoint{5.405301in}{1.119483in}}%
\pgfpathclose%
\pgfusepath{stroke,fill}%
\end{pgfscope}%
\begin{pgfscope}%
\pgfpathrectangle{\pgfqpoint{4.985294in}{0.500000in}}{\pgfqpoint{1.764706in}{1.700000in}}%
\pgfusepath{clip}%
\pgfsetbuttcap%
\pgfsetroundjoin%
\definecolor{currentfill}{rgb}{0.964920,0.695342,0.545192}%
\pgfsetfillcolor{currentfill}%
\pgfsetlinewidth{0.311001pt}%
\definecolor{currentstroke}{rgb}{1.000000,1.000000,1.000000}%
\pgfsetstrokecolor{currentstroke}%
\pgfsetdash{}{0pt}%
\pgfpathmoveto{\pgfqpoint{6.138532in}{0.925052in}}%
\pgfpathcurveto{\pgfqpoint{6.145665in}{0.925052in}}{\pgfqpoint{6.152507in}{0.927886in}}{\pgfqpoint{6.157551in}{0.932930in}}%
\pgfpathcurveto{\pgfqpoint{6.162594in}{0.937974in}}{\pgfqpoint{6.165428in}{0.944815in}}{\pgfqpoint{6.165428in}{0.951948in}}%
\pgfpathcurveto{\pgfqpoint{6.165428in}{0.959081in}}{\pgfqpoint{6.162594in}{0.965923in}}{\pgfqpoint{6.157551in}{0.970966in}}%
\pgfpathcurveto{\pgfqpoint{6.152507in}{0.976010in}}{\pgfqpoint{6.145665in}{0.978844in}}{\pgfqpoint{6.138532in}{0.978844in}}%
\pgfpathcurveto{\pgfqpoint{6.131400in}{0.978844in}}{\pgfqpoint{6.124558in}{0.976010in}}{\pgfqpoint{6.119514in}{0.970966in}}%
\pgfpathcurveto{\pgfqpoint{6.114471in}{0.965923in}}{\pgfqpoint{6.111637in}{0.959081in}}{\pgfqpoint{6.111637in}{0.951948in}}%
\pgfpathcurveto{\pgfqpoint{6.111637in}{0.944815in}}{\pgfqpoint{6.114471in}{0.937974in}}{\pgfqpoint{6.119514in}{0.932930in}}%
\pgfpathcurveto{\pgfqpoint{6.124558in}{0.927886in}}{\pgfqpoint{6.131400in}{0.925052in}}{\pgfqpoint{6.138532in}{0.925052in}}%
\pgfpathclose%
\pgfusepath{stroke,fill}%
\end{pgfscope}%
\begin{pgfscope}%
\pgfpathrectangle{\pgfqpoint{4.985294in}{0.500000in}}{\pgfqpoint{1.764706in}{1.700000in}}%
\pgfusepath{clip}%
\pgfsetbuttcap%
\pgfsetroundjoin%
\definecolor{currentfill}{rgb}{0.965753,0.732351,0.592427}%
\pgfsetfillcolor{currentfill}%
\pgfsetlinewidth{0.311001pt}%
\definecolor{currentstroke}{rgb}{1.000000,1.000000,1.000000}%
\pgfsetstrokecolor{currentstroke}%
\pgfsetdash{}{0pt}%
\pgfpathmoveto{\pgfqpoint{5.603311in}{0.960925in}}%
\pgfpathcurveto{\pgfqpoint{5.610443in}{0.960925in}}{\pgfqpoint{5.617285in}{0.963759in}}{\pgfqpoint{5.622329in}{0.968802in}}%
\pgfpathcurveto{\pgfqpoint{5.627372in}{0.973846in}}{\pgfqpoint{5.630206in}{0.980688in}}{\pgfqpoint{5.630206in}{0.987820in}}%
\pgfpathcurveto{\pgfqpoint{5.630206in}{0.994953in}}{\pgfqpoint{5.627372in}{1.001795in}}{\pgfqpoint{5.622329in}{1.006839in}}%
\pgfpathcurveto{\pgfqpoint{5.617285in}{1.011882in}}{\pgfqpoint{5.610443in}{1.014716in}}{\pgfqpoint{5.603311in}{1.014716in}}%
\pgfpathcurveto{\pgfqpoint{5.596178in}{1.014716in}}{\pgfqpoint{5.589336in}{1.011882in}}{\pgfqpoint{5.584293in}{1.006839in}}%
\pgfpathcurveto{\pgfqpoint{5.579249in}{1.001795in}}{\pgfqpoint{5.576415in}{0.994953in}}{\pgfqpoint{5.576415in}{0.987820in}}%
\pgfpathcurveto{\pgfqpoint{5.576415in}{0.980688in}}{\pgfqpoint{5.579249in}{0.973846in}}{\pgfqpoint{5.584293in}{0.968802in}}%
\pgfpathcurveto{\pgfqpoint{5.589336in}{0.963759in}}{\pgfqpoint{5.596178in}{0.960925in}}{\pgfqpoint{5.603311in}{0.960925in}}%
\pgfpathclose%
\pgfusepath{stroke,fill}%
\end{pgfscope}%
\begin{pgfscope}%
\pgfpathrectangle{\pgfqpoint{4.985294in}{0.500000in}}{\pgfqpoint{1.764706in}{1.700000in}}%
\pgfusepath{clip}%
\pgfsetbuttcap%
\pgfsetroundjoin%
\definecolor{currentfill}{rgb}{0.730358,0.086862,0.337485}%
\pgfsetfillcolor{currentfill}%
\pgfsetlinewidth{0.311001pt}%
\definecolor{currentstroke}{rgb}{1.000000,1.000000,1.000000}%
\pgfsetstrokecolor{currentstroke}%
\pgfsetdash{}{0pt}%
\pgfpathmoveto{\pgfqpoint{5.253089in}{1.079009in}}%
\pgfpathcurveto{\pgfqpoint{5.260222in}{1.079009in}}{\pgfqpoint{5.267063in}{1.081842in}}{\pgfqpoint{5.272107in}{1.086886in}}%
\pgfpathcurveto{\pgfqpoint{5.277151in}{1.091930in}}{\pgfqpoint{5.279985in}{1.098771in}}{\pgfqpoint{5.279985in}{1.105904in}}%
\pgfpathcurveto{\pgfqpoint{5.279985in}{1.113037in}}{\pgfqpoint{5.277151in}{1.119879in}}{\pgfqpoint{5.272107in}{1.124922in}}%
\pgfpathcurveto{\pgfqpoint{5.267063in}{1.129966in}}{\pgfqpoint{5.260222in}{1.132800in}}{\pgfqpoint{5.253089in}{1.132800in}}%
\pgfpathcurveto{\pgfqpoint{5.245956in}{1.132800in}}{\pgfqpoint{5.239114in}{1.129966in}}{\pgfqpoint{5.234071in}{1.124922in}}%
\pgfpathcurveto{\pgfqpoint{5.229027in}{1.119879in}}{\pgfqpoint{5.226193in}{1.113037in}}{\pgfqpoint{5.226193in}{1.105904in}}%
\pgfpathcurveto{\pgfqpoint{5.226193in}{1.098771in}}{\pgfqpoint{5.229027in}{1.091930in}}{\pgfqpoint{5.234071in}{1.086886in}}%
\pgfpathcurveto{\pgfqpoint{5.239114in}{1.081842in}}{\pgfqpoint{5.245956in}{1.079009in}}{\pgfqpoint{5.253089in}{1.079009in}}%
\pgfpathclose%
\pgfusepath{stroke,fill}%
\end{pgfscope}%
\begin{pgfscope}%
\pgfpathrectangle{\pgfqpoint{4.985294in}{0.500000in}}{\pgfqpoint{1.764706in}{1.700000in}}%
\pgfusepath{clip}%
\pgfsetbuttcap%
\pgfsetroundjoin%
\definecolor{currentfill}{rgb}{0.972726,0.844889,0.754401}%
\pgfsetfillcolor{currentfill}%
\pgfsetlinewidth{0.311001pt}%
\definecolor{currentstroke}{rgb}{1.000000,1.000000,1.000000}%
\pgfsetstrokecolor{currentstroke}%
\pgfsetdash{}{0pt}%
\pgfpathmoveto{\pgfqpoint{5.471744in}{1.017878in}}%
\pgfpathcurveto{\pgfqpoint{5.478877in}{1.017878in}}{\pgfqpoint{5.485719in}{1.020712in}}{\pgfqpoint{5.490763in}{1.025756in}}%
\pgfpathcurveto{\pgfqpoint{5.495806in}{1.030800in}}{\pgfqpoint{5.498640in}{1.037641in}}{\pgfqpoint{5.498640in}{1.044774in}}%
\pgfpathcurveto{\pgfqpoint{5.498640in}{1.051907in}}{\pgfqpoint{5.495806in}{1.058749in}}{\pgfqpoint{5.490763in}{1.063792in}}%
\pgfpathcurveto{\pgfqpoint{5.485719in}{1.068836in}}{\pgfqpoint{5.478877in}{1.071670in}}{\pgfqpoint{5.471744in}{1.071670in}}%
\pgfpathcurveto{\pgfqpoint{5.464612in}{1.071670in}}{\pgfqpoint{5.457770in}{1.068836in}}{\pgfqpoint{5.452726in}{1.063792in}}%
\pgfpathcurveto{\pgfqpoint{5.447683in}{1.058749in}}{\pgfqpoint{5.444849in}{1.051907in}}{\pgfqpoint{5.444849in}{1.044774in}}%
\pgfpathcurveto{\pgfqpoint{5.444849in}{1.037641in}}{\pgfqpoint{5.447683in}{1.030800in}}{\pgfqpoint{5.452726in}{1.025756in}}%
\pgfpathcurveto{\pgfqpoint{5.457770in}{1.020712in}}{\pgfqpoint{5.464612in}{1.017878in}}{\pgfqpoint{5.471744in}{1.017878in}}%
\pgfpathclose%
\pgfusepath{stroke,fill}%
\end{pgfscope}%
\begin{pgfscope}%
\pgfpathrectangle{\pgfqpoint{4.985294in}{0.500000in}}{\pgfqpoint{1.764706in}{1.700000in}}%
\pgfusepath{clip}%
\pgfsetbuttcap%
\pgfsetroundjoin%
\definecolor{currentfill}{rgb}{0.980678,0.914765,0.856766}%
\pgfsetfillcolor{currentfill}%
\pgfsetlinewidth{0.311001pt}%
\definecolor{currentstroke}{rgb}{1.000000,1.000000,1.000000}%
\pgfsetstrokecolor{currentstroke}%
\pgfsetdash{}{0pt}%
\pgfpathmoveto{\pgfqpoint{5.418257in}{1.271782in}}%
\pgfpathcurveto{\pgfqpoint{5.425390in}{1.271782in}}{\pgfqpoint{5.432232in}{1.274616in}}{\pgfqpoint{5.437276in}{1.279659in}}%
\pgfpathcurveto{\pgfqpoint{5.442319in}{1.284703in}}{\pgfqpoint{5.445153in}{1.291545in}}{\pgfqpoint{5.445153in}{1.298677in}}%
\pgfpathcurveto{\pgfqpoint{5.445153in}{1.305810in}}{\pgfqpoint{5.442319in}{1.312652in}}{\pgfqpoint{5.437276in}{1.317696in}}%
\pgfpathcurveto{\pgfqpoint{5.432232in}{1.322739in}}{\pgfqpoint{5.425390in}{1.325573in}}{\pgfqpoint{5.418257in}{1.325573in}}%
\pgfpathcurveto{\pgfqpoint{5.411125in}{1.325573in}}{\pgfqpoint{5.404283in}{1.322739in}}{\pgfqpoint{5.399239in}{1.317696in}}%
\pgfpathcurveto{\pgfqpoint{5.394196in}{1.312652in}}{\pgfqpoint{5.391362in}{1.305810in}}{\pgfqpoint{5.391362in}{1.298677in}}%
\pgfpathcurveto{\pgfqpoint{5.391362in}{1.291545in}}{\pgfqpoint{5.394196in}{1.284703in}}{\pgfqpoint{5.399239in}{1.279659in}}%
\pgfpathcurveto{\pgfqpoint{5.404283in}{1.274616in}}{\pgfqpoint{5.411125in}{1.271782in}}{\pgfqpoint{5.418257in}{1.271782in}}%
\pgfpathclose%
\pgfusepath{stroke,fill}%
\end{pgfscope}%
\begin{pgfscope}%
\pgfpathrectangle{\pgfqpoint{4.985294in}{0.500000in}}{\pgfqpoint{1.764706in}{1.700000in}}%
\pgfusepath{clip}%
\pgfsetbuttcap%
\pgfsetroundjoin%
\definecolor{currentfill}{rgb}{0.980678,0.914765,0.856766}%
\pgfsetfillcolor{currentfill}%
\pgfsetlinewidth{0.311001pt}%
\definecolor{currentstroke}{rgb}{1.000000,1.000000,1.000000}%
\pgfsetstrokecolor{currentstroke}%
\pgfsetdash{}{0pt}%
\pgfpathmoveto{\pgfqpoint{6.325218in}{1.261826in}}%
\pgfpathcurveto{\pgfqpoint{6.332351in}{1.261826in}}{\pgfqpoint{6.339192in}{1.264660in}}{\pgfqpoint{6.344236in}{1.269703in}}%
\pgfpathcurveto{\pgfqpoint{6.349280in}{1.274747in}}{\pgfqpoint{6.352114in}{1.281589in}}{\pgfqpoint{6.352114in}{1.288721in}}%
\pgfpathcurveto{\pgfqpoint{6.352114in}{1.295854in}}{\pgfqpoint{6.349280in}{1.302696in}}{\pgfqpoint{6.344236in}{1.307740in}}%
\pgfpathcurveto{\pgfqpoint{6.339192in}{1.312783in}}{\pgfqpoint{6.332351in}{1.315617in}}{\pgfqpoint{6.325218in}{1.315617in}}%
\pgfpathcurveto{\pgfqpoint{6.318085in}{1.315617in}}{\pgfqpoint{6.311243in}{1.312783in}}{\pgfqpoint{6.306200in}{1.307740in}}%
\pgfpathcurveto{\pgfqpoint{6.301156in}{1.302696in}}{\pgfqpoint{6.298322in}{1.295854in}}{\pgfqpoint{6.298322in}{1.288721in}}%
\pgfpathcurveto{\pgfqpoint{6.298322in}{1.281589in}}{\pgfqpoint{6.301156in}{1.274747in}}{\pgfqpoint{6.306200in}{1.269703in}}%
\pgfpathcurveto{\pgfqpoint{6.311243in}{1.264660in}}{\pgfqpoint{6.318085in}{1.261826in}}{\pgfqpoint{6.325218in}{1.261826in}}%
\pgfpathclose%
\pgfusepath{stroke,fill}%
\end{pgfscope}%
\begin{pgfscope}%
\pgfpathrectangle{\pgfqpoint{4.985294in}{0.500000in}}{\pgfqpoint{1.764706in}{1.700000in}}%
\pgfusepath{clip}%
\pgfsetbuttcap%
\pgfsetroundjoin%
\definecolor{currentfill}{rgb}{0.978376,0.897317,0.831308}%
\pgfsetfillcolor{currentfill}%
\pgfsetlinewidth{0.311001pt}%
\definecolor{currentstroke}{rgb}{1.000000,1.000000,1.000000}%
\pgfsetstrokecolor{currentstroke}%
\pgfsetdash{}{0pt}%
\pgfpathmoveto{\pgfqpoint{5.446431in}{1.304952in}}%
\pgfpathcurveto{\pgfqpoint{5.453564in}{1.304952in}}{\pgfqpoint{5.460406in}{1.307786in}}{\pgfqpoint{5.465449in}{1.312829in}}%
\pgfpathcurveto{\pgfqpoint{5.470493in}{1.317873in}}{\pgfqpoint{5.473327in}{1.324715in}}{\pgfqpoint{5.473327in}{1.331847in}}%
\pgfpathcurveto{\pgfqpoint{5.473327in}{1.338980in}}{\pgfqpoint{5.470493in}{1.345822in}}{\pgfqpoint{5.465449in}{1.350866in}}%
\pgfpathcurveto{\pgfqpoint{5.460406in}{1.355909in}}{\pgfqpoint{5.453564in}{1.358743in}}{\pgfqpoint{5.446431in}{1.358743in}}%
\pgfpathcurveto{\pgfqpoint{5.439298in}{1.358743in}}{\pgfqpoint{5.432457in}{1.355909in}}{\pgfqpoint{5.427413in}{1.350866in}}%
\pgfpathcurveto{\pgfqpoint{5.422369in}{1.345822in}}{\pgfqpoint{5.419536in}{1.338980in}}{\pgfqpoint{5.419536in}{1.331847in}}%
\pgfpathcurveto{\pgfqpoint{5.419536in}{1.324715in}}{\pgfqpoint{5.422369in}{1.317873in}}{\pgfqpoint{5.427413in}{1.312829in}}%
\pgfpathcurveto{\pgfqpoint{5.432457in}{1.307786in}}{\pgfqpoint{5.439298in}{1.304952in}}{\pgfqpoint{5.446431in}{1.304952in}}%
\pgfpathclose%
\pgfusepath{stroke,fill}%
\end{pgfscope}%
\begin{pgfscope}%
\pgfpathrectangle{\pgfqpoint{4.985294in}{0.500000in}}{\pgfqpoint{1.764706in}{1.700000in}}%
\pgfusepath{clip}%
\pgfsetbuttcap%
\pgfsetroundjoin%
\definecolor{currentfill}{rgb}{0.966120,0.744512,0.608720}%
\pgfsetfillcolor{currentfill}%
\pgfsetlinewidth{0.311001pt}%
\definecolor{currentstroke}{rgb}{1.000000,1.000000,1.000000}%
\pgfsetstrokecolor{currentstroke}%
\pgfsetdash{}{0pt}%
\pgfpathmoveto{\pgfqpoint{6.393971in}{1.205107in}}%
\pgfpathcurveto{\pgfqpoint{6.401104in}{1.205107in}}{\pgfqpoint{6.407946in}{1.207941in}}{\pgfqpoint{6.412990in}{1.212984in}}%
\pgfpathcurveto{\pgfqpoint{6.418033in}{1.218028in}}{\pgfqpoint{6.420867in}{1.224870in}}{\pgfqpoint{6.420867in}{1.232002in}}%
\pgfpathcurveto{\pgfqpoint{6.420867in}{1.239135in}}{\pgfqpoint{6.418033in}{1.245977in}}{\pgfqpoint{6.412990in}{1.251021in}}%
\pgfpathcurveto{\pgfqpoint{6.407946in}{1.256064in}}{\pgfqpoint{6.401104in}{1.258898in}}{\pgfqpoint{6.393971in}{1.258898in}}%
\pgfpathcurveto{\pgfqpoint{6.386839in}{1.258898in}}{\pgfqpoint{6.379997in}{1.256064in}}{\pgfqpoint{6.374953in}{1.251021in}}%
\pgfpathcurveto{\pgfqpoint{6.369910in}{1.245977in}}{\pgfqpoint{6.367076in}{1.239135in}}{\pgfqpoint{6.367076in}{1.232002in}}%
\pgfpathcurveto{\pgfqpoint{6.367076in}{1.224870in}}{\pgfqpoint{6.369910in}{1.218028in}}{\pgfqpoint{6.374953in}{1.212984in}}%
\pgfpathcurveto{\pgfqpoint{6.379997in}{1.207941in}}{\pgfqpoint{6.386839in}{1.205107in}}{\pgfqpoint{6.393971in}{1.205107in}}%
\pgfpathclose%
\pgfusepath{stroke,fill}%
\end{pgfscope}%
\begin{pgfscope}%
\pgfpathrectangle{\pgfqpoint{4.985294in}{0.500000in}}{\pgfqpoint{1.764706in}{1.700000in}}%
\pgfusepath{clip}%
\pgfsetbuttcap%
\pgfsetroundjoin%
\definecolor{currentfill}{rgb}{0.975018,0.868213,0.788710}%
\pgfsetfillcolor{currentfill}%
\pgfsetlinewidth{0.311001pt}%
\definecolor{currentstroke}{rgb}{1.000000,1.000000,1.000000}%
\pgfsetstrokecolor{currentstroke}%
\pgfsetdash{}{0pt}%
\pgfpathmoveto{\pgfqpoint{6.243701in}{1.166462in}}%
\pgfpathcurveto{\pgfqpoint{6.250834in}{1.166462in}}{\pgfqpoint{6.257675in}{1.169296in}}{\pgfqpoint{6.262719in}{1.174339in}}%
\pgfpathcurveto{\pgfqpoint{6.267763in}{1.179383in}}{\pgfqpoint{6.270597in}{1.186225in}}{\pgfqpoint{6.270597in}{1.193357in}}%
\pgfpathcurveto{\pgfqpoint{6.270597in}{1.200490in}}{\pgfqpoint{6.267763in}{1.207332in}}{\pgfqpoint{6.262719in}{1.212376in}}%
\pgfpathcurveto{\pgfqpoint{6.257675in}{1.217419in}}{\pgfqpoint{6.250834in}{1.220253in}}{\pgfqpoint{6.243701in}{1.220253in}}%
\pgfpathcurveto{\pgfqpoint{6.236568in}{1.220253in}}{\pgfqpoint{6.229726in}{1.217419in}}{\pgfqpoint{6.224683in}{1.212376in}}%
\pgfpathcurveto{\pgfqpoint{6.219639in}{1.207332in}}{\pgfqpoint{6.216805in}{1.200490in}}{\pgfqpoint{6.216805in}{1.193357in}}%
\pgfpathcurveto{\pgfqpoint{6.216805in}{1.186225in}}{\pgfqpoint{6.219639in}{1.179383in}}{\pgfqpoint{6.224683in}{1.174339in}}%
\pgfpathcurveto{\pgfqpoint{6.229726in}{1.169296in}}{\pgfqpoint{6.236568in}{1.166462in}}{\pgfqpoint{6.243701in}{1.166462in}}%
\pgfpathclose%
\pgfusepath{stroke,fill}%
\end{pgfscope}%
\begin{pgfscope}%
\pgfpathrectangle{\pgfqpoint{4.985294in}{0.500000in}}{\pgfqpoint{1.764706in}{1.700000in}}%
\pgfusepath{clip}%
\pgfsetbuttcap%
\pgfsetroundjoin%
\definecolor{currentfill}{rgb}{0.961734,0.579886,0.418445}%
\pgfsetfillcolor{currentfill}%
\pgfsetlinewidth{0.311001pt}%
\definecolor{currentstroke}{rgb}{1.000000,1.000000,1.000000}%
\pgfsetstrokecolor{currentstroke}%
\pgfsetdash{}{0pt}%
\pgfpathmoveto{\pgfqpoint{5.616762in}{1.730279in}}%
\pgfpathcurveto{\pgfqpoint{5.623895in}{1.730279in}}{\pgfqpoint{5.630737in}{1.733113in}}{\pgfqpoint{5.635780in}{1.738156in}}%
\pgfpathcurveto{\pgfqpoint{5.640824in}{1.743200in}}{\pgfqpoint{5.643658in}{1.750042in}}{\pgfqpoint{5.643658in}{1.757174in}}%
\pgfpathcurveto{\pgfqpoint{5.643658in}{1.764307in}}{\pgfqpoint{5.640824in}{1.771149in}}{\pgfqpoint{5.635780in}{1.776193in}}%
\pgfpathcurveto{\pgfqpoint{5.630737in}{1.781236in}}{\pgfqpoint{5.623895in}{1.784070in}}{\pgfqpoint{5.616762in}{1.784070in}}%
\pgfpathcurveto{\pgfqpoint{5.609629in}{1.784070in}}{\pgfqpoint{5.602788in}{1.781236in}}{\pgfqpoint{5.597744in}{1.776193in}}%
\pgfpathcurveto{\pgfqpoint{5.592700in}{1.771149in}}{\pgfqpoint{5.589866in}{1.764307in}}{\pgfqpoint{5.589866in}{1.757174in}}%
\pgfpathcurveto{\pgfqpoint{5.589866in}{1.750042in}}{\pgfqpoint{5.592700in}{1.743200in}}{\pgfqpoint{5.597744in}{1.738156in}}%
\pgfpathcurveto{\pgfqpoint{5.602788in}{1.733113in}}{\pgfqpoint{5.609629in}{1.730279in}}{\pgfqpoint{5.616762in}{1.730279in}}%
\pgfpathclose%
\pgfusepath{stroke,fill}%
\end{pgfscope}%
\begin{pgfscope}%
\pgfpathrectangle{\pgfqpoint{4.985294in}{0.500000in}}{\pgfqpoint{1.764706in}{1.700000in}}%
\pgfusepath{clip}%
\pgfsetbuttcap%
\pgfsetroundjoin%
\definecolor{currentfill}{rgb}{0.973271,0.850724,0.762998}%
\pgfsetfillcolor{currentfill}%
\pgfsetlinewidth{0.311001pt}%
\definecolor{currentstroke}{rgb}{1.000000,1.000000,1.000000}%
\pgfsetstrokecolor{currentstroke}%
\pgfsetdash{}{0pt}%
\pgfpathmoveto{\pgfqpoint{5.466789in}{1.593282in}}%
\pgfpathcurveto{\pgfqpoint{5.473922in}{1.593282in}}{\pgfqpoint{5.480764in}{1.596116in}}{\pgfqpoint{5.485807in}{1.601160in}}%
\pgfpathcurveto{\pgfqpoint{5.490851in}{1.606204in}}{\pgfqpoint{5.493685in}{1.613045in}}{\pgfqpoint{5.493685in}{1.620178in}}%
\pgfpathcurveto{\pgfqpoint{5.493685in}{1.627311in}}{\pgfqpoint{5.490851in}{1.634153in}}{\pgfqpoint{5.485807in}{1.639196in}}%
\pgfpathcurveto{\pgfqpoint{5.480764in}{1.644240in}}{\pgfqpoint{5.473922in}{1.647074in}}{\pgfqpoint{5.466789in}{1.647074in}}%
\pgfpathcurveto{\pgfqpoint{5.459656in}{1.647074in}}{\pgfqpoint{5.452815in}{1.644240in}}{\pgfqpoint{5.447771in}{1.639196in}}%
\pgfpathcurveto{\pgfqpoint{5.442727in}{1.634153in}}{\pgfqpoint{5.439894in}{1.627311in}}{\pgfqpoint{5.439894in}{1.620178in}}%
\pgfpathcurveto{\pgfqpoint{5.439894in}{1.613045in}}{\pgfqpoint{5.442727in}{1.606204in}}{\pgfqpoint{5.447771in}{1.601160in}}%
\pgfpathcurveto{\pgfqpoint{5.452815in}{1.596116in}}{\pgfqpoint{5.459656in}{1.593282in}}{\pgfqpoint{5.466789in}{1.593282in}}%
\pgfpathclose%
\pgfusepath{stroke,fill}%
\end{pgfscope}%
\begin{pgfscope}%
\pgfpathrectangle{\pgfqpoint{4.985294in}{0.500000in}}{\pgfqpoint{1.764706in}{1.700000in}}%
\pgfusepath{clip}%
\pgfsetbuttcap%
\pgfsetroundjoin%
\definecolor{currentfill}{rgb}{0.975018,0.868213,0.788710}%
\pgfsetfillcolor{currentfill}%
\pgfsetlinewidth{0.311001pt}%
\definecolor{currentstroke}{rgb}{1.000000,1.000000,1.000000}%
\pgfsetstrokecolor{currentstroke}%
\pgfsetdash{}{0pt}%
\pgfpathmoveto{\pgfqpoint{5.417974in}{1.121779in}}%
\pgfpathcurveto{\pgfqpoint{5.425106in}{1.121779in}}{\pgfqpoint{5.431948in}{1.124613in}}{\pgfqpoint{5.436992in}{1.129656in}}%
\pgfpathcurveto{\pgfqpoint{5.442035in}{1.134700in}}{\pgfqpoint{5.444869in}{1.141542in}}{\pgfqpoint{5.444869in}{1.148674in}}%
\pgfpathcurveto{\pgfqpoint{5.444869in}{1.155807in}}{\pgfqpoint{5.442035in}{1.162649in}}{\pgfqpoint{5.436992in}{1.167692in}}%
\pgfpathcurveto{\pgfqpoint{5.431948in}{1.172736in}}{\pgfqpoint{5.425106in}{1.175570in}}{\pgfqpoint{5.417974in}{1.175570in}}%
\pgfpathcurveto{\pgfqpoint{5.410841in}{1.175570in}}{\pgfqpoint{5.403999in}{1.172736in}}{\pgfqpoint{5.398956in}{1.167692in}}%
\pgfpathcurveto{\pgfqpoint{5.393912in}{1.162649in}}{\pgfqpoint{5.391078in}{1.155807in}}{\pgfqpoint{5.391078in}{1.148674in}}%
\pgfpathcurveto{\pgfqpoint{5.391078in}{1.141542in}}{\pgfqpoint{5.393912in}{1.134700in}}{\pgfqpoint{5.398956in}{1.129656in}}%
\pgfpathcurveto{\pgfqpoint{5.403999in}{1.124613in}}{\pgfqpoint{5.410841in}{1.121779in}}{\pgfqpoint{5.417974in}{1.121779in}}%
\pgfpathclose%
\pgfusepath{stroke,fill}%
\end{pgfscope}%
\begin{pgfscope}%
\pgfpathrectangle{\pgfqpoint{4.985294in}{0.500000in}}{\pgfqpoint{1.764706in}{1.700000in}}%
\pgfusepath{clip}%
\pgfsetbuttcap%
\pgfsetroundjoin%
\definecolor{currentfill}{rgb}{0.976961,0.885681,0.814303}%
\pgfsetfillcolor{currentfill}%
\pgfsetlinewidth{0.311001pt}%
\definecolor{currentstroke}{rgb}{1.000000,1.000000,1.000000}%
\pgfsetstrokecolor{currentstroke}%
\pgfsetdash{}{0pt}%
\pgfpathmoveto{\pgfqpoint{5.396928in}{1.213768in}}%
\pgfpathcurveto{\pgfqpoint{5.404061in}{1.213768in}}{\pgfqpoint{5.410902in}{1.216602in}}{\pgfqpoint{5.415946in}{1.221646in}}%
\pgfpathcurveto{\pgfqpoint{5.420990in}{1.226690in}}{\pgfqpoint{5.423824in}{1.233531in}}{\pgfqpoint{5.423824in}{1.240664in}}%
\pgfpathcurveto{\pgfqpoint{5.423824in}{1.247797in}}{\pgfqpoint{5.420990in}{1.254639in}}{\pgfqpoint{5.415946in}{1.259682in}}%
\pgfpathcurveto{\pgfqpoint{5.410902in}{1.264726in}}{\pgfqpoint{5.404061in}{1.267560in}}{\pgfqpoint{5.396928in}{1.267560in}}%
\pgfpathcurveto{\pgfqpoint{5.389795in}{1.267560in}}{\pgfqpoint{5.382953in}{1.264726in}}{\pgfqpoint{5.377910in}{1.259682in}}%
\pgfpathcurveto{\pgfqpoint{5.372866in}{1.254639in}}{\pgfqpoint{5.370032in}{1.247797in}}{\pgfqpoint{5.370032in}{1.240664in}}%
\pgfpathcurveto{\pgfqpoint{5.370032in}{1.233531in}}{\pgfqpoint{5.372866in}{1.226690in}}{\pgfqpoint{5.377910in}{1.221646in}}%
\pgfpathcurveto{\pgfqpoint{5.382953in}{1.216602in}}{\pgfqpoint{5.389795in}{1.213768in}}{\pgfqpoint{5.396928in}{1.213768in}}%
\pgfpathclose%
\pgfusepath{stroke,fill}%
\end{pgfscope}%
\begin{pgfscope}%
\pgfpathrectangle{\pgfqpoint{4.985294in}{0.500000in}}{\pgfqpoint{1.764706in}{1.700000in}}%
\pgfusepath{clip}%
\pgfsetbuttcap%
\pgfsetroundjoin%
\definecolor{currentfill}{rgb}{0.963559,0.632016,0.472047}%
\pgfsetfillcolor{currentfill}%
\pgfsetlinewidth{0.311001pt}%
\definecolor{currentstroke}{rgb}{1.000000,1.000000,1.000000}%
\pgfsetstrokecolor{currentstroke}%
\pgfsetdash{}{0pt}%
\pgfpathmoveto{\pgfqpoint{5.574236in}{1.744969in}}%
\pgfpathcurveto{\pgfqpoint{5.581369in}{1.744969in}}{\pgfqpoint{5.588210in}{1.747803in}}{\pgfqpoint{5.593254in}{1.752846in}}%
\pgfpathcurveto{\pgfqpoint{5.598298in}{1.757890in}}{\pgfqpoint{5.601132in}{1.764732in}}{\pgfqpoint{5.601132in}{1.771865in}}%
\pgfpathcurveto{\pgfqpoint{5.601132in}{1.778997in}}{\pgfqpoint{5.598298in}{1.785839in}}{\pgfqpoint{5.593254in}{1.790883in}}%
\pgfpathcurveto{\pgfqpoint{5.588210in}{1.795926in}}{\pgfqpoint{5.581369in}{1.798760in}}{\pgfqpoint{5.574236in}{1.798760in}}%
\pgfpathcurveto{\pgfqpoint{5.567103in}{1.798760in}}{\pgfqpoint{5.560261in}{1.795926in}}{\pgfqpoint{5.555218in}{1.790883in}}%
\pgfpathcurveto{\pgfqpoint{5.550174in}{1.785839in}}{\pgfqpoint{5.547340in}{1.778997in}}{\pgfqpoint{5.547340in}{1.771865in}}%
\pgfpathcurveto{\pgfqpoint{5.547340in}{1.764732in}}{\pgfqpoint{5.550174in}{1.757890in}}{\pgfqpoint{5.555218in}{1.752846in}}%
\pgfpathcurveto{\pgfqpoint{5.560261in}{1.747803in}}{\pgfqpoint{5.567103in}{1.744969in}}{\pgfqpoint{5.574236in}{1.744969in}}%
\pgfpathclose%
\pgfusepath{stroke,fill}%
\end{pgfscope}%
\begin{pgfscope}%
\pgfpathrectangle{\pgfqpoint{4.985294in}{0.500000in}}{\pgfqpoint{1.764706in}{1.700000in}}%
\pgfusepath{clip}%
\pgfsetbuttcap%
\pgfsetroundjoin%
\definecolor{currentfill}{rgb}{0.919781,0.275262,0.242460}%
\pgfsetfillcolor{currentfill}%
\pgfsetlinewidth{0.311001pt}%
\definecolor{currentstroke}{rgb}{1.000000,1.000000,1.000000}%
\pgfsetstrokecolor{currentstroke}%
\pgfsetdash{}{0pt}%
\pgfpathmoveto{\pgfqpoint{5.256063in}{1.287323in}}%
\pgfpathcurveto{\pgfqpoint{5.263196in}{1.287323in}}{\pgfqpoint{5.270037in}{1.290157in}}{\pgfqpoint{5.275081in}{1.295201in}}%
\pgfpathcurveto{\pgfqpoint{5.280125in}{1.300244in}}{\pgfqpoint{5.282959in}{1.307086in}}{\pgfqpoint{5.282959in}{1.314219in}}%
\pgfpathcurveto{\pgfqpoint{5.282959in}{1.321352in}}{\pgfqpoint{5.280125in}{1.328193in}}{\pgfqpoint{5.275081in}{1.333237in}}%
\pgfpathcurveto{\pgfqpoint{5.270037in}{1.338281in}}{\pgfqpoint{5.263196in}{1.341114in}}{\pgfqpoint{5.256063in}{1.341114in}}%
\pgfpathcurveto{\pgfqpoint{5.248930in}{1.341114in}}{\pgfqpoint{5.242088in}{1.338281in}}{\pgfqpoint{5.237045in}{1.333237in}}%
\pgfpathcurveto{\pgfqpoint{5.232001in}{1.328193in}}{\pgfqpoint{5.229167in}{1.321352in}}{\pgfqpoint{5.229167in}{1.314219in}}%
\pgfpathcurveto{\pgfqpoint{5.229167in}{1.307086in}}{\pgfqpoint{5.232001in}{1.300244in}}{\pgfqpoint{5.237045in}{1.295201in}}%
\pgfpathcurveto{\pgfqpoint{5.242088in}{1.290157in}}{\pgfqpoint{5.248930in}{1.287323in}}{\pgfqpoint{5.256063in}{1.287323in}}%
\pgfpathclose%
\pgfusepath{stroke,fill}%
\end{pgfscope}%
\begin{pgfscope}%
\pgfpathrectangle{\pgfqpoint{4.985294in}{0.500000in}}{\pgfqpoint{1.764706in}{1.700000in}}%
\pgfusepath{clip}%
\pgfsetbuttcap%
\pgfsetroundjoin%
\definecolor{currentfill}{rgb}{0.973832,0.856556,0.771584}%
\pgfsetfillcolor{currentfill}%
\pgfsetlinewidth{0.311001pt}%
\definecolor{currentstroke}{rgb}{1.000000,1.000000,1.000000}%
\pgfsetstrokecolor{currentstroke}%
\pgfsetdash{}{0pt}%
\pgfpathmoveto{\pgfqpoint{5.438034in}{1.559621in}}%
\pgfpathcurveto{\pgfqpoint{5.445167in}{1.559621in}}{\pgfqpoint{5.452009in}{1.562455in}}{\pgfqpoint{5.457052in}{1.567498in}}%
\pgfpathcurveto{\pgfqpoint{5.462096in}{1.572542in}}{\pgfqpoint{5.464930in}{1.579384in}}{\pgfqpoint{5.464930in}{1.586517in}}%
\pgfpathcurveto{\pgfqpoint{5.464930in}{1.593649in}}{\pgfqpoint{5.462096in}{1.600491in}}{\pgfqpoint{5.457052in}{1.605535in}}%
\pgfpathcurveto{\pgfqpoint{5.452009in}{1.610578in}}{\pgfqpoint{5.445167in}{1.613412in}}{\pgfqpoint{5.438034in}{1.613412in}}%
\pgfpathcurveto{\pgfqpoint{5.430901in}{1.613412in}}{\pgfqpoint{5.424060in}{1.610578in}}{\pgfqpoint{5.419016in}{1.605535in}}%
\pgfpathcurveto{\pgfqpoint{5.413972in}{1.600491in}}{\pgfqpoint{5.411139in}{1.593649in}}{\pgfqpoint{5.411139in}{1.586517in}}%
\pgfpathcurveto{\pgfqpoint{5.411139in}{1.579384in}}{\pgfqpoint{5.413972in}{1.572542in}}{\pgfqpoint{5.419016in}{1.567498in}}%
\pgfpathcurveto{\pgfqpoint{5.424060in}{1.562455in}}{\pgfqpoint{5.430901in}{1.559621in}}{\pgfqpoint{5.438034in}{1.559621in}}%
\pgfpathclose%
\pgfusepath{stroke,fill}%
\end{pgfscope}%
\begin{pgfscope}%
\pgfpathrectangle{\pgfqpoint{4.985294in}{0.500000in}}{\pgfqpoint{1.764706in}{1.700000in}}%
\pgfusepath{clip}%
\pgfsetbuttcap%
\pgfsetroundjoin%
\definecolor{currentfill}{rgb}{0.973271,0.850724,0.762998}%
\pgfsetfillcolor{currentfill}%
\pgfsetlinewidth{0.311001pt}%
\definecolor{currentstroke}{rgb}{1.000000,1.000000,1.000000}%
\pgfsetstrokecolor{currentstroke}%
\pgfsetdash{}{0pt}%
\pgfpathmoveto{\pgfqpoint{6.354920in}{1.180311in}}%
\pgfpathcurveto{\pgfqpoint{6.362053in}{1.180311in}}{\pgfqpoint{6.368894in}{1.183145in}}{\pgfqpoint{6.373938in}{1.188188in}}%
\pgfpathcurveto{\pgfqpoint{6.378982in}{1.193232in}}{\pgfqpoint{6.381816in}{1.200074in}}{\pgfqpoint{6.381816in}{1.207206in}}%
\pgfpathcurveto{\pgfqpoint{6.381816in}{1.214339in}}{\pgfqpoint{6.378982in}{1.221181in}}{\pgfqpoint{6.373938in}{1.226224in}}%
\pgfpathcurveto{\pgfqpoint{6.368894in}{1.231268in}}{\pgfqpoint{6.362053in}{1.234102in}}{\pgfqpoint{6.354920in}{1.234102in}}%
\pgfpathcurveto{\pgfqpoint{6.347787in}{1.234102in}}{\pgfqpoint{6.340945in}{1.231268in}}{\pgfqpoint{6.335902in}{1.226224in}}%
\pgfpathcurveto{\pgfqpoint{6.330858in}{1.221181in}}{\pgfqpoint{6.328024in}{1.214339in}}{\pgfqpoint{6.328024in}{1.207206in}}%
\pgfpathcurveto{\pgfqpoint{6.328024in}{1.200074in}}{\pgfqpoint{6.330858in}{1.193232in}}{\pgfqpoint{6.335902in}{1.188188in}}%
\pgfpathcurveto{\pgfqpoint{6.340945in}{1.183145in}}{\pgfqpoint{6.347787in}{1.180311in}}{\pgfqpoint{6.354920in}{1.180311in}}%
\pgfpathclose%
\pgfusepath{stroke,fill}%
\end{pgfscope}%
\begin{pgfscope}%
\pgfpathrectangle{\pgfqpoint{4.985294in}{0.500000in}}{\pgfqpoint{1.764706in}{1.700000in}}%
\pgfusepath{clip}%
\pgfsetbuttcap%
\pgfsetroundjoin%
\definecolor{currentfill}{rgb}{0.965302,0.713942,0.568499}%
\pgfsetfillcolor{currentfill}%
\pgfsetlinewidth{0.311001pt}%
\definecolor{currentstroke}{rgb}{1.000000,1.000000,1.000000}%
\pgfsetstrokecolor{currentstroke}%
\pgfsetdash{}{0pt}%
\pgfpathmoveto{\pgfqpoint{6.269386in}{1.690822in}}%
\pgfpathcurveto{\pgfqpoint{6.276519in}{1.690822in}}{\pgfqpoint{6.283361in}{1.693656in}}{\pgfqpoint{6.288405in}{1.698700in}}%
\pgfpathcurveto{\pgfqpoint{6.293448in}{1.703743in}}{\pgfqpoint{6.296282in}{1.710585in}}{\pgfqpoint{6.296282in}{1.717718in}}%
\pgfpathcurveto{\pgfqpoint{6.296282in}{1.724851in}}{\pgfqpoint{6.293448in}{1.731692in}}{\pgfqpoint{6.288405in}{1.736736in}}%
\pgfpathcurveto{\pgfqpoint{6.283361in}{1.741780in}}{\pgfqpoint{6.276519in}{1.744613in}}{\pgfqpoint{6.269386in}{1.744613in}}%
\pgfpathcurveto{\pgfqpoint{6.262254in}{1.744613in}}{\pgfqpoint{6.255412in}{1.741780in}}{\pgfqpoint{6.250368in}{1.736736in}}%
\pgfpathcurveto{\pgfqpoint{6.245325in}{1.731692in}}{\pgfqpoint{6.242491in}{1.724851in}}{\pgfqpoint{6.242491in}{1.717718in}}%
\pgfpathcurveto{\pgfqpoint{6.242491in}{1.710585in}}{\pgfqpoint{6.245325in}{1.703743in}}{\pgfqpoint{6.250368in}{1.698700in}}%
\pgfpathcurveto{\pgfqpoint{6.255412in}{1.693656in}}{\pgfqpoint{6.262254in}{1.690822in}}{\pgfqpoint{6.269386in}{1.690822in}}%
\pgfpathclose%
\pgfusepath{stroke,fill}%
\end{pgfscope}%
\begin{pgfscope}%
\pgfpathrectangle{\pgfqpoint{4.985294in}{0.500000in}}{\pgfqpoint{1.764706in}{1.700000in}}%
\pgfusepath{clip}%
\pgfsetbuttcap%
\pgfsetroundjoin%
\definecolor{currentfill}{rgb}{0.965169,0.707764,0.560659}%
\pgfsetfillcolor{currentfill}%
\pgfsetlinewidth{0.311001pt}%
\definecolor{currentstroke}{rgb}{1.000000,1.000000,1.000000}%
\pgfsetstrokecolor{currentstroke}%
\pgfsetdash{}{0pt}%
\pgfpathmoveto{\pgfqpoint{5.450283in}{0.943187in}}%
\pgfpathcurveto{\pgfqpoint{5.457416in}{0.943187in}}{\pgfqpoint{5.464257in}{0.946021in}}{\pgfqpoint{5.469301in}{0.951064in}}%
\pgfpathcurveto{\pgfqpoint{5.474345in}{0.956108in}}{\pgfqpoint{5.477179in}{0.962950in}}{\pgfqpoint{5.477179in}{0.970082in}}%
\pgfpathcurveto{\pgfqpoint{5.477179in}{0.977215in}}{\pgfqpoint{5.474345in}{0.984057in}}{\pgfqpoint{5.469301in}{0.989101in}}%
\pgfpathcurveto{\pgfqpoint{5.464257in}{0.994144in}}{\pgfqpoint{5.457416in}{0.996978in}}{\pgfqpoint{5.450283in}{0.996978in}}%
\pgfpathcurveto{\pgfqpoint{5.443150in}{0.996978in}}{\pgfqpoint{5.436308in}{0.994144in}}{\pgfqpoint{5.431265in}{0.989101in}}%
\pgfpathcurveto{\pgfqpoint{5.426221in}{0.984057in}}{\pgfqpoint{5.423387in}{0.977215in}}{\pgfqpoint{5.423387in}{0.970082in}}%
\pgfpathcurveto{\pgfqpoint{5.423387in}{0.962950in}}{\pgfqpoint{5.426221in}{0.956108in}}{\pgfqpoint{5.431265in}{0.951064in}}%
\pgfpathcurveto{\pgfqpoint{5.436308in}{0.946021in}}{\pgfqpoint{5.443150in}{0.943187in}}{\pgfqpoint{5.450283in}{0.943187in}}%
\pgfpathclose%
\pgfusepath{stroke,fill}%
\end{pgfscope}%
\begin{pgfscope}%
\pgfpathrectangle{\pgfqpoint{4.985294in}{0.500000in}}{\pgfqpoint{1.764706in}{1.700000in}}%
\pgfusepath{clip}%
\pgfsetbuttcap%
\pgfsetroundjoin%
\definecolor{currentfill}{rgb}{0.584229,0.109227,0.358485}%
\pgfsetfillcolor{currentfill}%
\pgfsetlinewidth{0.311001pt}%
\definecolor{currentstroke}{rgb}{1.000000,1.000000,1.000000}%
\pgfsetstrokecolor{currentstroke}%
\pgfsetdash{}{0pt}%
\pgfpathmoveto{\pgfqpoint{6.058169in}{1.435390in}}%
\pgfpathcurveto{\pgfqpoint{6.065302in}{1.435390in}}{\pgfqpoint{6.072143in}{1.438224in}}{\pgfqpoint{6.077187in}{1.443268in}}%
\pgfpathcurveto{\pgfqpoint{6.082231in}{1.448311in}}{\pgfqpoint{6.085064in}{1.455153in}}{\pgfqpoint{6.085064in}{1.462286in}}%
\pgfpathcurveto{\pgfqpoint{6.085064in}{1.469418in}}{\pgfqpoint{6.082231in}{1.476260in}}{\pgfqpoint{6.077187in}{1.481304in}}%
\pgfpathcurveto{\pgfqpoint{6.072143in}{1.486347in}}{\pgfqpoint{6.065302in}{1.489181in}}{\pgfqpoint{6.058169in}{1.489181in}}%
\pgfpathcurveto{\pgfqpoint{6.051036in}{1.489181in}}{\pgfqpoint{6.044194in}{1.486347in}}{\pgfqpoint{6.039151in}{1.481304in}}%
\pgfpathcurveto{\pgfqpoint{6.034107in}{1.476260in}}{\pgfqpoint{6.031273in}{1.469418in}}{\pgfqpoint{6.031273in}{1.462286in}}%
\pgfpathcurveto{\pgfqpoint{6.031273in}{1.455153in}}{\pgfqpoint{6.034107in}{1.448311in}}{\pgfqpoint{6.039151in}{1.443268in}}%
\pgfpathcurveto{\pgfqpoint{6.044194in}{1.438224in}}{\pgfqpoint{6.051036in}{1.435390in}}{\pgfqpoint{6.058169in}{1.435390in}}%
\pgfpathclose%
\pgfusepath{stroke,fill}%
\end{pgfscope}%
\begin{pgfscope}%
\pgfpathrectangle{\pgfqpoint{4.985294in}{0.500000in}}{\pgfqpoint{1.764706in}{1.700000in}}%
\pgfusepath{clip}%
\pgfsetbuttcap%
\pgfsetroundjoin%
\definecolor{currentfill}{rgb}{0.963190,0.619109,0.458249}%
\pgfsetfillcolor{currentfill}%
\pgfsetlinewidth{0.311001pt}%
\definecolor{currentstroke}{rgb}{1.000000,1.000000,1.000000}%
\pgfsetstrokecolor{currentstroke}%
\pgfsetdash{}{0pt}%
\pgfpathmoveto{\pgfqpoint{5.612628in}{1.049457in}}%
\pgfpathcurveto{\pgfqpoint{5.619761in}{1.049457in}}{\pgfqpoint{5.626602in}{1.052291in}}{\pgfqpoint{5.631646in}{1.057335in}}%
\pgfpathcurveto{\pgfqpoint{5.636690in}{1.062378in}}{\pgfqpoint{5.639524in}{1.069220in}}{\pgfqpoint{5.639524in}{1.076353in}}%
\pgfpathcurveto{\pgfqpoint{5.639524in}{1.083486in}}{\pgfqpoint{5.636690in}{1.090327in}}{\pgfqpoint{5.631646in}{1.095371in}}%
\pgfpathcurveto{\pgfqpoint{5.626602in}{1.100415in}}{\pgfqpoint{5.619761in}{1.103249in}}{\pgfqpoint{5.612628in}{1.103249in}}%
\pgfpathcurveto{\pgfqpoint{5.605495in}{1.103249in}}{\pgfqpoint{5.598653in}{1.100415in}}{\pgfqpoint{5.593610in}{1.095371in}}%
\pgfpathcurveto{\pgfqpoint{5.588566in}{1.090327in}}{\pgfqpoint{5.585732in}{1.083486in}}{\pgfqpoint{5.585732in}{1.076353in}}%
\pgfpathcurveto{\pgfqpoint{5.585732in}{1.069220in}}{\pgfqpoint{5.588566in}{1.062378in}}{\pgfqpoint{5.593610in}{1.057335in}}%
\pgfpathcurveto{\pgfqpoint{5.598653in}{1.052291in}}{\pgfqpoint{5.605495in}{1.049457in}}{\pgfqpoint{5.612628in}{1.049457in}}%
\pgfpathclose%
\pgfusepath{stroke,fill}%
\end{pgfscope}%
\begin{pgfscope}%
\pgfpathrectangle{\pgfqpoint{4.985294in}{0.500000in}}{\pgfqpoint{1.764706in}{1.700000in}}%
\pgfusepath{clip}%
\pgfsetbuttcap%
\pgfsetroundjoin%
\definecolor{currentfill}{rgb}{0.964433,0.670254,0.515093}%
\pgfsetfillcolor{currentfill}%
\pgfsetlinewidth{0.311001pt}%
\definecolor{currentstroke}{rgb}{1.000000,1.000000,1.000000}%
\pgfsetstrokecolor{currentstroke}%
\pgfsetdash{}{0pt}%
\pgfpathmoveto{\pgfqpoint{6.155473in}{0.896587in}}%
\pgfpathcurveto{\pgfqpoint{6.162605in}{0.896587in}}{\pgfqpoint{6.169447in}{0.899421in}}{\pgfqpoint{6.174491in}{0.904464in}}%
\pgfpathcurveto{\pgfqpoint{6.179534in}{0.909508in}}{\pgfqpoint{6.182368in}{0.916350in}}{\pgfqpoint{6.182368in}{0.923483in}}%
\pgfpathcurveto{\pgfqpoint{6.182368in}{0.930615in}}{\pgfqpoint{6.179534in}{0.937457in}}{\pgfqpoint{6.174491in}{0.942501in}}%
\pgfpathcurveto{\pgfqpoint{6.169447in}{0.947544in}}{\pgfqpoint{6.162605in}{0.950378in}}{\pgfqpoint{6.155473in}{0.950378in}}%
\pgfpathcurveto{\pgfqpoint{6.148340in}{0.950378in}}{\pgfqpoint{6.141498in}{0.947544in}}{\pgfqpoint{6.136454in}{0.942501in}}%
\pgfpathcurveto{\pgfqpoint{6.131411in}{0.937457in}}{\pgfqpoint{6.128577in}{0.930615in}}{\pgfqpoint{6.128577in}{0.923483in}}%
\pgfpathcurveto{\pgfqpoint{6.128577in}{0.916350in}}{\pgfqpoint{6.131411in}{0.909508in}}{\pgfqpoint{6.136454in}{0.904464in}}%
\pgfpathcurveto{\pgfqpoint{6.141498in}{0.899421in}}{\pgfqpoint{6.148340in}{0.896587in}}{\pgfqpoint{6.155473in}{0.896587in}}%
\pgfpathclose%
\pgfusepath{stroke,fill}%
\end{pgfscope}%
\begin{pgfscope}%
\pgfpathrectangle{\pgfqpoint{4.985294in}{0.500000in}}{\pgfqpoint{1.764706in}{1.700000in}}%
\pgfusepath{clip}%
\pgfsetbuttcap%
\pgfsetroundjoin%
\definecolor{currentfill}{rgb}{0.962532,0.599594,0.438051}%
\pgfsetfillcolor{currentfill}%
\pgfsetlinewidth{0.311001pt}%
\definecolor{currentstroke}{rgb}{1.000000,1.000000,1.000000}%
\pgfsetstrokecolor{currentstroke}%
\pgfsetdash{}{0pt}%
\pgfpathmoveto{\pgfqpoint{5.389983in}{0.973594in}}%
\pgfpathcurveto{\pgfqpoint{5.397116in}{0.973594in}}{\pgfqpoint{5.403958in}{0.976427in}}{\pgfqpoint{5.409001in}{0.981471in}}%
\pgfpathcurveto{\pgfqpoint{5.414045in}{0.986515in}}{\pgfqpoint{5.416879in}{0.993356in}}{\pgfqpoint{5.416879in}{1.000489in}}%
\pgfpathcurveto{\pgfqpoint{5.416879in}{1.007622in}}{\pgfqpoint{5.414045in}{1.014464in}}{\pgfqpoint{5.409001in}{1.019507in}}%
\pgfpathcurveto{\pgfqpoint{5.403958in}{1.024551in}}{\pgfqpoint{5.397116in}{1.027385in}}{\pgfqpoint{5.389983in}{1.027385in}}%
\pgfpathcurveto{\pgfqpoint{5.382850in}{1.027385in}}{\pgfqpoint{5.376009in}{1.024551in}}{\pgfqpoint{5.370965in}{1.019507in}}%
\pgfpathcurveto{\pgfqpoint{5.365921in}{1.014464in}}{\pgfqpoint{5.363087in}{1.007622in}}{\pgfqpoint{5.363087in}{1.000489in}}%
\pgfpathcurveto{\pgfqpoint{5.363087in}{0.993356in}}{\pgfqpoint{5.365921in}{0.986515in}}{\pgfqpoint{5.370965in}{0.981471in}}%
\pgfpathcurveto{\pgfqpoint{5.376009in}{0.976427in}}{\pgfqpoint{5.382850in}{0.973594in}}{\pgfqpoint{5.389983in}{0.973594in}}%
\pgfpathclose%
\pgfusepath{stroke,fill}%
\end{pgfscope}%
\begin{pgfscope}%
\pgfpathrectangle{\pgfqpoint{4.985294in}{0.500000in}}{\pgfqpoint{1.764706in}{1.700000in}}%
\pgfusepath{clip}%
\pgfsetbuttcap%
\pgfsetroundjoin%
\definecolor{currentfill}{rgb}{0.944085,0.383081,0.267220}%
\pgfsetfillcolor{currentfill}%
\pgfsetlinewidth{0.311001pt}%
\definecolor{currentstroke}{rgb}{1.000000,1.000000,1.000000}%
\pgfsetstrokecolor{currentstroke}%
\pgfsetdash{}{0pt}%
\pgfpathmoveto{\pgfqpoint{6.311421in}{0.903564in}}%
\pgfpathcurveto{\pgfqpoint{6.318554in}{0.903564in}}{\pgfqpoint{6.325396in}{0.906398in}}{\pgfqpoint{6.330439in}{0.911442in}}%
\pgfpathcurveto{\pgfqpoint{6.335483in}{0.916486in}}{\pgfqpoint{6.338317in}{0.923327in}}{\pgfqpoint{6.338317in}{0.930460in}}%
\pgfpathcurveto{\pgfqpoint{6.338317in}{0.937593in}}{\pgfqpoint{6.335483in}{0.944435in}}{\pgfqpoint{6.330439in}{0.949478in}}%
\pgfpathcurveto{\pgfqpoint{6.325396in}{0.954522in}}{\pgfqpoint{6.318554in}{0.957356in}}{\pgfqpoint{6.311421in}{0.957356in}}%
\pgfpathcurveto{\pgfqpoint{6.304288in}{0.957356in}}{\pgfqpoint{6.297447in}{0.954522in}}{\pgfqpoint{6.292403in}{0.949478in}}%
\pgfpathcurveto{\pgfqpoint{6.287359in}{0.944435in}}{\pgfqpoint{6.284525in}{0.937593in}}{\pgfqpoint{6.284525in}{0.930460in}}%
\pgfpathcurveto{\pgfqpoint{6.284525in}{0.923327in}}{\pgfqpoint{6.287359in}{0.916486in}}{\pgfqpoint{6.292403in}{0.911442in}}%
\pgfpathcurveto{\pgfqpoint{6.297447in}{0.906398in}}{\pgfqpoint{6.304288in}{0.903564in}}{\pgfqpoint{6.311421in}{0.903564in}}%
\pgfpathclose%
\pgfusepath{stroke,fill}%
\end{pgfscope}%
\begin{pgfscope}%
\pgfpathrectangle{\pgfqpoint{4.985294in}{0.500000in}}{\pgfqpoint{1.764706in}{1.700000in}}%
\pgfusepath{clip}%
\pgfsetbuttcap%
\pgfsetroundjoin%
\definecolor{currentfill}{rgb}{0.968931,0.798091,0.685123}%
\pgfsetfillcolor{currentfill}%
\pgfsetlinewidth{0.311001pt}%
\definecolor{currentstroke}{rgb}{1.000000,1.000000,1.000000}%
\pgfsetstrokecolor{currentstroke}%
\pgfsetdash{}{0pt}%
\pgfpathmoveto{\pgfqpoint{6.175786in}{0.993959in}}%
\pgfpathcurveto{\pgfqpoint{6.182919in}{0.993959in}}{\pgfqpoint{6.189761in}{0.996793in}}{\pgfqpoint{6.194804in}{1.001836in}}%
\pgfpathcurveto{\pgfqpoint{6.199848in}{1.006880in}}{\pgfqpoint{6.202682in}{1.013722in}}{\pgfqpoint{6.202682in}{1.020854in}}%
\pgfpathcurveto{\pgfqpoint{6.202682in}{1.027987in}}{\pgfqpoint{6.199848in}{1.034829in}}{\pgfqpoint{6.194804in}{1.039873in}}%
\pgfpathcurveto{\pgfqpoint{6.189761in}{1.044916in}}{\pgfqpoint{6.182919in}{1.047750in}}{\pgfqpoint{6.175786in}{1.047750in}}%
\pgfpathcurveto{\pgfqpoint{6.168653in}{1.047750in}}{\pgfqpoint{6.161812in}{1.044916in}}{\pgfqpoint{6.156768in}{1.039873in}}%
\pgfpathcurveto{\pgfqpoint{6.151724in}{1.034829in}}{\pgfqpoint{6.148891in}{1.027987in}}{\pgfqpoint{6.148891in}{1.020854in}}%
\pgfpathcurveto{\pgfqpoint{6.148891in}{1.013722in}}{\pgfqpoint{6.151724in}{1.006880in}}{\pgfqpoint{6.156768in}{1.001836in}}%
\pgfpathcurveto{\pgfqpoint{6.161812in}{0.996793in}}{\pgfqpoint{6.168653in}{0.993959in}}{\pgfqpoint{6.175786in}{0.993959in}}%
\pgfpathclose%
\pgfusepath{stroke,fill}%
\end{pgfscope}%
\begin{pgfscope}%
\pgfpathrectangle{\pgfqpoint{4.985294in}{0.500000in}}{\pgfqpoint{1.764706in}{1.700000in}}%
\pgfusepath{clip}%
\pgfsetbuttcap%
\pgfsetroundjoin%
\definecolor{currentfill}{rgb}{0.970718,0.821518,0.719872}%
\pgfsetfillcolor{currentfill}%
\pgfsetlinewidth{0.311001pt}%
\definecolor{currentstroke}{rgb}{1.000000,1.000000,1.000000}%
\pgfsetstrokecolor{currentstroke}%
\pgfsetdash{}{0pt}%
\pgfpathmoveto{\pgfqpoint{5.524044in}{1.509697in}}%
\pgfpathcurveto{\pgfqpoint{5.531176in}{1.509697in}}{\pgfqpoint{5.538018in}{1.512530in}}{\pgfqpoint{5.543062in}{1.517574in}}%
\pgfpathcurveto{\pgfqpoint{5.548105in}{1.522618in}}{\pgfqpoint{5.550939in}{1.529459in}}{\pgfqpoint{5.550939in}{1.536592in}}%
\pgfpathcurveto{\pgfqpoint{5.550939in}{1.543725in}}{\pgfqpoint{5.548105in}{1.550567in}}{\pgfqpoint{5.543062in}{1.555610in}}%
\pgfpathcurveto{\pgfqpoint{5.538018in}{1.560654in}}{\pgfqpoint{5.531176in}{1.563488in}}{\pgfqpoint{5.524044in}{1.563488in}}%
\pgfpathcurveto{\pgfqpoint{5.516911in}{1.563488in}}{\pgfqpoint{5.510069in}{1.560654in}}{\pgfqpoint{5.505025in}{1.555610in}}%
\pgfpathcurveto{\pgfqpoint{5.499982in}{1.550567in}}{\pgfqpoint{5.497148in}{1.543725in}}{\pgfqpoint{5.497148in}{1.536592in}}%
\pgfpathcurveto{\pgfqpoint{5.497148in}{1.529459in}}{\pgfqpoint{5.499982in}{1.522618in}}{\pgfqpoint{5.505025in}{1.517574in}}%
\pgfpathcurveto{\pgfqpoint{5.510069in}{1.512530in}}{\pgfqpoint{5.516911in}{1.509697in}}{\pgfqpoint{5.524044in}{1.509697in}}%
\pgfpathclose%
\pgfusepath{stroke,fill}%
\end{pgfscope}%
\begin{pgfscope}%
\pgfpathrectangle{\pgfqpoint{4.985294in}{0.500000in}}{\pgfqpoint{1.764706in}{1.700000in}}%
\pgfusepath{clip}%
\pgfsetbuttcap%
\pgfsetroundjoin%
\definecolor{currentfill}{rgb}{0.968509,0.792226,0.676405}%
\pgfsetfillcolor{currentfill}%
\pgfsetlinewidth{0.311001pt}%
\definecolor{currentstroke}{rgb}{1.000000,1.000000,1.000000}%
\pgfsetstrokecolor{currentstroke}%
\pgfsetdash{}{0pt}%
\pgfpathmoveto{\pgfqpoint{5.521475in}{1.457314in}}%
\pgfpathcurveto{\pgfqpoint{5.528608in}{1.457314in}}{\pgfqpoint{5.535450in}{1.460148in}}{\pgfqpoint{5.540493in}{1.465191in}}%
\pgfpathcurveto{\pgfqpoint{5.545537in}{1.470235in}}{\pgfqpoint{5.548371in}{1.477077in}}{\pgfqpoint{5.548371in}{1.484210in}}%
\pgfpathcurveto{\pgfqpoint{5.548371in}{1.491342in}}{\pgfqpoint{5.545537in}{1.498184in}}{\pgfqpoint{5.540493in}{1.503228in}}%
\pgfpathcurveto{\pgfqpoint{5.535450in}{1.508271in}}{\pgfqpoint{5.528608in}{1.511105in}}{\pgfqpoint{5.521475in}{1.511105in}}%
\pgfpathcurveto{\pgfqpoint{5.514342in}{1.511105in}}{\pgfqpoint{5.507501in}{1.508271in}}{\pgfqpoint{5.502457in}{1.503228in}}%
\pgfpathcurveto{\pgfqpoint{5.497413in}{1.498184in}}{\pgfqpoint{5.494580in}{1.491342in}}{\pgfqpoint{5.494580in}{1.484210in}}%
\pgfpathcurveto{\pgfqpoint{5.494580in}{1.477077in}}{\pgfqpoint{5.497413in}{1.470235in}}{\pgfqpoint{5.502457in}{1.465191in}}%
\pgfpathcurveto{\pgfqpoint{5.507501in}{1.460148in}}{\pgfqpoint{5.514342in}{1.457314in}}{\pgfqpoint{5.521475in}{1.457314in}}%
\pgfpathclose%
\pgfusepath{stroke,fill}%
\end{pgfscope}%
\begin{pgfscope}%
\pgfpathrectangle{\pgfqpoint{4.985294in}{0.500000in}}{\pgfqpoint{1.764706in}{1.700000in}}%
\pgfusepath{clip}%
\pgfsetbuttcap%
\pgfsetroundjoin%
\definecolor{currentfill}{rgb}{0.965302,0.713942,0.568499}%
\pgfsetfillcolor{currentfill}%
\pgfsetlinewidth{0.311001pt}%
\definecolor{currentstroke}{rgb}{1.000000,1.000000,1.000000}%
\pgfsetstrokecolor{currentstroke}%
\pgfsetdash{}{0pt}%
\pgfpathmoveto{\pgfqpoint{5.404578in}{1.006875in}}%
\pgfpathcurveto{\pgfqpoint{5.411711in}{1.006875in}}{\pgfqpoint{5.418553in}{1.009709in}}{\pgfqpoint{5.423596in}{1.014753in}}%
\pgfpathcurveto{\pgfqpoint{5.428640in}{1.019796in}}{\pgfqpoint{5.431474in}{1.026638in}}{\pgfqpoint{5.431474in}{1.033771in}}%
\pgfpathcurveto{\pgfqpoint{5.431474in}{1.040904in}}{\pgfqpoint{5.428640in}{1.047745in}}{\pgfqpoint{5.423596in}{1.052789in}}%
\pgfpathcurveto{\pgfqpoint{5.418553in}{1.057833in}}{\pgfqpoint{5.411711in}{1.060667in}}{\pgfqpoint{5.404578in}{1.060667in}}%
\pgfpathcurveto{\pgfqpoint{5.397445in}{1.060667in}}{\pgfqpoint{5.390604in}{1.057833in}}{\pgfqpoint{5.385560in}{1.052789in}}%
\pgfpathcurveto{\pgfqpoint{5.380516in}{1.047745in}}{\pgfqpoint{5.377682in}{1.040904in}}{\pgfqpoint{5.377682in}{1.033771in}}%
\pgfpathcurveto{\pgfqpoint{5.377682in}{1.026638in}}{\pgfqpoint{5.380516in}{1.019796in}}{\pgfqpoint{5.385560in}{1.014753in}}%
\pgfpathcurveto{\pgfqpoint{5.390604in}{1.009709in}}{\pgfqpoint{5.397445in}{1.006875in}}{\pgfqpoint{5.404578in}{1.006875in}}%
\pgfpathclose%
\pgfusepath{stroke,fill}%
\end{pgfscope}%
\begin{pgfscope}%
\pgfpathrectangle{\pgfqpoint{4.985294in}{0.500000in}}{\pgfqpoint{1.764706in}{1.700000in}}%
\pgfusepath{clip}%
\pgfsetbuttcap%
\pgfsetroundjoin%
\definecolor{currentfill}{rgb}{0.970255,0.815666,0.711203}%
\pgfsetfillcolor{currentfill}%
\pgfsetlinewidth{0.311001pt}%
\definecolor{currentstroke}{rgb}{1.000000,1.000000,1.000000}%
\pgfsetstrokecolor{currentstroke}%
\pgfsetdash{}{0pt}%
\pgfpathmoveto{\pgfqpoint{5.533709in}{1.601978in}}%
\pgfpathcurveto{\pgfqpoint{5.540841in}{1.601978in}}{\pgfqpoint{5.547683in}{1.604812in}}{\pgfqpoint{5.552727in}{1.609856in}}%
\pgfpathcurveto{\pgfqpoint{5.557770in}{1.614900in}}{\pgfqpoint{5.560604in}{1.621741in}}{\pgfqpoint{5.560604in}{1.628874in}}%
\pgfpathcurveto{\pgfqpoint{5.560604in}{1.636007in}}{\pgfqpoint{5.557770in}{1.642849in}}{\pgfqpoint{5.552727in}{1.647892in}}%
\pgfpathcurveto{\pgfqpoint{5.547683in}{1.652936in}}{\pgfqpoint{5.540841in}{1.655770in}}{\pgfqpoint{5.533709in}{1.655770in}}%
\pgfpathcurveto{\pgfqpoint{5.526576in}{1.655770in}}{\pgfqpoint{5.519734in}{1.652936in}}{\pgfqpoint{5.514690in}{1.647892in}}%
\pgfpathcurveto{\pgfqpoint{5.509647in}{1.642849in}}{\pgfqpoint{5.506813in}{1.636007in}}{\pgfqpoint{5.506813in}{1.628874in}}%
\pgfpathcurveto{\pgfqpoint{5.506813in}{1.621741in}}{\pgfqpoint{5.509647in}{1.614900in}}{\pgfqpoint{5.514690in}{1.609856in}}%
\pgfpathcurveto{\pgfqpoint{5.519734in}{1.604812in}}{\pgfqpoint{5.526576in}{1.601978in}}{\pgfqpoint{5.533709in}{1.601978in}}%
\pgfpathclose%
\pgfusepath{stroke,fill}%
\end{pgfscope}%
\begin{pgfscope}%
\pgfpathrectangle{\pgfqpoint{4.985294in}{0.500000in}}{\pgfqpoint{1.764706in}{1.700000in}}%
\pgfusepath{clip}%
\pgfsetbuttcap%
\pgfsetroundjoin%
\definecolor{currentfill}{rgb}{0.965302,0.713942,0.568499}%
\pgfsetfillcolor{currentfill}%
\pgfsetlinewidth{0.311001pt}%
\definecolor{currentstroke}{rgb}{1.000000,1.000000,1.000000}%
\pgfsetstrokecolor{currentstroke}%
\pgfsetdash{}{0pt}%
\pgfpathmoveto{\pgfqpoint{5.446576in}{1.675804in}}%
\pgfpathcurveto{\pgfqpoint{5.453709in}{1.675804in}}{\pgfqpoint{5.460550in}{1.678638in}}{\pgfqpoint{5.465594in}{1.683682in}}%
\pgfpathcurveto{\pgfqpoint{5.470638in}{1.688726in}}{\pgfqpoint{5.473472in}{1.695567in}}{\pgfqpoint{5.473472in}{1.702700in}}%
\pgfpathcurveto{\pgfqpoint{5.473472in}{1.709833in}}{\pgfqpoint{5.470638in}{1.716674in}}{\pgfqpoint{5.465594in}{1.721718in}}%
\pgfpathcurveto{\pgfqpoint{5.460550in}{1.726762in}}{\pgfqpoint{5.453709in}{1.729596in}}{\pgfqpoint{5.446576in}{1.729596in}}%
\pgfpathcurveto{\pgfqpoint{5.439443in}{1.729596in}}{\pgfqpoint{5.432601in}{1.726762in}}{\pgfqpoint{5.427558in}{1.721718in}}%
\pgfpathcurveto{\pgfqpoint{5.422514in}{1.716674in}}{\pgfqpoint{5.419680in}{1.709833in}}{\pgfqpoint{5.419680in}{1.702700in}}%
\pgfpathcurveto{\pgfqpoint{5.419680in}{1.695567in}}{\pgfqpoint{5.422514in}{1.688726in}}{\pgfqpoint{5.427558in}{1.683682in}}%
\pgfpathcurveto{\pgfqpoint{5.432601in}{1.678638in}}{\pgfqpoint{5.439443in}{1.675804in}}{\pgfqpoint{5.446576in}{1.675804in}}%
\pgfpathclose%
\pgfusepath{stroke,fill}%
\end{pgfscope}%
\begin{pgfscope}%
\pgfpathrectangle{\pgfqpoint{4.985294in}{0.500000in}}{\pgfqpoint{1.764706in}{1.700000in}}%
\pgfusepath{clip}%
\pgfsetbuttcap%
\pgfsetroundjoin%
\definecolor{currentfill}{rgb}{0.977657,0.891500,0.822809}%
\pgfsetfillcolor{currentfill}%
\pgfsetlinewidth{0.311001pt}%
\definecolor{currentstroke}{rgb}{1.000000,1.000000,1.000000}%
\pgfsetstrokecolor{currentstroke}%
\pgfsetdash{}{0pt}%
\pgfpathmoveto{\pgfqpoint{6.267005in}{1.126254in}}%
\pgfpathcurveto{\pgfqpoint{6.274138in}{1.126254in}}{\pgfqpoint{6.280980in}{1.129088in}}{\pgfqpoint{6.286023in}{1.134131in}}%
\pgfpathcurveto{\pgfqpoint{6.291067in}{1.139175in}}{\pgfqpoint{6.293901in}{1.146017in}}{\pgfqpoint{6.293901in}{1.153150in}}%
\pgfpathcurveto{\pgfqpoint{6.293901in}{1.160282in}}{\pgfqpoint{6.291067in}{1.167124in}}{\pgfqpoint{6.286023in}{1.172168in}}%
\pgfpathcurveto{\pgfqpoint{6.280980in}{1.177211in}}{\pgfqpoint{6.274138in}{1.180045in}}{\pgfqpoint{6.267005in}{1.180045in}}%
\pgfpathcurveto{\pgfqpoint{6.259873in}{1.180045in}}{\pgfqpoint{6.253031in}{1.177211in}}{\pgfqpoint{6.247987in}{1.172168in}}%
\pgfpathcurveto{\pgfqpoint{6.242944in}{1.167124in}}{\pgfqpoint{6.240110in}{1.160282in}}{\pgfqpoint{6.240110in}{1.153150in}}%
\pgfpathcurveto{\pgfqpoint{6.240110in}{1.146017in}}{\pgfqpoint{6.242944in}{1.139175in}}{\pgfqpoint{6.247987in}{1.134131in}}%
\pgfpathcurveto{\pgfqpoint{6.253031in}{1.129088in}}{\pgfqpoint{6.259873in}{1.126254in}}{\pgfqpoint{6.267005in}{1.126254in}}%
\pgfpathclose%
\pgfusepath{stroke,fill}%
\end{pgfscope}%
\begin{pgfscope}%
\pgfpathrectangle{\pgfqpoint{4.985294in}{0.500000in}}{\pgfqpoint{1.764706in}{1.700000in}}%
\pgfusepath{clip}%
\pgfsetbuttcap%
\pgfsetroundjoin%
\definecolor{currentfill}{rgb}{0.957344,0.505732,0.351309}%
\pgfsetfillcolor{currentfill}%
\pgfsetlinewidth{0.311001pt}%
\definecolor{currentstroke}{rgb}{1.000000,1.000000,1.000000}%
\pgfsetstrokecolor{currentstroke}%
\pgfsetdash{}{0pt}%
\pgfpathmoveto{\pgfqpoint{6.075140in}{0.889444in}}%
\pgfpathcurveto{\pgfqpoint{6.082273in}{0.889444in}}{\pgfqpoint{6.089115in}{0.892278in}}{\pgfqpoint{6.094158in}{0.897321in}}%
\pgfpathcurveto{\pgfqpoint{6.099202in}{0.902365in}}{\pgfqpoint{6.102036in}{0.909207in}}{\pgfqpoint{6.102036in}{0.916340in}}%
\pgfpathcurveto{\pgfqpoint{6.102036in}{0.923472in}}{\pgfqpoint{6.099202in}{0.930314in}}{\pgfqpoint{6.094158in}{0.935358in}}%
\pgfpathcurveto{\pgfqpoint{6.089115in}{0.940401in}}{\pgfqpoint{6.082273in}{0.943235in}}{\pgfqpoint{6.075140in}{0.943235in}}%
\pgfpathcurveto{\pgfqpoint{6.068007in}{0.943235in}}{\pgfqpoint{6.061166in}{0.940401in}}{\pgfqpoint{6.056122in}{0.935358in}}%
\pgfpathcurveto{\pgfqpoint{6.051078in}{0.930314in}}{\pgfqpoint{6.048244in}{0.923472in}}{\pgfqpoint{6.048244in}{0.916340in}}%
\pgfpathcurveto{\pgfqpoint{6.048244in}{0.909207in}}{\pgfqpoint{6.051078in}{0.902365in}}{\pgfqpoint{6.056122in}{0.897321in}}%
\pgfpathcurveto{\pgfqpoint{6.061166in}{0.892278in}}{\pgfqpoint{6.068007in}{0.889444in}}{\pgfqpoint{6.075140in}{0.889444in}}%
\pgfpathclose%
\pgfusepath{stroke,fill}%
\end{pgfscope}%
\begin{pgfscope}%
\pgfpathrectangle{\pgfqpoint{4.985294in}{0.500000in}}{\pgfqpoint{1.764706in}{1.700000in}}%
\pgfusepath{clip}%
\pgfsetbuttcap%
\pgfsetroundjoin%
\definecolor{currentfill}{rgb}{0.976287,0.879862,0.805788}%
\pgfsetfillcolor{currentfill}%
\pgfsetlinewidth{0.311001pt}%
\definecolor{currentstroke}{rgb}{1.000000,1.000000,1.000000}%
\pgfsetstrokecolor{currentstroke}%
\pgfsetdash{}{0pt}%
\pgfpathmoveto{\pgfqpoint{5.461421in}{1.374557in}}%
\pgfpathcurveto{\pgfqpoint{5.468554in}{1.374557in}}{\pgfqpoint{5.475396in}{1.377391in}}{\pgfqpoint{5.480439in}{1.382434in}}%
\pgfpathcurveto{\pgfqpoint{5.485483in}{1.387478in}}{\pgfqpoint{5.488317in}{1.394320in}}{\pgfqpoint{5.488317in}{1.401453in}}%
\pgfpathcurveto{\pgfqpoint{5.488317in}{1.408585in}}{\pgfqpoint{5.485483in}{1.415427in}}{\pgfqpoint{5.480439in}{1.420471in}}%
\pgfpathcurveto{\pgfqpoint{5.475396in}{1.425514in}}{\pgfqpoint{5.468554in}{1.428348in}}{\pgfqpoint{5.461421in}{1.428348in}}%
\pgfpathcurveto{\pgfqpoint{5.454288in}{1.428348in}}{\pgfqpoint{5.447447in}{1.425514in}}{\pgfqpoint{5.442403in}{1.420471in}}%
\pgfpathcurveto{\pgfqpoint{5.437359in}{1.415427in}}{\pgfqpoint{5.434525in}{1.408585in}}{\pgfqpoint{5.434525in}{1.401453in}}%
\pgfpathcurveto{\pgfqpoint{5.434525in}{1.394320in}}{\pgfqpoint{5.437359in}{1.387478in}}{\pgfqpoint{5.442403in}{1.382434in}}%
\pgfpathcurveto{\pgfqpoint{5.447447in}{1.377391in}}{\pgfqpoint{5.454288in}{1.374557in}}{\pgfqpoint{5.461421in}{1.374557in}}%
\pgfpathclose%
\pgfusepath{stroke,fill}%
\end{pgfscope}%
\begin{pgfscope}%
\pgfpathrectangle{\pgfqpoint{4.985294in}{0.500000in}}{\pgfqpoint{1.764706in}{1.700000in}}%
\pgfusepath{clip}%
\pgfsetbuttcap%
\pgfsetroundjoin%
\definecolor{currentfill}{rgb}{0.976961,0.885681,0.814303}%
\pgfsetfillcolor{currentfill}%
\pgfsetlinewidth{0.311001pt}%
\definecolor{currentstroke}{rgb}{1.000000,1.000000,1.000000}%
\pgfsetstrokecolor{currentstroke}%
\pgfsetdash{}{0pt}%
\pgfpathmoveto{\pgfqpoint{5.418177in}{1.479774in}}%
\pgfpathcurveto{\pgfqpoint{5.425310in}{1.479774in}}{\pgfqpoint{5.432151in}{1.482608in}}{\pgfqpoint{5.437195in}{1.487651in}}%
\pgfpathcurveto{\pgfqpoint{5.442239in}{1.492695in}}{\pgfqpoint{5.445072in}{1.499537in}}{\pgfqpoint{5.445072in}{1.506669in}}%
\pgfpathcurveto{\pgfqpoint{5.445072in}{1.513802in}}{\pgfqpoint{5.442239in}{1.520644in}}{\pgfqpoint{5.437195in}{1.525688in}}%
\pgfpathcurveto{\pgfqpoint{5.432151in}{1.530731in}}{\pgfqpoint{5.425310in}{1.533565in}}{\pgfqpoint{5.418177in}{1.533565in}}%
\pgfpathcurveto{\pgfqpoint{5.411044in}{1.533565in}}{\pgfqpoint{5.404202in}{1.530731in}}{\pgfqpoint{5.399159in}{1.525688in}}%
\pgfpathcurveto{\pgfqpoint{5.394115in}{1.520644in}}{\pgfqpoint{5.391281in}{1.513802in}}{\pgfqpoint{5.391281in}{1.506669in}}%
\pgfpathcurveto{\pgfqpoint{5.391281in}{1.499537in}}{\pgfqpoint{5.394115in}{1.492695in}}{\pgfqpoint{5.399159in}{1.487651in}}%
\pgfpathcurveto{\pgfqpoint{5.404202in}{1.482608in}}{\pgfqpoint{5.411044in}{1.479774in}}{\pgfqpoint{5.418177in}{1.479774in}}%
\pgfpathclose%
\pgfusepath{stroke,fill}%
\end{pgfscope}%
\begin{pgfscope}%
\pgfpathrectangle{\pgfqpoint{4.985294in}{0.500000in}}{\pgfqpoint{1.764706in}{1.700000in}}%
\pgfusepath{clip}%
\pgfsetbuttcap%
\pgfsetroundjoin%
\definecolor{currentfill}{rgb}{0.978376,0.897317,0.831308}%
\pgfsetfillcolor{currentfill}%
\pgfsetlinewidth{0.311001pt}%
\definecolor{currentstroke}{rgb}{1.000000,1.000000,1.000000}%
\pgfsetstrokecolor{currentstroke}%
\pgfsetdash{}{0pt}%
\pgfpathmoveto{\pgfqpoint{6.332915in}{1.216967in}}%
\pgfpathcurveto{\pgfqpoint{6.340048in}{1.216967in}}{\pgfqpoint{6.346889in}{1.219801in}}{\pgfqpoint{6.351933in}{1.224845in}}%
\pgfpathcurveto{\pgfqpoint{6.356977in}{1.229889in}}{\pgfqpoint{6.359811in}{1.236730in}}{\pgfqpoint{6.359811in}{1.243863in}}%
\pgfpathcurveto{\pgfqpoint{6.359811in}{1.250996in}}{\pgfqpoint{6.356977in}{1.257838in}}{\pgfqpoint{6.351933in}{1.262881in}}%
\pgfpathcurveto{\pgfqpoint{6.346889in}{1.267925in}}{\pgfqpoint{6.340048in}{1.270759in}}{\pgfqpoint{6.332915in}{1.270759in}}%
\pgfpathcurveto{\pgfqpoint{6.325782in}{1.270759in}}{\pgfqpoint{6.318940in}{1.267925in}}{\pgfqpoint{6.313897in}{1.262881in}}%
\pgfpathcurveto{\pgfqpoint{6.308853in}{1.257838in}}{\pgfqpoint{6.306019in}{1.250996in}}{\pgfqpoint{6.306019in}{1.243863in}}%
\pgfpathcurveto{\pgfqpoint{6.306019in}{1.236730in}}{\pgfqpoint{6.308853in}{1.229889in}}{\pgfqpoint{6.313897in}{1.224845in}}%
\pgfpathcurveto{\pgfqpoint{6.318940in}{1.219801in}}{\pgfqpoint{6.325782in}{1.216967in}}{\pgfqpoint{6.332915in}{1.216967in}}%
\pgfpathclose%
\pgfusepath{stroke,fill}%
\end{pgfscope}%
\begin{pgfscope}%
\pgfpathrectangle{\pgfqpoint{4.985294in}{0.500000in}}{\pgfqpoint{1.764706in}{1.700000in}}%
\pgfusepath{clip}%
\pgfsetbuttcap%
\pgfsetroundjoin%
\definecolor{currentfill}{rgb}{0.965753,0.732351,0.592427}%
\pgfsetfillcolor{currentfill}%
\pgfsetlinewidth{0.311001pt}%
\definecolor{currentstroke}{rgb}{1.000000,1.000000,1.000000}%
\pgfsetstrokecolor{currentstroke}%
\pgfsetdash{}{0pt}%
\pgfpathmoveto{\pgfqpoint{6.287577in}{0.989216in}}%
\pgfpathcurveto{\pgfqpoint{6.294710in}{0.989216in}}{\pgfqpoint{6.301552in}{0.992050in}}{\pgfqpoint{6.306595in}{0.997093in}}%
\pgfpathcurveto{\pgfqpoint{6.311639in}{1.002137in}}{\pgfqpoint{6.314473in}{1.008979in}}{\pgfqpoint{6.314473in}{1.016111in}}%
\pgfpathcurveto{\pgfqpoint{6.314473in}{1.023244in}}{\pgfqpoint{6.311639in}{1.030086in}}{\pgfqpoint{6.306595in}{1.035129in}}%
\pgfpathcurveto{\pgfqpoint{6.301552in}{1.040173in}}{\pgfqpoint{6.294710in}{1.043007in}}{\pgfqpoint{6.287577in}{1.043007in}}%
\pgfpathcurveto{\pgfqpoint{6.280444in}{1.043007in}}{\pgfqpoint{6.273603in}{1.040173in}}{\pgfqpoint{6.268559in}{1.035129in}}%
\pgfpathcurveto{\pgfqpoint{6.263515in}{1.030086in}}{\pgfqpoint{6.260682in}{1.023244in}}{\pgfqpoint{6.260682in}{1.016111in}}%
\pgfpathcurveto{\pgfqpoint{6.260682in}{1.008979in}}{\pgfqpoint{6.263515in}{1.002137in}}{\pgfqpoint{6.268559in}{0.997093in}}%
\pgfpathcurveto{\pgfqpoint{6.273603in}{0.992050in}}{\pgfqpoint{6.280444in}{0.989216in}}{\pgfqpoint{6.287577in}{0.989216in}}%
\pgfpathclose%
\pgfusepath{stroke,fill}%
\end{pgfscope}%
\begin{pgfscope}%
\pgfpathrectangle{\pgfqpoint{4.985294in}{0.500000in}}{\pgfqpoint{1.764706in}{1.700000in}}%
\pgfusepath{clip}%
\pgfsetbuttcap%
\pgfsetroundjoin%
\definecolor{currentfill}{rgb}{0.979124,0.903132,0.839793}%
\pgfsetfillcolor{currentfill}%
\pgfsetlinewidth{0.311001pt}%
\definecolor{currentstroke}{rgb}{1.000000,1.000000,1.000000}%
\pgfsetstrokecolor{currentstroke}%
\pgfsetdash{}{0pt}%
\pgfpathmoveto{\pgfqpoint{5.440985in}{1.194608in}}%
\pgfpathcurveto{\pgfqpoint{5.448118in}{1.194608in}}{\pgfqpoint{5.454960in}{1.197442in}}{\pgfqpoint{5.460003in}{1.202485in}}%
\pgfpathcurveto{\pgfqpoint{5.465047in}{1.207529in}}{\pgfqpoint{5.467881in}{1.214371in}}{\pgfqpoint{5.467881in}{1.221504in}}%
\pgfpathcurveto{\pgfqpoint{5.467881in}{1.228636in}}{\pgfqpoint{5.465047in}{1.235478in}}{\pgfqpoint{5.460003in}{1.240522in}}%
\pgfpathcurveto{\pgfqpoint{5.454960in}{1.245565in}}{\pgfqpoint{5.448118in}{1.248399in}}{\pgfqpoint{5.440985in}{1.248399in}}%
\pgfpathcurveto{\pgfqpoint{5.433852in}{1.248399in}}{\pgfqpoint{5.427011in}{1.245565in}}{\pgfqpoint{5.421967in}{1.240522in}}%
\pgfpathcurveto{\pgfqpoint{5.416923in}{1.235478in}}{\pgfqpoint{5.414089in}{1.228636in}}{\pgfqpoint{5.414089in}{1.221504in}}%
\pgfpathcurveto{\pgfqpoint{5.414089in}{1.214371in}}{\pgfqpoint{5.416923in}{1.207529in}}{\pgfqpoint{5.421967in}{1.202485in}}%
\pgfpathcurveto{\pgfqpoint{5.427011in}{1.197442in}}{\pgfqpoint{5.433852in}{1.194608in}}{\pgfqpoint{5.440985in}{1.194608in}}%
\pgfpathclose%
\pgfusepath{stroke,fill}%
\end{pgfscope}%
\begin{pgfscope}%
\pgfpathrectangle{\pgfqpoint{4.985294in}{0.500000in}}{\pgfqpoint{1.764706in}{1.700000in}}%
\pgfusepath{clip}%
\pgfsetbuttcap%
\pgfsetroundjoin%
\definecolor{currentfill}{rgb}{0.965928,0.738443,0.600540}%
\pgfsetfillcolor{currentfill}%
\pgfsetlinewidth{0.311001pt}%
\definecolor{currentstroke}{rgb}{1.000000,1.000000,1.000000}%
\pgfsetstrokecolor{currentstroke}%
\pgfsetdash{}{0pt}%
\pgfpathmoveto{\pgfqpoint{5.507875in}{1.707853in}}%
\pgfpathcurveto{\pgfqpoint{5.515008in}{1.707853in}}{\pgfqpoint{5.521850in}{1.710686in}}{\pgfqpoint{5.526893in}{1.715730in}}%
\pgfpathcurveto{\pgfqpoint{5.531937in}{1.720774in}}{\pgfqpoint{5.534771in}{1.727615in}}{\pgfqpoint{5.534771in}{1.734748in}}%
\pgfpathcurveto{\pgfqpoint{5.534771in}{1.741881in}}{\pgfqpoint{5.531937in}{1.748723in}}{\pgfqpoint{5.526893in}{1.753766in}}%
\pgfpathcurveto{\pgfqpoint{5.521850in}{1.758810in}}{\pgfqpoint{5.515008in}{1.761644in}}{\pgfqpoint{5.507875in}{1.761644in}}%
\pgfpathcurveto{\pgfqpoint{5.500743in}{1.761644in}}{\pgfqpoint{5.493901in}{1.758810in}}{\pgfqpoint{5.488857in}{1.753766in}}%
\pgfpathcurveto{\pgfqpoint{5.483814in}{1.748723in}}{\pgfqpoint{5.480980in}{1.741881in}}{\pgfqpoint{5.480980in}{1.734748in}}%
\pgfpathcurveto{\pgfqpoint{5.480980in}{1.727615in}}{\pgfqpoint{5.483814in}{1.720774in}}{\pgfqpoint{5.488857in}{1.715730in}}%
\pgfpathcurveto{\pgfqpoint{5.493901in}{1.710686in}}{\pgfqpoint{5.500743in}{1.707853in}}{\pgfqpoint{5.507875in}{1.707853in}}%
\pgfpathclose%
\pgfusepath{stroke,fill}%
\end{pgfscope}%
\begin{pgfscope}%
\pgfpathrectangle{\pgfqpoint{4.985294in}{0.500000in}}{\pgfqpoint{1.764706in}{1.700000in}}%
\pgfusepath{clip}%
\pgfsetbuttcap%
\pgfsetroundjoin%
\definecolor{currentfill}{rgb}{0.883342,0.198306,0.260142}%
\pgfsetfillcolor{currentfill}%
\pgfsetlinewidth{0.311001pt}%
\definecolor{currentstroke}{rgb}{1.000000,1.000000,1.000000}%
\pgfsetstrokecolor{currentstroke}%
\pgfsetdash{}{0pt}%
\pgfpathmoveto{\pgfqpoint{5.704487in}{1.021549in}}%
\pgfpathcurveto{\pgfqpoint{5.711620in}{1.021549in}}{\pgfqpoint{5.718462in}{1.024382in}}{\pgfqpoint{5.723505in}{1.029426in}}%
\pgfpathcurveto{\pgfqpoint{5.728549in}{1.034470in}}{\pgfqpoint{5.731383in}{1.041311in}}{\pgfqpoint{5.731383in}{1.048444in}}%
\pgfpathcurveto{\pgfqpoint{5.731383in}{1.055577in}}{\pgfqpoint{5.728549in}{1.062419in}}{\pgfqpoint{5.723505in}{1.067462in}}%
\pgfpathcurveto{\pgfqpoint{5.718462in}{1.072506in}}{\pgfqpoint{5.711620in}{1.075340in}}{\pgfqpoint{5.704487in}{1.075340in}}%
\pgfpathcurveto{\pgfqpoint{5.697354in}{1.075340in}}{\pgfqpoint{5.690513in}{1.072506in}}{\pgfqpoint{5.685469in}{1.067462in}}%
\pgfpathcurveto{\pgfqpoint{5.680425in}{1.062419in}}{\pgfqpoint{5.677591in}{1.055577in}}{\pgfqpoint{5.677591in}{1.048444in}}%
\pgfpathcurveto{\pgfqpoint{5.677591in}{1.041311in}}{\pgfqpoint{5.680425in}{1.034470in}}{\pgfqpoint{5.685469in}{1.029426in}}%
\pgfpathcurveto{\pgfqpoint{5.690513in}{1.024382in}}{\pgfqpoint{5.697354in}{1.021549in}}{\pgfqpoint{5.704487in}{1.021549in}}%
\pgfpathclose%
\pgfusepath{stroke,fill}%
\end{pgfscope}%
\begin{pgfscope}%
\pgfpathrectangle{\pgfqpoint{4.985294in}{0.500000in}}{\pgfqpoint{1.764706in}{1.700000in}}%
\pgfusepath{clip}%
\pgfsetbuttcap%
\pgfsetroundjoin%
\definecolor{currentfill}{rgb}{0.975644,0.874038,0.797253}%
\pgfsetfillcolor{currentfill}%
\pgfsetlinewidth{0.311001pt}%
\definecolor{currentstroke}{rgb}{1.000000,1.000000,1.000000}%
\pgfsetstrokecolor{currentstroke}%
\pgfsetdash{}{0pt}%
\pgfpathmoveto{\pgfqpoint{6.362449in}{1.324895in}}%
\pgfpathcurveto{\pgfqpoint{6.369581in}{1.324895in}}{\pgfqpoint{6.376423in}{1.327728in}}{\pgfqpoint{6.381467in}{1.332772in}}%
\pgfpathcurveto{\pgfqpoint{6.386510in}{1.337816in}}{\pgfqpoint{6.389344in}{1.344657in}}{\pgfqpoint{6.389344in}{1.351790in}}%
\pgfpathcurveto{\pgfqpoint{6.389344in}{1.358923in}}{\pgfqpoint{6.386510in}{1.365765in}}{\pgfqpoint{6.381467in}{1.370808in}}%
\pgfpathcurveto{\pgfqpoint{6.376423in}{1.375852in}}{\pgfqpoint{6.369581in}{1.378686in}}{\pgfqpoint{6.362449in}{1.378686in}}%
\pgfpathcurveto{\pgfqpoint{6.355316in}{1.378686in}}{\pgfqpoint{6.348474in}{1.375852in}}{\pgfqpoint{6.343430in}{1.370808in}}%
\pgfpathcurveto{\pgfqpoint{6.338387in}{1.365765in}}{\pgfqpoint{6.335553in}{1.358923in}}{\pgfqpoint{6.335553in}{1.351790in}}%
\pgfpathcurveto{\pgfqpoint{6.335553in}{1.344657in}}{\pgfqpoint{6.338387in}{1.337816in}}{\pgfqpoint{6.343430in}{1.332772in}}%
\pgfpathcurveto{\pgfqpoint{6.348474in}{1.327728in}}{\pgfqpoint{6.355316in}{1.324895in}}{\pgfqpoint{6.362449in}{1.324895in}}%
\pgfpathclose%
\pgfusepath{stroke,fill}%
\end{pgfscope}%
\begin{pgfscope}%
\pgfpathrectangle{\pgfqpoint{4.985294in}{0.500000in}}{\pgfqpoint{1.764706in}{1.700000in}}%
\pgfusepath{clip}%
\pgfsetbuttcap%
\pgfsetroundjoin%
\definecolor{currentfill}{rgb}{0.953126,0.456614,0.312398}%
\pgfsetfillcolor{currentfill}%
\pgfsetlinewidth{0.311001pt}%
\definecolor{currentstroke}{rgb}{1.000000,1.000000,1.000000}%
\pgfsetstrokecolor{currentstroke}%
\pgfsetdash{}{0pt}%
\pgfpathmoveto{\pgfqpoint{5.567241in}{1.331936in}}%
\pgfpathcurveto{\pgfqpoint{5.574373in}{1.331936in}}{\pgfqpoint{5.581215in}{1.334770in}}{\pgfqpoint{5.586259in}{1.339813in}}%
\pgfpathcurveto{\pgfqpoint{5.591302in}{1.344857in}}{\pgfqpoint{5.594136in}{1.351698in}}{\pgfqpoint{5.594136in}{1.358831in}}%
\pgfpathcurveto{\pgfqpoint{5.594136in}{1.365964in}}{\pgfqpoint{5.591302in}{1.372806in}}{\pgfqpoint{5.586259in}{1.377849in}}%
\pgfpathcurveto{\pgfqpoint{5.581215in}{1.382893in}}{\pgfqpoint{5.574373in}{1.385727in}}{\pgfqpoint{5.567241in}{1.385727in}}%
\pgfpathcurveto{\pgfqpoint{5.560108in}{1.385727in}}{\pgfqpoint{5.553266in}{1.382893in}}{\pgfqpoint{5.548222in}{1.377849in}}%
\pgfpathcurveto{\pgfqpoint{5.543179in}{1.372806in}}{\pgfqpoint{5.540345in}{1.365964in}}{\pgfqpoint{5.540345in}{1.358831in}}%
\pgfpathcurveto{\pgfqpoint{5.540345in}{1.351698in}}{\pgfqpoint{5.543179in}{1.344857in}}{\pgfqpoint{5.548222in}{1.339813in}}%
\pgfpathcurveto{\pgfqpoint{5.553266in}{1.334770in}}{\pgfqpoint{5.560108in}{1.331936in}}{\pgfqpoint{5.567241in}{1.331936in}}%
\pgfpathclose%
\pgfusepath{stroke,fill}%
\end{pgfscope}%
\begin{pgfscope}%
\pgfpathrectangle{\pgfqpoint{4.985294in}{0.500000in}}{\pgfqpoint{1.764706in}{1.700000in}}%
\pgfusepath{clip}%
\pgfsetbuttcap%
\pgfsetroundjoin%
\definecolor{currentfill}{rgb}{0.966560,0.756582,0.625273}%
\pgfsetfillcolor{currentfill}%
\pgfsetlinewidth{0.311001pt}%
\definecolor{currentstroke}{rgb}{1.000000,1.000000,1.000000}%
\pgfsetstrokecolor{currentstroke}%
\pgfsetdash{}{0pt}%
\pgfpathmoveto{\pgfqpoint{5.354021in}{1.169040in}}%
\pgfpathcurveto{\pgfqpoint{5.361154in}{1.169040in}}{\pgfqpoint{5.367996in}{1.171874in}}{\pgfqpoint{5.373039in}{1.176918in}}%
\pgfpathcurveto{\pgfqpoint{5.378083in}{1.181962in}}{\pgfqpoint{5.380917in}{1.188803in}}{\pgfqpoint{5.380917in}{1.195936in}}%
\pgfpathcurveto{\pgfqpoint{5.380917in}{1.203069in}}{\pgfqpoint{5.378083in}{1.209911in}}{\pgfqpoint{5.373039in}{1.214954in}}%
\pgfpathcurveto{\pgfqpoint{5.367996in}{1.219998in}}{\pgfqpoint{5.361154in}{1.222832in}}{\pgfqpoint{5.354021in}{1.222832in}}%
\pgfpathcurveto{\pgfqpoint{5.346888in}{1.222832in}}{\pgfqpoint{5.340047in}{1.219998in}}{\pgfqpoint{5.335003in}{1.214954in}}%
\pgfpathcurveto{\pgfqpoint{5.329959in}{1.209911in}}{\pgfqpoint{5.327125in}{1.203069in}}{\pgfqpoint{5.327125in}{1.195936in}}%
\pgfpathcurveto{\pgfqpoint{5.327125in}{1.188803in}}{\pgfqpoint{5.329959in}{1.181962in}}{\pgfqpoint{5.335003in}{1.176918in}}%
\pgfpathcurveto{\pgfqpoint{5.340047in}{1.171874in}}{\pgfqpoint{5.346888in}{1.169040in}}{\pgfqpoint{5.354021in}{1.169040in}}%
\pgfpathclose%
\pgfusepath{stroke,fill}%
\end{pgfscope}%
\begin{pgfscope}%
\pgfpathrectangle{\pgfqpoint{4.985294in}{0.500000in}}{\pgfqpoint{1.764706in}{1.700000in}}%
\pgfusepath{clip}%
\pgfsetbuttcap%
\pgfsetroundjoin%
\definecolor{currentfill}{rgb}{0.967092,0.768560,0.642079}%
\pgfsetfillcolor{currentfill}%
\pgfsetlinewidth{0.311001pt}%
\definecolor{currentstroke}{rgb}{1.000000,1.000000,1.000000}%
\pgfsetstrokecolor{currentstroke}%
\pgfsetdash{}{0pt}%
\pgfpathmoveto{\pgfqpoint{5.506975in}{1.339796in}}%
\pgfpathcurveto{\pgfqpoint{5.514108in}{1.339796in}}{\pgfqpoint{5.520949in}{1.342630in}}{\pgfqpoint{5.525993in}{1.347674in}}%
\pgfpathcurveto{\pgfqpoint{5.531037in}{1.352717in}}{\pgfqpoint{5.533871in}{1.359559in}}{\pgfqpoint{5.533871in}{1.366692in}}%
\pgfpathcurveto{\pgfqpoint{5.533871in}{1.373825in}}{\pgfqpoint{5.531037in}{1.380666in}}{\pgfqpoint{5.525993in}{1.385710in}}%
\pgfpathcurveto{\pgfqpoint{5.520949in}{1.390754in}}{\pgfqpoint{5.514108in}{1.393588in}}{\pgfqpoint{5.506975in}{1.393588in}}%
\pgfpathcurveto{\pgfqpoint{5.499842in}{1.393588in}}{\pgfqpoint{5.493000in}{1.390754in}}{\pgfqpoint{5.487957in}{1.385710in}}%
\pgfpathcurveto{\pgfqpoint{5.482913in}{1.380666in}}{\pgfqpoint{5.480079in}{1.373825in}}{\pgfqpoint{5.480079in}{1.366692in}}%
\pgfpathcurveto{\pgfqpoint{5.480079in}{1.359559in}}{\pgfqpoint{5.482913in}{1.352717in}}{\pgfqpoint{5.487957in}{1.347674in}}%
\pgfpathcurveto{\pgfqpoint{5.493000in}{1.342630in}}{\pgfqpoint{5.499842in}{1.339796in}}{\pgfqpoint{5.506975in}{1.339796in}}%
\pgfpathclose%
\pgfusepath{stroke,fill}%
\end{pgfscope}%
\begin{pgfscope}%
\pgfpathrectangle{\pgfqpoint{4.985294in}{0.500000in}}{\pgfqpoint{1.764706in}{1.700000in}}%
\pgfusepath{clip}%
\pgfsetbuttcap%
\pgfsetroundjoin%
\definecolor{currentfill}{rgb}{0.970718,0.821518,0.719872}%
\pgfsetfillcolor{currentfill}%
\pgfsetlinewidth{0.311001pt}%
\definecolor{currentstroke}{rgb}{1.000000,1.000000,1.000000}%
\pgfsetstrokecolor{currentstroke}%
\pgfsetdash{}{0pt}%
\pgfpathmoveto{\pgfqpoint{6.284664in}{1.040486in}}%
\pgfpathcurveto{\pgfqpoint{6.291797in}{1.040486in}}{\pgfqpoint{6.298638in}{1.043320in}}{\pgfqpoint{6.303682in}{1.048364in}}%
\pgfpathcurveto{\pgfqpoint{6.308726in}{1.053408in}}{\pgfqpoint{6.311560in}{1.060249in}}{\pgfqpoint{6.311560in}{1.067382in}}%
\pgfpathcurveto{\pgfqpoint{6.311560in}{1.074515in}}{\pgfqpoint{6.308726in}{1.081357in}}{\pgfqpoint{6.303682in}{1.086400in}}%
\pgfpathcurveto{\pgfqpoint{6.298638in}{1.091444in}}{\pgfqpoint{6.291797in}{1.094278in}}{\pgfqpoint{6.284664in}{1.094278in}}%
\pgfpathcurveto{\pgfqpoint{6.277531in}{1.094278in}}{\pgfqpoint{6.270690in}{1.091444in}}{\pgfqpoint{6.265646in}{1.086400in}}%
\pgfpathcurveto{\pgfqpoint{6.260602in}{1.081357in}}{\pgfqpoint{6.257768in}{1.074515in}}{\pgfqpoint{6.257768in}{1.067382in}}%
\pgfpathcurveto{\pgfqpoint{6.257768in}{1.060249in}}{\pgfqpoint{6.260602in}{1.053408in}}{\pgfqpoint{6.265646in}{1.048364in}}%
\pgfpathcurveto{\pgfqpoint{6.270690in}{1.043320in}}{\pgfqpoint{6.277531in}{1.040486in}}{\pgfqpoint{6.284664in}{1.040486in}}%
\pgfpathclose%
\pgfusepath{stroke,fill}%
\end{pgfscope}%
\begin{pgfscope}%
\pgfpathrectangle{\pgfqpoint{4.985294in}{0.500000in}}{\pgfqpoint{1.764706in}{1.700000in}}%
\pgfusepath{clip}%
\pgfsetbuttcap%
\pgfsetroundjoin%
\definecolor{currentfill}{rgb}{0.966812,0.762584,0.633643}%
\pgfsetfillcolor{currentfill}%
\pgfsetlinewidth{0.311001pt}%
\definecolor{currentstroke}{rgb}{1.000000,1.000000,1.000000}%
\pgfsetstrokecolor{currentstroke}%
\pgfsetdash{}{0pt}%
\pgfpathmoveto{\pgfqpoint{5.506855in}{1.261850in}}%
\pgfpathcurveto{\pgfqpoint{5.513988in}{1.261850in}}{\pgfqpoint{5.520829in}{1.264684in}}{\pgfqpoint{5.525873in}{1.269728in}}%
\pgfpathcurveto{\pgfqpoint{5.530917in}{1.274771in}}{\pgfqpoint{5.533750in}{1.281613in}}{\pgfqpoint{5.533750in}{1.288746in}}%
\pgfpathcurveto{\pgfqpoint{5.533750in}{1.295879in}}{\pgfqpoint{5.530917in}{1.302720in}}{\pgfqpoint{5.525873in}{1.307764in}}%
\pgfpathcurveto{\pgfqpoint{5.520829in}{1.312808in}}{\pgfqpoint{5.513988in}{1.315642in}}{\pgfqpoint{5.506855in}{1.315642in}}%
\pgfpathcurveto{\pgfqpoint{5.499722in}{1.315642in}}{\pgfqpoint{5.492880in}{1.312808in}}{\pgfqpoint{5.487837in}{1.307764in}}%
\pgfpathcurveto{\pgfqpoint{5.482793in}{1.302720in}}{\pgfqpoint{5.479959in}{1.295879in}}{\pgfqpoint{5.479959in}{1.288746in}}%
\pgfpathcurveto{\pgfqpoint{5.479959in}{1.281613in}}{\pgfqpoint{5.482793in}{1.274771in}}{\pgfqpoint{5.487837in}{1.269728in}}%
\pgfpathcurveto{\pgfqpoint{5.492880in}{1.264684in}}{\pgfqpoint{5.499722in}{1.261850in}}{\pgfqpoint{5.506855in}{1.261850in}}%
\pgfpathclose%
\pgfusepath{stroke,fill}%
\end{pgfscope}%
\begin{pgfscope}%
\pgfpathrectangle{\pgfqpoint{4.985294in}{0.500000in}}{\pgfqpoint{1.764706in}{1.700000in}}%
\pgfusepath{clip}%
\pgfsetbuttcap%
\pgfsetroundjoin%
\definecolor{currentfill}{rgb}{0.976961,0.885681,0.814303}%
\pgfsetfillcolor{currentfill}%
\pgfsetlinewidth{0.311001pt}%
\definecolor{currentstroke}{rgb}{1.000000,1.000000,1.000000}%
\pgfsetstrokecolor{currentstroke}%
\pgfsetdash{}{0pt}%
\pgfpathmoveto{\pgfqpoint{5.466032in}{1.122792in}}%
\pgfpathcurveto{\pgfqpoint{5.473165in}{1.122792in}}{\pgfqpoint{5.480007in}{1.125625in}}{\pgfqpoint{5.485050in}{1.130669in}}%
\pgfpathcurveto{\pgfqpoint{5.490094in}{1.135713in}}{\pgfqpoint{5.492928in}{1.142554in}}{\pgfqpoint{5.492928in}{1.149687in}}%
\pgfpathcurveto{\pgfqpoint{5.492928in}{1.156820in}}{\pgfqpoint{5.490094in}{1.163662in}}{\pgfqpoint{5.485050in}{1.168705in}}%
\pgfpathcurveto{\pgfqpoint{5.480007in}{1.173749in}}{\pgfqpoint{5.473165in}{1.176583in}}{\pgfqpoint{5.466032in}{1.176583in}}%
\pgfpathcurveto{\pgfqpoint{5.458899in}{1.176583in}}{\pgfqpoint{5.452058in}{1.173749in}}{\pgfqpoint{5.447014in}{1.168705in}}%
\pgfpathcurveto{\pgfqpoint{5.441970in}{1.163662in}}{\pgfqpoint{5.439136in}{1.156820in}}{\pgfqpoint{5.439136in}{1.149687in}}%
\pgfpathcurveto{\pgfqpoint{5.439136in}{1.142554in}}{\pgfqpoint{5.441970in}{1.135713in}}{\pgfqpoint{5.447014in}{1.130669in}}%
\pgfpathcurveto{\pgfqpoint{5.452058in}{1.125625in}}{\pgfqpoint{5.458899in}{1.122792in}}{\pgfqpoint{5.466032in}{1.122792in}}%
\pgfpathclose%
\pgfusepath{stroke,fill}%
\end{pgfscope}%
\begin{pgfscope}%
\pgfpathrectangle{\pgfqpoint{4.985294in}{0.500000in}}{\pgfqpoint{1.764706in}{1.700000in}}%
\pgfusepath{clip}%
\pgfsetbuttcap%
\pgfsetroundjoin%
\definecolor{currentfill}{rgb}{0.972726,0.844889,0.754401}%
\pgfsetfillcolor{currentfill}%
\pgfsetlinewidth{0.311001pt}%
\definecolor{currentstroke}{rgb}{1.000000,1.000000,1.000000}%
\pgfsetstrokecolor{currentstroke}%
\pgfsetdash{}{0pt}%
\pgfpathmoveto{\pgfqpoint{6.219376in}{1.140878in}}%
\pgfpathcurveto{\pgfqpoint{6.226509in}{1.140878in}}{\pgfqpoint{6.233351in}{1.143712in}}{\pgfqpoint{6.238395in}{1.148756in}}%
\pgfpathcurveto{\pgfqpoint{6.243438in}{1.153799in}}{\pgfqpoint{6.246272in}{1.160641in}}{\pgfqpoint{6.246272in}{1.167774in}}%
\pgfpathcurveto{\pgfqpoint{6.246272in}{1.174907in}}{\pgfqpoint{6.243438in}{1.181748in}}{\pgfqpoint{6.238395in}{1.186792in}}%
\pgfpathcurveto{\pgfqpoint{6.233351in}{1.191836in}}{\pgfqpoint{6.226509in}{1.194669in}}{\pgfqpoint{6.219376in}{1.194669in}}%
\pgfpathcurveto{\pgfqpoint{6.212244in}{1.194669in}}{\pgfqpoint{6.205402in}{1.191836in}}{\pgfqpoint{6.200358in}{1.186792in}}%
\pgfpathcurveto{\pgfqpoint{6.195315in}{1.181748in}}{\pgfqpoint{6.192481in}{1.174907in}}{\pgfqpoint{6.192481in}{1.167774in}}%
\pgfpathcurveto{\pgfqpoint{6.192481in}{1.160641in}}{\pgfqpoint{6.195315in}{1.153799in}}{\pgfqpoint{6.200358in}{1.148756in}}%
\pgfpathcurveto{\pgfqpoint{6.205402in}{1.143712in}}{\pgfqpoint{6.212244in}{1.140878in}}{\pgfqpoint{6.219376in}{1.140878in}}%
\pgfpathclose%
\pgfusepath{stroke,fill}%
\end{pgfscope}%
\begin{pgfscope}%
\pgfpathrectangle{\pgfqpoint{4.985294in}{0.500000in}}{\pgfqpoint{1.764706in}{1.700000in}}%
\pgfusepath{clip}%
\pgfsetbuttcap%
\pgfsetroundjoin%
\definecolor{currentfill}{rgb}{0.962985,0.612625,0.451451}%
\pgfsetfillcolor{currentfill}%
\pgfsetlinewidth{0.311001pt}%
\definecolor{currentstroke}{rgb}{1.000000,1.000000,1.000000}%
\pgfsetstrokecolor{currentstroke}%
\pgfsetdash{}{0pt}%
\pgfpathmoveto{\pgfqpoint{5.610750in}{1.602596in}}%
\pgfpathcurveto{\pgfqpoint{5.617883in}{1.602596in}}{\pgfqpoint{5.624724in}{1.605430in}}{\pgfqpoint{5.629768in}{1.610473in}}%
\pgfpathcurveto{\pgfqpoint{5.634812in}{1.615517in}}{\pgfqpoint{5.637646in}{1.622359in}}{\pgfqpoint{5.637646in}{1.629491in}}%
\pgfpathcurveto{\pgfqpoint{5.637646in}{1.636624in}}{\pgfqpoint{5.634812in}{1.643466in}}{\pgfqpoint{5.629768in}{1.648510in}}%
\pgfpathcurveto{\pgfqpoint{5.624724in}{1.653553in}}{\pgfqpoint{5.617883in}{1.656387in}}{\pgfqpoint{5.610750in}{1.656387in}}%
\pgfpathcurveto{\pgfqpoint{5.603617in}{1.656387in}}{\pgfqpoint{5.596776in}{1.653553in}}{\pgfqpoint{5.591732in}{1.648510in}}%
\pgfpathcurveto{\pgfqpoint{5.586688in}{1.643466in}}{\pgfqpoint{5.583854in}{1.636624in}}{\pgfqpoint{5.583854in}{1.629491in}}%
\pgfpathcurveto{\pgfqpoint{5.583854in}{1.622359in}}{\pgfqpoint{5.586688in}{1.615517in}}{\pgfqpoint{5.591732in}{1.610473in}}%
\pgfpathcurveto{\pgfqpoint{5.596776in}{1.605430in}}{\pgfqpoint{5.603617in}{1.602596in}}{\pgfqpoint{5.610750in}{1.602596in}}%
\pgfpathclose%
\pgfusepath{stroke,fill}%
\end{pgfscope}%
\begin{pgfscope}%
\pgfpathrectangle{\pgfqpoint{4.985294in}{0.500000in}}{\pgfqpoint{1.764706in}{1.700000in}}%
\pgfusepath{clip}%
\pgfsetbuttcap%
\pgfsetroundjoin%
\definecolor{currentfill}{rgb}{0.965928,0.738443,0.600540}%
\pgfsetfillcolor{currentfill}%
\pgfsetlinewidth{0.311001pt}%
\definecolor{currentstroke}{rgb}{1.000000,1.000000,1.000000}%
\pgfsetstrokecolor{currentstroke}%
\pgfsetdash{}{0pt}%
\pgfpathmoveto{\pgfqpoint{5.540784in}{1.134004in}}%
\pgfpathcurveto{\pgfqpoint{5.547917in}{1.134004in}}{\pgfqpoint{5.554758in}{1.136837in}}{\pgfqpoint{5.559802in}{1.141881in}}%
\pgfpathcurveto{\pgfqpoint{5.564846in}{1.146925in}}{\pgfqpoint{5.567680in}{1.153766in}}{\pgfqpoint{5.567680in}{1.160899in}}%
\pgfpathcurveto{\pgfqpoint{5.567680in}{1.168032in}}{\pgfqpoint{5.564846in}{1.174874in}}{\pgfqpoint{5.559802in}{1.179917in}}%
\pgfpathcurveto{\pgfqpoint{5.554758in}{1.184961in}}{\pgfqpoint{5.547917in}{1.187795in}}{\pgfqpoint{5.540784in}{1.187795in}}%
\pgfpathcurveto{\pgfqpoint{5.533651in}{1.187795in}}{\pgfqpoint{5.526809in}{1.184961in}}{\pgfqpoint{5.521766in}{1.179917in}}%
\pgfpathcurveto{\pgfqpoint{5.516722in}{1.174874in}}{\pgfqpoint{5.513888in}{1.168032in}}{\pgfqpoint{5.513888in}{1.160899in}}%
\pgfpathcurveto{\pgfqpoint{5.513888in}{1.153766in}}{\pgfqpoint{5.516722in}{1.146925in}}{\pgfqpoint{5.521766in}{1.141881in}}%
\pgfpathcurveto{\pgfqpoint{5.526809in}{1.136837in}}{\pgfqpoint{5.533651in}{1.134004in}}{\pgfqpoint{5.540784in}{1.134004in}}%
\pgfpathclose%
\pgfusepath{stroke,fill}%
\end{pgfscope}%
\begin{pgfscope}%
\pgfpathrectangle{\pgfqpoint{4.985294in}{0.500000in}}{\pgfqpoint{1.764706in}{1.700000in}}%
\pgfusepath{clip}%
\pgfsetbuttcap%
\pgfsetroundjoin%
\definecolor{currentfill}{rgb}{0.966120,0.744512,0.608720}%
\pgfsetfillcolor{currentfill}%
\pgfsetlinewidth{0.311001pt}%
\definecolor{currentstroke}{rgb}{1.000000,1.000000,1.000000}%
\pgfsetstrokecolor{currentstroke}%
\pgfsetdash{}{0pt}%
\pgfpathmoveto{\pgfqpoint{6.202624in}{1.263379in}}%
\pgfpathcurveto{\pgfqpoint{6.209757in}{1.263379in}}{\pgfqpoint{6.216598in}{1.266213in}}{\pgfqpoint{6.221642in}{1.271256in}}%
\pgfpathcurveto{\pgfqpoint{6.226686in}{1.276300in}}{\pgfqpoint{6.229520in}{1.283142in}}{\pgfqpoint{6.229520in}{1.290274in}}%
\pgfpathcurveto{\pgfqpoint{6.229520in}{1.297407in}}{\pgfqpoint{6.226686in}{1.304249in}}{\pgfqpoint{6.221642in}{1.309293in}}%
\pgfpathcurveto{\pgfqpoint{6.216598in}{1.314336in}}{\pgfqpoint{6.209757in}{1.317170in}}{\pgfqpoint{6.202624in}{1.317170in}}%
\pgfpathcurveto{\pgfqpoint{6.195491in}{1.317170in}}{\pgfqpoint{6.188650in}{1.314336in}}{\pgfqpoint{6.183606in}{1.309293in}}%
\pgfpathcurveto{\pgfqpoint{6.178562in}{1.304249in}}{\pgfqpoint{6.175728in}{1.297407in}}{\pgfqpoint{6.175728in}{1.290274in}}%
\pgfpathcurveto{\pgfqpoint{6.175728in}{1.283142in}}{\pgfqpoint{6.178562in}{1.276300in}}{\pgfqpoint{6.183606in}{1.271256in}}%
\pgfpathcurveto{\pgfqpoint{6.188650in}{1.266213in}}{\pgfqpoint{6.195491in}{1.263379in}}{\pgfqpoint{6.202624in}{1.263379in}}%
\pgfpathclose%
\pgfusepath{stroke,fill}%
\end{pgfscope}%
\begin{pgfscope}%
\pgfpathrectangle{\pgfqpoint{4.985294in}{0.500000in}}{\pgfqpoint{1.764706in}{1.700000in}}%
\pgfusepath{clip}%
\pgfsetbuttcap%
\pgfsetroundjoin%
\definecolor{currentfill}{rgb}{0.973271,0.850724,0.762998}%
\pgfsetfillcolor{currentfill}%
\pgfsetlinewidth{0.311001pt}%
\definecolor{currentstroke}{rgb}{1.000000,1.000000,1.000000}%
\pgfsetstrokecolor{currentstroke}%
\pgfsetdash{}{0pt}%
\pgfpathmoveto{\pgfqpoint{6.221907in}{1.035759in}}%
\pgfpathcurveto{\pgfqpoint{6.229040in}{1.035759in}}{\pgfqpoint{6.235882in}{1.038593in}}{\pgfqpoint{6.240925in}{1.043637in}}%
\pgfpathcurveto{\pgfqpoint{6.245969in}{1.048680in}}{\pgfqpoint{6.248803in}{1.055522in}}{\pgfqpoint{6.248803in}{1.062655in}}%
\pgfpathcurveto{\pgfqpoint{6.248803in}{1.069788in}}{\pgfqpoint{6.245969in}{1.076629in}}{\pgfqpoint{6.240925in}{1.081673in}}%
\pgfpathcurveto{\pgfqpoint{6.235882in}{1.086717in}}{\pgfqpoint{6.229040in}{1.089551in}}{\pgfqpoint{6.221907in}{1.089551in}}%
\pgfpathcurveto{\pgfqpoint{6.214774in}{1.089551in}}{\pgfqpoint{6.207933in}{1.086717in}}{\pgfqpoint{6.202889in}{1.081673in}}%
\pgfpathcurveto{\pgfqpoint{6.197845in}{1.076629in}}{\pgfqpoint{6.195012in}{1.069788in}}{\pgfqpoint{6.195012in}{1.062655in}}%
\pgfpathcurveto{\pgfqpoint{6.195012in}{1.055522in}}{\pgfqpoint{6.197845in}{1.048680in}}{\pgfqpoint{6.202889in}{1.043637in}}%
\pgfpathcurveto{\pgfqpoint{6.207933in}{1.038593in}}{\pgfqpoint{6.214774in}{1.035759in}}{\pgfqpoint{6.221907in}{1.035759in}}%
\pgfpathclose%
\pgfusepath{stroke,fill}%
\end{pgfscope}%
\begin{pgfscope}%
\pgfpathrectangle{\pgfqpoint{4.985294in}{0.500000in}}{\pgfqpoint{1.764706in}{1.700000in}}%
\pgfusepath{clip}%
\pgfsetbuttcap%
\pgfsetroundjoin%
\definecolor{currentfill}{rgb}{0.969359,0.803954,0.693832}%
\pgfsetfillcolor{currentfill}%
\pgfsetlinewidth{0.311001pt}%
\definecolor{currentstroke}{rgb}{1.000000,1.000000,1.000000}%
\pgfsetstrokecolor{currentstroke}%
\pgfsetdash{}{0pt}%
\pgfpathmoveto{\pgfqpoint{6.179774in}{1.672578in}}%
\pgfpathcurveto{\pgfqpoint{6.186906in}{1.672578in}}{\pgfqpoint{6.193748in}{1.675412in}}{\pgfqpoint{6.198792in}{1.680456in}}%
\pgfpathcurveto{\pgfqpoint{6.203835in}{1.685500in}}{\pgfqpoint{6.206669in}{1.692341in}}{\pgfqpoint{6.206669in}{1.699474in}}%
\pgfpathcurveto{\pgfqpoint{6.206669in}{1.706607in}}{\pgfqpoint{6.203835in}{1.713449in}}{\pgfqpoint{6.198792in}{1.718492in}}%
\pgfpathcurveto{\pgfqpoint{6.193748in}{1.723536in}}{\pgfqpoint{6.186906in}{1.726370in}}{\pgfqpoint{6.179774in}{1.726370in}}%
\pgfpathcurveto{\pgfqpoint{6.172641in}{1.726370in}}{\pgfqpoint{6.165799in}{1.723536in}}{\pgfqpoint{6.160755in}{1.718492in}}%
\pgfpathcurveto{\pgfqpoint{6.155712in}{1.713449in}}{\pgfqpoint{6.152878in}{1.706607in}}{\pgfqpoint{6.152878in}{1.699474in}}%
\pgfpathcurveto{\pgfqpoint{6.152878in}{1.692341in}}{\pgfqpoint{6.155712in}{1.685500in}}{\pgfqpoint{6.160755in}{1.680456in}}%
\pgfpathcurveto{\pgfqpoint{6.165799in}{1.675412in}}{\pgfqpoint{6.172641in}{1.672578in}}{\pgfqpoint{6.179774in}{1.672578in}}%
\pgfpathclose%
\pgfusepath{stroke,fill}%
\end{pgfscope}%
\begin{pgfscope}%
\pgfpathrectangle{\pgfqpoint{4.985294in}{0.500000in}}{\pgfqpoint{1.764706in}{1.700000in}}%
\pgfusepath{clip}%
\pgfsetbuttcap%
\pgfsetroundjoin%
\definecolor{currentfill}{rgb}{0.972726,0.844889,0.754401}%
\pgfsetfillcolor{currentfill}%
\pgfsetlinewidth{0.311001pt}%
\definecolor{currentstroke}{rgb}{1.000000,1.000000,1.000000}%
\pgfsetstrokecolor{currentstroke}%
\pgfsetdash{}{0pt}%
\pgfpathmoveto{\pgfqpoint{6.218376in}{1.549391in}}%
\pgfpathcurveto{\pgfqpoint{6.225509in}{1.549391in}}{\pgfqpoint{6.232351in}{1.552225in}}{\pgfqpoint{6.237394in}{1.557268in}}%
\pgfpathcurveto{\pgfqpoint{6.242438in}{1.562312in}}{\pgfqpoint{6.245272in}{1.569153in}}{\pgfqpoint{6.245272in}{1.576286in}}%
\pgfpathcurveto{\pgfqpoint{6.245272in}{1.583419in}}{\pgfqpoint{6.242438in}{1.590261in}}{\pgfqpoint{6.237394in}{1.595304in}}%
\pgfpathcurveto{\pgfqpoint{6.232351in}{1.600348in}}{\pgfqpoint{6.225509in}{1.603182in}}{\pgfqpoint{6.218376in}{1.603182in}}%
\pgfpathcurveto{\pgfqpoint{6.211243in}{1.603182in}}{\pgfqpoint{6.204402in}{1.600348in}}{\pgfqpoint{6.199358in}{1.595304in}}%
\pgfpathcurveto{\pgfqpoint{6.194314in}{1.590261in}}{\pgfqpoint{6.191481in}{1.583419in}}{\pgfqpoint{6.191481in}{1.576286in}}%
\pgfpathcurveto{\pgfqpoint{6.191481in}{1.569153in}}{\pgfqpoint{6.194314in}{1.562312in}}{\pgfqpoint{6.199358in}{1.557268in}}%
\pgfpathcurveto{\pgfqpoint{6.204402in}{1.552225in}}{\pgfqpoint{6.211243in}{1.549391in}}{\pgfqpoint{6.218376in}{1.549391in}}%
\pgfpathclose%
\pgfusepath{stroke,fill}%
\end{pgfscope}%
\begin{pgfscope}%
\pgfpathrectangle{\pgfqpoint{4.985294in}{0.500000in}}{\pgfqpoint{1.764706in}{1.700000in}}%
\pgfusepath{clip}%
\pgfsetbuttcap%
\pgfsetroundjoin%
\definecolor{currentfill}{rgb}{0.976961,0.885681,0.814303}%
\pgfsetfillcolor{currentfill}%
\pgfsetlinewidth{0.311001pt}%
\definecolor{currentstroke}{rgb}{1.000000,1.000000,1.000000}%
\pgfsetstrokecolor{currentstroke}%
\pgfsetdash{}{0pt}%
\pgfpathmoveto{\pgfqpoint{5.457670in}{1.359346in}}%
\pgfpathcurveto{\pgfqpoint{5.464802in}{1.359346in}}{\pgfqpoint{5.471644in}{1.362180in}}{\pgfqpoint{5.476688in}{1.367224in}}%
\pgfpathcurveto{\pgfqpoint{5.481731in}{1.372268in}}{\pgfqpoint{5.484565in}{1.379109in}}{\pgfqpoint{5.484565in}{1.386242in}}%
\pgfpathcurveto{\pgfqpoint{5.484565in}{1.393375in}}{\pgfqpoint{5.481731in}{1.400217in}}{\pgfqpoint{5.476688in}{1.405260in}}%
\pgfpathcurveto{\pgfqpoint{5.471644in}{1.410304in}}{\pgfqpoint{5.464802in}{1.413138in}}{\pgfqpoint{5.457670in}{1.413138in}}%
\pgfpathcurveto{\pgfqpoint{5.450537in}{1.413138in}}{\pgfqpoint{5.443695in}{1.410304in}}{\pgfqpoint{5.438651in}{1.405260in}}%
\pgfpathcurveto{\pgfqpoint{5.433608in}{1.400217in}}{\pgfqpoint{5.430774in}{1.393375in}}{\pgfqpoint{5.430774in}{1.386242in}}%
\pgfpathcurveto{\pgfqpoint{5.430774in}{1.379109in}}{\pgfqpoint{5.433608in}{1.372268in}}{\pgfqpoint{5.438651in}{1.367224in}}%
\pgfpathcurveto{\pgfqpoint{5.443695in}{1.362180in}}{\pgfqpoint{5.450537in}{1.359346in}}{\pgfqpoint{5.457670in}{1.359346in}}%
\pgfpathclose%
\pgfusepath{stroke,fill}%
\end{pgfscope}%
\begin{pgfscope}%
\pgfpathrectangle{\pgfqpoint{4.985294in}{0.500000in}}{\pgfqpoint{1.764706in}{1.700000in}}%
\pgfusepath{clip}%
\pgfsetbuttcap%
\pgfsetroundjoin%
\definecolor{currentfill}{rgb}{0.963559,0.632016,0.472047}%
\pgfsetfillcolor{currentfill}%
\pgfsetlinewidth{0.311001pt}%
\definecolor{currentstroke}{rgb}{1.000000,1.000000,1.000000}%
\pgfsetstrokecolor{currentstroke}%
\pgfsetdash{}{0pt}%
\pgfpathmoveto{\pgfqpoint{6.105474in}{1.720542in}}%
\pgfpathcurveto{\pgfqpoint{6.112606in}{1.720542in}}{\pgfqpoint{6.119448in}{1.723376in}}{\pgfqpoint{6.124492in}{1.728420in}}%
\pgfpathcurveto{\pgfqpoint{6.129535in}{1.733464in}}{\pgfqpoint{6.132369in}{1.740305in}}{\pgfqpoint{6.132369in}{1.747438in}}%
\pgfpathcurveto{\pgfqpoint{6.132369in}{1.754571in}}{\pgfqpoint{6.129535in}{1.761413in}}{\pgfqpoint{6.124492in}{1.766456in}}%
\pgfpathcurveto{\pgfqpoint{6.119448in}{1.771500in}}{\pgfqpoint{6.112606in}{1.774334in}}{\pgfqpoint{6.105474in}{1.774334in}}%
\pgfpathcurveto{\pgfqpoint{6.098341in}{1.774334in}}{\pgfqpoint{6.091499in}{1.771500in}}{\pgfqpoint{6.086455in}{1.766456in}}%
\pgfpathcurveto{\pgfqpoint{6.081412in}{1.761413in}}{\pgfqpoint{6.078578in}{1.754571in}}{\pgfqpoint{6.078578in}{1.747438in}}%
\pgfpathcurveto{\pgfqpoint{6.078578in}{1.740305in}}{\pgfqpoint{6.081412in}{1.733464in}}{\pgfqpoint{6.086455in}{1.728420in}}%
\pgfpathcurveto{\pgfqpoint{6.091499in}{1.723376in}}{\pgfqpoint{6.098341in}{1.720542in}}{\pgfqpoint{6.105474in}{1.720542in}}%
\pgfpathclose%
\pgfusepath{stroke,fill}%
\end{pgfscope}%
\begin{pgfscope}%
\pgfpathrectangle{\pgfqpoint{4.985294in}{0.500000in}}{\pgfqpoint{1.764706in}{1.700000in}}%
\pgfusepath{clip}%
\pgfsetbuttcap%
\pgfsetroundjoin%
\definecolor{currentfill}{rgb}{0.979124,0.903132,0.839793}%
\pgfsetfillcolor{currentfill}%
\pgfsetlinewidth{0.311001pt}%
\definecolor{currentstroke}{rgb}{1.000000,1.000000,1.000000}%
\pgfsetstrokecolor{currentstroke}%
\pgfsetdash{}{0pt}%
\pgfpathmoveto{\pgfqpoint{6.344278in}{1.282446in}}%
\pgfpathcurveto{\pgfqpoint{6.351411in}{1.282446in}}{\pgfqpoint{6.358253in}{1.285280in}}{\pgfqpoint{6.363297in}{1.290324in}}%
\pgfpathcurveto{\pgfqpoint{6.368340in}{1.295368in}}{\pgfqpoint{6.371174in}{1.302209in}}{\pgfqpoint{6.371174in}{1.309342in}}%
\pgfpathcurveto{\pgfqpoint{6.371174in}{1.316475in}}{\pgfqpoint{6.368340in}{1.323317in}}{\pgfqpoint{6.363297in}{1.328360in}}%
\pgfpathcurveto{\pgfqpoint{6.358253in}{1.333404in}}{\pgfqpoint{6.351411in}{1.336238in}}{\pgfqpoint{6.344278in}{1.336238in}}%
\pgfpathcurveto{\pgfqpoint{6.337146in}{1.336238in}}{\pgfqpoint{6.330304in}{1.333404in}}{\pgfqpoint{6.325260in}{1.328360in}}%
\pgfpathcurveto{\pgfqpoint{6.320217in}{1.323317in}}{\pgfqpoint{6.317383in}{1.316475in}}{\pgfqpoint{6.317383in}{1.309342in}}%
\pgfpathcurveto{\pgfqpoint{6.317383in}{1.302209in}}{\pgfqpoint{6.320217in}{1.295368in}}{\pgfqpoint{6.325260in}{1.290324in}}%
\pgfpathcurveto{\pgfqpoint{6.330304in}{1.285280in}}{\pgfqpoint{6.337146in}{1.282446in}}{\pgfqpoint{6.344278in}{1.282446in}}%
\pgfpathclose%
\pgfusepath{stroke,fill}%
\end{pgfscope}%
\begin{pgfscope}%
\pgfpathrectangle{\pgfqpoint{4.985294in}{0.500000in}}{\pgfqpoint{1.764706in}{1.700000in}}%
\pgfusepath{clip}%
\pgfsetbuttcap%
\pgfsetroundjoin%
\definecolor{currentfill}{rgb}{0.970718,0.821518,0.719872}%
\pgfsetfillcolor{currentfill}%
\pgfsetlinewidth{0.311001pt}%
\definecolor{currentstroke}{rgb}{1.000000,1.000000,1.000000}%
\pgfsetstrokecolor{currentstroke}%
\pgfsetdash{}{0pt}%
\pgfpathmoveto{\pgfqpoint{6.230760in}{1.240457in}}%
\pgfpathcurveto{\pgfqpoint{6.237893in}{1.240457in}}{\pgfqpoint{6.244735in}{1.243291in}}{\pgfqpoint{6.249778in}{1.248335in}}%
\pgfpathcurveto{\pgfqpoint{6.254822in}{1.253378in}}{\pgfqpoint{6.257656in}{1.260220in}}{\pgfqpoint{6.257656in}{1.267353in}}%
\pgfpathcurveto{\pgfqpoint{6.257656in}{1.274486in}}{\pgfqpoint{6.254822in}{1.281327in}}{\pgfqpoint{6.249778in}{1.286371in}}%
\pgfpathcurveto{\pgfqpoint{6.244735in}{1.291415in}}{\pgfqpoint{6.237893in}{1.294248in}}{\pgfqpoint{6.230760in}{1.294248in}}%
\pgfpathcurveto{\pgfqpoint{6.223627in}{1.294248in}}{\pgfqpoint{6.216786in}{1.291415in}}{\pgfqpoint{6.211742in}{1.286371in}}%
\pgfpathcurveto{\pgfqpoint{6.206698in}{1.281327in}}{\pgfqpoint{6.203865in}{1.274486in}}{\pgfqpoint{6.203865in}{1.267353in}}%
\pgfpathcurveto{\pgfqpoint{6.203865in}{1.260220in}}{\pgfqpoint{6.206698in}{1.253378in}}{\pgfqpoint{6.211742in}{1.248335in}}%
\pgfpathcurveto{\pgfqpoint{6.216786in}{1.243291in}}{\pgfqpoint{6.223627in}{1.240457in}}{\pgfqpoint{6.230760in}{1.240457in}}%
\pgfpathclose%
\pgfusepath{stroke,fill}%
\end{pgfscope}%
\begin{pgfscope}%
\pgfpathrectangle{\pgfqpoint{4.985294in}{0.500000in}}{\pgfqpoint{1.764706in}{1.700000in}}%
\pgfusepath{clip}%
\pgfsetbuttcap%
\pgfsetroundjoin%
\definecolor{currentfill}{rgb}{0.968105,0.786346,0.667739}%
\pgfsetfillcolor{currentfill}%
\pgfsetlinewidth{0.311001pt}%
\definecolor{currentstroke}{rgb}{1.000000,1.000000,1.000000}%
\pgfsetstrokecolor{currentstroke}%
\pgfsetdash{}{0pt}%
\pgfpathmoveto{\pgfqpoint{5.501524in}{1.250253in}}%
\pgfpathcurveto{\pgfqpoint{5.508657in}{1.250253in}}{\pgfqpoint{5.515499in}{1.253087in}}{\pgfqpoint{5.520542in}{1.258130in}}%
\pgfpathcurveto{\pgfqpoint{5.525586in}{1.263174in}}{\pgfqpoint{5.528420in}{1.270016in}}{\pgfqpoint{5.528420in}{1.277148in}}%
\pgfpathcurveto{\pgfqpoint{5.528420in}{1.284281in}}{\pgfqpoint{5.525586in}{1.291123in}}{\pgfqpoint{5.520542in}{1.296166in}}%
\pgfpathcurveto{\pgfqpoint{5.515499in}{1.301210in}}{\pgfqpoint{5.508657in}{1.304044in}}{\pgfqpoint{5.501524in}{1.304044in}}%
\pgfpathcurveto{\pgfqpoint{5.494391in}{1.304044in}}{\pgfqpoint{5.487550in}{1.301210in}}{\pgfqpoint{5.482506in}{1.296166in}}%
\pgfpathcurveto{\pgfqpoint{5.477462in}{1.291123in}}{\pgfqpoint{5.474628in}{1.284281in}}{\pgfqpoint{5.474628in}{1.277148in}}%
\pgfpathcurveto{\pgfqpoint{5.474628in}{1.270016in}}{\pgfqpoint{5.477462in}{1.263174in}}{\pgfqpoint{5.482506in}{1.258130in}}%
\pgfpathcurveto{\pgfqpoint{5.487550in}{1.253087in}}{\pgfqpoint{5.494391in}{1.250253in}}{\pgfqpoint{5.501524in}{1.250253in}}%
\pgfpathclose%
\pgfusepath{stroke,fill}%
\end{pgfscope}%
\begin{pgfscope}%
\pgfpathrectangle{\pgfqpoint{4.985294in}{0.500000in}}{\pgfqpoint{1.764706in}{1.700000in}}%
\pgfusepath{clip}%
\pgfsetbuttcap%
\pgfsetroundjoin%
\definecolor{currentfill}{rgb}{0.966328,0.750560,0.616961}%
\pgfsetfillcolor{currentfill}%
\pgfsetlinewidth{0.311001pt}%
\definecolor{currentstroke}{rgb}{1.000000,1.000000,1.000000}%
\pgfsetstrokecolor{currentstroke}%
\pgfsetdash{}{0pt}%
\pgfpathmoveto{\pgfqpoint{5.333223in}{1.322591in}}%
\pgfpathcurveto{\pgfqpoint{5.340356in}{1.322591in}}{\pgfqpoint{5.347198in}{1.325424in}}{\pgfqpoint{5.352241in}{1.330468in}}%
\pgfpathcurveto{\pgfqpoint{5.357285in}{1.335512in}}{\pgfqpoint{5.360119in}{1.342353in}}{\pgfqpoint{5.360119in}{1.349486in}}%
\pgfpathcurveto{\pgfqpoint{5.360119in}{1.356619in}}{\pgfqpoint{5.357285in}{1.363461in}}{\pgfqpoint{5.352241in}{1.368504in}}%
\pgfpathcurveto{\pgfqpoint{5.347198in}{1.373548in}}{\pgfqpoint{5.340356in}{1.376382in}}{\pgfqpoint{5.333223in}{1.376382in}}%
\pgfpathcurveto{\pgfqpoint{5.326090in}{1.376382in}}{\pgfqpoint{5.319249in}{1.373548in}}{\pgfqpoint{5.314205in}{1.368504in}}%
\pgfpathcurveto{\pgfqpoint{5.309161in}{1.363461in}}{\pgfqpoint{5.306327in}{1.356619in}}{\pgfqpoint{5.306327in}{1.349486in}}%
\pgfpathcurveto{\pgfqpoint{5.306327in}{1.342353in}}{\pgfqpoint{5.309161in}{1.335512in}}{\pgfqpoint{5.314205in}{1.330468in}}%
\pgfpathcurveto{\pgfqpoint{5.319249in}{1.325424in}}{\pgfqpoint{5.326090in}{1.322591in}}{\pgfqpoint{5.333223in}{1.322591in}}%
\pgfpathclose%
\pgfusepath{stroke,fill}%
\end{pgfscope}%
\begin{pgfscope}%
\pgfpathrectangle{\pgfqpoint{4.985294in}{0.500000in}}{\pgfqpoint{1.764706in}{1.700000in}}%
\pgfusepath{clip}%
\pgfsetbuttcap%
\pgfsetroundjoin%
\definecolor{currentfill}{rgb}{0.975644,0.874038,0.797253}%
\pgfsetfillcolor{currentfill}%
\pgfsetlinewidth{0.311001pt}%
\definecolor{currentstroke}{rgb}{1.000000,1.000000,1.000000}%
\pgfsetstrokecolor{currentstroke}%
\pgfsetdash{}{0pt}%
\pgfpathmoveto{\pgfqpoint{6.241260in}{1.122763in}}%
\pgfpathcurveto{\pgfqpoint{6.248393in}{1.122763in}}{\pgfqpoint{6.255235in}{1.125596in}}{\pgfqpoint{6.260278in}{1.130640in}}%
\pgfpathcurveto{\pgfqpoint{6.265322in}{1.135684in}}{\pgfqpoint{6.268156in}{1.142525in}}{\pgfqpoint{6.268156in}{1.149658in}}%
\pgfpathcurveto{\pgfqpoint{6.268156in}{1.156791in}}{\pgfqpoint{6.265322in}{1.163633in}}{\pgfqpoint{6.260278in}{1.168676in}}%
\pgfpathcurveto{\pgfqpoint{6.255235in}{1.173720in}}{\pgfqpoint{6.248393in}{1.176554in}}{\pgfqpoint{6.241260in}{1.176554in}}%
\pgfpathcurveto{\pgfqpoint{6.234127in}{1.176554in}}{\pgfqpoint{6.227286in}{1.173720in}}{\pgfqpoint{6.222242in}{1.168676in}}%
\pgfpathcurveto{\pgfqpoint{6.217198in}{1.163633in}}{\pgfqpoint{6.214365in}{1.156791in}}{\pgfqpoint{6.214365in}{1.149658in}}%
\pgfpathcurveto{\pgfqpoint{6.214365in}{1.142525in}}{\pgfqpoint{6.217198in}{1.135684in}}{\pgfqpoint{6.222242in}{1.130640in}}%
\pgfpathcurveto{\pgfqpoint{6.227286in}{1.125596in}}{\pgfqpoint{6.234127in}{1.122763in}}{\pgfqpoint{6.241260in}{1.122763in}}%
\pgfpathclose%
\pgfusepath{stroke,fill}%
\end{pgfscope}%
\begin{pgfscope}%
\pgfpathrectangle{\pgfqpoint{4.985294in}{0.500000in}}{\pgfqpoint{1.764706in}{1.700000in}}%
\pgfusepath{clip}%
\pgfsetbuttcap%
\pgfsetroundjoin%
\definecolor{currentfill}{rgb}{0.977657,0.891500,0.822809}%
\pgfsetfillcolor{currentfill}%
\pgfsetlinewidth{0.311001pt}%
\definecolor{currentstroke}{rgb}{1.000000,1.000000,1.000000}%
\pgfsetstrokecolor{currentstroke}%
\pgfsetdash{}{0pt}%
\pgfpathmoveto{\pgfqpoint{6.333764in}{1.451720in}}%
\pgfpathcurveto{\pgfqpoint{6.340897in}{1.451720in}}{\pgfqpoint{6.347738in}{1.454553in}}{\pgfqpoint{6.352782in}{1.459597in}}%
\pgfpathcurveto{\pgfqpoint{6.357826in}{1.464641in}}{\pgfqpoint{6.360660in}{1.471482in}}{\pgfqpoint{6.360660in}{1.478615in}}%
\pgfpathcurveto{\pgfqpoint{6.360660in}{1.485748in}}{\pgfqpoint{6.357826in}{1.492590in}}{\pgfqpoint{6.352782in}{1.497633in}}%
\pgfpathcurveto{\pgfqpoint{6.347738in}{1.502677in}}{\pgfqpoint{6.340897in}{1.505511in}}{\pgfqpoint{6.333764in}{1.505511in}}%
\pgfpathcurveto{\pgfqpoint{6.326631in}{1.505511in}}{\pgfqpoint{6.319789in}{1.502677in}}{\pgfqpoint{6.314746in}{1.497633in}}%
\pgfpathcurveto{\pgfqpoint{6.309702in}{1.492590in}}{\pgfqpoint{6.306868in}{1.485748in}}{\pgfqpoint{6.306868in}{1.478615in}}%
\pgfpathcurveto{\pgfqpoint{6.306868in}{1.471482in}}{\pgfqpoint{6.309702in}{1.464641in}}{\pgfqpoint{6.314746in}{1.459597in}}%
\pgfpathcurveto{\pgfqpoint{6.319789in}{1.454553in}}{\pgfqpoint{6.326631in}{1.451720in}}{\pgfqpoint{6.333764in}{1.451720in}}%
\pgfpathclose%
\pgfusepath{stroke,fill}%
\end{pgfscope}%
\begin{pgfscope}%
\pgfpathrectangle{\pgfqpoint{4.985294in}{0.500000in}}{\pgfqpoint{1.764706in}{1.700000in}}%
\pgfusepath{clip}%
\pgfsetbuttcap%
\pgfsetroundjoin%
\definecolor{currentfill}{rgb}{0.973271,0.850724,0.762998}%
\pgfsetfillcolor{currentfill}%
\pgfsetlinewidth{0.311001pt}%
\definecolor{currentstroke}{rgb}{1.000000,1.000000,1.000000}%
\pgfsetstrokecolor{currentstroke}%
\pgfsetdash{}{0pt}%
\pgfpathmoveto{\pgfqpoint{5.497846in}{1.016251in}}%
\pgfpathcurveto{\pgfqpoint{5.504979in}{1.016251in}}{\pgfqpoint{5.511820in}{1.019084in}}{\pgfqpoint{5.516864in}{1.024128in}}%
\pgfpathcurveto{\pgfqpoint{5.521908in}{1.029172in}}{\pgfqpoint{5.524742in}{1.036013in}}{\pgfqpoint{5.524742in}{1.043146in}}%
\pgfpathcurveto{\pgfqpoint{5.524742in}{1.050279in}}{\pgfqpoint{5.521908in}{1.057121in}}{\pgfqpoint{5.516864in}{1.062164in}}%
\pgfpathcurveto{\pgfqpoint{5.511820in}{1.067208in}}{\pgfqpoint{5.504979in}{1.070042in}}{\pgfqpoint{5.497846in}{1.070042in}}%
\pgfpathcurveto{\pgfqpoint{5.490713in}{1.070042in}}{\pgfqpoint{5.483871in}{1.067208in}}{\pgfqpoint{5.478828in}{1.062164in}}%
\pgfpathcurveto{\pgfqpoint{5.473784in}{1.057121in}}{\pgfqpoint{5.470950in}{1.050279in}}{\pgfqpoint{5.470950in}{1.043146in}}%
\pgfpathcurveto{\pgfqpoint{5.470950in}{1.036013in}}{\pgfqpoint{5.473784in}{1.029172in}}{\pgfqpoint{5.478828in}{1.024128in}}%
\pgfpathcurveto{\pgfqpoint{5.483871in}{1.019084in}}{\pgfqpoint{5.490713in}{1.016251in}}{\pgfqpoint{5.497846in}{1.016251in}}%
\pgfpathclose%
\pgfusepath{stroke,fill}%
\end{pgfscope}%
\begin{pgfscope}%
\pgfpathrectangle{\pgfqpoint{4.985294in}{0.500000in}}{\pgfqpoint{1.764706in}{1.700000in}}%
\pgfusepath{clip}%
\pgfsetbuttcap%
\pgfsetroundjoin%
\definecolor{currentfill}{rgb}{0.960778,0.559972,0.399412}%
\pgfsetfillcolor{currentfill}%
\pgfsetlinewidth{0.311001pt}%
\definecolor{currentstroke}{rgb}{1.000000,1.000000,1.000000}%
\pgfsetstrokecolor{currentstroke}%
\pgfsetdash{}{0pt}%
\pgfpathmoveto{\pgfqpoint{5.555378in}{1.248246in}}%
\pgfpathcurveto{\pgfqpoint{5.562511in}{1.248246in}}{\pgfqpoint{5.569353in}{1.251079in}}{\pgfqpoint{5.574396in}{1.256123in}}%
\pgfpathcurveto{\pgfqpoint{5.579440in}{1.261167in}}{\pgfqpoint{5.582274in}{1.268008in}}{\pgfqpoint{5.582274in}{1.275141in}}%
\pgfpathcurveto{\pgfqpoint{5.582274in}{1.282274in}}{\pgfqpoint{5.579440in}{1.289116in}}{\pgfqpoint{5.574396in}{1.294159in}}%
\pgfpathcurveto{\pgfqpoint{5.569353in}{1.299203in}}{\pgfqpoint{5.562511in}{1.302037in}}{\pgfqpoint{5.555378in}{1.302037in}}%
\pgfpathcurveto{\pgfqpoint{5.548245in}{1.302037in}}{\pgfqpoint{5.541404in}{1.299203in}}{\pgfqpoint{5.536360in}{1.294159in}}%
\pgfpathcurveto{\pgfqpoint{5.531316in}{1.289116in}}{\pgfqpoint{5.528482in}{1.282274in}}{\pgfqpoint{5.528482in}{1.275141in}}%
\pgfpathcurveto{\pgfqpoint{5.528482in}{1.268008in}}{\pgfqpoint{5.531316in}{1.261167in}}{\pgfqpoint{5.536360in}{1.256123in}}%
\pgfpathcurveto{\pgfqpoint{5.541404in}{1.251079in}}{\pgfqpoint{5.548245in}{1.248246in}}{\pgfqpoint{5.555378in}{1.248246in}}%
\pgfpathclose%
\pgfusepath{stroke,fill}%
\end{pgfscope}%
\begin{pgfscope}%
\pgfpathrectangle{\pgfqpoint{4.985294in}{0.500000in}}{\pgfqpoint{1.764706in}{1.700000in}}%
\pgfusepath{clip}%
\pgfsetbuttcap%
\pgfsetroundjoin%
\definecolor{currentfill}{rgb}{0.967398,0.774513,0.650573}%
\pgfsetfillcolor{currentfill}%
\pgfsetlinewidth{0.311001pt}%
\definecolor{currentstroke}{rgb}{1.000000,1.000000,1.000000}%
\pgfsetstrokecolor{currentstroke}%
\pgfsetdash{}{0pt}%
\pgfpathmoveto{\pgfqpoint{5.354694in}{1.189813in}}%
\pgfpathcurveto{\pgfqpoint{5.361827in}{1.189813in}}{\pgfqpoint{5.368668in}{1.192647in}}{\pgfqpoint{5.373712in}{1.197691in}}%
\pgfpathcurveto{\pgfqpoint{5.378756in}{1.202735in}}{\pgfqpoint{5.381590in}{1.209576in}}{\pgfqpoint{5.381590in}{1.216709in}}%
\pgfpathcurveto{\pgfqpoint{5.381590in}{1.223842in}}{\pgfqpoint{5.378756in}{1.230684in}}{\pgfqpoint{5.373712in}{1.235727in}}%
\pgfpathcurveto{\pgfqpoint{5.368668in}{1.240771in}}{\pgfqpoint{5.361827in}{1.243605in}}{\pgfqpoint{5.354694in}{1.243605in}}%
\pgfpathcurveto{\pgfqpoint{5.347561in}{1.243605in}}{\pgfqpoint{5.340719in}{1.240771in}}{\pgfqpoint{5.335676in}{1.235727in}}%
\pgfpathcurveto{\pgfqpoint{5.330632in}{1.230684in}}{\pgfqpoint{5.327798in}{1.223842in}}{\pgfqpoint{5.327798in}{1.216709in}}%
\pgfpathcurveto{\pgfqpoint{5.327798in}{1.209576in}}{\pgfqpoint{5.330632in}{1.202735in}}{\pgfqpoint{5.335676in}{1.197691in}}%
\pgfpathcurveto{\pgfqpoint{5.340719in}{1.192647in}}{\pgfqpoint{5.347561in}{1.189813in}}{\pgfqpoint{5.354694in}{1.189813in}}%
\pgfpathclose%
\pgfusepath{stroke,fill}%
\end{pgfscope}%
\begin{pgfscope}%
\pgfpathrectangle{\pgfqpoint{4.985294in}{0.500000in}}{\pgfqpoint{1.764706in}{1.700000in}}%
\pgfusepath{clip}%
\pgfsetbuttcap%
\pgfsetroundjoin%
\definecolor{currentfill}{rgb}{0.966120,0.744512,0.608720}%
\pgfsetfillcolor{currentfill}%
\pgfsetlinewidth{0.311001pt}%
\definecolor{currentstroke}{rgb}{1.000000,1.000000,1.000000}%
\pgfsetstrokecolor{currentstroke}%
\pgfsetdash{}{0pt}%
\pgfpathmoveto{\pgfqpoint{6.313483in}{1.024122in}}%
\pgfpathcurveto{\pgfqpoint{6.320616in}{1.024122in}}{\pgfqpoint{6.327458in}{1.026956in}}{\pgfqpoint{6.332501in}{1.031999in}}%
\pgfpathcurveto{\pgfqpoint{6.337545in}{1.037043in}}{\pgfqpoint{6.340379in}{1.043885in}}{\pgfqpoint{6.340379in}{1.051017in}}%
\pgfpathcurveto{\pgfqpoint{6.340379in}{1.058150in}}{\pgfqpoint{6.337545in}{1.064992in}}{\pgfqpoint{6.332501in}{1.070035in}}%
\pgfpathcurveto{\pgfqpoint{6.327458in}{1.075079in}}{\pgfqpoint{6.320616in}{1.077913in}}{\pgfqpoint{6.313483in}{1.077913in}}%
\pgfpathcurveto{\pgfqpoint{6.306350in}{1.077913in}}{\pgfqpoint{6.299509in}{1.075079in}}{\pgfqpoint{6.294465in}{1.070035in}}%
\pgfpathcurveto{\pgfqpoint{6.289422in}{1.064992in}}{\pgfqpoint{6.286588in}{1.058150in}}{\pgfqpoint{6.286588in}{1.051017in}}%
\pgfpathcurveto{\pgfqpoint{6.286588in}{1.043885in}}{\pgfqpoint{6.289422in}{1.037043in}}{\pgfqpoint{6.294465in}{1.031999in}}%
\pgfpathcurveto{\pgfqpoint{6.299509in}{1.026956in}}{\pgfqpoint{6.306350in}{1.024122in}}{\pgfqpoint{6.313483in}{1.024122in}}%
\pgfpathclose%
\pgfusepath{stroke,fill}%
\end{pgfscope}%
\begin{pgfscope}%
\pgfpathrectangle{\pgfqpoint{4.985294in}{0.500000in}}{\pgfqpoint{1.764706in}{1.700000in}}%
\pgfusepath{clip}%
\pgfsetbuttcap%
\pgfsetroundjoin%
\definecolor{currentfill}{rgb}{0.976287,0.879862,0.805788}%
\pgfsetfillcolor{currentfill}%
\pgfsetlinewidth{0.311001pt}%
\definecolor{currentstroke}{rgb}{1.000000,1.000000,1.000000}%
\pgfsetstrokecolor{currentstroke}%
\pgfsetdash{}{0pt}%
\pgfpathmoveto{\pgfqpoint{6.243704in}{1.094996in}}%
\pgfpathcurveto{\pgfqpoint{6.250837in}{1.094996in}}{\pgfqpoint{6.257679in}{1.097830in}}{\pgfqpoint{6.262722in}{1.102873in}}%
\pgfpathcurveto{\pgfqpoint{6.267766in}{1.107917in}}{\pgfqpoint{6.270600in}{1.114759in}}{\pgfqpoint{6.270600in}{1.121892in}}%
\pgfpathcurveto{\pgfqpoint{6.270600in}{1.129024in}}{\pgfqpoint{6.267766in}{1.135866in}}{\pgfqpoint{6.262722in}{1.140910in}}%
\pgfpathcurveto{\pgfqpoint{6.257679in}{1.145953in}}{\pgfqpoint{6.250837in}{1.148787in}}{\pgfqpoint{6.243704in}{1.148787in}}%
\pgfpathcurveto{\pgfqpoint{6.236571in}{1.148787in}}{\pgfqpoint{6.229730in}{1.145953in}}{\pgfqpoint{6.224686in}{1.140910in}}%
\pgfpathcurveto{\pgfqpoint{6.219642in}{1.135866in}}{\pgfqpoint{6.216808in}{1.129024in}}{\pgfqpoint{6.216808in}{1.121892in}}%
\pgfpathcurveto{\pgfqpoint{6.216808in}{1.114759in}}{\pgfqpoint{6.219642in}{1.107917in}}{\pgfqpoint{6.224686in}{1.102873in}}%
\pgfpathcurveto{\pgfqpoint{6.229730in}{1.097830in}}{\pgfqpoint{6.236571in}{1.094996in}}{\pgfqpoint{6.243704in}{1.094996in}}%
\pgfpathclose%
\pgfusepath{stroke,fill}%
\end{pgfscope}%
\begin{pgfscope}%
\pgfpathrectangle{\pgfqpoint{4.985294in}{0.500000in}}{\pgfqpoint{1.764706in}{1.700000in}}%
\pgfusepath{clip}%
\pgfsetbuttcap%
\pgfsetroundjoin%
\definecolor{currentfill}{rgb}{0.980678,0.914765,0.856766}%
\pgfsetfillcolor{currentfill}%
\pgfsetlinewidth{0.311001pt}%
\definecolor{currentstroke}{rgb}{1.000000,1.000000,1.000000}%
\pgfsetstrokecolor{currentstroke}%
\pgfsetdash{}{0pt}%
\pgfpathmoveto{\pgfqpoint{5.406057in}{1.357002in}}%
\pgfpathcurveto{\pgfqpoint{5.413189in}{1.357002in}}{\pgfqpoint{5.420031in}{1.359836in}}{\pgfqpoint{5.425075in}{1.364880in}}%
\pgfpathcurveto{\pgfqpoint{5.430118in}{1.369923in}}{\pgfqpoint{5.432952in}{1.376765in}}{\pgfqpoint{5.432952in}{1.383898in}}%
\pgfpathcurveto{\pgfqpoint{5.432952in}{1.391031in}}{\pgfqpoint{5.430118in}{1.397872in}}{\pgfqpoint{5.425075in}{1.402916in}}%
\pgfpathcurveto{\pgfqpoint{5.420031in}{1.407960in}}{\pgfqpoint{5.413189in}{1.410794in}}{\pgfqpoint{5.406057in}{1.410794in}}%
\pgfpathcurveto{\pgfqpoint{5.398924in}{1.410794in}}{\pgfqpoint{5.392082in}{1.407960in}}{\pgfqpoint{5.387039in}{1.402916in}}%
\pgfpathcurveto{\pgfqpoint{5.381995in}{1.397872in}}{\pgfqpoint{5.379161in}{1.391031in}}{\pgfqpoint{5.379161in}{1.383898in}}%
\pgfpathcurveto{\pgfqpoint{5.379161in}{1.376765in}}{\pgfqpoint{5.381995in}{1.369923in}}{\pgfqpoint{5.387039in}{1.364880in}}%
\pgfpathcurveto{\pgfqpoint{5.392082in}{1.359836in}}{\pgfqpoint{5.398924in}{1.357002in}}{\pgfqpoint{5.406057in}{1.357002in}}%
\pgfpathclose%
\pgfusepath{stroke,fill}%
\end{pgfscope}%
\begin{pgfscope}%
\pgfpathrectangle{\pgfqpoint{4.985294in}{0.500000in}}{\pgfqpoint{1.764706in}{1.700000in}}%
\pgfusepath{clip}%
\pgfsetbuttcap%
\pgfsetroundjoin%
\definecolor{currentfill}{rgb}{0.965042,0.701564,0.552889}%
\pgfsetfillcolor{currentfill}%
\pgfsetlinewidth{0.311001pt}%
\definecolor{currentstroke}{rgb}{1.000000,1.000000,1.000000}%
\pgfsetstrokecolor{currentstroke}%
\pgfsetdash{}{0pt}%
\pgfpathmoveto{\pgfqpoint{6.153490in}{1.540538in}}%
\pgfpathcurveto{\pgfqpoint{6.160623in}{1.540538in}}{\pgfqpoint{6.167464in}{1.543372in}}{\pgfqpoint{6.172508in}{1.548416in}}%
\pgfpathcurveto{\pgfqpoint{6.177552in}{1.553460in}}{\pgfqpoint{6.180385in}{1.560301in}}{\pgfqpoint{6.180385in}{1.567434in}}%
\pgfpathcurveto{\pgfqpoint{6.180385in}{1.574567in}}{\pgfqpoint{6.177552in}{1.581409in}}{\pgfqpoint{6.172508in}{1.586452in}}%
\pgfpathcurveto{\pgfqpoint{6.167464in}{1.591496in}}{\pgfqpoint{6.160623in}{1.594330in}}{\pgfqpoint{6.153490in}{1.594330in}}%
\pgfpathcurveto{\pgfqpoint{6.146357in}{1.594330in}}{\pgfqpoint{6.139515in}{1.591496in}}{\pgfqpoint{6.134472in}{1.586452in}}%
\pgfpathcurveto{\pgfqpoint{6.129428in}{1.581409in}}{\pgfqpoint{6.126594in}{1.574567in}}{\pgfqpoint{6.126594in}{1.567434in}}%
\pgfpathcurveto{\pgfqpoint{6.126594in}{1.560301in}}{\pgfqpoint{6.129428in}{1.553460in}}{\pgfqpoint{6.134472in}{1.548416in}}%
\pgfpathcurveto{\pgfqpoint{6.139515in}{1.543372in}}{\pgfqpoint{6.146357in}{1.540538in}}{\pgfqpoint{6.153490in}{1.540538in}}%
\pgfpathclose%
\pgfusepath{stroke,fill}%
\end{pgfscope}%
\begin{pgfscope}%
\pgfpathrectangle{\pgfqpoint{4.985294in}{0.500000in}}{\pgfqpoint{1.764706in}{1.700000in}}%
\pgfusepath{clip}%
\pgfsetbuttcap%
\pgfsetroundjoin%
\definecolor{currentfill}{rgb}{0.966812,0.762584,0.633643}%
\pgfsetfillcolor{currentfill}%
\pgfsetlinewidth{0.311001pt}%
\definecolor{currentstroke}{rgb}{1.000000,1.000000,1.000000}%
\pgfsetstrokecolor{currentstroke}%
\pgfsetdash{}{0pt}%
\pgfpathmoveto{\pgfqpoint{5.520283in}{1.180533in}}%
\pgfpathcurveto{\pgfqpoint{5.527416in}{1.180533in}}{\pgfqpoint{5.534258in}{1.183367in}}{\pgfqpoint{5.539301in}{1.188410in}}%
\pgfpathcurveto{\pgfqpoint{5.544345in}{1.193454in}}{\pgfqpoint{5.547179in}{1.200296in}}{\pgfqpoint{5.547179in}{1.207429in}}%
\pgfpathcurveto{\pgfqpoint{5.547179in}{1.214561in}}{\pgfqpoint{5.544345in}{1.221403in}}{\pgfqpoint{5.539301in}{1.226447in}}%
\pgfpathcurveto{\pgfqpoint{5.534258in}{1.231490in}}{\pgfqpoint{5.527416in}{1.234324in}}{\pgfqpoint{5.520283in}{1.234324in}}%
\pgfpathcurveto{\pgfqpoint{5.513150in}{1.234324in}}{\pgfqpoint{5.506309in}{1.231490in}}{\pgfqpoint{5.501265in}{1.226447in}}%
\pgfpathcurveto{\pgfqpoint{5.496221in}{1.221403in}}{\pgfqpoint{5.493387in}{1.214561in}}{\pgfqpoint{5.493387in}{1.207429in}}%
\pgfpathcurveto{\pgfqpoint{5.493387in}{1.200296in}}{\pgfqpoint{5.496221in}{1.193454in}}{\pgfqpoint{5.501265in}{1.188410in}}%
\pgfpathcurveto{\pgfqpoint{5.506309in}{1.183367in}}{\pgfqpoint{5.513150in}{1.180533in}}{\pgfqpoint{5.520283in}{1.180533in}}%
\pgfpathclose%
\pgfusepath{stroke,fill}%
\end{pgfscope}%
\begin{pgfscope}%
\pgfpathrectangle{\pgfqpoint{4.985294in}{0.500000in}}{\pgfqpoint{1.764706in}{1.700000in}}%
\pgfusepath{clip}%
\pgfsetbuttcap%
\pgfsetroundjoin%
\definecolor{currentfill}{rgb}{0.979124,0.903132,0.839793}%
\pgfsetfillcolor{currentfill}%
\pgfsetlinewidth{0.311001pt}%
\definecolor{currentstroke}{rgb}{1.000000,1.000000,1.000000}%
\pgfsetstrokecolor{currentstroke}%
\pgfsetdash{}{0pt}%
\pgfpathmoveto{\pgfqpoint{6.340262in}{1.264777in}}%
\pgfpathcurveto{\pgfqpoint{6.347395in}{1.264777in}}{\pgfqpoint{6.354237in}{1.267611in}}{\pgfqpoint{6.359280in}{1.272654in}}%
\pgfpathcurveto{\pgfqpoint{6.364324in}{1.277698in}}{\pgfqpoint{6.367158in}{1.284540in}}{\pgfqpoint{6.367158in}{1.291672in}}%
\pgfpathcurveto{\pgfqpoint{6.367158in}{1.298805in}}{\pgfqpoint{6.364324in}{1.305647in}}{\pgfqpoint{6.359280in}{1.310691in}}%
\pgfpathcurveto{\pgfqpoint{6.354237in}{1.315734in}}{\pgfqpoint{6.347395in}{1.318568in}}{\pgfqpoint{6.340262in}{1.318568in}}%
\pgfpathcurveto{\pgfqpoint{6.333130in}{1.318568in}}{\pgfqpoint{6.326288in}{1.315734in}}{\pgfqpoint{6.321244in}{1.310691in}}%
\pgfpathcurveto{\pgfqpoint{6.316201in}{1.305647in}}{\pgfqpoint{6.313367in}{1.298805in}}{\pgfqpoint{6.313367in}{1.291672in}}%
\pgfpathcurveto{\pgfqpoint{6.313367in}{1.284540in}}{\pgfqpoint{6.316201in}{1.277698in}}{\pgfqpoint{6.321244in}{1.272654in}}%
\pgfpathcurveto{\pgfqpoint{6.326288in}{1.267611in}}{\pgfqpoint{6.333130in}{1.264777in}}{\pgfqpoint{6.340262in}{1.264777in}}%
\pgfpathclose%
\pgfusepath{stroke,fill}%
\end{pgfscope}%
\begin{pgfscope}%
\pgfpathrectangle{\pgfqpoint{4.985294in}{0.500000in}}{\pgfqpoint{1.764706in}{1.700000in}}%
\pgfusepath{clip}%
\pgfsetbuttcap%
\pgfsetroundjoin%
\definecolor{currentfill}{rgb}{0.964032,0.651225,0.493258}%
\pgfsetfillcolor{currentfill}%
\pgfsetlinewidth{0.311001pt}%
\definecolor{currentstroke}{rgb}{1.000000,1.000000,1.000000}%
\pgfsetstrokecolor{currentstroke}%
\pgfsetdash{}{0pt}%
\pgfpathmoveto{\pgfqpoint{5.556970in}{1.743925in}}%
\pgfpathcurveto{\pgfqpoint{5.564103in}{1.743925in}}{\pgfqpoint{5.570944in}{1.746759in}}{\pgfqpoint{5.575988in}{1.751802in}}%
\pgfpathcurveto{\pgfqpoint{5.581032in}{1.756846in}}{\pgfqpoint{5.583865in}{1.763688in}}{\pgfqpoint{5.583865in}{1.770821in}}%
\pgfpathcurveto{\pgfqpoint{5.583865in}{1.777953in}}{\pgfqpoint{5.581032in}{1.784795in}}{\pgfqpoint{5.575988in}{1.789839in}}%
\pgfpathcurveto{\pgfqpoint{5.570944in}{1.794882in}}{\pgfqpoint{5.564103in}{1.797716in}}{\pgfqpoint{5.556970in}{1.797716in}}%
\pgfpathcurveto{\pgfqpoint{5.549837in}{1.797716in}}{\pgfqpoint{5.542995in}{1.794882in}}{\pgfqpoint{5.537952in}{1.789839in}}%
\pgfpathcurveto{\pgfqpoint{5.532908in}{1.784795in}}{\pgfqpoint{5.530074in}{1.777953in}}{\pgfqpoint{5.530074in}{1.770821in}}%
\pgfpathcurveto{\pgfqpoint{5.530074in}{1.763688in}}{\pgfqpoint{5.532908in}{1.756846in}}{\pgfqpoint{5.537952in}{1.751802in}}%
\pgfpathcurveto{\pgfqpoint{5.542995in}{1.746759in}}{\pgfqpoint{5.549837in}{1.743925in}}{\pgfqpoint{5.556970in}{1.743925in}}%
\pgfpathclose%
\pgfusepath{stroke,fill}%
\end{pgfscope}%
\begin{pgfscope}%
\pgfpathrectangle{\pgfqpoint{4.985294in}{0.500000in}}{\pgfqpoint{1.764706in}{1.700000in}}%
\pgfusepath{clip}%
\pgfsetbuttcap%
\pgfsetroundjoin%
\definecolor{currentfill}{rgb}{0.971694,0.833208,0.737161}%
\pgfsetfillcolor{currentfill}%
\pgfsetlinewidth{0.311001pt}%
\definecolor{currentstroke}{rgb}{1.000000,1.000000,1.000000}%
\pgfsetstrokecolor{currentstroke}%
\pgfsetdash{}{0pt}%
\pgfpathmoveto{\pgfqpoint{5.363994in}{1.238995in}}%
\pgfpathcurveto{\pgfqpoint{5.371127in}{1.238995in}}{\pgfqpoint{5.377969in}{1.241829in}}{\pgfqpoint{5.383013in}{1.246873in}}%
\pgfpathcurveto{\pgfqpoint{5.388056in}{1.251916in}}{\pgfqpoint{5.390890in}{1.258758in}}{\pgfqpoint{5.390890in}{1.265891in}}%
\pgfpathcurveto{\pgfqpoint{5.390890in}{1.273024in}}{\pgfqpoint{5.388056in}{1.279865in}}{\pgfqpoint{5.383013in}{1.284909in}}%
\pgfpathcurveto{\pgfqpoint{5.377969in}{1.289953in}}{\pgfqpoint{5.371127in}{1.292787in}}{\pgfqpoint{5.363994in}{1.292787in}}%
\pgfpathcurveto{\pgfqpoint{5.356862in}{1.292787in}}{\pgfqpoint{5.350020in}{1.289953in}}{\pgfqpoint{5.344976in}{1.284909in}}%
\pgfpathcurveto{\pgfqpoint{5.339933in}{1.279865in}}{\pgfqpoint{5.337099in}{1.273024in}}{\pgfqpoint{5.337099in}{1.265891in}}%
\pgfpathcurveto{\pgfqpoint{5.337099in}{1.258758in}}{\pgfqpoint{5.339933in}{1.251916in}}{\pgfqpoint{5.344976in}{1.246873in}}%
\pgfpathcurveto{\pgfqpoint{5.350020in}{1.241829in}}{\pgfqpoint{5.356862in}{1.238995in}}{\pgfqpoint{5.363994in}{1.238995in}}%
\pgfpathclose%
\pgfusepath{stroke,fill}%
\end{pgfscope}%
\begin{pgfscope}%
\pgfpathrectangle{\pgfqpoint{4.985294in}{0.500000in}}{\pgfqpoint{1.764706in}{1.700000in}}%
\pgfusepath{clip}%
\pgfsetbuttcap%
\pgfsetroundjoin%
\definecolor{currentfill}{rgb}{0.981377,0.920617,0.865369}%
\pgfsetfillcolor{currentfill}%
\pgfsetlinewidth{0.311001pt}%
\definecolor{currentstroke}{rgb}{1.000000,1.000000,1.000000}%
\pgfsetstrokecolor{currentstroke}%
\pgfsetdash{}{0pt}%
\pgfpathmoveto{\pgfqpoint{6.323115in}{1.299230in}}%
\pgfpathcurveto{\pgfqpoint{6.330247in}{1.299230in}}{\pgfqpoint{6.337089in}{1.302063in}}{\pgfqpoint{6.342133in}{1.307107in}}%
\pgfpathcurveto{\pgfqpoint{6.347176in}{1.312151in}}{\pgfqpoint{6.350010in}{1.318992in}}{\pgfqpoint{6.350010in}{1.326125in}}%
\pgfpathcurveto{\pgfqpoint{6.350010in}{1.333258in}}{\pgfqpoint{6.347176in}{1.340100in}}{\pgfqpoint{6.342133in}{1.345143in}}%
\pgfpathcurveto{\pgfqpoint{6.337089in}{1.350187in}}{\pgfqpoint{6.330247in}{1.353021in}}{\pgfqpoint{6.323115in}{1.353021in}}%
\pgfpathcurveto{\pgfqpoint{6.315982in}{1.353021in}}{\pgfqpoint{6.309140in}{1.350187in}}{\pgfqpoint{6.304096in}{1.345143in}}%
\pgfpathcurveto{\pgfqpoint{6.299053in}{1.340100in}}{\pgfqpoint{6.296219in}{1.333258in}}{\pgfqpoint{6.296219in}{1.326125in}}%
\pgfpathcurveto{\pgfqpoint{6.296219in}{1.318992in}}{\pgfqpoint{6.299053in}{1.312151in}}{\pgfqpoint{6.304096in}{1.307107in}}%
\pgfpathcurveto{\pgfqpoint{6.309140in}{1.302063in}}{\pgfqpoint{6.315982in}{1.299230in}}{\pgfqpoint{6.323115in}{1.299230in}}%
\pgfpathclose%
\pgfusepath{stroke,fill}%
\end{pgfscope}%
\begin{pgfscope}%
\pgfpathrectangle{\pgfqpoint{4.985294in}{0.500000in}}{\pgfqpoint{1.764706in}{1.700000in}}%
\pgfusepath{clip}%
\pgfsetbuttcap%
\pgfsetroundjoin%
\definecolor{currentfill}{rgb}{0.966560,0.756582,0.625273}%
\pgfsetfillcolor{currentfill}%
\pgfsetlinewidth{0.311001pt}%
\definecolor{currentstroke}{rgb}{1.000000,1.000000,1.000000}%
\pgfsetstrokecolor{currentstroke}%
\pgfsetdash{}{0pt}%
\pgfpathmoveto{\pgfqpoint{5.447272in}{0.971282in}}%
\pgfpathcurveto{\pgfqpoint{5.454404in}{0.971282in}}{\pgfqpoint{5.461246in}{0.974116in}}{\pgfqpoint{5.466290in}{0.979160in}}%
\pgfpathcurveto{\pgfqpoint{5.471333in}{0.984204in}}{\pgfqpoint{5.474167in}{0.991045in}}{\pgfqpoint{5.474167in}{0.998178in}}%
\pgfpathcurveto{\pgfqpoint{5.474167in}{1.005311in}}{\pgfqpoint{5.471333in}{1.012153in}}{\pgfqpoint{5.466290in}{1.017196in}}%
\pgfpathcurveto{\pgfqpoint{5.461246in}{1.022240in}}{\pgfqpoint{5.454404in}{1.025074in}}{\pgfqpoint{5.447272in}{1.025074in}}%
\pgfpathcurveto{\pgfqpoint{5.440139in}{1.025074in}}{\pgfqpoint{5.433297in}{1.022240in}}{\pgfqpoint{5.428254in}{1.017196in}}%
\pgfpathcurveto{\pgfqpoint{5.423210in}{1.012153in}}{\pgfqpoint{5.420376in}{1.005311in}}{\pgfqpoint{5.420376in}{0.998178in}}%
\pgfpathcurveto{\pgfqpoint{5.420376in}{0.991045in}}{\pgfqpoint{5.423210in}{0.984204in}}{\pgfqpoint{5.428254in}{0.979160in}}%
\pgfpathcurveto{\pgfqpoint{5.433297in}{0.974116in}}{\pgfqpoint{5.440139in}{0.971282in}}{\pgfqpoint{5.447272in}{0.971282in}}%
\pgfpathclose%
\pgfusepath{stroke,fill}%
\end{pgfscope}%
\begin{pgfscope}%
\pgfpathrectangle{\pgfqpoint{4.985294in}{0.500000in}}{\pgfqpoint{1.764706in}{1.700000in}}%
\pgfusepath{clip}%
\pgfsetbuttcap%
\pgfsetroundjoin%
\definecolor{currentfill}{rgb}{0.979124,0.903132,0.839793}%
\pgfsetfillcolor{currentfill}%
\pgfsetlinewidth{0.311001pt}%
\definecolor{currentstroke}{rgb}{1.000000,1.000000,1.000000}%
\pgfsetstrokecolor{currentstroke}%
\pgfsetdash{}{0pt}%
\pgfpathmoveto{\pgfqpoint{6.275640in}{1.196014in}}%
\pgfpathcurveto{\pgfqpoint{6.282773in}{1.196014in}}{\pgfqpoint{6.289615in}{1.198848in}}{\pgfqpoint{6.294658in}{1.203892in}}%
\pgfpathcurveto{\pgfqpoint{6.299702in}{1.208935in}}{\pgfqpoint{6.302536in}{1.215777in}}{\pgfqpoint{6.302536in}{1.222910in}}%
\pgfpathcurveto{\pgfqpoint{6.302536in}{1.230043in}}{\pgfqpoint{6.299702in}{1.236884in}}{\pgfqpoint{6.294658in}{1.241928in}}%
\pgfpathcurveto{\pgfqpoint{6.289615in}{1.246972in}}{\pgfqpoint{6.282773in}{1.249805in}}{\pgfqpoint{6.275640in}{1.249805in}}%
\pgfpathcurveto{\pgfqpoint{6.268508in}{1.249805in}}{\pgfqpoint{6.261666in}{1.246972in}}{\pgfqpoint{6.256622in}{1.241928in}}%
\pgfpathcurveto{\pgfqpoint{6.251579in}{1.236884in}}{\pgfqpoint{6.248745in}{1.230043in}}{\pgfqpoint{6.248745in}{1.222910in}}%
\pgfpathcurveto{\pgfqpoint{6.248745in}{1.215777in}}{\pgfqpoint{6.251579in}{1.208935in}}{\pgfqpoint{6.256622in}{1.203892in}}%
\pgfpathcurveto{\pgfqpoint{6.261666in}{1.198848in}}{\pgfqpoint{6.268508in}{1.196014in}}{\pgfqpoint{6.275640in}{1.196014in}}%
\pgfpathclose%
\pgfusepath{stroke,fill}%
\end{pgfscope}%
\begin{pgfscope}%
\pgfpathrectangle{\pgfqpoint{4.985294in}{0.500000in}}{\pgfqpoint{1.764706in}{1.700000in}}%
\pgfusepath{clip}%
\pgfsetbuttcap%
\pgfsetroundjoin%
\definecolor{currentfill}{rgb}{0.979891,0.908948,0.848279}%
\pgfsetfillcolor{currentfill}%
\pgfsetlinewidth{0.311001pt}%
\definecolor{currentstroke}{rgb}{1.000000,1.000000,1.000000}%
\pgfsetstrokecolor{currentstroke}%
\pgfsetdash{}{0pt}%
\pgfpathmoveto{\pgfqpoint{5.430154in}{1.224496in}}%
\pgfpathcurveto{\pgfqpoint{5.437286in}{1.224496in}}{\pgfqpoint{5.444128in}{1.227330in}}{\pgfqpoint{5.449172in}{1.232373in}}%
\pgfpathcurveto{\pgfqpoint{5.454215in}{1.237417in}}{\pgfqpoint{5.457049in}{1.244258in}}{\pgfqpoint{5.457049in}{1.251391in}}%
\pgfpathcurveto{\pgfqpoint{5.457049in}{1.258524in}}{\pgfqpoint{5.454215in}{1.265366in}}{\pgfqpoint{5.449172in}{1.270409in}}%
\pgfpathcurveto{\pgfqpoint{5.444128in}{1.275453in}}{\pgfqpoint{5.437286in}{1.278287in}}{\pgfqpoint{5.430154in}{1.278287in}}%
\pgfpathcurveto{\pgfqpoint{5.423021in}{1.278287in}}{\pgfqpoint{5.416179in}{1.275453in}}{\pgfqpoint{5.411136in}{1.270409in}}%
\pgfpathcurveto{\pgfqpoint{5.406092in}{1.265366in}}{\pgfqpoint{5.403258in}{1.258524in}}{\pgfqpoint{5.403258in}{1.251391in}}%
\pgfpathcurveto{\pgfqpoint{5.403258in}{1.244258in}}{\pgfqpoint{5.406092in}{1.237417in}}{\pgfqpoint{5.411136in}{1.232373in}}%
\pgfpathcurveto{\pgfqpoint{5.416179in}{1.227330in}}{\pgfqpoint{5.423021in}{1.224496in}}{\pgfqpoint{5.430154in}{1.224496in}}%
\pgfpathclose%
\pgfusepath{stroke,fill}%
\end{pgfscope}%
\begin{pgfscope}%
\pgfpathrectangle{\pgfqpoint{4.985294in}{0.500000in}}{\pgfqpoint{1.764706in}{1.700000in}}%
\pgfusepath{clip}%
\pgfsetbuttcap%
\pgfsetroundjoin%
\definecolor{currentfill}{rgb}{0.963884,0.644842,0.486120}%
\pgfsetfillcolor{currentfill}%
\pgfsetlinewidth{0.311001pt}%
\definecolor{currentstroke}{rgb}{1.000000,1.000000,1.000000}%
\pgfsetstrokecolor{currentstroke}%
\pgfsetdash{}{0pt}%
\pgfpathmoveto{\pgfqpoint{5.549112in}{1.748977in}}%
\pgfpathcurveto{\pgfqpoint{5.556245in}{1.748977in}}{\pgfqpoint{5.563087in}{1.751811in}}{\pgfqpoint{5.568130in}{1.756855in}}%
\pgfpathcurveto{\pgfqpoint{5.573174in}{1.761898in}}{\pgfqpoint{5.576008in}{1.768740in}}{\pgfqpoint{5.576008in}{1.775873in}}%
\pgfpathcurveto{\pgfqpoint{5.576008in}{1.783006in}}{\pgfqpoint{5.573174in}{1.789847in}}{\pgfqpoint{5.568130in}{1.794891in}}%
\pgfpathcurveto{\pgfqpoint{5.563087in}{1.799935in}}{\pgfqpoint{5.556245in}{1.802768in}}{\pgfqpoint{5.549112in}{1.802768in}}%
\pgfpathcurveto{\pgfqpoint{5.541979in}{1.802768in}}{\pgfqpoint{5.535138in}{1.799935in}}{\pgfqpoint{5.530094in}{1.794891in}}%
\pgfpathcurveto{\pgfqpoint{5.525050in}{1.789847in}}{\pgfqpoint{5.522216in}{1.783006in}}{\pgfqpoint{5.522216in}{1.775873in}}%
\pgfpathcurveto{\pgfqpoint{5.522216in}{1.768740in}}{\pgfqpoint{5.525050in}{1.761898in}}{\pgfqpoint{5.530094in}{1.756855in}}%
\pgfpathcurveto{\pgfqpoint{5.535138in}{1.751811in}}{\pgfqpoint{5.541979in}{1.748977in}}{\pgfqpoint{5.549112in}{1.748977in}}%
\pgfpathclose%
\pgfusepath{stroke,fill}%
\end{pgfscope}%
\begin{pgfscope}%
\pgfpathrectangle{\pgfqpoint{4.985294in}{0.500000in}}{\pgfqpoint{1.764706in}{1.700000in}}%
\pgfusepath{clip}%
\pgfsetbuttcap%
\pgfsetroundjoin%
\definecolor{currentfill}{rgb}{0.976961,0.885681,0.814303}%
\pgfsetfillcolor{currentfill}%
\pgfsetlinewidth{0.311001pt}%
\definecolor{currentstroke}{rgb}{1.000000,1.000000,1.000000}%
\pgfsetstrokecolor{currentstroke}%
\pgfsetdash{}{0pt}%
\pgfpathmoveto{\pgfqpoint{5.457511in}{1.376755in}}%
\pgfpathcurveto{\pgfqpoint{5.464643in}{1.376755in}}{\pgfqpoint{5.471485in}{1.379589in}}{\pgfqpoint{5.476529in}{1.384632in}}%
\pgfpathcurveto{\pgfqpoint{5.481572in}{1.389676in}}{\pgfqpoint{5.484406in}{1.396518in}}{\pgfqpoint{5.484406in}{1.403650in}}%
\pgfpathcurveto{\pgfqpoint{5.484406in}{1.410783in}}{\pgfqpoint{5.481572in}{1.417625in}}{\pgfqpoint{5.476529in}{1.422669in}}%
\pgfpathcurveto{\pgfqpoint{5.471485in}{1.427712in}}{\pgfqpoint{5.464643in}{1.430546in}}{\pgfqpoint{5.457511in}{1.430546in}}%
\pgfpathcurveto{\pgfqpoint{5.450378in}{1.430546in}}{\pgfqpoint{5.443536in}{1.427712in}}{\pgfqpoint{5.438492in}{1.422669in}}%
\pgfpathcurveto{\pgfqpoint{5.433449in}{1.417625in}}{\pgfqpoint{5.430615in}{1.410783in}}{\pgfqpoint{5.430615in}{1.403650in}}%
\pgfpathcurveto{\pgfqpoint{5.430615in}{1.396518in}}{\pgfqpoint{5.433449in}{1.389676in}}{\pgfqpoint{5.438492in}{1.384632in}}%
\pgfpathcurveto{\pgfqpoint{5.443536in}{1.379589in}}{\pgfqpoint{5.450378in}{1.376755in}}{\pgfqpoint{5.457511in}{1.376755in}}%
\pgfpathclose%
\pgfusepath{stroke,fill}%
\end{pgfscope}%
\begin{pgfscope}%
\pgfpathrectangle{\pgfqpoint{4.985294in}{0.500000in}}{\pgfqpoint{1.764706in}{1.700000in}}%
\pgfusepath{clip}%
\pgfsetbuttcap%
\pgfsetroundjoin%
\definecolor{currentfill}{rgb}{0.976961,0.885681,0.814303}%
\pgfsetfillcolor{currentfill}%
\pgfsetlinewidth{0.311001pt}%
\definecolor{currentstroke}{rgb}{1.000000,1.000000,1.000000}%
\pgfsetstrokecolor{currentstroke}%
\pgfsetdash{}{0pt}%
\pgfpathmoveto{\pgfqpoint{5.458853in}{1.181260in}}%
\pgfpathcurveto{\pgfqpoint{5.465986in}{1.181260in}}{\pgfqpoint{5.472828in}{1.184094in}}{\pgfqpoint{5.477871in}{1.189137in}}%
\pgfpathcurveto{\pgfqpoint{5.482915in}{1.194181in}}{\pgfqpoint{5.485749in}{1.201023in}}{\pgfqpoint{5.485749in}{1.208155in}}%
\pgfpathcurveto{\pgfqpoint{5.485749in}{1.215288in}}{\pgfqpoint{5.482915in}{1.222130in}}{\pgfqpoint{5.477871in}{1.227174in}}%
\pgfpathcurveto{\pgfqpoint{5.472828in}{1.232217in}}{\pgfqpoint{5.465986in}{1.235051in}}{\pgfqpoint{5.458853in}{1.235051in}}%
\pgfpathcurveto{\pgfqpoint{5.451720in}{1.235051in}}{\pgfqpoint{5.444879in}{1.232217in}}{\pgfqpoint{5.439835in}{1.227174in}}%
\pgfpathcurveto{\pgfqpoint{5.434791in}{1.222130in}}{\pgfqpoint{5.431957in}{1.215288in}}{\pgfqpoint{5.431957in}{1.208155in}}%
\pgfpathcurveto{\pgfqpoint{5.431957in}{1.201023in}}{\pgfqpoint{5.434791in}{1.194181in}}{\pgfqpoint{5.439835in}{1.189137in}}%
\pgfpathcurveto{\pgfqpoint{5.444879in}{1.184094in}}{\pgfqpoint{5.451720in}{1.181260in}}{\pgfqpoint{5.458853in}{1.181260in}}%
\pgfpathclose%
\pgfusepath{stroke,fill}%
\end{pgfscope}%
\begin{pgfscope}%
\pgfpathrectangle{\pgfqpoint{4.985294in}{0.500000in}}{\pgfqpoint{1.764706in}{1.700000in}}%
\pgfusepath{clip}%
\pgfsetbuttcap%
\pgfsetroundjoin%
\definecolor{currentfill}{rgb}{0.978376,0.897317,0.831308}%
\pgfsetfillcolor{currentfill}%
\pgfsetlinewidth{0.311001pt}%
\definecolor{currentstroke}{rgb}{1.000000,1.000000,1.000000}%
\pgfsetstrokecolor{currentstroke}%
\pgfsetdash{}{0pt}%
\pgfpathmoveto{\pgfqpoint{6.337894in}{1.233621in}}%
\pgfpathcurveto{\pgfqpoint{6.345026in}{1.233621in}}{\pgfqpoint{6.351868in}{1.236455in}}{\pgfqpoint{6.356912in}{1.241498in}}%
\pgfpathcurveto{\pgfqpoint{6.361955in}{1.246542in}}{\pgfqpoint{6.364789in}{1.253384in}}{\pgfqpoint{6.364789in}{1.260517in}}%
\pgfpathcurveto{\pgfqpoint{6.364789in}{1.267649in}}{\pgfqpoint{6.361955in}{1.274491in}}{\pgfqpoint{6.356912in}{1.279535in}}%
\pgfpathcurveto{\pgfqpoint{6.351868in}{1.284578in}}{\pgfqpoint{6.345026in}{1.287412in}}{\pgfqpoint{6.337894in}{1.287412in}}%
\pgfpathcurveto{\pgfqpoint{6.330761in}{1.287412in}}{\pgfqpoint{6.323919in}{1.284578in}}{\pgfqpoint{6.318875in}{1.279535in}}%
\pgfpathcurveto{\pgfqpoint{6.313832in}{1.274491in}}{\pgfqpoint{6.310998in}{1.267649in}}{\pgfqpoint{6.310998in}{1.260517in}}%
\pgfpathcurveto{\pgfqpoint{6.310998in}{1.253384in}}{\pgfqpoint{6.313832in}{1.246542in}}{\pgfqpoint{6.318875in}{1.241498in}}%
\pgfpathcurveto{\pgfqpoint{6.323919in}{1.236455in}}{\pgfqpoint{6.330761in}{1.233621in}}{\pgfqpoint{6.337894in}{1.233621in}}%
\pgfpathclose%
\pgfusepath{stroke,fill}%
\end{pgfscope}%
\begin{pgfscope}%
\pgfpathrectangle{\pgfqpoint{4.985294in}{0.500000in}}{\pgfqpoint{1.764706in}{1.700000in}}%
\pgfusepath{clip}%
\pgfsetbuttcap%
\pgfsetroundjoin%
\definecolor{currentfill}{rgb}{0.965440,0.720101,0.576404}%
\pgfsetfillcolor{currentfill}%
\pgfsetlinewidth{0.311001pt}%
\definecolor{currentstroke}{rgb}{1.000000,1.000000,1.000000}%
\pgfsetstrokecolor{currentstroke}%
\pgfsetdash{}{0pt}%
\pgfpathmoveto{\pgfqpoint{5.603950in}{0.936423in}}%
\pgfpathcurveto{\pgfqpoint{5.611083in}{0.936423in}}{\pgfqpoint{5.617924in}{0.939257in}}{\pgfqpoint{5.622968in}{0.944301in}}%
\pgfpathcurveto{\pgfqpoint{5.628012in}{0.949345in}}{\pgfqpoint{5.630845in}{0.956186in}}{\pgfqpoint{5.630845in}{0.963319in}}%
\pgfpathcurveto{\pgfqpoint{5.630845in}{0.970452in}}{\pgfqpoint{5.628012in}{0.977294in}}{\pgfqpoint{5.622968in}{0.982337in}}%
\pgfpathcurveto{\pgfqpoint{5.617924in}{0.987381in}}{\pgfqpoint{5.611083in}{0.990215in}}{\pgfqpoint{5.603950in}{0.990215in}}%
\pgfpathcurveto{\pgfqpoint{5.596817in}{0.990215in}}{\pgfqpoint{5.589975in}{0.987381in}}{\pgfqpoint{5.584932in}{0.982337in}}%
\pgfpathcurveto{\pgfqpoint{5.579888in}{0.977294in}}{\pgfqpoint{5.577054in}{0.970452in}}{\pgfqpoint{5.577054in}{0.963319in}}%
\pgfpathcurveto{\pgfqpoint{5.577054in}{0.956186in}}{\pgfqpoint{5.579888in}{0.949345in}}{\pgfqpoint{5.584932in}{0.944301in}}%
\pgfpathcurveto{\pgfqpoint{5.589975in}{0.939257in}}{\pgfqpoint{5.596817in}{0.936423in}}{\pgfqpoint{5.603950in}{0.936423in}}%
\pgfpathclose%
\pgfusepath{stroke,fill}%
\end{pgfscope}%
\begin{pgfscope}%
\pgfpathrectangle{\pgfqpoint{4.985294in}{0.500000in}}{\pgfqpoint{1.764706in}{1.700000in}}%
\pgfusepath{clip}%
\pgfsetbuttcap%
\pgfsetroundjoin%
\definecolor{currentfill}{rgb}{0.965042,0.701564,0.552889}%
\pgfsetfillcolor{currentfill}%
\pgfsetlinewidth{0.311001pt}%
\definecolor{currentstroke}{rgb}{1.000000,1.000000,1.000000}%
\pgfsetstrokecolor{currentstroke}%
\pgfsetdash{}{0pt}%
\pgfpathmoveto{\pgfqpoint{5.401125in}{1.007176in}}%
\pgfpathcurveto{\pgfqpoint{5.408258in}{1.007176in}}{\pgfqpoint{5.415100in}{1.010009in}}{\pgfqpoint{5.420143in}{1.015053in}}%
\pgfpathcurveto{\pgfqpoint{5.425187in}{1.020097in}}{\pgfqpoint{5.428021in}{1.026938in}}{\pgfqpoint{5.428021in}{1.034071in}}%
\pgfpathcurveto{\pgfqpoint{5.428021in}{1.041204in}}{\pgfqpoint{5.425187in}{1.048046in}}{\pgfqpoint{5.420143in}{1.053089in}}%
\pgfpathcurveto{\pgfqpoint{5.415100in}{1.058133in}}{\pgfqpoint{5.408258in}{1.060967in}}{\pgfqpoint{5.401125in}{1.060967in}}%
\pgfpathcurveto{\pgfqpoint{5.393992in}{1.060967in}}{\pgfqpoint{5.387151in}{1.058133in}}{\pgfqpoint{5.382107in}{1.053089in}}%
\pgfpathcurveto{\pgfqpoint{5.377063in}{1.048046in}}{\pgfqpoint{5.374230in}{1.041204in}}{\pgfqpoint{5.374230in}{1.034071in}}%
\pgfpathcurveto{\pgfqpoint{5.374230in}{1.026938in}}{\pgfqpoint{5.377063in}{1.020097in}}{\pgfqpoint{5.382107in}{1.015053in}}%
\pgfpathcurveto{\pgfqpoint{5.387151in}{1.010009in}}{\pgfqpoint{5.393992in}{1.007176in}}{\pgfqpoint{5.401125in}{1.007176in}}%
\pgfpathclose%
\pgfusepath{stroke,fill}%
\end{pgfscope}%
\begin{pgfscope}%
\pgfpathrectangle{\pgfqpoint{4.985294in}{0.500000in}}{\pgfqpoint{1.764706in}{1.700000in}}%
\pgfusepath{clip}%
\pgfsetbuttcap%
\pgfsetroundjoin%
\definecolor{currentfill}{rgb}{0.233340,0.102637,0.256977}%
\pgfsetfillcolor{currentfill}%
\pgfsetlinewidth{0.311001pt}%
\definecolor{currentstroke}{rgb}{1.000000,1.000000,1.000000}%
\pgfsetstrokecolor{currentstroke}%
\pgfsetdash{}{0pt}%
\pgfpathmoveto{\pgfqpoint{5.902230in}{1.568592in}}%
\pgfpathcurveto{\pgfqpoint{5.909362in}{1.568592in}}{\pgfqpoint{5.916204in}{1.571426in}}{\pgfqpoint{5.921248in}{1.576469in}}%
\pgfpathcurveto{\pgfqpoint{5.926291in}{1.581513in}}{\pgfqpoint{5.929125in}{1.588355in}}{\pgfqpoint{5.929125in}{1.595487in}}%
\pgfpathcurveto{\pgfqpoint{5.929125in}{1.602620in}}{\pgfqpoint{5.926291in}{1.609462in}}{\pgfqpoint{5.921248in}{1.614506in}}%
\pgfpathcurveto{\pgfqpoint{5.916204in}{1.619549in}}{\pgfqpoint{5.909362in}{1.622383in}}{\pgfqpoint{5.902230in}{1.622383in}}%
\pgfpathcurveto{\pgfqpoint{5.895097in}{1.622383in}}{\pgfqpoint{5.888255in}{1.619549in}}{\pgfqpoint{5.883211in}{1.614506in}}%
\pgfpathcurveto{\pgfqpoint{5.878168in}{1.609462in}}{\pgfqpoint{5.875334in}{1.602620in}}{\pgfqpoint{5.875334in}{1.595487in}}%
\pgfpathcurveto{\pgfqpoint{5.875334in}{1.588355in}}{\pgfqpoint{5.878168in}{1.581513in}}{\pgfqpoint{5.883211in}{1.576469in}}%
\pgfpathcurveto{\pgfqpoint{5.888255in}{1.571426in}}{\pgfqpoint{5.895097in}{1.568592in}}{\pgfqpoint{5.902230in}{1.568592in}}%
\pgfpathclose%
\pgfusepath{stroke,fill}%
\end{pgfscope}%
\begin{pgfscope}%
\pgfpathrectangle{\pgfqpoint{4.985294in}{0.500000in}}{\pgfqpoint{1.764706in}{1.700000in}}%
\pgfusepath{clip}%
\pgfsetbuttcap%
\pgfsetroundjoin%
\definecolor{currentfill}{rgb}{0.975018,0.868213,0.788710}%
\pgfsetfillcolor{currentfill}%
\pgfsetlinewidth{0.311001pt}%
\definecolor{currentstroke}{rgb}{1.000000,1.000000,1.000000}%
\pgfsetstrokecolor{currentstroke}%
\pgfsetdash{}{0pt}%
\pgfpathmoveto{\pgfqpoint{5.477555in}{1.145094in}}%
\pgfpathcurveto{\pgfqpoint{5.484688in}{1.145094in}}{\pgfqpoint{5.491529in}{1.147928in}}{\pgfqpoint{5.496573in}{1.152971in}}%
\pgfpathcurveto{\pgfqpoint{5.501617in}{1.158015in}}{\pgfqpoint{5.504451in}{1.164857in}}{\pgfqpoint{5.504451in}{1.171989in}}%
\pgfpathcurveto{\pgfqpoint{5.504451in}{1.179122in}}{\pgfqpoint{5.501617in}{1.185964in}}{\pgfqpoint{5.496573in}{1.191008in}}%
\pgfpathcurveto{\pgfqpoint{5.491529in}{1.196051in}}{\pgfqpoint{5.484688in}{1.198885in}}{\pgfqpoint{5.477555in}{1.198885in}}%
\pgfpathcurveto{\pgfqpoint{5.470422in}{1.198885in}}{\pgfqpoint{5.463580in}{1.196051in}}{\pgfqpoint{5.458537in}{1.191008in}}%
\pgfpathcurveto{\pgfqpoint{5.453493in}{1.185964in}}{\pgfqpoint{5.450659in}{1.179122in}}{\pgfqpoint{5.450659in}{1.171989in}}%
\pgfpathcurveto{\pgfqpoint{5.450659in}{1.164857in}}{\pgfqpoint{5.453493in}{1.158015in}}{\pgfqpoint{5.458537in}{1.152971in}}%
\pgfpathcurveto{\pgfqpoint{5.463580in}{1.147928in}}{\pgfqpoint{5.470422in}{1.145094in}}{\pgfqpoint{5.477555in}{1.145094in}}%
\pgfpathclose%
\pgfusepath{stroke,fill}%
\end{pgfscope}%
\begin{pgfscope}%
\pgfpathrectangle{\pgfqpoint{4.985294in}{0.500000in}}{\pgfqpoint{1.764706in}{1.700000in}}%
\pgfusepath{clip}%
\pgfsetbuttcap%
\pgfsetroundjoin%
\definecolor{currentfill}{rgb}{0.950851,0.435000,0.297228}%
\pgfsetfillcolor{currentfill}%
\pgfsetlinewidth{0.311001pt}%
\definecolor{currentstroke}{rgb}{1.000000,1.000000,1.000000}%
\pgfsetstrokecolor{currentstroke}%
\pgfsetdash{}{0pt}%
\pgfpathmoveto{\pgfqpoint{6.448308in}{1.388729in}}%
\pgfpathcurveto{\pgfqpoint{6.455441in}{1.388729in}}{\pgfqpoint{6.462282in}{1.391563in}}{\pgfqpoint{6.467326in}{1.396606in}}%
\pgfpathcurveto{\pgfqpoint{6.472370in}{1.401650in}}{\pgfqpoint{6.475204in}{1.408492in}}{\pgfqpoint{6.475204in}{1.415625in}}%
\pgfpathcurveto{\pgfqpoint{6.475204in}{1.422757in}}{\pgfqpoint{6.472370in}{1.429599in}}{\pgfqpoint{6.467326in}{1.434643in}}%
\pgfpathcurveto{\pgfqpoint{6.462282in}{1.439686in}}{\pgfqpoint{6.455441in}{1.442520in}}{\pgfqpoint{6.448308in}{1.442520in}}%
\pgfpathcurveto{\pgfqpoint{6.441175in}{1.442520in}}{\pgfqpoint{6.434333in}{1.439686in}}{\pgfqpoint{6.429290in}{1.434643in}}%
\pgfpathcurveto{\pgfqpoint{6.424246in}{1.429599in}}{\pgfqpoint{6.421412in}{1.422757in}}{\pgfqpoint{6.421412in}{1.415625in}}%
\pgfpathcurveto{\pgfqpoint{6.421412in}{1.408492in}}{\pgfqpoint{6.424246in}{1.401650in}}{\pgfqpoint{6.429290in}{1.396606in}}%
\pgfpathcurveto{\pgfqpoint{6.434333in}{1.391563in}}{\pgfqpoint{6.441175in}{1.388729in}}{\pgfqpoint{6.448308in}{1.388729in}}%
\pgfpathclose%
\pgfusepath{stroke,fill}%
\end{pgfscope}%
\begin{pgfscope}%
\pgfpathrectangle{\pgfqpoint{4.985294in}{0.500000in}}{\pgfqpoint{1.764706in}{1.700000in}}%
\pgfusepath{clip}%
\pgfsetbuttcap%
\pgfsetroundjoin%
\definecolor{currentfill}{rgb}{0.967092,0.768560,0.642079}%
\pgfsetfillcolor{currentfill}%
\pgfsetlinewidth{0.311001pt}%
\definecolor{currentstroke}{rgb}{1.000000,1.000000,1.000000}%
\pgfsetstrokecolor{currentstroke}%
\pgfsetdash{}{0pt}%
\pgfpathmoveto{\pgfqpoint{5.349110in}{1.208720in}}%
\pgfpathcurveto{\pgfqpoint{5.356243in}{1.208720in}}{\pgfqpoint{5.363084in}{1.211554in}}{\pgfqpoint{5.368128in}{1.216598in}}%
\pgfpathcurveto{\pgfqpoint{5.373171in}{1.221642in}}{\pgfqpoint{5.376005in}{1.228483in}}{\pgfqpoint{5.376005in}{1.235616in}}%
\pgfpathcurveto{\pgfqpoint{5.376005in}{1.242749in}}{\pgfqpoint{5.373171in}{1.249590in}}{\pgfqpoint{5.368128in}{1.254634in}}%
\pgfpathcurveto{\pgfqpoint{5.363084in}{1.259678in}}{\pgfqpoint{5.356243in}{1.262512in}}{\pgfqpoint{5.349110in}{1.262512in}}%
\pgfpathcurveto{\pgfqpoint{5.341977in}{1.262512in}}{\pgfqpoint{5.335135in}{1.259678in}}{\pgfqpoint{5.330092in}{1.254634in}}%
\pgfpathcurveto{\pgfqpoint{5.325048in}{1.249590in}}{\pgfqpoint{5.322214in}{1.242749in}}{\pgfqpoint{5.322214in}{1.235616in}}%
\pgfpathcurveto{\pgfqpoint{5.322214in}{1.228483in}}{\pgfqpoint{5.325048in}{1.221642in}}{\pgfqpoint{5.330092in}{1.216598in}}%
\pgfpathcurveto{\pgfqpoint{5.335135in}{1.211554in}}{\pgfqpoint{5.341977in}{1.208720in}}{\pgfqpoint{5.349110in}{1.208720in}}%
\pgfpathclose%
\pgfusepath{stroke,fill}%
\end{pgfscope}%
\begin{pgfscope}%
\pgfpathrectangle{\pgfqpoint{4.985294in}{0.500000in}}{\pgfqpoint{1.764706in}{1.700000in}}%
\pgfusepath{clip}%
\pgfsetbuttcap%
\pgfsetroundjoin%
\definecolor{currentfill}{rgb}{0.965928,0.738443,0.600540}%
\pgfsetfillcolor{currentfill}%
\pgfsetlinewidth{0.311001pt}%
\definecolor{currentstroke}{rgb}{1.000000,1.000000,1.000000}%
\pgfsetstrokecolor{currentstroke}%
\pgfsetdash{}{0pt}%
\pgfpathmoveto{\pgfqpoint{6.383993in}{1.459208in}}%
\pgfpathcurveto{\pgfqpoint{6.391126in}{1.459208in}}{\pgfqpoint{6.397968in}{1.462042in}}{\pgfqpoint{6.403012in}{1.467085in}}%
\pgfpathcurveto{\pgfqpoint{6.408055in}{1.472129in}}{\pgfqpoint{6.410889in}{1.478970in}}{\pgfqpoint{6.410889in}{1.486103in}}%
\pgfpathcurveto{\pgfqpoint{6.410889in}{1.493236in}}{\pgfqpoint{6.408055in}{1.500078in}}{\pgfqpoint{6.403012in}{1.505121in}}%
\pgfpathcurveto{\pgfqpoint{6.397968in}{1.510165in}}{\pgfqpoint{6.391126in}{1.512999in}}{\pgfqpoint{6.383993in}{1.512999in}}%
\pgfpathcurveto{\pgfqpoint{6.376861in}{1.512999in}}{\pgfqpoint{6.370019in}{1.510165in}}{\pgfqpoint{6.364975in}{1.505121in}}%
\pgfpathcurveto{\pgfqpoint{6.359932in}{1.500078in}}{\pgfqpoint{6.357098in}{1.493236in}}{\pgfqpoint{6.357098in}{1.486103in}}%
\pgfpathcurveto{\pgfqpoint{6.357098in}{1.478970in}}{\pgfqpoint{6.359932in}{1.472129in}}{\pgfqpoint{6.364975in}{1.467085in}}%
\pgfpathcurveto{\pgfqpoint{6.370019in}{1.462042in}}{\pgfqpoint{6.376861in}{1.459208in}}{\pgfqpoint{6.383993in}{1.459208in}}%
\pgfpathclose%
\pgfusepath{stroke,fill}%
\end{pgfscope}%
\begin{pgfscope}%
\pgfpathrectangle{\pgfqpoint{4.985294in}{0.500000in}}{\pgfqpoint{1.764706in}{1.700000in}}%
\pgfusepath{clip}%
\pgfsetbuttcap%
\pgfsetroundjoin%
\definecolor{currentfill}{rgb}{0.980678,0.914765,0.856766}%
\pgfsetfillcolor{currentfill}%
\pgfsetlinewidth{0.311001pt}%
\definecolor{currentstroke}{rgb}{1.000000,1.000000,1.000000}%
\pgfsetstrokecolor{currentstroke}%
\pgfsetdash{}{0pt}%
\pgfpathmoveto{\pgfqpoint{5.403773in}{1.315568in}}%
\pgfpathcurveto{\pgfqpoint{5.410906in}{1.315568in}}{\pgfqpoint{5.417747in}{1.318402in}}{\pgfqpoint{5.422791in}{1.323446in}}%
\pgfpathcurveto{\pgfqpoint{5.427835in}{1.328489in}}{\pgfqpoint{5.430668in}{1.335331in}}{\pgfqpoint{5.430668in}{1.342464in}}%
\pgfpathcurveto{\pgfqpoint{5.430668in}{1.349597in}}{\pgfqpoint{5.427835in}{1.356438in}}{\pgfqpoint{5.422791in}{1.361482in}}%
\pgfpathcurveto{\pgfqpoint{5.417747in}{1.366526in}}{\pgfqpoint{5.410906in}{1.369360in}}{\pgfqpoint{5.403773in}{1.369360in}}%
\pgfpathcurveto{\pgfqpoint{5.396640in}{1.369360in}}{\pgfqpoint{5.389798in}{1.366526in}}{\pgfqpoint{5.384755in}{1.361482in}}%
\pgfpathcurveto{\pgfqpoint{5.379711in}{1.356438in}}{\pgfqpoint{5.376877in}{1.349597in}}{\pgfqpoint{5.376877in}{1.342464in}}%
\pgfpathcurveto{\pgfqpoint{5.376877in}{1.335331in}}{\pgfqpoint{5.379711in}{1.328489in}}{\pgfqpoint{5.384755in}{1.323446in}}%
\pgfpathcurveto{\pgfqpoint{5.389798in}{1.318402in}}{\pgfqpoint{5.396640in}{1.315568in}}{\pgfqpoint{5.403773in}{1.315568in}}%
\pgfpathclose%
\pgfusepath{stroke,fill}%
\end{pgfscope}%
\begin{pgfscope}%
\pgfpathrectangle{\pgfqpoint{4.985294in}{0.500000in}}{\pgfqpoint{1.764706in}{1.700000in}}%
\pgfusepath{clip}%
\pgfsetbuttcap%
\pgfsetroundjoin%
\definecolor{currentfill}{rgb}{0.698038,0.088972,0.346299}%
\pgfsetfillcolor{currentfill}%
\pgfsetlinewidth{0.311001pt}%
\definecolor{currentstroke}{rgb}{1.000000,1.000000,1.000000}%
\pgfsetstrokecolor{currentstroke}%
\pgfsetdash{}{0pt}%
\pgfpathmoveto{\pgfqpoint{6.141820in}{1.902330in}}%
\pgfpathcurveto{\pgfqpoint{6.148952in}{1.902330in}}{\pgfqpoint{6.155794in}{1.905164in}}{\pgfqpoint{6.160838in}{1.910207in}}%
\pgfpathcurveto{\pgfqpoint{6.165881in}{1.915251in}}{\pgfqpoint{6.168715in}{1.922093in}}{\pgfqpoint{6.168715in}{1.929226in}}%
\pgfpathcurveto{\pgfqpoint{6.168715in}{1.936358in}}{\pgfqpoint{6.165881in}{1.943200in}}{\pgfqpoint{6.160838in}{1.948244in}}%
\pgfpathcurveto{\pgfqpoint{6.155794in}{1.953287in}}{\pgfqpoint{6.148952in}{1.956121in}}{\pgfqpoint{6.141820in}{1.956121in}}%
\pgfpathcurveto{\pgfqpoint{6.134687in}{1.956121in}}{\pgfqpoint{6.127845in}{1.953287in}}{\pgfqpoint{6.122801in}{1.948244in}}%
\pgfpathcurveto{\pgfqpoint{6.117758in}{1.943200in}}{\pgfqpoint{6.114924in}{1.936358in}}{\pgfqpoint{6.114924in}{1.929226in}}%
\pgfpathcurveto{\pgfqpoint{6.114924in}{1.922093in}}{\pgfqpoint{6.117758in}{1.915251in}}{\pgfqpoint{6.122801in}{1.910207in}}%
\pgfpathcurveto{\pgfqpoint{6.127845in}{1.905164in}}{\pgfqpoint{6.134687in}{1.902330in}}{\pgfqpoint{6.141820in}{1.902330in}}%
\pgfpathclose%
\pgfusepath{stroke,fill}%
\end{pgfscope}%
\begin{pgfscope}%
\pgfpathrectangle{\pgfqpoint{4.985294in}{0.500000in}}{\pgfqpoint{1.764706in}{1.700000in}}%
\pgfusepath{clip}%
\pgfsetbuttcap%
\pgfsetroundjoin%
\definecolor{currentfill}{rgb}{0.010608,0.018082,0.100187}%
\pgfsetfillcolor{currentfill}%
\pgfsetlinewidth{0.311001pt}%
\definecolor{currentstroke}{rgb}{1.000000,1.000000,1.000000}%
\pgfsetstrokecolor{currentstroke}%
\pgfsetdash{}{0pt}%
\pgfpathmoveto{\pgfqpoint{6.593933in}{1.299280in}}%
\pgfpathcurveto{\pgfqpoint{6.601066in}{1.299280in}}{\pgfqpoint{6.607908in}{1.302114in}}{\pgfqpoint{6.612952in}{1.307158in}}%
\pgfpathcurveto{\pgfqpoint{6.617995in}{1.312202in}}{\pgfqpoint{6.620829in}{1.319043in}}{\pgfqpoint{6.620829in}{1.326176in}}%
\pgfpathcurveto{\pgfqpoint{6.620829in}{1.333309in}}{\pgfqpoint{6.617995in}{1.340151in}}{\pgfqpoint{6.612952in}{1.345194in}}%
\pgfpathcurveto{\pgfqpoint{6.607908in}{1.350238in}}{\pgfqpoint{6.601066in}{1.353072in}}{\pgfqpoint{6.593933in}{1.353072in}}%
\pgfpathcurveto{\pgfqpoint{6.586801in}{1.353072in}}{\pgfqpoint{6.579959in}{1.350238in}}{\pgfqpoint{6.574915in}{1.345194in}}%
\pgfpathcurveto{\pgfqpoint{6.569872in}{1.340151in}}{\pgfqpoint{6.567038in}{1.333309in}}{\pgfqpoint{6.567038in}{1.326176in}}%
\pgfpathcurveto{\pgfqpoint{6.567038in}{1.319043in}}{\pgfqpoint{6.569872in}{1.312202in}}{\pgfqpoint{6.574915in}{1.307158in}}%
\pgfpathcurveto{\pgfqpoint{6.579959in}{1.302114in}}{\pgfqpoint{6.586801in}{1.299280in}}{\pgfqpoint{6.593933in}{1.299280in}}%
\pgfpathclose%
\pgfusepath{stroke,fill}%
\end{pgfscope}%
\begin{pgfscope}%
\pgfpathrectangle{\pgfqpoint{4.985294in}{0.500000in}}{\pgfqpoint{1.764706in}{1.700000in}}%
\pgfusepath{clip}%
\pgfsetbuttcap%
\pgfsetroundjoin%
\definecolor{currentfill}{rgb}{0.964306,0.663930,0.507747}%
\pgfsetfillcolor{currentfill}%
\pgfsetlinewidth{0.311001pt}%
\definecolor{currentstroke}{rgb}{1.000000,1.000000,1.000000}%
\pgfsetstrokecolor{currentstroke}%
\pgfsetdash{}{0pt}%
\pgfpathmoveto{\pgfqpoint{5.544988in}{1.184596in}}%
\pgfpathcurveto{\pgfqpoint{5.552121in}{1.184596in}}{\pgfqpoint{5.558963in}{1.187430in}}{\pgfqpoint{5.564006in}{1.192474in}}%
\pgfpathcurveto{\pgfqpoint{5.569050in}{1.197518in}}{\pgfqpoint{5.571884in}{1.204359in}}{\pgfqpoint{5.571884in}{1.211492in}}%
\pgfpathcurveto{\pgfqpoint{5.571884in}{1.218625in}}{\pgfqpoint{5.569050in}{1.225467in}}{\pgfqpoint{5.564006in}{1.230510in}}%
\pgfpathcurveto{\pgfqpoint{5.558963in}{1.235554in}}{\pgfqpoint{5.552121in}{1.238388in}}{\pgfqpoint{5.544988in}{1.238388in}}%
\pgfpathcurveto{\pgfqpoint{5.537855in}{1.238388in}}{\pgfqpoint{5.531014in}{1.235554in}}{\pgfqpoint{5.525970in}{1.230510in}}%
\pgfpathcurveto{\pgfqpoint{5.520926in}{1.225467in}}{\pgfqpoint{5.518093in}{1.218625in}}{\pgfqpoint{5.518093in}{1.211492in}}%
\pgfpathcurveto{\pgfqpoint{5.518093in}{1.204359in}}{\pgfqpoint{5.520926in}{1.197518in}}{\pgfqpoint{5.525970in}{1.192474in}}%
\pgfpathcurveto{\pgfqpoint{5.531014in}{1.187430in}}{\pgfqpoint{5.537855in}{1.184596in}}{\pgfqpoint{5.544988in}{1.184596in}}%
\pgfpathclose%
\pgfusepath{stroke,fill}%
\end{pgfscope}%
\begin{pgfscope}%
\pgfpathrectangle{\pgfqpoint{4.985294in}{0.500000in}}{\pgfqpoint{1.764706in}{1.700000in}}%
\pgfusepath{clip}%
\pgfsetbuttcap%
\pgfsetroundjoin%
\definecolor{currentfill}{rgb}{0.963559,0.632016,0.472047}%
\pgfsetfillcolor{currentfill}%
\pgfsetlinewidth{0.311001pt}%
\definecolor{currentstroke}{rgb}{1.000000,1.000000,1.000000}%
\pgfsetstrokecolor{currentstroke}%
\pgfsetdash{}{0pt}%
\pgfpathmoveto{\pgfqpoint{6.135429in}{1.082861in}}%
\pgfpathcurveto{\pgfqpoint{6.142562in}{1.082861in}}{\pgfqpoint{6.149404in}{1.085695in}}{\pgfqpoint{6.154448in}{1.090738in}}%
\pgfpathcurveto{\pgfqpoint{6.159491in}{1.095782in}}{\pgfqpoint{6.162325in}{1.102624in}}{\pgfqpoint{6.162325in}{1.109757in}}%
\pgfpathcurveto{\pgfqpoint{6.162325in}{1.116889in}}{\pgfqpoint{6.159491in}{1.123731in}}{\pgfqpoint{6.154448in}{1.128775in}}%
\pgfpathcurveto{\pgfqpoint{6.149404in}{1.133818in}}{\pgfqpoint{6.142562in}{1.136652in}}{\pgfqpoint{6.135429in}{1.136652in}}%
\pgfpathcurveto{\pgfqpoint{6.128297in}{1.136652in}}{\pgfqpoint{6.121455in}{1.133818in}}{\pgfqpoint{6.116411in}{1.128775in}}%
\pgfpathcurveto{\pgfqpoint{6.111368in}{1.123731in}}{\pgfqpoint{6.108534in}{1.116889in}}{\pgfqpoint{6.108534in}{1.109757in}}%
\pgfpathcurveto{\pgfqpoint{6.108534in}{1.102624in}}{\pgfqpoint{6.111368in}{1.095782in}}{\pgfqpoint{6.116411in}{1.090738in}}%
\pgfpathcurveto{\pgfqpoint{6.121455in}{1.085695in}}{\pgfqpoint{6.128297in}{1.082861in}}{\pgfqpoint{6.135429in}{1.082861in}}%
\pgfpathclose%
\pgfusepath{stroke,fill}%
\end{pgfscope}%
\begin{pgfscope}%
\pgfpathrectangle{\pgfqpoint{4.985294in}{0.500000in}}{\pgfqpoint{1.764706in}{1.700000in}}%
\pgfusepath{clip}%
\pgfsetbuttcap%
\pgfsetroundjoin%
\definecolor{currentfill}{rgb}{0.962018,0.586477,0.424918}%
\pgfsetfillcolor{currentfill}%
\pgfsetlinewidth{0.311001pt}%
\definecolor{currentstroke}{rgb}{1.000000,1.000000,1.000000}%
\pgfsetstrokecolor{currentstroke}%
\pgfsetdash{}{0pt}%
\pgfpathmoveto{\pgfqpoint{5.565631in}{1.768263in}}%
\pgfpathcurveto{\pgfqpoint{5.572764in}{1.768263in}}{\pgfqpoint{5.579606in}{1.771096in}}{\pgfqpoint{5.584649in}{1.776140in}}%
\pgfpathcurveto{\pgfqpoint{5.589693in}{1.781184in}}{\pgfqpoint{5.592527in}{1.788025in}}{\pgfqpoint{5.592527in}{1.795158in}}%
\pgfpathcurveto{\pgfqpoint{5.592527in}{1.802291in}}{\pgfqpoint{5.589693in}{1.809133in}}{\pgfqpoint{5.584649in}{1.814176in}}%
\pgfpathcurveto{\pgfqpoint{5.579606in}{1.819220in}}{\pgfqpoint{5.572764in}{1.822054in}}{\pgfqpoint{5.565631in}{1.822054in}}%
\pgfpathcurveto{\pgfqpoint{5.558498in}{1.822054in}}{\pgfqpoint{5.551657in}{1.819220in}}{\pgfqpoint{5.546613in}{1.814176in}}%
\pgfpathcurveto{\pgfqpoint{5.541569in}{1.809133in}}{\pgfqpoint{5.538736in}{1.802291in}}{\pgfqpoint{5.538736in}{1.795158in}}%
\pgfpathcurveto{\pgfqpoint{5.538736in}{1.788025in}}{\pgfqpoint{5.541569in}{1.781184in}}{\pgfqpoint{5.546613in}{1.776140in}}%
\pgfpathcurveto{\pgfqpoint{5.551657in}{1.771096in}}{\pgfqpoint{5.558498in}{1.768263in}}{\pgfqpoint{5.565631in}{1.768263in}}%
\pgfpathclose%
\pgfusepath{stroke,fill}%
\end{pgfscope}%
\begin{pgfscope}%
\pgfpathrectangle{\pgfqpoint{4.985294in}{0.500000in}}{\pgfqpoint{1.764706in}{1.700000in}}%
\pgfusepath{clip}%
\pgfsetbuttcap%
\pgfsetroundjoin%
\definecolor{currentfill}{rgb}{0.976287,0.879862,0.805788}%
\pgfsetfillcolor{currentfill}%
\pgfsetlinewidth{0.311001pt}%
\definecolor{currentstroke}{rgb}{1.000000,1.000000,1.000000}%
\pgfsetstrokecolor{currentstroke}%
\pgfsetdash{}{0pt}%
\pgfpathmoveto{\pgfqpoint{6.247461in}{1.095698in}}%
\pgfpathcurveto{\pgfqpoint{6.254593in}{1.095698in}}{\pgfqpoint{6.261435in}{1.098532in}}{\pgfqpoint{6.266479in}{1.103576in}}%
\pgfpathcurveto{\pgfqpoint{6.271522in}{1.108620in}}{\pgfqpoint{6.274356in}{1.115461in}}{\pgfqpoint{6.274356in}{1.122594in}}%
\pgfpathcurveto{\pgfqpoint{6.274356in}{1.129727in}}{\pgfqpoint{6.271522in}{1.136569in}}{\pgfqpoint{6.266479in}{1.141612in}}%
\pgfpathcurveto{\pgfqpoint{6.261435in}{1.146656in}}{\pgfqpoint{6.254593in}{1.149490in}}{\pgfqpoint{6.247461in}{1.149490in}}%
\pgfpathcurveto{\pgfqpoint{6.240328in}{1.149490in}}{\pgfqpoint{6.233486in}{1.146656in}}{\pgfqpoint{6.228442in}{1.141612in}}%
\pgfpathcurveto{\pgfqpoint{6.223399in}{1.136569in}}{\pgfqpoint{6.220565in}{1.129727in}}{\pgfqpoint{6.220565in}{1.122594in}}%
\pgfpathcurveto{\pgfqpoint{6.220565in}{1.115461in}}{\pgfqpoint{6.223399in}{1.108620in}}{\pgfqpoint{6.228442in}{1.103576in}}%
\pgfpathcurveto{\pgfqpoint{6.233486in}{1.098532in}}{\pgfqpoint{6.240328in}{1.095698in}}{\pgfqpoint{6.247461in}{1.095698in}}%
\pgfpathclose%
\pgfusepath{stroke,fill}%
\end{pgfscope}%
\begin{pgfscope}%
\pgfpathrectangle{\pgfqpoint{4.985294in}{0.500000in}}{\pgfqpoint{1.764706in}{1.700000in}}%
\pgfusepath{clip}%
\pgfsetbuttcap%
\pgfsetroundjoin%
\definecolor{currentfill}{rgb}{0.977657,0.891500,0.822809}%
\pgfsetfillcolor{currentfill}%
\pgfsetlinewidth{0.311001pt}%
\definecolor{currentstroke}{rgb}{1.000000,1.000000,1.000000}%
\pgfsetstrokecolor{currentstroke}%
\pgfsetdash{}{0pt}%
\pgfpathmoveto{\pgfqpoint{6.276254in}{1.125055in}}%
\pgfpathcurveto{\pgfqpoint{6.283387in}{1.125055in}}{\pgfqpoint{6.290228in}{1.127889in}}{\pgfqpoint{6.295272in}{1.132933in}}%
\pgfpathcurveto{\pgfqpoint{6.300316in}{1.137976in}}{\pgfqpoint{6.303149in}{1.144818in}}{\pgfqpoint{6.303149in}{1.151951in}}%
\pgfpathcurveto{\pgfqpoint{6.303149in}{1.159084in}}{\pgfqpoint{6.300316in}{1.165925in}}{\pgfqpoint{6.295272in}{1.170969in}}%
\pgfpathcurveto{\pgfqpoint{6.290228in}{1.176013in}}{\pgfqpoint{6.283387in}{1.178846in}}{\pgfqpoint{6.276254in}{1.178846in}}%
\pgfpathcurveto{\pgfqpoint{6.269121in}{1.178846in}}{\pgfqpoint{6.262279in}{1.176013in}}{\pgfqpoint{6.257236in}{1.170969in}}%
\pgfpathcurveto{\pgfqpoint{6.252192in}{1.165925in}}{\pgfqpoint{6.249358in}{1.159084in}}{\pgfqpoint{6.249358in}{1.151951in}}%
\pgfpathcurveto{\pgfqpoint{6.249358in}{1.144818in}}{\pgfqpoint{6.252192in}{1.137976in}}{\pgfqpoint{6.257236in}{1.132933in}}%
\pgfpathcurveto{\pgfqpoint{6.262279in}{1.127889in}}{\pgfqpoint{6.269121in}{1.125055in}}{\pgfqpoint{6.276254in}{1.125055in}}%
\pgfpathclose%
\pgfusepath{stroke,fill}%
\end{pgfscope}%
\begin{pgfscope}%
\pgfpathrectangle{\pgfqpoint{4.985294in}{0.500000in}}{\pgfqpoint{1.764706in}{1.700000in}}%
\pgfusepath{clip}%
\pgfsetbuttcap%
\pgfsetroundjoin%
\definecolor{currentfill}{rgb}{0.967092,0.768560,0.642079}%
\pgfsetfillcolor{currentfill}%
\pgfsetlinewidth{0.311001pt}%
\definecolor{currentstroke}{rgb}{1.000000,1.000000,1.000000}%
\pgfsetstrokecolor{currentstroke}%
\pgfsetdash{}{0pt}%
\pgfpathmoveto{\pgfqpoint{6.215922in}{1.330149in}}%
\pgfpathcurveto{\pgfqpoint{6.223055in}{1.330149in}}{\pgfqpoint{6.229897in}{1.332983in}}{\pgfqpoint{6.234940in}{1.338027in}}%
\pgfpathcurveto{\pgfqpoint{6.239984in}{1.343070in}}{\pgfqpoint{6.242818in}{1.349912in}}{\pgfqpoint{6.242818in}{1.357045in}}%
\pgfpathcurveto{\pgfqpoint{6.242818in}{1.364178in}}{\pgfqpoint{6.239984in}{1.371019in}}{\pgfqpoint{6.234940in}{1.376063in}}%
\pgfpathcurveto{\pgfqpoint{6.229897in}{1.381107in}}{\pgfqpoint{6.223055in}{1.383941in}}{\pgfqpoint{6.215922in}{1.383941in}}%
\pgfpathcurveto{\pgfqpoint{6.208790in}{1.383941in}}{\pgfqpoint{6.201948in}{1.381107in}}{\pgfqpoint{6.196904in}{1.376063in}}%
\pgfpathcurveto{\pgfqpoint{6.191861in}{1.371019in}}{\pgfqpoint{6.189027in}{1.364178in}}{\pgfqpoint{6.189027in}{1.357045in}}%
\pgfpathcurveto{\pgfqpoint{6.189027in}{1.349912in}}{\pgfqpoint{6.191861in}{1.343070in}}{\pgfqpoint{6.196904in}{1.338027in}}%
\pgfpathcurveto{\pgfqpoint{6.201948in}{1.332983in}}{\pgfqpoint{6.208790in}{1.330149in}}{\pgfqpoint{6.215922in}{1.330149in}}%
\pgfpathclose%
\pgfusepath{stroke,fill}%
\end{pgfscope}%
\begin{pgfscope}%
\pgfpathrectangle{\pgfqpoint{4.985294in}{0.500000in}}{\pgfqpoint{1.764706in}{1.700000in}}%
\pgfusepath{clip}%
\pgfsetbuttcap%
\pgfsetroundjoin%
\definecolor{currentfill}{rgb}{0.974412,0.862387,0.780156}%
\pgfsetfillcolor{currentfill}%
\pgfsetlinewidth{0.311001pt}%
\definecolor{currentstroke}{rgb}{1.000000,1.000000,1.000000}%
\pgfsetstrokecolor{currentstroke}%
\pgfsetdash{}{0pt}%
\pgfpathmoveto{\pgfqpoint{6.222910in}{1.087324in}}%
\pgfpathcurveto{\pgfqpoint{6.230043in}{1.087324in}}{\pgfqpoint{6.236885in}{1.090158in}}{\pgfqpoint{6.241928in}{1.095202in}}%
\pgfpathcurveto{\pgfqpoint{6.246972in}{1.100245in}}{\pgfqpoint{6.249806in}{1.107087in}}{\pgfqpoint{6.249806in}{1.114220in}}%
\pgfpathcurveto{\pgfqpoint{6.249806in}{1.121353in}}{\pgfqpoint{6.246972in}{1.128194in}}{\pgfqpoint{6.241928in}{1.133238in}}%
\pgfpathcurveto{\pgfqpoint{6.236885in}{1.138282in}}{\pgfqpoint{6.230043in}{1.141115in}}{\pgfqpoint{6.222910in}{1.141115in}}%
\pgfpathcurveto{\pgfqpoint{6.215777in}{1.141115in}}{\pgfqpoint{6.208936in}{1.138282in}}{\pgfqpoint{6.203892in}{1.133238in}}%
\pgfpathcurveto{\pgfqpoint{6.198848in}{1.128194in}}{\pgfqpoint{6.196014in}{1.121353in}}{\pgfqpoint{6.196014in}{1.114220in}}%
\pgfpathcurveto{\pgfqpoint{6.196014in}{1.107087in}}{\pgfqpoint{6.198848in}{1.100245in}}{\pgfqpoint{6.203892in}{1.095202in}}%
\pgfpathcurveto{\pgfqpoint{6.208936in}{1.090158in}}{\pgfqpoint{6.215777in}{1.087324in}}{\pgfqpoint{6.222910in}{1.087324in}}%
\pgfpathclose%
\pgfusepath{stroke,fill}%
\end{pgfscope}%
\begin{pgfscope}%
\pgfpathrectangle{\pgfqpoint{4.985294in}{0.500000in}}{\pgfqpoint{1.764706in}{1.700000in}}%
\pgfusepath{clip}%
\pgfsetbuttcap%
\pgfsetroundjoin%
\definecolor{currentfill}{rgb}{0.977657,0.891500,0.822809}%
\pgfsetfillcolor{currentfill}%
\pgfsetlinewidth{0.311001pt}%
\definecolor{currentstroke}{rgb}{1.000000,1.000000,1.000000}%
\pgfsetstrokecolor{currentstroke}%
\pgfsetdash{}{0pt}%
\pgfpathmoveto{\pgfqpoint{6.348352in}{1.375659in}}%
\pgfpathcurveto{\pgfqpoint{6.355485in}{1.375659in}}{\pgfqpoint{6.362326in}{1.378493in}}{\pgfqpoint{6.367370in}{1.383536in}}%
\pgfpathcurveto{\pgfqpoint{6.372414in}{1.388580in}}{\pgfqpoint{6.375248in}{1.395422in}}{\pgfqpoint{6.375248in}{1.402555in}}%
\pgfpathcurveto{\pgfqpoint{6.375248in}{1.409687in}}{\pgfqpoint{6.372414in}{1.416529in}}{\pgfqpoint{6.367370in}{1.421573in}}%
\pgfpathcurveto{\pgfqpoint{6.362326in}{1.426616in}}{\pgfqpoint{6.355485in}{1.429450in}}{\pgfqpoint{6.348352in}{1.429450in}}%
\pgfpathcurveto{\pgfqpoint{6.341219in}{1.429450in}}{\pgfqpoint{6.334378in}{1.426616in}}{\pgfqpoint{6.329334in}{1.421573in}}%
\pgfpathcurveto{\pgfqpoint{6.324290in}{1.416529in}}{\pgfqpoint{6.321456in}{1.409687in}}{\pgfqpoint{6.321456in}{1.402555in}}%
\pgfpathcurveto{\pgfqpoint{6.321456in}{1.395422in}}{\pgfqpoint{6.324290in}{1.388580in}}{\pgfqpoint{6.329334in}{1.383536in}}%
\pgfpathcurveto{\pgfqpoint{6.334378in}{1.378493in}}{\pgfqpoint{6.341219in}{1.375659in}}{\pgfqpoint{6.348352in}{1.375659in}}%
\pgfpathclose%
\pgfusepath{stroke,fill}%
\end{pgfscope}%
\begin{pgfscope}%
\pgfpathrectangle{\pgfqpoint{4.985294in}{0.500000in}}{\pgfqpoint{1.764706in}{1.700000in}}%
\pgfusepath{clip}%
\pgfsetbuttcap%
\pgfsetroundjoin%
\definecolor{currentfill}{rgb}{0.973832,0.856556,0.771584}%
\pgfsetfillcolor{currentfill}%
\pgfsetlinewidth{0.311001pt}%
\definecolor{currentstroke}{rgb}{1.000000,1.000000,1.000000}%
\pgfsetstrokecolor{currentstroke}%
\pgfsetdash{}{0pt}%
\pgfpathmoveto{\pgfqpoint{6.286636in}{1.077481in}}%
\pgfpathcurveto{\pgfqpoint{6.293769in}{1.077481in}}{\pgfqpoint{6.300611in}{1.080315in}}{\pgfqpoint{6.305655in}{1.085359in}}%
\pgfpathcurveto{\pgfqpoint{6.310698in}{1.090402in}}{\pgfqpoint{6.313532in}{1.097244in}}{\pgfqpoint{6.313532in}{1.104377in}}%
\pgfpathcurveto{\pgfqpoint{6.313532in}{1.111510in}}{\pgfqpoint{6.310698in}{1.118351in}}{\pgfqpoint{6.305655in}{1.123395in}}%
\pgfpathcurveto{\pgfqpoint{6.300611in}{1.128439in}}{\pgfqpoint{6.293769in}{1.131272in}}{\pgfqpoint{6.286636in}{1.131272in}}%
\pgfpathcurveto{\pgfqpoint{6.279504in}{1.131272in}}{\pgfqpoint{6.272662in}{1.128439in}}{\pgfqpoint{6.267618in}{1.123395in}}%
\pgfpathcurveto{\pgfqpoint{6.262575in}{1.118351in}}{\pgfqpoint{6.259741in}{1.111510in}}{\pgfqpoint{6.259741in}{1.104377in}}%
\pgfpathcurveto{\pgfqpoint{6.259741in}{1.097244in}}{\pgfqpoint{6.262575in}{1.090402in}}{\pgfqpoint{6.267618in}{1.085359in}}%
\pgfpathcurveto{\pgfqpoint{6.272662in}{1.080315in}}{\pgfqpoint{6.279504in}{1.077481in}}{\pgfqpoint{6.286636in}{1.077481in}}%
\pgfpathclose%
\pgfusepath{stroke,fill}%
\end{pgfscope}%
\begin{pgfscope}%
\pgfpathrectangle{\pgfqpoint{4.985294in}{0.500000in}}{\pgfqpoint{1.764706in}{1.700000in}}%
\pgfusepath{clip}%
\pgfsetbuttcap%
\pgfsetroundjoin%
\definecolor{currentfill}{rgb}{0.968509,0.792226,0.676405}%
\pgfsetfillcolor{currentfill}%
\pgfsetlinewidth{0.311001pt}%
\definecolor{currentstroke}{rgb}{1.000000,1.000000,1.000000}%
\pgfsetstrokecolor{currentstroke}%
\pgfsetdash{}{0pt}%
\pgfpathmoveto{\pgfqpoint{6.170064in}{1.031889in}}%
\pgfpathcurveto{\pgfqpoint{6.177197in}{1.031889in}}{\pgfqpoint{6.184038in}{1.034723in}}{\pgfqpoint{6.189082in}{1.039766in}}%
\pgfpathcurveto{\pgfqpoint{6.194126in}{1.044810in}}{\pgfqpoint{6.196960in}{1.051652in}}{\pgfqpoint{6.196960in}{1.058784in}}%
\pgfpathcurveto{\pgfqpoint{6.196960in}{1.065917in}}{\pgfqpoint{6.194126in}{1.072759in}}{\pgfqpoint{6.189082in}{1.077803in}}%
\pgfpathcurveto{\pgfqpoint{6.184038in}{1.082846in}}{\pgfqpoint{6.177197in}{1.085680in}}{\pgfqpoint{6.170064in}{1.085680in}}%
\pgfpathcurveto{\pgfqpoint{6.162931in}{1.085680in}}{\pgfqpoint{6.156089in}{1.082846in}}{\pgfqpoint{6.151046in}{1.077803in}}%
\pgfpathcurveto{\pgfqpoint{6.146002in}{1.072759in}}{\pgfqpoint{6.143168in}{1.065917in}}{\pgfqpoint{6.143168in}{1.058784in}}%
\pgfpathcurveto{\pgfqpoint{6.143168in}{1.051652in}}{\pgfqpoint{6.146002in}{1.044810in}}{\pgfqpoint{6.151046in}{1.039766in}}%
\pgfpathcurveto{\pgfqpoint{6.156089in}{1.034723in}}{\pgfqpoint{6.162931in}{1.031889in}}{\pgfqpoint{6.170064in}{1.031889in}}%
\pgfpathclose%
\pgfusepath{stroke,fill}%
\end{pgfscope}%
\begin{pgfscope}%
\pgfpathrectangle{\pgfqpoint{4.985294in}{0.500000in}}{\pgfqpoint{1.764706in}{1.700000in}}%
\pgfusepath{clip}%
\pgfsetbuttcap%
\pgfsetroundjoin%
\definecolor{currentfill}{rgb}{0.974412,0.862387,0.780156}%
\pgfsetfillcolor{currentfill}%
\pgfsetlinewidth{0.311001pt}%
\definecolor{currentstroke}{rgb}{1.000000,1.000000,1.000000}%
\pgfsetstrokecolor{currentstroke}%
\pgfsetdash{}{0pt}%
\pgfpathmoveto{\pgfqpoint{5.370446in}{1.351766in}}%
\pgfpathcurveto{\pgfqpoint{5.377579in}{1.351766in}}{\pgfqpoint{5.384420in}{1.354600in}}{\pgfqpoint{5.389464in}{1.359644in}}%
\pgfpathcurveto{\pgfqpoint{5.394508in}{1.364688in}}{\pgfqpoint{5.397342in}{1.371529in}}{\pgfqpoint{5.397342in}{1.378662in}}%
\pgfpathcurveto{\pgfqpoint{5.397342in}{1.385795in}}{\pgfqpoint{5.394508in}{1.392637in}}{\pgfqpoint{5.389464in}{1.397680in}}%
\pgfpathcurveto{\pgfqpoint{5.384420in}{1.402724in}}{\pgfqpoint{5.377579in}{1.405558in}}{\pgfqpoint{5.370446in}{1.405558in}}%
\pgfpathcurveto{\pgfqpoint{5.363313in}{1.405558in}}{\pgfqpoint{5.356471in}{1.402724in}}{\pgfqpoint{5.351428in}{1.397680in}}%
\pgfpathcurveto{\pgfqpoint{5.346384in}{1.392637in}}{\pgfqpoint{5.343550in}{1.385795in}}{\pgfqpoint{5.343550in}{1.378662in}}%
\pgfpathcurveto{\pgfqpoint{5.343550in}{1.371529in}}{\pgfqpoint{5.346384in}{1.364688in}}{\pgfqpoint{5.351428in}{1.359644in}}%
\pgfpathcurveto{\pgfqpoint{5.356471in}{1.354600in}}{\pgfqpoint{5.363313in}{1.351766in}}{\pgfqpoint{5.370446in}{1.351766in}}%
\pgfpathclose%
\pgfusepath{stroke,fill}%
\end{pgfscope}%
\begin{pgfscope}%
\pgfpathrectangle{\pgfqpoint{4.985294in}{0.500000in}}{\pgfqpoint{1.764706in}{1.700000in}}%
\pgfusepath{clip}%
\pgfsetbuttcap%
\pgfsetroundjoin%
\definecolor{currentfill}{rgb}{0.965169,0.707764,0.560659}%
\pgfsetfillcolor{currentfill}%
\pgfsetlinewidth{0.311001pt}%
\definecolor{currentstroke}{rgb}{1.000000,1.000000,1.000000}%
\pgfsetstrokecolor{currentstroke}%
\pgfsetdash{}{0pt}%
\pgfpathmoveto{\pgfqpoint{5.553873in}{1.712681in}}%
\pgfpathcurveto{\pgfqpoint{5.561006in}{1.712681in}}{\pgfqpoint{5.567847in}{1.715514in}}{\pgfqpoint{5.572891in}{1.720558in}}%
\pgfpathcurveto{\pgfqpoint{5.577935in}{1.725602in}}{\pgfqpoint{5.580769in}{1.732443in}}{\pgfqpoint{5.580769in}{1.739576in}}%
\pgfpathcurveto{\pgfqpoint{5.580769in}{1.746709in}}{\pgfqpoint{5.577935in}{1.753551in}}{\pgfqpoint{5.572891in}{1.758594in}}%
\pgfpathcurveto{\pgfqpoint{5.567847in}{1.763638in}}{\pgfqpoint{5.561006in}{1.766472in}}{\pgfqpoint{5.553873in}{1.766472in}}%
\pgfpathcurveto{\pgfqpoint{5.546740in}{1.766472in}}{\pgfqpoint{5.539898in}{1.763638in}}{\pgfqpoint{5.534855in}{1.758594in}}%
\pgfpathcurveto{\pgfqpoint{5.529811in}{1.753551in}}{\pgfqpoint{5.526977in}{1.746709in}}{\pgfqpoint{5.526977in}{1.739576in}}%
\pgfpathcurveto{\pgfqpoint{5.526977in}{1.732443in}}{\pgfqpoint{5.529811in}{1.725602in}}{\pgfqpoint{5.534855in}{1.720558in}}%
\pgfpathcurveto{\pgfqpoint{5.539898in}{1.715514in}}{\pgfqpoint{5.546740in}{1.712681in}}{\pgfqpoint{5.553873in}{1.712681in}}%
\pgfpathclose%
\pgfusepath{stroke,fill}%
\end{pgfscope}%
\begin{pgfscope}%
\pgfpathrectangle{\pgfqpoint{4.985294in}{0.500000in}}{\pgfqpoint{1.764706in}{1.700000in}}%
\pgfusepath{clip}%
\pgfsetbuttcap%
\pgfsetroundjoin%
\definecolor{currentfill}{rgb}{0.974412,0.862387,0.780156}%
\pgfsetfillcolor{currentfill}%
\pgfsetlinewidth{0.311001pt}%
\definecolor{currentstroke}{rgb}{1.000000,1.000000,1.000000}%
\pgfsetstrokecolor{currentstroke}%
\pgfsetdash{}{0pt}%
\pgfpathmoveto{\pgfqpoint{6.367349in}{1.301304in}}%
\pgfpathcurveto{\pgfqpoint{6.374482in}{1.301304in}}{\pgfqpoint{6.381323in}{1.304138in}}{\pgfqpoint{6.386367in}{1.309182in}}%
\pgfpathcurveto{\pgfqpoint{6.391411in}{1.314225in}}{\pgfqpoint{6.394245in}{1.321067in}}{\pgfqpoint{6.394245in}{1.328200in}}%
\pgfpathcurveto{\pgfqpoint{6.394245in}{1.335332in}}{\pgfqpoint{6.391411in}{1.342174in}}{\pgfqpoint{6.386367in}{1.347218in}}%
\pgfpathcurveto{\pgfqpoint{6.381323in}{1.352261in}}{\pgfqpoint{6.374482in}{1.355095in}}{\pgfqpoint{6.367349in}{1.355095in}}%
\pgfpathcurveto{\pgfqpoint{6.360216in}{1.355095in}}{\pgfqpoint{6.353374in}{1.352261in}}{\pgfqpoint{6.348331in}{1.347218in}}%
\pgfpathcurveto{\pgfqpoint{6.343287in}{1.342174in}}{\pgfqpoint{6.340453in}{1.335332in}}{\pgfqpoint{6.340453in}{1.328200in}}%
\pgfpathcurveto{\pgfqpoint{6.340453in}{1.321067in}}{\pgfqpoint{6.343287in}{1.314225in}}{\pgfqpoint{6.348331in}{1.309182in}}%
\pgfpathcurveto{\pgfqpoint{6.353374in}{1.304138in}}{\pgfqpoint{6.360216in}{1.301304in}}{\pgfqpoint{6.367349in}{1.301304in}}%
\pgfpathclose%
\pgfusepath{stroke,fill}%
\end{pgfscope}%
\begin{pgfscope}%
\pgfpathrectangle{\pgfqpoint{4.985294in}{0.500000in}}{\pgfqpoint{1.764706in}{1.700000in}}%
\pgfusepath{clip}%
\pgfsetbuttcap%
\pgfsetroundjoin%
\definecolor{currentfill}{rgb}{0.965302,0.713942,0.568499}%
\pgfsetfillcolor{currentfill}%
\pgfsetlinewidth{0.311001pt}%
\definecolor{currentstroke}{rgb}{1.000000,1.000000,1.000000}%
\pgfsetstrokecolor{currentstroke}%
\pgfsetdash{}{0pt}%
\pgfpathmoveto{\pgfqpoint{6.199761in}{1.360594in}}%
\pgfpathcurveto{\pgfqpoint{6.206894in}{1.360594in}}{\pgfqpoint{6.213735in}{1.363428in}}{\pgfqpoint{6.218779in}{1.368471in}}%
\pgfpathcurveto{\pgfqpoint{6.223823in}{1.373515in}}{\pgfqpoint{6.226657in}{1.380357in}}{\pgfqpoint{6.226657in}{1.387489in}}%
\pgfpathcurveto{\pgfqpoint{6.226657in}{1.394622in}}{\pgfqpoint{6.223823in}{1.401464in}}{\pgfqpoint{6.218779in}{1.406508in}}%
\pgfpathcurveto{\pgfqpoint{6.213735in}{1.411551in}}{\pgfqpoint{6.206894in}{1.414385in}}{\pgfqpoint{6.199761in}{1.414385in}}%
\pgfpathcurveto{\pgfqpoint{6.192628in}{1.414385in}}{\pgfqpoint{6.185787in}{1.411551in}}{\pgfqpoint{6.180743in}{1.406508in}}%
\pgfpathcurveto{\pgfqpoint{6.175699in}{1.401464in}}{\pgfqpoint{6.172865in}{1.394622in}}{\pgfqpoint{6.172865in}{1.387489in}}%
\pgfpathcurveto{\pgfqpoint{6.172865in}{1.380357in}}{\pgfqpoint{6.175699in}{1.373515in}}{\pgfqpoint{6.180743in}{1.368471in}}%
\pgfpathcurveto{\pgfqpoint{6.185787in}{1.363428in}}{\pgfqpoint{6.192628in}{1.360594in}}{\pgfqpoint{6.199761in}{1.360594in}}%
\pgfpathclose%
\pgfusepath{stroke,fill}%
\end{pgfscope}%
\begin{pgfscope}%
\pgfpathrectangle{\pgfqpoint{4.985294in}{0.500000in}}{\pgfqpoint{1.764706in}{1.700000in}}%
\pgfusepath{clip}%
\pgfsetbuttcap%
\pgfsetroundjoin%
\definecolor{currentfill}{rgb}{0.975018,0.868213,0.788710}%
\pgfsetfillcolor{currentfill}%
\pgfsetlinewidth{0.311001pt}%
\definecolor{currentstroke}{rgb}{1.000000,1.000000,1.000000}%
\pgfsetstrokecolor{currentstroke}%
\pgfsetdash{}{0pt}%
\pgfpathmoveto{\pgfqpoint{5.470025in}{1.553845in}}%
\pgfpathcurveto{\pgfqpoint{5.477158in}{1.553845in}}{\pgfqpoint{5.484000in}{1.556679in}}{\pgfqpoint{5.489043in}{1.561723in}}%
\pgfpathcurveto{\pgfqpoint{5.494087in}{1.566766in}}{\pgfqpoint{5.496921in}{1.573608in}}{\pgfqpoint{5.496921in}{1.580741in}}%
\pgfpathcurveto{\pgfqpoint{5.496921in}{1.587874in}}{\pgfqpoint{5.494087in}{1.594715in}}{\pgfqpoint{5.489043in}{1.599759in}}%
\pgfpathcurveto{\pgfqpoint{5.484000in}{1.604803in}}{\pgfqpoint{5.477158in}{1.607637in}}{\pgfqpoint{5.470025in}{1.607637in}}%
\pgfpathcurveto{\pgfqpoint{5.462892in}{1.607637in}}{\pgfqpoint{5.456051in}{1.604803in}}{\pgfqpoint{5.451007in}{1.599759in}}%
\pgfpathcurveto{\pgfqpoint{5.445964in}{1.594715in}}{\pgfqpoint{5.443130in}{1.587874in}}{\pgfqpoint{5.443130in}{1.580741in}}%
\pgfpathcurveto{\pgfqpoint{5.443130in}{1.573608in}}{\pgfqpoint{5.445964in}{1.566766in}}{\pgfqpoint{5.451007in}{1.561723in}}%
\pgfpathcurveto{\pgfqpoint{5.456051in}{1.556679in}}{\pgfqpoint{5.462892in}{1.553845in}}{\pgfqpoint{5.470025in}{1.553845in}}%
\pgfpathclose%
\pgfusepath{stroke,fill}%
\end{pgfscope}%
\begin{pgfscope}%
\pgfpathrectangle{\pgfqpoint{4.985294in}{0.500000in}}{\pgfqpoint{1.764706in}{1.700000in}}%
\pgfusepath{clip}%
\pgfsetbuttcap%
\pgfsetroundjoin%
\definecolor{currentfill}{rgb}{0.968105,0.786346,0.667739}%
\pgfsetfillcolor{currentfill}%
\pgfsetlinewidth{0.311001pt}%
\definecolor{currentstroke}{rgb}{1.000000,1.000000,1.000000}%
\pgfsetstrokecolor{currentstroke}%
\pgfsetdash{}{0pt}%
\pgfpathmoveto{\pgfqpoint{6.254677in}{0.987933in}}%
\pgfpathcurveto{\pgfqpoint{6.261809in}{0.987933in}}{\pgfqpoint{6.268651in}{0.990766in}}{\pgfqpoint{6.273695in}{0.995810in}}%
\pgfpathcurveto{\pgfqpoint{6.278738in}{1.000854in}}{\pgfqpoint{6.281572in}{1.007695in}}{\pgfqpoint{6.281572in}{1.014828in}}%
\pgfpathcurveto{\pgfqpoint{6.281572in}{1.021961in}}{\pgfqpoint{6.278738in}{1.028803in}}{\pgfqpoint{6.273695in}{1.033846in}}%
\pgfpathcurveto{\pgfqpoint{6.268651in}{1.038890in}}{\pgfqpoint{6.261809in}{1.041724in}}{\pgfqpoint{6.254677in}{1.041724in}}%
\pgfpathcurveto{\pgfqpoint{6.247544in}{1.041724in}}{\pgfqpoint{6.240702in}{1.038890in}}{\pgfqpoint{6.235659in}{1.033846in}}%
\pgfpathcurveto{\pgfqpoint{6.230615in}{1.028803in}}{\pgfqpoint{6.227781in}{1.021961in}}{\pgfqpoint{6.227781in}{1.014828in}}%
\pgfpathcurveto{\pgfqpoint{6.227781in}{1.007695in}}{\pgfqpoint{6.230615in}{1.000854in}}{\pgfqpoint{6.235659in}{0.995810in}}%
\pgfpathcurveto{\pgfqpoint{6.240702in}{0.990766in}}{\pgfqpoint{6.247544in}{0.987933in}}{\pgfqpoint{6.254677in}{0.987933in}}%
\pgfpathclose%
\pgfusepath{stroke,fill}%
\end{pgfscope}%
\begin{pgfscope}%
\pgfpathrectangle{\pgfqpoint{4.985294in}{0.500000in}}{\pgfqpoint{1.764706in}{1.700000in}}%
\pgfusepath{clip}%
\pgfsetbuttcap%
\pgfsetroundjoin%
\definecolor{currentfill}{rgb}{0.969803,0.809811,0.702523}%
\pgfsetfillcolor{currentfill}%
\pgfsetlinewidth{0.311001pt}%
\definecolor{currentstroke}{rgb}{1.000000,1.000000,1.000000}%
\pgfsetstrokecolor{currentstroke}%
\pgfsetdash{}{0pt}%
\pgfpathmoveto{\pgfqpoint{6.195820in}{1.127613in}}%
\pgfpathcurveto{\pgfqpoint{6.202952in}{1.127613in}}{\pgfqpoint{6.209794in}{1.130447in}}{\pgfqpoint{6.214838in}{1.135491in}}%
\pgfpathcurveto{\pgfqpoint{6.219881in}{1.140535in}}{\pgfqpoint{6.222715in}{1.147376in}}{\pgfqpoint{6.222715in}{1.154509in}}%
\pgfpathcurveto{\pgfqpoint{6.222715in}{1.161642in}}{\pgfqpoint{6.219881in}{1.168484in}}{\pgfqpoint{6.214838in}{1.173527in}}%
\pgfpathcurveto{\pgfqpoint{6.209794in}{1.178571in}}{\pgfqpoint{6.202952in}{1.181405in}}{\pgfqpoint{6.195820in}{1.181405in}}%
\pgfpathcurveto{\pgfqpoint{6.188687in}{1.181405in}}{\pgfqpoint{6.181845in}{1.178571in}}{\pgfqpoint{6.176801in}{1.173527in}}%
\pgfpathcurveto{\pgfqpoint{6.171758in}{1.168484in}}{\pgfqpoint{6.168924in}{1.161642in}}{\pgfqpoint{6.168924in}{1.154509in}}%
\pgfpathcurveto{\pgfqpoint{6.168924in}{1.147376in}}{\pgfqpoint{6.171758in}{1.140535in}}{\pgfqpoint{6.176801in}{1.135491in}}%
\pgfpathcurveto{\pgfqpoint{6.181845in}{1.130447in}}{\pgfqpoint{6.188687in}{1.127613in}}{\pgfqpoint{6.195820in}{1.127613in}}%
\pgfpathclose%
\pgfusepath{stroke,fill}%
\end{pgfscope}%
\begin{pgfscope}%
\pgfpathrectangle{\pgfqpoint{4.985294in}{0.500000in}}{\pgfqpoint{1.764706in}{1.700000in}}%
\pgfusepath{clip}%
\pgfsetbuttcap%
\pgfsetroundjoin%
\definecolor{currentfill}{rgb}{0.962532,0.599594,0.438051}%
\pgfsetfillcolor{currentfill}%
\pgfsetlinewidth{0.311001pt}%
\definecolor{currentstroke}{rgb}{1.000000,1.000000,1.000000}%
\pgfsetstrokecolor{currentstroke}%
\pgfsetdash{}{0pt}%
\pgfpathmoveto{\pgfqpoint{5.627778in}{1.645232in}}%
\pgfpathcurveto{\pgfqpoint{5.634910in}{1.645232in}}{\pgfqpoint{5.641752in}{1.648066in}}{\pgfqpoint{5.646796in}{1.653109in}}%
\pgfpathcurveto{\pgfqpoint{5.651839in}{1.658153in}}{\pgfqpoint{5.654673in}{1.664995in}}{\pgfqpoint{5.654673in}{1.672128in}}%
\pgfpathcurveto{\pgfqpoint{5.654673in}{1.679260in}}{\pgfqpoint{5.651839in}{1.686102in}}{\pgfqpoint{5.646796in}{1.691146in}}%
\pgfpathcurveto{\pgfqpoint{5.641752in}{1.696189in}}{\pgfqpoint{5.634910in}{1.699023in}}{\pgfqpoint{5.627778in}{1.699023in}}%
\pgfpathcurveto{\pgfqpoint{5.620645in}{1.699023in}}{\pgfqpoint{5.613803in}{1.696189in}}{\pgfqpoint{5.608760in}{1.691146in}}%
\pgfpathcurveto{\pgfqpoint{5.603716in}{1.686102in}}{\pgfqpoint{5.600882in}{1.679260in}}{\pgfqpoint{5.600882in}{1.672128in}}%
\pgfpathcurveto{\pgfqpoint{5.600882in}{1.664995in}}{\pgfqpoint{5.603716in}{1.658153in}}{\pgfqpoint{5.608760in}{1.653109in}}%
\pgfpathcurveto{\pgfqpoint{5.613803in}{1.648066in}}{\pgfqpoint{5.620645in}{1.645232in}}{\pgfqpoint{5.627778in}{1.645232in}}%
\pgfpathclose%
\pgfusepath{stroke,fill}%
\end{pgfscope}%
\begin{pgfscope}%
\pgfpathrectangle{\pgfqpoint{4.985294in}{0.500000in}}{\pgfqpoint{1.764706in}{1.700000in}}%
\pgfusepath{clip}%
\pgfsetbuttcap%
\pgfsetroundjoin%
\definecolor{currentfill}{rgb}{0.967092,0.768560,0.642079}%
\pgfsetfillcolor{currentfill}%
\pgfsetlinewidth{0.311001pt}%
\definecolor{currentstroke}{rgb}{1.000000,1.000000,1.000000}%
\pgfsetstrokecolor{currentstroke}%
\pgfsetdash{}{0pt}%
\pgfpathmoveto{\pgfqpoint{6.396744in}{1.267677in}}%
\pgfpathcurveto{\pgfqpoint{6.403877in}{1.267677in}}{\pgfqpoint{6.410719in}{1.270511in}}{\pgfqpoint{6.415762in}{1.275555in}}%
\pgfpathcurveto{\pgfqpoint{6.420806in}{1.280598in}}{\pgfqpoint{6.423640in}{1.287440in}}{\pgfqpoint{6.423640in}{1.294573in}}%
\pgfpathcurveto{\pgfqpoint{6.423640in}{1.301706in}}{\pgfqpoint{6.420806in}{1.308547in}}{\pgfqpoint{6.415762in}{1.313591in}}%
\pgfpathcurveto{\pgfqpoint{6.410719in}{1.318635in}}{\pgfqpoint{6.403877in}{1.321469in}}{\pgfqpoint{6.396744in}{1.321469in}}%
\pgfpathcurveto{\pgfqpoint{6.389611in}{1.321469in}}{\pgfqpoint{6.382770in}{1.318635in}}{\pgfqpoint{6.377726in}{1.313591in}}%
\pgfpathcurveto{\pgfqpoint{6.372682in}{1.308547in}}{\pgfqpoint{6.369848in}{1.301706in}}{\pgfqpoint{6.369848in}{1.294573in}}%
\pgfpathcurveto{\pgfqpoint{6.369848in}{1.287440in}}{\pgfqpoint{6.372682in}{1.280598in}}{\pgfqpoint{6.377726in}{1.275555in}}%
\pgfpathcurveto{\pgfqpoint{6.382770in}{1.270511in}}{\pgfqpoint{6.389611in}{1.267677in}}{\pgfqpoint{6.396744in}{1.267677in}}%
\pgfpathclose%
\pgfusepath{stroke,fill}%
\end{pgfscope}%
\begin{pgfscope}%
\pgfpathrectangle{\pgfqpoint{4.985294in}{0.500000in}}{\pgfqpoint{1.764706in}{1.700000in}}%
\pgfusepath{clip}%
\pgfsetbuttcap%
\pgfsetroundjoin%
\definecolor{currentfill}{rgb}{0.919781,0.275262,0.242460}%
\pgfsetfillcolor{currentfill}%
\pgfsetlinewidth{0.311001pt}%
\definecolor{currentstroke}{rgb}{1.000000,1.000000,1.000000}%
\pgfsetstrokecolor{currentstroke}%
\pgfsetdash{}{0pt}%
\pgfpathmoveto{\pgfqpoint{5.601269in}{1.835141in}}%
\pgfpathcurveto{\pgfqpoint{5.608402in}{1.835141in}}{\pgfqpoint{5.615244in}{1.837975in}}{\pgfqpoint{5.620287in}{1.843018in}}%
\pgfpathcurveto{\pgfqpoint{5.625331in}{1.848062in}}{\pgfqpoint{5.628165in}{1.854904in}}{\pgfqpoint{5.628165in}{1.862036in}}%
\pgfpathcurveto{\pgfqpoint{5.628165in}{1.869169in}}{\pgfqpoint{5.625331in}{1.876011in}}{\pgfqpoint{5.620287in}{1.881055in}}%
\pgfpathcurveto{\pgfqpoint{5.615244in}{1.886098in}}{\pgfqpoint{5.608402in}{1.888932in}}{\pgfqpoint{5.601269in}{1.888932in}}%
\pgfpathcurveto{\pgfqpoint{5.594136in}{1.888932in}}{\pgfqpoint{5.587295in}{1.886098in}}{\pgfqpoint{5.582251in}{1.881055in}}%
\pgfpathcurveto{\pgfqpoint{5.577207in}{1.876011in}}{\pgfqpoint{5.574373in}{1.869169in}}{\pgfqpoint{5.574373in}{1.862036in}}%
\pgfpathcurveto{\pgfqpoint{5.574373in}{1.854904in}}{\pgfqpoint{5.577207in}{1.848062in}}{\pgfqpoint{5.582251in}{1.843018in}}%
\pgfpathcurveto{\pgfqpoint{5.587295in}{1.837975in}}{\pgfqpoint{5.594136in}{1.835141in}}{\pgfqpoint{5.601269in}{1.835141in}}%
\pgfpathclose%
\pgfusepath{stroke,fill}%
\end{pgfscope}%
\begin{pgfscope}%
\pgfpathrectangle{\pgfqpoint{4.985294in}{0.500000in}}{\pgfqpoint{1.764706in}{1.700000in}}%
\pgfusepath{clip}%
\pgfsetbuttcap%
\pgfsetroundjoin%
\definecolor{currentfill}{rgb}{0.968105,0.786346,0.667739}%
\pgfsetfillcolor{currentfill}%
\pgfsetlinewidth{0.311001pt}%
\definecolor{currentstroke}{rgb}{1.000000,1.000000,1.000000}%
\pgfsetstrokecolor{currentstroke}%
\pgfsetdash{}{0pt}%
\pgfpathmoveto{\pgfqpoint{5.406904in}{1.051144in}}%
\pgfpathcurveto{\pgfqpoint{5.414036in}{1.051144in}}{\pgfqpoint{5.420878in}{1.053978in}}{\pgfqpoint{5.425922in}{1.059021in}}%
\pgfpathcurveto{\pgfqpoint{5.430965in}{1.064065in}}{\pgfqpoint{5.433799in}{1.070906in}}{\pgfqpoint{5.433799in}{1.078039in}}%
\pgfpathcurveto{\pgfqpoint{5.433799in}{1.085172in}}{\pgfqpoint{5.430965in}{1.092014in}}{\pgfqpoint{5.425922in}{1.097057in}}%
\pgfpathcurveto{\pgfqpoint{5.420878in}{1.102101in}}{\pgfqpoint{5.414036in}{1.104935in}}{\pgfqpoint{5.406904in}{1.104935in}}%
\pgfpathcurveto{\pgfqpoint{5.399771in}{1.104935in}}{\pgfqpoint{5.392929in}{1.102101in}}{\pgfqpoint{5.387885in}{1.097057in}}%
\pgfpathcurveto{\pgfqpoint{5.382842in}{1.092014in}}{\pgfqpoint{5.380008in}{1.085172in}}{\pgfqpoint{5.380008in}{1.078039in}}%
\pgfpathcurveto{\pgfqpoint{5.380008in}{1.070906in}}{\pgfqpoint{5.382842in}{1.064065in}}{\pgfqpoint{5.387885in}{1.059021in}}%
\pgfpathcurveto{\pgfqpoint{5.392929in}{1.053978in}}{\pgfqpoint{5.399771in}{1.051144in}}{\pgfqpoint{5.406904in}{1.051144in}}%
\pgfpathclose%
\pgfusepath{stroke,fill}%
\end{pgfscope}%
\begin{pgfscope}%
\pgfpathrectangle{\pgfqpoint{4.985294in}{0.500000in}}{\pgfqpoint{1.764706in}{1.700000in}}%
\pgfusepath{clip}%
\pgfsetbuttcap%
\pgfsetroundjoin%
\definecolor{currentfill}{rgb}{0.962018,0.586477,0.424918}%
\pgfsetfillcolor{currentfill}%
\pgfsetlinewidth{0.311001pt}%
\definecolor{currentstroke}{rgb}{1.000000,1.000000,1.000000}%
\pgfsetstrokecolor{currentstroke}%
\pgfsetdash{}{0pt}%
\pgfpathmoveto{\pgfqpoint{5.299513in}{1.322782in}}%
\pgfpathcurveto{\pgfqpoint{5.306646in}{1.322782in}}{\pgfqpoint{5.313488in}{1.325616in}}{\pgfqpoint{5.318531in}{1.330660in}}%
\pgfpathcurveto{\pgfqpoint{5.323575in}{1.335704in}}{\pgfqpoint{5.326409in}{1.342545in}}{\pgfqpoint{5.326409in}{1.349678in}}%
\pgfpathcurveto{\pgfqpoint{5.326409in}{1.356811in}}{\pgfqpoint{5.323575in}{1.363652in}}{\pgfqpoint{5.318531in}{1.368696in}}%
\pgfpathcurveto{\pgfqpoint{5.313488in}{1.373740in}}{\pgfqpoint{5.306646in}{1.376574in}}{\pgfqpoint{5.299513in}{1.376574in}}%
\pgfpathcurveto{\pgfqpoint{5.292380in}{1.376574in}}{\pgfqpoint{5.285539in}{1.373740in}}{\pgfqpoint{5.280495in}{1.368696in}}%
\pgfpathcurveto{\pgfqpoint{5.275452in}{1.363652in}}{\pgfqpoint{5.272618in}{1.356811in}}{\pgfqpoint{5.272618in}{1.349678in}}%
\pgfpathcurveto{\pgfqpoint{5.272618in}{1.342545in}}{\pgfqpoint{5.275452in}{1.335704in}}{\pgfqpoint{5.280495in}{1.330660in}}%
\pgfpathcurveto{\pgfqpoint{5.285539in}{1.325616in}}{\pgfqpoint{5.292380in}{1.322782in}}{\pgfqpoint{5.299513in}{1.322782in}}%
\pgfpathclose%
\pgfusepath{stroke,fill}%
\end{pgfscope}%
\begin{pgfscope}%
\pgfpathrectangle{\pgfqpoint{4.985294in}{0.500000in}}{\pgfqpoint{1.764706in}{1.700000in}}%
\pgfusepath{clip}%
\pgfsetbuttcap%
\pgfsetroundjoin%
\definecolor{currentfill}{rgb}{0.972201,0.839051,0.745789}%
\pgfsetfillcolor{currentfill}%
\pgfsetlinewidth{0.311001pt}%
\definecolor{currentstroke}{rgb}{1.000000,1.000000,1.000000}%
\pgfsetstrokecolor{currentstroke}%
\pgfsetdash{}{0pt}%
\pgfpathmoveto{\pgfqpoint{6.281699in}{1.627457in}}%
\pgfpathcurveto{\pgfqpoint{6.288832in}{1.627457in}}{\pgfqpoint{6.295674in}{1.630291in}}{\pgfqpoint{6.300718in}{1.635335in}}%
\pgfpathcurveto{\pgfqpoint{6.305761in}{1.640378in}}{\pgfqpoint{6.308595in}{1.647220in}}{\pgfqpoint{6.308595in}{1.654353in}}%
\pgfpathcurveto{\pgfqpoint{6.308595in}{1.661486in}}{\pgfqpoint{6.305761in}{1.668327in}}{\pgfqpoint{6.300718in}{1.673371in}}%
\pgfpathcurveto{\pgfqpoint{6.295674in}{1.678415in}}{\pgfqpoint{6.288832in}{1.681249in}}{\pgfqpoint{6.281699in}{1.681249in}}%
\pgfpathcurveto{\pgfqpoint{6.274567in}{1.681249in}}{\pgfqpoint{6.267725in}{1.678415in}}{\pgfqpoint{6.262681in}{1.673371in}}%
\pgfpathcurveto{\pgfqpoint{6.257638in}{1.668327in}}{\pgfqpoint{6.254804in}{1.661486in}}{\pgfqpoint{6.254804in}{1.654353in}}%
\pgfpathcurveto{\pgfqpoint{6.254804in}{1.647220in}}{\pgfqpoint{6.257638in}{1.640378in}}{\pgfqpoint{6.262681in}{1.635335in}}%
\pgfpathcurveto{\pgfqpoint{6.267725in}{1.630291in}}{\pgfqpoint{6.274567in}{1.627457in}}{\pgfqpoint{6.281699in}{1.627457in}}%
\pgfpathclose%
\pgfusepath{stroke,fill}%
\end{pgfscope}%
\begin{pgfscope}%
\pgfpathrectangle{\pgfqpoint{4.985294in}{0.500000in}}{\pgfqpoint{1.764706in}{1.700000in}}%
\pgfusepath{clip}%
\pgfsetbuttcap%
\pgfsetroundjoin%
\definecolor{currentfill}{rgb}{0.980678,0.914765,0.856766}%
\pgfsetfillcolor{currentfill}%
\pgfsetlinewidth{0.311001pt}%
\definecolor{currentstroke}{rgb}{1.000000,1.000000,1.000000}%
\pgfsetstrokecolor{currentstroke}%
\pgfsetdash{}{0pt}%
\pgfpathmoveto{\pgfqpoint{6.290019in}{1.457543in}}%
\pgfpathcurveto{\pgfqpoint{6.297152in}{1.457543in}}{\pgfqpoint{6.303993in}{1.460377in}}{\pgfqpoint{6.309037in}{1.465421in}}%
\pgfpathcurveto{\pgfqpoint{6.314081in}{1.470465in}}{\pgfqpoint{6.316914in}{1.477306in}}{\pgfqpoint{6.316914in}{1.484439in}}%
\pgfpathcurveto{\pgfqpoint{6.316914in}{1.491572in}}{\pgfqpoint{6.314081in}{1.498414in}}{\pgfqpoint{6.309037in}{1.503457in}}%
\pgfpathcurveto{\pgfqpoint{6.303993in}{1.508501in}}{\pgfqpoint{6.297152in}{1.511335in}}{\pgfqpoint{6.290019in}{1.511335in}}%
\pgfpathcurveto{\pgfqpoint{6.282886in}{1.511335in}}{\pgfqpoint{6.276044in}{1.508501in}}{\pgfqpoint{6.271001in}{1.503457in}}%
\pgfpathcurveto{\pgfqpoint{6.265957in}{1.498414in}}{\pgfqpoint{6.263123in}{1.491572in}}{\pgfqpoint{6.263123in}{1.484439in}}%
\pgfpathcurveto{\pgfqpoint{6.263123in}{1.477306in}}{\pgfqpoint{6.265957in}{1.470465in}}{\pgfqpoint{6.271001in}{1.465421in}}%
\pgfpathcurveto{\pgfqpoint{6.276044in}{1.460377in}}{\pgfqpoint{6.282886in}{1.457543in}}{\pgfqpoint{6.290019in}{1.457543in}}%
\pgfpathclose%
\pgfusepath{stroke,fill}%
\end{pgfscope}%
\begin{pgfscope}%
\pgfpathrectangle{\pgfqpoint{4.985294in}{0.500000in}}{\pgfqpoint{1.764706in}{1.700000in}}%
\pgfusepath{clip}%
\pgfsetbuttcap%
\pgfsetroundjoin%
\definecolor{currentfill}{rgb}{0.979891,0.908948,0.848279}%
\pgfsetfillcolor{currentfill}%
\pgfsetlinewidth{0.311001pt}%
\definecolor{currentstroke}{rgb}{1.000000,1.000000,1.000000}%
\pgfsetstrokecolor{currentstroke}%
\pgfsetdash{}{0pt}%
\pgfpathmoveto{\pgfqpoint{6.329402in}{1.248208in}}%
\pgfpathcurveto{\pgfqpoint{6.336534in}{1.248208in}}{\pgfqpoint{6.343376in}{1.251042in}}{\pgfqpoint{6.348420in}{1.256086in}}%
\pgfpathcurveto{\pgfqpoint{6.353463in}{1.261130in}}{\pgfqpoint{6.356297in}{1.267971in}}{\pgfqpoint{6.356297in}{1.275104in}}%
\pgfpathcurveto{\pgfqpoint{6.356297in}{1.282237in}}{\pgfqpoint{6.353463in}{1.289078in}}{\pgfqpoint{6.348420in}{1.294122in}}%
\pgfpathcurveto{\pgfqpoint{6.343376in}{1.299166in}}{\pgfqpoint{6.336534in}{1.302000in}}{\pgfqpoint{6.329402in}{1.302000in}}%
\pgfpathcurveto{\pgfqpoint{6.322269in}{1.302000in}}{\pgfqpoint{6.315427in}{1.299166in}}{\pgfqpoint{6.310384in}{1.294122in}}%
\pgfpathcurveto{\pgfqpoint{6.305340in}{1.289078in}}{\pgfqpoint{6.302506in}{1.282237in}}{\pgfqpoint{6.302506in}{1.275104in}}%
\pgfpathcurveto{\pgfqpoint{6.302506in}{1.267971in}}{\pgfqpoint{6.305340in}{1.261130in}}{\pgfqpoint{6.310384in}{1.256086in}}%
\pgfpathcurveto{\pgfqpoint{6.315427in}{1.251042in}}{\pgfqpoint{6.322269in}{1.248208in}}{\pgfqpoint{6.329402in}{1.248208in}}%
\pgfpathclose%
\pgfusepath{stroke,fill}%
\end{pgfscope}%
\begin{pgfscope}%
\pgfpathrectangle{\pgfqpoint{4.985294in}{0.500000in}}{\pgfqpoint{1.764706in}{1.700000in}}%
\pgfusepath{clip}%
\pgfsetbuttcap%
\pgfsetroundjoin%
\definecolor{currentfill}{rgb}{0.964920,0.695342,0.545192}%
\pgfsetfillcolor{currentfill}%
\pgfsetlinewidth{0.311001pt}%
\definecolor{currentstroke}{rgb}{1.000000,1.000000,1.000000}%
\pgfsetstrokecolor{currentstroke}%
\pgfsetdash{}{0pt}%
\pgfpathmoveto{\pgfqpoint{6.129132in}{0.981923in}}%
\pgfpathcurveto{\pgfqpoint{6.136265in}{0.981923in}}{\pgfqpoint{6.143106in}{0.984757in}}{\pgfqpoint{6.148150in}{0.989800in}}%
\pgfpathcurveto{\pgfqpoint{6.153194in}{0.994844in}}{\pgfqpoint{6.156028in}{1.001686in}}{\pgfqpoint{6.156028in}{1.008818in}}%
\pgfpathcurveto{\pgfqpoint{6.156028in}{1.015951in}}{\pgfqpoint{6.153194in}{1.022793in}}{\pgfqpoint{6.148150in}{1.027837in}}%
\pgfpathcurveto{\pgfqpoint{6.143106in}{1.032880in}}{\pgfqpoint{6.136265in}{1.035714in}}{\pgfqpoint{6.129132in}{1.035714in}}%
\pgfpathcurveto{\pgfqpoint{6.121999in}{1.035714in}}{\pgfqpoint{6.115157in}{1.032880in}}{\pgfqpoint{6.110114in}{1.027837in}}%
\pgfpathcurveto{\pgfqpoint{6.105070in}{1.022793in}}{\pgfqpoint{6.102236in}{1.015951in}}{\pgfqpoint{6.102236in}{1.008818in}}%
\pgfpathcurveto{\pgfqpoint{6.102236in}{1.001686in}}{\pgfqpoint{6.105070in}{0.994844in}}{\pgfqpoint{6.110114in}{0.989800in}}%
\pgfpathcurveto{\pgfqpoint{6.115157in}{0.984757in}}{\pgfqpoint{6.121999in}{0.981923in}}{\pgfqpoint{6.129132in}{0.981923in}}%
\pgfpathclose%
\pgfusepath{stroke,fill}%
\end{pgfscope}%
\begin{pgfscope}%
\pgfpathrectangle{\pgfqpoint{4.985294in}{0.500000in}}{\pgfqpoint{1.764706in}{1.700000in}}%
\pgfusepath{clip}%
\pgfsetbuttcap%
\pgfsetroundjoin%
\definecolor{currentfill}{rgb}{0.976287,0.879862,0.805788}%
\pgfsetfillcolor{currentfill}%
\pgfsetlinewidth{0.311001pt}%
\definecolor{currentstroke}{rgb}{1.000000,1.000000,1.000000}%
\pgfsetstrokecolor{currentstroke}%
\pgfsetdash{}{0pt}%
\pgfpathmoveto{\pgfqpoint{6.247032in}{1.107215in}}%
\pgfpathcurveto{\pgfqpoint{6.254165in}{1.107215in}}{\pgfqpoint{6.261006in}{1.110049in}}{\pgfqpoint{6.266050in}{1.115093in}}%
\pgfpathcurveto{\pgfqpoint{6.271094in}{1.120137in}}{\pgfqpoint{6.273928in}{1.126978in}}{\pgfqpoint{6.273928in}{1.134111in}}%
\pgfpathcurveto{\pgfqpoint{6.273928in}{1.141244in}}{\pgfqpoint{6.271094in}{1.148085in}}{\pgfqpoint{6.266050in}{1.153129in}}%
\pgfpathcurveto{\pgfqpoint{6.261006in}{1.158173in}}{\pgfqpoint{6.254165in}{1.161007in}}{\pgfqpoint{6.247032in}{1.161007in}}%
\pgfpathcurveto{\pgfqpoint{6.239899in}{1.161007in}}{\pgfqpoint{6.233057in}{1.158173in}}{\pgfqpoint{6.228014in}{1.153129in}}%
\pgfpathcurveto{\pgfqpoint{6.222970in}{1.148085in}}{\pgfqpoint{6.220136in}{1.141244in}}{\pgfqpoint{6.220136in}{1.134111in}}%
\pgfpathcurveto{\pgfqpoint{6.220136in}{1.126978in}}{\pgfqpoint{6.222970in}{1.120137in}}{\pgfqpoint{6.228014in}{1.115093in}}%
\pgfpathcurveto{\pgfqpoint{6.233057in}{1.110049in}}{\pgfqpoint{6.239899in}{1.107215in}}{\pgfqpoint{6.247032in}{1.107215in}}%
\pgfpathclose%
\pgfusepath{stroke,fill}%
\end{pgfscope}%
\begin{pgfscope}%
\pgfpathrectangle{\pgfqpoint{4.985294in}{0.500000in}}{\pgfqpoint{1.764706in}{1.700000in}}%
\pgfusepath{clip}%
\pgfsetbuttcap%
\pgfsetroundjoin%
\definecolor{currentfill}{rgb}{0.968931,0.798091,0.685123}%
\pgfsetfillcolor{currentfill}%
\pgfsetlinewidth{0.311001pt}%
\definecolor{currentstroke}{rgb}{1.000000,1.000000,1.000000}%
\pgfsetstrokecolor{currentstroke}%
\pgfsetdash{}{0pt}%
\pgfpathmoveto{\pgfqpoint{5.454781in}{1.635406in}}%
\pgfpathcurveto{\pgfqpoint{5.461914in}{1.635406in}}{\pgfqpoint{5.468756in}{1.638240in}}{\pgfqpoint{5.473799in}{1.643284in}}%
\pgfpathcurveto{\pgfqpoint{5.478843in}{1.648327in}}{\pgfqpoint{5.481677in}{1.655169in}}{\pgfqpoint{5.481677in}{1.662302in}}%
\pgfpathcurveto{\pgfqpoint{5.481677in}{1.669435in}}{\pgfqpoint{5.478843in}{1.676276in}}{\pgfqpoint{5.473799in}{1.681320in}}%
\pgfpathcurveto{\pgfqpoint{5.468756in}{1.686364in}}{\pgfqpoint{5.461914in}{1.689197in}}{\pgfqpoint{5.454781in}{1.689197in}}%
\pgfpathcurveto{\pgfqpoint{5.447648in}{1.689197in}}{\pgfqpoint{5.440807in}{1.686364in}}{\pgfqpoint{5.435763in}{1.681320in}}%
\pgfpathcurveto{\pgfqpoint{5.430719in}{1.676276in}}{\pgfqpoint{5.427886in}{1.669435in}}{\pgfqpoint{5.427886in}{1.662302in}}%
\pgfpathcurveto{\pgfqpoint{5.427886in}{1.655169in}}{\pgfqpoint{5.430719in}{1.648327in}}{\pgfqpoint{5.435763in}{1.643284in}}%
\pgfpathcurveto{\pgfqpoint{5.440807in}{1.638240in}}{\pgfqpoint{5.447648in}{1.635406in}}{\pgfqpoint{5.454781in}{1.635406in}}%
\pgfpathclose%
\pgfusepath{stroke,fill}%
\end{pgfscope}%
\begin{pgfscope}%
\pgfpathrectangle{\pgfqpoint{4.985294in}{0.500000in}}{\pgfqpoint{1.764706in}{1.700000in}}%
\pgfusepath{clip}%
\pgfsetbuttcap%
\pgfsetroundjoin%
\definecolor{currentfill}{rgb}{0.973271,0.850724,0.762998}%
\pgfsetfillcolor{currentfill}%
\pgfsetlinewidth{0.311001pt}%
\definecolor{currentstroke}{rgb}{1.000000,1.000000,1.000000}%
\pgfsetstrokecolor{currentstroke}%
\pgfsetdash{}{0pt}%
\pgfpathmoveto{\pgfqpoint{6.262179in}{1.375459in}}%
\pgfpathcurveto{\pgfqpoint{6.269312in}{1.375459in}}{\pgfqpoint{6.276154in}{1.378293in}}{\pgfqpoint{6.281197in}{1.383337in}}%
\pgfpathcurveto{\pgfqpoint{6.286241in}{1.388380in}}{\pgfqpoint{6.289075in}{1.395222in}}{\pgfqpoint{6.289075in}{1.402355in}}%
\pgfpathcurveto{\pgfqpoint{6.289075in}{1.409488in}}{\pgfqpoint{6.286241in}{1.416329in}}{\pgfqpoint{6.281197in}{1.421373in}}%
\pgfpathcurveto{\pgfqpoint{6.276154in}{1.426417in}}{\pgfqpoint{6.269312in}{1.429250in}}{\pgfqpoint{6.262179in}{1.429250in}}%
\pgfpathcurveto{\pgfqpoint{6.255046in}{1.429250in}}{\pgfqpoint{6.248205in}{1.426417in}}{\pgfqpoint{6.243161in}{1.421373in}}%
\pgfpathcurveto{\pgfqpoint{6.238117in}{1.416329in}}{\pgfqpoint{6.235283in}{1.409488in}}{\pgfqpoint{6.235283in}{1.402355in}}%
\pgfpathcurveto{\pgfqpoint{6.235283in}{1.395222in}}{\pgfqpoint{6.238117in}{1.388380in}}{\pgfqpoint{6.243161in}{1.383337in}}%
\pgfpathcurveto{\pgfqpoint{6.248205in}{1.378293in}}{\pgfqpoint{6.255046in}{1.375459in}}{\pgfqpoint{6.262179in}{1.375459in}}%
\pgfpathclose%
\pgfusepath{stroke,fill}%
\end{pgfscope}%
\begin{pgfscope}%
\pgfpathrectangle{\pgfqpoint{4.985294in}{0.500000in}}{\pgfqpoint{1.764706in}{1.700000in}}%
\pgfusepath{clip}%
\pgfsetbuttcap%
\pgfsetroundjoin%
\definecolor{currentfill}{rgb}{0.979891,0.908948,0.848279}%
\pgfsetfillcolor{currentfill}%
\pgfsetlinewidth{0.311001pt}%
\definecolor{currentstroke}{rgb}{1.000000,1.000000,1.000000}%
\pgfsetstrokecolor{currentstroke}%
\pgfsetdash{}{0pt}%
\pgfpathmoveto{\pgfqpoint{6.290352in}{1.228840in}}%
\pgfpathcurveto{\pgfqpoint{6.297485in}{1.228840in}}{\pgfqpoint{6.304326in}{1.231674in}}{\pgfqpoint{6.309370in}{1.236718in}}%
\pgfpathcurveto{\pgfqpoint{6.314414in}{1.241761in}}{\pgfqpoint{6.317248in}{1.248603in}}{\pgfqpoint{6.317248in}{1.255736in}}%
\pgfpathcurveto{\pgfqpoint{6.317248in}{1.262869in}}{\pgfqpoint{6.314414in}{1.269710in}}{\pgfqpoint{6.309370in}{1.274754in}}%
\pgfpathcurveto{\pgfqpoint{6.304326in}{1.279797in}}{\pgfqpoint{6.297485in}{1.282631in}}{\pgfqpoint{6.290352in}{1.282631in}}%
\pgfpathcurveto{\pgfqpoint{6.283219in}{1.282631in}}{\pgfqpoint{6.276377in}{1.279797in}}{\pgfqpoint{6.271334in}{1.274754in}}%
\pgfpathcurveto{\pgfqpoint{6.266290in}{1.269710in}}{\pgfqpoint{6.263456in}{1.262869in}}{\pgfqpoint{6.263456in}{1.255736in}}%
\pgfpathcurveto{\pgfqpoint{6.263456in}{1.248603in}}{\pgfqpoint{6.266290in}{1.241761in}}{\pgfqpoint{6.271334in}{1.236718in}}%
\pgfpathcurveto{\pgfqpoint{6.276377in}{1.231674in}}{\pgfqpoint{6.283219in}{1.228840in}}{\pgfqpoint{6.290352in}{1.228840in}}%
\pgfpathclose%
\pgfusepath{stroke,fill}%
\end{pgfscope}%
\begin{pgfscope}%
\pgfpathrectangle{\pgfqpoint{4.985294in}{0.500000in}}{\pgfqpoint{1.764706in}{1.700000in}}%
\pgfusepath{clip}%
\pgfsetbuttcap%
\pgfsetroundjoin%
\definecolor{currentfill}{rgb}{0.980678,0.914765,0.856766}%
\pgfsetfillcolor{currentfill}%
\pgfsetlinewidth{0.311001pt}%
\definecolor{currentstroke}{rgb}{1.000000,1.000000,1.000000}%
\pgfsetstrokecolor{currentstroke}%
\pgfsetdash{}{0pt}%
\pgfpathmoveto{\pgfqpoint{5.417021in}{1.387091in}}%
\pgfpathcurveto{\pgfqpoint{5.424154in}{1.387091in}}{\pgfqpoint{5.430995in}{1.389924in}}{\pgfqpoint{5.436039in}{1.394968in}}%
\pgfpathcurveto{\pgfqpoint{5.441083in}{1.400012in}}{\pgfqpoint{5.443917in}{1.406853in}}{\pgfqpoint{5.443917in}{1.413986in}}%
\pgfpathcurveto{\pgfqpoint{5.443917in}{1.421119in}}{\pgfqpoint{5.441083in}{1.427961in}}{\pgfqpoint{5.436039in}{1.433004in}}%
\pgfpathcurveto{\pgfqpoint{5.430995in}{1.438048in}}{\pgfqpoint{5.424154in}{1.440882in}}{\pgfqpoint{5.417021in}{1.440882in}}%
\pgfpathcurveto{\pgfqpoint{5.409888in}{1.440882in}}{\pgfqpoint{5.403046in}{1.438048in}}{\pgfqpoint{5.398003in}{1.433004in}}%
\pgfpathcurveto{\pgfqpoint{5.392959in}{1.427961in}}{\pgfqpoint{5.390125in}{1.421119in}}{\pgfqpoint{5.390125in}{1.413986in}}%
\pgfpathcurveto{\pgfqpoint{5.390125in}{1.406853in}}{\pgfqpoint{5.392959in}{1.400012in}}{\pgfqpoint{5.398003in}{1.394968in}}%
\pgfpathcurveto{\pgfqpoint{5.403046in}{1.389924in}}{\pgfqpoint{5.409888in}{1.387091in}}{\pgfqpoint{5.417021in}{1.387091in}}%
\pgfpathclose%
\pgfusepath{stroke,fill}%
\end{pgfscope}%
\begin{pgfscope}%
\pgfpathrectangle{\pgfqpoint{4.985294in}{0.500000in}}{\pgfqpoint{1.764706in}{1.700000in}}%
\pgfusepath{clip}%
\pgfsetbuttcap%
\pgfsetroundjoin%
\definecolor{currentfill}{rgb}{0.976287,0.879862,0.805788}%
\pgfsetfillcolor{currentfill}%
\pgfsetlinewidth{0.311001pt}%
\definecolor{currentstroke}{rgb}{1.000000,1.000000,1.000000}%
\pgfsetstrokecolor{currentstroke}%
\pgfsetdash{}{0pt}%
\pgfpathmoveto{\pgfqpoint{6.274996in}{1.338048in}}%
\pgfpathcurveto{\pgfqpoint{6.282129in}{1.338048in}}{\pgfqpoint{6.288970in}{1.340882in}}{\pgfqpoint{6.294014in}{1.345926in}}%
\pgfpathcurveto{\pgfqpoint{6.299058in}{1.350969in}}{\pgfqpoint{6.301892in}{1.357811in}}{\pgfqpoint{6.301892in}{1.364944in}}%
\pgfpathcurveto{\pgfqpoint{6.301892in}{1.372077in}}{\pgfqpoint{6.299058in}{1.378918in}}{\pgfqpoint{6.294014in}{1.383962in}}%
\pgfpathcurveto{\pgfqpoint{6.288970in}{1.389006in}}{\pgfqpoint{6.282129in}{1.391839in}}{\pgfqpoint{6.274996in}{1.391839in}}%
\pgfpathcurveto{\pgfqpoint{6.267863in}{1.391839in}}{\pgfqpoint{6.261021in}{1.389006in}}{\pgfqpoint{6.255978in}{1.383962in}}%
\pgfpathcurveto{\pgfqpoint{6.250934in}{1.378918in}}{\pgfqpoint{6.248100in}{1.372077in}}{\pgfqpoint{6.248100in}{1.364944in}}%
\pgfpathcurveto{\pgfqpoint{6.248100in}{1.357811in}}{\pgfqpoint{6.250934in}{1.350969in}}{\pgfqpoint{6.255978in}{1.345926in}}%
\pgfpathcurveto{\pgfqpoint{6.261021in}{1.340882in}}{\pgfqpoint{6.267863in}{1.338048in}}{\pgfqpoint{6.274996in}{1.338048in}}%
\pgfpathclose%
\pgfusepath{stroke,fill}%
\end{pgfscope}%
\begin{pgfscope}%
\pgfpathrectangle{\pgfqpoint{4.985294in}{0.500000in}}{\pgfqpoint{1.764706in}{1.700000in}}%
\pgfusepath{clip}%
\pgfsetbuttcap%
\pgfsetroundjoin%
\definecolor{currentfill}{rgb}{0.964799,0.689101,0.537560}%
\pgfsetfillcolor{currentfill}%
\pgfsetlinewidth{0.311001pt}%
\definecolor{currentstroke}{rgb}{1.000000,1.000000,1.000000}%
\pgfsetstrokecolor{currentstroke}%
\pgfsetdash{}{0pt}%
\pgfpathmoveto{\pgfqpoint{5.442169in}{1.680826in}}%
\pgfpathcurveto{\pgfqpoint{5.449302in}{1.680826in}}{\pgfqpoint{5.456144in}{1.683660in}}{\pgfqpoint{5.461188in}{1.688704in}}%
\pgfpathcurveto{\pgfqpoint{5.466231in}{1.693748in}}{\pgfqpoint{5.469065in}{1.700589in}}{\pgfqpoint{5.469065in}{1.707722in}}%
\pgfpathcurveto{\pgfqpoint{5.469065in}{1.714855in}}{\pgfqpoint{5.466231in}{1.721697in}}{\pgfqpoint{5.461188in}{1.726740in}}%
\pgfpathcurveto{\pgfqpoint{5.456144in}{1.731784in}}{\pgfqpoint{5.449302in}{1.734618in}}{\pgfqpoint{5.442169in}{1.734618in}}%
\pgfpathcurveto{\pgfqpoint{5.435037in}{1.734618in}}{\pgfqpoint{5.428195in}{1.731784in}}{\pgfqpoint{5.423151in}{1.726740in}}%
\pgfpathcurveto{\pgfqpoint{5.418108in}{1.721697in}}{\pgfqpoint{5.415274in}{1.714855in}}{\pgfqpoint{5.415274in}{1.707722in}}%
\pgfpathcurveto{\pgfqpoint{5.415274in}{1.700589in}}{\pgfqpoint{5.418108in}{1.693748in}}{\pgfqpoint{5.423151in}{1.688704in}}%
\pgfpathcurveto{\pgfqpoint{5.428195in}{1.683660in}}{\pgfqpoint{5.435037in}{1.680826in}}{\pgfqpoint{5.442169in}{1.680826in}}%
\pgfpathclose%
\pgfusepath{stroke,fill}%
\end{pgfscope}%
\begin{pgfscope}%
\pgfpathrectangle{\pgfqpoint{4.985294in}{0.500000in}}{\pgfqpoint{1.764706in}{1.700000in}}%
\pgfusepath{clip}%
\pgfsetbuttcap%
\pgfsetroundjoin%
\definecolor{currentfill}{rgb}{0.948235,0.413004,0.283323}%
\pgfsetfillcolor{currentfill}%
\pgfsetlinewidth{0.311001pt}%
\definecolor{currentstroke}{rgb}{1.000000,1.000000,1.000000}%
\pgfsetstrokecolor{currentstroke}%
\pgfsetdash{}{0pt}%
\pgfpathmoveto{\pgfqpoint{5.301993in}{1.108090in}}%
\pgfpathcurveto{\pgfqpoint{5.309126in}{1.108090in}}{\pgfqpoint{5.315967in}{1.110924in}}{\pgfqpoint{5.321011in}{1.115968in}}%
\pgfpathcurveto{\pgfqpoint{5.326055in}{1.121011in}}{\pgfqpoint{5.328888in}{1.127853in}}{\pgfqpoint{5.328888in}{1.134986in}}%
\pgfpathcurveto{\pgfqpoint{5.328888in}{1.142119in}}{\pgfqpoint{5.326055in}{1.148960in}}{\pgfqpoint{5.321011in}{1.154004in}}%
\pgfpathcurveto{\pgfqpoint{5.315967in}{1.159048in}}{\pgfqpoint{5.309126in}{1.161882in}}{\pgfqpoint{5.301993in}{1.161882in}}%
\pgfpathcurveto{\pgfqpoint{5.294860in}{1.161882in}}{\pgfqpoint{5.288018in}{1.159048in}}{\pgfqpoint{5.282975in}{1.154004in}}%
\pgfpathcurveto{\pgfqpoint{5.277931in}{1.148960in}}{\pgfqpoint{5.275097in}{1.142119in}}{\pgfqpoint{5.275097in}{1.134986in}}%
\pgfpathcurveto{\pgfqpoint{5.275097in}{1.127853in}}{\pgfqpoint{5.277931in}{1.121011in}}{\pgfqpoint{5.282975in}{1.115968in}}%
\pgfpathcurveto{\pgfqpoint{5.288018in}{1.110924in}}{\pgfqpoint{5.294860in}{1.108090in}}{\pgfqpoint{5.301993in}{1.108090in}}%
\pgfpathclose%
\pgfusepath{stroke,fill}%
\end{pgfscope}%
\begin{pgfscope}%
\pgfpathrectangle{\pgfqpoint{4.985294in}{0.500000in}}{\pgfqpoint{1.764706in}{1.700000in}}%
\pgfusepath{clip}%
\pgfsetbuttcap%
\pgfsetroundjoin%
\definecolor{currentfill}{rgb}{0.977657,0.891500,0.822809}%
\pgfsetfillcolor{currentfill}%
\pgfsetlinewidth{0.311001pt}%
\definecolor{currentstroke}{rgb}{1.000000,1.000000,1.000000}%
\pgfsetstrokecolor{currentstroke}%
\pgfsetdash{}{0pt}%
\pgfpathmoveto{\pgfqpoint{6.276046in}{1.473038in}}%
\pgfpathcurveto{\pgfqpoint{6.283179in}{1.473038in}}{\pgfqpoint{6.290021in}{1.475872in}}{\pgfqpoint{6.295064in}{1.480915in}}%
\pgfpathcurveto{\pgfqpoint{6.300108in}{1.485959in}}{\pgfqpoint{6.302942in}{1.492801in}}{\pgfqpoint{6.302942in}{1.499933in}}%
\pgfpathcurveto{\pgfqpoint{6.302942in}{1.507066in}}{\pgfqpoint{6.300108in}{1.513908in}}{\pgfqpoint{6.295064in}{1.518952in}}%
\pgfpathcurveto{\pgfqpoint{6.290021in}{1.523995in}}{\pgfqpoint{6.283179in}{1.526829in}}{\pgfqpoint{6.276046in}{1.526829in}}%
\pgfpathcurveto{\pgfqpoint{6.268914in}{1.526829in}}{\pgfqpoint{6.262072in}{1.523995in}}{\pgfqpoint{6.257028in}{1.518952in}}%
\pgfpathcurveto{\pgfqpoint{6.251985in}{1.513908in}}{\pgfqpoint{6.249151in}{1.507066in}}{\pgfqpoint{6.249151in}{1.499933in}}%
\pgfpathcurveto{\pgfqpoint{6.249151in}{1.492801in}}{\pgfqpoint{6.251985in}{1.485959in}}{\pgfqpoint{6.257028in}{1.480915in}}%
\pgfpathcurveto{\pgfqpoint{6.262072in}{1.475872in}}{\pgfqpoint{6.268914in}{1.473038in}}{\pgfqpoint{6.276046in}{1.473038in}}%
\pgfpathclose%
\pgfusepath{stroke,fill}%
\end{pgfscope}%
\begin{pgfscope}%
\pgfpathrectangle{\pgfqpoint{4.985294in}{0.500000in}}{\pgfqpoint{1.764706in}{1.700000in}}%
\pgfusepath{clip}%
\pgfsetbuttcap%
\pgfsetroundjoin%
\definecolor{currentfill}{rgb}{0.979124,0.903132,0.839793}%
\pgfsetfillcolor{currentfill}%
\pgfsetlinewidth{0.311001pt}%
\definecolor{currentstroke}{rgb}{1.000000,1.000000,1.000000}%
\pgfsetstrokecolor{currentstroke}%
\pgfsetdash{}{0pt}%
\pgfpathmoveto{\pgfqpoint{6.291097in}{1.363520in}}%
\pgfpathcurveto{\pgfqpoint{6.298230in}{1.363520in}}{\pgfqpoint{6.305071in}{1.366354in}}{\pgfqpoint{6.310115in}{1.371397in}}%
\pgfpathcurveto{\pgfqpoint{6.315158in}{1.376441in}}{\pgfqpoint{6.317992in}{1.383283in}}{\pgfqpoint{6.317992in}{1.390415in}}%
\pgfpathcurveto{\pgfqpoint{6.317992in}{1.397548in}}{\pgfqpoint{6.315158in}{1.404390in}}{\pgfqpoint{6.310115in}{1.409434in}}%
\pgfpathcurveto{\pgfqpoint{6.305071in}{1.414477in}}{\pgfqpoint{6.298230in}{1.417311in}}{\pgfqpoint{6.291097in}{1.417311in}}%
\pgfpathcurveto{\pgfqpoint{6.283964in}{1.417311in}}{\pgfqpoint{6.277122in}{1.414477in}}{\pgfqpoint{6.272079in}{1.409434in}}%
\pgfpathcurveto{\pgfqpoint{6.267035in}{1.404390in}}{\pgfqpoint{6.264201in}{1.397548in}}{\pgfqpoint{6.264201in}{1.390415in}}%
\pgfpathcurveto{\pgfqpoint{6.264201in}{1.383283in}}{\pgfqpoint{6.267035in}{1.376441in}}{\pgfqpoint{6.272079in}{1.371397in}}%
\pgfpathcurveto{\pgfqpoint{6.277122in}{1.366354in}}{\pgfqpoint{6.283964in}{1.363520in}}{\pgfqpoint{6.291097in}{1.363520in}}%
\pgfpathclose%
\pgfusepath{stroke,fill}%
\end{pgfscope}%
\begin{pgfscope}%
\pgfpathrectangle{\pgfqpoint{4.985294in}{0.500000in}}{\pgfqpoint{1.764706in}{1.700000in}}%
\pgfusepath{clip}%
\pgfsetbuttcap%
\pgfsetroundjoin%
\definecolor{currentfill}{rgb}{0.979891,0.908948,0.848279}%
\pgfsetfillcolor{currentfill}%
\pgfsetlinewidth{0.311001pt}%
\definecolor{currentstroke}{rgb}{1.000000,1.000000,1.000000}%
\pgfsetstrokecolor{currentstroke}%
\pgfsetdash{}{0pt}%
\pgfpathmoveto{\pgfqpoint{5.411031in}{1.398700in}}%
\pgfpathcurveto{\pgfqpoint{5.418163in}{1.398700in}}{\pgfqpoint{5.425005in}{1.401534in}}{\pgfqpoint{5.430049in}{1.406578in}}%
\pgfpathcurveto{\pgfqpoint{5.435092in}{1.411622in}}{\pgfqpoint{5.437926in}{1.418463in}}{\pgfqpoint{5.437926in}{1.425596in}}%
\pgfpathcurveto{\pgfqpoint{5.437926in}{1.432729in}}{\pgfqpoint{5.435092in}{1.439571in}}{\pgfqpoint{5.430049in}{1.444614in}}%
\pgfpathcurveto{\pgfqpoint{5.425005in}{1.449658in}}{\pgfqpoint{5.418163in}{1.452492in}}{\pgfqpoint{5.411031in}{1.452492in}}%
\pgfpathcurveto{\pgfqpoint{5.403898in}{1.452492in}}{\pgfqpoint{5.397056in}{1.449658in}}{\pgfqpoint{5.392012in}{1.444614in}}%
\pgfpathcurveto{\pgfqpoint{5.386969in}{1.439571in}}{\pgfqpoint{5.384135in}{1.432729in}}{\pgfqpoint{5.384135in}{1.425596in}}%
\pgfpathcurveto{\pgfqpoint{5.384135in}{1.418463in}}{\pgfqpoint{5.386969in}{1.411622in}}{\pgfqpoint{5.392012in}{1.406578in}}%
\pgfpathcurveto{\pgfqpoint{5.397056in}{1.401534in}}{\pgfqpoint{5.403898in}{1.398700in}}{\pgfqpoint{5.411031in}{1.398700in}}%
\pgfpathclose%
\pgfusepath{stroke,fill}%
\end{pgfscope}%
\begin{pgfscope}%
\pgfpathrectangle{\pgfqpoint{4.985294in}{0.500000in}}{\pgfqpoint{1.764706in}{1.700000in}}%
\pgfusepath{clip}%
\pgfsetbuttcap%
\pgfsetroundjoin%
\definecolor{currentfill}{rgb}{0.976287,0.879862,0.805788}%
\pgfsetfillcolor{currentfill}%
\pgfsetlinewidth{0.311001pt}%
\definecolor{currentstroke}{rgb}{1.000000,1.000000,1.000000}%
\pgfsetstrokecolor{currentstroke}%
\pgfsetdash{}{0pt}%
\pgfpathmoveto{\pgfqpoint{6.320206in}{1.515399in}}%
\pgfpathcurveto{\pgfqpoint{6.327339in}{1.515399in}}{\pgfqpoint{6.334180in}{1.518233in}}{\pgfqpoint{6.339224in}{1.523277in}}%
\pgfpathcurveto{\pgfqpoint{6.344268in}{1.528320in}}{\pgfqpoint{6.347102in}{1.535162in}}{\pgfqpoint{6.347102in}{1.542295in}}%
\pgfpathcurveto{\pgfqpoint{6.347102in}{1.549428in}}{\pgfqpoint{6.344268in}{1.556269in}}{\pgfqpoint{6.339224in}{1.561313in}}%
\pgfpathcurveto{\pgfqpoint{6.334180in}{1.566357in}}{\pgfqpoint{6.327339in}{1.569191in}}{\pgfqpoint{6.320206in}{1.569191in}}%
\pgfpathcurveto{\pgfqpoint{6.313073in}{1.569191in}}{\pgfqpoint{6.306231in}{1.566357in}}{\pgfqpoint{6.301188in}{1.561313in}}%
\pgfpathcurveto{\pgfqpoint{6.296144in}{1.556269in}}{\pgfqpoint{6.293310in}{1.549428in}}{\pgfqpoint{6.293310in}{1.542295in}}%
\pgfpathcurveto{\pgfqpoint{6.293310in}{1.535162in}}{\pgfqpoint{6.296144in}{1.528320in}}{\pgfqpoint{6.301188in}{1.523277in}}%
\pgfpathcurveto{\pgfqpoint{6.306231in}{1.518233in}}{\pgfqpoint{6.313073in}{1.515399in}}{\pgfqpoint{6.320206in}{1.515399in}}%
\pgfpathclose%
\pgfusepath{stroke,fill}%
\end{pgfscope}%
\begin{pgfscope}%
\pgfpathrectangle{\pgfqpoint{4.985294in}{0.500000in}}{\pgfqpoint{1.764706in}{1.700000in}}%
\pgfusepath{clip}%
\pgfsetbuttcap%
\pgfsetroundjoin%
\definecolor{currentfill}{rgb}{0.957344,0.505732,0.351309}%
\pgfsetfillcolor{currentfill}%
\pgfsetlinewidth{0.311001pt}%
\definecolor{currentstroke}{rgb}{1.000000,1.000000,1.000000}%
\pgfsetstrokecolor{currentstroke}%
\pgfsetdash{}{0pt}%
\pgfpathmoveto{\pgfqpoint{6.443656in}{1.331113in}}%
\pgfpathcurveto{\pgfqpoint{6.450789in}{1.331113in}}{\pgfqpoint{6.457631in}{1.333947in}}{\pgfqpoint{6.462674in}{1.338991in}}%
\pgfpathcurveto{\pgfqpoint{6.467718in}{1.344035in}}{\pgfqpoint{6.470552in}{1.350876in}}{\pgfqpoint{6.470552in}{1.358009in}}%
\pgfpathcurveto{\pgfqpoint{6.470552in}{1.365142in}}{\pgfqpoint{6.467718in}{1.371983in}}{\pgfqpoint{6.462674in}{1.377027in}}%
\pgfpathcurveto{\pgfqpoint{6.457631in}{1.382071in}}{\pgfqpoint{6.450789in}{1.384905in}}{\pgfqpoint{6.443656in}{1.384905in}}%
\pgfpathcurveto{\pgfqpoint{6.436524in}{1.384905in}}{\pgfqpoint{6.429682in}{1.382071in}}{\pgfqpoint{6.424638in}{1.377027in}}%
\pgfpathcurveto{\pgfqpoint{6.419595in}{1.371983in}}{\pgfqpoint{6.416761in}{1.365142in}}{\pgfqpoint{6.416761in}{1.358009in}}%
\pgfpathcurveto{\pgfqpoint{6.416761in}{1.350876in}}{\pgfqpoint{6.419595in}{1.344035in}}{\pgfqpoint{6.424638in}{1.338991in}}%
\pgfpathcurveto{\pgfqpoint{6.429682in}{1.333947in}}{\pgfqpoint{6.436524in}{1.331113in}}{\pgfqpoint{6.443656in}{1.331113in}}%
\pgfpathclose%
\pgfusepath{stroke,fill}%
\end{pgfscope}%
\begin{pgfscope}%
\pgfpathrectangle{\pgfqpoint{4.985294in}{0.500000in}}{\pgfqpoint{1.764706in}{1.700000in}}%
\pgfusepath{clip}%
\pgfsetbuttcap%
\pgfsetroundjoin%
\definecolor{currentfill}{rgb}{0.967398,0.774513,0.650573}%
\pgfsetfillcolor{currentfill}%
\pgfsetlinewidth{0.311001pt}%
\definecolor{currentstroke}{rgb}{1.000000,1.000000,1.000000}%
\pgfsetstrokecolor{currentstroke}%
\pgfsetdash{}{0pt}%
\pgfpathmoveto{\pgfqpoint{6.205911in}{1.232039in}}%
\pgfpathcurveto{\pgfqpoint{6.213044in}{1.232039in}}{\pgfqpoint{6.219886in}{1.234873in}}{\pgfqpoint{6.224930in}{1.239916in}}%
\pgfpathcurveto{\pgfqpoint{6.229973in}{1.244960in}}{\pgfqpoint{6.232807in}{1.251802in}}{\pgfqpoint{6.232807in}{1.258935in}}%
\pgfpathcurveto{\pgfqpoint{6.232807in}{1.266067in}}{\pgfqpoint{6.229973in}{1.272909in}}{\pgfqpoint{6.224930in}{1.277953in}}%
\pgfpathcurveto{\pgfqpoint{6.219886in}{1.282996in}}{\pgfqpoint{6.213044in}{1.285830in}}{\pgfqpoint{6.205911in}{1.285830in}}%
\pgfpathcurveto{\pgfqpoint{6.198779in}{1.285830in}}{\pgfqpoint{6.191937in}{1.282996in}}{\pgfqpoint{6.186893in}{1.277953in}}%
\pgfpathcurveto{\pgfqpoint{6.181850in}{1.272909in}}{\pgfqpoint{6.179016in}{1.266067in}}{\pgfqpoint{6.179016in}{1.258935in}}%
\pgfpathcurveto{\pgfqpoint{6.179016in}{1.251802in}}{\pgfqpoint{6.181850in}{1.244960in}}{\pgfqpoint{6.186893in}{1.239916in}}%
\pgfpathcurveto{\pgfqpoint{6.191937in}{1.234873in}}{\pgfqpoint{6.198779in}{1.232039in}}{\pgfqpoint{6.205911in}{1.232039in}}%
\pgfpathclose%
\pgfusepath{stroke,fill}%
\end{pgfscope}%
\begin{pgfscope}%
\pgfpathrectangle{\pgfqpoint{4.985294in}{0.500000in}}{\pgfqpoint{1.764706in}{1.700000in}}%
\pgfusepath{clip}%
\pgfsetbuttcap%
\pgfsetroundjoin%
\definecolor{currentfill}{rgb}{0.976287,0.879862,0.805788}%
\pgfsetfillcolor{currentfill}%
\pgfsetlinewidth{0.311001pt}%
\definecolor{currentstroke}{rgb}{1.000000,1.000000,1.000000}%
\pgfsetstrokecolor{currentstroke}%
\pgfsetdash{}{0pt}%
\pgfpathmoveto{\pgfqpoint{5.463529in}{1.207741in}}%
\pgfpathcurveto{\pgfqpoint{5.470662in}{1.207741in}}{\pgfqpoint{5.477504in}{1.210575in}}{\pgfqpoint{5.482547in}{1.215618in}}%
\pgfpathcurveto{\pgfqpoint{5.487591in}{1.220662in}}{\pgfqpoint{5.490425in}{1.227504in}}{\pgfqpoint{5.490425in}{1.234636in}}%
\pgfpathcurveto{\pgfqpoint{5.490425in}{1.241769in}}{\pgfqpoint{5.487591in}{1.248611in}}{\pgfqpoint{5.482547in}{1.253655in}}%
\pgfpathcurveto{\pgfqpoint{5.477504in}{1.258698in}}{\pgfqpoint{5.470662in}{1.261532in}}{\pgfqpoint{5.463529in}{1.261532in}}%
\pgfpathcurveto{\pgfqpoint{5.456396in}{1.261532in}}{\pgfqpoint{5.449555in}{1.258698in}}{\pgfqpoint{5.444511in}{1.253655in}}%
\pgfpathcurveto{\pgfqpoint{5.439467in}{1.248611in}}{\pgfqpoint{5.436633in}{1.241769in}}{\pgfqpoint{5.436633in}{1.234636in}}%
\pgfpathcurveto{\pgfqpoint{5.436633in}{1.227504in}}{\pgfqpoint{5.439467in}{1.220662in}}{\pgfqpoint{5.444511in}{1.215618in}}%
\pgfpathcurveto{\pgfqpoint{5.449555in}{1.210575in}}{\pgfqpoint{5.456396in}{1.207741in}}{\pgfqpoint{5.463529in}{1.207741in}}%
\pgfpathclose%
\pgfusepath{stroke,fill}%
\end{pgfscope}%
\begin{pgfscope}%
\pgfpathrectangle{\pgfqpoint{4.985294in}{0.500000in}}{\pgfqpoint{1.764706in}{1.700000in}}%
\pgfusepath{clip}%
\pgfsetbuttcap%
\pgfsetroundjoin%
\definecolor{currentfill}{rgb}{0.979891,0.908948,0.848279}%
\pgfsetfillcolor{currentfill}%
\pgfsetlinewidth{0.311001pt}%
\definecolor{currentstroke}{rgb}{1.000000,1.000000,1.000000}%
\pgfsetstrokecolor{currentstroke}%
\pgfsetdash{}{0pt}%
\pgfpathmoveto{\pgfqpoint{5.432516in}{1.242593in}}%
\pgfpathcurveto{\pgfqpoint{5.439648in}{1.242593in}}{\pgfqpoint{5.446490in}{1.245427in}}{\pgfqpoint{5.451534in}{1.250471in}}%
\pgfpathcurveto{\pgfqpoint{5.456577in}{1.255515in}}{\pgfqpoint{5.459411in}{1.262356in}}{\pgfqpoint{5.459411in}{1.269489in}}%
\pgfpathcurveto{\pgfqpoint{5.459411in}{1.276622in}}{\pgfqpoint{5.456577in}{1.283464in}}{\pgfqpoint{5.451534in}{1.288507in}}%
\pgfpathcurveto{\pgfqpoint{5.446490in}{1.293551in}}{\pgfqpoint{5.439648in}{1.296385in}}{\pgfqpoint{5.432516in}{1.296385in}}%
\pgfpathcurveto{\pgfqpoint{5.425383in}{1.296385in}}{\pgfqpoint{5.418541in}{1.293551in}}{\pgfqpoint{5.413497in}{1.288507in}}%
\pgfpathcurveto{\pgfqpoint{5.408454in}{1.283464in}}{\pgfqpoint{5.405620in}{1.276622in}}{\pgfqpoint{5.405620in}{1.269489in}}%
\pgfpathcurveto{\pgfqpoint{5.405620in}{1.262356in}}{\pgfqpoint{5.408454in}{1.255515in}}{\pgfqpoint{5.413497in}{1.250471in}}%
\pgfpathcurveto{\pgfqpoint{5.418541in}{1.245427in}}{\pgfqpoint{5.425383in}{1.242593in}}{\pgfqpoint{5.432516in}{1.242593in}}%
\pgfpathclose%
\pgfusepath{stroke,fill}%
\end{pgfscope}%
\begin{pgfscope}%
\pgfpathrectangle{\pgfqpoint{4.985294in}{0.500000in}}{\pgfqpoint{1.764706in}{1.700000in}}%
\pgfusepath{clip}%
\pgfsetbuttcap%
\pgfsetroundjoin%
\definecolor{currentfill}{rgb}{0.969359,0.803954,0.693832}%
\pgfsetfillcolor{currentfill}%
\pgfsetlinewidth{0.311001pt}%
\definecolor{currentstroke}{rgb}{1.000000,1.000000,1.000000}%
\pgfsetstrokecolor{currentstroke}%
\pgfsetdash{}{0pt}%
\pgfpathmoveto{\pgfqpoint{6.190552in}{0.982456in}}%
\pgfpathcurveto{\pgfqpoint{6.197685in}{0.982456in}}{\pgfqpoint{6.204526in}{0.985290in}}{\pgfqpoint{6.209570in}{0.990333in}}%
\pgfpathcurveto{\pgfqpoint{6.214614in}{0.995377in}}{\pgfqpoint{6.217447in}{1.002219in}}{\pgfqpoint{6.217447in}{1.009352in}}%
\pgfpathcurveto{\pgfqpoint{6.217447in}{1.016484in}}{\pgfqpoint{6.214614in}{1.023326in}}{\pgfqpoint{6.209570in}{1.028370in}}%
\pgfpathcurveto{\pgfqpoint{6.204526in}{1.033413in}}{\pgfqpoint{6.197685in}{1.036247in}}{\pgfqpoint{6.190552in}{1.036247in}}%
\pgfpathcurveto{\pgfqpoint{6.183419in}{1.036247in}}{\pgfqpoint{6.176577in}{1.033413in}}{\pgfqpoint{6.171534in}{1.028370in}}%
\pgfpathcurveto{\pgfqpoint{6.166490in}{1.023326in}}{\pgfqpoint{6.163656in}{1.016484in}}{\pgfqpoint{6.163656in}{1.009352in}}%
\pgfpathcurveto{\pgfqpoint{6.163656in}{1.002219in}}{\pgfqpoint{6.166490in}{0.995377in}}{\pgfqpoint{6.171534in}{0.990333in}}%
\pgfpathcurveto{\pgfqpoint{6.176577in}{0.985290in}}{\pgfqpoint{6.183419in}{0.982456in}}{\pgfqpoint{6.190552in}{0.982456in}}%
\pgfpathclose%
\pgfusepath{stroke,fill}%
\end{pgfscope}%
\begin{pgfscope}%
\pgfpathrectangle{\pgfqpoint{4.985294in}{0.500000in}}{\pgfqpoint{1.764706in}{1.700000in}}%
\pgfusepath{clip}%
\pgfsetbuttcap%
\pgfsetroundjoin%
\definecolor{currentfill}{rgb}{0.952404,0.449449,0.307210}%
\pgfsetfillcolor{currentfill}%
\pgfsetlinewidth{0.311001pt}%
\definecolor{currentstroke}{rgb}{1.000000,1.000000,1.000000}%
\pgfsetstrokecolor{currentstroke}%
\pgfsetdash{}{0pt}%
\pgfpathmoveto{\pgfqpoint{5.646822in}{1.056962in}}%
\pgfpathcurveto{\pgfqpoint{5.653955in}{1.056962in}}{\pgfqpoint{5.660796in}{1.059796in}}{\pgfqpoint{5.665840in}{1.064839in}}%
\pgfpathcurveto{\pgfqpoint{5.670884in}{1.069883in}}{\pgfqpoint{5.673718in}{1.076725in}}{\pgfqpoint{5.673718in}{1.083857in}}%
\pgfpathcurveto{\pgfqpoint{5.673718in}{1.090990in}}{\pgfqpoint{5.670884in}{1.097832in}}{\pgfqpoint{5.665840in}{1.102876in}}%
\pgfpathcurveto{\pgfqpoint{5.660796in}{1.107919in}}{\pgfqpoint{5.653955in}{1.110753in}}{\pgfqpoint{5.646822in}{1.110753in}}%
\pgfpathcurveto{\pgfqpoint{5.639689in}{1.110753in}}{\pgfqpoint{5.632848in}{1.107919in}}{\pgfqpoint{5.627804in}{1.102876in}}%
\pgfpathcurveto{\pgfqpoint{5.622760in}{1.097832in}}{\pgfqpoint{5.619926in}{1.090990in}}{\pgfqpoint{5.619926in}{1.083857in}}%
\pgfpathcurveto{\pgfqpoint{5.619926in}{1.076725in}}{\pgfqpoint{5.622760in}{1.069883in}}{\pgfqpoint{5.627804in}{1.064839in}}%
\pgfpathcurveto{\pgfqpoint{5.632848in}{1.059796in}}{\pgfqpoint{5.639689in}{1.056962in}}{\pgfqpoint{5.646822in}{1.056962in}}%
\pgfpathclose%
\pgfusepath{stroke,fill}%
\end{pgfscope}%
\begin{pgfscope}%
\pgfpathrectangle{\pgfqpoint{4.985294in}{0.500000in}}{\pgfqpoint{1.764706in}{1.700000in}}%
\pgfusepath{clip}%
\pgfsetbuttcap%
\pgfsetroundjoin%
\definecolor{currentfill}{rgb}{0.971202,0.827364,0.728520}%
\pgfsetfillcolor{currentfill}%
\pgfsetlinewidth{0.311001pt}%
\definecolor{currentstroke}{rgb}{1.000000,1.000000,1.000000}%
\pgfsetstrokecolor{currentstroke}%
\pgfsetdash{}{0pt}%
\pgfpathmoveto{\pgfqpoint{5.523490in}{1.073790in}}%
\pgfpathcurveto{\pgfqpoint{5.530623in}{1.073790in}}{\pgfqpoint{5.537465in}{1.076624in}}{\pgfqpoint{5.542509in}{1.081667in}}%
\pgfpathcurveto{\pgfqpoint{5.547552in}{1.086711in}}{\pgfqpoint{5.550386in}{1.093553in}}{\pgfqpoint{5.550386in}{1.100686in}}%
\pgfpathcurveto{\pgfqpoint{5.550386in}{1.107818in}}{\pgfqpoint{5.547552in}{1.114660in}}{\pgfqpoint{5.542509in}{1.119704in}}%
\pgfpathcurveto{\pgfqpoint{5.537465in}{1.124747in}}{\pgfqpoint{5.530623in}{1.127581in}}{\pgfqpoint{5.523490in}{1.127581in}}%
\pgfpathcurveto{\pgfqpoint{5.516358in}{1.127581in}}{\pgfqpoint{5.509516in}{1.124747in}}{\pgfqpoint{5.504472in}{1.119704in}}%
\pgfpathcurveto{\pgfqpoint{5.499429in}{1.114660in}}{\pgfqpoint{5.496595in}{1.107818in}}{\pgfqpoint{5.496595in}{1.100686in}}%
\pgfpathcurveto{\pgfqpoint{5.496595in}{1.093553in}}{\pgfqpoint{5.499429in}{1.086711in}}{\pgfqpoint{5.504472in}{1.081667in}}%
\pgfpathcurveto{\pgfqpoint{5.509516in}{1.076624in}}{\pgfqpoint{5.516358in}{1.073790in}}{\pgfqpoint{5.523490in}{1.073790in}}%
\pgfpathclose%
\pgfusepath{stroke,fill}%
\end{pgfscope}%
\begin{pgfscope}%
\pgfpathrectangle{\pgfqpoint{4.985294in}{0.500000in}}{\pgfqpoint{1.764706in}{1.700000in}}%
\pgfusepath{clip}%
\pgfsetbuttcap%
\pgfsetroundjoin%
\definecolor{currentfill}{rgb}{0.971202,0.827364,0.728520}%
\pgfsetfillcolor{currentfill}%
\pgfsetlinewidth{0.311001pt}%
\definecolor{currentstroke}{rgb}{1.000000,1.000000,1.000000}%
\pgfsetstrokecolor{currentstroke}%
\pgfsetdash{}{0pt}%
\pgfpathmoveto{\pgfqpoint{5.508381in}{1.626080in}}%
\pgfpathcurveto{\pgfqpoint{5.515514in}{1.626080in}}{\pgfqpoint{5.522356in}{1.628914in}}{\pgfqpoint{5.527400in}{1.633958in}}%
\pgfpathcurveto{\pgfqpoint{5.532443in}{1.639002in}}{\pgfqpoint{5.535277in}{1.645843in}}{\pgfqpoint{5.535277in}{1.652976in}}%
\pgfpathcurveto{\pgfqpoint{5.535277in}{1.660109in}}{\pgfqpoint{5.532443in}{1.666951in}}{\pgfqpoint{5.527400in}{1.671994in}}%
\pgfpathcurveto{\pgfqpoint{5.522356in}{1.677038in}}{\pgfqpoint{5.515514in}{1.679872in}}{\pgfqpoint{5.508381in}{1.679872in}}%
\pgfpathcurveto{\pgfqpoint{5.501249in}{1.679872in}}{\pgfqpoint{5.494407in}{1.677038in}}{\pgfqpoint{5.489363in}{1.671994in}}%
\pgfpathcurveto{\pgfqpoint{5.484320in}{1.666951in}}{\pgfqpoint{5.481486in}{1.660109in}}{\pgfqpoint{5.481486in}{1.652976in}}%
\pgfpathcurveto{\pgfqpoint{5.481486in}{1.645843in}}{\pgfqpoint{5.484320in}{1.639002in}}{\pgfqpoint{5.489363in}{1.633958in}}%
\pgfpathcurveto{\pgfqpoint{5.494407in}{1.628914in}}{\pgfqpoint{5.501249in}{1.626080in}}{\pgfqpoint{5.508381in}{1.626080in}}%
\pgfpathclose%
\pgfusepath{stroke,fill}%
\end{pgfscope}%
\begin{pgfscope}%
\pgfpathrectangle{\pgfqpoint{4.985294in}{0.500000in}}{\pgfqpoint{1.764706in}{1.700000in}}%
\pgfusepath{clip}%
\pgfsetbuttcap%
\pgfsetroundjoin%
\definecolor{currentfill}{rgb}{0.964679,0.682838,0.530002}%
\pgfsetfillcolor{currentfill}%
\pgfsetlinewidth{0.311001pt}%
\definecolor{currentstroke}{rgb}{1.000000,1.000000,1.000000}%
\pgfsetstrokecolor{currentstroke}%
\pgfsetdash{}{0pt}%
\pgfpathmoveto{\pgfqpoint{6.158284in}{0.900645in}}%
\pgfpathcurveto{\pgfqpoint{6.165417in}{0.900645in}}{\pgfqpoint{6.172258in}{0.903478in}}{\pgfqpoint{6.177302in}{0.908522in}}%
\pgfpathcurveto{\pgfqpoint{6.182346in}{0.913566in}}{\pgfqpoint{6.185180in}{0.920407in}}{\pgfqpoint{6.185180in}{0.927540in}}%
\pgfpathcurveto{\pgfqpoint{6.185180in}{0.934673in}}{\pgfqpoint{6.182346in}{0.941515in}}{\pgfqpoint{6.177302in}{0.946558in}}%
\pgfpathcurveto{\pgfqpoint{6.172258in}{0.951602in}}{\pgfqpoint{6.165417in}{0.954436in}}{\pgfqpoint{6.158284in}{0.954436in}}%
\pgfpathcurveto{\pgfqpoint{6.151151in}{0.954436in}}{\pgfqpoint{6.144309in}{0.951602in}}{\pgfqpoint{6.139266in}{0.946558in}}%
\pgfpathcurveto{\pgfqpoint{6.134222in}{0.941515in}}{\pgfqpoint{6.131388in}{0.934673in}}{\pgfqpoint{6.131388in}{0.927540in}}%
\pgfpathcurveto{\pgfqpoint{6.131388in}{0.920407in}}{\pgfqpoint{6.134222in}{0.913566in}}{\pgfqpoint{6.139266in}{0.908522in}}%
\pgfpathcurveto{\pgfqpoint{6.144309in}{0.903478in}}{\pgfqpoint{6.151151in}{0.900645in}}{\pgfqpoint{6.158284in}{0.900645in}}%
\pgfpathclose%
\pgfusepath{stroke,fill}%
\end{pgfscope}%
\begin{pgfscope}%
\pgfpathrectangle{\pgfqpoint{4.985294in}{0.500000in}}{\pgfqpoint{1.764706in}{1.700000in}}%
\pgfusepath{clip}%
\pgfsetbuttcap%
\pgfsetroundjoin%
\definecolor{currentfill}{rgb}{0.974412,0.862387,0.780156}%
\pgfsetfillcolor{currentfill}%
\pgfsetlinewidth{0.311001pt}%
\definecolor{currentstroke}{rgb}{1.000000,1.000000,1.000000}%
\pgfsetstrokecolor{currentstroke}%
\pgfsetdash{}{0pt}%
\pgfpathmoveto{\pgfqpoint{6.357737in}{1.211288in}}%
\pgfpathcurveto{\pgfqpoint{6.364870in}{1.211288in}}{\pgfqpoint{6.371711in}{1.214122in}}{\pgfqpoint{6.376755in}{1.219165in}}%
\pgfpathcurveto{\pgfqpoint{6.381799in}{1.224209in}}{\pgfqpoint{6.384633in}{1.231051in}}{\pgfqpoint{6.384633in}{1.238184in}}%
\pgfpathcurveto{\pgfqpoint{6.384633in}{1.245316in}}{\pgfqpoint{6.381799in}{1.252158in}}{\pgfqpoint{6.376755in}{1.257202in}}%
\pgfpathcurveto{\pgfqpoint{6.371711in}{1.262245in}}{\pgfqpoint{6.364870in}{1.265079in}}{\pgfqpoint{6.357737in}{1.265079in}}%
\pgfpathcurveto{\pgfqpoint{6.350604in}{1.265079in}}{\pgfqpoint{6.343762in}{1.262245in}}{\pgfqpoint{6.338719in}{1.257202in}}%
\pgfpathcurveto{\pgfqpoint{6.333675in}{1.252158in}}{\pgfqpoint{6.330841in}{1.245316in}}{\pgfqpoint{6.330841in}{1.238184in}}%
\pgfpathcurveto{\pgfqpoint{6.330841in}{1.231051in}}{\pgfqpoint{6.333675in}{1.224209in}}{\pgfqpoint{6.338719in}{1.219165in}}%
\pgfpathcurveto{\pgfqpoint{6.343762in}{1.214122in}}{\pgfqpoint{6.350604in}{1.211288in}}{\pgfqpoint{6.357737in}{1.211288in}}%
\pgfpathclose%
\pgfusepath{stroke,fill}%
\end{pgfscope}%
\begin{pgfscope}%
\pgfpathrectangle{\pgfqpoint{4.985294in}{0.500000in}}{\pgfqpoint{1.764706in}{1.700000in}}%
\pgfusepath{clip}%
\pgfsetbuttcap%
\pgfsetroundjoin%
\definecolor{currentfill}{rgb}{0.958791,0.526283,0.368909}%
\pgfsetfillcolor{currentfill}%
\pgfsetlinewidth{0.311001pt}%
\definecolor{currentstroke}{rgb}{1.000000,1.000000,1.000000}%
\pgfsetstrokecolor{currentstroke}%
\pgfsetdash{}{0pt}%
\pgfpathmoveto{\pgfqpoint{5.310059in}{1.466022in}}%
\pgfpathcurveto{\pgfqpoint{5.317192in}{1.466022in}}{\pgfqpoint{5.324034in}{1.468856in}}{\pgfqpoint{5.329077in}{1.473900in}}%
\pgfpathcurveto{\pgfqpoint{5.334121in}{1.478943in}}{\pgfqpoint{5.336955in}{1.485785in}}{\pgfqpoint{5.336955in}{1.492918in}}%
\pgfpathcurveto{\pgfqpoint{5.336955in}{1.500051in}}{\pgfqpoint{5.334121in}{1.506892in}}{\pgfqpoint{5.329077in}{1.511936in}}%
\pgfpathcurveto{\pgfqpoint{5.324034in}{1.516980in}}{\pgfqpoint{5.317192in}{1.519814in}}{\pgfqpoint{5.310059in}{1.519814in}}%
\pgfpathcurveto{\pgfqpoint{5.302926in}{1.519814in}}{\pgfqpoint{5.296085in}{1.516980in}}{\pgfqpoint{5.291041in}{1.511936in}}%
\pgfpathcurveto{\pgfqpoint{5.285997in}{1.506892in}}{\pgfqpoint{5.283163in}{1.500051in}}{\pgfqpoint{5.283163in}{1.492918in}}%
\pgfpathcurveto{\pgfqpoint{5.283163in}{1.485785in}}{\pgfqpoint{5.285997in}{1.478943in}}{\pgfqpoint{5.291041in}{1.473900in}}%
\pgfpathcurveto{\pgfqpoint{5.296085in}{1.468856in}}{\pgfqpoint{5.302926in}{1.466022in}}{\pgfqpoint{5.310059in}{1.466022in}}%
\pgfpathclose%
\pgfusepath{stroke,fill}%
\end{pgfscope}%
\begin{pgfscope}%
\pgfpathrectangle{\pgfqpoint{4.985294in}{0.500000in}}{\pgfqpoint{1.764706in}{1.700000in}}%
\pgfusepath{clip}%
\pgfsetbuttcap%
\pgfsetroundjoin%
\definecolor{currentfill}{rgb}{0.963379,0.625574,0.465113}%
\pgfsetfillcolor{currentfill}%
\pgfsetlinewidth{0.311001pt}%
\definecolor{currentstroke}{rgb}{1.000000,1.000000,1.000000}%
\pgfsetstrokecolor{currentstroke}%
\pgfsetdash{}{0pt}%
\pgfpathmoveto{\pgfqpoint{6.419013in}{1.213292in}}%
\pgfpathcurveto{\pgfqpoint{6.426145in}{1.213292in}}{\pgfqpoint{6.432987in}{1.216126in}}{\pgfqpoint{6.438031in}{1.221169in}}%
\pgfpathcurveto{\pgfqpoint{6.443074in}{1.226213in}}{\pgfqpoint{6.445908in}{1.233055in}}{\pgfqpoint{6.445908in}{1.240187in}}%
\pgfpathcurveto{\pgfqpoint{6.445908in}{1.247320in}}{\pgfqpoint{6.443074in}{1.254162in}}{\pgfqpoint{6.438031in}{1.259205in}}%
\pgfpathcurveto{\pgfqpoint{6.432987in}{1.264249in}}{\pgfqpoint{6.426145in}{1.267083in}}{\pgfqpoint{6.419013in}{1.267083in}}%
\pgfpathcurveto{\pgfqpoint{6.411880in}{1.267083in}}{\pgfqpoint{6.405038in}{1.264249in}}{\pgfqpoint{6.399994in}{1.259205in}}%
\pgfpathcurveto{\pgfqpoint{6.394951in}{1.254162in}}{\pgfqpoint{6.392117in}{1.247320in}}{\pgfqpoint{6.392117in}{1.240187in}}%
\pgfpathcurveto{\pgfqpoint{6.392117in}{1.233055in}}{\pgfqpoint{6.394951in}{1.226213in}}{\pgfqpoint{6.399994in}{1.221169in}}%
\pgfpathcurveto{\pgfqpoint{6.405038in}{1.216126in}}{\pgfqpoint{6.411880in}{1.213292in}}{\pgfqpoint{6.419013in}{1.213292in}}%
\pgfpathclose%
\pgfusepath{stroke,fill}%
\end{pgfscope}%
\begin{pgfscope}%
\pgfpathrectangle{\pgfqpoint{4.985294in}{0.500000in}}{\pgfqpoint{1.764706in}{1.700000in}}%
\pgfusepath{clip}%
\pgfsetbuttcap%
\pgfsetroundjoin%
\definecolor{currentfill}{rgb}{0.966560,0.756582,0.625273}%
\pgfsetfillcolor{currentfill}%
\pgfsetlinewidth{0.311001pt}%
\definecolor{currentstroke}{rgb}{1.000000,1.000000,1.000000}%
\pgfsetstrokecolor{currentstroke}%
\pgfsetdash{}{0pt}%
\pgfpathmoveto{\pgfqpoint{6.399468in}{1.265758in}}%
\pgfpathcurveto{\pgfqpoint{6.406601in}{1.265758in}}{\pgfqpoint{6.413443in}{1.268592in}}{\pgfqpoint{6.418486in}{1.273635in}}%
\pgfpathcurveto{\pgfqpoint{6.423530in}{1.278679in}}{\pgfqpoint{6.426364in}{1.285521in}}{\pgfqpoint{6.426364in}{1.292653in}}%
\pgfpathcurveto{\pgfqpoint{6.426364in}{1.299786in}}{\pgfqpoint{6.423530in}{1.306628in}}{\pgfqpoint{6.418486in}{1.311672in}}%
\pgfpathcurveto{\pgfqpoint{6.413443in}{1.316715in}}{\pgfqpoint{6.406601in}{1.319549in}}{\pgfqpoint{6.399468in}{1.319549in}}%
\pgfpathcurveto{\pgfqpoint{6.392335in}{1.319549in}}{\pgfqpoint{6.385494in}{1.316715in}}{\pgfqpoint{6.380450in}{1.311672in}}%
\pgfpathcurveto{\pgfqpoint{6.375406in}{1.306628in}}{\pgfqpoint{6.372573in}{1.299786in}}{\pgfqpoint{6.372573in}{1.292653in}}%
\pgfpathcurveto{\pgfqpoint{6.372573in}{1.285521in}}{\pgfqpoint{6.375406in}{1.278679in}}{\pgfqpoint{6.380450in}{1.273635in}}%
\pgfpathcurveto{\pgfqpoint{6.385494in}{1.268592in}}{\pgfqpoint{6.392335in}{1.265758in}}{\pgfqpoint{6.399468in}{1.265758in}}%
\pgfpathclose%
\pgfusepath{stroke,fill}%
\end{pgfscope}%
\begin{pgfscope}%
\pgfpathrectangle{\pgfqpoint{4.985294in}{0.500000in}}{\pgfqpoint{1.764706in}{1.700000in}}%
\pgfusepath{clip}%
\pgfsetbuttcap%
\pgfsetroundjoin%
\definecolor{currentfill}{rgb}{0.967398,0.774513,0.650573}%
\pgfsetfillcolor{currentfill}%
\pgfsetlinewidth{0.311001pt}%
\definecolor{currentstroke}{rgb}{1.000000,1.000000,1.000000}%
\pgfsetstrokecolor{currentstroke}%
\pgfsetdash{}{0pt}%
\pgfpathmoveto{\pgfqpoint{5.478161in}{1.671635in}}%
\pgfpathcurveto{\pgfqpoint{5.485294in}{1.671635in}}{\pgfqpoint{5.492136in}{1.674469in}}{\pgfqpoint{5.497180in}{1.679513in}}%
\pgfpathcurveto{\pgfqpoint{5.502223in}{1.684556in}}{\pgfqpoint{5.505057in}{1.691398in}}{\pgfqpoint{5.505057in}{1.698531in}}%
\pgfpathcurveto{\pgfqpoint{5.505057in}{1.705664in}}{\pgfqpoint{5.502223in}{1.712505in}}{\pgfqpoint{5.497180in}{1.717549in}}%
\pgfpathcurveto{\pgfqpoint{5.492136in}{1.722593in}}{\pgfqpoint{5.485294in}{1.725427in}}{\pgfqpoint{5.478161in}{1.725427in}}%
\pgfpathcurveto{\pgfqpoint{5.471029in}{1.725427in}}{\pgfqpoint{5.464187in}{1.722593in}}{\pgfqpoint{5.459143in}{1.717549in}}%
\pgfpathcurveto{\pgfqpoint{5.454100in}{1.712505in}}{\pgfqpoint{5.451266in}{1.705664in}}{\pgfqpoint{5.451266in}{1.698531in}}%
\pgfpathcurveto{\pgfqpoint{5.451266in}{1.691398in}}{\pgfqpoint{5.454100in}{1.684556in}}{\pgfqpoint{5.459143in}{1.679513in}}%
\pgfpathcurveto{\pgfqpoint{5.464187in}{1.674469in}}{\pgfqpoint{5.471029in}{1.671635in}}{\pgfqpoint{5.478161in}{1.671635in}}%
\pgfpathclose%
\pgfusepath{stroke,fill}%
\end{pgfscope}%
\begin{pgfscope}%
\pgfpathrectangle{\pgfqpoint{4.985294in}{0.500000in}}{\pgfqpoint{1.764706in}{1.700000in}}%
\pgfusepath{clip}%
\pgfsetbuttcap%
\pgfsetroundjoin%
\definecolor{currentfill}{rgb}{0.978376,0.897317,0.831308}%
\pgfsetfillcolor{currentfill}%
\pgfsetlinewidth{0.311001pt}%
\definecolor{currentstroke}{rgb}{1.000000,1.000000,1.000000}%
\pgfsetstrokecolor{currentstroke}%
\pgfsetdash{}{0pt}%
\pgfpathmoveto{\pgfqpoint{5.419222in}{1.180888in}}%
\pgfpathcurveto{\pgfqpoint{5.426355in}{1.180888in}}{\pgfqpoint{5.433197in}{1.183722in}}{\pgfqpoint{5.438240in}{1.188766in}}%
\pgfpathcurveto{\pgfqpoint{5.443284in}{1.193809in}}{\pgfqpoint{5.446118in}{1.200651in}}{\pgfqpoint{5.446118in}{1.207784in}}%
\pgfpathcurveto{\pgfqpoint{5.446118in}{1.214917in}}{\pgfqpoint{5.443284in}{1.221758in}}{\pgfqpoint{5.438240in}{1.226802in}}%
\pgfpathcurveto{\pgfqpoint{5.433197in}{1.231846in}}{\pgfqpoint{5.426355in}{1.234680in}}{\pgfqpoint{5.419222in}{1.234680in}}%
\pgfpathcurveto{\pgfqpoint{5.412089in}{1.234680in}}{\pgfqpoint{5.405248in}{1.231846in}}{\pgfqpoint{5.400204in}{1.226802in}}%
\pgfpathcurveto{\pgfqpoint{5.395160in}{1.221758in}}{\pgfqpoint{5.392326in}{1.214917in}}{\pgfqpoint{5.392326in}{1.207784in}}%
\pgfpathcurveto{\pgfqpoint{5.392326in}{1.200651in}}{\pgfqpoint{5.395160in}{1.193809in}}{\pgfqpoint{5.400204in}{1.188766in}}%
\pgfpathcurveto{\pgfqpoint{5.405248in}{1.183722in}}{\pgfqpoint{5.412089in}{1.180888in}}{\pgfqpoint{5.419222in}{1.180888in}}%
\pgfpathclose%
\pgfusepath{stroke,fill}%
\end{pgfscope}%
\begin{pgfscope}%
\pgfpathrectangle{\pgfqpoint{4.985294in}{0.500000in}}{\pgfqpoint{1.764706in}{1.700000in}}%
\pgfusepath{clip}%
\pgfsetbuttcap%
\pgfsetroundjoin%
\definecolor{currentfill}{rgb}{0.965592,0.726236,0.584384}%
\pgfsetfillcolor{currentfill}%
\pgfsetlinewidth{0.311001pt}%
\definecolor{currentstroke}{rgb}{1.000000,1.000000,1.000000}%
\pgfsetstrokecolor{currentstroke}%
\pgfsetdash{}{0pt}%
\pgfpathmoveto{\pgfqpoint{5.567139in}{1.556515in}}%
\pgfpathcurveto{\pgfqpoint{5.574272in}{1.556515in}}{\pgfqpoint{5.581113in}{1.559349in}}{\pgfqpoint{5.586157in}{1.564392in}}%
\pgfpathcurveto{\pgfqpoint{5.591201in}{1.569436in}}{\pgfqpoint{5.594034in}{1.576278in}}{\pgfqpoint{5.594034in}{1.583410in}}%
\pgfpathcurveto{\pgfqpoint{5.594034in}{1.590543in}}{\pgfqpoint{5.591201in}{1.597385in}}{\pgfqpoint{5.586157in}{1.602428in}}%
\pgfpathcurveto{\pgfqpoint{5.581113in}{1.607472in}}{\pgfqpoint{5.574272in}{1.610306in}}{\pgfqpoint{5.567139in}{1.610306in}}%
\pgfpathcurveto{\pgfqpoint{5.560006in}{1.610306in}}{\pgfqpoint{5.553164in}{1.607472in}}{\pgfqpoint{5.548121in}{1.602428in}}%
\pgfpathcurveto{\pgfqpoint{5.543077in}{1.597385in}}{\pgfqpoint{5.540243in}{1.590543in}}{\pgfqpoint{5.540243in}{1.583410in}}%
\pgfpathcurveto{\pgfqpoint{5.540243in}{1.576278in}}{\pgfqpoint{5.543077in}{1.569436in}}{\pgfqpoint{5.548121in}{1.564392in}}%
\pgfpathcurveto{\pgfqpoint{5.553164in}{1.559349in}}{\pgfqpoint{5.560006in}{1.556515in}}{\pgfqpoint{5.567139in}{1.556515in}}%
\pgfpathclose%
\pgfusepath{stroke,fill}%
\end{pgfscope}%
\begin{pgfscope}%
\pgfpathrectangle{\pgfqpoint{4.985294in}{0.500000in}}{\pgfqpoint{1.764706in}{1.700000in}}%
\pgfusepath{clip}%
\pgfsetbuttcap%
\pgfsetroundjoin%
\definecolor{currentfill}{rgb}{0.852817,0.156578,0.279098}%
\pgfsetfillcolor{currentfill}%
\pgfsetlinewidth{0.311001pt}%
\definecolor{currentstroke}{rgb}{1.000000,1.000000,1.000000}%
\pgfsetstrokecolor{currentstroke}%
\pgfsetdash{}{0pt}%
\pgfpathmoveto{\pgfqpoint{5.313907in}{1.624234in}}%
\pgfpathcurveto{\pgfqpoint{5.321040in}{1.624234in}}{\pgfqpoint{5.327882in}{1.627068in}}{\pgfqpoint{5.332925in}{1.632112in}}%
\pgfpathcurveto{\pgfqpoint{5.337969in}{1.637155in}}{\pgfqpoint{5.340803in}{1.643997in}}{\pgfqpoint{5.340803in}{1.651130in}}%
\pgfpathcurveto{\pgfqpoint{5.340803in}{1.658263in}}{\pgfqpoint{5.337969in}{1.665104in}}{\pgfqpoint{5.332925in}{1.670148in}}%
\pgfpathcurveto{\pgfqpoint{5.327882in}{1.675192in}}{\pgfqpoint{5.321040in}{1.678026in}}{\pgfqpoint{5.313907in}{1.678026in}}%
\pgfpathcurveto{\pgfqpoint{5.306774in}{1.678026in}}{\pgfqpoint{5.299933in}{1.675192in}}{\pgfqpoint{5.294889in}{1.670148in}}%
\pgfpathcurveto{\pgfqpoint{5.289845in}{1.665104in}}{\pgfqpoint{5.287012in}{1.658263in}}{\pgfqpoint{5.287012in}{1.651130in}}%
\pgfpathcurveto{\pgfqpoint{5.287012in}{1.643997in}}{\pgfqpoint{5.289845in}{1.637155in}}{\pgfqpoint{5.294889in}{1.632112in}}%
\pgfpathcurveto{\pgfqpoint{5.299933in}{1.627068in}}{\pgfqpoint{5.306774in}{1.624234in}}{\pgfqpoint{5.313907in}{1.624234in}}%
\pgfpathclose%
\pgfusepath{stroke,fill}%
\end{pgfscope}%
\begin{pgfscope}%
\pgfpathrectangle{\pgfqpoint{4.985294in}{0.500000in}}{\pgfqpoint{1.764706in}{1.700000in}}%
\pgfusepath{clip}%
\pgfsetbuttcap%
\pgfsetroundjoin%
\definecolor{currentfill}{rgb}{0.965302,0.713942,0.568499}%
\pgfsetfillcolor{currentfill}%
\pgfsetlinewidth{0.311001pt}%
\definecolor{currentstroke}{rgb}{1.000000,1.000000,1.000000}%
\pgfsetstrokecolor{currentstroke}%
\pgfsetdash{}{0pt}%
\pgfpathmoveto{\pgfqpoint{6.320532in}{1.627028in}}%
\pgfpathcurveto{\pgfqpoint{6.327665in}{1.627028in}}{\pgfqpoint{6.334507in}{1.629862in}}{\pgfqpoint{6.339550in}{1.634905in}}%
\pgfpathcurveto{\pgfqpoint{6.344594in}{1.639949in}}{\pgfqpoint{6.347428in}{1.646791in}}{\pgfqpoint{6.347428in}{1.653923in}}%
\pgfpathcurveto{\pgfqpoint{6.347428in}{1.661056in}}{\pgfqpoint{6.344594in}{1.667898in}}{\pgfqpoint{6.339550in}{1.672942in}}%
\pgfpathcurveto{\pgfqpoint{6.334507in}{1.677985in}}{\pgfqpoint{6.327665in}{1.680819in}}{\pgfqpoint{6.320532in}{1.680819in}}%
\pgfpathcurveto{\pgfqpoint{6.313399in}{1.680819in}}{\pgfqpoint{6.306558in}{1.677985in}}{\pgfqpoint{6.301514in}{1.672942in}}%
\pgfpathcurveto{\pgfqpoint{6.296471in}{1.667898in}}{\pgfqpoint{6.293637in}{1.661056in}}{\pgfqpoint{6.293637in}{1.653923in}}%
\pgfpathcurveto{\pgfqpoint{6.293637in}{1.646791in}}{\pgfqpoint{6.296471in}{1.639949in}}{\pgfqpoint{6.301514in}{1.634905in}}%
\pgfpathcurveto{\pgfqpoint{6.306558in}{1.629862in}}{\pgfqpoint{6.313399in}{1.627028in}}{\pgfqpoint{6.320532in}{1.627028in}}%
\pgfpathclose%
\pgfusepath{stroke,fill}%
\end{pgfscope}%
\begin{pgfscope}%
\pgfpathrectangle{\pgfqpoint{4.985294in}{0.500000in}}{\pgfqpoint{1.764706in}{1.700000in}}%
\pgfusepath{clip}%
\pgfsetbuttcap%
\pgfsetroundjoin%
\definecolor{currentfill}{rgb}{0.975644,0.874038,0.797253}%
\pgfsetfillcolor{currentfill}%
\pgfsetlinewidth{0.311001pt}%
\definecolor{currentstroke}{rgb}{1.000000,1.000000,1.000000}%
\pgfsetstrokecolor{currentstroke}%
\pgfsetdash{}{0pt}%
\pgfpathmoveto{\pgfqpoint{5.478956in}{1.124587in}}%
\pgfpathcurveto{\pgfqpoint{5.486089in}{1.124587in}}{\pgfqpoint{5.492931in}{1.127421in}}{\pgfqpoint{5.497974in}{1.132465in}}%
\pgfpathcurveto{\pgfqpoint{5.503018in}{1.137508in}}{\pgfqpoint{5.505852in}{1.144350in}}{\pgfqpoint{5.505852in}{1.151483in}}%
\pgfpathcurveto{\pgfqpoint{5.505852in}{1.158616in}}{\pgfqpoint{5.503018in}{1.165457in}}{\pgfqpoint{5.497974in}{1.170501in}}%
\pgfpathcurveto{\pgfqpoint{5.492931in}{1.175545in}}{\pgfqpoint{5.486089in}{1.178379in}}{\pgfqpoint{5.478956in}{1.178379in}}%
\pgfpathcurveto{\pgfqpoint{5.471823in}{1.178379in}}{\pgfqpoint{5.464982in}{1.175545in}}{\pgfqpoint{5.459938in}{1.170501in}}%
\pgfpathcurveto{\pgfqpoint{5.454894in}{1.165457in}}{\pgfqpoint{5.452060in}{1.158616in}}{\pgfqpoint{5.452060in}{1.151483in}}%
\pgfpathcurveto{\pgfqpoint{5.452060in}{1.144350in}}{\pgfqpoint{5.454894in}{1.137508in}}{\pgfqpoint{5.459938in}{1.132465in}}%
\pgfpathcurveto{\pgfqpoint{5.464982in}{1.127421in}}{\pgfqpoint{5.471823in}{1.124587in}}{\pgfqpoint{5.478956in}{1.124587in}}%
\pgfpathclose%
\pgfusepath{stroke,fill}%
\end{pgfscope}%
\begin{pgfscope}%
\pgfpathrectangle{\pgfqpoint{4.985294in}{0.500000in}}{\pgfqpoint{1.764706in}{1.700000in}}%
\pgfusepath{clip}%
\pgfsetbuttcap%
\pgfsetroundjoin%
\definecolor{currentfill}{rgb}{0.965928,0.738443,0.600540}%
\pgfsetfillcolor{currentfill}%
\pgfsetlinewidth{0.311001pt}%
\definecolor{currentstroke}{rgb}{1.000000,1.000000,1.000000}%
\pgfsetstrokecolor{currentstroke}%
\pgfsetdash{}{0pt}%
\pgfpathmoveto{\pgfqpoint{5.337340in}{1.400763in}}%
\pgfpathcurveto{\pgfqpoint{5.344472in}{1.400763in}}{\pgfqpoint{5.351314in}{1.403597in}}{\pgfqpoint{5.356358in}{1.408641in}}%
\pgfpathcurveto{\pgfqpoint{5.361401in}{1.413685in}}{\pgfqpoint{5.364235in}{1.420526in}}{\pgfqpoint{5.364235in}{1.427659in}}%
\pgfpathcurveto{\pgfqpoint{5.364235in}{1.434792in}}{\pgfqpoint{5.361401in}{1.441634in}}{\pgfqpoint{5.356358in}{1.446677in}}%
\pgfpathcurveto{\pgfqpoint{5.351314in}{1.451721in}}{\pgfqpoint{5.344472in}{1.454555in}}{\pgfqpoint{5.337340in}{1.454555in}}%
\pgfpathcurveto{\pgfqpoint{5.330207in}{1.454555in}}{\pgfqpoint{5.323365in}{1.451721in}}{\pgfqpoint{5.318321in}{1.446677in}}%
\pgfpathcurveto{\pgfqpoint{5.313278in}{1.441634in}}{\pgfqpoint{5.310444in}{1.434792in}}{\pgfqpoint{5.310444in}{1.427659in}}%
\pgfpathcurveto{\pgfqpoint{5.310444in}{1.420526in}}{\pgfqpoint{5.313278in}{1.413685in}}{\pgfqpoint{5.318321in}{1.408641in}}%
\pgfpathcurveto{\pgfqpoint{5.323365in}{1.403597in}}{\pgfqpoint{5.330207in}{1.400763in}}{\pgfqpoint{5.337340in}{1.400763in}}%
\pgfpathclose%
\pgfusepath{stroke,fill}%
\end{pgfscope}%
\begin{pgfscope}%
\pgfpathrectangle{\pgfqpoint{4.985294in}{0.500000in}}{\pgfqpoint{1.764706in}{1.700000in}}%
\pgfusepath{clip}%
\pgfsetbuttcap%
\pgfsetroundjoin%
\definecolor{currentfill}{rgb}{0.976961,0.885681,0.814303}%
\pgfsetfillcolor{currentfill}%
\pgfsetlinewidth{0.311001pt}%
\definecolor{currentstroke}{rgb}{1.000000,1.000000,1.000000}%
\pgfsetstrokecolor{currentstroke}%
\pgfsetdash{}{0pt}%
\pgfpathmoveto{\pgfqpoint{5.398930in}{1.209580in}}%
\pgfpathcurveto{\pgfqpoint{5.406063in}{1.209580in}}{\pgfqpoint{5.412905in}{1.212414in}}{\pgfqpoint{5.417948in}{1.217457in}}%
\pgfpathcurveto{\pgfqpoint{5.422992in}{1.222501in}}{\pgfqpoint{5.425826in}{1.229343in}}{\pgfqpoint{5.425826in}{1.236475in}}%
\pgfpathcurveto{\pgfqpoint{5.425826in}{1.243608in}}{\pgfqpoint{5.422992in}{1.250450in}}{\pgfqpoint{5.417948in}{1.255494in}}%
\pgfpathcurveto{\pgfqpoint{5.412905in}{1.260537in}}{\pgfqpoint{5.406063in}{1.263371in}}{\pgfqpoint{5.398930in}{1.263371in}}%
\pgfpathcurveto{\pgfqpoint{5.391797in}{1.263371in}}{\pgfqpoint{5.384956in}{1.260537in}}{\pgfqpoint{5.379912in}{1.255494in}}%
\pgfpathcurveto{\pgfqpoint{5.374868in}{1.250450in}}{\pgfqpoint{5.372035in}{1.243608in}}{\pgfqpoint{5.372035in}{1.236475in}}%
\pgfpathcurveto{\pgfqpoint{5.372035in}{1.229343in}}{\pgfqpoint{5.374868in}{1.222501in}}{\pgfqpoint{5.379912in}{1.217457in}}%
\pgfpathcurveto{\pgfqpoint{5.384956in}{1.212414in}}{\pgfqpoint{5.391797in}{1.209580in}}{\pgfqpoint{5.398930in}{1.209580in}}%
\pgfpathclose%
\pgfusepath{stroke,fill}%
\end{pgfscope}%
\begin{pgfscope}%
\pgfpathrectangle{\pgfqpoint{4.985294in}{0.500000in}}{\pgfqpoint{1.764706in}{1.700000in}}%
\pgfusepath{clip}%
\pgfsetbuttcap%
\pgfsetroundjoin%
\definecolor{currentfill}{rgb}{0.979891,0.908948,0.848279}%
\pgfsetfillcolor{currentfill}%
\pgfsetlinewidth{0.311001pt}%
\definecolor{currentstroke}{rgb}{1.000000,1.000000,1.000000}%
\pgfsetstrokecolor{currentstroke}%
\pgfsetdash{}{0pt}%
\pgfpathmoveto{\pgfqpoint{6.295347in}{1.208979in}}%
\pgfpathcurveto{\pgfqpoint{6.302480in}{1.208979in}}{\pgfqpoint{6.309322in}{1.211813in}}{\pgfqpoint{6.314365in}{1.216857in}}%
\pgfpathcurveto{\pgfqpoint{6.319409in}{1.221900in}}{\pgfqpoint{6.322243in}{1.228742in}}{\pgfqpoint{6.322243in}{1.235875in}}%
\pgfpathcurveto{\pgfqpoint{6.322243in}{1.243008in}}{\pgfqpoint{6.319409in}{1.249849in}}{\pgfqpoint{6.314365in}{1.254893in}}%
\pgfpathcurveto{\pgfqpoint{6.309322in}{1.259937in}}{\pgfqpoint{6.302480in}{1.262771in}}{\pgfqpoint{6.295347in}{1.262771in}}%
\pgfpathcurveto{\pgfqpoint{6.288214in}{1.262771in}}{\pgfqpoint{6.281373in}{1.259937in}}{\pgfqpoint{6.276329in}{1.254893in}}%
\pgfpathcurveto{\pgfqpoint{6.271286in}{1.249849in}}{\pgfqpoint{6.268452in}{1.243008in}}{\pgfqpoint{6.268452in}{1.235875in}}%
\pgfpathcurveto{\pgfqpoint{6.268452in}{1.228742in}}{\pgfqpoint{6.271286in}{1.221900in}}{\pgfqpoint{6.276329in}{1.216857in}}%
\pgfpathcurveto{\pgfqpoint{6.281373in}{1.211813in}}{\pgfqpoint{6.288214in}{1.208979in}}{\pgfqpoint{6.295347in}{1.208979in}}%
\pgfpathclose%
\pgfusepath{stroke,fill}%
\end{pgfscope}%
\begin{pgfscope}%
\pgfpathrectangle{\pgfqpoint{4.985294in}{0.500000in}}{\pgfqpoint{1.764706in}{1.700000in}}%
\pgfusepath{clip}%
\pgfsetbuttcap%
\pgfsetroundjoin%
\definecolor{currentfill}{rgb}{0.963884,0.644842,0.486120}%
\pgfsetfillcolor{currentfill}%
\pgfsetlinewidth{0.311001pt}%
\definecolor{currentstroke}{rgb}{1.000000,1.000000,1.000000}%
\pgfsetstrokecolor{currentstroke}%
\pgfsetdash{}{0pt}%
\pgfpathmoveto{\pgfqpoint{5.610678in}{1.673976in}}%
\pgfpathcurveto{\pgfqpoint{5.617811in}{1.673976in}}{\pgfqpoint{5.624652in}{1.676810in}}{\pgfqpoint{5.629696in}{1.681853in}}%
\pgfpathcurveto{\pgfqpoint{5.634740in}{1.686897in}}{\pgfqpoint{5.637573in}{1.693739in}}{\pgfqpoint{5.637573in}{1.700871in}}%
\pgfpathcurveto{\pgfqpoint{5.637573in}{1.708004in}}{\pgfqpoint{5.634740in}{1.714846in}}{\pgfqpoint{5.629696in}{1.719890in}}%
\pgfpathcurveto{\pgfqpoint{5.624652in}{1.724933in}}{\pgfqpoint{5.617811in}{1.727767in}}{\pgfqpoint{5.610678in}{1.727767in}}%
\pgfpathcurveto{\pgfqpoint{5.603545in}{1.727767in}}{\pgfqpoint{5.596703in}{1.724933in}}{\pgfqpoint{5.591660in}{1.719890in}}%
\pgfpathcurveto{\pgfqpoint{5.586616in}{1.714846in}}{\pgfqpoint{5.583782in}{1.708004in}}{\pgfqpoint{5.583782in}{1.700871in}}%
\pgfpathcurveto{\pgfqpoint{5.583782in}{1.693739in}}{\pgfqpoint{5.586616in}{1.686897in}}{\pgfqpoint{5.591660in}{1.681853in}}%
\pgfpathcurveto{\pgfqpoint{5.596703in}{1.676810in}}{\pgfqpoint{5.603545in}{1.673976in}}{\pgfqpoint{5.610678in}{1.673976in}}%
\pgfpathclose%
\pgfusepath{stroke,fill}%
\end{pgfscope}%
\begin{pgfscope}%
\pgfpathrectangle{\pgfqpoint{4.985294in}{0.500000in}}{\pgfqpoint{1.764706in}{1.700000in}}%
\pgfusepath{clip}%
\pgfsetbuttcap%
\pgfsetroundjoin%
\definecolor{currentfill}{rgb}{0.981377,0.920617,0.865369}%
\pgfsetfillcolor{currentfill}%
\pgfsetlinewidth{0.311001pt}%
\definecolor{currentstroke}{rgb}{1.000000,1.000000,1.000000}%
\pgfsetstrokecolor{currentstroke}%
\pgfsetdash{}{0pt}%
\pgfpathmoveto{\pgfqpoint{6.324646in}{1.355670in}}%
\pgfpathcurveto{\pgfqpoint{6.331779in}{1.355670in}}{\pgfqpoint{6.338620in}{1.358504in}}{\pgfqpoint{6.343664in}{1.363548in}}%
\pgfpathcurveto{\pgfqpoint{6.348708in}{1.368592in}}{\pgfqpoint{6.351542in}{1.375433in}}{\pgfqpoint{6.351542in}{1.382566in}}%
\pgfpathcurveto{\pgfqpoint{6.351542in}{1.389699in}}{\pgfqpoint{6.348708in}{1.396540in}}{\pgfqpoint{6.343664in}{1.401584in}}%
\pgfpathcurveto{\pgfqpoint{6.338620in}{1.406628in}}{\pgfqpoint{6.331779in}{1.409462in}}{\pgfqpoint{6.324646in}{1.409462in}}%
\pgfpathcurveto{\pgfqpoint{6.317513in}{1.409462in}}{\pgfqpoint{6.310671in}{1.406628in}}{\pgfqpoint{6.305628in}{1.401584in}}%
\pgfpathcurveto{\pgfqpoint{6.300584in}{1.396540in}}{\pgfqpoint{6.297750in}{1.389699in}}{\pgfqpoint{6.297750in}{1.382566in}}%
\pgfpathcurveto{\pgfqpoint{6.297750in}{1.375433in}}{\pgfqpoint{6.300584in}{1.368592in}}{\pgfqpoint{6.305628in}{1.363548in}}%
\pgfpathcurveto{\pgfqpoint{6.310671in}{1.358504in}}{\pgfqpoint{6.317513in}{1.355670in}}{\pgfqpoint{6.324646in}{1.355670in}}%
\pgfpathclose%
\pgfusepath{stroke,fill}%
\end{pgfscope}%
\begin{pgfscope}%
\pgfpathrectangle{\pgfqpoint{4.985294in}{0.500000in}}{\pgfqpoint{1.764706in}{1.700000in}}%
\pgfusepath{clip}%
\pgfsetbuttcap%
\pgfsetroundjoin%
\definecolor{currentfill}{rgb}{0.972201,0.839051,0.745789}%
\pgfsetfillcolor{currentfill}%
\pgfsetlinewidth{0.311001pt}%
\definecolor{currentstroke}{rgb}{1.000000,1.000000,1.000000}%
\pgfsetstrokecolor{currentstroke}%
\pgfsetdash{}{0pt}%
\pgfpathmoveto{\pgfqpoint{6.189570in}{1.655881in}}%
\pgfpathcurveto{\pgfqpoint{6.196703in}{1.655881in}}{\pgfqpoint{6.203545in}{1.658715in}}{\pgfqpoint{6.208589in}{1.663758in}}%
\pgfpathcurveto{\pgfqpoint{6.213632in}{1.668802in}}{\pgfqpoint{6.216466in}{1.675644in}}{\pgfqpoint{6.216466in}{1.682776in}}%
\pgfpathcurveto{\pgfqpoint{6.216466in}{1.689909in}}{\pgfqpoint{6.213632in}{1.696751in}}{\pgfqpoint{6.208589in}{1.701795in}}%
\pgfpathcurveto{\pgfqpoint{6.203545in}{1.706838in}}{\pgfqpoint{6.196703in}{1.709672in}}{\pgfqpoint{6.189570in}{1.709672in}}%
\pgfpathcurveto{\pgfqpoint{6.182438in}{1.709672in}}{\pgfqpoint{6.175596in}{1.706838in}}{\pgfqpoint{6.170552in}{1.701795in}}%
\pgfpathcurveto{\pgfqpoint{6.165509in}{1.696751in}}{\pgfqpoint{6.162675in}{1.689909in}}{\pgfqpoint{6.162675in}{1.682776in}}%
\pgfpathcurveto{\pgfqpoint{6.162675in}{1.675644in}}{\pgfqpoint{6.165509in}{1.668802in}}{\pgfqpoint{6.170552in}{1.663758in}}%
\pgfpathcurveto{\pgfqpoint{6.175596in}{1.658715in}}{\pgfqpoint{6.182438in}{1.655881in}}{\pgfqpoint{6.189570in}{1.655881in}}%
\pgfpathclose%
\pgfusepath{stroke,fill}%
\end{pgfscope}%
\begin{pgfscope}%
\pgfpathrectangle{\pgfqpoint{4.985294in}{0.500000in}}{\pgfqpoint{1.764706in}{1.700000in}}%
\pgfusepath{clip}%
\pgfsetbuttcap%
\pgfsetroundjoin%
\definecolor{currentfill}{rgb}{0.934351,0.329284,0.247753}%
\pgfsetfillcolor{currentfill}%
\pgfsetlinewidth{0.311001pt}%
\definecolor{currentstroke}{rgb}{1.000000,1.000000,1.000000}%
\pgfsetstrokecolor{currentstroke}%
\pgfsetdash{}{0pt}%
\pgfpathmoveto{\pgfqpoint{6.159666in}{1.825670in}}%
\pgfpathcurveto{\pgfqpoint{6.166799in}{1.825670in}}{\pgfqpoint{6.173641in}{1.828504in}}{\pgfqpoint{6.178684in}{1.833548in}}%
\pgfpathcurveto{\pgfqpoint{6.183728in}{1.838592in}}{\pgfqpoint{6.186562in}{1.845433in}}{\pgfqpoint{6.186562in}{1.852566in}}%
\pgfpathcurveto{\pgfqpoint{6.186562in}{1.859699in}}{\pgfqpoint{6.183728in}{1.866541in}}{\pgfqpoint{6.178684in}{1.871584in}}%
\pgfpathcurveto{\pgfqpoint{6.173641in}{1.876628in}}{\pgfqpoint{6.166799in}{1.879462in}}{\pgfqpoint{6.159666in}{1.879462in}}%
\pgfpathcurveto{\pgfqpoint{6.152533in}{1.879462in}}{\pgfqpoint{6.145692in}{1.876628in}}{\pgfqpoint{6.140648in}{1.871584in}}%
\pgfpathcurveto{\pgfqpoint{6.135604in}{1.866541in}}{\pgfqpoint{6.132770in}{1.859699in}}{\pgfqpoint{6.132770in}{1.852566in}}%
\pgfpathcurveto{\pgfqpoint{6.132770in}{1.845433in}}{\pgfqpoint{6.135604in}{1.838592in}}{\pgfqpoint{6.140648in}{1.833548in}}%
\pgfpathcurveto{\pgfqpoint{6.145692in}{1.828504in}}{\pgfqpoint{6.152533in}{1.825670in}}{\pgfqpoint{6.159666in}{1.825670in}}%
\pgfpathclose%
\pgfusepath{stroke,fill}%
\end{pgfscope}%
\begin{pgfscope}%
\pgfpathrectangle{\pgfqpoint{4.985294in}{0.500000in}}{\pgfqpoint{1.764706in}{1.700000in}}%
\pgfusepath{clip}%
\pgfsetbuttcap%
\pgfsetroundjoin%
\definecolor{currentfill}{rgb}{0.973271,0.850724,0.762998}%
\pgfsetfillcolor{currentfill}%
\pgfsetlinewidth{0.311001pt}%
\definecolor{currentstroke}{rgb}{1.000000,1.000000,1.000000}%
\pgfsetstrokecolor{currentstroke}%
\pgfsetdash{}{0pt}%
\pgfpathmoveto{\pgfqpoint{5.396985in}{1.488420in}}%
\pgfpathcurveto{\pgfqpoint{5.404118in}{1.488420in}}{\pgfqpoint{5.410959in}{1.491254in}}{\pgfqpoint{5.416003in}{1.496298in}}%
\pgfpathcurveto{\pgfqpoint{5.421047in}{1.501342in}}{\pgfqpoint{5.423881in}{1.508183in}}{\pgfqpoint{5.423881in}{1.515316in}}%
\pgfpathcurveto{\pgfqpoint{5.423881in}{1.522449in}}{\pgfqpoint{5.421047in}{1.529290in}}{\pgfqpoint{5.416003in}{1.534334in}}%
\pgfpathcurveto{\pgfqpoint{5.410959in}{1.539378in}}{\pgfqpoint{5.404118in}{1.542212in}}{\pgfqpoint{5.396985in}{1.542212in}}%
\pgfpathcurveto{\pgfqpoint{5.389852in}{1.542212in}}{\pgfqpoint{5.383010in}{1.539378in}}{\pgfqpoint{5.377967in}{1.534334in}}%
\pgfpathcurveto{\pgfqpoint{5.372923in}{1.529290in}}{\pgfqpoint{5.370089in}{1.522449in}}{\pgfqpoint{5.370089in}{1.515316in}}%
\pgfpathcurveto{\pgfqpoint{5.370089in}{1.508183in}}{\pgfqpoint{5.372923in}{1.501342in}}{\pgfqpoint{5.377967in}{1.496298in}}%
\pgfpathcurveto{\pgfqpoint{5.383010in}{1.491254in}}{\pgfqpoint{5.389852in}{1.488420in}}{\pgfqpoint{5.396985in}{1.488420in}}%
\pgfpathclose%
\pgfusepath{stroke,fill}%
\end{pgfscope}%
\begin{pgfscope}%
\pgfpathrectangle{\pgfqpoint{4.985294in}{0.500000in}}{\pgfqpoint{1.764706in}{1.700000in}}%
\pgfusepath{clip}%
\pgfsetbuttcap%
\pgfsetroundjoin%
\definecolor{currentfill}{rgb}{0.972726,0.844889,0.754401}%
\pgfsetfillcolor{currentfill}%
\pgfsetlinewidth{0.311001pt}%
\definecolor{currentstroke}{rgb}{1.000000,1.000000,1.000000}%
\pgfsetstrokecolor{currentstroke}%
\pgfsetdash{}{0pt}%
\pgfpathmoveto{\pgfqpoint{5.397425in}{1.499653in}}%
\pgfpathcurveto{\pgfqpoint{5.404558in}{1.499653in}}{\pgfqpoint{5.411399in}{1.502487in}}{\pgfqpoint{5.416443in}{1.507531in}}%
\pgfpathcurveto{\pgfqpoint{5.421487in}{1.512575in}}{\pgfqpoint{5.424321in}{1.519416in}}{\pgfqpoint{5.424321in}{1.526549in}}%
\pgfpathcurveto{\pgfqpoint{5.424321in}{1.533682in}}{\pgfqpoint{5.421487in}{1.540524in}}{\pgfqpoint{5.416443in}{1.545567in}}%
\pgfpathcurveto{\pgfqpoint{5.411399in}{1.550611in}}{\pgfqpoint{5.404558in}{1.553445in}}{\pgfqpoint{5.397425in}{1.553445in}}%
\pgfpathcurveto{\pgfqpoint{5.390292in}{1.553445in}}{\pgfqpoint{5.383450in}{1.550611in}}{\pgfqpoint{5.378407in}{1.545567in}}%
\pgfpathcurveto{\pgfqpoint{5.373363in}{1.540524in}}{\pgfqpoint{5.370529in}{1.533682in}}{\pgfqpoint{5.370529in}{1.526549in}}%
\pgfpathcurveto{\pgfqpoint{5.370529in}{1.519416in}}{\pgfqpoint{5.373363in}{1.512575in}}{\pgfqpoint{5.378407in}{1.507531in}}%
\pgfpathcurveto{\pgfqpoint{5.383450in}{1.502487in}}{\pgfqpoint{5.390292in}{1.499653in}}{\pgfqpoint{5.397425in}{1.499653in}}%
\pgfpathclose%
\pgfusepath{stroke,fill}%
\end{pgfscope}%
\begin{pgfscope}%
\pgfpathrectangle{\pgfqpoint{4.985294in}{0.500000in}}{\pgfqpoint{1.764706in}{1.700000in}}%
\pgfusepath{clip}%
\pgfsetbuttcap%
\pgfsetroundjoin%
\definecolor{currentfill}{rgb}{0.964920,0.695342,0.545192}%
\pgfsetfillcolor{currentfill}%
\pgfsetlinewidth{0.311001pt}%
\definecolor{currentstroke}{rgb}{1.000000,1.000000,1.000000}%
\pgfsetstrokecolor{currentstroke}%
\pgfsetdash{}{0pt}%
\pgfpathmoveto{\pgfqpoint{6.195360in}{1.335569in}}%
\pgfpathcurveto{\pgfqpoint{6.202493in}{1.335569in}}{\pgfqpoint{6.209335in}{1.338403in}}{\pgfqpoint{6.214378in}{1.343447in}}%
\pgfpathcurveto{\pgfqpoint{6.219422in}{1.348490in}}{\pgfqpoint{6.222256in}{1.355332in}}{\pgfqpoint{6.222256in}{1.362465in}}%
\pgfpathcurveto{\pgfqpoint{6.222256in}{1.369598in}}{\pgfqpoint{6.219422in}{1.376439in}}{\pgfqpoint{6.214378in}{1.381483in}}%
\pgfpathcurveto{\pgfqpoint{6.209335in}{1.386527in}}{\pgfqpoint{6.202493in}{1.389360in}}{\pgfqpoint{6.195360in}{1.389360in}}%
\pgfpathcurveto{\pgfqpoint{6.188227in}{1.389360in}}{\pgfqpoint{6.181386in}{1.386527in}}{\pgfqpoint{6.176342in}{1.381483in}}%
\pgfpathcurveto{\pgfqpoint{6.171298in}{1.376439in}}{\pgfqpoint{6.168464in}{1.369598in}}{\pgfqpoint{6.168464in}{1.362465in}}%
\pgfpathcurveto{\pgfqpoint{6.168464in}{1.355332in}}{\pgfqpoint{6.171298in}{1.348490in}}{\pgfqpoint{6.176342in}{1.343447in}}%
\pgfpathcurveto{\pgfqpoint{6.181386in}{1.338403in}}{\pgfqpoint{6.188227in}{1.335569in}}{\pgfqpoint{6.195360in}{1.335569in}}%
\pgfpathclose%
\pgfusepath{stroke,fill}%
\end{pgfscope}%
\begin{pgfscope}%
\pgfpathrectangle{\pgfqpoint{4.985294in}{0.500000in}}{\pgfqpoint{1.764706in}{1.700000in}}%
\pgfusepath{clip}%
\pgfsetbuttcap%
\pgfsetroundjoin%
\definecolor{currentfill}{rgb}{0.977657,0.891500,0.822809}%
\pgfsetfillcolor{currentfill}%
\pgfsetlinewidth{0.311001pt}%
\definecolor{currentstroke}{rgb}{1.000000,1.000000,1.000000}%
\pgfsetstrokecolor{currentstroke}%
\pgfsetdash{}{0pt}%
\pgfpathmoveto{\pgfqpoint{5.423726in}{1.163102in}}%
\pgfpathcurveto{\pgfqpoint{5.430859in}{1.163102in}}{\pgfqpoint{5.437701in}{1.165936in}}{\pgfqpoint{5.442745in}{1.170980in}}%
\pgfpathcurveto{\pgfqpoint{5.447788in}{1.176024in}}{\pgfqpoint{5.450622in}{1.182865in}}{\pgfqpoint{5.450622in}{1.189998in}}%
\pgfpathcurveto{\pgfqpoint{5.450622in}{1.197131in}}{\pgfqpoint{5.447788in}{1.203973in}}{\pgfqpoint{5.442745in}{1.209016in}}%
\pgfpathcurveto{\pgfqpoint{5.437701in}{1.214060in}}{\pgfqpoint{5.430859in}{1.216894in}}{\pgfqpoint{5.423726in}{1.216894in}}%
\pgfpathcurveto{\pgfqpoint{5.416594in}{1.216894in}}{\pgfqpoint{5.409752in}{1.214060in}}{\pgfqpoint{5.404708in}{1.209016in}}%
\pgfpathcurveto{\pgfqpoint{5.399665in}{1.203973in}}{\pgfqpoint{5.396831in}{1.197131in}}{\pgfqpoint{5.396831in}{1.189998in}}%
\pgfpathcurveto{\pgfqpoint{5.396831in}{1.182865in}}{\pgfqpoint{5.399665in}{1.176024in}}{\pgfqpoint{5.404708in}{1.170980in}}%
\pgfpathcurveto{\pgfqpoint{5.409752in}{1.165936in}}{\pgfqpoint{5.416594in}{1.163102in}}{\pgfqpoint{5.423726in}{1.163102in}}%
\pgfpathclose%
\pgfusepath{stroke,fill}%
\end{pgfscope}%
\begin{pgfscope}%
\pgfpathrectangle{\pgfqpoint{4.985294in}{0.500000in}}{\pgfqpoint{1.764706in}{1.700000in}}%
\pgfusepath{clip}%
\pgfsetbuttcap%
\pgfsetroundjoin%
\definecolor{currentfill}{rgb}{0.963728,0.638439,0.479050}%
\pgfsetfillcolor{currentfill}%
\pgfsetlinewidth{0.311001pt}%
\definecolor{currentstroke}{rgb}{1.000000,1.000000,1.000000}%
\pgfsetstrokecolor{currentstroke}%
\pgfsetdash{}{0pt}%
\pgfpathmoveto{\pgfqpoint{5.563349in}{0.877839in}}%
\pgfpathcurveto{\pgfqpoint{5.570482in}{0.877839in}}{\pgfqpoint{5.577324in}{0.880673in}}{\pgfqpoint{5.582368in}{0.885717in}}%
\pgfpathcurveto{\pgfqpoint{5.587411in}{0.890761in}}{\pgfqpoint{5.590245in}{0.897602in}}{\pgfqpoint{5.590245in}{0.904735in}}%
\pgfpathcurveto{\pgfqpoint{5.590245in}{0.911868in}}{\pgfqpoint{5.587411in}{0.918709in}}{\pgfqpoint{5.582368in}{0.923753in}}%
\pgfpathcurveto{\pgfqpoint{5.577324in}{0.928797in}}{\pgfqpoint{5.570482in}{0.931631in}}{\pgfqpoint{5.563349in}{0.931631in}}%
\pgfpathcurveto{\pgfqpoint{5.556217in}{0.931631in}}{\pgfqpoint{5.549375in}{0.928797in}}{\pgfqpoint{5.544331in}{0.923753in}}%
\pgfpathcurveto{\pgfqpoint{5.539288in}{0.918709in}}{\pgfqpoint{5.536454in}{0.911868in}}{\pgfqpoint{5.536454in}{0.904735in}}%
\pgfpathcurveto{\pgfqpoint{5.536454in}{0.897602in}}{\pgfqpoint{5.539288in}{0.890761in}}{\pgfqpoint{5.544331in}{0.885717in}}%
\pgfpathcurveto{\pgfqpoint{5.549375in}{0.880673in}}{\pgfqpoint{5.556217in}{0.877839in}}{\pgfqpoint{5.563349in}{0.877839in}}%
\pgfpathclose%
\pgfusepath{stroke,fill}%
\end{pgfscope}%
\begin{pgfscope}%
\pgfpathrectangle{\pgfqpoint{4.985294in}{0.500000in}}{\pgfqpoint{1.764706in}{1.700000in}}%
\pgfusepath{clip}%
\pgfsetbuttcap%
\pgfsetroundjoin%
\definecolor{currentfill}{rgb}{0.971694,0.833208,0.737161}%
\pgfsetfillcolor{currentfill}%
\pgfsetlinewidth{0.311001pt}%
\definecolor{currentstroke}{rgb}{1.000000,1.000000,1.000000}%
\pgfsetstrokecolor{currentstroke}%
\pgfsetdash{}{0pt}%
\pgfpathmoveto{\pgfqpoint{6.228496in}{1.507864in}}%
\pgfpathcurveto{\pgfqpoint{6.235629in}{1.507864in}}{\pgfqpoint{6.242471in}{1.510698in}}{\pgfqpoint{6.247514in}{1.515742in}}%
\pgfpathcurveto{\pgfqpoint{6.252558in}{1.520785in}}{\pgfqpoint{6.255392in}{1.527627in}}{\pgfqpoint{6.255392in}{1.534760in}}%
\pgfpathcurveto{\pgfqpoint{6.255392in}{1.541893in}}{\pgfqpoint{6.252558in}{1.548734in}}{\pgfqpoint{6.247514in}{1.553778in}}%
\pgfpathcurveto{\pgfqpoint{6.242471in}{1.558822in}}{\pgfqpoint{6.235629in}{1.561656in}}{\pgfqpoint{6.228496in}{1.561656in}}%
\pgfpathcurveto{\pgfqpoint{6.221363in}{1.561656in}}{\pgfqpoint{6.214522in}{1.558822in}}{\pgfqpoint{6.209478in}{1.553778in}}%
\pgfpathcurveto{\pgfqpoint{6.204434in}{1.548734in}}{\pgfqpoint{6.201601in}{1.541893in}}{\pgfqpoint{6.201601in}{1.534760in}}%
\pgfpathcurveto{\pgfqpoint{6.201601in}{1.527627in}}{\pgfqpoint{6.204434in}{1.520785in}}{\pgfqpoint{6.209478in}{1.515742in}}%
\pgfpathcurveto{\pgfqpoint{6.214522in}{1.510698in}}{\pgfqpoint{6.221363in}{1.507864in}}{\pgfqpoint{6.228496in}{1.507864in}}%
\pgfpathclose%
\pgfusepath{stroke,fill}%
\end{pgfscope}%
\begin{pgfscope}%
\pgfpathrectangle{\pgfqpoint{4.985294in}{0.500000in}}{\pgfqpoint{1.764706in}{1.700000in}}%
\pgfusepath{clip}%
\pgfsetbuttcap%
\pgfsetroundjoin%
\definecolor{currentfill}{rgb}{0.963379,0.625574,0.465113}%
\pgfsetfillcolor{currentfill}%
\pgfsetlinewidth{0.311001pt}%
\definecolor{currentstroke}{rgb}{1.000000,1.000000,1.000000}%
\pgfsetstrokecolor{currentstroke}%
\pgfsetdash{}{0pt}%
\pgfpathmoveto{\pgfqpoint{6.133981in}{1.082475in}}%
\pgfpathcurveto{\pgfqpoint{6.141114in}{1.082475in}}{\pgfqpoint{6.147955in}{1.085309in}}{\pgfqpoint{6.152999in}{1.090353in}}%
\pgfpathcurveto{\pgfqpoint{6.158043in}{1.095397in}}{\pgfqpoint{6.160877in}{1.102238in}}{\pgfqpoint{6.160877in}{1.109371in}}%
\pgfpathcurveto{\pgfqpoint{6.160877in}{1.116504in}}{\pgfqpoint{6.158043in}{1.123346in}}{\pgfqpoint{6.152999in}{1.128389in}}%
\pgfpathcurveto{\pgfqpoint{6.147955in}{1.133433in}}{\pgfqpoint{6.141114in}{1.136267in}}{\pgfqpoint{6.133981in}{1.136267in}}%
\pgfpathcurveto{\pgfqpoint{6.126848in}{1.136267in}}{\pgfqpoint{6.120007in}{1.133433in}}{\pgfqpoint{6.114963in}{1.128389in}}%
\pgfpathcurveto{\pgfqpoint{6.109919in}{1.123346in}}{\pgfqpoint{6.107085in}{1.116504in}}{\pgfqpoint{6.107085in}{1.109371in}}%
\pgfpathcurveto{\pgfqpoint{6.107085in}{1.102238in}}{\pgfqpoint{6.109919in}{1.095397in}}{\pgfqpoint{6.114963in}{1.090353in}}%
\pgfpathcurveto{\pgfqpoint{6.120007in}{1.085309in}}{\pgfqpoint{6.126848in}{1.082475in}}{\pgfqpoint{6.133981in}{1.082475in}}%
\pgfpathclose%
\pgfusepath{stroke,fill}%
\end{pgfscope}%
\begin{pgfscope}%
\pgfpathrectangle{\pgfqpoint{4.985294in}{0.500000in}}{\pgfqpoint{1.764706in}{1.700000in}}%
\pgfusepath{clip}%
\pgfsetbuttcap%
\pgfsetroundjoin%
\definecolor{currentfill}{rgb}{0.972726,0.844889,0.754401}%
\pgfsetfillcolor{currentfill}%
\pgfsetlinewidth{0.311001pt}%
\definecolor{currentstroke}{rgb}{1.000000,1.000000,1.000000}%
\pgfsetstrokecolor{currentstroke}%
\pgfsetdash{}{0pt}%
\pgfpathmoveto{\pgfqpoint{5.501854in}{1.114319in}}%
\pgfpathcurveto{\pgfqpoint{5.508987in}{1.114319in}}{\pgfqpoint{5.515829in}{1.117153in}}{\pgfqpoint{5.520873in}{1.122197in}}%
\pgfpathcurveto{\pgfqpoint{5.525916in}{1.127240in}}{\pgfqpoint{5.528750in}{1.134082in}}{\pgfqpoint{5.528750in}{1.141215in}}%
\pgfpathcurveto{\pgfqpoint{5.528750in}{1.148348in}}{\pgfqpoint{5.525916in}{1.155189in}}{\pgfqpoint{5.520873in}{1.160233in}}%
\pgfpathcurveto{\pgfqpoint{5.515829in}{1.165277in}}{\pgfqpoint{5.508987in}{1.168111in}}{\pgfqpoint{5.501854in}{1.168111in}}%
\pgfpathcurveto{\pgfqpoint{5.494722in}{1.168111in}}{\pgfqpoint{5.487880in}{1.165277in}}{\pgfqpoint{5.482836in}{1.160233in}}%
\pgfpathcurveto{\pgfqpoint{5.477793in}{1.155189in}}{\pgfqpoint{5.474959in}{1.148348in}}{\pgfqpoint{5.474959in}{1.141215in}}%
\pgfpathcurveto{\pgfqpoint{5.474959in}{1.134082in}}{\pgfqpoint{5.477793in}{1.127240in}}{\pgfqpoint{5.482836in}{1.122197in}}%
\pgfpathcurveto{\pgfqpoint{5.487880in}{1.117153in}}{\pgfqpoint{5.494722in}{1.114319in}}{\pgfqpoint{5.501854in}{1.114319in}}%
\pgfpathclose%
\pgfusepath{stroke,fill}%
\end{pgfscope}%
\begin{pgfscope}%
\pgfpathrectangle{\pgfqpoint{4.985294in}{0.500000in}}{\pgfqpoint{1.764706in}{1.700000in}}%
\pgfusepath{clip}%
\pgfsetbuttcap%
\pgfsetroundjoin%
\definecolor{currentfill}{rgb}{0.979891,0.908948,0.848279}%
\pgfsetfillcolor{currentfill}%
\pgfsetlinewidth{0.311001pt}%
\definecolor{currentstroke}{rgb}{1.000000,1.000000,1.000000}%
\pgfsetstrokecolor{currentstroke}%
\pgfsetdash{}{0pt}%
\pgfpathmoveto{\pgfqpoint{6.329363in}{1.240070in}}%
\pgfpathcurveto{\pgfqpoint{6.336496in}{1.240070in}}{\pgfqpoint{6.343337in}{1.242904in}}{\pgfqpoint{6.348381in}{1.247948in}}%
\pgfpathcurveto{\pgfqpoint{6.353424in}{1.252991in}}{\pgfqpoint{6.356258in}{1.259833in}}{\pgfqpoint{6.356258in}{1.266966in}}%
\pgfpathcurveto{\pgfqpoint{6.356258in}{1.274099in}}{\pgfqpoint{6.353424in}{1.280940in}}{\pgfqpoint{6.348381in}{1.285984in}}%
\pgfpathcurveto{\pgfqpoint{6.343337in}{1.291028in}}{\pgfqpoint{6.336496in}{1.293862in}}{\pgfqpoint{6.329363in}{1.293862in}}%
\pgfpathcurveto{\pgfqpoint{6.322230in}{1.293862in}}{\pgfqpoint{6.315388in}{1.291028in}}{\pgfqpoint{6.310345in}{1.285984in}}%
\pgfpathcurveto{\pgfqpoint{6.305301in}{1.280940in}}{\pgfqpoint{6.302467in}{1.274099in}}{\pgfqpoint{6.302467in}{1.266966in}}%
\pgfpathcurveto{\pgfqpoint{6.302467in}{1.259833in}}{\pgfqpoint{6.305301in}{1.252991in}}{\pgfqpoint{6.310345in}{1.247948in}}%
\pgfpathcurveto{\pgfqpoint{6.315388in}{1.242904in}}{\pgfqpoint{6.322230in}{1.240070in}}{\pgfqpoint{6.329363in}{1.240070in}}%
\pgfpathclose%
\pgfusepath{stroke,fill}%
\end{pgfscope}%
\begin{pgfscope}%
\pgfpathrectangle{\pgfqpoint{4.985294in}{0.500000in}}{\pgfqpoint{1.764706in}{1.700000in}}%
\pgfusepath{clip}%
\pgfsetbuttcap%
\pgfsetroundjoin%
\definecolor{currentfill}{rgb}{0.980678,0.914765,0.856766}%
\pgfsetfillcolor{currentfill}%
\pgfsetlinewidth{0.311001pt}%
\definecolor{currentstroke}{rgb}{1.000000,1.000000,1.000000}%
\pgfsetstrokecolor{currentstroke}%
\pgfsetdash{}{0pt}%
\pgfpathmoveto{\pgfqpoint{5.420786in}{1.307152in}}%
\pgfpathcurveto{\pgfqpoint{5.427919in}{1.307152in}}{\pgfqpoint{5.434761in}{1.309986in}}{\pgfqpoint{5.439805in}{1.315030in}}%
\pgfpathcurveto{\pgfqpoint{5.444848in}{1.320074in}}{\pgfqpoint{5.447682in}{1.326915in}}{\pgfqpoint{5.447682in}{1.334048in}}%
\pgfpathcurveto{\pgfqpoint{5.447682in}{1.341181in}}{\pgfqpoint{5.444848in}{1.348023in}}{\pgfqpoint{5.439805in}{1.353066in}}%
\pgfpathcurveto{\pgfqpoint{5.434761in}{1.358110in}}{\pgfqpoint{5.427919in}{1.360944in}}{\pgfqpoint{5.420786in}{1.360944in}}%
\pgfpathcurveto{\pgfqpoint{5.413654in}{1.360944in}}{\pgfqpoint{5.406812in}{1.358110in}}{\pgfqpoint{5.401768in}{1.353066in}}%
\pgfpathcurveto{\pgfqpoint{5.396725in}{1.348023in}}{\pgfqpoint{5.393891in}{1.341181in}}{\pgfqpoint{5.393891in}{1.334048in}}%
\pgfpathcurveto{\pgfqpoint{5.393891in}{1.326915in}}{\pgfqpoint{5.396725in}{1.320074in}}{\pgfqpoint{5.401768in}{1.315030in}}%
\pgfpathcurveto{\pgfqpoint{5.406812in}{1.309986in}}{\pgfqpoint{5.413654in}{1.307152in}}{\pgfqpoint{5.420786in}{1.307152in}}%
\pgfpathclose%
\pgfusepath{stroke,fill}%
\end{pgfscope}%
\begin{pgfscope}%
\pgfpathrectangle{\pgfqpoint{4.985294in}{0.500000in}}{\pgfqpoint{1.764706in}{1.700000in}}%
\pgfusepath{clip}%
\pgfsetbuttcap%
\pgfsetroundjoin%
\definecolor{currentfill}{rgb}{0.977657,0.891500,0.822809}%
\pgfsetfillcolor{currentfill}%
\pgfsetlinewidth{0.311001pt}%
\definecolor{currentstroke}{rgb}{1.000000,1.000000,1.000000}%
\pgfsetstrokecolor{currentstroke}%
\pgfsetdash{}{0pt}%
\pgfpathmoveto{\pgfqpoint{6.300245in}{1.140106in}}%
\pgfpathcurveto{\pgfqpoint{6.307377in}{1.140106in}}{\pgfqpoint{6.314219in}{1.142940in}}{\pgfqpoint{6.319263in}{1.147983in}}%
\pgfpathcurveto{\pgfqpoint{6.324306in}{1.153027in}}{\pgfqpoint{6.327140in}{1.159869in}}{\pgfqpoint{6.327140in}{1.167001in}}%
\pgfpathcurveto{\pgfqpoint{6.327140in}{1.174134in}}{\pgfqpoint{6.324306in}{1.180976in}}{\pgfqpoint{6.319263in}{1.186019in}}%
\pgfpathcurveto{\pgfqpoint{6.314219in}{1.191063in}}{\pgfqpoint{6.307377in}{1.193897in}}{\pgfqpoint{6.300245in}{1.193897in}}%
\pgfpathcurveto{\pgfqpoint{6.293112in}{1.193897in}}{\pgfqpoint{6.286270in}{1.191063in}}{\pgfqpoint{6.281226in}{1.186019in}}%
\pgfpathcurveto{\pgfqpoint{6.276183in}{1.180976in}}{\pgfqpoint{6.273349in}{1.174134in}}{\pgfqpoint{6.273349in}{1.167001in}}%
\pgfpathcurveto{\pgfqpoint{6.273349in}{1.159869in}}{\pgfqpoint{6.276183in}{1.153027in}}{\pgfqpoint{6.281226in}{1.147983in}}%
\pgfpathcurveto{\pgfqpoint{6.286270in}{1.142940in}}{\pgfqpoint{6.293112in}{1.140106in}}{\pgfqpoint{6.300245in}{1.140106in}}%
\pgfpathclose%
\pgfusepath{stroke,fill}%
\end{pgfscope}%
\begin{pgfscope}%
\pgfpathrectangle{\pgfqpoint{4.985294in}{0.500000in}}{\pgfqpoint{1.764706in}{1.700000in}}%
\pgfusepath{clip}%
\pgfsetbuttcap%
\pgfsetroundjoin%
\definecolor{currentfill}{rgb}{0.963884,0.644842,0.486120}%
\pgfsetfillcolor{currentfill}%
\pgfsetlinewidth{0.311001pt}%
\definecolor{currentstroke}{rgb}{1.000000,1.000000,1.000000}%
\pgfsetstrokecolor{currentstroke}%
\pgfsetdash{}{0pt}%
\pgfpathmoveto{\pgfqpoint{6.407219in}{1.447120in}}%
\pgfpathcurveto{\pgfqpoint{6.414352in}{1.447120in}}{\pgfqpoint{6.421194in}{1.449954in}}{\pgfqpoint{6.426238in}{1.454998in}}%
\pgfpathcurveto{\pgfqpoint{6.431281in}{1.460041in}}{\pgfqpoint{6.434115in}{1.466883in}}{\pgfqpoint{6.434115in}{1.474016in}}%
\pgfpathcurveto{\pgfqpoint{6.434115in}{1.481149in}}{\pgfqpoint{6.431281in}{1.487990in}}{\pgfqpoint{6.426238in}{1.493034in}}%
\pgfpathcurveto{\pgfqpoint{6.421194in}{1.498078in}}{\pgfqpoint{6.414352in}{1.500911in}}{\pgfqpoint{6.407219in}{1.500911in}}%
\pgfpathcurveto{\pgfqpoint{6.400087in}{1.500911in}}{\pgfqpoint{6.393245in}{1.498078in}}{\pgfqpoint{6.388201in}{1.493034in}}%
\pgfpathcurveto{\pgfqpoint{6.383158in}{1.487990in}}{\pgfqpoint{6.380324in}{1.481149in}}{\pgfqpoint{6.380324in}{1.474016in}}%
\pgfpathcurveto{\pgfqpoint{6.380324in}{1.466883in}}{\pgfqpoint{6.383158in}{1.460041in}}{\pgfqpoint{6.388201in}{1.454998in}}%
\pgfpathcurveto{\pgfqpoint{6.393245in}{1.449954in}}{\pgfqpoint{6.400087in}{1.447120in}}{\pgfqpoint{6.407219in}{1.447120in}}%
\pgfpathclose%
\pgfusepath{stroke,fill}%
\end{pgfscope}%
\begin{pgfscope}%
\pgfpathrectangle{\pgfqpoint{4.985294in}{0.500000in}}{\pgfqpoint{1.764706in}{1.700000in}}%
\pgfusepath{clip}%
\pgfsetbuttcap%
\pgfsetroundjoin%
\definecolor{currentfill}{rgb}{0.976961,0.885681,0.814303}%
\pgfsetfillcolor{currentfill}%
\pgfsetlinewidth{0.311001pt}%
\definecolor{currentstroke}{rgb}{1.000000,1.000000,1.000000}%
\pgfsetstrokecolor{currentstroke}%
\pgfsetdash{}{0pt}%
\pgfpathmoveto{\pgfqpoint{5.460506in}{1.447353in}}%
\pgfpathcurveto{\pgfqpoint{5.467639in}{1.447353in}}{\pgfqpoint{5.474481in}{1.450187in}}{\pgfqpoint{5.479524in}{1.455231in}}%
\pgfpathcurveto{\pgfqpoint{5.484568in}{1.460274in}}{\pgfqpoint{5.487402in}{1.467116in}}{\pgfqpoint{5.487402in}{1.474249in}}%
\pgfpathcurveto{\pgfqpoint{5.487402in}{1.481381in}}{\pgfqpoint{5.484568in}{1.488223in}}{\pgfqpoint{5.479524in}{1.493267in}}%
\pgfpathcurveto{\pgfqpoint{5.474481in}{1.498310in}}{\pgfqpoint{5.467639in}{1.501144in}}{\pgfqpoint{5.460506in}{1.501144in}}%
\pgfpathcurveto{\pgfqpoint{5.453373in}{1.501144in}}{\pgfqpoint{5.446532in}{1.498310in}}{\pgfqpoint{5.441488in}{1.493267in}}%
\pgfpathcurveto{\pgfqpoint{5.436444in}{1.488223in}}{\pgfqpoint{5.433610in}{1.481381in}}{\pgfqpoint{5.433610in}{1.474249in}}%
\pgfpathcurveto{\pgfqpoint{5.433610in}{1.467116in}}{\pgfqpoint{5.436444in}{1.460274in}}{\pgfqpoint{5.441488in}{1.455231in}}%
\pgfpathcurveto{\pgfqpoint{5.446532in}{1.450187in}}{\pgfqpoint{5.453373in}{1.447353in}}{\pgfqpoint{5.460506in}{1.447353in}}%
\pgfpathclose%
\pgfusepath{stroke,fill}%
\end{pgfscope}%
\begin{pgfscope}%
\pgfpathrectangle{\pgfqpoint{4.985294in}{0.500000in}}{\pgfqpoint{1.764706in}{1.700000in}}%
\pgfusepath{clip}%
\pgfsetbuttcap%
\pgfsetroundjoin%
\definecolor{currentfill}{rgb}{0.857426,0.162258,0.276275}%
\pgfsetfillcolor{currentfill}%
\pgfsetlinewidth{0.311001pt}%
\definecolor{currentstroke}{rgb}{1.000000,1.000000,1.000000}%
\pgfsetstrokecolor{currentstroke}%
\pgfsetdash{}{0pt}%
\pgfpathmoveto{\pgfqpoint{6.035908in}{1.826743in}}%
\pgfpathcurveto{\pgfqpoint{6.043041in}{1.826743in}}{\pgfqpoint{6.049882in}{1.829577in}}{\pgfqpoint{6.054926in}{1.834621in}}%
\pgfpathcurveto{\pgfqpoint{6.059969in}{1.839664in}}{\pgfqpoint{6.062803in}{1.846506in}}{\pgfqpoint{6.062803in}{1.853639in}}%
\pgfpathcurveto{\pgfqpoint{6.062803in}{1.860772in}}{\pgfqpoint{6.059969in}{1.867613in}}{\pgfqpoint{6.054926in}{1.872657in}}%
\pgfpathcurveto{\pgfqpoint{6.049882in}{1.877701in}}{\pgfqpoint{6.043041in}{1.880535in}}{\pgfqpoint{6.035908in}{1.880535in}}%
\pgfpathcurveto{\pgfqpoint{6.028775in}{1.880535in}}{\pgfqpoint{6.021933in}{1.877701in}}{\pgfqpoint{6.016890in}{1.872657in}}%
\pgfpathcurveto{\pgfqpoint{6.011846in}{1.867613in}}{\pgfqpoint{6.009012in}{1.860772in}}{\pgfqpoint{6.009012in}{1.853639in}}%
\pgfpathcurveto{\pgfqpoint{6.009012in}{1.846506in}}{\pgfqpoint{6.011846in}{1.839664in}}{\pgfqpoint{6.016890in}{1.834621in}}%
\pgfpathcurveto{\pgfqpoint{6.021933in}{1.829577in}}{\pgfqpoint{6.028775in}{1.826743in}}{\pgfqpoint{6.035908in}{1.826743in}}%
\pgfpathclose%
\pgfusepath{stroke,fill}%
\end{pgfscope}%
\begin{pgfscope}%
\pgfpathrectangle{\pgfqpoint{4.985294in}{0.500000in}}{\pgfqpoint{1.764706in}{1.700000in}}%
\pgfusepath{clip}%
\pgfsetbuttcap%
\pgfsetroundjoin%
\definecolor{currentfill}{rgb}{0.975018,0.868213,0.788710}%
\pgfsetfillcolor{currentfill}%
\pgfsetlinewidth{0.311001pt}%
\definecolor{currentstroke}{rgb}{1.000000,1.000000,1.000000}%
\pgfsetstrokecolor{currentstroke}%
\pgfsetdash{}{0pt}%
\pgfpathmoveto{\pgfqpoint{5.483654in}{1.133032in}}%
\pgfpathcurveto{\pgfqpoint{5.490787in}{1.133032in}}{\pgfqpoint{5.497629in}{1.135866in}}{\pgfqpoint{5.502673in}{1.140910in}}%
\pgfpathcurveto{\pgfqpoint{5.507716in}{1.145954in}}{\pgfqpoint{5.510550in}{1.152795in}}{\pgfqpoint{5.510550in}{1.159928in}}%
\pgfpathcurveto{\pgfqpoint{5.510550in}{1.167061in}}{\pgfqpoint{5.507716in}{1.173903in}}{\pgfqpoint{5.502673in}{1.178946in}}%
\pgfpathcurveto{\pgfqpoint{5.497629in}{1.183990in}}{\pgfqpoint{5.490787in}{1.186824in}}{\pgfqpoint{5.483654in}{1.186824in}}%
\pgfpathcurveto{\pgfqpoint{5.476522in}{1.186824in}}{\pgfqpoint{5.469680in}{1.183990in}}{\pgfqpoint{5.464636in}{1.178946in}}%
\pgfpathcurveto{\pgfqpoint{5.459593in}{1.173903in}}{\pgfqpoint{5.456759in}{1.167061in}}{\pgfqpoint{5.456759in}{1.159928in}}%
\pgfpathcurveto{\pgfqpoint{5.456759in}{1.152795in}}{\pgfqpoint{5.459593in}{1.145954in}}{\pgfqpoint{5.464636in}{1.140910in}}%
\pgfpathcurveto{\pgfqpoint{5.469680in}{1.135866in}}{\pgfqpoint{5.476522in}{1.133032in}}{\pgfqpoint{5.483654in}{1.133032in}}%
\pgfpathclose%
\pgfusepath{stroke,fill}%
\end{pgfscope}%
\begin{pgfscope}%
\pgfpathrectangle{\pgfqpoint{4.985294in}{0.500000in}}{\pgfqpoint{1.764706in}{1.700000in}}%
\pgfusepath{clip}%
\pgfsetbuttcap%
\pgfsetroundjoin%
\definecolor{currentfill}{rgb}{0.962765,0.606121,0.444717}%
\pgfsetfillcolor{currentfill}%
\pgfsetlinewidth{0.311001pt}%
\definecolor{currentstroke}{rgb}{1.000000,1.000000,1.000000}%
\pgfsetstrokecolor{currentstroke}%
\pgfsetdash{}{0pt}%
\pgfpathmoveto{\pgfqpoint{5.302943in}{1.297412in}}%
\pgfpathcurveto{\pgfqpoint{5.310076in}{1.297412in}}{\pgfqpoint{5.316918in}{1.300245in}}{\pgfqpoint{5.321961in}{1.305289in}}%
\pgfpathcurveto{\pgfqpoint{5.327005in}{1.310333in}}{\pgfqpoint{5.329839in}{1.317174in}}{\pgfqpoint{5.329839in}{1.324307in}}%
\pgfpathcurveto{\pgfqpoint{5.329839in}{1.331440in}}{\pgfqpoint{5.327005in}{1.338282in}}{\pgfqpoint{5.321961in}{1.343325in}}%
\pgfpathcurveto{\pgfqpoint{5.316918in}{1.348369in}}{\pgfqpoint{5.310076in}{1.351203in}}{\pgfqpoint{5.302943in}{1.351203in}}%
\pgfpathcurveto{\pgfqpoint{5.295810in}{1.351203in}}{\pgfqpoint{5.288969in}{1.348369in}}{\pgfqpoint{5.283925in}{1.343325in}}%
\pgfpathcurveto{\pgfqpoint{5.278881in}{1.338282in}}{\pgfqpoint{5.276048in}{1.331440in}}{\pgfqpoint{5.276048in}{1.324307in}}%
\pgfpathcurveto{\pgfqpoint{5.276048in}{1.317174in}}{\pgfqpoint{5.278881in}{1.310333in}}{\pgfqpoint{5.283925in}{1.305289in}}%
\pgfpathcurveto{\pgfqpoint{5.288969in}{1.300245in}}{\pgfqpoint{5.295810in}{1.297412in}}{\pgfqpoint{5.302943in}{1.297412in}}%
\pgfpathclose%
\pgfusepath{stroke,fill}%
\end{pgfscope}%
\begin{pgfscope}%
\pgfpathrectangle{\pgfqpoint{4.985294in}{0.500000in}}{\pgfqpoint{1.764706in}{1.700000in}}%
\pgfusepath{clip}%
\pgfsetbuttcap%
\pgfsetroundjoin%
\definecolor{currentfill}{rgb}{0.969359,0.803954,0.693832}%
\pgfsetfillcolor{currentfill}%
\pgfsetlinewidth{0.311001pt}%
\definecolor{currentstroke}{rgb}{1.000000,1.000000,1.000000}%
\pgfsetstrokecolor{currentstroke}%
\pgfsetdash{}{0pt}%
\pgfpathmoveto{\pgfqpoint{6.176222in}{1.667958in}}%
\pgfpathcurveto{\pgfqpoint{6.183355in}{1.667958in}}{\pgfqpoint{6.190196in}{1.670792in}}{\pgfqpoint{6.195240in}{1.675835in}}%
\pgfpathcurveto{\pgfqpoint{6.200284in}{1.680879in}}{\pgfqpoint{6.203118in}{1.687721in}}{\pgfqpoint{6.203118in}{1.694853in}}%
\pgfpathcurveto{\pgfqpoint{6.203118in}{1.701986in}}{\pgfqpoint{6.200284in}{1.708828in}}{\pgfqpoint{6.195240in}{1.713871in}}%
\pgfpathcurveto{\pgfqpoint{6.190196in}{1.718915in}}{\pgfqpoint{6.183355in}{1.721749in}}{\pgfqpoint{6.176222in}{1.721749in}}%
\pgfpathcurveto{\pgfqpoint{6.169089in}{1.721749in}}{\pgfqpoint{6.162247in}{1.718915in}}{\pgfqpoint{6.157204in}{1.713871in}}%
\pgfpathcurveto{\pgfqpoint{6.152160in}{1.708828in}}{\pgfqpoint{6.149326in}{1.701986in}}{\pgfqpoint{6.149326in}{1.694853in}}%
\pgfpathcurveto{\pgfqpoint{6.149326in}{1.687721in}}{\pgfqpoint{6.152160in}{1.680879in}}{\pgfqpoint{6.157204in}{1.675835in}}%
\pgfpathcurveto{\pgfqpoint{6.162247in}{1.670792in}}{\pgfqpoint{6.169089in}{1.667958in}}{\pgfqpoint{6.176222in}{1.667958in}}%
\pgfpathclose%
\pgfusepath{stroke,fill}%
\end{pgfscope}%
\begin{pgfscope}%
\pgfpathrectangle{\pgfqpoint{4.985294in}{0.500000in}}{\pgfqpoint{1.764706in}{1.700000in}}%
\pgfusepath{clip}%
\pgfsetbuttcap%
\pgfsetroundjoin%
\definecolor{currentfill}{rgb}{0.965753,0.732351,0.592427}%
\pgfsetfillcolor{currentfill}%
\pgfsetlinewidth{0.311001pt}%
\definecolor{currentstroke}{rgb}{1.000000,1.000000,1.000000}%
\pgfsetstrokecolor{currentstroke}%
\pgfsetdash{}{0pt}%
\pgfpathmoveto{\pgfqpoint{6.289459in}{0.989762in}}%
\pgfpathcurveto{\pgfqpoint{6.296591in}{0.989762in}}{\pgfqpoint{6.303433in}{0.992596in}}{\pgfqpoint{6.308477in}{0.997640in}}%
\pgfpathcurveto{\pgfqpoint{6.313520in}{1.002683in}}{\pgfqpoint{6.316354in}{1.009525in}}{\pgfqpoint{6.316354in}{1.016658in}}%
\pgfpathcurveto{\pgfqpoint{6.316354in}{1.023791in}}{\pgfqpoint{6.313520in}{1.030632in}}{\pgfqpoint{6.308477in}{1.035676in}}%
\pgfpathcurveto{\pgfqpoint{6.303433in}{1.040720in}}{\pgfqpoint{6.296591in}{1.043554in}}{\pgfqpoint{6.289459in}{1.043554in}}%
\pgfpathcurveto{\pgfqpoint{6.282326in}{1.043554in}}{\pgfqpoint{6.275484in}{1.040720in}}{\pgfqpoint{6.270440in}{1.035676in}}%
\pgfpathcurveto{\pgfqpoint{6.265397in}{1.030632in}}{\pgfqpoint{6.262563in}{1.023791in}}{\pgfqpoint{6.262563in}{1.016658in}}%
\pgfpathcurveto{\pgfqpoint{6.262563in}{1.009525in}}{\pgfqpoint{6.265397in}{1.002683in}}{\pgfqpoint{6.270440in}{0.997640in}}%
\pgfpathcurveto{\pgfqpoint{6.275484in}{0.992596in}}{\pgfqpoint{6.282326in}{0.989762in}}{\pgfqpoint{6.289459in}{0.989762in}}%
\pgfpathclose%
\pgfusepath{stroke,fill}%
\end{pgfscope}%
\begin{pgfscope}%
\pgfpathrectangle{\pgfqpoint{4.985294in}{0.500000in}}{\pgfqpoint{1.764706in}{1.700000in}}%
\pgfusepath{clip}%
\pgfsetbuttcap%
\pgfsetroundjoin%
\definecolor{currentfill}{rgb}{0.963559,0.632016,0.472047}%
\pgfsetfillcolor{currentfill}%
\pgfsetlinewidth{0.311001pt}%
\definecolor{currentstroke}{rgb}{1.000000,1.000000,1.000000}%
\pgfsetstrokecolor{currentstroke}%
\pgfsetdash{}{0pt}%
\pgfpathmoveto{\pgfqpoint{6.275198in}{0.937436in}}%
\pgfpathcurveto{\pgfqpoint{6.282331in}{0.937436in}}{\pgfqpoint{6.289173in}{0.940270in}}{\pgfqpoint{6.294216in}{0.945314in}}%
\pgfpathcurveto{\pgfqpoint{6.299260in}{0.950357in}}{\pgfqpoint{6.302094in}{0.957199in}}{\pgfqpoint{6.302094in}{0.964332in}}%
\pgfpathcurveto{\pgfqpoint{6.302094in}{0.971465in}}{\pgfqpoint{6.299260in}{0.978306in}}{\pgfqpoint{6.294216in}{0.983350in}}%
\pgfpathcurveto{\pgfqpoint{6.289173in}{0.988394in}}{\pgfqpoint{6.282331in}{0.991228in}}{\pgfqpoint{6.275198in}{0.991228in}}%
\pgfpathcurveto{\pgfqpoint{6.268065in}{0.991228in}}{\pgfqpoint{6.261224in}{0.988394in}}{\pgfqpoint{6.256180in}{0.983350in}}%
\pgfpathcurveto{\pgfqpoint{6.251136in}{0.978306in}}{\pgfqpoint{6.248302in}{0.971465in}}{\pgfqpoint{6.248302in}{0.964332in}}%
\pgfpathcurveto{\pgfqpoint{6.248302in}{0.957199in}}{\pgfqpoint{6.251136in}{0.950357in}}{\pgfqpoint{6.256180in}{0.945314in}}%
\pgfpathcurveto{\pgfqpoint{6.261224in}{0.940270in}}{\pgfqpoint{6.268065in}{0.937436in}}{\pgfqpoint{6.275198in}{0.937436in}}%
\pgfpathclose%
\pgfusepath{stroke,fill}%
\end{pgfscope}%
\begin{pgfscope}%
\pgfpathrectangle{\pgfqpoint{4.985294in}{0.500000in}}{\pgfqpoint{1.764706in}{1.700000in}}%
\pgfusepath{clip}%
\pgfsetbuttcap%
\pgfsetroundjoin%
\definecolor{currentfill}{rgb}{0.980678,0.914765,0.856766}%
\pgfsetfillcolor{currentfill}%
\pgfsetlinewidth{0.311001pt}%
\definecolor{currentstroke}{rgb}{1.000000,1.000000,1.000000}%
\pgfsetstrokecolor{currentstroke}%
\pgfsetdash{}{0pt}%
\pgfpathmoveto{\pgfqpoint{6.334984in}{1.303658in}}%
\pgfpathcurveto{\pgfqpoint{6.342117in}{1.303658in}}{\pgfqpoint{6.348958in}{1.306492in}}{\pgfqpoint{6.354002in}{1.311536in}}%
\pgfpathcurveto{\pgfqpoint{6.359046in}{1.316580in}}{\pgfqpoint{6.361880in}{1.323421in}}{\pgfqpoint{6.361880in}{1.330554in}}%
\pgfpathcurveto{\pgfqpoint{6.361880in}{1.337687in}}{\pgfqpoint{6.359046in}{1.344529in}}{\pgfqpoint{6.354002in}{1.349572in}}%
\pgfpathcurveto{\pgfqpoint{6.348958in}{1.354616in}}{\pgfqpoint{6.342117in}{1.357450in}}{\pgfqpoint{6.334984in}{1.357450in}}%
\pgfpathcurveto{\pgfqpoint{6.327851in}{1.357450in}}{\pgfqpoint{6.321009in}{1.354616in}}{\pgfqpoint{6.315966in}{1.349572in}}%
\pgfpathcurveto{\pgfqpoint{6.310922in}{1.344529in}}{\pgfqpoint{6.308088in}{1.337687in}}{\pgfqpoint{6.308088in}{1.330554in}}%
\pgfpathcurveto{\pgfqpoint{6.308088in}{1.323421in}}{\pgfqpoint{6.310922in}{1.316580in}}{\pgfqpoint{6.315966in}{1.311536in}}%
\pgfpathcurveto{\pgfqpoint{6.321009in}{1.306492in}}{\pgfqpoint{6.327851in}{1.303658in}}{\pgfqpoint{6.334984in}{1.303658in}}%
\pgfpathclose%
\pgfusepath{stroke,fill}%
\end{pgfscope}%
\begin{pgfscope}%
\pgfpathrectangle{\pgfqpoint{4.985294in}{0.500000in}}{\pgfqpoint{1.764706in}{1.700000in}}%
\pgfusepath{clip}%
\pgfsetbuttcap%
\pgfsetroundjoin%
\definecolor{currentfill}{rgb}{0.977657,0.891500,0.822809}%
\pgfsetfillcolor{currentfill}%
\pgfsetlinewidth{0.311001pt}%
\definecolor{currentstroke}{rgb}{1.000000,1.000000,1.000000}%
\pgfsetstrokecolor{currentstroke}%
\pgfsetdash{}{0pt}%
\pgfpathmoveto{\pgfqpoint{5.453518in}{1.172398in}}%
\pgfpathcurveto{\pgfqpoint{5.460651in}{1.172398in}}{\pgfqpoint{5.467492in}{1.175231in}}{\pgfqpoint{5.472536in}{1.180275in}}%
\pgfpathcurveto{\pgfqpoint{5.477580in}{1.185319in}}{\pgfqpoint{5.480414in}{1.192160in}}{\pgfqpoint{5.480414in}{1.199293in}}%
\pgfpathcurveto{\pgfqpoint{5.480414in}{1.206426in}}{\pgfqpoint{5.477580in}{1.213268in}}{\pgfqpoint{5.472536in}{1.218311in}}%
\pgfpathcurveto{\pgfqpoint{5.467492in}{1.223355in}}{\pgfqpoint{5.460651in}{1.226189in}}{\pgfqpoint{5.453518in}{1.226189in}}%
\pgfpathcurveto{\pgfqpoint{5.446385in}{1.226189in}}{\pgfqpoint{5.439543in}{1.223355in}}{\pgfqpoint{5.434500in}{1.218311in}}%
\pgfpathcurveto{\pgfqpoint{5.429456in}{1.213268in}}{\pgfqpoint{5.426622in}{1.206426in}}{\pgfqpoint{5.426622in}{1.199293in}}%
\pgfpathcurveto{\pgfqpoint{5.426622in}{1.192160in}}{\pgfqpoint{5.429456in}{1.185319in}}{\pgfqpoint{5.434500in}{1.180275in}}%
\pgfpathcurveto{\pgfqpoint{5.439543in}{1.175231in}}{\pgfqpoint{5.446385in}{1.172398in}}{\pgfqpoint{5.453518in}{1.172398in}}%
\pgfpathclose%
\pgfusepath{stroke,fill}%
\end{pgfscope}%
\begin{pgfscope}%
\pgfpathrectangle{\pgfqpoint{4.985294in}{0.500000in}}{\pgfqpoint{1.764706in}{1.700000in}}%
\pgfusepath{clip}%
\pgfsetbuttcap%
\pgfsetroundjoin%
\definecolor{currentfill}{rgb}{0.964799,0.689101,0.537560}%
\pgfsetfillcolor{currentfill}%
\pgfsetlinewidth{0.311001pt}%
\definecolor{currentstroke}{rgb}{1.000000,1.000000,1.000000}%
\pgfsetstrokecolor{currentstroke}%
\pgfsetdash{}{0pt}%
\pgfpathmoveto{\pgfqpoint{6.283417in}{0.965158in}}%
\pgfpathcurveto{\pgfqpoint{6.290550in}{0.965158in}}{\pgfqpoint{6.297392in}{0.967992in}}{\pgfqpoint{6.302435in}{0.973036in}}%
\pgfpathcurveto{\pgfqpoint{6.307479in}{0.978079in}}{\pgfqpoint{6.310313in}{0.984921in}}{\pgfqpoint{6.310313in}{0.992054in}}%
\pgfpathcurveto{\pgfqpoint{6.310313in}{0.999187in}}{\pgfqpoint{6.307479in}{1.006028in}}{\pgfqpoint{6.302435in}{1.011072in}}%
\pgfpathcurveto{\pgfqpoint{6.297392in}{1.016116in}}{\pgfqpoint{6.290550in}{1.018950in}}{\pgfqpoint{6.283417in}{1.018950in}}%
\pgfpathcurveto{\pgfqpoint{6.276284in}{1.018950in}}{\pgfqpoint{6.269443in}{1.016116in}}{\pgfqpoint{6.264399in}{1.011072in}}%
\pgfpathcurveto{\pgfqpoint{6.259355in}{1.006028in}}{\pgfqpoint{6.256521in}{0.999187in}}{\pgfqpoint{6.256521in}{0.992054in}}%
\pgfpathcurveto{\pgfqpoint{6.256521in}{0.984921in}}{\pgfqpoint{6.259355in}{0.978079in}}{\pgfqpoint{6.264399in}{0.973036in}}%
\pgfpathcurveto{\pgfqpoint{6.269443in}{0.967992in}}{\pgfqpoint{6.276284in}{0.965158in}}{\pgfqpoint{6.283417in}{0.965158in}}%
\pgfpathclose%
\pgfusepath{stroke,fill}%
\end{pgfscope}%
\begin{pgfscope}%
\pgfpathrectangle{\pgfqpoint{4.985294in}{0.500000in}}{\pgfqpoint{1.764706in}{1.700000in}}%
\pgfusepath{clip}%
\pgfsetbuttcap%
\pgfsetroundjoin%
\definecolor{currentfill}{rgb}{0.976287,0.879862,0.805788}%
\pgfsetfillcolor{currentfill}%
\pgfsetlinewidth{0.311001pt}%
\definecolor{currentstroke}{rgb}{1.000000,1.000000,1.000000}%
\pgfsetstrokecolor{currentstroke}%
\pgfsetdash{}{0pt}%
\pgfpathmoveto{\pgfqpoint{6.256296in}{1.537330in}}%
\pgfpathcurveto{\pgfqpoint{6.263429in}{1.537330in}}{\pgfqpoint{6.270271in}{1.540164in}}{\pgfqpoint{6.275314in}{1.545208in}}%
\pgfpathcurveto{\pgfqpoint{6.280358in}{1.550252in}}{\pgfqpoint{6.283192in}{1.557093in}}{\pgfqpoint{6.283192in}{1.564226in}}%
\pgfpathcurveto{\pgfqpoint{6.283192in}{1.571359in}}{\pgfqpoint{6.280358in}{1.578201in}}{\pgfqpoint{6.275314in}{1.583244in}}%
\pgfpathcurveto{\pgfqpoint{6.270271in}{1.588288in}}{\pgfqpoint{6.263429in}{1.591122in}}{\pgfqpoint{6.256296in}{1.591122in}}%
\pgfpathcurveto{\pgfqpoint{6.249163in}{1.591122in}}{\pgfqpoint{6.242322in}{1.588288in}}{\pgfqpoint{6.237278in}{1.583244in}}%
\pgfpathcurveto{\pgfqpoint{6.232234in}{1.578201in}}{\pgfqpoint{6.229400in}{1.571359in}}{\pgfqpoint{6.229400in}{1.564226in}}%
\pgfpathcurveto{\pgfqpoint{6.229400in}{1.557093in}}{\pgfqpoint{6.232234in}{1.550252in}}{\pgfqpoint{6.237278in}{1.545208in}}%
\pgfpathcurveto{\pgfqpoint{6.242322in}{1.540164in}}{\pgfqpoint{6.249163in}{1.537330in}}{\pgfqpoint{6.256296in}{1.537330in}}%
\pgfpathclose%
\pgfusepath{stroke,fill}%
\end{pgfscope}%
\begin{pgfscope}%
\pgfpathrectangle{\pgfqpoint{4.985294in}{0.500000in}}{\pgfqpoint{1.764706in}{1.700000in}}%
\pgfusepath{clip}%
\pgfsetbuttcap%
\pgfsetroundjoin%
\definecolor{currentfill}{rgb}{0.963379,0.625574,0.465113}%
\pgfsetfillcolor{currentfill}%
\pgfsetlinewidth{0.311001pt}%
\definecolor{currentstroke}{rgb}{1.000000,1.000000,1.000000}%
\pgfsetstrokecolor{currentstroke}%
\pgfsetdash{}{0pt}%
\pgfpathmoveto{\pgfqpoint{5.415994in}{0.948516in}}%
\pgfpathcurveto{\pgfqpoint{5.423127in}{0.948516in}}{\pgfqpoint{5.429969in}{0.951349in}}{\pgfqpoint{5.435012in}{0.956393in}}%
\pgfpathcurveto{\pgfqpoint{5.440056in}{0.961437in}}{\pgfqpoint{5.442890in}{0.968278in}}{\pgfqpoint{5.442890in}{0.975411in}}%
\pgfpathcurveto{\pgfqpoint{5.442890in}{0.982544in}}{\pgfqpoint{5.440056in}{0.989386in}}{\pgfqpoint{5.435012in}{0.994429in}}%
\pgfpathcurveto{\pgfqpoint{5.429969in}{0.999473in}}{\pgfqpoint{5.423127in}{1.002307in}}{\pgfqpoint{5.415994in}{1.002307in}}%
\pgfpathcurveto{\pgfqpoint{5.408861in}{1.002307in}}{\pgfqpoint{5.402020in}{0.999473in}}{\pgfqpoint{5.396976in}{0.994429in}}%
\pgfpathcurveto{\pgfqpoint{5.391932in}{0.989386in}}{\pgfqpoint{5.389098in}{0.982544in}}{\pgfqpoint{5.389098in}{0.975411in}}%
\pgfpathcurveto{\pgfqpoint{5.389098in}{0.968278in}}{\pgfqpoint{5.391932in}{0.961437in}}{\pgfqpoint{5.396976in}{0.956393in}}%
\pgfpathcurveto{\pgfqpoint{5.402020in}{0.951349in}}{\pgfqpoint{5.408861in}{0.948516in}}{\pgfqpoint{5.415994in}{0.948516in}}%
\pgfpathclose%
\pgfusepath{stroke,fill}%
\end{pgfscope}%
\begin{pgfscope}%
\pgfpathrectangle{\pgfqpoint{4.985294in}{0.500000in}}{\pgfqpoint{1.764706in}{1.700000in}}%
\pgfusepath{clip}%
\pgfsetbuttcap%
\pgfsetroundjoin%
\definecolor{currentfill}{rgb}{0.974412,0.862387,0.780156}%
\pgfsetfillcolor{currentfill}%
\pgfsetlinewidth{0.311001pt}%
\definecolor{currentstroke}{rgb}{1.000000,1.000000,1.000000}%
\pgfsetstrokecolor{currentstroke}%
\pgfsetdash{}{0pt}%
\pgfpathmoveto{\pgfqpoint{6.288242in}{1.601828in}}%
\pgfpathcurveto{\pgfqpoint{6.295375in}{1.601828in}}{\pgfqpoint{6.302217in}{1.604662in}}{\pgfqpoint{6.307260in}{1.609706in}}%
\pgfpathcurveto{\pgfqpoint{6.312304in}{1.614749in}}{\pgfqpoint{6.315138in}{1.621591in}}{\pgfqpoint{6.315138in}{1.628724in}}%
\pgfpathcurveto{\pgfqpoint{6.315138in}{1.635857in}}{\pgfqpoint{6.312304in}{1.642698in}}{\pgfqpoint{6.307260in}{1.647742in}}%
\pgfpathcurveto{\pgfqpoint{6.302217in}{1.652786in}}{\pgfqpoint{6.295375in}{1.655620in}}{\pgfqpoint{6.288242in}{1.655620in}}%
\pgfpathcurveto{\pgfqpoint{6.281109in}{1.655620in}}{\pgfqpoint{6.274268in}{1.652786in}}{\pgfqpoint{6.269224in}{1.647742in}}%
\pgfpathcurveto{\pgfqpoint{6.264180in}{1.642698in}}{\pgfqpoint{6.261347in}{1.635857in}}{\pgfqpoint{6.261347in}{1.628724in}}%
\pgfpathcurveto{\pgfqpoint{6.261347in}{1.621591in}}{\pgfqpoint{6.264180in}{1.614749in}}{\pgfqpoint{6.269224in}{1.609706in}}%
\pgfpathcurveto{\pgfqpoint{6.274268in}{1.604662in}}{\pgfqpoint{6.281109in}{1.601828in}}{\pgfqpoint{6.288242in}{1.601828in}}%
\pgfpathclose%
\pgfusepath{stroke,fill}%
\end{pgfscope}%
\begin{pgfscope}%
\pgfpathrectangle{\pgfqpoint{4.985294in}{0.500000in}}{\pgfqpoint{1.764706in}{1.700000in}}%
\pgfusepath{clip}%
\pgfsetbuttcap%
\pgfsetroundjoin%
\definecolor{currentfill}{rgb}{0.968105,0.786346,0.667739}%
\pgfsetfillcolor{currentfill}%
\pgfsetlinewidth{0.311001pt}%
\definecolor{currentstroke}{rgb}{1.000000,1.000000,1.000000}%
\pgfsetstrokecolor{currentstroke}%
\pgfsetdash{}{0pt}%
\pgfpathmoveto{\pgfqpoint{5.343579in}{1.349226in}}%
\pgfpathcurveto{\pgfqpoint{5.350712in}{1.349226in}}{\pgfqpoint{5.357554in}{1.352060in}}{\pgfqpoint{5.362598in}{1.357104in}}%
\pgfpathcurveto{\pgfqpoint{5.367641in}{1.362147in}}{\pgfqpoint{5.370475in}{1.368989in}}{\pgfqpoint{5.370475in}{1.376122in}}%
\pgfpathcurveto{\pgfqpoint{5.370475in}{1.383254in}}{\pgfqpoint{5.367641in}{1.390096in}}{\pgfqpoint{5.362598in}{1.395140in}}%
\pgfpathcurveto{\pgfqpoint{5.357554in}{1.400183in}}{\pgfqpoint{5.350712in}{1.403017in}}{\pgfqpoint{5.343579in}{1.403017in}}%
\pgfpathcurveto{\pgfqpoint{5.336447in}{1.403017in}}{\pgfqpoint{5.329605in}{1.400183in}}{\pgfqpoint{5.324561in}{1.395140in}}%
\pgfpathcurveto{\pgfqpoint{5.319518in}{1.390096in}}{\pgfqpoint{5.316684in}{1.383254in}}{\pgfqpoint{5.316684in}{1.376122in}}%
\pgfpathcurveto{\pgfqpoint{5.316684in}{1.368989in}}{\pgfqpoint{5.319518in}{1.362147in}}{\pgfqpoint{5.324561in}{1.357104in}}%
\pgfpathcurveto{\pgfqpoint{5.329605in}{1.352060in}}{\pgfqpoint{5.336447in}{1.349226in}}{\pgfqpoint{5.343579in}{1.349226in}}%
\pgfpathclose%
\pgfusepath{stroke,fill}%
\end{pgfscope}%
\begin{pgfscope}%
\pgfpathrectangle{\pgfqpoint{4.985294in}{0.500000in}}{\pgfqpoint{1.764706in}{1.700000in}}%
\pgfusepath{clip}%
\pgfsetbuttcap%
\pgfsetroundjoin%
\definecolor{currentfill}{rgb}{0.970718,0.821518,0.719872}%
\pgfsetfillcolor{currentfill}%
\pgfsetlinewidth{0.311001pt}%
\definecolor{currentstroke}{rgb}{1.000000,1.000000,1.000000}%
\pgfsetstrokecolor{currentstroke}%
\pgfsetdash{}{0pt}%
\pgfpathmoveto{\pgfqpoint{6.199190in}{1.117297in}}%
\pgfpathcurveto{\pgfqpoint{6.206323in}{1.117297in}}{\pgfqpoint{6.213164in}{1.120131in}}{\pgfqpoint{6.218208in}{1.125174in}}%
\pgfpathcurveto{\pgfqpoint{6.223252in}{1.130218in}}{\pgfqpoint{6.226085in}{1.137060in}}{\pgfqpoint{6.226085in}{1.144193in}}%
\pgfpathcurveto{\pgfqpoint{6.226085in}{1.151325in}}{\pgfqpoint{6.223252in}{1.158167in}}{\pgfqpoint{6.218208in}{1.163211in}}%
\pgfpathcurveto{\pgfqpoint{6.213164in}{1.168254in}}{\pgfqpoint{6.206323in}{1.171088in}}{\pgfqpoint{6.199190in}{1.171088in}}%
\pgfpathcurveto{\pgfqpoint{6.192057in}{1.171088in}}{\pgfqpoint{6.185215in}{1.168254in}}{\pgfqpoint{6.180172in}{1.163211in}}%
\pgfpathcurveto{\pgfqpoint{6.175128in}{1.158167in}}{\pgfqpoint{6.172294in}{1.151325in}}{\pgfqpoint{6.172294in}{1.144193in}}%
\pgfpathcurveto{\pgfqpoint{6.172294in}{1.137060in}}{\pgfqpoint{6.175128in}{1.130218in}}{\pgfqpoint{6.180172in}{1.125174in}}%
\pgfpathcurveto{\pgfqpoint{6.185215in}{1.120131in}}{\pgfqpoint{6.192057in}{1.117297in}}{\pgfqpoint{6.199190in}{1.117297in}}%
\pgfpathclose%
\pgfusepath{stroke,fill}%
\end{pgfscope}%
\begin{pgfscope}%
\pgfpathrectangle{\pgfqpoint{4.985294in}{0.500000in}}{\pgfqpoint{1.764706in}{1.700000in}}%
\pgfusepath{clip}%
\pgfsetbuttcap%
\pgfsetroundjoin%
\definecolor{currentfill}{rgb}{0.948235,0.413004,0.283323}%
\pgfsetfillcolor{currentfill}%
\pgfsetlinewidth{0.311001pt}%
\definecolor{currentstroke}{rgb}{1.000000,1.000000,1.000000}%
\pgfsetstrokecolor{currentstroke}%
\pgfsetdash{}{0pt}%
\pgfpathmoveto{\pgfqpoint{6.138072in}{1.424527in}}%
\pgfpathcurveto{\pgfqpoint{6.145205in}{1.424527in}}{\pgfqpoint{6.152047in}{1.427361in}}{\pgfqpoint{6.157091in}{1.432405in}}%
\pgfpathcurveto{\pgfqpoint{6.162134in}{1.437449in}}{\pgfqpoint{6.164968in}{1.444290in}}{\pgfqpoint{6.164968in}{1.451423in}}%
\pgfpathcurveto{\pgfqpoint{6.164968in}{1.458556in}}{\pgfqpoint{6.162134in}{1.465397in}}{\pgfqpoint{6.157091in}{1.470441in}}%
\pgfpathcurveto{\pgfqpoint{6.152047in}{1.475485in}}{\pgfqpoint{6.145205in}{1.478319in}}{\pgfqpoint{6.138072in}{1.478319in}}%
\pgfpathcurveto{\pgfqpoint{6.130940in}{1.478319in}}{\pgfqpoint{6.124098in}{1.475485in}}{\pgfqpoint{6.119054in}{1.470441in}}%
\pgfpathcurveto{\pgfqpoint{6.114011in}{1.465397in}}{\pgfqpoint{6.111177in}{1.458556in}}{\pgfqpoint{6.111177in}{1.451423in}}%
\pgfpathcurveto{\pgfqpoint{6.111177in}{1.444290in}}{\pgfqpoint{6.114011in}{1.437449in}}{\pgfqpoint{6.119054in}{1.432405in}}%
\pgfpathcurveto{\pgfqpoint{6.124098in}{1.427361in}}{\pgfqpoint{6.130940in}{1.424527in}}{\pgfqpoint{6.138072in}{1.424527in}}%
\pgfpathclose%
\pgfusepath{stroke,fill}%
\end{pgfscope}%
\begin{pgfscope}%
\pgfpathrectangle{\pgfqpoint{4.985294in}{0.500000in}}{\pgfqpoint{1.764706in}{1.700000in}}%
\pgfusepath{clip}%
\pgfsetbuttcap%
\pgfsetroundjoin%
\definecolor{currentfill}{rgb}{0.966328,0.750560,0.616961}%
\pgfsetfillcolor{currentfill}%
\pgfsetlinewidth{0.311001pt}%
\definecolor{currentstroke}{rgb}{1.000000,1.000000,1.000000}%
\pgfsetstrokecolor{currentstroke}%
\pgfsetdash{}{0pt}%
\pgfpathmoveto{\pgfqpoint{6.148712in}{0.998086in}}%
\pgfpathcurveto{\pgfqpoint{6.155845in}{0.998086in}}{\pgfqpoint{6.162687in}{1.000920in}}{\pgfqpoint{6.167731in}{1.005964in}}%
\pgfpathcurveto{\pgfqpoint{6.172774in}{1.011008in}}{\pgfqpoint{6.175608in}{1.017849in}}{\pgfqpoint{6.175608in}{1.024982in}}%
\pgfpathcurveto{\pgfqpoint{6.175608in}{1.032115in}}{\pgfqpoint{6.172774in}{1.038956in}}{\pgfqpoint{6.167731in}{1.044000in}}%
\pgfpathcurveto{\pgfqpoint{6.162687in}{1.049044in}}{\pgfqpoint{6.155845in}{1.051878in}}{\pgfqpoint{6.148712in}{1.051878in}}%
\pgfpathcurveto{\pgfqpoint{6.141580in}{1.051878in}}{\pgfqpoint{6.134738in}{1.049044in}}{\pgfqpoint{6.129694in}{1.044000in}}%
\pgfpathcurveto{\pgfqpoint{6.124651in}{1.038956in}}{\pgfqpoint{6.121817in}{1.032115in}}{\pgfqpoint{6.121817in}{1.024982in}}%
\pgfpathcurveto{\pgfqpoint{6.121817in}{1.017849in}}{\pgfqpoint{6.124651in}{1.011008in}}{\pgfqpoint{6.129694in}{1.005964in}}%
\pgfpathcurveto{\pgfqpoint{6.134738in}{1.000920in}}{\pgfqpoint{6.141580in}{0.998086in}}{\pgfqpoint{6.148712in}{0.998086in}}%
\pgfpathclose%
\pgfusepath{stroke,fill}%
\end{pgfscope}%
\begin{pgfscope}%
\pgfpathrectangle{\pgfqpoint{4.985294in}{0.500000in}}{\pgfqpoint{1.764706in}{1.700000in}}%
\pgfusepath{clip}%
\pgfsetbuttcap%
\pgfsetroundjoin%
\definecolor{currentfill}{rgb}{0.976961,0.885681,0.814303}%
\pgfsetfillcolor{currentfill}%
\pgfsetlinewidth{0.311001pt}%
\definecolor{currentstroke}{rgb}{1.000000,1.000000,1.000000}%
\pgfsetstrokecolor{currentstroke}%
\pgfsetdash{}{0pt}%
\pgfpathmoveto{\pgfqpoint{5.381849in}{1.354927in}}%
\pgfpathcurveto{\pgfqpoint{5.388982in}{1.354927in}}{\pgfqpoint{5.395824in}{1.357760in}}{\pgfqpoint{5.400867in}{1.362804in}}%
\pgfpathcurveto{\pgfqpoint{5.405911in}{1.367848in}}{\pgfqpoint{5.408745in}{1.374689in}}{\pgfqpoint{5.408745in}{1.381822in}}%
\pgfpathcurveto{\pgfqpoint{5.408745in}{1.388955in}}{\pgfqpoint{5.405911in}{1.395797in}}{\pgfqpoint{5.400867in}{1.400840in}}%
\pgfpathcurveto{\pgfqpoint{5.395824in}{1.405884in}}{\pgfqpoint{5.388982in}{1.408718in}}{\pgfqpoint{5.381849in}{1.408718in}}%
\pgfpathcurveto{\pgfqpoint{5.374716in}{1.408718in}}{\pgfqpoint{5.367875in}{1.405884in}}{\pgfqpoint{5.362831in}{1.400840in}}%
\pgfpathcurveto{\pgfqpoint{5.357787in}{1.395797in}}{\pgfqpoint{5.354953in}{1.388955in}}{\pgfqpoint{5.354953in}{1.381822in}}%
\pgfpathcurveto{\pgfqpoint{5.354953in}{1.374689in}}{\pgfqpoint{5.357787in}{1.367848in}}{\pgfqpoint{5.362831in}{1.362804in}}%
\pgfpathcurveto{\pgfqpoint{5.367875in}{1.357760in}}{\pgfqpoint{5.374716in}{1.354927in}}{\pgfqpoint{5.381849in}{1.354927in}}%
\pgfpathclose%
\pgfusepath{stroke,fill}%
\end{pgfscope}%
\begin{pgfscope}%
\pgfpathrectangle{\pgfqpoint{4.985294in}{0.500000in}}{\pgfqpoint{1.764706in}{1.700000in}}%
\pgfusepath{clip}%
\pgfsetbuttcap%
\pgfsetroundjoin%
\definecolor{currentfill}{rgb}{0.911533,0.252926,0.244703}%
\pgfsetfillcolor{currentfill}%
\pgfsetlinewidth{0.311001pt}%
\definecolor{currentstroke}{rgb}{1.000000,1.000000,1.000000}%
\pgfsetstrokecolor{currentstroke}%
\pgfsetdash{}{0pt}%
\pgfpathmoveto{\pgfqpoint{5.739960in}{1.681008in}}%
\pgfpathcurveto{\pgfqpoint{5.747093in}{1.681008in}}{\pgfqpoint{5.753934in}{1.683842in}}{\pgfqpoint{5.758978in}{1.688885in}}%
\pgfpathcurveto{\pgfqpoint{5.764022in}{1.693929in}}{\pgfqpoint{5.766855in}{1.700771in}}{\pgfqpoint{5.766855in}{1.707904in}}%
\pgfpathcurveto{\pgfqpoint{5.766855in}{1.715036in}}{\pgfqpoint{5.764022in}{1.721878in}}{\pgfqpoint{5.758978in}{1.726922in}}%
\pgfpathcurveto{\pgfqpoint{5.753934in}{1.731965in}}{\pgfqpoint{5.747093in}{1.734799in}}{\pgfqpoint{5.739960in}{1.734799in}}%
\pgfpathcurveto{\pgfqpoint{5.732827in}{1.734799in}}{\pgfqpoint{5.725985in}{1.731965in}}{\pgfqpoint{5.720942in}{1.726922in}}%
\pgfpathcurveto{\pgfqpoint{5.715898in}{1.721878in}}{\pgfqpoint{5.713064in}{1.715036in}}{\pgfqpoint{5.713064in}{1.707904in}}%
\pgfpathcurveto{\pgfqpoint{5.713064in}{1.700771in}}{\pgfqpoint{5.715898in}{1.693929in}}{\pgfqpoint{5.720942in}{1.688885in}}%
\pgfpathcurveto{\pgfqpoint{5.725985in}{1.683842in}}{\pgfqpoint{5.732827in}{1.681008in}}{\pgfqpoint{5.739960in}{1.681008in}}%
\pgfpathclose%
\pgfusepath{stroke,fill}%
\end{pgfscope}%
\begin{pgfscope}%
\pgfpathrectangle{\pgfqpoint{4.985294in}{0.500000in}}{\pgfqpoint{1.764706in}{1.700000in}}%
\pgfusepath{clip}%
\pgfsetbuttcap%
\pgfsetroundjoin%
\definecolor{currentfill}{rgb}{0.978376,0.897317,0.831308}%
\pgfsetfillcolor{currentfill}%
\pgfsetlinewidth{0.311001pt}%
\definecolor{currentstroke}{rgb}{1.000000,1.000000,1.000000}%
\pgfsetstrokecolor{currentstroke}%
\pgfsetdash{}{0pt}%
\pgfpathmoveto{\pgfqpoint{5.409822in}{1.438726in}}%
\pgfpathcurveto{\pgfqpoint{5.416955in}{1.438726in}}{\pgfqpoint{5.423797in}{1.441560in}}{\pgfqpoint{5.428841in}{1.446603in}}%
\pgfpathcurveto{\pgfqpoint{5.433884in}{1.451647in}}{\pgfqpoint{5.436718in}{1.458489in}}{\pgfqpoint{5.436718in}{1.465621in}}%
\pgfpathcurveto{\pgfqpoint{5.436718in}{1.472754in}}{\pgfqpoint{5.433884in}{1.479596in}}{\pgfqpoint{5.428841in}{1.484639in}}%
\pgfpathcurveto{\pgfqpoint{5.423797in}{1.489683in}}{\pgfqpoint{5.416955in}{1.492517in}}{\pgfqpoint{5.409822in}{1.492517in}}%
\pgfpathcurveto{\pgfqpoint{5.402690in}{1.492517in}}{\pgfqpoint{5.395848in}{1.489683in}}{\pgfqpoint{5.390804in}{1.484639in}}%
\pgfpathcurveto{\pgfqpoint{5.385761in}{1.479596in}}{\pgfqpoint{5.382927in}{1.472754in}}{\pgfqpoint{5.382927in}{1.465621in}}%
\pgfpathcurveto{\pgfqpoint{5.382927in}{1.458489in}}{\pgfqpoint{5.385761in}{1.451647in}}{\pgfqpoint{5.390804in}{1.446603in}}%
\pgfpathcurveto{\pgfqpoint{5.395848in}{1.441560in}}{\pgfqpoint{5.402690in}{1.438726in}}{\pgfqpoint{5.409822in}{1.438726in}}%
\pgfpathclose%
\pgfusepath{stroke,fill}%
\end{pgfscope}%
\begin{pgfscope}%
\pgfpathrectangle{\pgfqpoint{4.985294in}{0.500000in}}{\pgfqpoint{1.764706in}{1.700000in}}%
\pgfusepath{clip}%
\pgfsetbuttcap%
\pgfsetroundjoin%
\definecolor{currentfill}{rgb}{0.975018,0.868213,0.788710}%
\pgfsetfillcolor{currentfill}%
\pgfsetlinewidth{0.311001pt}%
\definecolor{currentstroke}{rgb}{1.000000,1.000000,1.000000}%
\pgfsetstrokecolor{currentstroke}%
\pgfsetdash{}{0pt}%
\pgfpathmoveto{\pgfqpoint{6.262349in}{1.260521in}}%
\pgfpathcurveto{\pgfqpoint{6.269482in}{1.260521in}}{\pgfqpoint{6.276324in}{1.263355in}}{\pgfqpoint{6.281367in}{1.268399in}}%
\pgfpathcurveto{\pgfqpoint{6.286411in}{1.273442in}}{\pgfqpoint{6.289245in}{1.280284in}}{\pgfqpoint{6.289245in}{1.287417in}}%
\pgfpathcurveto{\pgfqpoint{6.289245in}{1.294550in}}{\pgfqpoint{6.286411in}{1.301391in}}{\pgfqpoint{6.281367in}{1.306435in}}%
\pgfpathcurveto{\pgfqpoint{6.276324in}{1.311479in}}{\pgfqpoint{6.269482in}{1.314313in}}{\pgfqpoint{6.262349in}{1.314313in}}%
\pgfpathcurveto{\pgfqpoint{6.255216in}{1.314313in}}{\pgfqpoint{6.248375in}{1.311479in}}{\pgfqpoint{6.243331in}{1.306435in}}%
\pgfpathcurveto{\pgfqpoint{6.238287in}{1.301391in}}{\pgfqpoint{6.235454in}{1.294550in}}{\pgfqpoint{6.235454in}{1.287417in}}%
\pgfpathcurveto{\pgfqpoint{6.235454in}{1.280284in}}{\pgfqpoint{6.238287in}{1.273442in}}{\pgfqpoint{6.243331in}{1.268399in}}%
\pgfpathcurveto{\pgfqpoint{6.248375in}{1.263355in}}{\pgfqpoint{6.255216in}{1.260521in}}{\pgfqpoint{6.262349in}{1.260521in}}%
\pgfpathclose%
\pgfusepath{stroke,fill}%
\end{pgfscope}%
\begin{pgfscope}%
\pgfpathrectangle{\pgfqpoint{4.985294in}{0.500000in}}{\pgfqpoint{1.764706in}{1.700000in}}%
\pgfusepath{clip}%
\pgfsetbuttcap%
\pgfsetroundjoin%
\definecolor{currentfill}{rgb}{0.973832,0.856556,0.771584}%
\pgfsetfillcolor{currentfill}%
\pgfsetlinewidth{0.311001pt}%
\definecolor{currentstroke}{rgb}{1.000000,1.000000,1.000000}%
\pgfsetstrokecolor{currentstroke}%
\pgfsetdash{}{0pt}%
\pgfpathmoveto{\pgfqpoint{5.376709in}{1.220893in}}%
\pgfpathcurveto{\pgfqpoint{5.383842in}{1.220893in}}{\pgfqpoint{5.390684in}{1.223727in}}{\pgfqpoint{5.395728in}{1.228771in}}%
\pgfpathcurveto{\pgfqpoint{5.400771in}{1.233815in}}{\pgfqpoint{5.403605in}{1.240656in}}{\pgfqpoint{5.403605in}{1.247789in}}%
\pgfpathcurveto{\pgfqpoint{5.403605in}{1.254922in}}{\pgfqpoint{5.400771in}{1.261764in}}{\pgfqpoint{5.395728in}{1.266807in}}%
\pgfpathcurveto{\pgfqpoint{5.390684in}{1.271851in}}{\pgfqpoint{5.383842in}{1.274685in}}{\pgfqpoint{5.376709in}{1.274685in}}%
\pgfpathcurveto{\pgfqpoint{5.369577in}{1.274685in}}{\pgfqpoint{5.362735in}{1.271851in}}{\pgfqpoint{5.357691in}{1.266807in}}%
\pgfpathcurveto{\pgfqpoint{5.352648in}{1.261764in}}{\pgfqpoint{5.349814in}{1.254922in}}{\pgfqpoint{5.349814in}{1.247789in}}%
\pgfpathcurveto{\pgfqpoint{5.349814in}{1.240656in}}{\pgfqpoint{5.352648in}{1.233815in}}{\pgfqpoint{5.357691in}{1.228771in}}%
\pgfpathcurveto{\pgfqpoint{5.362735in}{1.223727in}}{\pgfqpoint{5.369577in}{1.220893in}}{\pgfqpoint{5.376709in}{1.220893in}}%
\pgfpathclose%
\pgfusepath{stroke,fill}%
\end{pgfscope}%
\begin{pgfscope}%
\pgfpathrectangle{\pgfqpoint{4.985294in}{0.500000in}}{\pgfqpoint{1.764706in}{1.700000in}}%
\pgfusepath{clip}%
\pgfsetbuttcap%
\pgfsetroundjoin%
\definecolor{currentfill}{rgb}{0.979891,0.908948,0.848279}%
\pgfsetfillcolor{currentfill}%
\pgfsetlinewidth{0.311001pt}%
\definecolor{currentstroke}{rgb}{1.000000,1.000000,1.000000}%
\pgfsetstrokecolor{currentstroke}%
\pgfsetdash{}{0pt}%
\pgfpathmoveto{\pgfqpoint{5.426554in}{1.400759in}}%
\pgfpathcurveto{\pgfqpoint{5.433687in}{1.400759in}}{\pgfqpoint{5.440529in}{1.403593in}}{\pgfqpoint{5.445573in}{1.408637in}}%
\pgfpathcurveto{\pgfqpoint{5.450616in}{1.413680in}}{\pgfqpoint{5.453450in}{1.420522in}}{\pgfqpoint{5.453450in}{1.427655in}}%
\pgfpathcurveto{\pgfqpoint{5.453450in}{1.434788in}}{\pgfqpoint{5.450616in}{1.441629in}}{\pgfqpoint{5.445573in}{1.446673in}}%
\pgfpathcurveto{\pgfqpoint{5.440529in}{1.451717in}}{\pgfqpoint{5.433687in}{1.454551in}}{\pgfqpoint{5.426554in}{1.454551in}}%
\pgfpathcurveto{\pgfqpoint{5.419422in}{1.454551in}}{\pgfqpoint{5.412580in}{1.451717in}}{\pgfqpoint{5.407536in}{1.446673in}}%
\pgfpathcurveto{\pgfqpoint{5.402493in}{1.441629in}}{\pgfqpoint{5.399659in}{1.434788in}}{\pgfqpoint{5.399659in}{1.427655in}}%
\pgfpathcurveto{\pgfqpoint{5.399659in}{1.420522in}}{\pgfqpoint{5.402493in}{1.413680in}}{\pgfqpoint{5.407536in}{1.408637in}}%
\pgfpathcurveto{\pgfqpoint{5.412580in}{1.403593in}}{\pgfqpoint{5.419422in}{1.400759in}}{\pgfqpoint{5.426554in}{1.400759in}}%
\pgfpathclose%
\pgfusepath{stroke,fill}%
\end{pgfscope}%
\begin{pgfscope}%
\pgfpathrectangle{\pgfqpoint{4.985294in}{0.500000in}}{\pgfqpoint{1.764706in}{1.700000in}}%
\pgfusepath{clip}%
\pgfsetbuttcap%
\pgfsetroundjoin%
\definecolor{currentfill}{rgb}{0.979124,0.903132,0.839793}%
\pgfsetfillcolor{currentfill}%
\pgfsetlinewidth{0.311001pt}%
\definecolor{currentstroke}{rgb}{1.000000,1.000000,1.000000}%
\pgfsetstrokecolor{currentstroke}%
\pgfsetdash{}{0pt}%
\pgfpathmoveto{\pgfqpoint{6.302103in}{1.531516in}}%
\pgfpathcurveto{\pgfqpoint{6.309236in}{1.531516in}}{\pgfqpoint{6.316078in}{1.534350in}}{\pgfqpoint{6.321122in}{1.539394in}}%
\pgfpathcurveto{\pgfqpoint{6.326165in}{1.544437in}}{\pgfqpoint{6.328999in}{1.551279in}}{\pgfqpoint{6.328999in}{1.558412in}}%
\pgfpathcurveto{\pgfqpoint{6.328999in}{1.565545in}}{\pgfqpoint{6.326165in}{1.572386in}}{\pgfqpoint{6.321122in}{1.577430in}}%
\pgfpathcurveto{\pgfqpoint{6.316078in}{1.582474in}}{\pgfqpoint{6.309236in}{1.585308in}}{\pgfqpoint{6.302103in}{1.585308in}}%
\pgfpathcurveto{\pgfqpoint{6.294971in}{1.585308in}}{\pgfqpoint{6.288129in}{1.582474in}}{\pgfqpoint{6.283085in}{1.577430in}}%
\pgfpathcurveto{\pgfqpoint{6.278042in}{1.572386in}}{\pgfqpoint{6.275208in}{1.565545in}}{\pgfqpoint{6.275208in}{1.558412in}}%
\pgfpathcurveto{\pgfqpoint{6.275208in}{1.551279in}}{\pgfqpoint{6.278042in}{1.544437in}}{\pgfqpoint{6.283085in}{1.539394in}}%
\pgfpathcurveto{\pgfqpoint{6.288129in}{1.534350in}}{\pgfqpoint{6.294971in}{1.531516in}}{\pgfqpoint{6.302103in}{1.531516in}}%
\pgfpathclose%
\pgfusepath{stroke,fill}%
\end{pgfscope}%
\begin{pgfscope}%
\pgfpathrectangle{\pgfqpoint{4.985294in}{0.500000in}}{\pgfqpoint{1.764706in}{1.700000in}}%
\pgfusepath{clip}%
\pgfsetbuttcap%
\pgfsetroundjoin%
\definecolor{currentfill}{rgb}{0.966812,0.762584,0.633643}%
\pgfsetfillcolor{currentfill}%
\pgfsetlinewidth{0.311001pt}%
\definecolor{currentstroke}{rgb}{1.000000,1.000000,1.000000}%
\pgfsetstrokecolor{currentstroke}%
\pgfsetdash{}{0pt}%
\pgfpathmoveto{\pgfqpoint{6.297974in}{1.015672in}}%
\pgfpathcurveto{\pgfqpoint{6.305107in}{1.015672in}}{\pgfqpoint{6.311948in}{1.018505in}}{\pgfqpoint{6.316992in}{1.023549in}}%
\pgfpathcurveto{\pgfqpoint{6.322036in}{1.028593in}}{\pgfqpoint{6.324870in}{1.035434in}}{\pgfqpoint{6.324870in}{1.042567in}}%
\pgfpathcurveto{\pgfqpoint{6.324870in}{1.049700in}}{\pgfqpoint{6.322036in}{1.056542in}}{\pgfqpoint{6.316992in}{1.061585in}}%
\pgfpathcurveto{\pgfqpoint{6.311948in}{1.066629in}}{\pgfqpoint{6.305107in}{1.069463in}}{\pgfqpoint{6.297974in}{1.069463in}}%
\pgfpathcurveto{\pgfqpoint{6.290841in}{1.069463in}}{\pgfqpoint{6.284000in}{1.066629in}}{\pgfqpoint{6.278956in}{1.061585in}}%
\pgfpathcurveto{\pgfqpoint{6.273912in}{1.056542in}}{\pgfqpoint{6.271078in}{1.049700in}}{\pgfqpoint{6.271078in}{1.042567in}}%
\pgfpathcurveto{\pgfqpoint{6.271078in}{1.035434in}}{\pgfqpoint{6.273912in}{1.028593in}}{\pgfqpoint{6.278956in}{1.023549in}}%
\pgfpathcurveto{\pgfqpoint{6.284000in}{1.018505in}}{\pgfqpoint{6.290841in}{1.015672in}}{\pgfqpoint{6.297974in}{1.015672in}}%
\pgfpathclose%
\pgfusepath{stroke,fill}%
\end{pgfscope}%
\begin{pgfscope}%
\pgfpathrectangle{\pgfqpoint{4.985294in}{0.500000in}}{\pgfqpoint{1.764706in}{1.700000in}}%
\pgfusepath{clip}%
\pgfsetbuttcap%
\pgfsetroundjoin%
\definecolor{currentfill}{rgb}{0.970718,0.821518,0.719872}%
\pgfsetfillcolor{currentfill}%
\pgfsetlinewidth{0.311001pt}%
\definecolor{currentstroke}{rgb}{1.000000,1.000000,1.000000}%
\pgfsetstrokecolor{currentstroke}%
\pgfsetdash{}{0pt}%
\pgfpathmoveto{\pgfqpoint{5.487883in}{1.269431in}}%
\pgfpathcurveto{\pgfqpoint{5.495016in}{1.269431in}}{\pgfqpoint{5.501857in}{1.272265in}}{\pgfqpoint{5.506901in}{1.277308in}}%
\pgfpathcurveto{\pgfqpoint{5.511945in}{1.282352in}}{\pgfqpoint{5.514778in}{1.289194in}}{\pgfqpoint{5.514778in}{1.296326in}}%
\pgfpathcurveto{\pgfqpoint{5.514778in}{1.303459in}}{\pgfqpoint{5.511945in}{1.310301in}}{\pgfqpoint{5.506901in}{1.315345in}}%
\pgfpathcurveto{\pgfqpoint{5.501857in}{1.320388in}}{\pgfqpoint{5.495016in}{1.323222in}}{\pgfqpoint{5.487883in}{1.323222in}}%
\pgfpathcurveto{\pgfqpoint{5.480750in}{1.323222in}}{\pgfqpoint{5.473908in}{1.320388in}}{\pgfqpoint{5.468865in}{1.315345in}}%
\pgfpathcurveto{\pgfqpoint{5.463821in}{1.310301in}}{\pgfqpoint{5.460987in}{1.303459in}}{\pgfqpoint{5.460987in}{1.296326in}}%
\pgfpathcurveto{\pgfqpoint{5.460987in}{1.289194in}}{\pgfqpoint{5.463821in}{1.282352in}}{\pgfqpoint{5.468865in}{1.277308in}}%
\pgfpathcurveto{\pgfqpoint{5.473908in}{1.272265in}}{\pgfqpoint{5.480750in}{1.269431in}}{\pgfqpoint{5.487883in}{1.269431in}}%
\pgfpathclose%
\pgfusepath{stroke,fill}%
\end{pgfscope}%
\begin{pgfscope}%
\pgfpathrectangle{\pgfqpoint{4.985294in}{0.500000in}}{\pgfqpoint{1.764706in}{1.700000in}}%
\pgfusepath{clip}%
\pgfsetbuttcap%
\pgfsetroundjoin%
\definecolor{currentfill}{rgb}{0.976961,0.885681,0.814303}%
\pgfsetfillcolor{currentfill}%
\pgfsetlinewidth{0.311001pt}%
\definecolor{currentstroke}{rgb}{1.000000,1.000000,1.000000}%
\pgfsetstrokecolor{currentstroke}%
\pgfsetdash{}{0pt}%
\pgfpathmoveto{\pgfqpoint{6.302004in}{1.129878in}}%
\pgfpathcurveto{\pgfqpoint{6.309137in}{1.129878in}}{\pgfqpoint{6.315979in}{1.132712in}}{\pgfqpoint{6.321022in}{1.137755in}}%
\pgfpathcurveto{\pgfqpoint{6.326066in}{1.142799in}}{\pgfqpoint{6.328900in}{1.149641in}}{\pgfqpoint{6.328900in}{1.156773in}}%
\pgfpathcurveto{\pgfqpoint{6.328900in}{1.163906in}}{\pgfqpoint{6.326066in}{1.170748in}}{\pgfqpoint{6.321022in}{1.175792in}}%
\pgfpathcurveto{\pgfqpoint{6.315979in}{1.180835in}}{\pgfqpoint{6.309137in}{1.183669in}}{\pgfqpoint{6.302004in}{1.183669in}}%
\pgfpathcurveto{\pgfqpoint{6.294871in}{1.183669in}}{\pgfqpoint{6.288030in}{1.180835in}}{\pgfqpoint{6.282986in}{1.175792in}}%
\pgfpathcurveto{\pgfqpoint{6.277942in}{1.170748in}}{\pgfqpoint{6.275108in}{1.163906in}}{\pgfqpoint{6.275108in}{1.156773in}}%
\pgfpathcurveto{\pgfqpoint{6.275108in}{1.149641in}}{\pgfqpoint{6.277942in}{1.142799in}}{\pgfqpoint{6.282986in}{1.137755in}}%
\pgfpathcurveto{\pgfqpoint{6.288030in}{1.132712in}}{\pgfqpoint{6.294871in}{1.129878in}}{\pgfqpoint{6.302004in}{1.129878in}}%
\pgfpathclose%
\pgfusepath{stroke,fill}%
\end{pgfscope}%
\begin{pgfscope}%
\pgfpathrectangle{\pgfqpoint{4.985294in}{0.500000in}}{\pgfqpoint{1.764706in}{1.700000in}}%
\pgfusepath{clip}%
\pgfsetbuttcap%
\pgfsetroundjoin%
\definecolor{currentfill}{rgb}{0.964173,0.657587,0.500469}%
\pgfsetfillcolor{currentfill}%
\pgfsetlinewidth{0.311001pt}%
\definecolor{currentstroke}{rgb}{1.000000,1.000000,1.000000}%
\pgfsetstrokecolor{currentstroke}%
\pgfsetdash{}{0pt}%
\pgfpathmoveto{\pgfqpoint{6.417970in}{1.245131in}}%
\pgfpathcurveto{\pgfqpoint{6.425103in}{1.245131in}}{\pgfqpoint{6.431944in}{1.247965in}}{\pgfqpoint{6.436988in}{1.253008in}}%
\pgfpathcurveto{\pgfqpoint{6.442032in}{1.258052in}}{\pgfqpoint{6.444866in}{1.264894in}}{\pgfqpoint{6.444866in}{1.272027in}}%
\pgfpathcurveto{\pgfqpoint{6.444866in}{1.279159in}}{\pgfqpoint{6.442032in}{1.286001in}}{\pgfqpoint{6.436988in}{1.291045in}}%
\pgfpathcurveto{\pgfqpoint{6.431944in}{1.296088in}}{\pgfqpoint{6.425103in}{1.298922in}}{\pgfqpoint{6.417970in}{1.298922in}}%
\pgfpathcurveto{\pgfqpoint{6.410837in}{1.298922in}}{\pgfqpoint{6.403995in}{1.296088in}}{\pgfqpoint{6.398952in}{1.291045in}}%
\pgfpathcurveto{\pgfqpoint{6.393908in}{1.286001in}}{\pgfqpoint{6.391074in}{1.279159in}}{\pgfqpoint{6.391074in}{1.272027in}}%
\pgfpathcurveto{\pgfqpoint{6.391074in}{1.264894in}}{\pgfqpoint{6.393908in}{1.258052in}}{\pgfqpoint{6.398952in}{1.253008in}}%
\pgfpathcurveto{\pgfqpoint{6.403995in}{1.247965in}}{\pgfqpoint{6.410837in}{1.245131in}}{\pgfqpoint{6.417970in}{1.245131in}}%
\pgfpathclose%
\pgfusepath{stroke,fill}%
\end{pgfscope}%
\begin{pgfscope}%
\pgfpathrectangle{\pgfqpoint{4.985294in}{0.500000in}}{\pgfqpoint{1.764706in}{1.700000in}}%
\pgfusepath{clip}%
\pgfsetbuttcap%
\pgfsetroundjoin%
\definecolor{currentfill}{rgb}{0.981377,0.920617,0.865369}%
\pgfsetfillcolor{currentfill}%
\pgfsetlinewidth{0.311001pt}%
\definecolor{currentstroke}{rgb}{1.000000,1.000000,1.000000}%
\pgfsetstrokecolor{currentstroke}%
\pgfsetdash{}{0pt}%
\pgfpathmoveto{\pgfqpoint{6.298237in}{1.463099in}}%
\pgfpathcurveto{\pgfqpoint{6.305370in}{1.463099in}}{\pgfqpoint{6.312211in}{1.465933in}}{\pgfqpoint{6.317255in}{1.470976in}}%
\pgfpathcurveto{\pgfqpoint{6.322299in}{1.476020in}}{\pgfqpoint{6.325132in}{1.482862in}}{\pgfqpoint{6.325132in}{1.489994in}}%
\pgfpathcurveto{\pgfqpoint{6.325132in}{1.497127in}}{\pgfqpoint{6.322299in}{1.503969in}}{\pgfqpoint{6.317255in}{1.509012in}}%
\pgfpathcurveto{\pgfqpoint{6.312211in}{1.514056in}}{\pgfqpoint{6.305370in}{1.516890in}}{\pgfqpoint{6.298237in}{1.516890in}}%
\pgfpathcurveto{\pgfqpoint{6.291104in}{1.516890in}}{\pgfqpoint{6.284262in}{1.514056in}}{\pgfqpoint{6.279219in}{1.509012in}}%
\pgfpathcurveto{\pgfqpoint{6.274175in}{1.503969in}}{\pgfqpoint{6.271341in}{1.497127in}}{\pgfqpoint{6.271341in}{1.489994in}}%
\pgfpathcurveto{\pgfqpoint{6.271341in}{1.482862in}}{\pgfqpoint{6.274175in}{1.476020in}}{\pgfqpoint{6.279219in}{1.470976in}}%
\pgfpathcurveto{\pgfqpoint{6.284262in}{1.465933in}}{\pgfqpoint{6.291104in}{1.463099in}}{\pgfqpoint{6.298237in}{1.463099in}}%
\pgfpathclose%
\pgfusepath{stroke,fill}%
\end{pgfscope}%
\begin{pgfscope}%
\pgfpathrectangle{\pgfqpoint{4.985294in}{0.500000in}}{\pgfqpoint{1.764706in}{1.700000in}}%
\pgfusepath{clip}%
\pgfsetbuttcap%
\pgfsetroundjoin%
\definecolor{currentfill}{rgb}{0.963884,0.644842,0.486120}%
\pgfsetfillcolor{currentfill}%
\pgfsetlinewidth{0.311001pt}%
\definecolor{currentstroke}{rgb}{1.000000,1.000000,1.000000}%
\pgfsetstrokecolor{currentstroke}%
\pgfsetdash{}{0pt}%
\pgfpathmoveto{\pgfqpoint{6.127014in}{1.739514in}}%
\pgfpathcurveto{\pgfqpoint{6.134146in}{1.739514in}}{\pgfqpoint{6.140988in}{1.742348in}}{\pgfqpoint{6.146032in}{1.747391in}}%
\pgfpathcurveto{\pgfqpoint{6.151075in}{1.752435in}}{\pgfqpoint{6.153909in}{1.759276in}}{\pgfqpoint{6.153909in}{1.766409in}}%
\pgfpathcurveto{\pgfqpoint{6.153909in}{1.773542in}}{\pgfqpoint{6.151075in}{1.780384in}}{\pgfqpoint{6.146032in}{1.785427in}}%
\pgfpathcurveto{\pgfqpoint{6.140988in}{1.790471in}}{\pgfqpoint{6.134146in}{1.793305in}}{\pgfqpoint{6.127014in}{1.793305in}}%
\pgfpathcurveto{\pgfqpoint{6.119881in}{1.793305in}}{\pgfqpoint{6.113039in}{1.790471in}}{\pgfqpoint{6.107995in}{1.785427in}}%
\pgfpathcurveto{\pgfqpoint{6.102952in}{1.780384in}}{\pgfqpoint{6.100118in}{1.773542in}}{\pgfqpoint{6.100118in}{1.766409in}}%
\pgfpathcurveto{\pgfqpoint{6.100118in}{1.759276in}}{\pgfqpoint{6.102952in}{1.752435in}}{\pgfqpoint{6.107995in}{1.747391in}}%
\pgfpathcurveto{\pgfqpoint{6.113039in}{1.742348in}}{\pgfqpoint{6.119881in}{1.739514in}}{\pgfqpoint{6.127014in}{1.739514in}}%
\pgfpathclose%
\pgfusepath{stroke,fill}%
\end{pgfscope}%
\begin{pgfscope}%
\pgfpathrectangle{\pgfqpoint{4.985294in}{0.500000in}}{\pgfqpoint{1.764706in}{1.700000in}}%
\pgfusepath{clip}%
\pgfsetbuttcap%
\pgfsetroundjoin%
\definecolor{currentfill}{rgb}{0.908486,0.245685,0.245983}%
\pgfsetfillcolor{currentfill}%
\pgfsetlinewidth{0.311001pt}%
\definecolor{currentstroke}{rgb}{1.000000,1.000000,1.000000}%
\pgfsetstrokecolor{currentstroke}%
\pgfsetdash{}{0pt}%
\pgfpathmoveto{\pgfqpoint{5.623062in}{1.837985in}}%
\pgfpathcurveto{\pgfqpoint{5.630195in}{1.837985in}}{\pgfqpoint{5.637037in}{1.840819in}}{\pgfqpoint{5.642081in}{1.845863in}}%
\pgfpathcurveto{\pgfqpoint{5.647124in}{1.850906in}}{\pgfqpoint{5.649958in}{1.857748in}}{\pgfqpoint{5.649958in}{1.864881in}}%
\pgfpathcurveto{\pgfqpoint{5.649958in}{1.872014in}}{\pgfqpoint{5.647124in}{1.878855in}}{\pgfqpoint{5.642081in}{1.883899in}}%
\pgfpathcurveto{\pgfqpoint{5.637037in}{1.888943in}}{\pgfqpoint{5.630195in}{1.891777in}}{\pgfqpoint{5.623062in}{1.891777in}}%
\pgfpathcurveto{\pgfqpoint{5.615930in}{1.891777in}}{\pgfqpoint{5.609088in}{1.888943in}}{\pgfqpoint{5.604044in}{1.883899in}}%
\pgfpathcurveto{\pgfqpoint{5.599001in}{1.878855in}}{\pgfqpoint{5.596167in}{1.872014in}}{\pgfqpoint{5.596167in}{1.864881in}}%
\pgfpathcurveto{\pgfqpoint{5.596167in}{1.857748in}}{\pgfqpoint{5.599001in}{1.850906in}}{\pgfqpoint{5.604044in}{1.845863in}}%
\pgfpathcurveto{\pgfqpoint{5.609088in}{1.840819in}}{\pgfqpoint{5.615930in}{1.837985in}}{\pgfqpoint{5.623062in}{1.837985in}}%
\pgfpathclose%
\pgfusepath{stroke,fill}%
\end{pgfscope}%
\begin{pgfscope}%
\pgfpathrectangle{\pgfqpoint{4.985294in}{0.500000in}}{\pgfqpoint{1.764706in}{1.700000in}}%
\pgfusepath{clip}%
\pgfsetbuttcap%
\pgfsetroundjoin%
\definecolor{currentfill}{rgb}{0.617923,0.103253,0.357601}%
\pgfsetfillcolor{currentfill}%
\pgfsetlinewidth{0.311001pt}%
\definecolor{currentstroke}{rgb}{1.000000,1.000000,1.000000}%
\pgfsetstrokecolor{currentstroke}%
\pgfsetdash{}{0pt}%
\pgfpathmoveto{\pgfqpoint{5.233986in}{1.094204in}}%
\pgfpathcurveto{\pgfqpoint{5.241119in}{1.094204in}}{\pgfqpoint{5.247961in}{1.097038in}}{\pgfqpoint{5.253005in}{1.102082in}}%
\pgfpathcurveto{\pgfqpoint{5.258048in}{1.107125in}}{\pgfqpoint{5.260882in}{1.113967in}}{\pgfqpoint{5.260882in}{1.121100in}}%
\pgfpathcurveto{\pgfqpoint{5.260882in}{1.128233in}}{\pgfqpoint{5.258048in}{1.135074in}}{\pgfqpoint{5.253005in}{1.140118in}}%
\pgfpathcurveto{\pgfqpoint{5.247961in}{1.145162in}}{\pgfqpoint{5.241119in}{1.147996in}}{\pgfqpoint{5.233986in}{1.147996in}}%
\pgfpathcurveto{\pgfqpoint{5.226854in}{1.147996in}}{\pgfqpoint{5.220012in}{1.145162in}}{\pgfqpoint{5.214968in}{1.140118in}}%
\pgfpathcurveto{\pgfqpoint{5.209925in}{1.135074in}}{\pgfqpoint{5.207091in}{1.128233in}}{\pgfqpoint{5.207091in}{1.121100in}}%
\pgfpathcurveto{\pgfqpoint{5.207091in}{1.113967in}}{\pgfqpoint{5.209925in}{1.107125in}}{\pgfqpoint{5.214968in}{1.102082in}}%
\pgfpathcurveto{\pgfqpoint{5.220012in}{1.097038in}}{\pgfqpoint{5.226854in}{1.094204in}}{\pgfqpoint{5.233986in}{1.094204in}}%
\pgfpathclose%
\pgfusepath{stroke,fill}%
\end{pgfscope}%
\begin{pgfscope}%
\pgfpathrectangle{\pgfqpoint{4.985294in}{0.500000in}}{\pgfqpoint{1.764706in}{1.700000in}}%
\pgfusepath{clip}%
\pgfsetbuttcap%
\pgfsetroundjoin%
\definecolor{currentfill}{rgb}{0.965302,0.713942,0.568499}%
\pgfsetfillcolor{currentfill}%
\pgfsetlinewidth{0.311001pt}%
\definecolor{currentstroke}{rgb}{1.000000,1.000000,1.000000}%
\pgfsetstrokecolor{currentstroke}%
\pgfsetdash{}{0pt}%
\pgfpathmoveto{\pgfqpoint{6.211824in}{0.922648in}}%
\pgfpathcurveto{\pgfqpoint{6.218956in}{0.922648in}}{\pgfqpoint{6.225798in}{0.925482in}}{\pgfqpoint{6.230842in}{0.930526in}}%
\pgfpathcurveto{\pgfqpoint{6.235885in}{0.935569in}}{\pgfqpoint{6.238719in}{0.942411in}}{\pgfqpoint{6.238719in}{0.949544in}}%
\pgfpathcurveto{\pgfqpoint{6.238719in}{0.956677in}}{\pgfqpoint{6.235885in}{0.963518in}}{\pgfqpoint{6.230842in}{0.968562in}}%
\pgfpathcurveto{\pgfqpoint{6.225798in}{0.973606in}}{\pgfqpoint{6.218956in}{0.976439in}}{\pgfqpoint{6.211824in}{0.976439in}}%
\pgfpathcurveto{\pgfqpoint{6.204691in}{0.976439in}}{\pgfqpoint{6.197849in}{0.973606in}}{\pgfqpoint{6.192805in}{0.968562in}}%
\pgfpathcurveto{\pgfqpoint{6.187762in}{0.963518in}}{\pgfqpoint{6.184928in}{0.956677in}}{\pgfqpoint{6.184928in}{0.949544in}}%
\pgfpathcurveto{\pgfqpoint{6.184928in}{0.942411in}}{\pgfqpoint{6.187762in}{0.935569in}}{\pgfqpoint{6.192805in}{0.930526in}}%
\pgfpathcurveto{\pgfqpoint{6.197849in}{0.925482in}}{\pgfqpoint{6.204691in}{0.922648in}}{\pgfqpoint{6.211824in}{0.922648in}}%
\pgfpathclose%
\pgfusepath{stroke,fill}%
\end{pgfscope}%
\begin{pgfscope}%
\pgfpathrectangle{\pgfqpoint{4.985294in}{0.500000in}}{\pgfqpoint{1.764706in}{1.700000in}}%
\pgfusepath{clip}%
\pgfsetbuttcap%
\pgfsetroundjoin%
\definecolor{currentfill}{rgb}{0.964173,0.657587,0.500469}%
\pgfsetfillcolor{currentfill}%
\pgfsetlinewidth{0.311001pt}%
\definecolor{currentstroke}{rgb}{1.000000,1.000000,1.000000}%
\pgfsetstrokecolor{currentstroke}%
\pgfsetdash{}{0pt}%
\pgfpathmoveto{\pgfqpoint{6.128437in}{1.041844in}}%
\pgfpathcurveto{\pgfqpoint{6.135570in}{1.041844in}}{\pgfqpoint{6.142411in}{1.044678in}}{\pgfqpoint{6.147455in}{1.049721in}}%
\pgfpathcurveto{\pgfqpoint{6.152499in}{1.054765in}}{\pgfqpoint{6.155332in}{1.061607in}}{\pgfqpoint{6.155332in}{1.068740in}}%
\pgfpathcurveto{\pgfqpoint{6.155332in}{1.075872in}}{\pgfqpoint{6.152499in}{1.082714in}}{\pgfqpoint{6.147455in}{1.087758in}}%
\pgfpathcurveto{\pgfqpoint{6.142411in}{1.092801in}}{\pgfqpoint{6.135570in}{1.095635in}}{\pgfqpoint{6.128437in}{1.095635in}}%
\pgfpathcurveto{\pgfqpoint{6.121304in}{1.095635in}}{\pgfqpoint{6.114462in}{1.092801in}}{\pgfqpoint{6.109419in}{1.087758in}}%
\pgfpathcurveto{\pgfqpoint{6.104375in}{1.082714in}}{\pgfqpoint{6.101541in}{1.075872in}}{\pgfqpoint{6.101541in}{1.068740in}}%
\pgfpathcurveto{\pgfqpoint{6.101541in}{1.061607in}}{\pgfqpoint{6.104375in}{1.054765in}}{\pgfqpoint{6.109419in}{1.049721in}}%
\pgfpathcurveto{\pgfqpoint{6.114462in}{1.044678in}}{\pgfqpoint{6.121304in}{1.041844in}}{\pgfqpoint{6.128437in}{1.041844in}}%
\pgfpathclose%
\pgfusepath{stroke,fill}%
\end{pgfscope}%
\begin{pgfscope}%
\pgfpathrectangle{\pgfqpoint{4.985294in}{0.500000in}}{\pgfqpoint{1.764706in}{1.700000in}}%
\pgfusepath{clip}%
\pgfsetbuttcap%
\pgfsetroundjoin%
\definecolor{currentfill}{rgb}{0.963190,0.619109,0.458249}%
\pgfsetfillcolor{currentfill}%
\pgfsetlinewidth{0.311001pt}%
\definecolor{currentstroke}{rgb}{1.000000,1.000000,1.000000}%
\pgfsetstrokecolor{currentstroke}%
\pgfsetdash{}{0pt}%
\pgfpathmoveto{\pgfqpoint{5.325801in}{1.154223in}}%
\pgfpathcurveto{\pgfqpoint{5.332934in}{1.154223in}}{\pgfqpoint{5.339775in}{1.157057in}}{\pgfqpoint{5.344819in}{1.162100in}}%
\pgfpathcurveto{\pgfqpoint{5.349863in}{1.167144in}}{\pgfqpoint{5.352697in}{1.173986in}}{\pgfqpoint{5.352697in}{1.181118in}}%
\pgfpathcurveto{\pgfqpoint{5.352697in}{1.188251in}}{\pgfqpoint{5.349863in}{1.195093in}}{\pgfqpoint{5.344819in}{1.200137in}}%
\pgfpathcurveto{\pgfqpoint{5.339775in}{1.205180in}}{\pgfqpoint{5.332934in}{1.208014in}}{\pgfqpoint{5.325801in}{1.208014in}}%
\pgfpathcurveto{\pgfqpoint{5.318668in}{1.208014in}}{\pgfqpoint{5.311826in}{1.205180in}}{\pgfqpoint{5.306783in}{1.200137in}}%
\pgfpathcurveto{\pgfqpoint{5.301739in}{1.195093in}}{\pgfqpoint{5.298905in}{1.188251in}}{\pgfqpoint{5.298905in}{1.181118in}}%
\pgfpathcurveto{\pgfqpoint{5.298905in}{1.173986in}}{\pgfqpoint{5.301739in}{1.167144in}}{\pgfqpoint{5.306783in}{1.162100in}}%
\pgfpathcurveto{\pgfqpoint{5.311826in}{1.157057in}}{\pgfqpoint{5.318668in}{1.154223in}}{\pgfqpoint{5.325801in}{1.154223in}}%
\pgfpathclose%
\pgfusepath{stroke,fill}%
\end{pgfscope}%
\begin{pgfscope}%
\pgfpathrectangle{\pgfqpoint{4.985294in}{0.500000in}}{\pgfqpoint{1.764706in}{1.700000in}}%
\pgfusepath{clip}%
\pgfsetbuttcap%
\pgfsetroundjoin%
\definecolor{currentfill}{rgb}{0.964433,0.670254,0.515093}%
\pgfsetfillcolor{currentfill}%
\pgfsetlinewidth{0.311001pt}%
\definecolor{currentstroke}{rgb}{1.000000,1.000000,1.000000}%
\pgfsetstrokecolor{currentstroke}%
\pgfsetdash{}{0pt}%
\pgfpathmoveto{\pgfqpoint{5.586331in}{0.894126in}}%
\pgfpathcurveto{\pgfqpoint{5.593464in}{0.894126in}}{\pgfqpoint{5.600306in}{0.896960in}}{\pgfqpoint{5.605350in}{0.902003in}}%
\pgfpathcurveto{\pgfqpoint{5.610393in}{0.907047in}}{\pgfqpoint{5.613227in}{0.913889in}}{\pgfqpoint{5.613227in}{0.921022in}}%
\pgfpathcurveto{\pgfqpoint{5.613227in}{0.928154in}}{\pgfqpoint{5.610393in}{0.934996in}}{\pgfqpoint{5.605350in}{0.940040in}}%
\pgfpathcurveto{\pgfqpoint{5.600306in}{0.945083in}}{\pgfqpoint{5.593464in}{0.947917in}}{\pgfqpoint{5.586331in}{0.947917in}}%
\pgfpathcurveto{\pgfqpoint{5.579199in}{0.947917in}}{\pgfqpoint{5.572357in}{0.945083in}}{\pgfqpoint{5.567313in}{0.940040in}}%
\pgfpathcurveto{\pgfqpoint{5.562270in}{0.934996in}}{\pgfqpoint{5.559436in}{0.928154in}}{\pgfqpoint{5.559436in}{0.921022in}}%
\pgfpathcurveto{\pgfqpoint{5.559436in}{0.913889in}}{\pgfqpoint{5.562270in}{0.907047in}}{\pgfqpoint{5.567313in}{0.902003in}}%
\pgfpathcurveto{\pgfqpoint{5.572357in}{0.896960in}}{\pgfqpoint{5.579199in}{0.894126in}}{\pgfqpoint{5.586331in}{0.894126in}}%
\pgfpathclose%
\pgfusepath{stroke,fill}%
\end{pgfscope}%
\begin{pgfscope}%
\pgfpathrectangle{\pgfqpoint{4.985294in}{0.500000in}}{\pgfqpoint{1.764706in}{1.700000in}}%
\pgfusepath{clip}%
\pgfsetbuttcap%
\pgfsetroundjoin%
\definecolor{currentfill}{rgb}{0.972726,0.844889,0.754401}%
\pgfsetfillcolor{currentfill}%
\pgfsetlinewidth{0.311001pt}%
\definecolor{currentstroke}{rgb}{1.000000,1.000000,1.000000}%
\pgfsetstrokecolor{currentstroke}%
\pgfsetdash{}{0pt}%
\pgfpathmoveto{\pgfqpoint{5.481761in}{1.373192in}}%
\pgfpathcurveto{\pgfqpoint{5.488894in}{1.373192in}}{\pgfqpoint{5.495736in}{1.376026in}}{\pgfqpoint{5.500780in}{1.381069in}}%
\pgfpathcurveto{\pgfqpoint{5.505823in}{1.386113in}}{\pgfqpoint{5.508657in}{1.392955in}}{\pgfqpoint{5.508657in}{1.400087in}}%
\pgfpathcurveto{\pgfqpoint{5.508657in}{1.407220in}}{\pgfqpoint{5.505823in}{1.414062in}}{\pgfqpoint{5.500780in}{1.419106in}}%
\pgfpathcurveto{\pgfqpoint{5.495736in}{1.424149in}}{\pgfqpoint{5.488894in}{1.426983in}}{\pgfqpoint{5.481761in}{1.426983in}}%
\pgfpathcurveto{\pgfqpoint{5.474629in}{1.426983in}}{\pgfqpoint{5.467787in}{1.424149in}}{\pgfqpoint{5.462743in}{1.419106in}}%
\pgfpathcurveto{\pgfqpoint{5.457700in}{1.414062in}}{\pgfqpoint{5.454866in}{1.407220in}}{\pgfqpoint{5.454866in}{1.400087in}}%
\pgfpathcurveto{\pgfqpoint{5.454866in}{1.392955in}}{\pgfqpoint{5.457700in}{1.386113in}}{\pgfqpoint{5.462743in}{1.381069in}}%
\pgfpathcurveto{\pgfqpoint{5.467787in}{1.376026in}}{\pgfqpoint{5.474629in}{1.373192in}}{\pgfqpoint{5.481761in}{1.373192in}}%
\pgfpathclose%
\pgfusepath{stroke,fill}%
\end{pgfscope}%
\begin{pgfscope}%
\pgfpathrectangle{\pgfqpoint{4.985294in}{0.500000in}}{\pgfqpoint{1.764706in}{1.700000in}}%
\pgfusepath{clip}%
\pgfsetbuttcap%
\pgfsetroundjoin%
\definecolor{currentfill}{rgb}{0.967735,0.780441,0.659127}%
\pgfsetfillcolor{currentfill}%
\pgfsetlinewidth{0.311001pt}%
\definecolor{currentstroke}{rgb}{1.000000,1.000000,1.000000}%
\pgfsetstrokecolor{currentstroke}%
\pgfsetdash{}{0pt}%
\pgfpathmoveto{\pgfqpoint{6.190676in}{0.954279in}}%
\pgfpathcurveto{\pgfqpoint{6.197809in}{0.954279in}}{\pgfqpoint{6.204651in}{0.957113in}}{\pgfqpoint{6.209694in}{0.962156in}}%
\pgfpathcurveto{\pgfqpoint{6.214738in}{0.967200in}}{\pgfqpoint{6.217572in}{0.974042in}}{\pgfqpoint{6.217572in}{0.981175in}}%
\pgfpathcurveto{\pgfqpoint{6.217572in}{0.988307in}}{\pgfqpoint{6.214738in}{0.995149in}}{\pgfqpoint{6.209694in}{1.000193in}}%
\pgfpathcurveto{\pgfqpoint{6.204651in}{1.005236in}}{\pgfqpoint{6.197809in}{1.008070in}}{\pgfqpoint{6.190676in}{1.008070in}}%
\pgfpathcurveto{\pgfqpoint{6.183543in}{1.008070in}}{\pgfqpoint{6.176702in}{1.005236in}}{\pgfqpoint{6.171658in}{1.000193in}}%
\pgfpathcurveto{\pgfqpoint{6.166615in}{0.995149in}}{\pgfqpoint{6.163781in}{0.988307in}}{\pgfqpoint{6.163781in}{0.981175in}}%
\pgfpathcurveto{\pgfqpoint{6.163781in}{0.974042in}}{\pgfqpoint{6.166615in}{0.967200in}}{\pgfqpoint{6.171658in}{0.962156in}}%
\pgfpathcurveto{\pgfqpoint{6.176702in}{0.957113in}}{\pgfqpoint{6.183543in}{0.954279in}}{\pgfqpoint{6.190676in}{0.954279in}}%
\pgfpathclose%
\pgfusepath{stroke,fill}%
\end{pgfscope}%
\begin{pgfscope}%
\pgfpathrectangle{\pgfqpoint{4.985294in}{0.500000in}}{\pgfqpoint{1.764706in}{1.700000in}}%
\pgfusepath{clip}%
\pgfsetbuttcap%
\pgfsetroundjoin%
\definecolor{currentfill}{rgb}{0.968509,0.792226,0.676405}%
\pgfsetfillcolor{currentfill}%
\pgfsetlinewidth{0.311001pt}%
\definecolor{currentstroke}{rgb}{1.000000,1.000000,1.000000}%
\pgfsetstrokecolor{currentstroke}%
\pgfsetdash{}{0pt}%
\pgfpathmoveto{\pgfqpoint{5.434931in}{1.011657in}}%
\pgfpathcurveto{\pgfqpoint{5.442064in}{1.011657in}}{\pgfqpoint{5.448905in}{1.014490in}}{\pgfqpoint{5.453949in}{1.019534in}}%
\pgfpathcurveto{\pgfqpoint{5.458993in}{1.024578in}}{\pgfqpoint{5.461827in}{1.031419in}}{\pgfqpoint{5.461827in}{1.038552in}}%
\pgfpathcurveto{\pgfqpoint{5.461827in}{1.045685in}}{\pgfqpoint{5.458993in}{1.052527in}}{\pgfqpoint{5.453949in}{1.057570in}}%
\pgfpathcurveto{\pgfqpoint{5.448905in}{1.062614in}}{\pgfqpoint{5.442064in}{1.065448in}}{\pgfqpoint{5.434931in}{1.065448in}}%
\pgfpathcurveto{\pgfqpoint{5.427798in}{1.065448in}}{\pgfqpoint{5.420956in}{1.062614in}}{\pgfqpoint{5.415913in}{1.057570in}}%
\pgfpathcurveto{\pgfqpoint{5.410869in}{1.052527in}}{\pgfqpoint{5.408035in}{1.045685in}}{\pgfqpoint{5.408035in}{1.038552in}}%
\pgfpathcurveto{\pgfqpoint{5.408035in}{1.031419in}}{\pgfqpoint{5.410869in}{1.024578in}}{\pgfqpoint{5.415913in}{1.019534in}}%
\pgfpathcurveto{\pgfqpoint{5.420956in}{1.014490in}}{\pgfqpoint{5.427798in}{1.011657in}}{\pgfqpoint{5.434931in}{1.011657in}}%
\pgfpathclose%
\pgfusepath{stroke,fill}%
\end{pgfscope}%
\begin{pgfscope}%
\pgfpathrectangle{\pgfqpoint{4.985294in}{0.500000in}}{\pgfqpoint{1.764706in}{1.700000in}}%
\pgfusepath{clip}%
\pgfsetbuttcap%
\pgfsetroundjoin%
\definecolor{currentfill}{rgb}{0.978376,0.897317,0.831308}%
\pgfsetfillcolor{currentfill}%
\pgfsetlinewidth{0.311001pt}%
\definecolor{currentstroke}{rgb}{1.000000,1.000000,1.000000}%
\pgfsetstrokecolor{currentstroke}%
\pgfsetdash{}{0pt}%
\pgfpathmoveto{\pgfqpoint{6.314122in}{1.175659in}}%
\pgfpathcurveto{\pgfqpoint{6.321255in}{1.175659in}}{\pgfqpoint{6.328096in}{1.178493in}}{\pgfqpoint{6.333140in}{1.183536in}}%
\pgfpathcurveto{\pgfqpoint{6.338184in}{1.188580in}}{\pgfqpoint{6.341017in}{1.195422in}}{\pgfqpoint{6.341017in}{1.202555in}}%
\pgfpathcurveto{\pgfqpoint{6.341017in}{1.209687in}}{\pgfqpoint{6.338184in}{1.216529in}}{\pgfqpoint{6.333140in}{1.221573in}}%
\pgfpathcurveto{\pgfqpoint{6.328096in}{1.226616in}}{\pgfqpoint{6.321255in}{1.229450in}}{\pgfqpoint{6.314122in}{1.229450in}}%
\pgfpathcurveto{\pgfqpoint{6.306989in}{1.229450in}}{\pgfqpoint{6.300147in}{1.226616in}}{\pgfqpoint{6.295104in}{1.221573in}}%
\pgfpathcurveto{\pgfqpoint{6.290060in}{1.216529in}}{\pgfqpoint{6.287226in}{1.209687in}}{\pgfqpoint{6.287226in}{1.202555in}}%
\pgfpathcurveto{\pgfqpoint{6.287226in}{1.195422in}}{\pgfqpoint{6.290060in}{1.188580in}}{\pgfqpoint{6.295104in}{1.183536in}}%
\pgfpathcurveto{\pgfqpoint{6.300147in}{1.178493in}}{\pgfqpoint{6.306989in}{1.175659in}}{\pgfqpoint{6.314122in}{1.175659in}}%
\pgfpathclose%
\pgfusepath{stroke,fill}%
\end{pgfscope}%
\begin{pgfscope}%
\pgfpathrectangle{\pgfqpoint{4.985294in}{0.500000in}}{\pgfqpoint{1.764706in}{1.700000in}}%
\pgfusepath{clip}%
\pgfsetbuttcap%
\pgfsetroundjoin%
\definecolor{currentfill}{rgb}{0.977657,0.891500,0.822809}%
\pgfsetfillcolor{currentfill}%
\pgfsetlinewidth{0.311001pt}%
\definecolor{currentstroke}{rgb}{1.000000,1.000000,1.000000}%
\pgfsetstrokecolor{currentstroke}%
\pgfsetdash{}{0pt}%
\pgfpathmoveto{\pgfqpoint{5.453770in}{1.215029in}}%
\pgfpathcurveto{\pgfqpoint{5.460903in}{1.215029in}}{\pgfqpoint{5.467744in}{1.217863in}}{\pgfqpoint{5.472788in}{1.222906in}}%
\pgfpathcurveto{\pgfqpoint{5.477832in}{1.227950in}}{\pgfqpoint{5.480666in}{1.234792in}}{\pgfqpoint{5.480666in}{1.241925in}}%
\pgfpathcurveto{\pgfqpoint{5.480666in}{1.249057in}}{\pgfqpoint{5.477832in}{1.255899in}}{\pgfqpoint{5.472788in}{1.260943in}}%
\pgfpathcurveto{\pgfqpoint{5.467744in}{1.265986in}}{\pgfqpoint{5.460903in}{1.268820in}}{\pgfqpoint{5.453770in}{1.268820in}}%
\pgfpathcurveto{\pgfqpoint{5.446637in}{1.268820in}}{\pgfqpoint{5.439795in}{1.265986in}}{\pgfqpoint{5.434752in}{1.260943in}}%
\pgfpathcurveto{\pgfqpoint{5.429708in}{1.255899in}}{\pgfqpoint{5.426874in}{1.249057in}}{\pgfqpoint{5.426874in}{1.241925in}}%
\pgfpathcurveto{\pgfqpoint{5.426874in}{1.234792in}}{\pgfqpoint{5.429708in}{1.227950in}}{\pgfqpoint{5.434752in}{1.222906in}}%
\pgfpathcurveto{\pgfqpoint{5.439795in}{1.217863in}}{\pgfqpoint{5.446637in}{1.215029in}}{\pgfqpoint{5.453770in}{1.215029in}}%
\pgfpathclose%
\pgfusepath{stroke,fill}%
\end{pgfscope}%
\begin{pgfscope}%
\pgfpathrectangle{\pgfqpoint{4.985294in}{0.500000in}}{\pgfqpoint{1.764706in}{1.700000in}}%
\pgfusepath{clip}%
\pgfsetbuttcap%
\pgfsetroundjoin%
\definecolor{currentfill}{rgb}{0.979124,0.903132,0.839793}%
\pgfsetfillcolor{currentfill}%
\pgfsetlinewidth{0.311001pt}%
\definecolor{currentstroke}{rgb}{1.000000,1.000000,1.000000}%
\pgfsetstrokecolor{currentstroke}%
\pgfsetdash{}{0pt}%
\pgfpathmoveto{\pgfqpoint{5.397909in}{1.264594in}}%
\pgfpathcurveto{\pgfqpoint{5.405042in}{1.264594in}}{\pgfqpoint{5.411884in}{1.267428in}}{\pgfqpoint{5.416927in}{1.272471in}}%
\pgfpathcurveto{\pgfqpoint{5.421971in}{1.277515in}}{\pgfqpoint{5.424805in}{1.284357in}}{\pgfqpoint{5.424805in}{1.291489in}}%
\pgfpathcurveto{\pgfqpoint{5.424805in}{1.298622in}}{\pgfqpoint{5.421971in}{1.305464in}}{\pgfqpoint{5.416927in}{1.310507in}}%
\pgfpathcurveto{\pgfqpoint{5.411884in}{1.315551in}}{\pgfqpoint{5.405042in}{1.318385in}}{\pgfqpoint{5.397909in}{1.318385in}}%
\pgfpathcurveto{\pgfqpoint{5.390776in}{1.318385in}}{\pgfqpoint{5.383935in}{1.315551in}}{\pgfqpoint{5.378891in}{1.310507in}}%
\pgfpathcurveto{\pgfqpoint{5.373847in}{1.305464in}}{\pgfqpoint{5.371013in}{1.298622in}}{\pgfqpoint{5.371013in}{1.291489in}}%
\pgfpathcurveto{\pgfqpoint{5.371013in}{1.284357in}}{\pgfqpoint{5.373847in}{1.277515in}}{\pgfqpoint{5.378891in}{1.272471in}}%
\pgfpathcurveto{\pgfqpoint{5.383935in}{1.267428in}}{\pgfqpoint{5.390776in}{1.264594in}}{\pgfqpoint{5.397909in}{1.264594in}}%
\pgfpathclose%
\pgfusepath{stroke,fill}%
\end{pgfscope}%
\begin{pgfscope}%
\pgfpathrectangle{\pgfqpoint{4.985294in}{0.500000in}}{\pgfqpoint{1.764706in}{1.700000in}}%
\pgfusepath{clip}%
\pgfsetbuttcap%
\pgfsetroundjoin%
\definecolor{currentfill}{rgb}{0.967735,0.780441,0.659127}%
\pgfsetfillcolor{currentfill}%
\pgfsetlinewidth{0.311001pt}%
\definecolor{currentstroke}{rgb}{1.000000,1.000000,1.000000}%
\pgfsetstrokecolor{currentstroke}%
\pgfsetdash{}{0pt}%
\pgfpathmoveto{\pgfqpoint{5.507575in}{1.215213in}}%
\pgfpathcurveto{\pgfqpoint{5.514708in}{1.215213in}}{\pgfqpoint{5.521549in}{1.218047in}}{\pgfqpoint{5.526593in}{1.223091in}}%
\pgfpathcurveto{\pgfqpoint{5.531637in}{1.228135in}}{\pgfqpoint{5.534470in}{1.234976in}}{\pgfqpoint{5.534470in}{1.242109in}}%
\pgfpathcurveto{\pgfqpoint{5.534470in}{1.249242in}}{\pgfqpoint{5.531637in}{1.256084in}}{\pgfqpoint{5.526593in}{1.261127in}}%
\pgfpathcurveto{\pgfqpoint{5.521549in}{1.266171in}}{\pgfqpoint{5.514708in}{1.269005in}}{\pgfqpoint{5.507575in}{1.269005in}}%
\pgfpathcurveto{\pgfqpoint{5.500442in}{1.269005in}}{\pgfqpoint{5.493600in}{1.266171in}}{\pgfqpoint{5.488557in}{1.261127in}}%
\pgfpathcurveto{\pgfqpoint{5.483513in}{1.256084in}}{\pgfqpoint{5.480679in}{1.249242in}}{\pgfqpoint{5.480679in}{1.242109in}}%
\pgfpathcurveto{\pgfqpoint{5.480679in}{1.234976in}}{\pgfqpoint{5.483513in}{1.228135in}}{\pgfqpoint{5.488557in}{1.223091in}}%
\pgfpathcurveto{\pgfqpoint{5.493600in}{1.218047in}}{\pgfqpoint{5.500442in}{1.215213in}}{\pgfqpoint{5.507575in}{1.215213in}}%
\pgfpathclose%
\pgfusepath{stroke,fill}%
\end{pgfscope}%
\begin{pgfscope}%
\pgfpathrectangle{\pgfqpoint{4.985294in}{0.500000in}}{\pgfqpoint{1.764706in}{1.700000in}}%
\pgfusepath{clip}%
\pgfsetbuttcap%
\pgfsetroundjoin%
\definecolor{currentfill}{rgb}{0.976961,0.885681,0.814303}%
\pgfsetfillcolor{currentfill}%
\pgfsetlinewidth{0.311001pt}%
\definecolor{currentstroke}{rgb}{1.000000,1.000000,1.000000}%
\pgfsetstrokecolor{currentstroke}%
\pgfsetdash{}{0pt}%
\pgfpathmoveto{\pgfqpoint{6.273839in}{1.286972in}}%
\pgfpathcurveto{\pgfqpoint{6.280972in}{1.286972in}}{\pgfqpoint{6.287814in}{1.289806in}}{\pgfqpoint{6.292858in}{1.294850in}}%
\pgfpathcurveto{\pgfqpoint{6.297901in}{1.299894in}}{\pgfqpoint{6.300735in}{1.306735in}}{\pgfqpoint{6.300735in}{1.313868in}}%
\pgfpathcurveto{\pgfqpoint{6.300735in}{1.321001in}}{\pgfqpoint{6.297901in}{1.327843in}}{\pgfqpoint{6.292858in}{1.332886in}}%
\pgfpathcurveto{\pgfqpoint{6.287814in}{1.337930in}}{\pgfqpoint{6.280972in}{1.340764in}}{\pgfqpoint{6.273839in}{1.340764in}}%
\pgfpathcurveto{\pgfqpoint{6.266707in}{1.340764in}}{\pgfqpoint{6.259865in}{1.337930in}}{\pgfqpoint{6.254821in}{1.332886in}}%
\pgfpathcurveto{\pgfqpoint{6.249778in}{1.327843in}}{\pgfqpoint{6.246944in}{1.321001in}}{\pgfqpoint{6.246944in}{1.313868in}}%
\pgfpathcurveto{\pgfqpoint{6.246944in}{1.306735in}}{\pgfqpoint{6.249778in}{1.299894in}}{\pgfqpoint{6.254821in}{1.294850in}}%
\pgfpathcurveto{\pgfqpoint{6.259865in}{1.289806in}}{\pgfqpoint{6.266707in}{1.286972in}}{\pgfqpoint{6.273839in}{1.286972in}}%
\pgfpathclose%
\pgfusepath{stroke,fill}%
\end{pgfscope}%
\begin{pgfscope}%
\pgfpathrectangle{\pgfqpoint{4.985294in}{0.500000in}}{\pgfqpoint{1.764706in}{1.700000in}}%
\pgfusepath{clip}%
\pgfsetbuttcap%
\pgfsetroundjoin%
\definecolor{currentfill}{rgb}{0.973832,0.856556,0.771584}%
\pgfsetfillcolor{currentfill}%
\pgfsetlinewidth{0.311001pt}%
\definecolor{currentstroke}{rgb}{1.000000,1.000000,1.000000}%
\pgfsetstrokecolor{currentstroke}%
\pgfsetdash{}{0pt}%
\pgfpathmoveto{\pgfqpoint{5.489945in}{1.461321in}}%
\pgfpathcurveto{\pgfqpoint{5.497078in}{1.461321in}}{\pgfqpoint{5.503919in}{1.464155in}}{\pgfqpoint{5.508963in}{1.469199in}}%
\pgfpathcurveto{\pgfqpoint{5.514007in}{1.474243in}}{\pgfqpoint{5.516841in}{1.481084in}}{\pgfqpoint{5.516841in}{1.488217in}}%
\pgfpathcurveto{\pgfqpoint{5.516841in}{1.495350in}}{\pgfqpoint{5.514007in}{1.502192in}}{\pgfqpoint{5.508963in}{1.507235in}}%
\pgfpathcurveto{\pgfqpoint{5.503919in}{1.512279in}}{\pgfqpoint{5.497078in}{1.515113in}}{\pgfqpoint{5.489945in}{1.515113in}}%
\pgfpathcurveto{\pgfqpoint{5.482812in}{1.515113in}}{\pgfqpoint{5.475970in}{1.512279in}}{\pgfqpoint{5.470927in}{1.507235in}}%
\pgfpathcurveto{\pgfqpoint{5.465883in}{1.502192in}}{\pgfqpoint{5.463049in}{1.495350in}}{\pgfqpoint{5.463049in}{1.488217in}}%
\pgfpathcurveto{\pgfqpoint{5.463049in}{1.481084in}}{\pgfqpoint{5.465883in}{1.474243in}}{\pgfqpoint{5.470927in}{1.469199in}}%
\pgfpathcurveto{\pgfqpoint{5.475970in}{1.464155in}}{\pgfqpoint{5.482812in}{1.461321in}}{\pgfqpoint{5.489945in}{1.461321in}}%
\pgfpathclose%
\pgfusepath{stroke,fill}%
\end{pgfscope}%
\begin{pgfscope}%
\pgfpathrectangle{\pgfqpoint{4.985294in}{0.500000in}}{\pgfqpoint{1.764706in}{1.700000in}}%
\pgfusepath{clip}%
\pgfsetbuttcap%
\pgfsetroundjoin%
\definecolor{currentfill}{rgb}{0.955103,0.477872,0.328626}%
\pgfsetfillcolor{currentfill}%
\pgfsetlinewidth{0.311001pt}%
\definecolor{currentstroke}{rgb}{1.000000,1.000000,1.000000}%
\pgfsetstrokecolor{currentstroke}%
\pgfsetdash{}{0pt}%
\pgfpathmoveto{\pgfqpoint{6.182098in}{1.777384in}}%
\pgfpathcurveto{\pgfqpoint{6.189231in}{1.777384in}}{\pgfqpoint{6.196073in}{1.780218in}}{\pgfqpoint{6.201116in}{1.785262in}}%
\pgfpathcurveto{\pgfqpoint{6.206160in}{1.790305in}}{\pgfqpoint{6.208994in}{1.797147in}}{\pgfqpoint{6.208994in}{1.804280in}}%
\pgfpathcurveto{\pgfqpoint{6.208994in}{1.811413in}}{\pgfqpoint{6.206160in}{1.818254in}}{\pgfqpoint{6.201116in}{1.823298in}}%
\pgfpathcurveto{\pgfqpoint{6.196073in}{1.828342in}}{\pgfqpoint{6.189231in}{1.831176in}}{\pgfqpoint{6.182098in}{1.831176in}}%
\pgfpathcurveto{\pgfqpoint{6.174965in}{1.831176in}}{\pgfqpoint{6.168124in}{1.828342in}}{\pgfqpoint{6.163080in}{1.823298in}}%
\pgfpathcurveto{\pgfqpoint{6.158036in}{1.818254in}}{\pgfqpoint{6.155203in}{1.811413in}}{\pgfqpoint{6.155203in}{1.804280in}}%
\pgfpathcurveto{\pgfqpoint{6.155203in}{1.797147in}}{\pgfqpoint{6.158036in}{1.790305in}}{\pgfqpoint{6.163080in}{1.785262in}}%
\pgfpathcurveto{\pgfqpoint{6.168124in}{1.780218in}}{\pgfqpoint{6.174965in}{1.777384in}}{\pgfqpoint{6.182098in}{1.777384in}}%
\pgfpathclose%
\pgfusepath{stroke,fill}%
\end{pgfscope}%
\begin{pgfscope}%
\pgfpathrectangle{\pgfqpoint{4.985294in}{0.500000in}}{\pgfqpoint{1.764706in}{1.700000in}}%
\pgfusepath{clip}%
\pgfsetbuttcap%
\pgfsetroundjoin%
\definecolor{currentfill}{rgb}{0.976961,0.885681,0.814303}%
\pgfsetfillcolor{currentfill}%
\pgfsetlinewidth{0.311001pt}%
\definecolor{currentstroke}{rgb}{1.000000,1.000000,1.000000}%
\pgfsetstrokecolor{currentstroke}%
\pgfsetdash{}{0pt}%
\pgfpathmoveto{\pgfqpoint{5.389043in}{1.247131in}}%
\pgfpathcurveto{\pgfqpoint{5.396176in}{1.247131in}}{\pgfqpoint{5.403018in}{1.249965in}}{\pgfqpoint{5.408061in}{1.255009in}}%
\pgfpathcurveto{\pgfqpoint{5.413105in}{1.260052in}}{\pgfqpoint{5.415939in}{1.266894in}}{\pgfqpoint{5.415939in}{1.274027in}}%
\pgfpathcurveto{\pgfqpoint{5.415939in}{1.281160in}}{\pgfqpoint{5.413105in}{1.288001in}}{\pgfqpoint{5.408061in}{1.293045in}}%
\pgfpathcurveto{\pgfqpoint{5.403018in}{1.298089in}}{\pgfqpoint{5.396176in}{1.300922in}}{\pgfqpoint{5.389043in}{1.300922in}}%
\pgfpathcurveto{\pgfqpoint{5.381910in}{1.300922in}}{\pgfqpoint{5.375069in}{1.298089in}}{\pgfqpoint{5.370025in}{1.293045in}}%
\pgfpathcurveto{\pgfqpoint{5.364981in}{1.288001in}}{\pgfqpoint{5.362147in}{1.281160in}}{\pgfqpoint{5.362147in}{1.274027in}}%
\pgfpathcurveto{\pgfqpoint{5.362147in}{1.266894in}}{\pgfqpoint{5.364981in}{1.260052in}}{\pgfqpoint{5.370025in}{1.255009in}}%
\pgfpathcurveto{\pgfqpoint{5.375069in}{1.249965in}}{\pgfqpoint{5.381910in}{1.247131in}}{\pgfqpoint{5.389043in}{1.247131in}}%
\pgfpathclose%
\pgfusepath{stroke,fill}%
\end{pgfscope}%
\begin{pgfscope}%
\pgfpathrectangle{\pgfqpoint{4.985294in}{0.500000in}}{\pgfqpoint{1.764706in}{1.700000in}}%
\pgfusepath{clip}%
\pgfsetbuttcap%
\pgfsetroundjoin%
\definecolor{currentfill}{rgb}{0.981377,0.920617,0.865369}%
\pgfsetfillcolor{currentfill}%
\pgfsetlinewidth{0.311001pt}%
\definecolor{currentstroke}{rgb}{1.000000,1.000000,1.000000}%
\pgfsetstrokecolor{currentstroke}%
\pgfsetdash{}{0pt}%
\pgfpathmoveto{\pgfqpoint{6.312653in}{1.431035in}}%
\pgfpathcurveto{\pgfqpoint{6.319786in}{1.431035in}}{\pgfqpoint{6.326627in}{1.433869in}}{\pgfqpoint{6.331671in}{1.438913in}}%
\pgfpathcurveto{\pgfqpoint{6.336715in}{1.443957in}}{\pgfqpoint{6.339549in}{1.450798in}}{\pgfqpoint{6.339549in}{1.457931in}}%
\pgfpathcurveto{\pgfqpoint{6.339549in}{1.465064in}}{\pgfqpoint{6.336715in}{1.471906in}}{\pgfqpoint{6.331671in}{1.476949in}}%
\pgfpathcurveto{\pgfqpoint{6.326627in}{1.481993in}}{\pgfqpoint{6.319786in}{1.484827in}}{\pgfqpoint{6.312653in}{1.484827in}}%
\pgfpathcurveto{\pgfqpoint{6.305520in}{1.484827in}}{\pgfqpoint{6.298679in}{1.481993in}}{\pgfqpoint{6.293635in}{1.476949in}}%
\pgfpathcurveto{\pgfqpoint{6.288591in}{1.471906in}}{\pgfqpoint{6.285757in}{1.465064in}}{\pgfqpoint{6.285757in}{1.457931in}}%
\pgfpathcurveto{\pgfqpoint{6.285757in}{1.450798in}}{\pgfqpoint{6.288591in}{1.443957in}}{\pgfqpoint{6.293635in}{1.438913in}}%
\pgfpathcurveto{\pgfqpoint{6.298679in}{1.433869in}}{\pgfqpoint{6.305520in}{1.431035in}}{\pgfqpoint{6.312653in}{1.431035in}}%
\pgfpathclose%
\pgfusepath{stroke,fill}%
\end{pgfscope}%
\begin{pgfscope}%
\pgfpathrectangle{\pgfqpoint{4.985294in}{0.500000in}}{\pgfqpoint{1.764706in}{1.700000in}}%
\pgfusepath{clip}%
\pgfsetbuttcap%
\pgfsetroundjoin%
\definecolor{currentfill}{rgb}{0.975018,0.868213,0.788710}%
\pgfsetfillcolor{currentfill}%
\pgfsetlinewidth{0.311001pt}%
\definecolor{currentstroke}{rgb}{1.000000,1.000000,1.000000}%
\pgfsetstrokecolor{currentstroke}%
\pgfsetdash{}{0pt}%
\pgfpathmoveto{\pgfqpoint{5.487919in}{1.103906in}}%
\pgfpathcurveto{\pgfqpoint{5.495052in}{1.103906in}}{\pgfqpoint{5.501893in}{1.106739in}}{\pgfqpoint{5.506937in}{1.111783in}}%
\pgfpathcurveto{\pgfqpoint{5.511981in}{1.116827in}}{\pgfqpoint{5.514815in}{1.123668in}}{\pgfqpoint{5.514815in}{1.130801in}}%
\pgfpathcurveto{\pgfqpoint{5.514815in}{1.137934in}}{\pgfqpoint{5.511981in}{1.144776in}}{\pgfqpoint{5.506937in}{1.149819in}}%
\pgfpathcurveto{\pgfqpoint{5.501893in}{1.154863in}}{\pgfqpoint{5.495052in}{1.157697in}}{\pgfqpoint{5.487919in}{1.157697in}}%
\pgfpathcurveto{\pgfqpoint{5.480786in}{1.157697in}}{\pgfqpoint{5.473944in}{1.154863in}}{\pgfqpoint{5.468901in}{1.149819in}}%
\pgfpathcurveto{\pgfqpoint{5.463857in}{1.144776in}}{\pgfqpoint{5.461023in}{1.137934in}}{\pgfqpoint{5.461023in}{1.130801in}}%
\pgfpathcurveto{\pgfqpoint{5.461023in}{1.123668in}}{\pgfqpoint{5.463857in}{1.116827in}}{\pgfqpoint{5.468901in}{1.111783in}}%
\pgfpathcurveto{\pgfqpoint{5.473944in}{1.106739in}}{\pgfqpoint{5.480786in}{1.103906in}}{\pgfqpoint{5.487919in}{1.103906in}}%
\pgfpathclose%
\pgfusepath{stroke,fill}%
\end{pgfscope}%
\begin{pgfscope}%
\pgfpathrectangle{\pgfqpoint{4.985294in}{0.500000in}}{\pgfqpoint{1.764706in}{1.700000in}}%
\pgfusepath{clip}%
\pgfsetbuttcap%
\pgfsetroundjoin%
\definecolor{currentfill}{rgb}{0.964679,0.682838,0.530002}%
\pgfsetfillcolor{currentfill}%
\pgfsetlinewidth{0.311001pt}%
\definecolor{currentstroke}{rgb}{1.000000,1.000000,1.000000}%
\pgfsetstrokecolor{currentstroke}%
\pgfsetdash{}{0pt}%
\pgfpathmoveto{\pgfqpoint{6.265431in}{1.703721in}}%
\pgfpathcurveto{\pgfqpoint{6.272564in}{1.703721in}}{\pgfqpoint{6.279406in}{1.706555in}}{\pgfqpoint{6.284450in}{1.711598in}}%
\pgfpathcurveto{\pgfqpoint{6.289493in}{1.716642in}}{\pgfqpoint{6.292327in}{1.723484in}}{\pgfqpoint{6.292327in}{1.730617in}}%
\pgfpathcurveto{\pgfqpoint{6.292327in}{1.737749in}}{\pgfqpoint{6.289493in}{1.744591in}}{\pgfqpoint{6.284450in}{1.749635in}}%
\pgfpathcurveto{\pgfqpoint{6.279406in}{1.754678in}}{\pgfqpoint{6.272564in}{1.757512in}}{\pgfqpoint{6.265431in}{1.757512in}}%
\pgfpathcurveto{\pgfqpoint{6.258299in}{1.757512in}}{\pgfqpoint{6.251457in}{1.754678in}}{\pgfqpoint{6.246413in}{1.749635in}}%
\pgfpathcurveto{\pgfqpoint{6.241370in}{1.744591in}}{\pgfqpoint{6.238536in}{1.737749in}}{\pgfqpoint{6.238536in}{1.730617in}}%
\pgfpathcurveto{\pgfqpoint{6.238536in}{1.723484in}}{\pgfqpoint{6.241370in}{1.716642in}}{\pgfqpoint{6.246413in}{1.711598in}}%
\pgfpathcurveto{\pgfqpoint{6.251457in}{1.706555in}}{\pgfqpoint{6.258299in}{1.703721in}}{\pgfqpoint{6.265431in}{1.703721in}}%
\pgfpathclose%
\pgfusepath{stroke,fill}%
\end{pgfscope}%
\begin{pgfscope}%
\pgfpathrectangle{\pgfqpoint{4.985294in}{0.500000in}}{\pgfqpoint{1.764706in}{1.700000in}}%
\pgfusepath{clip}%
\pgfsetbuttcap%
\pgfsetroundjoin%
\definecolor{currentfill}{rgb}{0.969359,0.803954,0.693832}%
\pgfsetfillcolor{currentfill}%
\pgfsetlinewidth{0.311001pt}%
\definecolor{currentstroke}{rgb}{1.000000,1.000000,1.000000}%
\pgfsetstrokecolor{currentstroke}%
\pgfsetdash{}{0pt}%
\pgfpathmoveto{\pgfqpoint{5.548452in}{0.970472in}}%
\pgfpathcurveto{\pgfqpoint{5.555584in}{0.970472in}}{\pgfqpoint{5.562426in}{0.973305in}}{\pgfqpoint{5.567470in}{0.978349in}}%
\pgfpathcurveto{\pgfqpoint{5.572513in}{0.983393in}}{\pgfqpoint{5.575347in}{0.990234in}}{\pgfqpoint{5.575347in}{0.997367in}}%
\pgfpathcurveto{\pgfqpoint{5.575347in}{1.004500in}}{\pgfqpoint{5.572513in}{1.011342in}}{\pgfqpoint{5.567470in}{1.016385in}}%
\pgfpathcurveto{\pgfqpoint{5.562426in}{1.021429in}}{\pgfqpoint{5.555584in}{1.024263in}}{\pgfqpoint{5.548452in}{1.024263in}}%
\pgfpathcurveto{\pgfqpoint{5.541319in}{1.024263in}}{\pgfqpoint{5.534477in}{1.021429in}}{\pgfqpoint{5.529433in}{1.016385in}}%
\pgfpathcurveto{\pgfqpoint{5.524390in}{1.011342in}}{\pgfqpoint{5.521556in}{1.004500in}}{\pgfqpoint{5.521556in}{0.997367in}}%
\pgfpathcurveto{\pgfqpoint{5.521556in}{0.990234in}}{\pgfqpoint{5.524390in}{0.983393in}}{\pgfqpoint{5.529433in}{0.978349in}}%
\pgfpathcurveto{\pgfqpoint{5.534477in}{0.973305in}}{\pgfqpoint{5.541319in}{0.970472in}}{\pgfqpoint{5.548452in}{0.970472in}}%
\pgfpathclose%
\pgfusepath{stroke,fill}%
\end{pgfscope}%
\begin{pgfscope}%
\pgfpathrectangle{\pgfqpoint{4.985294in}{0.500000in}}{\pgfqpoint{1.764706in}{1.700000in}}%
\pgfusepath{clip}%
\pgfsetbuttcap%
\pgfsetroundjoin%
\definecolor{currentfill}{rgb}{0.965302,0.713942,0.568499}%
\pgfsetfillcolor{currentfill}%
\pgfsetlinewidth{0.311001pt}%
\definecolor{currentstroke}{rgb}{1.000000,1.000000,1.000000}%
\pgfsetstrokecolor{currentstroke}%
\pgfsetdash{}{0pt}%
\pgfpathmoveto{\pgfqpoint{6.262549in}{0.958352in}}%
\pgfpathcurveto{\pgfqpoint{6.269682in}{0.958352in}}{\pgfqpoint{6.276523in}{0.961186in}}{\pgfqpoint{6.281567in}{0.966229in}}%
\pgfpathcurveto{\pgfqpoint{6.286611in}{0.971273in}}{\pgfqpoint{6.289445in}{0.978115in}}{\pgfqpoint{6.289445in}{0.985248in}}%
\pgfpathcurveto{\pgfqpoint{6.289445in}{0.992380in}}{\pgfqpoint{6.286611in}{0.999222in}}{\pgfqpoint{6.281567in}{1.004266in}}%
\pgfpathcurveto{\pgfqpoint{6.276523in}{1.009309in}}{\pgfqpoint{6.269682in}{1.012143in}}{\pgfqpoint{6.262549in}{1.012143in}}%
\pgfpathcurveto{\pgfqpoint{6.255416in}{1.012143in}}{\pgfqpoint{6.248575in}{1.009309in}}{\pgfqpoint{6.243531in}{1.004266in}}%
\pgfpathcurveto{\pgfqpoint{6.238487in}{0.999222in}}{\pgfqpoint{6.235653in}{0.992380in}}{\pgfqpoint{6.235653in}{0.985248in}}%
\pgfpathcurveto{\pgfqpoint{6.235653in}{0.978115in}}{\pgfqpoint{6.238487in}{0.971273in}}{\pgfqpoint{6.243531in}{0.966229in}}%
\pgfpathcurveto{\pgfqpoint{6.248575in}{0.961186in}}{\pgfqpoint{6.255416in}{0.958352in}}{\pgfqpoint{6.262549in}{0.958352in}}%
\pgfpathclose%
\pgfusepath{stroke,fill}%
\end{pgfscope}%
\begin{pgfscope}%
\pgfpathrectangle{\pgfqpoint{4.985294in}{0.500000in}}{\pgfqpoint{1.764706in}{1.700000in}}%
\pgfusepath{clip}%
\pgfsetbuttcap%
\pgfsetroundjoin%
\definecolor{currentfill}{rgb}{0.978376,0.897317,0.831308}%
\pgfsetfillcolor{currentfill}%
\pgfsetlinewidth{0.311001pt}%
\definecolor{currentstroke}{rgb}{1.000000,1.000000,1.000000}%
\pgfsetstrokecolor{currentstroke}%
\pgfsetdash{}{0pt}%
\pgfpathmoveto{\pgfqpoint{5.445482in}{1.341203in}}%
\pgfpathcurveto{\pgfqpoint{5.452614in}{1.341203in}}{\pgfqpoint{5.459456in}{1.344036in}}{\pgfqpoint{5.464500in}{1.349080in}}%
\pgfpathcurveto{\pgfqpoint{5.469543in}{1.354124in}}{\pgfqpoint{5.472377in}{1.360965in}}{\pgfqpoint{5.472377in}{1.368098in}}%
\pgfpathcurveto{\pgfqpoint{5.472377in}{1.375231in}}{\pgfqpoint{5.469543in}{1.382073in}}{\pgfqpoint{5.464500in}{1.387116in}}%
\pgfpathcurveto{\pgfqpoint{5.459456in}{1.392160in}}{\pgfqpoint{5.452614in}{1.394994in}}{\pgfqpoint{5.445482in}{1.394994in}}%
\pgfpathcurveto{\pgfqpoint{5.438349in}{1.394994in}}{\pgfqpoint{5.431507in}{1.392160in}}{\pgfqpoint{5.426463in}{1.387116in}}%
\pgfpathcurveto{\pgfqpoint{5.421420in}{1.382073in}}{\pgfqpoint{5.418586in}{1.375231in}}{\pgfqpoint{5.418586in}{1.368098in}}%
\pgfpathcurveto{\pgfqpoint{5.418586in}{1.360965in}}{\pgfqpoint{5.421420in}{1.354124in}}{\pgfqpoint{5.426463in}{1.349080in}}%
\pgfpathcurveto{\pgfqpoint{5.431507in}{1.344036in}}{\pgfqpoint{5.438349in}{1.341203in}}{\pgfqpoint{5.445482in}{1.341203in}}%
\pgfpathclose%
\pgfusepath{stroke,fill}%
\end{pgfscope}%
\begin{pgfscope}%
\pgfpathrectangle{\pgfqpoint{4.985294in}{0.500000in}}{\pgfqpoint{1.764706in}{1.700000in}}%
\pgfusepath{clip}%
\pgfsetbuttcap%
\pgfsetroundjoin%
\definecolor{currentfill}{rgb}{0.956268,0.491874,0.339856}%
\pgfsetfillcolor{currentfill}%
\pgfsetlinewidth{0.311001pt}%
\definecolor{currentstroke}{rgb}{1.000000,1.000000,1.000000}%
\pgfsetstrokecolor{currentstroke}%
\pgfsetdash{}{0pt}%
\pgfpathmoveto{\pgfqpoint{6.159394in}{1.322726in}}%
\pgfpathcurveto{\pgfqpoint{6.166526in}{1.322726in}}{\pgfqpoint{6.173368in}{1.325560in}}{\pgfqpoint{6.178412in}{1.330603in}}%
\pgfpathcurveto{\pgfqpoint{6.183455in}{1.335647in}}{\pgfqpoint{6.186289in}{1.342489in}}{\pgfqpoint{6.186289in}{1.349621in}}%
\pgfpathcurveto{\pgfqpoint{6.186289in}{1.356754in}}{\pgfqpoint{6.183455in}{1.363596in}}{\pgfqpoint{6.178412in}{1.368640in}}%
\pgfpathcurveto{\pgfqpoint{6.173368in}{1.373683in}}{\pgfqpoint{6.166526in}{1.376517in}}{\pgfqpoint{6.159394in}{1.376517in}}%
\pgfpathcurveto{\pgfqpoint{6.152261in}{1.376517in}}{\pgfqpoint{6.145419in}{1.373683in}}{\pgfqpoint{6.140375in}{1.368640in}}%
\pgfpathcurveto{\pgfqpoint{6.135332in}{1.363596in}}{\pgfqpoint{6.132498in}{1.356754in}}{\pgfqpoint{6.132498in}{1.349621in}}%
\pgfpathcurveto{\pgfqpoint{6.132498in}{1.342489in}}{\pgfqpoint{6.135332in}{1.335647in}}{\pgfqpoint{6.140375in}{1.330603in}}%
\pgfpathcurveto{\pgfqpoint{6.145419in}{1.325560in}}{\pgfqpoint{6.152261in}{1.322726in}}{\pgfqpoint{6.159394in}{1.322726in}}%
\pgfpathclose%
\pgfusepath{stroke,fill}%
\end{pgfscope}%
\begin{pgfscope}%
\pgfpathrectangle{\pgfqpoint{4.985294in}{0.500000in}}{\pgfqpoint{1.764706in}{1.700000in}}%
\pgfusepath{clip}%
\pgfsetbuttcap%
\pgfsetroundjoin%
\definecolor{currentfill}{rgb}{0.972201,0.839051,0.745789}%
\pgfsetfillcolor{currentfill}%
\pgfsetlinewidth{0.311001pt}%
\definecolor{currentstroke}{rgb}{1.000000,1.000000,1.000000}%
\pgfsetstrokecolor{currentstroke}%
\pgfsetdash{}{0pt}%
\pgfpathmoveto{\pgfqpoint{6.205731in}{1.024940in}}%
\pgfpathcurveto{\pgfqpoint{6.212864in}{1.024940in}}{\pgfqpoint{6.219705in}{1.027774in}}{\pgfqpoint{6.224749in}{1.032818in}}%
\pgfpathcurveto{\pgfqpoint{6.229793in}{1.037862in}}{\pgfqpoint{6.232627in}{1.044703in}}{\pgfqpoint{6.232627in}{1.051836in}}%
\pgfpathcurveto{\pgfqpoint{6.232627in}{1.058969in}}{\pgfqpoint{6.229793in}{1.065810in}}{\pgfqpoint{6.224749in}{1.070854in}}%
\pgfpathcurveto{\pgfqpoint{6.219705in}{1.075898in}}{\pgfqpoint{6.212864in}{1.078732in}}{\pgfqpoint{6.205731in}{1.078732in}}%
\pgfpathcurveto{\pgfqpoint{6.198598in}{1.078732in}}{\pgfqpoint{6.191756in}{1.075898in}}{\pgfqpoint{6.186713in}{1.070854in}}%
\pgfpathcurveto{\pgfqpoint{6.181669in}{1.065810in}}{\pgfqpoint{6.178835in}{1.058969in}}{\pgfqpoint{6.178835in}{1.051836in}}%
\pgfpathcurveto{\pgfqpoint{6.178835in}{1.044703in}}{\pgfqpoint{6.181669in}{1.037862in}}{\pgfqpoint{6.186713in}{1.032818in}}%
\pgfpathcurveto{\pgfqpoint{6.191756in}{1.027774in}}{\pgfqpoint{6.198598in}{1.024940in}}{\pgfqpoint{6.205731in}{1.024940in}}%
\pgfpathclose%
\pgfusepath{stroke,fill}%
\end{pgfscope}%
\begin{pgfscope}%
\pgfpathrectangle{\pgfqpoint{4.985294in}{0.500000in}}{\pgfqpoint{1.764706in}{1.700000in}}%
\pgfusepath{clip}%
\pgfsetbuttcap%
\pgfsetroundjoin%
\definecolor{currentfill}{rgb}{0.975644,0.874038,0.797253}%
\pgfsetfillcolor{currentfill}%
\pgfsetlinewidth{0.311001pt}%
\definecolor{currentstroke}{rgb}{1.000000,1.000000,1.000000}%
\pgfsetstrokecolor{currentstroke}%
\pgfsetdash{}{0pt}%
\pgfpathmoveto{\pgfqpoint{5.473584in}{1.433864in}}%
\pgfpathcurveto{\pgfqpoint{5.480717in}{1.433864in}}{\pgfqpoint{5.487559in}{1.436698in}}{\pgfqpoint{5.492602in}{1.441741in}}%
\pgfpathcurveto{\pgfqpoint{5.497646in}{1.446785in}}{\pgfqpoint{5.500480in}{1.453627in}}{\pgfqpoint{5.500480in}{1.460760in}}%
\pgfpathcurveto{\pgfqpoint{5.500480in}{1.467892in}}{\pgfqpoint{5.497646in}{1.474734in}}{\pgfqpoint{5.492602in}{1.479778in}}%
\pgfpathcurveto{\pgfqpoint{5.487559in}{1.484821in}}{\pgfqpoint{5.480717in}{1.487655in}}{\pgfqpoint{5.473584in}{1.487655in}}%
\pgfpathcurveto{\pgfqpoint{5.466451in}{1.487655in}}{\pgfqpoint{5.459610in}{1.484821in}}{\pgfqpoint{5.454566in}{1.479778in}}%
\pgfpathcurveto{\pgfqpoint{5.449523in}{1.474734in}}{\pgfqpoint{5.446689in}{1.467892in}}{\pgfqpoint{5.446689in}{1.460760in}}%
\pgfpathcurveto{\pgfqpoint{5.446689in}{1.453627in}}{\pgfqpoint{5.449523in}{1.446785in}}{\pgfqpoint{5.454566in}{1.441741in}}%
\pgfpathcurveto{\pgfqpoint{5.459610in}{1.436698in}}{\pgfqpoint{5.466451in}{1.433864in}}{\pgfqpoint{5.473584in}{1.433864in}}%
\pgfpathclose%
\pgfusepath{stroke,fill}%
\end{pgfscope}%
\begin{pgfscope}%
\pgfpathrectangle{\pgfqpoint{4.985294in}{0.500000in}}{\pgfqpoint{1.764706in}{1.700000in}}%
\pgfusepath{clip}%
\pgfsetbuttcap%
\pgfsetroundjoin%
\definecolor{currentfill}{rgb}{0.964799,0.689101,0.537560}%
\pgfsetfillcolor{currentfill}%
\pgfsetlinewidth{0.311001pt}%
\definecolor{currentstroke}{rgb}{1.000000,1.000000,1.000000}%
\pgfsetstrokecolor{currentstroke}%
\pgfsetdash{}{0pt}%
\pgfpathmoveto{\pgfqpoint{5.567826in}{0.896584in}}%
\pgfpathcurveto{\pgfqpoint{5.574959in}{0.896584in}}{\pgfqpoint{5.581800in}{0.899418in}}{\pgfqpoint{5.586844in}{0.904461in}}%
\pgfpathcurveto{\pgfqpoint{5.591888in}{0.909505in}}{\pgfqpoint{5.594722in}{0.916347in}}{\pgfqpoint{5.594722in}{0.923479in}}%
\pgfpathcurveto{\pgfqpoint{5.594722in}{0.930612in}}{\pgfqpoint{5.591888in}{0.937454in}}{\pgfqpoint{5.586844in}{0.942498in}}%
\pgfpathcurveto{\pgfqpoint{5.581800in}{0.947541in}}{\pgfqpoint{5.574959in}{0.950375in}}{\pgfqpoint{5.567826in}{0.950375in}}%
\pgfpathcurveto{\pgfqpoint{5.560693in}{0.950375in}}{\pgfqpoint{5.553851in}{0.947541in}}{\pgfqpoint{5.548808in}{0.942498in}}%
\pgfpathcurveto{\pgfqpoint{5.543764in}{0.937454in}}{\pgfqpoint{5.540930in}{0.930612in}}{\pgfqpoint{5.540930in}{0.923479in}}%
\pgfpathcurveto{\pgfqpoint{5.540930in}{0.916347in}}{\pgfqpoint{5.543764in}{0.909505in}}{\pgfqpoint{5.548808in}{0.904461in}}%
\pgfpathcurveto{\pgfqpoint{5.553851in}{0.899418in}}{\pgfqpoint{5.560693in}{0.896584in}}{\pgfqpoint{5.567826in}{0.896584in}}%
\pgfpathclose%
\pgfusepath{stroke,fill}%
\end{pgfscope}%
\begin{pgfscope}%
\pgfpathrectangle{\pgfqpoint{4.985294in}{0.500000in}}{\pgfqpoint{1.764706in}{1.700000in}}%
\pgfusepath{clip}%
\pgfsetbuttcap%
\pgfsetroundjoin%
\definecolor{currentfill}{rgb}{0.975018,0.868213,0.788710}%
\pgfsetfillcolor{currentfill}%
\pgfsetlinewidth{0.311001pt}%
\definecolor{currentstroke}{rgb}{1.000000,1.000000,1.000000}%
\pgfsetstrokecolor{currentstroke}%
\pgfsetdash{}{0pt}%
\pgfpathmoveto{\pgfqpoint{6.270323in}{1.390018in}}%
\pgfpathcurveto{\pgfqpoint{6.277456in}{1.390018in}}{\pgfqpoint{6.284298in}{1.392852in}}{\pgfqpoint{6.289342in}{1.397895in}}%
\pgfpathcurveto{\pgfqpoint{6.294385in}{1.402939in}}{\pgfqpoint{6.297219in}{1.409781in}}{\pgfqpoint{6.297219in}{1.416913in}}%
\pgfpathcurveto{\pgfqpoint{6.297219in}{1.424046in}}{\pgfqpoint{6.294385in}{1.430888in}}{\pgfqpoint{6.289342in}{1.435932in}}%
\pgfpathcurveto{\pgfqpoint{6.284298in}{1.440975in}}{\pgfqpoint{6.277456in}{1.443809in}}{\pgfqpoint{6.270323in}{1.443809in}}%
\pgfpathcurveto{\pgfqpoint{6.263191in}{1.443809in}}{\pgfqpoint{6.256349in}{1.440975in}}{\pgfqpoint{6.251305in}{1.435932in}}%
\pgfpathcurveto{\pgfqpoint{6.246262in}{1.430888in}}{\pgfqpoint{6.243428in}{1.424046in}}{\pgfqpoint{6.243428in}{1.416913in}}%
\pgfpathcurveto{\pgfqpoint{6.243428in}{1.409781in}}{\pgfqpoint{6.246262in}{1.402939in}}{\pgfqpoint{6.251305in}{1.397895in}}%
\pgfpathcurveto{\pgfqpoint{6.256349in}{1.392852in}}{\pgfqpoint{6.263191in}{1.390018in}}{\pgfqpoint{6.270323in}{1.390018in}}%
\pgfpathclose%
\pgfusepath{stroke,fill}%
\end{pgfscope}%
\begin{pgfscope}%
\pgfpathrectangle{\pgfqpoint{4.985294in}{0.500000in}}{\pgfqpoint{1.764706in}{1.700000in}}%
\pgfusepath{clip}%
\pgfsetbuttcap%
\pgfsetroundjoin%
\definecolor{currentfill}{rgb}{0.981377,0.920617,0.865369}%
\pgfsetfillcolor{currentfill}%
\pgfsetlinewidth{0.311001pt}%
\definecolor{currentstroke}{rgb}{1.000000,1.000000,1.000000}%
\pgfsetstrokecolor{currentstroke}%
\pgfsetdash{}{0pt}%
\pgfpathmoveto{\pgfqpoint{6.302868in}{1.278694in}}%
\pgfpathcurveto{\pgfqpoint{6.310001in}{1.278694in}}{\pgfqpoint{6.316843in}{1.281528in}}{\pgfqpoint{6.321886in}{1.286572in}}%
\pgfpathcurveto{\pgfqpoint{6.326930in}{1.291615in}}{\pgfqpoint{6.329764in}{1.298457in}}{\pgfqpoint{6.329764in}{1.305590in}}%
\pgfpathcurveto{\pgfqpoint{6.329764in}{1.312723in}}{\pgfqpoint{6.326930in}{1.319564in}}{\pgfqpoint{6.321886in}{1.324608in}}%
\pgfpathcurveto{\pgfqpoint{6.316843in}{1.329652in}}{\pgfqpoint{6.310001in}{1.332486in}}{\pgfqpoint{6.302868in}{1.332486in}}%
\pgfpathcurveto{\pgfqpoint{6.295736in}{1.332486in}}{\pgfqpoint{6.288894in}{1.329652in}}{\pgfqpoint{6.283850in}{1.324608in}}%
\pgfpathcurveto{\pgfqpoint{6.278807in}{1.319564in}}{\pgfqpoint{6.275973in}{1.312723in}}{\pgfqpoint{6.275973in}{1.305590in}}%
\pgfpathcurveto{\pgfqpoint{6.275973in}{1.298457in}}{\pgfqpoint{6.278807in}{1.291615in}}{\pgfqpoint{6.283850in}{1.286572in}}%
\pgfpathcurveto{\pgfqpoint{6.288894in}{1.281528in}}{\pgfqpoint{6.295736in}{1.278694in}}{\pgfqpoint{6.302868in}{1.278694in}}%
\pgfpathclose%
\pgfusepath{stroke,fill}%
\end{pgfscope}%
\begin{pgfscope}%
\pgfpathrectangle{\pgfqpoint{4.985294in}{0.500000in}}{\pgfqpoint{1.764706in}{1.700000in}}%
\pgfusepath{clip}%
\pgfsetbuttcap%
\pgfsetroundjoin%
\definecolor{currentfill}{rgb}{0.980678,0.914765,0.856766}%
\pgfsetfillcolor{currentfill}%
\pgfsetlinewidth{0.311001pt}%
\definecolor{currentstroke}{rgb}{1.000000,1.000000,1.000000}%
\pgfsetstrokecolor{currentstroke}%
\pgfsetdash{}{0pt}%
\pgfpathmoveto{\pgfqpoint{5.414749in}{1.326319in}}%
\pgfpathcurveto{\pgfqpoint{5.421882in}{1.326319in}}{\pgfqpoint{5.428724in}{1.329153in}}{\pgfqpoint{5.433767in}{1.334197in}}%
\pgfpathcurveto{\pgfqpoint{5.438811in}{1.339240in}}{\pgfqpoint{5.441645in}{1.346082in}}{\pgfqpoint{5.441645in}{1.353215in}}%
\pgfpathcurveto{\pgfqpoint{5.441645in}{1.360348in}}{\pgfqpoint{5.438811in}{1.367189in}}{\pgfqpoint{5.433767in}{1.372233in}}%
\pgfpathcurveto{\pgfqpoint{5.428724in}{1.377277in}}{\pgfqpoint{5.421882in}{1.380110in}}{\pgfqpoint{5.414749in}{1.380110in}}%
\pgfpathcurveto{\pgfqpoint{5.407616in}{1.380110in}}{\pgfqpoint{5.400775in}{1.377277in}}{\pgfqpoint{5.395731in}{1.372233in}}%
\pgfpathcurveto{\pgfqpoint{5.390687in}{1.367189in}}{\pgfqpoint{5.387853in}{1.360348in}}{\pgfqpoint{5.387853in}{1.353215in}}%
\pgfpathcurveto{\pgfqpoint{5.387853in}{1.346082in}}{\pgfqpoint{5.390687in}{1.339240in}}{\pgfqpoint{5.395731in}{1.334197in}}%
\pgfpathcurveto{\pgfqpoint{5.400775in}{1.329153in}}{\pgfqpoint{5.407616in}{1.326319in}}{\pgfqpoint{5.414749in}{1.326319in}}%
\pgfpathclose%
\pgfusepath{stroke,fill}%
\end{pgfscope}%
\begin{pgfscope}%
\pgfpathrectangle{\pgfqpoint{4.985294in}{0.500000in}}{\pgfqpoint{1.764706in}{1.700000in}}%
\pgfusepath{clip}%
\pgfsetbuttcap%
\pgfsetroundjoin%
\definecolor{currentfill}{rgb}{0.974412,0.862387,0.780156}%
\pgfsetfillcolor{currentfill}%
\pgfsetlinewidth{0.311001pt}%
\definecolor{currentstroke}{rgb}{1.000000,1.000000,1.000000}%
\pgfsetstrokecolor{currentstroke}%
\pgfsetdash{}{0pt}%
\pgfpathmoveto{\pgfqpoint{6.359830in}{1.387703in}}%
\pgfpathcurveto{\pgfqpoint{6.366963in}{1.387703in}}{\pgfqpoint{6.373805in}{1.390537in}}{\pgfqpoint{6.378848in}{1.395581in}}%
\pgfpathcurveto{\pgfqpoint{6.383892in}{1.400625in}}{\pgfqpoint{6.386726in}{1.407466in}}{\pgfqpoint{6.386726in}{1.414599in}}%
\pgfpathcurveto{\pgfqpoint{6.386726in}{1.421732in}}{\pgfqpoint{6.383892in}{1.428574in}}{\pgfqpoint{6.378848in}{1.433617in}}%
\pgfpathcurveto{\pgfqpoint{6.373805in}{1.438661in}}{\pgfqpoint{6.366963in}{1.441495in}}{\pgfqpoint{6.359830in}{1.441495in}}%
\pgfpathcurveto{\pgfqpoint{6.352697in}{1.441495in}}{\pgfqpoint{6.345856in}{1.438661in}}{\pgfqpoint{6.340812in}{1.433617in}}%
\pgfpathcurveto{\pgfqpoint{6.335768in}{1.428574in}}{\pgfqpoint{6.332934in}{1.421732in}}{\pgfqpoint{6.332934in}{1.414599in}}%
\pgfpathcurveto{\pgfqpoint{6.332934in}{1.407466in}}{\pgfqpoint{6.335768in}{1.400625in}}{\pgfqpoint{6.340812in}{1.395581in}}%
\pgfpathcurveto{\pgfqpoint{6.345856in}{1.390537in}}{\pgfqpoint{6.352697in}{1.387703in}}{\pgfqpoint{6.359830in}{1.387703in}}%
\pgfpathclose%
\pgfusepath{stroke,fill}%
\end{pgfscope}%
\begin{pgfscope}%
\pgfpathrectangle{\pgfqpoint{4.985294in}{0.500000in}}{\pgfqpoint{1.764706in}{1.700000in}}%
\pgfusepath{clip}%
\pgfsetbuttcap%
\pgfsetroundjoin%
\definecolor{currentfill}{rgb}{0.970718,0.821518,0.719872}%
\pgfsetfillcolor{currentfill}%
\pgfsetlinewidth{0.311001pt}%
\definecolor{currentstroke}{rgb}{1.000000,1.000000,1.000000}%
\pgfsetstrokecolor{currentstroke}%
\pgfsetdash{}{0pt}%
\pgfpathmoveto{\pgfqpoint{6.268203in}{1.649919in}}%
\pgfpathcurveto{\pgfqpoint{6.275336in}{1.649919in}}{\pgfqpoint{6.282178in}{1.652753in}}{\pgfqpoint{6.287222in}{1.657796in}}%
\pgfpathcurveto{\pgfqpoint{6.292265in}{1.662840in}}{\pgfqpoint{6.295099in}{1.669682in}}{\pgfqpoint{6.295099in}{1.676814in}}%
\pgfpathcurveto{\pgfqpoint{6.295099in}{1.683947in}}{\pgfqpoint{6.292265in}{1.690789in}}{\pgfqpoint{6.287222in}{1.695833in}}%
\pgfpathcurveto{\pgfqpoint{6.282178in}{1.700876in}}{\pgfqpoint{6.275336in}{1.703710in}}{\pgfqpoint{6.268203in}{1.703710in}}%
\pgfpathcurveto{\pgfqpoint{6.261071in}{1.703710in}}{\pgfqpoint{6.254229in}{1.700876in}}{\pgfqpoint{6.249185in}{1.695833in}}%
\pgfpathcurveto{\pgfqpoint{6.244142in}{1.690789in}}{\pgfqpoint{6.241308in}{1.683947in}}{\pgfqpoint{6.241308in}{1.676814in}}%
\pgfpathcurveto{\pgfqpoint{6.241308in}{1.669682in}}{\pgfqpoint{6.244142in}{1.662840in}}{\pgfqpoint{6.249185in}{1.657796in}}%
\pgfpathcurveto{\pgfqpoint{6.254229in}{1.652753in}}{\pgfqpoint{6.261071in}{1.649919in}}{\pgfqpoint{6.268203in}{1.649919in}}%
\pgfpathclose%
\pgfusepath{stroke,fill}%
\end{pgfscope}%
\begin{pgfscope}%
\pgfpathrectangle{\pgfqpoint{4.985294in}{0.500000in}}{\pgfqpoint{1.764706in}{1.700000in}}%
\pgfusepath{clip}%
\pgfsetbuttcap%
\pgfsetroundjoin%
\definecolor{currentfill}{rgb}{0.979124,0.903132,0.839793}%
\pgfsetfillcolor{currentfill}%
\pgfsetlinewidth{0.311001pt}%
\definecolor{currentstroke}{rgb}{1.000000,1.000000,1.000000}%
\pgfsetstrokecolor{currentstroke}%
\pgfsetdash{}{0pt}%
\pgfpathmoveto{\pgfqpoint{6.343662in}{1.289115in}}%
\pgfpathcurveto{\pgfqpoint{6.350794in}{1.289115in}}{\pgfqpoint{6.357636in}{1.291948in}}{\pgfqpoint{6.362680in}{1.296992in}}%
\pgfpathcurveto{\pgfqpoint{6.367723in}{1.302036in}}{\pgfqpoint{6.370557in}{1.308877in}}{\pgfqpoint{6.370557in}{1.316010in}}%
\pgfpathcurveto{\pgfqpoint{6.370557in}{1.323143in}}{\pgfqpoint{6.367723in}{1.329985in}}{\pgfqpoint{6.362680in}{1.335028in}}%
\pgfpathcurveto{\pgfqpoint{6.357636in}{1.340072in}}{\pgfqpoint{6.350794in}{1.342906in}}{\pgfqpoint{6.343662in}{1.342906in}}%
\pgfpathcurveto{\pgfqpoint{6.336529in}{1.342906in}}{\pgfqpoint{6.329687in}{1.340072in}}{\pgfqpoint{6.324643in}{1.335028in}}%
\pgfpathcurveto{\pgfqpoint{6.319600in}{1.329985in}}{\pgfqpoint{6.316766in}{1.323143in}}{\pgfqpoint{6.316766in}{1.316010in}}%
\pgfpathcurveto{\pgfqpoint{6.316766in}{1.308877in}}{\pgfqpoint{6.319600in}{1.302036in}}{\pgfqpoint{6.324643in}{1.296992in}}%
\pgfpathcurveto{\pgfqpoint{6.329687in}{1.291948in}}{\pgfqpoint{6.336529in}{1.289115in}}{\pgfqpoint{6.343662in}{1.289115in}}%
\pgfpathclose%
\pgfusepath{stroke,fill}%
\end{pgfscope}%
\begin{pgfscope}%
\pgfpathrectangle{\pgfqpoint{4.985294in}{0.500000in}}{\pgfqpoint{1.764706in}{1.700000in}}%
\pgfusepath{clip}%
\pgfsetbuttcap%
\pgfsetroundjoin%
\definecolor{currentfill}{rgb}{0.981377,0.920617,0.865369}%
\pgfsetfillcolor{currentfill}%
\pgfsetlinewidth{0.311001pt}%
\definecolor{currentstroke}{rgb}{1.000000,1.000000,1.000000}%
\pgfsetstrokecolor{currentstroke}%
\pgfsetdash{}{0pt}%
\pgfpathmoveto{\pgfqpoint{6.326505in}{1.358824in}}%
\pgfpathcurveto{\pgfqpoint{6.333638in}{1.358824in}}{\pgfqpoint{6.340479in}{1.361658in}}{\pgfqpoint{6.345523in}{1.366702in}}%
\pgfpathcurveto{\pgfqpoint{6.350567in}{1.371746in}}{\pgfqpoint{6.353400in}{1.378587in}}{\pgfqpoint{6.353400in}{1.385720in}}%
\pgfpathcurveto{\pgfqpoint{6.353400in}{1.392853in}}{\pgfqpoint{6.350567in}{1.399695in}}{\pgfqpoint{6.345523in}{1.404738in}}%
\pgfpathcurveto{\pgfqpoint{6.340479in}{1.409782in}}{\pgfqpoint{6.333638in}{1.412616in}}{\pgfqpoint{6.326505in}{1.412616in}}%
\pgfpathcurveto{\pgfqpoint{6.319372in}{1.412616in}}{\pgfqpoint{6.312530in}{1.409782in}}{\pgfqpoint{6.307487in}{1.404738in}}%
\pgfpathcurveto{\pgfqpoint{6.302443in}{1.399695in}}{\pgfqpoint{6.299609in}{1.392853in}}{\pgfqpoint{6.299609in}{1.385720in}}%
\pgfpathcurveto{\pgfqpoint{6.299609in}{1.378587in}}{\pgfqpoint{6.302443in}{1.371746in}}{\pgfqpoint{6.307487in}{1.366702in}}%
\pgfpathcurveto{\pgfqpoint{6.312530in}{1.361658in}}{\pgfqpoint{6.319372in}{1.358824in}}{\pgfqpoint{6.326505in}{1.358824in}}%
\pgfpathclose%
\pgfusepath{stroke,fill}%
\end{pgfscope}%
\begin{pgfscope}%
\pgfpathrectangle{\pgfqpoint{4.985294in}{0.500000in}}{\pgfqpoint{1.764706in}{1.700000in}}%
\pgfusepath{clip}%
\pgfsetbuttcap%
\pgfsetroundjoin%
\definecolor{currentfill}{rgb}{0.972201,0.839051,0.745789}%
\pgfsetfillcolor{currentfill}%
\pgfsetlinewidth{0.311001pt}%
\definecolor{currentstroke}{rgb}{1.000000,1.000000,1.000000}%
\pgfsetstrokecolor{currentstroke}%
\pgfsetdash{}{0pt}%
\pgfpathmoveto{\pgfqpoint{6.232277in}{1.205267in}}%
\pgfpathcurveto{\pgfqpoint{6.239410in}{1.205267in}}{\pgfqpoint{6.246251in}{1.208100in}}{\pgfqpoint{6.251295in}{1.213144in}}%
\pgfpathcurveto{\pgfqpoint{6.256339in}{1.218188in}}{\pgfqpoint{6.259173in}{1.225029in}}{\pgfqpoint{6.259173in}{1.232162in}}%
\pgfpathcurveto{\pgfqpoint{6.259173in}{1.239295in}}{\pgfqpoint{6.256339in}{1.246137in}}{\pgfqpoint{6.251295in}{1.251180in}}%
\pgfpathcurveto{\pgfqpoint{6.246251in}{1.256224in}}{\pgfqpoint{6.239410in}{1.259058in}}{\pgfqpoint{6.232277in}{1.259058in}}%
\pgfpathcurveto{\pgfqpoint{6.225144in}{1.259058in}}{\pgfqpoint{6.218302in}{1.256224in}}{\pgfqpoint{6.213259in}{1.251180in}}%
\pgfpathcurveto{\pgfqpoint{6.208215in}{1.246137in}}{\pgfqpoint{6.205381in}{1.239295in}}{\pgfqpoint{6.205381in}{1.232162in}}%
\pgfpathcurveto{\pgfqpoint{6.205381in}{1.225029in}}{\pgfqpoint{6.208215in}{1.218188in}}{\pgfqpoint{6.213259in}{1.213144in}}%
\pgfpathcurveto{\pgfqpoint{6.218302in}{1.208100in}}{\pgfqpoint{6.225144in}{1.205267in}}{\pgfqpoint{6.232277in}{1.205267in}}%
\pgfpathclose%
\pgfusepath{stroke,fill}%
\end{pgfscope}%
\begin{pgfscope}%
\pgfpathrectangle{\pgfqpoint{4.985294in}{0.500000in}}{\pgfqpoint{1.764706in}{1.700000in}}%
\pgfusepath{clip}%
\pgfsetbuttcap%
\pgfsetroundjoin%
\definecolor{currentfill}{rgb}{0.970718,0.821518,0.719872}%
\pgfsetfillcolor{currentfill}%
\pgfsetlinewidth{0.311001pt}%
\definecolor{currentstroke}{rgb}{1.000000,1.000000,1.000000}%
\pgfsetstrokecolor{currentstroke}%
\pgfsetdash{}{0pt}%
\pgfpathmoveto{\pgfqpoint{5.524604in}{1.080908in}}%
\pgfpathcurveto{\pgfqpoint{5.531737in}{1.080908in}}{\pgfqpoint{5.538579in}{1.083742in}}{\pgfqpoint{5.543622in}{1.088786in}}%
\pgfpathcurveto{\pgfqpoint{5.548666in}{1.093829in}}{\pgfqpoint{5.551500in}{1.100671in}}{\pgfqpoint{5.551500in}{1.107804in}}%
\pgfpathcurveto{\pgfqpoint{5.551500in}{1.114937in}}{\pgfqpoint{5.548666in}{1.121778in}}{\pgfqpoint{5.543622in}{1.126822in}}%
\pgfpathcurveto{\pgfqpoint{5.538579in}{1.131866in}}{\pgfqpoint{5.531737in}{1.134700in}}{\pgfqpoint{5.524604in}{1.134700in}}%
\pgfpathcurveto{\pgfqpoint{5.517471in}{1.134700in}}{\pgfqpoint{5.510630in}{1.131866in}}{\pgfqpoint{5.505586in}{1.126822in}}%
\pgfpathcurveto{\pgfqpoint{5.500542in}{1.121778in}}{\pgfqpoint{5.497709in}{1.114937in}}{\pgfqpoint{5.497709in}{1.107804in}}%
\pgfpathcurveto{\pgfqpoint{5.497709in}{1.100671in}}{\pgfqpoint{5.500542in}{1.093829in}}{\pgfqpoint{5.505586in}{1.088786in}}%
\pgfpathcurveto{\pgfqpoint{5.510630in}{1.083742in}}{\pgfqpoint{5.517471in}{1.080908in}}{\pgfqpoint{5.524604in}{1.080908in}}%
\pgfpathclose%
\pgfusepath{stroke,fill}%
\end{pgfscope}%
\begin{pgfscope}%
\pgfpathrectangle{\pgfqpoint{4.985294in}{0.500000in}}{\pgfqpoint{1.764706in}{1.700000in}}%
\pgfusepath{clip}%
\pgfsetbuttcap%
\pgfsetroundjoin%
\definecolor{currentfill}{rgb}{0.980678,0.914765,0.856766}%
\pgfsetfillcolor{currentfill}%
\pgfsetlinewidth{0.311001pt}%
\definecolor{currentstroke}{rgb}{1.000000,1.000000,1.000000}%
\pgfsetstrokecolor{currentstroke}%
\pgfsetdash{}{0pt}%
\pgfpathmoveto{\pgfqpoint{6.294584in}{1.414232in}}%
\pgfpathcurveto{\pgfqpoint{6.301717in}{1.414232in}}{\pgfqpoint{6.308559in}{1.417066in}}{\pgfqpoint{6.313603in}{1.422110in}}%
\pgfpathcurveto{\pgfqpoint{6.318646in}{1.427154in}}{\pgfqpoint{6.321480in}{1.433995in}}{\pgfqpoint{6.321480in}{1.441128in}}%
\pgfpathcurveto{\pgfqpoint{6.321480in}{1.448261in}}{\pgfqpoint{6.318646in}{1.455103in}}{\pgfqpoint{6.313603in}{1.460146in}}%
\pgfpathcurveto{\pgfqpoint{6.308559in}{1.465190in}}{\pgfqpoint{6.301717in}{1.468024in}}{\pgfqpoint{6.294584in}{1.468024in}}%
\pgfpathcurveto{\pgfqpoint{6.287452in}{1.468024in}}{\pgfqpoint{6.280610in}{1.465190in}}{\pgfqpoint{6.275566in}{1.460146in}}%
\pgfpathcurveto{\pgfqpoint{6.270523in}{1.455103in}}{\pgfqpoint{6.267689in}{1.448261in}}{\pgfqpoint{6.267689in}{1.441128in}}%
\pgfpathcurveto{\pgfqpoint{6.267689in}{1.433995in}}{\pgfqpoint{6.270523in}{1.427154in}}{\pgfqpoint{6.275566in}{1.422110in}}%
\pgfpathcurveto{\pgfqpoint{6.280610in}{1.417066in}}{\pgfqpoint{6.287452in}{1.414232in}}{\pgfqpoint{6.294584in}{1.414232in}}%
\pgfpathclose%
\pgfusepath{stroke,fill}%
\end{pgfscope}%
\begin{pgfscope}%
\pgfpathrectangle{\pgfqpoint{4.985294in}{0.500000in}}{\pgfqpoint{1.764706in}{1.700000in}}%
\pgfusepath{clip}%
\pgfsetbuttcap%
\pgfsetroundjoin%
\definecolor{currentfill}{rgb}{0.960421,0.553286,0.393191}%
\pgfsetfillcolor{currentfill}%
\pgfsetlinewidth{0.311001pt}%
\definecolor{currentstroke}{rgb}{1.000000,1.000000,1.000000}%
\pgfsetstrokecolor{currentstroke}%
\pgfsetdash{}{0pt}%
\pgfpathmoveto{\pgfqpoint{6.301195in}{0.935145in}}%
\pgfpathcurveto{\pgfqpoint{6.308328in}{0.935145in}}{\pgfqpoint{6.315169in}{0.937979in}}{\pgfqpoint{6.320213in}{0.943023in}}%
\pgfpathcurveto{\pgfqpoint{6.325257in}{0.948066in}}{\pgfqpoint{6.328091in}{0.954908in}}{\pgfqpoint{6.328091in}{0.962041in}}%
\pgfpathcurveto{\pgfqpoint{6.328091in}{0.969174in}}{\pgfqpoint{6.325257in}{0.976015in}}{\pgfqpoint{6.320213in}{0.981059in}}%
\pgfpathcurveto{\pgfqpoint{6.315169in}{0.986103in}}{\pgfqpoint{6.308328in}{0.988936in}}{\pgfqpoint{6.301195in}{0.988936in}}%
\pgfpathcurveto{\pgfqpoint{6.294062in}{0.988936in}}{\pgfqpoint{6.287220in}{0.986103in}}{\pgfqpoint{6.282177in}{0.981059in}}%
\pgfpathcurveto{\pgfqpoint{6.277133in}{0.976015in}}{\pgfqpoint{6.274299in}{0.969174in}}{\pgfqpoint{6.274299in}{0.962041in}}%
\pgfpathcurveto{\pgfqpoint{6.274299in}{0.954908in}}{\pgfqpoint{6.277133in}{0.948066in}}{\pgfqpoint{6.282177in}{0.943023in}}%
\pgfpathcurveto{\pgfqpoint{6.287220in}{0.937979in}}{\pgfqpoint{6.294062in}{0.935145in}}{\pgfqpoint{6.301195in}{0.935145in}}%
\pgfpathclose%
\pgfusepath{stroke,fill}%
\end{pgfscope}%
\begin{pgfscope}%
\pgfpathrectangle{\pgfqpoint{4.985294in}{0.500000in}}{\pgfqpoint{1.764706in}{1.700000in}}%
\pgfusepath{clip}%
\pgfsetbuttcap%
\pgfsetroundjoin%
\definecolor{currentfill}{rgb}{0.975644,0.874038,0.797253}%
\pgfsetfillcolor{currentfill}%
\pgfsetlinewidth{0.311001pt}%
\definecolor{currentstroke}{rgb}{1.000000,1.000000,1.000000}%
\pgfsetstrokecolor{currentstroke}%
\pgfsetdash{}{0pt}%
\pgfpathmoveto{\pgfqpoint{5.376944in}{1.364966in}}%
\pgfpathcurveto{\pgfqpoint{5.384077in}{1.364966in}}{\pgfqpoint{5.390918in}{1.367800in}}{\pgfqpoint{5.395962in}{1.372843in}}%
\pgfpathcurveto{\pgfqpoint{5.401006in}{1.377887in}}{\pgfqpoint{5.403840in}{1.384729in}}{\pgfqpoint{5.403840in}{1.391861in}}%
\pgfpathcurveto{\pgfqpoint{5.403840in}{1.398994in}}{\pgfqpoint{5.401006in}{1.405836in}}{\pgfqpoint{5.395962in}{1.410880in}}%
\pgfpathcurveto{\pgfqpoint{5.390918in}{1.415923in}}{\pgfqpoint{5.384077in}{1.418757in}}{\pgfqpoint{5.376944in}{1.418757in}}%
\pgfpathcurveto{\pgfqpoint{5.369811in}{1.418757in}}{\pgfqpoint{5.362969in}{1.415923in}}{\pgfqpoint{5.357926in}{1.410880in}}%
\pgfpathcurveto{\pgfqpoint{5.352882in}{1.405836in}}{\pgfqpoint{5.350048in}{1.398994in}}{\pgfqpoint{5.350048in}{1.391861in}}%
\pgfpathcurveto{\pgfqpoint{5.350048in}{1.384729in}}{\pgfqpoint{5.352882in}{1.377887in}}{\pgfqpoint{5.357926in}{1.372843in}}%
\pgfpathcurveto{\pgfqpoint{5.362969in}{1.367800in}}{\pgfqpoint{5.369811in}{1.364966in}}{\pgfqpoint{5.376944in}{1.364966in}}%
\pgfpathclose%
\pgfusepath{stroke,fill}%
\end{pgfscope}%
\begin{pgfscope}%
\pgfpathrectangle{\pgfqpoint{4.985294in}{0.500000in}}{\pgfqpoint{1.764706in}{1.700000in}}%
\pgfusepath{clip}%
\pgfsetbuttcap%
\pgfsetroundjoin%
\definecolor{currentfill}{rgb}{0.950017,0.427714,0.292447}%
\pgfsetfillcolor{currentfill}%
\pgfsetlinewidth{0.311001pt}%
\definecolor{currentstroke}{rgb}{1.000000,1.000000,1.000000}%
\pgfsetstrokecolor{currentstroke}%
\pgfsetdash{}{0pt}%
\pgfpathmoveto{\pgfqpoint{5.578639in}{1.237343in}}%
\pgfpathcurveto{\pgfqpoint{5.585772in}{1.237343in}}{\pgfqpoint{5.592614in}{1.240177in}}{\pgfqpoint{5.597657in}{1.245220in}}%
\pgfpathcurveto{\pgfqpoint{5.602701in}{1.250264in}}{\pgfqpoint{5.605535in}{1.257106in}}{\pgfqpoint{5.605535in}{1.264238in}}%
\pgfpathcurveto{\pgfqpoint{5.605535in}{1.271371in}}{\pgfqpoint{5.602701in}{1.278213in}}{\pgfqpoint{5.597657in}{1.283257in}}%
\pgfpathcurveto{\pgfqpoint{5.592614in}{1.288300in}}{\pgfqpoint{5.585772in}{1.291134in}}{\pgfqpoint{5.578639in}{1.291134in}}%
\pgfpathcurveto{\pgfqpoint{5.571506in}{1.291134in}}{\pgfqpoint{5.564665in}{1.288300in}}{\pgfqpoint{5.559621in}{1.283257in}}%
\pgfpathcurveto{\pgfqpoint{5.554577in}{1.278213in}}{\pgfqpoint{5.551743in}{1.271371in}}{\pgfqpoint{5.551743in}{1.264238in}}%
\pgfpathcurveto{\pgfqpoint{5.551743in}{1.257106in}}{\pgfqpoint{5.554577in}{1.250264in}}{\pgfqpoint{5.559621in}{1.245220in}}%
\pgfpathcurveto{\pgfqpoint{5.564665in}{1.240177in}}{\pgfqpoint{5.571506in}{1.237343in}}{\pgfqpoint{5.578639in}{1.237343in}}%
\pgfpathclose%
\pgfusepath{stroke,fill}%
\end{pgfscope}%
\begin{pgfscope}%
\pgfpathrectangle{\pgfqpoint{4.985294in}{0.500000in}}{\pgfqpoint{1.764706in}{1.700000in}}%
\pgfusepath{clip}%
\pgfsetbuttcap%
\pgfsetroundjoin%
\definecolor{currentfill}{rgb}{0.965440,0.720101,0.576404}%
\pgfsetfillcolor{currentfill}%
\pgfsetlinewidth{0.311001pt}%
\definecolor{currentstroke}{rgb}{1.000000,1.000000,1.000000}%
\pgfsetstrokecolor{currentstroke}%
\pgfsetdash{}{0pt}%
\pgfpathmoveto{\pgfqpoint{5.519602in}{1.260575in}}%
\pgfpathcurveto{\pgfqpoint{5.526735in}{1.260575in}}{\pgfqpoint{5.533576in}{1.263409in}}{\pgfqpoint{5.538620in}{1.268453in}}%
\pgfpathcurveto{\pgfqpoint{5.543664in}{1.273496in}}{\pgfqpoint{5.546498in}{1.280338in}}{\pgfqpoint{5.546498in}{1.287471in}}%
\pgfpathcurveto{\pgfqpoint{5.546498in}{1.294604in}}{\pgfqpoint{5.543664in}{1.301445in}}{\pgfqpoint{5.538620in}{1.306489in}}%
\pgfpathcurveto{\pgfqpoint{5.533576in}{1.311533in}}{\pgfqpoint{5.526735in}{1.314367in}}{\pgfqpoint{5.519602in}{1.314367in}}%
\pgfpathcurveto{\pgfqpoint{5.512469in}{1.314367in}}{\pgfqpoint{5.505627in}{1.311533in}}{\pgfqpoint{5.500584in}{1.306489in}}%
\pgfpathcurveto{\pgfqpoint{5.495540in}{1.301445in}}{\pgfqpoint{5.492706in}{1.294604in}}{\pgfqpoint{5.492706in}{1.287471in}}%
\pgfpathcurveto{\pgfqpoint{5.492706in}{1.280338in}}{\pgfqpoint{5.495540in}{1.273496in}}{\pgfqpoint{5.500584in}{1.268453in}}%
\pgfpathcurveto{\pgfqpoint{5.505627in}{1.263409in}}{\pgfqpoint{5.512469in}{1.260575in}}{\pgfqpoint{5.519602in}{1.260575in}}%
\pgfpathclose%
\pgfusepath{stroke,fill}%
\end{pgfscope}%
\begin{pgfscope}%
\pgfpathrectangle{\pgfqpoint{4.985294in}{0.500000in}}{\pgfqpoint{1.764706in}{1.700000in}}%
\pgfusepath{clip}%
\pgfsetbuttcap%
\pgfsetroundjoin%
\definecolor{currentfill}{rgb}{0.952404,0.449449,0.307210}%
\pgfsetfillcolor{currentfill}%
\pgfsetlinewidth{0.311001pt}%
\definecolor{currentstroke}{rgb}{1.000000,1.000000,1.000000}%
\pgfsetstrokecolor{currentstroke}%
\pgfsetdash{}{0pt}%
\pgfpathmoveto{\pgfqpoint{6.151338in}{1.368282in}}%
\pgfpathcurveto{\pgfqpoint{6.158471in}{1.368282in}}{\pgfqpoint{6.165313in}{1.371116in}}{\pgfqpoint{6.170356in}{1.376159in}}%
\pgfpathcurveto{\pgfqpoint{6.175400in}{1.381203in}}{\pgfqpoint{6.178234in}{1.388045in}}{\pgfqpoint{6.178234in}{1.395178in}}%
\pgfpathcurveto{\pgfqpoint{6.178234in}{1.402310in}}{\pgfqpoint{6.175400in}{1.409152in}}{\pgfqpoint{6.170356in}{1.414196in}}%
\pgfpathcurveto{\pgfqpoint{6.165313in}{1.419239in}}{\pgfqpoint{6.158471in}{1.422073in}}{\pgfqpoint{6.151338in}{1.422073in}}%
\pgfpathcurveto{\pgfqpoint{6.144205in}{1.422073in}}{\pgfqpoint{6.137364in}{1.419239in}}{\pgfqpoint{6.132320in}{1.414196in}}%
\pgfpathcurveto{\pgfqpoint{6.127276in}{1.409152in}}{\pgfqpoint{6.124443in}{1.402310in}}{\pgfqpoint{6.124443in}{1.395178in}}%
\pgfpathcurveto{\pgfqpoint{6.124443in}{1.388045in}}{\pgfqpoint{6.127276in}{1.381203in}}{\pgfqpoint{6.132320in}{1.376159in}}%
\pgfpathcurveto{\pgfqpoint{6.137364in}{1.371116in}}{\pgfqpoint{6.144205in}{1.368282in}}{\pgfqpoint{6.151338in}{1.368282in}}%
\pgfpathclose%
\pgfusepath{stroke,fill}%
\end{pgfscope}%
\begin{pgfscope}%
\pgfpathrectangle{\pgfqpoint{4.985294in}{0.500000in}}{\pgfqpoint{1.764706in}{1.700000in}}%
\pgfusepath{clip}%
\pgfsetbuttcap%
\pgfsetroundjoin%
\definecolor{currentfill}{rgb}{0.976287,0.879862,0.805788}%
\pgfsetfillcolor{currentfill}%
\pgfsetlinewidth{0.311001pt}%
\definecolor{currentstroke}{rgb}{1.000000,1.000000,1.000000}%
\pgfsetstrokecolor{currentstroke}%
\pgfsetdash{}{0pt}%
\pgfpathmoveto{\pgfqpoint{6.325118in}{1.154454in}}%
\pgfpathcurveto{\pgfqpoint{6.332251in}{1.154454in}}{\pgfqpoint{6.339092in}{1.157288in}}{\pgfqpoint{6.344136in}{1.162331in}}%
\pgfpathcurveto{\pgfqpoint{6.349180in}{1.167375in}}{\pgfqpoint{6.352014in}{1.174217in}}{\pgfqpoint{6.352014in}{1.181349in}}%
\pgfpathcurveto{\pgfqpoint{6.352014in}{1.188482in}}{\pgfqpoint{6.349180in}{1.195324in}}{\pgfqpoint{6.344136in}{1.200368in}}%
\pgfpathcurveto{\pgfqpoint{6.339092in}{1.205411in}}{\pgfqpoint{6.332251in}{1.208245in}}{\pgfqpoint{6.325118in}{1.208245in}}%
\pgfpathcurveto{\pgfqpoint{6.317985in}{1.208245in}}{\pgfqpoint{6.311143in}{1.205411in}}{\pgfqpoint{6.306100in}{1.200368in}}%
\pgfpathcurveto{\pgfqpoint{6.301056in}{1.195324in}}{\pgfqpoint{6.298222in}{1.188482in}}{\pgfqpoint{6.298222in}{1.181349in}}%
\pgfpathcurveto{\pgfqpoint{6.298222in}{1.174217in}}{\pgfqpoint{6.301056in}{1.167375in}}{\pgfqpoint{6.306100in}{1.162331in}}%
\pgfpathcurveto{\pgfqpoint{6.311143in}{1.157288in}}{\pgfqpoint{6.317985in}{1.154454in}}{\pgfqpoint{6.325118in}{1.154454in}}%
\pgfpathclose%
\pgfusepath{stroke,fill}%
\end{pgfscope}%
\begin{pgfscope}%
\pgfpathrectangle{\pgfqpoint{4.985294in}{0.500000in}}{\pgfqpoint{1.764706in}{1.700000in}}%
\pgfusepath{clip}%
\pgfsetbuttcap%
\pgfsetroundjoin%
\definecolor{currentfill}{rgb}{0.976961,0.885681,0.814303}%
\pgfsetfillcolor{currentfill}%
\pgfsetlinewidth{0.311001pt}%
\definecolor{currentstroke}{rgb}{1.000000,1.000000,1.000000}%
\pgfsetstrokecolor{currentstroke}%
\pgfsetdash{}{0pt}%
\pgfpathmoveto{\pgfqpoint{6.260084in}{1.596506in}}%
\pgfpathcurveto{\pgfqpoint{6.267216in}{1.596506in}}{\pgfqpoint{6.274058in}{1.599340in}}{\pgfqpoint{6.279102in}{1.604383in}}%
\pgfpathcurveto{\pgfqpoint{6.284145in}{1.609427in}}{\pgfqpoint{6.286979in}{1.616269in}}{\pgfqpoint{6.286979in}{1.623402in}}%
\pgfpathcurveto{\pgfqpoint{6.286979in}{1.630534in}}{\pgfqpoint{6.284145in}{1.637376in}}{\pgfqpoint{6.279102in}{1.642420in}}%
\pgfpathcurveto{\pgfqpoint{6.274058in}{1.647463in}}{\pgfqpoint{6.267216in}{1.650297in}}{\pgfqpoint{6.260084in}{1.650297in}}%
\pgfpathcurveto{\pgfqpoint{6.252951in}{1.650297in}}{\pgfqpoint{6.246109in}{1.647463in}}{\pgfqpoint{6.241065in}{1.642420in}}%
\pgfpathcurveto{\pgfqpoint{6.236022in}{1.637376in}}{\pgfqpoint{6.233188in}{1.630534in}}{\pgfqpoint{6.233188in}{1.623402in}}%
\pgfpathcurveto{\pgfqpoint{6.233188in}{1.616269in}}{\pgfqpoint{6.236022in}{1.609427in}}{\pgfqpoint{6.241065in}{1.604383in}}%
\pgfpathcurveto{\pgfqpoint{6.246109in}{1.599340in}}{\pgfqpoint{6.252951in}{1.596506in}}{\pgfqpoint{6.260084in}{1.596506in}}%
\pgfpathclose%
\pgfusepath{stroke,fill}%
\end{pgfscope}%
\begin{pgfscope}%
\pgfpathrectangle{\pgfqpoint{4.985294in}{0.500000in}}{\pgfqpoint{1.764706in}{1.700000in}}%
\pgfusepath{clip}%
\pgfsetbuttcap%
\pgfsetroundjoin%
\definecolor{currentfill}{rgb}{0.887314,0.204699,0.257695}%
\pgfsetfillcolor{currentfill}%
\pgfsetlinewidth{0.311001pt}%
\definecolor{currentstroke}{rgb}{1.000000,1.000000,1.000000}%
\pgfsetstrokecolor{currentstroke}%
\pgfsetdash{}{0pt}%
\pgfpathmoveto{\pgfqpoint{6.159865in}{0.788766in}}%
\pgfpathcurveto{\pgfqpoint{6.166998in}{0.788766in}}{\pgfqpoint{6.173840in}{0.791600in}}{\pgfqpoint{6.178883in}{0.796644in}}%
\pgfpathcurveto{\pgfqpoint{6.183927in}{0.801687in}}{\pgfqpoint{6.186761in}{0.808529in}}{\pgfqpoint{6.186761in}{0.815662in}}%
\pgfpathcurveto{\pgfqpoint{6.186761in}{0.822795in}}{\pgfqpoint{6.183927in}{0.829636in}}{\pgfqpoint{6.178883in}{0.834680in}}%
\pgfpathcurveto{\pgfqpoint{6.173840in}{0.839724in}}{\pgfqpoint{6.166998in}{0.842558in}}{\pgfqpoint{6.159865in}{0.842558in}}%
\pgfpathcurveto{\pgfqpoint{6.152732in}{0.842558in}}{\pgfqpoint{6.145891in}{0.839724in}}{\pgfqpoint{6.140847in}{0.834680in}}%
\pgfpathcurveto{\pgfqpoint{6.135803in}{0.829636in}}{\pgfqpoint{6.132970in}{0.822795in}}{\pgfqpoint{6.132970in}{0.815662in}}%
\pgfpathcurveto{\pgfqpoint{6.132970in}{0.808529in}}{\pgfqpoint{6.135803in}{0.801687in}}{\pgfqpoint{6.140847in}{0.796644in}}%
\pgfpathcurveto{\pgfqpoint{6.145891in}{0.791600in}}{\pgfqpoint{6.152732in}{0.788766in}}{\pgfqpoint{6.159865in}{0.788766in}}%
\pgfpathclose%
\pgfusepath{stroke,fill}%
\end{pgfscope}%
\begin{pgfscope}%
\pgfpathrectangle{\pgfqpoint{4.985294in}{0.500000in}}{\pgfqpoint{1.764706in}{1.700000in}}%
\pgfusepath{clip}%
\pgfsetbuttcap%
\pgfsetroundjoin%
\definecolor{currentfill}{rgb}{0.972726,0.844889,0.754401}%
\pgfsetfillcolor{currentfill}%
\pgfsetlinewidth{0.311001pt}%
\definecolor{currentstroke}{rgb}{1.000000,1.000000,1.000000}%
\pgfsetstrokecolor{currentstroke}%
\pgfsetdash{}{0pt}%
\pgfpathmoveto{\pgfqpoint{5.478160in}{1.331097in}}%
\pgfpathcurveto{\pgfqpoint{5.485293in}{1.331097in}}{\pgfqpoint{5.492135in}{1.333931in}}{\pgfqpoint{5.497178in}{1.338974in}}%
\pgfpathcurveto{\pgfqpoint{5.502222in}{1.344018in}}{\pgfqpoint{5.505056in}{1.350860in}}{\pgfqpoint{5.505056in}{1.357992in}}%
\pgfpathcurveto{\pgfqpoint{5.505056in}{1.365125in}}{\pgfqpoint{5.502222in}{1.371967in}}{\pgfqpoint{5.497178in}{1.377010in}}%
\pgfpathcurveto{\pgfqpoint{5.492135in}{1.382054in}}{\pgfqpoint{5.485293in}{1.384888in}}{\pgfqpoint{5.478160in}{1.384888in}}%
\pgfpathcurveto{\pgfqpoint{5.471027in}{1.384888in}}{\pgfqpoint{5.464186in}{1.382054in}}{\pgfqpoint{5.459142in}{1.377010in}}%
\pgfpathcurveto{\pgfqpoint{5.454098in}{1.371967in}}{\pgfqpoint{5.451264in}{1.365125in}}{\pgfqpoint{5.451264in}{1.357992in}}%
\pgfpathcurveto{\pgfqpoint{5.451264in}{1.350860in}}{\pgfqpoint{5.454098in}{1.344018in}}{\pgfqpoint{5.459142in}{1.338974in}}%
\pgfpathcurveto{\pgfqpoint{5.464186in}{1.333931in}}{\pgfqpoint{5.471027in}{1.331097in}}{\pgfqpoint{5.478160in}{1.331097in}}%
\pgfpathclose%
\pgfusepath{stroke,fill}%
\end{pgfscope}%
\begin{pgfscope}%
\pgfpathrectangle{\pgfqpoint{4.985294in}{0.500000in}}{\pgfqpoint{1.764706in}{1.700000in}}%
\pgfusepath{clip}%
\pgfsetbuttcap%
\pgfsetroundjoin%
\definecolor{currentfill}{rgb}{0.973832,0.856556,0.771584}%
\pgfsetfillcolor{currentfill}%
\pgfsetlinewidth{0.311001pt}%
\definecolor{currentstroke}{rgb}{1.000000,1.000000,1.000000}%
\pgfsetstrokecolor{currentstroke}%
\pgfsetdash{}{0pt}%
\pgfpathmoveto{\pgfqpoint{5.472754in}{1.275996in}}%
\pgfpathcurveto{\pgfqpoint{5.479887in}{1.275996in}}{\pgfqpoint{5.486729in}{1.278830in}}{\pgfqpoint{5.491773in}{1.283874in}}%
\pgfpathcurveto{\pgfqpoint{5.496816in}{1.288917in}}{\pgfqpoint{5.499650in}{1.295759in}}{\pgfqpoint{5.499650in}{1.302892in}}%
\pgfpathcurveto{\pgfqpoint{5.499650in}{1.310025in}}{\pgfqpoint{5.496816in}{1.316866in}}{\pgfqpoint{5.491773in}{1.321910in}}%
\pgfpathcurveto{\pgfqpoint{5.486729in}{1.326954in}}{\pgfqpoint{5.479887in}{1.329788in}}{\pgfqpoint{5.472754in}{1.329788in}}%
\pgfpathcurveto{\pgfqpoint{5.465622in}{1.329788in}}{\pgfqpoint{5.458780in}{1.326954in}}{\pgfqpoint{5.453736in}{1.321910in}}%
\pgfpathcurveto{\pgfqpoint{5.448693in}{1.316866in}}{\pgfqpoint{5.445859in}{1.310025in}}{\pgfqpoint{5.445859in}{1.302892in}}%
\pgfpathcurveto{\pgfqpoint{5.445859in}{1.295759in}}{\pgfqpoint{5.448693in}{1.288917in}}{\pgfqpoint{5.453736in}{1.283874in}}%
\pgfpathcurveto{\pgfqpoint{5.458780in}{1.278830in}}{\pgfqpoint{5.465622in}{1.275996in}}{\pgfqpoint{5.472754in}{1.275996in}}%
\pgfpathclose%
\pgfusepath{stroke,fill}%
\end{pgfscope}%
\begin{pgfscope}%
\pgfpathrectangle{\pgfqpoint{4.985294in}{0.500000in}}{\pgfqpoint{1.764706in}{1.700000in}}%
\pgfusepath{clip}%
\pgfsetbuttcap%
\pgfsetroundjoin%
\definecolor{currentfill}{rgb}{0.963559,0.632016,0.472047}%
\pgfsetfillcolor{currentfill}%
\pgfsetlinewidth{0.311001pt}%
\definecolor{currentstroke}{rgb}{1.000000,1.000000,1.000000}%
\pgfsetstrokecolor{currentstroke}%
\pgfsetdash{}{0pt}%
\pgfpathmoveto{\pgfqpoint{6.173539in}{1.429718in}}%
\pgfpathcurveto{\pgfqpoint{6.180672in}{1.429718in}}{\pgfqpoint{6.187514in}{1.432552in}}{\pgfqpoint{6.192557in}{1.437596in}}%
\pgfpathcurveto{\pgfqpoint{6.197601in}{1.442640in}}{\pgfqpoint{6.200435in}{1.449481in}}{\pgfqpoint{6.200435in}{1.456614in}}%
\pgfpathcurveto{\pgfqpoint{6.200435in}{1.463747in}}{\pgfqpoint{6.197601in}{1.470589in}}{\pgfqpoint{6.192557in}{1.475632in}}%
\pgfpathcurveto{\pgfqpoint{6.187514in}{1.480676in}}{\pgfqpoint{6.180672in}{1.483510in}}{\pgfqpoint{6.173539in}{1.483510in}}%
\pgfpathcurveto{\pgfqpoint{6.166406in}{1.483510in}}{\pgfqpoint{6.159565in}{1.480676in}}{\pgfqpoint{6.154521in}{1.475632in}}%
\pgfpathcurveto{\pgfqpoint{6.149478in}{1.470589in}}{\pgfqpoint{6.146644in}{1.463747in}}{\pgfqpoint{6.146644in}{1.456614in}}%
\pgfpathcurveto{\pgfqpoint{6.146644in}{1.449481in}}{\pgfqpoint{6.149478in}{1.442640in}}{\pgfqpoint{6.154521in}{1.437596in}}%
\pgfpathcurveto{\pgfqpoint{6.159565in}{1.432552in}}{\pgfqpoint{6.166406in}{1.429718in}}{\pgfqpoint{6.173539in}{1.429718in}}%
\pgfpathclose%
\pgfusepath{stroke,fill}%
\end{pgfscope}%
\begin{pgfscope}%
\pgfpathrectangle{\pgfqpoint{4.985294in}{0.500000in}}{\pgfqpoint{1.764706in}{1.700000in}}%
\pgfusepath{clip}%
\pgfsetbuttcap%
\pgfsetroundjoin%
\definecolor{currentfill}{rgb}{0.965440,0.720101,0.576404}%
\pgfsetfillcolor{currentfill}%
\pgfsetlinewidth{0.311001pt}%
\definecolor{currentstroke}{rgb}{1.000000,1.000000,1.000000}%
\pgfsetstrokecolor{currentstroke}%
\pgfsetdash{}{0pt}%
\pgfpathmoveto{\pgfqpoint{6.141814in}{1.693291in}}%
\pgfpathcurveto{\pgfqpoint{6.148947in}{1.693291in}}{\pgfqpoint{6.155788in}{1.696124in}}{\pgfqpoint{6.160832in}{1.701168in}}%
\pgfpathcurveto{\pgfqpoint{6.165875in}{1.706212in}}{\pgfqpoint{6.168709in}{1.713053in}}{\pgfqpoint{6.168709in}{1.720186in}}%
\pgfpathcurveto{\pgfqpoint{6.168709in}{1.727319in}}{\pgfqpoint{6.165875in}{1.734161in}}{\pgfqpoint{6.160832in}{1.739204in}}%
\pgfpathcurveto{\pgfqpoint{6.155788in}{1.744248in}}{\pgfqpoint{6.148947in}{1.747082in}}{\pgfqpoint{6.141814in}{1.747082in}}%
\pgfpathcurveto{\pgfqpoint{6.134681in}{1.747082in}}{\pgfqpoint{6.127839in}{1.744248in}}{\pgfqpoint{6.122796in}{1.739204in}}%
\pgfpathcurveto{\pgfqpoint{6.117752in}{1.734161in}}{\pgfqpoint{6.114918in}{1.727319in}}{\pgfqpoint{6.114918in}{1.720186in}}%
\pgfpathcurveto{\pgfqpoint{6.114918in}{1.713053in}}{\pgfqpoint{6.117752in}{1.706212in}}{\pgfqpoint{6.122796in}{1.701168in}}%
\pgfpathcurveto{\pgfqpoint{6.127839in}{1.696124in}}{\pgfqpoint{6.134681in}{1.693291in}}{\pgfqpoint{6.141814in}{1.693291in}}%
\pgfpathclose%
\pgfusepath{stroke,fill}%
\end{pgfscope}%
\begin{pgfscope}%
\pgfpathrectangle{\pgfqpoint{4.985294in}{0.500000in}}{\pgfqpoint{1.764706in}{1.700000in}}%
\pgfusepath{clip}%
\pgfsetbuttcap%
\pgfsetroundjoin%
\definecolor{currentfill}{rgb}{0.963728,0.638439,0.479050}%
\pgfsetfillcolor{currentfill}%
\pgfsetlinewidth{0.311001pt}%
\definecolor{currentstroke}{rgb}{1.000000,1.000000,1.000000}%
\pgfsetstrokecolor{currentstroke}%
\pgfsetdash{}{0pt}%
\pgfpathmoveto{\pgfqpoint{5.384130in}{1.611113in}}%
\pgfpathcurveto{\pgfqpoint{5.391263in}{1.611113in}}{\pgfqpoint{5.398105in}{1.613947in}}{\pgfqpoint{5.403148in}{1.618990in}}%
\pgfpathcurveto{\pgfqpoint{5.408192in}{1.624034in}}{\pgfqpoint{5.411026in}{1.630876in}}{\pgfqpoint{5.411026in}{1.638009in}}%
\pgfpathcurveto{\pgfqpoint{5.411026in}{1.645141in}}{\pgfqpoint{5.408192in}{1.651983in}}{\pgfqpoint{5.403148in}{1.657027in}}%
\pgfpathcurveto{\pgfqpoint{5.398105in}{1.662070in}}{\pgfqpoint{5.391263in}{1.664904in}}{\pgfqpoint{5.384130in}{1.664904in}}%
\pgfpathcurveto{\pgfqpoint{5.376997in}{1.664904in}}{\pgfqpoint{5.370156in}{1.662070in}}{\pgfqpoint{5.365112in}{1.657027in}}%
\pgfpathcurveto{\pgfqpoint{5.360068in}{1.651983in}}{\pgfqpoint{5.357235in}{1.645141in}}{\pgfqpoint{5.357235in}{1.638009in}}%
\pgfpathcurveto{\pgfqpoint{5.357235in}{1.630876in}}{\pgfqpoint{5.360068in}{1.624034in}}{\pgfqpoint{5.365112in}{1.618990in}}%
\pgfpathcurveto{\pgfqpoint{5.370156in}{1.613947in}}{\pgfqpoint{5.376997in}{1.611113in}}{\pgfqpoint{5.384130in}{1.611113in}}%
\pgfpathclose%
\pgfusepath{stroke,fill}%
\end{pgfscope}%
\begin{pgfscope}%
\pgfpathrectangle{\pgfqpoint{4.985294in}{0.500000in}}{\pgfqpoint{1.764706in}{1.700000in}}%
\pgfusepath{clip}%
\pgfsetbuttcap%
\pgfsetroundjoin%
\definecolor{currentfill}{rgb}{0.956817,0.498820,0.345554}%
\pgfsetfillcolor{currentfill}%
\pgfsetlinewidth{0.311001pt}%
\definecolor{currentstroke}{rgb}{1.000000,1.000000,1.000000}%
\pgfsetstrokecolor{currentstroke}%
\pgfsetdash{}{0pt}%
\pgfpathmoveto{\pgfqpoint{6.075459in}{1.027606in}}%
\pgfpathcurveto{\pgfqpoint{6.082592in}{1.027606in}}{\pgfqpoint{6.089434in}{1.030440in}}{\pgfqpoint{6.094477in}{1.035483in}}%
\pgfpathcurveto{\pgfqpoint{6.099521in}{1.040527in}}{\pgfqpoint{6.102355in}{1.047369in}}{\pgfqpoint{6.102355in}{1.054502in}}%
\pgfpathcurveto{\pgfqpoint{6.102355in}{1.061634in}}{\pgfqpoint{6.099521in}{1.068476in}}{\pgfqpoint{6.094477in}{1.073520in}}%
\pgfpathcurveto{\pgfqpoint{6.089434in}{1.078563in}}{\pgfqpoint{6.082592in}{1.081397in}}{\pgfqpoint{6.075459in}{1.081397in}}%
\pgfpathcurveto{\pgfqpoint{6.068326in}{1.081397in}}{\pgfqpoint{6.061485in}{1.078563in}}{\pgfqpoint{6.056441in}{1.073520in}}%
\pgfpathcurveto{\pgfqpoint{6.051397in}{1.068476in}}{\pgfqpoint{6.048564in}{1.061634in}}{\pgfqpoint{6.048564in}{1.054502in}}%
\pgfpathcurveto{\pgfqpoint{6.048564in}{1.047369in}}{\pgfqpoint{6.051397in}{1.040527in}}{\pgfqpoint{6.056441in}{1.035483in}}%
\pgfpathcurveto{\pgfqpoint{6.061485in}{1.030440in}}{\pgfqpoint{6.068326in}{1.027606in}}{\pgfqpoint{6.075459in}{1.027606in}}%
\pgfpathclose%
\pgfusepath{stroke,fill}%
\end{pgfscope}%
\begin{pgfscope}%
\pgfpathrectangle{\pgfqpoint{4.985294in}{0.500000in}}{\pgfqpoint{1.764706in}{1.700000in}}%
\pgfusepath{clip}%
\pgfsetbuttcap%
\pgfsetroundjoin%
\definecolor{currentfill}{rgb}{0.966328,0.750560,0.616961}%
\pgfsetfillcolor{currentfill}%
\pgfsetlinewidth{0.311001pt}%
\definecolor{currentstroke}{rgb}{1.000000,1.000000,1.000000}%
\pgfsetstrokecolor{currentstroke}%
\pgfsetdash{}{0pt}%
\pgfpathmoveto{\pgfqpoint{6.185609in}{1.167361in}}%
\pgfpathcurveto{\pgfqpoint{6.192741in}{1.167361in}}{\pgfqpoint{6.199583in}{1.170195in}}{\pgfqpoint{6.204627in}{1.175239in}}%
\pgfpathcurveto{\pgfqpoint{6.209670in}{1.180282in}}{\pgfqpoint{6.212504in}{1.187124in}}{\pgfqpoint{6.212504in}{1.194257in}}%
\pgfpathcurveto{\pgfqpoint{6.212504in}{1.201390in}}{\pgfqpoint{6.209670in}{1.208231in}}{\pgfqpoint{6.204627in}{1.213275in}}%
\pgfpathcurveto{\pgfqpoint{6.199583in}{1.218318in}}{\pgfqpoint{6.192741in}{1.221152in}}{\pgfqpoint{6.185609in}{1.221152in}}%
\pgfpathcurveto{\pgfqpoint{6.178476in}{1.221152in}}{\pgfqpoint{6.171634in}{1.218318in}}{\pgfqpoint{6.166590in}{1.213275in}}%
\pgfpathcurveto{\pgfqpoint{6.161547in}{1.208231in}}{\pgfqpoint{6.158713in}{1.201390in}}{\pgfqpoint{6.158713in}{1.194257in}}%
\pgfpathcurveto{\pgfqpoint{6.158713in}{1.187124in}}{\pgfqpoint{6.161547in}{1.180282in}}{\pgfqpoint{6.166590in}{1.175239in}}%
\pgfpathcurveto{\pgfqpoint{6.171634in}{1.170195in}}{\pgfqpoint{6.178476in}{1.167361in}}{\pgfqpoint{6.185609in}{1.167361in}}%
\pgfpathclose%
\pgfusepath{stroke,fill}%
\end{pgfscope}%
\begin{pgfscope}%
\pgfpathrectangle{\pgfqpoint{4.985294in}{0.500000in}}{\pgfqpoint{1.764706in}{1.700000in}}%
\pgfusepath{clip}%
\pgfsetbuttcap%
\pgfsetroundjoin%
\definecolor{currentfill}{rgb}{0.980678,0.914765,0.856766}%
\pgfsetfillcolor{currentfill}%
\pgfsetlinewidth{0.311001pt}%
\definecolor{currentstroke}{rgb}{1.000000,1.000000,1.000000}%
\pgfsetstrokecolor{currentstroke}%
\pgfsetdash{}{0pt}%
\pgfpathmoveto{\pgfqpoint{6.299431in}{1.368290in}}%
\pgfpathcurveto{\pgfqpoint{6.306564in}{1.368290in}}{\pgfqpoint{6.313406in}{1.371124in}}{\pgfqpoint{6.318450in}{1.376167in}}%
\pgfpathcurveto{\pgfqpoint{6.323493in}{1.381211in}}{\pgfqpoint{6.326327in}{1.388053in}}{\pgfqpoint{6.326327in}{1.395185in}}%
\pgfpathcurveto{\pgfqpoint{6.326327in}{1.402318in}}{\pgfqpoint{6.323493in}{1.409160in}}{\pgfqpoint{6.318450in}{1.414203in}}%
\pgfpathcurveto{\pgfqpoint{6.313406in}{1.419247in}}{\pgfqpoint{6.306564in}{1.422081in}}{\pgfqpoint{6.299431in}{1.422081in}}%
\pgfpathcurveto{\pgfqpoint{6.292299in}{1.422081in}}{\pgfqpoint{6.285457in}{1.419247in}}{\pgfqpoint{6.280413in}{1.414203in}}%
\pgfpathcurveto{\pgfqpoint{6.275370in}{1.409160in}}{\pgfqpoint{6.272536in}{1.402318in}}{\pgfqpoint{6.272536in}{1.395185in}}%
\pgfpathcurveto{\pgfqpoint{6.272536in}{1.388053in}}{\pgfqpoint{6.275370in}{1.381211in}}{\pgfqpoint{6.280413in}{1.376167in}}%
\pgfpathcurveto{\pgfqpoint{6.285457in}{1.371124in}}{\pgfqpoint{6.292299in}{1.368290in}}{\pgfqpoint{6.299431in}{1.368290in}}%
\pgfpathclose%
\pgfusepath{stroke,fill}%
\end{pgfscope}%
\begin{pgfscope}%
\pgfpathrectangle{\pgfqpoint{4.985294in}{0.500000in}}{\pgfqpoint{1.764706in}{1.700000in}}%
\pgfusepath{clip}%
\pgfsetbuttcap%
\pgfsetroundjoin%
\definecolor{currentfill}{rgb}{0.978376,0.897317,0.831308}%
\pgfsetfillcolor{currentfill}%
\pgfsetlinewidth{0.311001pt}%
\definecolor{currentstroke}{rgb}{1.000000,1.000000,1.000000}%
\pgfsetstrokecolor{currentstroke}%
\pgfsetdash{}{0pt}%
\pgfpathmoveto{\pgfqpoint{6.272077in}{1.199377in}}%
\pgfpathcurveto{\pgfqpoint{6.279210in}{1.199377in}}{\pgfqpoint{6.286051in}{1.202211in}}{\pgfqpoint{6.291095in}{1.207254in}}%
\pgfpathcurveto{\pgfqpoint{6.296139in}{1.212298in}}{\pgfqpoint{6.298972in}{1.219140in}}{\pgfqpoint{6.298972in}{1.226273in}}%
\pgfpathcurveto{\pgfqpoint{6.298972in}{1.233405in}}{\pgfqpoint{6.296139in}{1.240247in}}{\pgfqpoint{6.291095in}{1.245291in}}%
\pgfpathcurveto{\pgfqpoint{6.286051in}{1.250334in}}{\pgfqpoint{6.279210in}{1.253168in}}{\pgfqpoint{6.272077in}{1.253168in}}%
\pgfpathcurveto{\pgfqpoint{6.264944in}{1.253168in}}{\pgfqpoint{6.258102in}{1.250334in}}{\pgfqpoint{6.253059in}{1.245291in}}%
\pgfpathcurveto{\pgfqpoint{6.248015in}{1.240247in}}{\pgfqpoint{6.245181in}{1.233405in}}{\pgfqpoint{6.245181in}{1.226273in}}%
\pgfpathcurveto{\pgfqpoint{6.245181in}{1.219140in}}{\pgfqpoint{6.248015in}{1.212298in}}{\pgfqpoint{6.253059in}{1.207254in}}%
\pgfpathcurveto{\pgfqpoint{6.258102in}{1.202211in}}{\pgfqpoint{6.264944in}{1.199377in}}{\pgfqpoint{6.272077in}{1.199377in}}%
\pgfpathclose%
\pgfusepath{stroke,fill}%
\end{pgfscope}%
\begin{pgfscope}%
\pgfpathrectangle{\pgfqpoint{4.985294in}{0.500000in}}{\pgfqpoint{1.764706in}{1.700000in}}%
\pgfusepath{clip}%
\pgfsetbuttcap%
\pgfsetroundjoin%
\definecolor{currentfill}{rgb}{0.975018,0.868213,0.788710}%
\pgfsetfillcolor{currentfill}%
\pgfsetlinewidth{0.311001pt}%
\definecolor{currentstroke}{rgb}{1.000000,1.000000,1.000000}%
\pgfsetstrokecolor{currentstroke}%
\pgfsetdash{}{0pt}%
\pgfpathmoveto{\pgfqpoint{5.467955in}{1.361598in}}%
\pgfpathcurveto{\pgfqpoint{5.475088in}{1.361598in}}{\pgfqpoint{5.481930in}{1.364432in}}{\pgfqpoint{5.486974in}{1.369476in}}%
\pgfpathcurveto{\pgfqpoint{5.492017in}{1.374520in}}{\pgfqpoint{5.494851in}{1.381361in}}{\pgfqpoint{5.494851in}{1.388494in}}%
\pgfpathcurveto{\pgfqpoint{5.494851in}{1.395627in}}{\pgfqpoint{5.492017in}{1.402468in}}{\pgfqpoint{5.486974in}{1.407512in}}%
\pgfpathcurveto{\pgfqpoint{5.481930in}{1.412556in}}{\pgfqpoint{5.475088in}{1.415390in}}{\pgfqpoint{5.467955in}{1.415390in}}%
\pgfpathcurveto{\pgfqpoint{5.460823in}{1.415390in}}{\pgfqpoint{5.453981in}{1.412556in}}{\pgfqpoint{5.448937in}{1.407512in}}%
\pgfpathcurveto{\pgfqpoint{5.443894in}{1.402468in}}{\pgfqpoint{5.441060in}{1.395627in}}{\pgfqpoint{5.441060in}{1.388494in}}%
\pgfpathcurveto{\pgfqpoint{5.441060in}{1.381361in}}{\pgfqpoint{5.443894in}{1.374520in}}{\pgfqpoint{5.448937in}{1.369476in}}%
\pgfpathcurveto{\pgfqpoint{5.453981in}{1.364432in}}{\pgfqpoint{5.460823in}{1.361598in}}{\pgfqpoint{5.467955in}{1.361598in}}%
\pgfpathclose%
\pgfusepath{stroke,fill}%
\end{pgfscope}%
\begin{pgfscope}%
\pgfpathrectangle{\pgfqpoint{4.985294in}{0.500000in}}{\pgfqpoint{1.764706in}{1.700000in}}%
\pgfusepath{clip}%
\pgfsetbuttcap%
\pgfsetroundjoin%
\definecolor{currentfill}{rgb}{0.978376,0.897317,0.831308}%
\pgfsetfillcolor{currentfill}%
\pgfsetlinewidth{0.311001pt}%
\definecolor{currentstroke}{rgb}{1.000000,1.000000,1.000000}%
\pgfsetstrokecolor{currentstroke}%
\pgfsetdash{}{0pt}%
\pgfpathmoveto{\pgfqpoint{6.326010in}{1.460134in}}%
\pgfpathcurveto{\pgfqpoint{6.333143in}{1.460134in}}{\pgfqpoint{6.339985in}{1.462968in}}{\pgfqpoint{6.345029in}{1.468012in}}%
\pgfpathcurveto{\pgfqpoint{6.350072in}{1.473055in}}{\pgfqpoint{6.352906in}{1.479897in}}{\pgfqpoint{6.352906in}{1.487030in}}%
\pgfpathcurveto{\pgfqpoint{6.352906in}{1.494162in}}{\pgfqpoint{6.350072in}{1.501004in}}{\pgfqpoint{6.345029in}{1.506048in}}%
\pgfpathcurveto{\pgfqpoint{6.339985in}{1.511091in}}{\pgfqpoint{6.333143in}{1.513925in}}{\pgfqpoint{6.326010in}{1.513925in}}%
\pgfpathcurveto{\pgfqpoint{6.318878in}{1.513925in}}{\pgfqpoint{6.312036in}{1.511091in}}{\pgfqpoint{6.306992in}{1.506048in}}%
\pgfpathcurveto{\pgfqpoint{6.301949in}{1.501004in}}{\pgfqpoint{6.299115in}{1.494162in}}{\pgfqpoint{6.299115in}{1.487030in}}%
\pgfpathcurveto{\pgfqpoint{6.299115in}{1.479897in}}{\pgfqpoint{6.301949in}{1.473055in}}{\pgfqpoint{6.306992in}{1.468012in}}%
\pgfpathcurveto{\pgfqpoint{6.312036in}{1.462968in}}{\pgfqpoint{6.318878in}{1.460134in}}{\pgfqpoint{6.326010in}{1.460134in}}%
\pgfpathclose%
\pgfusepath{stroke,fill}%
\end{pgfscope}%
\begin{pgfscope}%
\pgfpathrectangle{\pgfqpoint{4.985294in}{0.500000in}}{\pgfqpoint{1.764706in}{1.700000in}}%
\pgfusepath{clip}%
\pgfsetbuttcap%
\pgfsetroundjoin%
\definecolor{currentfill}{rgb}{0.968509,0.792226,0.676405}%
\pgfsetfillcolor{currentfill}%
\pgfsetlinewidth{0.311001pt}%
\definecolor{currentstroke}{rgb}{1.000000,1.000000,1.000000}%
\pgfsetstrokecolor{currentstroke}%
\pgfsetdash{}{0pt}%
\pgfpathmoveto{\pgfqpoint{6.179585in}{1.081820in}}%
\pgfpathcurveto{\pgfqpoint{6.186718in}{1.081820in}}{\pgfqpoint{6.193559in}{1.084654in}}{\pgfqpoint{6.198603in}{1.089697in}}%
\pgfpathcurveto{\pgfqpoint{6.203647in}{1.094741in}}{\pgfqpoint{6.206481in}{1.101583in}}{\pgfqpoint{6.206481in}{1.108715in}}%
\pgfpathcurveto{\pgfqpoint{6.206481in}{1.115848in}}{\pgfqpoint{6.203647in}{1.122690in}}{\pgfqpoint{6.198603in}{1.127733in}}%
\pgfpathcurveto{\pgfqpoint{6.193559in}{1.132777in}}{\pgfqpoint{6.186718in}{1.135611in}}{\pgfqpoint{6.179585in}{1.135611in}}%
\pgfpathcurveto{\pgfqpoint{6.172452in}{1.135611in}}{\pgfqpoint{6.165611in}{1.132777in}}{\pgfqpoint{6.160567in}{1.127733in}}%
\pgfpathcurveto{\pgfqpoint{6.155523in}{1.122690in}}{\pgfqpoint{6.152689in}{1.115848in}}{\pgfqpoint{6.152689in}{1.108715in}}%
\pgfpathcurveto{\pgfqpoint{6.152689in}{1.101583in}}{\pgfqpoint{6.155523in}{1.094741in}}{\pgfqpoint{6.160567in}{1.089697in}}%
\pgfpathcurveto{\pgfqpoint{6.165611in}{1.084654in}}{\pgfqpoint{6.172452in}{1.081820in}}{\pgfqpoint{6.179585in}{1.081820in}}%
\pgfpathclose%
\pgfusepath{stroke,fill}%
\end{pgfscope}%
\begin{pgfscope}%
\pgfpathrectangle{\pgfqpoint{4.985294in}{0.500000in}}{\pgfqpoint{1.764706in}{1.700000in}}%
\pgfusepath{clip}%
\pgfsetbuttcap%
\pgfsetroundjoin%
\definecolor{currentfill}{rgb}{0.970255,0.815666,0.711203}%
\pgfsetfillcolor{currentfill}%
\pgfsetlinewidth{0.311001pt}%
\definecolor{currentstroke}{rgb}{1.000000,1.000000,1.000000}%
\pgfsetstrokecolor{currentstroke}%
\pgfsetdash{}{0pt}%
\pgfpathmoveto{\pgfqpoint{5.545385in}{0.987513in}}%
\pgfpathcurveto{\pgfqpoint{5.552518in}{0.987513in}}{\pgfqpoint{5.559359in}{0.990347in}}{\pgfqpoint{5.564403in}{0.995390in}}%
\pgfpathcurveto{\pgfqpoint{5.569447in}{1.000434in}}{\pgfqpoint{5.572281in}{1.007276in}}{\pgfqpoint{5.572281in}{1.014409in}}%
\pgfpathcurveto{\pgfqpoint{5.572281in}{1.021541in}}{\pgfqpoint{5.569447in}{1.028383in}}{\pgfqpoint{5.564403in}{1.033427in}}%
\pgfpathcurveto{\pgfqpoint{5.559359in}{1.038470in}}{\pgfqpoint{5.552518in}{1.041304in}}{\pgfqpoint{5.545385in}{1.041304in}}%
\pgfpathcurveto{\pgfqpoint{5.538252in}{1.041304in}}{\pgfqpoint{5.531410in}{1.038470in}}{\pgfqpoint{5.526367in}{1.033427in}}%
\pgfpathcurveto{\pgfqpoint{5.521323in}{1.028383in}}{\pgfqpoint{5.518489in}{1.021541in}}{\pgfqpoint{5.518489in}{1.014409in}}%
\pgfpathcurveto{\pgfqpoint{5.518489in}{1.007276in}}{\pgfqpoint{5.521323in}{1.000434in}}{\pgfqpoint{5.526367in}{0.995390in}}%
\pgfpathcurveto{\pgfqpoint{5.531410in}{0.990347in}}{\pgfqpoint{5.538252in}{0.987513in}}{\pgfqpoint{5.545385in}{0.987513in}}%
\pgfpathclose%
\pgfusepath{stroke,fill}%
\end{pgfscope}%
\begin{pgfscope}%
\pgfpathrectangle{\pgfqpoint{4.985294in}{0.500000in}}{\pgfqpoint{1.764706in}{1.700000in}}%
\pgfusepath{clip}%
\pgfsetbuttcap%
\pgfsetroundjoin%
\definecolor{currentfill}{rgb}{0.980678,0.914765,0.856766}%
\pgfsetfillcolor{currentfill}%
\pgfsetlinewidth{0.311001pt}%
\definecolor{currentstroke}{rgb}{1.000000,1.000000,1.000000}%
\pgfsetstrokecolor{currentstroke}%
\pgfsetdash{}{0pt}%
\pgfpathmoveto{\pgfqpoint{5.424667in}{1.330983in}}%
\pgfpathcurveto{\pgfqpoint{5.431800in}{1.330983in}}{\pgfqpoint{5.438641in}{1.333817in}}{\pgfqpoint{5.443685in}{1.338860in}}%
\pgfpathcurveto{\pgfqpoint{5.448729in}{1.343904in}}{\pgfqpoint{5.451562in}{1.350746in}}{\pgfqpoint{5.451562in}{1.357878in}}%
\pgfpathcurveto{\pgfqpoint{5.451562in}{1.365011in}}{\pgfqpoint{5.448729in}{1.371853in}}{\pgfqpoint{5.443685in}{1.376897in}}%
\pgfpathcurveto{\pgfqpoint{5.438641in}{1.381940in}}{\pgfqpoint{5.431800in}{1.384774in}}{\pgfqpoint{5.424667in}{1.384774in}}%
\pgfpathcurveto{\pgfqpoint{5.417534in}{1.384774in}}{\pgfqpoint{5.410692in}{1.381940in}}{\pgfqpoint{5.405649in}{1.376897in}}%
\pgfpathcurveto{\pgfqpoint{5.400605in}{1.371853in}}{\pgfqpoint{5.397771in}{1.365011in}}{\pgfqpoint{5.397771in}{1.357878in}}%
\pgfpathcurveto{\pgfqpoint{5.397771in}{1.350746in}}{\pgfqpoint{5.400605in}{1.343904in}}{\pgfqpoint{5.405649in}{1.338860in}}%
\pgfpathcurveto{\pgfqpoint{5.410692in}{1.333817in}}{\pgfqpoint{5.417534in}{1.330983in}}{\pgfqpoint{5.424667in}{1.330983in}}%
\pgfpathclose%
\pgfusepath{stroke,fill}%
\end{pgfscope}%
\begin{pgfscope}%
\pgfpathrectangle{\pgfqpoint{4.985294in}{0.500000in}}{\pgfqpoint{1.764706in}{1.700000in}}%
\pgfusepath{clip}%
\pgfsetbuttcap%
\pgfsetroundjoin%
\definecolor{currentfill}{rgb}{0.957344,0.505732,0.351309}%
\pgfsetfillcolor{currentfill}%
\pgfsetlinewidth{0.311001pt}%
\definecolor{currentstroke}{rgb}{1.000000,1.000000,1.000000}%
\pgfsetstrokecolor{currentstroke}%
\pgfsetdash{}{0pt}%
\pgfpathmoveto{\pgfqpoint{6.115620in}{1.102443in}}%
\pgfpathcurveto{\pgfqpoint{6.122753in}{1.102443in}}{\pgfqpoint{6.129595in}{1.105277in}}{\pgfqpoint{6.134638in}{1.110321in}}%
\pgfpathcurveto{\pgfqpoint{6.139682in}{1.115365in}}{\pgfqpoint{6.142516in}{1.122206in}}{\pgfqpoint{6.142516in}{1.129339in}}%
\pgfpathcurveto{\pgfqpoint{6.142516in}{1.136472in}}{\pgfqpoint{6.139682in}{1.143313in}}{\pgfqpoint{6.134638in}{1.148357in}}%
\pgfpathcurveto{\pgfqpoint{6.129595in}{1.153401in}}{\pgfqpoint{6.122753in}{1.156235in}}{\pgfqpoint{6.115620in}{1.156235in}}%
\pgfpathcurveto{\pgfqpoint{6.108487in}{1.156235in}}{\pgfqpoint{6.101646in}{1.153401in}}{\pgfqpoint{6.096602in}{1.148357in}}%
\pgfpathcurveto{\pgfqpoint{6.091558in}{1.143313in}}{\pgfqpoint{6.088725in}{1.136472in}}{\pgfqpoint{6.088725in}{1.129339in}}%
\pgfpathcurveto{\pgfqpoint{6.088725in}{1.122206in}}{\pgfqpoint{6.091558in}{1.115365in}}{\pgfqpoint{6.096602in}{1.110321in}}%
\pgfpathcurveto{\pgfqpoint{6.101646in}{1.105277in}}{\pgfqpoint{6.108487in}{1.102443in}}{\pgfqpoint{6.115620in}{1.102443in}}%
\pgfpathclose%
\pgfusepath{stroke,fill}%
\end{pgfscope}%
\begin{pgfscope}%
\pgfpathrectangle{\pgfqpoint{4.985294in}{0.500000in}}{\pgfqpoint{1.764706in}{1.700000in}}%
\pgfusepath{clip}%
\pgfsetbuttcap%
\pgfsetroundjoin%
\definecolor{currentfill}{rgb}{0.968931,0.798091,0.685123}%
\pgfsetfillcolor{currentfill}%
\pgfsetlinewidth{0.311001pt}%
\definecolor{currentstroke}{rgb}{1.000000,1.000000,1.000000}%
\pgfsetstrokecolor{currentstroke}%
\pgfsetdash{}{0pt}%
\pgfpathmoveto{\pgfqpoint{5.518486in}{1.659720in}}%
\pgfpathcurveto{\pgfqpoint{5.525619in}{1.659720in}}{\pgfqpoint{5.532460in}{1.662554in}}{\pgfqpoint{5.537504in}{1.667598in}}%
\pgfpathcurveto{\pgfqpoint{5.542548in}{1.672641in}}{\pgfqpoint{5.545381in}{1.679483in}}{\pgfqpoint{5.545381in}{1.686616in}}%
\pgfpathcurveto{\pgfqpoint{5.545381in}{1.693749in}}{\pgfqpoint{5.542548in}{1.700590in}}{\pgfqpoint{5.537504in}{1.705634in}}%
\pgfpathcurveto{\pgfqpoint{5.532460in}{1.710678in}}{\pgfqpoint{5.525619in}{1.713511in}}{\pgfqpoint{5.518486in}{1.713511in}}%
\pgfpathcurveto{\pgfqpoint{5.511353in}{1.713511in}}{\pgfqpoint{5.504511in}{1.710678in}}{\pgfqpoint{5.499468in}{1.705634in}}%
\pgfpathcurveto{\pgfqpoint{5.494424in}{1.700590in}}{\pgfqpoint{5.491590in}{1.693749in}}{\pgfqpoint{5.491590in}{1.686616in}}%
\pgfpathcurveto{\pgfqpoint{5.491590in}{1.679483in}}{\pgfqpoint{5.494424in}{1.672641in}}{\pgfqpoint{5.499468in}{1.667598in}}%
\pgfpathcurveto{\pgfqpoint{5.504511in}{1.662554in}}{\pgfqpoint{5.511353in}{1.659720in}}{\pgfqpoint{5.518486in}{1.659720in}}%
\pgfpathclose%
\pgfusepath{stroke,fill}%
\end{pgfscope}%
\begin{pgfscope}%
\pgfpathrectangle{\pgfqpoint{4.985294in}{0.500000in}}{\pgfqpoint{1.764706in}{1.700000in}}%
\pgfusepath{clip}%
\pgfsetbuttcap%
\pgfsetroundjoin%
\definecolor{currentfill}{rgb}{0.977657,0.891500,0.822809}%
\pgfsetfillcolor{currentfill}%
\pgfsetlinewidth{0.311001pt}%
\definecolor{currentstroke}{rgb}{1.000000,1.000000,1.000000}%
\pgfsetstrokecolor{currentstroke}%
\pgfsetdash{}{0pt}%
\pgfpathmoveto{\pgfqpoint{5.457852in}{1.434390in}}%
\pgfpathcurveto{\pgfqpoint{5.464985in}{1.434390in}}{\pgfqpoint{5.471827in}{1.437224in}}{\pgfqpoint{5.476870in}{1.442267in}}%
\pgfpathcurveto{\pgfqpoint{5.481914in}{1.447311in}}{\pgfqpoint{5.484748in}{1.454153in}}{\pgfqpoint{5.484748in}{1.461285in}}%
\pgfpathcurveto{\pgfqpoint{5.484748in}{1.468418in}}{\pgfqpoint{5.481914in}{1.475260in}}{\pgfqpoint{5.476870in}{1.480304in}}%
\pgfpathcurveto{\pgfqpoint{5.471827in}{1.485347in}}{\pgfqpoint{5.464985in}{1.488181in}}{\pgfqpoint{5.457852in}{1.488181in}}%
\pgfpathcurveto{\pgfqpoint{5.450719in}{1.488181in}}{\pgfqpoint{5.443878in}{1.485347in}}{\pgfqpoint{5.438834in}{1.480304in}}%
\pgfpathcurveto{\pgfqpoint{5.433790in}{1.475260in}}{\pgfqpoint{5.430957in}{1.468418in}}{\pgfqpoint{5.430957in}{1.461285in}}%
\pgfpathcurveto{\pgfqpoint{5.430957in}{1.454153in}}{\pgfqpoint{5.433790in}{1.447311in}}{\pgfqpoint{5.438834in}{1.442267in}}%
\pgfpathcurveto{\pgfqpoint{5.443878in}{1.437224in}}{\pgfqpoint{5.450719in}{1.434390in}}{\pgfqpoint{5.457852in}{1.434390in}}%
\pgfpathclose%
\pgfusepath{stroke,fill}%
\end{pgfscope}%
\begin{pgfscope}%
\pgfpathrectangle{\pgfqpoint{4.985294in}{0.500000in}}{\pgfqpoint{1.764706in}{1.700000in}}%
\pgfusepath{clip}%
\pgfsetbuttcap%
\pgfsetroundjoin%
\definecolor{currentfill}{rgb}{0.964433,0.670254,0.515093}%
\pgfsetfillcolor{currentfill}%
\pgfsetlinewidth{0.311001pt}%
\definecolor{currentstroke}{rgb}{1.000000,1.000000,1.000000}%
\pgfsetstrokecolor{currentstroke}%
\pgfsetdash{}{0pt}%
\pgfpathmoveto{\pgfqpoint{5.587491in}{1.697912in}}%
\pgfpathcurveto{\pgfqpoint{5.594624in}{1.697912in}}{\pgfqpoint{5.601466in}{1.700746in}}{\pgfqpoint{5.606509in}{1.705789in}}%
\pgfpathcurveto{\pgfqpoint{5.611553in}{1.710833in}}{\pgfqpoint{5.614387in}{1.717675in}}{\pgfqpoint{5.614387in}{1.724808in}}%
\pgfpathcurveto{\pgfqpoint{5.614387in}{1.731940in}}{\pgfqpoint{5.611553in}{1.738782in}}{\pgfqpoint{5.606509in}{1.743826in}}%
\pgfpathcurveto{\pgfqpoint{5.601466in}{1.748869in}}{\pgfqpoint{5.594624in}{1.751703in}}{\pgfqpoint{5.587491in}{1.751703in}}%
\pgfpathcurveto{\pgfqpoint{5.580358in}{1.751703in}}{\pgfqpoint{5.573517in}{1.748869in}}{\pgfqpoint{5.568473in}{1.743826in}}%
\pgfpathcurveto{\pgfqpoint{5.563429in}{1.738782in}}{\pgfqpoint{5.560595in}{1.731940in}}{\pgfqpoint{5.560595in}{1.724808in}}%
\pgfpathcurveto{\pgfqpoint{5.560595in}{1.717675in}}{\pgfqpoint{5.563429in}{1.710833in}}{\pgfqpoint{5.568473in}{1.705789in}}%
\pgfpathcurveto{\pgfqpoint{5.573517in}{1.700746in}}{\pgfqpoint{5.580358in}{1.697912in}}{\pgfqpoint{5.587491in}{1.697912in}}%
\pgfpathclose%
\pgfusepath{stroke,fill}%
\end{pgfscope}%
\begin{pgfscope}%
\pgfpathrectangle{\pgfqpoint{4.985294in}{0.500000in}}{\pgfqpoint{1.764706in}{1.700000in}}%
\pgfusepath{clip}%
\pgfsetbuttcap%
\pgfsetroundjoin%
\definecolor{currentfill}{rgb}{0.969803,0.809811,0.702523}%
\pgfsetfillcolor{currentfill}%
\pgfsetlinewidth{0.311001pt}%
\definecolor{currentstroke}{rgb}{1.000000,1.000000,1.000000}%
\pgfsetstrokecolor{currentstroke}%
\pgfsetdash{}{0pt}%
\pgfpathmoveto{\pgfqpoint{5.524221in}{1.098792in}}%
\pgfpathcurveto{\pgfqpoint{5.531354in}{1.098792in}}{\pgfqpoint{5.538196in}{1.101626in}}{\pgfqpoint{5.543239in}{1.106670in}}%
\pgfpathcurveto{\pgfqpoint{5.548283in}{1.111713in}}{\pgfqpoint{5.551117in}{1.118555in}}{\pgfqpoint{5.551117in}{1.125688in}}%
\pgfpathcurveto{\pgfqpoint{5.551117in}{1.132821in}}{\pgfqpoint{5.548283in}{1.139662in}}{\pgfqpoint{5.543239in}{1.144706in}}%
\pgfpathcurveto{\pgfqpoint{5.538196in}{1.149750in}}{\pgfqpoint{5.531354in}{1.152584in}}{\pgfqpoint{5.524221in}{1.152584in}}%
\pgfpathcurveto{\pgfqpoint{5.517088in}{1.152584in}}{\pgfqpoint{5.510247in}{1.149750in}}{\pgfqpoint{5.505203in}{1.144706in}}%
\pgfpathcurveto{\pgfqpoint{5.500159in}{1.139662in}}{\pgfqpoint{5.497326in}{1.132821in}}{\pgfqpoint{5.497326in}{1.125688in}}%
\pgfpathcurveto{\pgfqpoint{5.497326in}{1.118555in}}{\pgfqpoint{5.500159in}{1.111713in}}{\pgfqpoint{5.505203in}{1.106670in}}%
\pgfpathcurveto{\pgfqpoint{5.510247in}{1.101626in}}{\pgfqpoint{5.517088in}{1.098792in}}{\pgfqpoint{5.524221in}{1.098792in}}%
\pgfpathclose%
\pgfusepath{stroke,fill}%
\end{pgfscope}%
\begin{pgfscope}%
\pgfpathrectangle{\pgfqpoint{4.985294in}{0.500000in}}{\pgfqpoint{1.764706in}{1.700000in}}%
\pgfusepath{clip}%
\pgfsetbuttcap%
\pgfsetroundjoin%
\definecolor{currentfill}{rgb}{0.962765,0.606121,0.444717}%
\pgfsetfillcolor{currentfill}%
\pgfsetlinewidth{0.311001pt}%
\definecolor{currentstroke}{rgb}{1.000000,1.000000,1.000000}%
\pgfsetstrokecolor{currentstroke}%
\pgfsetdash{}{0pt}%
\pgfpathmoveto{\pgfqpoint{5.348250in}{1.544262in}}%
\pgfpathcurveto{\pgfqpoint{5.355383in}{1.544262in}}{\pgfqpoint{5.362225in}{1.547096in}}{\pgfqpoint{5.367269in}{1.552140in}}%
\pgfpathcurveto{\pgfqpoint{5.372312in}{1.557183in}}{\pgfqpoint{5.375146in}{1.564025in}}{\pgfqpoint{5.375146in}{1.571158in}}%
\pgfpathcurveto{\pgfqpoint{5.375146in}{1.578290in}}{\pgfqpoint{5.372312in}{1.585132in}}{\pgfqpoint{5.367269in}{1.590176in}}%
\pgfpathcurveto{\pgfqpoint{5.362225in}{1.595219in}}{\pgfqpoint{5.355383in}{1.598053in}}{\pgfqpoint{5.348250in}{1.598053in}}%
\pgfpathcurveto{\pgfqpoint{5.341118in}{1.598053in}}{\pgfqpoint{5.334276in}{1.595219in}}{\pgfqpoint{5.329232in}{1.590176in}}%
\pgfpathcurveto{\pgfqpoint{5.324189in}{1.585132in}}{\pgfqpoint{5.321355in}{1.578290in}}{\pgfqpoint{5.321355in}{1.571158in}}%
\pgfpathcurveto{\pgfqpoint{5.321355in}{1.564025in}}{\pgfqpoint{5.324189in}{1.557183in}}{\pgfqpoint{5.329232in}{1.552140in}}%
\pgfpathcurveto{\pgfqpoint{5.334276in}{1.547096in}}{\pgfqpoint{5.341118in}{1.544262in}}{\pgfqpoint{5.348250in}{1.544262in}}%
\pgfpathclose%
\pgfusepath{stroke,fill}%
\end{pgfscope}%
\begin{pgfscope}%
\pgfpathrectangle{\pgfqpoint{4.985294in}{0.500000in}}{\pgfqpoint{1.764706in}{1.700000in}}%
\pgfusepath{clip}%
\pgfsetbuttcap%
\pgfsetroundjoin%
\definecolor{currentfill}{rgb}{0.978376,0.897317,0.831308}%
\pgfsetfillcolor{currentfill}%
\pgfsetlinewidth{0.311001pt}%
\definecolor{currentstroke}{rgb}{1.000000,1.000000,1.000000}%
\pgfsetstrokecolor{currentstroke}%
\pgfsetdash{}{0pt}%
\pgfpathmoveto{\pgfqpoint{5.418518in}{1.192585in}}%
\pgfpathcurveto{\pgfqpoint{5.425651in}{1.192585in}}{\pgfqpoint{5.432493in}{1.195419in}}{\pgfqpoint{5.437536in}{1.200463in}}%
\pgfpathcurveto{\pgfqpoint{5.442580in}{1.205507in}}{\pgfqpoint{5.445414in}{1.212348in}}{\pgfqpoint{5.445414in}{1.219481in}}%
\pgfpathcurveto{\pgfqpoint{5.445414in}{1.226614in}}{\pgfqpoint{5.442580in}{1.233456in}}{\pgfqpoint{5.437536in}{1.238499in}}%
\pgfpathcurveto{\pgfqpoint{5.432493in}{1.243543in}}{\pgfqpoint{5.425651in}{1.246377in}}{\pgfqpoint{5.418518in}{1.246377in}}%
\pgfpathcurveto{\pgfqpoint{5.411385in}{1.246377in}}{\pgfqpoint{5.404544in}{1.243543in}}{\pgfqpoint{5.399500in}{1.238499in}}%
\pgfpathcurveto{\pgfqpoint{5.394456in}{1.233456in}}{\pgfqpoint{5.391622in}{1.226614in}}{\pgfqpoint{5.391622in}{1.219481in}}%
\pgfpathcurveto{\pgfqpoint{5.391622in}{1.212348in}}{\pgfqpoint{5.394456in}{1.205507in}}{\pgfqpoint{5.399500in}{1.200463in}}%
\pgfpathcurveto{\pgfqpoint{5.404544in}{1.195419in}}{\pgfqpoint{5.411385in}{1.192585in}}{\pgfqpoint{5.418518in}{1.192585in}}%
\pgfpathclose%
\pgfusepath{stroke,fill}%
\end{pgfscope}%
\begin{pgfscope}%
\pgfpathrectangle{\pgfqpoint{4.985294in}{0.500000in}}{\pgfqpoint{1.764706in}{1.700000in}}%
\pgfusepath{clip}%
\pgfsetbuttcap%
\pgfsetroundjoin%
\definecolor{currentfill}{rgb}{0.976961,0.885681,0.814303}%
\pgfsetfillcolor{currentfill}%
\pgfsetlinewidth{0.311001pt}%
\definecolor{currentstroke}{rgb}{1.000000,1.000000,1.000000}%
\pgfsetstrokecolor{currentstroke}%
\pgfsetdash{}{0pt}%
\pgfpathmoveto{\pgfqpoint{5.462067in}{1.178848in}}%
\pgfpathcurveto{\pgfqpoint{5.469200in}{1.178848in}}{\pgfqpoint{5.476041in}{1.181682in}}{\pgfqpoint{5.481085in}{1.186726in}}%
\pgfpathcurveto{\pgfqpoint{5.486129in}{1.191769in}}{\pgfqpoint{5.488962in}{1.198611in}}{\pgfqpoint{5.488962in}{1.205744in}}%
\pgfpathcurveto{\pgfqpoint{5.488962in}{1.212877in}}{\pgfqpoint{5.486129in}{1.219718in}}{\pgfqpoint{5.481085in}{1.224762in}}%
\pgfpathcurveto{\pgfqpoint{5.476041in}{1.229805in}}{\pgfqpoint{5.469200in}{1.232639in}}{\pgfqpoint{5.462067in}{1.232639in}}%
\pgfpathcurveto{\pgfqpoint{5.454934in}{1.232639in}}{\pgfqpoint{5.448092in}{1.229805in}}{\pgfqpoint{5.443049in}{1.224762in}}%
\pgfpathcurveto{\pgfqpoint{5.438005in}{1.219718in}}{\pgfqpoint{5.435171in}{1.212877in}}{\pgfqpoint{5.435171in}{1.205744in}}%
\pgfpathcurveto{\pgfqpoint{5.435171in}{1.198611in}}{\pgfqpoint{5.438005in}{1.191769in}}{\pgfqpoint{5.443049in}{1.186726in}}%
\pgfpathcurveto{\pgfqpoint{5.448092in}{1.181682in}}{\pgfqpoint{5.454934in}{1.178848in}}{\pgfqpoint{5.462067in}{1.178848in}}%
\pgfpathclose%
\pgfusepath{stroke,fill}%
\end{pgfscope}%
\begin{pgfscope}%
\pgfpathrectangle{\pgfqpoint{4.985294in}{0.500000in}}{\pgfqpoint{1.764706in}{1.700000in}}%
\pgfusepath{clip}%
\pgfsetbuttcap%
\pgfsetroundjoin%
\definecolor{currentfill}{rgb}{0.917171,0.267738,0.242941}%
\pgfsetfillcolor{currentfill}%
\pgfsetlinewidth{0.311001pt}%
\definecolor{currentstroke}{rgb}{1.000000,1.000000,1.000000}%
\pgfsetstrokecolor{currentstroke}%
\pgfsetdash{}{0pt}%
\pgfpathmoveto{\pgfqpoint{6.126303in}{1.313100in}}%
\pgfpathcurveto{\pgfqpoint{6.133436in}{1.313100in}}{\pgfqpoint{6.140277in}{1.315934in}}{\pgfqpoint{6.145321in}{1.320978in}}%
\pgfpathcurveto{\pgfqpoint{6.150365in}{1.326021in}}{\pgfqpoint{6.153198in}{1.332863in}}{\pgfqpoint{6.153198in}{1.339996in}}%
\pgfpathcurveto{\pgfqpoint{6.153198in}{1.347129in}}{\pgfqpoint{6.150365in}{1.353970in}}{\pgfqpoint{6.145321in}{1.359014in}}%
\pgfpathcurveto{\pgfqpoint{6.140277in}{1.364058in}}{\pgfqpoint{6.133436in}{1.366892in}}{\pgfqpoint{6.126303in}{1.366892in}}%
\pgfpathcurveto{\pgfqpoint{6.119170in}{1.366892in}}{\pgfqpoint{6.112328in}{1.364058in}}{\pgfqpoint{6.107285in}{1.359014in}}%
\pgfpathcurveto{\pgfqpoint{6.102241in}{1.353970in}}{\pgfqpoint{6.099407in}{1.347129in}}{\pgfqpoint{6.099407in}{1.339996in}}%
\pgfpathcurveto{\pgfqpoint{6.099407in}{1.332863in}}{\pgfqpoint{6.102241in}{1.326021in}}{\pgfqpoint{6.107285in}{1.320978in}}%
\pgfpathcurveto{\pgfqpoint{6.112328in}{1.315934in}}{\pgfqpoint{6.119170in}{1.313100in}}{\pgfqpoint{6.126303in}{1.313100in}}%
\pgfpathclose%
\pgfusepath{stroke,fill}%
\end{pgfscope}%
\begin{pgfscope}%
\pgfpathrectangle{\pgfqpoint{4.985294in}{0.500000in}}{\pgfqpoint{1.764706in}{1.700000in}}%
\pgfusepath{clip}%
\pgfsetbuttcap%
\pgfsetroundjoin%
\definecolor{currentfill}{rgb}{0.970255,0.815666,0.711203}%
\pgfsetfillcolor{currentfill}%
\pgfsetlinewidth{0.311001pt}%
\definecolor{currentstroke}{rgb}{1.000000,1.000000,1.000000}%
\pgfsetstrokecolor{currentstroke}%
\pgfsetdash{}{0pt}%
\pgfpathmoveto{\pgfqpoint{6.181445in}{1.597068in}}%
\pgfpathcurveto{\pgfqpoint{6.188577in}{1.597068in}}{\pgfqpoint{6.195419in}{1.599902in}}{\pgfqpoint{6.200463in}{1.604945in}}%
\pgfpathcurveto{\pgfqpoint{6.205506in}{1.609989in}}{\pgfqpoint{6.208340in}{1.616831in}}{\pgfqpoint{6.208340in}{1.623964in}}%
\pgfpathcurveto{\pgfqpoint{6.208340in}{1.631096in}}{\pgfqpoint{6.205506in}{1.637938in}}{\pgfqpoint{6.200463in}{1.642982in}}%
\pgfpathcurveto{\pgfqpoint{6.195419in}{1.648025in}}{\pgfqpoint{6.188577in}{1.650859in}}{\pgfqpoint{6.181445in}{1.650859in}}%
\pgfpathcurveto{\pgfqpoint{6.174312in}{1.650859in}}{\pgfqpoint{6.167470in}{1.648025in}}{\pgfqpoint{6.162426in}{1.642982in}}%
\pgfpathcurveto{\pgfqpoint{6.157383in}{1.637938in}}{\pgfqpoint{6.154549in}{1.631096in}}{\pgfqpoint{6.154549in}{1.623964in}}%
\pgfpathcurveto{\pgfqpoint{6.154549in}{1.616831in}}{\pgfqpoint{6.157383in}{1.609989in}}{\pgfqpoint{6.162426in}{1.604945in}}%
\pgfpathcurveto{\pgfqpoint{6.167470in}{1.599902in}}{\pgfqpoint{6.174312in}{1.597068in}}{\pgfqpoint{6.181445in}{1.597068in}}%
\pgfpathclose%
\pgfusepath{stroke,fill}%
\end{pgfscope}%
\begin{pgfscope}%
\pgfpathrectangle{\pgfqpoint{4.985294in}{0.500000in}}{\pgfqpoint{1.764706in}{1.700000in}}%
\pgfusepath{clip}%
\pgfsetbuttcap%
\pgfsetroundjoin%
\definecolor{currentfill}{rgb}{0.964799,0.689101,0.537560}%
\pgfsetfillcolor{currentfill}%
\pgfsetlinewidth{0.311001pt}%
\definecolor{currentstroke}{rgb}{1.000000,1.000000,1.000000}%
\pgfsetstrokecolor{currentstroke}%
\pgfsetdash{}{0pt}%
\pgfpathmoveto{\pgfqpoint{6.413418in}{1.265412in}}%
\pgfpathcurveto{\pgfqpoint{6.420551in}{1.265412in}}{\pgfqpoint{6.427393in}{1.268246in}}{\pgfqpoint{6.432437in}{1.273290in}}%
\pgfpathcurveto{\pgfqpoint{6.437480in}{1.278333in}}{\pgfqpoint{6.440314in}{1.285175in}}{\pgfqpoint{6.440314in}{1.292308in}}%
\pgfpathcurveto{\pgfqpoint{6.440314in}{1.299441in}}{\pgfqpoint{6.437480in}{1.306282in}}{\pgfqpoint{6.432437in}{1.311326in}}%
\pgfpathcurveto{\pgfqpoint{6.427393in}{1.316370in}}{\pgfqpoint{6.420551in}{1.319203in}}{\pgfqpoint{6.413418in}{1.319203in}}%
\pgfpathcurveto{\pgfqpoint{6.406286in}{1.319203in}}{\pgfqpoint{6.399444in}{1.316370in}}{\pgfqpoint{6.394400in}{1.311326in}}%
\pgfpathcurveto{\pgfqpoint{6.389357in}{1.306282in}}{\pgfqpoint{6.386523in}{1.299441in}}{\pgfqpoint{6.386523in}{1.292308in}}%
\pgfpathcurveto{\pgfqpoint{6.386523in}{1.285175in}}{\pgfqpoint{6.389357in}{1.278333in}}{\pgfqpoint{6.394400in}{1.273290in}}%
\pgfpathcurveto{\pgfqpoint{6.399444in}{1.268246in}}{\pgfqpoint{6.406286in}{1.265412in}}{\pgfqpoint{6.413418in}{1.265412in}}%
\pgfpathclose%
\pgfusepath{stroke,fill}%
\end{pgfscope}%
\begin{pgfscope}%
\pgfpathrectangle{\pgfqpoint{4.985294in}{0.500000in}}{\pgfqpoint{1.764706in}{1.700000in}}%
\pgfusepath{clip}%
\pgfsetbuttcap%
\pgfsetroundjoin%
\definecolor{currentfill}{rgb}{0.930781,0.313740,0.244688}%
\pgfsetfillcolor{currentfill}%
\pgfsetlinewidth{0.311001pt}%
\definecolor{currentstroke}{rgb}{1.000000,1.000000,1.000000}%
\pgfsetstrokecolor{currentstroke}%
\pgfsetdash{}{0pt}%
\pgfpathmoveto{\pgfqpoint{6.004165in}{1.717217in}}%
\pgfpathcurveto{\pgfqpoint{6.011298in}{1.717217in}}{\pgfqpoint{6.018139in}{1.720051in}}{\pgfqpoint{6.023183in}{1.725094in}}%
\pgfpathcurveto{\pgfqpoint{6.028227in}{1.730138in}}{\pgfqpoint{6.031060in}{1.736980in}}{\pgfqpoint{6.031060in}{1.744113in}}%
\pgfpathcurveto{\pgfqpoint{6.031060in}{1.751245in}}{\pgfqpoint{6.028227in}{1.758087in}}{\pgfqpoint{6.023183in}{1.763131in}}%
\pgfpathcurveto{\pgfqpoint{6.018139in}{1.768174in}}{\pgfqpoint{6.011298in}{1.771008in}}{\pgfqpoint{6.004165in}{1.771008in}}%
\pgfpathcurveto{\pgfqpoint{5.997032in}{1.771008in}}{\pgfqpoint{5.990190in}{1.768174in}}{\pgfqpoint{5.985147in}{1.763131in}}%
\pgfpathcurveto{\pgfqpoint{5.980103in}{1.758087in}}{\pgfqpoint{5.977269in}{1.751245in}}{\pgfqpoint{5.977269in}{1.744113in}}%
\pgfpathcurveto{\pgfqpoint{5.977269in}{1.736980in}}{\pgfqpoint{5.980103in}{1.730138in}}{\pgfqpoint{5.985147in}{1.725094in}}%
\pgfpathcurveto{\pgfqpoint{5.990190in}{1.720051in}}{\pgfqpoint{5.997032in}{1.717217in}}{\pgfqpoint{6.004165in}{1.717217in}}%
\pgfpathclose%
\pgfusepath{stroke,fill}%
\end{pgfscope}%
\begin{pgfscope}%
\pgfpathrectangle{\pgfqpoint{4.985294in}{0.500000in}}{\pgfqpoint{1.764706in}{1.700000in}}%
\pgfusepath{clip}%
\pgfsetbuttcap%
\pgfsetroundjoin%
\definecolor{currentfill}{rgb}{0.942910,0.375495,0.263698}%
\pgfsetfillcolor{currentfill}%
\pgfsetlinewidth{0.311001pt}%
\definecolor{currentstroke}{rgb}{1.000000,1.000000,1.000000}%
\pgfsetstrokecolor{currentstroke}%
\pgfsetdash{}{0pt}%
\pgfpathmoveto{\pgfqpoint{5.564737in}{1.819242in}}%
\pgfpathcurveto{\pgfqpoint{5.571870in}{1.819242in}}{\pgfqpoint{5.578711in}{1.822076in}}{\pgfqpoint{5.583755in}{1.827119in}}%
\pgfpathcurveto{\pgfqpoint{5.588799in}{1.832163in}}{\pgfqpoint{5.591633in}{1.839005in}}{\pgfqpoint{5.591633in}{1.846137in}}%
\pgfpathcurveto{\pgfqpoint{5.591633in}{1.853270in}}{\pgfqpoint{5.588799in}{1.860112in}}{\pgfqpoint{5.583755in}{1.865156in}}%
\pgfpathcurveto{\pgfqpoint{5.578711in}{1.870199in}}{\pgfqpoint{5.571870in}{1.873033in}}{\pgfqpoint{5.564737in}{1.873033in}}%
\pgfpathcurveto{\pgfqpoint{5.557604in}{1.873033in}}{\pgfqpoint{5.550762in}{1.870199in}}{\pgfqpoint{5.545719in}{1.865156in}}%
\pgfpathcurveto{\pgfqpoint{5.540675in}{1.860112in}}{\pgfqpoint{5.537841in}{1.853270in}}{\pgfqpoint{5.537841in}{1.846137in}}%
\pgfpathcurveto{\pgfqpoint{5.537841in}{1.839005in}}{\pgfqpoint{5.540675in}{1.832163in}}{\pgfqpoint{5.545719in}{1.827119in}}%
\pgfpathcurveto{\pgfqpoint{5.550762in}{1.822076in}}{\pgfqpoint{5.557604in}{1.819242in}}{\pgfqpoint{5.564737in}{1.819242in}}%
\pgfpathclose%
\pgfusepath{stroke,fill}%
\end{pgfscope}%
\begin{pgfscope}%
\pgfpathrectangle{\pgfqpoint{4.985294in}{0.500000in}}{\pgfqpoint{1.764706in}{1.700000in}}%
\pgfusepath{clip}%
\pgfsetbuttcap%
\pgfsetroundjoin%
\definecolor{currentfill}{rgb}{0.967092,0.768560,0.642079}%
\pgfsetfillcolor{currentfill}%
\pgfsetlinewidth{0.311001pt}%
\definecolor{currentstroke}{rgb}{1.000000,1.000000,1.000000}%
\pgfsetstrokecolor{currentstroke}%
\pgfsetdash{}{0pt}%
\pgfpathmoveto{\pgfqpoint{6.267249in}{0.988362in}}%
\pgfpathcurveto{\pgfqpoint{6.274381in}{0.988362in}}{\pgfqpoint{6.281223in}{0.991195in}}{\pgfqpoint{6.286267in}{0.996239in}}%
\pgfpathcurveto{\pgfqpoint{6.291310in}{1.001283in}}{\pgfqpoint{6.294144in}{1.008124in}}{\pgfqpoint{6.294144in}{1.015257in}}%
\pgfpathcurveto{\pgfqpoint{6.294144in}{1.022390in}}{\pgfqpoint{6.291310in}{1.029232in}}{\pgfqpoint{6.286267in}{1.034275in}}%
\pgfpathcurveto{\pgfqpoint{6.281223in}{1.039319in}}{\pgfqpoint{6.274381in}{1.042153in}}{\pgfqpoint{6.267249in}{1.042153in}}%
\pgfpathcurveto{\pgfqpoint{6.260116in}{1.042153in}}{\pgfqpoint{6.253274in}{1.039319in}}{\pgfqpoint{6.248230in}{1.034275in}}%
\pgfpathcurveto{\pgfqpoint{6.243187in}{1.029232in}}{\pgfqpoint{6.240353in}{1.022390in}}{\pgfqpoint{6.240353in}{1.015257in}}%
\pgfpathcurveto{\pgfqpoint{6.240353in}{1.008124in}}{\pgfqpoint{6.243187in}{1.001283in}}{\pgfqpoint{6.248230in}{0.996239in}}%
\pgfpathcurveto{\pgfqpoint{6.253274in}{0.991195in}}{\pgfqpoint{6.260116in}{0.988362in}}{\pgfqpoint{6.267249in}{0.988362in}}%
\pgfpathclose%
\pgfusepath{stroke,fill}%
\end{pgfscope}%
\begin{pgfscope}%
\pgfpathrectangle{\pgfqpoint{4.985294in}{0.500000in}}{\pgfqpoint{1.764706in}{1.700000in}}%
\pgfusepath{clip}%
\pgfsetbuttcap%
\pgfsetroundjoin%
\definecolor{currentfill}{rgb}{0.974412,0.862387,0.780156}%
\pgfsetfillcolor{currentfill}%
\pgfsetlinewidth{0.311001pt}%
\definecolor{currentstroke}{rgb}{1.000000,1.000000,1.000000}%
\pgfsetstrokecolor{currentstroke}%
\pgfsetdash{}{0pt}%
\pgfpathmoveto{\pgfqpoint{6.228093in}{1.062034in}}%
\pgfpathcurveto{\pgfqpoint{6.235226in}{1.062034in}}{\pgfqpoint{6.242067in}{1.064868in}}{\pgfqpoint{6.247111in}{1.069912in}}%
\pgfpathcurveto{\pgfqpoint{6.252155in}{1.074956in}}{\pgfqpoint{6.254989in}{1.081797in}}{\pgfqpoint{6.254989in}{1.088930in}}%
\pgfpathcurveto{\pgfqpoint{6.254989in}{1.096063in}}{\pgfqpoint{6.252155in}{1.102905in}}{\pgfqpoint{6.247111in}{1.107948in}}%
\pgfpathcurveto{\pgfqpoint{6.242067in}{1.112992in}}{\pgfqpoint{6.235226in}{1.115826in}}{\pgfqpoint{6.228093in}{1.115826in}}%
\pgfpathcurveto{\pgfqpoint{6.220960in}{1.115826in}}{\pgfqpoint{6.214118in}{1.112992in}}{\pgfqpoint{6.209075in}{1.107948in}}%
\pgfpathcurveto{\pgfqpoint{6.204031in}{1.102905in}}{\pgfqpoint{6.201197in}{1.096063in}}{\pgfqpoint{6.201197in}{1.088930in}}%
\pgfpathcurveto{\pgfqpoint{6.201197in}{1.081797in}}{\pgfqpoint{6.204031in}{1.074956in}}{\pgfqpoint{6.209075in}{1.069912in}}%
\pgfpathcurveto{\pgfqpoint{6.214118in}{1.064868in}}{\pgfqpoint{6.220960in}{1.062034in}}{\pgfqpoint{6.228093in}{1.062034in}}%
\pgfpathclose%
\pgfusepath{stroke,fill}%
\end{pgfscope}%
\begin{pgfscope}%
\pgfpathrectangle{\pgfqpoint{4.985294in}{0.500000in}}{\pgfqpoint{1.764706in}{1.700000in}}%
\pgfusepath{clip}%
\pgfsetbuttcap%
\pgfsetroundjoin%
\definecolor{currentfill}{rgb}{0.966328,0.750560,0.616961}%
\pgfsetfillcolor{currentfill}%
\pgfsetlinewidth{0.311001pt}%
\definecolor{currentstroke}{rgb}{1.000000,1.000000,1.000000}%
\pgfsetstrokecolor{currentstroke}%
\pgfsetdash{}{0pt}%
\pgfpathmoveto{\pgfqpoint{5.548380in}{1.680020in}}%
\pgfpathcurveto{\pgfqpoint{5.555513in}{1.680020in}}{\pgfqpoint{5.562354in}{1.682854in}}{\pgfqpoint{5.567398in}{1.687897in}}%
\pgfpathcurveto{\pgfqpoint{5.572442in}{1.692941in}}{\pgfqpoint{5.575276in}{1.699783in}}{\pgfqpoint{5.575276in}{1.706915in}}%
\pgfpathcurveto{\pgfqpoint{5.575276in}{1.714048in}}{\pgfqpoint{5.572442in}{1.720890in}}{\pgfqpoint{5.567398in}{1.725934in}}%
\pgfpathcurveto{\pgfqpoint{5.562354in}{1.730977in}}{\pgfqpoint{5.555513in}{1.733811in}}{\pgfqpoint{5.548380in}{1.733811in}}%
\pgfpathcurveto{\pgfqpoint{5.541247in}{1.733811in}}{\pgfqpoint{5.534405in}{1.730977in}}{\pgfqpoint{5.529362in}{1.725934in}}%
\pgfpathcurveto{\pgfqpoint{5.524318in}{1.720890in}}{\pgfqpoint{5.521484in}{1.714048in}}{\pgfqpoint{5.521484in}{1.706915in}}%
\pgfpathcurveto{\pgfqpoint{5.521484in}{1.699783in}}{\pgfqpoint{5.524318in}{1.692941in}}{\pgfqpoint{5.529362in}{1.687897in}}%
\pgfpathcurveto{\pgfqpoint{5.534405in}{1.682854in}}{\pgfqpoint{5.541247in}{1.680020in}}{\pgfqpoint{5.548380in}{1.680020in}}%
\pgfpathclose%
\pgfusepath{stroke,fill}%
\end{pgfscope}%
\begin{pgfscope}%
\pgfpathrectangle{\pgfqpoint{4.985294in}{0.500000in}}{\pgfqpoint{1.764706in}{1.700000in}}%
\pgfusepath{clip}%
\pgfsetbuttcap%
\pgfsetroundjoin%
\definecolor{currentfill}{rgb}{0.968105,0.786346,0.667739}%
\pgfsetfillcolor{currentfill}%
\pgfsetlinewidth{0.311001pt}%
\definecolor{currentstroke}{rgb}{1.000000,1.000000,1.000000}%
\pgfsetstrokecolor{currentstroke}%
\pgfsetdash{}{0pt}%
\pgfpathmoveto{\pgfqpoint{6.201612in}{1.192552in}}%
\pgfpathcurveto{\pgfqpoint{6.208745in}{1.192552in}}{\pgfqpoint{6.215586in}{1.195386in}}{\pgfqpoint{6.220630in}{1.200430in}}%
\pgfpathcurveto{\pgfqpoint{6.225674in}{1.205474in}}{\pgfqpoint{6.228508in}{1.212315in}}{\pgfqpoint{6.228508in}{1.219448in}}%
\pgfpathcurveto{\pgfqpoint{6.228508in}{1.226581in}}{\pgfqpoint{6.225674in}{1.233423in}}{\pgfqpoint{6.220630in}{1.238466in}}%
\pgfpathcurveto{\pgfqpoint{6.215586in}{1.243510in}}{\pgfqpoint{6.208745in}{1.246344in}}{\pgfqpoint{6.201612in}{1.246344in}}%
\pgfpathcurveto{\pgfqpoint{6.194479in}{1.246344in}}{\pgfqpoint{6.187638in}{1.243510in}}{\pgfqpoint{6.182594in}{1.238466in}}%
\pgfpathcurveto{\pgfqpoint{6.177550in}{1.233423in}}{\pgfqpoint{6.174716in}{1.226581in}}{\pgfqpoint{6.174716in}{1.219448in}}%
\pgfpathcurveto{\pgfqpoint{6.174716in}{1.212315in}}{\pgfqpoint{6.177550in}{1.205474in}}{\pgfqpoint{6.182594in}{1.200430in}}%
\pgfpathcurveto{\pgfqpoint{6.187638in}{1.195386in}}{\pgfqpoint{6.194479in}{1.192552in}}{\pgfqpoint{6.201612in}{1.192552in}}%
\pgfpathclose%
\pgfusepath{stroke,fill}%
\end{pgfscope}%
\begin{pgfscope}%
\pgfpathrectangle{\pgfqpoint{4.985294in}{0.500000in}}{\pgfqpoint{1.764706in}{1.700000in}}%
\pgfusepath{clip}%
\pgfsetbuttcap%
\pgfsetroundjoin%
\definecolor{currentfill}{rgb}{0.972726,0.844889,0.754401}%
\pgfsetfillcolor{currentfill}%
\pgfsetlinewidth{0.311001pt}%
\definecolor{currentstroke}{rgb}{1.000000,1.000000,1.000000}%
\pgfsetstrokecolor{currentstroke}%
\pgfsetdash{}{0pt}%
\pgfpathmoveto{\pgfqpoint{5.368706in}{1.402582in}}%
\pgfpathcurveto{\pgfqpoint{5.375838in}{1.402582in}}{\pgfqpoint{5.382680in}{1.405416in}}{\pgfqpoint{5.387724in}{1.410460in}}%
\pgfpathcurveto{\pgfqpoint{5.392767in}{1.415503in}}{\pgfqpoint{5.395601in}{1.422345in}}{\pgfqpoint{5.395601in}{1.429478in}}%
\pgfpathcurveto{\pgfqpoint{5.395601in}{1.436611in}}{\pgfqpoint{5.392767in}{1.443452in}}{\pgfqpoint{5.387724in}{1.448496in}}%
\pgfpathcurveto{\pgfqpoint{5.382680in}{1.453540in}}{\pgfqpoint{5.375838in}{1.456374in}}{\pgfqpoint{5.368706in}{1.456374in}}%
\pgfpathcurveto{\pgfqpoint{5.361573in}{1.456374in}}{\pgfqpoint{5.354731in}{1.453540in}}{\pgfqpoint{5.349687in}{1.448496in}}%
\pgfpathcurveto{\pgfqpoint{5.344644in}{1.443452in}}{\pgfqpoint{5.341810in}{1.436611in}}{\pgfqpoint{5.341810in}{1.429478in}}%
\pgfpathcurveto{\pgfqpoint{5.341810in}{1.422345in}}{\pgfqpoint{5.344644in}{1.415503in}}{\pgfqpoint{5.349687in}{1.410460in}}%
\pgfpathcurveto{\pgfqpoint{5.354731in}{1.405416in}}{\pgfqpoint{5.361573in}{1.402582in}}{\pgfqpoint{5.368706in}{1.402582in}}%
\pgfpathclose%
\pgfusepath{stroke,fill}%
\end{pgfscope}%
\begin{pgfscope}%
\pgfpathrectangle{\pgfqpoint{4.985294in}{0.500000in}}{\pgfqpoint{1.764706in}{1.700000in}}%
\pgfusepath{clip}%
\pgfsetbuttcap%
\pgfsetroundjoin%
\definecolor{currentfill}{rgb}{0.818205,0.120806,0.299261}%
\pgfsetfillcolor{currentfill}%
\pgfsetlinewidth{0.311001pt}%
\definecolor{currentstroke}{rgb}{1.000000,1.000000,1.000000}%
\pgfsetstrokecolor{currentstroke}%
\pgfsetdash{}{0pt}%
\pgfpathmoveto{\pgfqpoint{6.066855in}{1.128483in}}%
\pgfpathcurveto{\pgfqpoint{6.073988in}{1.128483in}}{\pgfqpoint{6.080829in}{1.131317in}}{\pgfqpoint{6.085873in}{1.136360in}}%
\pgfpathcurveto{\pgfqpoint{6.090917in}{1.141404in}}{\pgfqpoint{6.093750in}{1.148246in}}{\pgfqpoint{6.093750in}{1.155378in}}%
\pgfpathcurveto{\pgfqpoint{6.093750in}{1.162511in}}{\pgfqpoint{6.090917in}{1.169353in}}{\pgfqpoint{6.085873in}{1.174397in}}%
\pgfpathcurveto{\pgfqpoint{6.080829in}{1.179440in}}{\pgfqpoint{6.073988in}{1.182274in}}{\pgfqpoint{6.066855in}{1.182274in}}%
\pgfpathcurveto{\pgfqpoint{6.059722in}{1.182274in}}{\pgfqpoint{6.052880in}{1.179440in}}{\pgfqpoint{6.047837in}{1.174397in}}%
\pgfpathcurveto{\pgfqpoint{6.042793in}{1.169353in}}{\pgfqpoint{6.039959in}{1.162511in}}{\pgfqpoint{6.039959in}{1.155378in}}%
\pgfpathcurveto{\pgfqpoint{6.039959in}{1.148246in}}{\pgfqpoint{6.042793in}{1.141404in}}{\pgfqpoint{6.047837in}{1.136360in}}%
\pgfpathcurveto{\pgfqpoint{6.052880in}{1.131317in}}{\pgfqpoint{6.059722in}{1.128483in}}{\pgfqpoint{6.066855in}{1.128483in}}%
\pgfpathclose%
\pgfusepath{stroke,fill}%
\end{pgfscope}%
\begin{pgfscope}%
\pgfpathrectangle{\pgfqpoint{4.985294in}{0.500000in}}{\pgfqpoint{1.764706in}{1.700000in}}%
\pgfusepath{clip}%
\pgfsetbuttcap%
\pgfsetroundjoin%
\definecolor{currentfill}{rgb}{0.970255,0.815666,0.711203}%
\pgfsetfillcolor{currentfill}%
\pgfsetlinewidth{0.311001pt}%
\definecolor{currentstroke}{rgb}{1.000000,1.000000,1.000000}%
\pgfsetstrokecolor{currentstroke}%
\pgfsetdash{}{0pt}%
\pgfpathmoveto{\pgfqpoint{6.186128in}{1.055414in}}%
\pgfpathcurveto{\pgfqpoint{6.193261in}{1.055414in}}{\pgfqpoint{6.200103in}{1.058248in}}{\pgfqpoint{6.205146in}{1.063292in}}%
\pgfpathcurveto{\pgfqpoint{6.210190in}{1.068335in}}{\pgfqpoint{6.213024in}{1.075177in}}{\pgfqpoint{6.213024in}{1.082310in}}%
\pgfpathcurveto{\pgfqpoint{6.213024in}{1.089443in}}{\pgfqpoint{6.210190in}{1.096284in}}{\pgfqpoint{6.205146in}{1.101328in}}%
\pgfpathcurveto{\pgfqpoint{6.200103in}{1.106372in}}{\pgfqpoint{6.193261in}{1.109206in}}{\pgfqpoint{6.186128in}{1.109206in}}%
\pgfpathcurveto{\pgfqpoint{6.178996in}{1.109206in}}{\pgfqpoint{6.172154in}{1.106372in}}{\pgfqpoint{6.167110in}{1.101328in}}%
\pgfpathcurveto{\pgfqpoint{6.162067in}{1.096284in}}{\pgfqpoint{6.159233in}{1.089443in}}{\pgfqpoint{6.159233in}{1.082310in}}%
\pgfpathcurveto{\pgfqpoint{6.159233in}{1.075177in}}{\pgfqpoint{6.162067in}{1.068335in}}{\pgfqpoint{6.167110in}{1.063292in}}%
\pgfpathcurveto{\pgfqpoint{6.172154in}{1.058248in}}{\pgfqpoint{6.178996in}{1.055414in}}{\pgfqpoint{6.186128in}{1.055414in}}%
\pgfpathclose%
\pgfusepath{stroke,fill}%
\end{pgfscope}%
\begin{pgfscope}%
\pgfpathrectangle{\pgfqpoint{4.985294in}{0.500000in}}{\pgfqpoint{1.764706in}{1.700000in}}%
\pgfusepath{clip}%
\pgfsetbuttcap%
\pgfsetroundjoin%
\definecolor{currentfill}{rgb}{0.966812,0.762584,0.633643}%
\pgfsetfillcolor{currentfill}%
\pgfsetlinewidth{0.311001pt}%
\definecolor{currentstroke}{rgb}{1.000000,1.000000,1.000000}%
\pgfsetstrokecolor{currentstroke}%
\pgfsetdash{}{0pt}%
\pgfpathmoveto{\pgfqpoint{6.161061in}{1.045533in}}%
\pgfpathcurveto{\pgfqpoint{6.168194in}{1.045533in}}{\pgfqpoint{6.175036in}{1.048367in}}{\pgfqpoint{6.180079in}{1.053411in}}%
\pgfpathcurveto{\pgfqpoint{6.185123in}{1.058455in}}{\pgfqpoint{6.187957in}{1.065296in}}{\pgfqpoint{6.187957in}{1.072429in}}%
\pgfpathcurveto{\pgfqpoint{6.187957in}{1.079562in}}{\pgfqpoint{6.185123in}{1.086403in}}{\pgfqpoint{6.180079in}{1.091447in}}%
\pgfpathcurveto{\pgfqpoint{6.175036in}{1.096491in}}{\pgfqpoint{6.168194in}{1.099325in}}{\pgfqpoint{6.161061in}{1.099325in}}%
\pgfpathcurveto{\pgfqpoint{6.153928in}{1.099325in}}{\pgfqpoint{6.147087in}{1.096491in}}{\pgfqpoint{6.142043in}{1.091447in}}%
\pgfpathcurveto{\pgfqpoint{6.136999in}{1.086403in}}{\pgfqpoint{6.134165in}{1.079562in}}{\pgfqpoint{6.134165in}{1.072429in}}%
\pgfpathcurveto{\pgfqpoint{6.134165in}{1.065296in}}{\pgfqpoint{6.136999in}{1.058455in}}{\pgfqpoint{6.142043in}{1.053411in}}%
\pgfpathcurveto{\pgfqpoint{6.147087in}{1.048367in}}{\pgfqpoint{6.153928in}{1.045533in}}{\pgfqpoint{6.161061in}{1.045533in}}%
\pgfpathclose%
\pgfusepath{stroke,fill}%
\end{pgfscope}%
\begin{pgfscope}%
\pgfpathrectangle{\pgfqpoint{4.985294in}{0.500000in}}{\pgfqpoint{1.764706in}{1.700000in}}%
\pgfusepath{clip}%
\pgfsetbuttcap%
\pgfsetroundjoin%
\definecolor{currentfill}{rgb}{0.967092,0.768560,0.642079}%
\pgfsetfillcolor{currentfill}%
\pgfsetlinewidth{0.311001pt}%
\definecolor{currentstroke}{rgb}{1.000000,1.000000,1.000000}%
\pgfsetstrokecolor{currentstroke}%
\pgfsetdash{}{0pt}%
\pgfpathmoveto{\pgfqpoint{5.543438in}{1.094411in}}%
\pgfpathcurveto{\pgfqpoint{5.550570in}{1.094411in}}{\pgfqpoint{5.557412in}{1.097244in}}{\pgfqpoint{5.562456in}{1.102288in}}%
\pgfpathcurveto{\pgfqpoint{5.567499in}{1.107332in}}{\pgfqpoint{5.570333in}{1.114173in}}{\pgfqpoint{5.570333in}{1.121306in}}%
\pgfpathcurveto{\pgfqpoint{5.570333in}{1.128439in}}{\pgfqpoint{5.567499in}{1.135281in}}{\pgfqpoint{5.562456in}{1.140324in}}%
\pgfpathcurveto{\pgfqpoint{5.557412in}{1.145368in}}{\pgfqpoint{5.550570in}{1.148202in}}{\pgfqpoint{5.543438in}{1.148202in}}%
\pgfpathcurveto{\pgfqpoint{5.536305in}{1.148202in}}{\pgfqpoint{5.529463in}{1.145368in}}{\pgfqpoint{5.524419in}{1.140324in}}%
\pgfpathcurveto{\pgfqpoint{5.519376in}{1.135281in}}{\pgfqpoint{5.516542in}{1.128439in}}{\pgfqpoint{5.516542in}{1.121306in}}%
\pgfpathcurveto{\pgfqpoint{5.516542in}{1.114173in}}{\pgfqpoint{5.519376in}{1.107332in}}{\pgfqpoint{5.524419in}{1.102288in}}%
\pgfpathcurveto{\pgfqpoint{5.529463in}{1.097244in}}{\pgfqpoint{5.536305in}{1.094411in}}{\pgfqpoint{5.543438in}{1.094411in}}%
\pgfpathclose%
\pgfusepath{stroke,fill}%
\end{pgfscope}%
\begin{pgfscope}%
\pgfpathrectangle{\pgfqpoint{4.985294in}{0.500000in}}{\pgfqpoint{1.764706in}{1.700000in}}%
\pgfusepath{clip}%
\pgfsetbuttcap%
\pgfsetroundjoin%
\definecolor{currentfill}{rgb}{0.979124,0.903132,0.839793}%
\pgfsetfillcolor{currentfill}%
\pgfsetlinewidth{0.311001pt}%
\definecolor{currentstroke}{rgb}{1.000000,1.000000,1.000000}%
\pgfsetstrokecolor{currentstroke}%
\pgfsetdash{}{0pt}%
\pgfpathmoveto{\pgfqpoint{6.323361in}{1.456659in}}%
\pgfpathcurveto{\pgfqpoint{6.330494in}{1.456659in}}{\pgfqpoint{6.337336in}{1.459493in}}{\pgfqpoint{6.342379in}{1.464536in}}%
\pgfpathcurveto{\pgfqpoint{6.347423in}{1.469580in}}{\pgfqpoint{6.350257in}{1.476422in}}{\pgfqpoint{6.350257in}{1.483555in}}%
\pgfpathcurveto{\pgfqpoint{6.350257in}{1.490687in}}{\pgfqpoint{6.347423in}{1.497529in}}{\pgfqpoint{6.342379in}{1.502573in}}%
\pgfpathcurveto{\pgfqpoint{6.337336in}{1.507616in}}{\pgfqpoint{6.330494in}{1.510450in}}{\pgfqpoint{6.323361in}{1.510450in}}%
\pgfpathcurveto{\pgfqpoint{6.316228in}{1.510450in}}{\pgfqpoint{6.309387in}{1.507616in}}{\pgfqpoint{6.304343in}{1.502573in}}%
\pgfpathcurveto{\pgfqpoint{6.299299in}{1.497529in}}{\pgfqpoint{6.296465in}{1.490687in}}{\pgfqpoint{6.296465in}{1.483555in}}%
\pgfpathcurveto{\pgfqpoint{6.296465in}{1.476422in}}{\pgfqpoint{6.299299in}{1.469580in}}{\pgfqpoint{6.304343in}{1.464536in}}%
\pgfpathcurveto{\pgfqpoint{6.309387in}{1.459493in}}{\pgfqpoint{6.316228in}{1.456659in}}{\pgfqpoint{6.323361in}{1.456659in}}%
\pgfpathclose%
\pgfusepath{stroke,fill}%
\end{pgfscope}%
\begin{pgfscope}%
\pgfpathrectangle{\pgfqpoint{4.985294in}{0.500000in}}{\pgfqpoint{1.764706in}{1.700000in}}%
\pgfusepath{clip}%
\pgfsetbuttcap%
\pgfsetroundjoin%
\definecolor{currentfill}{rgb}{0.981377,0.920617,0.865369}%
\pgfsetfillcolor{currentfill}%
\pgfsetlinewidth{0.311001pt}%
\definecolor{currentstroke}{rgb}{1.000000,1.000000,1.000000}%
\pgfsetstrokecolor{currentstroke}%
\pgfsetdash{}{0pt}%
\pgfpathmoveto{\pgfqpoint{6.313848in}{1.310232in}}%
\pgfpathcurveto{\pgfqpoint{6.320981in}{1.310232in}}{\pgfqpoint{6.327823in}{1.313066in}}{\pgfqpoint{6.332866in}{1.318110in}}%
\pgfpathcurveto{\pgfqpoint{6.337910in}{1.323154in}}{\pgfqpoint{6.340744in}{1.329995in}}{\pgfqpoint{6.340744in}{1.337128in}}%
\pgfpathcurveto{\pgfqpoint{6.340744in}{1.344261in}}{\pgfqpoint{6.337910in}{1.351102in}}{\pgfqpoint{6.332866in}{1.356146in}}%
\pgfpathcurveto{\pgfqpoint{6.327823in}{1.361190in}}{\pgfqpoint{6.320981in}{1.364024in}}{\pgfqpoint{6.313848in}{1.364024in}}%
\pgfpathcurveto{\pgfqpoint{6.306715in}{1.364024in}}{\pgfqpoint{6.299874in}{1.361190in}}{\pgfqpoint{6.294830in}{1.356146in}}%
\pgfpathcurveto{\pgfqpoint{6.289786in}{1.351102in}}{\pgfqpoint{6.286952in}{1.344261in}}{\pgfqpoint{6.286952in}{1.337128in}}%
\pgfpathcurveto{\pgfqpoint{6.286952in}{1.329995in}}{\pgfqpoint{6.289786in}{1.323154in}}{\pgfqpoint{6.294830in}{1.318110in}}%
\pgfpathcurveto{\pgfqpoint{6.299874in}{1.313066in}}{\pgfqpoint{6.306715in}{1.310232in}}{\pgfqpoint{6.313848in}{1.310232in}}%
\pgfpathclose%
\pgfusepath{stroke,fill}%
\end{pgfscope}%
\begin{pgfscope}%
\pgfpathrectangle{\pgfqpoint{4.985294in}{0.500000in}}{\pgfqpoint{1.764706in}{1.700000in}}%
\pgfusepath{clip}%
\pgfsetbuttcap%
\pgfsetroundjoin%
\definecolor{currentfill}{rgb}{0.980678,0.914765,0.856766}%
\pgfsetfillcolor{currentfill}%
\pgfsetlinewidth{0.311001pt}%
\definecolor{currentstroke}{rgb}{1.000000,1.000000,1.000000}%
\pgfsetstrokecolor{currentstroke}%
\pgfsetdash{}{0pt}%
\pgfpathmoveto{\pgfqpoint{6.336181in}{1.347368in}}%
\pgfpathcurveto{\pgfqpoint{6.343314in}{1.347368in}}{\pgfqpoint{6.350156in}{1.350202in}}{\pgfqpoint{6.355199in}{1.355246in}}%
\pgfpathcurveto{\pgfqpoint{6.360243in}{1.360289in}}{\pgfqpoint{6.363077in}{1.367131in}}{\pgfqpoint{6.363077in}{1.374264in}}%
\pgfpathcurveto{\pgfqpoint{6.363077in}{1.381396in}}{\pgfqpoint{6.360243in}{1.388238in}}{\pgfqpoint{6.355199in}{1.393282in}}%
\pgfpathcurveto{\pgfqpoint{6.350156in}{1.398325in}}{\pgfqpoint{6.343314in}{1.401159in}}{\pgfqpoint{6.336181in}{1.401159in}}%
\pgfpathcurveto{\pgfqpoint{6.329048in}{1.401159in}}{\pgfqpoint{6.322207in}{1.398325in}}{\pgfqpoint{6.317163in}{1.393282in}}%
\pgfpathcurveto{\pgfqpoint{6.312119in}{1.388238in}}{\pgfqpoint{6.309285in}{1.381396in}}{\pgfqpoint{6.309285in}{1.374264in}}%
\pgfpathcurveto{\pgfqpoint{6.309285in}{1.367131in}}{\pgfqpoint{6.312119in}{1.360289in}}{\pgfqpoint{6.317163in}{1.355246in}}%
\pgfpathcurveto{\pgfqpoint{6.322207in}{1.350202in}}{\pgfqpoint{6.329048in}{1.347368in}}{\pgfqpoint{6.336181in}{1.347368in}}%
\pgfpathclose%
\pgfusepath{stroke,fill}%
\end{pgfscope}%
\begin{pgfscope}%
\pgfpathrectangle{\pgfqpoint{4.985294in}{0.500000in}}{\pgfqpoint{1.764706in}{1.700000in}}%
\pgfusepath{clip}%
\pgfsetbuttcap%
\pgfsetroundjoin%
\definecolor{currentfill}{rgb}{0.960421,0.553286,0.393191}%
\pgfsetfillcolor{currentfill}%
\pgfsetlinewidth{0.311001pt}%
\definecolor{currentstroke}{rgb}{1.000000,1.000000,1.000000}%
\pgfsetstrokecolor{currentstroke}%
\pgfsetdash{}{0pt}%
\pgfpathmoveto{\pgfqpoint{5.609565in}{0.862683in}}%
\pgfpathcurveto{\pgfqpoint{5.616698in}{0.862683in}}{\pgfqpoint{5.623539in}{0.865517in}}{\pgfqpoint{5.628583in}{0.870561in}}%
\pgfpathcurveto{\pgfqpoint{5.633627in}{0.875604in}}{\pgfqpoint{5.636460in}{0.882446in}}{\pgfqpoint{5.636460in}{0.889579in}}%
\pgfpathcurveto{\pgfqpoint{5.636460in}{0.896712in}}{\pgfqpoint{5.633627in}{0.903553in}}{\pgfqpoint{5.628583in}{0.908597in}}%
\pgfpathcurveto{\pgfqpoint{5.623539in}{0.913641in}}{\pgfqpoint{5.616698in}{0.916475in}}{\pgfqpoint{5.609565in}{0.916475in}}%
\pgfpathcurveto{\pgfqpoint{5.602432in}{0.916475in}}{\pgfqpoint{5.595590in}{0.913641in}}{\pgfqpoint{5.590547in}{0.908597in}}%
\pgfpathcurveto{\pgfqpoint{5.585503in}{0.903553in}}{\pgfqpoint{5.582669in}{0.896712in}}{\pgfqpoint{5.582669in}{0.889579in}}%
\pgfpathcurveto{\pgfqpoint{5.582669in}{0.882446in}}{\pgfqpoint{5.585503in}{0.875604in}}{\pgfqpoint{5.590547in}{0.870561in}}%
\pgfpathcurveto{\pgfqpoint{5.595590in}{0.865517in}}{\pgfqpoint{5.602432in}{0.862683in}}{\pgfqpoint{5.609565in}{0.862683in}}%
\pgfpathclose%
\pgfusepath{stroke,fill}%
\end{pgfscope}%
\begin{pgfscope}%
\pgfpathrectangle{\pgfqpoint{4.985294in}{0.500000in}}{\pgfqpoint{1.764706in}{1.700000in}}%
\pgfusepath{clip}%
\pgfsetbuttcap%
\pgfsetroundjoin%
\definecolor{currentfill}{rgb}{0.941676,0.367866,0.260395}%
\pgfsetfillcolor{currentfill}%
\pgfsetlinewidth{0.311001pt}%
\definecolor{currentstroke}{rgb}{1.000000,1.000000,1.000000}%
\pgfsetstrokecolor{currentstroke}%
\pgfsetdash{}{0pt}%
\pgfpathmoveto{\pgfqpoint{5.501853in}{0.825848in}}%
\pgfpathcurveto{\pgfqpoint{5.508986in}{0.825848in}}{\pgfqpoint{5.515828in}{0.828682in}}{\pgfqpoint{5.520871in}{0.833726in}}%
\pgfpathcurveto{\pgfqpoint{5.525915in}{0.838770in}}{\pgfqpoint{5.528749in}{0.845611in}}{\pgfqpoint{5.528749in}{0.852744in}}%
\pgfpathcurveto{\pgfqpoint{5.528749in}{0.859877in}}{\pgfqpoint{5.525915in}{0.866719in}}{\pgfqpoint{5.520871in}{0.871762in}}%
\pgfpathcurveto{\pgfqpoint{5.515828in}{0.876806in}}{\pgfqpoint{5.508986in}{0.879640in}}{\pgfqpoint{5.501853in}{0.879640in}}%
\pgfpathcurveto{\pgfqpoint{5.494721in}{0.879640in}}{\pgfqpoint{5.487879in}{0.876806in}}{\pgfqpoint{5.482835in}{0.871762in}}%
\pgfpathcurveto{\pgfqpoint{5.477792in}{0.866719in}}{\pgfqpoint{5.474958in}{0.859877in}}{\pgfqpoint{5.474958in}{0.852744in}}%
\pgfpathcurveto{\pgfqpoint{5.474958in}{0.845611in}}{\pgfqpoint{5.477792in}{0.838770in}}{\pgfqpoint{5.482835in}{0.833726in}}%
\pgfpathcurveto{\pgfqpoint{5.487879in}{0.828682in}}{\pgfqpoint{5.494721in}{0.825848in}}{\pgfqpoint{5.501853in}{0.825848in}}%
\pgfpathclose%
\pgfusepath{stroke,fill}%
\end{pgfscope}%
\begin{pgfscope}%
\pgfpathrectangle{\pgfqpoint{4.985294in}{0.500000in}}{\pgfqpoint{1.764706in}{1.700000in}}%
\pgfusepath{clip}%
\pgfsetbuttcap%
\pgfsetroundjoin%
\definecolor{currentfill}{rgb}{0.981377,0.920617,0.865369}%
\pgfsetfillcolor{currentfill}%
\pgfsetlinewidth{0.311001pt}%
\definecolor{currentstroke}{rgb}{1.000000,1.000000,1.000000}%
\pgfsetstrokecolor{currentstroke}%
\pgfsetdash{}{0pt}%
\pgfpathmoveto{\pgfqpoint{6.304780in}{1.422974in}}%
\pgfpathcurveto{\pgfqpoint{6.311913in}{1.422974in}}{\pgfqpoint{6.318755in}{1.425808in}}{\pgfqpoint{6.323798in}{1.430852in}}%
\pgfpathcurveto{\pgfqpoint{6.328842in}{1.435896in}}{\pgfqpoint{6.331676in}{1.442737in}}{\pgfqpoint{6.331676in}{1.449870in}}%
\pgfpathcurveto{\pgfqpoint{6.331676in}{1.457003in}}{\pgfqpoint{6.328842in}{1.463845in}}{\pgfqpoint{6.323798in}{1.468888in}}%
\pgfpathcurveto{\pgfqpoint{6.318755in}{1.473932in}}{\pgfqpoint{6.311913in}{1.476766in}}{\pgfqpoint{6.304780in}{1.476766in}}%
\pgfpathcurveto{\pgfqpoint{6.297647in}{1.476766in}}{\pgfqpoint{6.290806in}{1.473932in}}{\pgfqpoint{6.285762in}{1.468888in}}%
\pgfpathcurveto{\pgfqpoint{6.280718in}{1.463845in}}{\pgfqpoint{6.277884in}{1.457003in}}{\pgfqpoint{6.277884in}{1.449870in}}%
\pgfpathcurveto{\pgfqpoint{6.277884in}{1.442737in}}{\pgfqpoint{6.280718in}{1.435896in}}{\pgfqpoint{6.285762in}{1.430852in}}%
\pgfpathcurveto{\pgfqpoint{6.290806in}{1.425808in}}{\pgfqpoint{6.297647in}{1.422974in}}{\pgfqpoint{6.304780in}{1.422974in}}%
\pgfpathclose%
\pgfusepath{stroke,fill}%
\end{pgfscope}%
\begin{pgfscope}%
\pgfpathrectangle{\pgfqpoint{4.985294in}{0.500000in}}{\pgfqpoint{1.764706in}{1.700000in}}%
\pgfusepath{clip}%
\pgfsetbuttcap%
\pgfsetroundjoin%
\definecolor{currentfill}{rgb}{0.969803,0.809811,0.702523}%
\pgfsetfillcolor{currentfill}%
\pgfsetlinewidth{0.311001pt}%
\definecolor{currentstroke}{rgb}{1.000000,1.000000,1.000000}%
\pgfsetstrokecolor{currentstroke}%
\pgfsetdash{}{0pt}%
\pgfpathmoveto{\pgfqpoint{6.381607in}{1.373929in}}%
\pgfpathcurveto{\pgfqpoint{6.388740in}{1.373929in}}{\pgfqpoint{6.395582in}{1.376762in}}{\pgfqpoint{6.400625in}{1.381806in}}%
\pgfpathcurveto{\pgfqpoint{6.405669in}{1.386850in}}{\pgfqpoint{6.408503in}{1.393691in}}{\pgfqpoint{6.408503in}{1.400824in}}%
\pgfpathcurveto{\pgfqpoint{6.408503in}{1.407957in}}{\pgfqpoint{6.405669in}{1.414799in}}{\pgfqpoint{6.400625in}{1.419842in}}%
\pgfpathcurveto{\pgfqpoint{6.395582in}{1.424886in}}{\pgfqpoint{6.388740in}{1.427720in}}{\pgfqpoint{6.381607in}{1.427720in}}%
\pgfpathcurveto{\pgfqpoint{6.374474in}{1.427720in}}{\pgfqpoint{6.367633in}{1.424886in}}{\pgfqpoint{6.362589in}{1.419842in}}%
\pgfpathcurveto{\pgfqpoint{6.357545in}{1.414799in}}{\pgfqpoint{6.354711in}{1.407957in}}{\pgfqpoint{6.354711in}{1.400824in}}%
\pgfpathcurveto{\pgfqpoint{6.354711in}{1.393691in}}{\pgfqpoint{6.357545in}{1.386850in}}{\pgfqpoint{6.362589in}{1.381806in}}%
\pgfpathcurveto{\pgfqpoint{6.367633in}{1.376762in}}{\pgfqpoint{6.374474in}{1.373929in}}{\pgfqpoint{6.381607in}{1.373929in}}%
\pgfpathclose%
\pgfusepath{stroke,fill}%
\end{pgfscope}%
\begin{pgfscope}%
\pgfpathrectangle{\pgfqpoint{4.985294in}{0.500000in}}{\pgfqpoint{1.764706in}{1.700000in}}%
\pgfusepath{clip}%
\pgfsetbuttcap%
\pgfsetroundjoin%
\definecolor{currentfill}{rgb}{0.964920,0.695342,0.545192}%
\pgfsetfillcolor{currentfill}%
\pgfsetlinewidth{0.311001pt}%
\definecolor{currentstroke}{rgb}{1.000000,1.000000,1.000000}%
\pgfsetstrokecolor{currentstroke}%
\pgfsetdash{}{0pt}%
\pgfpathmoveto{\pgfqpoint{6.146219in}{0.916482in}}%
\pgfpathcurveto{\pgfqpoint{6.153352in}{0.916482in}}{\pgfqpoint{6.160193in}{0.919316in}}{\pgfqpoint{6.165237in}{0.924360in}}%
\pgfpathcurveto{\pgfqpoint{6.170281in}{0.929404in}}{\pgfqpoint{6.173114in}{0.936245in}}{\pgfqpoint{6.173114in}{0.943378in}}%
\pgfpathcurveto{\pgfqpoint{6.173114in}{0.950511in}}{\pgfqpoint{6.170281in}{0.957353in}}{\pgfqpoint{6.165237in}{0.962396in}}%
\pgfpathcurveto{\pgfqpoint{6.160193in}{0.967440in}}{\pgfqpoint{6.153352in}{0.970274in}}{\pgfqpoint{6.146219in}{0.970274in}}%
\pgfpathcurveto{\pgfqpoint{6.139086in}{0.970274in}}{\pgfqpoint{6.132244in}{0.967440in}}{\pgfqpoint{6.127201in}{0.962396in}}%
\pgfpathcurveto{\pgfqpoint{6.122157in}{0.957353in}}{\pgfqpoint{6.119323in}{0.950511in}}{\pgfqpoint{6.119323in}{0.943378in}}%
\pgfpathcurveto{\pgfqpoint{6.119323in}{0.936245in}}{\pgfqpoint{6.122157in}{0.929404in}}{\pgfqpoint{6.127201in}{0.924360in}}%
\pgfpathcurveto{\pgfqpoint{6.132244in}{0.919316in}}{\pgfqpoint{6.139086in}{0.916482in}}{\pgfqpoint{6.146219in}{0.916482in}}%
\pgfpathclose%
\pgfusepath{stroke,fill}%
\end{pgfscope}%
\begin{pgfscope}%
\pgfpathrectangle{\pgfqpoint{4.985294in}{0.500000in}}{\pgfqpoint{1.764706in}{1.700000in}}%
\pgfusepath{clip}%
\pgfsetbuttcap%
\pgfsetroundjoin%
\definecolor{currentfill}{rgb}{0.978376,0.897317,0.831308}%
\pgfsetfillcolor{currentfill}%
\pgfsetlinewidth{0.311001pt}%
\definecolor{currentstroke}{rgb}{1.000000,1.000000,1.000000}%
\pgfsetstrokecolor{currentstroke}%
\pgfsetdash{}{0pt}%
\pgfpathmoveto{\pgfqpoint{6.280636in}{1.578205in}}%
\pgfpathcurveto{\pgfqpoint{6.287769in}{1.578205in}}{\pgfqpoint{6.294611in}{1.581039in}}{\pgfqpoint{6.299654in}{1.586083in}}%
\pgfpathcurveto{\pgfqpoint{6.304698in}{1.591127in}}{\pgfqpoint{6.307532in}{1.597968in}}{\pgfqpoint{6.307532in}{1.605101in}}%
\pgfpathcurveto{\pgfqpoint{6.307532in}{1.612234in}}{\pgfqpoint{6.304698in}{1.619076in}}{\pgfqpoint{6.299654in}{1.624119in}}%
\pgfpathcurveto{\pgfqpoint{6.294611in}{1.629163in}}{\pgfqpoint{6.287769in}{1.631997in}}{\pgfqpoint{6.280636in}{1.631997in}}%
\pgfpathcurveto{\pgfqpoint{6.273503in}{1.631997in}}{\pgfqpoint{6.266662in}{1.629163in}}{\pgfqpoint{6.261618in}{1.624119in}}%
\pgfpathcurveto{\pgfqpoint{6.256574in}{1.619076in}}{\pgfqpoint{6.253741in}{1.612234in}}{\pgfqpoint{6.253741in}{1.605101in}}%
\pgfpathcurveto{\pgfqpoint{6.253741in}{1.597968in}}{\pgfqpoint{6.256574in}{1.591127in}}{\pgfqpoint{6.261618in}{1.586083in}}%
\pgfpathcurveto{\pgfqpoint{6.266662in}{1.581039in}}{\pgfqpoint{6.273503in}{1.578205in}}{\pgfqpoint{6.280636in}{1.578205in}}%
\pgfpathclose%
\pgfusepath{stroke,fill}%
\end{pgfscope}%
\begin{pgfscope}%
\pgfpathrectangle{\pgfqpoint{4.985294in}{0.500000in}}{\pgfqpoint{1.764706in}{1.700000in}}%
\pgfusepath{clip}%
\pgfsetbuttcap%
\pgfsetroundjoin%
\definecolor{currentfill}{rgb}{0.954476,0.470822,0.323110}%
\pgfsetfillcolor{currentfill}%
\pgfsetlinewidth{0.311001pt}%
\definecolor{currentstroke}{rgb}{1.000000,1.000000,1.000000}%
\pgfsetstrokecolor{currentstroke}%
\pgfsetdash{}{0pt}%
\pgfpathmoveto{\pgfqpoint{5.676818in}{0.887075in}}%
\pgfpathcurveto{\pgfqpoint{5.683951in}{0.887075in}}{\pgfqpoint{5.690793in}{0.889909in}}{\pgfqpoint{5.695836in}{0.894953in}}%
\pgfpathcurveto{\pgfqpoint{5.700880in}{0.899996in}}{\pgfqpoint{5.703714in}{0.906838in}}{\pgfqpoint{5.703714in}{0.913971in}}%
\pgfpathcurveto{\pgfqpoint{5.703714in}{0.921104in}}{\pgfqpoint{5.700880in}{0.927945in}}{\pgfqpoint{5.695836in}{0.932989in}}%
\pgfpathcurveto{\pgfqpoint{5.690793in}{0.938033in}}{\pgfqpoint{5.683951in}{0.940867in}}{\pgfqpoint{5.676818in}{0.940867in}}%
\pgfpathcurveto{\pgfqpoint{5.669685in}{0.940867in}}{\pgfqpoint{5.662844in}{0.938033in}}{\pgfqpoint{5.657800in}{0.932989in}}%
\pgfpathcurveto{\pgfqpoint{5.652756in}{0.927945in}}{\pgfqpoint{5.649923in}{0.921104in}}{\pgfqpoint{5.649923in}{0.913971in}}%
\pgfpathcurveto{\pgfqpoint{5.649923in}{0.906838in}}{\pgfqpoint{5.652756in}{0.899996in}}{\pgfqpoint{5.657800in}{0.894953in}}%
\pgfpathcurveto{\pgfqpoint{5.662844in}{0.889909in}}{\pgfqpoint{5.669685in}{0.887075in}}{\pgfqpoint{5.676818in}{0.887075in}}%
\pgfpathclose%
\pgfusepath{stroke,fill}%
\end{pgfscope}%
\begin{pgfscope}%
\pgfpathrectangle{\pgfqpoint{4.985294in}{0.500000in}}{\pgfqpoint{1.764706in}{1.700000in}}%
\pgfusepath{clip}%
\pgfsetbuttcap%
\pgfsetroundjoin%
\definecolor{currentfill}{rgb}{0.979124,0.903132,0.839793}%
\pgfsetfillcolor{currentfill}%
\pgfsetlinewidth{0.311001pt}%
\definecolor{currentstroke}{rgb}{1.000000,1.000000,1.000000}%
\pgfsetstrokecolor{currentstroke}%
\pgfsetdash{}{0pt}%
\pgfpathmoveto{\pgfqpoint{6.289712in}{1.308940in}}%
\pgfpathcurveto{\pgfqpoint{6.296845in}{1.308940in}}{\pgfqpoint{6.303687in}{1.311774in}}{\pgfqpoint{6.308730in}{1.316818in}}%
\pgfpathcurveto{\pgfqpoint{6.313774in}{1.321862in}}{\pgfqpoint{6.316608in}{1.328703in}}{\pgfqpoint{6.316608in}{1.335836in}}%
\pgfpathcurveto{\pgfqpoint{6.316608in}{1.342969in}}{\pgfqpoint{6.313774in}{1.349811in}}{\pgfqpoint{6.308730in}{1.354854in}}%
\pgfpathcurveto{\pgfqpoint{6.303687in}{1.359898in}}{\pgfqpoint{6.296845in}{1.362732in}}{\pgfqpoint{6.289712in}{1.362732in}}%
\pgfpathcurveto{\pgfqpoint{6.282579in}{1.362732in}}{\pgfqpoint{6.275738in}{1.359898in}}{\pgfqpoint{6.270694in}{1.354854in}}%
\pgfpathcurveto{\pgfqpoint{6.265650in}{1.349811in}}{\pgfqpoint{6.262816in}{1.342969in}}{\pgfqpoint{6.262816in}{1.335836in}}%
\pgfpathcurveto{\pgfqpoint{6.262816in}{1.328703in}}{\pgfqpoint{6.265650in}{1.321862in}}{\pgfqpoint{6.270694in}{1.316818in}}%
\pgfpathcurveto{\pgfqpoint{6.275738in}{1.311774in}}{\pgfqpoint{6.282579in}{1.308940in}}{\pgfqpoint{6.289712in}{1.308940in}}%
\pgfpathclose%
\pgfusepath{stroke,fill}%
\end{pgfscope}%
\begin{pgfscope}%
\pgfpathrectangle{\pgfqpoint{4.985294in}{0.500000in}}{\pgfqpoint{1.764706in}{1.700000in}}%
\pgfusepath{clip}%
\pgfsetbuttcap%
\pgfsetroundjoin%
\definecolor{currentfill}{rgb}{0.980678,0.914765,0.856766}%
\pgfsetfillcolor{currentfill}%
\pgfsetlinewidth{0.311001pt}%
\definecolor{currentstroke}{rgb}{1.000000,1.000000,1.000000}%
\pgfsetstrokecolor{currentstroke}%
\pgfsetdash{}{0pt}%
\pgfpathmoveto{\pgfqpoint{6.324491in}{1.244324in}}%
\pgfpathcurveto{\pgfqpoint{6.331624in}{1.244324in}}{\pgfqpoint{6.338465in}{1.247158in}}{\pgfqpoint{6.343509in}{1.252201in}}%
\pgfpathcurveto{\pgfqpoint{6.348553in}{1.257245in}}{\pgfqpoint{6.351387in}{1.264087in}}{\pgfqpoint{6.351387in}{1.271220in}}%
\pgfpathcurveto{\pgfqpoint{6.351387in}{1.278352in}}{\pgfqpoint{6.348553in}{1.285194in}}{\pgfqpoint{6.343509in}{1.290238in}}%
\pgfpathcurveto{\pgfqpoint{6.338465in}{1.295281in}}{\pgfqpoint{6.331624in}{1.298115in}}{\pgfqpoint{6.324491in}{1.298115in}}%
\pgfpathcurveto{\pgfqpoint{6.317358in}{1.298115in}}{\pgfqpoint{6.310516in}{1.295281in}}{\pgfqpoint{6.305473in}{1.290238in}}%
\pgfpathcurveto{\pgfqpoint{6.300429in}{1.285194in}}{\pgfqpoint{6.297595in}{1.278352in}}{\pgfqpoint{6.297595in}{1.271220in}}%
\pgfpathcurveto{\pgfqpoint{6.297595in}{1.264087in}}{\pgfqpoint{6.300429in}{1.257245in}}{\pgfqpoint{6.305473in}{1.252201in}}%
\pgfpathcurveto{\pgfqpoint{6.310516in}{1.247158in}}{\pgfqpoint{6.317358in}{1.244324in}}{\pgfqpoint{6.324491in}{1.244324in}}%
\pgfpathclose%
\pgfusepath{stroke,fill}%
\end{pgfscope}%
\begin{pgfscope}%
\pgfpathrectangle{\pgfqpoint{4.985294in}{0.500000in}}{\pgfqpoint{1.764706in}{1.700000in}}%
\pgfusepath{clip}%
\pgfsetbuttcap%
\pgfsetroundjoin%
\definecolor{currentfill}{rgb}{0.980678,0.914765,0.856766}%
\pgfsetfillcolor{currentfill}%
\pgfsetlinewidth{0.311001pt}%
\definecolor{currentstroke}{rgb}{1.000000,1.000000,1.000000}%
\pgfsetstrokecolor{currentstroke}%
\pgfsetdash{}{0pt}%
\pgfpathmoveto{\pgfqpoint{5.421899in}{1.322243in}}%
\pgfpathcurveto{\pgfqpoint{5.429032in}{1.322243in}}{\pgfqpoint{5.435873in}{1.325076in}}{\pgfqpoint{5.440917in}{1.330120in}}%
\pgfpathcurveto{\pgfqpoint{5.445961in}{1.335164in}}{\pgfqpoint{5.448794in}{1.342005in}}{\pgfqpoint{5.448794in}{1.349138in}}%
\pgfpathcurveto{\pgfqpoint{5.448794in}{1.356271in}}{\pgfqpoint{5.445961in}{1.363113in}}{\pgfqpoint{5.440917in}{1.368156in}}%
\pgfpathcurveto{\pgfqpoint{5.435873in}{1.373200in}}{\pgfqpoint{5.429032in}{1.376034in}}{\pgfqpoint{5.421899in}{1.376034in}}%
\pgfpathcurveto{\pgfqpoint{5.414766in}{1.376034in}}{\pgfqpoint{5.407924in}{1.373200in}}{\pgfqpoint{5.402881in}{1.368156in}}%
\pgfpathcurveto{\pgfqpoint{5.397837in}{1.363113in}}{\pgfqpoint{5.395003in}{1.356271in}}{\pgfqpoint{5.395003in}{1.349138in}}%
\pgfpathcurveto{\pgfqpoint{5.395003in}{1.342005in}}{\pgfqpoint{5.397837in}{1.335164in}}{\pgfqpoint{5.402881in}{1.330120in}}%
\pgfpathcurveto{\pgfqpoint{5.407924in}{1.325076in}}{\pgfqpoint{5.414766in}{1.322243in}}{\pgfqpoint{5.421899in}{1.322243in}}%
\pgfpathclose%
\pgfusepath{stroke,fill}%
\end{pgfscope}%
\begin{pgfscope}%
\pgfpathrectangle{\pgfqpoint{4.985294in}{0.500000in}}{\pgfqpoint{1.764706in}{1.700000in}}%
\pgfusepath{clip}%
\pgfsetbuttcap%
\pgfsetroundjoin%
\definecolor{currentfill}{rgb}{0.967398,0.774513,0.650573}%
\pgfsetfillcolor{currentfill}%
\pgfsetlinewidth{0.311001pt}%
\definecolor{currentstroke}{rgb}{1.000000,1.000000,1.000000}%
\pgfsetstrokecolor{currentstroke}%
\pgfsetdash{}{0pt}%
\pgfpathmoveto{\pgfqpoint{5.365910in}{1.147755in}}%
\pgfpathcurveto{\pgfqpoint{5.373043in}{1.147755in}}{\pgfqpoint{5.379884in}{1.150588in}}{\pgfqpoint{5.384928in}{1.155632in}}%
\pgfpathcurveto{\pgfqpoint{5.389972in}{1.160676in}}{\pgfqpoint{5.392805in}{1.167517in}}{\pgfqpoint{5.392805in}{1.174650in}}%
\pgfpathcurveto{\pgfqpoint{5.392805in}{1.181783in}}{\pgfqpoint{5.389972in}{1.188625in}}{\pgfqpoint{5.384928in}{1.193668in}}%
\pgfpathcurveto{\pgfqpoint{5.379884in}{1.198712in}}{\pgfqpoint{5.373043in}{1.201546in}}{\pgfqpoint{5.365910in}{1.201546in}}%
\pgfpathcurveto{\pgfqpoint{5.358777in}{1.201546in}}{\pgfqpoint{5.351935in}{1.198712in}}{\pgfqpoint{5.346892in}{1.193668in}}%
\pgfpathcurveto{\pgfqpoint{5.341848in}{1.188625in}}{\pgfqpoint{5.339014in}{1.181783in}}{\pgfqpoint{5.339014in}{1.174650in}}%
\pgfpathcurveto{\pgfqpoint{5.339014in}{1.167517in}}{\pgfqpoint{5.341848in}{1.160676in}}{\pgfqpoint{5.346892in}{1.155632in}}%
\pgfpathcurveto{\pgfqpoint{5.351935in}{1.150588in}}{\pgfqpoint{5.358777in}{1.147755in}}{\pgfqpoint{5.365910in}{1.147755in}}%
\pgfpathclose%
\pgfusepath{stroke,fill}%
\end{pgfscope}%
\begin{pgfscope}%
\pgfpathrectangle{\pgfqpoint{4.985294in}{0.500000in}}{\pgfqpoint{1.764706in}{1.700000in}}%
\pgfusepath{clip}%
\pgfsetbuttcap%
\pgfsetroundjoin%
\definecolor{currentfill}{rgb}{0.691463,0.089868,0.347769}%
\pgfsetfillcolor{currentfill}%
\pgfsetlinewidth{0.311001pt}%
\definecolor{currentstroke}{rgb}{1.000000,1.000000,1.000000}%
\pgfsetstrokecolor{currentstroke}%
\pgfsetdash{}{0pt}%
\pgfpathmoveto{\pgfqpoint{5.654872in}{1.193159in}}%
\pgfpathcurveto{\pgfqpoint{5.662005in}{1.193159in}}{\pgfqpoint{5.668847in}{1.195993in}}{\pgfqpoint{5.673891in}{1.201037in}}%
\pgfpathcurveto{\pgfqpoint{5.678934in}{1.206080in}}{\pgfqpoint{5.681768in}{1.212922in}}{\pgfqpoint{5.681768in}{1.220055in}}%
\pgfpathcurveto{\pgfqpoint{5.681768in}{1.227188in}}{\pgfqpoint{5.678934in}{1.234029in}}{\pgfqpoint{5.673891in}{1.239073in}}%
\pgfpathcurveto{\pgfqpoint{5.668847in}{1.244117in}}{\pgfqpoint{5.662005in}{1.246951in}}{\pgfqpoint{5.654872in}{1.246951in}}%
\pgfpathcurveto{\pgfqpoint{5.647740in}{1.246951in}}{\pgfqpoint{5.640898in}{1.244117in}}{\pgfqpoint{5.635854in}{1.239073in}}%
\pgfpathcurveto{\pgfqpoint{5.630811in}{1.234029in}}{\pgfqpoint{5.627977in}{1.227188in}}{\pgfqpoint{5.627977in}{1.220055in}}%
\pgfpathcurveto{\pgfqpoint{5.627977in}{1.212922in}}{\pgfqpoint{5.630811in}{1.206080in}}{\pgfqpoint{5.635854in}{1.201037in}}%
\pgfpathcurveto{\pgfqpoint{5.640898in}{1.195993in}}{\pgfqpoint{5.647740in}{1.193159in}}{\pgfqpoint{5.654872in}{1.193159in}}%
\pgfpathclose%
\pgfusepath{stroke,fill}%
\end{pgfscope}%
\begin{pgfscope}%
\pgfpathrectangle{\pgfqpoint{4.985294in}{0.500000in}}{\pgfqpoint{1.764706in}{1.700000in}}%
\pgfusepath{clip}%
\pgfsetbuttcap%
\pgfsetroundjoin%
\definecolor{currentfill}{rgb}{0.966120,0.744512,0.608720}%
\pgfsetfillcolor{currentfill}%
\pgfsetlinewidth{0.311001pt}%
\definecolor{currentstroke}{rgb}{1.000000,1.000000,1.000000}%
\pgfsetstrokecolor{currentstroke}%
\pgfsetdash{}{0pt}%
\pgfpathmoveto{\pgfqpoint{6.309125in}{1.632670in}}%
\pgfpathcurveto{\pgfqpoint{6.316258in}{1.632670in}}{\pgfqpoint{6.323099in}{1.635504in}}{\pgfqpoint{6.328143in}{1.640548in}}%
\pgfpathcurveto{\pgfqpoint{6.333187in}{1.645591in}}{\pgfqpoint{6.336021in}{1.652433in}}{\pgfqpoint{6.336021in}{1.659566in}}%
\pgfpathcurveto{\pgfqpoint{6.336021in}{1.666699in}}{\pgfqpoint{6.333187in}{1.673540in}}{\pgfqpoint{6.328143in}{1.678584in}}%
\pgfpathcurveto{\pgfqpoint{6.323099in}{1.683628in}}{\pgfqpoint{6.316258in}{1.686461in}}{\pgfqpoint{6.309125in}{1.686461in}}%
\pgfpathcurveto{\pgfqpoint{6.301992in}{1.686461in}}{\pgfqpoint{6.295150in}{1.683628in}}{\pgfqpoint{6.290107in}{1.678584in}}%
\pgfpathcurveto{\pgfqpoint{6.285063in}{1.673540in}}{\pgfqpoint{6.282229in}{1.666699in}}{\pgfqpoint{6.282229in}{1.659566in}}%
\pgfpathcurveto{\pgfqpoint{6.282229in}{1.652433in}}{\pgfqpoint{6.285063in}{1.645591in}}{\pgfqpoint{6.290107in}{1.640548in}}%
\pgfpathcurveto{\pgfqpoint{6.295150in}{1.635504in}}{\pgfqpoint{6.301992in}{1.632670in}}{\pgfqpoint{6.309125in}{1.632670in}}%
\pgfpathclose%
\pgfusepath{stroke,fill}%
\end{pgfscope}%
\begin{pgfscope}%
\pgfpathrectangle{\pgfqpoint{4.985294in}{0.500000in}}{\pgfqpoint{1.764706in}{1.700000in}}%
\pgfusepath{clip}%
\pgfsetbuttcap%
\pgfsetroundjoin%
\definecolor{currentfill}{rgb}{0.973832,0.856556,0.771584}%
\pgfsetfillcolor{currentfill}%
\pgfsetlinewidth{0.311001pt}%
\definecolor{currentstroke}{rgb}{1.000000,1.000000,1.000000}%
\pgfsetstrokecolor{currentstroke}%
\pgfsetdash{}{0pt}%
\pgfpathmoveto{\pgfqpoint{5.489317in}{1.444403in}}%
\pgfpathcurveto{\pgfqpoint{5.496450in}{1.444403in}}{\pgfqpoint{5.503291in}{1.447237in}}{\pgfqpoint{5.508335in}{1.452281in}}%
\pgfpathcurveto{\pgfqpoint{5.513379in}{1.457325in}}{\pgfqpoint{5.516213in}{1.464166in}}{\pgfqpoint{5.516213in}{1.471299in}}%
\pgfpathcurveto{\pgfqpoint{5.516213in}{1.478432in}}{\pgfqpoint{5.513379in}{1.485273in}}{\pgfqpoint{5.508335in}{1.490317in}}%
\pgfpathcurveto{\pgfqpoint{5.503291in}{1.495361in}}{\pgfqpoint{5.496450in}{1.498195in}}{\pgfqpoint{5.489317in}{1.498195in}}%
\pgfpathcurveto{\pgfqpoint{5.482184in}{1.498195in}}{\pgfqpoint{5.475342in}{1.495361in}}{\pgfqpoint{5.470299in}{1.490317in}}%
\pgfpathcurveto{\pgfqpoint{5.465255in}{1.485273in}}{\pgfqpoint{5.462421in}{1.478432in}}{\pgfqpoint{5.462421in}{1.471299in}}%
\pgfpathcurveto{\pgfqpoint{5.462421in}{1.464166in}}{\pgfqpoint{5.465255in}{1.457325in}}{\pgfqpoint{5.470299in}{1.452281in}}%
\pgfpathcurveto{\pgfqpoint{5.475342in}{1.447237in}}{\pgfqpoint{5.482184in}{1.444403in}}{\pgfqpoint{5.489317in}{1.444403in}}%
\pgfpathclose%
\pgfusepath{stroke,fill}%
\end{pgfscope}%
\begin{pgfscope}%
\pgfpathrectangle{\pgfqpoint{4.985294in}{0.500000in}}{\pgfqpoint{1.764706in}{1.700000in}}%
\pgfusepath{clip}%
\pgfsetbuttcap%
\pgfsetroundjoin%
\definecolor{currentfill}{rgb}{0.972726,0.844889,0.754401}%
\pgfsetfillcolor{currentfill}%
\pgfsetlinewidth{0.311001pt}%
\definecolor{currentstroke}{rgb}{1.000000,1.000000,1.000000}%
\pgfsetstrokecolor{currentstroke}%
\pgfsetdash{}{0pt}%
\pgfpathmoveto{\pgfqpoint{6.364628in}{1.213098in}}%
\pgfpathcurveto{\pgfqpoint{6.371761in}{1.213098in}}{\pgfqpoint{6.378602in}{1.215932in}}{\pgfqpoint{6.383646in}{1.220975in}}%
\pgfpathcurveto{\pgfqpoint{6.388690in}{1.226019in}}{\pgfqpoint{6.391523in}{1.232861in}}{\pgfqpoint{6.391523in}{1.239994in}}%
\pgfpathcurveto{\pgfqpoint{6.391523in}{1.247126in}}{\pgfqpoint{6.388690in}{1.253968in}}{\pgfqpoint{6.383646in}{1.259012in}}%
\pgfpathcurveto{\pgfqpoint{6.378602in}{1.264055in}}{\pgfqpoint{6.371761in}{1.266889in}}{\pgfqpoint{6.364628in}{1.266889in}}%
\pgfpathcurveto{\pgfqpoint{6.357495in}{1.266889in}}{\pgfqpoint{6.350653in}{1.264055in}}{\pgfqpoint{6.345610in}{1.259012in}}%
\pgfpathcurveto{\pgfqpoint{6.340566in}{1.253968in}}{\pgfqpoint{6.337732in}{1.247126in}}{\pgfqpoint{6.337732in}{1.239994in}}%
\pgfpathcurveto{\pgfqpoint{6.337732in}{1.232861in}}{\pgfqpoint{6.340566in}{1.226019in}}{\pgfqpoint{6.345610in}{1.220975in}}%
\pgfpathcurveto{\pgfqpoint{6.350653in}{1.215932in}}{\pgfqpoint{6.357495in}{1.213098in}}{\pgfqpoint{6.364628in}{1.213098in}}%
\pgfpathclose%
\pgfusepath{stroke,fill}%
\end{pgfscope}%
\begin{pgfscope}%
\pgfpathrectangle{\pgfqpoint{4.985294in}{0.500000in}}{\pgfqpoint{1.764706in}{1.700000in}}%
\pgfusepath{clip}%
\pgfsetbuttcap%
\pgfsetroundjoin%
\definecolor{currentfill}{rgb}{0.965302,0.713942,0.568499}%
\pgfsetfillcolor{currentfill}%
\pgfsetlinewidth{0.311001pt}%
\definecolor{currentstroke}{rgb}{1.000000,1.000000,1.000000}%
\pgfsetstrokecolor{currentstroke}%
\pgfsetdash{}{0pt}%
\pgfpathmoveto{\pgfqpoint{6.348084in}{1.577683in}}%
\pgfpathcurveto{\pgfqpoint{6.355217in}{1.577683in}}{\pgfqpoint{6.362058in}{1.580517in}}{\pgfqpoint{6.367102in}{1.585561in}}%
\pgfpathcurveto{\pgfqpoint{6.372146in}{1.590605in}}{\pgfqpoint{6.374980in}{1.597446in}}{\pgfqpoint{6.374980in}{1.604579in}}%
\pgfpathcurveto{\pgfqpoint{6.374980in}{1.611712in}}{\pgfqpoint{6.372146in}{1.618554in}}{\pgfqpoint{6.367102in}{1.623597in}}%
\pgfpathcurveto{\pgfqpoint{6.362058in}{1.628641in}}{\pgfqpoint{6.355217in}{1.631475in}}{\pgfqpoint{6.348084in}{1.631475in}}%
\pgfpathcurveto{\pgfqpoint{6.340951in}{1.631475in}}{\pgfqpoint{6.334109in}{1.628641in}}{\pgfqpoint{6.329066in}{1.623597in}}%
\pgfpathcurveto{\pgfqpoint{6.324022in}{1.618554in}}{\pgfqpoint{6.321188in}{1.611712in}}{\pgfqpoint{6.321188in}{1.604579in}}%
\pgfpathcurveto{\pgfqpoint{6.321188in}{1.597446in}}{\pgfqpoint{6.324022in}{1.590605in}}{\pgfqpoint{6.329066in}{1.585561in}}%
\pgfpathcurveto{\pgfqpoint{6.334109in}{1.580517in}}{\pgfqpoint{6.340951in}{1.577683in}}{\pgfqpoint{6.348084in}{1.577683in}}%
\pgfpathclose%
\pgfusepath{stroke,fill}%
\end{pgfscope}%
\begin{pgfscope}%
\pgfpathrectangle{\pgfqpoint{4.985294in}{0.500000in}}{\pgfqpoint{1.764706in}{1.700000in}}%
\pgfusepath{clip}%
\pgfsetbuttcap%
\pgfsetroundjoin%
\definecolor{currentfill}{rgb}{0.973832,0.856556,0.771584}%
\pgfsetfillcolor{currentfill}%
\pgfsetlinewidth{0.311001pt}%
\definecolor{currentstroke}{rgb}{1.000000,1.000000,1.000000}%
\pgfsetstrokecolor{currentstroke}%
\pgfsetdash{}{0pt}%
\pgfpathmoveto{\pgfqpoint{5.458561in}{1.575915in}}%
\pgfpathcurveto{\pgfqpoint{5.465694in}{1.575915in}}{\pgfqpoint{5.472536in}{1.578749in}}{\pgfqpoint{5.477579in}{1.583792in}}%
\pgfpathcurveto{\pgfqpoint{5.482623in}{1.588836in}}{\pgfqpoint{5.485457in}{1.595678in}}{\pgfqpoint{5.485457in}{1.602810in}}%
\pgfpathcurveto{\pgfqpoint{5.485457in}{1.609943in}}{\pgfqpoint{5.482623in}{1.616785in}}{\pgfqpoint{5.477579in}{1.621829in}}%
\pgfpathcurveto{\pgfqpoint{5.472536in}{1.626872in}}{\pgfqpoint{5.465694in}{1.629706in}}{\pgfqpoint{5.458561in}{1.629706in}}%
\pgfpathcurveto{\pgfqpoint{5.451428in}{1.629706in}}{\pgfqpoint{5.444587in}{1.626872in}}{\pgfqpoint{5.439543in}{1.621829in}}%
\pgfpathcurveto{\pgfqpoint{5.434499in}{1.616785in}}{\pgfqpoint{5.431665in}{1.609943in}}{\pgfqpoint{5.431665in}{1.602810in}}%
\pgfpathcurveto{\pgfqpoint{5.431665in}{1.595678in}}{\pgfqpoint{5.434499in}{1.588836in}}{\pgfqpoint{5.439543in}{1.583792in}}%
\pgfpathcurveto{\pgfqpoint{5.444587in}{1.578749in}}{\pgfqpoint{5.451428in}{1.575915in}}{\pgfqpoint{5.458561in}{1.575915in}}%
\pgfpathclose%
\pgfusepath{stroke,fill}%
\end{pgfscope}%
\begin{pgfscope}%
\pgfpathrectangle{\pgfqpoint{4.985294in}{0.500000in}}{\pgfqpoint{1.764706in}{1.700000in}}%
\pgfusepath{clip}%
\pgfsetbuttcap%
\pgfsetroundjoin%
\definecolor{currentfill}{rgb}{0.937528,0.344792,0.251999}%
\pgfsetfillcolor{currentfill}%
\pgfsetlinewidth{0.311001pt}%
\definecolor{currentstroke}{rgb}{1.000000,1.000000,1.000000}%
\pgfsetstrokecolor{currentstroke}%
\pgfsetdash{}{0pt}%
\pgfpathmoveto{\pgfqpoint{6.131813in}{0.811560in}}%
\pgfpathcurveto{\pgfqpoint{6.138946in}{0.811560in}}{\pgfqpoint{6.145787in}{0.814394in}}{\pgfqpoint{6.150831in}{0.819438in}}%
\pgfpathcurveto{\pgfqpoint{6.155875in}{0.824481in}}{\pgfqpoint{6.158709in}{0.831323in}}{\pgfqpoint{6.158709in}{0.838456in}}%
\pgfpathcurveto{\pgfqpoint{6.158709in}{0.845589in}}{\pgfqpoint{6.155875in}{0.852430in}}{\pgfqpoint{6.150831in}{0.857474in}}%
\pgfpathcurveto{\pgfqpoint{6.145787in}{0.862518in}}{\pgfqpoint{6.138946in}{0.865352in}}{\pgfqpoint{6.131813in}{0.865352in}}%
\pgfpathcurveto{\pgfqpoint{6.124680in}{0.865352in}}{\pgfqpoint{6.117838in}{0.862518in}}{\pgfqpoint{6.112795in}{0.857474in}}%
\pgfpathcurveto{\pgfqpoint{6.107751in}{0.852430in}}{\pgfqpoint{6.104917in}{0.845589in}}{\pgfqpoint{6.104917in}{0.838456in}}%
\pgfpathcurveto{\pgfqpoint{6.104917in}{0.831323in}}{\pgfqpoint{6.107751in}{0.824481in}}{\pgfqpoint{6.112795in}{0.819438in}}%
\pgfpathcurveto{\pgfqpoint{6.117838in}{0.814394in}}{\pgfqpoint{6.124680in}{0.811560in}}{\pgfqpoint{6.131813in}{0.811560in}}%
\pgfpathclose%
\pgfusepath{stroke,fill}%
\end{pgfscope}%
\begin{pgfscope}%
\pgfpathrectangle{\pgfqpoint{4.985294in}{0.500000in}}{\pgfqpoint{1.764706in}{1.700000in}}%
\pgfusepath{clip}%
\pgfsetbuttcap%
\pgfsetroundjoin%
\definecolor{currentfill}{rgb}{0.976961,0.885681,0.814303}%
\pgfsetfillcolor{currentfill}%
\pgfsetlinewidth{0.311001pt}%
\definecolor{currentstroke}{rgb}{1.000000,1.000000,1.000000}%
\pgfsetstrokecolor{currentstroke}%
\pgfsetdash{}{0pt}%
\pgfpathmoveto{\pgfqpoint{5.415265in}{1.169173in}}%
\pgfpathcurveto{\pgfqpoint{5.422398in}{1.169173in}}{\pgfqpoint{5.429240in}{1.172007in}}{\pgfqpoint{5.434284in}{1.177051in}}%
\pgfpathcurveto{\pgfqpoint{5.439327in}{1.182094in}}{\pgfqpoint{5.442161in}{1.188936in}}{\pgfqpoint{5.442161in}{1.196069in}}%
\pgfpathcurveto{\pgfqpoint{5.442161in}{1.203202in}}{\pgfqpoint{5.439327in}{1.210043in}}{\pgfqpoint{5.434284in}{1.215087in}}%
\pgfpathcurveto{\pgfqpoint{5.429240in}{1.220131in}}{\pgfqpoint{5.422398in}{1.222964in}}{\pgfqpoint{5.415265in}{1.222964in}}%
\pgfpathcurveto{\pgfqpoint{5.408133in}{1.222964in}}{\pgfqpoint{5.401291in}{1.220131in}}{\pgfqpoint{5.396247in}{1.215087in}}%
\pgfpathcurveto{\pgfqpoint{5.391204in}{1.210043in}}{\pgfqpoint{5.388370in}{1.203202in}}{\pgfqpoint{5.388370in}{1.196069in}}%
\pgfpathcurveto{\pgfqpoint{5.388370in}{1.188936in}}{\pgfqpoint{5.391204in}{1.182094in}}{\pgfqpoint{5.396247in}{1.177051in}}%
\pgfpathcurveto{\pgfqpoint{5.401291in}{1.172007in}}{\pgfqpoint{5.408133in}{1.169173in}}{\pgfqpoint{5.415265in}{1.169173in}}%
\pgfpathclose%
\pgfusepath{stroke,fill}%
\end{pgfscope}%
\begin{pgfscope}%
\pgfpathrectangle{\pgfqpoint{4.985294in}{0.500000in}}{\pgfqpoint{1.764706in}{1.700000in}}%
\pgfusepath{clip}%
\pgfsetbuttcap%
\pgfsetroundjoin%
\definecolor{currentfill}{rgb}{0.965440,0.720101,0.576404}%
\pgfsetfillcolor{currentfill}%
\pgfsetlinewidth{0.311001pt}%
\definecolor{currentstroke}{rgb}{1.000000,1.000000,1.000000}%
\pgfsetstrokecolor{currentstroke}%
\pgfsetdash{}{0pt}%
\pgfpathmoveto{\pgfqpoint{5.565888in}{1.546436in}}%
\pgfpathcurveto{\pgfqpoint{5.573021in}{1.546436in}}{\pgfqpoint{5.579863in}{1.549270in}}{\pgfqpoint{5.584906in}{1.554314in}}%
\pgfpathcurveto{\pgfqpoint{5.589950in}{1.559357in}}{\pgfqpoint{5.592784in}{1.566199in}}{\pgfqpoint{5.592784in}{1.573332in}}%
\pgfpathcurveto{\pgfqpoint{5.592784in}{1.580465in}}{\pgfqpoint{5.589950in}{1.587306in}}{\pgfqpoint{5.584906in}{1.592350in}}%
\pgfpathcurveto{\pgfqpoint{5.579863in}{1.597394in}}{\pgfqpoint{5.573021in}{1.600228in}}{\pgfqpoint{5.565888in}{1.600228in}}%
\pgfpathcurveto{\pgfqpoint{5.558755in}{1.600228in}}{\pgfqpoint{5.551914in}{1.597394in}}{\pgfqpoint{5.546870in}{1.592350in}}%
\pgfpathcurveto{\pgfqpoint{5.541826in}{1.587306in}}{\pgfqpoint{5.538992in}{1.580465in}}{\pgfqpoint{5.538992in}{1.573332in}}%
\pgfpathcurveto{\pgfqpoint{5.538992in}{1.566199in}}{\pgfqpoint{5.541826in}{1.559357in}}{\pgfqpoint{5.546870in}{1.554314in}}%
\pgfpathcurveto{\pgfqpoint{5.551914in}{1.549270in}}{\pgfqpoint{5.558755in}{1.546436in}}{\pgfqpoint{5.565888in}{1.546436in}}%
\pgfpathclose%
\pgfusepath{stroke,fill}%
\end{pgfscope}%
\begin{pgfscope}%
\pgfpathrectangle{\pgfqpoint{4.985294in}{0.500000in}}{\pgfqpoint{1.764706in}{1.700000in}}%
\pgfusepath{clip}%
\pgfsetbuttcap%
\pgfsetroundjoin%
\definecolor{currentfill}{rgb}{0.977657,0.891500,0.822809}%
\pgfsetfillcolor{currentfill}%
\pgfsetlinewidth{0.311001pt}%
\definecolor{currentstroke}{rgb}{1.000000,1.000000,1.000000}%
\pgfsetstrokecolor{currentstroke}%
\pgfsetdash{}{0pt}%
\pgfpathmoveto{\pgfqpoint{5.424415in}{1.153438in}}%
\pgfpathcurveto{\pgfqpoint{5.431548in}{1.153438in}}{\pgfqpoint{5.438389in}{1.156272in}}{\pgfqpoint{5.443433in}{1.161316in}}%
\pgfpathcurveto{\pgfqpoint{5.448477in}{1.166360in}}{\pgfqpoint{5.451311in}{1.173201in}}{\pgfqpoint{5.451311in}{1.180334in}}%
\pgfpathcurveto{\pgfqpoint{5.451311in}{1.187467in}}{\pgfqpoint{5.448477in}{1.194309in}}{\pgfqpoint{5.443433in}{1.199352in}}%
\pgfpathcurveto{\pgfqpoint{5.438389in}{1.204396in}}{\pgfqpoint{5.431548in}{1.207230in}}{\pgfqpoint{5.424415in}{1.207230in}}%
\pgfpathcurveto{\pgfqpoint{5.417282in}{1.207230in}}{\pgfqpoint{5.410440in}{1.204396in}}{\pgfqpoint{5.405397in}{1.199352in}}%
\pgfpathcurveto{\pgfqpoint{5.400353in}{1.194309in}}{\pgfqpoint{5.397519in}{1.187467in}}{\pgfqpoint{5.397519in}{1.180334in}}%
\pgfpathcurveto{\pgfqpoint{5.397519in}{1.173201in}}{\pgfqpoint{5.400353in}{1.166360in}}{\pgfqpoint{5.405397in}{1.161316in}}%
\pgfpathcurveto{\pgfqpoint{5.410440in}{1.156272in}}{\pgfqpoint{5.417282in}{1.153438in}}{\pgfqpoint{5.424415in}{1.153438in}}%
\pgfpathclose%
\pgfusepath{stroke,fill}%
\end{pgfscope}%
\begin{pgfscope}%
\pgfpathrectangle{\pgfqpoint{4.985294in}{0.500000in}}{\pgfqpoint{1.764706in}{1.700000in}}%
\pgfusepath{clip}%
\pgfsetbuttcap%
\pgfsetroundjoin%
\definecolor{currentfill}{rgb}{0.972201,0.839051,0.745789}%
\pgfsetfillcolor{currentfill}%
\pgfsetlinewidth{0.311001pt}%
\definecolor{currentstroke}{rgb}{1.000000,1.000000,1.000000}%
\pgfsetstrokecolor{currentstroke}%
\pgfsetdash{}{0pt}%
\pgfpathmoveto{\pgfqpoint{5.473424in}{1.611785in}}%
\pgfpathcurveto{\pgfqpoint{5.480557in}{1.611785in}}{\pgfqpoint{5.487398in}{1.614619in}}{\pgfqpoint{5.492442in}{1.619663in}}%
\pgfpathcurveto{\pgfqpoint{5.497486in}{1.624707in}}{\pgfqpoint{5.500319in}{1.631548in}}{\pgfqpoint{5.500319in}{1.638681in}}%
\pgfpathcurveto{\pgfqpoint{5.500319in}{1.645814in}}{\pgfqpoint{5.497486in}{1.652656in}}{\pgfqpoint{5.492442in}{1.657699in}}%
\pgfpathcurveto{\pgfqpoint{5.487398in}{1.662743in}}{\pgfqpoint{5.480557in}{1.665577in}}{\pgfqpoint{5.473424in}{1.665577in}}%
\pgfpathcurveto{\pgfqpoint{5.466291in}{1.665577in}}{\pgfqpoint{5.459449in}{1.662743in}}{\pgfqpoint{5.454406in}{1.657699in}}%
\pgfpathcurveto{\pgfqpoint{5.449362in}{1.652656in}}{\pgfqpoint{5.446528in}{1.645814in}}{\pgfqpoint{5.446528in}{1.638681in}}%
\pgfpathcurveto{\pgfqpoint{5.446528in}{1.631548in}}{\pgfqpoint{5.449362in}{1.624707in}}{\pgfqpoint{5.454406in}{1.619663in}}%
\pgfpathcurveto{\pgfqpoint{5.459449in}{1.614619in}}{\pgfqpoint{5.466291in}{1.611785in}}{\pgfqpoint{5.473424in}{1.611785in}}%
\pgfpathclose%
\pgfusepath{stroke,fill}%
\end{pgfscope}%
\begin{pgfscope}%
\pgfpathrectangle{\pgfqpoint{4.985294in}{0.500000in}}{\pgfqpoint{1.764706in}{1.700000in}}%
\pgfusepath{clip}%
\pgfsetbuttcap%
\pgfsetroundjoin%
\definecolor{currentfill}{rgb}{0.970718,0.821518,0.719872}%
\pgfsetfillcolor{currentfill}%
\pgfsetlinewidth{0.311001pt}%
\definecolor{currentstroke}{rgb}{1.000000,1.000000,1.000000}%
\pgfsetstrokecolor{currentstroke}%
\pgfsetdash{}{0pt}%
\pgfpathmoveto{\pgfqpoint{6.213653in}{1.173665in}}%
\pgfpathcurveto{\pgfqpoint{6.220786in}{1.173665in}}{\pgfqpoint{6.227627in}{1.176499in}}{\pgfqpoint{6.232671in}{1.181542in}}%
\pgfpathcurveto{\pgfqpoint{6.237715in}{1.186586in}}{\pgfqpoint{6.240549in}{1.193428in}}{\pgfqpoint{6.240549in}{1.200560in}}%
\pgfpathcurveto{\pgfqpoint{6.240549in}{1.207693in}}{\pgfqpoint{6.237715in}{1.214535in}}{\pgfqpoint{6.232671in}{1.219579in}}%
\pgfpathcurveto{\pgfqpoint{6.227627in}{1.224622in}}{\pgfqpoint{6.220786in}{1.227456in}}{\pgfqpoint{6.213653in}{1.227456in}}%
\pgfpathcurveto{\pgfqpoint{6.206520in}{1.227456in}}{\pgfqpoint{6.199679in}{1.224622in}}{\pgfqpoint{6.194635in}{1.219579in}}%
\pgfpathcurveto{\pgfqpoint{6.189591in}{1.214535in}}{\pgfqpoint{6.186757in}{1.207693in}}{\pgfqpoint{6.186757in}{1.200560in}}%
\pgfpathcurveto{\pgfqpoint{6.186757in}{1.193428in}}{\pgfqpoint{6.189591in}{1.186586in}}{\pgfqpoint{6.194635in}{1.181542in}}%
\pgfpathcurveto{\pgfqpoint{6.199679in}{1.176499in}}{\pgfqpoint{6.206520in}{1.173665in}}{\pgfqpoint{6.213653in}{1.173665in}}%
\pgfpathclose%
\pgfusepath{stroke,fill}%
\end{pgfscope}%
\begin{pgfscope}%
\pgfpathrectangle{\pgfqpoint{4.985294in}{0.500000in}}{\pgfqpoint{1.764706in}{1.700000in}}%
\pgfusepath{clip}%
\pgfsetbuttcap%
\pgfsetroundjoin%
\definecolor{currentfill}{rgb}{0.964920,0.695342,0.545192}%
\pgfsetfillcolor{currentfill}%
\pgfsetlinewidth{0.311001pt}%
\definecolor{currentstroke}{rgb}{1.000000,1.000000,1.000000}%
\pgfsetstrokecolor{currentstroke}%
\pgfsetdash{}{0pt}%
\pgfpathmoveto{\pgfqpoint{6.361755in}{1.560577in}}%
\pgfpathcurveto{\pgfqpoint{6.368887in}{1.560577in}}{\pgfqpoint{6.375729in}{1.563411in}}{\pgfqpoint{6.380773in}{1.568454in}}%
\pgfpathcurveto{\pgfqpoint{6.385816in}{1.573498in}}{\pgfqpoint{6.388650in}{1.580340in}}{\pgfqpoint{6.388650in}{1.587472in}}%
\pgfpathcurveto{\pgfqpoint{6.388650in}{1.594605in}}{\pgfqpoint{6.385816in}{1.601447in}}{\pgfqpoint{6.380773in}{1.606491in}}%
\pgfpathcurveto{\pgfqpoint{6.375729in}{1.611534in}}{\pgfqpoint{6.368887in}{1.614368in}}{\pgfqpoint{6.361755in}{1.614368in}}%
\pgfpathcurveto{\pgfqpoint{6.354622in}{1.614368in}}{\pgfqpoint{6.347780in}{1.611534in}}{\pgfqpoint{6.342737in}{1.606491in}}%
\pgfpathcurveto{\pgfqpoint{6.337693in}{1.601447in}}{\pgfqpoint{6.334859in}{1.594605in}}{\pgfqpoint{6.334859in}{1.587472in}}%
\pgfpathcurveto{\pgfqpoint{6.334859in}{1.580340in}}{\pgfqpoint{6.337693in}{1.573498in}}{\pgfqpoint{6.342737in}{1.568454in}}%
\pgfpathcurveto{\pgfqpoint{6.347780in}{1.563411in}}{\pgfqpoint{6.354622in}{1.560577in}}{\pgfqpoint{6.361755in}{1.560577in}}%
\pgfpathclose%
\pgfusepath{stroke,fill}%
\end{pgfscope}%
\begin{pgfscope}%
\pgfpathrectangle{\pgfqpoint{4.985294in}{0.500000in}}{\pgfqpoint{1.764706in}{1.700000in}}%
\pgfusepath{clip}%
\pgfsetbuttcap%
\pgfsetroundjoin%
\definecolor{currentfill}{rgb}{0.966328,0.750560,0.616961}%
\pgfsetfillcolor{currentfill}%
\pgfsetlinewidth{0.311001pt}%
\definecolor{currentstroke}{rgb}{1.000000,1.000000,1.000000}%
\pgfsetstrokecolor{currentstroke}%
\pgfsetdash{}{0pt}%
\pgfpathmoveto{\pgfqpoint{6.194980in}{1.481250in}}%
\pgfpathcurveto{\pgfqpoint{6.202112in}{1.481250in}}{\pgfqpoint{6.208954in}{1.484083in}}{\pgfqpoint{6.213998in}{1.489127in}}%
\pgfpathcurveto{\pgfqpoint{6.219041in}{1.494171in}}{\pgfqpoint{6.221875in}{1.501012in}}{\pgfqpoint{6.221875in}{1.508145in}}%
\pgfpathcurveto{\pgfqpoint{6.221875in}{1.515278in}}{\pgfqpoint{6.219041in}{1.522120in}}{\pgfqpoint{6.213998in}{1.527163in}}%
\pgfpathcurveto{\pgfqpoint{6.208954in}{1.532207in}}{\pgfqpoint{6.202112in}{1.535041in}}{\pgfqpoint{6.194980in}{1.535041in}}%
\pgfpathcurveto{\pgfqpoint{6.187847in}{1.535041in}}{\pgfqpoint{6.181005in}{1.532207in}}{\pgfqpoint{6.175961in}{1.527163in}}%
\pgfpathcurveto{\pgfqpoint{6.170918in}{1.522120in}}{\pgfqpoint{6.168084in}{1.515278in}}{\pgfqpoint{6.168084in}{1.508145in}}%
\pgfpathcurveto{\pgfqpoint{6.168084in}{1.501012in}}{\pgfqpoint{6.170918in}{1.494171in}}{\pgfqpoint{6.175961in}{1.489127in}}%
\pgfpathcurveto{\pgfqpoint{6.181005in}{1.484083in}}{\pgfqpoint{6.187847in}{1.481250in}}{\pgfqpoint{6.194980in}{1.481250in}}%
\pgfpathclose%
\pgfusepath{stroke,fill}%
\end{pgfscope}%
\begin{pgfscope}%
\pgfpathrectangle{\pgfqpoint{4.985294in}{0.500000in}}{\pgfqpoint{1.764706in}{1.700000in}}%
\pgfusepath{clip}%
\pgfsetbuttcap%
\pgfsetroundjoin%
\definecolor{currentfill}{rgb}{0.976287,0.879862,0.805788}%
\pgfsetfillcolor{currentfill}%
\pgfsetlinewidth{0.311001pt}%
\definecolor{currentstroke}{rgb}{1.000000,1.000000,1.000000}%
\pgfsetstrokecolor{currentstroke}%
\pgfsetdash{}{0pt}%
\pgfpathmoveto{\pgfqpoint{5.459139in}{1.347704in}}%
\pgfpathcurveto{\pgfqpoint{5.466272in}{1.347704in}}{\pgfqpoint{5.473113in}{1.350538in}}{\pgfqpoint{5.478157in}{1.355582in}}%
\pgfpathcurveto{\pgfqpoint{5.483201in}{1.360626in}}{\pgfqpoint{5.486034in}{1.367467in}}{\pgfqpoint{5.486034in}{1.374600in}}%
\pgfpathcurveto{\pgfqpoint{5.486034in}{1.381733in}}{\pgfqpoint{5.483201in}{1.388575in}}{\pgfqpoint{5.478157in}{1.393618in}}%
\pgfpathcurveto{\pgfqpoint{5.473113in}{1.398662in}}{\pgfqpoint{5.466272in}{1.401496in}}{\pgfqpoint{5.459139in}{1.401496in}}%
\pgfpathcurveto{\pgfqpoint{5.452006in}{1.401496in}}{\pgfqpoint{5.445164in}{1.398662in}}{\pgfqpoint{5.440121in}{1.393618in}}%
\pgfpathcurveto{\pgfqpoint{5.435077in}{1.388575in}}{\pgfqpoint{5.432243in}{1.381733in}}{\pgfqpoint{5.432243in}{1.374600in}}%
\pgfpathcurveto{\pgfqpoint{5.432243in}{1.367467in}}{\pgfqpoint{5.435077in}{1.360626in}}{\pgfqpoint{5.440121in}{1.355582in}}%
\pgfpathcurveto{\pgfqpoint{5.445164in}{1.350538in}}{\pgfqpoint{5.452006in}{1.347704in}}{\pgfqpoint{5.459139in}{1.347704in}}%
\pgfpathclose%
\pgfusepath{stroke,fill}%
\end{pgfscope}%
\begin{pgfscope}%
\pgfpathrectangle{\pgfqpoint{4.985294in}{0.500000in}}{\pgfqpoint{1.764706in}{1.700000in}}%
\pgfusepath{clip}%
\pgfsetbuttcap%
\pgfsetroundjoin%
\definecolor{currentfill}{rgb}{0.965169,0.707764,0.560659}%
\pgfsetfillcolor{currentfill}%
\pgfsetlinewidth{0.311001pt}%
\definecolor{currentstroke}{rgb}{1.000000,1.000000,1.000000}%
\pgfsetstrokecolor{currentstroke}%
\pgfsetdash{}{0pt}%
\pgfpathmoveto{\pgfqpoint{5.570923in}{0.904585in}}%
\pgfpathcurveto{\pgfqpoint{5.578055in}{0.904585in}}{\pgfqpoint{5.584897in}{0.907419in}}{\pgfqpoint{5.589941in}{0.912463in}}%
\pgfpathcurveto{\pgfqpoint{5.594984in}{0.917507in}}{\pgfqpoint{5.597818in}{0.924348in}}{\pgfqpoint{5.597818in}{0.931481in}}%
\pgfpathcurveto{\pgfqpoint{5.597818in}{0.938614in}}{\pgfqpoint{5.594984in}{0.945456in}}{\pgfqpoint{5.589941in}{0.950499in}}%
\pgfpathcurveto{\pgfqpoint{5.584897in}{0.955543in}}{\pgfqpoint{5.578055in}{0.958377in}}{\pgfqpoint{5.570923in}{0.958377in}}%
\pgfpathcurveto{\pgfqpoint{5.563790in}{0.958377in}}{\pgfqpoint{5.556948in}{0.955543in}}{\pgfqpoint{5.551904in}{0.950499in}}%
\pgfpathcurveto{\pgfqpoint{5.546861in}{0.945456in}}{\pgfqpoint{5.544027in}{0.938614in}}{\pgfqpoint{5.544027in}{0.931481in}}%
\pgfpathcurveto{\pgfqpoint{5.544027in}{0.924348in}}{\pgfqpoint{5.546861in}{0.917507in}}{\pgfqpoint{5.551904in}{0.912463in}}%
\pgfpathcurveto{\pgfqpoint{5.556948in}{0.907419in}}{\pgfqpoint{5.563790in}{0.904585in}}{\pgfqpoint{5.570923in}{0.904585in}}%
\pgfpathclose%
\pgfusepath{stroke,fill}%
\end{pgfscope}%
\begin{pgfscope}%
\pgfpathrectangle{\pgfqpoint{4.985294in}{0.500000in}}{\pgfqpoint{1.764706in}{1.700000in}}%
\pgfusepath{clip}%
\pgfsetbuttcap%
\pgfsetroundjoin%
\definecolor{currentfill}{rgb}{0.957344,0.505732,0.351309}%
\pgfsetfillcolor{currentfill}%
\pgfsetlinewidth{0.311001pt}%
\definecolor{currentstroke}{rgb}{1.000000,1.000000,1.000000}%
\pgfsetstrokecolor{currentstroke}%
\pgfsetdash{}{0pt}%
\pgfpathmoveto{\pgfqpoint{6.412984in}{1.529243in}}%
\pgfpathcurveto{\pgfqpoint{6.420117in}{1.529243in}}{\pgfqpoint{6.426959in}{1.532077in}}{\pgfqpoint{6.432003in}{1.537120in}}%
\pgfpathcurveto{\pgfqpoint{6.437046in}{1.542164in}}{\pgfqpoint{6.439880in}{1.549006in}}{\pgfqpoint{6.439880in}{1.556138in}}%
\pgfpathcurveto{\pgfqpoint{6.439880in}{1.563271in}}{\pgfqpoint{6.437046in}{1.570113in}}{\pgfqpoint{6.432003in}{1.575157in}}%
\pgfpathcurveto{\pgfqpoint{6.426959in}{1.580200in}}{\pgfqpoint{6.420117in}{1.583034in}}{\pgfqpoint{6.412984in}{1.583034in}}%
\pgfpathcurveto{\pgfqpoint{6.405852in}{1.583034in}}{\pgfqpoint{6.399010in}{1.580200in}}{\pgfqpoint{6.393966in}{1.575157in}}%
\pgfpathcurveto{\pgfqpoint{6.388923in}{1.570113in}}{\pgfqpoint{6.386089in}{1.563271in}}{\pgfqpoint{6.386089in}{1.556138in}}%
\pgfpathcurveto{\pgfqpoint{6.386089in}{1.549006in}}{\pgfqpoint{6.388923in}{1.542164in}}{\pgfqpoint{6.393966in}{1.537120in}}%
\pgfpathcurveto{\pgfqpoint{6.399010in}{1.532077in}}{\pgfqpoint{6.405852in}{1.529243in}}{\pgfqpoint{6.412984in}{1.529243in}}%
\pgfpathclose%
\pgfusepath{stroke,fill}%
\end{pgfscope}%
\begin{pgfscope}%
\pgfpathrectangle{\pgfqpoint{4.985294in}{0.500000in}}{\pgfqpoint{1.764706in}{1.700000in}}%
\pgfusepath{clip}%
\pgfsetbuttcap%
\pgfsetroundjoin%
\definecolor{currentfill}{rgb}{0.963190,0.619109,0.458249}%
\pgfsetfillcolor{currentfill}%
\pgfsetlinewidth{0.311001pt}%
\definecolor{currentstroke}{rgb}{1.000000,1.000000,1.000000}%
\pgfsetstrokecolor{currentstroke}%
\pgfsetdash{}{0pt}%
\pgfpathmoveto{\pgfqpoint{5.576042in}{1.749359in}}%
\pgfpathcurveto{\pgfqpoint{5.583175in}{1.749359in}}{\pgfqpoint{5.590016in}{1.752193in}}{\pgfqpoint{5.595060in}{1.757236in}}%
\pgfpathcurveto{\pgfqpoint{5.600104in}{1.762280in}}{\pgfqpoint{5.602938in}{1.769122in}}{\pgfqpoint{5.602938in}{1.776254in}}%
\pgfpathcurveto{\pgfqpoint{5.602938in}{1.783387in}}{\pgfqpoint{5.600104in}{1.790229in}}{\pgfqpoint{5.595060in}{1.795273in}}%
\pgfpathcurveto{\pgfqpoint{5.590016in}{1.800316in}}{\pgfqpoint{5.583175in}{1.803150in}}{\pgfqpoint{5.576042in}{1.803150in}}%
\pgfpathcurveto{\pgfqpoint{5.568909in}{1.803150in}}{\pgfqpoint{5.562067in}{1.800316in}}{\pgfqpoint{5.557024in}{1.795273in}}%
\pgfpathcurveto{\pgfqpoint{5.551980in}{1.790229in}}{\pgfqpoint{5.549146in}{1.783387in}}{\pgfqpoint{5.549146in}{1.776254in}}%
\pgfpathcurveto{\pgfqpoint{5.549146in}{1.769122in}}{\pgfqpoint{5.551980in}{1.762280in}}{\pgfqpoint{5.557024in}{1.757236in}}%
\pgfpathcurveto{\pgfqpoint{5.562067in}{1.752193in}}{\pgfqpoint{5.568909in}{1.749359in}}{\pgfqpoint{5.576042in}{1.749359in}}%
\pgfpathclose%
\pgfusepath{stroke,fill}%
\end{pgfscope}%
\begin{pgfscope}%
\pgfpathrectangle{\pgfqpoint{4.985294in}{0.500000in}}{\pgfqpoint{1.764706in}{1.700000in}}%
\pgfusepath{clip}%
\pgfsetbuttcap%
\pgfsetroundjoin%
\definecolor{currentfill}{rgb}{0.981377,0.920617,0.865369}%
\pgfsetfillcolor{currentfill}%
\pgfsetlinewidth{0.311001pt}%
\definecolor{currentstroke}{rgb}{1.000000,1.000000,1.000000}%
\pgfsetstrokecolor{currentstroke}%
\pgfsetdash{}{0pt}%
\pgfpathmoveto{\pgfqpoint{6.314209in}{1.442572in}}%
\pgfpathcurveto{\pgfqpoint{6.321342in}{1.442572in}}{\pgfqpoint{6.328183in}{1.445406in}}{\pgfqpoint{6.333227in}{1.450450in}}%
\pgfpathcurveto{\pgfqpoint{6.338271in}{1.455493in}}{\pgfqpoint{6.341105in}{1.462335in}}{\pgfqpoint{6.341105in}{1.469468in}}%
\pgfpathcurveto{\pgfqpoint{6.341105in}{1.476601in}}{\pgfqpoint{6.338271in}{1.483442in}}{\pgfqpoint{6.333227in}{1.488486in}}%
\pgfpathcurveto{\pgfqpoint{6.328183in}{1.493530in}}{\pgfqpoint{6.321342in}{1.496363in}}{\pgfqpoint{6.314209in}{1.496363in}}%
\pgfpathcurveto{\pgfqpoint{6.307076in}{1.496363in}}{\pgfqpoint{6.300234in}{1.493530in}}{\pgfqpoint{6.295191in}{1.488486in}}%
\pgfpathcurveto{\pgfqpoint{6.290147in}{1.483442in}}{\pgfqpoint{6.287313in}{1.476601in}}{\pgfqpoint{6.287313in}{1.469468in}}%
\pgfpathcurveto{\pgfqpoint{6.287313in}{1.462335in}}{\pgfqpoint{6.290147in}{1.455493in}}{\pgfqpoint{6.295191in}{1.450450in}}%
\pgfpathcurveto{\pgfqpoint{6.300234in}{1.445406in}}{\pgfqpoint{6.307076in}{1.442572in}}{\pgfqpoint{6.314209in}{1.442572in}}%
\pgfpathclose%
\pgfusepath{stroke,fill}%
\end{pgfscope}%
\begin{pgfscope}%
\pgfpathrectangle{\pgfqpoint{4.985294in}{0.500000in}}{\pgfqpoint{1.764706in}{1.700000in}}%
\pgfusepath{clip}%
\pgfsetbuttcap%
\pgfsetroundjoin%
\definecolor{currentfill}{rgb}{0.976961,0.885681,0.814303}%
\pgfsetfillcolor{currentfill}%
\pgfsetlinewidth{0.311001pt}%
\definecolor{currentstroke}{rgb}{1.000000,1.000000,1.000000}%
\pgfsetstrokecolor{currentstroke}%
\pgfsetdash{}{0pt}%
\pgfpathmoveto{\pgfqpoint{6.267081in}{1.496371in}}%
\pgfpathcurveto{\pgfqpoint{6.274214in}{1.496371in}}{\pgfqpoint{6.281056in}{1.499204in}}{\pgfqpoint{6.286100in}{1.504248in}}%
\pgfpathcurveto{\pgfqpoint{6.291143in}{1.509292in}}{\pgfqpoint{6.293977in}{1.516133in}}{\pgfqpoint{6.293977in}{1.523266in}}%
\pgfpathcurveto{\pgfqpoint{6.293977in}{1.530399in}}{\pgfqpoint{6.291143in}{1.537241in}}{\pgfqpoint{6.286100in}{1.542284in}}%
\pgfpathcurveto{\pgfqpoint{6.281056in}{1.547328in}}{\pgfqpoint{6.274214in}{1.550162in}}{\pgfqpoint{6.267081in}{1.550162in}}%
\pgfpathcurveto{\pgfqpoint{6.259949in}{1.550162in}}{\pgfqpoint{6.253107in}{1.547328in}}{\pgfqpoint{6.248063in}{1.542284in}}%
\pgfpathcurveto{\pgfqpoint{6.243020in}{1.537241in}}{\pgfqpoint{6.240186in}{1.530399in}}{\pgfqpoint{6.240186in}{1.523266in}}%
\pgfpathcurveto{\pgfqpoint{6.240186in}{1.516133in}}{\pgfqpoint{6.243020in}{1.509292in}}{\pgfqpoint{6.248063in}{1.504248in}}%
\pgfpathcurveto{\pgfqpoint{6.253107in}{1.499204in}}{\pgfqpoint{6.259949in}{1.496371in}}{\pgfqpoint{6.267081in}{1.496371in}}%
\pgfpathclose%
\pgfusepath{stroke,fill}%
\end{pgfscope}%
\begin{pgfscope}%
\pgfpathrectangle{\pgfqpoint{4.985294in}{0.500000in}}{\pgfqpoint{1.764706in}{1.700000in}}%
\pgfusepath{clip}%
\pgfsetbuttcap%
\pgfsetroundjoin%
\definecolor{currentfill}{rgb}{0.962765,0.606121,0.444717}%
\pgfsetfillcolor{currentfill}%
\pgfsetlinewidth{0.311001pt}%
\definecolor{currentstroke}{rgb}{1.000000,1.000000,1.000000}%
\pgfsetstrokecolor{currentstroke}%
\pgfsetdash{}{0pt}%
\pgfpathmoveto{\pgfqpoint{5.314852in}{1.192590in}}%
\pgfpathcurveto{\pgfqpoint{5.321985in}{1.192590in}}{\pgfqpoint{5.328826in}{1.195424in}}{\pgfqpoint{5.333870in}{1.200468in}}%
\pgfpathcurveto{\pgfqpoint{5.338914in}{1.205511in}}{\pgfqpoint{5.341747in}{1.212353in}}{\pgfqpoint{5.341747in}{1.219486in}}%
\pgfpathcurveto{\pgfqpoint{5.341747in}{1.226619in}}{\pgfqpoint{5.338914in}{1.233460in}}{\pgfqpoint{5.333870in}{1.238504in}}%
\pgfpathcurveto{\pgfqpoint{5.328826in}{1.243548in}}{\pgfqpoint{5.321985in}{1.246382in}}{\pgfqpoint{5.314852in}{1.246382in}}%
\pgfpathcurveto{\pgfqpoint{5.307719in}{1.246382in}}{\pgfqpoint{5.300877in}{1.243548in}}{\pgfqpoint{5.295834in}{1.238504in}}%
\pgfpathcurveto{\pgfqpoint{5.290790in}{1.233460in}}{\pgfqpoint{5.287956in}{1.226619in}}{\pgfqpoint{5.287956in}{1.219486in}}%
\pgfpathcurveto{\pgfqpoint{5.287956in}{1.212353in}}{\pgfqpoint{5.290790in}{1.205511in}}{\pgfqpoint{5.295834in}{1.200468in}}%
\pgfpathcurveto{\pgfqpoint{5.300877in}{1.195424in}}{\pgfqpoint{5.307719in}{1.192590in}}{\pgfqpoint{5.314852in}{1.192590in}}%
\pgfpathclose%
\pgfusepath{stroke,fill}%
\end{pgfscope}%
\begin{pgfscope}%
\pgfpathrectangle{\pgfqpoint{4.985294in}{0.500000in}}{\pgfqpoint{1.764706in}{1.700000in}}%
\pgfusepath{clip}%
\pgfsetbuttcap%
\pgfsetroundjoin%
\definecolor{currentfill}{rgb}{0.978376,0.897317,0.831308}%
\pgfsetfillcolor{currentfill}%
\pgfsetlinewidth{0.311001pt}%
\definecolor{currentstroke}{rgb}{1.000000,1.000000,1.000000}%
\pgfsetstrokecolor{currentstroke}%
\pgfsetdash{}{0pt}%
\pgfpathmoveto{\pgfqpoint{5.408937in}{1.440388in}}%
\pgfpathcurveto{\pgfqpoint{5.416070in}{1.440388in}}{\pgfqpoint{5.422912in}{1.443222in}}{\pgfqpoint{5.427955in}{1.448266in}}%
\pgfpathcurveto{\pgfqpoint{5.432999in}{1.453310in}}{\pgfqpoint{5.435833in}{1.460151in}}{\pgfqpoint{5.435833in}{1.467284in}}%
\pgfpathcurveto{\pgfqpoint{5.435833in}{1.474417in}}{\pgfqpoint{5.432999in}{1.481259in}}{\pgfqpoint{5.427955in}{1.486302in}}%
\pgfpathcurveto{\pgfqpoint{5.422912in}{1.491346in}}{\pgfqpoint{5.416070in}{1.494180in}}{\pgfqpoint{5.408937in}{1.494180in}}%
\pgfpathcurveto{\pgfqpoint{5.401804in}{1.494180in}}{\pgfqpoint{5.394963in}{1.491346in}}{\pgfqpoint{5.389919in}{1.486302in}}%
\pgfpathcurveto{\pgfqpoint{5.384875in}{1.481259in}}{\pgfqpoint{5.382041in}{1.474417in}}{\pgfqpoint{5.382041in}{1.467284in}}%
\pgfpathcurveto{\pgfqpoint{5.382041in}{1.460151in}}{\pgfqpoint{5.384875in}{1.453310in}}{\pgfqpoint{5.389919in}{1.448266in}}%
\pgfpathcurveto{\pgfqpoint{5.394963in}{1.443222in}}{\pgfqpoint{5.401804in}{1.440388in}}{\pgfqpoint{5.408937in}{1.440388in}}%
\pgfpathclose%
\pgfusepath{stroke,fill}%
\end{pgfscope}%
\begin{pgfscope}%
\pgfpathrectangle{\pgfqpoint{4.985294in}{0.500000in}}{\pgfqpoint{1.764706in}{1.700000in}}%
\pgfusepath{clip}%
\pgfsetbuttcap%
\pgfsetroundjoin%
\definecolor{currentfill}{rgb}{0.971694,0.833208,0.737161}%
\pgfsetfillcolor{currentfill}%
\pgfsetlinewidth{0.311001pt}%
\definecolor{currentstroke}{rgb}{1.000000,1.000000,1.000000}%
\pgfsetstrokecolor{currentstroke}%
\pgfsetdash{}{0pt}%
\pgfpathmoveto{\pgfqpoint{6.244352in}{1.465003in}}%
\pgfpathcurveto{\pgfqpoint{6.251484in}{1.465003in}}{\pgfqpoint{6.258326in}{1.467837in}}{\pgfqpoint{6.263370in}{1.472880in}}%
\pgfpathcurveto{\pgfqpoint{6.268413in}{1.477924in}}{\pgfqpoint{6.271247in}{1.484766in}}{\pgfqpoint{6.271247in}{1.491898in}}%
\pgfpathcurveto{\pgfqpoint{6.271247in}{1.499031in}}{\pgfqpoint{6.268413in}{1.505873in}}{\pgfqpoint{6.263370in}{1.510917in}}%
\pgfpathcurveto{\pgfqpoint{6.258326in}{1.515960in}}{\pgfqpoint{6.251484in}{1.518794in}}{\pgfqpoint{6.244352in}{1.518794in}}%
\pgfpathcurveto{\pgfqpoint{6.237219in}{1.518794in}}{\pgfqpoint{6.230377in}{1.515960in}}{\pgfqpoint{6.225334in}{1.510917in}}%
\pgfpathcurveto{\pgfqpoint{6.220290in}{1.505873in}}{\pgfqpoint{6.217456in}{1.499031in}}{\pgfqpoint{6.217456in}{1.491898in}}%
\pgfpathcurveto{\pgfqpoint{6.217456in}{1.484766in}}{\pgfqpoint{6.220290in}{1.477924in}}{\pgfqpoint{6.225334in}{1.472880in}}%
\pgfpathcurveto{\pgfqpoint{6.230377in}{1.467837in}}{\pgfqpoint{6.237219in}{1.465003in}}{\pgfqpoint{6.244352in}{1.465003in}}%
\pgfpathclose%
\pgfusepath{stroke,fill}%
\end{pgfscope}%
\begin{pgfscope}%
\pgfpathrectangle{\pgfqpoint{4.985294in}{0.500000in}}{\pgfqpoint{1.764706in}{1.700000in}}%
\pgfusepath{clip}%
\pgfsetbuttcap%
\pgfsetroundjoin%
\definecolor{currentfill}{rgb}{0.969359,0.803954,0.693832}%
\pgfsetfillcolor{currentfill}%
\pgfsetlinewidth{0.311001pt}%
\definecolor{currentstroke}{rgb}{1.000000,1.000000,1.000000}%
\pgfsetstrokecolor{currentstroke}%
\pgfsetdash{}{0pt}%
\pgfpathmoveto{\pgfqpoint{5.525704in}{1.102162in}}%
\pgfpathcurveto{\pgfqpoint{5.532837in}{1.102162in}}{\pgfqpoint{5.539679in}{1.104996in}}{\pgfqpoint{5.544722in}{1.110040in}}%
\pgfpathcurveto{\pgfqpoint{5.549766in}{1.115083in}}{\pgfqpoint{5.552600in}{1.121925in}}{\pgfqpoint{5.552600in}{1.129058in}}%
\pgfpathcurveto{\pgfqpoint{5.552600in}{1.136191in}}{\pgfqpoint{5.549766in}{1.143032in}}{\pgfqpoint{5.544722in}{1.148076in}}%
\pgfpathcurveto{\pgfqpoint{5.539679in}{1.153120in}}{\pgfqpoint{5.532837in}{1.155954in}}{\pgfqpoint{5.525704in}{1.155954in}}%
\pgfpathcurveto{\pgfqpoint{5.518572in}{1.155954in}}{\pgfqpoint{5.511730in}{1.153120in}}{\pgfqpoint{5.506686in}{1.148076in}}%
\pgfpathcurveto{\pgfqpoint{5.501643in}{1.143032in}}{\pgfqpoint{5.498809in}{1.136191in}}{\pgfqpoint{5.498809in}{1.129058in}}%
\pgfpathcurveto{\pgfqpoint{5.498809in}{1.121925in}}{\pgfqpoint{5.501643in}{1.115083in}}{\pgfqpoint{5.506686in}{1.110040in}}%
\pgfpathcurveto{\pgfqpoint{5.511730in}{1.104996in}}{\pgfqpoint{5.518572in}{1.102162in}}{\pgfqpoint{5.525704in}{1.102162in}}%
\pgfpathclose%
\pgfusepath{stroke,fill}%
\end{pgfscope}%
\begin{pgfscope}%
\pgfpathrectangle{\pgfqpoint{4.985294in}{0.500000in}}{\pgfqpoint{1.764706in}{1.700000in}}%
\pgfusepath{clip}%
\pgfsetbuttcap%
\pgfsetroundjoin%
\definecolor{currentfill}{rgb}{0.969803,0.809811,0.702523}%
\pgfsetfillcolor{currentfill}%
\pgfsetlinewidth{0.311001pt}%
\definecolor{currentstroke}{rgb}{1.000000,1.000000,1.000000}%
\pgfsetstrokecolor{currentstroke}%
\pgfsetdash{}{0pt}%
\pgfpathmoveto{\pgfqpoint{6.386325in}{1.302022in}}%
\pgfpathcurveto{\pgfqpoint{6.393458in}{1.302022in}}{\pgfqpoint{6.400300in}{1.304855in}}{\pgfqpoint{6.405344in}{1.309899in}}%
\pgfpathcurveto{\pgfqpoint{6.410387in}{1.314943in}}{\pgfqpoint{6.413221in}{1.321784in}}{\pgfqpoint{6.413221in}{1.328917in}}%
\pgfpathcurveto{\pgfqpoint{6.413221in}{1.336050in}}{\pgfqpoint{6.410387in}{1.342892in}}{\pgfqpoint{6.405344in}{1.347935in}}%
\pgfpathcurveto{\pgfqpoint{6.400300in}{1.352979in}}{\pgfqpoint{6.393458in}{1.355813in}}{\pgfqpoint{6.386325in}{1.355813in}}%
\pgfpathcurveto{\pgfqpoint{6.379193in}{1.355813in}}{\pgfqpoint{6.372351in}{1.352979in}}{\pgfqpoint{6.367307in}{1.347935in}}%
\pgfpathcurveto{\pgfqpoint{6.362264in}{1.342892in}}{\pgfqpoint{6.359430in}{1.336050in}}{\pgfqpoint{6.359430in}{1.328917in}}%
\pgfpathcurveto{\pgfqpoint{6.359430in}{1.321784in}}{\pgfqpoint{6.362264in}{1.314943in}}{\pgfqpoint{6.367307in}{1.309899in}}%
\pgfpathcurveto{\pgfqpoint{6.372351in}{1.304855in}}{\pgfqpoint{6.379193in}{1.302022in}}{\pgfqpoint{6.386325in}{1.302022in}}%
\pgfpathclose%
\pgfusepath{stroke,fill}%
\end{pgfscope}%
\begin{pgfscope}%
\pgfpathrectangle{\pgfqpoint{4.985294in}{0.500000in}}{\pgfqpoint{1.764706in}{1.700000in}}%
\pgfusepath{clip}%
\pgfsetbuttcap%
\pgfsetroundjoin%
\definecolor{currentfill}{rgb}{0.972726,0.844889,0.754401}%
\pgfsetfillcolor{currentfill}%
\pgfsetlinewidth{0.311001pt}%
\definecolor{currentstroke}{rgb}{1.000000,1.000000,1.000000}%
\pgfsetstrokecolor{currentstroke}%
\pgfsetdash{}{0pt}%
\pgfpathmoveto{\pgfqpoint{5.476381in}{1.275714in}}%
\pgfpathcurveto{\pgfqpoint{5.483514in}{1.275714in}}{\pgfqpoint{5.490355in}{1.278548in}}{\pgfqpoint{5.495399in}{1.283592in}}%
\pgfpathcurveto{\pgfqpoint{5.500443in}{1.288635in}}{\pgfqpoint{5.503276in}{1.295477in}}{\pgfqpoint{5.503276in}{1.302610in}}%
\pgfpathcurveto{\pgfqpoint{5.503276in}{1.309743in}}{\pgfqpoint{5.500443in}{1.316584in}}{\pgfqpoint{5.495399in}{1.321628in}}%
\pgfpathcurveto{\pgfqpoint{5.490355in}{1.326672in}}{\pgfqpoint{5.483514in}{1.329505in}}{\pgfqpoint{5.476381in}{1.329505in}}%
\pgfpathcurveto{\pgfqpoint{5.469248in}{1.329505in}}{\pgfqpoint{5.462406in}{1.326672in}}{\pgfqpoint{5.457363in}{1.321628in}}%
\pgfpathcurveto{\pgfqpoint{5.452319in}{1.316584in}}{\pgfqpoint{5.449485in}{1.309743in}}{\pgfqpoint{5.449485in}{1.302610in}}%
\pgfpathcurveto{\pgfqpoint{5.449485in}{1.295477in}}{\pgfqpoint{5.452319in}{1.288635in}}{\pgfqpoint{5.457363in}{1.283592in}}%
\pgfpathcurveto{\pgfqpoint{5.462406in}{1.278548in}}{\pgfqpoint{5.469248in}{1.275714in}}{\pgfqpoint{5.476381in}{1.275714in}}%
\pgfpathclose%
\pgfusepath{stroke,fill}%
\end{pgfscope}%
\begin{pgfscope}%
\pgfpathrectangle{\pgfqpoint{4.985294in}{0.500000in}}{\pgfqpoint{1.764706in}{1.700000in}}%
\pgfusepath{clip}%
\pgfsetbuttcap%
\pgfsetroundjoin%
\definecolor{currentfill}{rgb}{0.981377,0.920617,0.865369}%
\pgfsetfillcolor{currentfill}%
\pgfsetlinewidth{0.311001pt}%
\definecolor{currentstroke}{rgb}{1.000000,1.000000,1.000000}%
\pgfsetstrokecolor{currentstroke}%
\pgfsetdash{}{0pt}%
\pgfpathmoveto{\pgfqpoint{6.318860in}{1.410854in}}%
\pgfpathcurveto{\pgfqpoint{6.325993in}{1.410854in}}{\pgfqpoint{6.332835in}{1.413688in}}{\pgfqpoint{6.337878in}{1.418732in}}%
\pgfpathcurveto{\pgfqpoint{6.342922in}{1.423776in}}{\pgfqpoint{6.345756in}{1.430617in}}{\pgfqpoint{6.345756in}{1.437750in}}%
\pgfpathcurveto{\pgfqpoint{6.345756in}{1.444883in}}{\pgfqpoint{6.342922in}{1.451725in}}{\pgfqpoint{6.337878in}{1.456768in}}%
\pgfpathcurveto{\pgfqpoint{6.332835in}{1.461812in}}{\pgfqpoint{6.325993in}{1.464646in}}{\pgfqpoint{6.318860in}{1.464646in}}%
\pgfpathcurveto{\pgfqpoint{6.311727in}{1.464646in}}{\pgfqpoint{6.304886in}{1.461812in}}{\pgfqpoint{6.299842in}{1.456768in}}%
\pgfpathcurveto{\pgfqpoint{6.294798in}{1.451725in}}{\pgfqpoint{6.291964in}{1.444883in}}{\pgfqpoint{6.291964in}{1.437750in}}%
\pgfpathcurveto{\pgfqpoint{6.291964in}{1.430617in}}{\pgfqpoint{6.294798in}{1.423776in}}{\pgfqpoint{6.299842in}{1.418732in}}%
\pgfpathcurveto{\pgfqpoint{6.304886in}{1.413688in}}{\pgfqpoint{6.311727in}{1.410854in}}{\pgfqpoint{6.318860in}{1.410854in}}%
\pgfpathclose%
\pgfusepath{stroke,fill}%
\end{pgfscope}%
\begin{pgfscope}%
\pgfpathrectangle{\pgfqpoint{4.985294in}{0.500000in}}{\pgfqpoint{1.764706in}{1.700000in}}%
\pgfusepath{clip}%
\pgfsetbuttcap%
\pgfsetroundjoin%
\definecolor{currentfill}{rgb}{0.977657,0.891500,0.822809}%
\pgfsetfillcolor{currentfill}%
\pgfsetlinewidth{0.311001pt}%
\definecolor{currentstroke}{rgb}{1.000000,1.000000,1.000000}%
\pgfsetstrokecolor{currentstroke}%
\pgfsetdash{}{0pt}%
\pgfpathmoveto{\pgfqpoint{6.335456in}{1.437731in}}%
\pgfpathcurveto{\pgfqpoint{6.342589in}{1.437731in}}{\pgfqpoint{6.349431in}{1.440565in}}{\pgfqpoint{6.354474in}{1.445609in}}%
\pgfpathcurveto{\pgfqpoint{6.359518in}{1.450652in}}{\pgfqpoint{6.362352in}{1.457494in}}{\pgfqpoint{6.362352in}{1.464627in}}%
\pgfpathcurveto{\pgfqpoint{6.362352in}{1.471760in}}{\pgfqpoint{6.359518in}{1.478601in}}{\pgfqpoint{6.354474in}{1.483645in}}%
\pgfpathcurveto{\pgfqpoint{6.349431in}{1.488689in}}{\pgfqpoint{6.342589in}{1.491523in}}{\pgfqpoint{6.335456in}{1.491523in}}%
\pgfpathcurveto{\pgfqpoint{6.328323in}{1.491523in}}{\pgfqpoint{6.321482in}{1.488689in}}{\pgfqpoint{6.316438in}{1.483645in}}%
\pgfpathcurveto{\pgfqpoint{6.311394in}{1.478601in}}{\pgfqpoint{6.308561in}{1.471760in}}{\pgfqpoint{6.308561in}{1.464627in}}%
\pgfpathcurveto{\pgfqpoint{6.308561in}{1.457494in}}{\pgfqpoint{6.311394in}{1.450652in}}{\pgfqpoint{6.316438in}{1.445609in}}%
\pgfpathcurveto{\pgfqpoint{6.321482in}{1.440565in}}{\pgfqpoint{6.328323in}{1.437731in}}{\pgfqpoint{6.335456in}{1.437731in}}%
\pgfpathclose%
\pgfusepath{stroke,fill}%
\end{pgfscope}%
\begin{pgfscope}%
\pgfpathrectangle{\pgfqpoint{4.985294in}{0.500000in}}{\pgfqpoint{1.764706in}{1.700000in}}%
\pgfusepath{clip}%
\pgfsetbuttcap%
\pgfsetroundjoin%
\definecolor{currentfill}{rgb}{0.968105,0.786346,0.667739}%
\pgfsetfillcolor{currentfill}%
\pgfsetlinewidth{0.311001pt}%
\definecolor{currentstroke}{rgb}{1.000000,1.000000,1.000000}%
\pgfsetstrokecolor{currentstroke}%
\pgfsetdash{}{0pt}%
\pgfpathmoveto{\pgfqpoint{6.211499in}{1.471570in}}%
\pgfpathcurveto{\pgfqpoint{6.218632in}{1.471570in}}{\pgfqpoint{6.225474in}{1.474404in}}{\pgfqpoint{6.230518in}{1.479447in}}%
\pgfpathcurveto{\pgfqpoint{6.235561in}{1.484491in}}{\pgfqpoint{6.238395in}{1.491333in}}{\pgfqpoint{6.238395in}{1.498466in}}%
\pgfpathcurveto{\pgfqpoint{6.238395in}{1.505598in}}{\pgfqpoint{6.235561in}{1.512440in}}{\pgfqpoint{6.230518in}{1.517484in}}%
\pgfpathcurveto{\pgfqpoint{6.225474in}{1.522527in}}{\pgfqpoint{6.218632in}{1.525361in}}{\pgfqpoint{6.211499in}{1.525361in}}%
\pgfpathcurveto{\pgfqpoint{6.204367in}{1.525361in}}{\pgfqpoint{6.197525in}{1.522527in}}{\pgfqpoint{6.192481in}{1.517484in}}%
\pgfpathcurveto{\pgfqpoint{6.187438in}{1.512440in}}{\pgfqpoint{6.184604in}{1.505598in}}{\pgfqpoint{6.184604in}{1.498466in}}%
\pgfpathcurveto{\pgfqpoint{6.184604in}{1.491333in}}{\pgfqpoint{6.187438in}{1.484491in}}{\pgfqpoint{6.192481in}{1.479447in}}%
\pgfpathcurveto{\pgfqpoint{6.197525in}{1.474404in}}{\pgfqpoint{6.204367in}{1.471570in}}{\pgfqpoint{6.211499in}{1.471570in}}%
\pgfpathclose%
\pgfusepath{stroke,fill}%
\end{pgfscope}%
\begin{pgfscope}%
\pgfpathrectangle{\pgfqpoint{4.985294in}{0.500000in}}{\pgfqpoint{1.764706in}{1.700000in}}%
\pgfusepath{clip}%
\pgfsetbuttcap%
\pgfsetroundjoin%
\definecolor{currentfill}{rgb}{0.964920,0.695342,0.545192}%
\pgfsetfillcolor{currentfill}%
\pgfsetlinewidth{0.311001pt}%
\definecolor{currentstroke}{rgb}{1.000000,1.000000,1.000000}%
\pgfsetstrokecolor{currentstroke}%
\pgfsetdash{}{0pt}%
\pgfpathmoveto{\pgfqpoint{6.398922in}{1.438013in}}%
\pgfpathcurveto{\pgfqpoint{6.406055in}{1.438013in}}{\pgfqpoint{6.412896in}{1.440847in}}{\pgfqpoint{6.417940in}{1.445891in}}%
\pgfpathcurveto{\pgfqpoint{6.422984in}{1.450934in}}{\pgfqpoint{6.425818in}{1.457776in}}{\pgfqpoint{6.425818in}{1.464909in}}%
\pgfpathcurveto{\pgfqpoint{6.425818in}{1.472042in}}{\pgfqpoint{6.422984in}{1.478883in}}{\pgfqpoint{6.417940in}{1.483927in}}%
\pgfpathcurveto{\pgfqpoint{6.412896in}{1.488971in}}{\pgfqpoint{6.406055in}{1.491804in}}{\pgfqpoint{6.398922in}{1.491804in}}%
\pgfpathcurveto{\pgfqpoint{6.391789in}{1.491804in}}{\pgfqpoint{6.384948in}{1.488971in}}{\pgfqpoint{6.379904in}{1.483927in}}%
\pgfpathcurveto{\pgfqpoint{6.374860in}{1.478883in}}{\pgfqpoint{6.372026in}{1.472042in}}{\pgfqpoint{6.372026in}{1.464909in}}%
\pgfpathcurveto{\pgfqpoint{6.372026in}{1.457776in}}{\pgfqpoint{6.374860in}{1.450934in}}{\pgfqpoint{6.379904in}{1.445891in}}%
\pgfpathcurveto{\pgfqpoint{6.384948in}{1.440847in}}{\pgfqpoint{6.391789in}{1.438013in}}{\pgfqpoint{6.398922in}{1.438013in}}%
\pgfpathclose%
\pgfusepath{stroke,fill}%
\end{pgfscope}%
\begin{pgfscope}%
\pgfpathrectangle{\pgfqpoint{4.985294in}{0.500000in}}{\pgfqpoint{1.764706in}{1.700000in}}%
\pgfusepath{clip}%
\pgfsetbuttcap%
\pgfsetroundjoin%
\definecolor{currentfill}{rgb}{0.973832,0.856556,0.771584}%
\pgfsetfillcolor{currentfill}%
\pgfsetlinewidth{0.311001pt}%
\definecolor{currentstroke}{rgb}{1.000000,1.000000,1.000000}%
\pgfsetstrokecolor{currentstroke}%
\pgfsetdash{}{0pt}%
\pgfpathmoveto{\pgfqpoint{6.333044in}{1.500922in}}%
\pgfpathcurveto{\pgfqpoint{6.340177in}{1.500922in}}{\pgfqpoint{6.347019in}{1.503756in}}{\pgfqpoint{6.352062in}{1.508800in}}%
\pgfpathcurveto{\pgfqpoint{6.357106in}{1.513843in}}{\pgfqpoint{6.359940in}{1.520685in}}{\pgfqpoint{6.359940in}{1.527818in}}%
\pgfpathcurveto{\pgfqpoint{6.359940in}{1.534951in}}{\pgfqpoint{6.357106in}{1.541792in}}{\pgfqpoint{6.352062in}{1.546836in}}%
\pgfpathcurveto{\pgfqpoint{6.347019in}{1.551880in}}{\pgfqpoint{6.340177in}{1.554714in}}{\pgfqpoint{6.333044in}{1.554714in}}%
\pgfpathcurveto{\pgfqpoint{6.325911in}{1.554714in}}{\pgfqpoint{6.319070in}{1.551880in}}{\pgfqpoint{6.314026in}{1.546836in}}%
\pgfpathcurveto{\pgfqpoint{6.308982in}{1.541792in}}{\pgfqpoint{6.306149in}{1.534951in}}{\pgfqpoint{6.306149in}{1.527818in}}%
\pgfpathcurveto{\pgfqpoint{6.306149in}{1.520685in}}{\pgfqpoint{6.308982in}{1.513843in}}{\pgfqpoint{6.314026in}{1.508800in}}%
\pgfpathcurveto{\pgfqpoint{6.319070in}{1.503756in}}{\pgfqpoint{6.325911in}{1.500922in}}{\pgfqpoint{6.333044in}{1.500922in}}%
\pgfpathclose%
\pgfusepath{stroke,fill}%
\end{pgfscope}%
\begin{pgfscope}%
\pgfpathrectangle{\pgfqpoint{4.985294in}{0.500000in}}{\pgfqpoint{1.764706in}{1.700000in}}%
\pgfusepath{clip}%
\pgfsetbuttcap%
\pgfsetroundjoin%
\definecolor{currentfill}{rgb}{0.972726,0.844889,0.754401}%
\pgfsetfillcolor{currentfill}%
\pgfsetlinewidth{0.311001pt}%
\definecolor{currentstroke}{rgb}{1.000000,1.000000,1.000000}%
\pgfsetstrokecolor{currentstroke}%
\pgfsetdash{}{0pt}%
\pgfpathmoveto{\pgfqpoint{5.478959in}{1.265706in}}%
\pgfpathcurveto{\pgfqpoint{5.486092in}{1.265706in}}{\pgfqpoint{5.492933in}{1.268540in}}{\pgfqpoint{5.497977in}{1.273583in}}%
\pgfpathcurveto{\pgfqpoint{5.503021in}{1.278627in}}{\pgfqpoint{5.505855in}{1.285469in}}{\pgfqpoint{5.505855in}{1.292601in}}%
\pgfpathcurveto{\pgfqpoint{5.505855in}{1.299734in}}{\pgfqpoint{5.503021in}{1.306576in}}{\pgfqpoint{5.497977in}{1.311620in}}%
\pgfpathcurveto{\pgfqpoint{5.492933in}{1.316663in}}{\pgfqpoint{5.486092in}{1.319497in}}{\pgfqpoint{5.478959in}{1.319497in}}%
\pgfpathcurveto{\pgfqpoint{5.471826in}{1.319497in}}{\pgfqpoint{5.464984in}{1.316663in}}{\pgfqpoint{5.459941in}{1.311620in}}%
\pgfpathcurveto{\pgfqpoint{5.454897in}{1.306576in}}{\pgfqpoint{5.452063in}{1.299734in}}{\pgfqpoint{5.452063in}{1.292601in}}%
\pgfpathcurveto{\pgfqpoint{5.452063in}{1.285469in}}{\pgfqpoint{5.454897in}{1.278627in}}{\pgfqpoint{5.459941in}{1.273583in}}%
\pgfpathcurveto{\pgfqpoint{5.464984in}{1.268540in}}{\pgfqpoint{5.471826in}{1.265706in}}{\pgfqpoint{5.478959in}{1.265706in}}%
\pgfpathclose%
\pgfusepath{stroke,fill}%
\end{pgfscope}%
\begin{pgfscope}%
\pgfpathrectangle{\pgfqpoint{4.985294in}{0.500000in}}{\pgfqpoint{1.764706in}{1.700000in}}%
\pgfusepath{clip}%
\pgfsetbuttcap%
\pgfsetroundjoin%
\definecolor{currentfill}{rgb}{0.965928,0.738443,0.600540}%
\pgfsetfillcolor{currentfill}%
\pgfsetlinewidth{0.311001pt}%
\definecolor{currentstroke}{rgb}{1.000000,1.000000,1.000000}%
\pgfsetstrokecolor{currentstroke}%
\pgfsetdash{}{0pt}%
\pgfpathmoveto{\pgfqpoint{6.385436in}{1.161667in}}%
\pgfpathcurveto{\pgfqpoint{6.392569in}{1.161667in}}{\pgfqpoint{6.399410in}{1.164501in}}{\pgfqpoint{6.404454in}{1.169544in}}%
\pgfpathcurveto{\pgfqpoint{6.409498in}{1.174588in}}{\pgfqpoint{6.412332in}{1.181430in}}{\pgfqpoint{6.412332in}{1.188563in}}%
\pgfpathcurveto{\pgfqpoint{6.412332in}{1.195695in}}{\pgfqpoint{6.409498in}{1.202537in}}{\pgfqpoint{6.404454in}{1.207581in}}%
\pgfpathcurveto{\pgfqpoint{6.399410in}{1.212624in}}{\pgfqpoint{6.392569in}{1.215458in}}{\pgfqpoint{6.385436in}{1.215458in}}%
\pgfpathcurveto{\pgfqpoint{6.378303in}{1.215458in}}{\pgfqpoint{6.371461in}{1.212624in}}{\pgfqpoint{6.366418in}{1.207581in}}%
\pgfpathcurveto{\pgfqpoint{6.361374in}{1.202537in}}{\pgfqpoint{6.358540in}{1.195695in}}{\pgfqpoint{6.358540in}{1.188563in}}%
\pgfpathcurveto{\pgfqpoint{6.358540in}{1.181430in}}{\pgfqpoint{6.361374in}{1.174588in}}{\pgfqpoint{6.366418in}{1.169544in}}%
\pgfpathcurveto{\pgfqpoint{6.371461in}{1.164501in}}{\pgfqpoint{6.378303in}{1.161667in}}{\pgfqpoint{6.385436in}{1.161667in}}%
\pgfpathclose%
\pgfusepath{stroke,fill}%
\end{pgfscope}%
\begin{pgfscope}%
\pgfpathrectangle{\pgfqpoint{4.985294in}{0.500000in}}{\pgfqpoint{1.764706in}{1.700000in}}%
\pgfusepath{clip}%
\pgfsetbuttcap%
\pgfsetroundjoin%
\definecolor{currentfill}{rgb}{0.966812,0.762584,0.633643}%
\pgfsetfillcolor{currentfill}%
\pgfsetlinewidth{0.311001pt}%
\definecolor{currentstroke}{rgb}{1.000000,1.000000,1.000000}%
\pgfsetstrokecolor{currentstroke}%
\pgfsetdash{}{0pt}%
\pgfpathmoveto{\pgfqpoint{5.347179in}{1.422544in}}%
\pgfpathcurveto{\pgfqpoint{5.354312in}{1.422544in}}{\pgfqpoint{5.361154in}{1.425378in}}{\pgfqpoint{5.366198in}{1.430421in}}%
\pgfpathcurveto{\pgfqpoint{5.371241in}{1.435465in}}{\pgfqpoint{5.374075in}{1.442307in}}{\pgfqpoint{5.374075in}{1.449440in}}%
\pgfpathcurveto{\pgfqpoint{5.374075in}{1.456572in}}{\pgfqpoint{5.371241in}{1.463414in}}{\pgfqpoint{5.366198in}{1.468458in}}%
\pgfpathcurveto{\pgfqpoint{5.361154in}{1.473501in}}{\pgfqpoint{5.354312in}{1.476335in}}{\pgfqpoint{5.347179in}{1.476335in}}%
\pgfpathcurveto{\pgfqpoint{5.340047in}{1.476335in}}{\pgfqpoint{5.333205in}{1.473501in}}{\pgfqpoint{5.328161in}{1.468458in}}%
\pgfpathcurveto{\pgfqpoint{5.323118in}{1.463414in}}{\pgfqpoint{5.320284in}{1.456572in}}{\pgfqpoint{5.320284in}{1.449440in}}%
\pgfpathcurveto{\pgfqpoint{5.320284in}{1.442307in}}{\pgfqpoint{5.323118in}{1.435465in}}{\pgfqpoint{5.328161in}{1.430421in}}%
\pgfpathcurveto{\pgfqpoint{5.333205in}{1.425378in}}{\pgfqpoint{5.340047in}{1.422544in}}{\pgfqpoint{5.347179in}{1.422544in}}%
\pgfpathclose%
\pgfusepath{stroke,fill}%
\end{pgfscope}%
\begin{pgfscope}%
\pgfpathrectangle{\pgfqpoint{4.985294in}{0.500000in}}{\pgfqpoint{1.764706in}{1.700000in}}%
\pgfusepath{clip}%
\pgfsetbuttcap%
\pgfsetroundjoin%
\definecolor{currentfill}{rgb}{0.964558,0.676556,0.522514}%
\pgfsetfillcolor{currentfill}%
\pgfsetlinewidth{0.311001pt}%
\definecolor{currentstroke}{rgb}{1.000000,1.000000,1.000000}%
\pgfsetstrokecolor{currentstroke}%
\pgfsetdash{}{0pt}%
\pgfpathmoveto{\pgfqpoint{5.621362in}{0.992910in}}%
\pgfpathcurveto{\pgfqpoint{5.628495in}{0.992910in}}{\pgfqpoint{5.635337in}{0.995744in}}{\pgfqpoint{5.640380in}{1.000788in}}%
\pgfpathcurveto{\pgfqpoint{5.645424in}{1.005832in}}{\pgfqpoint{5.648258in}{1.012673in}}{\pgfqpoint{5.648258in}{1.019806in}}%
\pgfpathcurveto{\pgfqpoint{5.648258in}{1.026939in}}{\pgfqpoint{5.645424in}{1.033781in}}{\pgfqpoint{5.640380in}{1.038824in}}%
\pgfpathcurveto{\pgfqpoint{5.635337in}{1.043868in}}{\pgfqpoint{5.628495in}{1.046702in}}{\pgfqpoint{5.621362in}{1.046702in}}%
\pgfpathcurveto{\pgfqpoint{5.614229in}{1.046702in}}{\pgfqpoint{5.607388in}{1.043868in}}{\pgfqpoint{5.602344in}{1.038824in}}%
\pgfpathcurveto{\pgfqpoint{5.597300in}{1.033781in}}{\pgfqpoint{5.594467in}{1.026939in}}{\pgfqpoint{5.594467in}{1.019806in}}%
\pgfpathcurveto{\pgfqpoint{5.594467in}{1.012673in}}{\pgfqpoint{5.597300in}{1.005832in}}{\pgfqpoint{5.602344in}{1.000788in}}%
\pgfpathcurveto{\pgfqpoint{5.607388in}{0.995744in}}{\pgfqpoint{5.614229in}{0.992910in}}{\pgfqpoint{5.621362in}{0.992910in}}%
\pgfpathclose%
\pgfusepath{stroke,fill}%
\end{pgfscope}%
\begin{pgfscope}%
\pgfpathrectangle{\pgfqpoint{4.985294in}{0.500000in}}{\pgfqpoint{1.764706in}{1.700000in}}%
\pgfusepath{clip}%
\pgfsetbuttcap%
\pgfsetroundjoin%
\definecolor{currentfill}{rgb}{0.964799,0.689101,0.537560}%
\pgfsetfillcolor{currentfill}%
\pgfsetlinewidth{0.311001pt}%
\definecolor{currentstroke}{rgb}{1.000000,1.000000,1.000000}%
\pgfsetstrokecolor{currentstroke}%
\pgfsetdash{}{0pt}%
\pgfpathmoveto{\pgfqpoint{5.325266in}{1.395087in}}%
\pgfpathcurveto{\pgfqpoint{5.332399in}{1.395087in}}{\pgfqpoint{5.339240in}{1.397921in}}{\pgfqpoint{5.344284in}{1.402965in}}%
\pgfpathcurveto{\pgfqpoint{5.349328in}{1.408008in}}{\pgfqpoint{5.352161in}{1.414850in}}{\pgfqpoint{5.352161in}{1.421983in}}%
\pgfpathcurveto{\pgfqpoint{5.352161in}{1.429115in}}{\pgfqpoint{5.349328in}{1.435957in}}{\pgfqpoint{5.344284in}{1.441001in}}%
\pgfpathcurveto{\pgfqpoint{5.339240in}{1.446044in}}{\pgfqpoint{5.332399in}{1.448878in}}{\pgfqpoint{5.325266in}{1.448878in}}%
\pgfpathcurveto{\pgfqpoint{5.318133in}{1.448878in}}{\pgfqpoint{5.311291in}{1.446044in}}{\pgfqpoint{5.306248in}{1.441001in}}%
\pgfpathcurveto{\pgfqpoint{5.301204in}{1.435957in}}{\pgfqpoint{5.298370in}{1.429115in}}{\pgfqpoint{5.298370in}{1.421983in}}%
\pgfpathcurveto{\pgfqpoint{5.298370in}{1.414850in}}{\pgfqpoint{5.301204in}{1.408008in}}{\pgfqpoint{5.306248in}{1.402965in}}%
\pgfpathcurveto{\pgfqpoint{5.311291in}{1.397921in}}{\pgfqpoint{5.318133in}{1.395087in}}{\pgfqpoint{5.325266in}{1.395087in}}%
\pgfpathclose%
\pgfusepath{stroke,fill}%
\end{pgfscope}%
\begin{pgfscope}%
\pgfpathrectangle{\pgfqpoint{4.985294in}{0.500000in}}{\pgfqpoint{1.764706in}{1.700000in}}%
\pgfusepath{clip}%
\pgfsetbuttcap%
\pgfsetroundjoin%
\definecolor{currentfill}{rgb}{0.979124,0.903132,0.839793}%
\pgfsetfillcolor{currentfill}%
\pgfsetlinewidth{0.311001pt}%
\definecolor{currentstroke}{rgb}{1.000000,1.000000,1.000000}%
\pgfsetstrokecolor{currentstroke}%
\pgfsetdash{}{0pt}%
\pgfpathmoveto{\pgfqpoint{6.318381in}{1.483505in}}%
\pgfpathcurveto{\pgfqpoint{6.325514in}{1.483505in}}{\pgfqpoint{6.332356in}{1.486339in}}{\pgfqpoint{6.337399in}{1.491383in}}%
\pgfpathcurveto{\pgfqpoint{6.342443in}{1.496427in}}{\pgfqpoint{6.345277in}{1.503268in}}{\pgfqpoint{6.345277in}{1.510401in}}%
\pgfpathcurveto{\pgfqpoint{6.345277in}{1.517534in}}{\pgfqpoint{6.342443in}{1.524375in}}{\pgfqpoint{6.337399in}{1.529419in}}%
\pgfpathcurveto{\pgfqpoint{6.332356in}{1.534463in}}{\pgfqpoint{6.325514in}{1.537297in}}{\pgfqpoint{6.318381in}{1.537297in}}%
\pgfpathcurveto{\pgfqpoint{6.311248in}{1.537297in}}{\pgfqpoint{6.304407in}{1.534463in}}{\pgfqpoint{6.299363in}{1.529419in}}%
\pgfpathcurveto{\pgfqpoint{6.294319in}{1.524375in}}{\pgfqpoint{6.291485in}{1.517534in}}{\pgfqpoint{6.291485in}{1.510401in}}%
\pgfpathcurveto{\pgfqpoint{6.291485in}{1.503268in}}{\pgfqpoint{6.294319in}{1.496427in}}{\pgfqpoint{6.299363in}{1.491383in}}%
\pgfpathcurveto{\pgfqpoint{6.304407in}{1.486339in}}{\pgfqpoint{6.311248in}{1.483505in}}{\pgfqpoint{6.318381in}{1.483505in}}%
\pgfpathclose%
\pgfusepath{stroke,fill}%
\end{pgfscope}%
\begin{pgfscope}%
\pgfpathrectangle{\pgfqpoint{4.985294in}{0.500000in}}{\pgfqpoint{1.764706in}{1.700000in}}%
\pgfusepath{clip}%
\pgfsetbuttcap%
\pgfsetroundjoin%
\definecolor{currentfill}{rgb}{0.950017,0.427714,0.292447}%
\pgfsetfillcolor{currentfill}%
\pgfsetlinewidth{0.311001pt}%
\definecolor{currentstroke}{rgb}{1.000000,1.000000,1.000000}%
\pgfsetstrokecolor{currentstroke}%
\pgfsetdash{}{0pt}%
\pgfpathmoveto{\pgfqpoint{6.316808in}{1.717736in}}%
\pgfpathcurveto{\pgfqpoint{6.323941in}{1.717736in}}{\pgfqpoint{6.330782in}{1.720570in}}{\pgfqpoint{6.335826in}{1.725613in}}%
\pgfpathcurveto{\pgfqpoint{6.340870in}{1.730657in}}{\pgfqpoint{6.343704in}{1.737498in}}{\pgfqpoint{6.343704in}{1.744631in}}%
\pgfpathcurveto{\pgfqpoint{6.343704in}{1.751764in}}{\pgfqpoint{6.340870in}{1.758606in}}{\pgfqpoint{6.335826in}{1.763649in}}%
\pgfpathcurveto{\pgfqpoint{6.330782in}{1.768693in}}{\pgfqpoint{6.323941in}{1.771527in}}{\pgfqpoint{6.316808in}{1.771527in}}%
\pgfpathcurveto{\pgfqpoint{6.309675in}{1.771527in}}{\pgfqpoint{6.302833in}{1.768693in}}{\pgfqpoint{6.297790in}{1.763649in}}%
\pgfpathcurveto{\pgfqpoint{6.292746in}{1.758606in}}{\pgfqpoint{6.289912in}{1.751764in}}{\pgfqpoint{6.289912in}{1.744631in}}%
\pgfpathcurveto{\pgfqpoint{6.289912in}{1.737498in}}{\pgfqpoint{6.292746in}{1.730657in}}{\pgfqpoint{6.297790in}{1.725613in}}%
\pgfpathcurveto{\pgfqpoint{6.302833in}{1.720570in}}{\pgfqpoint{6.309675in}{1.717736in}}{\pgfqpoint{6.316808in}{1.717736in}}%
\pgfpathclose%
\pgfusepath{stroke,fill}%
\end{pgfscope}%
\begin{pgfscope}%
\pgfpathrectangle{\pgfqpoint{4.985294in}{0.500000in}}{\pgfqpoint{1.764706in}{1.700000in}}%
\pgfusepath{clip}%
\pgfsetbuttcap%
\pgfsetroundjoin%
\definecolor{currentfill}{rgb}{0.966120,0.744512,0.608720}%
\pgfsetfillcolor{currentfill}%
\pgfsetlinewidth{0.311001pt}%
\definecolor{currentstroke}{rgb}{1.000000,1.000000,1.000000}%
\pgfsetstrokecolor{currentstroke}%
\pgfsetdash{}{0pt}%
\pgfpathmoveto{\pgfqpoint{5.511973in}{1.318845in}}%
\pgfpathcurveto{\pgfqpoint{5.519106in}{1.318845in}}{\pgfqpoint{5.525947in}{1.321678in}}{\pgfqpoint{5.530991in}{1.326722in}}%
\pgfpathcurveto{\pgfqpoint{5.536035in}{1.331766in}}{\pgfqpoint{5.538869in}{1.338607in}}{\pgfqpoint{5.538869in}{1.345740in}}%
\pgfpathcurveto{\pgfqpoint{5.538869in}{1.352873in}}{\pgfqpoint{5.536035in}{1.359715in}}{\pgfqpoint{5.530991in}{1.364758in}}%
\pgfpathcurveto{\pgfqpoint{5.525947in}{1.369802in}}{\pgfqpoint{5.519106in}{1.372636in}}{\pgfqpoint{5.511973in}{1.372636in}}%
\pgfpathcurveto{\pgfqpoint{5.504840in}{1.372636in}}{\pgfqpoint{5.497998in}{1.369802in}}{\pgfqpoint{5.492955in}{1.364758in}}%
\pgfpathcurveto{\pgfqpoint{5.487911in}{1.359715in}}{\pgfqpoint{5.485077in}{1.352873in}}{\pgfqpoint{5.485077in}{1.345740in}}%
\pgfpathcurveto{\pgfqpoint{5.485077in}{1.338607in}}{\pgfqpoint{5.487911in}{1.331766in}}{\pgfqpoint{5.492955in}{1.326722in}}%
\pgfpathcurveto{\pgfqpoint{5.497998in}{1.321678in}}{\pgfqpoint{5.504840in}{1.318845in}}{\pgfqpoint{5.511973in}{1.318845in}}%
\pgfpathclose%
\pgfusepath{stroke,fill}%
\end{pgfscope}%
\begin{pgfscope}%
\pgfpathrectangle{\pgfqpoint{4.985294in}{0.500000in}}{\pgfqpoint{1.764706in}{1.700000in}}%
\pgfusepath{clip}%
\pgfsetbuttcap%
\pgfsetroundjoin%
\definecolor{currentfill}{rgb}{0.973832,0.856556,0.771584}%
\pgfsetfillcolor{currentfill}%
\pgfsetlinewidth{0.311001pt}%
\definecolor{currentstroke}{rgb}{1.000000,1.000000,1.000000}%
\pgfsetstrokecolor{currentstroke}%
\pgfsetdash{}{0pt}%
\pgfpathmoveto{\pgfqpoint{5.498245in}{1.528744in}}%
\pgfpathcurveto{\pgfqpoint{5.505378in}{1.528744in}}{\pgfqpoint{5.512220in}{1.531578in}}{\pgfqpoint{5.517263in}{1.536622in}}%
\pgfpathcurveto{\pgfqpoint{5.522307in}{1.541665in}}{\pgfqpoint{5.525141in}{1.548507in}}{\pgfqpoint{5.525141in}{1.555640in}}%
\pgfpathcurveto{\pgfqpoint{5.525141in}{1.562773in}}{\pgfqpoint{5.522307in}{1.569614in}}{\pgfqpoint{5.517263in}{1.574658in}}%
\pgfpathcurveto{\pgfqpoint{5.512220in}{1.579702in}}{\pgfqpoint{5.505378in}{1.582536in}}{\pgfqpoint{5.498245in}{1.582536in}}%
\pgfpathcurveto{\pgfqpoint{5.491112in}{1.582536in}}{\pgfqpoint{5.484271in}{1.579702in}}{\pgfqpoint{5.479227in}{1.574658in}}%
\pgfpathcurveto{\pgfqpoint{5.474183in}{1.569614in}}{\pgfqpoint{5.471349in}{1.562773in}}{\pgfqpoint{5.471349in}{1.555640in}}%
\pgfpathcurveto{\pgfqpoint{5.471349in}{1.548507in}}{\pgfqpoint{5.474183in}{1.541665in}}{\pgfqpoint{5.479227in}{1.536622in}}%
\pgfpathcurveto{\pgfqpoint{5.484271in}{1.531578in}}{\pgfqpoint{5.491112in}{1.528744in}}{\pgfqpoint{5.498245in}{1.528744in}}%
\pgfpathclose%
\pgfusepath{stroke,fill}%
\end{pgfscope}%
\begin{pgfscope}%
\pgfpathrectangle{\pgfqpoint{4.985294in}{0.500000in}}{\pgfqpoint{1.764706in}{1.700000in}}%
\pgfusepath{clip}%
\pgfsetbuttcap%
\pgfsetroundjoin%
\definecolor{currentfill}{rgb}{0.973832,0.856556,0.771584}%
\pgfsetfillcolor{currentfill}%
\pgfsetlinewidth{0.311001pt}%
\definecolor{currentstroke}{rgb}{1.000000,1.000000,1.000000}%
\pgfsetstrokecolor{currentstroke}%
\pgfsetdash{}{0pt}%
\pgfpathmoveto{\pgfqpoint{6.360819in}{1.219298in}}%
\pgfpathcurveto{\pgfqpoint{6.367952in}{1.219298in}}{\pgfqpoint{6.374794in}{1.222132in}}{\pgfqpoint{6.379838in}{1.227175in}}%
\pgfpathcurveto{\pgfqpoint{6.384881in}{1.232219in}}{\pgfqpoint{6.387715in}{1.239061in}}{\pgfqpoint{6.387715in}{1.246194in}}%
\pgfpathcurveto{\pgfqpoint{6.387715in}{1.253326in}}{\pgfqpoint{6.384881in}{1.260168in}}{\pgfqpoint{6.379838in}{1.265212in}}%
\pgfpathcurveto{\pgfqpoint{6.374794in}{1.270255in}}{\pgfqpoint{6.367952in}{1.273089in}}{\pgfqpoint{6.360819in}{1.273089in}}%
\pgfpathcurveto{\pgfqpoint{6.353687in}{1.273089in}}{\pgfqpoint{6.346845in}{1.270255in}}{\pgfqpoint{6.341801in}{1.265212in}}%
\pgfpathcurveto{\pgfqpoint{6.336758in}{1.260168in}}{\pgfqpoint{6.333924in}{1.253326in}}{\pgfqpoint{6.333924in}{1.246194in}}%
\pgfpathcurveto{\pgfqpoint{6.333924in}{1.239061in}}{\pgfqpoint{6.336758in}{1.232219in}}{\pgfqpoint{6.341801in}{1.227175in}}%
\pgfpathcurveto{\pgfqpoint{6.346845in}{1.222132in}}{\pgfqpoint{6.353687in}{1.219298in}}{\pgfqpoint{6.360819in}{1.219298in}}%
\pgfpathclose%
\pgfusepath{stroke,fill}%
\end{pgfscope}%
\begin{pgfscope}%
\pgfpathrectangle{\pgfqpoint{4.985294in}{0.500000in}}{\pgfqpoint{1.764706in}{1.700000in}}%
\pgfusepath{clip}%
\pgfsetbuttcap%
\pgfsetroundjoin%
\definecolor{currentfill}{rgb}{0.930781,0.313740,0.244688}%
\pgfsetfillcolor{currentfill}%
\pgfsetlinewidth{0.311001pt}%
\definecolor{currentstroke}{rgb}{1.000000,1.000000,1.000000}%
\pgfsetstrokecolor{currentstroke}%
\pgfsetdash{}{0pt}%
\pgfpathmoveto{\pgfqpoint{6.467640in}{1.279445in}}%
\pgfpathcurveto{\pgfqpoint{6.474773in}{1.279445in}}{\pgfqpoint{6.481614in}{1.282279in}}{\pgfqpoint{6.486658in}{1.287323in}}%
\pgfpathcurveto{\pgfqpoint{6.491702in}{1.292367in}}{\pgfqpoint{6.494536in}{1.299208in}}{\pgfqpoint{6.494536in}{1.306341in}}%
\pgfpathcurveto{\pgfqpoint{6.494536in}{1.313474in}}{\pgfqpoint{6.491702in}{1.320316in}}{\pgfqpoint{6.486658in}{1.325359in}}%
\pgfpathcurveto{\pgfqpoint{6.481614in}{1.330403in}}{\pgfqpoint{6.474773in}{1.333237in}}{\pgfqpoint{6.467640in}{1.333237in}}%
\pgfpathcurveto{\pgfqpoint{6.460507in}{1.333237in}}{\pgfqpoint{6.453665in}{1.330403in}}{\pgfqpoint{6.448622in}{1.325359in}}%
\pgfpathcurveto{\pgfqpoint{6.443578in}{1.320316in}}{\pgfqpoint{6.440744in}{1.313474in}}{\pgfqpoint{6.440744in}{1.306341in}}%
\pgfpathcurveto{\pgfqpoint{6.440744in}{1.299208in}}{\pgfqpoint{6.443578in}{1.292367in}}{\pgfqpoint{6.448622in}{1.287323in}}%
\pgfpathcurveto{\pgfqpoint{6.453665in}{1.282279in}}{\pgfqpoint{6.460507in}{1.279445in}}{\pgfqpoint{6.467640in}{1.279445in}}%
\pgfpathclose%
\pgfusepath{stroke,fill}%
\end{pgfscope}%
\begin{pgfscope}%
\pgfpathrectangle{\pgfqpoint{4.985294in}{0.500000in}}{\pgfqpoint{1.764706in}{1.700000in}}%
\pgfusepath{clip}%
\pgfsetbuttcap%
\pgfsetroundjoin%
\definecolor{currentfill}{rgb}{0.963559,0.632016,0.472047}%
\pgfsetfillcolor{currentfill}%
\pgfsetlinewidth{0.311001pt}%
\definecolor{currentstroke}{rgb}{1.000000,1.000000,1.000000}%
\pgfsetstrokecolor{currentstroke}%
\pgfsetdash{}{0pt}%
\pgfpathmoveto{\pgfqpoint{5.308669in}{1.325469in}}%
\pgfpathcurveto{\pgfqpoint{5.315802in}{1.325469in}}{\pgfqpoint{5.322643in}{1.328303in}}{\pgfqpoint{5.327687in}{1.333347in}}%
\pgfpathcurveto{\pgfqpoint{5.332731in}{1.338390in}}{\pgfqpoint{5.335565in}{1.345232in}}{\pgfqpoint{5.335565in}{1.352365in}}%
\pgfpathcurveto{\pgfqpoint{5.335565in}{1.359498in}}{\pgfqpoint{5.332731in}{1.366339in}}{\pgfqpoint{5.327687in}{1.371383in}}%
\pgfpathcurveto{\pgfqpoint{5.322643in}{1.376427in}}{\pgfqpoint{5.315802in}{1.379260in}}{\pgfqpoint{5.308669in}{1.379260in}}%
\pgfpathcurveto{\pgfqpoint{5.301536in}{1.379260in}}{\pgfqpoint{5.294695in}{1.376427in}}{\pgfqpoint{5.289651in}{1.371383in}}%
\pgfpathcurveto{\pgfqpoint{5.284607in}{1.366339in}}{\pgfqpoint{5.281773in}{1.359498in}}{\pgfqpoint{5.281773in}{1.352365in}}%
\pgfpathcurveto{\pgfqpoint{5.281773in}{1.345232in}}{\pgfqpoint{5.284607in}{1.338390in}}{\pgfqpoint{5.289651in}{1.333347in}}%
\pgfpathcurveto{\pgfqpoint{5.294695in}{1.328303in}}{\pgfqpoint{5.301536in}{1.325469in}}{\pgfqpoint{5.308669in}{1.325469in}}%
\pgfpathclose%
\pgfusepath{stroke,fill}%
\end{pgfscope}%
\begin{pgfscope}%
\pgfpathrectangle{\pgfqpoint{4.985294in}{0.500000in}}{\pgfqpoint{1.764706in}{1.700000in}}%
\pgfusepath{clip}%
\pgfsetbuttcap%
\pgfsetroundjoin%
\definecolor{currentfill}{rgb}{0.968931,0.798091,0.685123}%
\pgfsetfillcolor{currentfill}%
\pgfsetlinewidth{0.311001pt}%
\definecolor{currentstroke}{rgb}{1.000000,1.000000,1.000000}%
\pgfsetstrokecolor{currentstroke}%
\pgfsetdash{}{0pt}%
\pgfpathmoveto{\pgfqpoint{6.296695in}{1.629220in}}%
\pgfpathcurveto{\pgfqpoint{6.303827in}{1.629220in}}{\pgfqpoint{6.310669in}{1.632054in}}{\pgfqpoint{6.315713in}{1.637098in}}%
\pgfpathcurveto{\pgfqpoint{6.320756in}{1.642141in}}{\pgfqpoint{6.323590in}{1.648983in}}{\pgfqpoint{6.323590in}{1.656116in}}%
\pgfpathcurveto{\pgfqpoint{6.323590in}{1.663249in}}{\pgfqpoint{6.320756in}{1.670090in}}{\pgfqpoint{6.315713in}{1.675134in}}%
\pgfpathcurveto{\pgfqpoint{6.310669in}{1.680177in}}{\pgfqpoint{6.303827in}{1.683011in}}{\pgfqpoint{6.296695in}{1.683011in}}%
\pgfpathcurveto{\pgfqpoint{6.289562in}{1.683011in}}{\pgfqpoint{6.282720in}{1.680177in}}{\pgfqpoint{6.277676in}{1.675134in}}%
\pgfpathcurveto{\pgfqpoint{6.272633in}{1.670090in}}{\pgfqpoint{6.269799in}{1.663249in}}{\pgfqpoint{6.269799in}{1.656116in}}%
\pgfpathcurveto{\pgfqpoint{6.269799in}{1.648983in}}{\pgfqpoint{6.272633in}{1.642141in}}{\pgfqpoint{6.277676in}{1.637098in}}%
\pgfpathcurveto{\pgfqpoint{6.282720in}{1.632054in}}{\pgfqpoint{6.289562in}{1.629220in}}{\pgfqpoint{6.296695in}{1.629220in}}%
\pgfpathclose%
\pgfusepath{stroke,fill}%
\end{pgfscope}%
\begin{pgfscope}%
\pgfpathrectangle{\pgfqpoint{4.985294in}{0.500000in}}{\pgfqpoint{1.764706in}{1.700000in}}%
\pgfusepath{clip}%
\pgfsetbuttcap%
\pgfsetroundjoin%
\definecolor{currentfill}{rgb}{0.968105,0.786346,0.667739}%
\pgfsetfillcolor{currentfill}%
\pgfsetlinewidth{0.311001pt}%
\definecolor{currentstroke}{rgb}{1.000000,1.000000,1.000000}%
\pgfsetstrokecolor{currentstroke}%
\pgfsetdash{}{0pt}%
\pgfpathmoveto{\pgfqpoint{5.563734in}{1.023784in}}%
\pgfpathcurveto{\pgfqpoint{5.570867in}{1.023784in}}{\pgfqpoint{5.577708in}{1.026618in}}{\pgfqpoint{5.582752in}{1.031661in}}%
\pgfpathcurveto{\pgfqpoint{5.587796in}{1.036705in}}{\pgfqpoint{5.590630in}{1.043547in}}{\pgfqpoint{5.590630in}{1.050679in}}%
\pgfpathcurveto{\pgfqpoint{5.590630in}{1.057812in}}{\pgfqpoint{5.587796in}{1.064654in}}{\pgfqpoint{5.582752in}{1.069698in}}%
\pgfpathcurveto{\pgfqpoint{5.577708in}{1.074741in}}{\pgfqpoint{5.570867in}{1.077575in}}{\pgfqpoint{5.563734in}{1.077575in}}%
\pgfpathcurveto{\pgfqpoint{5.556601in}{1.077575in}}{\pgfqpoint{5.549759in}{1.074741in}}{\pgfqpoint{5.544716in}{1.069698in}}%
\pgfpathcurveto{\pgfqpoint{5.539672in}{1.064654in}}{\pgfqpoint{5.536838in}{1.057812in}}{\pgfqpoint{5.536838in}{1.050679in}}%
\pgfpathcurveto{\pgfqpoint{5.536838in}{1.043547in}}{\pgfqpoint{5.539672in}{1.036705in}}{\pgfqpoint{5.544716in}{1.031661in}}%
\pgfpathcurveto{\pgfqpoint{5.549759in}{1.026618in}}{\pgfqpoint{5.556601in}{1.023784in}}{\pgfqpoint{5.563734in}{1.023784in}}%
\pgfpathclose%
\pgfusepath{stroke,fill}%
\end{pgfscope}%
\begin{pgfscope}%
\pgfpathrectangle{\pgfqpoint{4.985294in}{0.500000in}}{\pgfqpoint{1.764706in}{1.700000in}}%
\pgfusepath{clip}%
\pgfsetbuttcap%
\pgfsetroundjoin%
\definecolor{currentfill}{rgb}{0.938993,0.352507,0.254528}%
\pgfsetfillcolor{currentfill}%
\pgfsetlinewidth{0.311001pt}%
\definecolor{currentstroke}{rgb}{1.000000,1.000000,1.000000}%
\pgfsetstrokecolor{currentstroke}%
\pgfsetdash{}{0pt}%
\pgfpathmoveto{\pgfqpoint{5.624923in}{1.129501in}}%
\pgfpathcurveto{\pgfqpoint{5.632056in}{1.129501in}}{\pgfqpoint{5.638898in}{1.132335in}}{\pgfqpoint{5.643941in}{1.137378in}}%
\pgfpathcurveto{\pgfqpoint{5.648985in}{1.142422in}}{\pgfqpoint{5.651819in}{1.149264in}}{\pgfqpoint{5.651819in}{1.156397in}}%
\pgfpathcurveto{\pgfqpoint{5.651819in}{1.163529in}}{\pgfqpoint{5.648985in}{1.170371in}}{\pgfqpoint{5.643941in}{1.175415in}}%
\pgfpathcurveto{\pgfqpoint{5.638898in}{1.180458in}}{\pgfqpoint{5.632056in}{1.183292in}}{\pgfqpoint{5.624923in}{1.183292in}}%
\pgfpathcurveto{\pgfqpoint{5.617790in}{1.183292in}}{\pgfqpoint{5.610949in}{1.180458in}}{\pgfqpoint{5.605905in}{1.175415in}}%
\pgfpathcurveto{\pgfqpoint{5.600861in}{1.170371in}}{\pgfqpoint{5.598028in}{1.163529in}}{\pgfqpoint{5.598028in}{1.156397in}}%
\pgfpathcurveto{\pgfqpoint{5.598028in}{1.149264in}}{\pgfqpoint{5.600861in}{1.142422in}}{\pgfqpoint{5.605905in}{1.137378in}}%
\pgfpathcurveto{\pgfqpoint{5.610949in}{1.132335in}}{\pgfqpoint{5.617790in}{1.129501in}}{\pgfqpoint{5.624923in}{1.129501in}}%
\pgfpathclose%
\pgfusepath{stroke,fill}%
\end{pgfscope}%
\begin{pgfscope}%
\pgfpathrectangle{\pgfqpoint{4.985294in}{0.500000in}}{\pgfqpoint{1.764706in}{1.700000in}}%
\pgfusepath{clip}%
\pgfsetbuttcap%
\pgfsetroundjoin%
\definecolor{currentfill}{rgb}{0.979891,0.908948,0.848279}%
\pgfsetfillcolor{currentfill}%
\pgfsetlinewidth{0.311001pt}%
\definecolor{currentstroke}{rgb}{1.000000,1.000000,1.000000}%
\pgfsetstrokecolor{currentstroke}%
\pgfsetdash{}{0pt}%
\pgfpathmoveto{\pgfqpoint{5.421395in}{1.414688in}}%
\pgfpathcurveto{\pgfqpoint{5.428528in}{1.414688in}}{\pgfqpoint{5.435370in}{1.417522in}}{\pgfqpoint{5.440413in}{1.422566in}}%
\pgfpathcurveto{\pgfqpoint{5.445457in}{1.427610in}}{\pgfqpoint{5.448291in}{1.434451in}}{\pgfqpoint{5.448291in}{1.441584in}}%
\pgfpathcurveto{\pgfqpoint{5.448291in}{1.448717in}}{\pgfqpoint{5.445457in}{1.455559in}}{\pgfqpoint{5.440413in}{1.460602in}}%
\pgfpathcurveto{\pgfqpoint{5.435370in}{1.465646in}}{\pgfqpoint{5.428528in}{1.468480in}}{\pgfqpoint{5.421395in}{1.468480in}}%
\pgfpathcurveto{\pgfqpoint{5.414262in}{1.468480in}}{\pgfqpoint{5.407421in}{1.465646in}}{\pgfqpoint{5.402377in}{1.460602in}}%
\pgfpathcurveto{\pgfqpoint{5.397333in}{1.455559in}}{\pgfqpoint{5.394499in}{1.448717in}}{\pgfqpoint{5.394499in}{1.441584in}}%
\pgfpathcurveto{\pgfqpoint{5.394499in}{1.434451in}}{\pgfqpoint{5.397333in}{1.427610in}}{\pgfqpoint{5.402377in}{1.422566in}}%
\pgfpathcurveto{\pgfqpoint{5.407421in}{1.417522in}}{\pgfqpoint{5.414262in}{1.414688in}}{\pgfqpoint{5.421395in}{1.414688in}}%
\pgfpathclose%
\pgfusepath{stroke,fill}%
\end{pgfscope}%
\begin{pgfscope}%
\pgfpathrectangle{\pgfqpoint{4.985294in}{0.500000in}}{\pgfqpoint{1.764706in}{1.700000in}}%
\pgfusepath{clip}%
\pgfsetbuttcap%
\pgfsetroundjoin%
\definecolor{currentfill}{rgb}{0.967735,0.780441,0.659127}%
\pgfsetfillcolor{currentfill}%
\pgfsetlinewidth{0.311001pt}%
\definecolor{currentstroke}{rgb}{1.000000,1.000000,1.000000}%
\pgfsetstrokecolor{currentstroke}%
\pgfsetdash{}{0pt}%
\pgfpathmoveto{\pgfqpoint{6.391160in}{1.357018in}}%
\pgfpathcurveto{\pgfqpoint{6.398293in}{1.357018in}}{\pgfqpoint{6.405134in}{1.359851in}}{\pgfqpoint{6.410178in}{1.364895in}}%
\pgfpathcurveto{\pgfqpoint{6.415222in}{1.369939in}}{\pgfqpoint{6.418055in}{1.376780in}}{\pgfqpoint{6.418055in}{1.383913in}}%
\pgfpathcurveto{\pgfqpoint{6.418055in}{1.391046in}}{\pgfqpoint{6.415222in}{1.397888in}}{\pgfqpoint{6.410178in}{1.402931in}}%
\pgfpathcurveto{\pgfqpoint{6.405134in}{1.407975in}}{\pgfqpoint{6.398293in}{1.410809in}}{\pgfqpoint{6.391160in}{1.410809in}}%
\pgfpathcurveto{\pgfqpoint{6.384027in}{1.410809in}}{\pgfqpoint{6.377185in}{1.407975in}}{\pgfqpoint{6.372142in}{1.402931in}}%
\pgfpathcurveto{\pgfqpoint{6.367098in}{1.397888in}}{\pgfqpoint{6.364264in}{1.391046in}}{\pgfqpoint{6.364264in}{1.383913in}}%
\pgfpathcurveto{\pgfqpoint{6.364264in}{1.376780in}}{\pgfqpoint{6.367098in}{1.369939in}}{\pgfqpoint{6.372142in}{1.364895in}}%
\pgfpathcurveto{\pgfqpoint{6.377185in}{1.359851in}}{\pgfqpoint{6.384027in}{1.357018in}}{\pgfqpoint{6.391160in}{1.357018in}}%
\pgfpathclose%
\pgfusepath{stroke,fill}%
\end{pgfscope}%
\begin{pgfscope}%
\pgfpathrectangle{\pgfqpoint{4.985294in}{0.500000in}}{\pgfqpoint{1.764706in}{1.700000in}}%
\pgfusepath{clip}%
\pgfsetbuttcap%
\pgfsetroundjoin%
\definecolor{currentfill}{rgb}{0.977657,0.891500,0.822809}%
\pgfsetfillcolor{currentfill}%
\pgfsetlinewidth{0.311001pt}%
\definecolor{currentstroke}{rgb}{1.000000,1.000000,1.000000}%
\pgfsetstrokecolor{currentstroke}%
\pgfsetdash{}{0pt}%
\pgfpathmoveto{\pgfqpoint{5.451648in}{1.248358in}}%
\pgfpathcurveto{\pgfqpoint{5.458781in}{1.248358in}}{\pgfqpoint{5.465622in}{1.251192in}}{\pgfqpoint{5.470666in}{1.256236in}}%
\pgfpathcurveto{\pgfqpoint{5.475710in}{1.261280in}}{\pgfqpoint{5.478543in}{1.268121in}}{\pgfqpoint{5.478543in}{1.275254in}}%
\pgfpathcurveto{\pgfqpoint{5.478543in}{1.282387in}}{\pgfqpoint{5.475710in}{1.289229in}}{\pgfqpoint{5.470666in}{1.294272in}}%
\pgfpathcurveto{\pgfqpoint{5.465622in}{1.299316in}}{\pgfqpoint{5.458781in}{1.302150in}}{\pgfqpoint{5.451648in}{1.302150in}}%
\pgfpathcurveto{\pgfqpoint{5.444515in}{1.302150in}}{\pgfqpoint{5.437673in}{1.299316in}}{\pgfqpoint{5.432630in}{1.294272in}}%
\pgfpathcurveto{\pgfqpoint{5.427586in}{1.289229in}}{\pgfqpoint{5.424752in}{1.282387in}}{\pgfqpoint{5.424752in}{1.275254in}}%
\pgfpathcurveto{\pgfqpoint{5.424752in}{1.268121in}}{\pgfqpoint{5.427586in}{1.261280in}}{\pgfqpoint{5.432630in}{1.256236in}}%
\pgfpathcurveto{\pgfqpoint{5.437673in}{1.251192in}}{\pgfqpoint{5.444515in}{1.248358in}}{\pgfqpoint{5.451648in}{1.248358in}}%
\pgfpathclose%
\pgfusepath{stroke,fill}%
\end{pgfscope}%
\begin{pgfscope}%
\pgfpathrectangle{\pgfqpoint{4.985294in}{0.500000in}}{\pgfqpoint{1.764706in}{1.700000in}}%
\pgfusepath{clip}%
\pgfsetbuttcap%
\pgfsetroundjoin%
\definecolor{currentfill}{rgb}{0.973271,0.850724,0.762998}%
\pgfsetfillcolor{currentfill}%
\pgfsetlinewidth{0.311001pt}%
\definecolor{currentstroke}{rgb}{1.000000,1.000000,1.000000}%
\pgfsetstrokecolor{currentstroke}%
\pgfsetdash{}{0pt}%
\pgfpathmoveto{\pgfqpoint{5.498946in}{1.015215in}}%
\pgfpathcurveto{\pgfqpoint{5.506079in}{1.015215in}}{\pgfqpoint{5.512921in}{1.018048in}}{\pgfqpoint{5.517965in}{1.023092in}}%
\pgfpathcurveto{\pgfqpoint{5.523008in}{1.028136in}}{\pgfqpoint{5.525842in}{1.034977in}}{\pgfqpoint{5.525842in}{1.042110in}}%
\pgfpathcurveto{\pgfqpoint{5.525842in}{1.049243in}}{\pgfqpoint{5.523008in}{1.056085in}}{\pgfqpoint{5.517965in}{1.061128in}}%
\pgfpathcurveto{\pgfqpoint{5.512921in}{1.066172in}}{\pgfqpoint{5.506079in}{1.069006in}}{\pgfqpoint{5.498946in}{1.069006in}}%
\pgfpathcurveto{\pgfqpoint{5.491814in}{1.069006in}}{\pgfqpoint{5.484972in}{1.066172in}}{\pgfqpoint{5.479928in}{1.061128in}}%
\pgfpathcurveto{\pgfqpoint{5.474885in}{1.056085in}}{\pgfqpoint{5.472051in}{1.049243in}}{\pgfqpoint{5.472051in}{1.042110in}}%
\pgfpathcurveto{\pgfqpoint{5.472051in}{1.034977in}}{\pgfqpoint{5.474885in}{1.028136in}}{\pgfqpoint{5.479928in}{1.023092in}}%
\pgfpathcurveto{\pgfqpoint{5.484972in}{1.018048in}}{\pgfqpoint{5.491814in}{1.015215in}}{\pgfqpoint{5.498946in}{1.015215in}}%
\pgfpathclose%
\pgfusepath{stroke,fill}%
\end{pgfscope}%
\begin{pgfscope}%
\pgfpathrectangle{\pgfqpoint{4.985294in}{0.500000in}}{\pgfqpoint{1.764706in}{1.700000in}}%
\pgfusepath{clip}%
\pgfsetbuttcap%
\pgfsetroundjoin%
\definecolor{currentfill}{rgb}{0.972201,0.839051,0.745789}%
\pgfsetfillcolor{currentfill}%
\pgfsetlinewidth{0.311001pt}%
\definecolor{currentstroke}{rgb}{1.000000,1.000000,1.000000}%
\pgfsetstrokecolor{currentstroke}%
\pgfsetdash{}{0pt}%
\pgfpathmoveto{\pgfqpoint{5.519295in}{1.568614in}}%
\pgfpathcurveto{\pgfqpoint{5.526428in}{1.568614in}}{\pgfqpoint{5.533269in}{1.571448in}}{\pgfqpoint{5.538313in}{1.576492in}}%
\pgfpathcurveto{\pgfqpoint{5.543357in}{1.581536in}}{\pgfqpoint{5.546190in}{1.588377in}}{\pgfqpoint{5.546190in}{1.595510in}}%
\pgfpathcurveto{\pgfqpoint{5.546190in}{1.602643in}}{\pgfqpoint{5.543357in}{1.609484in}}{\pgfqpoint{5.538313in}{1.614528in}}%
\pgfpathcurveto{\pgfqpoint{5.533269in}{1.619572in}}{\pgfqpoint{5.526428in}{1.622406in}}{\pgfqpoint{5.519295in}{1.622406in}}%
\pgfpathcurveto{\pgfqpoint{5.512162in}{1.622406in}}{\pgfqpoint{5.505320in}{1.619572in}}{\pgfqpoint{5.500277in}{1.614528in}}%
\pgfpathcurveto{\pgfqpoint{5.495233in}{1.609484in}}{\pgfqpoint{5.492399in}{1.602643in}}{\pgfqpoint{5.492399in}{1.595510in}}%
\pgfpathcurveto{\pgfqpoint{5.492399in}{1.588377in}}{\pgfqpoint{5.495233in}{1.581536in}}{\pgfqpoint{5.500277in}{1.576492in}}%
\pgfpathcurveto{\pgfqpoint{5.505320in}{1.571448in}}{\pgfqpoint{5.512162in}{1.568614in}}{\pgfqpoint{5.519295in}{1.568614in}}%
\pgfpathclose%
\pgfusepath{stroke,fill}%
\end{pgfscope}%
\begin{pgfscope}%
\pgfpathrectangle{\pgfqpoint{4.985294in}{0.500000in}}{\pgfqpoint{1.764706in}{1.700000in}}%
\pgfusepath{clip}%
\pgfsetbuttcap%
\pgfsetroundjoin%
\definecolor{currentfill}{rgb}{0.954476,0.470822,0.323110}%
\pgfsetfillcolor{currentfill}%
\pgfsetlinewidth{0.311001pt}%
\definecolor{currentstroke}{rgb}{1.000000,1.000000,1.000000}%
\pgfsetstrokecolor{currentstroke}%
\pgfsetdash{}{0pt}%
\pgfpathmoveto{\pgfqpoint{5.658225in}{0.864146in}}%
\pgfpathcurveto{\pgfqpoint{5.665357in}{0.864146in}}{\pgfqpoint{5.672199in}{0.866980in}}{\pgfqpoint{5.677243in}{0.872023in}}%
\pgfpathcurveto{\pgfqpoint{5.682286in}{0.877067in}}{\pgfqpoint{5.685120in}{0.883908in}}{\pgfqpoint{5.685120in}{0.891041in}}%
\pgfpathcurveto{\pgfqpoint{5.685120in}{0.898174in}}{\pgfqpoint{5.682286in}{0.905016in}}{\pgfqpoint{5.677243in}{0.910059in}}%
\pgfpathcurveto{\pgfqpoint{5.672199in}{0.915103in}}{\pgfqpoint{5.665357in}{0.917937in}}{\pgfqpoint{5.658225in}{0.917937in}}%
\pgfpathcurveto{\pgfqpoint{5.651092in}{0.917937in}}{\pgfqpoint{5.644250in}{0.915103in}}{\pgfqpoint{5.639207in}{0.910059in}}%
\pgfpathcurveto{\pgfqpoint{5.634163in}{0.905016in}}{\pgfqpoint{5.631329in}{0.898174in}}{\pgfqpoint{5.631329in}{0.891041in}}%
\pgfpathcurveto{\pgfqpoint{5.631329in}{0.883908in}}{\pgfqpoint{5.634163in}{0.877067in}}{\pgfqpoint{5.639207in}{0.872023in}}%
\pgfpathcurveto{\pgfqpoint{5.644250in}{0.866980in}}{\pgfqpoint{5.651092in}{0.864146in}}{\pgfqpoint{5.658225in}{0.864146in}}%
\pgfpathclose%
\pgfusepath{stroke,fill}%
\end{pgfscope}%
\begin{pgfscope}%
\pgfpathrectangle{\pgfqpoint{4.985294in}{0.500000in}}{\pgfqpoint{1.764706in}{1.700000in}}%
\pgfusepath{clip}%
\pgfsetbuttcap%
\pgfsetroundjoin%
\definecolor{currentfill}{rgb}{0.962532,0.599594,0.438051}%
\pgfsetfillcolor{currentfill}%
\pgfsetlinewidth{0.311001pt}%
\definecolor{currentstroke}{rgb}{1.000000,1.000000,1.000000}%
\pgfsetstrokecolor{currentstroke}%
\pgfsetdash{}{0pt}%
\pgfpathmoveto{\pgfqpoint{6.103039in}{1.735014in}}%
\pgfpathcurveto{\pgfqpoint{6.110171in}{1.735014in}}{\pgfqpoint{6.117013in}{1.737848in}}{\pgfqpoint{6.122057in}{1.742892in}}%
\pgfpathcurveto{\pgfqpoint{6.127100in}{1.747935in}}{\pgfqpoint{6.129934in}{1.754777in}}{\pgfqpoint{6.129934in}{1.761910in}}%
\pgfpathcurveto{\pgfqpoint{6.129934in}{1.769043in}}{\pgfqpoint{6.127100in}{1.775884in}}{\pgfqpoint{6.122057in}{1.780928in}}%
\pgfpathcurveto{\pgfqpoint{6.117013in}{1.785971in}}{\pgfqpoint{6.110171in}{1.788805in}}{\pgfqpoint{6.103039in}{1.788805in}}%
\pgfpathcurveto{\pgfqpoint{6.095906in}{1.788805in}}{\pgfqpoint{6.089064in}{1.785971in}}{\pgfqpoint{6.084020in}{1.780928in}}%
\pgfpathcurveto{\pgfqpoint{6.078977in}{1.775884in}}{\pgfqpoint{6.076143in}{1.769043in}}{\pgfqpoint{6.076143in}{1.761910in}}%
\pgfpathcurveto{\pgfqpoint{6.076143in}{1.754777in}}{\pgfqpoint{6.078977in}{1.747935in}}{\pgfqpoint{6.084020in}{1.742892in}}%
\pgfpathcurveto{\pgfqpoint{6.089064in}{1.737848in}}{\pgfqpoint{6.095906in}{1.735014in}}{\pgfqpoint{6.103039in}{1.735014in}}%
\pgfpathclose%
\pgfusepath{stroke,fill}%
\end{pgfscope}%
\begin{pgfscope}%
\pgfpathrectangle{\pgfqpoint{4.985294in}{0.500000in}}{\pgfqpoint{1.764706in}{1.700000in}}%
\pgfusepath{clip}%
\pgfsetbuttcap%
\pgfsetroundjoin%
\definecolor{currentfill}{rgb}{0.977657,0.891500,0.822809}%
\pgfsetfillcolor{currentfill}%
\pgfsetlinewidth{0.311001pt}%
\definecolor{currentstroke}{rgb}{1.000000,1.000000,1.000000}%
\pgfsetstrokecolor{currentstroke}%
\pgfsetdash{}{0pt}%
\pgfpathmoveto{\pgfqpoint{6.336360in}{1.441071in}}%
\pgfpathcurveto{\pgfqpoint{6.343493in}{1.441071in}}{\pgfqpoint{6.350334in}{1.443905in}}{\pgfqpoint{6.355378in}{1.448948in}}%
\pgfpathcurveto{\pgfqpoint{6.360422in}{1.453992in}}{\pgfqpoint{6.363256in}{1.460834in}}{\pgfqpoint{6.363256in}{1.467966in}}%
\pgfpathcurveto{\pgfqpoint{6.363256in}{1.475099in}}{\pgfqpoint{6.360422in}{1.481941in}}{\pgfqpoint{6.355378in}{1.486984in}}%
\pgfpathcurveto{\pgfqpoint{6.350334in}{1.492028in}}{\pgfqpoint{6.343493in}{1.494862in}}{\pgfqpoint{6.336360in}{1.494862in}}%
\pgfpathcurveto{\pgfqpoint{6.329227in}{1.494862in}}{\pgfqpoint{6.322385in}{1.492028in}}{\pgfqpoint{6.317342in}{1.486984in}}%
\pgfpathcurveto{\pgfqpoint{6.312298in}{1.481941in}}{\pgfqpoint{6.309464in}{1.475099in}}{\pgfqpoint{6.309464in}{1.467966in}}%
\pgfpathcurveto{\pgfqpoint{6.309464in}{1.460834in}}{\pgfqpoint{6.312298in}{1.453992in}}{\pgfqpoint{6.317342in}{1.448948in}}%
\pgfpathcurveto{\pgfqpoint{6.322385in}{1.443905in}}{\pgfqpoint{6.329227in}{1.441071in}}{\pgfqpoint{6.336360in}{1.441071in}}%
\pgfpathclose%
\pgfusepath{stroke,fill}%
\end{pgfscope}%
\begin{pgfscope}%
\pgfpathrectangle{\pgfqpoint{4.985294in}{0.500000in}}{\pgfqpoint{1.764706in}{1.700000in}}%
\pgfusepath{clip}%
\pgfsetbuttcap%
\pgfsetroundjoin%
\definecolor{currentfill}{rgb}{0.971202,0.827364,0.728520}%
\pgfsetfillcolor{currentfill}%
\pgfsetlinewidth{0.311001pt}%
\definecolor{currentstroke}{rgb}{1.000000,1.000000,1.000000}%
\pgfsetstrokecolor{currentstroke}%
\pgfsetdash{}{0pt}%
\pgfpathmoveto{\pgfqpoint{5.368542in}{1.206663in}}%
\pgfpathcurveto{\pgfqpoint{5.375675in}{1.206663in}}{\pgfqpoint{5.382517in}{1.209497in}}{\pgfqpoint{5.387560in}{1.214541in}}%
\pgfpathcurveto{\pgfqpoint{5.392604in}{1.219584in}}{\pgfqpoint{5.395438in}{1.226426in}}{\pgfqpoint{5.395438in}{1.233559in}}%
\pgfpathcurveto{\pgfqpoint{5.395438in}{1.240692in}}{\pgfqpoint{5.392604in}{1.247533in}}{\pgfqpoint{5.387560in}{1.252577in}}%
\pgfpathcurveto{\pgfqpoint{5.382517in}{1.257620in}}{\pgfqpoint{5.375675in}{1.260454in}}{\pgfqpoint{5.368542in}{1.260454in}}%
\pgfpathcurveto{\pgfqpoint{5.361409in}{1.260454in}}{\pgfqpoint{5.354568in}{1.257620in}}{\pgfqpoint{5.349524in}{1.252577in}}%
\pgfpathcurveto{\pgfqpoint{5.344480in}{1.247533in}}{\pgfqpoint{5.341647in}{1.240692in}}{\pgfqpoint{5.341647in}{1.233559in}}%
\pgfpathcurveto{\pgfqpoint{5.341647in}{1.226426in}}{\pgfqpoint{5.344480in}{1.219584in}}{\pgfqpoint{5.349524in}{1.214541in}}%
\pgfpathcurveto{\pgfqpoint{5.354568in}{1.209497in}}{\pgfqpoint{5.361409in}{1.206663in}}{\pgfqpoint{5.368542in}{1.206663in}}%
\pgfpathclose%
\pgfusepath{stroke,fill}%
\end{pgfscope}%
\begin{pgfscope}%
\pgfpathrectangle{\pgfqpoint{4.985294in}{0.500000in}}{\pgfqpoint{1.764706in}{1.700000in}}%
\pgfusepath{clip}%
\pgfsetbuttcap%
\pgfsetroundjoin%
\definecolor{currentfill}{rgb}{0.965302,0.713942,0.568499}%
\pgfsetfillcolor{currentfill}%
\pgfsetlinewidth{0.311001pt}%
\definecolor{currentstroke}{rgb}{1.000000,1.000000,1.000000}%
\pgfsetstrokecolor{currentstroke}%
\pgfsetdash{}{0pt}%
\pgfpathmoveto{\pgfqpoint{6.141145in}{1.029041in}}%
\pgfpathcurveto{\pgfqpoint{6.148278in}{1.029041in}}{\pgfqpoint{6.155120in}{1.031875in}}{\pgfqpoint{6.160163in}{1.036919in}}%
\pgfpathcurveto{\pgfqpoint{6.165207in}{1.041962in}}{\pgfqpoint{6.168041in}{1.048804in}}{\pgfqpoint{6.168041in}{1.055937in}}%
\pgfpathcurveto{\pgfqpoint{6.168041in}{1.063070in}}{\pgfqpoint{6.165207in}{1.069911in}}{\pgfqpoint{6.160163in}{1.074955in}}%
\pgfpathcurveto{\pgfqpoint{6.155120in}{1.079999in}}{\pgfqpoint{6.148278in}{1.082832in}}{\pgfqpoint{6.141145in}{1.082832in}}%
\pgfpathcurveto{\pgfqpoint{6.134012in}{1.082832in}}{\pgfqpoint{6.127171in}{1.079999in}}{\pgfqpoint{6.122127in}{1.074955in}}%
\pgfpathcurveto{\pgfqpoint{6.117083in}{1.069911in}}{\pgfqpoint{6.114249in}{1.063070in}}{\pgfqpoint{6.114249in}{1.055937in}}%
\pgfpathcurveto{\pgfqpoint{6.114249in}{1.048804in}}{\pgfqpoint{6.117083in}{1.041962in}}{\pgfqpoint{6.122127in}{1.036919in}}%
\pgfpathcurveto{\pgfqpoint{6.127171in}{1.031875in}}{\pgfqpoint{6.134012in}{1.029041in}}{\pgfqpoint{6.141145in}{1.029041in}}%
\pgfpathclose%
\pgfusepath{stroke,fill}%
\end{pgfscope}%
\begin{pgfscope}%
\pgfpathrectangle{\pgfqpoint{4.985294in}{0.500000in}}{\pgfqpoint{1.764706in}{1.700000in}}%
\pgfusepath{clip}%
\pgfsetbuttcap%
\pgfsetroundjoin%
\definecolor{currentfill}{rgb}{0.970255,0.815666,0.711203}%
\pgfsetfillcolor{currentfill}%
\pgfsetlinewidth{0.311001pt}%
\definecolor{currentstroke}{rgb}{1.000000,1.000000,1.000000}%
\pgfsetstrokecolor{currentstroke}%
\pgfsetdash{}{0pt}%
\pgfpathmoveto{\pgfqpoint{5.503961in}{1.646149in}}%
\pgfpathcurveto{\pgfqpoint{5.511094in}{1.646149in}}{\pgfqpoint{5.517936in}{1.648983in}}{\pgfqpoint{5.522979in}{1.654027in}}%
\pgfpathcurveto{\pgfqpoint{5.528023in}{1.659070in}}{\pgfqpoint{5.530857in}{1.665912in}}{\pgfqpoint{5.530857in}{1.673045in}}%
\pgfpathcurveto{\pgfqpoint{5.530857in}{1.680178in}}{\pgfqpoint{5.528023in}{1.687019in}}{\pgfqpoint{5.522979in}{1.692063in}}%
\pgfpathcurveto{\pgfqpoint{5.517936in}{1.697107in}}{\pgfqpoint{5.511094in}{1.699941in}}{\pgfqpoint{5.503961in}{1.699941in}}%
\pgfpathcurveto{\pgfqpoint{5.496828in}{1.699941in}}{\pgfqpoint{5.489987in}{1.697107in}}{\pgfqpoint{5.484943in}{1.692063in}}%
\pgfpathcurveto{\pgfqpoint{5.479899in}{1.687019in}}{\pgfqpoint{5.477065in}{1.680178in}}{\pgfqpoint{5.477065in}{1.673045in}}%
\pgfpathcurveto{\pgfqpoint{5.477065in}{1.665912in}}{\pgfqpoint{5.479899in}{1.659070in}}{\pgfqpoint{5.484943in}{1.654027in}}%
\pgfpathcurveto{\pgfqpoint{5.489987in}{1.648983in}}{\pgfqpoint{5.496828in}{1.646149in}}{\pgfqpoint{5.503961in}{1.646149in}}%
\pgfpathclose%
\pgfusepath{stroke,fill}%
\end{pgfscope}%
\begin{pgfscope}%
\pgfpathrectangle{\pgfqpoint{4.985294in}{0.500000in}}{\pgfqpoint{1.764706in}{1.700000in}}%
\pgfusepath{clip}%
\pgfsetbuttcap%
\pgfsetroundjoin%
\definecolor{currentfill}{rgb}{0.962283,0.593046,0.431453}%
\pgfsetfillcolor{currentfill}%
\pgfsetlinewidth{0.311001pt}%
\definecolor{currentstroke}{rgb}{1.000000,1.000000,1.000000}%
\pgfsetstrokecolor{currentstroke}%
\pgfsetdash{}{0pt}%
\pgfpathmoveto{\pgfqpoint{6.429779in}{1.260796in}}%
\pgfpathcurveto{\pgfqpoint{6.436912in}{1.260796in}}{\pgfqpoint{6.443754in}{1.263629in}}{\pgfqpoint{6.448797in}{1.268673in}}%
\pgfpathcurveto{\pgfqpoint{6.453841in}{1.273717in}}{\pgfqpoint{6.456675in}{1.280558in}}{\pgfqpoint{6.456675in}{1.287691in}}%
\pgfpathcurveto{\pgfqpoint{6.456675in}{1.294824in}}{\pgfqpoint{6.453841in}{1.301666in}}{\pgfqpoint{6.448797in}{1.306709in}}%
\pgfpathcurveto{\pgfqpoint{6.443754in}{1.311753in}}{\pgfqpoint{6.436912in}{1.314587in}}{\pgfqpoint{6.429779in}{1.314587in}}%
\pgfpathcurveto{\pgfqpoint{6.422646in}{1.314587in}}{\pgfqpoint{6.415805in}{1.311753in}}{\pgfqpoint{6.410761in}{1.306709in}}%
\pgfpathcurveto{\pgfqpoint{6.405717in}{1.301666in}}{\pgfqpoint{6.402883in}{1.294824in}}{\pgfqpoint{6.402883in}{1.287691in}}%
\pgfpathcurveto{\pgfqpoint{6.402883in}{1.280558in}}{\pgfqpoint{6.405717in}{1.273717in}}{\pgfqpoint{6.410761in}{1.268673in}}%
\pgfpathcurveto{\pgfqpoint{6.415805in}{1.263629in}}{\pgfqpoint{6.422646in}{1.260796in}}{\pgfqpoint{6.429779in}{1.260796in}}%
\pgfpathclose%
\pgfusepath{stroke,fill}%
\end{pgfscope}%
\begin{pgfscope}%
\pgfpathrectangle{\pgfqpoint{4.985294in}{0.500000in}}{\pgfqpoint{1.764706in}{1.700000in}}%
\pgfusepath{clip}%
\pgfsetbuttcap%
\pgfsetroundjoin%
\definecolor{currentfill}{rgb}{0.963379,0.625574,0.465113}%
\pgfsetfillcolor{currentfill}%
\pgfsetlinewidth{0.311001pt}%
\definecolor{currentstroke}{rgb}{1.000000,1.000000,1.000000}%
\pgfsetstrokecolor{currentstroke}%
\pgfsetdash{}{0pt}%
\pgfpathmoveto{\pgfqpoint{6.393695in}{1.518710in}}%
\pgfpathcurveto{\pgfqpoint{6.400828in}{1.518710in}}{\pgfqpoint{6.407670in}{1.521544in}}{\pgfqpoint{6.412714in}{1.526588in}}%
\pgfpathcurveto{\pgfqpoint{6.417757in}{1.531631in}}{\pgfqpoint{6.420591in}{1.538473in}}{\pgfqpoint{6.420591in}{1.545606in}}%
\pgfpathcurveto{\pgfqpoint{6.420591in}{1.552739in}}{\pgfqpoint{6.417757in}{1.559580in}}{\pgfqpoint{6.412714in}{1.564624in}}%
\pgfpathcurveto{\pgfqpoint{6.407670in}{1.569667in}}{\pgfqpoint{6.400828in}{1.572501in}}{\pgfqpoint{6.393695in}{1.572501in}}%
\pgfpathcurveto{\pgfqpoint{6.386563in}{1.572501in}}{\pgfqpoint{6.379721in}{1.569667in}}{\pgfqpoint{6.374677in}{1.564624in}}%
\pgfpathcurveto{\pgfqpoint{6.369634in}{1.559580in}}{\pgfqpoint{6.366800in}{1.552739in}}{\pgfqpoint{6.366800in}{1.545606in}}%
\pgfpathcurveto{\pgfqpoint{6.366800in}{1.538473in}}{\pgfqpoint{6.369634in}{1.531631in}}{\pgfqpoint{6.374677in}{1.526588in}}%
\pgfpathcurveto{\pgfqpoint{6.379721in}{1.521544in}}{\pgfqpoint{6.386563in}{1.518710in}}{\pgfqpoint{6.393695in}{1.518710in}}%
\pgfpathclose%
\pgfusepath{stroke,fill}%
\end{pgfscope}%
\begin{pgfscope}%
\pgfpathrectangle{\pgfqpoint{4.985294in}{0.500000in}}{\pgfqpoint{1.764706in}{1.700000in}}%
\pgfusepath{clip}%
\pgfsetbuttcap%
\pgfsetroundjoin%
\definecolor{currentfill}{rgb}{0.978376,0.897317,0.831308}%
\pgfsetfillcolor{currentfill}%
\pgfsetlinewidth{0.311001pt}%
\definecolor{currentstroke}{rgb}{1.000000,1.000000,1.000000}%
\pgfsetstrokecolor{currentstroke}%
\pgfsetdash{}{0pt}%
\pgfpathmoveto{\pgfqpoint{6.267927in}{1.141812in}}%
\pgfpathcurveto{\pgfqpoint{6.275060in}{1.141812in}}{\pgfqpoint{6.281902in}{1.144646in}}{\pgfqpoint{6.286945in}{1.149690in}}%
\pgfpathcurveto{\pgfqpoint{6.291989in}{1.154733in}}{\pgfqpoint{6.294823in}{1.161575in}}{\pgfqpoint{6.294823in}{1.168708in}}%
\pgfpathcurveto{\pgfqpoint{6.294823in}{1.175841in}}{\pgfqpoint{6.291989in}{1.182682in}}{\pgfqpoint{6.286945in}{1.187726in}}%
\pgfpathcurveto{\pgfqpoint{6.281902in}{1.192770in}}{\pgfqpoint{6.275060in}{1.195604in}}{\pgfqpoint{6.267927in}{1.195604in}}%
\pgfpathcurveto{\pgfqpoint{6.260794in}{1.195604in}}{\pgfqpoint{6.253953in}{1.192770in}}{\pgfqpoint{6.248909in}{1.187726in}}%
\pgfpathcurveto{\pgfqpoint{6.243865in}{1.182682in}}{\pgfqpoint{6.241032in}{1.175841in}}{\pgfqpoint{6.241032in}{1.168708in}}%
\pgfpathcurveto{\pgfqpoint{6.241032in}{1.161575in}}{\pgfqpoint{6.243865in}{1.154733in}}{\pgfqpoint{6.248909in}{1.149690in}}%
\pgfpathcurveto{\pgfqpoint{6.253953in}{1.144646in}}{\pgfqpoint{6.260794in}{1.141812in}}{\pgfqpoint{6.267927in}{1.141812in}}%
\pgfpathclose%
\pgfusepath{stroke,fill}%
\end{pgfscope}%
\begin{pgfscope}%
\pgfpathrectangle{\pgfqpoint{4.985294in}{0.500000in}}{\pgfqpoint{1.764706in}{1.700000in}}%
\pgfusepath{clip}%
\pgfsetbuttcap%
\pgfsetroundjoin%
\definecolor{currentfill}{rgb}{0.950017,0.427714,0.292447}%
\pgfsetfillcolor{currentfill}%
\pgfsetlinewidth{0.311001pt}%
\definecolor{currentstroke}{rgb}{1.000000,1.000000,1.000000}%
\pgfsetstrokecolor{currentstroke}%
\pgfsetdash{}{0pt}%
\pgfpathmoveto{\pgfqpoint{5.634799in}{1.084000in}}%
\pgfpathcurveto{\pgfqpoint{5.641931in}{1.084000in}}{\pgfqpoint{5.648773in}{1.086834in}}{\pgfqpoint{5.653817in}{1.091878in}}%
\pgfpathcurveto{\pgfqpoint{5.658860in}{1.096921in}}{\pgfqpoint{5.661694in}{1.103763in}}{\pgfqpoint{5.661694in}{1.110896in}}%
\pgfpathcurveto{\pgfqpoint{5.661694in}{1.118029in}}{\pgfqpoint{5.658860in}{1.124870in}}{\pgfqpoint{5.653817in}{1.129914in}}%
\pgfpathcurveto{\pgfqpoint{5.648773in}{1.134958in}}{\pgfqpoint{5.641931in}{1.137792in}}{\pgfqpoint{5.634799in}{1.137792in}}%
\pgfpathcurveto{\pgfqpoint{5.627666in}{1.137792in}}{\pgfqpoint{5.620824in}{1.134958in}}{\pgfqpoint{5.615781in}{1.129914in}}%
\pgfpathcurveto{\pgfqpoint{5.610737in}{1.124870in}}{\pgfqpoint{5.607903in}{1.118029in}}{\pgfqpoint{5.607903in}{1.110896in}}%
\pgfpathcurveto{\pgfqpoint{5.607903in}{1.103763in}}{\pgfqpoint{5.610737in}{1.096921in}}{\pgfqpoint{5.615781in}{1.091878in}}%
\pgfpathcurveto{\pgfqpoint{5.620824in}{1.086834in}}{\pgfqpoint{5.627666in}{1.084000in}}{\pgfqpoint{5.634799in}{1.084000in}}%
\pgfpathclose%
\pgfusepath{stroke,fill}%
\end{pgfscope}%
\begin{pgfscope}%
\pgfpathrectangle{\pgfqpoint{4.985294in}{0.500000in}}{\pgfqpoint{1.764706in}{1.700000in}}%
\pgfusepath{clip}%
\pgfsetbuttcap%
\pgfsetroundjoin%
\definecolor{currentfill}{rgb}{0.968105,0.786346,0.667739}%
\pgfsetfillcolor{currentfill}%
\pgfsetlinewidth{0.311001pt}%
\definecolor{currentstroke}{rgb}{1.000000,1.000000,1.000000}%
\pgfsetstrokecolor{currentstroke}%
\pgfsetdash{}{0pt}%
\pgfpathmoveto{\pgfqpoint{5.351558in}{1.218765in}}%
\pgfpathcurveto{\pgfqpoint{5.358691in}{1.218765in}}{\pgfqpoint{5.365533in}{1.221599in}}{\pgfqpoint{5.370577in}{1.226643in}}%
\pgfpathcurveto{\pgfqpoint{5.375620in}{1.231687in}}{\pgfqpoint{5.378454in}{1.238528in}}{\pgfqpoint{5.378454in}{1.245661in}}%
\pgfpathcurveto{\pgfqpoint{5.378454in}{1.252794in}}{\pgfqpoint{5.375620in}{1.259636in}}{\pgfqpoint{5.370577in}{1.264679in}}%
\pgfpathcurveto{\pgfqpoint{5.365533in}{1.269723in}}{\pgfqpoint{5.358691in}{1.272557in}}{\pgfqpoint{5.351558in}{1.272557in}}%
\pgfpathcurveto{\pgfqpoint{5.344426in}{1.272557in}}{\pgfqpoint{5.337584in}{1.269723in}}{\pgfqpoint{5.332540in}{1.264679in}}%
\pgfpathcurveto{\pgfqpoint{5.327497in}{1.259636in}}{\pgfqpoint{5.324663in}{1.252794in}}{\pgfqpoint{5.324663in}{1.245661in}}%
\pgfpathcurveto{\pgfqpoint{5.324663in}{1.238528in}}{\pgfqpoint{5.327497in}{1.231687in}}{\pgfqpoint{5.332540in}{1.226643in}}%
\pgfpathcurveto{\pgfqpoint{5.337584in}{1.221599in}}{\pgfqpoint{5.344426in}{1.218765in}}{\pgfqpoint{5.351558in}{1.218765in}}%
\pgfpathclose%
\pgfusepath{stroke,fill}%
\end{pgfscope}%
\begin{pgfscope}%
\pgfpathrectangle{\pgfqpoint{4.985294in}{0.500000in}}{\pgfqpoint{1.764706in}{1.700000in}}%
\pgfusepath{clip}%
\pgfsetbuttcap%
\pgfsetroundjoin%
\definecolor{currentfill}{rgb}{0.973832,0.856556,0.771584}%
\pgfsetfillcolor{currentfill}%
\pgfsetlinewidth{0.311001pt}%
\definecolor{currentstroke}{rgb}{1.000000,1.000000,1.000000}%
\pgfsetstrokecolor{currentstroke}%
\pgfsetdash{}{0pt}%
\pgfpathmoveto{\pgfqpoint{5.428173in}{1.544799in}}%
\pgfpathcurveto{\pgfqpoint{5.435306in}{1.544799in}}{\pgfqpoint{5.442148in}{1.547633in}}{\pgfqpoint{5.447191in}{1.552676in}}%
\pgfpathcurveto{\pgfqpoint{5.452235in}{1.557720in}}{\pgfqpoint{5.455069in}{1.564561in}}{\pgfqpoint{5.455069in}{1.571694in}}%
\pgfpathcurveto{\pgfqpoint{5.455069in}{1.578827in}}{\pgfqpoint{5.452235in}{1.585669in}}{\pgfqpoint{5.447191in}{1.590712in}}%
\pgfpathcurveto{\pgfqpoint{5.442148in}{1.595756in}}{\pgfqpoint{5.435306in}{1.598590in}}{\pgfqpoint{5.428173in}{1.598590in}}%
\pgfpathcurveto{\pgfqpoint{5.421040in}{1.598590in}}{\pgfqpoint{5.414199in}{1.595756in}}{\pgfqpoint{5.409155in}{1.590712in}}%
\pgfpathcurveto{\pgfqpoint{5.404111in}{1.585669in}}{\pgfqpoint{5.401278in}{1.578827in}}{\pgfqpoint{5.401278in}{1.571694in}}%
\pgfpathcurveto{\pgfqpoint{5.401278in}{1.564561in}}{\pgfqpoint{5.404111in}{1.557720in}}{\pgfqpoint{5.409155in}{1.552676in}}%
\pgfpathcurveto{\pgfqpoint{5.414199in}{1.547633in}}{\pgfqpoint{5.421040in}{1.544799in}}{\pgfqpoint{5.428173in}{1.544799in}}%
\pgfpathclose%
\pgfusepath{stroke,fill}%
\end{pgfscope}%
\begin{pgfscope}%
\pgfpathrectangle{\pgfqpoint{4.985294in}{0.500000in}}{\pgfqpoint{1.764706in}{1.700000in}}%
\pgfusepath{clip}%
\pgfsetbuttcap%
\pgfsetroundjoin%
\definecolor{currentfill}{rgb}{0.965169,0.707764,0.560659}%
\pgfsetfillcolor{currentfill}%
\pgfsetlinewidth{0.311001pt}%
\definecolor{currentstroke}{rgb}{1.000000,1.000000,1.000000}%
\pgfsetstrokecolor{currentstroke}%
\pgfsetdash{}{0pt}%
\pgfpathmoveto{\pgfqpoint{5.341351in}{1.456872in}}%
\pgfpathcurveto{\pgfqpoint{5.348484in}{1.456872in}}{\pgfqpoint{5.355325in}{1.459706in}}{\pgfqpoint{5.360369in}{1.464749in}}%
\pgfpathcurveto{\pgfqpoint{5.365413in}{1.469793in}}{\pgfqpoint{5.368247in}{1.476635in}}{\pgfqpoint{5.368247in}{1.483768in}}%
\pgfpathcurveto{\pgfqpoint{5.368247in}{1.490900in}}{\pgfqpoint{5.365413in}{1.497742in}}{\pgfqpoint{5.360369in}{1.502786in}}%
\pgfpathcurveto{\pgfqpoint{5.355325in}{1.507829in}}{\pgfqpoint{5.348484in}{1.510663in}}{\pgfqpoint{5.341351in}{1.510663in}}%
\pgfpathcurveto{\pgfqpoint{5.334218in}{1.510663in}}{\pgfqpoint{5.327376in}{1.507829in}}{\pgfqpoint{5.322333in}{1.502786in}}%
\pgfpathcurveto{\pgfqpoint{5.317289in}{1.497742in}}{\pgfqpoint{5.314455in}{1.490900in}}{\pgfqpoint{5.314455in}{1.483768in}}%
\pgfpathcurveto{\pgfqpoint{5.314455in}{1.476635in}}{\pgfqpoint{5.317289in}{1.469793in}}{\pgfqpoint{5.322333in}{1.464749in}}%
\pgfpathcurveto{\pgfqpoint{5.327376in}{1.459706in}}{\pgfqpoint{5.334218in}{1.456872in}}{\pgfqpoint{5.341351in}{1.456872in}}%
\pgfpathclose%
\pgfusepath{stroke,fill}%
\end{pgfscope}%
\begin{pgfscope}%
\pgfpathrectangle{\pgfqpoint{4.985294in}{0.500000in}}{\pgfqpoint{1.764706in}{1.700000in}}%
\pgfusepath{clip}%
\pgfsetbuttcap%
\pgfsetroundjoin%
\definecolor{currentfill}{rgb}{0.964920,0.695342,0.545192}%
\pgfsetfillcolor{currentfill}%
\pgfsetlinewidth{0.311001pt}%
\definecolor{currentstroke}{rgb}{1.000000,1.000000,1.000000}%
\pgfsetstrokecolor{currentstroke}%
\pgfsetdash{}{0pt}%
\pgfpathmoveto{\pgfqpoint{6.393511in}{1.463666in}}%
\pgfpathcurveto{\pgfqpoint{6.400643in}{1.463666in}}{\pgfqpoint{6.407485in}{1.466500in}}{\pgfqpoint{6.412529in}{1.471544in}}%
\pgfpathcurveto{\pgfqpoint{6.417572in}{1.476587in}}{\pgfqpoint{6.420406in}{1.483429in}}{\pgfqpoint{6.420406in}{1.490562in}}%
\pgfpathcurveto{\pgfqpoint{6.420406in}{1.497695in}}{\pgfqpoint{6.417572in}{1.504536in}}{\pgfqpoint{6.412529in}{1.509580in}}%
\pgfpathcurveto{\pgfqpoint{6.407485in}{1.514624in}}{\pgfqpoint{6.400643in}{1.517457in}}{\pgfqpoint{6.393511in}{1.517457in}}%
\pgfpathcurveto{\pgfqpoint{6.386378in}{1.517457in}}{\pgfqpoint{6.379536in}{1.514624in}}{\pgfqpoint{6.374493in}{1.509580in}}%
\pgfpathcurveto{\pgfqpoint{6.369449in}{1.504536in}}{\pgfqpoint{6.366615in}{1.497695in}}{\pgfqpoint{6.366615in}{1.490562in}}%
\pgfpathcurveto{\pgfqpoint{6.366615in}{1.483429in}}{\pgfqpoint{6.369449in}{1.476587in}}{\pgfqpoint{6.374493in}{1.471544in}}%
\pgfpathcurveto{\pgfqpoint{6.379536in}{1.466500in}}{\pgfqpoint{6.386378in}{1.463666in}}{\pgfqpoint{6.393511in}{1.463666in}}%
\pgfpathclose%
\pgfusepath{stroke,fill}%
\end{pgfscope}%
\begin{pgfscope}%
\pgfpathrectangle{\pgfqpoint{4.985294in}{0.500000in}}{\pgfqpoint{1.764706in}{1.700000in}}%
\pgfusepath{clip}%
\pgfsetbuttcap%
\pgfsetroundjoin%
\definecolor{currentfill}{rgb}{0.962765,0.606121,0.444717}%
\pgfsetfillcolor{currentfill}%
\pgfsetlinewidth{0.311001pt}%
\definecolor{currentstroke}{rgb}{1.000000,1.000000,1.000000}%
\pgfsetstrokecolor{currentstroke}%
\pgfsetdash{}{0pt}%
\pgfpathmoveto{\pgfqpoint{5.524344in}{0.871338in}}%
\pgfpathcurveto{\pgfqpoint{5.531477in}{0.871338in}}{\pgfqpoint{5.538319in}{0.874172in}}{\pgfqpoint{5.543362in}{0.879215in}}%
\pgfpathcurveto{\pgfqpoint{5.548406in}{0.884259in}}{\pgfqpoint{5.551240in}{0.891101in}}{\pgfqpoint{5.551240in}{0.898233in}}%
\pgfpathcurveto{\pgfqpoint{5.551240in}{0.905366in}}{\pgfqpoint{5.548406in}{0.912208in}}{\pgfqpoint{5.543362in}{0.917252in}}%
\pgfpathcurveto{\pgfqpoint{5.538319in}{0.922295in}}{\pgfqpoint{5.531477in}{0.925129in}}{\pgfqpoint{5.524344in}{0.925129in}}%
\pgfpathcurveto{\pgfqpoint{5.517211in}{0.925129in}}{\pgfqpoint{5.510370in}{0.922295in}}{\pgfqpoint{5.505326in}{0.917252in}}%
\pgfpathcurveto{\pgfqpoint{5.500282in}{0.912208in}}{\pgfqpoint{5.497448in}{0.905366in}}{\pgfqpoint{5.497448in}{0.898233in}}%
\pgfpathcurveto{\pgfqpoint{5.497448in}{0.891101in}}{\pgfqpoint{5.500282in}{0.884259in}}{\pgfqpoint{5.505326in}{0.879215in}}%
\pgfpathcurveto{\pgfqpoint{5.510370in}{0.874172in}}{\pgfqpoint{5.517211in}{0.871338in}}{\pgfqpoint{5.524344in}{0.871338in}}%
\pgfpathclose%
\pgfusepath{stroke,fill}%
\end{pgfscope}%
\begin{pgfscope}%
\pgfpathrectangle{\pgfqpoint{4.985294in}{0.500000in}}{\pgfqpoint{1.764706in}{1.700000in}}%
\pgfusepath{clip}%
\pgfsetbuttcap%
\pgfsetroundjoin%
\definecolor{currentfill}{rgb}{0.973271,0.850724,0.762998}%
\pgfsetfillcolor{currentfill}%
\pgfsetlinewidth{0.311001pt}%
\definecolor{currentstroke}{rgb}{1.000000,1.000000,1.000000}%
\pgfsetstrokecolor{currentstroke}%
\pgfsetdash{}{0pt}%
\pgfpathmoveto{\pgfqpoint{6.235864in}{1.637927in}}%
\pgfpathcurveto{\pgfqpoint{6.242997in}{1.637927in}}{\pgfqpoint{6.249838in}{1.640761in}}{\pgfqpoint{6.254882in}{1.645805in}}%
\pgfpathcurveto{\pgfqpoint{6.259926in}{1.650849in}}{\pgfqpoint{6.262759in}{1.657690in}}{\pgfqpoint{6.262759in}{1.664823in}}%
\pgfpathcurveto{\pgfqpoint{6.262759in}{1.671956in}}{\pgfqpoint{6.259926in}{1.678797in}}{\pgfqpoint{6.254882in}{1.683841in}}%
\pgfpathcurveto{\pgfqpoint{6.249838in}{1.688885in}}{\pgfqpoint{6.242997in}{1.691719in}}{\pgfqpoint{6.235864in}{1.691719in}}%
\pgfpathcurveto{\pgfqpoint{6.228731in}{1.691719in}}{\pgfqpoint{6.221889in}{1.688885in}}{\pgfqpoint{6.216846in}{1.683841in}}%
\pgfpathcurveto{\pgfqpoint{6.211802in}{1.678797in}}{\pgfqpoint{6.208968in}{1.671956in}}{\pgfqpoint{6.208968in}{1.664823in}}%
\pgfpathcurveto{\pgfqpoint{6.208968in}{1.657690in}}{\pgfqpoint{6.211802in}{1.650849in}}{\pgfqpoint{6.216846in}{1.645805in}}%
\pgfpathcurveto{\pgfqpoint{6.221889in}{1.640761in}}{\pgfqpoint{6.228731in}{1.637927in}}{\pgfqpoint{6.235864in}{1.637927in}}%
\pgfpathclose%
\pgfusepath{stroke,fill}%
\end{pgfscope}%
\begin{pgfscope}%
\pgfpathrectangle{\pgfqpoint{4.985294in}{0.500000in}}{\pgfqpoint{1.764706in}{1.700000in}}%
\pgfusepath{clip}%
\pgfsetbuttcap%
\pgfsetroundjoin%
\definecolor{currentfill}{rgb}{0.972201,0.839051,0.745789}%
\pgfsetfillcolor{currentfill}%
\pgfsetlinewidth{0.311001pt}%
\definecolor{currentstroke}{rgb}{1.000000,1.000000,1.000000}%
\pgfsetstrokecolor{currentstroke}%
\pgfsetdash{}{0pt}%
\pgfpathmoveto{\pgfqpoint{5.514591in}{0.996768in}}%
\pgfpathcurveto{\pgfqpoint{5.521723in}{0.996768in}}{\pgfqpoint{5.528565in}{0.999602in}}{\pgfqpoint{5.533609in}{1.004645in}}%
\pgfpathcurveto{\pgfqpoint{5.538652in}{1.009689in}}{\pgfqpoint{5.541486in}{1.016531in}}{\pgfqpoint{5.541486in}{1.023664in}}%
\pgfpathcurveto{\pgfqpoint{5.541486in}{1.030796in}}{\pgfqpoint{5.538652in}{1.037638in}}{\pgfqpoint{5.533609in}{1.042682in}}%
\pgfpathcurveto{\pgfqpoint{5.528565in}{1.047725in}}{\pgfqpoint{5.521723in}{1.050559in}}{\pgfqpoint{5.514591in}{1.050559in}}%
\pgfpathcurveto{\pgfqpoint{5.507458in}{1.050559in}}{\pgfqpoint{5.500616in}{1.047725in}}{\pgfqpoint{5.495573in}{1.042682in}}%
\pgfpathcurveto{\pgfqpoint{5.490529in}{1.037638in}}{\pgfqpoint{5.487695in}{1.030796in}}{\pgfqpoint{5.487695in}{1.023664in}}%
\pgfpathcurveto{\pgfqpoint{5.487695in}{1.016531in}}{\pgfqpoint{5.490529in}{1.009689in}}{\pgfqpoint{5.495573in}{1.004645in}}%
\pgfpathcurveto{\pgfqpoint{5.500616in}{0.999602in}}{\pgfqpoint{5.507458in}{0.996768in}}{\pgfqpoint{5.514591in}{0.996768in}}%
\pgfpathclose%
\pgfusepath{stroke,fill}%
\end{pgfscope}%
\begin{pgfscope}%
\pgfpathrectangle{\pgfqpoint{4.985294in}{0.500000in}}{\pgfqpoint{1.764706in}{1.700000in}}%
\pgfusepath{clip}%
\pgfsetbuttcap%
\pgfsetroundjoin%
\definecolor{currentfill}{rgb}{0.953126,0.456614,0.312398}%
\pgfsetfillcolor{currentfill}%
\pgfsetlinewidth{0.311001pt}%
\definecolor{currentstroke}{rgb}{1.000000,1.000000,1.000000}%
\pgfsetstrokecolor{currentstroke}%
\pgfsetdash{}{0pt}%
\pgfpathmoveto{\pgfqpoint{5.646436in}{1.607105in}}%
\pgfpathcurveto{\pgfqpoint{5.653568in}{1.607105in}}{\pgfqpoint{5.660410in}{1.609939in}}{\pgfqpoint{5.665454in}{1.614983in}}%
\pgfpathcurveto{\pgfqpoint{5.670497in}{1.620027in}}{\pgfqpoint{5.673331in}{1.626868in}}{\pgfqpoint{5.673331in}{1.634001in}}%
\pgfpathcurveto{\pgfqpoint{5.673331in}{1.641134in}}{\pgfqpoint{5.670497in}{1.647976in}}{\pgfqpoint{5.665454in}{1.653019in}}%
\pgfpathcurveto{\pgfqpoint{5.660410in}{1.658063in}}{\pgfqpoint{5.653568in}{1.660897in}}{\pgfqpoint{5.646436in}{1.660897in}}%
\pgfpathcurveto{\pgfqpoint{5.639303in}{1.660897in}}{\pgfqpoint{5.632461in}{1.658063in}}{\pgfqpoint{5.627417in}{1.653019in}}%
\pgfpathcurveto{\pgfqpoint{5.622374in}{1.647976in}}{\pgfqpoint{5.619540in}{1.641134in}}{\pgfqpoint{5.619540in}{1.634001in}}%
\pgfpathcurveto{\pgfqpoint{5.619540in}{1.626868in}}{\pgfqpoint{5.622374in}{1.620027in}}{\pgfqpoint{5.627417in}{1.614983in}}%
\pgfpathcurveto{\pgfqpoint{5.632461in}{1.609939in}}{\pgfqpoint{5.639303in}{1.607105in}}{\pgfqpoint{5.646436in}{1.607105in}}%
\pgfpathclose%
\pgfusepath{stroke,fill}%
\end{pgfscope}%
\begin{pgfscope}%
\pgfpathrectangle{\pgfqpoint{4.985294in}{0.500000in}}{\pgfqpoint{1.764706in}{1.700000in}}%
\pgfusepath{clip}%
\pgfsetbuttcap%
\pgfsetroundjoin%
\definecolor{currentfill}{rgb}{0.969359,0.803954,0.693832}%
\pgfsetfillcolor{currentfill}%
\pgfsetlinewidth{0.311001pt}%
\definecolor{currentstroke}{rgb}{1.000000,1.000000,1.000000}%
\pgfsetstrokecolor{currentstroke}%
\pgfsetdash{}{0pt}%
\pgfpathmoveto{\pgfqpoint{5.405450in}{1.558870in}}%
\pgfpathcurveto{\pgfqpoint{5.412583in}{1.558870in}}{\pgfqpoint{5.419425in}{1.561704in}}{\pgfqpoint{5.424468in}{1.566748in}}%
\pgfpathcurveto{\pgfqpoint{5.429512in}{1.571792in}}{\pgfqpoint{5.432346in}{1.578633in}}{\pgfqpoint{5.432346in}{1.585766in}}%
\pgfpathcurveto{\pgfqpoint{5.432346in}{1.592899in}}{\pgfqpoint{5.429512in}{1.599740in}}{\pgfqpoint{5.424468in}{1.604784in}}%
\pgfpathcurveto{\pgfqpoint{5.419425in}{1.609828in}}{\pgfqpoint{5.412583in}{1.612662in}}{\pgfqpoint{5.405450in}{1.612662in}}%
\pgfpathcurveto{\pgfqpoint{5.398317in}{1.612662in}}{\pgfqpoint{5.391476in}{1.609828in}}{\pgfqpoint{5.386432in}{1.604784in}}%
\pgfpathcurveto{\pgfqpoint{5.381388in}{1.599740in}}{\pgfqpoint{5.378554in}{1.592899in}}{\pgfqpoint{5.378554in}{1.585766in}}%
\pgfpathcurveto{\pgfqpoint{5.378554in}{1.578633in}}{\pgfqpoint{5.381388in}{1.571792in}}{\pgfqpoint{5.386432in}{1.566748in}}%
\pgfpathcurveto{\pgfqpoint{5.391476in}{1.561704in}}{\pgfqpoint{5.398317in}{1.558870in}}{\pgfqpoint{5.405450in}{1.558870in}}%
\pgfpathclose%
\pgfusepath{stroke,fill}%
\end{pgfscope}%
\begin{pgfscope}%
\pgfpathrectangle{\pgfqpoint{4.985294in}{0.500000in}}{\pgfqpoint{1.764706in}{1.700000in}}%
\pgfusepath{clip}%
\pgfsetbuttcap%
\pgfsetroundjoin%
\definecolor{currentfill}{rgb}{0.965042,0.701564,0.552889}%
\pgfsetfillcolor{currentfill}%
\pgfsetlinewidth{0.311001pt}%
\definecolor{currentstroke}{rgb}{1.000000,1.000000,1.000000}%
\pgfsetstrokecolor{currentstroke}%
\pgfsetdash{}{0pt}%
\pgfpathmoveto{\pgfqpoint{5.425801in}{1.655171in}}%
\pgfpathcurveto{\pgfqpoint{5.432934in}{1.655171in}}{\pgfqpoint{5.439776in}{1.658005in}}{\pgfqpoint{5.444820in}{1.663049in}}%
\pgfpathcurveto{\pgfqpoint{5.449863in}{1.668093in}}{\pgfqpoint{5.452697in}{1.674934in}}{\pgfqpoint{5.452697in}{1.682067in}}%
\pgfpathcurveto{\pgfqpoint{5.452697in}{1.689200in}}{\pgfqpoint{5.449863in}{1.696041in}}{\pgfqpoint{5.444820in}{1.701085in}}%
\pgfpathcurveto{\pgfqpoint{5.439776in}{1.706129in}}{\pgfqpoint{5.432934in}{1.708963in}}{\pgfqpoint{5.425801in}{1.708963in}}%
\pgfpathcurveto{\pgfqpoint{5.418669in}{1.708963in}}{\pgfqpoint{5.411827in}{1.706129in}}{\pgfqpoint{5.406783in}{1.701085in}}%
\pgfpathcurveto{\pgfqpoint{5.401740in}{1.696041in}}{\pgfqpoint{5.398906in}{1.689200in}}{\pgfqpoint{5.398906in}{1.682067in}}%
\pgfpathcurveto{\pgfqpoint{5.398906in}{1.674934in}}{\pgfqpoint{5.401740in}{1.668093in}}{\pgfqpoint{5.406783in}{1.663049in}}%
\pgfpathcurveto{\pgfqpoint{5.411827in}{1.658005in}}{\pgfqpoint{5.418669in}{1.655171in}}{\pgfqpoint{5.425801in}{1.655171in}}%
\pgfpathclose%
\pgfusepath{stroke,fill}%
\end{pgfscope}%
\begin{pgfscope}%
\pgfpathrectangle{\pgfqpoint{4.985294in}{0.500000in}}{\pgfqpoint{1.764706in}{1.700000in}}%
\pgfusepath{clip}%
\pgfsetbuttcap%
\pgfsetroundjoin%
\definecolor{currentfill}{rgb}{0.972201,0.839051,0.745789}%
\pgfsetfillcolor{currentfill}%
\pgfsetlinewidth{0.311001pt}%
\definecolor{currentstroke}{rgb}{1.000000,1.000000,1.000000}%
\pgfsetstrokecolor{currentstroke}%
\pgfsetdash{}{0pt}%
\pgfpathmoveto{\pgfqpoint{6.374039in}{1.356716in}}%
\pgfpathcurveto{\pgfqpoint{6.381172in}{1.356716in}}{\pgfqpoint{6.388014in}{1.359549in}}{\pgfqpoint{6.393057in}{1.364593in}}%
\pgfpathcurveto{\pgfqpoint{6.398101in}{1.369637in}}{\pgfqpoint{6.400935in}{1.376478in}}{\pgfqpoint{6.400935in}{1.383611in}}%
\pgfpathcurveto{\pgfqpoint{6.400935in}{1.390744in}}{\pgfqpoint{6.398101in}{1.397586in}}{\pgfqpoint{6.393057in}{1.402629in}}%
\pgfpathcurveto{\pgfqpoint{6.388014in}{1.407673in}}{\pgfqpoint{6.381172in}{1.410507in}}{\pgfqpoint{6.374039in}{1.410507in}}%
\pgfpathcurveto{\pgfqpoint{6.366906in}{1.410507in}}{\pgfqpoint{6.360065in}{1.407673in}}{\pgfqpoint{6.355021in}{1.402629in}}%
\pgfpathcurveto{\pgfqpoint{6.349977in}{1.397586in}}{\pgfqpoint{6.347143in}{1.390744in}}{\pgfqpoint{6.347143in}{1.383611in}}%
\pgfpathcurveto{\pgfqpoint{6.347143in}{1.376478in}}{\pgfqpoint{6.349977in}{1.369637in}}{\pgfqpoint{6.355021in}{1.364593in}}%
\pgfpathcurveto{\pgfqpoint{6.360065in}{1.359549in}}{\pgfqpoint{6.366906in}{1.356716in}}{\pgfqpoint{6.374039in}{1.356716in}}%
\pgfpathclose%
\pgfusepath{stroke,fill}%
\end{pgfscope}%
\begin{pgfscope}%
\pgfpathrectangle{\pgfqpoint{4.985294in}{0.500000in}}{\pgfqpoint{1.764706in}{1.700000in}}%
\pgfusepath{clip}%
\pgfsetbuttcap%
\pgfsetroundjoin%
\definecolor{currentfill}{rgb}{0.975018,0.868213,0.788710}%
\pgfsetfillcolor{currentfill}%
\pgfsetlinewidth{0.311001pt}%
\definecolor{currentstroke}{rgb}{1.000000,1.000000,1.000000}%
\pgfsetstrokecolor{currentstroke}%
\pgfsetdash{}{0pt}%
\pgfpathmoveto{\pgfqpoint{6.233116in}{1.095468in}}%
\pgfpathcurveto{\pgfqpoint{6.240248in}{1.095468in}}{\pgfqpoint{6.247090in}{1.098302in}}{\pgfqpoint{6.252134in}{1.103345in}}%
\pgfpathcurveto{\pgfqpoint{6.257177in}{1.108389in}}{\pgfqpoint{6.260011in}{1.115231in}}{\pgfqpoint{6.260011in}{1.122363in}}%
\pgfpathcurveto{\pgfqpoint{6.260011in}{1.129496in}}{\pgfqpoint{6.257177in}{1.136338in}}{\pgfqpoint{6.252134in}{1.141382in}}%
\pgfpathcurveto{\pgfqpoint{6.247090in}{1.146425in}}{\pgfqpoint{6.240248in}{1.149259in}}{\pgfqpoint{6.233116in}{1.149259in}}%
\pgfpathcurveto{\pgfqpoint{6.225983in}{1.149259in}}{\pgfqpoint{6.219141in}{1.146425in}}{\pgfqpoint{6.214097in}{1.141382in}}%
\pgfpathcurveto{\pgfqpoint{6.209054in}{1.136338in}}{\pgfqpoint{6.206220in}{1.129496in}}{\pgfqpoint{6.206220in}{1.122363in}}%
\pgfpathcurveto{\pgfqpoint{6.206220in}{1.115231in}}{\pgfqpoint{6.209054in}{1.108389in}}{\pgfqpoint{6.214097in}{1.103345in}}%
\pgfpathcurveto{\pgfqpoint{6.219141in}{1.098302in}}{\pgfqpoint{6.225983in}{1.095468in}}{\pgfqpoint{6.233116in}{1.095468in}}%
\pgfpathclose%
\pgfusepath{stroke,fill}%
\end{pgfscope}%
\begin{pgfscope}%
\pgfpathrectangle{\pgfqpoint{4.985294in}{0.500000in}}{\pgfqpoint{1.764706in}{1.700000in}}%
\pgfusepath{clip}%
\pgfsetbuttcap%
\pgfsetroundjoin%
\definecolor{currentfill}{rgb}{0.730358,0.086862,0.337485}%
\pgfsetfillcolor{currentfill}%
\pgfsetlinewidth{0.311001pt}%
\definecolor{currentstroke}{rgb}{1.000000,1.000000,1.000000}%
\pgfsetstrokecolor{currentstroke}%
\pgfsetdash{}{0pt}%
\pgfpathmoveto{\pgfqpoint{6.099105in}{1.245361in}}%
\pgfpathcurveto{\pgfqpoint{6.106238in}{1.245361in}}{\pgfqpoint{6.113080in}{1.248195in}}{\pgfqpoint{6.118124in}{1.253239in}}%
\pgfpathcurveto{\pgfqpoint{6.123167in}{1.258283in}}{\pgfqpoint{6.126001in}{1.265124in}}{\pgfqpoint{6.126001in}{1.272257in}}%
\pgfpathcurveto{\pgfqpoint{6.126001in}{1.279390in}}{\pgfqpoint{6.123167in}{1.286232in}}{\pgfqpoint{6.118124in}{1.291275in}}%
\pgfpathcurveto{\pgfqpoint{6.113080in}{1.296319in}}{\pgfqpoint{6.106238in}{1.299153in}}{\pgfqpoint{6.099105in}{1.299153in}}%
\pgfpathcurveto{\pgfqpoint{6.091973in}{1.299153in}}{\pgfqpoint{6.085131in}{1.296319in}}{\pgfqpoint{6.080087in}{1.291275in}}%
\pgfpathcurveto{\pgfqpoint{6.075044in}{1.286232in}}{\pgfqpoint{6.072210in}{1.279390in}}{\pgfqpoint{6.072210in}{1.272257in}}%
\pgfpathcurveto{\pgfqpoint{6.072210in}{1.265124in}}{\pgfqpoint{6.075044in}{1.258283in}}{\pgfqpoint{6.080087in}{1.253239in}}%
\pgfpathcurveto{\pgfqpoint{6.085131in}{1.248195in}}{\pgfqpoint{6.091973in}{1.245361in}}{\pgfqpoint{6.099105in}{1.245361in}}%
\pgfpathclose%
\pgfusepath{stroke,fill}%
\end{pgfscope}%
\begin{pgfscope}%
\pgfpathrectangle{\pgfqpoint{4.985294in}{0.500000in}}{\pgfqpoint{1.764706in}{1.700000in}}%
\pgfusepath{clip}%
\pgfsetbuttcap%
\pgfsetroundjoin%
\definecolor{currentfill}{rgb}{0.944085,0.383081,0.267220}%
\pgfsetfillcolor{currentfill}%
\pgfsetlinewidth{0.311001pt}%
\definecolor{currentstroke}{rgb}{1.000000,1.000000,1.000000}%
\pgfsetstrokecolor{currentstroke}%
\pgfsetdash{}{0pt}%
\pgfpathmoveto{\pgfqpoint{6.137842in}{1.271815in}}%
\pgfpathcurveto{\pgfqpoint{6.144975in}{1.271815in}}{\pgfqpoint{6.151816in}{1.274649in}}{\pgfqpoint{6.156860in}{1.279692in}}%
\pgfpathcurveto{\pgfqpoint{6.161904in}{1.284736in}}{\pgfqpoint{6.164738in}{1.291578in}}{\pgfqpoint{6.164738in}{1.298711in}}%
\pgfpathcurveto{\pgfqpoint{6.164738in}{1.305843in}}{\pgfqpoint{6.161904in}{1.312685in}}{\pgfqpoint{6.156860in}{1.317729in}}%
\pgfpathcurveto{\pgfqpoint{6.151816in}{1.322772in}}{\pgfqpoint{6.144975in}{1.325606in}}{\pgfqpoint{6.137842in}{1.325606in}}%
\pgfpathcurveto{\pgfqpoint{6.130709in}{1.325606in}}{\pgfqpoint{6.123867in}{1.322772in}}{\pgfqpoint{6.118824in}{1.317729in}}%
\pgfpathcurveto{\pgfqpoint{6.113780in}{1.312685in}}{\pgfqpoint{6.110946in}{1.305843in}}{\pgfqpoint{6.110946in}{1.298711in}}%
\pgfpathcurveto{\pgfqpoint{6.110946in}{1.291578in}}{\pgfqpoint{6.113780in}{1.284736in}}{\pgfqpoint{6.118824in}{1.279692in}}%
\pgfpathcurveto{\pgfqpoint{6.123867in}{1.274649in}}{\pgfqpoint{6.130709in}{1.271815in}}{\pgfqpoint{6.137842in}{1.271815in}}%
\pgfpathclose%
\pgfusepath{stroke,fill}%
\end{pgfscope}%
\begin{pgfscope}%
\pgfpathrectangle{\pgfqpoint{4.985294in}{0.500000in}}{\pgfqpoint{1.764706in}{1.700000in}}%
\pgfusepath{clip}%
\pgfsetbuttcap%
\pgfsetroundjoin%
\definecolor{currentfill}{rgb}{0.975018,0.868213,0.788710}%
\pgfsetfillcolor{currentfill}%
\pgfsetlinewidth{0.311001pt}%
\definecolor{currentstroke}{rgb}{1.000000,1.000000,1.000000}%
\pgfsetstrokecolor{currentstroke}%
\pgfsetdash{}{0pt}%
\pgfpathmoveto{\pgfqpoint{5.490067in}{1.069572in}}%
\pgfpathcurveto{\pgfqpoint{5.497199in}{1.069572in}}{\pgfqpoint{5.504041in}{1.072406in}}{\pgfqpoint{5.509085in}{1.077449in}}%
\pgfpathcurveto{\pgfqpoint{5.514128in}{1.082493in}}{\pgfqpoint{5.516962in}{1.089335in}}{\pgfqpoint{5.516962in}{1.096467in}}%
\pgfpathcurveto{\pgfqpoint{5.516962in}{1.103600in}}{\pgfqpoint{5.514128in}{1.110442in}}{\pgfqpoint{5.509085in}{1.115486in}}%
\pgfpathcurveto{\pgfqpoint{5.504041in}{1.120529in}}{\pgfqpoint{5.497199in}{1.123363in}}{\pgfqpoint{5.490067in}{1.123363in}}%
\pgfpathcurveto{\pgfqpoint{5.482934in}{1.123363in}}{\pgfqpoint{5.476092in}{1.120529in}}{\pgfqpoint{5.471048in}{1.115486in}}%
\pgfpathcurveto{\pgfqpoint{5.466005in}{1.110442in}}{\pgfqpoint{5.463171in}{1.103600in}}{\pgfqpoint{5.463171in}{1.096467in}}%
\pgfpathcurveto{\pgfqpoint{5.463171in}{1.089335in}}{\pgfqpoint{5.466005in}{1.082493in}}{\pgfqpoint{5.471048in}{1.077449in}}%
\pgfpathcurveto{\pgfqpoint{5.476092in}{1.072406in}}{\pgfqpoint{5.482934in}{1.069572in}}{\pgfqpoint{5.490067in}{1.069572in}}%
\pgfpathclose%
\pgfusepath{stroke,fill}%
\end{pgfscope}%
\begin{pgfscope}%
\pgfpathrectangle{\pgfqpoint{4.985294in}{0.500000in}}{\pgfqpoint{1.764706in}{1.700000in}}%
\pgfusepath{clip}%
\pgfsetbuttcap%
\pgfsetroundjoin%
\definecolor{currentfill}{rgb}{0.969359,0.803954,0.693832}%
\pgfsetfillcolor{currentfill}%
\pgfsetlinewidth{0.311001pt}%
\definecolor{currentstroke}{rgb}{1.000000,1.000000,1.000000}%
\pgfsetstrokecolor{currentstroke}%
\pgfsetdash{}{0pt}%
\pgfpathmoveto{\pgfqpoint{5.543081in}{1.054099in}}%
\pgfpathcurveto{\pgfqpoint{5.550214in}{1.054099in}}{\pgfqpoint{5.557056in}{1.056933in}}{\pgfqpoint{5.562099in}{1.061977in}}%
\pgfpathcurveto{\pgfqpoint{5.567143in}{1.067020in}}{\pgfqpoint{5.569977in}{1.073862in}}{\pgfqpoint{5.569977in}{1.080995in}}%
\pgfpathcurveto{\pgfqpoint{5.569977in}{1.088128in}}{\pgfqpoint{5.567143in}{1.094969in}}{\pgfqpoint{5.562099in}{1.100013in}}%
\pgfpathcurveto{\pgfqpoint{5.557056in}{1.105057in}}{\pgfqpoint{5.550214in}{1.107890in}}{\pgfqpoint{5.543081in}{1.107890in}}%
\pgfpathcurveto{\pgfqpoint{5.535948in}{1.107890in}}{\pgfqpoint{5.529107in}{1.105057in}}{\pgfqpoint{5.524063in}{1.100013in}}%
\pgfpathcurveto{\pgfqpoint{5.519019in}{1.094969in}}{\pgfqpoint{5.516186in}{1.088128in}}{\pgfqpoint{5.516186in}{1.080995in}}%
\pgfpathcurveto{\pgfqpoint{5.516186in}{1.073862in}}{\pgfqpoint{5.519019in}{1.067020in}}{\pgfqpoint{5.524063in}{1.061977in}}%
\pgfpathcurveto{\pgfqpoint{5.529107in}{1.056933in}}{\pgfqpoint{5.535948in}{1.054099in}}{\pgfqpoint{5.543081in}{1.054099in}}%
\pgfpathclose%
\pgfusepath{stroke,fill}%
\end{pgfscope}%
\begin{pgfscope}%
\pgfpathrectangle{\pgfqpoint{4.985294in}{0.500000in}}{\pgfqpoint{1.764706in}{1.700000in}}%
\pgfusepath{clip}%
\pgfsetbuttcap%
\pgfsetroundjoin%
\definecolor{currentfill}{rgb}{0.975018,0.868213,0.788710}%
\pgfsetfillcolor{currentfill}%
\pgfsetlinewidth{0.311001pt}%
\definecolor{currentstroke}{rgb}{1.000000,1.000000,1.000000}%
\pgfsetstrokecolor{currentstroke}%
\pgfsetdash{}{0pt}%
\pgfpathmoveto{\pgfqpoint{5.456273in}{1.550791in}}%
\pgfpathcurveto{\pgfqpoint{5.463406in}{1.550791in}}{\pgfqpoint{5.470248in}{1.553625in}}{\pgfqpoint{5.475291in}{1.558668in}}%
\pgfpathcurveto{\pgfqpoint{5.480335in}{1.563712in}}{\pgfqpoint{5.483169in}{1.570554in}}{\pgfqpoint{5.483169in}{1.577686in}}%
\pgfpathcurveto{\pgfqpoint{5.483169in}{1.584819in}}{\pgfqpoint{5.480335in}{1.591661in}}{\pgfqpoint{5.475291in}{1.596705in}}%
\pgfpathcurveto{\pgfqpoint{5.470248in}{1.601748in}}{\pgfqpoint{5.463406in}{1.604582in}}{\pgfqpoint{5.456273in}{1.604582in}}%
\pgfpathcurveto{\pgfqpoint{5.449140in}{1.604582in}}{\pgfqpoint{5.442299in}{1.601748in}}{\pgfqpoint{5.437255in}{1.596705in}}%
\pgfpathcurveto{\pgfqpoint{5.432211in}{1.591661in}}{\pgfqpoint{5.429377in}{1.584819in}}{\pgfqpoint{5.429377in}{1.577686in}}%
\pgfpathcurveto{\pgfqpoint{5.429377in}{1.570554in}}{\pgfqpoint{5.432211in}{1.563712in}}{\pgfqpoint{5.437255in}{1.558668in}}%
\pgfpathcurveto{\pgfqpoint{5.442299in}{1.553625in}}{\pgfqpoint{5.449140in}{1.550791in}}{\pgfqpoint{5.456273in}{1.550791in}}%
\pgfpathclose%
\pgfusepath{stroke,fill}%
\end{pgfscope}%
\begin{pgfscope}%
\pgfpathrectangle{\pgfqpoint{4.985294in}{0.500000in}}{\pgfqpoint{1.764706in}{1.700000in}}%
\pgfusepath{clip}%
\pgfsetbuttcap%
\pgfsetroundjoin%
\definecolor{currentfill}{rgb}{0.965169,0.707764,0.560659}%
\pgfsetfillcolor{currentfill}%
\pgfsetlinewidth{0.311001pt}%
\definecolor{currentstroke}{rgb}{1.000000,1.000000,1.000000}%
\pgfsetstrokecolor{currentstroke}%
\pgfsetdash{}{0pt}%
\pgfpathmoveto{\pgfqpoint{6.195394in}{1.413298in}}%
\pgfpathcurveto{\pgfqpoint{6.202527in}{1.413298in}}{\pgfqpoint{6.209368in}{1.416132in}}{\pgfqpoint{6.214412in}{1.421176in}}%
\pgfpathcurveto{\pgfqpoint{6.219456in}{1.426220in}}{\pgfqpoint{6.222289in}{1.433061in}}{\pgfqpoint{6.222289in}{1.440194in}}%
\pgfpathcurveto{\pgfqpoint{6.222289in}{1.447327in}}{\pgfqpoint{6.219456in}{1.454169in}}{\pgfqpoint{6.214412in}{1.459212in}}%
\pgfpathcurveto{\pgfqpoint{6.209368in}{1.464256in}}{\pgfqpoint{6.202527in}{1.467090in}}{\pgfqpoint{6.195394in}{1.467090in}}%
\pgfpathcurveto{\pgfqpoint{6.188261in}{1.467090in}}{\pgfqpoint{6.181419in}{1.464256in}}{\pgfqpoint{6.176376in}{1.459212in}}%
\pgfpathcurveto{\pgfqpoint{6.171332in}{1.454169in}}{\pgfqpoint{6.168498in}{1.447327in}}{\pgfqpoint{6.168498in}{1.440194in}}%
\pgfpathcurveto{\pgfqpoint{6.168498in}{1.433061in}}{\pgfqpoint{6.171332in}{1.426220in}}{\pgfqpoint{6.176376in}{1.421176in}}%
\pgfpathcurveto{\pgfqpoint{6.181419in}{1.416132in}}{\pgfqpoint{6.188261in}{1.413298in}}{\pgfqpoint{6.195394in}{1.413298in}}%
\pgfpathclose%
\pgfusepath{stroke,fill}%
\end{pgfscope}%
\begin{pgfscope}%
\pgfpathrectangle{\pgfqpoint{4.985294in}{0.500000in}}{\pgfqpoint{1.764706in}{1.700000in}}%
\pgfusepath{clip}%
\pgfsetbuttcap%
\pgfsetroundjoin%
\definecolor{currentfill}{rgb}{0.973832,0.856556,0.771584}%
\pgfsetfillcolor{currentfill}%
\pgfsetlinewidth{0.311001pt}%
\definecolor{currentstroke}{rgb}{1.000000,1.000000,1.000000}%
\pgfsetstrokecolor{currentstroke}%
\pgfsetdash{}{0pt}%
\pgfpathmoveto{\pgfqpoint{6.362789in}{1.380396in}}%
\pgfpathcurveto{\pgfqpoint{6.369922in}{1.380396in}}{\pgfqpoint{6.376763in}{1.383230in}}{\pgfqpoint{6.381807in}{1.388274in}}%
\pgfpathcurveto{\pgfqpoint{6.386851in}{1.393318in}}{\pgfqpoint{6.389684in}{1.400159in}}{\pgfqpoint{6.389684in}{1.407292in}}%
\pgfpathcurveto{\pgfqpoint{6.389684in}{1.414425in}}{\pgfqpoint{6.386851in}{1.421267in}}{\pgfqpoint{6.381807in}{1.426310in}}%
\pgfpathcurveto{\pgfqpoint{6.376763in}{1.431354in}}{\pgfqpoint{6.369922in}{1.434188in}}{\pgfqpoint{6.362789in}{1.434188in}}%
\pgfpathcurveto{\pgfqpoint{6.355656in}{1.434188in}}{\pgfqpoint{6.348814in}{1.431354in}}{\pgfqpoint{6.343771in}{1.426310in}}%
\pgfpathcurveto{\pgfqpoint{6.338727in}{1.421267in}}{\pgfqpoint{6.335893in}{1.414425in}}{\pgfqpoint{6.335893in}{1.407292in}}%
\pgfpathcurveto{\pgfqpoint{6.335893in}{1.400159in}}{\pgfqpoint{6.338727in}{1.393318in}}{\pgfqpoint{6.343771in}{1.388274in}}%
\pgfpathcurveto{\pgfqpoint{6.348814in}{1.383230in}}{\pgfqpoint{6.355656in}{1.380396in}}{\pgfqpoint{6.362789in}{1.380396in}}%
\pgfpathclose%
\pgfusepath{stroke,fill}%
\end{pgfscope}%
\begin{pgfscope}%
\pgfpathrectangle{\pgfqpoint{4.985294in}{0.500000in}}{\pgfqpoint{1.764706in}{1.700000in}}%
\pgfusepath{clip}%
\pgfsetbuttcap%
\pgfsetroundjoin%
\definecolor{currentfill}{rgb}{0.975018,0.868213,0.788710}%
\pgfsetfillcolor{currentfill}%
\pgfsetlinewidth{0.311001pt}%
\definecolor{currentstroke}{rgb}{1.000000,1.000000,1.000000}%
\pgfsetstrokecolor{currentstroke}%
\pgfsetdash{}{0pt}%
\pgfpathmoveto{\pgfqpoint{5.371937in}{1.317901in}}%
\pgfpathcurveto{\pgfqpoint{5.379070in}{1.317901in}}{\pgfqpoint{5.385912in}{1.320735in}}{\pgfqpoint{5.390956in}{1.325779in}}%
\pgfpathcurveto{\pgfqpoint{5.395999in}{1.330823in}}{\pgfqpoint{5.398833in}{1.337664in}}{\pgfqpoint{5.398833in}{1.344797in}}%
\pgfpathcurveto{\pgfqpoint{5.398833in}{1.351930in}}{\pgfqpoint{5.395999in}{1.358772in}}{\pgfqpoint{5.390956in}{1.363815in}}%
\pgfpathcurveto{\pgfqpoint{5.385912in}{1.368859in}}{\pgfqpoint{5.379070in}{1.371693in}}{\pgfqpoint{5.371937in}{1.371693in}}%
\pgfpathcurveto{\pgfqpoint{5.364805in}{1.371693in}}{\pgfqpoint{5.357963in}{1.368859in}}{\pgfqpoint{5.352919in}{1.363815in}}%
\pgfpathcurveto{\pgfqpoint{5.347876in}{1.358772in}}{\pgfqpoint{5.345042in}{1.351930in}}{\pgfqpoint{5.345042in}{1.344797in}}%
\pgfpathcurveto{\pgfqpoint{5.345042in}{1.337664in}}{\pgfqpoint{5.347876in}{1.330823in}}{\pgfqpoint{5.352919in}{1.325779in}}%
\pgfpathcurveto{\pgfqpoint{5.357963in}{1.320735in}}{\pgfqpoint{5.364805in}{1.317901in}}{\pgfqpoint{5.371937in}{1.317901in}}%
\pgfpathclose%
\pgfusepath{stroke,fill}%
\end{pgfscope}%
\begin{pgfscope}%
\pgfpathrectangle{\pgfqpoint{4.985294in}{0.500000in}}{\pgfqpoint{1.764706in}{1.700000in}}%
\pgfusepath{clip}%
\pgfsetbuttcap%
\pgfsetroundjoin%
\definecolor{currentfill}{rgb}{0.973832,0.856556,0.771584}%
\pgfsetfillcolor{currentfill}%
\pgfsetlinewidth{0.311001pt}%
\definecolor{currentstroke}{rgb}{1.000000,1.000000,1.000000}%
\pgfsetstrokecolor{currentstroke}%
\pgfsetdash{}{0pt}%
\pgfpathmoveto{\pgfqpoint{6.196934in}{1.615482in}}%
\pgfpathcurveto{\pgfqpoint{6.204067in}{1.615482in}}{\pgfqpoint{6.210909in}{1.618315in}}{\pgfqpoint{6.215953in}{1.623359in}}%
\pgfpathcurveto{\pgfqpoint{6.220996in}{1.628403in}}{\pgfqpoint{6.223830in}{1.635244in}}{\pgfqpoint{6.223830in}{1.642377in}}%
\pgfpathcurveto{\pgfqpoint{6.223830in}{1.649510in}}{\pgfqpoint{6.220996in}{1.656352in}}{\pgfqpoint{6.215953in}{1.661395in}}%
\pgfpathcurveto{\pgfqpoint{6.210909in}{1.666439in}}{\pgfqpoint{6.204067in}{1.669273in}}{\pgfqpoint{6.196934in}{1.669273in}}%
\pgfpathcurveto{\pgfqpoint{6.189802in}{1.669273in}}{\pgfqpoint{6.182960in}{1.666439in}}{\pgfqpoint{6.177916in}{1.661395in}}%
\pgfpathcurveto{\pgfqpoint{6.172873in}{1.656352in}}{\pgfqpoint{6.170039in}{1.649510in}}{\pgfqpoint{6.170039in}{1.642377in}}%
\pgfpathcurveto{\pgfqpoint{6.170039in}{1.635244in}}{\pgfqpoint{6.172873in}{1.628403in}}{\pgfqpoint{6.177916in}{1.623359in}}%
\pgfpathcurveto{\pgfqpoint{6.182960in}{1.618315in}}{\pgfqpoint{6.189802in}{1.615482in}}{\pgfqpoint{6.196934in}{1.615482in}}%
\pgfpathclose%
\pgfusepath{stroke,fill}%
\end{pgfscope}%
\begin{pgfscope}%
\pgfpathrectangle{\pgfqpoint{4.985294in}{0.500000in}}{\pgfqpoint{1.764706in}{1.700000in}}%
\pgfusepath{clip}%
\pgfsetbuttcap%
\pgfsetroundjoin%
\definecolor{currentfill}{rgb}{0.961734,0.579886,0.418445}%
\pgfsetfillcolor{currentfill}%
\pgfsetlinewidth{0.311001pt}%
\definecolor{currentstroke}{rgb}{1.000000,1.000000,1.000000}%
\pgfsetstrokecolor{currentstroke}%
\pgfsetdash{}{0pt}%
\pgfpathmoveto{\pgfqpoint{6.231373in}{0.887208in}}%
\pgfpathcurveto{\pgfqpoint{6.238506in}{0.887208in}}{\pgfqpoint{6.245347in}{0.890042in}}{\pgfqpoint{6.250391in}{0.895085in}}%
\pgfpathcurveto{\pgfqpoint{6.255435in}{0.900129in}}{\pgfqpoint{6.258269in}{0.906971in}}{\pgfqpoint{6.258269in}{0.914103in}}%
\pgfpathcurveto{\pgfqpoint{6.258269in}{0.921236in}}{\pgfqpoint{6.255435in}{0.928078in}}{\pgfqpoint{6.250391in}{0.933122in}}%
\pgfpathcurveto{\pgfqpoint{6.245347in}{0.938165in}}{\pgfqpoint{6.238506in}{0.940999in}}{\pgfqpoint{6.231373in}{0.940999in}}%
\pgfpathcurveto{\pgfqpoint{6.224240in}{0.940999in}}{\pgfqpoint{6.217399in}{0.938165in}}{\pgfqpoint{6.212355in}{0.933122in}}%
\pgfpathcurveto{\pgfqpoint{6.207311in}{0.928078in}}{\pgfqpoint{6.204477in}{0.921236in}}{\pgfqpoint{6.204477in}{0.914103in}}%
\pgfpathcurveto{\pgfqpoint{6.204477in}{0.906971in}}{\pgfqpoint{6.207311in}{0.900129in}}{\pgfqpoint{6.212355in}{0.895085in}}%
\pgfpathcurveto{\pgfqpoint{6.217399in}{0.890042in}}{\pgfqpoint{6.224240in}{0.887208in}}{\pgfqpoint{6.231373in}{0.887208in}}%
\pgfpathclose%
\pgfusepath{stroke,fill}%
\end{pgfscope}%
\begin{pgfscope}%
\pgfpathrectangle{\pgfqpoint{4.985294in}{0.500000in}}{\pgfqpoint{1.764706in}{1.700000in}}%
\pgfusepath{clip}%
\pgfsetbuttcap%
\pgfsetroundjoin%
\definecolor{currentfill}{rgb}{0.968105,0.786346,0.667739}%
\pgfsetfillcolor{currentfill}%
\pgfsetlinewidth{0.311001pt}%
\definecolor{currentstroke}{rgb}{1.000000,1.000000,1.000000}%
\pgfsetstrokecolor{currentstroke}%
\pgfsetdash{}{0pt}%
\pgfpathmoveto{\pgfqpoint{5.515047in}{1.426792in}}%
\pgfpathcurveto{\pgfqpoint{5.522180in}{1.426792in}}{\pgfqpoint{5.529021in}{1.429626in}}{\pgfqpoint{5.534065in}{1.434669in}}%
\pgfpathcurveto{\pgfqpoint{5.539109in}{1.439713in}}{\pgfqpoint{5.541943in}{1.446555in}}{\pgfqpoint{5.541943in}{1.453688in}}%
\pgfpathcurveto{\pgfqpoint{5.541943in}{1.460820in}}{\pgfqpoint{5.539109in}{1.467662in}}{\pgfqpoint{5.534065in}{1.472706in}}%
\pgfpathcurveto{\pgfqpoint{5.529021in}{1.477749in}}{\pgfqpoint{5.522180in}{1.480583in}}{\pgfqpoint{5.515047in}{1.480583in}}%
\pgfpathcurveto{\pgfqpoint{5.507914in}{1.480583in}}{\pgfqpoint{5.501072in}{1.477749in}}{\pgfqpoint{5.496029in}{1.472706in}}%
\pgfpathcurveto{\pgfqpoint{5.490985in}{1.467662in}}{\pgfqpoint{5.488151in}{1.460820in}}{\pgfqpoint{5.488151in}{1.453688in}}%
\pgfpathcurveto{\pgfqpoint{5.488151in}{1.446555in}}{\pgfqpoint{5.490985in}{1.439713in}}{\pgfqpoint{5.496029in}{1.434669in}}%
\pgfpathcurveto{\pgfqpoint{5.501072in}{1.429626in}}{\pgfqpoint{5.507914in}{1.426792in}}{\pgfqpoint{5.515047in}{1.426792in}}%
\pgfpathclose%
\pgfusepath{stroke,fill}%
\end{pgfscope}%
\begin{pgfscope}%
\pgfpathrectangle{\pgfqpoint{4.985294in}{0.500000in}}{\pgfqpoint{1.764706in}{1.700000in}}%
\pgfusepath{clip}%
\pgfsetbuttcap%
\pgfsetroundjoin%
\definecolor{currentfill}{rgb}{0.966328,0.750560,0.616961}%
\pgfsetfillcolor{currentfill}%
\pgfsetlinewidth{0.311001pt}%
\definecolor{currentstroke}{rgb}{1.000000,1.000000,1.000000}%
\pgfsetstrokecolor{currentstroke}%
\pgfsetdash{}{0pt}%
\pgfpathmoveto{\pgfqpoint{6.154479in}{0.960493in}}%
\pgfpathcurveto{\pgfqpoint{6.161612in}{0.960493in}}{\pgfqpoint{6.168454in}{0.963327in}}{\pgfqpoint{6.173497in}{0.968371in}}%
\pgfpathcurveto{\pgfqpoint{6.178541in}{0.973414in}}{\pgfqpoint{6.181375in}{0.980256in}}{\pgfqpoint{6.181375in}{0.987389in}}%
\pgfpathcurveto{\pgfqpoint{6.181375in}{0.994522in}}{\pgfqpoint{6.178541in}{1.001363in}}{\pgfqpoint{6.173497in}{1.006407in}}%
\pgfpathcurveto{\pgfqpoint{6.168454in}{1.011451in}}{\pgfqpoint{6.161612in}{1.014284in}}{\pgfqpoint{6.154479in}{1.014284in}}%
\pgfpathcurveto{\pgfqpoint{6.147346in}{1.014284in}}{\pgfqpoint{6.140505in}{1.011451in}}{\pgfqpoint{6.135461in}{1.006407in}}%
\pgfpathcurveto{\pgfqpoint{6.130417in}{1.001363in}}{\pgfqpoint{6.127584in}{0.994522in}}{\pgfqpoint{6.127584in}{0.987389in}}%
\pgfpathcurveto{\pgfqpoint{6.127584in}{0.980256in}}{\pgfqpoint{6.130417in}{0.973414in}}{\pgfqpoint{6.135461in}{0.968371in}}%
\pgfpathcurveto{\pgfqpoint{6.140505in}{0.963327in}}{\pgfqpoint{6.147346in}{0.960493in}}{\pgfqpoint{6.154479in}{0.960493in}}%
\pgfpathclose%
\pgfusepath{stroke,fill}%
\end{pgfscope}%
\begin{pgfscope}%
\pgfpathrectangle{\pgfqpoint{4.985294in}{0.500000in}}{\pgfqpoint{1.764706in}{1.700000in}}%
\pgfusepath{clip}%
\pgfsetbuttcap%
\pgfsetroundjoin%
\definecolor{currentfill}{rgb}{0.971694,0.833208,0.737161}%
\pgfsetfillcolor{currentfill}%
\pgfsetlinewidth{0.311001pt}%
\definecolor{currentstroke}{rgb}{1.000000,1.000000,1.000000}%
\pgfsetstrokecolor{currentstroke}%
\pgfsetdash{}{0pt}%
\pgfpathmoveto{\pgfqpoint{5.518429in}{1.523246in}}%
\pgfpathcurveto{\pgfqpoint{5.525562in}{1.523246in}}{\pgfqpoint{5.532404in}{1.526080in}}{\pgfqpoint{5.537448in}{1.531123in}}%
\pgfpathcurveto{\pgfqpoint{5.542491in}{1.536167in}}{\pgfqpoint{5.545325in}{1.543009in}}{\pgfqpoint{5.545325in}{1.550141in}}%
\pgfpathcurveto{\pgfqpoint{5.545325in}{1.557274in}}{\pgfqpoint{5.542491in}{1.564116in}}{\pgfqpoint{5.537448in}{1.569160in}}%
\pgfpathcurveto{\pgfqpoint{5.532404in}{1.574203in}}{\pgfqpoint{5.525562in}{1.577037in}}{\pgfqpoint{5.518429in}{1.577037in}}%
\pgfpathcurveto{\pgfqpoint{5.511297in}{1.577037in}}{\pgfqpoint{5.504455in}{1.574203in}}{\pgfqpoint{5.499411in}{1.569160in}}%
\pgfpathcurveto{\pgfqpoint{5.494368in}{1.564116in}}{\pgfqpoint{5.491534in}{1.557274in}}{\pgfqpoint{5.491534in}{1.550141in}}%
\pgfpathcurveto{\pgfqpoint{5.491534in}{1.543009in}}{\pgfqpoint{5.494368in}{1.536167in}}{\pgfqpoint{5.499411in}{1.531123in}}%
\pgfpathcurveto{\pgfqpoint{5.504455in}{1.526080in}}{\pgfqpoint{5.511297in}{1.523246in}}{\pgfqpoint{5.518429in}{1.523246in}}%
\pgfpathclose%
\pgfusepath{stroke,fill}%
\end{pgfscope}%
\begin{pgfscope}%
\pgfpathrectangle{\pgfqpoint{4.985294in}{0.500000in}}{\pgfqpoint{1.764706in}{1.700000in}}%
\pgfusepath{clip}%
\pgfsetbuttcap%
\pgfsetroundjoin%
\definecolor{currentfill}{rgb}{0.975018,0.868213,0.788710}%
\pgfsetfillcolor{currentfill}%
\pgfsetlinewidth{0.311001pt}%
\definecolor{currentstroke}{rgb}{1.000000,1.000000,1.000000}%
\pgfsetstrokecolor{currentstroke}%
\pgfsetdash{}{0pt}%
\pgfpathmoveto{\pgfqpoint{6.308427in}{1.560803in}}%
\pgfpathcurveto{\pgfqpoint{6.315560in}{1.560803in}}{\pgfqpoint{6.322402in}{1.563637in}}{\pgfqpoint{6.327445in}{1.568681in}}%
\pgfpathcurveto{\pgfqpoint{6.332489in}{1.573724in}}{\pgfqpoint{6.335323in}{1.580566in}}{\pgfqpoint{6.335323in}{1.587699in}}%
\pgfpathcurveto{\pgfqpoint{6.335323in}{1.594831in}}{\pgfqpoint{6.332489in}{1.601673in}}{\pgfqpoint{6.327445in}{1.606717in}}%
\pgfpathcurveto{\pgfqpoint{6.322402in}{1.611760in}}{\pgfqpoint{6.315560in}{1.614594in}}{\pgfqpoint{6.308427in}{1.614594in}}%
\pgfpathcurveto{\pgfqpoint{6.301294in}{1.614594in}}{\pgfqpoint{6.294453in}{1.611760in}}{\pgfqpoint{6.289409in}{1.606717in}}%
\pgfpathcurveto{\pgfqpoint{6.284365in}{1.601673in}}{\pgfqpoint{6.281531in}{1.594831in}}{\pgfqpoint{6.281531in}{1.587699in}}%
\pgfpathcurveto{\pgfqpoint{6.281531in}{1.580566in}}{\pgfqpoint{6.284365in}{1.573724in}}{\pgfqpoint{6.289409in}{1.568681in}}%
\pgfpathcurveto{\pgfqpoint{6.294453in}{1.563637in}}{\pgfqpoint{6.301294in}{1.560803in}}{\pgfqpoint{6.308427in}{1.560803in}}%
\pgfpathclose%
\pgfusepath{stroke,fill}%
\end{pgfscope}%
\begin{pgfscope}%
\pgfpathrectangle{\pgfqpoint{4.985294in}{0.500000in}}{\pgfqpoint{1.764706in}{1.700000in}}%
\pgfusepath{clip}%
\pgfsetbuttcap%
\pgfsetroundjoin%
\definecolor{currentfill}{rgb}{0.975644,0.874038,0.797253}%
\pgfsetfillcolor{currentfill}%
\pgfsetlinewidth{0.311001pt}%
\definecolor{currentstroke}{rgb}{1.000000,1.000000,1.000000}%
\pgfsetstrokecolor{currentstroke}%
\pgfsetdash{}{0pt}%
\pgfpathmoveto{\pgfqpoint{5.470739in}{1.423456in}}%
\pgfpathcurveto{\pgfqpoint{5.477872in}{1.423456in}}{\pgfqpoint{5.484714in}{1.426289in}}{\pgfqpoint{5.489757in}{1.431333in}}%
\pgfpathcurveto{\pgfqpoint{5.494801in}{1.436377in}}{\pgfqpoint{5.497635in}{1.443218in}}{\pgfqpoint{5.497635in}{1.450351in}}%
\pgfpathcurveto{\pgfqpoint{5.497635in}{1.457484in}}{\pgfqpoint{5.494801in}{1.464326in}}{\pgfqpoint{5.489757in}{1.469369in}}%
\pgfpathcurveto{\pgfqpoint{5.484714in}{1.474413in}}{\pgfqpoint{5.477872in}{1.477247in}}{\pgfqpoint{5.470739in}{1.477247in}}%
\pgfpathcurveto{\pgfqpoint{5.463606in}{1.477247in}}{\pgfqpoint{5.456765in}{1.474413in}}{\pgfqpoint{5.451721in}{1.469369in}}%
\pgfpathcurveto{\pgfqpoint{5.446677in}{1.464326in}}{\pgfqpoint{5.443844in}{1.457484in}}{\pgfqpoint{5.443844in}{1.450351in}}%
\pgfpathcurveto{\pgfqpoint{5.443844in}{1.443218in}}{\pgfqpoint{5.446677in}{1.436377in}}{\pgfqpoint{5.451721in}{1.431333in}}%
\pgfpathcurveto{\pgfqpoint{5.456765in}{1.426289in}}{\pgfqpoint{5.463606in}{1.423456in}}{\pgfqpoint{5.470739in}{1.423456in}}%
\pgfpathclose%
\pgfusepath{stroke,fill}%
\end{pgfscope}%
\begin{pgfscope}%
\pgfpathrectangle{\pgfqpoint{4.985294in}{0.500000in}}{\pgfqpoint{1.764706in}{1.700000in}}%
\pgfusepath{clip}%
\pgfsetbuttcap%
\pgfsetroundjoin%
\definecolor{currentfill}{rgb}{0.978376,0.897317,0.831308}%
\pgfsetfillcolor{currentfill}%
\pgfsetlinewidth{0.311001pt}%
\definecolor{currentstroke}{rgb}{1.000000,1.000000,1.000000}%
\pgfsetstrokecolor{currentstroke}%
\pgfsetdash{}{0pt}%
\pgfpathmoveto{\pgfqpoint{5.428014in}{1.470992in}}%
\pgfpathcurveto{\pgfqpoint{5.435146in}{1.470992in}}{\pgfqpoint{5.441988in}{1.473826in}}{\pgfqpoint{5.447032in}{1.478869in}}%
\pgfpathcurveto{\pgfqpoint{5.452075in}{1.483913in}}{\pgfqpoint{5.454909in}{1.490754in}}{\pgfqpoint{5.454909in}{1.497887in}}%
\pgfpathcurveto{\pgfqpoint{5.454909in}{1.505020in}}{\pgfqpoint{5.452075in}{1.511862in}}{\pgfqpoint{5.447032in}{1.516905in}}%
\pgfpathcurveto{\pgfqpoint{5.441988in}{1.521949in}}{\pgfqpoint{5.435146in}{1.524783in}}{\pgfqpoint{5.428014in}{1.524783in}}%
\pgfpathcurveto{\pgfqpoint{5.420881in}{1.524783in}}{\pgfqpoint{5.414039in}{1.521949in}}{\pgfqpoint{5.408995in}{1.516905in}}%
\pgfpathcurveto{\pgfqpoint{5.403952in}{1.511862in}}{\pgfqpoint{5.401118in}{1.505020in}}{\pgfqpoint{5.401118in}{1.497887in}}%
\pgfpathcurveto{\pgfqpoint{5.401118in}{1.490754in}}{\pgfqpoint{5.403952in}{1.483913in}}{\pgfqpoint{5.408995in}{1.478869in}}%
\pgfpathcurveto{\pgfqpoint{5.414039in}{1.473826in}}{\pgfqpoint{5.420881in}{1.470992in}}{\pgfqpoint{5.428014in}{1.470992in}}%
\pgfpathclose%
\pgfusepath{stroke,fill}%
\end{pgfscope}%
\begin{pgfscope}%
\pgfpathrectangle{\pgfqpoint{4.985294in}{0.500000in}}{\pgfqpoint{1.764706in}{1.700000in}}%
\pgfusepath{clip}%
\pgfsetbuttcap%
\pgfsetroundjoin%
\definecolor{currentfill}{rgb}{0.979891,0.908948,0.848279}%
\pgfsetfillcolor{currentfill}%
\pgfsetlinewidth{0.311001pt}%
\definecolor{currentstroke}{rgb}{1.000000,1.000000,1.000000}%
\pgfsetstrokecolor{currentstroke}%
\pgfsetdash{}{0pt}%
\pgfpathmoveto{\pgfqpoint{6.334734in}{1.249527in}}%
\pgfpathcurveto{\pgfqpoint{6.341866in}{1.249527in}}{\pgfqpoint{6.348708in}{1.252361in}}{\pgfqpoint{6.353752in}{1.257405in}}%
\pgfpathcurveto{\pgfqpoint{6.358795in}{1.262449in}}{\pgfqpoint{6.361629in}{1.269290in}}{\pgfqpoint{6.361629in}{1.276423in}}%
\pgfpathcurveto{\pgfqpoint{6.361629in}{1.283556in}}{\pgfqpoint{6.358795in}{1.290397in}}{\pgfqpoint{6.353752in}{1.295441in}}%
\pgfpathcurveto{\pgfqpoint{6.348708in}{1.300485in}}{\pgfqpoint{6.341866in}{1.303319in}}{\pgfqpoint{6.334734in}{1.303319in}}%
\pgfpathcurveto{\pgfqpoint{6.327601in}{1.303319in}}{\pgfqpoint{6.320759in}{1.300485in}}{\pgfqpoint{6.315715in}{1.295441in}}%
\pgfpathcurveto{\pgfqpoint{6.310672in}{1.290397in}}{\pgfqpoint{6.307838in}{1.283556in}}{\pgfqpoint{6.307838in}{1.276423in}}%
\pgfpathcurveto{\pgfqpoint{6.307838in}{1.269290in}}{\pgfqpoint{6.310672in}{1.262449in}}{\pgfqpoint{6.315715in}{1.257405in}}%
\pgfpathcurveto{\pgfqpoint{6.320759in}{1.252361in}}{\pgfqpoint{6.327601in}{1.249527in}}{\pgfqpoint{6.334734in}{1.249527in}}%
\pgfpathclose%
\pgfusepath{stroke,fill}%
\end{pgfscope}%
\begin{pgfscope}%
\pgfpathrectangle{\pgfqpoint{4.985294in}{0.500000in}}{\pgfqpoint{1.764706in}{1.700000in}}%
\pgfusepath{clip}%
\pgfsetbuttcap%
\pgfsetroundjoin%
\definecolor{currentfill}{rgb}{0.963559,0.632016,0.472047}%
\pgfsetfillcolor{currentfill}%
\pgfsetlinewidth{0.311001pt}%
\definecolor{currentstroke}{rgb}{1.000000,1.000000,1.000000}%
\pgfsetstrokecolor{currentstroke}%
\pgfsetdash{}{0pt}%
\pgfpathmoveto{\pgfqpoint{5.618007in}{1.655059in}}%
\pgfpathcurveto{\pgfqpoint{5.625140in}{1.655059in}}{\pgfqpoint{5.631981in}{1.657893in}}{\pgfqpoint{5.637025in}{1.662937in}}%
\pgfpathcurveto{\pgfqpoint{5.642069in}{1.667980in}}{\pgfqpoint{5.644903in}{1.674822in}}{\pgfqpoint{5.644903in}{1.681955in}}%
\pgfpathcurveto{\pgfqpoint{5.644903in}{1.689088in}}{\pgfqpoint{5.642069in}{1.695929in}}{\pgfqpoint{5.637025in}{1.700973in}}%
\pgfpathcurveto{\pgfqpoint{5.631981in}{1.706017in}}{\pgfqpoint{5.625140in}{1.708851in}}{\pgfqpoint{5.618007in}{1.708851in}}%
\pgfpathcurveto{\pgfqpoint{5.610874in}{1.708851in}}{\pgfqpoint{5.604032in}{1.706017in}}{\pgfqpoint{5.598989in}{1.700973in}}%
\pgfpathcurveto{\pgfqpoint{5.593945in}{1.695929in}}{\pgfqpoint{5.591111in}{1.689088in}}{\pgfqpoint{5.591111in}{1.681955in}}%
\pgfpathcurveto{\pgfqpoint{5.591111in}{1.674822in}}{\pgfqpoint{5.593945in}{1.667980in}}{\pgfqpoint{5.598989in}{1.662937in}}%
\pgfpathcurveto{\pgfqpoint{5.604032in}{1.657893in}}{\pgfqpoint{5.610874in}{1.655059in}}{\pgfqpoint{5.618007in}{1.655059in}}%
\pgfpathclose%
\pgfusepath{stroke,fill}%
\end{pgfscope}%
\begin{pgfscope}%
\pgfpathrectangle{\pgfqpoint{4.985294in}{0.500000in}}{\pgfqpoint{1.764706in}{1.700000in}}%
\pgfusepath{clip}%
\pgfsetbuttcap%
\pgfsetroundjoin%
\definecolor{currentfill}{rgb}{0.971202,0.827364,0.728520}%
\pgfsetfillcolor{currentfill}%
\pgfsetlinewidth{0.311001pt}%
\definecolor{currentstroke}{rgb}{1.000000,1.000000,1.000000}%
\pgfsetstrokecolor{currentstroke}%
\pgfsetdash{}{0pt}%
\pgfpathmoveto{\pgfqpoint{6.305002in}{1.595790in}}%
\pgfpathcurveto{\pgfqpoint{6.312135in}{1.595790in}}{\pgfqpoint{6.318976in}{1.598624in}}{\pgfqpoint{6.324020in}{1.603668in}}%
\pgfpathcurveto{\pgfqpoint{6.329064in}{1.608711in}}{\pgfqpoint{6.331898in}{1.615553in}}{\pgfqpoint{6.331898in}{1.622686in}}%
\pgfpathcurveto{\pgfqpoint{6.331898in}{1.629819in}}{\pgfqpoint{6.329064in}{1.636660in}}{\pgfqpoint{6.324020in}{1.641704in}}%
\pgfpathcurveto{\pgfqpoint{6.318976in}{1.646748in}}{\pgfqpoint{6.312135in}{1.649582in}}{\pgfqpoint{6.305002in}{1.649582in}}%
\pgfpathcurveto{\pgfqpoint{6.297869in}{1.649582in}}{\pgfqpoint{6.291028in}{1.646748in}}{\pgfqpoint{6.285984in}{1.641704in}}%
\pgfpathcurveto{\pgfqpoint{6.280940in}{1.636660in}}{\pgfqpoint{6.278106in}{1.629819in}}{\pgfqpoint{6.278106in}{1.622686in}}%
\pgfpathcurveto{\pgfqpoint{6.278106in}{1.615553in}}{\pgfqpoint{6.280940in}{1.608711in}}{\pgfqpoint{6.285984in}{1.603668in}}%
\pgfpathcurveto{\pgfqpoint{6.291028in}{1.598624in}}{\pgfqpoint{6.297869in}{1.595790in}}{\pgfqpoint{6.305002in}{1.595790in}}%
\pgfpathclose%
\pgfusepath{stroke,fill}%
\end{pgfscope}%
\begin{pgfscope}%
\pgfpathrectangle{\pgfqpoint{4.985294in}{0.500000in}}{\pgfqpoint{1.764706in}{1.700000in}}%
\pgfusepath{clip}%
\pgfsetbuttcap%
\pgfsetroundjoin%
\definecolor{currentfill}{rgb}{0.977657,0.891500,0.822809}%
\pgfsetfillcolor{currentfill}%
\pgfsetlinewidth{0.311001pt}%
\definecolor{currentstroke}{rgb}{1.000000,1.000000,1.000000}%
\pgfsetstrokecolor{currentstroke}%
\pgfsetdash{}{0pt}%
\pgfpathmoveto{\pgfqpoint{6.347250in}{1.248659in}}%
\pgfpathcurveto{\pgfqpoint{6.354382in}{1.248659in}}{\pgfqpoint{6.361224in}{1.251493in}}{\pgfqpoint{6.366268in}{1.256536in}}%
\pgfpathcurveto{\pgfqpoint{6.371311in}{1.261580in}}{\pgfqpoint{6.374145in}{1.268422in}}{\pgfqpoint{6.374145in}{1.275555in}}%
\pgfpathcurveto{\pgfqpoint{6.374145in}{1.282687in}}{\pgfqpoint{6.371311in}{1.289529in}}{\pgfqpoint{6.366268in}{1.294573in}}%
\pgfpathcurveto{\pgfqpoint{6.361224in}{1.299616in}}{\pgfqpoint{6.354382in}{1.302450in}}{\pgfqpoint{6.347250in}{1.302450in}}%
\pgfpathcurveto{\pgfqpoint{6.340117in}{1.302450in}}{\pgfqpoint{6.333275in}{1.299616in}}{\pgfqpoint{6.328232in}{1.294573in}}%
\pgfpathcurveto{\pgfqpoint{6.323188in}{1.289529in}}{\pgfqpoint{6.320354in}{1.282687in}}{\pgfqpoint{6.320354in}{1.275555in}}%
\pgfpathcurveto{\pgfqpoint{6.320354in}{1.268422in}}{\pgfqpoint{6.323188in}{1.261580in}}{\pgfqpoint{6.328232in}{1.256536in}}%
\pgfpathcurveto{\pgfqpoint{6.333275in}{1.251493in}}{\pgfqpoint{6.340117in}{1.248659in}}{\pgfqpoint{6.347250in}{1.248659in}}%
\pgfpathclose%
\pgfusepath{stroke,fill}%
\end{pgfscope}%
\begin{pgfscope}%
\pgfpathrectangle{\pgfqpoint{4.985294in}{0.500000in}}{\pgfqpoint{1.764706in}{1.700000in}}%
\pgfusepath{clip}%
\pgfsetbuttcap%
\pgfsetroundjoin%
\definecolor{currentfill}{rgb}{0.974412,0.862387,0.780156}%
\pgfsetfillcolor{currentfill}%
\pgfsetlinewidth{0.311001pt}%
\definecolor{currentstroke}{rgb}{1.000000,1.000000,1.000000}%
\pgfsetstrokecolor{currentstroke}%
\pgfsetdash{}{0pt}%
\pgfpathmoveto{\pgfqpoint{6.226406in}{1.606740in}}%
\pgfpathcurveto{\pgfqpoint{6.233539in}{1.606740in}}{\pgfqpoint{6.240381in}{1.609574in}}{\pgfqpoint{6.245424in}{1.614618in}}%
\pgfpathcurveto{\pgfqpoint{6.250468in}{1.619661in}}{\pgfqpoint{6.253302in}{1.626503in}}{\pgfqpoint{6.253302in}{1.633636in}}%
\pgfpathcurveto{\pgfqpoint{6.253302in}{1.640769in}}{\pgfqpoint{6.250468in}{1.647610in}}{\pgfqpoint{6.245424in}{1.652654in}}%
\pgfpathcurveto{\pgfqpoint{6.240381in}{1.657698in}}{\pgfqpoint{6.233539in}{1.660532in}}{\pgfqpoint{6.226406in}{1.660532in}}%
\pgfpathcurveto{\pgfqpoint{6.219273in}{1.660532in}}{\pgfqpoint{6.212432in}{1.657698in}}{\pgfqpoint{6.207388in}{1.652654in}}%
\pgfpathcurveto{\pgfqpoint{6.202344in}{1.647610in}}{\pgfqpoint{6.199510in}{1.640769in}}{\pgfqpoint{6.199510in}{1.633636in}}%
\pgfpathcurveto{\pgfqpoint{6.199510in}{1.626503in}}{\pgfqpoint{6.202344in}{1.619661in}}{\pgfqpoint{6.207388in}{1.614618in}}%
\pgfpathcurveto{\pgfqpoint{6.212432in}{1.609574in}}{\pgfqpoint{6.219273in}{1.606740in}}{\pgfqpoint{6.226406in}{1.606740in}}%
\pgfpathclose%
\pgfusepath{stroke,fill}%
\end{pgfscope}%
\begin{pgfscope}%
\pgfpathrectangle{\pgfqpoint{4.985294in}{0.500000in}}{\pgfqpoint{1.764706in}{1.700000in}}%
\pgfusepath{clip}%
\pgfsetbuttcap%
\pgfsetroundjoin%
\definecolor{currentfill}{rgb}{0.980678,0.914765,0.856766}%
\pgfsetfillcolor{currentfill}%
\pgfsetlinewidth{0.311001pt}%
\definecolor{currentstroke}{rgb}{1.000000,1.000000,1.000000}%
\pgfsetstrokecolor{currentstroke}%
\pgfsetdash{}{0pt}%
\pgfpathmoveto{\pgfqpoint{5.416149in}{1.357176in}}%
\pgfpathcurveto{\pgfqpoint{5.423282in}{1.357176in}}{\pgfqpoint{5.430123in}{1.360010in}}{\pgfqpoint{5.435167in}{1.365053in}}%
\pgfpathcurveto{\pgfqpoint{5.440211in}{1.370097in}}{\pgfqpoint{5.443044in}{1.376939in}}{\pgfqpoint{5.443044in}{1.384071in}}%
\pgfpathcurveto{\pgfqpoint{5.443044in}{1.391204in}}{\pgfqpoint{5.440211in}{1.398046in}}{\pgfqpoint{5.435167in}{1.403090in}}%
\pgfpathcurveto{\pgfqpoint{5.430123in}{1.408133in}}{\pgfqpoint{5.423282in}{1.410967in}}{\pgfqpoint{5.416149in}{1.410967in}}%
\pgfpathcurveto{\pgfqpoint{5.409016in}{1.410967in}}{\pgfqpoint{5.402174in}{1.408133in}}{\pgfqpoint{5.397131in}{1.403090in}}%
\pgfpathcurveto{\pgfqpoint{5.392087in}{1.398046in}}{\pgfqpoint{5.389253in}{1.391204in}}{\pgfqpoint{5.389253in}{1.384071in}}%
\pgfpathcurveto{\pgfqpoint{5.389253in}{1.376939in}}{\pgfqpoint{5.392087in}{1.370097in}}{\pgfqpoint{5.397131in}{1.365053in}}%
\pgfpathcurveto{\pgfqpoint{5.402174in}{1.360010in}}{\pgfqpoint{5.409016in}{1.357176in}}{\pgfqpoint{5.416149in}{1.357176in}}%
\pgfpathclose%
\pgfusepath{stroke,fill}%
\end{pgfscope}%
\begin{pgfscope}%
\pgfpathrectangle{\pgfqpoint{4.985294in}{0.500000in}}{\pgfqpoint{1.764706in}{1.700000in}}%
\pgfusepath{clip}%
\pgfsetbuttcap%
\pgfsetroundjoin%
\definecolor{currentfill}{rgb}{0.966560,0.756582,0.625273}%
\pgfsetfillcolor{currentfill}%
\pgfsetlinewidth{0.311001pt}%
\definecolor{currentstroke}{rgb}{1.000000,1.000000,1.000000}%
\pgfsetstrokecolor{currentstroke}%
\pgfsetdash{}{0pt}%
\pgfpathmoveto{\pgfqpoint{5.524458in}{1.165920in}}%
\pgfpathcurveto{\pgfqpoint{5.531591in}{1.165920in}}{\pgfqpoint{5.538433in}{1.168754in}}{\pgfqpoint{5.543477in}{1.173797in}}%
\pgfpathcurveto{\pgfqpoint{5.548520in}{1.178841in}}{\pgfqpoint{5.551354in}{1.185683in}}{\pgfqpoint{5.551354in}{1.192815in}}%
\pgfpathcurveto{\pgfqpoint{5.551354in}{1.199948in}}{\pgfqpoint{5.548520in}{1.206790in}}{\pgfqpoint{5.543477in}{1.211834in}}%
\pgfpathcurveto{\pgfqpoint{5.538433in}{1.216877in}}{\pgfqpoint{5.531591in}{1.219711in}}{\pgfqpoint{5.524458in}{1.219711in}}%
\pgfpathcurveto{\pgfqpoint{5.517326in}{1.219711in}}{\pgfqpoint{5.510484in}{1.216877in}}{\pgfqpoint{5.505440in}{1.211834in}}%
\pgfpathcurveto{\pgfqpoint{5.500397in}{1.206790in}}{\pgfqpoint{5.497563in}{1.199948in}}{\pgfqpoint{5.497563in}{1.192815in}}%
\pgfpathcurveto{\pgfqpoint{5.497563in}{1.185683in}}{\pgfqpoint{5.500397in}{1.178841in}}{\pgfqpoint{5.505440in}{1.173797in}}%
\pgfpathcurveto{\pgfqpoint{5.510484in}{1.168754in}}{\pgfqpoint{5.517326in}{1.165920in}}{\pgfqpoint{5.524458in}{1.165920in}}%
\pgfpathclose%
\pgfusepath{stroke,fill}%
\end{pgfscope}%
\begin{pgfscope}%
\pgfpathrectangle{\pgfqpoint{4.985294in}{0.500000in}}{\pgfqpoint{1.764706in}{1.700000in}}%
\pgfusepath{clip}%
\pgfsetbuttcap%
\pgfsetroundjoin%
\definecolor{currentfill}{rgb}{0.796501,0.105066,0.310630}%
\pgfsetfillcolor{currentfill}%
\pgfsetlinewidth{0.311001pt}%
\definecolor{currentstroke}{rgb}{1.000000,1.000000,1.000000}%
\pgfsetstrokecolor{currentstroke}%
\pgfsetdash{}{0pt}%
\pgfpathmoveto{\pgfqpoint{5.990270in}{0.992481in}}%
\pgfpathcurveto{\pgfqpoint{5.997403in}{0.992481in}}{\pgfqpoint{6.004245in}{0.995315in}}{\pgfqpoint{6.009288in}{1.000358in}}%
\pgfpathcurveto{\pgfqpoint{6.014332in}{1.005402in}}{\pgfqpoint{6.017166in}{1.012244in}}{\pgfqpoint{6.017166in}{1.019376in}}%
\pgfpathcurveto{\pgfqpoint{6.017166in}{1.026509in}}{\pgfqpoint{6.014332in}{1.033351in}}{\pgfqpoint{6.009288in}{1.038394in}}%
\pgfpathcurveto{\pgfqpoint{6.004245in}{1.043438in}}{\pgfqpoint{5.997403in}{1.046272in}}{\pgfqpoint{5.990270in}{1.046272in}}%
\pgfpathcurveto{\pgfqpoint{5.983137in}{1.046272in}}{\pgfqpoint{5.976296in}{1.043438in}}{\pgfqpoint{5.971252in}{1.038394in}}%
\pgfpathcurveto{\pgfqpoint{5.966208in}{1.033351in}}{\pgfqpoint{5.963374in}{1.026509in}}{\pgfqpoint{5.963374in}{1.019376in}}%
\pgfpathcurveto{\pgfqpoint{5.963374in}{1.012244in}}{\pgfqpoint{5.966208in}{1.005402in}}{\pgfqpoint{5.971252in}{1.000358in}}%
\pgfpathcurveto{\pgfqpoint{5.976296in}{0.995315in}}{\pgfqpoint{5.983137in}{0.992481in}}{\pgfqpoint{5.990270in}{0.992481in}}%
\pgfpathclose%
\pgfusepath{stroke,fill}%
\end{pgfscope}%
\begin{pgfscope}%
\pgfpathrectangle{\pgfqpoint{4.985294in}{0.500000in}}{\pgfqpoint{1.764706in}{1.700000in}}%
\pgfusepath{clip}%
\pgfsetbuttcap%
\pgfsetroundjoin%
\definecolor{currentfill}{rgb}{0.942910,0.375495,0.263698}%
\pgfsetfillcolor{currentfill}%
\pgfsetlinewidth{0.311001pt}%
\definecolor{currentstroke}{rgb}{1.000000,1.000000,1.000000}%
\pgfsetstrokecolor{currentstroke}%
\pgfsetdash{}{0pt}%
\pgfpathmoveto{\pgfqpoint{6.072928in}{1.793765in}}%
\pgfpathcurveto{\pgfqpoint{6.080061in}{1.793765in}}{\pgfqpoint{6.086903in}{1.796599in}}{\pgfqpoint{6.091947in}{1.801643in}}%
\pgfpathcurveto{\pgfqpoint{6.096990in}{1.806686in}}{\pgfqpoint{6.099824in}{1.813528in}}{\pgfqpoint{6.099824in}{1.820661in}}%
\pgfpathcurveto{\pgfqpoint{6.099824in}{1.827794in}}{\pgfqpoint{6.096990in}{1.834635in}}{\pgfqpoint{6.091947in}{1.839679in}}%
\pgfpathcurveto{\pgfqpoint{6.086903in}{1.844723in}}{\pgfqpoint{6.080061in}{1.847557in}}{\pgfqpoint{6.072928in}{1.847557in}}%
\pgfpathcurveto{\pgfqpoint{6.065796in}{1.847557in}}{\pgfqpoint{6.058954in}{1.844723in}}{\pgfqpoint{6.053910in}{1.839679in}}%
\pgfpathcurveto{\pgfqpoint{6.048867in}{1.834635in}}{\pgfqpoint{6.046033in}{1.827794in}}{\pgfqpoint{6.046033in}{1.820661in}}%
\pgfpathcurveto{\pgfqpoint{6.046033in}{1.813528in}}{\pgfqpoint{6.048867in}{1.806686in}}{\pgfqpoint{6.053910in}{1.801643in}}%
\pgfpathcurveto{\pgfqpoint{6.058954in}{1.796599in}}{\pgfqpoint{6.065796in}{1.793765in}}{\pgfqpoint{6.072928in}{1.793765in}}%
\pgfpathclose%
\pgfusepath{stroke,fill}%
\end{pgfscope}%
\begin{pgfscope}%
\pgfpathrectangle{\pgfqpoint{4.985294in}{0.500000in}}{\pgfqpoint{1.764706in}{1.700000in}}%
\pgfusepath{clip}%
\pgfsetbuttcap%
\pgfsetroundjoin%
\definecolor{currentfill}{rgb}{0.978376,0.897317,0.831308}%
\pgfsetfillcolor{currentfill}%
\pgfsetlinewidth{0.311001pt}%
\definecolor{currentstroke}{rgb}{1.000000,1.000000,1.000000}%
\pgfsetstrokecolor{currentstroke}%
\pgfsetdash{}{0pt}%
\pgfpathmoveto{\pgfqpoint{6.280787in}{1.269627in}}%
\pgfpathcurveto{\pgfqpoint{6.287920in}{1.269627in}}{\pgfqpoint{6.294762in}{1.272460in}}{\pgfqpoint{6.299806in}{1.277504in}}%
\pgfpathcurveto{\pgfqpoint{6.304849in}{1.282548in}}{\pgfqpoint{6.307683in}{1.289389in}}{\pgfqpoint{6.307683in}{1.296522in}}%
\pgfpathcurveto{\pgfqpoint{6.307683in}{1.303655in}}{\pgfqpoint{6.304849in}{1.310497in}}{\pgfqpoint{6.299806in}{1.315540in}}%
\pgfpathcurveto{\pgfqpoint{6.294762in}{1.320584in}}{\pgfqpoint{6.287920in}{1.323418in}}{\pgfqpoint{6.280787in}{1.323418in}}%
\pgfpathcurveto{\pgfqpoint{6.273655in}{1.323418in}}{\pgfqpoint{6.266813in}{1.320584in}}{\pgfqpoint{6.261769in}{1.315540in}}%
\pgfpathcurveto{\pgfqpoint{6.256726in}{1.310497in}}{\pgfqpoint{6.253892in}{1.303655in}}{\pgfqpoint{6.253892in}{1.296522in}}%
\pgfpathcurveto{\pgfqpoint{6.253892in}{1.289389in}}{\pgfqpoint{6.256726in}{1.282548in}}{\pgfqpoint{6.261769in}{1.277504in}}%
\pgfpathcurveto{\pgfqpoint{6.266813in}{1.272460in}}{\pgfqpoint{6.273655in}{1.269627in}}{\pgfqpoint{6.280787in}{1.269627in}}%
\pgfpathclose%
\pgfusepath{stroke,fill}%
\end{pgfscope}%
\begin{pgfscope}%
\pgfpathrectangle{\pgfqpoint{4.985294in}{0.500000in}}{\pgfqpoint{1.764706in}{1.700000in}}%
\pgfusepath{clip}%
\pgfsetbuttcap%
\pgfsetroundjoin%
\definecolor{currentfill}{rgb}{0.973271,0.850724,0.762998}%
\pgfsetfillcolor{currentfill}%
\pgfsetlinewidth{0.311001pt}%
\definecolor{currentstroke}{rgb}{1.000000,1.000000,1.000000}%
\pgfsetstrokecolor{currentstroke}%
\pgfsetdash{}{0pt}%
\pgfpathmoveto{\pgfqpoint{5.510874in}{1.540816in}}%
\pgfpathcurveto{\pgfqpoint{5.518007in}{1.540816in}}{\pgfqpoint{5.524848in}{1.543650in}}{\pgfqpoint{5.529892in}{1.548694in}}%
\pgfpathcurveto{\pgfqpoint{5.534935in}{1.553738in}}{\pgfqpoint{5.537769in}{1.560579in}}{\pgfqpoint{5.537769in}{1.567712in}}%
\pgfpathcurveto{\pgfqpoint{5.537769in}{1.574845in}}{\pgfqpoint{5.534935in}{1.581687in}}{\pgfqpoint{5.529892in}{1.586730in}}%
\pgfpathcurveto{\pgfqpoint{5.524848in}{1.591774in}}{\pgfqpoint{5.518007in}{1.594608in}}{\pgfqpoint{5.510874in}{1.594608in}}%
\pgfpathcurveto{\pgfqpoint{5.503741in}{1.594608in}}{\pgfqpoint{5.496899in}{1.591774in}}{\pgfqpoint{5.491856in}{1.586730in}}%
\pgfpathcurveto{\pgfqpoint{5.486812in}{1.581687in}}{\pgfqpoint{5.483978in}{1.574845in}}{\pgfqpoint{5.483978in}{1.567712in}}%
\pgfpathcurveto{\pgfqpoint{5.483978in}{1.560579in}}{\pgfqpoint{5.486812in}{1.553738in}}{\pgfqpoint{5.491856in}{1.548694in}}%
\pgfpathcurveto{\pgfqpoint{5.496899in}{1.543650in}}{\pgfqpoint{5.503741in}{1.540816in}}{\pgfqpoint{5.510874in}{1.540816in}}%
\pgfpathclose%
\pgfusepath{stroke,fill}%
\end{pgfscope}%
\begin{pgfscope}%
\pgfpathrectangle{\pgfqpoint{4.985294in}{0.500000in}}{\pgfqpoint{1.764706in}{1.700000in}}%
\pgfusepath{clip}%
\pgfsetbuttcap%
\pgfsetroundjoin%
\definecolor{currentfill}{rgb}{0.970255,0.815666,0.711203}%
\pgfsetfillcolor{currentfill}%
\pgfsetlinewidth{0.311001pt}%
\definecolor{currentstroke}{rgb}{1.000000,1.000000,1.000000}%
\pgfsetstrokecolor{currentstroke}%
\pgfsetdash{}{0pt}%
\pgfpathmoveto{\pgfqpoint{5.521370in}{0.972507in}}%
\pgfpathcurveto{\pgfqpoint{5.528503in}{0.972507in}}{\pgfqpoint{5.535344in}{0.975341in}}{\pgfqpoint{5.540388in}{0.980385in}}%
\pgfpathcurveto{\pgfqpoint{5.545432in}{0.985429in}}{\pgfqpoint{5.548266in}{0.992270in}}{\pgfqpoint{5.548266in}{0.999403in}}%
\pgfpathcurveto{\pgfqpoint{5.548266in}{1.006536in}}{\pgfqpoint{5.545432in}{1.013378in}}{\pgfqpoint{5.540388in}{1.018421in}}%
\pgfpathcurveto{\pgfqpoint{5.535344in}{1.023465in}}{\pgfqpoint{5.528503in}{1.026299in}}{\pgfqpoint{5.521370in}{1.026299in}}%
\pgfpathcurveto{\pgfqpoint{5.514237in}{1.026299in}}{\pgfqpoint{5.507395in}{1.023465in}}{\pgfqpoint{5.502352in}{1.018421in}}%
\pgfpathcurveto{\pgfqpoint{5.497308in}{1.013378in}}{\pgfqpoint{5.494474in}{1.006536in}}{\pgfqpoint{5.494474in}{0.999403in}}%
\pgfpathcurveto{\pgfqpoint{5.494474in}{0.992270in}}{\pgfqpoint{5.497308in}{0.985429in}}{\pgfqpoint{5.502352in}{0.980385in}}%
\pgfpathcurveto{\pgfqpoint{5.507395in}{0.975341in}}{\pgfqpoint{5.514237in}{0.972507in}}{\pgfqpoint{5.521370in}{0.972507in}}%
\pgfpathclose%
\pgfusepath{stroke,fill}%
\end{pgfscope}%
\begin{pgfscope}%
\pgfpathrectangle{\pgfqpoint{4.985294in}{0.500000in}}{\pgfqpoint{1.764706in}{1.700000in}}%
\pgfusepath{clip}%
\pgfsetbuttcap%
\pgfsetroundjoin%
\definecolor{currentfill}{rgb}{0.978376,0.897317,0.831308}%
\pgfsetfillcolor{currentfill}%
\pgfsetlinewidth{0.311001pt}%
\definecolor{currentstroke}{rgb}{1.000000,1.000000,1.000000}%
\pgfsetstrokecolor{currentstroke}%
\pgfsetdash{}{0pt}%
\pgfpathmoveto{\pgfqpoint{6.280828in}{1.266007in}}%
\pgfpathcurveto{\pgfqpoint{6.287961in}{1.266007in}}{\pgfqpoint{6.294802in}{1.268841in}}{\pgfqpoint{6.299846in}{1.273885in}}%
\pgfpathcurveto{\pgfqpoint{6.304890in}{1.278928in}}{\pgfqpoint{6.307724in}{1.285770in}}{\pgfqpoint{6.307724in}{1.292903in}}%
\pgfpathcurveto{\pgfqpoint{6.307724in}{1.300036in}}{\pgfqpoint{6.304890in}{1.306877in}}{\pgfqpoint{6.299846in}{1.311921in}}%
\pgfpathcurveto{\pgfqpoint{6.294802in}{1.316965in}}{\pgfqpoint{6.287961in}{1.319799in}}{\pgfqpoint{6.280828in}{1.319799in}}%
\pgfpathcurveto{\pgfqpoint{6.273695in}{1.319799in}}{\pgfqpoint{6.266853in}{1.316965in}}{\pgfqpoint{6.261810in}{1.311921in}}%
\pgfpathcurveto{\pgfqpoint{6.256766in}{1.306877in}}{\pgfqpoint{6.253932in}{1.300036in}}{\pgfqpoint{6.253932in}{1.292903in}}%
\pgfpathcurveto{\pgfqpoint{6.253932in}{1.285770in}}{\pgfqpoint{6.256766in}{1.278928in}}{\pgfqpoint{6.261810in}{1.273885in}}%
\pgfpathcurveto{\pgfqpoint{6.266853in}{1.268841in}}{\pgfqpoint{6.273695in}{1.266007in}}{\pgfqpoint{6.280828in}{1.266007in}}%
\pgfpathclose%
\pgfusepath{stroke,fill}%
\end{pgfscope}%
\begin{pgfscope}%
\pgfpathrectangle{\pgfqpoint{4.985294in}{0.500000in}}{\pgfqpoint{1.764706in}{1.700000in}}%
\pgfusepath{clip}%
\pgfsetbuttcap%
\pgfsetroundjoin%
\definecolor{currentfill}{rgb}{0.966328,0.750560,0.616961}%
\pgfsetfillcolor{currentfill}%
\pgfsetlinewidth{0.311001pt}%
\definecolor{currentstroke}{rgb}{1.000000,1.000000,1.000000}%
\pgfsetstrokecolor{currentstroke}%
\pgfsetdash{}{0pt}%
\pgfpathmoveto{\pgfqpoint{6.192593in}{1.486773in}}%
\pgfpathcurveto{\pgfqpoint{6.199726in}{1.486773in}}{\pgfqpoint{6.206567in}{1.489607in}}{\pgfqpoint{6.211611in}{1.494651in}}%
\pgfpathcurveto{\pgfqpoint{6.216655in}{1.499695in}}{\pgfqpoint{6.219488in}{1.506536in}}{\pgfqpoint{6.219488in}{1.513669in}}%
\pgfpathcurveto{\pgfqpoint{6.219488in}{1.520802in}}{\pgfqpoint{6.216655in}{1.527644in}}{\pgfqpoint{6.211611in}{1.532687in}}%
\pgfpathcurveto{\pgfqpoint{6.206567in}{1.537731in}}{\pgfqpoint{6.199726in}{1.540565in}}{\pgfqpoint{6.192593in}{1.540565in}}%
\pgfpathcurveto{\pgfqpoint{6.185460in}{1.540565in}}{\pgfqpoint{6.178618in}{1.537731in}}{\pgfqpoint{6.173575in}{1.532687in}}%
\pgfpathcurveto{\pgfqpoint{6.168531in}{1.527644in}}{\pgfqpoint{6.165697in}{1.520802in}}{\pgfqpoint{6.165697in}{1.513669in}}%
\pgfpathcurveto{\pgfqpoint{6.165697in}{1.506536in}}{\pgfqpoint{6.168531in}{1.499695in}}{\pgfqpoint{6.173575in}{1.494651in}}%
\pgfpathcurveto{\pgfqpoint{6.178618in}{1.489607in}}{\pgfqpoint{6.185460in}{1.486773in}}{\pgfqpoint{6.192593in}{1.486773in}}%
\pgfpathclose%
\pgfusepath{stroke,fill}%
\end{pgfscope}%
\begin{pgfscope}%
\pgfpathrectangle{\pgfqpoint{4.985294in}{0.500000in}}{\pgfqpoint{1.764706in}{1.700000in}}%
\pgfusepath{clip}%
\pgfsetbuttcap%
\pgfsetroundjoin%
\definecolor{currentfill}{rgb}{0.965753,0.732351,0.592427}%
\pgfsetfillcolor{currentfill}%
\pgfsetlinewidth{0.311001pt}%
\definecolor{currentstroke}{rgb}{1.000000,1.000000,1.000000}%
\pgfsetstrokecolor{currentstroke}%
\pgfsetdash{}{0pt}%
\pgfpathmoveto{\pgfqpoint{6.404960in}{1.328449in}}%
\pgfpathcurveto{\pgfqpoint{6.412093in}{1.328449in}}{\pgfqpoint{6.418934in}{1.331282in}}{\pgfqpoint{6.423978in}{1.336326in}}%
\pgfpathcurveto{\pgfqpoint{6.429022in}{1.341370in}}{\pgfqpoint{6.431855in}{1.348211in}}{\pgfqpoint{6.431855in}{1.355344in}}%
\pgfpathcurveto{\pgfqpoint{6.431855in}{1.362477in}}{\pgfqpoint{6.429022in}{1.369319in}}{\pgfqpoint{6.423978in}{1.374362in}}%
\pgfpathcurveto{\pgfqpoint{6.418934in}{1.379406in}}{\pgfqpoint{6.412093in}{1.382240in}}{\pgfqpoint{6.404960in}{1.382240in}}%
\pgfpathcurveto{\pgfqpoint{6.397827in}{1.382240in}}{\pgfqpoint{6.390985in}{1.379406in}}{\pgfqpoint{6.385942in}{1.374362in}}%
\pgfpathcurveto{\pgfqpoint{6.380898in}{1.369319in}}{\pgfqpoint{6.378064in}{1.362477in}}{\pgfqpoint{6.378064in}{1.355344in}}%
\pgfpathcurveto{\pgfqpoint{6.378064in}{1.348211in}}{\pgfqpoint{6.380898in}{1.341370in}}{\pgfqpoint{6.385942in}{1.336326in}}%
\pgfpathcurveto{\pgfqpoint{6.390985in}{1.331282in}}{\pgfqpoint{6.397827in}{1.328449in}}{\pgfqpoint{6.404960in}{1.328449in}}%
\pgfpathclose%
\pgfusepath{stroke,fill}%
\end{pgfscope}%
\begin{pgfscope}%
\pgfpathrectangle{\pgfqpoint{4.985294in}{0.500000in}}{\pgfqpoint{1.764706in}{1.700000in}}%
\pgfusepath{clip}%
\pgfsetbuttcap%
\pgfsetroundjoin%
\definecolor{currentfill}{rgb}{0.962283,0.593046,0.431453}%
\pgfsetfillcolor{currentfill}%
\pgfsetlinewidth{0.311001pt}%
\definecolor{currentstroke}{rgb}{1.000000,1.000000,1.000000}%
\pgfsetstrokecolor{currentstroke}%
\pgfsetdash{}{0pt}%
\pgfpathmoveto{\pgfqpoint{6.278180in}{1.714686in}}%
\pgfpathcurveto{\pgfqpoint{6.285313in}{1.714686in}}{\pgfqpoint{6.292154in}{1.717520in}}{\pgfqpoint{6.297198in}{1.722564in}}%
\pgfpathcurveto{\pgfqpoint{6.302242in}{1.727607in}}{\pgfqpoint{6.305076in}{1.734449in}}{\pgfqpoint{6.305076in}{1.741582in}}%
\pgfpathcurveto{\pgfqpoint{6.305076in}{1.748715in}}{\pgfqpoint{6.302242in}{1.755556in}}{\pgfqpoint{6.297198in}{1.760600in}}%
\pgfpathcurveto{\pgfqpoint{6.292154in}{1.765644in}}{\pgfqpoint{6.285313in}{1.768478in}}{\pgfqpoint{6.278180in}{1.768478in}}%
\pgfpathcurveto{\pgfqpoint{6.271047in}{1.768478in}}{\pgfqpoint{6.264205in}{1.765644in}}{\pgfqpoint{6.259162in}{1.760600in}}%
\pgfpathcurveto{\pgfqpoint{6.254118in}{1.755556in}}{\pgfqpoint{6.251284in}{1.748715in}}{\pgfqpoint{6.251284in}{1.741582in}}%
\pgfpathcurveto{\pgfqpoint{6.251284in}{1.734449in}}{\pgfqpoint{6.254118in}{1.727607in}}{\pgfqpoint{6.259162in}{1.722564in}}%
\pgfpathcurveto{\pgfqpoint{6.264205in}{1.717520in}}{\pgfqpoint{6.271047in}{1.714686in}}{\pgfqpoint{6.278180in}{1.714686in}}%
\pgfpathclose%
\pgfusepath{stroke,fill}%
\end{pgfscope}%
\begin{pgfscope}%
\pgfpathrectangle{\pgfqpoint{4.985294in}{0.500000in}}{\pgfqpoint{1.764706in}{1.700000in}}%
\pgfusepath{clip}%
\pgfsetbuttcap%
\pgfsetroundjoin%
\definecolor{currentfill}{rgb}{0.965753,0.732351,0.592427}%
\pgfsetfillcolor{currentfill}%
\pgfsetlinewidth{0.311001pt}%
\definecolor{currentstroke}{rgb}{1.000000,1.000000,1.000000}%
\pgfsetstrokecolor{currentstroke}%
\pgfsetdash{}{0pt}%
\pgfpathmoveto{\pgfqpoint{5.450634in}{1.672618in}}%
\pgfpathcurveto{\pgfqpoint{5.457767in}{1.672618in}}{\pgfqpoint{5.464609in}{1.675452in}}{\pgfqpoint{5.469652in}{1.680495in}}%
\pgfpathcurveto{\pgfqpoint{5.474696in}{1.685539in}}{\pgfqpoint{5.477530in}{1.692381in}}{\pgfqpoint{5.477530in}{1.699513in}}%
\pgfpathcurveto{\pgfqpoint{5.477530in}{1.706646in}}{\pgfqpoint{5.474696in}{1.713488in}}{\pgfqpoint{5.469652in}{1.718532in}}%
\pgfpathcurveto{\pgfqpoint{5.464609in}{1.723575in}}{\pgfqpoint{5.457767in}{1.726409in}}{\pgfqpoint{5.450634in}{1.726409in}}%
\pgfpathcurveto{\pgfqpoint{5.443501in}{1.726409in}}{\pgfqpoint{5.436660in}{1.723575in}}{\pgfqpoint{5.431616in}{1.718532in}}%
\pgfpathcurveto{\pgfqpoint{5.426572in}{1.713488in}}{\pgfqpoint{5.423739in}{1.706646in}}{\pgfqpoint{5.423739in}{1.699513in}}%
\pgfpathcurveto{\pgfqpoint{5.423739in}{1.692381in}}{\pgfqpoint{5.426572in}{1.685539in}}{\pgfqpoint{5.431616in}{1.680495in}}%
\pgfpathcurveto{\pgfqpoint{5.436660in}{1.675452in}}{\pgfqpoint{5.443501in}{1.672618in}}{\pgfqpoint{5.450634in}{1.672618in}}%
\pgfpathclose%
\pgfusepath{stroke,fill}%
\end{pgfscope}%
\begin{pgfscope}%
\pgfpathrectangle{\pgfqpoint{4.985294in}{0.500000in}}{\pgfqpoint{1.764706in}{1.700000in}}%
\pgfusepath{clip}%
\pgfsetbuttcap%
\pgfsetroundjoin%
\definecolor{currentfill}{rgb}{0.818205,0.120806,0.299261}%
\pgfsetfillcolor{currentfill}%
\pgfsetlinewidth{0.311001pt}%
\definecolor{currentstroke}{rgb}{1.000000,1.000000,1.000000}%
\pgfsetstrokecolor{currentstroke}%
\pgfsetdash{}{0pt}%
\pgfpathmoveto{\pgfqpoint{6.309889in}{0.848409in}}%
\pgfpathcurveto{\pgfqpoint{6.317022in}{0.848409in}}{\pgfqpoint{6.323863in}{0.851243in}}{\pgfqpoint{6.328907in}{0.856287in}}%
\pgfpathcurveto{\pgfqpoint{6.333951in}{0.861330in}}{\pgfqpoint{6.336785in}{0.868172in}}{\pgfqpoint{6.336785in}{0.875305in}}%
\pgfpathcurveto{\pgfqpoint{6.336785in}{0.882438in}}{\pgfqpoint{6.333951in}{0.889279in}}{\pgfqpoint{6.328907in}{0.894323in}}%
\pgfpathcurveto{\pgfqpoint{6.323863in}{0.899367in}}{\pgfqpoint{6.317022in}{0.902200in}}{\pgfqpoint{6.309889in}{0.902200in}}%
\pgfpathcurveto{\pgfqpoint{6.302756in}{0.902200in}}{\pgfqpoint{6.295914in}{0.899367in}}{\pgfqpoint{6.290871in}{0.894323in}}%
\pgfpathcurveto{\pgfqpoint{6.285827in}{0.889279in}}{\pgfqpoint{6.282993in}{0.882438in}}{\pgfqpoint{6.282993in}{0.875305in}}%
\pgfpathcurveto{\pgfqpoint{6.282993in}{0.868172in}}{\pgfqpoint{6.285827in}{0.861330in}}{\pgfqpoint{6.290871in}{0.856287in}}%
\pgfpathcurveto{\pgfqpoint{6.295914in}{0.851243in}}{\pgfqpoint{6.302756in}{0.848409in}}{\pgfqpoint{6.309889in}{0.848409in}}%
\pgfpathclose%
\pgfusepath{stroke,fill}%
\end{pgfscope}%
\begin{pgfscope}%
\pgfpathrectangle{\pgfqpoint{4.985294in}{0.500000in}}{\pgfqpoint{1.764706in}{1.700000in}}%
\pgfusepath{clip}%
\pgfsetbuttcap%
\pgfsetroundjoin%
\definecolor{currentfill}{rgb}{0.953816,0.463738,0.317699}%
\pgfsetfillcolor{currentfill}%
\pgfsetlinewidth{0.311001pt}%
\definecolor{currentstroke}{rgb}{1.000000,1.000000,1.000000}%
\pgfsetstrokecolor{currentstroke}%
\pgfsetdash{}{0pt}%
\pgfpathmoveto{\pgfqpoint{5.565771in}{1.801365in}}%
\pgfpathcurveto{\pgfqpoint{5.572904in}{1.801365in}}{\pgfqpoint{5.579745in}{1.804199in}}{\pgfqpoint{5.584789in}{1.809242in}}%
\pgfpathcurveto{\pgfqpoint{5.589833in}{1.814286in}}{\pgfqpoint{5.592666in}{1.821128in}}{\pgfqpoint{5.592666in}{1.828260in}}%
\pgfpathcurveto{\pgfqpoint{5.592666in}{1.835393in}}{\pgfqpoint{5.589833in}{1.842235in}}{\pgfqpoint{5.584789in}{1.847279in}}%
\pgfpathcurveto{\pgfqpoint{5.579745in}{1.852322in}}{\pgfqpoint{5.572904in}{1.855156in}}{\pgfqpoint{5.565771in}{1.855156in}}%
\pgfpathcurveto{\pgfqpoint{5.558638in}{1.855156in}}{\pgfqpoint{5.551796in}{1.852322in}}{\pgfqpoint{5.546753in}{1.847279in}}%
\pgfpathcurveto{\pgfqpoint{5.541709in}{1.842235in}}{\pgfqpoint{5.538875in}{1.835393in}}{\pgfqpoint{5.538875in}{1.828260in}}%
\pgfpathcurveto{\pgfqpoint{5.538875in}{1.821128in}}{\pgfqpoint{5.541709in}{1.814286in}}{\pgfqpoint{5.546753in}{1.809242in}}%
\pgfpathcurveto{\pgfqpoint{5.551796in}{1.804199in}}{\pgfqpoint{5.558638in}{1.801365in}}{\pgfqpoint{5.565771in}{1.801365in}}%
\pgfpathclose%
\pgfusepath{stroke,fill}%
\end{pgfscope}%
\begin{pgfscope}%
\pgfpathrectangle{\pgfqpoint{4.985294in}{0.500000in}}{\pgfqpoint{1.764706in}{1.700000in}}%
\pgfusepath{clip}%
\pgfsetbuttcap%
\pgfsetroundjoin%
\definecolor{currentfill}{rgb}{0.972726,0.844889,0.754401}%
\pgfsetfillcolor{currentfill}%
\pgfsetlinewidth{0.311001pt}%
\definecolor{currentstroke}{rgb}{1.000000,1.000000,1.000000}%
\pgfsetstrokecolor{currentstroke}%
\pgfsetdash{}{0pt}%
\pgfpathmoveto{\pgfqpoint{6.349864in}{1.164701in}}%
\pgfpathcurveto{\pgfqpoint{6.356997in}{1.164701in}}{\pgfqpoint{6.363839in}{1.167535in}}{\pgfqpoint{6.368883in}{1.172579in}}%
\pgfpathcurveto{\pgfqpoint{6.373926in}{1.177623in}}{\pgfqpoint{6.376760in}{1.184464in}}{\pgfqpoint{6.376760in}{1.191597in}}%
\pgfpathcurveto{\pgfqpoint{6.376760in}{1.198730in}}{\pgfqpoint{6.373926in}{1.205571in}}{\pgfqpoint{6.368883in}{1.210615in}}%
\pgfpathcurveto{\pgfqpoint{6.363839in}{1.215659in}}{\pgfqpoint{6.356997in}{1.218493in}}{\pgfqpoint{6.349864in}{1.218493in}}%
\pgfpathcurveto{\pgfqpoint{6.342732in}{1.218493in}}{\pgfqpoint{6.335890in}{1.215659in}}{\pgfqpoint{6.330846in}{1.210615in}}%
\pgfpathcurveto{\pgfqpoint{6.325803in}{1.205571in}}{\pgfqpoint{6.322969in}{1.198730in}}{\pgfqpoint{6.322969in}{1.191597in}}%
\pgfpathcurveto{\pgfqpoint{6.322969in}{1.184464in}}{\pgfqpoint{6.325803in}{1.177623in}}{\pgfqpoint{6.330846in}{1.172579in}}%
\pgfpathcurveto{\pgfqpoint{6.335890in}{1.167535in}}{\pgfqpoint{6.342732in}{1.164701in}}{\pgfqpoint{6.349864in}{1.164701in}}%
\pgfpathclose%
\pgfusepath{stroke,fill}%
\end{pgfscope}%
\begin{pgfscope}%
\pgfpathrectangle{\pgfqpoint{4.985294in}{0.500000in}}{\pgfqpoint{1.764706in}{1.700000in}}%
\pgfusepath{clip}%
\pgfsetbuttcap%
\pgfsetroundjoin%
\definecolor{currentfill}{rgb}{0.965302,0.713942,0.568499}%
\pgfsetfillcolor{currentfill}%
\pgfsetlinewidth{0.311001pt}%
\definecolor{currentstroke}{rgb}{1.000000,1.000000,1.000000}%
\pgfsetstrokecolor{currentstroke}%
\pgfsetdash{}{0pt}%
\pgfpathmoveto{\pgfqpoint{6.402694in}{1.216596in}}%
\pgfpathcurveto{\pgfqpoint{6.409826in}{1.216596in}}{\pgfqpoint{6.416668in}{1.219430in}}{\pgfqpoint{6.421712in}{1.224474in}}%
\pgfpathcurveto{\pgfqpoint{6.426755in}{1.229517in}}{\pgfqpoint{6.429589in}{1.236359in}}{\pgfqpoint{6.429589in}{1.243492in}}%
\pgfpathcurveto{\pgfqpoint{6.429589in}{1.250625in}}{\pgfqpoint{6.426755in}{1.257466in}}{\pgfqpoint{6.421712in}{1.262510in}}%
\pgfpathcurveto{\pgfqpoint{6.416668in}{1.267554in}}{\pgfqpoint{6.409826in}{1.270388in}}{\pgfqpoint{6.402694in}{1.270388in}}%
\pgfpathcurveto{\pgfqpoint{6.395561in}{1.270388in}}{\pgfqpoint{6.388719in}{1.267554in}}{\pgfqpoint{6.383675in}{1.262510in}}%
\pgfpathcurveto{\pgfqpoint{6.378632in}{1.257466in}}{\pgfqpoint{6.375798in}{1.250625in}}{\pgfqpoint{6.375798in}{1.243492in}}%
\pgfpathcurveto{\pgfqpoint{6.375798in}{1.236359in}}{\pgfqpoint{6.378632in}{1.229517in}}{\pgfqpoint{6.383675in}{1.224474in}}%
\pgfpathcurveto{\pgfqpoint{6.388719in}{1.219430in}}{\pgfqpoint{6.395561in}{1.216596in}}{\pgfqpoint{6.402694in}{1.216596in}}%
\pgfpathclose%
\pgfusepath{stroke,fill}%
\end{pgfscope}%
\begin{pgfscope}%
\pgfpathrectangle{\pgfqpoint{4.985294in}{0.500000in}}{\pgfqpoint{1.764706in}{1.700000in}}%
\pgfusepath{clip}%
\pgfsetbuttcap%
\pgfsetroundjoin%
\definecolor{currentfill}{rgb}{0.965928,0.738443,0.600540}%
\pgfsetfillcolor{currentfill}%
\pgfsetlinewidth{0.311001pt}%
\definecolor{currentstroke}{rgb}{1.000000,1.000000,1.000000}%
\pgfsetstrokecolor{currentstroke}%
\pgfsetdash{}{0pt}%
\pgfpathmoveto{\pgfqpoint{5.572881in}{1.052576in}}%
\pgfpathcurveto{\pgfqpoint{5.580014in}{1.052576in}}{\pgfqpoint{5.586856in}{1.055410in}}{\pgfqpoint{5.591900in}{1.060453in}}%
\pgfpathcurveto{\pgfqpoint{5.596943in}{1.065497in}}{\pgfqpoint{5.599777in}{1.072339in}}{\pgfqpoint{5.599777in}{1.079471in}}%
\pgfpathcurveto{\pgfqpoint{5.599777in}{1.086604in}}{\pgfqpoint{5.596943in}{1.093446in}}{\pgfqpoint{5.591900in}{1.098490in}}%
\pgfpathcurveto{\pgfqpoint{5.586856in}{1.103533in}}{\pgfqpoint{5.580014in}{1.106367in}}{\pgfqpoint{5.572881in}{1.106367in}}%
\pgfpathcurveto{\pgfqpoint{5.565749in}{1.106367in}}{\pgfqpoint{5.558907in}{1.103533in}}{\pgfqpoint{5.553863in}{1.098490in}}%
\pgfpathcurveto{\pgfqpoint{5.548820in}{1.093446in}}{\pgfqpoint{5.545986in}{1.086604in}}{\pgfqpoint{5.545986in}{1.079471in}}%
\pgfpathcurveto{\pgfqpoint{5.545986in}{1.072339in}}{\pgfqpoint{5.548820in}{1.065497in}}{\pgfqpoint{5.553863in}{1.060453in}}%
\pgfpathcurveto{\pgfqpoint{5.558907in}{1.055410in}}{\pgfqpoint{5.565749in}{1.052576in}}{\pgfqpoint{5.572881in}{1.052576in}}%
\pgfpathclose%
\pgfusepath{stroke,fill}%
\end{pgfscope}%
\begin{pgfscope}%
\pgfpathrectangle{\pgfqpoint{4.985294in}{0.500000in}}{\pgfqpoint{1.764706in}{1.700000in}}%
\pgfusepath{clip}%
\pgfsetbuttcap%
\pgfsetroundjoin%
\definecolor{currentfill}{rgb}{0.698038,0.088972,0.346299}%
\pgfsetfillcolor{currentfill}%
\pgfsetlinewidth{0.311001pt}%
\definecolor{currentstroke}{rgb}{1.000000,1.000000,1.000000}%
\pgfsetstrokecolor{currentstroke}%
\pgfsetdash{}{0pt}%
\pgfpathmoveto{\pgfqpoint{6.403966in}{1.766837in}}%
\pgfpathcurveto{\pgfqpoint{6.411099in}{1.766837in}}{\pgfqpoint{6.417941in}{1.769671in}}{\pgfqpoint{6.422984in}{1.774714in}}%
\pgfpathcurveto{\pgfqpoint{6.428028in}{1.779758in}}{\pgfqpoint{6.430862in}{1.786600in}}{\pgfqpoint{6.430862in}{1.793733in}}%
\pgfpathcurveto{\pgfqpoint{6.430862in}{1.800865in}}{\pgfqpoint{6.428028in}{1.807707in}}{\pgfqpoint{6.422984in}{1.812751in}}%
\pgfpathcurveto{\pgfqpoint{6.417941in}{1.817794in}}{\pgfqpoint{6.411099in}{1.820628in}}{\pgfqpoint{6.403966in}{1.820628in}}%
\pgfpathcurveto{\pgfqpoint{6.396833in}{1.820628in}}{\pgfqpoint{6.389992in}{1.817794in}}{\pgfqpoint{6.384948in}{1.812751in}}%
\pgfpathcurveto{\pgfqpoint{6.379904in}{1.807707in}}{\pgfqpoint{6.377070in}{1.800865in}}{\pgfqpoint{6.377070in}{1.793733in}}%
\pgfpathcurveto{\pgfqpoint{6.377070in}{1.786600in}}{\pgfqpoint{6.379904in}{1.779758in}}{\pgfqpoint{6.384948in}{1.774714in}}%
\pgfpathcurveto{\pgfqpoint{6.389992in}{1.769671in}}{\pgfqpoint{6.396833in}{1.766837in}}{\pgfqpoint{6.403966in}{1.766837in}}%
\pgfpathclose%
\pgfusepath{stroke,fill}%
\end{pgfscope}%
\begin{pgfscope}%
\pgfpathrectangle{\pgfqpoint{4.985294in}{0.500000in}}{\pgfqpoint{1.764706in}{1.700000in}}%
\pgfusepath{clip}%
\pgfsetbuttcap%
\pgfsetroundjoin%
\definecolor{currentfill}{rgb}{0.975018,0.868213,0.788710}%
\pgfsetfillcolor{currentfill}%
\pgfsetlinewidth{0.311001pt}%
\definecolor{currentstroke}{rgb}{1.000000,1.000000,1.000000}%
\pgfsetstrokecolor{currentstroke}%
\pgfsetdash{}{0pt}%
\pgfpathmoveto{\pgfqpoint{6.278780in}{1.612755in}}%
\pgfpathcurveto{\pgfqpoint{6.285913in}{1.612755in}}{\pgfqpoint{6.292755in}{1.615589in}}{\pgfqpoint{6.297798in}{1.620633in}}%
\pgfpathcurveto{\pgfqpoint{6.302842in}{1.625676in}}{\pgfqpoint{6.305676in}{1.632518in}}{\pgfqpoint{6.305676in}{1.639651in}}%
\pgfpathcurveto{\pgfqpoint{6.305676in}{1.646784in}}{\pgfqpoint{6.302842in}{1.653625in}}{\pgfqpoint{6.297798in}{1.658669in}}%
\pgfpathcurveto{\pgfqpoint{6.292755in}{1.663713in}}{\pgfqpoint{6.285913in}{1.666547in}}{\pgfqpoint{6.278780in}{1.666547in}}%
\pgfpathcurveto{\pgfqpoint{6.271647in}{1.666547in}}{\pgfqpoint{6.264806in}{1.663713in}}{\pgfqpoint{6.259762in}{1.658669in}}%
\pgfpathcurveto{\pgfqpoint{6.254718in}{1.653625in}}{\pgfqpoint{6.251884in}{1.646784in}}{\pgfqpoint{6.251884in}{1.639651in}}%
\pgfpathcurveto{\pgfqpoint{6.251884in}{1.632518in}}{\pgfqpoint{6.254718in}{1.625676in}}{\pgfqpoint{6.259762in}{1.620633in}}%
\pgfpathcurveto{\pgfqpoint{6.264806in}{1.615589in}}{\pgfqpoint{6.271647in}{1.612755in}}{\pgfqpoint{6.278780in}{1.612755in}}%
\pgfpathclose%
\pgfusepath{stroke,fill}%
\end{pgfscope}%
\begin{pgfscope}%
\pgfpathrectangle{\pgfqpoint{4.985294in}{0.500000in}}{\pgfqpoint{1.764706in}{1.700000in}}%
\pgfusepath{clip}%
\pgfsetbuttcap%
\pgfsetroundjoin%
\definecolor{currentfill}{rgb}{0.973271,0.850724,0.762998}%
\pgfsetfillcolor{currentfill}%
\pgfsetlinewidth{0.311001pt}%
\definecolor{currentstroke}{rgb}{1.000000,1.000000,1.000000}%
\pgfsetstrokecolor{currentstroke}%
\pgfsetdash{}{0pt}%
\pgfpathmoveto{\pgfqpoint{6.241762in}{1.508968in}}%
\pgfpathcurveto{\pgfqpoint{6.248895in}{1.508968in}}{\pgfqpoint{6.255737in}{1.511802in}}{\pgfqpoint{6.260780in}{1.516846in}}%
\pgfpathcurveto{\pgfqpoint{6.265824in}{1.521889in}}{\pgfqpoint{6.268658in}{1.528731in}}{\pgfqpoint{6.268658in}{1.535864in}}%
\pgfpathcurveto{\pgfqpoint{6.268658in}{1.542996in}}{\pgfqpoint{6.265824in}{1.549838in}}{\pgfqpoint{6.260780in}{1.554882in}}%
\pgfpathcurveto{\pgfqpoint{6.255737in}{1.559925in}}{\pgfqpoint{6.248895in}{1.562759in}}{\pgfqpoint{6.241762in}{1.562759in}}%
\pgfpathcurveto{\pgfqpoint{6.234629in}{1.562759in}}{\pgfqpoint{6.227788in}{1.559925in}}{\pgfqpoint{6.222744in}{1.554882in}}%
\pgfpathcurveto{\pgfqpoint{6.217700in}{1.549838in}}{\pgfqpoint{6.214867in}{1.542996in}}{\pgfqpoint{6.214867in}{1.535864in}}%
\pgfpathcurveto{\pgfqpoint{6.214867in}{1.528731in}}{\pgfqpoint{6.217700in}{1.521889in}}{\pgfqpoint{6.222744in}{1.516846in}}%
\pgfpathcurveto{\pgfqpoint{6.227788in}{1.511802in}}{\pgfqpoint{6.234629in}{1.508968in}}{\pgfqpoint{6.241762in}{1.508968in}}%
\pgfpathclose%
\pgfusepath{stroke,fill}%
\end{pgfscope}%
\begin{pgfscope}%
\pgfpathrectangle{\pgfqpoint{4.985294in}{0.500000in}}{\pgfqpoint{1.764706in}{1.700000in}}%
\pgfusepath{clip}%
\pgfsetbuttcap%
\pgfsetroundjoin%
\definecolor{currentfill}{rgb}{0.975644,0.874038,0.797253}%
\pgfsetfillcolor{currentfill}%
\pgfsetlinewidth{0.311001pt}%
\definecolor{currentstroke}{rgb}{1.000000,1.000000,1.000000}%
\pgfsetstrokecolor{currentstroke}%
\pgfsetdash{}{0pt}%
\pgfpathmoveto{\pgfqpoint{5.467069in}{1.540785in}}%
\pgfpathcurveto{\pgfqpoint{5.474202in}{1.540785in}}{\pgfqpoint{5.481043in}{1.543619in}}{\pgfqpoint{5.486087in}{1.548662in}}%
\pgfpathcurveto{\pgfqpoint{5.491131in}{1.553706in}}{\pgfqpoint{5.493965in}{1.560548in}}{\pgfqpoint{5.493965in}{1.567680in}}%
\pgfpathcurveto{\pgfqpoint{5.493965in}{1.574813in}}{\pgfqpoint{5.491131in}{1.581655in}}{\pgfqpoint{5.486087in}{1.586699in}}%
\pgfpathcurveto{\pgfqpoint{5.481043in}{1.591742in}}{\pgfqpoint{5.474202in}{1.594576in}}{\pgfqpoint{5.467069in}{1.594576in}}%
\pgfpathcurveto{\pgfqpoint{5.459936in}{1.594576in}}{\pgfqpoint{5.453094in}{1.591742in}}{\pgfqpoint{5.448051in}{1.586699in}}%
\pgfpathcurveto{\pgfqpoint{5.443007in}{1.581655in}}{\pgfqpoint{5.440173in}{1.574813in}}{\pgfqpoint{5.440173in}{1.567680in}}%
\pgfpathcurveto{\pgfqpoint{5.440173in}{1.560548in}}{\pgfqpoint{5.443007in}{1.553706in}}{\pgfqpoint{5.448051in}{1.548662in}}%
\pgfpathcurveto{\pgfqpoint{5.453094in}{1.543619in}}{\pgfqpoint{5.459936in}{1.540785in}}{\pgfqpoint{5.467069in}{1.540785in}}%
\pgfpathclose%
\pgfusepath{stroke,fill}%
\end{pgfscope}%
\begin{pgfscope}%
\pgfpathrectangle{\pgfqpoint{4.985294in}{0.500000in}}{\pgfqpoint{1.764706in}{1.700000in}}%
\pgfusepath{clip}%
\pgfsetbuttcap%
\pgfsetroundjoin%
\definecolor{currentfill}{rgb}{0.958791,0.526283,0.368909}%
\pgfsetfillcolor{currentfill}%
\pgfsetlinewidth{0.311001pt}%
\definecolor{currentstroke}{rgb}{1.000000,1.000000,1.000000}%
\pgfsetstrokecolor{currentstroke}%
\pgfsetdash{}{0pt}%
\pgfpathmoveto{\pgfqpoint{6.292082in}{1.716619in}}%
\pgfpathcurveto{\pgfqpoint{6.299215in}{1.716619in}}{\pgfqpoint{6.306056in}{1.719453in}}{\pgfqpoint{6.311100in}{1.724497in}}%
\pgfpathcurveto{\pgfqpoint{6.316144in}{1.729540in}}{\pgfqpoint{6.318977in}{1.736382in}}{\pgfqpoint{6.318977in}{1.743515in}}%
\pgfpathcurveto{\pgfqpoint{6.318977in}{1.750648in}}{\pgfqpoint{6.316144in}{1.757489in}}{\pgfqpoint{6.311100in}{1.762533in}}%
\pgfpathcurveto{\pgfqpoint{6.306056in}{1.767577in}}{\pgfqpoint{6.299215in}{1.770411in}}{\pgfqpoint{6.292082in}{1.770411in}}%
\pgfpathcurveto{\pgfqpoint{6.284949in}{1.770411in}}{\pgfqpoint{6.278107in}{1.767577in}}{\pgfqpoint{6.273064in}{1.762533in}}%
\pgfpathcurveto{\pgfqpoint{6.268020in}{1.757489in}}{\pgfqpoint{6.265186in}{1.750648in}}{\pgfqpoint{6.265186in}{1.743515in}}%
\pgfpathcurveto{\pgfqpoint{6.265186in}{1.736382in}}{\pgfqpoint{6.268020in}{1.729540in}}{\pgfqpoint{6.273064in}{1.724497in}}%
\pgfpathcurveto{\pgfqpoint{6.278107in}{1.719453in}}{\pgfqpoint{6.284949in}{1.716619in}}{\pgfqpoint{6.292082in}{1.716619in}}%
\pgfpathclose%
\pgfusepath{stroke,fill}%
\end{pgfscope}%
\begin{pgfscope}%
\pgfpathrectangle{\pgfqpoint{4.985294in}{0.500000in}}{\pgfqpoint{1.764706in}{1.700000in}}%
\pgfusepath{clip}%
\pgfsetbuttcap%
\pgfsetroundjoin%
\definecolor{currentfill}{rgb}{0.980678,0.914765,0.856766}%
\pgfsetfillcolor{currentfill}%
\pgfsetlinewidth{0.311001pt}%
\definecolor{currentstroke}{rgb}{1.000000,1.000000,1.000000}%
\pgfsetstrokecolor{currentstroke}%
\pgfsetdash{}{0pt}%
\pgfpathmoveto{\pgfqpoint{6.287016in}{1.498160in}}%
\pgfpathcurveto{\pgfqpoint{6.294149in}{1.498160in}}{\pgfqpoint{6.300991in}{1.500994in}}{\pgfqpoint{6.306034in}{1.506038in}}%
\pgfpathcurveto{\pgfqpoint{6.311078in}{1.511081in}}{\pgfqpoint{6.313912in}{1.517923in}}{\pgfqpoint{6.313912in}{1.525056in}}%
\pgfpathcurveto{\pgfqpoint{6.313912in}{1.532188in}}{\pgfqpoint{6.311078in}{1.539030in}}{\pgfqpoint{6.306034in}{1.544074in}}%
\pgfpathcurveto{\pgfqpoint{6.300991in}{1.549117in}}{\pgfqpoint{6.294149in}{1.551951in}}{\pgfqpoint{6.287016in}{1.551951in}}%
\pgfpathcurveto{\pgfqpoint{6.279883in}{1.551951in}}{\pgfqpoint{6.273042in}{1.549117in}}{\pgfqpoint{6.267998in}{1.544074in}}%
\pgfpathcurveto{\pgfqpoint{6.262954in}{1.539030in}}{\pgfqpoint{6.260121in}{1.532188in}}{\pgfqpoint{6.260121in}{1.525056in}}%
\pgfpathcurveto{\pgfqpoint{6.260121in}{1.517923in}}{\pgfqpoint{6.262954in}{1.511081in}}{\pgfqpoint{6.267998in}{1.506038in}}%
\pgfpathcurveto{\pgfqpoint{6.273042in}{1.500994in}}{\pgfqpoint{6.279883in}{1.498160in}}{\pgfqpoint{6.287016in}{1.498160in}}%
\pgfpathclose%
\pgfusepath{stroke,fill}%
\end{pgfscope}%
\begin{pgfscope}%
\pgfpathrectangle{\pgfqpoint{4.985294in}{0.500000in}}{\pgfqpoint{1.764706in}{1.700000in}}%
\pgfusepath{clip}%
\pgfsetbuttcap%
\pgfsetroundjoin%
\definecolor{currentfill}{rgb}{0.981377,0.920617,0.865369}%
\pgfsetfillcolor{currentfill}%
\pgfsetlinewidth{0.311001pt}%
\definecolor{currentstroke}{rgb}{1.000000,1.000000,1.000000}%
\pgfsetstrokecolor{currentstroke}%
\pgfsetdash{}{0pt}%
\pgfpathmoveto{\pgfqpoint{6.299635in}{1.451417in}}%
\pgfpathcurveto{\pgfqpoint{6.306768in}{1.451417in}}{\pgfqpoint{6.313610in}{1.454251in}}{\pgfqpoint{6.318653in}{1.459295in}}%
\pgfpathcurveto{\pgfqpoint{6.323697in}{1.464339in}}{\pgfqpoint{6.326531in}{1.471180in}}{\pgfqpoint{6.326531in}{1.478313in}}%
\pgfpathcurveto{\pgfqpoint{6.326531in}{1.485446in}}{\pgfqpoint{6.323697in}{1.492288in}}{\pgfqpoint{6.318653in}{1.497331in}}%
\pgfpathcurveto{\pgfqpoint{6.313610in}{1.502375in}}{\pgfqpoint{6.306768in}{1.505209in}}{\pgfqpoint{6.299635in}{1.505209in}}%
\pgfpathcurveto{\pgfqpoint{6.292502in}{1.505209in}}{\pgfqpoint{6.285661in}{1.502375in}}{\pgfqpoint{6.280617in}{1.497331in}}%
\pgfpathcurveto{\pgfqpoint{6.275573in}{1.492288in}}{\pgfqpoint{6.272739in}{1.485446in}}{\pgfqpoint{6.272739in}{1.478313in}}%
\pgfpathcurveto{\pgfqpoint{6.272739in}{1.471180in}}{\pgfqpoint{6.275573in}{1.464339in}}{\pgfqpoint{6.280617in}{1.459295in}}%
\pgfpathcurveto{\pgfqpoint{6.285661in}{1.454251in}}{\pgfqpoint{6.292502in}{1.451417in}}{\pgfqpoint{6.299635in}{1.451417in}}%
\pgfpathclose%
\pgfusepath{stroke,fill}%
\end{pgfscope}%
\begin{pgfscope}%
\pgfpathrectangle{\pgfqpoint{4.985294in}{0.500000in}}{\pgfqpoint{1.764706in}{1.700000in}}%
\pgfusepath{clip}%
\pgfsetbuttcap%
\pgfsetroundjoin%
\definecolor{currentfill}{rgb}{0.969803,0.809811,0.702523}%
\pgfsetfillcolor{currentfill}%
\pgfsetlinewidth{0.311001pt}%
\definecolor{currentstroke}{rgb}{1.000000,1.000000,1.000000}%
\pgfsetstrokecolor{currentstroke}%
\pgfsetdash{}{0pt}%
\pgfpathmoveto{\pgfqpoint{6.203014in}{1.153181in}}%
\pgfpathcurveto{\pgfqpoint{6.210147in}{1.153181in}}{\pgfqpoint{6.216988in}{1.156015in}}{\pgfqpoint{6.222032in}{1.161059in}}%
\pgfpathcurveto{\pgfqpoint{6.227076in}{1.166103in}}{\pgfqpoint{6.229909in}{1.172944in}}{\pgfqpoint{6.229909in}{1.180077in}}%
\pgfpathcurveto{\pgfqpoint{6.229909in}{1.187210in}}{\pgfqpoint{6.227076in}{1.194052in}}{\pgfqpoint{6.222032in}{1.199095in}}%
\pgfpathcurveto{\pgfqpoint{6.216988in}{1.204139in}}{\pgfqpoint{6.210147in}{1.206973in}}{\pgfqpoint{6.203014in}{1.206973in}}%
\pgfpathcurveto{\pgfqpoint{6.195881in}{1.206973in}}{\pgfqpoint{6.189039in}{1.204139in}}{\pgfqpoint{6.183996in}{1.199095in}}%
\pgfpathcurveto{\pgfqpoint{6.178952in}{1.194052in}}{\pgfqpoint{6.176118in}{1.187210in}}{\pgfqpoint{6.176118in}{1.180077in}}%
\pgfpathcurveto{\pgfqpoint{6.176118in}{1.172944in}}{\pgfqpoint{6.178952in}{1.166103in}}{\pgfqpoint{6.183996in}{1.161059in}}%
\pgfpathcurveto{\pgfqpoint{6.189039in}{1.156015in}}{\pgfqpoint{6.195881in}{1.153181in}}{\pgfqpoint{6.203014in}{1.153181in}}%
\pgfpathclose%
\pgfusepath{stroke,fill}%
\end{pgfscope}%
\begin{pgfscope}%
\pgfpathrectangle{\pgfqpoint{4.985294in}{0.500000in}}{\pgfqpoint{1.764706in}{1.700000in}}%
\pgfusepath{clip}%
\pgfsetbuttcap%
\pgfsetroundjoin%
\definecolor{currentfill}{rgb}{0.953126,0.456614,0.312398}%
\pgfsetfillcolor{currentfill}%
\pgfsetlinewidth{0.311001pt}%
\definecolor{currentstroke}{rgb}{1.000000,1.000000,1.000000}%
\pgfsetstrokecolor{currentstroke}%
\pgfsetdash{}{0pt}%
\pgfpathmoveto{\pgfqpoint{6.214090in}{0.848006in}}%
\pgfpathcurveto{\pgfqpoint{6.221223in}{0.848006in}}{\pgfqpoint{6.228065in}{0.850840in}}{\pgfqpoint{6.233109in}{0.855883in}}%
\pgfpathcurveto{\pgfqpoint{6.238152in}{0.860927in}}{\pgfqpoint{6.240986in}{0.867769in}}{\pgfqpoint{6.240986in}{0.874901in}}%
\pgfpathcurveto{\pgfqpoint{6.240986in}{0.882034in}}{\pgfqpoint{6.238152in}{0.888876in}}{\pgfqpoint{6.233109in}{0.893920in}}%
\pgfpathcurveto{\pgfqpoint{6.228065in}{0.898963in}}{\pgfqpoint{6.221223in}{0.901797in}}{\pgfqpoint{6.214090in}{0.901797in}}%
\pgfpathcurveto{\pgfqpoint{6.206958in}{0.901797in}}{\pgfqpoint{6.200116in}{0.898963in}}{\pgfqpoint{6.195072in}{0.893920in}}%
\pgfpathcurveto{\pgfqpoint{6.190029in}{0.888876in}}{\pgfqpoint{6.187195in}{0.882034in}}{\pgfqpoint{6.187195in}{0.874901in}}%
\pgfpathcurveto{\pgfqpoint{6.187195in}{0.867769in}}{\pgfqpoint{6.190029in}{0.860927in}}{\pgfqpoint{6.195072in}{0.855883in}}%
\pgfpathcurveto{\pgfqpoint{6.200116in}{0.850840in}}{\pgfqpoint{6.206958in}{0.848006in}}{\pgfqpoint{6.214090in}{0.848006in}}%
\pgfpathclose%
\pgfusepath{stroke,fill}%
\end{pgfscope}%
\begin{pgfscope}%
\pgfpathrectangle{\pgfqpoint{4.985294in}{0.500000in}}{\pgfqpoint{1.764706in}{1.700000in}}%
\pgfusepath{clip}%
\pgfsetbuttcap%
\pgfsetroundjoin%
\definecolor{currentfill}{rgb}{0.976961,0.885681,0.814303}%
\pgfsetfillcolor{currentfill}%
\pgfsetlinewidth{0.311001pt}%
\definecolor{currentstroke}{rgb}{1.000000,1.000000,1.000000}%
\pgfsetstrokecolor{currentstroke}%
\pgfsetdash{}{0pt}%
\pgfpathmoveto{\pgfqpoint{6.319600in}{1.508657in}}%
\pgfpathcurveto{\pgfqpoint{6.326733in}{1.508657in}}{\pgfqpoint{6.333574in}{1.511490in}}{\pgfqpoint{6.338618in}{1.516534in}}%
\pgfpathcurveto{\pgfqpoint{6.343662in}{1.521578in}}{\pgfqpoint{6.346496in}{1.528419in}}{\pgfqpoint{6.346496in}{1.535552in}}%
\pgfpathcurveto{\pgfqpoint{6.346496in}{1.542685in}}{\pgfqpoint{6.343662in}{1.549527in}}{\pgfqpoint{6.338618in}{1.554570in}}%
\pgfpathcurveto{\pgfqpoint{6.333574in}{1.559614in}}{\pgfqpoint{6.326733in}{1.562448in}}{\pgfqpoint{6.319600in}{1.562448in}}%
\pgfpathcurveto{\pgfqpoint{6.312467in}{1.562448in}}{\pgfqpoint{6.305625in}{1.559614in}}{\pgfqpoint{6.300582in}{1.554570in}}%
\pgfpathcurveto{\pgfqpoint{6.295538in}{1.549527in}}{\pgfqpoint{6.292704in}{1.542685in}}{\pgfqpoint{6.292704in}{1.535552in}}%
\pgfpathcurveto{\pgfqpoint{6.292704in}{1.528419in}}{\pgfqpoint{6.295538in}{1.521578in}}{\pgfqpoint{6.300582in}{1.516534in}}%
\pgfpathcurveto{\pgfqpoint{6.305625in}{1.511490in}}{\pgfqpoint{6.312467in}{1.508657in}}{\pgfqpoint{6.319600in}{1.508657in}}%
\pgfpathclose%
\pgfusepath{stroke,fill}%
\end{pgfscope}%
\begin{pgfscope}%
\pgfpathrectangle{\pgfqpoint{4.985294in}{0.500000in}}{\pgfqpoint{1.764706in}{1.700000in}}%
\pgfusepath{clip}%
\pgfsetbuttcap%
\pgfsetroundjoin%
\definecolor{currentfill}{rgb}{0.914423,0.260289,0.243694}%
\pgfsetfillcolor{currentfill}%
\pgfsetlinewidth{0.311001pt}%
\definecolor{currentstroke}{rgb}{1.000000,1.000000,1.000000}%
\pgfsetstrokecolor{currentstroke}%
\pgfsetdash{}{0pt}%
\pgfpathmoveto{\pgfqpoint{5.749996in}{1.697010in}}%
\pgfpathcurveto{\pgfqpoint{5.757129in}{1.697010in}}{\pgfqpoint{5.763971in}{1.699844in}}{\pgfqpoint{5.769014in}{1.704888in}}%
\pgfpathcurveto{\pgfqpoint{5.774058in}{1.709932in}}{\pgfqpoint{5.776892in}{1.716773in}}{\pgfqpoint{5.776892in}{1.723906in}}%
\pgfpathcurveto{\pgfqpoint{5.776892in}{1.731039in}}{\pgfqpoint{5.774058in}{1.737880in}}{\pgfqpoint{5.769014in}{1.742924in}}%
\pgfpathcurveto{\pgfqpoint{5.763971in}{1.747968in}}{\pgfqpoint{5.757129in}{1.750802in}}{\pgfqpoint{5.749996in}{1.750802in}}%
\pgfpathcurveto{\pgfqpoint{5.742863in}{1.750802in}}{\pgfqpoint{5.736022in}{1.747968in}}{\pgfqpoint{5.730978in}{1.742924in}}%
\pgfpathcurveto{\pgfqpoint{5.725934in}{1.737880in}}{\pgfqpoint{5.723101in}{1.731039in}}{\pgfqpoint{5.723101in}{1.723906in}}%
\pgfpathcurveto{\pgfqpoint{5.723101in}{1.716773in}}{\pgfqpoint{5.725934in}{1.709932in}}{\pgfqpoint{5.730978in}{1.704888in}}%
\pgfpathcurveto{\pgfqpoint{5.736022in}{1.699844in}}{\pgfqpoint{5.742863in}{1.697010in}}{\pgfqpoint{5.749996in}{1.697010in}}%
\pgfpathclose%
\pgfusepath{stroke,fill}%
\end{pgfscope}%
\begin{pgfscope}%
\pgfpathrectangle{\pgfqpoint{4.985294in}{0.500000in}}{\pgfqpoint{1.764706in}{1.700000in}}%
\pgfusepath{clip}%
\pgfsetbuttcap%
\pgfsetroundjoin%
\definecolor{currentfill}{rgb}{0.972201,0.839051,0.745789}%
\pgfsetfillcolor{currentfill}%
\pgfsetlinewidth{0.311001pt}%
\definecolor{currentstroke}{rgb}{1.000000,1.000000,1.000000}%
\pgfsetstrokecolor{currentstroke}%
\pgfsetdash{}{0pt}%
\pgfpathmoveto{\pgfqpoint{6.206933in}{1.558982in}}%
\pgfpathcurveto{\pgfqpoint{6.214065in}{1.558982in}}{\pgfqpoint{6.220907in}{1.561816in}}{\pgfqpoint{6.225951in}{1.566860in}}%
\pgfpathcurveto{\pgfqpoint{6.230994in}{1.571903in}}{\pgfqpoint{6.233828in}{1.578745in}}{\pgfqpoint{6.233828in}{1.585878in}}%
\pgfpathcurveto{\pgfqpoint{6.233828in}{1.593011in}}{\pgfqpoint{6.230994in}{1.599852in}}{\pgfqpoint{6.225951in}{1.604896in}}%
\pgfpathcurveto{\pgfqpoint{6.220907in}{1.609940in}}{\pgfqpoint{6.214065in}{1.612774in}}{\pgfqpoint{6.206933in}{1.612774in}}%
\pgfpathcurveto{\pgfqpoint{6.199800in}{1.612774in}}{\pgfqpoint{6.192958in}{1.609940in}}{\pgfqpoint{6.187914in}{1.604896in}}%
\pgfpathcurveto{\pgfqpoint{6.182871in}{1.599852in}}{\pgfqpoint{6.180037in}{1.593011in}}{\pgfqpoint{6.180037in}{1.585878in}}%
\pgfpathcurveto{\pgfqpoint{6.180037in}{1.578745in}}{\pgfqpoint{6.182871in}{1.571903in}}{\pgfqpoint{6.187914in}{1.566860in}}%
\pgfpathcurveto{\pgfqpoint{6.192958in}{1.561816in}}{\pgfqpoint{6.199800in}{1.558982in}}{\pgfqpoint{6.206933in}{1.558982in}}%
\pgfpathclose%
\pgfusepath{stroke,fill}%
\end{pgfscope}%
\begin{pgfscope}%
\pgfpathrectangle{\pgfqpoint{4.985294in}{0.500000in}}{\pgfqpoint{1.764706in}{1.700000in}}%
\pgfusepath{clip}%
\pgfsetbuttcap%
\pgfsetroundjoin%
\definecolor{currentfill}{rgb}{0.979891,0.908948,0.848279}%
\pgfsetfillcolor{currentfill}%
\pgfsetlinewidth{0.311001pt}%
\definecolor{currentstroke}{rgb}{1.000000,1.000000,1.000000}%
\pgfsetstrokecolor{currentstroke}%
\pgfsetdash{}{0pt}%
\pgfpathmoveto{\pgfqpoint{6.292427in}{1.232070in}}%
\pgfpathcurveto{\pgfqpoint{6.299560in}{1.232070in}}{\pgfqpoint{6.306402in}{1.234904in}}{\pgfqpoint{6.311445in}{1.239947in}}%
\pgfpathcurveto{\pgfqpoint{6.316489in}{1.244991in}}{\pgfqpoint{6.319323in}{1.251833in}}{\pgfqpoint{6.319323in}{1.258966in}}%
\pgfpathcurveto{\pgfqpoint{6.319323in}{1.266098in}}{\pgfqpoint{6.316489in}{1.272940in}}{\pgfqpoint{6.311445in}{1.277984in}}%
\pgfpathcurveto{\pgfqpoint{6.306402in}{1.283027in}}{\pgfqpoint{6.299560in}{1.285861in}}{\pgfqpoint{6.292427in}{1.285861in}}%
\pgfpathcurveto{\pgfqpoint{6.285294in}{1.285861in}}{\pgfqpoint{6.278453in}{1.283027in}}{\pgfqpoint{6.273409in}{1.277984in}}%
\pgfpathcurveto{\pgfqpoint{6.268365in}{1.272940in}}{\pgfqpoint{6.265531in}{1.266098in}}{\pgfqpoint{6.265531in}{1.258966in}}%
\pgfpathcurveto{\pgfqpoint{6.265531in}{1.251833in}}{\pgfqpoint{6.268365in}{1.244991in}}{\pgfqpoint{6.273409in}{1.239947in}}%
\pgfpathcurveto{\pgfqpoint{6.278453in}{1.234904in}}{\pgfqpoint{6.285294in}{1.232070in}}{\pgfqpoint{6.292427in}{1.232070in}}%
\pgfpathclose%
\pgfusepath{stroke,fill}%
\end{pgfscope}%
\begin{pgfscope}%
\pgfpathrectangle{\pgfqpoint{4.985294in}{0.500000in}}{\pgfqpoint{1.764706in}{1.700000in}}%
\pgfusepath{clip}%
\pgfsetbuttcap%
\pgfsetroundjoin%
\definecolor{currentfill}{rgb}{0.972201,0.839051,0.745789}%
\pgfsetfillcolor{currentfill}%
\pgfsetlinewidth{0.311001pt}%
\definecolor{currentstroke}{rgb}{1.000000,1.000000,1.000000}%
\pgfsetstrokecolor{currentstroke}%
\pgfsetdash{}{0pt}%
\pgfpathmoveto{\pgfqpoint{5.516716in}{1.521466in}}%
\pgfpathcurveto{\pgfqpoint{5.523849in}{1.521466in}}{\pgfqpoint{5.530691in}{1.524300in}}{\pgfqpoint{5.535734in}{1.529344in}}%
\pgfpathcurveto{\pgfqpoint{5.540778in}{1.534387in}}{\pgfqpoint{5.543612in}{1.541229in}}{\pgfqpoint{5.543612in}{1.548362in}}%
\pgfpathcurveto{\pgfqpoint{5.543612in}{1.555495in}}{\pgfqpoint{5.540778in}{1.562336in}}{\pgfqpoint{5.535734in}{1.567380in}}%
\pgfpathcurveto{\pgfqpoint{5.530691in}{1.572424in}}{\pgfqpoint{5.523849in}{1.575257in}}{\pgfqpoint{5.516716in}{1.575257in}}%
\pgfpathcurveto{\pgfqpoint{5.509583in}{1.575257in}}{\pgfqpoint{5.502742in}{1.572424in}}{\pgfqpoint{5.497698in}{1.567380in}}%
\pgfpathcurveto{\pgfqpoint{5.492654in}{1.562336in}}{\pgfqpoint{5.489820in}{1.555495in}}{\pgfqpoint{5.489820in}{1.548362in}}%
\pgfpathcurveto{\pgfqpoint{5.489820in}{1.541229in}}{\pgfqpoint{5.492654in}{1.534387in}}{\pgfqpoint{5.497698in}{1.529344in}}%
\pgfpathcurveto{\pgfqpoint{5.502742in}{1.524300in}}{\pgfqpoint{5.509583in}{1.521466in}}{\pgfqpoint{5.516716in}{1.521466in}}%
\pgfpathclose%
\pgfusepath{stroke,fill}%
\end{pgfscope}%
\begin{pgfscope}%
\pgfpathrectangle{\pgfqpoint{4.985294in}{0.500000in}}{\pgfqpoint{1.764706in}{1.700000in}}%
\pgfusepath{clip}%
\pgfsetbuttcap%
\pgfsetroundjoin%
\definecolor{currentfill}{rgb}{0.978376,0.897317,0.831308}%
\pgfsetfillcolor{currentfill}%
\pgfsetlinewidth{0.311001pt}%
\definecolor{currentstroke}{rgb}{1.000000,1.000000,1.000000}%
\pgfsetstrokecolor{currentstroke}%
\pgfsetdash{}{0pt}%
\pgfpathmoveto{\pgfqpoint{6.280681in}{1.140479in}}%
\pgfpathcurveto{\pgfqpoint{6.287814in}{1.140479in}}{\pgfqpoint{6.294656in}{1.143312in}}{\pgfqpoint{6.299699in}{1.148356in}}%
\pgfpathcurveto{\pgfqpoint{6.304743in}{1.153400in}}{\pgfqpoint{6.307577in}{1.160241in}}{\pgfqpoint{6.307577in}{1.167374in}}%
\pgfpathcurveto{\pgfqpoint{6.307577in}{1.174507in}}{\pgfqpoint{6.304743in}{1.181349in}}{\pgfqpoint{6.299699in}{1.186392in}}%
\pgfpathcurveto{\pgfqpoint{6.294656in}{1.191436in}}{\pgfqpoint{6.287814in}{1.194270in}}{\pgfqpoint{6.280681in}{1.194270in}}%
\pgfpathcurveto{\pgfqpoint{6.273548in}{1.194270in}}{\pgfqpoint{6.266707in}{1.191436in}}{\pgfqpoint{6.261663in}{1.186392in}}%
\pgfpathcurveto{\pgfqpoint{6.256619in}{1.181349in}}{\pgfqpoint{6.253785in}{1.174507in}}{\pgfqpoint{6.253785in}{1.167374in}}%
\pgfpathcurveto{\pgfqpoint{6.253785in}{1.160241in}}{\pgfqpoint{6.256619in}{1.153400in}}{\pgfqpoint{6.261663in}{1.148356in}}%
\pgfpathcurveto{\pgfqpoint{6.266707in}{1.143312in}}{\pgfqpoint{6.273548in}{1.140479in}}{\pgfqpoint{6.280681in}{1.140479in}}%
\pgfpathclose%
\pgfusepath{stroke,fill}%
\end{pgfscope}%
\begin{pgfscope}%
\pgfpathrectangle{\pgfqpoint{4.985294in}{0.500000in}}{\pgfqpoint{1.764706in}{1.700000in}}%
\pgfusepath{clip}%
\pgfsetbuttcap%
\pgfsetroundjoin%
\definecolor{currentfill}{rgb}{0.966328,0.750560,0.616961}%
\pgfsetfillcolor{currentfill}%
\pgfsetlinewidth{0.311001pt}%
\definecolor{currentstroke}{rgb}{1.000000,1.000000,1.000000}%
\pgfsetstrokecolor{currentstroke}%
\pgfsetdash{}{0pt}%
\pgfpathmoveto{\pgfqpoint{5.544127in}{0.923459in}}%
\pgfpathcurveto{\pgfqpoint{5.551260in}{0.923459in}}{\pgfqpoint{5.558102in}{0.926293in}}{\pgfqpoint{5.563145in}{0.931336in}}%
\pgfpathcurveto{\pgfqpoint{5.568189in}{0.936380in}}{\pgfqpoint{5.571023in}{0.943222in}}{\pgfqpoint{5.571023in}{0.950354in}}%
\pgfpathcurveto{\pgfqpoint{5.571023in}{0.957487in}}{\pgfqpoint{5.568189in}{0.964329in}}{\pgfqpoint{5.563145in}{0.969373in}}%
\pgfpathcurveto{\pgfqpoint{5.558102in}{0.974416in}}{\pgfqpoint{5.551260in}{0.977250in}}{\pgfqpoint{5.544127in}{0.977250in}}%
\pgfpathcurveto{\pgfqpoint{5.536994in}{0.977250in}}{\pgfqpoint{5.530153in}{0.974416in}}{\pgfqpoint{5.525109in}{0.969373in}}%
\pgfpathcurveto{\pgfqpoint{5.520065in}{0.964329in}}{\pgfqpoint{5.517231in}{0.957487in}}{\pgfqpoint{5.517231in}{0.950354in}}%
\pgfpathcurveto{\pgfqpoint{5.517231in}{0.943222in}}{\pgfqpoint{5.520065in}{0.936380in}}{\pgfqpoint{5.525109in}{0.931336in}}%
\pgfpathcurveto{\pgfqpoint{5.530153in}{0.926293in}}{\pgfqpoint{5.536994in}{0.923459in}}{\pgfqpoint{5.544127in}{0.923459in}}%
\pgfpathclose%
\pgfusepath{stroke,fill}%
\end{pgfscope}%
\begin{pgfscope}%
\pgfpathrectangle{\pgfqpoint{4.985294in}{0.500000in}}{\pgfqpoint{1.764706in}{1.700000in}}%
\pgfusepath{clip}%
\pgfsetbuttcap%
\pgfsetroundjoin%
\definecolor{currentfill}{rgb}{0.942910,0.375495,0.263698}%
\pgfsetfillcolor{currentfill}%
\pgfsetlinewidth{0.311001pt}%
\definecolor{currentstroke}{rgb}{1.000000,1.000000,1.000000}%
\pgfsetstrokecolor{currentstroke}%
\pgfsetdash{}{0pt}%
\pgfpathmoveto{\pgfqpoint{6.022663in}{0.908866in}}%
\pgfpathcurveto{\pgfqpoint{6.029796in}{0.908866in}}{\pgfqpoint{6.036638in}{0.911700in}}{\pgfqpoint{6.041681in}{0.916743in}}%
\pgfpathcurveto{\pgfqpoint{6.046725in}{0.921787in}}{\pgfqpoint{6.049559in}{0.928629in}}{\pgfqpoint{6.049559in}{0.935762in}}%
\pgfpathcurveto{\pgfqpoint{6.049559in}{0.942894in}}{\pgfqpoint{6.046725in}{0.949736in}}{\pgfqpoint{6.041681in}{0.954780in}}%
\pgfpathcurveto{\pgfqpoint{6.036638in}{0.959823in}}{\pgfqpoint{6.029796in}{0.962657in}}{\pgfqpoint{6.022663in}{0.962657in}}%
\pgfpathcurveto{\pgfqpoint{6.015530in}{0.962657in}}{\pgfqpoint{6.008689in}{0.959823in}}{\pgfqpoint{6.003645in}{0.954780in}}%
\pgfpathcurveto{\pgfqpoint{5.998601in}{0.949736in}}{\pgfqpoint{5.995767in}{0.942894in}}{\pgfqpoint{5.995767in}{0.935762in}}%
\pgfpathcurveto{\pgfqpoint{5.995767in}{0.928629in}}{\pgfqpoint{5.998601in}{0.921787in}}{\pgfqpoint{6.003645in}{0.916743in}}%
\pgfpathcurveto{\pgfqpoint{6.008689in}{0.911700in}}{\pgfqpoint{6.015530in}{0.908866in}}{\pgfqpoint{6.022663in}{0.908866in}}%
\pgfpathclose%
\pgfusepath{stroke,fill}%
\end{pgfscope}%
\begin{pgfscope}%
\pgfpathrectangle{\pgfqpoint{4.985294in}{0.500000in}}{\pgfqpoint{1.764706in}{1.700000in}}%
\pgfusepath{clip}%
\pgfsetbuttcap%
\pgfsetroundjoin%
\definecolor{currentfill}{rgb}{0.948235,0.413004,0.283323}%
\pgfsetfillcolor{currentfill}%
\pgfsetlinewidth{0.311001pt}%
\definecolor{currentstroke}{rgb}{1.000000,1.000000,1.000000}%
\pgfsetstrokecolor{currentstroke}%
\pgfsetdash{}{0pt}%
\pgfpathmoveto{\pgfqpoint{6.350717in}{1.697460in}}%
\pgfpathcurveto{\pgfqpoint{6.357849in}{1.697460in}}{\pgfqpoint{6.364691in}{1.700294in}}{\pgfqpoint{6.369735in}{1.705337in}}%
\pgfpathcurveto{\pgfqpoint{6.374778in}{1.710381in}}{\pgfqpoint{6.377612in}{1.717223in}}{\pgfqpoint{6.377612in}{1.724355in}}%
\pgfpathcurveto{\pgfqpoint{6.377612in}{1.731488in}}{\pgfqpoint{6.374778in}{1.738330in}}{\pgfqpoint{6.369735in}{1.743374in}}%
\pgfpathcurveto{\pgfqpoint{6.364691in}{1.748417in}}{\pgfqpoint{6.357849in}{1.751251in}}{\pgfqpoint{6.350717in}{1.751251in}}%
\pgfpathcurveto{\pgfqpoint{6.343584in}{1.751251in}}{\pgfqpoint{6.336742in}{1.748417in}}{\pgfqpoint{6.331698in}{1.743374in}}%
\pgfpathcurveto{\pgfqpoint{6.326655in}{1.738330in}}{\pgfqpoint{6.323821in}{1.731488in}}{\pgfqpoint{6.323821in}{1.724355in}}%
\pgfpathcurveto{\pgfqpoint{6.323821in}{1.717223in}}{\pgfqpoint{6.326655in}{1.710381in}}{\pgfqpoint{6.331698in}{1.705337in}}%
\pgfpathcurveto{\pgfqpoint{6.336742in}{1.700294in}}{\pgfqpoint{6.343584in}{1.697460in}}{\pgfqpoint{6.350717in}{1.697460in}}%
\pgfpathclose%
\pgfusepath{stroke,fill}%
\end{pgfscope}%
\begin{pgfscope}%
\pgfpathrectangle{\pgfqpoint{4.985294in}{0.500000in}}{\pgfqpoint{1.764706in}{1.700000in}}%
\pgfusepath{clip}%
\pgfsetbuttcap%
\pgfsetroundjoin%
\definecolor{currentfill}{rgb}{0.922239,0.282873,0.242296}%
\pgfsetfillcolor{currentfill}%
\pgfsetlinewidth{0.311001pt}%
\definecolor{currentstroke}{rgb}{1.000000,1.000000,1.000000}%
\pgfsetstrokecolor{currentstroke}%
\pgfsetdash{}{0pt}%
\pgfpathmoveto{\pgfqpoint{6.178689in}{0.803970in}}%
\pgfpathcurveto{\pgfqpoint{6.185821in}{0.803970in}}{\pgfqpoint{6.192663in}{0.806804in}}{\pgfqpoint{6.197707in}{0.811848in}}%
\pgfpathcurveto{\pgfqpoint{6.202750in}{0.816891in}}{\pgfqpoint{6.205584in}{0.823733in}}{\pgfqpoint{6.205584in}{0.830866in}}%
\pgfpathcurveto{\pgfqpoint{6.205584in}{0.837999in}}{\pgfqpoint{6.202750in}{0.844840in}}{\pgfqpoint{6.197707in}{0.849884in}}%
\pgfpathcurveto{\pgfqpoint{6.192663in}{0.854928in}}{\pgfqpoint{6.185821in}{0.857762in}}{\pgfqpoint{6.178689in}{0.857762in}}%
\pgfpathcurveto{\pgfqpoint{6.171556in}{0.857762in}}{\pgfqpoint{6.164714in}{0.854928in}}{\pgfqpoint{6.159670in}{0.849884in}}%
\pgfpathcurveto{\pgfqpoint{6.154627in}{0.844840in}}{\pgfqpoint{6.151793in}{0.837999in}}{\pgfqpoint{6.151793in}{0.830866in}}%
\pgfpathcurveto{\pgfqpoint{6.151793in}{0.823733in}}{\pgfqpoint{6.154627in}{0.816891in}}{\pgfqpoint{6.159670in}{0.811848in}}%
\pgfpathcurveto{\pgfqpoint{6.164714in}{0.806804in}}{\pgfqpoint{6.171556in}{0.803970in}}{\pgfqpoint{6.178689in}{0.803970in}}%
\pgfpathclose%
\pgfusepath{stroke,fill}%
\end{pgfscope}%
\begin{pgfscope}%
\pgfpathrectangle{\pgfqpoint{4.985294in}{0.500000in}}{\pgfqpoint{1.764706in}{1.700000in}}%
\pgfusepath{clip}%
\pgfsetbuttcap%
\pgfsetroundjoin%
\definecolor{currentfill}{rgb}{0.970255,0.815666,0.711203}%
\pgfsetfillcolor{currentfill}%
\pgfsetlinewidth{0.311001pt}%
\definecolor{currentstroke}{rgb}{1.000000,1.000000,1.000000}%
\pgfsetstrokecolor{currentstroke}%
\pgfsetdash{}{0pt}%
\pgfpathmoveto{\pgfqpoint{6.382075in}{1.255915in}}%
\pgfpathcurveto{\pgfqpoint{6.389208in}{1.255915in}}{\pgfqpoint{6.396050in}{1.258749in}}{\pgfqpoint{6.401094in}{1.263793in}}%
\pgfpathcurveto{\pgfqpoint{6.406137in}{1.268837in}}{\pgfqpoint{6.408971in}{1.275678in}}{\pgfqpoint{6.408971in}{1.282811in}}%
\pgfpathcurveto{\pgfqpoint{6.408971in}{1.289944in}}{\pgfqpoint{6.406137in}{1.296785in}}{\pgfqpoint{6.401094in}{1.301829in}}%
\pgfpathcurveto{\pgfqpoint{6.396050in}{1.306873in}}{\pgfqpoint{6.389208in}{1.309707in}}{\pgfqpoint{6.382075in}{1.309707in}}%
\pgfpathcurveto{\pgfqpoint{6.374943in}{1.309707in}}{\pgfqpoint{6.368101in}{1.306873in}}{\pgfqpoint{6.363057in}{1.301829in}}%
\pgfpathcurveto{\pgfqpoint{6.358014in}{1.296785in}}{\pgfqpoint{6.355180in}{1.289944in}}{\pgfqpoint{6.355180in}{1.282811in}}%
\pgfpathcurveto{\pgfqpoint{6.355180in}{1.275678in}}{\pgfqpoint{6.358014in}{1.268837in}}{\pgfqpoint{6.363057in}{1.263793in}}%
\pgfpathcurveto{\pgfqpoint{6.368101in}{1.258749in}}{\pgfqpoint{6.374943in}{1.255915in}}{\pgfqpoint{6.382075in}{1.255915in}}%
\pgfpathclose%
\pgfusepath{stroke,fill}%
\end{pgfscope}%
\begin{pgfscope}%
\pgfpathrectangle{\pgfqpoint{4.985294in}{0.500000in}}{\pgfqpoint{1.764706in}{1.700000in}}%
\pgfusepath{clip}%
\pgfsetbuttcap%
\pgfsetroundjoin%
\definecolor{currentfill}{rgb}{0.966328,0.750560,0.616961}%
\pgfsetfillcolor{currentfill}%
\pgfsetlinewidth{0.311001pt}%
\definecolor{currentstroke}{rgb}{1.000000,1.000000,1.000000}%
\pgfsetstrokecolor{currentstroke}%
\pgfsetdash{}{0pt}%
\pgfpathmoveto{\pgfqpoint{5.522353in}{1.699880in}}%
\pgfpathcurveto{\pgfqpoint{5.529485in}{1.699880in}}{\pgfqpoint{5.536327in}{1.702714in}}{\pgfqpoint{5.541371in}{1.707757in}}%
\pgfpathcurveto{\pgfqpoint{5.546414in}{1.712801in}}{\pgfqpoint{5.549248in}{1.719643in}}{\pgfqpoint{5.549248in}{1.726776in}}%
\pgfpathcurveto{\pgfqpoint{5.549248in}{1.733908in}}{\pgfqpoint{5.546414in}{1.740750in}}{\pgfqpoint{5.541371in}{1.745794in}}%
\pgfpathcurveto{\pgfqpoint{5.536327in}{1.750837in}}{\pgfqpoint{5.529485in}{1.753671in}}{\pgfqpoint{5.522353in}{1.753671in}}%
\pgfpathcurveto{\pgfqpoint{5.515220in}{1.753671in}}{\pgfqpoint{5.508378in}{1.750837in}}{\pgfqpoint{5.503334in}{1.745794in}}%
\pgfpathcurveto{\pgfqpoint{5.498291in}{1.740750in}}{\pgfqpoint{5.495457in}{1.733908in}}{\pgfqpoint{5.495457in}{1.726776in}}%
\pgfpathcurveto{\pgfqpoint{5.495457in}{1.719643in}}{\pgfqpoint{5.498291in}{1.712801in}}{\pgfqpoint{5.503334in}{1.707757in}}%
\pgfpathcurveto{\pgfqpoint{5.508378in}{1.702714in}}{\pgfqpoint{5.515220in}{1.699880in}}{\pgfqpoint{5.522353in}{1.699880in}}%
\pgfpathclose%
\pgfusepath{stroke,fill}%
\end{pgfscope}%
\begin{pgfscope}%
\pgfpathrectangle{\pgfqpoint{4.985294in}{0.500000in}}{\pgfqpoint{1.764706in}{1.700000in}}%
\pgfusepath{clip}%
\pgfsetbuttcap%
\pgfsetroundjoin%
\definecolor{currentfill}{rgb}{0.976287,0.879862,0.805788}%
\pgfsetfillcolor{currentfill}%
\pgfsetlinewidth{0.311001pt}%
\definecolor{currentstroke}{rgb}{1.000000,1.000000,1.000000}%
\pgfsetstrokecolor{currentstroke}%
\pgfsetdash{}{0pt}%
\pgfpathmoveto{\pgfqpoint{6.354500in}{1.377516in}}%
\pgfpathcurveto{\pgfqpoint{6.361633in}{1.377516in}}{\pgfqpoint{6.368475in}{1.380349in}}{\pgfqpoint{6.373518in}{1.385393in}}%
\pgfpathcurveto{\pgfqpoint{6.378562in}{1.390437in}}{\pgfqpoint{6.381396in}{1.397278in}}{\pgfqpoint{6.381396in}{1.404411in}}%
\pgfpathcurveto{\pgfqpoint{6.381396in}{1.411544in}}{\pgfqpoint{6.378562in}{1.418386in}}{\pgfqpoint{6.373518in}{1.423429in}}%
\pgfpathcurveto{\pgfqpoint{6.368475in}{1.428473in}}{\pgfqpoint{6.361633in}{1.431307in}}{\pgfqpoint{6.354500in}{1.431307in}}%
\pgfpathcurveto{\pgfqpoint{6.347367in}{1.431307in}}{\pgfqpoint{6.340526in}{1.428473in}}{\pgfqpoint{6.335482in}{1.423429in}}%
\pgfpathcurveto{\pgfqpoint{6.330438in}{1.418386in}}{\pgfqpoint{6.327604in}{1.411544in}}{\pgfqpoint{6.327604in}{1.404411in}}%
\pgfpathcurveto{\pgfqpoint{6.327604in}{1.397278in}}{\pgfqpoint{6.330438in}{1.390437in}}{\pgfqpoint{6.335482in}{1.385393in}}%
\pgfpathcurveto{\pgfqpoint{6.340526in}{1.380349in}}{\pgfqpoint{6.347367in}{1.377516in}}{\pgfqpoint{6.354500in}{1.377516in}}%
\pgfpathclose%
\pgfusepath{stroke,fill}%
\end{pgfscope}%
\begin{pgfscope}%
\pgfpathrectangle{\pgfqpoint{4.985294in}{0.500000in}}{\pgfqpoint{1.764706in}{1.700000in}}%
\pgfusepath{clip}%
\pgfsetbuttcap%
\pgfsetroundjoin%
\definecolor{currentfill}{rgb}{0.971202,0.827364,0.728520}%
\pgfsetfillcolor{currentfill}%
\pgfsetlinewidth{0.311001pt}%
\definecolor{currentstroke}{rgb}{1.000000,1.000000,1.000000}%
\pgfsetstrokecolor{currentstroke}%
\pgfsetdash{}{0pt}%
\pgfpathmoveto{\pgfqpoint{5.505369in}{1.624877in}}%
\pgfpathcurveto{\pgfqpoint{5.512502in}{1.624877in}}{\pgfqpoint{5.519343in}{1.627711in}}{\pgfqpoint{5.524387in}{1.632755in}}%
\pgfpathcurveto{\pgfqpoint{5.529431in}{1.637798in}}{\pgfqpoint{5.532264in}{1.644640in}}{\pgfqpoint{5.532264in}{1.651773in}}%
\pgfpathcurveto{\pgfqpoint{5.532264in}{1.658906in}}{\pgfqpoint{5.529431in}{1.665747in}}{\pgfqpoint{5.524387in}{1.670791in}}%
\pgfpathcurveto{\pgfqpoint{5.519343in}{1.675835in}}{\pgfqpoint{5.512502in}{1.678669in}}{\pgfqpoint{5.505369in}{1.678669in}}%
\pgfpathcurveto{\pgfqpoint{5.498236in}{1.678669in}}{\pgfqpoint{5.491394in}{1.675835in}}{\pgfqpoint{5.486351in}{1.670791in}}%
\pgfpathcurveto{\pgfqpoint{5.481307in}{1.665747in}}{\pgfqpoint{5.478473in}{1.658906in}}{\pgfqpoint{5.478473in}{1.651773in}}%
\pgfpathcurveto{\pgfqpoint{5.478473in}{1.644640in}}{\pgfqpoint{5.481307in}{1.637798in}}{\pgfqpoint{5.486351in}{1.632755in}}%
\pgfpathcurveto{\pgfqpoint{5.491394in}{1.627711in}}{\pgfqpoint{5.498236in}{1.624877in}}{\pgfqpoint{5.505369in}{1.624877in}}%
\pgfpathclose%
\pgfusepath{stroke,fill}%
\end{pgfscope}%
\begin{pgfscope}%
\pgfpathrectangle{\pgfqpoint{4.985294in}{0.500000in}}{\pgfqpoint{1.764706in}{1.700000in}}%
\pgfusepath{clip}%
\pgfsetbuttcap%
\pgfsetroundjoin%
\definecolor{currentfill}{rgb}{0.971694,0.833208,0.737161}%
\pgfsetfillcolor{currentfill}%
\pgfsetlinewidth{0.311001pt}%
\definecolor{currentstroke}{rgb}{1.000000,1.000000,1.000000}%
\pgfsetstrokecolor{currentstroke}%
\pgfsetdash{}{0pt}%
\pgfpathmoveto{\pgfqpoint{6.358182in}{1.450964in}}%
\pgfpathcurveto{\pgfqpoint{6.365315in}{1.450964in}}{\pgfqpoint{6.372157in}{1.453797in}}{\pgfqpoint{6.377200in}{1.458841in}}%
\pgfpathcurveto{\pgfqpoint{6.382244in}{1.463885in}}{\pgfqpoint{6.385078in}{1.470726in}}{\pgfqpoint{6.385078in}{1.477859in}}%
\pgfpathcurveto{\pgfqpoint{6.385078in}{1.484992in}}{\pgfqpoint{6.382244in}{1.491834in}}{\pgfqpoint{6.377200in}{1.496877in}}%
\pgfpathcurveto{\pgfqpoint{6.372157in}{1.501921in}}{\pgfqpoint{6.365315in}{1.504755in}}{\pgfqpoint{6.358182in}{1.504755in}}%
\pgfpathcurveto{\pgfqpoint{6.351049in}{1.504755in}}{\pgfqpoint{6.344208in}{1.501921in}}{\pgfqpoint{6.339164in}{1.496877in}}%
\pgfpathcurveto{\pgfqpoint{6.334120in}{1.491834in}}{\pgfqpoint{6.331287in}{1.484992in}}{\pgfqpoint{6.331287in}{1.477859in}}%
\pgfpathcurveto{\pgfqpoint{6.331287in}{1.470726in}}{\pgfqpoint{6.334120in}{1.463885in}}{\pgfqpoint{6.339164in}{1.458841in}}%
\pgfpathcurveto{\pgfqpoint{6.344208in}{1.453797in}}{\pgfqpoint{6.351049in}{1.450964in}}{\pgfqpoint{6.358182in}{1.450964in}}%
\pgfpathclose%
\pgfusepath{stroke,fill}%
\end{pgfscope}%
\begin{pgfscope}%
\pgfpathrectangle{\pgfqpoint{4.985294in}{0.500000in}}{\pgfqpoint{1.764706in}{1.700000in}}%
\pgfusepath{clip}%
\pgfsetbuttcap%
\pgfsetroundjoin%
\definecolor{currentfill}{rgb}{0.976961,0.885681,0.814303}%
\pgfsetfillcolor{currentfill}%
\pgfsetlinewidth{0.311001pt}%
\definecolor{currentstroke}{rgb}{1.000000,1.000000,1.000000}%
\pgfsetstrokecolor{currentstroke}%
\pgfsetdash{}{0pt}%
\pgfpathmoveto{\pgfqpoint{5.459986in}{1.188632in}}%
\pgfpathcurveto{\pgfqpoint{5.467119in}{1.188632in}}{\pgfqpoint{5.473961in}{1.191466in}}{\pgfqpoint{5.479005in}{1.196510in}}%
\pgfpathcurveto{\pgfqpoint{5.484048in}{1.201553in}}{\pgfqpoint{5.486882in}{1.208395in}}{\pgfqpoint{5.486882in}{1.215528in}}%
\pgfpathcurveto{\pgfqpoint{5.486882in}{1.222661in}}{\pgfqpoint{5.484048in}{1.229502in}}{\pgfqpoint{5.479005in}{1.234546in}}%
\pgfpathcurveto{\pgfqpoint{5.473961in}{1.239590in}}{\pgfqpoint{5.467119in}{1.242423in}}{\pgfqpoint{5.459986in}{1.242423in}}%
\pgfpathcurveto{\pgfqpoint{5.452854in}{1.242423in}}{\pgfqpoint{5.446012in}{1.239590in}}{\pgfqpoint{5.440968in}{1.234546in}}%
\pgfpathcurveto{\pgfqpoint{5.435925in}{1.229502in}}{\pgfqpoint{5.433091in}{1.222661in}}{\pgfqpoint{5.433091in}{1.215528in}}%
\pgfpathcurveto{\pgfqpoint{5.433091in}{1.208395in}}{\pgfqpoint{5.435925in}{1.201553in}}{\pgfqpoint{5.440968in}{1.196510in}}%
\pgfpathcurveto{\pgfqpoint{5.446012in}{1.191466in}}{\pgfqpoint{5.452854in}{1.188632in}}{\pgfqpoint{5.459986in}{1.188632in}}%
\pgfpathclose%
\pgfusepath{stroke,fill}%
\end{pgfscope}%
\begin{pgfscope}%
\pgfpathrectangle{\pgfqpoint{4.985294in}{0.500000in}}{\pgfqpoint{1.764706in}{1.700000in}}%
\pgfusepath{clip}%
\pgfsetbuttcap%
\pgfsetroundjoin%
\definecolor{currentfill}{rgb}{0.432143,0.121800,0.339663}%
\pgfsetfillcolor{currentfill}%
\pgfsetlinewidth{0.311001pt}%
\definecolor{currentstroke}{rgb}{1.000000,1.000000,1.000000}%
\pgfsetstrokecolor{currentstroke}%
\pgfsetdash{}{0pt}%
\pgfpathmoveto{\pgfqpoint{5.667445in}{1.287658in}}%
\pgfpathcurveto{\pgfqpoint{5.674578in}{1.287658in}}{\pgfqpoint{5.681419in}{1.290492in}}{\pgfqpoint{5.686463in}{1.295535in}}%
\pgfpathcurveto{\pgfqpoint{5.691507in}{1.300579in}}{\pgfqpoint{5.694341in}{1.307421in}}{\pgfqpoint{5.694341in}{1.314554in}}%
\pgfpathcurveto{\pgfqpoint{5.694341in}{1.321686in}}{\pgfqpoint{5.691507in}{1.328528in}}{\pgfqpoint{5.686463in}{1.333572in}}%
\pgfpathcurveto{\pgfqpoint{5.681419in}{1.338615in}}{\pgfqpoint{5.674578in}{1.341449in}}{\pgfqpoint{5.667445in}{1.341449in}}%
\pgfpathcurveto{\pgfqpoint{5.660312in}{1.341449in}}{\pgfqpoint{5.653470in}{1.338615in}}{\pgfqpoint{5.648427in}{1.333572in}}%
\pgfpathcurveto{\pgfqpoint{5.643383in}{1.328528in}}{\pgfqpoint{5.640549in}{1.321686in}}{\pgfqpoint{5.640549in}{1.314554in}}%
\pgfpathcurveto{\pgfqpoint{5.640549in}{1.307421in}}{\pgfqpoint{5.643383in}{1.300579in}}{\pgfqpoint{5.648427in}{1.295535in}}%
\pgfpathcurveto{\pgfqpoint{5.653470in}{1.290492in}}{\pgfqpoint{5.660312in}{1.287658in}}{\pgfqpoint{5.667445in}{1.287658in}}%
\pgfpathclose%
\pgfusepath{stroke,fill}%
\end{pgfscope}%
\begin{pgfscope}%
\pgfpathrectangle{\pgfqpoint{4.985294in}{0.500000in}}{\pgfqpoint{1.764706in}{1.700000in}}%
\pgfusepath{clip}%
\pgfsetbuttcap%
\pgfsetroundjoin%
\definecolor{currentfill}{rgb}{0.979891,0.908948,0.848279}%
\pgfsetfillcolor{currentfill}%
\pgfsetlinewidth{0.311001pt}%
\definecolor{currentstroke}{rgb}{1.000000,1.000000,1.000000}%
\pgfsetstrokecolor{currentstroke}%
\pgfsetdash{}{0pt}%
\pgfpathmoveto{\pgfqpoint{5.410433in}{1.256706in}}%
\pgfpathcurveto{\pgfqpoint{5.417566in}{1.256706in}}{\pgfqpoint{5.424407in}{1.259540in}}{\pgfqpoint{5.429451in}{1.264583in}}%
\pgfpathcurveto{\pgfqpoint{5.434495in}{1.269627in}}{\pgfqpoint{5.437328in}{1.276469in}}{\pgfqpoint{5.437328in}{1.283602in}}%
\pgfpathcurveto{\pgfqpoint{5.437328in}{1.290734in}}{\pgfqpoint{5.434495in}{1.297576in}}{\pgfqpoint{5.429451in}{1.302620in}}%
\pgfpathcurveto{\pgfqpoint{5.424407in}{1.307663in}}{\pgfqpoint{5.417566in}{1.310497in}}{\pgfqpoint{5.410433in}{1.310497in}}%
\pgfpathcurveto{\pgfqpoint{5.403300in}{1.310497in}}{\pgfqpoint{5.396458in}{1.307663in}}{\pgfqpoint{5.391415in}{1.302620in}}%
\pgfpathcurveto{\pgfqpoint{5.386371in}{1.297576in}}{\pgfqpoint{5.383537in}{1.290734in}}{\pgfqpoint{5.383537in}{1.283602in}}%
\pgfpathcurveto{\pgfqpoint{5.383537in}{1.276469in}}{\pgfqpoint{5.386371in}{1.269627in}}{\pgfqpoint{5.391415in}{1.264583in}}%
\pgfpathcurveto{\pgfqpoint{5.396458in}{1.259540in}}{\pgfqpoint{5.403300in}{1.256706in}}{\pgfqpoint{5.410433in}{1.256706in}}%
\pgfpathclose%
\pgfusepath{stroke,fill}%
\end{pgfscope}%
\begin{pgfscope}%
\pgfpathrectangle{\pgfqpoint{4.985294in}{0.500000in}}{\pgfqpoint{1.764706in}{1.700000in}}%
\pgfusepath{clip}%
\pgfsetbuttcap%
\pgfsetroundjoin%
\definecolor{currentfill}{rgb}{0.978376,0.897317,0.831308}%
\pgfsetfillcolor{currentfill}%
\pgfsetlinewidth{0.311001pt}%
\definecolor{currentstroke}{rgb}{1.000000,1.000000,1.000000}%
\pgfsetstrokecolor{currentstroke}%
\pgfsetdash{}{0pt}%
\pgfpathmoveto{\pgfqpoint{5.448028in}{1.431188in}}%
\pgfpathcurveto{\pgfqpoint{5.455161in}{1.431188in}}{\pgfqpoint{5.462003in}{1.434022in}}{\pgfqpoint{5.467047in}{1.439066in}}%
\pgfpathcurveto{\pgfqpoint{5.472090in}{1.444109in}}{\pgfqpoint{5.474924in}{1.450951in}}{\pgfqpoint{5.474924in}{1.458084in}}%
\pgfpathcurveto{\pgfqpoint{5.474924in}{1.465217in}}{\pgfqpoint{5.472090in}{1.472058in}}{\pgfqpoint{5.467047in}{1.477102in}}%
\pgfpathcurveto{\pgfqpoint{5.462003in}{1.482146in}}{\pgfqpoint{5.455161in}{1.484980in}}{\pgfqpoint{5.448028in}{1.484980in}}%
\pgfpathcurveto{\pgfqpoint{5.440896in}{1.484980in}}{\pgfqpoint{5.434054in}{1.482146in}}{\pgfqpoint{5.429010in}{1.477102in}}%
\pgfpathcurveto{\pgfqpoint{5.423967in}{1.472058in}}{\pgfqpoint{5.421133in}{1.465217in}}{\pgfqpoint{5.421133in}{1.458084in}}%
\pgfpathcurveto{\pgfqpoint{5.421133in}{1.450951in}}{\pgfqpoint{5.423967in}{1.444109in}}{\pgfqpoint{5.429010in}{1.439066in}}%
\pgfpathcurveto{\pgfqpoint{5.434054in}{1.434022in}}{\pgfqpoint{5.440896in}{1.431188in}}{\pgfqpoint{5.448028in}{1.431188in}}%
\pgfpathclose%
\pgfusepath{stroke,fill}%
\end{pgfscope}%
\begin{pgfscope}%
\pgfpathrectangle{\pgfqpoint{4.985294in}{0.500000in}}{\pgfqpoint{1.764706in}{1.700000in}}%
\pgfusepath{clip}%
\pgfsetbuttcap%
\pgfsetroundjoin%
\definecolor{currentfill}{rgb}{0.966328,0.750560,0.616961}%
\pgfsetfillcolor{currentfill}%
\pgfsetlinewidth{0.311001pt}%
\definecolor{currentstroke}{rgb}{1.000000,1.000000,1.000000}%
\pgfsetstrokecolor{currentstroke}%
\pgfsetdash{}{0pt}%
\pgfpathmoveto{\pgfqpoint{5.561674in}{1.068249in}}%
\pgfpathcurveto{\pgfqpoint{5.568807in}{1.068249in}}{\pgfqpoint{5.575648in}{1.071083in}}{\pgfqpoint{5.580692in}{1.076127in}}%
\pgfpathcurveto{\pgfqpoint{5.585736in}{1.081170in}}{\pgfqpoint{5.588569in}{1.088012in}}{\pgfqpoint{5.588569in}{1.095145in}}%
\pgfpathcurveto{\pgfqpoint{5.588569in}{1.102277in}}{\pgfqpoint{5.585736in}{1.109119in}}{\pgfqpoint{5.580692in}{1.114163in}}%
\pgfpathcurveto{\pgfqpoint{5.575648in}{1.119206in}}{\pgfqpoint{5.568807in}{1.122040in}}{\pgfqpoint{5.561674in}{1.122040in}}%
\pgfpathcurveto{\pgfqpoint{5.554541in}{1.122040in}}{\pgfqpoint{5.547699in}{1.119206in}}{\pgfqpoint{5.542656in}{1.114163in}}%
\pgfpathcurveto{\pgfqpoint{5.537612in}{1.109119in}}{\pgfqpoint{5.534778in}{1.102277in}}{\pgfqpoint{5.534778in}{1.095145in}}%
\pgfpathcurveto{\pgfqpoint{5.534778in}{1.088012in}}{\pgfqpoint{5.537612in}{1.081170in}}{\pgfqpoint{5.542656in}{1.076127in}}%
\pgfpathcurveto{\pgfqpoint{5.547699in}{1.071083in}}{\pgfqpoint{5.554541in}{1.068249in}}{\pgfqpoint{5.561674in}{1.068249in}}%
\pgfpathclose%
\pgfusepath{stroke,fill}%
\end{pgfscope}%
\begin{pgfscope}%
\pgfpathrectangle{\pgfqpoint{4.985294in}{0.500000in}}{\pgfqpoint{1.764706in}{1.700000in}}%
\pgfusepath{clip}%
\pgfsetbuttcap%
\pgfsetroundjoin%
\definecolor{currentfill}{rgb}{0.979891,0.908948,0.848279}%
\pgfsetfillcolor{currentfill}%
\pgfsetlinewidth{0.311001pt}%
\definecolor{currentstroke}{rgb}{1.000000,1.000000,1.000000}%
\pgfsetstrokecolor{currentstroke}%
\pgfsetdash{}{0pt}%
\pgfpathmoveto{\pgfqpoint{6.284072in}{1.497946in}}%
\pgfpathcurveto{\pgfqpoint{6.291205in}{1.497946in}}{\pgfqpoint{6.298046in}{1.500780in}}{\pgfqpoint{6.303090in}{1.505824in}}%
\pgfpathcurveto{\pgfqpoint{6.308134in}{1.510867in}}{\pgfqpoint{6.310968in}{1.517709in}}{\pgfqpoint{6.310968in}{1.524842in}}%
\pgfpathcurveto{\pgfqpoint{6.310968in}{1.531975in}}{\pgfqpoint{6.308134in}{1.538816in}}{\pgfqpoint{6.303090in}{1.543860in}}%
\pgfpathcurveto{\pgfqpoint{6.298046in}{1.548904in}}{\pgfqpoint{6.291205in}{1.551738in}}{\pgfqpoint{6.284072in}{1.551738in}}%
\pgfpathcurveto{\pgfqpoint{6.276939in}{1.551738in}}{\pgfqpoint{6.270097in}{1.548904in}}{\pgfqpoint{6.265054in}{1.543860in}}%
\pgfpathcurveto{\pgfqpoint{6.260010in}{1.538816in}}{\pgfqpoint{6.257176in}{1.531975in}}{\pgfqpoint{6.257176in}{1.524842in}}%
\pgfpathcurveto{\pgfqpoint{6.257176in}{1.517709in}}{\pgfqpoint{6.260010in}{1.510867in}}{\pgfqpoint{6.265054in}{1.505824in}}%
\pgfpathcurveto{\pgfqpoint{6.270097in}{1.500780in}}{\pgfqpoint{6.276939in}{1.497946in}}{\pgfqpoint{6.284072in}{1.497946in}}%
\pgfpathclose%
\pgfusepath{stroke,fill}%
\end{pgfscope}%
\begin{pgfscope}%
\pgfpathrectangle{\pgfqpoint{4.985294in}{0.500000in}}{\pgfqpoint{1.764706in}{1.700000in}}%
\pgfusepath{clip}%
\pgfsetbuttcap%
\pgfsetroundjoin%
\definecolor{currentfill}{rgb}{0.973271,0.850724,0.762998}%
\pgfsetfillcolor{currentfill}%
\pgfsetlinewidth{0.311001pt}%
\definecolor{currentstroke}{rgb}{1.000000,1.000000,1.000000}%
\pgfsetstrokecolor{currentstroke}%
\pgfsetdash{}{0pt}%
\pgfpathmoveto{\pgfqpoint{6.320220in}{1.110666in}}%
\pgfpathcurveto{\pgfqpoint{6.327353in}{1.110666in}}{\pgfqpoint{6.334195in}{1.113500in}}{\pgfqpoint{6.339238in}{1.118544in}}%
\pgfpathcurveto{\pgfqpoint{6.344282in}{1.123588in}}{\pgfqpoint{6.347116in}{1.130429in}}{\pgfqpoint{6.347116in}{1.137562in}}%
\pgfpathcurveto{\pgfqpoint{6.347116in}{1.144695in}}{\pgfqpoint{6.344282in}{1.151536in}}{\pgfqpoint{6.339238in}{1.156580in}}%
\pgfpathcurveto{\pgfqpoint{6.334195in}{1.161624in}}{\pgfqpoint{6.327353in}{1.164458in}}{\pgfqpoint{6.320220in}{1.164458in}}%
\pgfpathcurveto{\pgfqpoint{6.313087in}{1.164458in}}{\pgfqpoint{6.306246in}{1.161624in}}{\pgfqpoint{6.301202in}{1.156580in}}%
\pgfpathcurveto{\pgfqpoint{6.296158in}{1.151536in}}{\pgfqpoint{6.293324in}{1.144695in}}{\pgfqpoint{6.293324in}{1.137562in}}%
\pgfpathcurveto{\pgfqpoint{6.293324in}{1.130429in}}{\pgfqpoint{6.296158in}{1.123588in}}{\pgfqpoint{6.301202in}{1.118544in}}%
\pgfpathcurveto{\pgfqpoint{6.306246in}{1.113500in}}{\pgfqpoint{6.313087in}{1.110666in}}{\pgfqpoint{6.320220in}{1.110666in}}%
\pgfpathclose%
\pgfusepath{stroke,fill}%
\end{pgfscope}%
\begin{pgfscope}%
\pgfpathrectangle{\pgfqpoint{4.985294in}{0.500000in}}{\pgfqpoint{1.764706in}{1.700000in}}%
\pgfusepath{clip}%
\pgfsetbuttcap%
\pgfsetroundjoin%
\definecolor{currentfill}{rgb}{0.597702,0.106938,0.358380}%
\pgfsetfillcolor{currentfill}%
\pgfsetlinewidth{0.311001pt}%
\definecolor{currentstroke}{rgb}{1.000000,1.000000,1.000000}%
\pgfsetstrokecolor{currentstroke}%
\pgfsetdash{}{0pt}%
\pgfpathmoveto{\pgfqpoint{5.990102in}{1.050100in}}%
\pgfpathcurveto{\pgfqpoint{5.997235in}{1.050100in}}{\pgfqpoint{6.004077in}{1.052934in}}{\pgfqpoint{6.009120in}{1.057978in}}%
\pgfpathcurveto{\pgfqpoint{6.014164in}{1.063022in}}{\pgfqpoint{6.016998in}{1.069863in}}{\pgfqpoint{6.016998in}{1.076996in}}%
\pgfpathcurveto{\pgfqpoint{6.016998in}{1.084129in}}{\pgfqpoint{6.014164in}{1.090971in}}{\pgfqpoint{6.009120in}{1.096014in}}%
\pgfpathcurveto{\pgfqpoint{6.004077in}{1.101058in}}{\pgfqpoint{5.997235in}{1.103892in}}{\pgfqpoint{5.990102in}{1.103892in}}%
\pgfpathcurveto{\pgfqpoint{5.982969in}{1.103892in}}{\pgfqpoint{5.976128in}{1.101058in}}{\pgfqpoint{5.971084in}{1.096014in}}%
\pgfpathcurveto{\pgfqpoint{5.966040in}{1.090971in}}{\pgfqpoint{5.963206in}{1.084129in}}{\pgfqpoint{5.963206in}{1.076996in}}%
\pgfpathcurveto{\pgfqpoint{5.963206in}{1.069863in}}{\pgfqpoint{5.966040in}{1.063022in}}{\pgfqpoint{5.971084in}{1.057978in}}%
\pgfpathcurveto{\pgfqpoint{5.976128in}{1.052934in}}{\pgfqpoint{5.982969in}{1.050100in}}{\pgfqpoint{5.990102in}{1.050100in}}%
\pgfpathclose%
\pgfusepath{stroke,fill}%
\end{pgfscope}%
\begin{pgfscope}%
\pgfpathrectangle{\pgfqpoint{4.985294in}{0.500000in}}{\pgfqpoint{1.764706in}{1.700000in}}%
\pgfusepath{clip}%
\pgfsetbuttcap%
\pgfsetroundjoin%
\definecolor{currentfill}{rgb}{0.976287,0.879862,0.805788}%
\pgfsetfillcolor{currentfill}%
\pgfsetlinewidth{0.311001pt}%
\definecolor{currentstroke}{rgb}{1.000000,1.000000,1.000000}%
\pgfsetstrokecolor{currentstroke}%
\pgfsetdash{}{0pt}%
\pgfpathmoveto{\pgfqpoint{6.308891in}{1.131023in}}%
\pgfpathcurveto{\pgfqpoint{6.316023in}{1.131023in}}{\pgfqpoint{6.322865in}{1.133857in}}{\pgfqpoint{6.327909in}{1.138901in}}%
\pgfpathcurveto{\pgfqpoint{6.332952in}{1.143944in}}{\pgfqpoint{6.335786in}{1.150786in}}{\pgfqpoint{6.335786in}{1.157919in}}%
\pgfpathcurveto{\pgfqpoint{6.335786in}{1.165052in}}{\pgfqpoint{6.332952in}{1.171893in}}{\pgfqpoint{6.327909in}{1.176937in}}%
\pgfpathcurveto{\pgfqpoint{6.322865in}{1.181981in}}{\pgfqpoint{6.316023in}{1.184815in}}{\pgfqpoint{6.308891in}{1.184815in}}%
\pgfpathcurveto{\pgfqpoint{6.301758in}{1.184815in}}{\pgfqpoint{6.294916in}{1.181981in}}{\pgfqpoint{6.289872in}{1.176937in}}%
\pgfpathcurveto{\pgfqpoint{6.284829in}{1.171893in}}{\pgfqpoint{6.281995in}{1.165052in}}{\pgfqpoint{6.281995in}{1.157919in}}%
\pgfpathcurveto{\pgfqpoint{6.281995in}{1.150786in}}{\pgfqpoint{6.284829in}{1.143944in}}{\pgfqpoint{6.289872in}{1.138901in}}%
\pgfpathcurveto{\pgfqpoint{6.294916in}{1.133857in}}{\pgfqpoint{6.301758in}{1.131023in}}{\pgfqpoint{6.308891in}{1.131023in}}%
\pgfpathclose%
\pgfusepath{stroke,fill}%
\end{pgfscope}%
\begin{pgfscope}%
\pgfpathrectangle{\pgfqpoint{4.985294in}{0.500000in}}{\pgfqpoint{1.764706in}{1.700000in}}%
\pgfusepath{clip}%
\pgfsetbuttcap%
\pgfsetroundjoin%
\definecolor{currentfill}{rgb}{0.950017,0.427714,0.292447}%
\pgfsetfillcolor{currentfill}%
\pgfsetlinewidth{0.311001pt}%
\definecolor{currentstroke}{rgb}{1.000000,1.000000,1.000000}%
\pgfsetstrokecolor{currentstroke}%
\pgfsetdash{}{0pt}%
\pgfpathmoveto{\pgfqpoint{5.645360in}{0.843607in}}%
\pgfpathcurveto{\pgfqpoint{5.652493in}{0.843607in}}{\pgfqpoint{5.659334in}{0.846441in}}{\pgfqpoint{5.664378in}{0.851485in}}%
\pgfpathcurveto{\pgfqpoint{5.669422in}{0.856528in}}{\pgfqpoint{5.672255in}{0.863370in}}{\pgfqpoint{5.672255in}{0.870503in}}%
\pgfpathcurveto{\pgfqpoint{5.672255in}{0.877636in}}{\pgfqpoint{5.669422in}{0.884477in}}{\pgfqpoint{5.664378in}{0.889521in}}%
\pgfpathcurveto{\pgfqpoint{5.659334in}{0.894565in}}{\pgfqpoint{5.652493in}{0.897399in}}{\pgfqpoint{5.645360in}{0.897399in}}%
\pgfpathcurveto{\pgfqpoint{5.638227in}{0.897399in}}{\pgfqpoint{5.631385in}{0.894565in}}{\pgfqpoint{5.626342in}{0.889521in}}%
\pgfpathcurveto{\pgfqpoint{5.621298in}{0.884477in}}{\pgfqpoint{5.618464in}{0.877636in}}{\pgfqpoint{5.618464in}{0.870503in}}%
\pgfpathcurveto{\pgfqpoint{5.618464in}{0.863370in}}{\pgfqpoint{5.621298in}{0.856528in}}{\pgfqpoint{5.626342in}{0.851485in}}%
\pgfpathcurveto{\pgfqpoint{5.631385in}{0.846441in}}{\pgfqpoint{5.638227in}{0.843607in}}{\pgfqpoint{5.645360in}{0.843607in}}%
\pgfpathclose%
\pgfusepath{stroke,fill}%
\end{pgfscope}%
\begin{pgfscope}%
\pgfpathrectangle{\pgfqpoint{4.985294in}{0.500000in}}{\pgfqpoint{1.764706in}{1.700000in}}%
\pgfusepath{clip}%
\pgfsetbuttcap%
\pgfsetroundjoin%
\definecolor{currentfill}{rgb}{0.704578,0.088213,0.344730}%
\pgfsetfillcolor{currentfill}%
\pgfsetlinewidth{0.311001pt}%
\definecolor{currentstroke}{rgb}{1.000000,1.000000,1.000000}%
\pgfsetstrokecolor{currentstroke}%
\pgfsetdash{}{0pt}%
\pgfpathmoveto{\pgfqpoint{6.339184in}{1.793387in}}%
\pgfpathcurveto{\pgfqpoint{6.346317in}{1.793387in}}{\pgfqpoint{6.353158in}{1.796221in}}{\pgfqpoint{6.358202in}{1.801265in}}%
\pgfpathcurveto{\pgfqpoint{6.363246in}{1.806308in}}{\pgfqpoint{6.366080in}{1.813150in}}{\pgfqpoint{6.366080in}{1.820283in}}%
\pgfpathcurveto{\pgfqpoint{6.366080in}{1.827416in}}{\pgfqpoint{6.363246in}{1.834257in}}{\pgfqpoint{6.358202in}{1.839301in}}%
\pgfpathcurveto{\pgfqpoint{6.353158in}{1.844345in}}{\pgfqpoint{6.346317in}{1.847179in}}{\pgfqpoint{6.339184in}{1.847179in}}%
\pgfpathcurveto{\pgfqpoint{6.332051in}{1.847179in}}{\pgfqpoint{6.325209in}{1.844345in}}{\pgfqpoint{6.320166in}{1.839301in}}%
\pgfpathcurveto{\pgfqpoint{6.315122in}{1.834257in}}{\pgfqpoint{6.312288in}{1.827416in}}{\pgfqpoint{6.312288in}{1.820283in}}%
\pgfpathcurveto{\pgfqpoint{6.312288in}{1.813150in}}{\pgfqpoint{6.315122in}{1.806308in}}{\pgfqpoint{6.320166in}{1.801265in}}%
\pgfpathcurveto{\pgfqpoint{6.325209in}{1.796221in}}{\pgfqpoint{6.332051in}{1.793387in}}{\pgfqpoint{6.339184in}{1.793387in}}%
\pgfpathclose%
\pgfusepath{stroke,fill}%
\end{pgfscope}%
\begin{pgfscope}%
\pgfpathrectangle{\pgfqpoint{4.985294in}{0.500000in}}{\pgfqpoint{1.764706in}{1.700000in}}%
\pgfusepath{clip}%
\pgfsetbuttcap%
\pgfsetroundjoin%
\definecolor{currentfill}{rgb}{0.965753,0.732351,0.592427}%
\pgfsetfillcolor{currentfill}%
\pgfsetlinewidth{0.311001pt}%
\definecolor{currentstroke}{rgb}{1.000000,1.000000,1.000000}%
\pgfsetstrokecolor{currentstroke}%
\pgfsetdash{}{0pt}%
\pgfpathmoveto{\pgfqpoint{6.128859in}{1.649283in}}%
\pgfpathcurveto{\pgfqpoint{6.135992in}{1.649283in}}{\pgfqpoint{6.142833in}{1.652117in}}{\pgfqpoint{6.147877in}{1.657161in}}%
\pgfpathcurveto{\pgfqpoint{6.152921in}{1.662204in}}{\pgfqpoint{6.155754in}{1.669046in}}{\pgfqpoint{6.155754in}{1.676179in}}%
\pgfpathcurveto{\pgfqpoint{6.155754in}{1.683312in}}{\pgfqpoint{6.152921in}{1.690153in}}{\pgfqpoint{6.147877in}{1.695197in}}%
\pgfpathcurveto{\pgfqpoint{6.142833in}{1.700240in}}{\pgfqpoint{6.135992in}{1.703074in}}{\pgfqpoint{6.128859in}{1.703074in}}%
\pgfpathcurveto{\pgfqpoint{6.121726in}{1.703074in}}{\pgfqpoint{6.114884in}{1.700240in}}{\pgfqpoint{6.109841in}{1.695197in}}%
\pgfpathcurveto{\pgfqpoint{6.104797in}{1.690153in}}{\pgfqpoint{6.101963in}{1.683312in}}{\pgfqpoint{6.101963in}{1.676179in}}%
\pgfpathcurveto{\pgfqpoint{6.101963in}{1.669046in}}{\pgfqpoint{6.104797in}{1.662204in}}{\pgfqpoint{6.109841in}{1.657161in}}%
\pgfpathcurveto{\pgfqpoint{6.114884in}{1.652117in}}{\pgfqpoint{6.121726in}{1.649283in}}{\pgfqpoint{6.128859in}{1.649283in}}%
\pgfpathclose%
\pgfusepath{stroke,fill}%
\end{pgfscope}%
\begin{pgfscope}%
\pgfpathrectangle{\pgfqpoint{4.985294in}{0.500000in}}{\pgfqpoint{1.764706in}{1.700000in}}%
\pgfusepath{clip}%
\pgfsetbuttcap%
\pgfsetroundjoin%
\definecolor{currentfill}{rgb}{0.977657,0.891500,0.822809}%
\pgfsetfillcolor{currentfill}%
\pgfsetlinewidth{0.311001pt}%
\definecolor{currentstroke}{rgb}{1.000000,1.000000,1.000000}%
\pgfsetstrokecolor{currentstroke}%
\pgfsetdash{}{0pt}%
\pgfpathmoveto{\pgfqpoint{6.326814in}{1.181789in}}%
\pgfpathcurveto{\pgfqpoint{6.333947in}{1.181789in}}{\pgfqpoint{6.340789in}{1.184623in}}{\pgfqpoint{6.345832in}{1.189667in}}%
\pgfpathcurveto{\pgfqpoint{6.350876in}{1.194710in}}{\pgfqpoint{6.353710in}{1.201552in}}{\pgfqpoint{6.353710in}{1.208685in}}%
\pgfpathcurveto{\pgfqpoint{6.353710in}{1.215818in}}{\pgfqpoint{6.350876in}{1.222659in}}{\pgfqpoint{6.345832in}{1.227703in}}%
\pgfpathcurveto{\pgfqpoint{6.340789in}{1.232746in}}{\pgfqpoint{6.333947in}{1.235580in}}{\pgfqpoint{6.326814in}{1.235580in}}%
\pgfpathcurveto{\pgfqpoint{6.319681in}{1.235580in}}{\pgfqpoint{6.312840in}{1.232746in}}{\pgfqpoint{6.307796in}{1.227703in}}%
\pgfpathcurveto{\pgfqpoint{6.302752in}{1.222659in}}{\pgfqpoint{6.299918in}{1.215818in}}{\pgfqpoint{6.299918in}{1.208685in}}%
\pgfpathcurveto{\pgfqpoint{6.299918in}{1.201552in}}{\pgfqpoint{6.302752in}{1.194710in}}{\pgfqpoint{6.307796in}{1.189667in}}%
\pgfpathcurveto{\pgfqpoint{6.312840in}{1.184623in}}{\pgfqpoint{6.319681in}{1.181789in}}{\pgfqpoint{6.326814in}{1.181789in}}%
\pgfpathclose%
\pgfusepath{stroke,fill}%
\end{pgfscope}%
\begin{pgfscope}%
\pgfpathrectangle{\pgfqpoint{4.985294in}{0.500000in}}{\pgfqpoint{1.764706in}{1.700000in}}%
\pgfusepath{clip}%
\pgfsetbuttcap%
\pgfsetroundjoin%
\definecolor{currentfill}{rgb}{0.956817,0.498820,0.345554}%
\pgfsetfillcolor{currentfill}%
\pgfsetlinewidth{0.311001pt}%
\definecolor{currentstroke}{rgb}{1.000000,1.000000,1.000000}%
\pgfsetstrokecolor{currentstroke}%
\pgfsetdash{}{0pt}%
\pgfpathmoveto{\pgfqpoint{5.671496in}{1.692483in}}%
\pgfpathcurveto{\pgfqpoint{5.678629in}{1.692483in}}{\pgfqpoint{5.685470in}{1.695316in}}{\pgfqpoint{5.690514in}{1.700360in}}%
\pgfpathcurveto{\pgfqpoint{5.695558in}{1.705404in}}{\pgfqpoint{5.698392in}{1.712245in}}{\pgfqpoint{5.698392in}{1.719378in}}%
\pgfpathcurveto{\pgfqpoint{5.698392in}{1.726511in}}{\pgfqpoint{5.695558in}{1.733353in}}{\pgfqpoint{5.690514in}{1.738396in}}%
\pgfpathcurveto{\pgfqpoint{5.685470in}{1.743440in}}{\pgfqpoint{5.678629in}{1.746274in}}{\pgfqpoint{5.671496in}{1.746274in}}%
\pgfpathcurveto{\pgfqpoint{5.664363in}{1.746274in}}{\pgfqpoint{5.657521in}{1.743440in}}{\pgfqpoint{5.652478in}{1.738396in}}%
\pgfpathcurveto{\pgfqpoint{5.647434in}{1.733353in}}{\pgfqpoint{5.644600in}{1.726511in}}{\pgfqpoint{5.644600in}{1.719378in}}%
\pgfpathcurveto{\pgfqpoint{5.644600in}{1.712245in}}{\pgfqpoint{5.647434in}{1.705404in}}{\pgfqpoint{5.652478in}{1.700360in}}%
\pgfpathcurveto{\pgfqpoint{5.657521in}{1.695316in}}{\pgfqpoint{5.664363in}{1.692483in}}{\pgfqpoint{5.671496in}{1.692483in}}%
\pgfpathclose%
\pgfusepath{stroke,fill}%
\end{pgfscope}%
\begin{pgfscope}%
\pgfpathrectangle{\pgfqpoint{4.985294in}{0.500000in}}{\pgfqpoint{1.764706in}{1.700000in}}%
\pgfusepath{clip}%
\pgfsetbuttcap%
\pgfsetroundjoin%
\definecolor{currentfill}{rgb}{0.967398,0.774513,0.650573}%
\pgfsetfillcolor{currentfill}%
\pgfsetlinewidth{0.311001pt}%
\definecolor{currentstroke}{rgb}{1.000000,1.000000,1.000000}%
\pgfsetstrokecolor{currentstroke}%
\pgfsetdash{}{0pt}%
\pgfpathmoveto{\pgfqpoint{6.207278in}{1.470304in}}%
\pgfpathcurveto{\pgfqpoint{6.214411in}{1.470304in}}{\pgfqpoint{6.221253in}{1.473138in}}{\pgfqpoint{6.226296in}{1.478182in}}%
\pgfpathcurveto{\pgfqpoint{6.231340in}{1.483225in}}{\pgfqpoint{6.234174in}{1.490067in}}{\pgfqpoint{6.234174in}{1.497200in}}%
\pgfpathcurveto{\pgfqpoint{6.234174in}{1.504333in}}{\pgfqpoint{6.231340in}{1.511174in}}{\pgfqpoint{6.226296in}{1.516218in}}%
\pgfpathcurveto{\pgfqpoint{6.221253in}{1.521262in}}{\pgfqpoint{6.214411in}{1.524095in}}{\pgfqpoint{6.207278in}{1.524095in}}%
\pgfpathcurveto{\pgfqpoint{6.200145in}{1.524095in}}{\pgfqpoint{6.193304in}{1.521262in}}{\pgfqpoint{6.188260in}{1.516218in}}%
\pgfpathcurveto{\pgfqpoint{6.183216in}{1.511174in}}{\pgfqpoint{6.180382in}{1.504333in}}{\pgfqpoint{6.180382in}{1.497200in}}%
\pgfpathcurveto{\pgfqpoint{6.180382in}{1.490067in}}{\pgfqpoint{6.183216in}{1.483225in}}{\pgfqpoint{6.188260in}{1.478182in}}%
\pgfpathcurveto{\pgfqpoint{6.193304in}{1.473138in}}{\pgfqpoint{6.200145in}{1.470304in}}{\pgfqpoint{6.207278in}{1.470304in}}%
\pgfpathclose%
\pgfusepath{stroke,fill}%
\end{pgfscope}%
\begin{pgfscope}%
\pgfpathrectangle{\pgfqpoint{4.985294in}{0.500000in}}{\pgfqpoint{1.764706in}{1.700000in}}%
\pgfusepath{clip}%
\pgfsetbuttcap%
\pgfsetroundjoin%
\definecolor{currentfill}{rgb}{0.963884,0.644842,0.486120}%
\pgfsetfillcolor{currentfill}%
\pgfsetlinewidth{0.311001pt}%
\definecolor{currentstroke}{rgb}{1.000000,1.000000,1.000000}%
\pgfsetstrokecolor{currentstroke}%
\pgfsetdash{}{0pt}%
\pgfpathmoveto{\pgfqpoint{5.549498in}{1.427522in}}%
\pgfpathcurveto{\pgfqpoint{5.556631in}{1.427522in}}{\pgfqpoint{5.563472in}{1.430356in}}{\pgfqpoint{5.568516in}{1.435399in}}%
\pgfpathcurveto{\pgfqpoint{5.573559in}{1.440443in}}{\pgfqpoint{5.576393in}{1.447285in}}{\pgfqpoint{5.576393in}{1.454417in}}%
\pgfpathcurveto{\pgfqpoint{5.576393in}{1.461550in}}{\pgfqpoint{5.573559in}{1.468392in}}{\pgfqpoint{5.568516in}{1.473436in}}%
\pgfpathcurveto{\pgfqpoint{5.563472in}{1.478479in}}{\pgfqpoint{5.556631in}{1.481313in}}{\pgfqpoint{5.549498in}{1.481313in}}%
\pgfpathcurveto{\pgfqpoint{5.542365in}{1.481313in}}{\pgfqpoint{5.535523in}{1.478479in}}{\pgfqpoint{5.530480in}{1.473436in}}%
\pgfpathcurveto{\pgfqpoint{5.525436in}{1.468392in}}{\pgfqpoint{5.522602in}{1.461550in}}{\pgfqpoint{5.522602in}{1.454417in}}%
\pgfpathcurveto{\pgfqpoint{5.522602in}{1.447285in}}{\pgfqpoint{5.525436in}{1.440443in}}{\pgfqpoint{5.530480in}{1.435399in}}%
\pgfpathcurveto{\pgfqpoint{5.535523in}{1.430356in}}{\pgfqpoint{5.542365in}{1.427522in}}{\pgfqpoint{5.549498in}{1.427522in}}%
\pgfpathclose%
\pgfusepath{stroke,fill}%
\end{pgfscope}%
\begin{pgfscope}%
\pgfpathrectangle{\pgfqpoint{4.985294in}{0.500000in}}{\pgfqpoint{1.764706in}{1.700000in}}%
\pgfusepath{clip}%
\pgfsetbuttcap%
\pgfsetroundjoin%
\definecolor{currentfill}{rgb}{0.971202,0.827364,0.728520}%
\pgfsetfillcolor{currentfill}%
\pgfsetlinewidth{0.311001pt}%
\definecolor{currentstroke}{rgb}{1.000000,1.000000,1.000000}%
\pgfsetstrokecolor{currentstroke}%
\pgfsetdash{}{0pt}%
\pgfpathmoveto{\pgfqpoint{6.184724in}{1.660260in}}%
\pgfpathcurveto{\pgfqpoint{6.191857in}{1.660260in}}{\pgfqpoint{6.198699in}{1.663094in}}{\pgfqpoint{6.203742in}{1.668137in}}%
\pgfpathcurveto{\pgfqpoint{6.208786in}{1.673181in}}{\pgfqpoint{6.211620in}{1.680023in}}{\pgfqpoint{6.211620in}{1.687155in}}%
\pgfpathcurveto{\pgfqpoint{6.211620in}{1.694288in}}{\pgfqpoint{6.208786in}{1.701130in}}{\pgfqpoint{6.203742in}{1.706174in}}%
\pgfpathcurveto{\pgfqpoint{6.198699in}{1.711217in}}{\pgfqpoint{6.191857in}{1.714051in}}{\pgfqpoint{6.184724in}{1.714051in}}%
\pgfpathcurveto{\pgfqpoint{6.177592in}{1.714051in}}{\pgfqpoint{6.170750in}{1.711217in}}{\pgfqpoint{6.165706in}{1.706174in}}%
\pgfpathcurveto{\pgfqpoint{6.160663in}{1.701130in}}{\pgfqpoint{6.157829in}{1.694288in}}{\pgfqpoint{6.157829in}{1.687155in}}%
\pgfpathcurveto{\pgfqpoint{6.157829in}{1.680023in}}{\pgfqpoint{6.160663in}{1.673181in}}{\pgfqpoint{6.165706in}{1.668137in}}%
\pgfpathcurveto{\pgfqpoint{6.170750in}{1.663094in}}{\pgfqpoint{6.177592in}{1.660260in}}{\pgfqpoint{6.184724in}{1.660260in}}%
\pgfpathclose%
\pgfusepath{stroke,fill}%
\end{pgfscope}%
\begin{pgfscope}%
\pgfpathrectangle{\pgfqpoint{4.985294in}{0.500000in}}{\pgfqpoint{1.764706in}{1.700000in}}%
\pgfusepath{clip}%
\pgfsetbuttcap%
\pgfsetroundjoin%
\definecolor{currentfill}{rgb}{0.952404,0.449449,0.307210}%
\pgfsetfillcolor{currentfill}%
\pgfsetlinewidth{0.311001pt}%
\definecolor{currentstroke}{rgb}{1.000000,1.000000,1.000000}%
\pgfsetstrokecolor{currentstroke}%
\pgfsetdash{}{0pt}%
\pgfpathmoveto{\pgfqpoint{6.420531in}{1.533302in}}%
\pgfpathcurveto{\pgfqpoint{6.427664in}{1.533302in}}{\pgfqpoint{6.434505in}{1.536136in}}{\pgfqpoint{6.439549in}{1.541180in}}%
\pgfpathcurveto{\pgfqpoint{6.444593in}{1.546223in}}{\pgfqpoint{6.447427in}{1.553065in}}{\pgfqpoint{6.447427in}{1.560198in}}%
\pgfpathcurveto{\pgfqpoint{6.447427in}{1.567331in}}{\pgfqpoint{6.444593in}{1.574172in}}{\pgfqpoint{6.439549in}{1.579216in}}%
\pgfpathcurveto{\pgfqpoint{6.434505in}{1.584260in}}{\pgfqpoint{6.427664in}{1.587094in}}{\pgfqpoint{6.420531in}{1.587094in}}%
\pgfpathcurveto{\pgfqpoint{6.413398in}{1.587094in}}{\pgfqpoint{6.406556in}{1.584260in}}{\pgfqpoint{6.401513in}{1.579216in}}%
\pgfpathcurveto{\pgfqpoint{6.396469in}{1.574172in}}{\pgfqpoint{6.393635in}{1.567331in}}{\pgfqpoint{6.393635in}{1.560198in}}%
\pgfpathcurveto{\pgfqpoint{6.393635in}{1.553065in}}{\pgfqpoint{6.396469in}{1.546223in}}{\pgfqpoint{6.401513in}{1.541180in}}%
\pgfpathcurveto{\pgfqpoint{6.406556in}{1.536136in}}{\pgfqpoint{6.413398in}{1.533302in}}{\pgfqpoint{6.420531in}{1.533302in}}%
\pgfpathclose%
\pgfusepath{stroke,fill}%
\end{pgfscope}%
\begin{pgfscope}%
\pgfpathrectangle{\pgfqpoint{4.985294in}{0.500000in}}{\pgfqpoint{1.764706in}{1.700000in}}%
\pgfusepath{clip}%
\pgfsetbuttcap%
\pgfsetroundjoin%
\definecolor{currentfill}{rgb}{0.969803,0.809811,0.702523}%
\pgfsetfillcolor{currentfill}%
\pgfsetlinewidth{0.311001pt}%
\definecolor{currentstroke}{rgb}{1.000000,1.000000,1.000000}%
\pgfsetstrokecolor{currentstroke}%
\pgfsetdash{}{0pt}%
\pgfpathmoveto{\pgfqpoint{6.227923in}{1.255374in}}%
\pgfpathcurveto{\pgfqpoint{6.235056in}{1.255374in}}{\pgfqpoint{6.241897in}{1.258208in}}{\pgfqpoint{6.246941in}{1.263252in}}%
\pgfpathcurveto{\pgfqpoint{6.251985in}{1.268295in}}{\pgfqpoint{6.254818in}{1.275137in}}{\pgfqpoint{6.254818in}{1.282270in}}%
\pgfpathcurveto{\pgfqpoint{6.254818in}{1.289403in}}{\pgfqpoint{6.251985in}{1.296244in}}{\pgfqpoint{6.246941in}{1.301288in}}%
\pgfpathcurveto{\pgfqpoint{6.241897in}{1.306332in}}{\pgfqpoint{6.235056in}{1.309166in}}{\pgfqpoint{6.227923in}{1.309166in}}%
\pgfpathcurveto{\pgfqpoint{6.220790in}{1.309166in}}{\pgfqpoint{6.213948in}{1.306332in}}{\pgfqpoint{6.208905in}{1.301288in}}%
\pgfpathcurveto{\pgfqpoint{6.203861in}{1.296244in}}{\pgfqpoint{6.201027in}{1.289403in}}{\pgfqpoint{6.201027in}{1.282270in}}%
\pgfpathcurveto{\pgfqpoint{6.201027in}{1.275137in}}{\pgfqpoint{6.203861in}{1.268295in}}{\pgfqpoint{6.208905in}{1.263252in}}%
\pgfpathcurveto{\pgfqpoint{6.213948in}{1.258208in}}{\pgfqpoint{6.220790in}{1.255374in}}{\pgfqpoint{6.227923in}{1.255374in}}%
\pgfpathclose%
\pgfusepath{stroke,fill}%
\end{pgfscope}%
\begin{pgfscope}%
\pgfpathrectangle{\pgfqpoint{4.985294in}{0.500000in}}{\pgfqpoint{1.764706in}{1.700000in}}%
\pgfusepath{clip}%
\pgfsetbuttcap%
\pgfsetroundjoin%
\definecolor{currentfill}{rgb}{0.966328,0.750560,0.616961}%
\pgfsetfillcolor{currentfill}%
\pgfsetlinewidth{0.311001pt}%
\definecolor{currentstroke}{rgb}{1.000000,1.000000,1.000000}%
\pgfsetstrokecolor{currentstroke}%
\pgfsetdash{}{0pt}%
\pgfpathmoveto{\pgfqpoint{5.382696in}{1.076182in}}%
\pgfpathcurveto{\pgfqpoint{5.389829in}{1.076182in}}{\pgfqpoint{5.396670in}{1.079016in}}{\pgfqpoint{5.401714in}{1.084060in}}%
\pgfpathcurveto{\pgfqpoint{5.406758in}{1.089104in}}{\pgfqpoint{5.409592in}{1.095945in}}{\pgfqpoint{5.409592in}{1.103078in}}%
\pgfpathcurveto{\pgfqpoint{5.409592in}{1.110211in}}{\pgfqpoint{5.406758in}{1.117053in}}{\pgfqpoint{5.401714in}{1.122096in}}%
\pgfpathcurveto{\pgfqpoint{5.396670in}{1.127140in}}{\pgfqpoint{5.389829in}{1.129974in}}{\pgfqpoint{5.382696in}{1.129974in}}%
\pgfpathcurveto{\pgfqpoint{5.375563in}{1.129974in}}{\pgfqpoint{5.368721in}{1.127140in}}{\pgfqpoint{5.363678in}{1.122096in}}%
\pgfpathcurveto{\pgfqpoint{5.358634in}{1.117053in}}{\pgfqpoint{5.355800in}{1.110211in}}{\pgfqpoint{5.355800in}{1.103078in}}%
\pgfpathcurveto{\pgfqpoint{5.355800in}{1.095945in}}{\pgfqpoint{5.358634in}{1.089104in}}{\pgfqpoint{5.363678in}{1.084060in}}%
\pgfpathcurveto{\pgfqpoint{5.368721in}{1.079016in}}{\pgfqpoint{5.375563in}{1.076182in}}{\pgfqpoint{5.382696in}{1.076182in}}%
\pgfpathclose%
\pgfusepath{stroke,fill}%
\end{pgfscope}%
\begin{pgfscope}%
\pgfpathrectangle{\pgfqpoint{4.985294in}{0.500000in}}{\pgfqpoint{1.764706in}{1.700000in}}%
\pgfusepath{clip}%
\pgfsetbuttcap%
\pgfsetroundjoin%
\definecolor{currentfill}{rgb}{0.955103,0.477872,0.328626}%
\pgfsetfillcolor{currentfill}%
\pgfsetlinewidth{0.311001pt}%
\definecolor{currentstroke}{rgb}{1.000000,1.000000,1.000000}%
\pgfsetstrokecolor{currentstroke}%
\pgfsetdash{}{0pt}%
\pgfpathmoveto{\pgfqpoint{5.575641in}{1.424882in}}%
\pgfpathcurveto{\pgfqpoint{5.582774in}{1.424882in}}{\pgfqpoint{5.589615in}{1.427716in}}{\pgfqpoint{5.594659in}{1.432760in}}%
\pgfpathcurveto{\pgfqpoint{5.599703in}{1.437803in}}{\pgfqpoint{5.602537in}{1.444645in}}{\pgfqpoint{5.602537in}{1.451778in}}%
\pgfpathcurveto{\pgfqpoint{5.602537in}{1.458911in}}{\pgfqpoint{5.599703in}{1.465752in}}{\pgfqpoint{5.594659in}{1.470796in}}%
\pgfpathcurveto{\pgfqpoint{5.589615in}{1.475840in}}{\pgfqpoint{5.582774in}{1.478673in}}{\pgfqpoint{5.575641in}{1.478673in}}%
\pgfpathcurveto{\pgfqpoint{5.568508in}{1.478673in}}{\pgfqpoint{5.561666in}{1.475840in}}{\pgfqpoint{5.556623in}{1.470796in}}%
\pgfpathcurveto{\pgfqpoint{5.551579in}{1.465752in}}{\pgfqpoint{5.548745in}{1.458911in}}{\pgfqpoint{5.548745in}{1.451778in}}%
\pgfpathcurveto{\pgfqpoint{5.548745in}{1.444645in}}{\pgfqpoint{5.551579in}{1.437803in}}{\pgfqpoint{5.556623in}{1.432760in}}%
\pgfpathcurveto{\pgfqpoint{5.561666in}{1.427716in}}{\pgfqpoint{5.568508in}{1.424882in}}{\pgfqpoint{5.575641in}{1.424882in}}%
\pgfpathclose%
\pgfusepath{stroke,fill}%
\end{pgfscope}%
\begin{pgfscope}%
\pgfpathrectangle{\pgfqpoint{4.985294in}{0.500000in}}{\pgfqpoint{1.764706in}{1.700000in}}%
\pgfusepath{clip}%
\pgfsetbuttcap%
\pgfsetroundjoin%
\definecolor{currentfill}{rgb}{0.967735,0.780441,0.659127}%
\pgfsetfillcolor{currentfill}%
\pgfsetlinewidth{0.311001pt}%
\definecolor{currentstroke}{rgb}{1.000000,1.000000,1.000000}%
\pgfsetstrokecolor{currentstroke}%
\pgfsetdash{}{0pt}%
\pgfpathmoveto{\pgfqpoint{5.529088in}{1.124463in}}%
\pgfpathcurveto{\pgfqpoint{5.536220in}{1.124463in}}{\pgfqpoint{5.543062in}{1.127297in}}{\pgfqpoint{5.548106in}{1.132341in}}%
\pgfpathcurveto{\pgfqpoint{5.553149in}{1.137384in}}{\pgfqpoint{5.555983in}{1.144226in}}{\pgfqpoint{5.555983in}{1.151359in}}%
\pgfpathcurveto{\pgfqpoint{5.555983in}{1.158492in}}{\pgfqpoint{5.553149in}{1.165333in}}{\pgfqpoint{5.548106in}{1.170377in}}%
\pgfpathcurveto{\pgfqpoint{5.543062in}{1.175421in}}{\pgfqpoint{5.536220in}{1.178254in}}{\pgfqpoint{5.529088in}{1.178254in}}%
\pgfpathcurveto{\pgfqpoint{5.521955in}{1.178254in}}{\pgfqpoint{5.515113in}{1.175421in}}{\pgfqpoint{5.510069in}{1.170377in}}%
\pgfpathcurveto{\pgfqpoint{5.505026in}{1.165333in}}{\pgfqpoint{5.502192in}{1.158492in}}{\pgfqpoint{5.502192in}{1.151359in}}%
\pgfpathcurveto{\pgfqpoint{5.502192in}{1.144226in}}{\pgfqpoint{5.505026in}{1.137384in}}{\pgfqpoint{5.510069in}{1.132341in}}%
\pgfpathcurveto{\pgfqpoint{5.515113in}{1.127297in}}{\pgfqpoint{5.521955in}{1.124463in}}{\pgfqpoint{5.529088in}{1.124463in}}%
\pgfpathclose%
\pgfusepath{stroke,fill}%
\end{pgfscope}%
\begin{pgfscope}%
\pgfpathrectangle{\pgfqpoint{4.985294in}{0.500000in}}{\pgfqpoint{1.764706in}{1.700000in}}%
\pgfusepath{clip}%
\pgfsetbuttcap%
\pgfsetroundjoin%
\definecolor{currentfill}{rgb}{0.879259,0.192033,0.262681}%
\pgfsetfillcolor{currentfill}%
\pgfsetlinewidth{0.311001pt}%
\definecolor{currentstroke}{rgb}{1.000000,1.000000,1.000000}%
\pgfsetstrokecolor{currentstroke}%
\pgfsetdash{}{0pt}%
\pgfpathmoveto{\pgfqpoint{5.244888in}{1.310977in}}%
\pgfpathcurveto{\pgfqpoint{5.252021in}{1.310977in}}{\pgfqpoint{5.258863in}{1.313811in}}{\pgfqpoint{5.263906in}{1.318855in}}%
\pgfpathcurveto{\pgfqpoint{5.268950in}{1.323899in}}{\pgfqpoint{5.271784in}{1.330740in}}{\pgfqpoint{5.271784in}{1.337873in}}%
\pgfpathcurveto{\pgfqpoint{5.271784in}{1.345006in}}{\pgfqpoint{5.268950in}{1.351848in}}{\pgfqpoint{5.263906in}{1.356891in}}%
\pgfpathcurveto{\pgfqpoint{5.258863in}{1.361935in}}{\pgfqpoint{5.252021in}{1.364769in}}{\pgfqpoint{5.244888in}{1.364769in}}%
\pgfpathcurveto{\pgfqpoint{5.237755in}{1.364769in}}{\pgfqpoint{5.230914in}{1.361935in}}{\pgfqpoint{5.225870in}{1.356891in}}%
\pgfpathcurveto{\pgfqpoint{5.220826in}{1.351848in}}{\pgfqpoint{5.217992in}{1.345006in}}{\pgfqpoint{5.217992in}{1.337873in}}%
\pgfpathcurveto{\pgfqpoint{5.217992in}{1.330740in}}{\pgfqpoint{5.220826in}{1.323899in}}{\pgfqpoint{5.225870in}{1.318855in}}%
\pgfpathcurveto{\pgfqpoint{5.230914in}{1.313811in}}{\pgfqpoint{5.237755in}{1.310977in}}{\pgfqpoint{5.244888in}{1.310977in}}%
\pgfpathclose%
\pgfusepath{stroke,fill}%
\end{pgfscope}%
\begin{pgfscope}%
\pgfpathrectangle{\pgfqpoint{4.985294in}{0.500000in}}{\pgfqpoint{1.764706in}{1.700000in}}%
\pgfusepath{clip}%
\pgfsetbuttcap%
\pgfsetroundjoin%
\definecolor{currentfill}{rgb}{0.970718,0.821518,0.719872}%
\pgfsetfillcolor{currentfill}%
\pgfsetlinewidth{0.311001pt}%
\definecolor{currentstroke}{rgb}{1.000000,1.000000,1.000000}%
\pgfsetstrokecolor{currentstroke}%
\pgfsetdash{}{0pt}%
\pgfpathmoveto{\pgfqpoint{6.244725in}{1.338158in}}%
\pgfpathcurveto{\pgfqpoint{6.251858in}{1.338158in}}{\pgfqpoint{6.258700in}{1.340991in}}{\pgfqpoint{6.263743in}{1.346035in}}%
\pgfpathcurveto{\pgfqpoint{6.268787in}{1.351079in}}{\pgfqpoint{6.271621in}{1.357920in}}{\pgfqpoint{6.271621in}{1.365053in}}%
\pgfpathcurveto{\pgfqpoint{6.271621in}{1.372186in}}{\pgfqpoint{6.268787in}{1.379028in}}{\pgfqpoint{6.263743in}{1.384071in}}%
\pgfpathcurveto{\pgfqpoint{6.258700in}{1.389115in}}{\pgfqpoint{6.251858in}{1.391949in}}{\pgfqpoint{6.244725in}{1.391949in}}%
\pgfpathcurveto{\pgfqpoint{6.237592in}{1.391949in}}{\pgfqpoint{6.230751in}{1.389115in}}{\pgfqpoint{6.225707in}{1.384071in}}%
\pgfpathcurveto{\pgfqpoint{6.220663in}{1.379028in}}{\pgfqpoint{6.217829in}{1.372186in}}{\pgfqpoint{6.217829in}{1.365053in}}%
\pgfpathcurveto{\pgfqpoint{6.217829in}{1.357920in}}{\pgfqpoint{6.220663in}{1.351079in}}{\pgfqpoint{6.225707in}{1.346035in}}%
\pgfpathcurveto{\pgfqpoint{6.230751in}{1.340991in}}{\pgfqpoint{6.237592in}{1.338158in}}{\pgfqpoint{6.244725in}{1.338158in}}%
\pgfpathclose%
\pgfusepath{stroke,fill}%
\end{pgfscope}%
\begin{pgfscope}%
\pgfpathrectangle{\pgfqpoint{4.985294in}{0.500000in}}{\pgfqpoint{1.764706in}{1.700000in}}%
\pgfusepath{clip}%
\pgfsetbuttcap%
\pgfsetroundjoin%
\definecolor{currentfill}{rgb}{0.965592,0.726236,0.584384}%
\pgfsetfillcolor{currentfill}%
\pgfsetlinewidth{0.311001pt}%
\definecolor{currentstroke}{rgb}{1.000000,1.000000,1.000000}%
\pgfsetstrokecolor{currentstroke}%
\pgfsetdash{}{0pt}%
\pgfpathmoveto{\pgfqpoint{5.328724in}{1.285844in}}%
\pgfpathcurveto{\pgfqpoint{5.335856in}{1.285844in}}{\pgfqpoint{5.342698in}{1.288678in}}{\pgfqpoint{5.347742in}{1.293722in}}%
\pgfpathcurveto{\pgfqpoint{5.352785in}{1.298765in}}{\pgfqpoint{5.355619in}{1.305607in}}{\pgfqpoint{5.355619in}{1.312740in}}%
\pgfpathcurveto{\pgfqpoint{5.355619in}{1.319873in}}{\pgfqpoint{5.352785in}{1.326714in}}{\pgfqpoint{5.347742in}{1.331758in}}%
\pgfpathcurveto{\pgfqpoint{5.342698in}{1.336801in}}{\pgfqpoint{5.335856in}{1.339635in}}{\pgfqpoint{5.328724in}{1.339635in}}%
\pgfpathcurveto{\pgfqpoint{5.321591in}{1.339635in}}{\pgfqpoint{5.314749in}{1.336801in}}{\pgfqpoint{5.309705in}{1.331758in}}%
\pgfpathcurveto{\pgfqpoint{5.304662in}{1.326714in}}{\pgfqpoint{5.301828in}{1.319873in}}{\pgfqpoint{5.301828in}{1.312740in}}%
\pgfpathcurveto{\pgfqpoint{5.301828in}{1.305607in}}{\pgfqpoint{5.304662in}{1.298765in}}{\pgfqpoint{5.309705in}{1.293722in}}%
\pgfpathcurveto{\pgfqpoint{5.314749in}{1.288678in}}{\pgfqpoint{5.321591in}{1.285844in}}{\pgfqpoint{5.328724in}{1.285844in}}%
\pgfpathclose%
\pgfusepath{stroke,fill}%
\end{pgfscope}%
\begin{pgfscope}%
\pgfpathrectangle{\pgfqpoint{4.985294in}{0.500000in}}{\pgfqpoint{1.764706in}{1.700000in}}%
\pgfusepath{clip}%
\pgfsetbuttcap%
\pgfsetroundjoin%
\definecolor{currentfill}{rgb}{0.981377,0.920617,0.865369}%
\pgfsetfillcolor{currentfill}%
\pgfsetlinewidth{0.311001pt}%
\definecolor{currentstroke}{rgb}{1.000000,1.000000,1.000000}%
\pgfsetstrokecolor{currentstroke}%
\pgfsetdash{}{0pt}%
\pgfpathmoveto{\pgfqpoint{6.310316in}{1.269648in}}%
\pgfpathcurveto{\pgfqpoint{6.317449in}{1.269648in}}{\pgfqpoint{6.324291in}{1.272481in}}{\pgfqpoint{6.329335in}{1.277525in}}%
\pgfpathcurveto{\pgfqpoint{6.334378in}{1.282569in}}{\pgfqpoint{6.337212in}{1.289410in}}{\pgfqpoint{6.337212in}{1.296543in}}%
\pgfpathcurveto{\pgfqpoint{6.337212in}{1.303676in}}{\pgfqpoint{6.334378in}{1.310518in}}{\pgfqpoint{6.329335in}{1.315561in}}%
\pgfpathcurveto{\pgfqpoint{6.324291in}{1.320605in}}{\pgfqpoint{6.317449in}{1.323439in}}{\pgfqpoint{6.310316in}{1.323439in}}%
\pgfpathcurveto{\pgfqpoint{6.303184in}{1.323439in}}{\pgfqpoint{6.296342in}{1.320605in}}{\pgfqpoint{6.291298in}{1.315561in}}%
\pgfpathcurveto{\pgfqpoint{6.286255in}{1.310518in}}{\pgfqpoint{6.283421in}{1.303676in}}{\pgfqpoint{6.283421in}{1.296543in}}%
\pgfpathcurveto{\pgfqpoint{6.283421in}{1.289410in}}{\pgfqpoint{6.286255in}{1.282569in}}{\pgfqpoint{6.291298in}{1.277525in}}%
\pgfpathcurveto{\pgfqpoint{6.296342in}{1.272481in}}{\pgfqpoint{6.303184in}{1.269648in}}{\pgfqpoint{6.310316in}{1.269648in}}%
\pgfpathclose%
\pgfusepath{stroke,fill}%
\end{pgfscope}%
\begin{pgfscope}%
\pgfpathrectangle{\pgfqpoint{4.985294in}{0.500000in}}{\pgfqpoint{1.764706in}{1.700000in}}%
\pgfusepath{clip}%
\pgfsetbuttcap%
\pgfsetroundjoin%
\definecolor{currentfill}{rgb}{0.966328,0.750560,0.616961}%
\pgfsetfillcolor{currentfill}%
\pgfsetlinewidth{0.311001pt}%
\definecolor{currentstroke}{rgb}{1.000000,1.000000,1.000000}%
\pgfsetstrokecolor{currentstroke}%
\pgfsetdash{}{0pt}%
\pgfpathmoveto{\pgfqpoint{5.523757in}{0.923959in}}%
\pgfpathcurveto{\pgfqpoint{5.530890in}{0.923959in}}{\pgfqpoint{5.537731in}{0.926792in}}{\pgfqpoint{5.542775in}{0.931836in}}%
\pgfpathcurveto{\pgfqpoint{5.547819in}{0.936880in}}{\pgfqpoint{5.550652in}{0.943721in}}{\pgfqpoint{5.550652in}{0.950854in}}%
\pgfpathcurveto{\pgfqpoint{5.550652in}{0.957987in}}{\pgfqpoint{5.547819in}{0.964829in}}{\pgfqpoint{5.542775in}{0.969872in}}%
\pgfpathcurveto{\pgfqpoint{5.537731in}{0.974916in}}{\pgfqpoint{5.530890in}{0.977750in}}{\pgfqpoint{5.523757in}{0.977750in}}%
\pgfpathcurveto{\pgfqpoint{5.516624in}{0.977750in}}{\pgfqpoint{5.509782in}{0.974916in}}{\pgfqpoint{5.504739in}{0.969872in}}%
\pgfpathcurveto{\pgfqpoint{5.499695in}{0.964829in}}{\pgfqpoint{5.496861in}{0.957987in}}{\pgfqpoint{5.496861in}{0.950854in}}%
\pgfpathcurveto{\pgfqpoint{5.496861in}{0.943721in}}{\pgfqpoint{5.499695in}{0.936880in}}{\pgfqpoint{5.504739in}{0.931836in}}%
\pgfpathcurveto{\pgfqpoint{5.509782in}{0.926792in}}{\pgfqpoint{5.516624in}{0.923959in}}{\pgfqpoint{5.523757in}{0.923959in}}%
\pgfpathclose%
\pgfusepath{stroke,fill}%
\end{pgfscope}%
\begin{pgfscope}%
\pgfpathrectangle{\pgfqpoint{4.985294in}{0.500000in}}{\pgfqpoint{1.764706in}{1.700000in}}%
\pgfusepath{clip}%
\pgfsetbuttcap%
\pgfsetroundjoin%
\definecolor{currentfill}{rgb}{0.966812,0.762584,0.633643}%
\pgfsetfillcolor{currentfill}%
\pgfsetlinewidth{0.311001pt}%
\definecolor{currentstroke}{rgb}{1.000000,1.000000,1.000000}%
\pgfsetstrokecolor{currentstroke}%
\pgfsetdash{}{0pt}%
\pgfpathmoveto{\pgfqpoint{5.380491in}{1.091270in}}%
\pgfpathcurveto{\pgfqpoint{5.387624in}{1.091270in}}{\pgfqpoint{5.394465in}{1.094103in}}{\pgfqpoint{5.399509in}{1.099147in}}%
\pgfpathcurveto{\pgfqpoint{5.404553in}{1.104191in}}{\pgfqpoint{5.407386in}{1.111032in}}{\pgfqpoint{5.407386in}{1.118165in}}%
\pgfpathcurveto{\pgfqpoint{5.407386in}{1.125298in}}{\pgfqpoint{5.404553in}{1.132140in}}{\pgfqpoint{5.399509in}{1.137183in}}%
\pgfpathcurveto{\pgfqpoint{5.394465in}{1.142227in}}{\pgfqpoint{5.387624in}{1.145061in}}{\pgfqpoint{5.380491in}{1.145061in}}%
\pgfpathcurveto{\pgfqpoint{5.373358in}{1.145061in}}{\pgfqpoint{5.366516in}{1.142227in}}{\pgfqpoint{5.361473in}{1.137183in}}%
\pgfpathcurveto{\pgfqpoint{5.356429in}{1.132140in}}{\pgfqpoint{5.353595in}{1.125298in}}{\pgfqpoint{5.353595in}{1.118165in}}%
\pgfpathcurveto{\pgfqpoint{5.353595in}{1.111032in}}{\pgfqpoint{5.356429in}{1.104191in}}{\pgfqpoint{5.361473in}{1.099147in}}%
\pgfpathcurveto{\pgfqpoint{5.366516in}{1.094103in}}{\pgfqpoint{5.373358in}{1.091270in}}{\pgfqpoint{5.380491in}{1.091270in}}%
\pgfpathclose%
\pgfusepath{stroke,fill}%
\end{pgfscope}%
\begin{pgfscope}%
\pgfpathrectangle{\pgfqpoint{4.985294in}{0.500000in}}{\pgfqpoint{1.764706in}{1.700000in}}%
\pgfusepath{clip}%
\pgfsetbuttcap%
\pgfsetroundjoin%
\definecolor{currentfill}{rgb}{0.976287,0.879862,0.805788}%
\pgfsetfillcolor{currentfill}%
\pgfsetlinewidth{0.311001pt}%
\definecolor{currentstroke}{rgb}{1.000000,1.000000,1.000000}%
\pgfsetstrokecolor{currentstroke}%
\pgfsetdash{}{0pt}%
\pgfpathmoveto{\pgfqpoint{6.260151in}{1.215349in}}%
\pgfpathcurveto{\pgfqpoint{6.267284in}{1.215349in}}{\pgfqpoint{6.274125in}{1.218183in}}{\pgfqpoint{6.279169in}{1.223226in}}%
\pgfpathcurveto{\pgfqpoint{6.284213in}{1.228270in}}{\pgfqpoint{6.287046in}{1.235112in}}{\pgfqpoint{6.287046in}{1.242244in}}%
\pgfpathcurveto{\pgfqpoint{6.287046in}{1.249377in}}{\pgfqpoint{6.284213in}{1.256219in}}{\pgfqpoint{6.279169in}{1.261263in}}%
\pgfpathcurveto{\pgfqpoint{6.274125in}{1.266306in}}{\pgfqpoint{6.267284in}{1.269140in}}{\pgfqpoint{6.260151in}{1.269140in}}%
\pgfpathcurveto{\pgfqpoint{6.253018in}{1.269140in}}{\pgfqpoint{6.246176in}{1.266306in}}{\pgfqpoint{6.241133in}{1.261263in}}%
\pgfpathcurveto{\pgfqpoint{6.236089in}{1.256219in}}{\pgfqpoint{6.233255in}{1.249377in}}{\pgfqpoint{6.233255in}{1.242244in}}%
\pgfpathcurveto{\pgfqpoint{6.233255in}{1.235112in}}{\pgfqpoint{6.236089in}{1.228270in}}{\pgfqpoint{6.241133in}{1.223226in}}%
\pgfpathcurveto{\pgfqpoint{6.246176in}{1.218183in}}{\pgfqpoint{6.253018in}{1.215349in}}{\pgfqpoint{6.260151in}{1.215349in}}%
\pgfpathclose%
\pgfusepath{stroke,fill}%
\end{pgfscope}%
\begin{pgfscope}%
\pgfpathrectangle{\pgfqpoint{4.985294in}{0.500000in}}{\pgfqpoint{1.764706in}{1.700000in}}%
\pgfusepath{clip}%
\pgfsetbuttcap%
\pgfsetroundjoin%
\definecolor{currentfill}{rgb}{0.958331,0.519463,0.362986}%
\pgfsetfillcolor{currentfill}%
\pgfsetlinewidth{0.311001pt}%
\definecolor{currentstroke}{rgb}{1.000000,1.000000,1.000000}%
\pgfsetstrokecolor{currentstroke}%
\pgfsetdash{}{0pt}%
\pgfpathmoveto{\pgfqpoint{5.621655in}{1.761458in}}%
\pgfpathcurveto{\pgfqpoint{5.628788in}{1.761458in}}{\pgfqpoint{5.635629in}{1.764292in}}{\pgfqpoint{5.640673in}{1.769335in}}%
\pgfpathcurveto{\pgfqpoint{5.645717in}{1.774379in}}{\pgfqpoint{5.648551in}{1.781221in}}{\pgfqpoint{5.648551in}{1.788354in}}%
\pgfpathcurveto{\pgfqpoint{5.648551in}{1.795486in}}{\pgfqpoint{5.645717in}{1.802328in}}{\pgfqpoint{5.640673in}{1.807372in}}%
\pgfpathcurveto{\pgfqpoint{5.635629in}{1.812415in}}{\pgfqpoint{5.628788in}{1.815249in}}{\pgfqpoint{5.621655in}{1.815249in}}%
\pgfpathcurveto{\pgfqpoint{5.614522in}{1.815249in}}{\pgfqpoint{5.607681in}{1.812415in}}{\pgfqpoint{5.602637in}{1.807372in}}%
\pgfpathcurveto{\pgfqpoint{5.597593in}{1.802328in}}{\pgfqpoint{5.594759in}{1.795486in}}{\pgfqpoint{5.594759in}{1.788354in}}%
\pgfpathcurveto{\pgfqpoint{5.594759in}{1.781221in}}{\pgfqpoint{5.597593in}{1.774379in}}{\pgfqpoint{5.602637in}{1.769335in}}%
\pgfpathcurveto{\pgfqpoint{5.607681in}{1.764292in}}{\pgfqpoint{5.614522in}{1.761458in}}{\pgfqpoint{5.621655in}{1.761458in}}%
\pgfpathclose%
\pgfusepath{stroke,fill}%
\end{pgfscope}%
\begin{pgfscope}%
\pgfpathrectangle{\pgfqpoint{4.985294in}{0.500000in}}{\pgfqpoint{1.764706in}{1.700000in}}%
\pgfusepath{clip}%
\pgfsetbuttcap%
\pgfsetroundjoin%
\definecolor{currentfill}{rgb}{0.973271,0.850724,0.762998}%
\pgfsetfillcolor{currentfill}%
\pgfsetlinewidth{0.311001pt}%
\definecolor{currentstroke}{rgb}{1.000000,1.000000,1.000000}%
\pgfsetstrokecolor{currentstroke}%
\pgfsetdash{}{0pt}%
\pgfpathmoveto{\pgfqpoint{6.363136in}{1.214405in}}%
\pgfpathcurveto{\pgfqpoint{6.370269in}{1.214405in}}{\pgfqpoint{6.377111in}{1.217239in}}{\pgfqpoint{6.382155in}{1.222283in}}%
\pgfpathcurveto{\pgfqpoint{6.387198in}{1.227326in}}{\pgfqpoint{6.390032in}{1.234168in}}{\pgfqpoint{6.390032in}{1.241301in}}%
\pgfpathcurveto{\pgfqpoint{6.390032in}{1.248433in}}{\pgfqpoint{6.387198in}{1.255275in}}{\pgfqpoint{6.382155in}{1.260319in}}%
\pgfpathcurveto{\pgfqpoint{6.377111in}{1.265362in}}{\pgfqpoint{6.370269in}{1.268196in}}{\pgfqpoint{6.363136in}{1.268196in}}%
\pgfpathcurveto{\pgfqpoint{6.356004in}{1.268196in}}{\pgfqpoint{6.349162in}{1.265362in}}{\pgfqpoint{6.344118in}{1.260319in}}%
\pgfpathcurveto{\pgfqpoint{6.339075in}{1.255275in}}{\pgfqpoint{6.336241in}{1.248433in}}{\pgfqpoint{6.336241in}{1.241301in}}%
\pgfpathcurveto{\pgfqpoint{6.336241in}{1.234168in}}{\pgfqpoint{6.339075in}{1.227326in}}{\pgfqpoint{6.344118in}{1.222283in}}%
\pgfpathcurveto{\pgfqpoint{6.349162in}{1.217239in}}{\pgfqpoint{6.356004in}{1.214405in}}{\pgfqpoint{6.363136in}{1.214405in}}%
\pgfpathclose%
\pgfusepath{stroke,fill}%
\end{pgfscope}%
\begin{pgfscope}%
\pgfpathrectangle{\pgfqpoint{4.985294in}{0.500000in}}{\pgfqpoint{1.764706in}{1.700000in}}%
\pgfusepath{clip}%
\pgfsetbuttcap%
\pgfsetroundjoin%
\definecolor{currentfill}{rgb}{0.970255,0.815666,0.711203}%
\pgfsetfillcolor{currentfill}%
\pgfsetlinewidth{0.311001pt}%
\definecolor{currentstroke}{rgb}{1.000000,1.000000,1.000000}%
\pgfsetstrokecolor{currentstroke}%
\pgfsetdash{}{0pt}%
\pgfpathmoveto{\pgfqpoint{6.383199in}{1.257808in}}%
\pgfpathcurveto{\pgfqpoint{6.390332in}{1.257808in}}{\pgfqpoint{6.397174in}{1.260641in}}{\pgfqpoint{6.402217in}{1.265685in}}%
\pgfpathcurveto{\pgfqpoint{6.407261in}{1.270729in}}{\pgfqpoint{6.410095in}{1.277570in}}{\pgfqpoint{6.410095in}{1.284703in}}%
\pgfpathcurveto{\pgfqpoint{6.410095in}{1.291836in}}{\pgfqpoint{6.407261in}{1.298678in}}{\pgfqpoint{6.402217in}{1.303721in}}%
\pgfpathcurveto{\pgfqpoint{6.397174in}{1.308765in}}{\pgfqpoint{6.390332in}{1.311599in}}{\pgfqpoint{6.383199in}{1.311599in}}%
\pgfpathcurveto{\pgfqpoint{6.376066in}{1.311599in}}{\pgfqpoint{6.369225in}{1.308765in}}{\pgfqpoint{6.364181in}{1.303721in}}%
\pgfpathcurveto{\pgfqpoint{6.359137in}{1.298678in}}{\pgfqpoint{6.356304in}{1.291836in}}{\pgfqpoint{6.356304in}{1.284703in}}%
\pgfpathcurveto{\pgfqpoint{6.356304in}{1.277570in}}{\pgfqpoint{6.359137in}{1.270729in}}{\pgfqpoint{6.364181in}{1.265685in}}%
\pgfpathcurveto{\pgfqpoint{6.369225in}{1.260641in}}{\pgfqpoint{6.376066in}{1.257808in}}{\pgfqpoint{6.383199in}{1.257808in}}%
\pgfpathclose%
\pgfusepath{stroke,fill}%
\end{pgfscope}%
\begin{pgfscope}%
\pgfpathrectangle{\pgfqpoint{4.985294in}{0.500000in}}{\pgfqpoint{1.764706in}{1.700000in}}%
\pgfusepath{clip}%
\pgfsetbuttcap%
\pgfsetroundjoin%
\definecolor{currentfill}{rgb}{0.973832,0.856556,0.771584}%
\pgfsetfillcolor{currentfill}%
\pgfsetlinewidth{0.311001pt}%
\definecolor{currentstroke}{rgb}{1.000000,1.000000,1.000000}%
\pgfsetstrokecolor{currentstroke}%
\pgfsetdash{}{0pt}%
\pgfpathmoveto{\pgfqpoint{6.248983in}{1.047749in}}%
\pgfpathcurveto{\pgfqpoint{6.256115in}{1.047749in}}{\pgfqpoint{6.262957in}{1.050583in}}{\pgfqpoint{6.268001in}{1.055627in}}%
\pgfpathcurveto{\pgfqpoint{6.273044in}{1.060670in}}{\pgfqpoint{6.275878in}{1.067512in}}{\pgfqpoint{6.275878in}{1.074645in}}%
\pgfpathcurveto{\pgfqpoint{6.275878in}{1.081777in}}{\pgfqpoint{6.273044in}{1.088619in}}{\pgfqpoint{6.268001in}{1.093663in}}%
\pgfpathcurveto{\pgfqpoint{6.262957in}{1.098706in}}{\pgfqpoint{6.256115in}{1.101540in}}{\pgfqpoint{6.248983in}{1.101540in}}%
\pgfpathcurveto{\pgfqpoint{6.241850in}{1.101540in}}{\pgfqpoint{6.235008in}{1.098706in}}{\pgfqpoint{6.229964in}{1.093663in}}%
\pgfpathcurveto{\pgfqpoint{6.224921in}{1.088619in}}{\pgfqpoint{6.222087in}{1.081777in}}{\pgfqpoint{6.222087in}{1.074645in}}%
\pgfpathcurveto{\pgfqpoint{6.222087in}{1.067512in}}{\pgfqpoint{6.224921in}{1.060670in}}{\pgfqpoint{6.229964in}{1.055627in}}%
\pgfpathcurveto{\pgfqpoint{6.235008in}{1.050583in}}{\pgfqpoint{6.241850in}{1.047749in}}{\pgfqpoint{6.248983in}{1.047749in}}%
\pgfpathclose%
\pgfusepath{stroke,fill}%
\end{pgfscope}%
\begin{pgfscope}%
\pgfpathrectangle{\pgfqpoint{4.985294in}{0.500000in}}{\pgfqpoint{1.764706in}{1.700000in}}%
\pgfusepath{clip}%
\pgfsetbuttcap%
\pgfsetroundjoin%
\definecolor{currentfill}{rgb}{0.968105,0.786346,0.667739}%
\pgfsetfillcolor{currentfill}%
\pgfsetlinewidth{0.311001pt}%
\definecolor{currentstroke}{rgb}{1.000000,1.000000,1.000000}%
\pgfsetstrokecolor{currentstroke}%
\pgfsetdash{}{0pt}%
\pgfpathmoveto{\pgfqpoint{5.499201in}{1.674995in}}%
\pgfpathcurveto{\pgfqpoint{5.506334in}{1.674995in}}{\pgfqpoint{5.513175in}{1.677829in}}{\pgfqpoint{5.518219in}{1.682872in}}%
\pgfpathcurveto{\pgfqpoint{5.523263in}{1.687916in}}{\pgfqpoint{5.526096in}{1.694758in}}{\pgfqpoint{5.526096in}{1.701890in}}%
\pgfpathcurveto{\pgfqpoint{5.526096in}{1.709023in}}{\pgfqpoint{5.523263in}{1.715865in}}{\pgfqpoint{5.518219in}{1.720908in}}%
\pgfpathcurveto{\pgfqpoint{5.513175in}{1.725952in}}{\pgfqpoint{5.506334in}{1.728786in}}{\pgfqpoint{5.499201in}{1.728786in}}%
\pgfpathcurveto{\pgfqpoint{5.492068in}{1.728786in}}{\pgfqpoint{5.485226in}{1.725952in}}{\pgfqpoint{5.480183in}{1.720908in}}%
\pgfpathcurveto{\pgfqpoint{5.475139in}{1.715865in}}{\pgfqpoint{5.472305in}{1.709023in}}{\pgfqpoint{5.472305in}{1.701890in}}%
\pgfpathcurveto{\pgfqpoint{5.472305in}{1.694758in}}{\pgfqpoint{5.475139in}{1.687916in}}{\pgfqpoint{5.480183in}{1.682872in}}%
\pgfpathcurveto{\pgfqpoint{5.485226in}{1.677829in}}{\pgfqpoint{5.492068in}{1.674995in}}{\pgfqpoint{5.499201in}{1.674995in}}%
\pgfpathclose%
\pgfusepath{stroke,fill}%
\end{pgfscope}%
\begin{pgfscope}%
\pgfpathrectangle{\pgfqpoint{4.985294in}{0.500000in}}{\pgfqpoint{1.764706in}{1.700000in}}%
\pgfusepath{clip}%
\pgfsetbuttcap%
\pgfsetroundjoin%
\definecolor{currentfill}{rgb}{0.956817,0.498820,0.345554}%
\pgfsetfillcolor{currentfill}%
\pgfsetlinewidth{0.311001pt}%
\definecolor{currentstroke}{rgb}{1.000000,1.000000,1.000000}%
\pgfsetstrokecolor{currentstroke}%
\pgfsetdash{}{0pt}%
\pgfpathmoveto{\pgfqpoint{6.266445in}{1.746948in}}%
\pgfpathcurveto{\pgfqpoint{6.273578in}{1.746948in}}{\pgfqpoint{6.280420in}{1.749782in}}{\pgfqpoint{6.285463in}{1.754826in}}%
\pgfpathcurveto{\pgfqpoint{6.290507in}{1.759869in}}{\pgfqpoint{6.293341in}{1.766711in}}{\pgfqpoint{6.293341in}{1.773844in}}%
\pgfpathcurveto{\pgfqpoint{6.293341in}{1.780976in}}{\pgfqpoint{6.290507in}{1.787818in}}{\pgfqpoint{6.285463in}{1.792862in}}%
\pgfpathcurveto{\pgfqpoint{6.280420in}{1.797905in}}{\pgfqpoint{6.273578in}{1.800739in}}{\pgfqpoint{6.266445in}{1.800739in}}%
\pgfpathcurveto{\pgfqpoint{6.259312in}{1.800739in}}{\pgfqpoint{6.252471in}{1.797905in}}{\pgfqpoint{6.247427in}{1.792862in}}%
\pgfpathcurveto{\pgfqpoint{6.242383in}{1.787818in}}{\pgfqpoint{6.239550in}{1.780976in}}{\pgfqpoint{6.239550in}{1.773844in}}%
\pgfpathcurveto{\pgfqpoint{6.239550in}{1.766711in}}{\pgfqpoint{6.242383in}{1.759869in}}{\pgfqpoint{6.247427in}{1.754826in}}%
\pgfpathcurveto{\pgfqpoint{6.252471in}{1.749782in}}{\pgfqpoint{6.259312in}{1.746948in}}{\pgfqpoint{6.266445in}{1.746948in}}%
\pgfpathclose%
\pgfusepath{stroke,fill}%
\end{pgfscope}%
\begin{pgfscope}%
\pgfpathrectangle{\pgfqpoint{4.985294in}{0.500000in}}{\pgfqpoint{1.764706in}{1.700000in}}%
\pgfusepath{clip}%
\pgfsetbuttcap%
\pgfsetroundjoin%
\definecolor{currentfill}{rgb}{0.978376,0.897317,0.831308}%
\pgfsetfillcolor{currentfill}%
\pgfsetlinewidth{0.311001pt}%
\definecolor{currentstroke}{rgb}{1.000000,1.000000,1.000000}%
\pgfsetstrokecolor{currentstroke}%
\pgfsetdash{}{0pt}%
\pgfpathmoveto{\pgfqpoint{6.285561in}{1.349068in}}%
\pgfpathcurveto{\pgfqpoint{6.292694in}{1.349068in}}{\pgfqpoint{6.299536in}{1.351902in}}{\pgfqpoint{6.304579in}{1.356946in}}%
\pgfpathcurveto{\pgfqpoint{6.309623in}{1.361989in}}{\pgfqpoint{6.312457in}{1.368831in}}{\pgfqpoint{6.312457in}{1.375964in}}%
\pgfpathcurveto{\pgfqpoint{6.312457in}{1.383097in}}{\pgfqpoint{6.309623in}{1.389938in}}{\pgfqpoint{6.304579in}{1.394982in}}%
\pgfpathcurveto{\pgfqpoint{6.299536in}{1.400026in}}{\pgfqpoint{6.292694in}{1.402860in}}{\pgfqpoint{6.285561in}{1.402860in}}%
\pgfpathcurveto{\pgfqpoint{6.278428in}{1.402860in}}{\pgfqpoint{6.271587in}{1.400026in}}{\pgfqpoint{6.266543in}{1.394982in}}%
\pgfpathcurveto{\pgfqpoint{6.261499in}{1.389938in}}{\pgfqpoint{6.258665in}{1.383097in}}{\pgfqpoint{6.258665in}{1.375964in}}%
\pgfpathcurveto{\pgfqpoint{6.258665in}{1.368831in}}{\pgfqpoint{6.261499in}{1.361989in}}{\pgfqpoint{6.266543in}{1.356946in}}%
\pgfpathcurveto{\pgfqpoint{6.271587in}{1.351902in}}{\pgfqpoint{6.278428in}{1.349068in}}{\pgfqpoint{6.285561in}{1.349068in}}%
\pgfpathclose%
\pgfusepath{stroke,fill}%
\end{pgfscope}%
\begin{pgfscope}%
\pgfpathrectangle{\pgfqpoint{4.985294in}{0.500000in}}{\pgfqpoint{1.764706in}{1.700000in}}%
\pgfusepath{clip}%
\pgfsetbuttcap%
\pgfsetroundjoin%
\definecolor{currentfill}{rgb}{0.981377,0.920617,0.865369}%
\pgfsetfillcolor{currentfill}%
\pgfsetlinewidth{0.311001pt}%
\definecolor{currentstroke}{rgb}{1.000000,1.000000,1.000000}%
\pgfsetstrokecolor{currentstroke}%
\pgfsetdash{}{0pt}%
\pgfpathmoveto{\pgfqpoint{6.310832in}{1.382327in}}%
\pgfpathcurveto{\pgfqpoint{6.317965in}{1.382327in}}{\pgfqpoint{6.324807in}{1.385161in}}{\pgfqpoint{6.329850in}{1.390205in}}%
\pgfpathcurveto{\pgfqpoint{6.334894in}{1.395249in}}{\pgfqpoint{6.337728in}{1.402090in}}{\pgfqpoint{6.337728in}{1.409223in}}%
\pgfpathcurveto{\pgfqpoint{6.337728in}{1.416356in}}{\pgfqpoint{6.334894in}{1.423197in}}{\pgfqpoint{6.329850in}{1.428241in}}%
\pgfpathcurveto{\pgfqpoint{6.324807in}{1.433285in}}{\pgfqpoint{6.317965in}{1.436119in}}{\pgfqpoint{6.310832in}{1.436119in}}%
\pgfpathcurveto{\pgfqpoint{6.303700in}{1.436119in}}{\pgfqpoint{6.296858in}{1.433285in}}{\pgfqpoint{6.291814in}{1.428241in}}%
\pgfpathcurveto{\pgfqpoint{6.286771in}{1.423197in}}{\pgfqpoint{6.283937in}{1.416356in}}{\pgfqpoint{6.283937in}{1.409223in}}%
\pgfpathcurveto{\pgfqpoint{6.283937in}{1.402090in}}{\pgfqpoint{6.286771in}{1.395249in}}{\pgfqpoint{6.291814in}{1.390205in}}%
\pgfpathcurveto{\pgfqpoint{6.296858in}{1.385161in}}{\pgfqpoint{6.303700in}{1.382327in}}{\pgfqpoint{6.310832in}{1.382327in}}%
\pgfpathclose%
\pgfusepath{stroke,fill}%
\end{pgfscope}%
\begin{pgfscope}%
\pgfpathrectangle{\pgfqpoint{4.985294in}{0.500000in}}{\pgfqpoint{1.764706in}{1.700000in}}%
\pgfusepath{clip}%
\pgfsetbuttcap%
\pgfsetroundjoin%
\definecolor{currentfill}{rgb}{0.980678,0.914765,0.856766}%
\pgfsetfillcolor{currentfill}%
\pgfsetlinewidth{0.311001pt}%
\definecolor{currentstroke}{rgb}{1.000000,1.000000,1.000000}%
\pgfsetstrokecolor{currentstroke}%
\pgfsetdash{}{0pt}%
\pgfpathmoveto{\pgfqpoint{6.301985in}{1.241922in}}%
\pgfpathcurveto{\pgfqpoint{6.309118in}{1.241922in}}{\pgfqpoint{6.315959in}{1.244756in}}{\pgfqpoint{6.321003in}{1.249800in}}%
\pgfpathcurveto{\pgfqpoint{6.326047in}{1.254843in}}{\pgfqpoint{6.328880in}{1.261685in}}{\pgfqpoint{6.328880in}{1.268818in}}%
\pgfpathcurveto{\pgfqpoint{6.328880in}{1.275951in}}{\pgfqpoint{6.326047in}{1.282792in}}{\pgfqpoint{6.321003in}{1.287836in}}%
\pgfpathcurveto{\pgfqpoint{6.315959in}{1.292880in}}{\pgfqpoint{6.309118in}{1.295714in}}{\pgfqpoint{6.301985in}{1.295714in}}%
\pgfpathcurveto{\pgfqpoint{6.294852in}{1.295714in}}{\pgfqpoint{6.288010in}{1.292880in}}{\pgfqpoint{6.282967in}{1.287836in}}%
\pgfpathcurveto{\pgfqpoint{6.277923in}{1.282792in}}{\pgfqpoint{6.275089in}{1.275951in}}{\pgfqpoint{6.275089in}{1.268818in}}%
\pgfpathcurveto{\pgfqpoint{6.275089in}{1.261685in}}{\pgfqpoint{6.277923in}{1.254843in}}{\pgfqpoint{6.282967in}{1.249800in}}%
\pgfpathcurveto{\pgfqpoint{6.288010in}{1.244756in}}{\pgfqpoint{6.294852in}{1.241922in}}{\pgfqpoint{6.301985in}{1.241922in}}%
\pgfpathclose%
\pgfusepath{stroke,fill}%
\end{pgfscope}%
\begin{pgfscope}%
\pgfpathrectangle{\pgfqpoint{4.985294in}{0.500000in}}{\pgfqpoint{1.764706in}{1.700000in}}%
\pgfusepath{clip}%
\pgfsetbuttcap%
\pgfsetroundjoin%
\definecolor{currentfill}{rgb}{0.976287,0.879862,0.805788}%
\pgfsetfillcolor{currentfill}%
\pgfsetlinewidth{0.311001pt}%
\definecolor{currentstroke}{rgb}{1.000000,1.000000,1.000000}%
\pgfsetstrokecolor{currentstroke}%
\pgfsetdash{}{0pt}%
\pgfpathmoveto{\pgfqpoint{5.461870in}{1.233642in}}%
\pgfpathcurveto{\pgfqpoint{5.469003in}{1.233642in}}{\pgfqpoint{5.475844in}{1.236476in}}{\pgfqpoint{5.480888in}{1.241520in}}%
\pgfpathcurveto{\pgfqpoint{5.485932in}{1.246564in}}{\pgfqpoint{5.488766in}{1.253405in}}{\pgfqpoint{5.488766in}{1.260538in}}%
\pgfpathcurveto{\pgfqpoint{5.488766in}{1.267671in}}{\pgfqpoint{5.485932in}{1.274512in}}{\pgfqpoint{5.480888in}{1.279556in}}%
\pgfpathcurveto{\pgfqpoint{5.475844in}{1.284600in}}{\pgfqpoint{5.469003in}{1.287434in}}{\pgfqpoint{5.461870in}{1.287434in}}%
\pgfpathcurveto{\pgfqpoint{5.454737in}{1.287434in}}{\pgfqpoint{5.447895in}{1.284600in}}{\pgfqpoint{5.442852in}{1.279556in}}%
\pgfpathcurveto{\pgfqpoint{5.437808in}{1.274512in}}{\pgfqpoint{5.434974in}{1.267671in}}{\pgfqpoint{5.434974in}{1.260538in}}%
\pgfpathcurveto{\pgfqpoint{5.434974in}{1.253405in}}{\pgfqpoint{5.437808in}{1.246564in}}{\pgfqpoint{5.442852in}{1.241520in}}%
\pgfpathcurveto{\pgfqpoint{5.447895in}{1.236476in}}{\pgfqpoint{5.454737in}{1.233642in}}{\pgfqpoint{5.461870in}{1.233642in}}%
\pgfpathclose%
\pgfusepath{stroke,fill}%
\end{pgfscope}%
\begin{pgfscope}%
\pgfpathrectangle{\pgfqpoint{4.985294in}{0.500000in}}{\pgfqpoint{1.764706in}{1.700000in}}%
\pgfusepath{clip}%
\pgfsetbuttcap%
\pgfsetroundjoin%
\definecolor{currentfill}{rgb}{0.973271,0.850724,0.762998}%
\pgfsetfillcolor{currentfill}%
\pgfsetlinewidth{0.311001pt}%
\definecolor{currentstroke}{rgb}{1.000000,1.000000,1.000000}%
\pgfsetstrokecolor{currentstroke}%
\pgfsetdash{}{0pt}%
\pgfpathmoveto{\pgfqpoint{6.238257in}{1.514575in}}%
\pgfpathcurveto{\pgfqpoint{6.245390in}{1.514575in}}{\pgfqpoint{6.252232in}{1.517409in}}{\pgfqpoint{6.257275in}{1.522453in}}%
\pgfpathcurveto{\pgfqpoint{6.262319in}{1.527496in}}{\pgfqpoint{6.265153in}{1.534338in}}{\pgfqpoint{6.265153in}{1.541471in}}%
\pgfpathcurveto{\pgfqpoint{6.265153in}{1.548604in}}{\pgfqpoint{6.262319in}{1.555445in}}{\pgfqpoint{6.257275in}{1.560489in}}%
\pgfpathcurveto{\pgfqpoint{6.252232in}{1.565533in}}{\pgfqpoint{6.245390in}{1.568366in}}{\pgfqpoint{6.238257in}{1.568366in}}%
\pgfpathcurveto{\pgfqpoint{6.231124in}{1.568366in}}{\pgfqpoint{6.224283in}{1.565533in}}{\pgfqpoint{6.219239in}{1.560489in}}%
\pgfpathcurveto{\pgfqpoint{6.214195in}{1.555445in}}{\pgfqpoint{6.211362in}{1.548604in}}{\pgfqpoint{6.211362in}{1.541471in}}%
\pgfpathcurveto{\pgfqpoint{6.211362in}{1.534338in}}{\pgfqpoint{6.214195in}{1.527496in}}{\pgfqpoint{6.219239in}{1.522453in}}%
\pgfpathcurveto{\pgfqpoint{6.224283in}{1.517409in}}{\pgfqpoint{6.231124in}{1.514575in}}{\pgfqpoint{6.238257in}{1.514575in}}%
\pgfpathclose%
\pgfusepath{stroke,fill}%
\end{pgfscope}%
\begin{pgfscope}%
\pgfpathrectangle{\pgfqpoint{4.985294in}{0.500000in}}{\pgfqpoint{1.764706in}{1.700000in}}%
\pgfusepath{clip}%
\pgfsetbuttcap%
\pgfsetroundjoin%
\definecolor{currentfill}{rgb}{0.960043,0.546576,0.387029}%
\pgfsetfillcolor{currentfill}%
\pgfsetlinewidth{0.311001pt}%
\definecolor{currentstroke}{rgb}{1.000000,1.000000,1.000000}%
\pgfsetstrokecolor{currentstroke}%
\pgfsetdash{}{0pt}%
\pgfpathmoveto{\pgfqpoint{5.325618in}{1.508102in}}%
\pgfpathcurveto{\pgfqpoint{5.332751in}{1.508102in}}{\pgfqpoint{5.339592in}{1.510936in}}{\pgfqpoint{5.344636in}{1.515979in}}%
\pgfpathcurveto{\pgfqpoint{5.349680in}{1.521023in}}{\pgfqpoint{5.352514in}{1.527865in}}{\pgfqpoint{5.352514in}{1.534998in}}%
\pgfpathcurveto{\pgfqpoint{5.352514in}{1.542130in}}{\pgfqpoint{5.349680in}{1.548972in}}{\pgfqpoint{5.344636in}{1.554016in}}%
\pgfpathcurveto{\pgfqpoint{5.339592in}{1.559059in}}{\pgfqpoint{5.332751in}{1.561893in}}{\pgfqpoint{5.325618in}{1.561893in}}%
\pgfpathcurveto{\pgfqpoint{5.318485in}{1.561893in}}{\pgfqpoint{5.311643in}{1.559059in}}{\pgfqpoint{5.306600in}{1.554016in}}%
\pgfpathcurveto{\pgfqpoint{5.301556in}{1.548972in}}{\pgfqpoint{5.298722in}{1.542130in}}{\pgfqpoint{5.298722in}{1.534998in}}%
\pgfpathcurveto{\pgfqpoint{5.298722in}{1.527865in}}{\pgfqpoint{5.301556in}{1.521023in}}{\pgfqpoint{5.306600in}{1.515979in}}%
\pgfpathcurveto{\pgfqpoint{5.311643in}{1.510936in}}{\pgfqpoint{5.318485in}{1.508102in}}{\pgfqpoint{5.325618in}{1.508102in}}%
\pgfpathclose%
\pgfusepath{stroke,fill}%
\end{pgfscope}%
\begin{pgfscope}%
\pgfpathrectangle{\pgfqpoint{4.985294in}{0.500000in}}{\pgfqpoint{1.764706in}{1.700000in}}%
\pgfusepath{clip}%
\pgfsetbuttcap%
\pgfsetroundjoin%
\definecolor{currentfill}{rgb}{0.966328,0.750560,0.616961}%
\pgfsetfillcolor{currentfill}%
\pgfsetlinewidth{0.311001pt}%
\definecolor{currentstroke}{rgb}{1.000000,1.000000,1.000000}%
\pgfsetstrokecolor{currentstroke}%
\pgfsetdash{}{0pt}%
\pgfpathmoveto{\pgfqpoint{5.371857in}{1.520412in}}%
\pgfpathcurveto{\pgfqpoint{5.378990in}{1.520412in}}{\pgfqpoint{5.385831in}{1.523246in}}{\pgfqpoint{5.390875in}{1.528290in}}%
\pgfpathcurveto{\pgfqpoint{5.395919in}{1.533333in}}{\pgfqpoint{5.398753in}{1.540175in}}{\pgfqpoint{5.398753in}{1.547308in}}%
\pgfpathcurveto{\pgfqpoint{5.398753in}{1.554441in}}{\pgfqpoint{5.395919in}{1.561282in}}{\pgfqpoint{5.390875in}{1.566326in}}%
\pgfpathcurveto{\pgfqpoint{5.385831in}{1.571370in}}{\pgfqpoint{5.378990in}{1.574203in}}{\pgfqpoint{5.371857in}{1.574203in}}%
\pgfpathcurveto{\pgfqpoint{5.364724in}{1.574203in}}{\pgfqpoint{5.357882in}{1.571370in}}{\pgfqpoint{5.352839in}{1.566326in}}%
\pgfpathcurveto{\pgfqpoint{5.347795in}{1.561282in}}{\pgfqpoint{5.344961in}{1.554441in}}{\pgfqpoint{5.344961in}{1.547308in}}%
\pgfpathcurveto{\pgfqpoint{5.344961in}{1.540175in}}{\pgfqpoint{5.347795in}{1.533333in}}{\pgfqpoint{5.352839in}{1.528290in}}%
\pgfpathcurveto{\pgfqpoint{5.357882in}{1.523246in}}{\pgfqpoint{5.364724in}{1.520412in}}{\pgfqpoint{5.371857in}{1.520412in}}%
\pgfpathclose%
\pgfusepath{stroke,fill}%
\end{pgfscope}%
\begin{pgfscope}%
\pgfpathrectangle{\pgfqpoint{4.985294in}{0.500000in}}{\pgfqpoint{1.764706in}{1.700000in}}%
\pgfusepath{clip}%
\pgfsetbuttcap%
\pgfsetroundjoin%
\definecolor{currentfill}{rgb}{0.979124,0.903132,0.839793}%
\pgfsetfillcolor{currentfill}%
\pgfsetlinewidth{0.311001pt}%
\definecolor{currentstroke}{rgb}{1.000000,1.000000,1.000000}%
\pgfsetstrokecolor{currentstroke}%
\pgfsetdash{}{0pt}%
\pgfpathmoveto{\pgfqpoint{6.271500in}{1.561965in}}%
\pgfpathcurveto{\pgfqpoint{6.278633in}{1.561965in}}{\pgfqpoint{6.285475in}{1.564799in}}{\pgfqpoint{6.290518in}{1.569842in}}%
\pgfpathcurveto{\pgfqpoint{6.295562in}{1.574886in}}{\pgfqpoint{6.298396in}{1.581728in}}{\pgfqpoint{6.298396in}{1.588860in}}%
\pgfpathcurveto{\pgfqpoint{6.298396in}{1.595993in}}{\pgfqpoint{6.295562in}{1.602835in}}{\pgfqpoint{6.290518in}{1.607879in}}%
\pgfpathcurveto{\pgfqpoint{6.285475in}{1.612922in}}{\pgfqpoint{6.278633in}{1.615756in}}{\pgfqpoint{6.271500in}{1.615756in}}%
\pgfpathcurveto{\pgfqpoint{6.264368in}{1.615756in}}{\pgfqpoint{6.257526in}{1.612922in}}{\pgfqpoint{6.252482in}{1.607879in}}%
\pgfpathcurveto{\pgfqpoint{6.247439in}{1.602835in}}{\pgfqpoint{6.244605in}{1.595993in}}{\pgfqpoint{6.244605in}{1.588860in}}%
\pgfpathcurveto{\pgfqpoint{6.244605in}{1.581728in}}{\pgfqpoint{6.247439in}{1.574886in}}{\pgfqpoint{6.252482in}{1.569842in}}%
\pgfpathcurveto{\pgfqpoint{6.257526in}{1.564799in}}{\pgfqpoint{6.264368in}{1.561965in}}{\pgfqpoint{6.271500in}{1.561965in}}%
\pgfpathclose%
\pgfusepath{stroke,fill}%
\end{pgfscope}%
\begin{pgfscope}%
\pgfpathrectangle{\pgfqpoint{4.985294in}{0.500000in}}{\pgfqpoint{1.764706in}{1.700000in}}%
\pgfusepath{clip}%
\pgfsetbuttcap%
\pgfsetroundjoin%
\definecolor{currentfill}{rgb}{0.967735,0.780441,0.659127}%
\pgfsetfillcolor{currentfill}%
\pgfsetlinewidth{0.311001pt}%
\definecolor{currentstroke}{rgb}{1.000000,1.000000,1.000000}%
\pgfsetstrokecolor{currentstroke}%
\pgfsetdash{}{0pt}%
\pgfpathmoveto{\pgfqpoint{5.452209in}{1.645124in}}%
\pgfpathcurveto{\pgfqpoint{5.459342in}{1.645124in}}{\pgfqpoint{5.466183in}{1.647958in}}{\pgfqpoint{5.471227in}{1.653002in}}%
\pgfpathcurveto{\pgfqpoint{5.476271in}{1.658045in}}{\pgfqpoint{5.479105in}{1.664887in}}{\pgfqpoint{5.479105in}{1.672020in}}%
\pgfpathcurveto{\pgfqpoint{5.479105in}{1.679153in}}{\pgfqpoint{5.476271in}{1.685994in}}{\pgfqpoint{5.471227in}{1.691038in}}%
\pgfpathcurveto{\pgfqpoint{5.466183in}{1.696082in}}{\pgfqpoint{5.459342in}{1.698915in}}{\pgfqpoint{5.452209in}{1.698915in}}%
\pgfpathcurveto{\pgfqpoint{5.445076in}{1.698915in}}{\pgfqpoint{5.438234in}{1.696082in}}{\pgfqpoint{5.433191in}{1.691038in}}%
\pgfpathcurveto{\pgfqpoint{5.428147in}{1.685994in}}{\pgfqpoint{5.425313in}{1.679153in}}{\pgfqpoint{5.425313in}{1.672020in}}%
\pgfpathcurveto{\pgfqpoint{5.425313in}{1.664887in}}{\pgfqpoint{5.428147in}{1.658045in}}{\pgfqpoint{5.433191in}{1.653002in}}%
\pgfpathcurveto{\pgfqpoint{5.438234in}{1.647958in}}{\pgfqpoint{5.445076in}{1.645124in}}{\pgfqpoint{5.452209in}{1.645124in}}%
\pgfpathclose%
\pgfusepath{stroke,fill}%
\end{pgfscope}%
\begin{pgfscope}%
\pgfpathrectangle{\pgfqpoint{4.985294in}{0.500000in}}{\pgfqpoint{1.764706in}{1.700000in}}%
\pgfusepath{clip}%
\pgfsetbuttcap%
\pgfsetroundjoin%
\definecolor{currentfill}{rgb}{0.967398,0.774513,0.650573}%
\pgfsetfillcolor{currentfill}%
\pgfsetlinewidth{0.311001pt}%
\definecolor{currentstroke}{rgb}{1.000000,1.000000,1.000000}%
\pgfsetstrokecolor{currentstroke}%
\pgfsetdash{}{0pt}%
\pgfpathmoveto{\pgfqpoint{6.171870in}{1.576067in}}%
\pgfpathcurveto{\pgfqpoint{6.179003in}{1.576067in}}{\pgfqpoint{6.185844in}{1.578901in}}{\pgfqpoint{6.190888in}{1.583945in}}%
\pgfpathcurveto{\pgfqpoint{6.195932in}{1.588989in}}{\pgfqpoint{6.198765in}{1.595830in}}{\pgfqpoint{6.198765in}{1.602963in}}%
\pgfpathcurveto{\pgfqpoint{6.198765in}{1.610096in}}{\pgfqpoint{6.195932in}{1.616938in}}{\pgfqpoint{6.190888in}{1.621981in}}%
\pgfpathcurveto{\pgfqpoint{6.185844in}{1.627025in}}{\pgfqpoint{6.179003in}{1.629859in}}{\pgfqpoint{6.171870in}{1.629859in}}%
\pgfpathcurveto{\pgfqpoint{6.164737in}{1.629859in}}{\pgfqpoint{6.157895in}{1.627025in}}{\pgfqpoint{6.152852in}{1.621981in}}%
\pgfpathcurveto{\pgfqpoint{6.147808in}{1.616938in}}{\pgfqpoint{6.144974in}{1.610096in}}{\pgfqpoint{6.144974in}{1.602963in}}%
\pgfpathcurveto{\pgfqpoint{6.144974in}{1.595830in}}{\pgfqpoint{6.147808in}{1.588989in}}{\pgfqpoint{6.152852in}{1.583945in}}%
\pgfpathcurveto{\pgfqpoint{6.157895in}{1.578901in}}{\pgfqpoint{6.164737in}{1.576067in}}{\pgfqpoint{6.171870in}{1.576067in}}%
\pgfpathclose%
\pgfusepath{stroke,fill}%
\end{pgfscope}%
\begin{pgfscope}%
\pgfpathrectangle{\pgfqpoint{4.985294in}{0.500000in}}{\pgfqpoint{1.764706in}{1.700000in}}%
\pgfusepath{clip}%
\pgfsetbuttcap%
\pgfsetroundjoin%
\definecolor{currentfill}{rgb}{0.973832,0.856556,0.771584}%
\pgfsetfillcolor{currentfill}%
\pgfsetlinewidth{0.311001pt}%
\definecolor{currentstroke}{rgb}{1.000000,1.000000,1.000000}%
\pgfsetstrokecolor{currentstroke}%
\pgfsetdash{}{0pt}%
\pgfpathmoveto{\pgfqpoint{5.482497in}{1.585944in}}%
\pgfpathcurveto{\pgfqpoint{5.489630in}{1.585944in}}{\pgfqpoint{5.496471in}{1.588777in}}{\pgfqpoint{5.501515in}{1.593821in}}%
\pgfpathcurveto{\pgfqpoint{5.506558in}{1.598865in}}{\pgfqpoint{5.509392in}{1.605706in}}{\pgfqpoint{5.509392in}{1.612839in}}%
\pgfpathcurveto{\pgfqpoint{5.509392in}{1.619972in}}{\pgfqpoint{5.506558in}{1.626814in}}{\pgfqpoint{5.501515in}{1.631857in}}%
\pgfpathcurveto{\pgfqpoint{5.496471in}{1.636901in}}{\pgfqpoint{5.489630in}{1.639735in}}{\pgfqpoint{5.482497in}{1.639735in}}%
\pgfpathcurveto{\pgfqpoint{5.475364in}{1.639735in}}{\pgfqpoint{5.468522in}{1.636901in}}{\pgfqpoint{5.463479in}{1.631857in}}%
\pgfpathcurveto{\pgfqpoint{5.458435in}{1.626814in}}{\pgfqpoint{5.455601in}{1.619972in}}{\pgfqpoint{5.455601in}{1.612839in}}%
\pgfpathcurveto{\pgfqpoint{5.455601in}{1.605706in}}{\pgfqpoint{5.458435in}{1.598865in}}{\pgfqpoint{5.463479in}{1.593821in}}%
\pgfpathcurveto{\pgfqpoint{5.468522in}{1.588777in}}{\pgfqpoint{5.475364in}{1.585944in}}{\pgfqpoint{5.482497in}{1.585944in}}%
\pgfpathclose%
\pgfusepath{stroke,fill}%
\end{pgfscope}%
\begin{pgfscope}%
\pgfpathrectangle{\pgfqpoint{4.985294in}{0.500000in}}{\pgfqpoint{1.764706in}{1.700000in}}%
\pgfusepath{clip}%
\pgfsetbuttcap%
\pgfsetroundjoin%
\definecolor{currentfill}{rgb}{0.962765,0.606121,0.444717}%
\pgfsetfillcolor{currentfill}%
\pgfsetlinewidth{0.311001pt}%
\definecolor{currentstroke}{rgb}{1.000000,1.000000,1.000000}%
\pgfsetstrokecolor{currentstroke}%
\pgfsetdash{}{0pt}%
\pgfpathmoveto{\pgfqpoint{5.344400in}{1.081034in}}%
\pgfpathcurveto{\pgfqpoint{5.351533in}{1.081034in}}{\pgfqpoint{5.358375in}{1.083868in}}{\pgfqpoint{5.363418in}{1.088911in}}%
\pgfpathcurveto{\pgfqpoint{5.368462in}{1.093955in}}{\pgfqpoint{5.371296in}{1.100797in}}{\pgfqpoint{5.371296in}{1.107930in}}%
\pgfpathcurveto{\pgfqpoint{5.371296in}{1.115062in}}{\pgfqpoint{5.368462in}{1.121904in}}{\pgfqpoint{5.363418in}{1.126948in}}%
\pgfpathcurveto{\pgfqpoint{5.358375in}{1.131991in}}{\pgfqpoint{5.351533in}{1.134825in}}{\pgfqpoint{5.344400in}{1.134825in}}%
\pgfpathcurveto{\pgfqpoint{5.337267in}{1.134825in}}{\pgfqpoint{5.330426in}{1.131991in}}{\pgfqpoint{5.325382in}{1.126948in}}%
\pgfpathcurveto{\pgfqpoint{5.320338in}{1.121904in}}{\pgfqpoint{5.317504in}{1.115062in}}{\pgfqpoint{5.317504in}{1.107930in}}%
\pgfpathcurveto{\pgfqpoint{5.317504in}{1.100797in}}{\pgfqpoint{5.320338in}{1.093955in}}{\pgfqpoint{5.325382in}{1.088911in}}%
\pgfpathcurveto{\pgfqpoint{5.330426in}{1.083868in}}{\pgfqpoint{5.337267in}{1.081034in}}{\pgfqpoint{5.344400in}{1.081034in}}%
\pgfpathclose%
\pgfusepath{stroke,fill}%
\end{pgfscope}%
\begin{pgfscope}%
\pgfpathrectangle{\pgfqpoint{4.985294in}{0.500000in}}{\pgfqpoint{1.764706in}{1.700000in}}%
\pgfusepath{clip}%
\pgfsetbuttcap%
\pgfsetroundjoin%
\definecolor{currentfill}{rgb}{0.973832,0.856556,0.771584}%
\pgfsetfillcolor{currentfill}%
\pgfsetlinewidth{0.311001pt}%
\definecolor{currentstroke}{rgb}{1.000000,1.000000,1.000000}%
\pgfsetstrokecolor{currentstroke}%
\pgfsetdash{}{0pt}%
\pgfpathmoveto{\pgfqpoint{6.263595in}{1.630039in}}%
\pgfpathcurveto{\pgfqpoint{6.270728in}{1.630039in}}{\pgfqpoint{6.277570in}{1.632873in}}{\pgfqpoint{6.282614in}{1.637917in}}%
\pgfpathcurveto{\pgfqpoint{6.287657in}{1.642960in}}{\pgfqpoint{6.290491in}{1.649802in}}{\pgfqpoint{6.290491in}{1.656935in}}%
\pgfpathcurveto{\pgfqpoint{6.290491in}{1.664068in}}{\pgfqpoint{6.287657in}{1.670909in}}{\pgfqpoint{6.282614in}{1.675953in}}%
\pgfpathcurveto{\pgfqpoint{6.277570in}{1.680997in}}{\pgfqpoint{6.270728in}{1.683831in}}{\pgfqpoint{6.263595in}{1.683831in}}%
\pgfpathcurveto{\pgfqpoint{6.256463in}{1.683831in}}{\pgfqpoint{6.249621in}{1.680997in}}{\pgfqpoint{6.244577in}{1.675953in}}%
\pgfpathcurveto{\pgfqpoint{6.239534in}{1.670909in}}{\pgfqpoint{6.236700in}{1.664068in}}{\pgfqpoint{6.236700in}{1.656935in}}%
\pgfpathcurveto{\pgfqpoint{6.236700in}{1.649802in}}{\pgfqpoint{6.239534in}{1.642960in}}{\pgfqpoint{6.244577in}{1.637917in}}%
\pgfpathcurveto{\pgfqpoint{6.249621in}{1.632873in}}{\pgfqpoint{6.256463in}{1.630039in}}{\pgfqpoint{6.263595in}{1.630039in}}%
\pgfpathclose%
\pgfusepath{stroke,fill}%
\end{pgfscope}%
\begin{pgfscope}%
\pgfpathrectangle{\pgfqpoint{4.985294in}{0.500000in}}{\pgfqpoint{1.764706in}{1.700000in}}%
\pgfusepath{clip}%
\pgfsetbuttcap%
\pgfsetroundjoin%
\definecolor{currentfill}{rgb}{0.972726,0.844889,0.754401}%
\pgfsetfillcolor{currentfill}%
\pgfsetlinewidth{0.311001pt}%
\definecolor{currentstroke}{rgb}{1.000000,1.000000,1.000000}%
\pgfsetstrokecolor{currentstroke}%
\pgfsetdash{}{0pt}%
\pgfpathmoveto{\pgfqpoint{5.513209in}{1.065865in}}%
\pgfpathcurveto{\pgfqpoint{5.520341in}{1.065865in}}{\pgfqpoint{5.527183in}{1.068699in}}{\pgfqpoint{5.532227in}{1.073743in}}%
\pgfpathcurveto{\pgfqpoint{5.537270in}{1.078786in}}{\pgfqpoint{5.540104in}{1.085628in}}{\pgfqpoint{5.540104in}{1.092761in}}%
\pgfpathcurveto{\pgfqpoint{5.540104in}{1.099894in}}{\pgfqpoint{5.537270in}{1.106735in}}{\pgfqpoint{5.532227in}{1.111779in}}%
\pgfpathcurveto{\pgfqpoint{5.527183in}{1.116823in}}{\pgfqpoint{5.520341in}{1.119657in}}{\pgfqpoint{5.513209in}{1.119657in}}%
\pgfpathcurveto{\pgfqpoint{5.506076in}{1.119657in}}{\pgfqpoint{5.499234in}{1.116823in}}{\pgfqpoint{5.494190in}{1.111779in}}%
\pgfpathcurveto{\pgfqpoint{5.489147in}{1.106735in}}{\pgfqpoint{5.486313in}{1.099894in}}{\pgfqpoint{5.486313in}{1.092761in}}%
\pgfpathcurveto{\pgfqpoint{5.486313in}{1.085628in}}{\pgfqpoint{5.489147in}{1.078786in}}{\pgfqpoint{5.494190in}{1.073743in}}%
\pgfpathcurveto{\pgfqpoint{5.499234in}{1.068699in}}{\pgfqpoint{5.506076in}{1.065865in}}{\pgfqpoint{5.513209in}{1.065865in}}%
\pgfpathclose%
\pgfusepath{stroke,fill}%
\end{pgfscope}%
\begin{pgfscope}%
\pgfpathrectangle{\pgfqpoint{4.985294in}{0.500000in}}{\pgfqpoint{1.764706in}{1.700000in}}%
\pgfusepath{clip}%
\pgfsetbuttcap%
\pgfsetroundjoin%
\definecolor{currentfill}{rgb}{0.975018,0.868213,0.788710}%
\pgfsetfillcolor{currentfill}%
\pgfsetlinewidth{0.311001pt}%
\definecolor{currentstroke}{rgb}{1.000000,1.000000,1.000000}%
\pgfsetstrokecolor{currentstroke}%
\pgfsetdash{}{0pt}%
\pgfpathmoveto{\pgfqpoint{5.482632in}{1.132118in}}%
\pgfpathcurveto{\pgfqpoint{5.489765in}{1.132118in}}{\pgfqpoint{5.496606in}{1.134952in}}{\pgfqpoint{5.501650in}{1.139996in}}%
\pgfpathcurveto{\pgfqpoint{5.506694in}{1.145040in}}{\pgfqpoint{5.509528in}{1.151881in}}{\pgfqpoint{5.509528in}{1.159014in}}%
\pgfpathcurveto{\pgfqpoint{5.509528in}{1.166147in}}{\pgfqpoint{5.506694in}{1.172988in}}{\pgfqpoint{5.501650in}{1.178032in}}%
\pgfpathcurveto{\pgfqpoint{5.496606in}{1.183076in}}{\pgfqpoint{5.489765in}{1.185910in}}{\pgfqpoint{5.482632in}{1.185910in}}%
\pgfpathcurveto{\pgfqpoint{5.475499in}{1.185910in}}{\pgfqpoint{5.468657in}{1.183076in}}{\pgfqpoint{5.463614in}{1.178032in}}%
\pgfpathcurveto{\pgfqpoint{5.458570in}{1.172988in}}{\pgfqpoint{5.455736in}{1.166147in}}{\pgfqpoint{5.455736in}{1.159014in}}%
\pgfpathcurveto{\pgfqpoint{5.455736in}{1.151881in}}{\pgfqpoint{5.458570in}{1.145040in}}{\pgfqpoint{5.463614in}{1.139996in}}%
\pgfpathcurveto{\pgfqpoint{5.468657in}{1.134952in}}{\pgfqpoint{5.475499in}{1.132118in}}{\pgfqpoint{5.482632in}{1.132118in}}%
\pgfpathclose%
\pgfusepath{stroke,fill}%
\end{pgfscope}%
\begin{pgfscope}%
\pgfpathrectangle{\pgfqpoint{4.985294in}{0.500000in}}{\pgfqpoint{1.764706in}{1.700000in}}%
\pgfusepath{clip}%
\pgfsetbuttcap%
\pgfsetroundjoin%
\definecolor{currentfill}{rgb}{0.950851,0.435000,0.297228}%
\pgfsetfillcolor{currentfill}%
\pgfsetlinewidth{0.311001pt}%
\definecolor{currentstroke}{rgb}{1.000000,1.000000,1.000000}%
\pgfsetstrokecolor{currentstroke}%
\pgfsetdash{}{0pt}%
\pgfpathmoveto{\pgfqpoint{6.351440in}{1.689398in}}%
\pgfpathcurveto{\pgfqpoint{6.358573in}{1.689398in}}{\pgfqpoint{6.365414in}{1.692232in}}{\pgfqpoint{6.370458in}{1.697276in}}%
\pgfpathcurveto{\pgfqpoint{6.375502in}{1.702320in}}{\pgfqpoint{6.378336in}{1.709161in}}{\pgfqpoint{6.378336in}{1.716294in}}%
\pgfpathcurveto{\pgfqpoint{6.378336in}{1.723427in}}{\pgfqpoint{6.375502in}{1.730268in}}{\pgfqpoint{6.370458in}{1.735312in}}%
\pgfpathcurveto{\pgfqpoint{6.365414in}{1.740356in}}{\pgfqpoint{6.358573in}{1.743190in}}{\pgfqpoint{6.351440in}{1.743190in}}%
\pgfpathcurveto{\pgfqpoint{6.344307in}{1.743190in}}{\pgfqpoint{6.337465in}{1.740356in}}{\pgfqpoint{6.332422in}{1.735312in}}%
\pgfpathcurveto{\pgfqpoint{6.327378in}{1.730268in}}{\pgfqpoint{6.324544in}{1.723427in}}{\pgfqpoint{6.324544in}{1.716294in}}%
\pgfpathcurveto{\pgfqpoint{6.324544in}{1.709161in}}{\pgfqpoint{6.327378in}{1.702320in}}{\pgfqpoint{6.332422in}{1.697276in}}%
\pgfpathcurveto{\pgfqpoint{6.337465in}{1.692232in}}{\pgfqpoint{6.344307in}{1.689398in}}{\pgfqpoint{6.351440in}{1.689398in}}%
\pgfpathclose%
\pgfusepath{stroke,fill}%
\end{pgfscope}%
\begin{pgfscope}%
\pgfpathrectangle{\pgfqpoint{4.985294in}{0.500000in}}{\pgfqpoint{1.764706in}{1.700000in}}%
\pgfusepath{clip}%
\pgfsetbuttcap%
\pgfsetroundjoin%
\definecolor{currentfill}{rgb}{0.968931,0.798091,0.685123}%
\pgfsetfillcolor{currentfill}%
\pgfsetlinewidth{0.311001pt}%
\definecolor{currentstroke}{rgb}{1.000000,1.000000,1.000000}%
\pgfsetstrokecolor{currentstroke}%
\pgfsetdash{}{0pt}%
\pgfpathmoveto{\pgfqpoint{6.363144in}{1.471940in}}%
\pgfpathcurveto{\pgfqpoint{6.370277in}{1.471940in}}{\pgfqpoint{6.377119in}{1.474774in}}{\pgfqpoint{6.382162in}{1.479818in}}%
\pgfpathcurveto{\pgfqpoint{6.387206in}{1.484861in}}{\pgfqpoint{6.390040in}{1.491703in}}{\pgfqpoint{6.390040in}{1.498836in}}%
\pgfpathcurveto{\pgfqpoint{6.390040in}{1.505969in}}{\pgfqpoint{6.387206in}{1.512810in}}{\pgfqpoint{6.382162in}{1.517854in}}%
\pgfpathcurveto{\pgfqpoint{6.377119in}{1.522898in}}{\pgfqpoint{6.370277in}{1.525732in}}{\pgfqpoint{6.363144in}{1.525732in}}%
\pgfpathcurveto{\pgfqpoint{6.356011in}{1.525732in}}{\pgfqpoint{6.349170in}{1.522898in}}{\pgfqpoint{6.344126in}{1.517854in}}%
\pgfpathcurveto{\pgfqpoint{6.339082in}{1.512810in}}{\pgfqpoint{6.336248in}{1.505969in}}{\pgfqpoint{6.336248in}{1.498836in}}%
\pgfpathcurveto{\pgfqpoint{6.336248in}{1.491703in}}{\pgfqpoint{6.339082in}{1.484861in}}{\pgfqpoint{6.344126in}{1.479818in}}%
\pgfpathcurveto{\pgfqpoint{6.349170in}{1.474774in}}{\pgfqpoint{6.356011in}{1.471940in}}{\pgfqpoint{6.363144in}{1.471940in}}%
\pgfpathclose%
\pgfusepath{stroke,fill}%
\end{pgfscope}%
\begin{pgfscope}%
\pgfpathrectangle{\pgfqpoint{4.985294in}{0.500000in}}{\pgfqpoint{1.764706in}{1.700000in}}%
\pgfusepath{clip}%
\pgfsetbuttcap%
\pgfsetroundjoin%
\definecolor{currentfill}{rgb}{0.962283,0.593046,0.431453}%
\pgfsetfillcolor{currentfill}%
\pgfsetlinewidth{0.311001pt}%
\definecolor{currentstroke}{rgb}{1.000000,1.000000,1.000000}%
\pgfsetstrokecolor{currentstroke}%
\pgfsetdash{}{0pt}%
\pgfpathmoveto{\pgfqpoint{6.136754in}{1.518821in}}%
\pgfpathcurveto{\pgfqpoint{6.143887in}{1.518821in}}{\pgfqpoint{6.150729in}{1.521655in}}{\pgfqpoint{6.155773in}{1.526698in}}%
\pgfpathcurveto{\pgfqpoint{6.160816in}{1.531742in}}{\pgfqpoint{6.163650in}{1.538584in}}{\pgfqpoint{6.163650in}{1.545716in}}%
\pgfpathcurveto{\pgfqpoint{6.163650in}{1.552849in}}{\pgfqpoint{6.160816in}{1.559691in}}{\pgfqpoint{6.155773in}{1.564735in}}%
\pgfpathcurveto{\pgfqpoint{6.150729in}{1.569778in}}{\pgfqpoint{6.143887in}{1.572612in}}{\pgfqpoint{6.136754in}{1.572612in}}%
\pgfpathcurveto{\pgfqpoint{6.129622in}{1.572612in}}{\pgfqpoint{6.122780in}{1.569778in}}{\pgfqpoint{6.117736in}{1.564735in}}%
\pgfpathcurveto{\pgfqpoint{6.112693in}{1.559691in}}{\pgfqpoint{6.109859in}{1.552849in}}{\pgfqpoint{6.109859in}{1.545716in}}%
\pgfpathcurveto{\pgfqpoint{6.109859in}{1.538584in}}{\pgfqpoint{6.112693in}{1.531742in}}{\pgfqpoint{6.117736in}{1.526698in}}%
\pgfpathcurveto{\pgfqpoint{6.122780in}{1.521655in}}{\pgfqpoint{6.129622in}{1.518821in}}{\pgfqpoint{6.136754in}{1.518821in}}%
\pgfpathclose%
\pgfusepath{stroke,fill}%
\end{pgfscope}%
\begin{pgfscope}%
\pgfpathrectangle{\pgfqpoint{4.985294in}{0.500000in}}{\pgfqpoint{1.764706in}{1.700000in}}%
\pgfusepath{clip}%
\pgfsetbuttcap%
\pgfsetroundjoin%
\definecolor{currentfill}{rgb}{0.966560,0.756582,0.625273}%
\pgfsetfillcolor{currentfill}%
\pgfsetlinewidth{0.311001pt}%
\definecolor{currentstroke}{rgb}{1.000000,1.000000,1.000000}%
\pgfsetstrokecolor{currentstroke}%
\pgfsetdash{}{0pt}%
\pgfpathmoveto{\pgfqpoint{5.449451in}{0.970324in}}%
\pgfpathcurveto{\pgfqpoint{5.456584in}{0.970324in}}{\pgfqpoint{5.463426in}{0.973158in}}{\pgfqpoint{5.468469in}{0.978201in}}%
\pgfpathcurveto{\pgfqpoint{5.473513in}{0.983245in}}{\pgfqpoint{5.476347in}{0.990087in}}{\pgfqpoint{5.476347in}{0.997220in}}%
\pgfpathcurveto{\pgfqpoint{5.476347in}{1.004352in}}{\pgfqpoint{5.473513in}{1.011194in}}{\pgfqpoint{5.468469in}{1.016238in}}%
\pgfpathcurveto{\pgfqpoint{5.463426in}{1.021281in}}{\pgfqpoint{5.456584in}{1.024115in}}{\pgfqpoint{5.449451in}{1.024115in}}%
\pgfpathcurveto{\pgfqpoint{5.442318in}{1.024115in}}{\pgfqpoint{5.435477in}{1.021281in}}{\pgfqpoint{5.430433in}{1.016238in}}%
\pgfpathcurveto{\pgfqpoint{5.425389in}{1.011194in}}{\pgfqpoint{5.422555in}{1.004352in}}{\pgfqpoint{5.422555in}{0.997220in}}%
\pgfpathcurveto{\pgfqpoint{5.422555in}{0.990087in}}{\pgfqpoint{5.425389in}{0.983245in}}{\pgfqpoint{5.430433in}{0.978201in}}%
\pgfpathcurveto{\pgfqpoint{5.435477in}{0.973158in}}{\pgfqpoint{5.442318in}{0.970324in}}{\pgfqpoint{5.449451in}{0.970324in}}%
\pgfpathclose%
\pgfusepath{stroke,fill}%
\end{pgfscope}%
\begin{pgfscope}%
\pgfpathrectangle{\pgfqpoint{4.985294in}{0.500000in}}{\pgfqpoint{1.764706in}{1.700000in}}%
\pgfusepath{clip}%
\pgfsetbuttcap%
\pgfsetroundjoin%
\definecolor{currentfill}{rgb}{0.953816,0.463738,0.317699}%
\pgfsetfillcolor{currentfill}%
\pgfsetlinewidth{0.311001pt}%
\definecolor{currentstroke}{rgb}{1.000000,1.000000,1.000000}%
\pgfsetstrokecolor{currentstroke}%
\pgfsetdash{}{0pt}%
\pgfpathmoveto{\pgfqpoint{5.369349in}{1.648261in}}%
\pgfpathcurveto{\pgfqpoint{5.376482in}{1.648261in}}{\pgfqpoint{5.383323in}{1.651095in}}{\pgfqpoint{5.388367in}{1.656139in}}%
\pgfpathcurveto{\pgfqpoint{5.393411in}{1.661182in}}{\pgfqpoint{5.396245in}{1.668024in}}{\pgfqpoint{5.396245in}{1.675157in}}%
\pgfpathcurveto{\pgfqpoint{5.396245in}{1.682290in}}{\pgfqpoint{5.393411in}{1.689131in}}{\pgfqpoint{5.388367in}{1.694175in}}%
\pgfpathcurveto{\pgfqpoint{5.383323in}{1.699219in}}{\pgfqpoint{5.376482in}{1.702053in}}{\pgfqpoint{5.369349in}{1.702053in}}%
\pgfpathcurveto{\pgfqpoint{5.362216in}{1.702053in}}{\pgfqpoint{5.355374in}{1.699219in}}{\pgfqpoint{5.350331in}{1.694175in}}%
\pgfpathcurveto{\pgfqpoint{5.345287in}{1.689131in}}{\pgfqpoint{5.342453in}{1.682290in}}{\pgfqpoint{5.342453in}{1.675157in}}%
\pgfpathcurveto{\pgfqpoint{5.342453in}{1.668024in}}{\pgfqpoint{5.345287in}{1.661182in}}{\pgfqpoint{5.350331in}{1.656139in}}%
\pgfpathcurveto{\pgfqpoint{5.355374in}{1.651095in}}{\pgfqpoint{5.362216in}{1.648261in}}{\pgfqpoint{5.369349in}{1.648261in}}%
\pgfpathclose%
\pgfusepath{stroke,fill}%
\end{pgfscope}%
\begin{pgfscope}%
\pgfpathrectangle{\pgfqpoint{4.985294in}{0.500000in}}{\pgfqpoint{1.764706in}{1.700000in}}%
\pgfusepath{clip}%
\pgfsetbuttcap%
\pgfsetroundjoin%
\definecolor{currentfill}{rgb}{0.979891,0.908948,0.848279}%
\pgfsetfillcolor{currentfill}%
\pgfsetlinewidth{0.311001pt}%
\definecolor{currentstroke}{rgb}{1.000000,1.000000,1.000000}%
\pgfsetstrokecolor{currentstroke}%
\pgfsetdash{}{0pt}%
\pgfpathmoveto{\pgfqpoint{6.341865in}{1.327498in}}%
\pgfpathcurveto{\pgfqpoint{6.348998in}{1.327498in}}{\pgfqpoint{6.355840in}{1.330332in}}{\pgfqpoint{6.360883in}{1.335376in}}%
\pgfpathcurveto{\pgfqpoint{6.365927in}{1.340419in}}{\pgfqpoint{6.368761in}{1.347261in}}{\pgfqpoint{6.368761in}{1.354394in}}%
\pgfpathcurveto{\pgfqpoint{6.368761in}{1.361527in}}{\pgfqpoint{6.365927in}{1.368368in}}{\pgfqpoint{6.360883in}{1.373412in}}%
\pgfpathcurveto{\pgfqpoint{6.355840in}{1.378456in}}{\pgfqpoint{6.348998in}{1.381289in}}{\pgfqpoint{6.341865in}{1.381289in}}%
\pgfpathcurveto{\pgfqpoint{6.334732in}{1.381289in}}{\pgfqpoint{6.327891in}{1.378456in}}{\pgfqpoint{6.322847in}{1.373412in}}%
\pgfpathcurveto{\pgfqpoint{6.317803in}{1.368368in}}{\pgfqpoint{6.314969in}{1.361527in}}{\pgfqpoint{6.314969in}{1.354394in}}%
\pgfpathcurveto{\pgfqpoint{6.314969in}{1.347261in}}{\pgfqpoint{6.317803in}{1.340419in}}{\pgfqpoint{6.322847in}{1.335376in}}%
\pgfpathcurveto{\pgfqpoint{6.327891in}{1.330332in}}{\pgfqpoint{6.334732in}{1.327498in}}{\pgfqpoint{6.341865in}{1.327498in}}%
\pgfpathclose%
\pgfusepath{stroke,fill}%
\end{pgfscope}%
\begin{pgfscope}%
\pgfpathrectangle{\pgfqpoint{4.985294in}{0.500000in}}{\pgfqpoint{1.764706in}{1.700000in}}%
\pgfusepath{clip}%
\pgfsetbuttcap%
\pgfsetroundjoin%
\definecolor{currentfill}{rgb}{0.977657,0.891500,0.822809}%
\pgfsetfillcolor{currentfill}%
\pgfsetlinewidth{0.311001pt}%
\definecolor{currentstroke}{rgb}{1.000000,1.000000,1.000000}%
\pgfsetstrokecolor{currentstroke}%
\pgfsetdash{}{0pt}%
\pgfpathmoveto{\pgfqpoint{5.454687in}{1.463079in}}%
\pgfpathcurveto{\pgfqpoint{5.461820in}{1.463079in}}{\pgfqpoint{5.468661in}{1.465913in}}{\pgfqpoint{5.473705in}{1.470956in}}%
\pgfpathcurveto{\pgfqpoint{5.478749in}{1.476000in}}{\pgfqpoint{5.481583in}{1.482842in}}{\pgfqpoint{5.481583in}{1.489974in}}%
\pgfpathcurveto{\pgfqpoint{5.481583in}{1.497107in}}{\pgfqpoint{5.478749in}{1.503949in}}{\pgfqpoint{5.473705in}{1.508993in}}%
\pgfpathcurveto{\pgfqpoint{5.468661in}{1.514036in}}{\pgfqpoint{5.461820in}{1.516870in}}{\pgfqpoint{5.454687in}{1.516870in}}%
\pgfpathcurveto{\pgfqpoint{5.447554in}{1.516870in}}{\pgfqpoint{5.440712in}{1.514036in}}{\pgfqpoint{5.435669in}{1.508993in}}%
\pgfpathcurveto{\pgfqpoint{5.430625in}{1.503949in}}{\pgfqpoint{5.427791in}{1.497107in}}{\pgfqpoint{5.427791in}{1.489974in}}%
\pgfpathcurveto{\pgfqpoint{5.427791in}{1.482842in}}{\pgfqpoint{5.430625in}{1.476000in}}{\pgfqpoint{5.435669in}{1.470956in}}%
\pgfpathcurveto{\pgfqpoint{5.440712in}{1.465913in}}{\pgfqpoint{5.447554in}{1.463079in}}{\pgfqpoint{5.454687in}{1.463079in}}%
\pgfpathclose%
\pgfusepath{stroke,fill}%
\end{pgfscope}%
\begin{pgfscope}%
\pgfpathrectangle{\pgfqpoint{4.985294in}{0.500000in}}{\pgfqpoint{1.764706in}{1.700000in}}%
\pgfusepath{clip}%
\pgfsetbuttcap%
\pgfsetroundjoin%
\definecolor{currentfill}{rgb}{0.965042,0.701564,0.552889}%
\pgfsetfillcolor{currentfill}%
\pgfsetlinewidth{0.311001pt}%
\definecolor{currentstroke}{rgb}{1.000000,1.000000,1.000000}%
\pgfsetstrokecolor{currentstroke}%
\pgfsetdash{}{0pt}%
\pgfpathmoveto{\pgfqpoint{5.365939in}{1.538191in}}%
\pgfpathcurveto{\pgfqpoint{5.373072in}{1.538191in}}{\pgfqpoint{5.379914in}{1.541025in}}{\pgfqpoint{5.384957in}{1.546069in}}%
\pgfpathcurveto{\pgfqpoint{5.390001in}{1.551112in}}{\pgfqpoint{5.392835in}{1.557954in}}{\pgfqpoint{5.392835in}{1.565087in}}%
\pgfpathcurveto{\pgfqpoint{5.392835in}{1.572219in}}{\pgfqpoint{5.390001in}{1.579061in}}{\pgfqpoint{5.384957in}{1.584105in}}%
\pgfpathcurveto{\pgfqpoint{5.379914in}{1.589148in}}{\pgfqpoint{5.373072in}{1.591982in}}{\pgfqpoint{5.365939in}{1.591982in}}%
\pgfpathcurveto{\pgfqpoint{5.358806in}{1.591982in}}{\pgfqpoint{5.351965in}{1.589148in}}{\pgfqpoint{5.346921in}{1.584105in}}%
\pgfpathcurveto{\pgfqpoint{5.341877in}{1.579061in}}{\pgfqpoint{5.339044in}{1.572219in}}{\pgfqpoint{5.339044in}{1.565087in}}%
\pgfpathcurveto{\pgfqpoint{5.339044in}{1.557954in}}{\pgfqpoint{5.341877in}{1.551112in}}{\pgfqpoint{5.346921in}{1.546069in}}%
\pgfpathcurveto{\pgfqpoint{5.351965in}{1.541025in}}{\pgfqpoint{5.358806in}{1.538191in}}{\pgfqpoint{5.365939in}{1.538191in}}%
\pgfpathclose%
\pgfusepath{stroke,fill}%
\end{pgfscope}%
\begin{pgfscope}%
\pgfpathrectangle{\pgfqpoint{4.985294in}{0.500000in}}{\pgfqpoint{1.764706in}{1.700000in}}%
\pgfusepath{clip}%
\pgfsetbuttcap%
\pgfsetroundjoin%
\definecolor{currentfill}{rgb}{0.958791,0.526283,0.368909}%
\pgfsetfillcolor{currentfill}%
\pgfsetlinewidth{0.311001pt}%
\definecolor{currentstroke}{rgb}{1.000000,1.000000,1.000000}%
\pgfsetstrokecolor{currentstroke}%
\pgfsetdash{}{0pt}%
\pgfpathmoveto{\pgfqpoint{6.422250in}{1.476573in}}%
\pgfpathcurveto{\pgfqpoint{6.429383in}{1.476573in}}{\pgfqpoint{6.436225in}{1.479407in}}{\pgfqpoint{6.441268in}{1.484450in}}%
\pgfpathcurveto{\pgfqpoint{6.446312in}{1.489494in}}{\pgfqpoint{6.449146in}{1.496336in}}{\pgfqpoint{6.449146in}{1.503469in}}%
\pgfpathcurveto{\pgfqpoint{6.449146in}{1.510601in}}{\pgfqpoint{6.446312in}{1.517443in}}{\pgfqpoint{6.441268in}{1.522487in}}%
\pgfpathcurveto{\pgfqpoint{6.436225in}{1.527530in}}{\pgfqpoint{6.429383in}{1.530364in}}{\pgfqpoint{6.422250in}{1.530364in}}%
\pgfpathcurveto{\pgfqpoint{6.415117in}{1.530364in}}{\pgfqpoint{6.408276in}{1.527530in}}{\pgfqpoint{6.403232in}{1.522487in}}%
\pgfpathcurveto{\pgfqpoint{6.398188in}{1.517443in}}{\pgfqpoint{6.395354in}{1.510601in}}{\pgfqpoint{6.395354in}{1.503469in}}%
\pgfpathcurveto{\pgfqpoint{6.395354in}{1.496336in}}{\pgfqpoint{6.398188in}{1.489494in}}{\pgfqpoint{6.403232in}{1.484450in}}%
\pgfpathcurveto{\pgfqpoint{6.408276in}{1.479407in}}{\pgfqpoint{6.415117in}{1.476573in}}{\pgfqpoint{6.422250in}{1.476573in}}%
\pgfpathclose%
\pgfusepath{stroke,fill}%
\end{pgfscope}%
\begin{pgfscope}%
\pgfpathrectangle{\pgfqpoint{4.985294in}{0.500000in}}{\pgfqpoint{1.764706in}{1.700000in}}%
\pgfusepath{clip}%
\pgfsetbuttcap%
\pgfsetroundjoin%
\definecolor{currentfill}{rgb}{0.287537,0.111919,0.286609}%
\pgfsetfillcolor{currentfill}%
\pgfsetlinewidth{0.803000pt}%
\definecolor{currentstroke}{rgb}{0.287537,0.111919,0.286609}%
\pgfsetstrokecolor{currentstroke}%
\pgfsetdash{}{0pt}%
\pgfsys@defobject{currentmarker}{\pgfqpoint{-0.033333in}{-0.033333in}}{\pgfqpoint{0.033333in}{0.033333in}}{%
\pgfpathmoveto{\pgfqpoint{0.000000in}{-0.033333in}}%
\pgfpathcurveto{\pgfqpoint{0.008840in}{-0.033333in}}{\pgfqpoint{0.017319in}{-0.029821in}}{\pgfqpoint{0.023570in}{-0.023570in}}%
\pgfpathcurveto{\pgfqpoint{0.029821in}{-0.017319in}}{\pgfqpoint{0.033333in}{-0.008840in}}{\pgfqpoint{0.033333in}{0.000000in}}%
\pgfpathcurveto{\pgfqpoint{0.033333in}{0.008840in}}{\pgfqpoint{0.029821in}{0.017319in}}{\pgfqpoint{0.023570in}{0.023570in}}%
\pgfpathcurveto{\pgfqpoint{0.017319in}{0.029821in}}{\pgfqpoint{0.008840in}{0.033333in}}{\pgfqpoint{0.000000in}{0.033333in}}%
\pgfpathcurveto{\pgfqpoint{-0.008840in}{0.033333in}}{\pgfqpoint{-0.017319in}{0.029821in}}{\pgfqpoint{-0.023570in}{0.023570in}}%
\pgfpathcurveto{\pgfqpoint{-0.029821in}{0.017319in}}{\pgfqpoint{-0.033333in}{0.008840in}}{\pgfqpoint{-0.033333in}{0.000000in}}%
\pgfpathcurveto{\pgfqpoint{-0.033333in}{-0.008840in}}{\pgfqpoint{-0.029821in}{-0.017319in}}{\pgfqpoint{-0.023570in}{-0.023570in}}%
\pgfpathcurveto{\pgfqpoint{-0.017319in}{-0.029821in}}{\pgfqpoint{-0.008840in}{-0.033333in}}{\pgfqpoint{0.000000in}{-0.033333in}}%
\pgfpathclose%
\pgfusepath{stroke,fill}%
}%
\end{pgfscope}%
\begin{pgfscope}%
\pgfpathrectangle{\pgfqpoint{4.985294in}{0.500000in}}{\pgfqpoint{1.764706in}{1.700000in}}%
\pgfusepath{clip}%
\pgfsetbuttcap%
\pgfsetroundjoin%
\definecolor{currentfill}{rgb}{0.638121,0.099382,0.356038}%
\pgfsetfillcolor{currentfill}%
\pgfsetlinewidth{0.803000pt}%
\definecolor{currentstroke}{rgb}{0.638121,0.099382,0.356038}%
\pgfsetstrokecolor{currentstroke}%
\pgfsetdash{}{0pt}%
\pgfsys@defobject{currentmarker}{\pgfqpoint{-0.033333in}{-0.033333in}}{\pgfqpoint{0.033333in}{0.033333in}}{%
\pgfpathmoveto{\pgfqpoint{0.000000in}{-0.033333in}}%
\pgfpathcurveto{\pgfqpoint{0.008840in}{-0.033333in}}{\pgfqpoint{0.017319in}{-0.029821in}}{\pgfqpoint{0.023570in}{-0.023570in}}%
\pgfpathcurveto{\pgfqpoint{0.029821in}{-0.017319in}}{\pgfqpoint{0.033333in}{-0.008840in}}{\pgfqpoint{0.033333in}{0.000000in}}%
\pgfpathcurveto{\pgfqpoint{0.033333in}{0.008840in}}{\pgfqpoint{0.029821in}{0.017319in}}{\pgfqpoint{0.023570in}{0.023570in}}%
\pgfpathcurveto{\pgfqpoint{0.017319in}{0.029821in}}{\pgfqpoint{0.008840in}{0.033333in}}{\pgfqpoint{0.000000in}{0.033333in}}%
\pgfpathcurveto{\pgfqpoint{-0.008840in}{0.033333in}}{\pgfqpoint{-0.017319in}{0.029821in}}{\pgfqpoint{-0.023570in}{0.023570in}}%
\pgfpathcurveto{\pgfqpoint{-0.029821in}{0.017319in}}{\pgfqpoint{-0.033333in}{0.008840in}}{\pgfqpoint{-0.033333in}{0.000000in}}%
\pgfpathcurveto{\pgfqpoint{-0.033333in}{-0.008840in}}{\pgfqpoint{-0.029821in}{-0.017319in}}{\pgfqpoint{-0.023570in}{-0.023570in}}%
\pgfpathcurveto{\pgfqpoint{-0.017319in}{-0.029821in}}{\pgfqpoint{-0.008840in}{-0.033333in}}{\pgfqpoint{0.000000in}{-0.033333in}}%
\pgfpathclose%
\pgfusepath{stroke,fill}%
}%
\end{pgfscope}%
\begin{pgfscope}%
\pgfpathrectangle{\pgfqpoint{4.985294in}{0.500000in}}{\pgfqpoint{1.764706in}{1.700000in}}%
\pgfusepath{clip}%
\pgfsetbuttcap%
\pgfsetroundjoin%
\definecolor{currentfill}{rgb}{0.919781,0.275262,0.242460}%
\pgfsetfillcolor{currentfill}%
\pgfsetlinewidth{0.803000pt}%
\definecolor{currentstroke}{rgb}{0.919781,0.275262,0.242460}%
\pgfsetstrokecolor{currentstroke}%
\pgfsetdash{}{0pt}%
\pgfsys@defobject{currentmarker}{\pgfqpoint{-0.033333in}{-0.033333in}}{\pgfqpoint{0.033333in}{0.033333in}}{%
\pgfpathmoveto{\pgfqpoint{0.000000in}{-0.033333in}}%
\pgfpathcurveto{\pgfqpoint{0.008840in}{-0.033333in}}{\pgfqpoint{0.017319in}{-0.029821in}}{\pgfqpoint{0.023570in}{-0.023570in}}%
\pgfpathcurveto{\pgfqpoint{0.029821in}{-0.017319in}}{\pgfqpoint{0.033333in}{-0.008840in}}{\pgfqpoint{0.033333in}{0.000000in}}%
\pgfpathcurveto{\pgfqpoint{0.033333in}{0.008840in}}{\pgfqpoint{0.029821in}{0.017319in}}{\pgfqpoint{0.023570in}{0.023570in}}%
\pgfpathcurveto{\pgfqpoint{0.017319in}{0.029821in}}{\pgfqpoint{0.008840in}{0.033333in}}{\pgfqpoint{0.000000in}{0.033333in}}%
\pgfpathcurveto{\pgfqpoint{-0.008840in}{0.033333in}}{\pgfqpoint{-0.017319in}{0.029821in}}{\pgfqpoint{-0.023570in}{0.023570in}}%
\pgfpathcurveto{\pgfqpoint{-0.029821in}{0.017319in}}{\pgfqpoint{-0.033333in}{0.008840in}}{\pgfqpoint{-0.033333in}{0.000000in}}%
\pgfpathcurveto{\pgfqpoint{-0.033333in}{-0.008840in}}{\pgfqpoint{-0.029821in}{-0.017319in}}{\pgfqpoint{-0.023570in}{-0.023570in}}%
\pgfpathcurveto{\pgfqpoint{-0.017319in}{-0.029821in}}{\pgfqpoint{-0.008840in}{-0.033333in}}{\pgfqpoint{0.000000in}{-0.033333in}}%
\pgfpathclose%
\pgfusepath{stroke,fill}%
}%
\end{pgfscope}%
\begin{pgfscope}%
\pgfpathrectangle{\pgfqpoint{4.985294in}{0.500000in}}{\pgfqpoint{1.764706in}{1.700000in}}%
\pgfusepath{clip}%
\pgfsetbuttcap%
\pgfsetroundjoin%
\definecolor{currentfill}{rgb}{0.964306,0.663930,0.507747}%
\pgfsetfillcolor{currentfill}%
\pgfsetlinewidth{0.803000pt}%
\definecolor{currentstroke}{rgb}{0.964306,0.663930,0.507747}%
\pgfsetstrokecolor{currentstroke}%
\pgfsetdash{}{0pt}%
\pgfsys@defobject{currentmarker}{\pgfqpoint{-0.033333in}{-0.033333in}}{\pgfqpoint{0.033333in}{0.033333in}}{%
\pgfpathmoveto{\pgfqpoint{0.000000in}{-0.033333in}}%
\pgfpathcurveto{\pgfqpoint{0.008840in}{-0.033333in}}{\pgfqpoint{0.017319in}{-0.029821in}}{\pgfqpoint{0.023570in}{-0.023570in}}%
\pgfpathcurveto{\pgfqpoint{0.029821in}{-0.017319in}}{\pgfqpoint{0.033333in}{-0.008840in}}{\pgfqpoint{0.033333in}{0.000000in}}%
\pgfpathcurveto{\pgfqpoint{0.033333in}{0.008840in}}{\pgfqpoint{0.029821in}{0.017319in}}{\pgfqpoint{0.023570in}{0.023570in}}%
\pgfpathcurveto{\pgfqpoint{0.017319in}{0.029821in}}{\pgfqpoint{0.008840in}{0.033333in}}{\pgfqpoint{0.000000in}{0.033333in}}%
\pgfpathcurveto{\pgfqpoint{-0.008840in}{0.033333in}}{\pgfqpoint{-0.017319in}{0.029821in}}{\pgfqpoint{-0.023570in}{0.023570in}}%
\pgfpathcurveto{\pgfqpoint{-0.029821in}{0.017319in}}{\pgfqpoint{-0.033333in}{0.008840in}}{\pgfqpoint{-0.033333in}{0.000000in}}%
\pgfpathcurveto{\pgfqpoint{-0.033333in}{-0.008840in}}{\pgfqpoint{-0.029821in}{-0.017319in}}{\pgfqpoint{-0.023570in}{-0.023570in}}%
\pgfpathcurveto{\pgfqpoint{-0.017319in}{-0.029821in}}{\pgfqpoint{-0.008840in}{-0.033333in}}{\pgfqpoint{0.000000in}{-0.033333in}}%
\pgfpathclose%
\pgfusepath{stroke,fill}%
}%
\end{pgfscope}%
\begin{pgfscope}%
\pgfsetrectcap%
\pgfsetmiterjoin%
\pgfsetlinewidth{1.003750pt}%
\definecolor{currentstroke}{rgb}{0.150000,0.150000,0.150000}%
\pgfsetstrokecolor{currentstroke}%
\pgfsetdash{}{0pt}%
\pgfpathmoveto{\pgfqpoint{4.985294in}{0.500000in}}%
\pgfpathlineto{\pgfqpoint{4.985294in}{2.200000in}}%
\pgfusepath{stroke}%
\end{pgfscope}%
\begin{pgfscope}%
\pgfsetrectcap%
\pgfsetmiterjoin%
\pgfsetlinewidth{1.003750pt}%
\definecolor{currentstroke}{rgb}{0.150000,0.150000,0.150000}%
\pgfsetstrokecolor{currentstroke}%
\pgfsetdash{}{0pt}%
\pgfpathmoveto{\pgfqpoint{6.750000in}{0.500000in}}%
\pgfpathlineto{\pgfqpoint{6.750000in}{2.200000in}}%
\pgfusepath{stroke}%
\end{pgfscope}%
\begin{pgfscope}%
\pgfsetrectcap%
\pgfsetmiterjoin%
\pgfsetlinewidth{1.003750pt}%
\definecolor{currentstroke}{rgb}{0.150000,0.150000,0.150000}%
\pgfsetstrokecolor{currentstroke}%
\pgfsetdash{}{0pt}%
\pgfpathmoveto{\pgfqpoint{4.985294in}{0.500000in}}%
\pgfpathlineto{\pgfqpoint{6.750000in}{0.500000in}}%
\pgfusepath{stroke}%
\end{pgfscope}%
\begin{pgfscope}%
\pgfsetrectcap%
\pgfsetmiterjoin%
\pgfsetlinewidth{1.003750pt}%
\definecolor{currentstroke}{rgb}{0.150000,0.150000,0.150000}%
\pgfsetstrokecolor{currentstroke}%
\pgfsetdash{}{0pt}%
\pgfpathmoveto{\pgfqpoint{4.985294in}{2.200000in}}%
\pgfpathlineto{\pgfqpoint{6.750000in}{2.200000in}}%
\pgfusepath{stroke}%
\end{pgfscope}%
\begin{pgfscope}%
\definecolor{textcolor}{rgb}{0.150000,0.150000,0.150000}%
\pgfsetstrokecolor{textcolor}%
\pgfsetfillcolor{textcolor}%
\pgftext[x=5.867647in,y=2.283333in,,base]{\color{textcolor}\rmfamily\fontsize{9.600000}{11.520000}\selectfont Iteration 6000}%
\end{pgfscope}%
\end{pgfpicture}%
\makeatother%
\endgroup%

    %  \includegraphics{figures/elbow_example/elbow.pdf}
    \caption{Training curve for a normalizing flow with 4 planar flows. When variational inference is performed on this bi-modal target distribution, an `elbow' forms in the ELBO as the variational approximation changes shape to `snap' to the new mode\cite{blei2017variational}.}
    \label{fig:training_curve_elbow_example}    
    \end{adjustwidth}
\end{figure}

\begin{figure}
    \begin{adjustwidth}{-2.5cm}{-2.5cm}
    \centering
     \includegraphics{figures/elbow_example/elbow.pdf}
    \caption{Training curve for a normalizing flow with 4 planar flows. When variational inference is performed on this bi-modal target distribution, an `elbow' forms in the ELBO as the variational approximation changes shape to `snap' to the new mode\cite{blei2017variational}.}
    % \label{fig:training_curve_elbow_example}   
    \end{adjustwidth}
\end{figure}

From the log term in \cref{eq:kl_divergence} it is clear that minimizing the KL-divergence heavily penalizes candidate distributions $q$ for which the probability at $z$, $q(z)$, is high and $p(z|x)$ is low.
This is why it takes some time for the second mode to be fitted --- the space between the two modes has a low $p(z)$, leading to a penalization on the ELBO for approximations which traverse this space.
Indeed, it can be seen in \cref{fig:elbow_example_10000} that in iteration 10000, the variational approximation deviates from a uni-modal approximation, and this iteration corresponds to the area in the training curve in \cref{fig:training_curve_elbow_example} where the ELBO drops, before in the second mode is captured and the ELBO rises again in iteration 11000.

\subsection{Poisson regression}

\begin{figure}
    \centering
    % \tikz{
    % nodes
    \node[obs] (y) {$y$};
    \node[obs,above=of y] (x) {$\vec x$};
    \node[latent,left=of y] (w) {$\vec{w}$};
    \node[latent,above=of w,fill] (kappa2) {$\kappa^2$};
    % plate
    \plate [inner sep=.25cm,yshift=.2cm] {P} {(w)} {$P$};
    \plate [inner sep=.25cm,yshift=.2cm] {N} {(y)(x)} {$N$};
    % edges
    \edge {kappa2} {w}
    \edge {x, w} {y}
    
    \node [right=of N, anchor=west] {
    $\begin{aligned}
        \kappa^2 & \sim \mathrm{LogNormal}(0, 1) \\
        \vec{w} & \sim \normal(0, \kappa^2) \\
         y & \sim \mathrm{Poisson}(\exp(\vec w\T \vec x))
    \end{aligned}$
    };  

}

% \tikz{
%     % nodes
%     \node[obs] (y) {$y$};
%     \node[latent,above=of y] (mu) {$\vec{\mu}$};
%     \node[latent,above=of mu] (lambda) {$\vec{\lambda}$};
%     \node[obs,above=of lambda] (x) {$\vec x$};
%     \node[latent,left=of lambda] (w) {$\vec{w}$};
%     \node[latent,above=of w,fill] (kappa2) {$\kappa^2$};
%     % plate
%     \plate [inner sep=.25cm,yshift=.2cm] {P} {(w)} {$P$};
%     \plate [inner sep=.25cm,yshift=.2cm] {N} {(y)(x)(lambda)(mu)} {$N$};
%     % edges
%     \edge {kappa2} {w}
%     \edge {x, w} {lambda}
%     \edge {lambda} {mu}
%     \edge {mu} {y}
    
%     \node [right=of N, anchor=west] {
%     $\begin{aligned}
%         \kappa^2 & \sim \mathrm{LogNormal}(0, 1) \\
%         \vec{w} & \sim \normal(0, \kappa^2) \\
%         \vec \lambda & = \vec w\T \vec x \\
%         \vec \mu & = \exp(\vec \lambda) \\
%          y & \sim \mathrm{Poisson}(\vec \mu)
%     \end{aligned}$
%     };  

% }
    \caption{Probabilistic graphical model for Poisson regression problem.}
    \label{fig:poisson-pgm}
\end{figure}

\begin{figure}
    \centering
    % % This file was created with tikzplotlib v0.9.15.
\begin{tikzpicture}

\definecolor{color0}{rgb}{0.768627450980392,0.305882352941176,0.32156862745098}

\begin{axis}[
axis line style={white!15!black},
legend cell align={left},
legend style={
  fill opacity=0.8,
  draw opacity=1,
  text opacity=1,
  at={(0.03,0.97)},
  anchor=north west,
  draw=white!80!black
},
tick align=outside,
tick pos=left,
x grid style={white!80!black},
xlabel=\textcolor{white!15!black}{Age},
xmin=33.1877806451613, xmax=65.8122193548387,
xtick style={color=white!15!black},
xtick={30,40,50,60,70},
xticklabels={
  \(\displaystyle 30\),
  \(\displaystyle 40\),
  \(\displaystyle 50\),
  \(\displaystyle 60\),
  \(\displaystyle 70\)
},
y grid style={white!80!black},
ylabel=\textcolor{white!15!black}{Number of deaths},
ymin=-1.4, ymax=29.4,
ytick style={color=white!15!black},
ytick={-5,0,5,10,15,20,25,30},
yticklabels={
  \(\displaystyle -5\),
  \(\displaystyle 0\),
  \(\displaystyle 5\),
  \(\displaystyle 10\),
  \(\displaystyle 15\),
  \(\displaystyle 20\),
  \(\displaystyle 25\),
  \(\displaystyle 30\)
}
]
\path [draw=color0, fill=color0, opacity=0.25, line width=0.32pt]
(axis cs:35,9)
--(axis cs:35,0)
--(axis cs:36,0)
--(axis cs:37,0)
--(axis cs:38,0)
--(axis cs:39,0)
--(axis cs:40,0)
--(axis cs:41,0)
--(axis cs:42,0)
--(axis cs:43,0)
--(axis cs:44,0)
--(axis cs:45,0)
--(axis cs:46,0)
--(axis cs:47,1)
--(axis cs:48,1)
--(axis cs:49,1)
--(axis cs:50,1)
--(axis cs:51,1)
--(axis cs:52,1)
--(axis cs:53,2)
--(axis cs:54,2)
--(axis cs:55,2)
--(axis cs:56,3)
--(axis cs:57,3)
--(axis cs:58,3)
--(axis cs:59,3)
--(axis cs:60,3)
--(axis cs:61,4)
--(axis cs:62,4)
--(axis cs:63,5)
--(axis cs:64,5)
--(axis cs:64,28)
--(axis cs:64,28)
--(axis cs:63,25)
--(axis cs:62,25)
--(axis cs:61,23)
--(axis cs:60,21.5250000000001)
--(axis cs:59,21)
--(axis cs:58,20)
--(axis cs:57,19)
--(axis cs:56,18)
--(axis cs:55,18)
--(axis cs:54,17)
--(axis cs:53,16.5250000000001)
--(axis cs:52,15)
--(axis cs:51,16)
--(axis cs:50,14)
--(axis cs:49,14)
--(axis cs:48,14)
--(axis cs:47,13)
--(axis cs:46,12)
--(axis cs:45,12)
--(axis cs:44,12)
--(axis cs:43,11)
--(axis cs:42,11)
--(axis cs:41,10)
--(axis cs:40,10)
--(axis cs:39,10)
--(axis cs:38,10)
--(axis cs:37,9.52500000000009)
--(axis cs:36,9)
--(axis cs:35,9)
--cycle;
\addlegendimage{area legend, draw=color0, fill=color0, opacity=0.25, line width=0.32pt}
\addlegendentry{99% Interval}

\path [draw=color0, fill=color0, opacity=0.35, line width=0.32pt]
(axis cs:35,7)
--(axis cs:35,0)
--(axis cs:36,0)
--(axis cs:37,0)
--(axis cs:38,0)
--(axis cs:39,1)
--(axis cs:40,1)
--(axis cs:41,1)
--(axis cs:42,1)
--(axis cs:43,1)
--(axis cs:44,1)
--(axis cs:45,1)
--(axis cs:46,1)
--(axis cs:47,2)
--(axis cs:48,2)
--(axis cs:49,2)
--(axis cs:50,2)
--(axis cs:51,2)
--(axis cs:52,2)
--(axis cs:53,3)
--(axis cs:54,3)
--(axis cs:55,3)
--(axis cs:56,4)
--(axis cs:57,4)
--(axis cs:58,4)
--(axis cs:59,5)
--(axis cs:60,5)
--(axis cs:61,5)
--(axis cs:62,6)
--(axis cs:63,6)
--(axis cs:64,7)
--(axis cs:64,24)
--(axis cs:64,24)
--(axis cs:63,22)
--(axis cs:62,22)
--(axis cs:61,20)
--(axis cs:60,19)
--(axis cs:59,18)
--(axis cs:58,17)
--(axis cs:57,16.625)
--(axis cs:56,16)
--(axis cs:55,15)
--(axis cs:54,15)
--(axis cs:53,14)
--(axis cs:52,13)
--(axis cs:51,13)
--(axis cs:50,12)
--(axis cs:49,12)
--(axis cs:48,12)
--(axis cs:47,11)
--(axis cs:46,10)
--(axis cs:45,10)
--(axis cs:44,10)
--(axis cs:43,10)
--(axis cs:42,9)
--(axis cs:41,9)
--(axis cs:40,9)
--(axis cs:39,8)
--(axis cs:38,8)
--(axis cs:37,8)
--(axis cs:36,8)
--(axis cs:35,7)
--cycle;
\addlegendimage{area legend, draw=color0, fill=color0, opacity=0.35, line width=0.32pt}
\addlegendentry{95% Interval}

\path [draw=color0, fill=color0, opacity=0.6, line width=0.32pt]
(axis cs:35,5)
--(axis cs:35,1)
--(axis cs:36,1)
--(axis cs:37,1)
--(axis cs:38,1)
--(axis cs:39,2)
--(axis cs:40,2)
--(axis cs:41,2)
--(axis cs:42,2)
--(axis cs:43,2)
--(axis cs:44,2)
--(axis cs:45,3)
--(axis cs:46,3)
--(axis cs:47,3)
--(axis cs:48,3)
--(axis cs:49,4)
--(axis cs:50,4)
--(axis cs:51,4)
--(axis cs:52,4)
--(axis cs:53,5)
--(axis cs:54,5)
--(axis cs:55,5)
--(axis cs:56,6)
--(axis cs:57,6)
--(axis cs:58,7)
--(axis cs:59,7)
--(axis cs:60,7)
--(axis cs:61,8)
--(axis cs:62,8)
--(axis cs:63,9)
--(axis cs:64,9)
--(axis cs:64,19)
--(axis cs:64,19)
--(axis cs:63,18)
--(axis cs:62,17)
--(axis cs:61,16)
--(axis cs:60,16)
--(axis cs:59,15)
--(axis cs:58,14)
--(axis cs:57,14)
--(axis cs:56,13)
--(axis cs:55,12)
--(axis cs:54,12)
--(axis cs:53,11)
--(axis cs:52,11)
--(axis cs:51,10)
--(axis cs:50,10)
--(axis cs:49,9)
--(axis cs:48,9)
--(axis cs:47,9)
--(axis cs:46,8)
--(axis cs:45,8)
--(axis cs:44,8)
--(axis cs:43,7)
--(axis cs:42,7)
--(axis cs:41,7)
--(axis cs:40,6)
--(axis cs:39,6)
--(axis cs:38,6)
--(axis cs:37,6)
--(axis cs:36,6)
--(axis cs:35,5)
--cycle;
\addlegendimage{area legend, draw=color0, fill=color0, opacity=0.6, line width=0.32pt}
\addlegendentry{75% Interval}

\addplot [draw=white, fill=white!10!black, mark=*, only marks]
table{%
x  y
35 3
36 1
37 3
38 2
39 2
40 4
41 4
42 7
43 5
44 2
45 8
46 13
47 8
48 2
49 7
50 4
51 7
52 4
53 4
54 11
55 11
56 13
57 12
58 12
59 19
60 12
61 16
62 12
63 6
64 10
};
\addlegendentry{Observations}
\addplot [ultra thick, white!10!black, opacity=0.7]
table {%
35 3.181396484375
36 3.295166015625
37 3.479248046875
38 3.71826171875
39 3.85302734375
40 4.015380859375
41 4.262451171875
42 4.51171875
43 4.74951171875
44 4.970947265625
45 5.24658203125
46 5.474853515625
47 5.838623046875
48 6.13720703125
49 6.446044921875
50 6.787109375
51 7.227294921875
52 7.513671875
53 7.851806640625
54 8.397705078125
55 8.800537109375
56 9.30810546875
57 9.724853515625
58 10.349853515625
59 10.875732421875
60 11.5087890625
61 12.117431640625
62 12.701416015625
63 13.41162109375
64 14.21533203125
};
\addlegendentry{Mean}
\end{axis}

\end{tikzpicture}

    \caption{Posterior predictive distribution $t_{*}|\mathbf{t}$}
\end{figure}

\subsection{Eight schools}

The eight schools problem\cite{rubin1981estimation} is a classic test problem in Bayesian statistics\cite{gelman2013bayesian, carpenter2017stan}.

\begin{figure}
    \centering
    % \tikz{
    % nodes
    \node[obs] (y) {$y$};
    \node[obs,above=of y] (x) {$\vec x$};
    \node[latent,left=of y] (w) {$\vec{w}$};
    \node[latent,above=of w,fill] (kappa2) {$\kappa^2$};
    % plate
    \plate [inner sep=.25cm,yshift=.2cm] {P} {(w)} {$P$};
    \plate [inner sep=.25cm,yshift=.2cm] {N} {(y)(x)} {$N$};
    % edges
    \edge {kappa2} {w}
    \edge {x, w} {y}
    
    \node [right=of N, anchor=west] {
    $\begin{aligned}
        \kappa^2 & \sim \mathrm{LogNormal}(0, 1) \\
        \vec{w} & \sim \normal(0, \kappa^2) \\
         y & \sim \mathrm{Poisson}(\exp(\vec w\T \vec x))
    \end{aligned}$
    };  

}

% \tikz{
%     % nodes
%     \node[obs] (y) {$y$};
%     \node[latent,above=of y] (mu) {$\vec{\mu}$};
%     \node[latent,above=of mu] (lambda) {$\vec{\lambda}$};
%     \node[obs,above=of lambda] (x) {$\vec x$};
%     \node[latent,left=of lambda] (w) {$\vec{w}$};
%     \node[latent,above=of w,fill] (kappa2) {$\kappa^2$};
%     % plate
%     \plate [inner sep=.25cm,yshift=.2cm] {P} {(w)} {$P$};
%     \plate [inner sep=.25cm,yshift=.2cm] {N} {(y)(x)(lambda)(mu)} {$N$};
%     % edges
%     \edge {kappa2} {w}
%     \edge {x, w} {lambda}
%     \edge {lambda} {mu}
%     \edge {mu} {y}
    
%     \node [right=of N, anchor=west] {
%     $\begin{aligned}
%         \kappa^2 & \sim \mathrm{LogNormal}(0, 1) \\
%         \vec{w} & \sim \normal(0, \kappa^2) \\
%         \vec \lambda & = \vec w\T \vec x \\
%         \vec \mu & = \exp(\vec \lambda) \\
%          y & \sim \mathrm{Poisson}(\vec \mu)
%     \end{aligned}$
%     };  

% }
    \caption{Probabilistic graphical model for eight schools problem.}
    \label{fig:8s-pgm}
\end{figure}

The results in \cref{tab:poi} were generated for different variational approximations for 10k iterations, an \textsc{Adam} \cite{kingma2014adam} learning rate of $10^{-2}$, and 256 Monte Carlo ELBO gradient estimation samples.

\begin{table}
\centering
\caption{ELBO and $\hat{k}$-statistic\cite{yao2018yes} for different variational inference algorithms trained on the eight schools test problem. It can be seen that there is a loosely monotonic relationship between the ELBO and the $\hat{k}$-statistic.}
\label{tab:poi}
% \begin{tabular}{ccccccccccc} 
% \toprule
% \multirow{2}{*}{} & \multirow{2}{*}{Mean-field} & \multirow{2}{*}{Full-rank} & \multicolumn{4}{c}{Planar}            & \multicolumn{4}{c}{Radial}             \\
%                                                                               \cmidrule(lr){4-7}                      \cmidrule(lr){8-11}
%                   &                             &                            & 4       & 8       & 16      & 32      & 4       & 8       & 16      & 32       \\ 
% \midrule
% ELBO              & -33.406                     & -32.598                    & -33.297 & -32.728 & -32.412 & -31.791 & -37.756 & -37.683 & -35.904 & -33.292  \\
% $\hat{k}$         & 0.8710                      & 0.8226                     & 0.8329  & 0.8066  & 0.7700  & 0.6379  & 0.9209  & 0.8979  & 0.8760  & 0.8203   \\
% \bottomrule
% \end{tabular}


\begin{tabular}{lrrr} 
\toprule
      &                         & ELBO    & $\hat{k}$  \\
\midrule
\multicolumn{2}{l}{Mean-field}   & -33.406 & 0.8710                        \\\addlinespace
\multicolumn{2}{l}{Full-rank}    & -32.598 & 0.8226                        \\\addlinespace
Planar & 4                       & -33.297 & 0.8329                        \\
      & 8                       & -32.728 & 0.8066                        \\
      & 16                      & -32.412 & 0.7700                        \\
      & 32                      & -31.791 & 0.6379                        \\\addlinespace
Radial & 4                       & -37.756 & 0.9209                        \\
      & 8                       & -37.683 & 0.8979                        \\
      & 16                      & -35.904 & 0.8760                        \\
      & 32                      & -33.292 & 0.8203                        \\
\bottomrule
\end{tabular}
\end{table}

A normalizing flow variational approximation with 32 planar flows which was trained for 1000 iterations achieved a final ELBO of -32.746 and a $\hat{k}$-statistic of 0.7723, which means this approximation performs worse than a fully-converged full-rank family.
This suggests that a more complex model whose training is terminated before convergence generally performs worse than a simpler model which has been trained to convergence.

\subsection{Non-negative matrix factorization}

\begin{figure}
    \centering
    % \tikz{
    % nodes
    \node[obs] (y) {$y$};
    \node[obs,above=of y] (x) {$\vec x$};
    \node[latent,left=of y] (w) {$\vec{w}$};
    \node[latent,above=of w,fill] (kappa2) {$\kappa^2$};
    % plate
    \plate [inner sep=.25cm,yshift=.2cm] {P} {(w)} {$P$};
    \plate [inner sep=.25cm,yshift=.2cm] {N} {(y)(x)} {$N$};
    % edges
    \edge {kappa2} {w}
    \edge {x, w} {y}
    
    \node [right=of N, anchor=west] {
    $\begin{aligned}
        \kappa^2 & \sim \mathrm{LogNormal}(0, 1) \\
        \vec{w} & \sim \normal(0, \kappa^2) \\
         y & \sim \mathrm{Poisson}(\exp(\vec w\T \vec x))
    \end{aligned}$
    };  

}

% \tikz{
%     % nodes
%     \node[obs] (y) {$y$};
%     \node[latent,above=of y] (mu) {$\vec{\mu}$};
%     \node[latent,above=of mu] (lambda) {$\vec{\lambda}$};
%     \node[obs,above=of lambda] (x) {$\vec x$};
%     \node[latent,left=of lambda] (w) {$\vec{w}$};
%     \node[latent,above=of w,fill] (kappa2) {$\kappa^2$};
%     % plate
%     \plate [inner sep=.25cm,yshift=.2cm] {P} {(w)} {$P$};
%     \plate [inner sep=.25cm,yshift=.2cm] {N} {(y)(x)(lambda)(mu)} {$N$};
%     % edges
%     \edge {kappa2} {w}
%     \edge {x, w} {lambda}
%     \edge {lambda} {mu}
%     \edge {mu} {y}
    
%     \node [right=of N, anchor=west] {
%     $\begin{aligned}
%         \kappa^2 & \sim \mathrm{LogNormal}(0, 1) \\
%         \vec{w} & \sim \normal(0, \kappa^2) \\
%         \vec \lambda & = \vec w\T \vec x \\
%         \vec \mu & = \exp(\vec \lambda) \\
%          y & \sim \mathrm{Poisson}(\vec \mu)
%     \end{aligned}$
%     };  

% }
    \caption{Probabilistic graphical model for non-negative matrix factorization.}
    \label{fig:nnmf-pgm}
\end{figure}

\subsection{Radon}

\subsection{Bayesian neural network}



